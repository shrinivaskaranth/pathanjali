\fancyhead[RL]{}
\chapter*{\center ಸಹಸ್ರನಾಮಸ್ತೋತ್ರಾಣಿ}
\section{ದತ್ತಾತ್ರೇಯಸಹಸ್ರನಾಮಸ್ತೋತ್ರಮ್ ॥}
\addcontentsline{toc}{section}{ದತ್ತಾತ್ರೇಯಸಹಸ್ರನಾಮಸ್ತೋತ್ರಮ್ ॥}
ಶ್ರೀದತ್ತಾತ್ರೇಯಾಯ ಸಚ್ಚಿದಾನಂದಾಯ ಸರ್ವಾಂತರಾತ್ಮನೇ ಸದ್ಗುರವೇ ಪರಬ್ರಹ್ಮಣೇ ನಮಃ ।\\
ಓಂ ಅಸ್ಯ ಶ್ರೀದತ್ತಾತ್ರೇಯಸಹಸ್ರನಾಮಸ್ತೋತ್ರಮಂತ್ರಸ್ಯ ಬ್ರಹ್ಮಾಋಷಿಃ . ಅನುಷ್ಟುಪ್ಛಂದಃ . ಶ್ರೀದತ್ತಪುರುಷಃ ಪರಮಾತ್ಮಾ ದೇವತಾ. ಓಂ ಹಂಸಹಂಸಾಯ ವಿದ್ಮಹೇ ಇತಿ ಬೀಜಂ . ಸೋಽಹಂ ಸೋಽಹಂ ಚ ಧೀಮಹಿ ಇತಿ ಶಕ್ತಿಃ. ಹಂಸಃ ಸೋಽಹಂ ಚ ಪ್ರಚೋದಯಾತ್ ಇತಿ ಕೀಲಕಂ .ಶ್ರೀಪರಮಪುರುಷ ಪರಮಹಂಸ ಪರಮಾತ್ಮ ಪ್ರೀತ್ಯರ್ಥೇ ಜಪೇ ವಿನಿಯೋಗಃ ॥

ಅಥಃ ನ್ಯಾಸಃ ।\\
ಓಂ ಹಂಸಾಂ ಗಣೇಶಾಯ ಅಂಗುಷ್ಠಾಭ್ಯಾಂ ನಮಃ ।\\
ಓಂ ಹಂಸೀ ಪ್ರಜಾಪತಯೇ ತರ್ಜನೀಭ್ಯಾಂ ನಮಃ ।\\
ಓಂ ಹಂಸೂಂ ಮಹಾವಿಷ್ಣವೇ ಮಧ್ಯಮಾಭ್ಯಾಂ ನಮಃ ।\\
ಓಂ ಹಂಸೈಂ ಶಂಭವೇ ಅನಾಮಿಕಾಭ್ಯಾಂ ನಮಃ ।\\
ಓಂ ಹಂಸೌಂ ಜೀವಾತ್ಮನೇ ಕನಿಷ್ಠಿಕಾಮ್ಯಾಂ ನಗಃ ।\\
ಓಂ ಹಂಸಃ ಪರಮಾತ್ಮನೇ ಕರತಲಕರಪೃಷ್ಠಾಭ್ಯಾಂ ನಮಃ ।\\
ಏವಂ ಹೃದಯಾದಿಷಡಂಗನ್ಯಾಸಃ ।\\
ಓಂ ಹಂಸಃ ಸೋಽಹಂ ಹಂಸಃ ಇತಿ ದಿಗ್ಬಂಧಃ ॥

\as{ಬಾಲಾರ್ಕಪ್ರಭಮಿಂದ್ರನೀಲಜಟಿಲಂ ಭಸ್ಮಾಂಗರಾಗೋಜ್ಜ್ವಲ।\\
ಶಾಂತಂ ನಾದವಿಲೀನಚಿತ್ತಪವನಂ ಶಾರ್ದೂಲಚರ್ಮಾಂಬರಂ ।\\
ಬ್ರಹ್ಮಜ್ಞೈಃ ಸನಕಾದಿಭಿಃ ಪರಿವೃತಂ ಸಿದ್ಧೈರ್ಮಹಾಯೋಗಿಭಿ।\\
ರ್ದತ್ತಾತ್ರೇಯಮುಪಾಸ್ಮಹೇ ಹೃದಿ ಮುದಾ ಧ್ಯೇಯಂ ಸದಾ ಯೋಗಿನಾಂ ॥}

ಓಂ ಶ್ರೀಮಾನ್ ದೇವೋ ವಿರೂಪಾಕ್ಷೋ ಪುರಾಣಪುರುಷೋತ್ತಮಃ ।\\
ಬ್ರಹ್ಮಾ ಪರೋ ಯತೀನಾಥೋ ದೀನಬಂಧುಃ ಕೃಪಾನಿಧಿಃ ॥೧॥

ಸಾರಸ್ವತೋ ಮುನಿರ್ಮುಖ್ಯಸ್ತೇಜಸ್ವೀ ಭಕ್ತವತ್ಸಲಃ ।\\
ಧರ್ಮೋ ಧರ್ಮಮಯೋ ಧರ್ಮೀ ಧರ್ಮದೋ ಧರ್ಮಭಾವನಃ ॥೨॥

ಭಾಗ್ಯದೋ ಭೋಗದೋ ಭೋಗೀ ಭಾಗ್ಯವಾನ್ ಭಾನುರಂಜನಃ ।\\
ಭಾಸ್ಕರೋ ಭಯಹಾ ಭರ್ತಾ ಭಾವಭೂರ್ಭವತಾರಣಃ ॥೩॥

ಕೃಷ್ಣೋ ಲಕ್ಷ್ಮೀಪತಿರ್ದೇವಃ ಪಾರಿಜಾತಾಪಹಾರಕಃ ।\\
ಸಿಂಹಾದ್ರಿನಿಲಯಃ ಶಂಭುರ್ವ್ಯಂಕಟಾಚಲವಾಸಕಃ ॥೪॥

ಕೋಲ್ಹಾಪುರಃ ಶ್ರೀಜಪವಾನ್ ಮಾಹುರಾರ್ಜಿತಭಿಕ್ಷುಕಃ ।\\
ಸೇತುತೀರ್ಥವಿಶುದ್ಧಾತ್ಮಾ ರಾಮಧ್ಯಾನಪರಾಯಣಃ ॥೫॥

ರಾಮಾರ್ಚಿತೋ ರಾಮಗುರುಃ ರಾಮಾತ್ಮಾ ರಾಮದೈವತಃ ।\\
ಶ್ರೀರಾಮಶಿಷ್ಯೋ ರಾಮಜ್ಞೋ ರಾಮೈಕಾಕ್ಷರತತ್ಪರಃ ॥೬॥

ಶ್ರೀರಾಮಮಂತ್ರವಿಖ್ಯಾತೋ ರಾಮಮಂತ್ರಾಬ್ಧಿಪಾರಗಃ ।\\
ರಾಮಭಕ್ತೋ ರಾಮಸಖಾ ರಾಮವಾನ್ ರಾಮಹರ್ಷಣಃ ॥೭॥

ಅನಸೂಯಾತ್ಮಜೋ ದೇವದತ್ತಶ್ಚಾತ್ರೇಯನಾಮಕಃ ।\\
ಸುರೂಪಃ ಸುಮತಿಃ ಪ್ರಾಜ್ಞಃ ಶ್ರೀದೋ ವೈಕುಂಠವಲ್ಲಭಃ ॥೮॥

ವಿರಜಸ್ಥಾನಕಃ ಶ್ರೇಷ್ಠಃ ಸರ್ವೋ ನಾರಾಯಣಃ ಪ್ರಭುಃ ।\\
ಕರ್ಮಜ್ಞಃ ಕರ್ಮನಿರತೋ ನೃಸಿಂಹೋ ವಾಮನೋಽಚ್ಯುತಃ ॥೯॥

ಕವಿಃ ಕಾವ್ಯೋ ಜಗನ್ನಾಥೋ ಜಗನ್ಮೂರ್ತಿರನಾಮಯಃ ।\\
ಮತ್ಸ್ಯಃ ಕೂರ್ಮೋ ವರಾಹಶ್ಚ ಹರಿಃ ಕೃಷ್ಣೋ ಮಹಾಸ್ಮಯಃ ॥೧೦॥

ರಾಮೋ ರಾಮೋ ರಘುಪತಿರ್ಬುದ್ಧಃ ಕಲ್ಕೀ ಜನಾರ್ದನಃ ।\\
ಗೋವಿಂದೋ ಮಾಧವೋ ವಿಷ್ಣುಃ ಶ್ರೀಧರೋ ದೇವನಾಯಕಃ ॥೧೧॥

ತ್ರಿವಿಕ್ರಮಃ ಕೇಶವಶ್ಚ ವಾಸುದೇವೋ ಮಹೇಶ್ವರಃ ।\\
ಸಂಕರ್ಷಣಃ ಪದ್ಮನಾಭೋ ದಾಮೋದರಪರಃ ಶುಚಿಃ ॥೧೨॥

ಶ್ರೀಶೈಲವನಚಾರೀ ಚ ಭಾರ್ಗವಸ್ಥಾನಕೋವಿದಃ ।\\
ಶೇಷಾಚಲನಿವಾಸೀ ಚ ಸ್ವಾಮೀ ಪುಷ್ಕರಿಣೀಪ್ರಿಯಃ ॥೧೩॥

ಕುಂಭಕೋಣನಿವಾಸೀ ಚ ಕಾಂಚಿವಾಸೀ ರಸೇಶ್ವರಃ ।\\
ರಸಾನುಭೋಕ್ತಾ ಸಿದ್ಧೇಶಃ ಸಿದ್ಧಿಮಾನ್ ಸಿದ್ಧವತ್ಸಲಃ ॥೧೪॥

ಸಿದ್ಧರೂಪಃ ಸಿದ್ಧವಿಧಿಃ ಸಿದ್ಧಾಚಾರಪ್ರವರ್ತಕಃ ।\\
ರಸಾಹಾರೋ ವಿಷಾಹಾರೋ ಗಂಧಕಾದಿ ಪ್ರಸೇವಕಃ ॥೧೫॥

ಯೋಗೀ ಯೋಗಪರೋ ರಾಜಾ ಧೃತಿಮಾನ್ ಮತಿಮಾನ್ಸುಖೀ ।\\
ಬುದ್ಧಿಮಾನ್ನೀತಿಮಾನ್ ಬಾಲೋ ಹ್ಯುನ್ಮತ್ತೋ ಜ್ಞಾನಸಾಗರಃ ॥೧೬॥

ಯೋಗಿಸ್ತುತೋ ಯೋಗಿಚಂದ್ರೋ ಯೋಗಿವಂದ್ಯೋ ಯತೀಶ್ವರಃ ।\\
ಯೋಗಾದಿಮಾನ್ ಯೋಗರೂಪೋ ಯೋಗೀಶೋ ಯೋಗಿಪೂಜಿತಃ ॥೧೭॥

ಕಾಷ್ಠಾಯೋಗೀ ದೃಢಪ್ರಜ್ಞೋ ಲಂಬಿಕಾಯೋಗವಾನ್ ದೃಢಃ ।\\
ಖೇಚರಶ್ಚ ಖಗಃ ಪೂಷಾ ರಶ್ಮಿವಾನ್ಭೂತಭಾವನಃ ॥೧೮॥

ಬ್ರಹ್ಮಜ್ಞಃ ಸನಕಾದಿಭ್ಯಃ ಶ್ರೀಪತಿಃ ಕಾರ್ಯಸಿದ್ಧಿಮಾನ್ ।\\
ಸ್ಪೃಷ್ಟಾಸ್ಪೃಷ್ಟವಿಹೀನಾತ್ಮಾ ಯೋಗಜ್ಞೋ ಯೋಗಮೂರ್ತಿಮಾನ್ ॥೧೯॥

ಮೋಕ್ಷಶ್ರೀರ್ಮೋಕ್ಷದೋ ಮೋಕ್ಷೀ ಮೋಕ್ಷರೂಪೋ ವಿಶೇಷವಾನ್ ।\\
ಸುಖಪ್ರದಃ ಸುಖಃ ಸೌಖ್ಯಃ ಸುಖರೂಪಃ ಸುಖಾತ್ಮಕಃ ॥೨೦॥

ರಾತ್ರಿರೂಪೋ ದಿವಾರೂಪಃ ಸಂಧ್ಯಾಽಽತ್ಮಾ ಕಾಲರೂಪಕಃ ।\\
ಕಾಲಃ ಕಾಲವಿವರ್ಣಶ್ಚ ಬಾಲಃ ಪ್ರಭುರತುಲ್ಯಕಃ ॥೨೧॥

ಸಹಸ್ರಶೀರ್ಷಾ ಪುರುಷೋ ವೇದಾತ್ಮಾ ವೇದಪಾರಗಃ ।\\
ಸಹಸ್ರಚರಣೋಽನಂತಃ ಸಹಸ್ರಾಕ್ಷೋ ಜಿತೇಂದ್ರಿಯಃ ॥೨೨॥

ಸ್ಥೂಲಸೂಕ್ಷ್ಮೋ ನಿರಾಕಾರೋ ನಿರ್ಮೋಹೋ ಭಕ್ತಮೋಹವಾನ್ ।\\
ಮಹೀಯಾನ್ಪರಮಾಣುಶ್ಚ ಜಿತಕ್ರೋಧೋ ಭಯಾಪಹಃ ॥೨೩॥

ಯೋಗಾನಂದಪ್ರದಾತಾ ಚ ಯೋಗೋ ಯೋಗವಿಶಾರದಃ ।\\
ನಿತ್ಯೋ ನಿತ್ಯಾತ್ಮವಾನ್ ಯೋಗೀ ನಿತ್ಯಪೂರ್ಣೋ ನಿರಾಮಯಃ ॥೨೪॥

ದತ್ತಾತ್ರೇಯೋ ದೇಯದತ್ತೋ ಯೋಗೀ ಪರಮಭಾಸ್ಕರಃ ।\\
ಅವಧೂತಃ ಸರ್ವನಾಥಃ ಸತ್ಕರ್ತಾ ಪುರುಷೋತ್ತಮಃ ॥೨೫॥

ಜ್ಞಾನೀ ಲೋಕವಿಭುಃ ಕಾಂತಃ ಶೀತೋಷ್ಣಸಮಬುದ್ಧಿಕಃ ।\\
ವಿದ್ವೇಷಿ ಜನಸಂಹರ್ತಾ ಧರ್ಮಬುದ್ಧಿವಿಚಕ್ಷಣಃ ॥೨೬॥

ನಿತ್ಯತೃಪ್ತೋ ವಿಶೋಕಶ್ಚ ದ್ವಿಭುಜಃ ಕಾಮರೂಪಕಃ ।\\
ಕಲ್ಯಾಣೋಽಭಿಜನೋ ಧೀರೋ ವಿಶಿಷ್ಟಃ ಸುವಿಚಕ್ಷಣಃ ॥೨೭॥

ಶ್ರೀಮದ್ಭಾಗವತಾರ್ಥಜ್ಞೋ ರಾಮಾಯಣವಿಶೇಷವಾನ್ ।\\
ಅಷ್ಟಾದಶಪುರಾಣಜ್ಞೋ ಷಡ್ದರ್ಶನವಿಜೃಂಭಕಃ ॥೨೮॥

ನಿರ್ವಿಕಲ್ಪಃ ಸುರಶ್ರೇಷ್ಠೋ ಹ್ಯುತ್ತಮೋ ಲೋಕಪೂಜಿತಃ ।\\
ಗುಣಾತೀತಃ ಪೂರ್ಣಗುಣೋ ಬ್ರಹ್ಮಣ್ಯೋ ದ್ವಿಜಸಂವೃತಃ ॥೨೯॥

ದಿಗಂಬರೋ ಮಹಾಜ್ಞೇಯೋ ವಿಶ್ವಾತ್ಮಾಽಽತ್ಮಪರಾಯಣಃ ।\\
ವೇದಾಂತಶ್ರವಣೋ ವೇದೀ ಕಲಾವಾನ್ನಿಷ್ಕಲಂಕವಾನ್ ॥೩೦॥

ಮಿತಭಾಷ್ಯಮಿತಭಾಷೀ ಚ ಸೌಮ್ಯೋ ರಾಮೋ ಜಯಃ ಶಿವಃ ।\\
ಸರ್ವಜಿತ್ ಸರ್ವತೋಭದ್ರೋ ಜಯಕಾಂಕ್ಷೀ ಸುಖಾವಹಃ ॥೩೧॥

ಪ್ರತ್ಯರ್ಥಿಕೀರ್ತಿಸಂಹರ್ತಾ ಮಂದರಾರ್ಚಿತಪಾದುಕಃ ।\\
ವೈಕುಂಠವಾಸೀ ದೇವೇಶೋ ವಿರಜಾಸ್ನಾತಮಾನಸಃ ॥೩೨॥

ಶ್ರೀಮೇರುನಿಲಯೋ ಯೋಗೀ ಬಾಲಾರ್ಕಸಮಕಾಂತಿಮಾನ್ ।\\
ರಕ್ತಾಂಗಃ ಶ್ಯಾಮಲಾಂಗಶ್ಚ ಬಹುವೇಷೋ ಬಹುಪ್ರಿಯಃ ॥೩೩॥

ಮಹಾಲಕ್ಷ್ಮ್ಯನ್ನಪೂರ್ಣೇಶಃ ಸ್ವಧಾಕಾರೋ ಯತೀಶ್ವರಃ ।\\
ಸ್ವರ್ಣರೂಪಃ ಸ್ವರ್ಣದಾಯೀ ಮೂಲಿಕಾಯಂತ್ರಕೋವಿದಃ ॥೩೪॥

ಆನೀತಮೂಲಿಕಾಯಂತ್ರೋ ಭಕ್ತಾಭೀಷ್ಟಪ್ರದೋ ಮಹಾನ್ ।\\
ಶಾಂತಾಕಾರೋ ಮಹಾಮಾಯೋ ಮಾಹುರಸ್ಥೋ ಜಗನ್ಮಯಃ ॥೩೫॥

ಬದ್ಧಾಶನಶ್ಚ ಸೂಕ್ಷ್ಮಾಂಶೀ ಮಿತಾಹಾರೋ ನಿರುದ್ಯಮಃ ।\\
ಧ್ಯಾನಾತ್ಮಾ ಧ್ಯಾನಯೋಗಾತ್ಮಾ ಧ್ಯಾನಸ್ಥೋ ಧ್ಯಾನಸತ್ಪ್ರಿಯಃ ॥೩೬॥

ಸತ್ಯಧ್ಯಾನಃ ಸತ್ಯಮಯಃ ಸತ್ಯರೂಪೋ ನಿಜಾಕೃತಿಃ ।\\
ತ್ರಿಲೋಕಗುರುರೇಕಾತ್ಮಾ ಭಸ್ಮೋದ್ಧೂಲಿತವಿಗ್ರಹಃ ॥೩೭॥

ಪ್ರಿಯಾಪ್ರಿಯಸಮಃ ಪೂರ್ಣೋ ಲಾಭಾಲಾಭಸಮಪ್ರಿಯಃ ।\\
ಸುಖದುಃಖಸಮೋ ಹ್ರೀಮಾನ್ ಹಿತಾಹಿತಸಮಃ ಪರಃ ॥೩೮॥

ಗುರುರ್ಬ್ರಹ್ಮಾ ಚ ವಿಷ್ಣುಶ್ಚ ಮಹಾವಿಷ್ಣುಃ ಸನಾತನಃ ।\\
ಸದಾಶಿವೋ ಮಹೇಂದ್ರಶ್ಚ ಗೋವಿಂದೋ ಮಧುಸೂದನಃ ॥೩೯॥

ಕರ್ತಾ ಕಾರಯಿತಾ ರುದ್ರಃ ಸರ್ವಚಾರೀ ತು ಯಾಚಕಃ ।\\
ಸಂಪತ್ಪ್ರದೋ ವೃಷ್ಟಿರೂಪೋ ಮೇಘರೂಪಸ್ತಪಃಪ್ರಿಯಃ ॥೪೦॥

ತಪೋಮೂರ್ತಿ ಸ್ತಪೋರಾಶಿ ಸ್ತಪಸ್ವೀ ಚ ತಪೋಧನಃ ।\\
ತಪೋಮಯ ಸ್ತಪಃಶುದ್ಧೋ ಜನಕೋ ವಿಶ್ವಸೃಗ್ವಿಧಿಃ ॥೪೧॥

ತಪಃಸಿದ್ಧ ಸ್ತಪಃಸಾಧ್ಯ ಸ್ತಪಃಕರ್ತಾ ತಪಃಕ್ರತುಃ ।\\
ತಪಃಶಮ ಸ್ತಪಃಕೀರ್ತಿ ಸ್ತಪೋದಾರ ಸ್ತಪೋಽತ್ಯಯಃ ॥೪೨॥

ತಪೋರೇತ ಸ್ತಪೋಜ್ಯೋತಿ ಸ್ತಪಾತ್ಮಾ ಚಾತ್ರಿನಂದನಃ ।\\
ನಿಷ್ಕಲ್ಮಷೋ ನಿಷ್ಕಪಟೋ ನಿರ್ವಿಘ್ನೋ ಧರ್ಮಭೀರುಕಃ ॥೪೩॥

ವೈದ್ಯುತಸ್ತಾರಕಃ ಕರ್ಮವೈದಿಕೋ ಬ್ರಾಹ್ಮಣೋ ಯತಿಃ ।\\
ನಕ್ಷತ್ರತೇಜಾ ದೀಪ್ತಾತ್ಮಾ ಪರಿಶುದ್ಧೋ ವಿಮತ್ಸರಃ ॥೪೪॥

ಜಟೀ ಕೃಷ್ಣಾಜಿನಪದೋ ವ್ಯಾಘ್ರಚರ್ಮಧರೋ ವಶೀ ।\\
ಜಿತೇಂದ್ರಿಯಶ್ಚೀರವಾಸಾಃ ಶುಕ್ಲವಸ್ತ್ರಾಂಬರೋ ಹರಿಃ ॥೪೫॥

ಚಂದ್ರಾನುಜ ಶ್ಚಂದ್ರಮುಖಃ ಶುಕಯೋಗೀ ವರಪ್ರದಃ ।\\
ದಿವ್ಯಯೋಗೀ ಪಂಚತಪೋ ಮಾಸರ್ತುವತ್ಸರಾನನಃ ॥೪೬॥

ಭೂತಜ್ಞೋ ವರ್ತಮಾನಜ್ಞೋ ಭಾವಿಜ್ಞೋ ಧರ್ಮವತ್ಸಲಃ ।\\
ಪ್ರಜಾಹಿತಃ ಸರ್ವಹಿತ ಅನಿಂದ್ಯೋ ಲೋಕವಂದಿತಃ ॥೪೭॥

ಆಕುಂಚಯೋಗ ಸಂಬದ್ಧ ಮಲಮೂತ್ರ ರಸಾದಿಕಃ ।\\
ಕನಕೀಭೂತ ಮಲವಾನ್ ರಾಜಯೋಗ ವಿಚಕ್ಷಣಃ ॥೪೮॥

ಶಕಟಾದಿ ವಿಶೇಷಜ್ಞೋ ಲಂಬಿಕಾನೀತಿತತ್ಪರಃ ।\\
ಪ್ರಪಂಚರೂಪೀ ಬಲವಾನ್ ಏಕಕೌಪೀನವಸ್ತ್ರಕಃ ॥೪೯॥

ದಿಗಂಬರಃ ಸೋತ್ತರೀಯಃ ಸಜಟಃ ಸಕಮಂಡಲುಃ ।\\
ನಿರ್ದಂಡಶ್ಚಾಸಿದಂಡಶ್ಚ ಸ್ತ್ರೀವೇಷಃ ಪುರುಷಾಕೃತಿಃ ॥೫೦॥

ತುಲಸೀಕಾಷ್ಠಮಾಲೀ ಚ ರೌದ್ರಃ ಸ್ಫಟಿಕಮಾಲಿಕಃ ।\\
ನಿರ್ಮಾಲಿಕಃ ಶುದ್ಧತರಃ ಸ್ವೇಚ್ಛಾ ಅಮರವಾನ್ ಪರಃ ॥೫೧॥

ಉರ್ಧ್ವಪುಂಡ್ರಸ್ತ್ರಿಪುಂಡ್ರಾಂಕೋ ದ್ವಂದ್ವಹೀನಃ ಸುನಿರ್ಮಲಃ ।\\
ನಿರ್ಜಟಃ ಸುಜಟೋ ಹೇಯೋ ಭಸ್ಮಶಾಯೀ ಸುಭೋಗವಾನ್ ॥೫೨॥

ಮೂತ್ರಸ್ಪರ್ಶೋ ಮಲಸ್ಪರ್ಶೋ ಜಾತಿಹೀನಃ ಸುಜಾತಿಕಃ ।\\
ಅಭಕ್ಷ್ಯಭಕ್ಷೋ ನಿರ್ಭಕ್ಷೋ ಜಗದ್ವಂದಿತ ದೇಹವಾನ್ ॥೫೩॥

ಭೂಷಣೋ ದೂಷಣಸಮಃ ಕಾಲಾಕಾಲೋ ದಯಾನಿಧಿಃ ।\\
ಬಾಲಪ್ರಿಯೋ ಬಾಲರುಚಿರ್ಬಾಲವಾನತಿಬಾಲಕಃ ॥೫೪॥

ಬಾಲಕ್ರೀಡೋ ಬಾಲರತೋ ಬಾಲಸಂಘವೃತೋ ಬಲೀ ।\\
ಬಾಲಲೀಲಾವಿನೋದಶ್ಚ ಕರ್ಣಾಕರ್ಷಣಕಾರಕಃ ॥೫೫॥

ಕ್ರಯಾನೀತವಣಿಕ್ಪಣ್ಯೋ ಗುಡಸೂಪಾದಿಭಕ್ಷಕಃ ।\\
ಬಾಲವದ್ಗೀತಹೃಷ್ಟಶ್ಚ ಮುಷ್ಟಿಯುದ್ಧಕರಶ್ಚಲಃ ॥೫೬॥

ಅದೃಶ್ಯೋ ದೃಶ್ಯಮಾನಶ್ಚ ದ್ವಂದ್ವಯುದ್ಧಪ್ರವರ್ತಕಃ ।\\
ಪಲಾಯಮಾನೋ ಬಾಲಾಢ್ಯೋ ಬಾಲಹಾಸಃ ಸುಸಂಗತಃ ॥೫೭॥

ಪ್ರತ್ಯಾಗತಃ ಪುನರ್ಗಚ್ಛಚ್ಚಕ್ರವದ್ಗಮನಾಕುಲಃ ।\\
ಚೋರವದ್ಧೃತಸರ್ವಸ್ವೋ ಜನತಾಽಽರ್ತಿಕದೇಹವಾನ್ ॥೫೮॥

ಪ್ರಹಸನ್ಪ್ರವದಂದತ್ತೋ ದಿವ್ಯಮಂಗಲವಿಗ್ರಹಃ ।\\
ಮಾಯಾಬಾಲಶ್ಚ ಮಾಯಾವೀ ಪೂರ್ಣಲೀಲೋ ಮುನೀಶ್ವರಃ ॥೫೯॥

ಮಾಹುರೇಶೋ ವಿಶುದ್ಧಾತ್ಮಾ ಯಶಸ್ವೀ ಕೀರ್ತಿಮಾನ್ ಯುವಾ ।\\
ಸವಿಕಲ್ಪಃ ಸಚ್ಚಿದಾಭೋ ಗುಣವಾನ್ ಸೌಮ್ಯಭಾವನಃ ॥೬೦॥

ಪಿನಾಕೀ ಶಶಿಮೌಲೀ ಚ ವಾಸುದೇವೋ ದಿವಸ್ಪತಿಃ ।\\
ಸುಶಿರಾಃ ಸೂರ್ಯತೇಜಶ್ಚ ಶ್ರೀಗಂಭೀರೋಷ್ಠ ಉನ್ನತಿಃ ॥೬೧॥

ದಶಪದ್ಮಾ ತ್ರಿಶೀರ್ಷಶ್ಚ ತ್ರಿಭಿರ್ವ್ಯಾಪ್ತೋ ದ್ವಿಶುಕ್ಲವಾನ್ ।\\
ತ್ರಿಸಮಶ್ಚ ತ್ರಿತಾತ್ಮಾಚ ತ್ರಿಲೋಕಶ್ಚ ತ್ರಯಂಬಕಃ ॥೬೨॥

ಚತುರ್ದ್ವಂದ್ವಸ್ತ್ರಿಯವನಸ್ತ್ರಿಕಾಮೋ ಹಂಸವಾಹನಃ ।\\
ಚತುಷ್ಕಲಶ್ಚತುರ್ದಂಷ್ಟ್ರೋ ಗತಿಃ ಶಂಭುಃ ಪ್ರಿಯಾನನಃ ॥೬೩॥

ಚತುರ್ಮತಿರ್ಮಹಾದಂಷ್ಟ್ರೋ ವೇದಾಂಗೀ ಚತುರಾನನಃ ।\\
ಪಂಚಶುದ್ಧೋ ಮಹಾಯೋಗೀ ಮಹಾದ್ವಾದಶವಾನಕಃ ॥೬೪॥

ಚತುರ್ಮುಖೋ ನರತನುರಜೇಯಶ್ಚಾಷ್ಟವಂಶವಾನ್ ।\\
ಚತುರ್ದಶಸಮದ್ವಂದ್ವೋ ಮುಕುರಾಂಕೋ ದಶಾಂಶವಾನ್ ॥೬೫॥

ವೃಷಾಂಕೋ ವೃಷಭಾರೂಢಶ್ಚಂದ್ರತೇಜಾಃ ಸುದರ್ಶನಃ ।\\
ಸಾಮಪ್ರಿಯೋ ಮಹೇಶಾನಶ್ಚಿದಾಕಾರೋಃ ನರೋತ್ತಮಃ ॥೬೬॥

ದಯಾವಾನ್ ಕರುಣಾಪೂರ್ಣೋ ಮಹೇಂದ್ರೋ ಮಾಹುರೇಶ್ವರಃ ।\\
ವೀರಾಸನಸಮಾಸೀನೋ ರಾಮೋ ರಾಮಪರಾಯಣಃ ॥೬೭॥

ಇಂದ್ರೋ ವಹ್ನಿರ್ಯಮಃ ಕಾಲೋ ನಿರೃತಿರ್ವರುಣೋ ಯಮಃ ।\\
ವಾಯುಶ್ಚ ರುದ್ರಶ್ಚೇಶಾನೋ ಲೋಕಪಾಲೋ ಮಹಾಯಶಾಃ ॥೬೮॥

ಯಕ್ಷಗಂಧರ್ವನಾಗಶ್ಚ ಕಿನ್ನರಃ ಶುದ್ಧರೂಪಕಃ ।\\
ವಿದ್ಯಾಧರಶ್ಚಾಹಿಪತಿಶ್ಚಾರಣಃ ಪನ್ನಗೇಶ್ವರಃ ॥೬೯॥

ಚಂಡಿಕೇಶಃ ಪ್ರಚಂಡಶ್ಚ ಘಂಟಾನಾದರತಃ ಪ್ರಿಯಃ ।\\
ವೀಣಾಧ್ವನಿರ್ವೈನತೇಯೋ ನಾರದಸ್ತುಂಬರುರ್ಹರಃ ॥೭೦॥

ವೀಣಾಪ್ರಚಂಡಸೌಂದರ್ಯೋ ರಾಜೀವಾಕ್ಷಶ್ಚ ಮನ್ಮಥಃ ।\\
ಚಂದ್ರೋ ದಿವಾಕರೋ ಗೋಪಃ ಕೇಸರೀ ಸೋಮಸೋದರಃ ॥೭೧॥

ಸನಕಃ ಶುಕಯೋಗೀ ಚ ನಂದೀ ಷಣ್ಮುಖರಾಗಕಃ ।\\
ಗಣೇಶೋ ವಿಘ್ನರಾಜಶ್ಚ ಚಂದ್ರಾಭೋ ವಿಜಯೋ ಜಯಃ ॥೭೨॥

ಅತೀತಕಾಲಚಕ್ರಶ್ಚ ತಾಮಸಃ ಕಾಲದಂಡವಾನ್ ।\\
ವಿಷ್ಣುಚಕ್ರಃ ತ್ರಿಶೂಲೇಂದ್ರೋ ಬ್ರಹ್ಮದಂಡೋ ವಿರುದ್ಧಕಃ ॥೭೩॥

ಬ್ರಹ್ಮಾಸ್ತ್ರರೂಪಃ ಸತ್ಯೇಂದ್ರಃ ಕೀರ್ತಿಮಾನ್ಗೋಪತಿರ್ಭವಃ ।\\
ವಸಿಷ್ಠೋ ವಾಮದೇವಶ್ಚ ಜಾಬಾಲೀ ಕಣ್ವರೂಪಕಃ ॥೭೪॥

ಸಂವರ್ತರೂಪೋ ಮೌದ್ಗಲ್ಯೋ ಮಾರ್ಕಂಡೇಯಶ್ಚ ಕಶ್ಯಪಃ ।\\
ತ್ರಿಜಟೋ ಗಾರ್ಗ್ಯರೂಪೀ ಚ ವಿಷನಾಥೋ ಮಹೋದಯಃ ॥೭೫॥

ತ್ವಷ್ಟಾ ನಿಶಾಕರಃ ಕರ್ಮಕಾಶ್ಯಪಶ್ಚ ತ್ರಿರೂಪವಾನ್ ।\\
ಜಮದಗ್ನಿಃ ಸರ್ವರೂಪಃ ಸರ್ವನಾದೋ ಯತೀಶ್ವರಃ ॥೭೬॥

ಅಶ್ವರೂಪೀ ವೈದ್ಯಪತಿರ್ಗರಕಂಠೋಽಮ್ಬಿಕಾರ್ಚಿತಃ ।\\
ಚಿಂತಾಮಣಿಃ ಕಲ್ಪವೃಕ್ಷೋ ರತ್ನಾದ್ರಿರುದಧಿಪ್ರಿಯಃ ॥೭೭॥

ಮಹಾಮಂಡೂಕರೂಪೀ ಚ ಕಾಲಾಗ್ನಿಸಮವಿಗ್ರಹಃ ।\\
ಆಧಾರಶಕ್ತಿರೂಪೀ ಚ ಕೂರ್ಮಃ ಪಂಚಾಗ್ನಿರೂಪಕಃ ॥೭೮॥

ಕ್ಷೀರಾರ್ಣವೋ ಮಹಾರೂಪೀ ವರಾಹಶ್ಚ ಧೃತಾವನಿಃ ।\\
ಐರಾವತೋ ಜನಃ ಪದ್ಮೋ ವಾಮನಃ ಕುಮುದಾತ್ಮವಾನ್ ॥೭೯॥

ಪುಂಡರೀಕಃ ಪುಷ್ಪದಂತೋ ಮೇಘಚ್ಛನ್ನೋಽಭ್ರಚಾರಕಃ ।\\
ಸಿತೋತ್ಪಲಾಭೋ ದ್ಯುತಿಮಾನ್ ದೃಢೋರಸ್ಕಃ ಸುರಾರ್ಚಿತಃ ॥೮೦॥

ಪದ್ಮನಾಭಃ ಸುನಾಭಶ್ಚ ದಶಶೀರ್ಷಃ ಶತೋದರಃ ।\\
ಅವಾಙ್ಮುಖೋ ಪಂಚವಕ್ತ್ರೋ ರಕ್ಷಾಖ್ಯಾತ್ಮಾ ದ್ವಿರೂಪಕಃ ॥೮೧॥

ಸ್ವರ್ಣಮಂಡಲಸಂಚಾರೀ ವೇದಿಸ್ಥಃ ಸರ್ವಪೂಜಿತಃ ।\\
ಸ್ವಪ್ರಸನ್ನಃ ಪ್ರಸನ್ನಾತ್ಮಾ ಸ್ವಭಕ್ತಾಭಿಮುಖೋ ಮೃದುಃ ॥೮೨॥

ಆವಾಹಿತಃ ಸನ್ನಿಹಿತೋ ವರದೋ ಜ್ಞಾನಿವತ್ಸ್ಥಿತಃ ।\\
ಶಾಲಿಗ್ರಾಮಾತ್ಮಕೋ ಧ್ಯಾತೋ ರತ್ನಸಿಂಹಾಸನಸ್ಥಿತಃ ॥೮೩॥

ಅರ್ಘ್ಯಪ್ರಿಯಃ ಪಾದ್ಯತುಷ್ಟಶ್ಚಾಚಮ್ಯಾರ್ಚಿತಪಾದುಕಃ ।\\
ಪಂಚಾಮೃತಃ ಸ್ನಾನವಿಧಿಃ ಶುದ್ಧೋದಕಸುಸಂಚಿತಃ ॥೮೪॥

ಗಂಧಾಕ್ಷತಸುಸಂಪ್ರೀತಃ ಪುಷ್ಪಾಲಂಕಾರಭೂಷಣಃ ।\\
ಅಂಗಪೂಜಾಪ್ರಿಯಃ ಸರ್ವೋ ಮಹಾಕೀರ್ತಿರ್ಮಹಾಭುಜಃ ॥೮೫॥

ನಾಮಪೂಜಾವಿಶೇಷಜ್ಞಃ ಸರ್ವನಾಮಸ್ವರೂಪಕಃ ।\\
ಧೂಪಿತೋ ದಿವ್ಯಧೂಪಾತ್ಮಾ ದೀಪಿತೋ ಬಹುದೀಪವಾನ್ ॥೮೬॥

ಬಹುನೈವೇದ್ಯಸಂಹೃಷ್ಟೋ ನಿರಾಜನವಿರಾಜಿತಃ ।\\
ಸರ್ವಾತಿರಂಜಿತಾನಂದಃ ಸೌಖ್ಯವಾನ್ ಧವಲಾರ್ಜುನಃ ॥೮೭॥

ವಿರಾಗೋ ನಿರ್ವಿರಾಗಶ್ಚ ಯಜ್ಞಾರ್ಚಾಂಗೋ ವಿಭೂತಿಕಃ ।\\
ಉನ್ಮತ್ತೋ ಭ್ರಾಂತಚಿತ್ತಶ್ಚ ಶುಭಚಿತ್ತಃ ಶುಭಾಹುತಿಃ ॥೮೮॥

ಸುರೈರಿಷ್ಟೋ ಲಘಿಷ್ಠಶ್ಚ ಬಂಹಿಷ್ಠೋ ಬಹುದಾಯಕಃ ।\\
ಮಹಿಷ್ಠಃ ಸುಮಹೌಜಾಶ್ಚ ಬಲಿಷ್ಠಃ ಸುಪ್ರತಿಷ್ಠಿತಃ ॥೮೯॥

ಕಾಶೀಗಂಗಾಂಬುಮಜ್ಜಶ್ಚ ಕುಲಶ್ರೀಮಂತ್ರಜಾಪಕಃ ।\\
ಚಿಕುರಾನ್ವಿತಭಾಲಶ್ಚ ಸರ್ವಾಂಗಾಲಿಪ್ತಭೂತಿಕಃ ॥೯೦॥

ಅನಾದಿನಿಧನೋ ಜ್ಯೋತಿರ್ಭಾರ್ಗವಾದ್ಯಃ ಸನಾತನಃ ।\\
ತಾಪತ್ರಯೋಪಶಮನೋ ಮಾನವಾಸೋ ಮಹೋದಯಃ ॥೯೧॥

ಜ್ಯೇಷ್ಠಃ ಶ್ರೇಷ್ಠೋ ಮಹಾರೌದ್ರಃ ಕಾಲಮೂರ್ತಿಃ ಸುನಿಶ್ಚಯಃ ।\\
ಊರ್ಧ್ವಃ ಸಮೂರ್ಧ್ವಲಿಂಗಶ್ಚ ಹಿರಣ್ಯೋ ಹೇಮಲಿಂಗವಾನ್ ॥೯೨॥

ಸುವರ್ಣಃ ಸ್ವರ್ಣಲಿಂಗಶ್ಚ ದಿವ್ಯಸೂತಿರ್ದಿವಸ್ಪತಿಃ ।\\
ದಿವ್ಯಲಿಂಗೋ ಭವೋ ಭವ್ಯಃ ಸರ್ವಲಿಂಗಸ್ತು ಸರ್ವಕಃ ॥೯೩॥

ಶಿವಲಿಂಗಃ ಶಿವೋ ಮಾಯೋ ಜ್ವಲಸ್ತೂಜ್ಜ್ವಲಲಿಂಗವಾನ್ ।\\
ಆತ್ಮಾ ಚೈವಾತ್ಮಲಿಂಗಶ್ಚ ಪರಮೋ ಲಿಂಗಪಾರಗಃ ॥೯೪॥

ಸೋಮಃ ಸೂರ್ಯಃ ಸರ್ವಲಿಂಗಃ ಪಾಣಿಯಂತ್ರಪವಿತ್ರವಾನ್ ।\\
ಸದ್ಯೋಜಾತೋ ತಪೋರೂಪೋ ಭವೋದ್ಭವ ಅನೀಶ್ವರಃ ॥೯೫॥

ತತ್ಸವಿದ್ರೂಪಸವಿತಾ ವರೇಣ್ಯಶ್ಚ ಪ್ರಚೋದಯಾತ್ ।\\
ದೂರದೃಷ್ಟಿರ್ದೂರಗತೋ ದೂರಶ್ರವಣತರ್ಪಿತಃ ॥೯೬॥

ಯೋಗಪೀಠಸ್ಥಿತೋ ವಿದ್ವಾನ್ ನಮಸ್ಕಾರಿತರಾಸಭಃ ।\\
ನಮಸ್ಕೃತಶುನಶ್ಚಾಪಿ ವಜ್ರಕಷ್ಟ್ಯಾತಿಭೀಷಣಃ ॥೯೭॥

ಜ್ವಲನ್ಮುಖಃ ಪ್ರತಿವೀಣಾ ಸಖಡ್ಗೋ ದ್ರಾವಿತಪ್ರಜಃ ।\\
ಪಶುಘ್ನಶ್ಚ ರಸೋನ್ಮತ್ತೋ ರಸೋರ್ಧ್ವಮುಖರಂಜಿತಃ ॥೯೮॥

ರಸಪ್ರಿಯೋ ರಸಾತ್ಮಾ ಚ ರಸರೂಪೀ ರಸೇಶ್ವರಃ ।\\
ರಸಾಧಿದೈವತೋ ಭೌಮೋ ರಸಾಂಗೋ ರಸಭಾವನಃ ॥೯೯॥

ರಸೋನ್ಮಯೋ ರಸಕರೋ ರಸೇಂದ್ರೋ ರಸಪೂಜಕಃ ।\\
ರಸಸಿದ್ಧಃ ಸಿದ್ಧರಸೋ ರಸದ್ರವ್ಯೋ ರಸೋನ್ಮುಖಃ ॥೧೦೦॥

ರಸಾಂಕಿತೋ ರಸಾಪೂರ್ಣೋ ರಸದೋ ರಸಿಕೋ ರಸೀ ।\\
ಗಂಧಕಾದಸ್ತಾಲಕಾದೋ ಗೌರಃಸ್ಫಟಿಕಸೇವನಃ ॥೧೦೧॥

ಕಾರ್ಯಸಿದ್ಧಃ ಕಾರ್ಯರುಚಿರ್ಬಹುಕಾರ್ಯೋ ನ ಕಾರ್ಯವಾನ್ ।\\
ಅಭೇದೀ ಜನಕರ್ತಾ ಚ ಶಂಖಚಕ್ರಗದಾಧರಃ ॥೧೦೨॥

ಕೃಷ್ಣಾಜಿನಕಿರೀಟೀ ಚ ಶ್ರೀಕೃಷ್ಣಾಜಿನಕಂಚುಕಃ ।\\
ಮೃಗಯಾಯೀ ಮೃಗೇಂದ್ರಶ್ಚ ಗಜರೂಪೀ ಗಜೇಶ್ವರಃ ॥೧೦೩॥

ದೃಢವ್ರತಃ ಸತ್ಯವಾದೀ ಕೃತಜ್ಞೋ ಬಲವಾನ್ಬಲಃ ।\\
ಗುಣವಾನ್ ಕಾರ್ಯವಾನ್ ದಾಂತಃ ಕೃತಶೋಭೋ ದುರಾಸದಃ ॥೧೦೪॥

ಸುಕಾಲೋ ಭೂತನಿಹಿತಃ ಸಮರ್ಥಶ್ಚಾಂಡನಾಯಕಃ ।\\
ಸಂಪೂರ್ಣದೃಷ್ಟಿರಕ್ಷುಬ್ಧೋ ಜನೈಕಪ್ರಿಯದರ್ಶನಃ ॥೧೦೫॥

ನಿಯತಾತ್ಮಾ ಪದ್ಮಧರೋ ಬ್ರಹ್ಮವಾಂಶ್ಚಾನಸೂಯಕಃ ।\\
ಉಂಚ್ಛವೃತ್ತಿರನೀಶಶ್ಚ ರಾಜಭೋಗೀ ಸುಮಾಲಿಕಃ ॥೧೦೬॥

ಸುಕುಮಾರೋ ಜರಾಹೀನೇ ಚೋರಘ್ನೋ ಮಂಜುಲಕ್ಷಣಃ ।\\
ಸುಪದಃ ಸ್ವಂಗುಲೀಕಶ್ಚ ಸುಜಂಘಃ ಶುಭಜಾನುಕಃ ॥೧೦೭॥

ಶುಭೋರುಃ ಶುಭಲಿಂಗಶ್ಚ ಸುನಾಭೋ ಜಘನೋತ್ತಮಃ ।\\
ಸುಪಾರ್ಶ್ವಃ ಸುಸ್ತನೋ ನೀಲಃ ಸುವಕ್ಷಾಶ್ಚ ಸುಜತ್ರುಕಃ ॥೧೦೮॥

ನೀಲಗ್ರೀವೋ ಮಹಾಸ್ಕಂಧಃ ಸುಭುಜೋ ದಿವ್ಯಜಂಘಕಃ ।\\
ಸುಹಸ್ತರೇಖೋ ಲಕ್ಷ್ಮೀವಾನ್ ದೀರ್ಘಪೃಷ್ಠೋ ಯತಿಶ್ಚಲಃ ॥೧೦೯॥

ಬಿಂಬೋಷ್ಠಃ ಶುಭದಂತಶ್ಚ ವಿದ್ಯುಜ್ಜಿಹ್ವಃ ಸುತಾಲುಕಃ ।\\
ದೀರ್ಘನಾಸಃ ಸುತಾಮ್ರಾಕ್ಷಃ ಸುಕಪೋಲಃ ಸುಕರ್ಣಕಃ ॥೧೧೦॥

ನಿಮೀಲಿತೋನ್ಮೀಲಿತಶ್ಚ ವಿಶಾಲಾಕ್ಷಶ್ಚ ಶುಭ್ರಕಃ ।\\
ಶುಭಮಧ್ಯಃ ಸುಭಾಲಶ್ಚ ಸುಶಿರಾ ನೀಲರೋಮಕಃ ॥೧೧೧॥

ವಿಶಿಷ್ಟಗ್ರಾಮಣಿಸ್ಕಂಧಃ ಶಿಖಿವರ್ಣೋ ವಿಭಾವಸುಃ ।\\
ಕೈಲಾಸೇಶೋ ವಿಚಿತ್ರಜ್ಞೋ ವೈಕುಂಠೇಂದ್ರೋ ವಿಚಿತ್ರವಾನ್ ॥೧೧೨॥

ಮನಸೇಂದ್ರಶ್ಚಕ್ರವಾಲೋ ಮಹೇಂದ್ರೋ ಮಂದಾರಧಿಪಃ ।\\
ಮಲಯೋ ವಿಂಧ್ಯರೂಪಶ್ಚ ಹಿಮವಾನ್ ಮೇರುರೂಪಕಃ ॥೧೧೩॥

ಸುವೇಷೋ ನವ್ಯರೂಪಾತ್ಮಾ ಮೈನಾಕೋ ಗಂಧಮಾದನಃ ।\\
ಸಿಂಹಲಶ್ಚೈವ ವೇದಾದ್ರಿಃ ಶ್ರೀಶೈಲಃ ಕ್ರಕಚಾತ್ಮಕಃ ॥೧೧೪॥

ನಾನಾಚಲಶ್ಚಿತ್ರಕೂಟೋ ದುರ್ವಾಸಾಃ ಪರ್ವತಾತ್ಮಜಃ ।\\
ಯಮುನಾಕೃಷ್ಣವೇಣೀಶೋ ಭದ್ರೇಶೋ ಗೌತಮೀಪತಿಃ ॥೧೧೫॥

ಗೋದಾವರೀಶೋ ಗಂಗಾತ್ಮಾ ಶೋಣಕಃ ಕೌಶಿಕೀಪತಿಃ ।\\
ನರ್ಮದೇಶಸ್ತು ಕಾವೇರೀತಾಮ್ರಪರ್ಣೀಶ್ವರೋ ಜಟೀ ॥೧೧೬॥

ಸರಿದ್ರೂಪಾ ನದಾತ್ಮಾ ಚ ಸಮುದ್ರಃ ಸರಿದೀಶ್ವರಃ ।\\
ಹ್ರಾದಿನೀಶಃ ಪಾವನೀಶೋ ನಲಿನೀಶಃ ಸುಚಕ್ಷುಮಾನ್ ॥೧೧೭॥

ಸೀತಾನದೀಪತಿಃ ಸಿಂಧೂ ರೇವೇಶೋ ಮುರಲೀಪತಿಃ ।\\
ಲವಣೇಕ್ಷುಃ ಕ್ಷೀರನಿಧಿಃ ಸುರಾಬ್ಧಿಃ ಸರ್ಪಿರಂಬುಧಿಃ ॥೧೧೮॥

ದಯಾಬ್ಧಿಶುದ್ಧಜಲಧಿಸ್ತತ್ವರೂಪೋ ಧನಾಧಿಪಃ ।\\
ಭೂಪಾಲಮಧುರಾಗಜ್ಞೋ ಮಾಲತೀರಾಗಕೋವಿದಃ ॥೧೧೯॥

ಪೌಂಡ್ರಕ್ರಿಯಾಜ್ಞಃ ಶ್ರೀರಾಗೋ ನಾನಾರಾಗಾರ್ಣವಾಂತಕಃ ।\\
ವೇದಾದಿರೂಪೋ ಹ್ರೀರೂಪೋ ಕ್ಲಂರೂಪಃ ಕ್ಲೀಂವಿಕಾರಕಃ ॥೧೨೦॥

ವ್ರುಮ್ಮಯಃ ಕ್ಲೀಮ್ಮಯಃ ಪ್ರಖ್ಯೋ ಹುಮ್ಮಯಃ ಕ್ರೋಮ್ಮಯೋ ಭಟಃ ।\\
ಧ್ರೀಮ್ಮಯೋ ಲುಂಗ್ಮಯೋ ಗಾಂಗೋ ಘಮ್ಮಯಃ ಖಮ್ಮಯಃ ಖಗಃ ॥೧೨೧॥

ಖಮ್ಮಯೋ ಜ್ಞಮ್ಮಯಶ್ಚಾಂಗೋ ಬೀಜಾಂಗೋ ಬೀಜಜಮ್ಮಯಃ ।\\
ಝಂಕರಷ್ಟಂಕರಃ ಷ್ಟಂಗೋ ಡಂಕರೀ ಠಂಕರೋಽಣುಕಃ ॥೧೨೨॥

ತಂಕರಸ್ಥಂಕರಸ್ತುಂಗೋ ದ್ರಾಮ್ಮುದ್ರಾರೂಪಕಃ ಸುದಃ ।\\
ದಕ್ಷೋ ದಂಡೀ ದಾನವಘ್ನೋ ಅಪ್ರತಿದ್ವಂದ್ವವಾಮದಃ ॥೧೨೩॥

ಧಂರೂಪೋ ನಂಸ್ವರೂಪಶ್ಚ ಪಂಕಜಾಕ್ಷಶ್ಚ ಫಮ್ಮಯಃ ।\\
ಮಹೇಂದ್ರೋ ಮಧುಭೋಕ್ತಾ ಚ ಮಂದರೇತಾಸ್ತು ಭಮ್ಮಯಃ ॥೧೨೪॥

ರಮ್ಮಯೋ ರಿಂಕರೋ ರಂಗೋ ಲಂಕರಃ ವಮ್ಮಯಃ ಶರಃ ।\\
ಶಂಕರಃಷಣ್ಮುಖೋ ಹಂಸಃ ಶಂಕರಃ ಶಂಕರೋಽಕ್ಷಯಃ ॥೧೨೫॥

ಓಮಿತ್ಯೇಕಾಕ್ಷರಾತ್ಮಾ ಚ ಸರ್ವಬೀಜಸ್ವರೂಪಕಃ ।\\
ಶ್ರೀಕರಃ ಶ್ರೀಪದಃ ಶ್ರೀಶಃ ಶ್ರೀನಿಧಿಃ ಶ್ರೀನಿಕೇತನಃ ॥೧೨೬॥

ಪುರುಷೋತ್ತಮಃ ಸುಖೀ ಯೋಗೀ ದತ್ತಾತ್ರೇಯೋ ಹೃದಿಪ್ರಿಯಃ ।\\
ತತ್ಸಂಯುತಃ ಸದಾಯೋಗೀ ಧೀರತಂತ್ರಸುಸಾಧಕಃ ॥೧೨೭॥

ಪುರುಷೋತ್ತಮೋ ಯತಿಶ್ರೇಷ್ಠೋ ದತ್ತಾತ್ರೇಯಃ ಸಖೀತ್ವವಾನ್ ।\\
ವಸಿಷ್ಠವಾಮದೇವಾಭ್ಯಾಂ ದತ್ತಃ ಪುರುಷಃ ಈರಿತಃ ॥೧೨೮॥

ಯಾವತ್ತಿಷ್ಠತೇ ಹ್ಯಸ್ಮಿನ್ ತಾವತ್ತಿಷ್ಠತಿ ತತ್ಸುಖೀ ।\\
ಯ ಇದಂ ಶೃಣುಯಾನ್ನಿತ್ಯಂ ಬ್ರಹ್ಮಸಾಯುಜ್ಯತಾಂ ವ್ರಜೇತ್ ॥೧೨೯॥

ಭುಕ್ತಿಮುಕ್ತಿಕರಂ ತಸ್ಯ ನಾತ್ರಕಾರ್ಯಾ ವಿಚಾರಣಾ ।\\
ಆಯುಷ್ಮತ್ಪುತ್ರಪೌತ್ರಾಂಶ್ಚ ದತ್ತಾತ್ರೇಯಃ ಪ್ರದರ್ಶಯೇತ್ ॥೧೩೦॥

ಧನ್ಯಂ ಯಶಸ್ಯಮಾಯುಷ್ಯಂ ಪುತ್ರಭಾಗ್ಯವಿವರ್ಧನಂ ।\\
ಕರೋತಿ ಲೇಖನಾದೇವ ಪರಾರ್ಥಂ ವಾ ನ ಸಂಶಯಃ ॥೧೩೧॥

ಯಃ ಕರೋತ್ಯುಪದೇಶಂ ಚ ನಾಮದತ್ತಸಹಸ್ರಕಂ ।\\
ಸ ಚ ಯಾತಿ ಚ ಸಾಯುಜ್ಯಂ ಶ್ರೀಮಾನ್ ಶ್ರೀಮಾನ್ ನ ಸಂಶಯಃ ॥೧೩೨॥

ಪಠನಾಚ್ಛ್ರವಣಾದ್ವಾಪಿ ಸರ್ವಾನ್ಕಾಮಾನವಾಪ್ನುಯಾತ್ ।\\
ಖೇಚರತ್ವಂ ಕಾರ್ಯಸಿದ್ಧಿಂ ಯೋಗಸಿದ್ಧಿಮವಾಪ್ನುಯಾತ್ ॥೧೩೬॥

ಬ್ರಹ್ಮರಾಕ್ಷಸವೇತಾಲೈಃ ಪಿಶಾಚೈಃ ಕಾಮಿನೀಮುಖೈಃ ।\\
ಪೀಡಾಕರೈಃ ಸುಖಕರೈರ್ಗ್ರಹೈರ್ದುಷ್ಟೈರ್ನ ಬಾಧ್ಯತೇ ॥೧೩೪॥

ದೇವೈಃ ಪಿಶಾಚೈರ್ಮುಚ್ಯೇತ ಸಕೃದುಚ್ಚಾರಣೇನ ತು ।\\
ಯಸ್ಮಿಂದೇಶೇ ಸ್ಥಿತಂ ಚೈತತ್ಪುಸ್ತಕಂ ದತ್ತನಾಮಕಂ ॥೧೩೫॥

ಪಂಚಯೋಜನವಿಸ್ತಾರಂ ರಕ್ಷಣಂ ನಾತ್ರ ಸಂಶಯಃ ।\\
ಸರ್ವಬೀಜಸಮಾಯುಕ್ತಂ ಸ್ತೋತ್ರಂ ನಾಮಸಹಸ್ರಕಂ ॥೧೩೬॥

ಸರ್ವಮಂತ್ರಸ್ವರೂಪಂ ಚ ದತ್ತಾತ್ರೇಯಸ್ವರೂಪಕಂ ।\\
ಏಕವಾರಂ ಪಠಿತ್ವಾ ತು ತಾಮ್ರಪಾತ್ರೇ ಜಲಂ ಸ್ಪೃಶೇತ್ ॥೧೩೭॥

ಪೀತ್ವಾ ಚೇತ್ಸರ್ವರೋಗೈಶ್ಚ ಮುಚ್ಯತೇ ನಾತ್ರ ಸಂಶಯಃ ।\\
ಸ್ತ್ರೀವಶ್ಯಂ ಪುರುಷವಶ್ಯಂ ರಾಜವಶ್ಯಂ ಜಯಾವಹಂ ॥೧೩೮॥

ಸಂಪತ್ಪ್ರದಂ ಮೋಕ್ಷಕರಂ ಪಠೇನ್ನಿತ್ಯಮತಂದ್ರಿತಃ ।\\
ಲೀಯತೇಽಸ್ಮಿನ್ಪ್ರಪಂಚಾರ್ಥಾನ್ ವೈರಿಶೋಕಾದಿಕಾರಿತಃ ॥೧೩೯॥

ಪಠನಾತ್ತು ಪ್ರಸನ್ನೋಽಹಂ ಶಂಕರಾಚಾರ್ಯ ಬುದ್ಧಿಮಾನ್ ।\\
ಭವಿಷ್ಯಸಿ ನ ಸಂದೇಹಃ ಪಠಿತಃ ಪ್ರಾತರೇವ ಮಾಂ ॥೧೪೦॥

ಉಪದೇಕ್ಷ್ಯೇ ಸರ್ವಯೋಗಾನ್ ಲಂಬಿಕಾದಿಬಹೂನ್ವರಾನ್ ।\\
ದತ್ತಾತ್ರೇಯಸ್ತು ಚೇತ್ಯುಕ್ತ್ವಾ ಸ್ವಪ್ನೇ ಚಾಂತರಧೀಯತ ॥೧೪೧॥

ಸ್ವಪ್ನಾದುತ್ಥಾಯ ಚಾಚಾರ್ಯಃ ಶಂಕರೋ ವಿಸ್ಮಯಂ ಗತಃ ।\\
ಸ್ವಪ್ನೋಪದೇಶಿತಂ ಸ್ತೋತ್ರಂ ದತ್ತಾತ್ರೇಯೇನ ಯೋಗಿನಾ ॥೧೪೨॥

ಸಹಸ್ರನಾಮಕಂ ದಿವ್ಯಂ ಪಠಿತ್ವಾ ಯೋಗವಾನ್ಭವೇತ್ ।\\
ಜ್ಞಾನಯೋಗಯತಿತ್ವಂ ಚ ಪರಾಕಾಯಪ್ರವೇಶನಂ ॥೧೪೩॥

ಬಹುವಿದ್ಯಾಖೇಚರತ್ವಂ ದೀರ್ಘಾಯುಸ್ತತ್ಪ್ರಸಾದತಃ ।\\
ತದಾರಭ್ಯ ಭುವಿ ಶ್ರೇಷ್ಠಃ ಪ್ರಸಿದ್ಧಶ್ಚಾಭವದ್ಯತೀ ॥೧೪೪॥

\authorline{ಇತಿ ಶ್ರೀಶಂಕರಾಚಾರ್ಯಸ್ವಪ್ನಾವಸ್ಥಾಯಾಂ ದತ್ತಾತ್ರೇಯೋಪದೇಶಿತಂ ಸಕಲಪುರಾಣವೇದೋಕ್ತಪ್ರಪಂಚಾರ್ಥಸಾರವತ್ಸ್ತೋತ್ರಂ ಸಂಪೂರ್ಣಂ ॥೧೪೫॥}
%====================================================================================================
\section{ಶ್ರೀದತ್ತಾತ್ರೇಯಾಷ್ಟೋತ್ತರಶತನಾಮಸ್ತೋತ್ರಂ}
\addcontentsline{toc}{section}{ಶ್ರೀದತ್ತಾತ್ರೇಯಾಷ್ಟೋತ್ತರಶತನಾಮಸ್ತೋತ್ರಂ}
\addcontentsline{toc}{section}{ಶ್ರೀದತ್ತಾತ್ರೇಯಾಷ್ಟೋತ್ತರಶತನಾಮಸ್ತೋತ್ರಂ}
ಅಸ್ಯ ದತ್ತಾತ್ರೇಯಾಷ್ಟೋತ್ತರ ಶತನಾಮಸ್ತೋತ್ರ ಮಹಾಮಂತ್ರಸ್ಯ, ಬ್ರಹ್ಮವಿಷ್ಣುಮಹೇಶ್ವರಾ ಋಷಯಃ~। ಶ್ರೀದತ್ತಾತ್ರೇಯೋ ದೇವತಾ~। ಅನುಷ್ಟುಪ್ಛಂದಃ~। ಶ್ರೀದತ್ತಾತ್ರೇಯಪ್ರೀತ್ಯರ್ಥೇ ಜಪೇ ವಿನಿಯೋಗಃ~।\\
\dhyana{ದಿಗಂಬರಂ ಭಸ್ಮವಿಲೇಪಿತಾಂಗಂ ಚಕ್ರಂ ತ್ರಿಶೂಲಂ ಡಮರುಂ ಗದಾಂ ಚ~।\\
ಪದ್ಮಾನನಂಯೋಗಿಮುನೀಂದ್ರವಂದ್ಯಂ~ಧ್ಯಾಯಾಮಿತಂದತ್ತಮಭೀಷ್ಟಸಿದ್ಧ್ಯೈ॥}

ಓಂ ಅನಸೂಯಾಸುತೋ ದತ್ತೋ ಹ್ಯತ್ರಿಪುತ್ರೋ ಮಹಾಮುನಿಃ~।\\
ಯೋಗೀಂದ್ರಃ ಪುಣ್ಯಪುರುಷೋ ದೇವೇಶೋ ಜಗದೀಶ್ವರಃ ॥೧॥

ಪರಮಾತ್ಮಾ ಪರಂ ಬ್ರಹ್ಮ ಸದಾನಂದೋ ಜಗದ್ಗುರುಃ~।\\
ನಿತ್ಯತೃಪ್ತೋ ನಿರ್ವಿಕಾರೋ ನಿರ್ವಿಕಲ್ಪೋ ನಿರಂಜನಃ ॥೨॥

ಗುಣಾತ್ಮಕೋ ಗುಣಾತೀತೋ ಬ್ರಹ್ಮವಿಷ್ಣುಶಿವಾತ್ಮಕಃ~।\\
ನಾನಾರೂಪಧರೋ ನಿತ್ಯಃ ಶಾಂತೋ ದಾಂತಃ ಕೃಪಾನಿಧಿಃ ॥೩॥

ಭಕ್ತಪ್ರಿಯೋ ಭವಹರೋ ಭಗವಾನ್ಭವನಾಶನಃ~।\\
ಆದಿದೇವೋ ಮಹಾದೇವಃ ಸರ್ವೇಶೋ ಭುವನೇಶ್ವರಃ ॥೪॥

ವೇದಾಂತವೇದ್ಯೋ ವರದೋ ವಿಶ್ವರೂಪೋಽವ್ಯಯೋ ಹರಿಃ~।\\
ಸಚ್ಚಿದಾನಂದಃ ಸರ್ವೇಶೋ ಯೋಗೀಶೋ ಭಕ್ತವತ್ಸಲಃ ॥೫॥

ದಿಗಂಬರೋ ದಿವ್ಯಮೂರ್ತಿರ್ದಿವ್ಯಭೂತಿವಿಭೂಷಣಃ~।\\
ಅನಾದಿಃ ಸಿದ್ಧಸುಲಭೋ ಭಕ್ತವಾಂಛಿತದಾಯಕಃ ॥೬॥

ಏಕೋಽನೇಕೋ ಹ್ಯದ್ವಿತೀಯೋ ನಿಗಮಾಗಮಪಂಡಿತಃ~।\\
ಭುಕ್ತಿಮುಕ್ತಿಪ್ರದಾತಾ ಚ ಕಾರ್ತವೀರ್ಯವರಪ್ರದಃ ॥೭॥

ಶಾಶ್ವತಾಂಗೋ ವಿಶುದ್ಧಾತ್ಮಾ ವಿಶ್ವಾತ್ಮಾ ವಿಶ್ವತೋ ಮುಖಃ~।\\
ಸರ್ವೇಶ್ವರಃ ಸದಾತುಷ್ಟಃ ಸರ್ವಮಂಗಲದಾಯಕಃ ॥೮॥

ನಿಷ್ಕಲಂಕೋ ನಿರಾಭಾಸೋ ನಿರ್ವಿಕಲ್ಪೋ ನಿರಾಶ್ರಯಃ~।\\
ಪುರುಷೋತ್ತಮೋ ಲೋಕನಾಥಃ ಪುರಾಣಪುರುಷೋಽನಘಃ ॥೯॥

ಅಪಾರಮಹಿಮಾಽನಂತೋ ಹ್ಯಾದ್ಯಂತರಹಿತಾಕೃತಿಃ~।\\
ಸಂಸಾರವನದಾವಾಗ್ನಿರ್ಭವಸಾಗರತಾರಕಃ ॥೧೦॥

ಶ್ರೀನಿವಾಸೋ ವಿಶಾಲಾಕ್ಷಃ ಕ್ಷೀರಾಬ್ಧಿಶಯನೋಽಚ್ಯುತಃ~।\\
ಸರ್ವಪಾಪಕ್ಷಯಕರಸ್ತಾಪತ್ರಯನಿವಾರಣಃ ॥೧೧॥

ಲೋಕೇಶಃ ಸರ್ವಭೂತೇಶೋ ವ್ಯಾಪಕಃ ಕರುಣಾಮಯಃ~।\\
ಬ್ರಹ್ಮಾದಿವಂದಿತಪದೋ ಮುನಿವಂದ್ಯಃ ಸ್ತುತಿಪ್ರಿಯಃ ॥೧೨॥

ನಾಮರೂಪಕ್ರಿಯಾತೀತೋ ನಿಃಸ್ಪೃಹೋ ನಿರ್ಮಲಾತ್ಮಕಃ~।\\
ಮಾಯಾಧೀಶೋ ಮಹಾತ್ಮಾ ಚ ಮಹಾದೇವೋ ಮಹೇಶ್ವರಃ ॥೧೩॥

ವ್ಯಾಘ್ರಚರ್ಮಾಂಬರಧರೋ ನಾಗಕುಂಡಲಭೂಷಣಃ~।\\
ಸರ್ವಲಕ್ಷಣಸಂಪೂರ್ಣಃ ಸರ್ವಸಿದ್ಧಿಪ್ರದಾಯಕಃ ॥೧೪॥

ಸರ್ವಜ್ಞಃ ಕರುಣಾಸಿಂಧುಃ ಸರ್ಪಹಾರಃ ಸದಾಶಿವಃ~।\\
ಸಹ್ಯಾದ್ರಿವಾಸಃ ಸರ್ವಾತ್ಮಾ ಭವಬಂಧವಿಮೋಚನಃ ॥೧೫॥

ವಿಶ್ವಂಭರೋ ವಿಶ್ವನಾಥೋ ಜಗನ್ನಾಥೋ ಜಗತ್ಪ್ರಭುಃ~।\\
ನಿತ್ಯಂ ಪಠತಿ ಯೋ ಭಕ್ತ್ಯಾ ಸರ್ವಪಾಪೈಃ ಪ್ರಮುಚ್ಯತೇ ॥೧೬॥

ಸರ್ವದುಃಖಪ್ರಶಮನಂ ಸರ್ವಾರಿಷ್ಟನಿವಾರಣಂ~।\\
ಭೋಗಮೋಕ್ಷಪ್ರದಂ ನೄಣಾಂ ದತ್ತಸಾಯುಜ್ಯದಾಯಕಂ~।\\
ಪಠಂತಿ ಯೇ ಪ್ರಯತ್ನೇನ ಸತ್ಯಂ ಸತ್ಯಂ ವದಾಮ್ಯಹಂ ॥೧೭॥
\authorline{॥ಇತಿ ಬ್ರಹ್ಮಾಂಡಪುರಾಣೇ ಬ್ರಹ್ಮನಾರದಸಂವಾದೇ ಶ್ರೀದತ್ತಾತ್ರೇಯಾಷ್ಟೋತ್ತರ ಶತನಾಮಸ್ತೋತ್ರಂ ॥}
%=====================================================================================================
\section{ಶ್ರೀಮೇಧಾದಕ್ಷಿಣಾಮೂರ್ತಿಸಹಸ್ರನಾಮಸ್ತೋತ್ರಂ}
\addcontentsline{toc}{section}{ಶ್ರೀಮೇಧಾದಕ್ಷಿಣಾಮೂರ್ತಿಸಹಸ್ರನಾಮಸ್ತೋತ್ರಂ}
ಅಸ್ಯ ಶ್ರೀ ಮೇಧಾದಕ್ಷಿಣಾಮೂರ್ತಿಸಹಸ್ರನಾಮಸ್ತೋತ್ರಸ್ಯ ಬ್ರಹ್ಮಾ ಋಷಿಃ~। ಗಾಯತ್ರೀ ಛಂದಃ~। ದಕ್ಷಿಣಾಮೂರ್ತಿರ್ದೇವತಾ~। ಓಂ ಬೀಜಂ~। ಸ್ವಾಹಾ ಶಕ್ತಿಃ~। ನಮಃ ಕೀಲಕಂ~। ದಕ್ಷಿಣಾಮೂರ್ತಿಪ್ರೀತ್ಯರ್ಥೇ ಜಪೇ ವಿನಿಯೋಗಃ~॥

\dhyana{ಸಿದ್ಧಿತೋಯನಿಧೇರ್ಮಧ್ಯೇ ರತ್ನಗ್ರೀವೇ ಮನೋರಮೇ~।\\
ಕದಂಬವನಿಕಾಮಧ್ಯೇ ಶ್ರೀಮದ್ವಟತರೋರಧಃ ॥೧॥

	ಆಸೀನಮಾದ್ಯಂ ಪುರುಷಮಾದಿಮಧ್ಯಾಂತವರ್ಜಿತಂ~।\\
	ಶುದ್ಧಸ್ಫಟಿಕ ಗೋಕ್ಷೀರ ಶರತ್ಪೂರ್ಣೇಂದು ಶೇಖರಂ ॥೨॥

ದಕ್ಷಿಣೇ ಚಾಕ್ಷಮಾಲಾಂ ಚ ವಹ್ನಿಂ ವೈ ವಾಮಹಸ್ತಕೇ~।\\
ಜಟಾಮಂಡಲ ಸಂಲಗ್ನ ಶೀತಾಂಶುಕರ ಮಂಡಿತಂ ॥೩॥

	ನಾಗಹಾರಧರಂ ಚಾರುಕಂಕಣೈಃ ಕಟಿಸೂತ್ರಕೈಃ~।\\
	ವಿರಾಜಮಾನ ವೃಷಭಂ ವ್ಯಾಘ್ರ ಚರ್ಮಾಂಬರಾವೃತಂ ॥೪॥

ಚಿಂತಾಮಣಿ ಮಹಾಬೃಂದೈಃ ಕಲ್ಪಕೈಃ ಕಾಮಧೇನುಭಿಃ~।\\
ಚತುಷ್ಷಷ್ಟಿ ಕಲಾವಿದ್ಯಾ ಮೂರ್ತಿಭಿಃ ಶ್ರುತಿಮಸ್ತಕೈಃ ॥೫॥

	ರತ್ನಸಿಂಹಾಸನೇ ಸಾಧುದ್ವೀಪಿಚರ್ಮ ಸಮಾಯುತಂ~।\\
	ತತ್ರಾಷ್ಟದಲಪದ್ಮಸ್ಯ ಕರ್ಣಿಕಾಯಾಂ ಸುಶೋಭನೇ ॥೬॥

ವೀರಾಸನೇ ಸಮಾಸೀನಂ ಲಂಬದಕ್ಷಪದಾಂಬುಜಂ~।\\
ಜ್ಞಾನಮುದ್ರಾಂ ಪುಸ್ತಕಂ ಚ ವರಾಭೀತಿಧರಂ ಹರಂ ॥೭॥

	ಪಾದಮೂಲ ಸಮಾಕ್ರಾಂತ ಮಹಾಪಸ್ಮಾರ ವೈಭವಂ~।\\
	ರುದ್ರಾಕ್ಷಮಾಲಾಭರಣ ಭೂಷಿತಂ ಭೂತಿಭಾಸುರಂ ॥೮॥

ಗಜಚರ್ಮೋತ್ತರೀಯಂ ಚ ಮಂದಸ್ಮಿತ ಮುಖಾಂಬುಜಂ~।\\
ಸಿದ್ಧವೃಂದೈ ರ್ಯೋಗಿವೃಂದೈ ರ್ಮುನಿವೃಂದೈ ರ್ನಿಷೇವಿತಂ ॥೯॥

	ಆರಾಧ್ಯಮಾನವೃಷಭಂ ಅಗ್ನೀಂದುರವಿಲೋಚನಂ~।\\
	ಪೂರಯಂತಂ ಕೃಪಾದೃಷ್ಟ್ಯಾ  ಪುಮರ್ಥಾನಾಶ್ರಿತೇ ಜನೇ~।\\
ಏವಂ ವಿಭಾವಯೇದೀಶಂ ಸರ್ವವಿದ್ಯಾ ಕಲಾನಿಧಿಂ ॥೧೦॥}

	ದೇವದೇವೋ ಮಹಾದೇವೋ ದೇವಾನಾಮಪಿ ದೇಶಿಕಃ~।\\
	ದಕ್ಷಿಣಾಮೂರ್ತಿರೀಶಾನೋ ದಯಾಪೂರಿತ ದಿಙ್ಮುಖಃ ॥೧॥

ಕೈಲಾಸಶಿಖರೋತ್ತುಂಗ ಕಮನೀಯನಿಜಾಕೃತಿಃ~।\\
ವಟದ್ರುಮತಟೀ ದಿವ್ಯಕನಕಾಸನ ಸಂಸ್ಥಿತಃ ॥೨॥

	ಕಟೀತಟ ಪಟೀಭೂತ ಕರಿಚರ್ಮೋಜ್ಜ್ವಲಾಕೃತಿಃ~।\\
	ಪಾಟೀರಾಪಾಂಡುರಾಕಾರ ಪರಿಪೂರ್ಣಸುಧಾಧಿಪಃ~।೩॥

ಜಟಾಕೋಟೀರಘಟಿತ ಸುಧಾಕರ ಸುಧಾಪ್ಲುತಃ~।\\
ಪಶ್ಯಲ್ಲಲಾಟಸುಭಗ ಸುಂದರ ಭ್ರೂವಿಲಾಸವಾನ್ ॥೪॥

	ಕಟಾಕ್ಷಸರಣೀ ನಿರ್ಯತ್ಕರುಣಾಪೂರ್ಣ ಲೋಚನಃ~।\\
	ಕರ್ಣಾಲೋಲ ತಟಿದ್ವರ್ಣ ಕುಂಡಲೋಜ್ಜ್ವಲ ಗಂಡಭೂಃ ॥೫॥

ತಿಲಪ್ರಸೂನ ಸಂಕಾಶ ನಾಸಿಕಾಪುಟ ಭಾಸುರಃ~।\\
ಮಂದಸ್ಮಿತ ಸ್ಫುರನ್ಮುಗ್ಧ ಮಹನೀಯ ಮುಖಾಂಬುಜಃ ॥೬॥

	ಕುಂದಕುಡ್ಮಲ ಸಂಸ್ಪರ್ಧಿ ದಂತಪಂಕ್ತಿ ವಿರಾಜಿತಃ~।\\
	ಸಿಂದೂರಾರುಣ ಸುಸ್ನಿಗ್ಧ ಕೋಮಲಾಧರ ಪಲ್ಲವಃ ॥೭॥

ಶಂಖಾಟೋಪ ಗಲದ್ದಿವ್ಯ ಗಳವೈಭವಮಂಜುಲಃ~।\\
ಕರಕಂದಲಿತ ಜ್ಞಾನಮುದ್ರಾ ರುದ್ರಾಕ್ಷಮಾಲಿಕಃ ॥೮॥

	ಅನ್ಯಹಸ್ತ ತಲನ್ಯಸ್ತ ವೀಣಾ ಪುಸ್ತೋಲ್ಲಸದ್ವಪುಃ~।\\
	ವಿಶಾಲ ರುಚಿರೋರಸ್ಕ ವಲಿಮತ್ಪಲ್ಲವೋದರಃ ॥೯॥

ಬೃಹತ್ಕಟಿ ನಿತಂಬಾಢ್ಯಃ ಪೀವರೋರು ದ್ವಯಾನ್ವಿತಃ~।\\
ಜಂಘಾವಿಜಿತ ತೂಣೀರಸ್ತುಂಗಗುಲ್ಫ ಯುಗೋಜ್ಜ್ವಲಃ ॥೧೦॥

	ಮೃದು ಪಾಟಲ ಪಾದಾಬ್ಜಶ್ಚಂದ್ರಾಭ ನಖದೀಧಿತಿಃ~।\\
	ಅಪಸವ್ಯೋರು ವಿನ್ಯಸ್ತ ಸವ್ಯಪಾದ ಸರೋರುಹಃ ॥೧೧॥

ಘೋರಾಪಸ್ಮಾರ ನಿಕ್ಷಿಪ್ತ ಧೀರದಕ್ಷ ಪದಾಂಬುಜಃ~।\\
ಸನಕಾದಿ ಮುನಿಧ್ಯೇಯಃ ಸರ್ವಾಭರಣ ಭೂಷಿತಃ ॥೧೨॥

	ದಿವ್ಯಚಂದನ ಲಿಪ್ತಾಂಗಶ್ಚಾರುಹಾಸ ಪರಿಷ್ಕೃತಃ~।\\
	ಕರ್ಪೂರ ಧವಲಾಕಾರಃ ಕಂದರ್ಪಶತ ಸುಂದರಃ ॥೧೩॥

ಕಾತ್ಯಾಯನೀ ಪ್ರೇಮನಿಧಿಃ ಕರುಣಾರಸ ವಾರಿಧಿಃ~।\\
ಕಾಮಿತಾರ್ಥಪ್ರದಃ ಶ್ರೀಮತ್ಕಮಲಾ ವಲ್ಲಭಪ್ರಿಯಃ ॥೧೪॥

	ಕಟಾಕ್ಷಿತಾತ್ಮವಿಜ್ಞಾನಃ ಕೈವಲ್ಯಾನಂದಕಂದಲಃ~।\\
	ಮಂದಹಾಸ ಸಮಾನೇಂದುಃ ಛಿನ್ನಾಜ್ಞಾನ ತಮಸ್ತತಿಃ ॥೧೫॥

ಸಂಸಾರಾನಲ ಸಂತಪ್ತಜನತಾಮೃತ ಸಾಗರಃ~।\\
ಗಂಭೀರ ಹೃದಯಾಂಭೋಜ ನಭೋಮಣಿ ನಿಭಾಕೃತಿಃ ॥೧೬॥

	ನಿಶಾಕರಕರಾಕಾರ ವಶೀಕೃತಜಗತ್ತ್ರಯಃ~।\\
	ತಾಪಸಾರಾಧ್ಯ ಪಾದಾಬ್ಜಸ್ತರುಣಾನಂದ ವಿಗ್ರಹಃ ॥೧೭॥

ಭೂತಿ ಭೂಷಿತ ಸರ್ವಾಂಗೋ ಭೂತಾಧಿಪತಿರೀಶ್ವರಃ~।\\
ವದನೇಂದು ಸ್ಮಿತಜ್ಯೋತ್ಸ್ನಾನಿಲೀನ ತ್ರಿಪುರಾಕೃತಿಃ ॥೧೮॥

	ತಾಪತ್ರಯತ ಮೋಭಾನುಃ ಪಾಪಾರಣ್ಯ ದವಾನಲಃ~।\\
	ಸಂಸಾರ ಸಾಗರೋದ್ಧರ್ತಾ ಹಂಸಾಗ್ರ್ಯೋಪಾಸ್ಯ ವಿಗ್ರಹಃ ॥೧೯॥

ಲಲಾಟ ಹುತಭುಗ್ದಗ್ಧ ಮನೋಭವ ಶುಭಾಕೃತಿಃ~।\\
ತುಚ್ಛೀಕೃತ ಜಗಜ್ಜಾಲ ಸ್ತುಷಾರಕರ ಶೀತಲಃ ॥೨೦॥

	ಅಸ್ತಂಗತ ಸಮಸ್ತೇಚ್ಛೋ ನಿಸ್ತುಲಾನಂದ ಮಂಥರಃ~।\\
	ಧೀರೋದಾತ್ತ ಗುಣಾಧಾರ ಉದಾರ ವರವೈಭವಃ ॥೨೧॥

ಅಪಾರ ಕರುಣಾಮೂರ್ತಿರಜ್ಞಾನ ಧ್ವಾಂತಭಾಸ್ಕರಃ~।\\
ಭಕ್ತಮಾನಸ ಹಂಸಾಗ್ರ್ಯ ಭವಾಮಯ ಭಿಷಕ್ತಮಃ ॥೨೨॥

	ಯೋಗೀಂದ್ರಪೂಜ್ಯಪಾದಾಬ್ಜೋ ಯೋಗಪಟ್ಟೋಲ್ಲಸತ್ಕಟಿಃ~।\\
	ಶುದ್ಧಸ್ಫಟಿಕಸಂಕಾಶೋ ಬದ್ಧಪನ್ನಗಭೂಷಣಃ ॥೨೩॥

ನಾನಾಮುನಿ ಸಮಾಕೀರ್ಣೋ ನಾಸಾಗ್ರನ್ಯಸ್ತಲೋಚನಃ~।\\
ವೇದಮೂರ್ಧೈಕಸಂವೇದ್ಯೋ ನಾದಧ್ಯಾನಪರಾಯಣಃ ॥೨೪॥

	ಧರಾಧರೇಂದು ರಾನಂದಸಂದೋಹ ರಸಸಾಗರಃ~।\\
	ದ್ವೈತವೃಂದವಿಮೋಹಾಂಧ್ಯ ಪರಾಕೃತ ದೃಗದ್ಭುತಃ ॥೨೫॥

ಪ್ರತ್ಯಗಾತ್ಮಾ ಪರಂಜ್ಯೋತಿಃ ಪುರಾಣಃ ಪರಮೇಶ್ವರಃ~।\\
ಪ್ರಪಂಚೋಪಶಮಃ ಪ್ರಾಜ್ಞಃ ಪುಣ್ಯಕೀರ್ತಿಃ ಪುರಾತನಃ ॥೨೬॥

	ಸರ್ವಾಧಿಷ್ಠಾನ ಸನ್ಮಾತ್ರಸ್ಸ್ವಾತ್ಮ ಬಂಧಹರೋ ಹರಃ~।\\
	ಸರ್ವಪ್ರೇಮನಿಜಾಹಾಸಃ ಸರ್ವಾನುಗ್ರಹಕೃತ್ ಶಿವಃ ॥೨೭॥

ಸರ್ವೇಂದ್ರಿಯಗುಣಾಭಾಸಃ ಸರ್ವಭೂತಗುಣಾಶ್ರಯಃ~।\\
ಸಚ್ಚಿದಾನಂದಪೂರ್ಣಾತ್ಮಾ ಸ್ವೇ ಮಹಿಮ್ನಿ ಪ್ರತಿಷ್ಠಿತಃ ॥೨೮॥

	ಸರ್ವಭೂತಾಂತರಸ್ಸಾಕ್ಷೀ ಸರ್ವಜ್ಞಸ್ಸರ್ವಕಾಮದಃ~।\\
	ಸನಕಾದಿಮಹಾಯೋಗಿಸಮಾರಾಧಿತಪಾದುಕಃ ॥೨೯॥

ಆದಿದೇವೋ ದಯಾಸಿಂಧುಃ ಶಿಕ್ಷಿತಾಸುರವಿಗ್ರಹಃ~।\\
ಯಕ್ಷಕಿನ್ನರಗಂಧರ್ವಸ್ತೂಯಮಾನಾತ್ಮವೈಭವಃ ॥೩೦॥

	ಬ್ರಹ್ಮಾದಿದೇವವಿನುತೋ ಯೋಗಮಾಯಾನಿಯೋಜಕಃ~।\\
	ಶಿವಯೋಗೀ ಶಿವಾನಂದಃ ಶಿವಭಕ್ತಸಮುದ್ಧರಃ ॥೩೧॥

ವೇದಾಂತಸಾರಸಂದೋಹಃ ಸರ್ವಸತ್ತ್ವಾವಲಂಬನಃ~।\\
ವಟಮೂಲಾಶ್ರಯೋ ವಾಗ್ಮೀ ಮಾನ್ಯೋ ಮಲಯಜಪ್ರಿಯಃ ॥೩೨॥

	ಸುಶೀಲೋ ವಾಂಛಿತಾರ್ಥಜ್ಞಃ ಪ್ರಸನ್ನವದನೇಕ್ಷಣಃ ।\\
	ನೃತ್ತಗೀತಕಲಾಭಿಜ್ಞಃ ಕರ್ಮವಿತ್ ಕರ್ಮಮೋಚಕಃ ॥೩೩॥

ಕರ್ಮಸಾಕ್ಷೀ ಕರ್ಮಮಯಃ ಕರ್ಮಣಾಂ ಚ ಫಲಪ್ರದಃ~।\\
ಜ್ಞಾನದಾತಾ ಸದಾಚಾರಃ ಸರ್ವೋಪದ್ರವಮೋಚಕಃ ॥೩೪॥

	ಅನಾಥನಾಥೋ ಭಗವಾನಾಶ್ರಿತಾಮರಪಾದಪಃ~।\\
	ವರಪ್ರದಃ ಪ್ರಕಾಶಾತ್ಮಾ ಸರ್ವಭೂತಹಿತೇ ರತಃ ॥೩೫॥

ವ್ಯಾಘ್ರಚರ್ಮಾಸನಾಸೀನ ಆದಿಕರ್ತಾ ಮಹೇಶ್ವರಃ~।\\
ಸುವಿಕ್ರಮಃ ಸರ್ವಗತೋ ವಿಶಿಷ್ಟಜನವತ್ಸಲಃ ॥೩೬॥

	ಚಿಂತಾಶೋಕಪ್ರಶಮನೋ ಜಗದಾನಂದಕಾರಕಃ~।\\
	ರಶ್ಮಿಮಾನ್ ಭುವನೇಶಶ್ಚ ದೇವಾಸುರಸುಪೂಜಿತಃ ॥೩೭॥

ಮೃತ್ಯುಂಜಯೋ ವ್ಯೋಮಕೇಶಃ ಷಟ್ತ್ರಿಂಶತ್ತತ್ತ್ವಸಂಗ್ರಹಃ~।\\
ಅಜ್ಞಾತಸಂಭವೋ ಭಿಕ್ಷುರದ್ವಿತೀಯೋ ದಿಗಂಬರಃ ॥೩೮॥

	ಸಮಸ್ತದೇವತಾಮೂರ್ತಿಃ ಸೋಮಸೂರ್ಯಾಗ್ನಿಲೋಚನಃ~।\\
	ಸರ್ವಸಾಮ್ರಾಜ್ಯನಿಪುಣೋ ಧರ್ಮಮಾರ್ಗಪ್ರವರ್ತಕಃ ॥೩೯॥

ವಿಶ್ವಾಧಿಕಃ ಪಶುಪತಿಃ ಪಶುಪಾಶವಿಮೋಚಕಃ~।\\
ಅಷ್ಟಮೂರ್ತಿರ್ದೀಪ್ತಮೂರ್ತಿಃ ನಾಮೋಚ್ಚಾರಣಮುಕ್ತಿದಃ ॥೪೦॥

	ಸಹಸ್ರಾದಿತ್ಯಸಂಕಾಶಃ ಸದಾಷೋಡಶವಾರ್ಷಿಕಃ~।\\
	ದಿವ್ಯಕೇಲೀಸಮಾಯುಕ್ತೋ ದಿವ್ಯಮಾಲ್ಯಾಂಬರಾವೃತಃ ॥೪೧॥

ಅನರ್ಘರತ್ನಸಂಪೂರ್ಣೋ ಮಲ್ಲಿಕಾಕುಸುಮಪ್ರಿಯಃ~।\\
ತಪ್ತಚಾಮೀಕರಾಕಾರೋ ಜಿತದಾವಾನಲಾಕೃತಿಃ ॥೪೨॥

	ನಿರಂಜನೋ ನಿರ್ವಿಕಾರೋ ನಿಜಾವಾಸೋ ನಿರಾಕೃತಿಃ~।\\
	ಜಗದ್ಗುರುರ್ಜಗತ್ಕರ್ತಾ ಜಗದೀಶೋ ಜಗತ್ಪತಿಃ ॥೪೩॥

ಕಾಮಹಂತಾ ಕಾಮಮೂರ್ತಿಃ ಕಲ್ಯಾಣವೃಷವಾಹನಃ~।\\
ಗಂಗಾಧರೋ ಮಹಾದೇವೋ ದೀನಬಂಧವಿಮೋಚಕಃ ॥೪೪॥

	ಧೂರ್ಜಟಿಃ ಖಂಡಪರಶುಃ ಸದ್ಗುಣೋ ಗಿರಿಜಾಸಖಃ~।\\
	ಅವ್ಯಯೋ ಭೂತಸೇನೇಶಃ ಪಾಪಘ್ನಃ ಪುಣ್ಯದಾಯಕಃ ॥೪೫॥

ಉಪದೇಷ್ಟಾ ದೃಢಪ್ರಜ್ಞೋ ರುದ್ರೋ ರೋಗವಿನಾಶನಃ~।\\
ನಿತ್ಯಾನಂದೋ ನಿರಾಧಾರೋ ಹರೋ ದೇವಶಿಖಾಮಣಿಃ ॥೪೬॥

	ಪ್ರಣತಾರ್ತಿಹರಃ ಸೋಮಃ ಸಾಂದ್ರಾನಂದೋ ಮಹಾಮತಿಃ~।\\
	ಆಶ್ಚರ್ಯವೈಭವೋ ದೇವಃ ಸಂಸಾರಾರ್ಣವತಾರಕಃ ॥೪೭॥

ಯಜ್ಞೇಶೋ ರಾಜರಾಜೇಶೋ ಭಸ್ಮರುದ್ರಾಕ್ಷಲಾಂಛನಃ~।\\
ಅನಂತಸ್ತಾರಕಃ ಸ್ಥಾಣುಃ ಸರ್ವವಿದ್ಯೇಶ್ವರೋ ಹರಿಃ ॥೪೮॥

	ವಿಶ್ವರೂಪೋ ವಿರೂಪಾಕ್ಷಃ ಪ್ರಭುಃ ಪರಿವೃಢೋ ದೃಢಃ~।\\
	ಭವ್ಯೋ ಜಿತಾರಿಷಡ್ವರ್ಗೋ ಮಹೋದಾರೋ ವಿಷಾಶನಃ ॥೪೯॥

ಸುಕೀರ್ತಿರಾದಿಪುರುಷೋ ಜರಾಮರಣವರ್ಜಿತಃ~।\\
ಪ್ರಮಾಣಭೂತೋ ದುರ್ಜ್ಞೇಯಃ ಪುಣ್ಯಃ ಪರಪುರಂಜಯಃ ॥೫೦॥

	ಗುಣಾಕಾರೋ ಗುಣಶ್ರೇಷ್ಠಃ ಸಚ್ಚಿದಾನಂದವಿಗ್ರಹಃ~।\\
	ಸುಖದಃ ಕಾರಣಂ ಕರ್ತಾ ಭವಬಂಧವಿಮೋಚಕಃ ॥೫೧॥

ಅನಿರ್ವಿಣ್ಣೋ ಗುಣಗ್ರಾಹೀ ನಿಷ್ಕಲಂಕಃ ಕಲಂಕಹಾ~।\\
ಪುರುಷಃ ಶಾಶ್ವತೋ ಯೋಗೀ ವ್ಯಕ್ತಾವ್ಯಕ್ತಃ ಸನಾತನಃ ॥೫೨॥

	ಚರಾಚರಾತ್ಮಾ ಸೂಕ್ಷ್ಮಾತ್ಮಾ ವಿಶ್ವಕರ್ಮಾ ತಮೋಽಪಹೃತ್~।\\
	ಭುಜಂಗಭೂಷಣೋ ಭರ್ಗಸ್ತರುಣಃ ಕರುಣಾಲಯಃ ॥೫೩॥

ಅಣಿಮಾದಿಗುಣೋಪೇತೋ ಲೋಕವಶ್ಯವಿಧಾಯಕಃ~।\\
ಯೋಗಪಟ್ಟಧರೋ ಮುಕ್ತೋ ಮುಕ್ತಾನಾಂ ಪರಮಾ ಗತಿಃ ॥೫೪॥

	ಗುರುರೂಪಧರಃ ಶ್ರೀಮತ್ಪರಮಾನಂದಸಾಗರಃ~।\\
	ಸಹಸ್ರಬಾಹುಃ ಸರ್ವೇಶಃ ಸಹಸ್ರಾವಯವಾನ್ವಿತಃ ॥೫೫॥

ಸಹಸ್ರಮೂರ್ಧಾ ಸರ್ವಾತ್ಮಾ ಸಹಸ್ರಾಕ್ಷಃ ಸಹಸ್ರಪಾತ್~।\\
ನಿರಾಭಾಸಃ ಸೂಕ್ಷ್ಮತನುರ್ಹೃದಿ ಜ್ಞಾತಃ ಪರಾತ್ಪರಃ ॥೫೬॥

	ಸರ್ವಾತ್ಮಗಃ ಸರ್ವಸಾಕ್ಷೀ ನಿಃಸಂಗೋ ನಿರುಪದ್ರವಃ~।\\
	ನಿಷ್ಕಲಃ ಸಕಲಾಧ್ಯಕ್ಷಶ್ಚಿನ್ಮಯಸ್ತಮಸಃ ಪರಃ ॥೫೭॥

ಜ್ಞಾನವೈರಾಗ್ಯಸಂಪನ್ನೋ ಯೋಗಾನಂದಮಯಃ ಶಿವಃ~।\\
ಶಾಶ್ವತೈಶ್ವರ್ಯಸಂಪೂರ್ಣೋ ಮಹಾಯೋಗೀಶ್ವರೇಶ್ವರಃ ॥೫೮॥

	ಸಹಸ್ರಶಕ್ತಿಸಂಯುಕ್ತಃ ಪುಣ್ಯಕಾಯೋ ದುರಾಸದಃ~।\\
	ತಾರಕಬ್ರಹ್ಮಸಂಪೂರ್ಣಸ್ತಪಸ್ವಿಜನಸಂವೃತಃ ॥೫೯॥

ವಿಧೀಂದ್ರಾಮರಸಂಪೂಜ್ಯೋ ಜ್ಯೋತಿಷಾಂ ಜ್ಯೋತಿರುತ್ತಮಃ~।\\
ನಿರಕ್ಷರೋ ನಿರಾಲಂಬಃ ಸ್ವಾತ್ಮಾರಾಮೋ ವಿಕರ್ತನಃ ॥೬೦॥

	ನಿರವದ್ಯೋ ನಿರಾತಂಕೋ ಭೀಮೋ ಭೀಮಪರಾಕ್ರಮಃ~।\\
	ವೀರಭದ್ರಃ ಪುರಾರಾತಿರ್ಜಲಂಧರಶಿರೋಹರಃ ॥೬೧॥

ಅಂಧಕಾಸುರಸಂಹರ್ತಾ ಭಗನೇತ್ರಭಿದದ್ಭುತಃ~।\\
ವಿಶ್ವಗ್ರಾಸೋಽಧರ್ಮಶತ್ರುರ್ಬ್ರಹ್ಮಜ್ಞಾನೈಕಮಂಥರಃ ॥೬೨॥

	ಅಗ್ರೇಸರಸ್ತೀರ್ಥಭೂತಃ ಸಿತಭಸ್ಮಾವಕುಂಠನಃ~।\\
	ಅಕುಂಠಮೇಧಾಃ ಶ್ರೀಕಂಠೋ ವೈಕುಂಠಪರಮಪ್ರಿಯಃ ॥೬೩॥

ಲಲಾಟೋಜ್ಜ್ವಲನೇತ್ರಾಬ್ಜಸ್ತುಷಾರಕರಶೇಖರಃ~।\\
ಗಜಾಸುರಶಿರಶ್ಛೇತ್ತಾ ಗಂಗೋದ್ಭಾಸಿತಮೂರ್ಧಜಃ ॥೬೪॥

	ಕಲ್ಯಾಣಾಚಲಕೋದಂಡಃ ಕಮಲಾಪತಿಸಾಯಕಃ~।\\
	ವಾರಾಂ ಶೇವಧಿ ತೂಣೀರಃ ಸರೋಜಾಸನ ಸಾರಥಿಃ ॥೬೫॥

ತ್ರಯೀತುರಂಗಸಂಕ್ರಾಂತೋ ವಾಸುಕಿಜ್ಯಾವಿರಾಜಿತಃ~।\\
ರವೀಂದುಚರಣಾಚಾರಿಧರಾರಥವಿರಾಜಿತಃ ॥೬೬॥

	ತ್ರಯ್ಯಂತಪ್ರಗ್ರಹೋದಾರಚಾರುಘಂಟಾರವೋಜ್ಜ್ವಲಃ~।\\
	ಉತ್ತಾನಪರ್ವಲೋಮಾಢ್ಯೋ ಲೀಲಾವಿಜಿತಮನ್ಮಥಃ ॥೬೭॥

ಜಾತುಪ್ರಪನ್ನಜನತಾಜೀವನೋಪಾಯನೋತ್ಸುಕಃ~।\\
ಸಂಸಾರಾರ್ಣವನಿರ್ಮಗ್ನಸಮುದ್ಧರಣಪಂಡಿತಃ ॥೬೮॥

	ಮದದ್ವಿರದಧಿಕ್ಕಾರಿಗತಿಮಂಜುಲವೈಭವಃ~।\\
	ಮತ್ತಕೋಕಿಲಮಾಧುರ್ಯರಸನಿರ್ಭರಗೀರ್ಗಣಃ ॥೬೯॥

ಕೈವಲ್ಯೋದಧಿಕಲ್ಲೋಲಲೀಲಾತಾಂಡವಪಂಡಿತಃ~।\\
ವಿಷ್ಣುರ್ಜಿಷ್ಣುರ್ವಾಸುದೇವಃ ಪ್ರಭವಿಷ್ಣುಃ ಪುರಾತನಃ ॥೭೦॥

	ವರ್ಧಿಷ್ಣುರ್ವರದೋ ವೈದ್ಯೋ ಹರಿರ್ನಾರಾಯಣೋಽಚ್ಯುತಃ~।\\
	ಅಜ್ಞಾನವನದಾವಾಗ್ನಿಃ ಪ್ರಜ್ಞಾಪ್ರಾಸಾದಭೂಪತಿಃ ॥೭೧॥

ಸರ್ಪಭೂಷಿತಸರ್ವಾಂಗಃ ಕರ್ಪೂರೋಜ್ಜ್ವಲಿತಾಕೃತಿಃ~।\\
ಅನಾದಿಮಧ್ಯನಿಧನೋ ಗಿರೀಶೋ ಗಿರಿಜಾಪತಿಃ ॥೭೨॥

	ವೀತರಾಗೋ ವಿನೀತಾತ್ಮಾ ತಪಸ್ವೀ ಭೂತಭಾವನಃ~।\\
	ದೇವಾಸುರಗುರುಧ್ಯೇಯೋ ದೇವಾಸುರನಮಸ್ಕೃತಃ ॥೭೩॥

ದೇವಾದಿದೇವೋ ದೇವರ್ಷಿರ್ದೇವಾಸುರವರಪ್ರದಃ~।\\
ಸರ್ವದೇವಮಯೋಽಚಿಂತ್ಯೋ ದೇವಾತ್ಮಾ ಚಾತ್ಮಸಂಭವಃ ॥೭೪॥

	ನಿರ್ಲೇಪೋ ನಿಷ್ಪ್ರಪಂಚಾತ್ಮಾ ನಿರ್ವಿಘ್ನೋ ವಿಘ್ನನಾಶಕಃ~।\\
	ಏಕಜ್ಯೋತಿರ್ನಿರಾತಂಕೋ ವ್ಯಾಪ್ತಮೂರ್ತಿರನಾಕುಲಃ ॥೭೫॥

ನಿರವದ್ಯಪದೋಪಾಧಿರ್ವಿದ್ಯಾರಾಶಿರನುತ್ತಮಃ~।\\
ನಿತ್ಯಾನಂದಃ ಸುರಾಧ್ಯಕ್ಷೋ ನಿಃಸಂಕಲ್ಪೋ ನಿರಂಜನಃ ॥೭೬॥

	ನಿಷ್ಕಲಂಕೋ ನಿರಾಕಾರೋ ನಿಷ್ಪ್ರಪಂಚೋ ನಿರಾಮಯಃ~।\\
	ವಿದ್ಯಾಧರೋ ವಿಯತ್ಕೇಶೋ ಮಾರ್ಕಂಡೇಯವರಪ್ರದಃ ॥೭೭॥

ಭೈರವೋ ಭೈರವೀನಾಥಃ ಕಾಮದಃ ಕಮಲಾಸನಃ~।\\
ವೇದವೇದ್ಯಃ ಸುರಾನಂದೋ ಲಸಜ್ಜ್ಯೋತಿಃ ಪ್ರಭಾಕರಃ ॥೭೮॥

	ಚೂಡಾಮಣಿಃ ಸುರಾಧೀಶೋ ಯಜ್ಞಗೇಯೋ ಹರಿಪ್ರಿಯಃ~।\\
	ನಿರ್ಲೇಪೋ ನೀತಿಮಾನ್ ಸೂತ್ರೀ ಶ್ರೀಹಾಲಾಹಲಸುಂದರಃ ॥೭೯॥

ಧರ್ಮದಕ್ಷೋ ಮಹಾರಾಜಃ ಕಿರೀಟೀ ವಂದಿತೋ ಗುಹಃ~।\\
ಮಾಧವೋ ಯಾಮಿನೀನಾಥಃ ಶಂಬರಃ ಶಬರೀಪ್ರಿಯಃ ॥೮೦॥

	ಸಂಗೀತವೇತ್ತಾ ಲೋಕಜ್ಞಃ ಶಾಂತಃ ಕಲಶಸಂಭವಃ~।\\
	ಬ್ರಹ್ಮಣ್ಯೋ ವರದೋ ನಿತ್ಯಃ ಶೂಲೀ ಗುರುವರೋ ಹರಃ ॥೮೧॥

ಮಾರ್ತಾಂಡಃ ಪುಂಡರೀಕಾಕ್ಷೋ ಲೋಕನಾಯಕವಿಕ್ರಮಃ~।\\
ಮುಕುಂದಾರ್ಚ್ಯೋ ವೈದ್ಯನಾಥಃ ಪುರಂದರವರಪ್ರದಃ ॥೮೨॥

	ಭಾಷಾವಿಹೀನೋ ಭಾಷಾಜ್ಞೋ ವಿಘ್ನೇಶೋ ವಿಘ್ನನಾಶನಃ~।\\
	ಕಿನ್ನರೇಶೋ ಬೃಹದ್ಭಾನುಃ ಶ್ರೀನಿವಾಸಃ ಕಪಾಲಭೃತ್ ॥೮೩॥

ವಿಜಯೋ ಭೂತಭಾವಜ್ಞೋ ಭೀಮಸೇನೋ ದಿವಾಕರಃ~।\\
ಬಿಲ್ವಪ್ರಿಯೋ ವಸಿಷ್ಠೇಶಃ ಸರ್ವಮಾರ್ಗಪ್ರವರ್ತಕಃ ॥೮೪॥

	ಓಷಧೀಶೋ ವಾಮದೇವೋ ಗೋವಿಂದೋ ನೀಲಲೋಹಿತಃ~।\\
	ಷಡರ್ಧನಯನಃ ಶ್ರೀಮನ್ಮಹಾದೇವೋ ವೃಷಧ್ವಜಃ ॥೮೫॥

ಕರ್ಪೂರದೀಪಿಕಾಲೋಲಃ ಕರ್ಪೂರರಸಚರ್ಚಿತಃ~।\\
ಅವ್ಯಾಜಕರುಣಾಮೂರ್ತಿಸ್ತ್ಯಾಗರಾಜಃ ಕ್ಷಪಾಕರಃ ॥೮೬॥

	ಆಶ್ಚರ್ಯವಿಗ್ರಹಃ ಸೂಕ್ಷ್ಮಃ ಸಿದ್ಧೇಶಃ ಸ್ವರ್ಣಭೈರವಃ~।\\
	ದೇವರಾಜಃ ಕೃಪಾಸಿಂಧುರದ್ವಯೋಽಮಿತವಿಕ್ರಮಃ ॥೮೭॥

ನಿರ್ಭೇದೋ ನಿತ್ಯಸತ್ವಸ್ಥೋ ನಿರ್ಯೋಗಕ್ಷೇಮ ಆತ್ಮವಾನ್~।\\
ನಿರಪಾಯೋ ನಿರಾಸಂಗೋ ನಿಃಶಬ್ದೋ ನಿರುಪಾಧಿಕಃ ॥೮೮॥

	ಭವಃ ಸರ್ವೇಶ್ವರಃ ಸ್ವಾಮೀ ಭವಭೀತಿವಿಭಂಜನಃ~।\\
	ದಾರಿದ್ರ್ಯತೃಣಕೂಟಾಗ್ನಿರ್ದಾರಿತಾಸುರಸಂತತಿಃ ॥೮೯॥

ಮುಕ್ತಿದೋ ಮುದಿತೋಽಕುಬ್ಜೋ ಧಾರ್ಮಿಕೋ ಭಕ್ತವತ್ಸಲಃ~।\\
ಅಭ್ಯಾಸಾತಿಶಯಜ್ಞೇಯಶ್ಚಂದ್ರಮೌಲಿಃ ಕಲಾಧರಃ ॥೯೦॥

	ಮಹಾಬಲೋ ಮಹಾವೀರ್ಯೋ ವಿಭುಃ ಶ್ರೀಶಃ ಶುಭಪ್ರದಃ~।\\
	ಸಿದ್ಧಃ ಪುರಾಣಪುರುಷೋ ರಣಮಂಡಲಭೈರವಃ ॥೯೧॥

ಸದ್ಯೋಜಾತೋ ವಟಾರಣ್ಯವಾಸೀ ಪುರುಷವಲ್ಲಭಃ~।\\
ಹರಿಕೇಶೋ ಮಹಾತ್ರಾತಾ ನೀಲಗ್ರೀವಸ್ಸುಮಂಗಲಃ ॥೯೨॥

	ಹಿರಣ್ಯಬಾಹುಸ್ತೀಕ್ಷ್ಣಾಂಶುಃ ಕಾಮೇಶಃ ಸೋಮವಿಗ್ರಹಃ~।\\
	ಸರ್ವಾತ್ಮಾ ಸರ್ವಕರ್ತಾ ಚ ತಾಂಡವೋ ಮುಂಡಮಾಲಿಕಃ ॥೯೩॥

ಅಗ್ರಗಣ್ಯಃ ಸುಗಂಭೀರೋ ದೇಶಿಕೋ ವೈದಿಕೋತ್ತಮಃ~।\\
ಪ್ರಸನ್ನದೇವೋ ವಾಗೀಶಶ್ಚಿಂತಾತಿಮಿರಭಾಸ್ಕರಃ ॥೯೪॥

	ಗೌರೀಪತಿಸ್ತುಂಗಮೌಲಿರ್ಮಖರಾಜೋ ಮಹಾಕವಿಃ~।\\
	ಶ್ರೀಧರಸ್ಸರ್ವಸಿದ್ಧೇಶೋ ವಿಶ್ವನಾಥೋ ದಯಾನಿಧಿಃ ॥೯೫॥

ಅಂತರ್ಮುಖೋ ಬಹಿರ್ದೃಷ್ಟಿಃ ಸಿದ್ಧವೇಷಮನೋಹರಃ~।\\
ಕೃತ್ತಿವಾಸಾಃ ಕೃಪಾಸಿಂಧುರ್ಮಂತ್ರಸಿದ್ಧೋ ಮತಿಪ್ರದಃ ॥೯೬॥

	ಮಹೋತ್ಕೃಷ್ಟಃ ಪುಣ್ಯಕರೋ ಜಗತ್ಸಾಕ್ಷೀ ಸದಾಶಿವಃ~।\\
	ಮಹಾಕ್ರತುರ್ಮಹಾಯಜ್ವಾ ವಿಶ್ವಕರ್ಮಾ ತಪೋನಿಧಿಃ ॥೯೭॥

ಛಂದೋಮಯೋ ಮಹಾಜ್ಞಾನೀ ಸರ್ವಜ್ಞೋ ದೇವವಂದಿತಃ~।\\
ಸಾರ್ವಭೌಮಸ್ಸದಾನಂದಃ ಕರುಣಾಮೃತವಾರಿಧಿಃ ॥೯೮॥

	ಕಾಲಕಾಲಃ ಕಲಿಧ್ವಂಸೀ ಜರಾಮರಣನಾಶಕಃ~।\\
	ಶಿತಿಕಂಠಶ್ಚಿದಾನಂದೋ ಯೋಗಿನೀಗಣಸೇವಿತಃ ॥೯೯॥

ಚಂಡೀಶಃ ಶುಕಸಂವೇದ್ಯಃ ಪುಣ್ಯಶ್ಲೋಕೋ ದಿವಸ್ಪತಿಃ~।\\
ಸ್ಥಾಯೀ ಸಕಲತತ್ತ್ವಾತ್ಮಾ ಸದಾಸೇವಕವರ್ಧನಃ ॥೧೦೦॥

	ರೋಹಿತಾಶ್ವಃ ಕ್ಷಮಾರೂಪೀ ತಪ್ತಚಾಮೀಕರಪ್ರಭಃ~।\\
	ತ್ರಿಯಂಬಕೋ ವರರುಚಿರ್ದೇವದೇವಶ್ಚತುರ್ಭುಜಃ ॥೧೦೧।\\

ವಿಶ್ವಂಭರೋ ವಿಚಿತ್ರಾಂಗೋ ವಿಧಾತಾ ಪುರಶಾಸನಃ~।\\
ಸುಬ್ರಹ್ಮಣ್ಯೋ ಜಗತ್ಸ್ವಾಮೀ ರೋಹಿತಾಕ್ಷಃ ಶಿವೋತ್ತಮಃ ॥೧೦೨॥

	ನಕ್ಷತ್ರಮಾಲಾಭರಣೋ ಮಘವಾನ್ ಅಘನಾಶನಃ~।\\
	ವಿಧಿಕರ್ತಾ ವಿಧಾನಜ್ಞಃ ಪ್ರಧಾನಪುರುಷೇಶ್ವರಃ ॥೧೦೩॥

ಚಿಂತಾಮಣಿಃ ಸುರಗುರುರ್ಧ್ಯೇಯೋ ನೀರಾಜನಪ್ರಿಯಃ~।\\
ಗೋವಿಂದೋ ರಾಜರಾಜೇಶೋ ಬಹುಪುಷ್ಪಾರ್ಚನಪ್ರಿಯಃ ॥೧೦೪॥।\\

ಸರ್ವಾನಂದೋ ದಯಾರೂಪೀ ಶೈಲಜಾಸುಮನೋಹರಃ~।\\
ಸುವಿಕ್ರಮಃ ಸರ್ವಗತೋ ಹೇತುಸಾಧನವರ್ಜಿತಃ ॥೧೦೫॥

	ವೃಷಾಂಕೋ ರಮಣೀಯಾಂಗಃ ಸದಂಘ್ರಿಃ ಸಾಮಪಾರಗಃ~।\\
	ಮಂತ್ರಾತ್ಮಾ ಕೋಟಿಕಂದರ್ಪಸೌಂದರ್ಯರಸವಾರಿಧಿಃ ॥೧೦೬ ॥

ಯಜ್ಞೇಶೋ ಯಜ್ಞಪುರುಷಃ ಸೃಷ್ಟಿಸ್ಥಿತ್ಯಂತಕಾರಣಂ~।\\
ಪರಹಂಸೈಕಜಿಜ್ಞಾಸ್ಯಃ ಸ್ವಪ್ರಕಾಶಸ್ವರೂಪವಾನ್ ॥೧೦೭॥

	ಮುನಿಮೃಗ್ಯೋ ದೇವಮೃಗ್ಯೋ ಮೃಗಹಸ್ತೋ ಮೃಗೇಶ್ವರಃ~।\\
	ಮೃಗೇಂದ್ರಚರ್ಮವಸನೋ ನರಸಿಂಹನಿಪಾತನಃ ॥೧೦೮॥

ಮುನಿವಂದ್ಯೋ ಮುನಿಶ್ರೇಷ್ಠೋ ಮುನಿಬೃಂದನಿಷೇವಿತಃ~।\\
ದುಷ್ಟಮೃತ್ಯುರದುಷ್ಟೇಹೋ ಮೃತ್ಯುಹಾ ಮೃತ್ಯುಪೂಜಿತಃ ॥೧೦೯॥

	ಅವ್ಯಕ್ತೋಽಮ್ಬುಜಜನ್ಮಾದಿಕೋಟಿಕೋಟಿಸುಪೂಜಿತಃ~।\\
	ಲಿಂಗಮೂರ್ತಿರಲಿಂಗಾತ್ಮಾ ಲಿಂಗಾತ್ಮಾ ಲಿಂಗವಿಗ್ರಹಃ ॥೧೧೦॥

ಯಜುರ್ಮೂರ್ತಿಃ ಸಾಮಮೂರ್ತಿರೃಙ್ಮೂರ್ತಿರ್ಮೂರ್ತಿವರ್ಜಿತಃ~।\\
ವಿಶ್ವೇಶೋ ಗಜಚರ್ಮೈಕಚೇಲಾಂಚಿತಕಟೀತಟಃ ॥೧೧೧॥

	ಪಾವನಾಂತೇವಸದ್ಯೋಗಿಜನಸಾರ್ಥಸುಧಾಕರಃ~।\\
	ಅನಂತಸೋಮಸೂರ್ಯಾಗ್ನಿಮಂಡಲಪ್ರತಿಮಪ್ರಭಃ ॥೧೧೨॥

ಚಿಂತಾಶೋಕಪ್ರಶಮನಃ ಸರ್ವವಿದ್ಯಾವಿಶಾರದಃ~।\\
ಭಕ್ತವಿಜ್ಞಪ್ತಿಸಂಧಾತಾ ಕರ್ತಾ ಗಿರಿವರಾಕೃತಿಃ ॥೧೧೩॥

	ಜ್ಞಾನಪ್ರದೋ ಮನೋವಾಸಃ ಕ್ಷೇಮ್ಯೋ ಮೋಹವಿನಾಶನಃ~।\\
	ಸುರೋತ್ತಮಶ್ಚಿತ್ರಭಾನುಃ ಸದಾವೈಭವತತ್ಪರಃ ॥೧೧೪॥

ಸುಹೃದಗ್ರೇಸರಃ ಸಿದ್ಧಜ್ಞಾನಮುದ್ರೋ ಗಣಾಧಿಪಃ~।\\
ಆಗಮಶ್ಚರ್ಮವಸನೋ ವಾಂಛಿತಾರ್ಥಫಲಪ್ರದಃ ॥೧೧೫॥

	ಅಂತರ್ಹಿತೋಽಸಮಾನಶ್ಚ ದೇವಸಿಂಹಾಸನಾಧಿಪಃ~।\\
	ವಿವಾದಹಂತಾ ಸರ್ವಾತ್ಮಾ ಕಾಲಃ ಕಾಲವಿವರ್ಜಿತಃ ॥೧೧೬॥

ವಿಶ್ವಾತೀತೋ ವಿಶ್ವಕರ್ತಾ ವಿಶ್ವೇಶೋ ವಿಶ್ವಕಾರಣಂ~।\\
ಯೋಗಿಧ್ಯೇಯೋ ಯೋಗನಿಷ್ಠೋ ಯೋಗಾತ್ಮಾ ಯೋಗವಿತ್ತಮಃ ॥೧೧೭॥

	ಓಂಕಾರರೂಪೋ ಭಗವಾನ್ ಬಿಂದುನಾದಮಯಃ ಶಿವಃ~।\\
	ಚತುರ್ಮುಖಾದಿಸಂಸ್ತುತ್ಯಶ್ಚತುರ್ವರ್ಗಫಲಪ್ರದಃ ॥೧೧೮॥

ಸಹ್ಯಾಚಲಗುಹಾವಾಸೀ ಸಾಕ್ಷಾನ್ಮೋಕ್ಷರಸಾಮೃತಃ~।\\
ದಕ್ಷಾಧ್ವರಸಮುಚ್ಛೇತ್ತಾ ಪಕ್ಷಪಾತವಿವರ್ಜಿತಃ ॥೧೧೯॥

	ಓಂಕಾರವಾಚಕಃ ಶಂಭುಃ ಶಂಕರಃ ಶಶಿಶೀತಲಃ~।\\
	ಪಂಕಜಾಸನಸಂಸೇವ್ಯಃ ಕಿಂಕರಾಮರವತ್ಸಲಃ ॥೧೨೦॥

ನತದೌರ್ಭಾಗ್ಯತೂಲಾಗ್ನಿಃ ಕೃತಕೌತುಕಮಂಗಲಃ~।\\
ತ್ರಿಲೋಕಮೋಹನಃ ಶ್ರೀಮತ್ತ್ರಿಪುಂಡ್ರಾಂಕಿತಮಸ್ತಕಃ ॥೧೨೧॥

	ಕ್ರೌಂಚಾರಿಜನಕಃ ಶ್ರೀಮದ್ಗಣನಾಥಸುತಾನ್ವಿತಃ~।\\
	ಅದ್ಭುತಾನಂತವರದೋಽಪರಿಚ್ಛಿನ್ನಾತ್ಮವೈಭವಃ ॥೧೨೨॥

ಇಷ್ಟಾಪೂರ್ತಪ್ರಿಯಃ ಶರ್ವ ಏಕವೀರಃ ಪ್ರಿಯಂವದಃ~।\\
ಊಹಾಪೋಹವಿನಿರ್ಮುಕ್ತ ಓಂಕಾರೇಶ್ವರಪೂಜಿತಃ ॥೧೨೩॥

	ರುದ್ರಾಕ್ಷವಕ್ಷಾ ರುದ್ರಾಕ್ಷರೂಪೋ ರುದ್ರಾಕ್ಷಪಕ್ಷಕಃ~।\\
	ಭುಜಗೇಂದ್ರಲಸತ್ಕಂಠೋ ಭುಜಂಗಾಭರಣಪ್ರಿಯಃ ॥೧೨೪॥

ಕಲ್ಯಾಣರೂಪಃ ಕಲ್ಯಾಣಃ ಕಲ್ಯಾಣಗುಣಸಂಶ್ರಯಃ~।\\
ಸುಂದರಭ್ರೂಃ ಸುನಯನಃ ಸುಲಲಾಟಃ ಸುಕಂಧರಃ ॥೧೨೫॥

	ವಿದ್ವಜ್ಜನಾಶ್ರಯೋ ವಿದ್ವಜ್ಜನಸ್ತವ್ಯಪರಾಕ್ರಮಃ~।\\
	ವಿನೀತವತ್ಸಲೋ ನೀತಿಸ್ವರೂಪೋ ನೀತಿಸಂಶ್ರಯಃ ॥೧೨೬॥

ಅತಿರಾಗೀ ವೀತರಾಗೀ ರಾಗಹೇತುರ್ವಿರಾಗವಿತ್~।\\
ರಾಗಹಾ ರಾಗಶಮನೋ ರಾಗದೋ ರಾಗಿರಾಗವಿತ್ ॥೧೨೭॥

	ಮನೋನ್ಮನೋ ಮನೋರೂಪೋ ಬಲಪ್ರಮಥನೋ ಬಲಃ~।\\
	ವಿದ್ಯಾಕರೋ ಮಹಾವಿದ್ಯೋ ವಿದ್ಯಾವಿದ್ಯಾವಿಶಾರದಃ ॥೧೨೮॥

ವಸಂತಕೃದ್ವಸಂತಾತ್ಮಾ ವಸಂತೇಶೋ ವಸಂತದಃ~।\\
ಪ್ರಾವೃಟ್ಕೃತ್ ಪ್ರಾವೃಡಾಕಾರಃ ಪ್ರಾವೃಟ್ಕಾಲಪ್ರವರ್ತಕಃ ॥೧೨೯॥

	ಶರನ್ನಾಥೋ ಶರತ್ಕಾಲನಾಶಕಃ ಶರದಾಶ್ರಯಃ~।\\
	ಕುಂದಮಂದಾರಪುಷ್ಪೌಘಲಸದ್ವಾಯುನಿಷೇವಿತಃ ॥೧೩೦॥

ದಿವ್ಯದೇಹಪ್ರಭಾಕೂಟಸಂದೀಪಿತದಿಗಂತರಃ~।\\
ದೇವಾಸುರಗುರುಸ್ತವ್ಯೋ ದೇವಾಸುರನಮಸ್ಕೃತಃ ॥೧೩೧॥

	ವಾಮಾಂಗಭಾಗವಿಲಸಚ್ಛ್ಯಾಮಲಾವೀಕ್ಷಣಪ್ರಿಯಃ~।\\
	ಕೀರ್ತ್ಯಾಧಾರಃ ಕೀರ್ತಿಕರಃ ಕೀರ್ತಿಹೇತುರಹೇತುಕಃ ॥೧೩೨॥

ಶರಣಾಗತದೀನಾರ್ತಪರಿತ್ರಾಣಪರಾಯಣಃ~।\\
ಮಹಾಪ್ರೇತಾಸನಾಸೀನೋ ಜಿತಸರ್ವಪಿತಾಮಹಃ ॥೧೩೩॥

	ಮುಕ್ತಾದಾಮಪರೀತಾಂಗೋ ನಾನಾಗಾನವಿಶಾರದಃ~।\\
	ವಿಷ್ಣುಬ್ರಹ್ಮಾದಿವಂದ್ಯಾಂಘ್ರಿರ್ನಾನಾದೇಶೈಕನಾಯಕಃ ॥೧೩೪॥

ಧೀರೋದಾತ್ತೋ ಮಹಾಧೀರೋ ಧೈರ್ಯದೋ ಧೈರ್ಯವರ್ಧಕಃ~।\\
ವಿಜ್ಞಾನಮಯ ಆನಂದಮಯಃ ಪ್ರಾಣಮಯೋಽನ್ನದಃ ॥೧೩೫॥

	ಭವಾಬ್ಧಿತರಣೋಪಾಯಃ ಕವಿರ್ದುಃಸ್ವಪ್ನನಾಶನಃ~।\\
	ಗೌರೀವಿಲಾಸಸದನಃ ಪಿಶಾಚಾನುಚರಾವೃತಃ ॥೧೩೬॥

ದಕ್ಷಿಣಾಪ್ರೇಮಸಂತುಷ್ಟೋ ದಾರಿದ್ರ್ಯವಡವಾನಲಃ~।\\
ಅದ್ಭುತಾನಂತ ಸಂಗ್ರಾಮೋ ಢಕ್ಕಾವಾದನ ತತ್ಪರಃ ॥೧೩೭॥

	ಪ್ರಾಚ್ಯಾತ್ಮಾ ದಕ್ಷಿಣಾಕಾರಃ ಪ್ರತೀಚ್ಯಾತ್ಮೋತ್ತರಾಕೃತಿಃ~।\\
	ಊರ್ಧ್ವಾದ್ಯನ್ಯದಿಗಾಕಾರೋ ಮರ್ಮಜ್ಞಃ ಸರ್ವಶಿಕ್ಷಕಃ ॥೧೩೮॥

ಯುಗಾವಹೋ ಯುಗಾಧೀಶೋ ಯುಗಾತ್ಮಾ ಯುಗನಾಯಕಃ~।\\
ಜಂಗಮಃ ಸ್ಥಾವರಾಕಾರಃ ಕೈಲಾಸಶಿಖರಪ್ರಿಯಃ ॥೧೩೯॥

	ಹಸ್ತರಾಜತ್ಪುಂಡರೀಕಃ ಪುಂಡರೀಕನಿಭೇಕ್ಷಣಃ~।\\
	ಲೀಲಾವಿಡಂಬಿತವಪುರ್ಭಕ್ತಮಾನಸಮಂಡಿತಃ ॥೧೪೦॥

ಬೃಂದಾರಕಪ್ರಿಯತಮೋ ಬೃಂದಾರಕವರಾರ್ಚಿತಃ~।\\
ನಾನಾವಿಧಾನೇಕರತ್ನಲಸತ್ಕುಂಡಲಮಂಡಿತಃ ॥೧೪೧॥

	ನಿಃಸೀಮಮಹಿಮಾ ನಿತ್ಯಲೀಲಾವಿಗ್ರಹರೂಪಧೃತ್~।\\
	ಚಂದನದ್ರವದಿಗ್ಧಾಂಗಶ್ಚಾಂಪೇಯಕುಸುಮಾರ್ಚಿತಃ ॥೧೪೨॥

ಸಮಸ್ತಭಕ್ತಸುಖದಃ ಪರಮಾಣುರ್ಮಹಾಹ್ರದಃ~।\\
ಅಲೌಕಿಕೋ ದುಷ್ಪ್ರಧರ್ಷಃ ಕಪಿಲಃ ಕಾಲಕಂಧರಃ ॥೧೪೩॥

	ಕರ್ಪೂರಗೌರಃ ಕುಶಲಃ ಸತ್ಯಸಂಧೋ ಜಿತೇಂದ್ರಿಯಃ~।\\
	ಶಾಶ್ವತೈಶ್ವರ್ಯವಿಭವಃ ಪೋಷಕಃ ಸುಸಮಾಹಿತಃ ॥೧೪೪॥

ಮಹರ್ಷಿನಾಥಿತೋ ಬ್ರಹ್ಮಯೋನಿಃ ಸರ್ವೋತ್ತಮೋತ್ತಮಃ~।\\
ಭೂತಿಭಾರಾರ್ತಿಸಂಹರ್ತಾ ಷಡೂರ್ಮಿರಹಿತೋ ಮೃಡಃ ॥೧೪೫॥

	ತ್ರಿವಿಷ್ಟಪೇಶ್ವರಃ ಸರ್ವಹೃದಯಾಂಬುಜಮಧ್ಯಗಃ~।\\
	ಸಹಸ್ರದಲಪದ್ಮಸ್ಥಃ ಸರ್ವವರ್ಣೋಪಶೋಭಿತಃ ॥೧೪೬॥

ಪುಣ್ಯಮೂರ್ತಿಃ ಪುಣ್ಯಲಭ್ಯಃ ಪುಣ್ಯಶ್ರವಣಕೀರ್ತನಃ~।\\
ಸೂರ್ಯಮಂಡಲಮಧ್ಯಸ್ಥಶ್ಚಂದ್ರಮಂಡಲಮಧ್ಯಗಃ ॥೧೪೭॥

	ಸದ್ಭಕ್ತಧ್ಯಾನನಿಗಲಃ ಶರಣಾಗತಪಾಲಕಃ~।\\
	ಶ್ವೇತಾತಪತ್ರರುಚಿರಃ ಶ್ವೇತಚಾಮರವೀಜಿತಃ ॥೧೪೮॥

ಸರ್ವಾವಯವಸಂಪೂರ್ಣಃ ಸರ್ವಲಕ್ಷಣಲಕ್ಷಿತಃ~।\\
ಸರ್ವಮಂಗಲಮಾಂಗಲ್ಯಃ ಸರ್ವಕಾರಣಕಾರಣಃ ॥೧೪೯॥

	ಅಮೋದೋ ಮೋದಜನಕಃ ಸರ್ಪರಾಜೋತ್ತರೀಯಕಃ~।\\
	ಕಪಾಲೀ ಕೋವಿದಃ ಸಿದ್ಧಕಾಂತಿಸಂವಲಿತಾನನಃ ॥೧೫೦॥

ಸರ್ವಸದ್ಗುರುಸಂಸೇವ್ಯೋ ದಿವ್ಯಚಂದನಚರ್ಚಿತಃ~।\\
ವಿಲಾಸಿನೀಕೃತೋಲ್ಲಾಸ ಇಚ್ಛಾಶಕ್ತಿನಿಷೇವಿತಃ ॥೧೫೧॥

	ಅನಂತಾನಂದಸುಖದೋ ನಂದನಃ ಶ್ರೀನಿಕೇತನಃ~।\\
	ಅಮೃತಾಬ್ಧಿಕೃತಾವಾಸೋ ನಿತ್ಯಕ್ಲೀಬೋ ನಿರಾಮಯಃ ॥೧೫೨॥

ಅನಪಾಯೋಽನಂತದೃಷ್ಟಿರಪ್ರಮೇಯೋಽಜರೋಽಮರಃ~।\\
ತಮೋಮೋಹಪ್ರತಿಹತಿರಪ್ರತರ್ಕ್ಯೋಽಮೃತೋಽಕ್ಷರಃ ॥೧೫೩॥

	ಅಮೋಘಬುದ್ಧಿರಾಧಾರ ಆಧಾರಾಧೇಯವರ್ಜಿತಃ~।\\
	ಈಷಣಾತ್ರಯನಿರ್ಮುಕ್ತ ಇಹಾಮುತ್ರವಿವರ್ಜಿತಃ ॥೧೫೪॥

ಋಗ್ಯಜುಃಸಾಮನಯನೋ ಬುದ್ಧಿಸಿದ್ಧಿಸಮೃದ್ಧಿದಃ~।\\
ಔದಾರ್ಯನಿಧಿರಾಪೂರ್ಣ ಐಹಿಕಾಮುಷ್ಮಿಕಪ್ರದಃ ॥೧೫೫॥

	ಶುದ್ಧಸನ್ಮಾತ್ರಸಂವಿದ್ಧೀಸ್ವರೂಪಃ ಸುಖವಿಗ್ರಹಃ~।\\
	ದರ್ಶನಪ್ರಥಮಾಭಾಸೋ ದೃಷ್ಟಿದೃಶ್ಯವಿವರ್ಜಿತಃ ॥೧೫೬॥

ಅಗ್ರಗಣ್ಯೋಽಚಿಂತ್ಯರೂಪಃ ಕಲಿಕಲ್ಮಷನಾಶನಃ~।\\
ವಿಮರ್ಶರೂಪೋ ವಿಮಲೋ ನಿತ್ಯರೂಪೋ ನಿರಾಶ್ರಯಃ ॥೧೫೭॥

ನಿತ್ಯಶುದ್ಧೋ ನಿತ್ಯಬುದ್ಧೋ ನಿತ್ಯಮುಕ್ತೋಽಪರಾಕೃತಃ~।\\
ಮೈತ್ರ್ಯಾದಿವಾಸನಾಲಭ್ಯೋ ಮಹಾಪ್ರಲಯಸಂಸ್ಥಿತಃ ॥೧೫೮॥

ಮಹಾಕೈಲಾಸನಿಲಯಃ ಪ್ರಜ್ಞಾನಘನವಿಗ್ರಹಃ~।\\
ಶ್ರೀಮಾನ್ ವ್ಯಾಘ್ರಪುರಾವಾಸೋ ಭುಕ್ತಿಮುಕ್ತಿಪ್ರದಾಯಕಃ ॥೧೫೯॥

	ಜಗದ್ಯೋನಿರ್ಜಗತ್ಸಾಕ್ಷೀ ಜಗದೀಶೋ ಜಗನ್ಮಯಃ~।\\
	ಜಪೋ ಜಪಪರೋ ಜಪ್ಯೋ ವಿದ್ಯಾಸಿಂಹಾಸನಪ್ರಭುಃ ॥೧೬೦॥

ತತ್ತ್ವಾನಾಂ ಪ್ರಕೃತಿಸ್ತತ್ತ್ವಂ ತತ್ತ್ವಂಪದನಿರೂಪಿತಃ~।\\
ದಿಕ್ಕಾಲಾದ್ಯನವಚ್ಛಿನ್ನಃ ಸಹಜಾನಂದಸಾಗರಃ ॥೧೬೧॥

	ಪ್ರಕೃತಿಃ ಪ್ರಾಕೃತಾತೀತೋ ವಿಜ್ಞಾನೈಕರಸಾಕೃತಿಃ~।\\
	ನಿಃಶಂಕಮತಿದೂರಸ್ಥಶ್ಚೈತ್ಯಚೇತನಚಿಂತನಃ ॥೧೬೨॥

ತಾರಕಾನಾಂ ಹೃದಂತಸ್ಥಸ್ತಾರಕಸ್ತಾರಕಾಂತಕಃ~।\\
ಧ್ಯಾನೈಕಪ್ರಕಟೋ ಧ್ಯೇಯೋ ಧ್ಯಾನೀ ಧ್ಯಾನವಿಭೂಷಣಃ ॥೧೬೩॥

	ಪರಂ ವ್ಯೋಮ ಪರಂ ಧಾಮ ಪರಮಾತ್ಮಾ ಪರಂ ಪದಂ~।\\
	ಪೂರ್ಣಾನಂದಃ ಸದಾನಂದೋ ನಾದಮಧ್ಯಪ್ರತಿಷ್ಠಿತಃ ॥೧೬೪॥

ಪ್ರಮಾವಿಪರ್ಯಯಾತೀತಃ ಪ್ರಣತಾಜ್ಞಾನನಾಶಕಃ~।\\
ಬಾಣಾರ್ಚಿತಾಂಘ್ರಿರ್ಬಹುದೋ ಬಾಲಕೇಲಿಕುತೂಹಲೀ ॥೧೬೫॥

	ಬ್ರಹ್ಮರೂಪೀ ಬ್ರಹ್ಮಪದಂ ಬ್ರಹ್ಮವಿದ್ ಬ್ರಾಹ್ಮಣಪ್ರಿಯಃ~।\\
	ಭ್ರೂಕ್ಷೇಪದತ್ತಲಕ್ಷ್ಮೀಕೋ ಭ್ರೂಮಧ್ಯಧ್ಯಾನಲಕ್ಷಿತಃ ॥೧೬೬॥

ಯಶಸ್ಕರೋ ರತ್ನಗರ್ಭೋ ಮಹಾರಾಜ್ಯಸುಖಪ್ರದಃ~।\\
ಶಬ್ದಬ್ರಹ್ಮ ಶಮಪ್ರಾಪ್ಯೋ ಲಾಭಕೃಲ್ಲೋಕವಿಶ್ರುತಃ ॥೧೬೭॥

	ಶಾಸ್ತಾ ಶಿವಾದ್ರಿನಿಲಯಃ ಶರಣ್ಯೋ ಯಾಜಕಪ್ರಿಯಃ~।\\
	ಸಂಸಾರವೈದ್ಯಃ ಸರ್ವಜ್ಞಃ ಸಭೇಷಜವಿಭೇಷಜಃ ॥೧೬೮॥

ಮನೋವಚೋಭಿರಗ್ರಾಹ್ಯಃ ಪಂಚಕೋಶವಿಲಕ್ಷಣಃ~।\\
ಅವಸ್ಥಾತ್ರಯನಿರ್ಮುಕ್ತಸ್ತ್ವವಸ್ಥಾಸಾಕ್ಷಿತುರ್ಯಕಃ ॥೧೬೯॥

	ಪಂಚಭೂತಾದಿದೂರಸ್ಥಃ ಪ್ರತ್ಯಗೇಕರಸೋಽವ್ಯಯಃ~।\\
	ಷಟ್ಚಕ್ರಾಂತರ್ಗತೋಲ್ಲಾಸೀ ಷಡ್ವಿಕಾರವಿವರ್ಜಿತಃ ॥೧೭೦॥

ವಿಜ್ಞಾನಘನಸಂಪೂರ್ಣೋ ವೀಣಾವಾದನತತ್ಪರಃ~।\\
ನೀಹಾರಾಕಾರಗೌರಾಂಗೋ ಮಹಾಲಾವಣ್ಯವಾರಿಧಿಃ ॥೧೭೧॥

	ಪರಾಭಿಚಾರಶಮನಃ ಷಡಧ್ವೋಪರಿಸಂಸ್ಥಿತಃ~।\\
	ಸುಷುಮ್ನಾಮಾರ್ಗಸಂಚಾರೀ ಬಿಸತಂತುನಿಭಾಕೃತಿಃ ॥೧೭೨॥

ಪಿನಾಕೀ ಲಿಂಗರೂಪಶ್ರೀಃ ಮಂಗಲಾವಯವೋಜ್ಜ್ವಲಃ~।\\
ಕ್ಷೇತ್ರಾಧಿಪಃ ಸುಸಂವೇದ್ಯಃ ಶ್ರೀಪ್ರದೋ ವಿಭವಪ್ರದಃ ॥೧೭೩॥

	ಸರ್ವವಶ್ಯಕರಃ ಸರ್ವದೋಷಹಾ ಪುತ್ರಪೌತ್ರದಃ~।\\
	ತೈಲದೀಪಪ್ರಿಯಸ್ತೈಲಪಕ್ವಾನ್ನಪ್ರೀತಮಾನಸಃ ॥೧೭೪॥

ತೈಲಾಭಿಷೇಕಸಂತುಷ್ಟಸ್ತಿಲಭಕ್ಷಣತತ್ಪರಃ~।\\
ಆಪಾದಕನಿಕಾಮುಕ್ತಾಭೂಷಾಶತಮನೋಹರಃ ॥೧೭೫॥

	ಶಾಣೋಲ್ಲೀಢಮಣಿಶ್ರೇಣೀರಮ್ಯಾಂಘ್ರಿನಖಮಂಡಲಃ~।\\
	ಮಣಿಮಂಜೀರಕಿರಣಕಿಂಜಲ್ಕಿತಪದಾಂಬುಜಃ ॥೧೭೬॥

ಅಪಸ್ಮಾರೋಪರಿನ್ಯಸ್ತಸವ್ಯಪಾದಸರೋರುಹಃ~।\\
ಕಂದರ್ಪತೂಣಾಭಜಂಘೋ ಗುಲ್ಫೋದಂಚಿತನೂಪುರಃ ॥೧೭೭॥

	ಕರಿಹಸ್ತೋಪಮೇಯೋರುರಾದರ್ಶೋಜ್ಜ್ವಲಜಾನುಭೃತ್~।\\
	ವಿಶಂಕಟಕಟಿನ್ಯಸ್ತವಾಚಾಲಮಣಿಮೇಖಲಃ ॥೧೭೮॥

ಆವರ್ತನಾಭಿರೋಮಾಲಿವಲಿಮತ್ಪಲ್ಲವೋದರಃ~।\\
ಮುಕ್ತಾಹಾರಲಸತ್ತುಂಗವಿಪುಲೋರಸ್ಕರಂಜಿತಃ ॥೧೭೯॥

	ವೀರಾಸನಸಮಾಸೀನೋ ವೀಣಾಪುಸ್ತೋಲ್ಲಸತ್ಕರಃ~।\\
	ಅಕ್ಷಮಾಲಾಲಸತ್ಪಾಣಿಶ್ಚಿನ್ಮುದ್ರಿತಕರಾಂಬುಜಃ ॥೧೮೦॥

ಮಾಣಿಕ್ಯಕಂಕಣೋಲ್ಲಾಸಿಕರಾಂಬುಜವಿರಾಜಿತಃ~।\\
ಅನರ್ಘರತ್ನಗ್ರೈವೇಯವಿಲಸತ್ಕಂಬುಕಂಧರಃ ॥೧೮೧॥

	ಅನಾಕಲಿತಸಾದೃಶ್ಯಚಿಬುಕಶ್ರೀವಿರಾಜಿತಃ~।\\
	ಮುಗ್ಧಸ್ಮಿತಪರೀಪಾಕಪ್ರಕಾಶಿತರದಾಂಕುರಃ ॥೧೮೨॥

ಚಾರುಚಾಂಪೇಯಪುಷ್ಪಾಭನಾಸಿಕಾಪುಟರಂಜಿತಃ~।\\
ವರವಜ್ರಶಿಲಾದರ್ಶಪರಿಭಾವಿಕಪೋಲಭೂಃ ॥೧೮೩॥

	ಕರ್ಣದ್ವಯೋಲ್ಲಸದ್ದಿವ್ಯಮಣಿಕುಂಡಲಮಂಡಿತಃ~।\\
	ಕರುಣಾಲಹರೀಪೂರ್ಣಕರ್ಣಾಂತಾಯತಲೋಚನಃ ॥೧೮೪॥

ಅರ್ಧಚಂದ್ರಾಭನಿಟಿಲಪಾಟೀರತಿಲಕೋಜ್ಜ್ವಲಃ~।\\
ಚಾರುಚಾಮೀಕರಾಕಾರಜಟಾಚರ್ಚಿತಚಂದನಃ~।\\
ಕೈಲಾಸಶಿಖರಸ್ಫರ್ಧಿಕಮನೀಯನಿಜಾಕೃತಿಃ ॥೧೮೫॥
\authorline{ಇತಿ ಶ್ರೀದಕ್ಷಿಣಾಮೂರ್ತಿಸಹಸ್ರನಾಮಸ್ತೋತ್ರಂ ಸಂಪೂರ್ಣಂ ॥}
%=================================================================================================================
\section{ಶ್ರೀದಕ್ಷಿಣಾಮೂರ್ತಿ ಅಷ್ಟೋತ್ತರ ಶತನಾಮಸ್ತೋತ್ರಂ }
\addcontentsline{toc}{section}{ಶ್ರೀದಕ್ಷಿಣಾಮೂರ್ತಿ ಅಷ್ಟೋತ್ತರ ಶತನಾಮಸ್ತೋತ್ರಂ }
\dhyana{ವಟವೃಕ್ಷ ತಟಾಸೀನಂ ಯೋಗಿ ಧ್ಯೇಯಾಂಘ್ರಿ ಪಂಕಜಂ।\\
ಶರಚ್ಚಂದ್ರ ನಿಭಂ ಪೂಜ್ಯಂ ಜಟಾಮುಕುಟ ಮಂಡಿತಂ ॥

ಗಂಗಾಧರಂ ಲಲಾಟಾಕ್ಷಂ ವ್ಯಾಘ್ರ ಚರ್ಮಾಂಬರಾವೃತಂ।\\
ನಾಗಭೂಷಂ ಪರಂಬ್ರಹ್ಮ ದ್ವಿಜರಾಜಾವತಂಸಕಂ ॥

ಅಕ್ಷಮಾಲಾ ಜ್ಞಾನಮುದ್ರಾ ವೀಣಾ ಪುಸ್ತಕ ಶೋಭಿತಂ।\\
ಶುಕಾದಿ ವೃದ್ಧ ಶಿಷ್ಯಾಢ್ಯಂ ವೇದ ವೇದಾಂತಗೋಚರಂ॥\\
ಯುವಾನಂ ಮನ್ಮಥಾರಾತಿಂ ದಕ್ಷಿಣಾಮೂರ್ತಿಮಾಶ್ರಯೇ॥}

ಓಂ ವಿದ್ಯಾರೂಪೀ ಮಹಾಯೋಗೀ ಶುದ್ಧ ಜ್ಞಾನೀ ಪಿನಾಕಧೃತ್~।\\
ರತ್ನಾಲಂಕೃತ ಸರ್ವಾಂಗೀ ರತ್ನಮೌಳಿರ್ಜಟಾಧರಃ ॥೧॥

ಗಂಗಾಧಾರ್ಯಚಲಾವಾಸೀ ಮಹಾಜ್ಞಾನೀ ಸಮಾಧಿಕೃತ್।\\
ಅಪ್ರಮೇಯೋ ಯೋಗನಿಧಿಸ್ತಾರಕೋ ಭಕ್ತವತ್ಸಲಃ॥೨॥

ಬ್ರಹ್ಮರೂಪೀ ಜಗದ್ವ್ಯಾಪೀ ವಿಷ್ಣುಮೂರ್ತಿಃ ಪುರಾತನಃ~।\\
ಉಕ್ಷವಾಹಶ್ಚರ್ಮವಾಸಾಃ ಪೀತಾಂಬರ ವಿಭೂಷಣಃ॥೩॥

ಮೋಕ್ಷದಾಯೀ ಮೋಕ್ಷ ನಿಧಿಶ್ಚಾಂಧಕಾರಿರ್ಜಗತ್ಪತಿಃ।\\
ವಿದ್ಯಾಧಾರೀ ಶುಕ್ಲ ತನುಃ ವಿದ್ಯಾದಾಯೀ ಗಣಾಧಿಪಃ॥೪॥

ಪ್ರೌಢಾಪಸ್ಮೃತಿ ಸಂಹರ್ತಾ ಶಶಿಮೌಳಿರ್ಮಹಾಸ್ವನಃ~।\\
ಸಾಮ ಪ್ರಿಯೋಽವ್ಯಯಃ ಸಾಧುಃ ಸರ್ವ ವೇದೈರಲಂಕೃತಃ ॥೫॥

ಹಸ್ತೇ ವಹ್ನಿಧರಃ ಶ್ರೀಮಾನ್ ಮೃಗಧಾರೀ ವಶಂಕರಃ~।\\
ಯಜ್ಞನಾಥಃ ಕ್ರತುಧ್ವಂಸೀ ಯಜ್ಞಭೋಕ್ತಾ ಯಮಾಂತಕಃ॥೬॥

ಭಕ್ತಾನುಗ್ರಹ ಮೂರ್ತಿಶ್ಚ ಭಕ್ತಸೇವ್ಯೋ ವೃಷಧ್ವಜಃ~।\\
ಭಸ್ಮೋದ್ಧೂಲಿತ ಸರ್ವಾಂಗಃ ಚಾಕ್ಷಮಾಲಾಧರೋ ಮಹಾನ್ ॥೭॥

ತ್ರಯೀಮೂರ್ತಿಃ ಪರಂಬ್ರಹ್ಮ ನಾಗರಾಜೈರಲಂಕೃತಃ~।\\
ಶಾಂತರೂಪೋ ಮಹಾಜ್ಞಾನೀ ಸರ್ವ ಲೋಕ ವಿಭೂಷಣಃ ॥೮॥

ಅರ್ಧನಾರೀಶ್ವರೋ ದೇವೋ ಮುನಿಸೇವ್ಯಸ್ಸುರೋತ್ತಮಃ~।\\
ವ್ಯಾಖ್ಯಾನದೇವೋ ಭಗವಾನ್ ರವಿ ಚಂದ್ರಾಗ್ನಿ ಲೋಚನಃ ॥೯॥

ಜಗದ್ಗುರುರ್ಮಹಾದೇವೋ ಮಹಾನಂದ ಪರಾಯಣಃ~।\\
ಜಟಾಧಾರೀ ಮಹಾಯೋಗೀ ಜ್ಞಾನಮಾಲೈರಲಂಕೃತಃ ॥೧೦॥

ವ್ಯೋಮಗಂಗಾ ಜಲ ಸ್ಥಾನಃ ವಿಶುದ್ಧೋ ಯತಿರೂರ್ಜಿತಃ~।\\
ತತ್ತ್ವಮೂರ್ತಿರ್ಮಹಾಯೋಗೀ ಮಹಾಸಾರಸ್ವತಪ್ರದಃ ॥೧೧॥\\

ವ್ಯೋಮಮೂರ್ತಿಶ್ಚ ಭಕ್ತಾನಾಂ ಇಷ್ಟಃ ಕಾಮಫಲಪ್ರದಃ~।\\
ಪರಮೂರ್ತಿಃ ಚಿತ್ಸ್ವರೂಪೀ ತೇಜೋಮೂರ್ತಿರನಾಮಯಃ ॥೧೨॥

ವೇದವೇದಾಂಗ ತತ್ತ್ವಜ್ಞಃ ಚತುಃಷ್ಷಷ್ಟಿ ಕಲಾನಿಧಿಃ~।\\
ಭವರೋಗ ಭಯಧ್ವಂಸೀ ಭಕ್ತಾನಾಮಭಯಪ್ರದಃ ॥೧೩॥

ನೀಲಗ್ರೀವೋ ಲಲಾಟಾಕ್ಷೋ ಗಜ ಚರ್ಮಾಗತಿಪ್ರದಃ~।\\
ಅರಾಗೀ ಕಾಮದಶ್ಚಾಥ ತಪಸ್ವೀ ವಿಷ್ಣುವಲ್ಲಭಃ ॥೧೪॥

ಬ್ರಹ್ಮಚಾರೀ ಚ ಸನ್ಯಾಸೀ ಗೃಹಸ್ಥಾಶ್ರಮ ಕಾರಣಃ~।\\
ದಾಂತಃ ಶಮವತಾಂ ಶ್ರೇಷ್ಠೋ ಸತ್ಯರೂಪೋ ದಯಾಪರಃ ॥೧೫॥

ಯೋಗಪಟ್ಟಾಭಿರಾಮಶ್ಚ ವೀಣಾಧಾರೀ ವಿಚೇತನಃ~।\\
ಮತಿಪ್ರಜ್ಞಾ ಸುಧಾಧಾರೀ ಮುದ್ರಾಪುಸ್ತಕ ಧಾರಣಃ ॥೧೬॥

ವೇತಾಲಾದಿ ಪಿಶಾಚೌಘ ರಾಕ್ಷಸೌಘ ವಿನಾಶನಃ~।\\
ರಾಜ ಯಕ್ಷ್ಮಾದಿ ರೋಗಾಣಾಂ ವಿನಿಹಂತಾ ಸುರೇಶ್ವರಃ ॥೧೭॥
\authorline{॥ಇತಿ ಶ್ರೀ ದಕ್ಷಿಣಾಮೂರ್ತಿ ಅಷ್ಟೋತ್ತರ ಶತನಾಮ ಸ್ತೋತ್ರಂ ಸಂಪೂರ್ಣಂ ॥}
%==================================================================================================================
\section{ಮಹಾಗಣಪತಿಸಹಸ್ರನಾಮಸ್ತೋತ್ರಂ }
\addcontentsline{toc}{section}{ಮಹಾಗಣಪತಿಸಹಸ್ರನಾಮಸ್ತೋತ್ರಂ }
ಮುನಿರುವಾಚ ।\\
ಕಥಂ ನಾಮ್ನಾಂ ಸಹಸ್ರಂ ತಂ ಗಣೇಶ ಉಪದಿಷ್ಟವಾನ್ ।\\
ಶಿವದಂ ತನ್ಮಮಾಚಕ್ಷ್ವ ಲೋಕಾನುಗ್ರಹತತ್ಪರ ॥೧॥

। ಬ್ರಹ್ಮೋವಾಚ ।\\
ದೇವಃ ಪೂರ್ವಂ ಪುರಾರಾತಿಃ ಪುರತ್ರಯಜಯೋದ್ಯಮೇ ।\\
ಅನರ್ಚನಾದ್ಗಣೇಶಸ್ಯ ಜಾತೋ ವಿಘ್ನಾಕುಲಃ ಕಿಲ ॥೨॥

ಮನಸಾ ಸ ವಿನಿರ್ಧಾರ್ಯ ದದೃಶೇ ವಿಘ್ನಕಾರಣಂ ।\\
ಮಹಾಗಣಪತಿಂ ಭಕ್ತ್ಯಾ ಸಮಭ್ಯರ್ಚ್ಯ ಯಥಾವಿಧಿ ॥೩॥

ವಿಘ್ನಪ್ರಶಮನೋಪಾಯಮಪೃಚ್ಛದಪರಿಶ್ರಮಂ ।\\
ಸಂತುಷ್ಟಃ ಪೂಜಯಾ ಶಂಭೋರ್ಮಹಾಗಣಪತಿಃ ಸ್ವಯಂ ॥೪॥

ಸರ್ವವಿಘ್ನಪ್ರಶಮನಂ ಸರ್ವಕಾಮಫಲಪ್ರದಂ ।\\
ತತಸ್ತಸ್ಮೈ ಸ್ವಯಂ ನಾಮ್ನಾಂ ಸಹಸ್ರಮಿದಮಬ್ರವೀತ್ ॥೫॥

ಅಸ್ಯ ಶ್ರೀಮಹಾಗಣಪತಿಸಹಸ್ರನಾಮಸ್ತೋತ್ರಮಾಲಾಮಂತ್ರಸ್ಯ । ಗಣೇಶ ಋಷಿಃ । ಮಹಾಗಣಪತಿರ್ದೇವತಾ । ನಾನಾವಿಧಾನಿಚ್ಛಂದಾಂಸಿ । ಹುಮಿತಿ ಬೀಜಂ । ತುಂಗಮಿತಿ ಶಕ್ತಿಃ । ಸ್ವಾಹಾಶಕ್ತಿರಿತಿ ಕೀಲಕಂ ॥

ಅಥ ಕರನ್ಯಾಸಃ ।\\
ಗಣೇಶ್ವರೋ ಗಣಕ್ರೀಡ ಇತ್ಯಂಗುಷ್ಠಾಭ್ಯಾಂ ನಮಃ ।\\
ಕುಮಾರಗುರುರೀಶಾನ ಇತಿ ತರ್ಜನೀಭ್ಯಾಂ ನಮಃ ॥೧॥

ಬ್ರಹ್ಮಾಂಡಕುಂಭಶ್ಚಿದ್ವ್ಯೋಮೇತಿ ಮಧ್ಯಮಾಭ್ಯಾಂ ನಮಃ ।\\
ರಕ್ತೋ ರಕ್ತಾಂಬರಧರ ಇತ್ಯನಾಮಿಕಾಭ್ಯಾಂ ನಮಃ ॥೨॥

ಸರ್ವಸದ್ಗುರುಸಂಸೇವ್ಯ ಇತಿ ಕನಿಷ್ಠಿಕಾಭ್ಯಾಂ ನಮಃ ।\\
ಲುಪ್ತವಿಘ್ನಃ ಸ್ವಭಕ್ತಾನಾಮಿತಿ ಕರತಲಕರಪೃಷ್ಠಾಭ್ಯಾಂ ನಮಃ ॥೩॥

ಅಥ ಹೃದಯಾದಿನ್ಯಾಸಃ ।\\
ಛಂದಶ್ಛಂದೋದ್ಭವ ಇತಿ ಹೃದಯಾಯ ನಮಃ ।\\
ನಿಷ್ಕಲೋ ನಿರ್ಮಲ ಇತಿ ಶಿರಸೇ ಸ್ವಾಹಾ ।\\
ಸೃಷ್ಟಿಸ್ಥಿತಿಲಯಕ್ರೀಡ ಇತಿ ಶಿಖಾಯೈ ವಷಟ್ ।\\
ಜ್ಞಾನಂ ವಿಜ್ಞಾನಮಾನಂದ ಇತಿ ಕವಚಾಯ ಹುಂ ।\\
ಅಷ್ಟಾಂಗಯೋಗಫಲಭೃದಿತಿ ನೇತ್ರತ್ರಯಾಯ ವೌಷಟ್ ।\\
ಅನಂತಶಕ್ತಿಸಹಿತ ಇತ್ಯಸ್ತ್ರಾಯ ಫಟ್ ।\\
ಭೂರ್ಭುವಃ ಸ್ವರೋಂ ಇತಿ ದಿಗ್ಬಂಧಃ ॥

ಅಥ ಧ್ಯಾನಂ ।\\
ಗಜವದನಮಚಿಂತ್ಯಂ ತೀಕ್ಷ್ಣದಂಷ್ಟ್ರಂ ತ್ರಿನೇತ್ರಂ\\
ಬೃಹದುದರಮಶೇಷಂ ಭೂತಿರಾಜಂ ಪುರಾಣಂ ।\\
ಅಮರವರಸುಪೂಜ್ಯಂ ರಕ್ತವರ್ಣಂ ಸುರೇಶಂ\\
ಪಶುಪತಿಸುತಮೀಶಂ ವಿಘ್ನರಾಜಂ ನಮಾಮಿ ॥೧॥

ಸಕಲವಿಘ್ನವಿನಾಶನದ್ವಾರಾ ಶ್ರೀಮಹಾಗಣಪತಿಪ್ರಸಾದಸಿದ್ಧ್ಯರ್ಥೇ ಜಪೇ ವಿನಿಯೋಗಃ ॥

ಶ್ರೀಗಣಪತಿರುವಾಚ ।\\
ಓಂ ಗಣೇಶ್ವರೋ ಗಣಕ್ರೀಡೋ ಗಣನಾಥೋ ಗಣಾಧಿಪಃ ।\\
ಏಕದಂತೋ ವಕ್ರತುಂಡೋ ಗಜವಕ್ತ್ರೋ ಮಹೋದರಃ ॥೧॥

ಲಂಬೋದರೋ ಧೂಮ್ರವರ್ಣೋ ವಿಕಟೋ ವಿಘ್ನನಾಶನಃ ।\\
ಸುಮುಖೋ ದುರ್ಮುಖೋ ಬುದ್ಧೋ ವಿಘ್ನರಾಜೋ ಗಜಾನನಃ ॥೨॥

ಭೀಮಃ ಪ್ರಮೋದ ಆಮೋದಃ ಸುರಾನಂದೋ ಮದೋತ್ಕಟಃ ।\\
ಹೇರಂಬಃ ಶಂಬರಃ ಶಂಭುರ್ಲಂಬಕರ್ಣೋ ಮಹಾಬಲಃ ॥೩॥

ನಂದನೋ ಲಂಪಟೋ ಭೀಮೋ ಮೇಘನಾದೋ ಗಣಂಜಯಃ ।\\
ವಿನಾಯಕೋ ವಿರೂಪಾಕ್ಷೋ ವೀರಃ ಶೂರವರಪ್ರದಃ ॥೪॥

ಮಹಾಗಣಪತಿರ್ಬುದ್ಧಿಪ್ರಿಯಃ ಕ್ಷಿಪ್ರಪ್ರಸಾದನಃ ।\\
ರುದ್ರಪ್ರಿಯೋ ಗಣಾಧ್ಯಕ್ಷ ಉಮಾಪುತ್ರೋಽಘನಾಶನಃ ॥೫॥

ಕುಮಾರಗುರುರೀಶಾನಪುತ್ರೋ ಮೂಷಕವಾಹನಃ ।\\
ಸಿದ್ಧಿಪ್ರಿಯಃ ಸಿದ್ಧಿಪತಿಃ ಸಿದ್ಧಃ ಸಿದ್ಧಿವಿನಾಯಕಃ ॥೬॥

ಅವಿಘ್ನಸ್ತುಂಬುರುಃ ಸಿಂಹವಾಹನೋ ಮೋಹಿನೀಪ್ರಿಯಃ ।\\
ಕಟಂಕಟೋ ರಾಜಪುತ್ರಃ ಶಾಕಲಃ ಸಂಮಿತೋಽಮಿತಃ ॥೭॥

ಕೂಷ್ಮಾಂಡಸಾಮಸಂಭೂತಿರ್ದುರ್ಜಯೋ ಧೂರ್ಜಯೋ ಜಯಃ ।\\
ಭೂಪತಿರ್ಭುವನಪತಿರ್ಭೂತಾನಾಂ ಪತಿರವ್ಯಯಃ ॥೮॥

ವಿಶ್ವಕರ್ತಾ ವಿಶ್ವಮುಖೋ ವಿಶ್ವರೂಪೋ ನಿಧಿರ್ಗುಣಃ ।\\
ಕವಿಃ ಕವೀನಾಮೃಷಭೋ ಬ್ರಹ್ಮಣ್ಯೋ ಬ್ರಹ್ಮವಿತ್ಪ್ರಿಯಃ ॥೯॥

ಜ್ಯೇಷ್ಠರಾಜೋ ನಿಧಿಪತಿರ್ನಿಧಿಪ್ರಿಯಪತಿಪ್ರಿಯಃ ।\\
ಹಿರಣ್ಮಯಪುರಾಂತಃಸ್ಥಃ ಸೂರ್ಯಮಂಡಲಮಧ್ಯಗಃ ॥೧೦॥

ಕರಾಹತಿಧ್ವಸ್ತಸಿಂಧುಸಲಿಲಃ ಪೂಷದಂತಭಿತ್ ।\\
ಉಮಾಂಕಕೇಲಿಕುತುಕೀ ಮುಕ್ತಿದಃ ಕುಲಪಾವನಃ ॥೧೧॥

ಕಿರೀಟೀ ಕುಂಡಲೀ ಹಾರೀ ವನಮಾಲೀ ಮನೋಮಯಃ ।\\
ವೈಮುಖ್ಯಹತದೈತ್ಯಶ್ರೀಃ ಪಾದಾಹತಿಜಿತಕ್ಷಿತಿಃ ॥೧೨॥

ಸದ್ಯೋಜಾತಃ ಸ್ವರ್ಣಮುಂಜಮೇಖಲೀ ದುರ್ನಿಮಿತ್ತಹೃತ್ ।\\
ದುಃಸ್ವಪ್ನಹೃತ್ಪ್ರಸಹನೋ ಗುಣೀ ನಾದಪ್ರತಿಷ್ಠಿತಃ ॥೧೩॥

ಸುರೂಪಃ ಸರ್ವನೇತ್ರಾಧಿವಾಸೋ ವೀರಾಸನಾಶ್ರಯಃ ।\\
ಪೀತಾಂಬರಃ ಖಂಡರದಃ ಖಂಡವೈಶಾಖಸಂಸ್ಥಿತಃ ॥೧೪॥

ಚಿತ್ರಾಂಗಃ ಶ್ಯಾಮದಶನೋ ಭಾಲಚಂದ್ರೋ ಹವಿರ್ಭುಜಃ ।\\
ಯೋಗಾಧಿಪಸ್ತಾರಕಸ್ಥಃ ಪುರುಷೋ ಗಜಕರ್ಣಕಃ ॥೧೫॥

ಗಣಾಧಿರಾಜೋ ವಿಜಯಃ ಸ್ಥಿರೋ ಗಜಪತಿರ್ಧ್ವಜೀ ।\\
ದೇವದೇವಃ ಸ್ಮರಃ ಪ್ರಾಣದೀಪಕೋ ವಾಯುಕೀಲಕಃ ॥೧೬॥

ವಿಪಶ್ಚಿದ್ವರದೋ ನಾದೋ ನಾದಭಿನ್ನಮಹಾಚಲಃ ।\\
ವರಾಹರದನೋ ಮೃತ್ಯುಂಜಯೋ ವ್ಯಾಘ್ರಾಜಿನಾಂಬರಃ ॥೧೭॥

ಇಚ್ಛಾಶಕ್ತಿಭವೋ ದೇವತ್ರಾತಾ ದೈತ್ಯವಿಮರ್ದನಃ ।\\
ಶಂಭುವಕ್ತ್ರೋದ್ಭವಃ ಶಂಭುಕೋಪಹಾ ಶಂಭುಹಾಸ್ಯಭೂಃ ॥೧೮॥

ಶಂಭುತೇಜಾಃ ಶಿವಾಶೋಕಹಾರೀ ಗೌರೀಸುಖಾವಹಃ ।\\
ಉಮಾಂಗಮಲಜೋ ಗೌರೀತೇಜೋಭೂಃ ಸ್ವರ್ಧುನೀಭವಃ ॥೧೯॥

ಯಜ್ಞಕಾಯೋ ಮಹಾನಾದೋ ಗಿರಿವರ್ಷ್ಮಾ ಶುಭಾನನಃ ।\\
ಸರ್ವಾತ್ಮಾ ಸರ್ವದೇವಾತ್ಮಾ ಬ್ರಹ್ಮಮೂರ್ಧಾ ಕಕುಪ್ಶ್ರುತಿಃ ॥೨೦॥

ಬ್ರಹ್ಮಾಂಡಕುಂಭಶ್ಚಿದ್ವ್ಯೋಮಭಾಲಃಸತ್ಯಶಿರೋರುಹಃ ।\\
ಜಗಜ್ಜನ್ಮಲಯೋನ್ಮೇಷನಿಮೇಷೋಽಗ್ನ್ಯರ್ಕಸೋಮದೃಕ್ ॥೨೧॥

ಗಿರೀಂದ್ರೈಕರದೋ ಧರ್ಮಾಧರ್ಮೋಷ್ಠಃ ಸಾಮಬೃಂಹಿತಃ ।\\
ಗ್ರಹರ್ಕ್ಷದಶನೋ ವಾಣೀಜಿಹ್ವೋ ವಾಸವನಾಸಿಕಃ ॥೨೨॥

ಭ್ರೂಮಧ್ಯಸಂಸ್ಥಿತಕರೋ ಬ್ರಹ್ಮವಿದ್ಯಾಮದೋದಕಃ ।\\
ಕುಲಾಚಲಾಂಸಃ ಸೋಮಾರ್ಕಘಂಟೋ ರುದ್ರಶಿರೋಧರಃ ॥೨೩॥

ನದೀನದಭುಜಃ ಸರ್ಪಾಂಗುಲೀಕಸ್ತಾರಕಾನಖಃ ।\\
ವ್ಯೋಮನಾಭಿಃ ಶ್ರೀಹೃದಯೋ ಮೇರುಪೃಷ್ಠೋಽರ್ಣವೋದರಃ ॥೨೪॥

ಕುಕ್ಷಿಸ್ಥಯಕ್ಷಗಂಧರ್ವರಕ್ಷಃಕಿನ್ನರಮಾನುಷಃ ।\\
ಪೃಥ್ವೀಕಟಿಃ ಸೃಷ್ಟಿಲಿಂಗಃ ಶೈಲೋರುರ್ದಸ್ರಜಾನುಕಃ ॥೨೫॥

ಪಾತಾಲಜಂಘೋ ಮುನಿಪಾತ್ಕಾಲಾಂಗುಷ್ಠಸ್ತ್ರಯೀತನುಃ ।\\
ಜ್ಯೋತಿರ್ಮಂಡಲಲಾಂಗೂಲೋ ಹೃದಯಾಲಾನನಿಶ್ಚಲಃ ॥೨೬॥

ಹೃತ್ಪದ್ಮಕರ್ಣಿಕಾಶಾಲೀ ವಿಯತ್ಕೇಲಿಸರೋವರಃ ।\\
ಸದ್ಭಕ್ತಧ್ಯಾನನಿಗಡಃ ಪೂಜಾವಾರಿನಿವಾರಿತಃ ॥೨೭॥

ಪ್ರತಾಪೀ ಕಾಶ್ಯಪೋ ಮಂತಾ ಗಣಕೋ ವಿಷ್ಟಪೀ ಬಲೀ ।\\
ಯಶಸ್ವೀ ಧಾರ್ಮಿಕೋ ಜೇತಾ ಪ್ರಥಮಃ ಪ್ರಮಥೇಶ್ವರಃ ॥೨೮॥

ಚಿಂತಾಮಣಿರ್ದ್ವೀಪಪತಿಃ ಕಲ್ಪದ್ರುಮವನಾಲಯಃ ।\\
ರತ್ನಮಂಡಪಮಧ್ಯಸ್ಥೋ ರತ್ನಸಿಂಹಾಸನಾಶ್ರಯಃ ॥೨೯॥

ತೀವ್ರಾಶಿರೋದ್ಧೃತಪದೋ ಜ್ವಾಲಿನೀಮೌಲಿಲಾಲಿತಃ ।\\
ನಂದಾನಂದಿತಪೀಠಶ್ರೀರ್ಭೋಗದೋ ಭೂಷಿತಾಸನಃ ॥೩೦॥

ಸಕಾಮದಾಯಿನೀಪೀಠಃ ಸ್ಫುರದುಗ್ರಾಸನಾಶ್ರಯಃ ।\\
ತೇಜೋವತೀಶಿರೋರತ್ನಂ ಸತ್ಯಾನಿತ್ಯಾವತಂಸಿತಃ ॥೩೧॥

ಸವಿಘ್ನನಾಶಿನೀಪೀಠಃ ಸರ್ವಶಕ್ತ್ಯಂಬುಜಾಲಯಃ ।\\
ಲಿಪಿಪದ್ಮಾಸನಾಧಾರೋ ವಹ್ನಿಧಾಮತ್ರಯಾಲಯಃ ॥೩೨॥

ಉನ್ನತಪ್ರಪದೋ ಗೂಢಗುಲ್ಫಃ ಸಂವೃತಪಾರ್ಷ್ಣಿಕಃ ।\\
ಪೀನಜಂಘಃ ಶ್ಲಿಷ್ಟಜಾನುಃ ಸ್ಥೂಲೋರುಃ ಪ್ರೋನ್ನಮತ್ಕಟಿಃ ॥೩೩॥

ನಿಮ್ನನಾಭಿಃ ಸ್ಥೂಲಕುಕ್ಷಿಃ ಪೀನವಕ್ಷಾ ಬೃಹದ್ಭುಜಃ ।\\
ಪೀನಸ್ಕಂಧಃ ಕಂಬುಕಂಠೋ ಲಂಬೋಷ್ಠೋ ಲಂಬನಾಸಿಕಃ ॥೩೪॥

ಭಗ್ನವಾಮರದಸ್ತುಂಗಸವ್ಯದಂತೋ ಮಹಾಹನುಃ ।\\
ಹ್ರಸ್ವನೇತ್ರತ್ರಯಃ ಶೂರ್ಪಕರ್ಣೋ ನಿಬಿಡಮಸ್ತಕಃ ॥೩೫॥

ಸ್ತಬಕಾಕಾರಕುಂಭಾಗ್ರೋ ರತ್ನಮೌಲಿರ್ನಿರಂಕುಶಃ ।\\
ಸರ್ಪಹಾರಕಟೀಸೂತ್ರಃ ಸರ್ಪಯಜ್ಞೋಪವೀತವಾನ್ ॥೩೬॥

ಸರ್ಪಕೋಟೀರಕಟಕಃ ಸರ್ಪಗ್ರೈವೇಯಕಾಂಗದಃ ।\\
ಸರ್ಪಕಕ್ಷೋದರಾಬಂಧಃ ಸರ್ಪರಾಜೋತ್ತರಚ್ಛದಃ ॥೩೭॥

ರಕ್ತೋ ರಕ್ತಾಂಬರಧರೋ ರಕ್ತಮಾಲಾವಿಭೂಷಣಃ ।\\
ರಕ್ತೇಕ್ಷಣೋ ರಕ್ತಕರೋ ರಕ್ತತಾಲ್ವೋಷ್ಠಪಲ್ಲವಃ ॥೩೮॥

ಶ್ವೇತಃ ಶ್ವೇತಾಂಬರಧರಃ ಶ್ವೇತಮಾಲಾವಿಭೂಷಣಃ ।\\
ಶ್ವೇತಾತಪತ್ರರುಚಿರಃ ಶ್ವೇತಚಾಮರವೀಜಿತಃ ॥೩೯॥

ಸರ್ವಾವಯವಸಂಪೂರ್ಣಃ ಸರ್ವಲಕ್ಷಣಲಕ್ಷಿತಃ ।\\
ಸರ್ವಾಭರಣಶೋಭಾಢ್ಯಃ ಸರ್ವಶೋಭಾಸಮನ್ವಿತಃ ॥೪೦॥

ಸರ್ವಮಂಗಲಮಾಂಗಲ್ಯಃ ಸರ್ವಕಾರಣಕಾರಣಂ ।\\
ಸರ್ವದೇವವರಃ ಶಾರ್ಙ್ಗೀ ಬೀಜಪೂರೀ ಗದಾಧರಃ ॥೪೧॥

ಶುಭಾಂಗೋ ಲೋಕಸಾರಂಗಃ ಸುತಂತುಸ್ತಂತುವರ್ಧನಃ ।\\
ಕಿರೀಟೀ ಕುಂಡಲೀ ಹಾರೀ ವನಮಾಲೀ ಶುಭಾಂಗದಃ ॥೪೨॥

ಇಕ್ಷುಚಾಪಧರಃ ಶೂಲೀ ಚಕ್ರಪಾಣಿಃ ಸರೋಜಭೃತ್ ।\\
ಪಾಶೀ ಧೃತೋತ್ಪಲಃ ಶಾಲಿಮಂಜರೀಭೃತ್ಸ್ವದಂತಭೃತ್ ॥೪೩॥

ಕಲ್ಪವಲ್ಲೀಧರೋ ವಿಶ್ವಾಭಯದೈಕಕರೋ ವಶೀ ।\\
ಅಕ್ಷಮಾಲಾಧರೋ ಜ್ಞಾನಮುದ್ರಾವಾನ್ ಮುದ್ಗರಾಯುಧಃ ॥೪೪॥

ಪೂರ್ಣಪಾತ್ರೀ ಕಂಬುಧರೋ ವಿಧೃತಾಂಕುಶಮೂಲಕಃ ।\\
ಕರಸ್ಥಾಮ್ರಫಲಶ್ಚೂತಕಲಿಕಾಭೃತ್ಕುಠಾರವಾನ್ ॥೪೫॥

ಪುಷ್ಕರಸ್ಥಸ್ವರ್ಣಘಟೀಪೂರ್ಣರತ್ನಾಭಿವರ್ಷಕಃ ।\\
ಭಾರತೀಸುಂದರೀನಾಥೋ ವಿನಾಯಕರತಿಪ್ರಿಯಃ ॥೪೬॥

ಮಹಾಲಕ್ಷ್ಮೀಪ್ರಿಯತಮಃ ಸಿದ್ಧಲಕ್ಷ್ಮೀಮನೋರಮಃ ।\\
ರಮಾರಮೇಶಪೂರ್ವಾಂಗೋ ದಕ್ಷಿಣೋಮಾಮಹೇಶ್ವರಃ ॥೪೭॥

ಮಹೀವರಾಹವಾಮಾಂಗೋ ರತಿಕಂದರ್ಪಪಶ್ಚಿಮಃ ।\\
ಆಮೋದಮೋದಜನನಃ ಸಂಪ್ರಮೋದಪ್ರಮೋದನಃ ॥೪೮॥

ಸಂವರ್ಧಿತಮಹಾವೃದ್ಧಿರೃದ್ಧಿಸಿದ್ಧಿಪ್ರವರ್ಧನಃ ।\\
ದಂತಸೌಮುಖ್ಯಸುಮುಖಃ ಕಾಂತಿಕಂದಲಿತಾಶ್ರಯಃ ॥೪೯॥

ಮದನಾವತ್ಯಾಶ್ರಿತಾಂಘ್ರಿಃ ಕೃತವೈಮುಖ್ಯದುರ್ಮುಖಃ ।\\
ವಿಘ್ನಸಂಪಲ್ಲವಃ ಪದ್ಮಃ ಸರ್ವೋನ್ನತಮದದ್ರವಃ ॥೫೦॥

ವಿಘ್ನಕೃನ್ನಿಮ್ನಚರಣೋ ದ್ರಾವಿಣೀಶಕ್ತಿಸತ್ಕೃತಃ ।\\
ತೀವ್ರಾಪ್ರಸನ್ನನಯನೋ ಜ್ವಾಲಿನೀಪಾಲಿತೈಕದೃಕ್ ॥೫೧॥

ಮೋಹಿನೀಮೋಹನೋ ಭೋಗದಾಯಿನೀಕಾಂತಿಮಂಡನಃ ।\\
ಕಾಮಿನೀಕಾಂತವಕ್ತ್ರಶ್ರೀರಧಿಷ್ಠಿತವಸುಂಧರಃ ॥೫೨॥

ವಸುಧಾರಾಮದೋನ್ನಾದೋ ಮಹಾಶಂಖನಿಧಿಪ್ರಿಯಃ ।\\
ನಮದ್ವಸುಮತೀಮಾಲೀ ಮಹಾಪದ್ಮನಿಧಿಃ ಪ್ರಭುಃ ॥೫೩॥

ಸರ್ವಸದ್ಗುರುಸಂಸೇವ್ಯಃ ಶೋಚಿಷ್ಕೇಶಹೃದಾಶ್ರಯಃ ।\\
ಈಶಾನಮೂರ್ಧಾ ದೇವೇಂದ್ರಶಿಖಃ ಪವನನಂದನಃ ॥೫೪॥

ಪ್ರತ್ಯುಗ್ರನಯನೋ ದಿವ್ಯೋ ದಿವ್ಯಾಸ್ತ್ರಶತಪರ್ವಧೃಕ್ ।\\
ಐರಾವತಾದಿಸರ್ವಾಶಾವಾರಣೋ ವಾರಣಪ್ರಿಯಃ ॥೫೫॥

ವಜ್ರಾದ್ಯಸ್ತ್ರಪರೀವಾರೋ ಗಣಚಂಡಸಮಾಶ್ರಯಃ ।\\
ಜಯಾಜಯಪರಿಕರೋ ವಿಜಯಾವಿಜಯಾವಹಃ ॥೫೬॥

ಅಜಯಾರ್ಚಿತಪಾದಾಬ್ಜೋ ನಿತ್ಯಾನಂದವನಸ್ಥಿತಃ ।\\
ವಿಲಾಸಿನೀಕೃತೋಲ್ಲಾಸಃ ಶೌಂಡೀ ಸೌಂದರ್ಯಮಂಡಿತಃ ॥೫೭॥

ಅನಂತಾನಂತಸುಖದಃ ಸುಮಂಗಲಸುಮಂಗಲಃ ।\\
ಜ್ಞಾನಾಶ್ರಯಃ ಕ್ರಿಯಾಧಾರ ಇಚ್ಛಾಶಕ್ತಿನಿಷೇವಿತಃ ॥೫೮॥

ಸುಭಗಾಸಂಶ್ರಿತಪದೋ ಲಲಿತಾಲಲಿತಾಶ್ರಯಃ ।\\
ಕಾಮಿನೀಪಾಲನಃ ಕಾಮಕಾಮಿನೀಕೇಲಿಲಾಲಿತಃ ॥೫೯॥

ಸರಸ್ವತ್ಯಾಶ್ರಯೋ ಗೌರೀನಂದನಃ ಶ್ರೀನಿಕೇತನಃ ।\\
ಗುರುಗುಪ್ತಪದೋ ವಾಚಾಸಿದ್ಧೋ ವಾಗೀಶ್ವರೀಪತಿಃ ॥೬೦॥

ನಲಿನೀಕಾಮುಕೋ ವಾಮಾರಾಮೋ ಜ್ಯೇಷ್ಠಾಮನೋರಮಃ ।\\
ರೌದ್ರೀಮುದ್ರಿತಪಾದಾಬ್ಜೋ ಹುಂಬೀಜಸ್ತುಂಗಶಕ್ತಿಕಃ ॥೬೧॥

ವಿಶ್ವಾದಿಜನನತ್ರಾಣಃ ಸ್ವಾಹಾಶಕ್ತಿಃ ಸಕೀಲಕಃ ।\\
ಅಮೃತಾಬ್ಧಿಕೃತಾವಾಸೋ ಮದಘೂರ್ಣಿತಲೋಚನಃ ॥೬೨॥

ಉಚ್ಛಿಷ್ಟೋಚ್ಛಿಷ್ಟಗಣಕೋ ಗಣೇಶೋ ಗಣನಾಯಕಃ ।\\
ಸಾರ್ವಕಾಲಿಕಸಂಸಿದ್ಧಿರ್ನಿತ್ಯಸೇವ್ಯೋ ದಿಗಂಬರಃ ॥೬೩॥

ಅನಪಾಯೋಽನಂತದೃಷ್ಟಿರಪ್ರಮೇಯೋಽಜರಾಮರಃ ।\\
ಅನಾವಿಲೋಽಪ್ರತಿಹತಿರಚ್ಯುತೋಽಮೃತಮಕ್ಷರಃ ॥೬೪॥

ಅಪ್ರತರ್ಕ್ಯೋಽಕ್ಷಯೋಽಜಯ್ಯೋಽನಾಧಾರೋಽನಾಮಯೋಽಮಲಃ ।\\
ಅಮೇಯಸಿದ್ಧಿರದ್ವೈತಮಘೋರೋಽಗ್ನಿಸಮಾನನಃ ॥೬೫॥

ಅನಾಕಾರೋಽಬ್ಧಿಭೂಮ್ಯಗ್ನಿಬಲಘ್ನೋಽವ್ಯಕ್ತಲಕ್ಷಣಃ ।\\
ಆಧಾರಪೀಠಮಾಧಾರ ಆಧಾರಾಧೇಯವರ್ಜಿತಃ ॥೬೬॥

ಆಖುಕೇತನ ಆಶಾಪೂರಕ ಆಖುಮಹಾರಥಃ ।\\
ಇಕ್ಷುಸಾಗರಮಧ್ಯಸ್ಥ ಇಕ್ಷುಭಕ್ಷಣಲಾಲಸಃ ॥೬೭॥

ಇಕ್ಷುಚಾಪಾತಿರೇಕಶ್ರೀರಿಕ್ಷುಚಾಪನಿಷೇವಿತಃ ।\\
ಇಂದ್ರಗೋಪಸಮಾನಶ್ರೀರಿಂದ್ರನೀಲಸಮದ್ಯುತಿಃ ॥೬೮॥

ಇಂದೀವರದಲಶ್ಯಾಮ ಇಂದುಮಂಡಲಮಂಡಿತಃ ।\\
ಇಧ್ಮಪ್ರಿಯ ಇಡಾಭಾಗ ಇಡಾವಾನಿಂದಿರಾಪ್ರಿಯಃ ॥೬೯॥

ಇಕ್ಷ್ವಾಕುವಿಘ್ನವಿಧ್ವಂಸೀ ಇತಿಕರ್ತವ್ಯತೇಪ್ಸಿತಃ ।\\
ಈಶಾನಮೌಲಿರೀಶಾನ ಈಶಾನಪ್ರಿಯ ಈತಿಹಾ ॥೭೦॥

ಈಷಣಾತ್ರಯಕಲ್ಪಾಂತ ಈಹಾಮಾತ್ರವಿವರ್ಜಿತಃ ।\\
ಉಪೇಂದ್ರ ಉಡುಭೃನ್ಮೌಲಿರುಡುನಾಥಕರಪ್ರಿಯಃ ॥೭೧॥

ಉನ್ನತಾನನ ಉತ್ತುಂಗ ಉದಾರಸ್ತ್ರಿದಶಾಗ್ರಣೀಃ ।\\
ಊರ್ಜಸ್ವಾನೂಷ್ಮಲಮದ ಊಹಾಪೋಹದುರಾಸದಃ ॥೭೨॥

ಋಗ್ಯಜುಃಸಾಮನಯನ ಋದ್ಧಿಸಿದ್ಧಿಸಮರ್ಪಕಃ ।\\
ಋಜುಚಿತ್ತೈಕಸುಲಭೋ ಋಣತ್ರಯವಿಮೋಚನಃ ॥೭೩॥

ಲುಪ್ತವಿಘ್ನಃ ಸ್ವಭಕ್ತಾನಾಂ ಲುಪ್ತಶಕ್ತಿಃ ಸುರದ್ವಿಷಾಂ ।\\
ಲುಪ್ತಶ್ರೀರ್ವಿಮುಖಾರ್ಚಾನಾಂ ಲೂತಾವಿಸ್ಫೋಟನಾಶನಃ ॥೭೪॥

ಏಕಾರಪೀಠಮಧ್ಯಸ್ಥ ಏಕಪಾದಕೃತಾಸನಃ ।\\
ಏಜಿತಾಖಿಲದೈತ್ಯಶ್ರೀರೇಧಿತಾಖಿಲಸಂಶ್ರಯಃ ॥೭೫॥

ಐಶ್ವರ್ಯನಿಧಿರೈಶ್ವರ್ಯಮೈಹಿಕಾಮುಷ್ಮಿಕಪ್ರದಃ ।\\
ಐರಂಮದಸಮೋನ್ಮೇಷ ಐರಾವತಸಮಾನನಃ ॥೭೬॥

ಓಂಕಾರವಾಚ್ಯ ಓಂಕಾರ ಓಜಸ್ವಾನೋಷಧೀಪತಿಃ ।\\
ಔದಾರ್ಯನಿಧಿರೌದ್ಧತ್ಯಧೈರ್ಯ ಔನ್ನತ್ಯನಿಃಸಮಃ ॥೭೭॥

ಅಂಕುಶಃ ಸುರನಾಗಾನಾಮಂಕುಶಾಕಾರಸಂಸ್ಥಿತಃ ।\\
ಅಃ ಸಮಸ್ತವಿಸರ್ಗಾಂತಪದೇಷು ಪರಿಕೀರ್ತಿತಃ ॥೭೮॥

ಕಮಂಡಲುಧರಃ ಕಲ್ಪಃ ಕಪರ್ದೀ ಕಲಭಾನನಃ ।\\
ಕರ್ಮಸಾಕ್ಷೀ ಕರ್ಮಕರ್ತಾ ಕರ್ಮಾಕರ್ಮಫಲಪ್ರದಃ ॥೭೯॥

ಕದಂಬಗೋಲಕಾಕಾರಃ ಕೂಷ್ಮಾಂಡಗಣನಾಯಕಃ ।\\
ಕಾರುಣ್ಯದೇಹಃ ಕಪಿಲಃ ಕಥಕಃ ಕಟಿಸೂತ್ರಭೃತ್ ॥೮೦॥

ಖರ್ವಃ ಖಡ್ಗಪ್ರಿಯಃ ಖಡ್ಗಃ ಖಾಂತಾಂತಃಸ್ಥಃ ಖನಿರ್ಮಲಃ ।\\
ಖಲ್ವಾಟಶೃಂಗನಿಲಯಃ ಖಟ್ವಾಂಗೀ ಖದುರಾಸದಃ ॥೮೧॥

ಗುಣಾಢ್ಯೋ ಗಹನೋ ಗದ್ಯೋ ಗದ್ಯಪದ್ಯಸುಧಾರ್ಣವಃ ।\\
ಗದ್ಯಗಾನಪ್ರಿಯೋ ಗರ್ಜೋ ಗೀತಗೀರ್ವಾಣಪೂರ್ವಜಃ ॥೮೨॥

ಗುಹ್ಯಾಚಾರರತೋ ಗುಹ್ಯೋ ಗುಹ್ಯಾಗಮನಿರೂಪಿತಃ ।\\
ಗುಹಾಶಯೋ ಗುಡಾಬ್ಧಿಸ್ಥೋ ಗುರುಗಮ್ಯೋ ಗುರುರ್ಗುರುಃ ॥೮೩॥

ಘಂಟಾಘರ್ಘರಿಕಾಮಾಲೀ ಘಟಕುಂಭೋ ಘಟೋದರಃ ।\\
ಙಕಾರವಾಚ್ಯೋ ಙಾಕಾರೋ ಙಕಾರಾಕಾರಶುಂಡಭೃತ್ ॥೮೪॥

ಚಂಡಶ್ಚಂಡೇಶ್ವರಶ್ಚಂಡೀ ಚಂಡೇಶಶ್ಚಂಡವಿಕ್ರಮಃ ।\\
ಚರಾಚರಪಿತಾ ಚಿಂತಾಮಣಿಶ್ಚರ್ವಣಲಾಲಸಃ ॥೮೫॥

ಛಂದಶ್ಛಂದೋದ್ಭವಶ್ಛಂದೋ ದುರ್ಲಕ್ಷ್ಯಶ್ಛಂದವಿಗ್ರಹಃ ।\\
ಜಗದ್ಯೋನಿರ್ಜಗತ್ಸಾಕ್ಷೀ ಜಗದೀಶೋ ಜಗನ್ಮಯಃ ॥೮೬॥

ಜಪ್ಯೋ ಜಪಪರೋ ಜಾಪ್ಯೋ ಜಿಹ್ವಾಸಿಂಹಾಸನಪ್ರಭುಃ ।\\
ಸ್ರವದ್ಗಂಡೋಲ್ಲಸದ್ಧಾನಝಂಕಾರಿಭ್ರಮರಾಕುಲಃ ॥೮೭॥

ಟಂಕಾರಸ್ಫಾರಸಂರಾವಷ್ಟಂಕಾರಮಣಿನೂಪುರಃ ।\\
ಠದ್ವಯೀಪಲ್ಲವಾಂತಸ್ಥಸರ್ವಮಂತ್ರೇಷು ಸಿದ್ಧಿದಃ ॥೮೮॥

ಡಿಂಡಿಮುಂಡೋ ಡಾಕಿನೀಶೋ ಡಾಮರೋ ಡಿಂಡಿಮಪ್ರಿಯಃ ।\\
ಢಕ್ಕಾನಿನಾದಮುದಿತೋ ಢೌಂಕೋ ಢುಂಢಿವಿನಾಯಕಃ ॥೮೯॥

ತತ್ತ್ವಾನಾಂ ಪ್ರಕೃತಿಸ್ತತ್ತ್ವಂ ತತ್ತ್ವಂಪದನಿರೂಪಿತಃ ।\\
ತಾರಕಾಂತರಸಂಸ್ಥಾನಸ್ತಾರಕಸ್ತಾರಕಾಂತಕಃ ॥೯೦॥

ಸ್ಥಾಣುಃ ಸ್ಥಾಣುಪ್ರಿಯಃ ಸ್ಥಾತಾ ಸ್ಥಾವರಂ ಜಂಗಮಂ ಜಗತ್ ।\\
ದಕ್ಷಯಜ್ಞಪ್ರಮಥನೋ ದಾತಾ ದಾನಂ ದಮೋ ದಯಾ ॥೯೧॥

ದಯಾವಾಂದಿವ್ಯವಿಭವೋ ದಂಡಭೃದ್ದಂಡನಾಯಕಃ ।\\
ದಂತಪ್ರಭಿನ್ನಾಭ್ರಮಾಲೋ ದೈತ್ಯವಾರಣದಾರಣಃ ॥೯೨॥

ದಂಷ್ಟ್ರಾಲಗ್ನದ್ವೀಪಘಟೋ ದೇವಾರ್ಥನೃಗಜಾಕೃತಿಃ ।\\
ಧನಂ ಧನಪತೇರ್ಬಂಧುರ್ಧನದೋ ಧರಣೀಧರಃ ॥೯೩॥

ಧ್ಯಾನೈಕಪ್ರಕಟೋ ಧ್ಯೇಯೋ ಧ್ಯಾನಂ ಧ್ಯಾನಪರಾಯಣಃ ।\\
ಧ್ವನಿಪ್ರಕೃತಿಚೀತ್ಕಾರೋ ಬ್ರಹ್ಮಾಂಡಾವಲಿಮೇಖಲಃ ॥೯೪॥

ನಂದ್ಯೋ ನಂದಿಪ್ರಿಯೋ ನಾದೋ ನಾದಮಧ್ಯಪ್ರತಿಷ್ಠಿತಃ ।\\
ನಿಷ್ಕಲೋ ನಿರ್ಮಲೋ ನಿತ್ಯೋ ನಿತ್ಯಾನಿತ್ಯೋ ನಿರಾಮಯಃ ॥೯೫॥

ಪರಂ ವ್ಯೋಮ ಪರಂ ಧಾಮ ಪರಮಾತ್ಮಾ ಪರಂ ಪದಂ ॥೯೬॥

ಪರಾತ್ಪರಃ ಪಶುಪತಿಃ ಪಶುಪಾಶವಿಮೋಚನಃ ।\\
ಪೂರ್ಣಾನಂದಃ ಪರಾನಂದಃ ಪುರಾಣಪುರುಷೋತ್ತಮಃ ॥೯೭॥

ಪದ್ಮಪ್ರಸನ್ನವದನಃ ಪ್ರಣತಾಜ್ಞಾನನಾಶನಃ ।\\
ಪ್ರಮಾಣಪ್ರತ್ಯಯಾತೀತಃ ಪ್ರಣತಾರ್ತಿನಿವಾರಣಃ ॥೯೮॥

ಫಣಿಹಸ್ತಃ ಫಣಿಪತಿಃ ಫೂತ್ಕಾರಃ ಫಣಿತಪ್ರಿಯಃ ।\\
ಬಾಣಾರ್ಚಿತಾಂಘ್ರಿಯುಗಲೋ ಬಾಲಕೇಲಿಕುತೂಹಲೀ ।\\
ಬ್ರಹ್ಮ ಬ್ರಹ್ಮಾರ್ಚಿತಪದೋ ಬ್ರಹ್ಮಚಾರೀ ಬೃಹಸ್ಪತಿಃ ॥೯೯॥

ಬೃಹತ್ತಮೋ ಬ್ರಹ್ಮಪರೋ ಬ್ರಹ್ಮಣ್ಯೋ ಬ್ರಹ್ಮವಿತ್ಪ್ರಿಯಃ ।\\
ಬೃಹನ್ನಾದಾಗ್ರ್ಯಚೀತ್ಕಾರೋ ಬ್ರಹ್ಮಾಂಡಾವಲಿಮೇಖಲಃ ॥೧೦೦॥

ಭ್ರೂಕ್ಷೇಪದತ್ತಲಕ್ಷ್ಮೀಕೋ ಭರ್ಗೋ ಭದ್ರೋ ಭಯಾಪಹಃ ।\\
ಭಗವಾನ್ ಭಕ್ತಿಸುಲಭೋ ಭೂತಿದೋ ಭೂತಿಭೂಷಣಃ ॥೧೦೧॥

ಭವ್ಯೋ ಭೂತಾಲಯೋ ಭೋಗದಾತಾ ಭ್ರೂಮಧ್ಯಗೋಚರಃ ।\\
ಮಂತ್ರೋ ಮಂತ್ರಪತಿರ್ಮಂತ್ರೀ ಮದಮತ್ತೋ ಮನೋ ಮಯಃ ॥೧೦೨॥

ಮೇಖಲಾಹೀಶ್ವರೋ ಮಂದಗತಿರ್ಮಂದನಿಭೇಕ್ಷಣಃ ।\\
ಮಹಾಬಲೋ ಮಹಾವೀರ್ಯೋ ಮಹಾಪ್ರಾಣೋ ಮಹಾಮನಾಃ ॥೧೦೩॥

ಯಜ್ಞೋ ಯಜ್ಞಪತಿರ್ಯಜ್ಞಗೋಪ್ತಾ ಯಜ್ಞಫಲಪ್ರದಃ ।\\
ಯಶಸ್ಕರೋ ಯೋಗಗಮ್ಯೋ ಯಾಜ್ಞಿಕೋ ಯಾಜಕಪ್ರಿಯಃ ॥೧೦೪॥

ರಸೋ ರಸಪ್ರಿಯೋ ರಸ್ಯೋ ರಂಜಕೋ ರಾವಣಾರ್ಚಿತಃ ।\\
ರಾಜ್ಯರಕ್ಷಾಕರೋ ರತ್ನಗರ್ಭೋ ರಾಜ್ಯಸುಖಪ್ರದಃ ॥೧೦೫॥

ಲಕ್ಷೋ ಲಕ್ಷಪತಿರ್ಲಕ್ಷ್ಯೋ ಲಯಸ್ಥೋ ಲಡ್ಡುಕಪ್ರಿಯಃ ।\\
ಲಾಸಪ್ರಿಯೋ ಲಾಸ್ಯಪರೋ ಲಾಭಕೃಲ್ಲೋಕವಿಶ್ರುತಃ ॥೧೦೬॥

ವರೇಣ್ಯೋ ವಹ್ನಿವದನೋ ವಂದ್ಯೋ ವೇದಾಂತಗೋಚರಃ ।\\
ವಿಕರ್ತಾ ವಿಶ್ವತಶ್ಚಕ್ಷುರ್ವಿಧಾತಾ ವಿಶ್ವತೋಮುಖಃ ॥೧೦೭॥

ವಾಮದೇವೋ ವಿಶ್ವನೇತಾ ವಜ್ರಿವಜ್ರನಿವಾರಣಃ ।\\
ವಿವಸ್ವದ್ಬಂಧನೋ ವಿಶ್ವಾಧಾರೋ ವಿಶ್ವೇಶ್ವರೋ ವಿಭುಃ ॥೧೦೮॥

ಶಬ್ದಬ್ರಹ್ಮ ಶಮಪ್ರಾಪ್ಯಃ ಶಂಭುಶಕ್ತಿಗಣೇಶ್ವರಃ ।\\
ಶಾಸ್ತಾ ಶಿಖಾಗ್ರನಿಲಯಃ ಶರಣ್ಯಃ ಶಂಬರೇಶ್ವರಃ ॥೧೦೯॥

ಷಡೃತುಕುಸುಮಸ್ರಗ್ವೀ ಷಡಾಧಾರಃ ಷಡಕ್ಷರಃ ।\\
ಸಂಸಾರವೈದ್ಯಃ ಸರ್ವಜ್ಞಃ ಸರ್ವಭೇಷಜಭೇಷಜಂ ॥೧೧೦॥

ಸೃಷ್ಟಿಸ್ಥಿತಿಲಯಕ್ರೀಡಃ ಸುರಕುಂಜರಭೇದಕಃ ।\\
ಸಿಂದೂರಿತಮಹಾಕುಂಭಃ ಸದಸದ್ಭಕ್ತಿದಾಯಕಃ ॥೧೧೧॥

ಸಾಕ್ಷೀ ಸಮುದ್ರಮಥನಃ ಸ್ವಯಂವೇದ್ಯಃ ಸ್ವದಕ್ಷಿಣಃ ।\\
ಸ್ವತಂತ್ರಃ ಸತ್ಯಸಂಕಲ್ಪಃ ಸಾಮಗಾನರತಃ ಸುಖೀ ॥೧೧೨॥

ಹಂಸೋ ಹಸ್ತಿಪಿಶಾಚೀಶೋ ಹವನಂ ಹವ್ಯಕವ್ಯಭುಕ್ ।\\
ಹವ್ಯಂ ಹುತಪ್ರಿಯೋ ಹೃಷ್ಟೋ ಹೃಲ್ಲೇಖಾಮಂತ್ರಮಧ್ಯಗಃ ॥೧೧೩॥

ಕ್ಷೇತ್ರಾಧಿಪಃ ಕ್ಷಮಾಭರ್ತಾ ಕ್ಷಮಾಕ್ಷಮಪರಾಯಣಃ ।\\
ಕ್ಷಿಪ್ರಕ್ಷೇಮಕರಃ ಕ್ಷೇಮಾನಂದಃ ಕ್ಷೋಣೀಸುರದ್ರುಮಃ ॥೧೧೪॥

ಧರ್ಮಪ್ರದೋಽರ್ಥದಃ ಕಾಮದಾತಾ ಸೌಭಾಗ್ಯವರ್ಧನಃ ।\\
ವಿದ್ಯಾಪ್ರದೋ ವಿಭವದೋ ಭುಕ್ತಿಮುಕ್ತಿಫಲಪ್ರದಃ ॥೧೧೫॥

ಆಭಿರೂಪ್ಯಕರೋ ವೀರಶ್ರೀಪ್ರದೋ ವಿಜಯಪ್ರದಃ ।\\
ಸರ್ವವಶ್ಯಕರೋ ಗರ್ಭದೋಷಹಾ ಪುತ್ರಪೌತ್ರದಃ ॥೧೧೬॥

ಮೇಧಾದಃ ಕೀರ್ತಿದಃ ಶೋಕಹಾರೀ ದೌರ್ಭಾಗ್ಯನಾಶನಃ ।\\
ಪ್ರತಿವಾದಿಮುಖಸ್ತಂಭೋ ರುಷ್ಟಚಿತ್ತಪ್ರಸಾದನಃ ॥೧೧೭॥

ಪರಾಭಿಚಾರಶಮನೋ ದುಃಖಹಾ ಬಂಧಮೋಕ್ಷದಃ ।\\
ಲವಸ್ತ್ರುಟಿಃ ಕಲಾ ಕಾಷ್ಠಾ ನಿಮೇಷಸ್ತತ್ಪರಕ್ಷಣಃ ॥೧೧೮॥

ಘಟೀ ಮುಹೂರ್ತಃ ಪ್ರಹರೋ ದಿವಾ ನಕ್ತಮಹರ್ನಿಶಂ ।\\
ಪಕ್ಷೋ ಮಾಸರ್ತ್ವಯನಾಬ್ದಯುಗಂ ಕಲ್ಪೋ ಮಹಾಲಯಃ ॥೧೧೯॥

ರಾಶಿಸ್ತಾರಾ ತಿಥಿರ್ಯೋಗೋ ವಾರಃ ಕರಣಮಂಶಕಂ ।\\
ಲಗ್ನಂ ಹೋರಾ ಕಾಲಚಕ್ರಂ ಮೇರುಃ ಸಪ್ತರ್ಷಯೋ ಧ್ರುವಃ ॥೧೨೦॥

ರಾಹುರ್ಮಂದಃ ಕವಿರ್ಜೀವೋ ಬುಧೋ ಭೌಮಃ ಶಶೀ ರವಿಃ ।\\
ಕಾಲಃ ಸೃಷ್ಟಿಃ ಸ್ಥಿತಿರ್ವಿಶ್ವಂ ಸ್ಥಾವರಂ ಜಂಗಮಂ ಜಗತ್ ॥೧೨೧॥

ಭೂರಾಪೋಽಗ್ನಿರ್ಮರುದ್ವ್ಯೋಮಾಹಂಕೃತಿಃ ಪ್ರಕೃತಿಃ ಪುಮಾನ್ ।\\
ಬ್ರಹ್ಮಾ ವಿಷ್ಣುಃ ಶಿವೋ ರುದ್ರ ಈಶಃ ಶಕ್ತಿಃ ಸದಾಶಿವಃ ॥೧೨೨॥

ತ್ರಿದಶಾಃ ಪಿತರಃ ಸಿದ್ಧಾ ಯಕ್ಷಾ ರಕ್ಷಾಂಸಿ ಕಿನ್ನರಾಃ ।\\
ಸಿದ್ಧವಿದ್ಯಾಧರಾ ಭೂತಾ ಮನುಷ್ಯಾಃ ಪಶವಃ ಖಗಾಃ ॥೧೨೩॥

ಸಮುದ್ರಾಃ ಸರಿತಃ ಶೈಲಾ ಭೂತಂ ಭವ್ಯಂ ಭವೋದ್ಭವಃ ।\\
ಸಾಂಖ್ಯಂ ಪಾತಂಜಲಂ ಯೋಗಂ ಪುರಾಣಾನಿ ಶ್ರುತಿಃ ಸ್ಮೃತಿಃ ॥೧೨೪॥

ವೇದಾಂಗಾನಿ ಸದಾಚಾರೋ ಮೀಮಾಂಸಾ ನ್ಯಾಯವಿಸ್ತರಃ ।\\
ಆಯುರ್ವೇದೋ ಧನುರ್ವೇದೋ ಗಾಂಧರ್ವಂ ಕಾವ್ಯನಾಟಕಂ ॥೧೨೫॥

ವೈಖಾನಸಂ ಭಾಗವತಂ ಮಾನುಷಂ ಪಾಂಚರಾತ್ರಕಂ ।\\
ಶೈವಂ ಪಾಶುಪತಂ ಕಾಲಾಮುಖಂಭೈರವಶಾಸನಂ ॥೧೨೬॥

ಶಾಕ್ತಂ ವೈನಾಯಕಂ ಸೌರಂ ಜೈನಮಾರ್ಹತಸಂಹಿತಾ ।\\
ಸದಸದ್ವ್ಯಕ್ತಮವ್ಯಕ್ತಂ ಸಚೇತನಮಚೇತನಂ ॥೧೨೭॥

ಬಂಧೋ ಮೋಕ್ಷಃ ಸುಖಂ ಭೋಗೋ ಯೋಗಃ ಸತ್ಯಮಣುರ್ಮಹಾನ್ ।\\
ಸ್ವಸ್ತಿ ಹುಂ ಫಟ್ ಸ್ವಧಾ ಸ್ವಾಹಾ ಶ್ರೌಷಡ್ ವೌಷಡ್ ವಷಣ್ಣಮಃ ॥೧೨೮॥

ಜ್ಞಾನಂ ವಿಜ್ಞಾನಮಾನಂದೋ ಬೋಧಃ ಸಂವಿತ್ಸಮೋಽಸಮಃ ।\\
ಏಕ ಏಕಾಕ್ಷರಾಧಾರ ಏಕಾಕ್ಷರಪರಾಯಣಃ ॥೧೨೯॥

ಏಕಾಗ್ರಧೀರೇಕವೀರ ಏಕೋಽನೇಕಸ್ವರೂಪಧೃಕ್ ।\\
ದ್ವಿರೂಪೋ ದ್ವಿಭುಜೋ ದ್ವ್ಯಕ್ಷೋ ದ್ವಿರದೋ ದ್ವೀಪರಕ್ಷಕಃ ॥೧೩೦॥

ದ್ವೈಮಾತುರೋ ದ್ವಿವದನೋ ದ್ವಂದ್ವಹೀನೋ ದ್ವಯಾತಿಗಃ ।\\
ತ್ರಿಧಾಮಾ ತ್ರಿಕರಸ್ತ್ರೇತಾ ತ್ರಿವರ್ಗಫಲದಾಯಕಃ ॥೧೩೧॥

ತ್ರಿಗುಣಾತ್ಮಾ ತ್ರಿಲೋಕಾದಿಸ್ತ್ರಿಶಕ್ತೀಶಸ್ತ್ರಿಲೋಚನಃ ।\\
ಚತುರ್ವಿಧವಚೋವೃತ್ತಿಪರಿವೃತ್ತಿಪ್ರವರ್ತಕಃ ॥೧೩೨॥

ಚತುರ್ಬಾಹುಶ್ಚತುರ್ದಂತಶ್ಚತುರಾತ್ಮಾ ಚತುರ್ಭುಜಃ ।\\
ಚತುರ್ವಿಧೋಪಾಯಮಯಶ್ಚತುರ್ವರ್ಣಾಶ್ರಮಾಶ್ರಯಃ ।\\
ಚತುರ್ಥೀಪೂಜನಪ್ರೀತಶ್ಚತುರ್ಥೀತಿಥಿಸಂಭವಃ ॥೧೩೩॥

ಪಂಚಾಕ್ಷರಾತ್ಮಾ ಪಂಚಾತ್ಮಾ ಪಂಚಾಸ್ಯಃ ಪಂಚಕೃತ್ತಮಃ ॥೧೩೪॥

ಪಂಚಾಧಾರಃ ಪಂಚವರ್ಣಃ ಪಂಚಾಕ್ಷರಪರಾಯಣಃ ।\\
ಪಂಚತಾಲಃ ಪಂಚಕರಃ ಪಂಚಪ್ರಣವಮಾತೃಕಃ ॥೧೩೫॥

ಪಂಚಬ್ರಹ್ಮಮಯಸ್ಫೂರ್ತಿಃ ಪಂಚಾವರಣವಾರಿತಃ ।\\
ಪಂಚಭಕ್ಷಪ್ರಿಯಃ ಪಂಚಬಾಣಃ ಪಂಚಶಿಖಾತ್ಮಕಃ ॥೧೩೬॥

ಷಟ್ಕೋಣಪೀಠಃ ಷಟ್ಚಕ್ರಧಾಮಾ ಷಡ್ಗ್ರಂಥಿಭೇದಕಃ ।\\
ಷಡಂಗಧ್ವಾಂತವಿಧ್ವಂಸೀ ಷಡಂಗುಲಮಹಾಹ್ರದಃ ॥೧೩೭॥

ಷಣ್ಮುಖಃ ಷಣ್ಮುಖಭ್ರಾತಾ ಷಟ್ಶಕ್ತಿಪರಿವಾರಿತಃ ।\\
ಷಡ್ವೈರಿವರ್ಗವಿಧ್ವಂಸೀ ಷಡೂರ್ಮಿಭಯಭಂಜನಃ ॥೧೩೮॥

ಷಟ್ತರ್ಕದೂರಃ ಷಟ್ಕರ್ಮಾ ಷಡ್ಗುಣಃ ಷಡ್ರಸಾಶ್ರಯಃ ।\\
ಸಪ್ತಪಾತಾಲಚರಣಃ ಸಪ್ತದ್ವೀಪೋರುಮಂಡಲಃ ॥೧೩೯॥

ಸಪ್ತಸ್ವರ್ಲೋಕಮುಕುಟಃ ಸಪ್ತಸಪ್ತಿವರಪ್ರದಃ ।\\
ಸಪ್ತಾಂಗರಾಜ್ಯಸುಖದಃ ಸಪ್ತರ್ಷಿಗಣವಂದಿತಃ ॥೧೪೦॥

ಸಪ್ತಚ್ಛಂದೋನಿಧಿಃ ಸಪ್ತಹೋತ್ರಃ ಸಪ್ತಸ್ವರಾಶ್ರಯಃ ।\\
ಸಪ್ತಾಬ್ಧಿಕೇಲಿಕಾಸಾರಃ ಸಪ್ತಮಾತೃನಿಷೇವಿತಃ ॥೧೪೧॥

ಸಪ್ತಚ್ಛಂದೋ ಮೋದಮದಃ ಸಪ್ತಚ್ಛಂದೋ ಮಖಪ್ರಭುಃ ।\\
ಅಷ್ಟಮೂರ್ತಿರ್ಧ್ಯೇಯಮೂರ್ತಿರಷ್ಟಪ್ರಕೃತಿಕಾರಣಂ ॥೧೪೨॥

ಅಷ್ಟಾಂಗಯೋಗಫಲಭೃದಷ್ಟಪತ್ರಾಂಬುಜಾಸನಃ ।\\
ಅಷ್ಟಶಕ್ತಿಸಮಾನಶ್ರೀರಷ್ಟೈಶ್ವರ್ಯಪ್ರವರ್ಧನಃ ॥೧೪೩॥

ಅಷ್ಟಪೀಠೋಪಪೀಠಶ್ರೀರಷ್ಟಮಾತೃಸಮಾವೃತಃ ।\\
ಅಷ್ಟಭೈರವಸೇವ್ಯೋಽಷ್ಟವಸುವಂದ್ಯೋಽಷ್ಟಮೂರ್ತಿಭೃತ್ ॥೧೪೪॥

ಅಷ್ಟಚಕ್ರಸ್ಫುರನ್ಮೂರ್ತಿರಷ್ಟದ್ರವ್ಯಹವಿಃಪ್ರಿಯಃ ।\\
ಅಷ್ಟಶ್ರೀರಷ್ಟಸಾಮಶ್ರೀರಷ್ಟೈಶ್ವರ್ಯಪ್ರದಾಯಕಃ ।\\
ನವನಾಗಾಸನಾಧ್ಯಾಸೀ ನವನಿಧ್ಯನುಶಾಸಿತಃ ॥೧೪೫॥

ನವದ್ವಾರಪುರಾವೃತ್ತೋ ನವದ್ವಾರನಿಕೇತನಃ ।\\
ನವನಾಥಮಹಾನಾಥೋ ನವನಾಗವಿಭೂಷಿತಃ ॥೧೪೬॥

ನವನಾರಾಯಣಸ್ತುಲ್ಯೋ ನವದುರ್ಗಾನಿಷೇವಿತಃ ।\\
ನವರತ್ನವಿಚಿತ್ರಾಂಗೋ ನವಶಕ್ತಿಶಿರೋದ್ಧೃತಃ ॥೧೪೭॥

ದಶಾತ್ಮಕೋ ದಶಭುಜೋ ದಶದಿಕ್ಪತಿವಂದಿತಃ ।\\
ದಶಾಧ್ಯಾಯೋ ದಶಪ್ರಾಣೋ ದಶೇಂದ್ರಿಯನಿಯಾಮಕಃ ॥೧೪೮॥

ದಶಾಕ್ಷರಮಹಾಮಂತ್ರೋ ದಶಾಶಾವ್ಯಾಪಿವಿಗ್ರಹಃ ।\\
ಏಕಾದಶಮಹಾರುದ್ರೈಃಸ್ತುತಶ್ಚೈಕಾದಶಾಕ್ಷರಃ ॥೧೪೯॥

ದ್ವಾದಶದ್ವಿದಶಾಷ್ಟಾದಿದೋರ್ದಂಡಾಸ್ತ್ರನಿಕೇತನಃ ।\\
ತ್ರಯೋದಶಭಿದಾಭಿನ್ನೋ ವಿಶ್ವೇದೇವಾಧಿದೈವತಂ ॥೧೫೦॥

ಚತುರ್ದಶೇಂದ್ರವರದಶ್ಚತುರ್ದಶಮನುಪ್ರಭುಃ ।\\
ಚತುರ್ದಶಾದ್ಯವಿದ್ಯಾಢ್ಯಶ್ಚತುರ್ದಶಜಗತ್ಪತಿಃ ॥೧೫೧॥

ಸಾಮಪಂಚದಶಃ ಪಂಚದಶೀಶೀತಾಂಶುನಿರ್ಮಲಃ ।\\
ತಿಥಿಪಂಚದಶಾಕಾರಸ್ತಿಥ್ಯಾ ಪಂಚದಶಾರ್ಚಿತಃ ॥೧೫೨॥

ಷೋಡಶಾಧಾರನಿಲಯಃ ಷೋಡಶಸ್ವರಮಾತೃಕಃ ।\\
ಷೋಡಶಾಂತಪದಾವಾಸಃ ಷೋಡಶೇಂದುಕಲಾತ್ಮಕಃ ॥೧೫೩॥

ಕಲಾಸಪ್ತದಶೀ ಸಪ್ತದಶಸಪ್ತದಶಾಕ್ಷರಃ ।\\
ಅಷ್ಟಾದಶದ್ವೀಪಪತಿರಷ್ಟಾದಶಪುರಾಣಕೃತ್ ॥೧೫೪॥

ಅಷ್ಟಾದಶೌಷಧೀಸೃಷ್ಟಿರಷ್ಟಾದಶವಿಧಿಃ ಸ್ಮೃತಃ ।\\
ಅಷ್ಟಾದಶಲಿಪಿವ್ಯಷ್ಟಿಸಮಷ್ಟಿಜ್ಞಾನಕೋವಿದಃ ॥೧೫೫॥

ಅಷ್ಟಾದಶಾನ್ನಸಂಪತ್ತಿರಷ್ಟಾದಶವಿಜಾತಿಕೃತ್ ।\\
ಏಕವಿಂಶಃ ಪುಮಾನೇಕವಿಂಶತ್ಯಂಗುಲಿಪಲ್ಲವಃ ॥೧೫೬॥

ಚತುರ್ವಿಂಶತಿತತ್ತ್ವಾತ್ಮಾ ಪಂಚವಿಂಶಾಖ್ಯಪೂರುಷಃ ।\\
ಸಪ್ತವಿಂಶತಿತಾರೇಶಃ ಸಪ್ತವಿಂಶತಿಯೋಗಕೃತ್ ॥೧೫೭॥

ದ್ವಾತ್ರಿಂಶದ್ಭೈರವಾಧೀಶಶ್ಚತುಸ್ತ್ರಿಂಶನ್ಮಹಾಹ್ರದಃ ।\\
ಷಟ್ತ್ರಿಂಶತ್ತತ್ತ್ವಸಂಭೂತಿರಷ್ಟತ್ರಿಂಶತ್ಕಲಾತ್ಮಕಃ ॥೧೫೮॥

ಪಂಚಾಶದ್ವಿಷ್ಣುಶಕ್ತೀಶಃ ಪಂಚಾಶನ್ಮಾತೃಕಾಲಯಃ ।\\
ದ್ವಿಪಂಚಾಶದ್ವಪುಃಶ್ರೇಣೀತ್ರಿಷಷ್ಟ್ಯಕ್ಷರಸಂಶ್ರಯಃ ।\\
ಪಂಚಾಶದಕ್ಷರಶ್ರೇಣೀಪಂಚಾಶದ್ರುದ್ರವಿಗ್ರಹಃ ॥೧೫೯॥

ಚತುಃಷಷ್ಟಿಮಹಾಸಿದ್ಧಿಯೋಗಿನೀವೃಂದವಂದಿತಃ ।\\
ನಮದೇಕೋನಪಂಚಾಶನ್ಮರುದ್ವರ್ಗನಿರರ್ಗಲಃ ॥೧೬೦॥

ಚತುಃಷಷ್ಟ್ಯರ್ಥನಿರ್ಣೇತಾ ಚತುಃಷಷ್ಟಿಕಲಾನಿಧಿಃ ।\\
ಅಷ್ಟಷಷ್ಟಿಮಹಾತೀರ್ಥಕ್ಷೇತ್ರಭೈರವವಂದಿತಃ ॥೧೬೧॥

ಚತುರ್ನವತಿಮಂತ್ರಾತ್ಮಾ ಷಣ್ಣವತ್ಯಧಿಕಪ್ರಭುಃ ।\\
ಶತಾನಂದಃ ಶತಧೃತಿಃ ಶತಪತ್ರಾಯತೇಕ್ಷಣಃ ॥೧೬೨॥

ಶತಾನೀಕಃ ಶತಮಖಃ ಶತಧಾರಾವರಾಯುಧಃ ।\\
ಸಹಸ್ರಪತ್ರನಿಲಯಃ ಸಹಸ್ರಫಣಿಭೂಷಣಃ ॥೧೬೩॥

ಸಹಸ್ರಶೀರ್ಷಾ ಪುರುಷಃ ಸಹಸ್ರಾಕ್ಷಃ ಸಹಸ್ರಪಾತ್ ।\\
ಸಹಸ್ರನಾಮಸಂಸ್ತುತ್ಯಃ ಸಹಸ್ರಾಕ್ಷಬಲಾಪಹಃ ॥೧೬೪॥

ದಶಸಾಹಸ್ರಫಣಿಭೃತ್ಫಣಿರಾಜಕೃತಾಸನಃ ।\\
ಅಷ್ಟಾಶೀತಿಸಹಸ್ರಾದ್ಯಮಹರ್ಷಿಸ್ತೋತ್ರಪಾಠಿತಃ ॥೧೬೫॥

ಲಕ್ಷಾಧಾರಃ ಪ್ರಿಯಾಧಾರೋ ಲಕ್ಷಾಧಾರಮನೋಮಯಃ ।\\
ಚತುರ್ಲಕ್ಷಜಪಪ್ರೀತಶ್ಚತುರ್ಲಕ್ಷಪ್ರಕಾಶಕಃ ॥೧೬೬॥

ಚತುರಶೀತಿಲಕ್ಷಾಣಾಂ ಜೀವಾನಾಂ ದೇಹಸಂಸ್ಥಿತಃ ।\\
ಕೋಟಿಸೂರ್ಯಪ್ರತೀಕಾಶಃ ಕೋಟಿಚಂದ್ರಾಂಶುನಿರ್ಮಲಃ ॥೧೬೭॥

ಶಿವೋದ್ಭವಾದ್ಯಷ್ಟಕೋಟಿವೈನಾಯಕಧುರಂಧರಃ ।\\
ಸಪ್ತಕೋಟಿಮಹಾಮಂತ್ರಮಂತ್ರಿತಾವಯವದ್ಯುತಿಃ ॥೧೬೮॥

ತ್ರಯಸ್ತ್ರಿಂಶತ್ಕೋಟಿಸುರಶ್ರೇಣೀಪ್ರಣತಪಾದುಕಃ ।\\
ಅನಂತದೇವತಾಸೇವ್ಯೋ ಹ್ಯನಂತಶುಭದಾಯಕಃ ॥೧೬೯॥

ಅನಂತನಾಮಾನಂತಶ್ರೀರನಂತೋಽನಂತಸೌಖ್ಯದಃ ।\\
ಅನಂತಶಕ್ತಿಸಹಿತೋ ಹ್ಯನಂತಮುನಿಸಂಸ್ತುತಃ ॥೧೭೦॥


ಇತಿ ವೈನಾಯಕಂ ನಾಮ್ನಾಂ ಸಹಸ್ರಮಿದಮೀರಿತಂ ।\\
ಇದಂ ಬ್ರಾಹ್ಮೇ ಮುಹೂರ್ತೇ ಯಃ ಪಠತಿ ಪ್ರತ್ಯಹಂ ನರಃ ॥೧೭೧॥

ಕರಸ್ಥಂ ತಸ್ಯ ಸಕಲಮೈಹಿಕಾಮುಷ್ಮಿಕಂ ಸುಖಂ ।\\
ಆಯುರಾರೋಗ್ಯಮೈಶ್ವರ್ಯಂ ಧೈರ್ಯಂ ಶೌರ್ಯಂ ಬಲಂ ಯಶಃ ॥೧೭೨॥

ಮೇಧಾ ಪ್ರಜ್ಞಾ ಧೃತಿಃ ಕಾಂತಿಃ ಸೌಭಾಗ್ಯಮಭಿರೂಪತಾ ।\\
ಸತ್ಯಂ ದಯಾ ಕ್ಷಮಾ ಶಾಂತಿರ್ದಾಕ್ಷಿಣ್ಯಂ ಧರ್ಮಶೀಲತಾ ॥೧೭೩॥

ಜಗತ್ಸಂವನನಂ ವಿಶ್ವಸಂವಾದೋ ವೇದಪಾಟವಂ ।\\
ಸಭಾಪಾಂಡಿತ್ಯಮೌದಾರ್ಯಂ ಗಾಂಭೀರ್ಯಂ ಬ್ರಹ್ಮವರ್ಚಸಂ ॥೧೭೪॥

ಓಜಸ್ತೇಜಃ ಕುಲಂ ಶೀಲಂ ಪ್ರತಾಪೋ ವೀರ್ಯಮಾರ್ಯತಾ ।\\
ಜ್ಞಾನಂ ವಿಜ್ಞಾನಮಾಸ್ತಿಕ್ಯಂ ಸ್ಥೈರ್ಯಂ ವಿಶ್ವಾಸತಾ ತಥಾ ॥೧೭೫॥

ಧನಧಾನ್ಯಾದಿವೃದ್ಧಿಶ್ಚ ಸಕೃದಸ್ಯ ಜಪಾದ್ಭವೇತ್ ।\\
ವಶ್ಯಂ ಚತುರ್ವಿಧಂ ವಿಶ್ವಂ ಜಪಾದಸ್ಯ ಪ್ರಜಾಯತೇ ॥೧೭೬॥

ರಾಜ್ಞೋ ರಾಜಕಲತ್ರಸ್ಯ ರಾಜಪುತ್ರಸ್ಯ ಮಂತ್ರಿಣಃ ।\\
ಜಪ್ಯತೇ ಯಸ್ಯ ವಶ್ಯಾರ್ಥೇ ಸ ದಾಸಸ್ತಸ್ಯ ಜಾಯತೇ ॥೧೭೭॥

ಧರ್ಮಾರ್ಥಕಾಮಮೋಕ್ಷಾಣಾಮನಾಯಾಸೇನ ಸಾಧನಂ ।\\
ಶಾಕಿನೀಡಾಕಿನೀರಕ್ಷೋಯಕ್ಷಗ್ರಹಭಯಾಪಹಂ ॥೧೭೮॥

ಸಾಮ್ರಾಜ್ಯಸುಖದಂ ಸರ್ವಸಪತ್ನಮದಮರ್ದನಂ ।\\
ಸಮಸ್ತಕಲಹಧ್ವಂಸಿ ದಗ್ಧಬೀಜಪ್ರರೋಹಣಂ ॥೧೭೯॥

ದುಃಸ್ವಪ್ನಶಮನಂ ಕ್ರುದ್ಧಸ್ವಾಮಿಚಿತ್ತಪ್ರಸಾದನಂ ।\\
ಷಡ್ವರ್ಗಾಷ್ಟಮಹಾಸಿದ್ಧಿತ್ರಿಕಾಲಜ್ಞಾನಕಾರಣಂ ॥೧೮೦॥

ಪರಕೃತ್ಯಪ್ರಶಮನಂ ಪರಚಕ್ರಪ್ರಮರ್ದನಂ ।\\
ಸಂಗ್ರಾಮಮಾರ್ಗೇ ಸರ್ವೇಷಾಮಿದಮೇಕಂ ಜಯಾವಹಂ ॥೧೮೧॥

ಸರ್ವವಂಧ್ಯತ್ವದೋಷಘ್ನಂ ಗರ್ಭರಕ್ಷೈಕಕಾರಣಂ ।\\
ಪಠ್ಯತೇ ಪ್ರತ್ಯಹಂ ಯತ್ರ ಸ್ತೋತ್ರಂ ಗಣಪತೇರಿದಂ ॥೧೮೨॥

ದೇಶೇ ತತ್ರ ನ ದುರ್ಭಿಕ್ಷಮೀತಯೋ ದುರಿತಾನಿ ಚ ।\\
ನ ತದ್ಗೇಹಂ ಜಹಾತಿ ಶ್ರೀರ್ಯತ್ರಾಯಂ ಜಪ್ಯತೇ ಸ್ತವಃ ॥೧೮೩॥

ಕ್ಷಯಕುಷ್ಠಪ್ರಮೇಹಾರ್ಶಭಗಂದರವಿಷೂಚಿಕಾಃ ।\\
ಗುಲ್ಮಂ ಪ್ಲೀಹಾನಮಶಮಾನಮತಿಸಾರಂ ಮಹೋದರಂ ॥೧೮೪॥

ಕಾಸಂ ಶ್ವಾಸಮುದಾವರ್ತಂ ಶೂಲಂ ಶೋಫಾಮಯೋದರಂ ।\\
ಶಿರೋರೋಗಂ ವಮಿಂ ಹಿಕ್ಕಾಂ ಗಂಡಮಾಲಾಮರೋಚಕಂ ॥೧೮೫॥

ವಾತಪಿತ್ತಕಫದ್ವಂದ್ವತ್ರಿದೋಷಜನಿತಜ್ವರಂ ।\\
ಆಗಂತುವಿಷಮಂ ಶೀತಮುಷ್ಣಂ ಚೈಕಾಹಿಕಾದಿಕಂ ॥೧೮೬॥

ಇತ್ಯಾದ್ಯುಕ್ತಮನುಕ್ತಂ ವಾ ರೋಗದೋಷಾದಿಸಂಭವಂ ।\\
ಸರ್ವಂ ಪ್ರಶಮಯತ್ಯಾಶು ಸ್ತೋತ್ರಸ್ಯಾಸ್ಯ ಸಕೃಜ್ಜಪಃ ॥೧೮೭॥

ಪ್ರಾಪ್ಯತೇಽಸ್ಯ ಜಪಾತ್ಸಿದ್ಧಿಃ ಸ್ತ್ರೀಶೂದ್ರೈಃ ಪತಿತೈರಪಿ ।\\
ಸಹಸ್ರನಾಮಮಂತ್ರೋಽಯಂ ಜಪಿತವ್ಯಃ ಶುಭಾಪ್ತಯೇ ॥೧೮೮॥

ಮಹಾಗಣಪತೇಃ ಸ್ತೋತ್ರಂ ಸಕಾಮಃ ಪ್ರಜಪನ್ನಿದಂ ।\\
ಇಚ್ಛಯಾ ಸಕಲಾನ್ ಭೋಗಾನುಪಭುಜ್ಯೇಹ ಪಾರ್ಥಿವಾನ್ ॥೧೮೯॥

ಮನೋರಥಫಲೈರ್ದಿವ್ಯೈರ್ವ್ಯೋಮಯಾನೈರ್ಮನೋರಮೈಃ ।\\
ಚಂದ್ರೇಂದ್ರಭಾಸ್ಕರೋಪೇಂದ್ರಬ್ರಹ್ಮಶರ್ವಾದಿಸದ್ಮಸು ॥೧೯೦॥

ಕಾಮರೂಪಃ ಕಾಮಗತಿಃ ಕಾಮದಃ ಕಾಮದೇಶ್ವರಃ ।\\
ಭುಕ್ತ್ವಾ ಯಥೇಪ್ಸಿತಾನ್ಭೋಗಾನಭೀಷ್ಟೈಃ ಸಹ ಬಂಧುಭಿಃ ॥೧೯೧॥

ಗಣೇಶಾನುಚರೋ ಭೂತ್ವಾ ಗಣೋ ಗಣಪತಿಪ್ರಿಯಃ ।\\
ನಂದೀಶ್ವರಾದಿಸಾನಂದೈರ್ನಂದಿತಃ ಸಕಲೈರ್ಗಣೈಃ ॥೧೯೨॥

ಶಿವಾಭ್ಯಾಂ ಕೃಪಯಾ ಪುತ್ರನಿರ್ವಿಶೇಷಂ ಚ ಲಾಲಿತಃ ।\\
ಶಿವಭಕ್ತಃ ಪೂರ್ಣಕಾಮೋ ಗಣೇಶ್ವರವರಾತ್ಪುನಃ ॥೧೯೩॥

ಜಾತಿಸ್ಮರೋ ಧರ್ಮಪರಃ ಸಾರ್ವಭೌಮೋಽಭಿಜಾಯತೇ ।\\
ನಿಷ್ಕಾಮಸ್ತು ಜಪನ್ನಿತ್ಯಂ ಭಕ್ತ್ಯಾ ವಿಘ್ನೇಶತತ್ಪರಃ ॥೧೯೪॥

ಯೋಗಸಿದ್ಧಿಂ ಪರಾಂ ಪ್ರಾಪ್ಯ ಜ್ಞಾನವೈರಾಗ್ಯಸಂಯುತಃ ।\\
ನಿರಂತರೇ ನಿರಾಬಾಧೇ ಪರಮಾನಂದಸಂಜ್ಞಿತೇ ॥೧೯೫॥

ವಿಶ್ವೋತ್ತೀರ್ಣೇ ಪರೇ ಪೂರ್ಣೇ ಪುನರಾವೃತ್ತಿವರ್ಜಿತೇ ।\\
ಲೀನೋ ವೈನಾಯಕೇ ಧಾಮ್ನಿ ರಮತೇ ನಿತ್ಯನಿರ್ವೃತೇ ॥೧೯೬॥

ಯೋ ನಾಮಭಿರ್ಹುತೈರ್ದತ್ತೈಃ ಪೂಜಯೇದರ್ಚಯೀನ್ನರಃ ।\\
ರಾಜಾನೋ ವಶ್ಯತಾಂ ಯಾಂತಿ ರಿಪವೋ ಯಾಂತಿ ದಾಸತಾಂ ॥೧೯೭॥

ತಸ್ಯ ಸಿಧ್ಯಂತಿ ಮಂತ್ರಾಣಾಂ ದುರ್ಲಭಾಶ್ಚೇಷ್ಟಸಿದ್ಧಯಃ ।\\
ಮೂಲಮಂತ್ರಾದಪಿ ಸ್ತೋತ್ರಮಿದಂ ಪ್ರಿಯತಮಂ ಮಮ ॥೧೯೮॥

ನಭಸ್ಯೇ ಮಾಸಿ ಶುಕ್ಲಾಯಾಂ ಚತುರ್ಥ್ಯಾಂ ಮಮ ಜನ್ಮನಿ ।\\
ದೂರ್ವಾಭಿರ್ನಾಮಭಿಃ ಪೂಜಾಂ ತರ್ಪಣಂ ವಿಧಿವಚ್ಚರೇತ್ ॥೧೯೯॥

ಅಷ್ಟದ್ರವ್ಯೈರ್ವಿಶೇಷೇಣ ಕುರ್ಯಾದ್ಭಕ್ತಿಸುಸಂಯುತಃ ।\\
ತಸ್ಯೇಪ್ಸಿತಂ ಧನಂ ಧಾನ್ಯಮೈಶ್ವರ್ಯಂ ವಿಜಯೋ ಯಶಃ ॥೨೦೦॥

ಭವಿಷ್ಯತಿ ನ ಸಂದೇಹಃ ಪುತ್ರಪೌತ್ರಾದಿಕಂ ಸುಖಂ ।\\
ಇದಂ ಪ್ರಜಪಿತಂ ಸ್ತೋತ್ರಂ ಪಠಿತಂ ಶ್ರಾವಿತಂ ಶ್ರುತಂ ॥೨೦೧॥

ವ್ಯಾಕೃತಂ ಚರ್ಚಿತಂ ಧ್ಯಾತಂ ವಿಮೃಷ್ಟಮಭಿವಂದಿತಂ ।\\
ಇಹಾಮುತ್ರ ಚ ವಿಶ್ವೇಷಾಂ ವಿಶ್ವೈಶ್ವರ್ಯಪ್ರದಾಯಕಂ ॥೨೦೨॥

ಸ್ವಚ್ಛಂದಚಾರಿಣಾಪ್ಯೇಷ ಯೇನ ಸಂಧಾರ್ಯತೇ ಸ್ತವಃ ।\\
ಸ ರಕ್ಷ್ಯತೇ ಶಿವೋದ್ಭೂತೈರ್ಗಣೈರಧ್ಯಷ್ಟಕೋಟಿಭಿಃ ॥೨೦೩॥

ಲಿಖಿತಂ ಪುಸ್ತಕಸ್ತೋತ್ರಂ ಮಂತ್ರಭೂತಂ ಪ್ರಪೂಜಯೇತ್ ।\\
ತತ್ರ ಸರ್ವೋತ್ತಮಾ ಲಕ್ಷ್ಮೀಃ ಸನ್ನಿಧತ್ತೇ ನಿರಂತರಂ ॥೨೦೪॥

ದಾನೈರಶೇಷೈರಖಿಲೈರ್ವ್ರತೈಶ್ಚ\\
ತೀರ್ಥೈರಶೇಷೈರಖಿಲೈರ್ಮಖೈಶ್ಚ ।\\
ನ ತತ್ಫಲಂ ವಿಂದತಿ ಯದ್ಗಣೇಶ\\
ಸಹಸ್ರನಾಮಸ್ಮರಣೇನ ಸದ್ಯಃ ॥೨೦೫॥

ಏತನ್ನಾಮ್ನಾಂ ಸಹಸ್ರಂ ಪಠತಿ ದಿನಮಣೌ ಪ್ರತ್ಯಹಂ ಪ್ರೋಜ್ಜಿಹಾನೇ\\
 ಸಾಯಂ ಮಧ್ಯಂದಿನೇ ವಾ ತ್ರಿಷವಣಮಥವಾ ಸಂತತಂ ವಾ ಜನೋ ಯಃ ।\\
ಸ ಸ್ಯಾದೈಶ್ವರ್ಯಧುರ್ಯಃ ಪ್ರಭವತಿ ವಚಸಾಂ ಕೀರ್ತಿಮುಚ್ಚೈಸ್ತನೋತಿ\\
 ದಾರಿದ್ರ್ಯಂ ಹಂತಿ ವಿಶ್ವಂ ವಶಯತಿ ಸುಚಿರಂ ವರ್ಧತೇ ಪುತ್ರಪೌತ್ರೈಃ ॥೨೦೬॥

ಅಕಿಂಚನೋಽಪ್ಯೇಕಚಿತ್ತೋ ನಿಯತೋ ನಿಯತಾಸನಃ ।\\
ಪ್ರಜಪಂಶ್ಚತುರೋ ಮಾಸಾನ್ ಗಣೇಶಾರ್ಚನತತ್ಪರಃ ॥೨೦೭॥

ದರಿದ್ರತಾಂ ಸಮುನ್ಮೂಲ್ಯ ಸಪ್ತಜನ್ಮಾನುಗಾಮಪಿ ।\\
ಲಭತೇ ಮಹತೀಂ ಲಕ್ಷ್ಮೀಮಿತ್ಯಾಜ್ಞಾ ಪಾರಮೇಶ್ವರೀ ॥೨೦೮॥

ಆಯುಷ್ಯಂ ವೀತರೋಗಂ ಕುಲಮತಿವಿಮಲಂ ಸಂಪದಶ್ಚಾರ್ತಿನಾಶಃ\\
ಕೀರ್ತಿರ್ನಿತ್ಯಾವದಾತಾ ಭವತಿ ಖಲು ನವಾ ಕಾಂತಿರವ್ಯಾಜಭವ್ಯಾ ।\\
ಪುತ್ರಾಃ ಸಂತಃ ಕಲತ್ರಂ ಗುಣವದಭಿಮತಂ ಯದ್ಯದನ್ಯಚ್ಚ ತತ್ತನ್\\
ನಿತ್ಯಂ ಯಃ ಸ್ತೋತ್ರಮೇತತ್ ಪಠತಿ ಗಣಪತೇಸ್ತಸ್ಯ ಹಸ್ತೇ ಸಮಸ್ತಂ ॥೨೦೯॥

ಗಣಂಜಯೋ ಗಣಪತಿರ್ಹೇರಂಬೋ ಧರಣೀಧರಃ ।\\
ಮಹಾಗಣಪತಿರ್ಬುದ್ಧಿಪ್ರಿಯಃ ಕ್ಷಿಪ್ರಪ್ರಸಾದನಃ ॥೨೧೦॥

ಅಮೋಘಸಿದ್ಧಿರಮೃತಮಂತ್ರಶ್ಚಿಂತಾಮಣಿರ್ನಿಧಿಃ ।\\
ಸುಮಂಗಲೋ ಬೀಜಮಾಶಾಪೂರಕೋ ವರದಃ ಕಲಃ ॥೨೧೧॥

ಕಾಶ್ಯಪೋ ನಂದನೋ ವಾಚಾಸಿದ್ಧೋ ಢುಂಢಿರ್ವಿನಾಯಕಃ ।\\
ಮೋದಕೈರೇಭಿರತ್ರೈಕವಿಂಶತ್ಯಾ ನಾಮಭಿಃ ಪುಮಾನ್ ॥೨೧೨॥

ಉಪಾಯನಂ ದದೇದ್ಭಕ್ತ್ಯಾ ಮತ್ಪ್ರಸಾದಂ ಚಿಕೀರ್ಷತಿ ।\\
ವತ್ಸರಂ ವಿಘ್ನರಾಜೋಽಸ್ಯ ತಥ್ಯಮಿಷ್ಟಾರ್ಥಸಿದ್ಧಯೇ ॥೨೧೩॥

ಯಃ ಸ್ತೌತಿ ಮದ್ಗತಮನಾ ಮಮಾರಾಧನತತ್ಪರಃ ।\\
ಸ್ತುತೋ ನಾಮ್ನಾ ಸಹಸ್ರೇಣ ತೇನಾಹಂ ನಾತ್ರ ಸಂಶಯಃ ॥೨೧೪॥

ನಮೋ ನಮಃ ಸುರವರಪೂಜಿತಾಂಘ್ರಯೇ\\
ನಮೋ ನಮೋ ನಿರುಪಮಮಂಗಲಾತ್ಮನೇ ।\\
ನಮೋ ನಮೋ ವಿಪುಲದಯೈಕಸಿದ್ಧಯೇ\\
ನಮೋ ನಮಃ ಕರಿಕಲಭಾನನಾಯ ತೇ ॥೨೧೫॥

ಕಿಂಕಿಣೀಗಣರಚಿತಚರಣಃ\\
ಪ್ರಕಟಿತಗುರುಮಿತಚಾರುಕರಣಃ ।\\
ಮದಜಲಲಹರೀಕಲಿತಕಪೋಲಃ\\
ಶಮಯತು ದುರಿತಂ ಗಣಪತಿನಾಮ್ನಾ ॥೨೧೬॥
\authorline{॥ಇತಿ ಶ್ರೀಗಣೇಶಪುರಾಣೇ ಉಪಾಸನಾಖಂಡೇ ಈಶ್ವರಗಣೇಶಸಂವಾದೇ ಗಣೇಶಸಹಸ್ರನಾಮಸ್ತೋತ್ರಂ ನಾಮ ಷಟ್ಚತ್ವಾರಿಂಶೋಽಧ್ಯಾಯಃ ॥}
%=============================================================================================
\section{ಶ್ರೀವಿಘ್ನೇಶ್ವರಾಷ್ಟೋತ್ತರ ಶತನಾಮಸ್ತೋತ್ರಂ}
\addcontentsline{toc}{section}{ಶ್ರೀವಿಘ್ನೇಶ್ವರಾಷ್ಟೋತ್ತರ ಶತನಾಮಸ್ತೋತ್ರಂ}
\dhyana{ಗಜವದನಮಚಿಂತ್ಯಂ ತೀಕ್ಷ್ಣದಂಷ್ಟ್ರಂ ತ್ರಿಣೇತ್ರಂ\\
ಬೃಹದುದರಮಶೇಷಂ ಭೂತಿರೂಪಂ ಪುರಾಣಮ್~।\\
ಅಮರವರಸುಪೂಜ್ಯಂ ರಕ್ತವರ್ಣಂ ಸುರೇಶಂ\\
 ಪಶುಪತಿಸುತಮೀಶಂ ವಿಘ್ನರಾಜಂ ನಮಾಮಿ ॥}

ವಿನಾಯಕೋ ವಿಘ್ನರಾಜೋ ಗೌರೀಪುತ್ರೋ ಗಣೇಶ್ವರಃ~।\\
ಸ್ಕಂದಾಗ್ರಜೋಽವ್ಯಯಃ ಪೂತೋ ದಕ್ಷೋಽಧ್ಯಕ್ಷೋ ದ್ವಿಜಪ್ರಿಯಃ ॥೧॥

ಅಗ್ನಿಗರ್ಭಚ್ಛಿದಿಂದ್ರಶ್ರೀಪ್ರದೋ ವಾಣೀಬಲಪ್ರದಃ~।\\
ಸರ್ವಸಿದ್ಧಿಪ್ರದಶ್ಶರ್ವತನಯಃ ಶರ್ವರೀಪ್ರಿಯಃ ॥೨॥

ಸರ್ವಾತ್ಮಕಃ ಸೃಷ್ಟಿಕರ್ತಾ ದೇವಾನೀಕಾರ್ಚಿತಶ್ಶಿವಃ~।\\
ಶುದ್ಧೋ ಬುದ್ಧಿಪ್ರಿಯಶ್ಶಾಂತೋ ಬ್ರಹ್ಮಚಾರೀ ಗಜಾನನಃ ॥೩॥

ದ್ವೈಮಾತ್ರೇಯೋ ಮುನಿಸ್ತುತ್ಯೋ ಭಕ್ತವಿಘ್ನವಿನಾಶನಃ~।\\
ಏಕದಂತಶ್ಚತುರ್ಬಾಹುಶ್ಚತುರಶ್ಶಕ್ತಿಸಂಯುತಃ ॥೪॥

ಲಂಬೋದರಶ್ಶೂರ್ಪಕರ್ಣೋ ಹರಿರ್ಬ್ರಹ್ಮ ವಿದುತ್ತಮಃ~।\\
ಕಾಲೋ ಗ್ರಹಪತಿಃ ಕಾಮೀ ಸೋಮಸೂರ್ಯಾಗ್ನಿಲೋಚನಃ ॥೫॥

ಪಾಶಾಂಕುಶಧರಶ್ಚಂಡೋ ಗುಣಾತೀತೋ ನಿರಂಜನಃ~।\\
ಅಕಲ್ಮಷಸ್ಸ್ವಯಂಸಿದ್ಧಸ್ಸಿದ್ಧಾರ್ಚಿತಪದಾಂಬುಜಃ ॥೬॥

ಬೀಜಪೂರಫಲಾಸಕ್ತೋ ವರದಶ್ಶಾಶ್ವತಃ ಕೃತೀ~।\\
ದ್ವಿಜಪ್ರಿಯೋ ವೀತಭಯೋ ಗದೀ ಚಕ್ರೀಕ್ಷುಚಾಪಧೃತ್ ॥೭॥

ಶ್ರೀದೋಽಜ ಉತ್ಪಲಕರಃ ಶ್ರೀಪತಿಃ ಸ್ತುತಿಹರ್ಷಿತಃ~।\\
ಕುಲಾದ್ರಿಭೇತ್ತಾ ಜಟಿಲಃ ಕಲಿಕಲ್ಮಷನಾಶನಃ ॥೮॥

ಚಂದ್ರಚೂಡಾಮಣಿಃ ಕಾಂತಃ ಪಾಪಹಾರೀ ಸಮಾಹಿತಃ~।\\
ಆಶ್ರಿತಶ್ಶ್ರೀಕರಸ್ಸೌಮ್ಯೋ ಭಕ್ತವಾಂಛಿತದಾಯಕಃ ॥೯॥

ಶಾಂತಃ ಕೈವಲ್ಯಸುಖದಸ್ಸಚ್ಚಿದಾನಂದವಿಗ್ರಹಃ~।\\
ಜ್ಞಾನೀ ದಯಾಯುತೋ ದಾಂತೋ ಬ್ರಹ್ಮ ದ್ವೇಷವಿವರ್ಜಿತಃ ॥೧೦॥

ಪ್ರಮತ್ತದೈತ್ಯಭಯದಃ ಶ್ರೀಕಂಠೋ ವಿಬುಧೇಶ್ವರಃ~।\\
ರಮಾರ್ಚಿತೋವಿಧಿರ್ನಾಗರಾಜಯಜ್ಞೋಪವೀತಕಃ ॥೧೧॥

ಸ್ಥೂಲಕಂಠಃ ಸ್ವಯಂಕರ್ತಾ ಸಾಮಘೋಷಪ್ರಿಯಃ ಪರಃ~।\\
ಸ್ಥೂಲತುಂಡೋಽಗ್ರಣೀರ್ಧೀರೋ ವಾಗೀಶಸ್ಸಿದ್ಧಿದಾಯಕಃ ॥೧೨॥

ದೂರ್ವಾಬಿಲ್ವಪ್ರಿಯೋಽವ್ಯಕ್ತಮೂರ್ತಿರದ್ಭುತಮೂರ್ತಿಮಾನ್~।\\
ಶೈಲೇಂದ್ರತನುಜೋತ್ಸಂಗಖೇಲನೋತ್ಸುಕಮಾನಸಃ ॥೧೩॥

ಸ್ವಲಾವಣ್ಯಸುಧಾಸಾರೋ ಜಿತಮನ್ಮಥವಿಗ್ರಹಃ~।\\
ಸಮಸ್ತಜಗದಾಧಾರೋ ಮಾಯೀ ಮೂಷಕವಾಹನಃ ॥೧೪॥

ಹೃಷ್ಟಸ್ತುಷ್ಟಃ ಪ್ರಸನ್ನಾತ್ಮಾ ಸರ್ವಸಿದ್ಧಿಪ್ರದಾಯಕಃ~।\\
ಅಷ್ಟೋತ್ತರಶತೇನೈವಂ ನಾಮ್ನಾಂ ವಿಘ್ನೇಶ್ವರಂ ವಿಭುಂ ॥೧೫॥

ತುಷ್ಟಾವ ಶಂಕರಃ ಪುತ್ರಂ ತ್ರಿಪುರಂ ಹಂತುಮುದ್ಯತಃ~।\\
ಯಃ ಪೂಜಯೇದನೇನೈವ ಭಕ್ತ್ಯಾ ಸಿದ್ಧಿವಿನಾಯಕಂ ॥೧೬॥

ದೂರ್ವಾದಲೈರ್ಬಿಲ್ವಪತ್ರೈಃ ಪುಷ್ಪೈರ್ವಾ ಚಂದನಾಕ್ಷತೈಃ~।\\
ಸರ್ವಾನ್ಕಾಮಾನವಾಪ್ನೋತಿ ಸರ್ವವಿಘ್ನೈಃ ಪ್ರಮುಚ್ಯತೇ ॥
\authorline{ಇತಿ ಶ್ರೀವಿಘ್ನೇಶ್ವರಾಷ್ಟೋತ್ತರ ಶತನಾಮಸ್ತೋತ್ರಂ}
%========================================================================================
\section{ಶಿವಸಹಸ್ರನಾಮ ಸ್ತೋತ್ರಮ್}
\addcontentsline{toc}{section}{ಶಿವಸಹಸ್ರನಾಮ ಸ್ತೋತ್ರಮ್}
\dhyana{ಶಾಂತಂ ಪದ್ಮಾಸನಸ್ಥಂ ಶಶಧರಮುಕುಟಂ ಪಂಚವಕ್ತ್ರಂ ತ್ರಿನೇತ್ರಂ\\
ಶೂಲಂ ವಜ್ರಂ ಚ ಖಡ್ಗಂ ಪರಶುಮಭಯದಂ ದಕ್ಷಭಾಗೇ ವಹಂತಂ~।\\
ನಾಗಂ ಪಾಶಂ ಚ ಘಂಟಾಂ ಪ್ರಲಯಹುತವಹಂ ಚಾಂಕುಶಂ ವಾಮಭಾಗೇ\\
ನಾನಾಲಂಕಾರಯುಕ್ತಂ ಸ್ಫಟಿಕಮಣಿನಿಭಂ ಪಾರ್ವತೀಶಂ ನಮಾಮಿ ॥}

ಸ್ಥಿರಃ ಸ್ಥಾಣುಃ ಪ್ರಭುರ್ಭೀಮಃ ಪ್ರವರೋ ವರದೋ ವರಃ ।\\
ಸರ್ವಾತ್ಮಾ ಸರ್ವವಿಖ್ಯಾತಃ ಸರ್ವಃ ಸರ್ವಕರೋ ಭವಃ ॥೩೧॥

ಜಟೀ ಚರ್ಮೀ ಶಿಖೀ ಖಡ್ಗೀ ಸರ್ವಾಂಗಃ ಸರ್ವಭಾವನಃ ।\\
ಹರಶ್ಚ ಹರಿಣಾಕ್ಷಶ್ಚ ಸರ್ವಭೂತಹರಃ ಪ್ರಭುಃ ॥೩೨॥

ಪ್ರವೃತ್ತಿಶ್ಚ ನಿವೃತ್ತಿಶ್ಚ ನಿಯತಃ ಶಾಶ್ವತೋ ಧ್ರುವಃ ।\\
ಶ್ಮಶಾನವಾಸೀ ಭಗವಾನ್ಖಚರೋ ಗೋಚರೋಽರ್ದನಃ ॥೩೩॥

ಅಭಿವಾದ್ಯೋ ಮಹಾಕರ್ಮಾ ತಪಸ್ವೀ ಭೂತಭಾವನಃ ।\\
ಉನ್ಮತ್ತವೇಷಪ್ರಚ್ಛನ್ನಃ ಸರ್ವಲೋಕಪ್ರಜಾಪತಿಃ ॥೩೪॥

ಮಹಾರೂಪೋ ಮಹಾಕಾಯೋ ವೃಷರೂಪೋ ಮಹಾಯಶಾಃ ।\\
ಮಹಾತ್ಮಾ ಸರ್ವಭೂತಾತ್ಮಾ ವಿಶ್ವರೂಪೋ ಮಹಾಹನುಃ ॥೩೫॥

ಲೋಕಪಾಲೋಽನ್ತರ್ಹಿತಾತ್ಮಾ ಪ್ರಸಾದೋ ಹಯಗರ್ದಭಿಃ ।\\
ಪವಿತ್ರಂ ಚ ಮಹಾಂಶ್ಚೈವ ನಿಯಮೋ ನಿಯಮಾಶ್ರಿತಃ ॥೩೬॥

ಸರ್ವಕರ್ಮಾ ಸ್ವಯಂಭೂತ ಆದಿರಾದಿಕರೋ ನಿಧಿಃ ।\\
ಸಹಸ್ರಾಕ್ಷೋ ವಿಶಾಲಾಕ್ಷಃ ಸೋಮೋ ನಕ್ಷತ್ರಸಾಧಕಃ ॥೩೭॥

ಚಂದ್ರಃ ಸೂರ್ಯಃ ಶನಿಃ ಕೇತುರ್ಗ್ರಹೋ ಗ್ರಹಪತಿರ್ವರಃ ।\\
ಅತ್ರಿರತ್ರ್ಯಾನಮಸ್ಕರ್ತಾ ಮೃಗಬಾಣಾರ್ಪಣೋಽನಘಃ ॥೩೮॥

ಮಹಾತಪಾ ಘೋರತಪಾ ಅದೀನೋ ದೀನಸಾಧಕಃ ।\\
ಸಂವತ್ಸರಕರೋ ಮಂತ್ರಃ ಪ್ರಮಾಣಂ ಪರಮಂ ತಪಃ ॥೩೯॥

ಯೋಗೀ ಯೋಜ್ಯೋ ಮಹಾಬೀಜೋ ಮಹಾರೇತಾ ಮಹಾಬಲಃ ।\\
ಸುವರ್ಣರೇತಾಃ ಸರ್ವಜ್ಞಃ ಸುಬೀಜೋ ಬೀಜವಾಹನಃ ॥೪೦॥

ದಶಬಾಹುಸ್ತ್ವನಿಮಿಷೋ ನೀಲಕಂಠ ಉಮಾಪತಿಃ ।\\
ವಿಶ್ವರೂಪಃ ಸ್ವಯಂಶ್ರೇಷ್ಠೋ ಬಲವೀರೋ ಬಲೋ ಗಣಃ ॥೪೧॥

ಗಣಕರ್ತಾ ಗಣಪತಿರ್ದಿಗ್ವಾಸಾಃ ಕಾಮ ಏವ ಚ ।\\
ಮಂತ್ರವಿತ್ಪರಮೋ ಮಂತ್ರಃ ಸರ್ವಭಾವಕರೋ ಹರಃ ॥೪೨॥

ಕಮಂಡಲುಧರೋ ಧನ್ವೀ ಬಾಣಹಸ್ತಃ ಕಪಾಲವಾನ್ ।\\
ಅಶನೀ ಶತಘ್ನೀ ಖಡ್ಗೀ ಪಟ್ಟಿಶೀ ಚಾಯುಧೀ ಮಹಾನ್ ॥೪೩॥

ಸ್ರುವಹಸ್ತಃ ಸುರೂಪಶ್ಚ ತೇಜಸ್ತೇಜಸ್ಕರೋ ನಿಧಿಃ ।\\
ಉಷ್ಣೀಷೀ ಚ ಸುವಕ್ತ್ರಶ್ಚ ಉದಗ್ರೋ ವಿನತಸ್ತಥಾ ॥೪೪॥

ದೀರ್ಘಶ್ಚ ಹರಿಕೇಶಶ್ಚ ಸುತೀರ್ಥಃ ಕೃಷ್ಣ ಏವ ಚ ।\\
ಶೃಗಾಲರೂಪಃ ಸಿದ್ಧಾರ್ಥೋ ಮುಂಡಃ ಸರ್ವಶುಭಂಕರಃ ॥೪೫॥

ಅಜಶ್ಚ ಬಹುರೂಪಶ್ಚ ಗಂಧಧಾರೀ ಕಪರ್ದ್ಯಪಿ ।\\
ಊರ್ಧ್ವರೇತಾ ಊರ್ಧ್ವಲಿಂಗ ಊರ್ಧ್ವಶಾಯೀ ನಭಃಸ್ಥಲಃ ॥೪೬॥

ತ್ರಿಜಟೀ ಚೀರವಾಸಾಶ್ಚ ರುದ್ರಃ ಸೇನಾಪತಿರ್ವಿಭುಃ ।\\
ಅಹಶ್ಚರೋ ನಕ್ತಂಚರಸ್ತಿಗ್ಮಮನ್ಯುಃ ಸುವರ್ಚಸಃ ॥೪೭॥

ಗಜಹಾ ದೈತ್ಯಹಾ ಕಾಲೋ ಲೋಕಧಾತಾ ಗುಣಾಕರಃ ।\\
ಸಿಂಹಶಾರ್ದೂಲರೂಪಶ್ಚ ಆರ್ದ್ರಚರ್ಮಾಂಬರಾವೃತಃ ॥೪೮॥

ಕಾಲಯೋಗೀ ಮಹಾನಾದಃ ಸರ್ವಕಾಮಶ್ಚತುಷ್ಪಥಃ ।\\
ನಿಶಾಚರಃ ಪ್ರೇತಚಾರೀ ಭೂತಚಾರೀ ಮಹೇಶ್ವರಃ ॥೪೯॥

ಬಹುಭೂತೋ ಬಹುಧರಃ ಸ್ವರ್ಭಾನುರಮಿತೋ ಗತಿಃ ।\\
ನೃತ್ಯಪ್ರಿಯೋ ನಿತ್ಯನರ್ತೋ ನರ್ತಕಃ ಸರ್ವಲಾಲಸಃ ॥೫೦॥

ಘೋರೋ ಮಹಾತಪಾಃ ಪಾಶೋ ನಿತ್ಯೋ ಗಿರಿರುಹೋ ನಭಃ ।\\
ಸಹಸ್ರಹಸ್ತೋ ವಿಜಯೋ ವ್ಯವಸಾಯೋ ಹ್ಯತಂದ್ರಿತಃ ॥೫೧॥

ಅಧರ್ಷಣೋ ಧರ್ಷಣಾತ್ಮಾ ಯಜ್ಞಹಾ ಕಾಮನಾಶಕಃ ।\\
ದಕ್ಷಯಾಗಾಪಹಾರೀ ಚ ಸುಸಹೋ ಮಧ್ಯಮಸ್ತಥಾ ॥೫೨॥

ತೇಜೋಪಹಾರೀ ಬಲಹಾ ಮುದಿತೋಽರ್ಥೋಽಜಿತೋಽವರಃ ।\\
ಗಂಭೀರಘೋಷಾ ಗಂಭೀರೋ ಗಂಭೀರಬಲವಾಹನಃ ॥೫೩॥

ನ್ಯಗ್ರೋಧರೂಪೋ ನ್ಯಗ್ರೋಧೋ ವೃಕ್ಷಪರ್ಣಸ್ಥಿತಿರ್ವಿಭುಃ ।\\
ಸುತೀಕ್ಷ್ಣದಶನಶ್ಚೈವ ಮಹಾಕಾಯೋ ಮಹಾನನಃ ॥೫೪॥

ವಿಷ್ವಕ್ಸೇನೋ ಹರಿರ್ಯಜ್ಞಃ ಸಂಯುಗಾಪೀಡವಾಹನಃ ।\\
ತೀಕ್ಷ್ಣತಾಪಶ್ಚ ಹರ್ಯಶ್ವಃ ಸಹಾಯಃ ಕರ್ಮಕಾಲವಿತ್ ॥೫೫॥

ವಿಷ್ಣುಪ್ರಸಾದಿತೋ ಯಜ್ಞಃ ಸಮುದ್ರೋ ವಡವಾಮುಖಃ ।\\
ಹುತಾಶನಸಹಾಯಶ್ಚ ಪ್ರಶಾಂತಾತ್ಮಾ ಹುತಾಶನಃ ॥೫೬॥

ಉಗ್ರತೇಜಾ ಮಹಾತೇಜಾ ಜನ್ಯೋ ವಿಜಯಕಾಲವಿತ್ ।\\
ಜ್ಯೋತಿಷಾಮಯನಂ ಸಿದ್ಧಿಃ ಸರ್ವವಿಗ್ರಹ ಏವ ಚ ॥೫೭॥

ಶಿಖೀ ಮುಂಡೀ ಜಟೀ ಜ್ವಾಲೀ ಮೂರ್ತಿಜೋ ಮೂರ್ಧಗೋ ಬಲೀ ।\\
ವೇಣವೀ ಪಣವೀ ತಾಲೀ ಖಲೀ ಕಾಲಕಟಂಕಟಃ ॥೫೮॥

ನಕ್ಷತ್ರವಿಗ್ರಹಮತಿರ್ಗುಣಬುದ್ಧಿರ್ಲಯೋ ಗಮಃ ।\\
ಪ್ರಜಾಪತಿರ್ವಿಶ್ವಬಾಹುರ್ವಿಭಾಗಃ ಸರ್ವಗೋಮುಖಃ ॥೫೯॥

ವಿಮೋಚನಃ ಸುಸರಣೋ ಹಿರಣ್ಯಕವಚೋದ್ಭವಃ ।\\
ಮೇಢ್ರಜೋ ಬಲಚಾರೀ ಚ ಮಹೀಚಾರೀ ಸ್ರುತಸ್ತಥಾ ॥೬೦॥

ಸರ್ವತೂರ್ಯನಿನಾದೀ ಚ ಸರ್ವಾತೋದ್ಯಪರಿಗ್ರಹಃ ।\\
ವ್ಯಾಲರೂಪೋ ಗುಹಾವಾಸೀ ಗುಹೋ ಮಾಲೀ ತರಂಗವಿತ್ ॥೬೧॥

ತ್ರಿದಶಸ್ತ್ರಿಕಾಲಧೃಕ್ಕರ್ಮಸರ್ವಬಂಧವಿಮೋಚನಃ ।\\
ಬಂಧನಸ್ತ್ವಸುರೇಂದ್ರಾಣಾಂ ಯುಧಿ ಶತ್ರುವಿನಾಶನಃ ॥೬೨॥

ಸಾಂಖ್ಯಪ್ರಸಾದೋ ದುರ್ವಾಸಾಃ ಸರ್ವಸಾಧುನಿಷೇವಿತಃ ।\\
ಪ್ರಸ್ಕಂದನೋ ವಿಭಾಗಜ್ಞೋ ಅತುಲ್ಯೋ ಯಜ್ಞಭಾಗವಿತ್ ॥೬೩॥

ಸರ್ವವಾಸಃ ಸರ್ವಚಾರೀ ದುರ್ವಾಸಾ ವಾಸವೋಽಮರಃ ।\\
ಹೈಮೋ ಹೇಮಕರೋ ಯಜ್ಞಃ ಸರ್ವಧಾರೀ ಧರೋತ್ತಮಃ ॥೬೪॥

ಲೋಹಿತಾಕ್ಷೋ ಮಹಾಕ್ಷಶ್ಚ ವಿಜಯಾಕ್ಷೋ ವಿಶಾರದಃ ।\\
ಸಂಗ್ರಹೋ ನಿಗ್ರಹಃ ಕರ್ತಾ ಸರ್ಪಚೀರನಿವಾಸನಃ ॥೬೫॥

ಮುಖ್ಯೋಽಮುಖ್ಯಶ್ಚ ದೇಹಶ್ಚ ಕಾಹಲಿಃ ಸರ್ವಕಾಮದಃ ।\\
ಸರ್ವಕಾಸಪ್ರಸಾದಶ್ಚ ಸುಬಲೋ ಬಲರೂಪಧೃತ್ ॥೬೬॥

ಸರ್ವಕಾಮವರಶ್ಚೈವ ಸರ್ವದಃ ಸರ್ವತೋಮುಖಃ ।\\
ಆಕಾಶನಿರ್ವಿರೂಪಶ್ಚ ನಿಪಾತೀ ಹ್ಯವಶಃ ಖಗಃ ॥೬೭॥

ರೌದ್ರರೂಪೋಂಽಶುರಾದಿತ್ಯೋ ಬಹುರಶ್ಮಿಃ ಸುವರ್ಚಸೀ ।\\
ವಸುವೇಗೋ ಮಹಾವೇಗೋ ಮನೋವೇಗೋ ನಿಶಾಚರಃ ॥೬೮॥

ಸರ್ವವಾಸೀ ಶ್ರಿಯಾವಾಸೀ ಉಪದೇಶಕರೋಽಕರಃ ।\\
ಮುನಿರಾತ್ಮನಿರಾಲೋಕಃ ಸಂಭಗ್ನಶ್ಚ ಸಹಸ್ರದಃ ॥೬೯॥

ಪಕ್ಷೀ ಚ ಪಕ್ಷರೂಪಶ್ಚ ಅತಿದೀಪ್ತೋ ವಿಶಾಂಪತಿಃ ।\\
ಉನ್ಮಾದೋ ಮದನಃ ಕಾಮೋ ಹ್ಯಶ್ವತ್ಥೋಽರ್ಥಕರೋ ಯಶಃ ॥೭೦॥

ವಾಮದೇವಶ್ಚ ವಾಮಶ್ಚ ಪ್ರಾಗ್ದಕ್ಷಿಣಶ್ಚ ವಾಮನಃ ।\\
ಸಿದ್ಧಯೋಗೀ ಮಹರ್ಷಿಶ್ಚ ಸಿದ್ಧಾರ್ಥಃ ಸಿದ್ಧಸಾಧಕಃ ॥೭೧॥

ಭಿಕ್ಷುಶ್ಚ ಭಿಕ್ಷುರೂಪಶ್ಚ ವಿಪಣೋ ಮೃದುರವ್ಯಯಃ ।\\
ಮಹಾಸೇನೋ ವಿಶಾಖಶ್ಚ ಷಷ್ಟಿಭಾಗೋ ಗವಾಂಪತಿಃ ॥೭೨॥

ವಜ್ರಹಸ್ತಶ್ಚ ವಿಷ್ಕಂಭೀ ಚಮೂಸ್ತಂಭನ ಏವ ಚ ।\\
ವೃತ್ತಾವೃತ್ತಕರಸ್ತಾಲೋ ಮಧುರ್ಮಧುಕಲೋಚನಃ ॥೭೩॥

ವಾಚಸ್ಪತ್ಯೋ ವಾಜಸನೋ ನಿತ್ಯಮಾಶ್ರಮಪೂಜಿತಃ ।\\
ಬ್ರಹ್ಮಚಾರೀ ಲೋಕಚಾರೀ ಸರ್ವಚಾರೀ ವಿಚಾರವಿತ್ ॥೭೪॥

ಈಶಾನ ಈಶ್ವರಃ ಕಾಲೋ ನಿಶಾಚಾರೀ ಪಿನಾಕವಾನ್ ।\\
ನಿಮಿತ್ತಸ್ಥೋ ನಿಮಿತ್ತಂ ಚ ನಂದಿರ್ನಂದಿಕರೋ ಹರಿಃ ॥೭೫॥

ನಂದೀಶ್ವರಶ್ಚ ನಂದೀ ಚ ನಂದನೋ ನಂದಿವರ್ಧನಃ ।\\
ಭಗಹಾರೀ ನಿಹಂತಾ ಚ ಕಾಲೋ ಬ್ರಹ್ಮಾ ಪಿತಾಮಹಃ ॥೭೬॥

ಚತುರ್ಮುಖೋ ಮಹಾಲಿಂಗಶ್ಚಾರುಲಿಂಗಸ್ತಥೈವ ಚ ।\\
ಲಿಂಗಾಧ್ಯಕ್ಷಃ ಸುರಾಧ್ಯಕ್ಷೋ ಯೋಗಾಧ್ಯಕ್ಷೋ ಯುಗಾವಹಃ ॥೭೭॥

ಬೀಜಾಧ್ಯಕ್ಷೋ ಬೀಜಕರ್ತಾ ಅವ್ಯಾತ್ಮಾಽನುಗತೋ ಬಲಃ ।\\
ಇತಿಹಾಸಃ ಸಕಲ್ಪಶ್ಚ ಗೌತಮೋಽಥ ನಿಶಾಕರಃ ॥೭೮॥

ದಂಭೋ ಹ್ಯದಂಭೋ ವೈದಂಭೋ ವಶ್ಯೋ ವಶಕರಃ ಕಲಿಃ ।\\
ಲೋಕಕರ್ತಾ ಪಶುಪತಿರ್ಮಹಾಕರ್ತಾ ಹ್ಯನೌಷಧಃ ॥೭೯॥

ಅಕ್ಷರಂ ಪರಮಂ ಬ್ರಹ್ಮ ಬಲವಚ್ಛಕ್ರ ಏವ ಚ ।\\
ನೀತಿರ್ಹ್ಯನೀತಿಃ ಶುದ್ಧಾತ್ಮಾ ಶುದ್ಧೋ ಮಾನ್ಯೋ ಗತಾಗತಃ ॥೮೦॥

ಬಹುಪ್ರಸಾದಃ ಸುಸ್ವಪ್ನೋ ದರ್ಪಣೋಽಥ ತ್ವಮಿತ್ರಜಿತ್ ।\\
ವೇದಕಾರೋ ಮಂತ್ರಕಾರೋ ವಿದ್ವಾನ್ಸಮರಮರ್ದನಃ ॥೮೧॥

ಮಹಾಮೇಘನಿವಾಸೀ ಚ ಮಹಾಘೋರೋ ವಶೀಕರಃ ।\\
ಅಗ್ನಿರ್ಜ್ವಾಲೋ ಮಹಾಜ್ವಾಲೋ ಅತಿಧೂಮ್ರೋ ಹುತೋ ಹವಿಃ ॥೮೨॥

ವೃಷಣಃ ಶಂಕರೋ ನಿತ್ಯಂ ವರ್ಚಸ್ವೀ ಧೂಮಕೇತನಃ ।\\
ನೀಲಸ್ತಥಾಂಗಲುಬ್ಧಶ್ಚ ಶೋಭನೋ ನಿರವಗ್ರಹಃ ॥೮೩॥

ಸ್ವಸ್ತಿದಃ ಸ್ವಸ್ತಿಭಾವಶ್ಚ ಭಾಗೀ ಭಾಗಕರೋ ಲಘುಃ ।\\
ಉತ್ಸಂಗಶ್ಚ ಮಹಾಂಗಶ್ಚ ಮಹಾಗರ್ಭಪರಾಯಣಃ ॥೮೪॥

ಕೃಷ್ಣವರ್ಣಃ ಸುವರ್ಣಶ್ಚ ಇಂದ್ರಿಯಂ ಸರ್ವದೇಹಿನಾಂ ।\\
ಮಹಾಪಾದೋ ಮಹಾಹಸ್ತೋ ಮಹಾಕಾಯೋ ಮಹಾಯಶಾಃ ॥೮೫॥

ಮಹಾಮೂರ್ಧಾ ಮಹಾಮಾತ್ರೋ ಮಹಾನೇತ್ರೋ ನಿಶಾಲಯಃ ।\\
ಮಹಾಂತಕೋ ಮಹಾಕರ್ಣೋ ಮಹೋಷ್ಠಶ್ಚ ಮಹಾಹನುಃ ॥೮೬॥

ಮಹಾನಾಸೋ ಮಹಾಕಂಬುರ್ಮಹಾಗ್ರೀವಃ ಶ್ಮಶಾನಭಾಕ್ ।\\
ಮಹಾವಕ್ಷಾ ಮಹೋರಸ್ಕೋ ಹ್ಯಂತರಾತ್ಮಾ ಮೃಗಾಲಯಃ ॥೮೭॥

ಲಂಬನೋ ಲಂಬಿತೋಷ್ಠಶ್ಚ ಮಹಾಮಾಯಃ ಪಯೋನಿಧಿಃ ।\\
ಮಹಾದಂತೋ ಮಹಾದಂಷ್ಟ್ರೋ ಮಹಾಜಿಹ್ವೋ ಮಹಾಮುಖಃ ॥೮೮॥

ಮಹಾನಖೋ ಮಹಾರೋಮಾ ಮಹಾಕೇಶೋ ಮಹಾಜಟಃ ।\\
ಪ್ರಸನ್ನಶ್ಚ ಪ್ರಸಾದಶ್ಚ ಪ್ರತ್ಯಯೋ ಗಿರಿಸಾಧನಃ ॥೮೯॥

ಸ್ನೇಹನೋಽಸ್ನೇಹನಶ್ಚೈವ ಅಜಿತಶ್ಚ ಮಹಾಮುನಿಃ ।\\
ವೃಕ್ಷಾಕಾರೋ ವೃಕ್ಷಕೇತುರನಲೋ ವಾಯುವಾಹನಃ ॥೯೦॥

ಗಂಡಲೀ ಮೇರುಧಾಮಾ ಚ ದೇವಾಧಿಪತಿರೇವ ಚ ।\\
ಅಥರ್ವಶೀರ್ಷಃ ಸಾಮಾಸ್ಯ ಋಕ್ಸಹಸ್ರಾಮಿತೇಕ್ಷಣಃ ॥೯೧॥

ಯಜುಃಪಾದಭುಜೋ ಗುಹ್ಯಃ ಪ್ರಕಾಶೋ ಜಂಗಮಸ್ತಥಾ ।\\
ಅಮೋಘಾರ್ಥಃ ಪ್ರಸಾದಶ್ಚ ಅಭಿಗಮ್ಯಃ ಸುದರ್ಶನಃ ॥೯೨॥

ಉಪಕಾರಃ ಪ್ರಿಯಃ ಸರ್ವಃ ಕನಕಃ ಕಾಂಚನಚ್ಛವಿಃ ।\\
ನಾಭಿರ್ನಂದಿಕರೋ ಭಾವಃ ಪುಷ್ಕರಸ್ಥಪತಿಃ ಸ್ಥಿರಃ ॥೯೩॥

ದ್ವಾದಶಸ್ತ್ರಾಸನಶ್ಚಾದ್ಯೋ ಯಜ್ಞೋ ಯಜ್ಞಸಮಾಹಿತಃ ।\\
ನಕ್ತಂ ಕಲಿಶ್ಚ ಕಾಲಶ್ಚ ಮಕರಃ ಕಾಲಪೂಜಿತಃ ॥೯೪॥

ಸಗಣೋ ಗಣಕಾರಶ್ಚ ಭೂತವಾಹನಸಾರಥಿಃ ।\\
ಭಸ್ಮಶಯೋ ಭಸ್ಮಗೋಪ್ತಾ ಭಸ್ಮಭೂತಸ್ತರುರ್ಗಣಃ ॥೯೫॥

ಲೋಕಪಾಲಸ್ತಥಾ ಲೋಕೋ ಮಹಾತ್ಮಾ ಸರ್ವಪೂಜಿತಃ ।\\
ಶುಕ್ಲಸ್ತ್ರಿಶುಕ್ಲಃ ಸಂಪನ್ನಃ ಶುಚಿರ್ಭೂತನಿಷೇವಿತಃ ॥೯೬॥

ಆಶ್ರಮಸ್ಥಃ ಕ್ರಿಯಾವಸ್ಥೋ ವಿಶ್ವಕರ್ಮಮತಿರ್ವರಃ ।\\
ವಿಶಾಲಶಾಖಸ್ತಾಮ್ರೋಷ್ಠೋ ಹ್ಯಂಬುಜಾಲಃ ಸುನಿಶ್ಚಲಃ ॥೯೭॥

ಕಪಿಲಃ ಕಪಿಶಃ ಶುಕ್ಲ ಆಯುಶ್ಚೈವಿ ಪರೋಽಪರಃ ।\\
ಗಂಧರ್ವೋ ಹ್ಯದಿತಿಸ್ತಾರ್ಕ್ಷ್ಯಃ ಸುವಿಜ್ಞೇಯಃ ಸುಶಾರದಃ ॥೯೮॥

ಪರಶ್ವಧಾಯುಧೋ ದೇವ ಅನುಕಾರೀ ಸುಬಾಂಧವಃ ।\\
ತುಂಬವೀಣೋ ಮಹಾಕ್ರೋಧ ಊರ್ಧ್ವರೇತಾ ಜಲೇಶಯಃ ॥೯೯॥

ಉಗ್ರೋ ವಂಶಕರೋ ವಂಶೋ ವಂಶನಾದೋ ಹ್ಯನಿಂದಿತಃ ।\\
ಸರ್ವಾಂಗರೂಪೋ ಮಾಯಾವೀ ಸುಹೃದೋ ಹ್ಯನಿಲೋಽನಲಃ ॥೧೦೦॥

ಬಂಧನೋ ಬಂಧಕರ್ತಾ ಚ ಸುಬಂಧನವಿಮೋಚನಃ ।\\
ಸ ಯಜ್ಞಾರಿಃ ಸ ಕಾಮಾರಿರ್ಮಹಾದಂಷ್ಟ್ರೋ ಮಹಾಯುಧಃ ॥೧೦೧॥

ಬಹುಧಾನಿಂದಿತಃ ಶರ್ವಃ ಶಂಕರಃ ಶಂಕರೋಽಧನಃ ।\\
ಅಮರೇಶೋ ಮಹಾದೇವೋ ವಿಶ್ವದೇವಃ ಸುರಾರಿಹಾ ॥೧೦೨॥

ಅಹಿರ್ಬುಧ್ನ್ಯೋಽನಿಲಾಭಶ್ಚ ಚೇಕಿತಾನೋ ಹವಿಸ್ತಥಾ ।\\
ಅಜೈಕಪಾಚ್ಚ ಕಾಪಾಲೀ ತ್ರಿಶಂಕುರಜಿತಃ ಶಿವಃ ॥೧೦೩॥

ಧನ್ವಂತರಿರ್ಧೂಮಕೇತುಃ ಸ್ಕಂದೋ ವೈಶ್ರವಣಸ್ತಥಾ ।\\
ಧಾತಾ ಶಕ್ರಶ್ಚ ವಿಷ್ಣುಶ್ಚ ಮಿತ್ರಸ್ತ್ವಷ್ಟಾ ಧ್ರುವೋ ಧರಃ ॥೧೦೪॥

ಪ್ರಭಾವಃ ಸರ್ವಗೋ ವಾಯುರರ್ಯಮಾ ಸವಿತಾ ರವಿಃ ।\\
ಉಷಂಗುಶ್ಚ ವಿಧಾತಾ ಚ ಮಾಂಧಾತಾ ಭೂತಭಾವನಃ ॥೧೦೫॥

ವಿಭುರ್ವರ್ಣವಿಭಾವೀ ಚ ಸರ್ವಕಾಮಗುಣಾವಹಃ ।\\
ಪದ್ಮನಾಭೋ ಮಹಾಗರ್ಭಶ್ಚಂದ್ರವಕ್ತ್ರೋಽನಿಲೋಽನಲಃ ॥೧೦೬॥

ಬಲವಾಂಶ್ಚೋಪಶಾಂತಶ್ಚ ಪುರಾಣಃ ಪುಣ್ಯಚಂಚುರೀ ।\\
ಕುರುಕರ್ತಾ ಕುರುವಾಸೀ ಕುರುಭೂತೋ ಗುಣೌಷಧಃ ॥೧೦೭॥

ಸರ್ವಾಶಯೋ ದರ್ಭಚಾರೀ ಸರ್ವೇಷಾಂ ಪ್ರಾಣಿನಾಂ ಪತಿಃ ।\\
ದೇವದೇವಃ ಸುಖಾಸಕ್ತಃ ಸದಸತ್ಸರ್ವರತ್ನವಿತ್ ॥೧೦೮॥

ಕೈಲಾಸಗಿರಿವಾಸೀ ಚ ಹಿಮವದ್ಗಿರಿಸಂಶ್ರಯಃ ।\\
ಕೂಲಹಾರೀ ಕೂಲಕರ್ತಾ ಬಹುವಿದ್ಯೋ ಬಹುಪ್ರದಃ ॥೧೦೯॥

ವಣಿಜೋ ವರ್ಧಕೀ ವೃಕ್ಷೋ ಬಕುಲಶ್ಚಂದನಶ್ಛದಃ ।\\
ಸಾರಗ್ರೀವೋ ಮಹಾಜತ್ರುರಲೋಲಶ್ಚ ಮಹೌಷಧಃ ॥೧೧೦॥

ಸಿದ್ಧಾರ್ಥಕಾರೀ ಸಿದ್ಧಾರ್ಥಶ್ಛಂದೋವ್ಯಾಕರಣೋತ್ತರಃ ।\\
ಸಿಂಹನಾದಃ ಸಿಂಹದಂಷ್ಟ್ರಃ ಸಿಂಹಗಃ ಸಿಂಹವಾಹನಃ ॥೧೧೧॥

ಪ್ರಭಾವಾತ್ಮಾ ಜಗತ್ಕಾಲಸ್ಥಾಲೋ ಲೋಕಹಿತಸ್ತರುಃ ।\\
ಸಾರಂಗೋ ನವಚಕ್ರಾಂಗಃ ಕೇತುಮಾಲೀ ಸಭಾವನಃ ॥೧೧೨॥

ಭೂತಾಲಯೋ ಭೂತಪತಿರಹೋರಾತ್ರಮನಿಂದಿತಃ ॥೧೧೩॥

ವಾಹಿತಾ ಸರ್ವಭೂತಾನಾಂ ನಿಲಯಶ್ಚ ವಿಭುರ್ಭವಃ ।\\
ಅಮೋಘಃ ಸಂಯತೋ ಹ್ಯಶ್ವೋ ಭೋಜನಃ ಪ್ರಾಣಧಾರಣಃ ॥೧೧೪॥

ಧೃತಿಮಾನ್ಮತಿಮಾಂದಕ್ಷಃ ಸತ್ಕೃತಶ್ಚ ಯುಗಾಧಿಪಃ ।\\
ಗೋಪಾಲಿರ್ಗೋಪತಿರ್ಗ್ರಾಮೋ ಗೋಚರ್ಮವಸನೋ ಹರಿಃ। ೧೧೫॥

ಹಿರಣ್ಯಬಾಹುಶ್ಚ ತಥಾ ಗುಹಾಪಾಲಃ ಪ್ರವೇಶಿನಾಂ ।\\
ಪ್ರಕೃಷ್ಟಾರಿರ್ಮಹಾಹರ್ಷೋ ಜಿತಕಾಮೋ ಜಿತೇಂದ್ರಿಯಃ ॥೧೧೬॥

ಗಾಂಧಾರಶ್ಚ ಸುವಾಸಶ್ಚ ತಪಃಸಕ್ತೋ ರತಿರ್ನರಃ ।\\
ಮಹಾಗೀತೋ ಮಹಾನೃತ್ಯೋ ಹ್ಯಪ್ಸರೋಗಣಸೇವಿತಃ ॥೧೧೭॥

ಮಹಾಕೇತುರ್ಮಹಾಧಾತುರ್ನೈಕಸಾನುಚರಶ್ಚಲಃ ।\\
ಆವೇದನೀಯ ಆದೇಶಃ ಸರ್ವಗಂಧಸುಖಾವಹಃ ॥೧೧೮॥

ತೋರಣಸ್ತಾರಣೋ ವಾತಃ ಪರಿಧೀ ಪತಿಖೇಚರಃ ।\\
ಸಂಯೋಗೋ ವರ್ಧನೋ ವೃದ್ಧೋ ಅತಿವೃದ್ಧೋ ಗುಣಾಧಿಕಃ ॥೧೧೯॥

ನಿತ್ಯ ಆತ್ಮಸಹಾಯಶ್ಚ ದೇವಾಸುರಪತಿಃ ಪತಿಃ ।\\
ಯುಕ್ತಶ್ಚ ಯುಕ್ತಬಾಹುಶ್ಚ ದೇವೋ ದಿವಿ ಸುಪರ್ವಣಃ ॥೧೨೦॥

ಆಷಾಢಶ್ಚ ಸುಷಾಂಢಶ್ಚ ಧ್ರುವೋಽಥ ಹರಿಣೋ ಹರಃ ।\\
ವಪುರಾವರ್ತಮಾನೇಭ್ಯೋ ವಸುಶ್ರೇಷ್ಠೋ ಮಹಾಪಥಃ ॥೧೨೧॥

ಶಿರೋಹಾರೀ ವಿಮರ್ಶಶ್ಚ ಸರ್ವಲಕ್ಷಣಲಕ್ಷಿತಃ ।\\
ಅಕ್ಷಶ್ಚ ರಥಯೋಗೀ ಚ ಸರ್ವಯೋಗೀ ಮಹಾಬಲಃ ॥೧೨೨॥

ಸಮಾಮ್ನಾಯೋಽಸಮಾಮ್ನಾಯಸ್ತೀರ್ಥದೇವೋ ಮಹಾರಥಃ ।\\
ನಿರ್ಜೀವೋ ಜೀವನೋ ಮಂತ್ರಃ ಶುಭಾಕ್ಷೋ ಬಹುಕರ್ಕಶಃ ॥೧೨೩॥

ರತ್ನಪ್ರಭೂತೋ ರತ್ನಾಂಗೋ ಮಹಾರ್ಣವನಿಪಾನವಿತ್ ।\\
ಮೂಲಂ ವಿಶಾಲೋ ಹ್ಯಮೃತೋ ವ್ಯಕ್ತಾವ್ಯಕ್ತಸ್ತಪೋನಿಧಿಃ ॥೧೨೪॥

ಆರೋಹಣೋಽಧಿರೋಹಶ್ಚ ಶೀಲಧಾರೀ ಮಹಾಯಶಾಃ ।\\
ಸೇನಾಕಲ್ಪೋ ಮಹಾಕಲ್ಪೋ ಯೋಗೋ ಯುಗಕರೋ ಹರಿಃ ॥೧೨೫॥

ಯುಗರೂಪೋ ಮಹಾರೂಪೋ ಮಹಾನಾಗಹನೋ ವಧಃ ।\\
ನ್ಯಾಯನಿರ್ವಪಣಃ ಪಾದಃ ಪಂಡಿತೋ ಹ್ಯಚಲೋಪಮಃ ॥೧೨೬॥

ಬಹುಮಾಲೋ ಮಹಾಮಾಲಃ ಶಶೀ ಹರಸುಲೋಚನಃ ।\\
ವಿಸ್ತಾರೋ ಲವಣಃ ಕೂಪಸ್ತ್ರಿಯುಗಃ ಸಫಲೋದಯಃ ॥೧೨೭॥

ತ್ರಿಲೋಚನೋ ವಿಷಣ್ಣಾಂಗೋ ಮಣಿವಿದ್ಧೋ ಜಟಾಧರಃ ।\\
ಬಿಂದುರ್ವಿಸರ್ಗಃ ಸುಮುಖಃ ಶರಃ ಸರ್ವಾಯುಧಃ ಸಹಃ ॥೧೨೮॥

ನಿವೇದನಃ ಸುಖಾಜಾತಃ ಸುಗಂಧಾರೋ ಮಹಾಧನುಃ ।\\
ಗಂಧಪಾಲೀ ಚ ಭಗವಾನುತ್ಥಾನಃ ಸರ್ವಕರ್ಮಣಾಂ ॥೧೨೯॥

ಮಂಥಾನೋ ಬಹುಲೋ ವಾಯುಃ ಸಕಲಃ ಸರ್ವಲೋಚನಃ ।\\
ತಲಸ್ತಾಲಃ ಕರಸ್ಥಾಲೀ ಊರ್ಧ್ವಸಂಹನನೋ ಮಹಾನ್ ॥೧೩೦॥

ಛತ್ರಂ ಸುಚ್ಛತ್ರೋ ವಿಖ್ಯಾತೋ ಲೋಕಃ ಸರ್ವಾಶ್ರಯಃ ಕ್ರಮಃ ।\\
ಮುಂಡೋ ವಿರೂಪೋ ವಿಕೃತೋ ದಂಡೀ ಕುಂಡೀ ವಿಕುರ್ವಣಃ। ೧೩೧॥

ಹರ್ಯಕ್ಷಃ ಕಕುಭೋ ವಜ್ರೋ ಶತಜಿಹ್ವಃ ಸಹಸ್ರಪಾತ್ ।\\
ಸಹಸ್ರಮೂರ್ಧಾ ದೇವೇಂದ್ರಃ ಸರ್ವದೇವಮಯೋ ಗುರುಃ ॥೧೩೨॥

ಸಹಸ್ರಬಾಹುಃ ಸರ್ವಾಂಗಃ ಶರಣ್ಯಃ ಸರ್ವಲೋಕಕೃತ್ ।\\
ಪವಿತ್ರಂ ತ್ರಿಕಕುನ್ಮಂತ್ರಃ ಕನಿಷ್ಠಃ ಕೃಷ್ಣಪಿಂಗಲಃ। ೧೩೩॥

ಬ್ರಹ್ಮದಂಡವಿನಿರ್ಮಾತಾ ಶತಘ್ನೀಪಾಶಶಕ್ತಿಮಾನ್ ।\\
ಪದ್ಮಗರ್ಭೋ ಮಹಾಗರ್ಭೋ ಬ್ರಹ್ಮಗರ್ಭೋ ಜಲೋದ್ಭವಃ ॥೧೩೪॥

ಗಭಸ್ತಿರ್ಬ್ರಹ್ಮಕೃದ್ಬ್ರಹ್ಮೀ ಬ್ರಹ್ಮವಿದ್ಬ್ರಾಹ್ಮಣೋ ಗತಿಃ ।\\
ಅನಂತರೂಪೋ ನೈಕಾತ್ಮಾ ತಿಗ್ಮತೇಜಾಃ ಸ್ವಯಂಭುವಃ ॥೧೩೫॥

ಊರ್ಧ್ವಗಾತ್ಮಾ ಪಶುಪತಿರ್ವಾತರಂಹಾ ಮನೋಜವಃ ।\\
ಚಂದನೀ ಪದ್ಮನಾಲಾಗ್ರಃ ಸುರಭ್ಯುತ್ತರಣೋ ನರಃ ॥೧೩೬॥

ಕರ್ಣಿಕಾರಮಹಾಸ್ರಗ್ವೀ ನೀಲಮೌಲಿಃ ಪಿನಾಕಧೃತ್ ।\\
ಉಮಾಪತಿರುಮಾಕಾಂತೋ ಜಾಹ್ನವೀಧೃಗುಮಾಧವಃ ॥೧೩೭॥

ವರೋ ವರಾಹೋ ವರದೋ ವರೇಣ್ಯಃ ಸುಮಹಾಸ್ವನಃ ।\\
ಮಹಾಪ್ರಸಾದೋ ದಮನಃ ಶತ್ರುಹಾ ಶ್ವೇತಪಿಂಗಲಃ ॥೧೩೮॥

ಪೀತಾತ್ಮಾ ಪರಮಾತ್ಮಾ ಚ ಪ್ರಯತಾತ್ಮಾ ಪ್ರಧಾನಧೃತ್ ।\\
ಸರ್ವಪಾರ್ಶ್ವಮುಖಸ್ತ್ರ್ಯಕ್ಷೋ ಧರ್ಮಸಾಧಾರಣೋ ವರಃ ॥೧೩೯॥

ಚರಾಚರಾತ್ಮಾ ಸೂಕ್ಷ್ಮಾತ್ಮಾ ಅಮೃತೋ ಗೋವೃಷೇಶ್ವರಃ ।\\
ಸಾಧ್ಯರ್ಷಿರ್ವಸುರಾದಿತ್ಯೋ ವಿವಸ್ವಾನ್ಸವಿತಾಽಮೃತಃ ೧೪೦॥

ವ್ಯಾಸಃ ಸರ್ಗಃ ಸುಸಂಕ್ಷೇಪೋ ವಿಸ್ತರಃ ಪರ್ಯಯೋ ನರಃ ।\\
ಋತು ಸಂವತ್ಸರೋ ಮಾಸಃ ಪಕ್ಷಃ ಸಂಖ್ಯಾಸಮಾಪನಃ ॥೧೪೧॥

ಕಲಾ ಕಾಷ್ಠಾ ಲವಾ ಮಾತ್ರಾ ಮುಹೂರ್ತಾಹಃಕ್ಷಪಾಃ ಕ್ಷಣಾಃ ।\\
ವಿಶ್ವಕ್ಷೇತ್ರಂ ಪ್ರಜಾಬೀಜಂ ಲಿಂಗಮಾದ್ಯಸ್ತು ನಿರ್ಗಮಃ ॥೧೪೨॥

ಸದಸದ್ವ್ಯಕ್ತಮವ್ಯಕ್ತಂ ಪಿತಾ ಮಾತಾ ಪಿತಾಮಹಃ ।\\
ಸ್ವರ್ಗದ್ವಾರಂ ಪ್ರಜಾದ್ವಾರಂ ಮೋಕ್ಷದ್ವಾರಂ ತ್ರಿವಿಷ್ಟಪಂ ॥೧೪೩॥

ನಿರ್ವಾಣಂ ಹ್ಲಾದನಶ್ಚೈವ ಬ್ರಹ್ಮಲೋಕಃ ಪರಾ ಗತಿಃ ।\\
ದೇವಾಸುರವಿನಿರ್ಮಾತಾ ದೇವಾಸುರಪರಾಯಣಃ ॥೧೪೪॥

ದೇವಾಸುರಗುರುರ್ದೇವೋ ದೇವಾಸುರನಮಸ್ಕೃತಃ ।\\
ದೇವಾಸುರಮಹಾಮಾತ್ರೋ ದೇವಾಸುಗಣಾಶ್ರಯಃ ॥೧೪೫॥

ದೇವಾಸುರಗಣಾಧ್ಯಕ್ಷೋ ದೇವಾಸುರಗಣಾಗ್ರಣೀಃ ।\\
ದೇವಾತಿದೇವೋ ದೇವರ್ಷಿರ್ದೇವಾಸುರವರಪ್ರದಃ ॥೧೪೬॥

ದೇವಾಸುರೇಶ್ವರೋ ವಿಶ್ವೋ ದೇವಾಸುರಮಹೇಶ್ವರಃ ।\\
ಸರ್ವದೇವಮಯೋಽಚಿಂತ್ಯೋ ದೇವತಾತ್ಮಾಽಽತ್ಮಸಂಭವಃ ॥೧೪೭॥

ಉದ್ಭಿತ್ತ್ರಿವಿಕ್ರಮೋ ವೈದ್ಯೋ ವಿರಜೋ ನೀರಜೋಽಮರಃ ॥

ಈಡ್ಯೋ ಹಸ್ತೀಶ್ವರೋ ವ್ಯಾಘ್ರೋ ದೇವಸಿಂಹೋ ನರರ್ಷಭಃ ॥೧೪೮॥

ವಿಬುಧೋಽಗ್ರವರಃ ಸೂಕ್ಷ್ಮಃ ಸರ್ವದೇವಸ್ತಪೋಮಯಃ ।\\
ಸುಯುಕ್ತಃ ಶೋಭನೋ ವಜ್ರೀ ಪ್ರಾಸಾನಾಂ ಪ್ರಭವೋಽವ್ಯಯಃ ॥೧೪೯॥

ಗುಹಃ ಕಾಂತೋ ನಿಜಃ ಸರ್ಗಃ ಪವಿತ್ರಂ ಸರ್ವಪಾವನಃ ।\\
ಶೃಂಗೀ ಶೃಂಗಪ್ರಿಯೋ ಬಭ್ರೂ ರಾಜರಾಜೋ ನಿರಾಮಯಃ ॥೧೫೦॥

ಅಭಿರಾಮಃ ಸುರಗಣೋ ವಿರಾಮಃ ಸರ್ವಸಾಧನಃ ।\\
ಲಲಾಟಾಕ್ಷೋ ವಿಶ್ವದೇವೋ ಹರಿಣೋ ಬ್ರಹ್ಮವರ್ಚಸಃ ॥೧೫೧॥

ಸ್ಥಾವರಾಣಾಂ ಪತಿಶ್ಚೈವ ನಿಯಮೇಂದ್ರಿಯವರ್ಧನಃ ।\\
ಸಿದ್ಧಾರ್ಥಃ ಸಿದ್ಧಭೂತಾರ್ಥೋಽಚಿಂತ್ಯಃ ಸತ್ಯವ್ರತಃ ಶುಚಿಃ ॥೧೫೨॥

ವ್ರತಾಧಿಪಃ ಪರಂ ಬ್ರಹ್ಮ ಭಕ್ತಾನಾಂ ಪರಮಾ ಗತಿಃ ।\\
ವಿಮುಕ್ತೋ ಮುಕ್ತತೇಜಾಶ್ಚ ಶ್ರೀಮಾನ್ಶ್ರೀವರ್ಧನೋ ಜಗತ್ ॥೧೫೩॥
\authorline{{ಇತಿ ಶಿವ ಸಹಸ್ರನಾಮಸ್ತೋತ್ರಮ್}}
%==================================================================================================================================
\section{ಅಥ ಶಿವಾಷ್ಟೋತ್ತರಶತನಾಮಸ್ತೋತ್ರಮ್ }
\addcontentsline{toc}{section}{ಅಥ ಶಿವಾಷ್ಟೋತ್ತರಶತನಾಮಸ್ತೋತ್ರಮ್ }
ಶಿವೋ ಮಹೇಶ್ವರಃ ಶಂಭುಃ ಪಿನಾಕೀ ಶಶಿಶೇಖರಃ~।\\
ವಾಮದೇವೋ ವಿರೂಪಾಕ್ಷಃ ಕಪರ್ದೀ ನೀಲಲೋಹಿತಃ ॥೧ ॥

ಶಂಕರಃ ಶೂಲಪಾಣಿಶ್ಚ ಖಟ್ವಾಂಗೀ ವಿಷ್ಣುವಲ್ಲಭಃ~।\\
ಶಿಪಿವಿಷ್ಟೋಂಬಿಕಾನಾಥಃ ಶ್ರೀಕಂಠೋ ಭಕ್ತವತ್ಸಲಃ ॥೨ ॥

ಭವಃ ಶರ್ವಸ್ತ್ರಿಲೋಕೇಶಃ ಶಿತಿಕಂಠಃ ಶಿವಾಪ್ರಿಯಃ।\\
ಉಗ್ರಃ ಕಪಾಲೀ ಕಾಮಾರಿರಂಧಕಾಸುರಸೂದನಃ ॥೩ ॥

ಗಂಗಾಧರೋ ಲಲಾಟಾಕ್ಷಃ ಕಾಲಕಾಲಃ ಕೃಪಾನಿಧಿಃ~।\\
ಭೀಮಃ ಪರಶುಹಸ್ತಶ್ಚ ಮೃಗಪಾಣಿರ್ಜಟಾಧರಃ ॥೪ ॥

ಕೈಲಾಸವಾಸೀ ಕವಚೀ ಕಠೋರಸ್ತ್ರಿಪುರಾಂತಕಃ।\\
ವೃಷಾಂಕೋ ವೃಷಭಾರೂಢೋ ಭಸ್ಮೋದ್ಧೂಲಿತವಿಗ್ರಹಃ ॥೫ ॥

ಸಾಮಪ್ರಿಯಃ ಸ್ವರಮಯಸ್ತ್ರಯೀಮೂರ್ತಿರನೀಶ್ವರಃ~।\\
ಸರ್ವಜ್ಞಃ ಪರಮಾತ್ಮಾ ಚ ಸೋಮಸೂರ್ಯಾಗ್ನಿಲೋಚನಃ ॥೬॥

ಹವಿರ್ಯಜ್ಞಮಯಃ ಸೋಮಃ ಪಂಚವಕ್ತ್ರಃ ಸದಾಶಿವಃ।\\
ವಿಶ್ವೇಶ್ವರೋ ವೀರಭದ್ರೋ ಗಣನಾಥಃ ಪ್ರಜಾಪತಿಃ ॥೭॥

ಹಿರಣ್ಯರೇತಾ ದುರ್ಧರ್ಷೋ ಗಿರೀಶೋ ಗಿರಿಶೋನಘಃ।\\
ಭುಜಂಗಭೂಷಣೋ ಭರ್ಗೋ ಗಿರಿಧನ್ವಾ ಗಿರಿಪ್ರಿಯಃ ॥೮॥

ಕೃತ್ತಿವಾಸಾಃ ಪುರಾರಾತಿರ್ಭಗವಾನ್ ಪ್ರಮಥಾಧಿಪಃ।\\
ಮೃತ್ಯುಂಜಯಃ ಸೂಕ್ಷ್ಮತನುರ್ಜಗದ್ವ್ಯಾಪೀ ಜಗದ್ಗುರುಃ ॥೯॥

ವ್ಯೋಮಕೇಶೋ ಮಹಾಸೇನಜನಕಶ್ಚಾರುವಿಕ್ರಮಃ।\\
ರುದ್ರೋ ಭೂತಪತಿಃ ಸ್ಥಾಣುರಹಿರ್ಬುಧ್ನ್ಯೋ ದಿಗಂಬರಃ ॥೧೦॥

ಅಷ್ಟಮೂರ್ತಿರನೇಕಾತ್ಮಾ ಸಾತ್ವಿಕಃ ಶುದ್ಧವಿಗ್ರಹಃ।\\
ಶಾಶ್ವತಃ ಖಂಡಪರಶುರಜಃ ಪಾಶವಿಮೋಚನಃ ॥೧೧॥

ಮೃಡಃ ಪಶುಪತಿರ್ದೇವೋ ಮಹಾದೇವೋಽವ್ಯಯೋ ಹರಿಃ।\\
ಪೂಷದಂತಭಿದವ್ಯಗ್ರೋ ದಕ್ಷಾಧ್ವರಹರೋ ಹರಃ ॥೧೨॥

ಭಗನೇತ್ರಭಿದವ್ಯಕ್ತಃ ಸಹಸ್ರಾಕ್ಷಃ ಸಹಸ್ರಪಾತ್।\\
ಅಪವರ್ಗಪ್ರದೋಽನಂತಸ್ತಾರಕಃ ಪರಮೇಶ್ವರಃ ॥೧೩॥
\authorline{॥ಇತಿ ಶಿವಾಷ್ಟೋತ್ತರ ಶತನಾಮಸ್ತೋತ್ರಂ ಸಂಪೂರ್ಣಂ॥}
%===================================================================================

\section{ಶ್ರೀರಾಮಸಹಸ್ರನಾಮಸ್ತೋತ್ರಮ್}
\addcontentsline{toc}{section}{ಶ್ರೀರಾಮಸಹಸ್ರನಾಮಸ್ತೋತ್ರಮ್}
ಓಂ ಅಸ್ಯ ಶ್ರೀರಾಮಸಹಸ್ರನಾಮಮಾಲಾಮಂತ್ರಸ್ಯ ವಿನಾಯಕ ಋಷಿಃ । ಅನುಷ್ಟುಪ್ ಛಂದಃ । ಶ್ರೀರಾಮೋ ದೇವತಾ । ಮಹಾವಿಷ್ಣುರಿತಿ ಬೀಜಂ । ಗುಣಭೃನ್ನಿರ್ಗುಣೋ ಮಹಾನಿತಿ ಶಕ್ತಿಃ । ಸಚ್ಚಿದಾನಂದವಿಗ್ರಹ ಇತಿ ಕೀಲಕಂ । ಶ್ರೀರಾಮಪ್ರೀತ್ಯರ್ಥೇ ಜಪೇ ವಿನಿಯೋಗಃ ॥\\
ನ್ಯಾಸಃ : ೧.ಓಂ ಶ್ರೀರಾಮಚಂದ್ರಾಯ ೨.ಸೀತಾಪತಯೇ ೩.ರಘುನಾಥಾಯ ೪.ಭರತಾಗ್ರಜಾಯ  ೫.ದಶರಥಾತ್ಮಜಾಯ  ೬.ಹನುಮತ್ಪ್ರಭವೇ

ಅಥ ಧ್ಯಾನಂ ।\\
ಧ್ಯಾಯೇದಾಜಾನುಬಾಹುಂ ಧೃತಶರಧನುಷಂ ಬದ್ಧಪದ್ಮಾಸನಸ್ಥಂ\\
ಪೀತಂ ವಾಸೋ ವಸಾನಂ ನವಕಮಲಸ್ಪರ್ಧಿ ನೇತ್ರಂ ಪ್ರಸನ್ನಂ ।\\
ವಾಮಾಂಕಾರೂಢಸೀತಾಮುಖಕಮಲಮಿಲಲ್ಲೋಚನಂ ನೀರದಾಭಂ\\
ನಾನಾಲಂಕಾರದೀಪ್ತಂ ದಧತಮುರುಜಟಾಮಂಡಲಂ ರಾಮಚಂದ್ರಂ ॥೩೧॥

ವೈದೇಹೀಸಹಿತಂ ಸುರದ್ರುಮತಲೇ ಹೈಮೇ ಮಹಾಮಂಡಪೇ\\
ಮಧ್ಯೇ ಪುಷ್ಪಕಮಾಸನೇ ಮಣಿಮಯೇ ವೀರಾಸನೇ ಸಂಸ್ಥಿತಂ ।\\
ಅಗ್ರೇ ವಾಚಯತಿ ಪ್ರಭಂಜನೇಸುತೇ ತತ್ತ್ವಂ ಮುನಿಭ್ಯಃ ಪರಂ\\
ವ್ಯಾಖ್ಯಾಂತಂ ಭರತಾದಿಭಿಃ ಪರಿವೃತಂ ರಾಮಂ ಭಜೇ ಶ್ಯಾಮಲಂ ॥೩೨॥

ಸೌವರ್ಣಮಂಡಪೇ ದಿವ್ಯೇ ಪುಷ್ಪಕೇ ಸುವಿರಾಜಿತೇ ।\\
ಮೂಲೇ ಕಲ್ಪತರೋಃ ಸ್ವರ್ಣಪೀಠೇ ಸಿಂಹಾಷ್ಟಸಂಯುತೇ ॥೩೩॥

ಮೃದುಶ್ಲಕ್ಷ್ಣತರೇ ತತ್ರ ಜಾನಕ್ಯಾ ಸಹ ಸಂಸ್ಥಿತಂ ।\\
ರಾಮಂ ನೀಲೋತ್ಪಲಶ್ಯಾಮಂ ದ್ವಿಭುಜಂ ಪೀತವಾಸಸಂ ॥೩೪॥

ಸ್ಮಿತವಕ್ತ್ರಂ ಸುಖಾಸೀನಂ ಪದ್ಮಪತ್ರನಿಭೇಕ್ಷಣಂ ।\\
ಕಿರೀಟಹಾರಕೇಯೂರಕುಂಡಲೈಃ ಕಟಕಾದಿಭಿಃ ॥೩೫॥

ಭ್ರಾಜಮಾನಂ ಜ್ಞಾನಮುದ್ರಾಧರಂ ವೀರಾಸನಸ್ಥಿತಂ ।\\
ಸ್ಪೃಶಂತಂ ಸ್ತನಯೋರಗ್ರೇ ಜಾನಕ್ಯಾಃ ಸವ್ಯಪಾಣಿನಾ ॥೩೬॥

ವಸಿಷ್ಠವಾಮದೇವಾದ್ಯೈಃ ಸೇವಿತಂ ಲಕ್ಷ್ಮಣಾದಿಭಿಃ ।\\
ಅಯೋಧ್ಯಾನಗರೇ ರಮ್ಯೇ ಹ್ಯಭಿಷಿಕ್ತಂ ರಘೂದ್ವಹಂ ॥೩೭॥

ಏವಂ ಧ್ಯಾತ್ವಾ ಜಪೇನ್ನಿತ್ಯಂ ರಾಮನಾಮಸಹಸ್ರಕಂ ।\\
ಹತ್ಯಾಕೋಟಿಯುತೋ ವಾಪಿ ಮುಚ್ಯತೇ ನಾತ್ರ ಸಂಶಯಃ ॥೩೮॥

ಅಥ ಸಹಸ್ರನಾಮ ಸ್ತೋತ್ರಮ್\\
ಓಂ ರಾಮಃ ಶ್ರೀಮಾನ್ಮಹಾವಿಷ್ಣುರ್ಜಿಷ್ಣುರ್ದೇವಹಿತಾವಹಃ ।\\
ತತ್ತ್ವಾತ್ಮಾ ತಾರಕಬ್ರಹ್ಮ ಶಾಶ್ವತಃ ಸರ್ವಸಿದ್ಧಿದಃ ॥೩೯॥

ರಾಜೀವಲೋಚನಃ ಶ್ರೀಮಾನ್ ಶ್ರೀರಾಮೋ ರಘುಪುಂಗವಃ ।\\
ರಾಮಭದ್ರಃ ಸದಾಚಾರೋ ರಾಜೇಂದ್ರೋ ಜಾನಕೀಪತಿಃ ॥೪೦॥

ಅಗ್ರಗಣ್ಯೋ ವರೇಣ್ಯಶ್ಚ ವರದಃ ಪರಮೇಶ್ವರಃ ।\\
ಜನಾರ್ದನೋ ಜಿತಾಮಿತ್ರಃ ಪರಾರ್ಥೈಕಪ್ರಯೋಜನಃ ॥೪೧॥

ವಿಶ್ವಾಮಿತ್ರಪ್ರಿಯೋ ದಾತಾ ಶತ್ರುಜಿಚ್ಛತ್ರುತಾಪನಃ ।\\
ಸರ್ವಜ್ಞಃ ಸರ್ವವೇದಾದಿಃ ಶರಣ್ಯೋ ವಾಲಿಮರ್ದನಃ ॥೪೨॥

ಜ್ಞಾನಭವ್ಯೋಽಪರಿಚ್ಛೇದ್ಯೋ ವಾಗ್ಮೀ ಸತ್ಯವ್ರತಃ ಶುಚಿಃ ।\\
ಜ್ಞಾನಗಮ್ಯೋ ದೃಢಪ್ರಜ್ಞಃ ಖರಧ್ವಂಸಃ ಪ್ರತಾಪವಾನ್ ॥೪೩॥

ದ್ಯುತಿಮಾನಾತ್ಮವಾನ್ ವೀರೋ ಜಿತಕ್ರೋಧೋಽರಿಮರ್ದನಃ ।\\
ವಿಶ್ವರೂಪೋ ವಿಶಾಲಾಕ್ಷಃ ಪ್ರಭುಃ ಪರಿವೃಢೋ ದೃಢಃ ॥೪೪॥

ಈಶಃ ಖಡ್ಗಧರಃ ಶ್ರೀಮಾನ್ ಕೌಸಲ್ಯೇಯೋಽನಸೂಯಕಃ ।\\
ವಿಪುಲಾಂಸೋ ಮಹೋರಸ್ಕಃ ಪರಮೇಷ್ಠೀ ಪರಾಯಣಃ ॥೪೫॥

ಸತ್ಯವ್ರತಃ ಸತ್ಯಸಂಧೋ ಗುರುಃ ಪರಮಧಾರ್ಮಿಕಃ ।\\
ಲೋಕೇಶೋ ಲೋಕವಂದ್ಯಶ್ಚ ಲೋಕಾತ್ಮಾ ಲೋಕಕೃದ್ವಿಭುಃ ॥೪೬॥

ಅನಾದಿರ್ಭಗವಾನ್ ಸೇವ್ಯೋ ಜಿತಮಾಯೋ ರಘೂದ್ವಹಃ ।\\
ರಾಮೋ ದಯಾಕರೋ ದಕ್ಷಃ ಸರ್ವಜ್ಞಃ ಸರ್ವಪಾವನಃ ॥೪೭॥

ಬ್ರಹ್ಮಣ್ಯೋ ನೀತಿಮಾನ್ ಗೋಪ್ತಾ ಸರ್ವದೇವಮಯೋ ಹರಿಃ ।\\
ಸುಂದರಃ ಪೀತವಾಸಾಶ್ಚ ಸೂತ್ರಕಾರಃ ಪುರಾತನಃ ॥೪೮॥

ಸೌಮ್ಯೋ ಮಹರ್ಷಿಃ ಕೋದಂಡಃ ಸರ್ವಜ್ಞಃ ಸರ್ವಕೋವಿದಃ ।\\
ಕವಿಃ ಸುಗ್ರೀವವರದಃ ಸರ್ವಪುಣ್ಯಾಧಿಕಪ್ರದಃ ॥೪೯॥

ಭವ್ಯೋ ಜಿತಾರಿಷಡ್ವರ್ಗೋ ಮಹೋದಾರೋಽಘನಾಶನಃ ।\\
ಸುಕೀರ್ತಿರಾದಿಪುರುಷಃ ಕಾಂತಃ ಪುಣ್ಯಕೃತಾಗಮಃ ॥೫೦॥

ಅಕಲ್ಮಷಶ್ಚತುರ್ಬಾಹುಃ ಸರ್ವಾವಾಸೋ ದುರಾಸದಃ ( ೧೦೦)।\\
ಸ್ಮಿತಭಾಷೀ ನಿವೃತ್ತಾತ್ಮಾ ಸ್ಮೃತಿಮಾನ್ ವೀರ್ಯವಾನ್ ಪ್ರಭುಃ ॥೫೧॥

ಧೀರೋ ದಾಂತೋ ಘನಶ್ಯಾಮಃ ಸರ್ವಾಯುಧವಿಶಾರದಃ ।\\
ಅಧ್ಯಾತ್ಮಯೋಗನಿಲಯಃ ಸುಮನಾ ಲಕ್ಷ್ಮಣಾಗ್ರಜಃ ॥೫೨॥

ಸರ್ವತೀರ್ಥಮಯಃ ಶೂರಃ ಸರ್ವಯಜ್ಞಫಲಪ್ರದಃ ।\\
ಯಜ್ಞಸ್ವರೂಪೋ ಯಜ್ಞೇಶೋ ಜರಾಮರಣವರ್ಜಿತಃ ॥೫೩॥

ವರ್ಣಾಶ್ರಮಗುರುರ್ವರ್ಣೀ ಶತ್ರುಜಿತ್ಪುರುಷೋತ್ತಮಃ ।\\
ಶಿವಲಿಂಗಪ್ರತಿಷ್ಠಾತಾ ಪರಮಾತ್ಮಾ ಪರಾಪರಃ ॥೫೪॥

ಪ್ರಮಾಣಭೂತೋ ದುರ್ಜ್ಞೇಯಃ ಪೂರ್ಣಃ ಪರಪುರಂಜಯಃ ।\\
ಅನಂತದೃಷ್ಟಿರಾನಂದೋ ಧನುರ್ವೇದೋ ಧನುರ್ಧರಃ ॥೫೫॥

ಗುಣಾಕಾರೋ ಗುಣಶ್ರೇಷ್ಠಃ ಸಚ್ಚಿದಾನಂದವಿಗ್ರಹಃ ।\\
ಅಭಿವಾದ್ಯೋ ಮಹಾಕಾಯೋ ವಿಶ್ವಕರ್ಮಾ ವಿಶಾರದಃ ॥೫೬॥

ವಿನೀತಾತ್ಮಾ ವೀತರಾಗಸ್ತಪಸ್ವೀಶೋ ಜನೇಶ್ವರಃ ।\\
ಕಲ್ಯಾಣಃ ಪ್ರಹ್ವತಿಃ ಕಲ್ಪಃ ಸರ್ವೇಶಃ ಸರ್ವಕಾಮದಃ ॥೫೭॥

ಅಕ್ಷಯಃ ಪುರುಷಃ ಸಾಕ್ಷೀ ಕೇಶವಃ ಪುರುಷೋತ್ತಮಃ ।\\
ಲೋಕಾಧ್ಯಕ್ಷೋ ಮಹಾಕಾರ್ಯೋ ವಿಭೀಷಣವರಪ್ರದಃ ॥೫೮॥

ಆನಂದವಿಗ್ರಹೋ ಜ್ಯೋತಿರ್ಹನುಮತ್ಪ್ರಭುರವ್ಯಯಃ ।\\
ಭ್ರಾಜಿಷ್ಣುಃ ಸಹನೋ ಭೋಕ್ತಾ ಸತ್ಯವಾದೀ ಬಹುಶ್ರುತಃ ॥೫೯॥

ಸುಖದಃ ಕಾರಣಂ ಕರ್ತಾ ಭವಬಂಧವಿಮೋಚನಃ ।\\
ದೇವಚೂಡಾಮಣಿರ್ನೇತಾ ಬ್ರಹ್ಮಣ್ಯೋ ಬ್ರಹ್ಮವರ್ಧನಃ ॥೬೦॥

ಸಂಸಾರತಾರಕೋ ರಾಮಃ ಸರ್ವದುಃಖವಿಮೋಕ್ಷಕೃತ್ ।\\
ವಿದ್ವತ್ತಮೋ ವಿಶ್ವಕರ್ತಾ ವಿಶ್ವಕೃದ್ವಿಶ್ವಕರ್ಮ ಚ ॥೬೧॥

ನಿತ್ಯೋ ನಿಯತಕಲ್ಯಾಣಃ ಸೀತಾಶೋಕವಿನಾಶಕೃತ್ ।\\
ಕಾಕುತ್ಸ್ಥಃ ಪುಂಡರೀಕಾಕ್ಷೋ ವಿಶ್ವಾಮಿತ್ರಭಯಾಪಹಃ ॥೬೨॥

ಮಾರೀಚಮಥನೋ ರಾಮೋ ವಿರಾಧವಧಪಂಡಿತಃ ।\\
ದುಃಸ್ವಪ್ನನಾಶನೋ ರಮ್ಯಃ ಕಿರೀಟೀ ತ್ರಿದಶಾಧಿಪಃ ॥೬೩॥

ಮಹಾಧನುರ್ಮಹಾಕಾಯೋ ಭೀಮೋ ಭೀಮಪರಾಕ್ರಮಃ ।\\
ತತ್ತ್ವಸ್ವರೂಪಸ್ತತ್ತ್ವಜ್ಞಸ್ತತ್ತ್ವವಾದೀ ಸುವಿಕ್ರಮಃ ॥೬೪॥

ಭೂತಾತ್ಮ ಭೂತಕೃತ್ಸ್ವಾಮೀ ಕಾಲಜ್ಞಾನೀ ಮಹಾವಪುಃ ।\\
ಅನಿರ್ವಿಣ್ಣೋ ಗುಣಗ್ರಾಮೋ ನಿಷ್ಕಲಂಕಃ ಕಲಂಕಹಾ ॥೬೫॥

ಸ್ವಭಾವಭದ್ರಃ ಶತ್ರುಘ್ನಃ ಕೇಶವಃ ಸ್ಥಾಣುರೀಶ್ವರಃ ।\\
ಭೂತಾದಿಃ ಶಂಭುರಾದಿತ್ಯಃ ಸ್ಥವಿಷ್ಠಃ ಶಾಶ್ವತೋ ಧ್ರುವಃ ॥೬೬॥

ಕವಚೀ ಕುಂಡಲೀ ಚಕ್ರೀ ಖಡ್ಗೀ ಭಕ್ತಜನಪ್ರಿಯಃ ।\\
ಅಮೃತ್ಯುರ್ಜನ್ಮರಹಿತಃ ಸರ್ವಜಿತ್ಸರ್ವಗೋಚರಃ ॥೬೭॥

ಅನುತ್ತಮೋಽಪ್ರಮೇಯಾತ್ಮಾ ಸರ್ವಾತ್ಮಾ ಗುಣಸಾಗರಃ ( ೨೦೦)।\\
ರಾಮಃ ಸಮಾತ್ಮಾ ಸಮಗೋ ಜಟಾಮುಕುಟಮಂಡಿತಃ ॥೬೮॥

ಅಜೇಯಃ ಸರ್ವಭೂತಾತ್ಮಾ ವಿಷ್ವಕ್ಸೇನೋ ಮಹಾತಪಾಃ ।\\
ಲೋಕಾಧ್ಯಕ್ಷೋ ಮಹಾಬಾಹುರಮೃತೋ ವೇದವಿತ್ತಮಃ ॥೬೯॥

ಸಹಿಷ್ಣುಃ ಸದ್ಗತಿಃ ಶಾಸ್ತಾ ವಿಶ್ವಯೋನಿರ್ಮಹಾದ್ಯುತಿಃ ।\\
ಅತೀಂದ್ರ ಊರ್ಜಿತಃ ಪ್ರಾಂಶುರುಪೇಂದ್ರೋ ವಾಮನೋ ಬಲಿಃ ॥೭೦॥

ಧನುರ್ವೇದೋ ವಿಧಾತಾ ಚ ಬ್ರಹ್ಮಾ ವಿಷ್ಣುಶ್ಚ ಶಂಕರಃ ।\\
ಹಂಸೋ ಮರೀಚಿರ್ಗೋವಿಂದೋ ರತ್ನಗರ್ಭೋ ಮಹದ್ದ್ಯುತಿಃ ॥೭೧॥

ವ್ಯಾಸೋ ವಾಚಸ್ಪತಿಃ ಸರ್ವದರ್ಪಿತಾಸುರಮರ್ದನಃ ।\\
ಜಾನಕೀವಲ್ಲಭಃ ಶ್ರೀಮಾನ್ ಪ್ರಕಟಃ ಪ್ರೀತಿವರ್ಧನಃ ॥೭೨॥

ಸಂಭವೋಽತೀಂದ್ರಿಯೋ ವೇದ್ಯೋ ನಿರ್ದೇಶೋ ಜಾಂಬವತ್ಪ್ರಭುಃ ।\\
ಮದನೋ ಮನ್ಮಥೋ ವ್ಯಾಪೀ ವಿಶ್ವರೂಪೋ ನಿರಂಜನಃ ॥೭೩॥

ನಾರಾಯಣೋಽಗ್ರಣೀ ಸಾಧುರ್ಜಟಾಯುಪ್ರೀತಿವರ್ಧನಃ ।\\
ನೈಕರೂಪೋ ಜಗನ್ನಾಥಃ ಸುರಕಾರ್ಯಹಿತಃ ಪ್ರಭುಃ ॥೭೪॥

ಜಿತಕ್ರೋಧೋ ಜಿತಾರಾತಿಃ ಪ್ಲವಗಾಧಿಪರಾಜ್ಯದಃ ।\\
ವಸುದಃ ಸುಭುಜೋ ನೈಕಮಾಯೋ ಭವ್ಯಃ ಪ್ರಮೋದನಃ ॥೭೫॥

ಚಂಡಾಂಶುಃ ಸಿದ್ಧಿದಃ ಕಲ್ಪಃ ಶರಣಾಗತವತ್ಸಲಃ ।\\
ಅಗದೋ ರೋಗಹರ್ತಾ ಚ ಮಂತ್ರಜ್ಞೋ ಮಂತ್ರಭಾವನಃ ॥೭೬॥

ಸೌಮಿತ್ರಿವತ್ಸಲೋ ಧುರ್ಯೋ ವ್ಯಕ್ತಾವ್ಯಕ್ತಸ್ವರೂಪಧೃಕ್ ।\\
ವಸಿಷ್ಠೋ ಗ್ರಾಮಣೀಃ ಶ್ರೀಮಾನನುಕೂಲಃ ಪ್ರಿಯಂವದಃ ॥೭೭॥

ಅತುಲಃ ಸಾತ್ತ್ವಿಕೋ ಧೀರಃ ಶರಾಸನವಿಶಾರದಃ ।\\
ಜ್ಯೇಷ್ಠಃ ಸರ್ವಗುಣೋಪೇತಃ ಶಕ್ತಿಮಾಂಸ್ತಾಟಕಾಂತಕಃ ॥೭೮॥

ವೈಕುಂಠಃ ಪ್ರಾಣಿನಾಂ ಪ್ರಾಣಃ ಕಮಲಃ ಕಮಲಾಧಿಪಃ ।\\
ಗೋವರ್ಧನಧರೋ ಮತ್ಸ್ಯರೂಪಃ ಕಾರುಣ್ಯಸಾಗರಃ ॥೭೯॥

ಕುಂಭಕರ್ಣಪ್ರಭೇತ್ತಾ ಚ ಗೋಪಿಗೋಪಾಲಸಂವೃತಃ ( ೩೦೦)।\\
ಮಾಯಾವೀ ವ್ಯಾಪಕೋ ವ್ಯಾಪೀ ರೇಣುಕೇಯಬಲಾಪಹಃ ॥೮೦॥

ಪಿನಾಕಮಥನೋ ವಂದ್ಯಃ ಸಮರ್ಥೋ ಗರುಡಧ್ವಜಃ ।\\
ಲೋಕತ್ರಯಾಶ್ರಯೋ ಲೋಕಭರಿತೋ ಭರತಾಗ್ರಜಃ ॥೮೧॥

ಶ್ರೀಧರಃ ಸಂಗತಿರ್ಲೋಕಸಾಕ್ಷೀ ನಾರಾಯಣೋ ವಿಭುಃ ।\\
ಮನೋರೂಪೀ ಮನೋವೇಗೀ ಪೂರ್ಣಃ ಪುರುಷಪುಂಗವಃ ॥೮೨॥

ಯದುಶ್ರೇಷ್ಠೋ ಯದುಪತಿರ್ಭೂತಾವಾಸಃ ಸುವಿಕ್ರಮಃ ।\\
ತೇಜೋಧರೋ ಧರಾಧರಶ್ಚತುರ್ಮೂರ್ತಿರ್ಮಹಾನಿಧಿಃ ॥೮೩॥

ಚಾಣೂರಮಥನೋ ವಂದ್ಯಃ ಶಾಂತೋ ಭರತವಂದಿತಃ ।\\
ಶಬ್ದಾತಿಗೋ ಗಭೀರಾತ್ಮಾ ಕೋಮಲಾಂಗಃ ಪ್ರಜಾಗರಃ ॥೮೪॥

ಲೋಕೋರ್ಧ್ವಗಃ ಶೇಷಶಾಯೀ ಕ್ಷೀರಾಬ್ಧಿನಿಲಯೋಽಮಲಃ ।\\
ಆತ್ಮಜ್ಯೋತಿರದೀನಾತ್ಮಾ ಸಹಸ್ರಾರ್ಚಿಃ ಸಹಸ್ರಪಾತ್ ॥೮೫॥

ಅಮೃತಾಂಶುರ್ಮಹೀಗರ್ತೋ ನಿವೃತ್ತವಿಷಯಸ್ಪೃಹಃ ।\\
ತ್ರಿಕಾಲಜ್ಞೋ ಮುನಿಃ ಸಾಕ್ಷೀ ವಿಹಾಯಸಗತಿಃ ಕೃತೀ ॥೮೬॥

ಪರ್ಜನ್ಯಃ ಕುಮುದೋ ಭೂತಾವಾಸಃ ಕಮಲಲೋಚನಃ ।\\
ಶ್ರೀವತ್ಸವಕ್ಷಾಃ ಶ್ರೀವಾಸೋ ವೀರಹಾ ಲಕ್ಷ್ಮಣಾಗ್ರಜಃ ॥೮೭॥

ಲೋಕಾಭಿರಾಮೋ ಲೋಕಾರಿಮರ್ದನಃ ಸೇವಕಪ್ರಿಯಃ ।\\
ಸನಾತನತಮೋ ಮೇಘಶ್ಯಾಮಲೋ ರಾಕ್ಷಸಾಂತಕಃ ॥೮೮॥

ದಿವ್ಯಾಯುಧಧರಃ ಶ್ರೀಮಾನಪ್ರಮೇಯೋ ಜಿತೇಂದ್ರಿಯಃ ।\\
ಭೂದೇವವಂದ್ಯೋ ಜನಕಪ್ರಿಯಕೃತ್ಪ್ರಪಿತಾಮಹಃ ॥೮೯॥

ಉತ್ತಮಃ ಸಾತ್ವಿಕಃ ಸತ್ಯಃ ಸತ್ಯಸಂಧಸ್ತ್ರಿವಿಕ್ರಮಃ ।\\
ಸುವೃತ್ತಃ ಸುಗಮಃ ಸೂಕ್ಷ್ಮಃ ಸುಘೋಷಃ ಸುಖದಃ ಸುಹೃತ್ ॥೯೦॥

ದಾಮೋದರೋಽಚ್ಯುತಃ ಶಾರ್ಙ್ಗೀ ವಾಮನೋ ಮಥುರಾಧಿಪಃ ।\\
ದೇವಕೀನಂದನಃ ಶೌರಿಃ ಶೂರಃ ಕೈಟಭಮರ್ದನಃ ॥೯೧॥

ಸಪ್ತತಾಲಪ್ರಭೇತ್ತಾ ಚ ಮಿತ್ರವಂಶಪ್ರವರ್ಧನಃ ।\\
ಕಾಲಸ್ವರೂಪೀ ಕಾಲಾತ್ಮಾ ಕಾಲಃ ಕಲ್ಯಾಣದಃ ಕಲಿಃ (೪೦೦)॥೯೨॥

ಸಂವತ್ಸರೋ ಋತುಃ ಪಕ್ಷೋ ಹ್ಯಯನಂ ದಿವಸೋ ಯುಗಃ ।\\
ಸ್ತವ್ಯೋ ವಿವಿಕ್ತೋ ನಿರ್ಲೇಪಃ ಸರ್ವವ್ಯಾಪೀ ನಿರಾಕುಲಃ ॥೯೩॥

ಅನಾದಿನಿಧನಃ ಸರ್ವಲೋಕಪೂಜ್ಯೋ ನಿರಾಮಯಃ ।\\
ರಸೋ ರಸಜ್ಞಃ ಸಾರಜ್ಞೋ ಲೋಕಸಾರೋ ರಸಾತ್ಮಕಃ ॥೯೪॥

ಸರ್ವದುಃಖಾತಿಗೋ ವಿದ್ಯಾರಾಶಿಃ ಪರಮಗೋಚರಃ ।\\
ಶೇಷೋ ವಿಶೇಷೋ ವಿಗತಕಲ್ಮಷೋ ರಘುಪುಂಗವಃ ॥೯೫॥

ವರ್ಣಶ್ರೇಷ್ಠೋ ವರ್ಣಭಾವ್ಯೋ ವರ್ಣೋ ವರ್ಣಗುಣೋಜ್ಜ್ವಲಃ ।\\
ಕರ್ಮಸಾಕ್ಷೀ ಗುಣಶ್ರೇಷ್ಠೋ ದೇವಃ ಸುರವರಪ್ರದಃ ॥೯೬॥

ದೇವಾಧಿದೇವೋ ದೇವರ್ಷಿರ್ದೇವಾಸುರನಮಸ್ಕೃತಃ ।\\
ಸರ್ವದೇವಮಯಶ್ಚಕ್ರೀ ಶಾರ್ಙ್ಗಪಾಣೀ ರಘೂತ್ತಮಃ ॥೯೭॥

ಮನೋಗುಪ್ತಿರಹಂಕಾರಃ ಪ್ರಕೃತಿಃ ಪುರುಷೋಽವ್ಯಯಃ ।\\
ನ್ಯಾಯೋ ನ್ಯಾಯೀ ನಯೀ ಶ್ರೀಮಾನ್ ನಯೋ ನಗಧರೋ ಧ್ರುವಃ ॥೯೮॥

ಲಕ್ಷ್ಮೀವಿಶ್ವಂಭರೋ ಭರ್ತಾ ದೇವೇಂದ್ರೋ ಬಲಿಮರ್ದನಃ ।\\
ಬಾಣಾರಿಮರ್ದನೋ ಯಜ್ವಾನುತ್ತಮೋ ಮುನಿಸೇವಿತಃ ॥೯೯॥

ದೇವಾಗ್ರಣೀಃ ಶಿವಧ್ಯಾನತತ್ಪರಃ ಪರಮಃ ಪರಃ ।\\
ಸಾಮಗೇಯಃ ಪ್ರಿಯಃ ಶೂರಃ ಪೂರ್ಣಕೀರ್ತಿಃ ಸುಲೋಚನಃ ॥೧೦೦॥

ಅವ್ಯಕ್ತಲಕ್ಷಣೋ ವ್ಯಕ್ತೋ ದಶಾಸ್ಯದ್ವಿಪಕೇಸರೀ ।\\
ಕಲಾನಿಧಿಃ ಕಲಾನಾಥಃ ಕಮಲಾನಂದವರ್ಧನಃ ॥೧೦೧॥

ಪುಣ್ಯಃ ಪುಣ್ಯಾಧಿಕಃ ಪೂರ್ಣಃ ಪೂರ್ವಃ ಪೂರಯಿತಾ ರವಿಃ ।\\
ಜಟಿಲಃ ಕಲ್ಮಷಧ್ವಾಂತಪ್ರಭಂಜನವಿಭಾವಸುಃ ॥೧೦೨॥

ಜಯೀ ಜಿತಾರಿಃ ಸರ್ವಾದಿಃ ಶಮನೋ ಭವಭಂಜನಃ ।\\
ಅಲಂಕರಿಷ್ಣುರಚಲೋ ರೋಚಿಷ್ಣುರ್ವಿಕ್ರಮೋತ್ತಮಃ ॥೧೦೩॥

ಆಶುಃ ಶಬ್ದಪತಿಃ ಶಬ್ದಗೋಚರೋ ರಂಜನೋ ಲಘುಃ ।\\
ನಿಃಶಬ್ದಪುರುಷೋ ಮಾಯೋ ಸ್ಥೂಲಃ ಸೂಕ್ಷ್ಮೋ ವಿಲಕ್ಷಣಃ (೫೦೦)॥೧೦೪॥

ಆತ್ಮಯೋನಿರಯೋನಿಶ್ಚ ಸಪ್ತಜಿಹ್ವಃ ಸಹಸ್ರಪಾತ್ ।\\
ಸನಾತನತಮಃ ಸ್ರಗ್ವೀ ಪೇಶಲೋ ವಿಜಿತಾಂಬರಃ ॥೧೦೫॥

ಶಕ್ತಿಮಾನ್ ಶಂಖಭೃನ್ನಾಥೋ ಗದಾಧರರಥಾಂಗಭೃತ್ ।\\
ನಿರೀಹೋ ನಿರ್ವಿಕಲ್ಪಶ್ಚ ಚಿದ್ರೂಪೋ ವೀತಸಾಧ್ವಸಃ ॥೧೦೬॥

ಸನಾತನಃ ಸಹಸ್ರಾಕ್ಷಃ ಶತಮೂರ್ತಿರ್ಘನಪ್ರಭಃ ।\\
ಹೃತ್ಪುಂಡರೀಕಶಯನಃ ಕಠಿನೋ ದ್ರವ ಏವ ಚ ॥೧೦೭॥

ಸೂರ್ಯೋ ಗ್ರಹಪತಿಃ ಶ್ರೀಮಾನ್ ಸಮರ್ಥೋಽನರ್ಥನಾಶನಃ ।\\
ಅಧರ್ಮಶತ್ರೂ ರಕ್ಷೋಘ್ನಃ ಪುರುಹೂತಃ ಪುರಸ್ತುತಃ ॥೧೦೮॥

ಬ್ರಹ್ಮಗರ್ಭೋ ಬೃಹದ್ಗರ್ಭೋ ಧರ್ಮಧೇನುರ್ಧನಾಗಮಃ ।\\
ಹಿರಣ್ಯಗರ್ಭೋ ಜ್ಯೋತಿಷ್ಮಾನ್ ಸುಲಲಾಟಃ ಸುವಿಕ್ರಮಃ ॥೧೦೯॥

ಶಿವಪೂಜಾರತಃ ಶ್ರೀಮಾನ್ ಭವಾನೀಪ್ರಿಯಕೃದ್ವಶೀ ।\\
ನರೋ ನಾರಾಯಣಃ ಶ್ಯಾಮಃ ಕಪರ್ದೀ ನೀಲಲೋಹಿತಃ ॥೧೧೦॥

ರುದ್ರಃ ಪಶುಪತಿಃ ಸ್ಥಾಣುರ್ವಿಶ್ವಾಮಿತ್ರೋ ದ್ವಿಜೇಶ್ವರಃ ।\\
ಮಾತಾಮಹೋ ಮಾತರಿಶ್ವಾ ವಿರಿಂಚಿರ್ವಿಷ್ಟರಶ್ರವಾಃ ॥೧೧೧॥

ಅಕ್ಷೋಭ್ಯಃ ಸರ್ವಭೂತಾನಾಂ ಚಂಡಃ ಸತ್ಯಪರಾಕ್ರಮಃ ।\\
ವಾಲಖಿಲ್ಯೋ ಮಹಾಕಲ್ಪಃ ಕಲ್ಪವೃಕ್ಷಃ ಕಲಾಧರಃ ॥೧೧೨॥

ನಿದಾಘಸ್ತಪನೋ ಮೇಘಃ ಶುಕ್ರಃ ಪರಬಲಾಪಹೃತ್ ।\\
ವಸುಶ್ರವಾಃ ಕವ್ಯವಾಹಃ ಪ್ರತಪ್ತೋ ವಿಶ್ವಭೋಜನಃ ॥೧೧೩॥

ರಾಮೋ ನೀಲೋತ್ಪಲಶ್ಯಾಮೋ ಜ್ಞಾನಸ್ಕಂದೋ ಮಹಾದ್ಯುತಿಃ ।\\
ಕಬಂಧಮಥನೋ ದಿವ್ಯಃ ಕಂಬುಗ್ರೀವಃ ಶಿವಪ್ರಿಯಃ ॥೧೧೪॥

ಸುಖೀ ನೀಲಃ ಸುನಿಷ್ಪನ್ನಃ ಸುಲಭಃ ಶಿಶಿರಾತ್ಮಕಃ ।\\
ಅಸಂಸೃಷ್ಟೋಽತಿಥಿಃ ಶೂರಃ ಪ್ರಮಾಥೀ ಪಾಪನಾಶಕೃತ್ ॥೧೧೫॥

ಪವಿತ್ರಪಾದಃ ಪಾಪಾರಿರ್ಮಣಿಪೂರೋ ನಭೋಗತಿಃ ।\\
ಉತ್ತಾರಣೋ ದುಷ್ಕೃತಿಹಾ ದುರ್ಧರ್ಷೋ ದುಃಸಹೋ ಬಲಃ (೬೦೦)॥೧೧೬॥

ಅಮೃತೇಶೋಽಮೃತವಪುರ್ಧರ್ಮೀ ಧರ್ಮಃ ಕೃಪಾಕರಃ ।\\
ಭಗೋ ವಿವಸ್ವಾನಾದಿತ್ಯೋ ಯೋಗಾಚಾರ್ಯೋ ದಿವಸ್ಪತಿಃ ॥೧೧೭॥

ಉದಾರಕೀರ್ತಿರುದ್ಯೋಗೀ ವಾಙ್ಮಯಃ ಸದಸನ್ಮಯಃ ।\\
ನಕ್ಷತ್ರಮಾನೀ ನಾಕೇಶಃ ಸ್ವಾಧಿಷ್ಠಾನಃ ಷಡಾಶ್ರಯಃ ॥೧೧೮॥

ಚತುರ್ವರ್ಗಫಲಂ ವರ್ಣಶಕ್ತಿತ್ರಯಫಲಂ ನಿಧಿಃ ।\\
ನಿಧಾನಗರ್ಭೋ ನಿರ್ವ್ಯಾಜೋ ನಿರೀಶೋ ವ್ಯಾಲಮರ್ದನಃ ॥೧೧೯॥

ಶ್ರೀವಲ್ಲಭಃ ಶಿವಾರಂಭಃ ಶಾಂತೋ ಭದ್ರಃ ಸಮಂಜಯಃ ।\\
ಭೂಶಾಯೀ ಭೂತಕೃದ್ಭೂತಿರ್ಭೂಷಣೋ ಭೂತಭಾವನಃ ॥೧೨೦॥

ಅಕಾಯೋ ಭಕ್ತಕಾಯಸ್ಥಃ ಕಾಲಜ್ಞಾನೀ ಮಹಾಪಟುಃ ।\\
ಪರಾರ್ಧವೃತ್ತಿರಚಲೋ ವಿವಿಕ್ತಃ ಶ್ರುತಿಸಾಗರಃ ॥೧೨೧॥

ಸ್ವಭಾವಭದ್ರೋ ಮಧ್ಯಸ್ಥಃ ಸಂಸಾರಭಯನಾಶನಃ ।\\
ವೇದ್ಯೋ ವೈದ್ಯೋ ವಿಯದ್ಗೋಪ್ತಾ ಸರ್ವಾಮರಮುನೀಶ್ವರಃ ॥೧೨೨॥

ಸುರೇಂದ್ರಃ ಕಾರಣಂ ಕರ್ಮಕರಃ ಕರ್ಮೀ ಹ್ಯಧೋಕ್ಷಜಃ ।\\
ಧೈರ್ಯೋಽಗ್ರಧುರ್ಯೋ ಧಾತ್ರೀಶಃ ಸಂಕಲ್ಪಃ ಶರ್ವರೀಪತಿಃ ॥೧೨೩॥

ಪರಮಾರ್ಥಗುರುರ್ದೃಷ್ಟಿಃ ಸುಚಿರಾಶ್ರಿತವತ್ಸಲಃ ।\\
ವಿಷ್ಣುರ್ಜಿಷ್ಣುರ್ವಿಭುರ್ಯಜ್ಞೋ ಯಜ್ಞೇಶೋ ಯಜ್ಞಪಾಲಕಃ ॥೧೨೪॥

ಪ್ರಭುರ್ವಿಷ್ಣುರ್ಗ್ರಸಿಷ್ಣುಶ್ಚ ಲೋಕಾತ್ಮಾ ಲೋಕಪಾಲಕಃ ।\\
ಕೇಶವಃ ಕೇಶಿಹಾ ಕಾವ್ಯಃ ಕವಿಃ ಕಾರಣಕಾರಣಂ ॥೧೨೫॥

ಕಾಲಕರ್ತಾ ಕಾಲಶೇಷೋ ವಾಸುದೇವಃ ಪುರುಷ್ಟುತಃ ।\\
ಆದಿಕರ್ತಾ ವರಾಹಶ್ಚ ವಾಮನೋ ಮಧುಸೂದನಃ ॥೧೨೬॥

ನಾರಾಯಣೋ ನರೋ ಹಂಸೋ ವಿಷ್ವಕ್ಸೇನೋ ಜನಾರ್ದನಃ ।\\
ವಿಶ್ವಕರ್ತಾ ಮಹಾಯಜ್ಞೋ ಜ್ಯೋತಿಷ್ಮಾನ್ಪುರುಷೋತ್ತಮಃ ( ೭೦೦)॥೧೨೭॥

ವೈಕುಂಠಃ ಪುಂಡರೀಕಾಕ್ಷಃ ಕೃಷ್ಣಃ ಸೂರ್ಯಃ ಸುರಾರ್ಚಿತಃ ।\\
ನಾರಸಿಂಹೋ ಮಹಾಭೀಮೋ ವಜ್ರದಂಷ್ಟ್ರೋ ನಖಾಯುಧಃ ॥೧೨೮॥

ಆದಿದೇವೋ ಜಗತ್ಕರ್ತಾ ಯೋಗೀಶೋ ಗರುಡಧ್ವಜಃ ।\\
ಗೋವಿಂದೋ ಗೋಪತಿರ್ಗೋಪ್ತಾ ಭೂಪತಿರ್ಭುವನೇಶ್ವರಃ ॥೧೨೯॥

ಪದ್ಮನಾಭೋ ಹೃಷೀಕೇಶೋ ಧಾತಾ ದಾಮೋದರಃ ಪ್ರಭುಃ ।\\
ತ್ರಿವಿಕ್ರಮಸ್ತ್ರಿಲೋಕೇಶೋ ಬ್ರಹ್ಮೇಶಃ ಪ್ರೀತಿವರ್ಧನಃ ॥೧೩೦॥

ಸಂನ್ಯಾಸೀ ಶಾಸ್ತ್ರತತ್ತ್ವಜ್ಞೋ ಮಂದಿರೋ ಗಿರಿಶೋ ನತಃ ।\\
ವಾಮನೋ ದುಷ್ಟದಮನೋ ಗೋವಿಂದೋ ಗೋಪವಲ್ಲಭಃ ॥೧೩೧॥

ಭಕ್ತಪ್ರಿಯೋಽಚ್ಯುತಃ ಸತ್ಯಃ ಸತ್ಯಕೀರ್ತಿರ್ಧೃತಿಃ ಸ್ಮೃತಿಃ ।\\
ಕಾರುಣ್ಯಃ ಕರುಣೋ ವ್ಯಾಸಃ ಪಾಪಹಾ ಶಾಂತಿವರ್ಧನಃ ॥೧೩೨॥

ಬದರೀನಿಲಯಃ ಶಾಂತಸ್ತಪಸ್ವೀ ವೈದ್ಯುತಃ ಪ್ರಭುಃ ।\\
ಭೂತಾವಾಸೋ ಮಹಾವಾಸೋ ಶ್ರೀನಿವಾಸಃ ಶ್ರಿಯಃ ಪತಿಃ ॥೧೩೩॥

ತಪೋವಾಸೋ ಮುದಾವಾಸಃ ಸತ್ಯವಾಸಃ ಸನಾತನಃ ।\\
ಪುರುಷಃ ಪುಷ್ಕರಃ ಪುಣ್ಯಃ ಪುಷ್ಕರಾಕ್ಷೋ ಮಹೇಶ್ವರಃ ॥೧೩೪॥

ಪೂರ್ಣಮೂರ್ತಿಃ ಪುರಾಣಜ್ಞಃ ಪುಣ್ಯದಃ ಪ್ರೀತಿವರ್ಧನಃ ।\\
ಪೂರ್ಣರೂಪಃ ಕಾಲಚಕ್ರಪ್ರವರ್ತನಸಮಾಹಿತಃ ॥೧೩೫॥

ನಾರಾಯಣಃ ಪರಂಜ್ಯೋತಿಃ ಪರಮಾತ್ಮಾ ಸದಾಶಿವಃ ।\\
ಶಂಖೀ ಚಕ್ರೀ ಗದೀ ಶಾರ್ಙ್ಗೀ ಲಾಂಗಲೀ ಮುಸಲೀ ಹಲೀ ॥೧೩೬॥

ಕಿರೀಟೀ ಕುಂಡಲೀ ಹಾರೀ ಮೇಖಲೀ ಕವಚೀ ಧ್ವಜೀ ।\\
ಯೋದ್ಧಾ ಜೇತಾ ಮಹಾವೀರ್ಯಃ ಶತ್ರುಘ್ನಃ ಶತ್ರುತಾಪನಃ ॥೧೩೭॥

ಶಾಸ್ತಾ ಶಾಸ್ತ್ರಕರಃ ಶಾಸ್ತ್ರಂ ಶಂಕರಃ ಶಂಕರಸ್ತುತಃ ।\\
ಸಾರಥೀ ಸಾತ್ತ್ವಿಕಃ ಸ್ವಾಮೀ ಸಾಮವೇದಪ್ರಿಯಃ ಸಮಃ ( ೮೦೦)॥೧೩೮॥

ಪವನಃ ಸಂಹಿತಃ ಶಕ್ತಿಃ ಸಂಪೂರ್ಣಾಂಗಃ ಸಮೃದ್ಧಿಮಾನ್ ।\\
ಸ್ವರ್ಗದಃ ಕಾಮದಃ ಶ್ರೀದಃ ಕೀರ್ತಿದಃ ಕೀರ್ತಿದಾಯಕಃ ॥೧೩೯॥

ಮೋಕ್ಷದಃ ಪುಂಡರೀಕಾಕ್ಷಃ ಕ್ಷೀರಾಬ್ಧಿಕೃತಕೇತನಃ ।\\
ಸರ್ವಾತ್ಮಾ ಸರ್ವಲೋಕೇಶಃ ಪ್ರೇರಕಃ ಪಾಪನಾಶನಃ ॥೧೪೦॥

ವೈಕುಂಠಃ ಪುಂಡರೀಕಾಕ್ಷಃ ಸರ್ವದೇವನಮಸ್ಕೃತಃ ।\\
ಸರ್ವವ್ಯಾಪೀ ಜಗನ್ನಾಥಃ ಸರ್ವಲೋಕಮಹೇಶ್ವರಃ ॥೧೪೧॥

ಸರ್ಗಸ್ಥಿತ್ಯಂತಕೃದ್ದೇವಃ ಸರ್ವಲೋಕಸುಖಾವಹಃ ।\\
ಅಕ್ಷಯಃ ಶಾಶ್ವತೋಽನಂತಃ ಕ್ಷಯವೃದ್ಧಿವಿವರ್ಜಿತಃ ॥೧೪೨॥

ನಿರ್ಲೇಪೋ ನಿರ್ಗುಣಃ ಸೂಕ್ಷ್ಮೋ ನಿರ್ವಿಕಾರೋ ನಿರಂಜನಃ ।\\
ಸರ್ವೋಪಾಧಿವಿನಿರ್ಮುಕ್ತಃ ಸತ್ತಾಮಾತ್ರವ್ಯವಸ್ಥಿತಃ ॥೧೪೩॥

ಅಧಿಕಾರೀ ವಿಭುರ್ನಿತ್ಯಃ ಪರಮಾತ್ಮಾ ಸನಾತನಃ ।\\
ಅಚಲೋ ನಿಶ್ಚಲೋ ವ್ಯಾಪೀ ನಿತ್ಯತೃಪ್ತೋ ನಿರಾಶ್ರಯಃ ॥೧೪೪॥

ಶ್ಯಾಮೀ ಯುವಾ ಲೋಹಿತಾಕ್ಷೋ ದೀಪ್ತ್ಯಾ ಶೋಭಿತಭಾಷಣಃ ।\\
ಆಜಾನುಬಾಹುಃ ಸುಮುಖಃ ಸಿಂಹಸ್ಕಂಧೋ ಮಹಾಭುಜಃ ॥೧೪೫॥

ಸತ್ತ್ವವಾನ್ ಗುಣಸಂಪನ್ನೋ ದೀಪ್ಯಮಾನಃ ಸ್ವತೇಜಸಾ ।\\
ಕಾಲಾತ್ಮಾ ಭಗವಾನ್ ಕಾಲಃ ಕಾಲಚಕ್ರಪ್ರವರ್ತಕಃ ॥೧೪೬॥

ನಾರಾಯಣಃ ಪರಂಜ್ಯೋತಿಃ ಪರಮಾತ್ಮಾ ಸನಾತನಃ ।\\
ವಿಶ್ವಕೃದ್ವಿಶ್ವಭೋಕ್ತಾ ಚ ವಿಶ್ವಗೋಪ್ತಾ ಚ ಶಾಶ್ವತಃ ॥೧೪೭॥

ವಿಶ್ವೇಶ್ವರೋ ವಿಶ್ವಮೂರ್ತಿರ್ವಿಶ್ವಾತ್ಮಾ ವಿಶ್ವಭಾವನಃ ।\\
ಸರ್ವಭೂತಸುಹೃಚ್ಛಾಂತಃ ಸರ್ವಭೂತಾನುಕಂಪನಃ ॥೧೪೮॥

ಸರ್ವೇಶ್ವರಃ ಸರ್ವಶರ್ವಃ ಸರ್ವದಾಽಽಶ್ರಿತವತ್ಸಲಃ ।\\
ಸರ್ವಗಃ ಸರ್ವಭೂತೇಶಃ ಸರ್ವಭೂತಾಶಯಸ್ಥಿತಃ ॥೧೪೯॥

ಅಭ್ಯಂತರಸ್ಥಸ್ತಮಸಶ್ಛೇತ್ತಾ ನಾರಾಯಣಃ ಪರಃ ।\\
ಅನಾದಿನಿಧನಃ ಸ್ರಷ್ಟಾ ಪ್ರಜಾಪತಿಪತಿರ್ಹರಿಃ ॥೧೫೦॥

ನರಸಿಂಹೋ ಹೃಷೀಕೇಶಃ ಸರ್ವಾತ್ಮಾ ಸರ್ವದೃಗ್ವಶೀ ।\\
ಜಗತಸ್ತಸ್ಥುಷಶ್ಚೈವ ಪ್ರಭುರ್ನೇತಾ ಸನಾತನಃ ( ೯೦೦)॥೧೫೧॥

ಕರ್ತಾ ಧಾತಾ ವಿಧಾತಾ ಚ ಸರ್ವೇಷಾಂ ಪತಿರೀಶ್ವರಃ ।\\
ಸಹಸ್ರಮೂರ್ಧಾ ವಿಶ್ವಾತ್ಮಾ ವಿಷ್ಣುರ್ವಿಶ್ವದೃಗವ್ಯಯಃ ॥೧೫೨॥

ಪುರಾಣಪುರುಷಃ ಶ್ರೇಷ್ಠಃ ಸಹಸ್ರಾಕ್ಷಃ ಸಹಸ್ರಪಾತ್ ।\\
ತತ್ತ್ವಂ ನಾರಾಯಣೋ ವಿಷ್ಣುರ್ವಾಸುದೇವಃ ಸನಾತನಃ ॥೧೫೩॥

ಪರಮಾತ್ಮಾ ಪರಂಬ್ರಹ್ಮ ಸಚ್ಚಿದಾನಂದವಿಗ್ರಹಃ ।\\
ಪರಂಜ್ಯೋತಿಃ ಪರಂಧಾಮ ಪರಾಕಾಶಃ ಪರಾತ್ಪರಃ ॥೧೫೪॥

ಅಚ್ಯುತಃ ಪುರುಷಃ ಕೃಷ್ಣಃ ಶಾಶ್ವತಃ ಶಿವ ಈಶ್ವರಃ ।\\
ನಿತ್ಯಃ ಸರ್ವಗತಃ ಸ್ಥಾಣೂ ರುದ್ರಃ ಸಾಕ್ಷೀ ಪ್ರಜಾಪತಿಃ ॥೧೫೫॥

ಹಿರಣ್ಯಗರ್ಭಃ ಸವಿತಾ ಲೋಕಕೃಲ್ಲೋಕಭುಗ್ವಿಭುಃ ।\\
ಓಂಕಾರವಾಚ್ಯೋ ಭಗವಾನ್ ಶ್ರೀಭೂಲೀಲಾಪತಿಃ ಪ್ರಭುಃ ॥೧೫೬॥

ಸರ್ವಲೋಕೇಶ್ವರಃ ಶ್ರೀಮಾನ್ ಸರ್ವಜ್ಞಃ ಸರ್ವತೋಮುಖಃ ।\\
ಸ್ವಾಮೀ ಸುಶೀಲಃ ಸುಲಭಃ ಸರ್ವಗಃ ಸರ್ವಶಕ್ತಿಮಾನ್ ॥೧೫೭॥

ನಿತ್ಯಃ ಸಂಪೂರ್ಣಕಾಮಶ್ಚ ನೈಸರ್ಗಿಕಸುಹೃತ್ಸುಖೀ ।\\
ಕೃಪಾಪೀಯೂಷಜಲಧಿಃ ಶರಣ್ಯಃ ಸರ್ವಶಕ್ತಿಮಾನ್ ॥೧೫೮॥

ಶ್ರೀಮಾನ್ನಾರಾಯಣಃ ಸ್ವಾಮೀ ಜಗತಾಂ ಪ್ರಭುರೀಶ್ವರಃ ।\\
ಮತ್ಸ್ಯಃ ಕೂರ್ಮೋ ವರಾಹಶ್ಚ ನಾರಸಿಂಹೋಽಥ ವಾಮನಃ ॥೧೫೯॥

ರಾಮೋ ರಾಮಶ್ಚ ಕೃಷ್ಣಶ್ಚ ಬೌದ್ಧಃ ಕಲ್ಕೀ ಪರಾತ್ಪರಃ ।\\
ಅಯೋಧ್ಯೇಶೋ ನೃಪಶ್ರೇಷ್ಠಃ ಕುಶಬಾಲಃ ಪರಂತಪಃ ॥೧೬೦॥

ಲವಬಾಲಃ ಕಂಜನೇತ್ರಃ ಕಂಜಾಂಘ್ರಿಃ ಪಂಕಜಾನನಃ ।\\
ಸೀತಾಕಾಂತಃ ಸೌಮ್ಯರೂಪಃ ಶಿಶುಜೀವನತತ್ಪರಃ ॥೧೬೧॥

ಸೇತುಕೃಚ್ಚಿತ್ರಕೂಟಸ್ಥಃ ಶಬರೀಸಂಸ್ತುತಃ ಪ್ರಭುಃ ।\\
ಯೋಗಿಧ್ಯೇಯಃ ಶಿವಧ್ಯೇಯಃ ಶಾಸ್ತಾ ರಾವಣದರ್ಪಹಾ ॥೧೬೨॥

ಶ್ರೀಶಃ ಶರಣ್ಯೋ ಭೂತಾನಾಂ ಸಂಶ್ರಿತಾಭೀಷ್ಟದಾಯಕಃ ।\\
ಅನಂತಃ ಶ್ರೀಪತೀ ರಾಮೋ ಗುಣಭೃನ್ನಿರ್ಗುಣೋ ಮಹಾನ್ (೧೦೦೦)॥೧೬೩॥

ಏವಮಾದೀನಿ ನಾಮಾನಿ ಹ್ಯಸಂಖ್ಯಾನ್ಯಪರಾಣಿ ಚ ।\\
ಏಕೈಕಂ ನಾಮ ರಾಮಸ್ಯ ಸರ್ವಪಾಪಪ್ರಣಾಶನಂ ॥೧೬೪॥

ಸಹಸ್ರನಾಮಫಲದಂ ಸರ್ವೈಶ್ವರ್ಯಪ್ರದಾಯಕಂ ।\\
ಸರ್ವಸಿದ್ಧಿಕರಂ ಪುಣ್ಯಂ ಭುಕ್ತಿಮುಕ್ತಿಫಲಪ್ರದಂ ॥೧೬೫॥

ಮಂತ್ರಾತ್ಮಕಮಿದಂ ಸರ್ವಂ ವ್ಯಾಖ್ಯಾತಂ ಸರ್ವಮಂಗಲಂ ।\\
ಉಕ್ತಾನಿ ತವ ಪುತ್ರೇಣ ವಿಘ್ನರಾಜೇನ ಧೀಮತಾ ॥೧೬೬॥

ಸನತ್ಕುಮಾರಾಯ ಪುರಾ ತಾನ್ಯುಕ್ತಾನಿ ಮಯಾ ತವ ।\\
ಯಃ ಪಠೇಚ್ಛೃಣುಯಾದ್ವಾಪಿ ಸ ತು ಬ್ರಹ್ಮಪದಂ ಲಭೇತ್ ॥೧೬೭॥

ತಾವದೇವ ಬಲಂ ತೇಷಾಂ ಮಹಾಪಾತಕದಂತಿನಾಂ ।\\
ಯಾವನ್ನ ಶ್ರೂಯತೇ ರಾಮನಾಮಪಂಚಾನನಧ್ವನಿಃ ॥೧೬೮॥

ಬ್ರಹ್ಮಘ್ನಶ್ಚ ಸುರಾಪಶ್ಚ ಸ್ತೇಯೀ ಚ ಗುರುತಲ್ಪಗಃ ।\\
ಶರಣಾಗತಘಾತೀ ಚ ಮಿತ್ರವಿಶ್ವಾಸಘಾತಕಃ ॥೧೬೯॥

ಮಾತೃಹಾ ಪಿತೃಹಾ ಚೈವ ಭ್ರೂಣಹಾ ವೀರಹಾ ತಥಾ ।\\
ಕೋಟಿಕೋಟಿಸಹಸ್ರಾಣಿ ಹ್ಯುಪಪಾಪಾನಿ ಯಾನ್ಯಪಿ ॥೧೭೦॥

ಸಂವತ್ಸರಂ ಕ್ರಮಾಜ್ಜಪ್ತ್ವಾ ಪ್ರತ್ಯಹಂ ರಾಮಸನ್ನಿಧೌ ।\\
ನಿಷ್ಕಂಟಕಂ ಸುಖಂ ಭುಕ್ತ್ವಾ ತತೋ ಮೋಕ್ಷಮವಾಪ್ನುಯಾತ್ ॥೧೭೧॥

ಶ್ರೀರಾಮನಾಮ್ನಾಂ ಪರಮಂ ಸಹಸ್ರಕಂ\\ ಪಾಪಾಪಹಂ ಸೌಖ್ಯವಿವೃದ್ಧಿಕಾರಕಂ ।\\
ಭವಾಪಹಂ ಭಕ್ತಜನೈಕಪಾಲಕಂ \\ಸ್ತ್ರೀಪುತ್ರಪೌತ್ರಪ್ರದಮೃದ್ಧಿದಾಯಕಂ ॥

\authorline{ಇತಿ ಶ್ರೀಮದಾನಂದರಾಮಾಯಣೇ ವಾಲ್ಮೀಕೀಯೇ ರಾಜ್ಯಕಾಂಡೇ ಪೂರ್ವಾರ್ಧೇ ಶ್ರೀರಾಮಸಹಸ್ರನಾಮಸ್ತ್ರೋತ್ರಮ್ ॥}

%======================================================================================================
\section{ಶ್ರೀರಾಮಾಷ್ಟೋತ್ತರಶತನಾಮಸ್ತೋತ್ರಂ}
\addcontentsline{toc}{section}{ಶ್ರೀರಾಮಾಷ್ಟೋತ್ತರಶತನಾಮಸ್ತೋತ್ರಂ}

ರಾಮೋ ರಾವಣಸಂಹಾರಕೃತಮಾನುಷವಿಗ್ರಹಃ ।\\
ಕೌಸಲ್ಯಾಸುಕೃತವ್ರಾತಫಲಂ ದಶರಥಾತ್ಮಜಃ ॥೧॥

ಲಕ್ಷ್ಮಣಾರ್ಚಿತಪಾದಾಬ್ಜಃಸರ್ವಲೋಕಪ್ರಿಯಂಕರಃ ।\\
ಸಾಕೇತವಾಸಿನೇತ್ರಾಬ್ಜಸಂಪ್ರೀಣನದಿವಾಕರಃ ॥೨॥

ವಿಶ್ವಾಮಿತ್ರಪ್ರಿಯಶ್ಶಾಂತಃ ತಾಟಕಾಧ್ವಾಂತಭಾಸ್ಕರಃ ।\\
ಸುಬಾಹುರಾಕ್ಷಸರಿಪುಃ ಕೌಶಿಕಾಧ್ವರಪಾಲಕಃ ॥೩॥

ಅಹಲ್ಯಾಪಾಪಸಂಹರ್ತಾ ಜನಕೇಂದ್ರಪ್ರಿಯಾತಿಥಿಃ ।\\
ಪುರಾರಿಚಾಪದಲನೋ ವೀರಲಕ್ಷ್ಮೀಸಮಾಶ್ರಯಃ ॥೪॥

ಸೀತಾವರಣಮಾಲ್ಯಾಢ್ಯೋ ಜಾಮದಗ್ನ್ಯಮದಾಪಹಃ ।\\
ವೈದೇಹೀಕೃತಶೃಂಗಾರಃ ಪಿತೃಪ್ರೀತಿವಿವರ್ಧನಃ ॥೫॥

ತಾತಾಜ್ಞೋತ್ಸೃಷ್ಟಹಸ್ತಸ್ಥರಾಜ್ಯಸ್ಸತ್ಯಪ್ರತಿಶ್ರವಃ ।\\
ತಮಸಾತೀರಸಂವಾಸೀ ಗುಹಾನುಗ್ರಹತತ್ಪರಃ ॥೬॥

ಸುಮಂತ್ರಸೇವಿತಪದೋ ಭರದ್ವಾಜಪ್ರಿಯಾತಿಥಿಃ ।\\
ಚಿತ್ರಕೂಟಪ್ರಿಯಾವಾಸಃ ಪಾದುಕಾನ್ಯಸ್ತಭೂಭರಃ ॥೭॥

ಅನಸೂಯಾಂಗರಾಗಾಂಕಸೀತಾಸಾಹಿತ್ಯಶೋಭಿತಃ ।\\
ದಂಡಕಾರಣ್ಯಸಂಚಾರೀ ವಿರಾಧಸ್ವರ್ಗದಾಯಕಃ ॥೮॥

ರಕ್ಷಃಕಾಲಾಂತಕಸ್ಸರ್ವಮುನಿಸಂಘಮುದಾವಹಃ ।\\
ಪ್ರತಿಜ್ಞಾತಾಸುರವಧಃ ಶರಭಂಗಗತಿಪ್ರದಃ ॥೯॥

ಅಗಸ್ತ್ಯಾರ್ಪಿತಬಾಣಾಸಖಡ್ಗತೂಣೀರಮಂಡಿತಃ ।\\
ಪ್ರಾಪ್ತಪಂಚವಟೀವಾಸೋ ಗೃಧ್ರರಾಜಸಹಾಯವಾನ್ ॥೧೦॥

ಕಾಮಿಶೂರ್ಪಣಖಾಕರ್ಣನಾಸಾಚ್ಛೇದನಿಯಾಮಕಃ ।\\
ಖರಾದಿರಾಕ್ಷಸವ್ರಾತಖಂಡನಾವಿತಸಜ್ಜನಃ ॥೧೧॥

ಸೀತಾಸಂಶ್ಲಿಷ್ಟಕಾಯಾಭಾಜಿತವಿದ್ಯುದ್ಯುತಾಂಬುದಃ ।\\
ಮಾರೀಚಹಂತಾ ಮಾಯಾಢ್ಯೋ ಜಟಾಯುರ್ಮೋಕ್ಷದಾಯಕಃ ॥೧೨॥

ಕಬಂಧಬಾಹುದಲನಶ್ಶಬರೀಪ್ರಾರ್ಥಿತಾತಿಥಿಃ ।\\
ಹನುಮದ್ವಂದಿತಪದಸ್ಸುಗ್ರೀವಸುಹೃದವ್ಯಯಃ ॥೧೩॥

ದೈತ್ಯಕಂಕಾಲವಿಕ್ಷೇಪೀ ಸಪ್ತತಾಲಪ್ರಭೇದಕಃ ।\\
ಏಕೇಷುಹತವಾಲೀ ಚ ತಾರಾಸಂಸ್ತುತಸದ್ಗುಣಃ ॥೧೪॥

ಕಪೀಂದ್ರೀಕೃತಸುಗ್ರೀವಸ್ಸರ್ವವಾನರಪೂಜಿತಃ ।\\
ವಾಯುಸೂನುಸಮಾನೀತಸೀತಾಸಂದೇಶನಂದಿತಃ ॥೧೫॥

ಜೈತ್ರಯಾತ್ರೋತ್ಸವಃ ಜಿಷ್ಣುರ್ವಿಷ್ಣುರೂಪೋ ನಿರಾಕೃತಿಃ ।\\
ಕಂಪಿತಾಂಭೋನಿಧಿಸ್ಸಂಪತ್ಪ್ರದಸ್ಸೇತುನಿಬಂಧನಃ ॥೧೬॥

ಲಂಕಾವಿಭೇದನಪಟುರ್ನಿಶಾಚರವಿನಾಶಕಃ ।\\
ಕುಂಭಕರ್ಣಾಖ್ಯಕುಂಭೀಂದ್ರಮೃಗರಾಜಪರಾಕ್ರಮಃ ॥೧೭॥

ಮೇಘನಾದವಧೋದ್ಯುಕ್ತಲಕ್ಷ್ಮಣಾಸ್ತ್ರಬಲಪ್ರದಃ ।\\
ದಶಗ್ರೀವಾಂಧತಾಮಿಸ್ರಪ್ರಮಾಪಣಪ್ರಭಾಕರಃ ॥೧೮॥

ಇಂದ್ರಾದಿದೇವತಾಸ್ತುತ್ಯಶ್ಚಂದ್ರಾಭಮುಖಮಂಡಲಃ ।\\
ವಿಭೀಷಣಾರ್ಪಿತನಿಶಾಚರರಾಜ್ಯೋ ವೃಷಪ್ರಿಯಃ ॥೧೯॥

ವೈಶ್ವಾನರಸ್ತುತಗುಣಾವನಿಪುತ್ರೀಸಮಾಗತಃ ।\\
ಪುಷ್ಪಕಸ್ಥಾನಸುಭಗಃ ಪುಣ್ಯವತ್ಪ್ರಾಪ್ಯದರ್ಶನಃ ॥೨೦॥

ರಾಜ್ಯಾಭಿಷಿಕ್ತೋ ರಾಜೇಂದ್ರೋ ರಾಜೀವಸದೃಶೇಕ್ಷಣಃ ।\\
ಲೋಕತಾಪಪರಿಹಂತಾ ಧರ್ಮಸಂಸ್ಥಾಪನೋದ್ಯತಃ ॥೨೧॥

ಶರಣ್ಯಃ ಕೀರ್ತಿಮಾನ್ನಿತ್ಯೋ ವದಾನ್ಯಃ ಕರುಣಾರ್ಣವಃ ।\\
ಸಂಸಾರಸಿಂಧುಸಮ್ಮಗ್ನತಾರಕಾಖ್ಯಾಮಹೋಜ್ಜವಲಃ ॥೨೨॥

ಮಧುರೋಮಧುರೋಕ್ತಿಶ್ಚ ಮಧುರಾನಾಯಕಾಗ್ರಜಃ ।\\
ಶಂಬೂಕದತ್ತಸ್ವರ್ಲೋಕಶ್ಶಂಬರಾರಾತಿಸುಂದರಃ ॥೨೩॥

ಅಶ್ವಮೇಧಮಹಾಯಾಜೀ ವಾಲ್ಮೀಕಿಪ್ರೀತಿಮಾನ್ವಶೀ ।\\
ಸ್ವಯಂರಾಮಾಯಣಶ್ರೋತಾ ಪುತ್ರಪ್ರಾಪ್ತಿಪ್ರಮೋದಿತಃ ॥೨೪॥

ಬ್ರಹ್ಮಾದಿಸ್ತುತಮಾಹಾತ್ಮ್ಯೋ ಬ್ರಹ್ಮರ್ಷಿಗಣಪೂಜಿತಃ ।\\
ವರ್ಣಾಶ್ರಮರತೋ ವರ್ಣಾಶ್ರಮಧರ್ಮನಿಯಾಮಕಃ ॥೨೫॥

ರಕ್ಷಾಪರೋ ರಾಜವಂಶಪ್ರತಿಷ್ಠಾಪನತತ್ಪರಃ ।\\
ಗಂಧರ್ವಹಿಂಸಾಸಂಹಾರೀ ಧೃತಿಮಾಂದೀನವತ್ಸಲಃ ॥೨೬॥

ಜ್ಞಾನೋಪದೇಷ್ಟಾ ವೇದಾಂತವೇದ್ಯೋ ಭಕ್ತಪ್ರಿಯಂಕರಃ ।\\
ವೈಕುಂಠವಾಸೀ ಪಾಯಾನ್ನಶ್ಚರಾಚರವಿಮುಕ್ತಿದಃ ॥೨೭॥
\authorline{ಇತಿ ಶ್ರೀರಾಮರಹಸ್ಯೋಕ್ತಂ ಶ್ರೀರಾಮಾಷ್ಟೋತ್ತರಶತನಾಮಸ್ತ್ತೋರಂ ಸಂಪೂರ್ಣಂ ।}
%=============================================================================================

\section{ಶ್ರೀವಿಷ್ಣುಸಹಸ್ರನಾಮಸ್ತೋತ್ರಂ}
\addcontentsline{toc}{section}{ಶ್ರೀವಿಷ್ಣುಸಹಸ್ರನಾಮಸ್ತೋತ್ರಂ}


ನಾರಾಯಣಂ ನಮಸ್ಕೃತ್ಯ ನರಂ ಚೈವ ನರೋತ್ತಮಂ ।\\
ದೇವೀಂ ಸರಸ್ವತೀಂ ವ್ಯಾಸಂ ತತೋ ಜಯಮುದೀರಯೇತ್॥

ಓಂ ಅಥ ಸಕಲಸೌಭಾಗ್ಯದಾಯಕ ಶ್ರೀವಿಷ್ಣುಸಹಸ್ರನಾಮಸ್ತೋತ್ರಂ ।\\

ಶುಕ್ಲಾಂಬರಧರಂ ವಿಷ್ಣುಂ ಶಶಿವರ್ಣಂ ಚತುರ್ಭುಜಂ ।\\
ಪ್ರಸನ್ನವದನಂ ಧ್ಯಾಯೇತ್ ಸರ್ವವಿಘ್ನೋಪಶಾಂತಯೇ ॥೧॥

ಯಸ್ಯ ದ್ವಿರದವಕ್ತ್ರಾದ್ಯಾಃ ಪಾರಿಷದ್ಯಾಃ ಪರಃ ಶತಂ ।\\
ವಿಘ್ನಂ ನಿಘ್ನಂತಿ ಸತತಂ ವಿಷ್ವಕ್ಸೇನಂ ತಮಾಶ್ರಯೇ ॥೨॥

ವ್ಯಾಸಂ ವಸಿಷ್ಠನಪ್ತಾರಂ ಶಕ್ತೇಃ ಪೌತ್ರಮಕಲ್ಮಷಂ ।\\
ಪರಾಶರಾತ್ಮಜಂ ವಂದೇ ಶುಕತಾತಂ ತಪೋನಿಧಿಂ ॥೩॥

ವ್ಯಾಸಾಯ ವಿಷ್ಣುರೂಪಾಯ ವ್ಯಾಸರೂಪಾಯ ವಿಷ್ಣವೇ ।\\
ನಮೋ ವೈ ಬ್ರಹ್ಮನಿಧಯೇ ವಾಸಿಷ್ಠಾಯ ನಮೋ ನಮಃ ॥೪॥

ಅವಿಕಾರಾಯ ಶುದ್ಧಾಯ ನಿತ್ಯಾಯ ಪರಮಾತ್ಮನೇ ।\\
ಸದೈಕರೂಪರೂಪಾಯ ವಿಷ್ಣವೇ ಸರ್ವಜಿಷ್ಣವೇ ॥೫॥

ಯಸ್ಯ ಸ್ಮರಣಮಾತ್ರೇಣ ಜನ್ಮಸಂಸಾರಬಂಧನಾತ್ ।\\
ವಿಮುಚ್ಯತೇ ನಮಸ್ತಸ್ಮೈ ವಿಷ್ಣವೇ ಪ್ರಭವಿಷ್ಣವೇ ॥೬॥

ಓಂ ನಮೋ ವಿಷ್ಣವೇ ಪ್ರಭವಿಷ್ಣವೇ ।\\
ಶ್ರೀವೈಶಂಪಾಯನ ಉವಾಚ॥
ಶ್ರುತ್ವಾ ಧರ್ಮಾನಶೇಷೇಣ ಪಾವನಾನಿ ಚ ಸರ್ವಶಃ ।\\
ಯುಧಿಷ್ಠಿರಃ ಶಾಂತನವಂ ಪುನರೇವಾಭ್ಯಭಾಷತ ॥೭॥

ಯುಧಿಷ್ಠಿರ ಉವಾಚ॥
ಕಿಮೇಕಂ ದೈವತಂ ಲೋಕೇ ಕಿಂ ವಾಪ್ಯೇಕಂ ಪರಾಯಣಂ ।\\
ಸ್ತುವಂತಃ ಕಂ ಕಮರ್ಚಂತಃ ಪ್ರಾಪ್ನುಯುರ್ಮಾನವಾಃ ಶುಭಂ ॥೮॥

ಕೋ ಧರ್ಮಃ ಸರ್ವಧರ್ಮಾಣಾಂ ಭವತಃ ಪರಮೋ ಮತಃ ।\\
ಕಿಂ ಜಪನ್ಮುಚ್ಯತೇ ಜಂತುರ್ಜನ್ಮಸಂಸಾರಬಂಧನಾತ್ ॥೯॥

ಭೀಷ್ಮ ಉವಾಚ॥
ಜಗತ್ಪ್ರಭುಂ ದೇವದೇವಮನಂತಂ ಪುರುಷೋತ್ತಮಂ ।\\
ಸ್ತುವನ್ ನಾಮಸಹಸ್ರೇಣ ಪುರುಷಃ ಸತತೋತ್ಥಿತಃ ॥೧೦॥

ತಮೇವ ಚಾರ್ಚಯನ್ನಿತ್ಯಂ ಭಕ್ತ್ಯಾ ಪುರುಷಮವ್ಯಯಂ ।\\
ಧ್ಯಾಯನ್ ಸ್ತುವನ್ ನಮಸ್ಯಂಶ್ಚ ಯಜಮಾನಸ್ತಮೇವ ಚ ॥೧೧॥

ಅನಾದಿನಿಧನಂ ವಿಷ್ಣುಂ ಸರ್ವಲೋಕಮಹೇಶ್ವರಂ ।\\
ಲೋಕಾಧ್ಯಕ್ಷಂ ಸ್ತುವನ್ನಿತ್ಯಂ ಸರ್ವದುಃಖಾತಿಗೋ ಭವೇತ್ ॥೧೨॥

ಬ್ರಹ್ಮಣ್ಯಂ ಸರ್ವಧರ್ಮಜ್ಞಂ ಲೋಕಾನಾಂ ಕೀರ್ತಿವರ್ಧನಂ ।\\
ಲೋಕನಾಥಂ ಮಹದ್ಭೂತಂ ಸರ್ವಭೂತಭವೋದ್ಭವಂ ॥೧೩॥

ಏಷ ಮೇ ಸರ್ವಧರ್ಮಾಣಾಂ ಧರ್ಮೋಽಧಿಕತಮೋ ಮತಃ ।\\
ಯದ್ಭಕ್ತ್ಯಾ ಪುಂಡರೀಕಾಕ್ಷಂ ಸ್ತವೈರರ್ಚೇನ್ನರಃ ಸದಾ ॥೧೪॥

ಪರಮಂ ಯೋ ಮಹತ್ತೇಜಃ ಪರಮಂ ಯೋ ಮಹತ್ತಪಃ ।\\
ಪರಮಂ ಯೋ ಮಹದ್ಬ್ರಹ್ಮ ಪರಮಂ ಯಃ ಪರಾಯಣಂ ॥೧೫॥

ಪವಿತ್ರಾಣಾಂ ಪವಿತ್ರಂ ಯೋ ಮಂಗಲಾನಾಂ ಚ ಮಂಗಲಂ ।\\
ದೈವತಂ ದೈವತಾನಾಂ ಚ ಭೂತಾನಾಂ ಯೋಽವ್ಯಯಃ ಪಿತಾ ॥೧೬॥

ಯತಃ ಸರ್ವಾಣಿ ಭೂತಾನಿ ಭವಂತ್ಯಾದಿಯುಗಾಗಮೇ ।\\
ಯಸ್ಮಿಂಶ್ಚ ಪ್ರಲಯಂ ಯಾಂತಿ ಪುನರೇವ ಯುಗಕ್ಷಯೇ ॥೧೭॥

ತಸ್ಯ ಲೋಕಪ್ರಧಾನಸ್ಯ ಜಗನ್ನಾಥಸ್ಯ ಭೂಪತೇ ।\\
ವಿಷ್ಣೋರ್ನಾಮಸಹಸ್ರಂ ಮೇ ಶೃಣು ಪಾಪಭಯಾಪಹಂ ॥೧೮॥

ಯಾನಿ ನಾಮಾನಿ ಗೌಣಾನಿ ವಿಖ್ಯಾತಾನಿ ಮಹಾತ್ಮನಃ ।\\
ಋಷಿಭಿಃ ಪರಿಗೀತಾನಿ ತಾನಿ ವಕ್ಷ್ಯಾಮಿ ಭೂತಯೇ ॥೧೯॥

ಋಷಿರ್ನಾಮ್ನಾಂ ಸಹಸ್ರಸ್ಯ ವೇದವ್ಯಾಸೋ ಮಹಾಮುನಿಃ॥

ಛಂದೋಽನುಷ್ಟುಪ್ ತಥಾ ದೇವೋ ಭಗವಾನ್ ದೇವಕೀಸುತಃ ॥೨೦॥

ಅಮೃತಾಂಶೂದ್ಭವೋ ಬೀಜಂ ಶಕ್ತಿರ್ದೇವಕಿನಂದನಃ ।\\
ತ್ರಿಸಾಮಾ ಹೃದಯಂ ತಸ್ಯ ಶಾಂತ್ಯರ್ಥೇ ವಿನಿಯೋಜ್ಯತೇ ॥೨೧॥

ವಿಷ್ಣುಂ ಜಿಷ್ಣುಂ ಮಹಾವಿಷ್ಣುಂ ಪ್ರಭವಿಷ್ಣುಂ ಮಹೇಶ್ವರಂ॥

ಅನೇಕರೂಪ ದೈತ್ಯಾಂತಂ ನಮಾಮಿ ಪುರುಷೋತ್ತಮಂ ॥೨೨॥

ಪೂರ್ವನ್ಯಾಸಃ ।\\
ಶ್ರೀವೇದವ್ಯಾಸ ಉವಾಚ॥
ಓಂ ಅಸ್ಯ ಶ್ರೀವಿಷ್ಣೋರ್ದಿವ್ಯಸಹಸ್ರನಾಮಸ್ತೋತ್ರಮಹಾಮಂತ್ರಸ್ಯ ।\\
ಶ್ರೀ ವೇದವ್ಯಾಸೋ ಭಗವಾನ್ ಋಷಿಃ ।\\
ಅನುಷ್ಟುಪ್ ಛಂದಃ ।\\
ಶ್ರೀಮಹಾವಿಷ್ಣುಃ ಪರಮಾತ್ಮಾ ಶ್ರೀಮನ್ನಾರಾಯಣೋ ದೇವತಾ ।\\
ಅಮೃತಾಂಶೂದ್ಭವೋ ಭಾನುರಿತಿ ಬೀಜಂ ।\\
ದೇವಕೀನಂದನಃ ಸ್ರಷ್ಟೇತಿ ಶಕ್ತಿಃ ।\\
ಉದ್ಭವಃ ಕ್ಷೋಭಣೋ ದೇವ ಇತಿ ಪರಮೋ ಮಂತ್ರಃ ।\\
ಶಂಖಭೃನ್ನಂದಕೀ ಚಕ್ರೀತಿ ಕೀಲಕಂ ।\\
ಶಾರ್ಙ್ಗಧನ್ವಾ ಗದಾಧರ ಇತ್ಯಸ್ತ್ರಂ ।\\
ರಥಾಂಗಪಾಣಿರಕ್ಷೋಭ್ಯ ಇತಿ ನೇತ್ರಂ ।\\
ತ್ರಿಸಾಮಾ ಸಾಮಗಃ ಸಾಮೇತಿ ಕವಚಂ ।\\
ಆನಂದಂ ಪರಬ್ರಹ್ಮೇತಿ ಯೋನಿಃ ।\\
ಋತುಃ ಸುದರ್ಶನಃ ಕಾಲ ಇತಿ ದಿಗ್ಬಂಧಃ॥

ಶ್ರೀವಿಶ್ವರೂಪ ಇತಿ ಧ್ಯಾನಂ ।\\
ಶ್ರೀಮಹಾವಿಷ್ಣುಪ್ರೀತ್ಯರ್ಥೇ ಸಹಸ್ರನಾಮಸ್ತೋತ್ರಪಾಠೇ ವಿನಿಯೋಗಃ॥

ಅಥ ನ್ಯಾಸಃ ।\\
ಓಂ ಶಿರಸಿ ವೇದವ್ಯಾಸಋಷಯೇ ನಮಃ ।\\
ಮುಖೇ ಅನುಷ್ಟುಪ್ಛಂದಸೇ ನಮಃ ।\\
ಹೃದಿ ಶ್ರೀಕೃಷ್ಣಪರಮಾತ್ಮದೇವತಾಯೈ ನಮಃ ।\\
ಗುಹ್ಯೇ ಅಮೃತಾಂಶೂದ್ಭವೋ ಭಾನುರಿತಿ ಬೀಜಾಯ ನಮಃ ।\\
ಪಾದಯೋರ್ದೇವಕೀನಂದನಃ ಸ್ರಷ್ಟೇತಿ ಶಕ್ತಯೇ ನಮಃ ।\\
ಸರ್ವಾಂಗೇ ಶಂಖಭೃನ್ನಂದಕೀ ಚಕ್ರೀತಿ ಕೀಲಕಾಯ ನಮಃ ।\\
ಕರಸಂಪೂಟೇ ಮಮ ಶ್ರೀಕೃಷ್ಣಪ್ರೀತ್ಯರ್ಥೇ ಜಪೇ ವಿನಿಯೋಗಾಯ ನಮಃ॥

ಇತಿ ಋಷಯಾದಿನ್ಯಾಸಃ॥

ಅಥ ಕರನ್ಯಾಸಃ ।\\
ಓಂ ವಿಶ್ವಂ ವಿಷ್ಣುರ್ವಷಟ್ಕಾರ ಇತ್ಯಂಗುಷ್ಠಾಭ್ಯಾಂ ನಮಃ ।\\
ಅಮೃತಾಂಶೂದ್ಭವೋ ಭಾನುರಿತಿ ತರ್ಜನೀಭ್ಯಾಂ ನಮಃ ।\\
ಬ್ರಹ್ಮಣ್ಯೋ ಬ್ರಹ್ಮಕೃದ್ಬ್ರಹ್ಮೇತಿ ಮಧ್ಯಮಾಭ್ಯಾಂ ನಮಃ ।\\
ಸುವರ್ಣಬಿಂದುರಕ್ಷೋಭ್ಯ ಇತ್ಯನಾಮಿಕಾಭ್ಯಾಂ ನಮಃ ।\\
ನಿಮಿಷೋಽನಿಮಿಷಃ ಸ್ರಗ್ವೀತಿ ಕನಿಷ್ಠಿಕಾಭ್ಯಾಂ ನಮಃ ।\\
ರಥಾಂಗಪಾಣಿರಕ್ಷೋಭ್ಯ ಇತಿ ಕರತಲಕರಪೃಷ್ಠಾಭ್ಯಾಂ ನಮಃ ।\\
ಇತಿ ಕರನ್ಯಾಸಃ ।\\
ಅಥ ಷಡಂಗನ್ಯಾಸಃ ।\\
ಓಂ ವಿಶ್ವಂ ವಿಷ್ಣುರ್ವಷಟ್ಕಾರ ಇತಿ ಹೃದಯಾಯ ನಮಃ ।\\
ಅಮೃತಾಂಶೂದ್ಭವೋ ಭಾನುರಿತಿ ಶಿರಸೇ ಸ್ವಾಹಾ ।\\
ಬ್ರಹ್ಮಣ್ಯೋ ಬ್ರಹ್ಮಕೃದ್ಬ್ರಹ್ಮೇತಿ ಶಿಖಾಯೈ ವಷಟ್ ।\\
ಸುವರ್ಣಬಿಂದುರಕ್ಷೋಭ್ಯ ಇತಿ ಕವಚಾಯ ಹುಂ ।\\
ನಿಮಿಷೋಽನಿಮಿಷಃ ಸ್ರಗ್ವೀತಿ ನೇತ್ರತ್ರಯಾಯ ವೌಷಟ್ ।\\
ರಥಾಂಗಪಾಣಿರಕ್ಷೋಭ್ಯ ಇತ್ಯಸ್ತ್ರಾಯ ಫಟ್ ।\\
ಇತಿ ಷಡಂಗನ್ಯಾಸಃ॥

ಶ್ರೀಕೃಷ್ಣಪ್ರೀತ್ಯರ್ಥೇ ವಿಷ್ಣೋರ್ದಿವ್ಯಸಹಸ್ರನಾಮಜಪಮಹಂ ಕರಿಷ್ಯೇ ಇತಿ ಸಂಕಲ್ಪಃ ।\\
ಅಥ ಧ್ಯಾನಂ ।\\
ಕ್ಷೀರೋದನ್ವತ್ಪ್ರದೇಶೇ ಶುಚಿಮಣಿವಿಲಸತ್ಸೈಕತೇರ್ಮೌಕ್ತಿಕಾನಾ\\
ಮಾಲಾಕೢಪ್ತಾಸನಸ್ಥಃ ಸ್ಫಟಿಕಮಣಿನಿಭೈರ್ಮೌಕ್ತಿಕೈರ್ಮಂಡಿತಾಂಗಃ ।\\
ಶುಭ್ರೈರಭ್ರೈರದಭ್ರೈರುಪರಿವಿರಚಿತೈರ್ಮುಕ್ತಪೀಯೂಷ ವರ್ಷೈ\\
ಆನಂದೀ ನಃ ಪುನೀಯಾದರಿನಲಿನಗದಾ ಶಂಖಪಾಣಿರ್ಮುಕುಂದಃ ॥೧॥

ಭೂಃ ಪಾದೌ ಯಸ್ಯ ನಾಭಿರ್ವಿಯದಸುರನಿಲಶ್ಚಂದ್ರ ಸೂರ್ಯೌ ಚ ನೇತ್ರ\\
ಕರ್ಣಾವಾಶಾಃ ಶಿರೋ ದ್ಯೌರ್ಮುಖಮಪಿ ದಹನೋ ಯಸ್ಯ ವಾಸ್ತೇಯಮಬ್ಧಿಃ ।\\
ಅಂತಃಸ್ಥಂ ಯಸ್ಯ ವಿಶ್ವಂ ಸುರನರಖಗಗೋಭೋಗಿಗಂಧರ್ವದೈತ್ಯೈ\\
ಚಿತ್ರಂ ರಂರಮ್ಯತೇ ತಂ ತ್ರಿಭುವನ ವಪುಷಂ ವಿಷ್ಣುಮೀಶಂ ನಮಾಮಿ ॥೨॥

ಓಂ ಶಾಂತಾಕಾರಂ ಭುಜಗಶಯನಂ ಪದ್ಮನಾಭಂ ಸುರೇಶ\\
ವಿಶ್ವಾಧಾರಂ ಗಗನಸದೃಶಂ ಮೇಘವರ್ಣಂ ಶುಭಾಂಗಂ ।\\
ಲಕ್ಷ್ಮೀಕಾಂತಂ ಕಮಲನಯನಂ ಯೋಗಿಭಿರ್ಧ್ಯಾನಗಮ್ಯ\\
ವಂದೇ ವಿಷ್ಣುಂ ಭವಭಯಹರಂ ಸರ್ವಲೋಕೈಕನಾಥಂ ॥೩॥

ಮೇಘಶ್ಯಾಮಂ ಪೀತಕೌಶೇಯವಾಸ\\
ಶ್ರೀವತ್ಸಾಂಕಂ ಕೌಸ್ತುಭೋದ್ಭಾಸಿತಾಂಗಂ ।\\
ಪುಣ್ಯೋಪೇತಂ ಪುಂಡರೀಕಾಯತಾಕ್ಷ\\
ವಿಷ್ಣುಂ ವಂದೇ ಸರ್ವಲೋಕೈಕನಾಥಂ ॥೪॥

ನಮಃ ಸಮಸ್ತಭೂತಾನಾಮಾದಿಭೂತಾಯ ಭೂಭೃತೇ ।\\
ಅನೇಕರೂಪರೂಪಾಯ ವಿಷ್ಣವೇ ಪ್ರಭವಿಷ್ಣವೇ ॥೫॥

ಸಶಂಖಚಕ್ರಂ ಸಕಿರೀಟಕುಂಡಲ\\
ಸಪೀತವಸ್ತ್ರಂ ಸರಸೀರುಹೇಕ್ಷಣಂ ।\\
ಸಹಾರವಕ್ಷಃಸ್ಥಲಕೌಸ್ತುಭಶ್ರಿಯ\\
ನಮಾಮಿ ವಿಷ್ಣುಂ ಶಿರಸಾ ಚತುರ್ಭುಜಂ ॥೬॥

ಛಾಯಾಯಾಂ ಪಾರಿಜಾತಸ್ಯ ಹೇಮಸಿಂಹಾಸನೋಪರ\\
ಆಸೀನಮಂಬುದಶ್ಯಾಮಮಾಯತಾಕ್ಷಮಲಂಕೃತಂ ।\\
ಚಂದ್ರಾನನಂ ಚತುರ್ಬಾಹುಂ ಶ್ರೀವತ್ಸಾಂಕಿತ ವಕ್ಷಸ\\
ರುಕ್ಮಿಣೀ ಸತ್ಯಭಾಮಾಭ್ಯಾಂ ಸಹಿತಂ ಕೃಷ್ಣಮಾಶ್ರಯೇ ॥೭॥


ಸ್ತೋತ್ರಂ ।\\
ಹರಿಃ ಓಂ ।\\
ವಿಶ್ವಂ ವಿಷ್ಣುರ್ವಷಟ್ಕಾರೋ ಭೂತಭವ್ಯಭವತ್ಪ್ರಭುಃ ।\\
ಭೂತಕೃದ್ಭೂತಭೃದ್ಭಾವೋ ಭೂತಾತ್ಮಾ ಭೂತಭಾವನಃ ॥೧॥

ಪೂತಾತ್ಮಾ ಪರಮಾತ್ಮಾ ಚ ಮುಕ್ತಾನಾಂ ಪರಮಾ ಗತಿಃ ।\\
ಅವ್ಯಯಃ ಪುರುಷಃ ಸಾಕ್ಷೀ ಕ್ಷೇತ್ರಜ್ಞೋಽಕ್ಷರ ಏವ ಚ ॥೨॥

ಯೋಗೋ ಯೋಗವಿದಾಂ ನೇತಾ ಪ್ರಧಾನಪುರುಷೇಶ್ವರಃ ।\\
ನಾರಸಿಂಹವಪುಃ ಶ್ರೀಮಾನ್ ಕೇಶವಃ ಪುರುಷೋತ್ತಮಃ ॥೩॥

ಸರ್ವಃ ಶರ್ವಃ ಶಿವಃ ಸ್ಥಾಣುರ್ಭೂತಾದಿರ್ನಿಧಿರವ್ಯಯಃ ।\\
ಸಂಭವೋ ಭಾವನೋ ಭರ್ತಾ ಪ್ರಭವಃ ಪ್ರಭುರೀಶ್ವರಃ ॥೪॥

ಸ್ವಯಂಭೂಃ ಶಂಭುರಾದಿತ್ಯಃ ಪುಷ್ಕರಾಕ್ಷೋ ಮಹಾಸ್ವನಃ ।\\
ಅನಾದಿನಿಧನೋ ಧಾತಾ ವಿಧಾತಾ ಧಾತುರುತ್ತಮಃ ॥೫॥

ಅಪ್ರಮೇಯೋ ಹೃಷೀಕೇಶಃ ಪದ್ಮನಾಭೋಽಮರಪ್ರಭುಃ ।\\
ವಿಶ್ವಕರ್ಮಾ ಮನುಸ್ತ್ವಷ್ಟಾ ಸ್ಥವಿಷ್ಠಃ ಸ್ಥವಿರೋ ಧ್ರುವಃ ॥೬॥

ಅಗ್ರಾಹ್ಯಃ ಶಾಶ್ವತಃ ಕೃಷ್ಣೋ ಲೋಹಿತಾಕ್ಷಃ ಪ್ರತರ್ದನಃ ।\\
ಪ್ರಭೂತಸ್ತ್ರಿಕಕುಬ್ಧಾಮ ಪವಿತ್ರಂ ಮಂಗಲಂ ಪರಂ ॥೭॥

ಈಶಾನಃ ಪ್ರಾಣದಃ ಪ್ರಾಣೋ ಜ್ಯೇಷ್ಠಃ ಶ್ರೇಷ್ಠಃ ಪ್ರಜಾಪತಿಃ ।\\
ಹಿರಣ್ಯಗರ್ಭೋ ಭೂಗರ್ಭೋ ಮಾಧವೋ ಮಧುಸೂದನಃ ॥೮॥

ಈಶ್ವರೋ ವಿಕ್ರಮೀ ಧನ್ವೀ ಮೇಧಾವೀ ವಿಕ್ರಮಃ ಕ್ರಮಃ ।\\
ಅನುತ್ತಮೋ ದುರಾಧರ್ಷಃ ಕೃತಜ್ಞಃ ಕೃತಿರಾತ್ಮವಾನ್ ॥೯॥

ಸುರೇಶಃ ಶರಣಂ ಶರ್ಮ ವಿಶ್ವರೇತಾಃ ಪ್ರಜಾಭವಃ ।\\
ಅಹಃ ಸಂವತ್ಸರೋ ವ್ಯಾಲಃ ಪ್ರತ್ಯಯಃ ಸರ್ವದರ್ಶನಃ ॥೧೦॥

ಅಜಃ ಸರ್ವೇಶ್ವರಃ ಸಿದ್ಧಃ ಸಿದ್ಧಿಃ ಸರ್ವಾದಿರಚ್ಯುತಃ ।\\
ವೃಷಾಕಪಿರಮೇಯಾತ್ಮಾ ಸರ್ವಯೋಗವಿನಿಃಸೃತಃ ॥೧೧॥

ವಸುರ್ವಸುಮನಾಃ ಸತ್ಯಃ ಸಮಾತ್ಮಾಽಸಮ್ಮಿತಃ ಸಮಃ ।\\
ಅಮೋಘಃ ಪುಂಡರೀಕಾಕ್ಷೋ ವೃಷಕರ್ಮಾ ವೃಷಾಕೃತಿಃ ॥೧೨॥

ರುದ್ರೋ ಬಹುಶಿರಾ ಬಭ್ರುರ್ವಿಶ್ವಯೋನಿಃ ಶುಚಿಶ್ರವಾಃ ।\\
ಅಮೃತಃ ಶಾಶ್ವತಸ್ಥಾಣುರ್ವರಾರೋಹೋ ಮಹಾತಪಾಃ ॥೧೩॥

ಸರ್ವಗಃ ಸರ್ವವಿದ್ಭಾನುರ್ವಿಷ್ವಕ್ಸೇನೋ ಜನಾರ್ದನಃ ।\\
ವೇದೋ ವೇದವಿದವ್ಯಂಗೋ ವೇದಾಂಗೋ ವೇದವಿತ್ ಕವಿಃ ॥೧೪॥

ಲೋಕಾಧ್ಯಕ್ಷಃ ಸುರಾಧ್ಯಕ್ಷೋ ಧರ್ಮಾಧ್ಯಕ್ಷಃ ಕೃತಾಕೃತಃ ।\\
ಚತುರಾತ್ಮಾ ಚತುರ್ವ್ಯೂಹಶ್ಚತುರ್ದಂಷ್ಟ್ರಶ್ಚತುರ್ಭುಜಃ ॥೧೫॥

ಭ್ರಾಜಿಷ್ಣುರ್ಭೋಜನಂ ಭೋಕ್ತಾ ಸಹಿಷ್ಣುರ್ಜಗದಾದಿಜಃ ।\\
ಅನಘೋ ವಿಜಯೋ ಜೇತಾ ವಿಶ್ವಯೋನಿಃ ಪುನರ್ವಸುಃ ॥೧೬॥

ಉಪೇಂದ್ರೋ ವಾಮನಃ ಪ್ರಾಂಶುರಮೋಘಃ ಶುಚಿರೂರ್ಜಿತಃ ।\\
ಅತೀಂದ್ರಃ ಸಂಗ್ರಹಃ ಸರ್ಗೋ ಧೃತಾತ್ಮಾ ನಿಯಮೋ ಯಮಃ ॥೧೭॥

ವೇದ್ಯೋ ವೈದ್ಯಃ ಸದಾಯೋಗೀ ವೀರಹಾ ಮಾಧವೋ ಮಧುಃ ।\\
ಅತೀಂದ್ರಿಯೋ ಮಹಾಮಾಯೋ ಮಹೋತ್ಸಾಹೋ ಮಹಾಬಲಃ ॥೧೮॥

ಮಹಾಬುದ್ಧಿರ್ಮಹಾವೀರ್ಯೋ ಮಹಾಶಕ್ತಿರ್ಮಹಾದ್ಯುತಿಃ ।\\
ಅನಿರ್ದೇಶ್ಯವಪುಃ ಶ್ರೀಮಾನಮೇಯಾತ್ಮಾ ಮಹಾದ್ರಿಧೃಕ್ ॥೧೯॥

ಮಹೇಷ್ವಾಸೋ ಮಹೀಭರ್ತಾ ಶ್ರೀನಿವಾಸಃ ಸತಾಂ ಗತಿಃ ।\\
ಅನಿರುದ್ಧಃ ಸುರಾನಂದೋ ಗೋವಿಂದೋ ಗೋವಿದಾಂ ಪತಿಃ ॥೨೦॥

ಮರೀಚಿರ್ದಮನೋ ಹಂಸಃ ಸುಪರ್ಣೋ ಭುಜಗೋತ್ತಮಃ ।\\
ಹಿರಣ್ಯನಾಭಃ ಸುತಪಾಃ ಪದ್ಮನಾಭಃ ಪ್ರಜಾಪತಿಃ ॥೨೧॥

ಅಮೃತ್ಯುಃ ಸರ್ವದೃಕ್ ಸಿಂಹಃ ಸಂಧಾತಾ ಸಂಧಿಮಾನ್ ಸ್ಥಿರಃ ।\\
ಅಜೋ ದುರ್ಮರ್ಷಣಃ ಶಾಸ್ತಾ ವಿಶ್ರುತಾತ್ಮಾ ಸುರಾರಿಹಾ ॥೨೨॥

ಗುರುರ್ಗುರುತಮೋ ಧಾಮ ಸತ್ಯಃ ಸತ್ಯಪರಾಕ್ರಮಃ ।\\
ನಿಮಿಷೋಽನಿಮಿಷಃ ಸ್ರಗ್ವೀ ವಾಚಸ್ಪತಿರುದಾರಧೀಃ ॥೨೩॥

ಅಗ್ರಣೀರ್ಗ್ರಾಮಣೀಃ ಶ್ರೀಮಾನ್ ನ್ಯಾಯೋ ನೇತಾ ಸಮೀರಣಃ ।\\
ಸಹಸ್ರಮೂರ್ಧಾ ವಿಶ್ವಾತ್ಮಾ ಸಹಸ್ರಾಕ್ಷಃ ಸಹಸ್ರಪಾತ್ ॥೨೪॥

ಆವರ್ತನೋ ನಿವೃತ್ತಾತ್ಮಾ ಸಂವೃತಃ ಸಂಪ್ರಮರ್ದನಃ ।\\
ಅಹಃ ಸಂವರ್ತಕೋ ವಹ್ನಿರನಿಲೋ ಧರಣೀಧರಃ ॥೨೫॥

ಸುಪ್ರಸಾದಃ ಪ್ರಸನ್ನಾತ್ಮಾ ವಿಶ್ವಧೃಗ್ವಿಶ್ವಭುಗ್ವಿಭುಃ ।\\
ಸತ್ಕರ್ತಾ ಸತ್ಕೃತಃ ಸಾಧುರ್ಜಹ್ನುರ್ನಾರಾಯಣೋ ನರಃ ॥೨೬॥

ಅಸಂಖ್ಯೇಯೋಽಪ್ರಮೇಯಾತ್ಮಾ ವಿಶಿಷ್ಟಃ ಶಿಷ್ಟಕೃಚ್ಛುಚಿಃ ।\\
ಸಿದ್ಧಾರ್ಥಃ ಸಿದ್ಧಸಂಕಲ್ಪಃ ಸಿದ್ಧಿದಃ ಸಿದ್ಧಿಸಾಧನಃ ॥೨೭॥

ವೃಷಾಹೀ ವೃಷಭೋ ವಿಷ್ಣುರ್ವೃಷಪರ್ವಾ ವೃಷೋದರಃ ।\\
ವರ್ಧನೋ ವರ್ಧಮಾನಶ್ಚ ವಿವಿಕ್ತಃ ಶ್ರುತಿಸಾಗರಃ ॥೨೮॥

ಸುಭುಜೋ ದುರ್ಧರೋ ವಾಗ್ಮೀ ಮಹೇಂದ್ರೋ ವಸುದೋ ವಸುಃ ।\\
ನೈಕರೂಪೋ ಬೃಹದ್ರೂಪಃ ಶಿಪಿವಿಷ್ಟಃ ಪ್ರಕಾಶನಃ ॥೨೯॥

ಓಜಸ್ತೇಜೋದ್ಯುತಿಧರಃ ಪ್ರಕಾಶಾತ್ಮಾ ಪ್ರತಾಪನಃ ।\\
ಋದ್ಧಃ ಸ್ಪಷ್ಟಾಕ್ಷರೋ ಮಂತ್ರಶ್ಚಂದ್ರಾಂಶುರ್ಭಾಸ್ಕರದ್ಯುತಿಃ ॥೩೦॥

ಅಮೃತಾಂಶೂದ್ಭವೋ ಭಾನುಃ ಶಶಬಿಂದುಃ ಸುರೇಶ್ವರಃ ।\\
ಔಷಧಂ ಜಗತಃ ಸೇತುಃ ಸತ್ಯಧರ್ಮಪರಾಕ್ರಮಃ ॥೩೧॥

ಭೂತಭವ್ಯಭವನ್ನಾಥಃ ಪವನಃ ಪಾವನೋಽನಲಃ ।\\
ಕಾಮಹಾ ಕಾಮಕೃತ್ಕಾಂತಃ ಕಾಮಃ ಕಾಮಪ್ರದಃ ಪ್ರಭುಃ ॥೩೨॥

ಯುಗಾದಿಕೃದ್ಯುಗಾವರ್ತೋ ನೈಕಮಾಯೋ ಮಹಾಶನಃ ।\\
ಅದೃಶ್ಯೋ ವ್ಯಕ್ತರೂಪಶ್ಚ ಸಹಸ್ರಜಿದನಂತಜಿತ್ ॥೩೩॥

ಇಷ್ಟೋಽವಿಶಿಷ್ಟಃ ಶಿಷ್ಟೇಷ್ಟಃ ಶಿಖಂಡೀ ನಹುಷೋ ವೃಷಃ ।\\
ಕ್ರೋಧಹಾ ಕ್ರೋಧಕೃತ್ಕರ್ತಾ ವಿಶ್ವಬಾಹುರ್ಮಹೀಧರಃ ॥೩೪॥

ಅಚ್ಯುತಃ ಪ್ರಥಿತಃ ಪ್ರಾಣಃ ಪ್ರಾಣದೋ ವಾಸವಾನುಜಃ ।\\
ಅಪಾಂನಿಧಿರಧಿಷ್ಠಾನಮಪ್ರಮತ್ತಃ ಪ್ರತಿಷ್ಠಿತಃ ॥೩೫॥

ಸ್ಕಂದಃ ಸ್ಕಂದಧರೋ ಧುರ್ಯೋ ವರದೋ ವಾಯುವಾಹನಃ ।\\
ವಾಸುದೇವೋ ಬೃಹದ್ಭಾನುರಾದಿದೇವಃ ಪುರಂದರಃ ॥೩೬॥

ಅಶೋಕಸ್ತಾರಣಸ್ತಾರಃ ಶೂರಃ ಶೌರಿರ್ಜನೇಶ್ವರಃ ।\\
ಅನುಕೂಲಃ ಶತಾವರ್ತಃ ಪದ್ಮೀ ಪದ್ಮನಿಭೇಕ್ಷಣಃ ॥೩೭॥

ಪದ್ಮನಾಭೋಽರವಿಂದಾಕ್ಷಃ ಪದ್ಮಗರ್ಭಃ ಶರೀರಭೃತ್ ।\\
ಮಹರ್ದ್ಧಿರೃದ್ಧೋ ವೃದ್ಧಾತ್ಮಾ ಮಹಾಕ್ಷೋ ಗರುಡಧ್ವಜಃ ॥೩೮॥

ಅತುಲಃ ಶರಭೋ ಭೀಮಃ ಸಮಯಜ್ಞೋ ಹವಿರ್ಹರಿಃ ।\\
ಸರ್ವಲಕ್ಷಣಲಕ್ಷಣ್ಯೋ ಲಕ್ಷ್ಮೀವಾನ್ ಸಮಿತಿಂಜಯಃ ॥೩೯॥

ವಿಕ್ಷರೋ ರೋಹಿತೋ ಮಾರ್ಗೋ ಹೇತುರ್ದಾಮೋದರಃ ಸಹಃ ।\\
ಮಹೀಧರೋ ಮಹಾಭಾಗೋ ವೇಗವಾನಮಿತಾಶನಃ ॥೪೦॥

ಉದ್ಭವಃ ಕ್ಷೋಭಣೋ ದೇವಃ ಶ್ರೀಗರ್ಭಃ ಪರಮೇಶ್ವರಃ ।\\
ಕರಣಂ ಕಾರಣಂ ಕರ್ತಾ ವಿಕರ್ತಾ ಗಹನೋ ಗುಹಃ ॥೪೧॥

ವ್ಯವಸಾಯೋ ವ್ಯವಸ್ಥಾನಃ ಸಂಸ್ಥಾನಃ ಸ್ಥಾನದೋ ಧ್ರುವಃ ।\\
ಪರರ್ದ್ಧಿಃ ಪರಮಸ್ಪಷ್ಟಸ್ತುಷ್ಟಃ ಪುಷ್ಟಃ ಶುಭೇಕ್ಷಣಃ ॥೪೨॥

ರಾಮೋ ವಿರಾಮೋ ವಿರಜೋ ಮಾರ್ಗೋ ನೇಯೋ ನಯೋಽನಯಃ ।\\
ವೀರಃ ಶಕ್ತಿಮತಾಂ ಶ್ರೇಷ್ಠೋ ಧರ್ಮೋ ಧರ್ಮವಿದುತ್ತಮಃ ॥೪೩॥

ವೈಕುಂಠಃ ಪುರುಷಃ ಪ್ರಾಣಃ ಪ್ರಾಣದಃ ಪ್ರಣವಃ ಪೃಥುಃ ।\\
ಹಿರಣ್ಯಗರ್ಭಃ ಶತ್ರುಘ್ನೋ ವ್ಯಾಪ್ತೋ ವಾಯುರಧೋಕ್ಷಜಃ ॥೪೪॥

ಋತುಃ ಸುದರ್ಶನಃ ಕಾಲಃ ಪರಮೇಷ್ಠೀ ಪರಿಗ್ರಹಃ ।\\
ಉಗ್ರಃ ಸಂವತ್ಸರೋ ದಕ್ಷೋ ವಿಶ್ರಾಮೋ ವಿಶ್ವದಕ್ಷಿಣಃ ॥೪೫॥

ವಿಸ್ತಾರಃ ಸ್ಥಾವರಸ್ಥಾಣುಃ ಪ್ರಮಾಣಂ ಬೀಜಮವ್ಯಯಂ ।\\
ಅರ್ಥೋಽನರ್ಥೋ ಮಹಾಕೋಶೋ ಮಹಾಭೋಗೋ ಮಹಾಧನಃ ॥೪೬॥

ಅನಿರ್ವಿಣ್ಣಃ ಸ್ಥವಿಷ್ಠೋಽಭೂರ್ಧರ್ಮಯೂಪೋ ಮಹಾಮಖಃ ।\\
ನಕ್ಷತ್ರನೇಮಿರ್ನಕ್ಷತ್ರೀ ಕ್ಷಮಃ ಕ್ಷಾಮಃ ಸಮೀಹನಃ ॥೪೭॥

ಯಜ್ಞ ಇಜ್ಯೋ ಮಹೇಜ್ಯಶ್ಚ ಕ್ರತುಃ ಸತ್ರಂ ಸತಾಂ ಗತಿಃ ।\\
ಸರ್ವದರ್ಶೀ ವಿಮುಕ್ತಾತ್ಮಾ ಸರ್ವಜ್ಞೋ ಜ್ಞಾನಮುತ್ತಮಂ ॥೪೮॥

ಸುವ್ರತಃ ಸುಮುಖಃ ಸೂಕ್ಷ್ಮಃ ಸುಘೋಷಃ ಸುಖದಃ ಸುಹೃತ್ ।\\
ಮನೋಹರೋ ಜಿತಕ್ರೋಧೋ ವೀರಬಾಹುರ್ವಿದಾರಣಃ ॥೪೯॥

ಸ್ವಾಪನಃ ಸ್ವವಶೋ ವ್ಯಾಪೀ ನೈಕಾತ್ಮಾ ನೈಕಕರ್ಮಕೃತ್ ।\\
ವತ್ಸರೋ ವತ್ಸಲೋ ವತ್ಸೀ ರತ್ನಗರ್ಭೋ ಧನೇಶ್ವರಃ ॥೫೦॥

ಧರ್ಮಗುಬ್ಧರ್ಮಕೃದ್ಧರ್ಮೀ ಸದಸತ್ಕ್ಷರಮಕ್ಷರಂ ।\\
ಅವಿಜ್ಞಾತಾ ಸಹಸ್ರಾಂಶುರ್ವಿಧಾತಾ ಕೃತಲಕ್ಷಣಃ ॥೫೧॥

ಗಭಸ್ತಿನೇಮಿಃ ಸತ್ತ್ವಸ್ಥಃ ಸಿಂಹೋ ಭೂತಮಹೇಶ್ವರಃ ।\\
ಆದಿದೇವೋ ಮಹಾದೇವೋ ದೇವೇಶೋ ದೇವಭೃದ್ಗುರುಃ ॥೫೨॥

ಉತ್ತರೋ ಗೋಪತಿರ್ಗೋಪ್ತಾ ಜ್ಞಾನಗಮ್ಯಃ ಪುರಾತನಃ ।\\
ಶರೀರಭೂತಭೃದ್ಭೋಕ್ತಾ ಕಪೀಂದ್ರೋ ಭೂರಿದಕ್ಷಿಣಃ ॥೫೩॥

ಸೋಮಪೋಽಮೃತಪಃ ಸೋಮಃ ಪುರುಜಿತ್ಪುರುಸತ್ತಮಃ ।\\
ವಿನಯೋ ಜಯಃ ಸತ್ಯಸಂಧೋ ದಾಶಾರ್ಹಃ ಸಾತ್ವತಾಂಪತಿಃ ॥೫೪॥

ಜೀವೋ ವಿನಯಿತಾ ಸಾಕ್ಷೀ ಮುಕುಂದೋಽಮಿತವಿಕ್ರಮಃ ।\\
ಅಂಭೋನಿಧಿರನಂತಾತ್ಮಾ ಮಹೋದಧಿಶಯೋಽನ್ತಕಃ ॥೫೫॥

ಅಜೋ ಮಹಾರ್ಹಃ ಸ್ವಾಭಾವ್ಯೋ ಜಿತಾಮಿತ್ರಃ ಪ್ರಮೋದನಃ ।\\
ಆನಂದೋ ನಂದನೋ ನಂದಃ ಸತ್ಯಧರ್ಮಾ ತ್ರಿವಿಕ್ರಮಃ ॥೫೬॥

ಮಹರ್ಷಿಃ ಕಪಿಲಾಚಾರ್ಯಃ ಕೃತಜ್ಞೋ ಮೇದಿನೀಪತಿಃ ।\\
ತ್ರಿಪದಸ್ತ್ರಿದಶಾಧ್ಯಕ್ಷೋ ಮಹಾಶೃಂಗಃ ಕೃತಾಂತಕೃತ್ ॥೫೭॥

ಮಹಾವರಾಹೋ ಗೋವಿಂದಃ ಸುಷೇಣಃ ಕನಕಾಂಗದೀ ।\\
ಗುಹ್ಯೋ ಗಭೀರೋ ಗಹನೋ ಗುಪ್ತಶ್ಚಕ್ರಗದಾಧರಃ ॥೫೮॥

ವೇಧಾಃ ಸ್ವಾಂಗೋಽಜಿತಃ ಕೃಷ್ಣೋ ದೃಢಃ ಸಂಕರ್ಷಣೋಽಚ್ಯುತಃ ।\\
ವರುಣೋ ವಾರುಣೋ ವೃಕ್ಷಃ ಪುಷ್ಕರಾಕ್ಷೋ ಮಹಾಮನಾಃ ॥೫೯॥

ಭಗವಾನ್ ಭಗಹಾಽಽನಂದೀ ವನಮಾಲೀ ಹಲಾಯುಧಃ ।\\
ಆದಿತ್ಯೋ ಜ್ಯೋತಿರಾದಿತ್ಯಃ ಸಹಿಷ್ಣುರ್ಗತಿಸತ್ತಮಃ ॥೬೦॥

ಸುಧನ್ವಾ ಖಂಡಪರಶುರ್ದಾರುಣೋ ದ್ರವಿಣಪ್ರದಃ ।\\
ದಿವಸ್ಪೃಕ್ ಸರ್ವದೃಗ್ವ್ಯಾಸೋ ವಾಚಸ್ಪತಿರಯೋನಿಜಃ ॥೬೧॥
ತ್ರಿಸಾಮಾ ಸಾಮಗಃ ಸಾಮ ನಿರ್ವಾಣಂ ಭೇಷಜಂ ಭಿಷಕ್ ।\\
ಸಂನ್ಯಾಸಕೃಚ್ಛಮಃ ಶಾಂತೋ ನಿಷ್ಠಾ ಶಾಂತಿಃ ಪರಾಯಣಂ ॥೬೨॥

ಶುಭಾಂಗಃ ಶಾಂತಿದಃ ಸ್ರಷ್ಟಾ ಕುಮುದಃ ಕುವಲೇಶಯಃ ।\\
ಗೋಹಿತೋ ಗೋಪತಿರ್ಗೋಪ್ತಾ ವೃಷಭಾಕ್ಷೋ ವೃಷಪ್ರಿಯಃ ॥೬೩॥

ಅನಿವರ್ತೀ ನಿವೃತ್ತಾತ್ಮಾ ಸಂಕ್ಷೇಪ್ತಾ ಕ್ಷೇಮಕೃಚ್ಛಿವಃ ।\\
ಶ್ರೀವತ್ಸವಕ್ಷಾಃ ಶ್ರೀವಾಸಃ ಶ್ರೀಪತಿಃ ಶ್ರೀಮತಾಂವರಃ ॥೬೪॥

ಶ್ರೀದಃ ಶ್ರೀಶಃ ಶ್ರೀನಿವಾಸಃ ಶ್ರೀನಿಧಿಃ ಶ್ರೀವಿಭಾವನಃ ।\\
ಶ್ರೀಧರಃ ಶ್ರೀಕರಃ ಶ್ರೇಯಃ ಶ್ರೀಮಾಁಲ್ಲೋಕತ್ರಯಾಶ್ರಯಃ ॥೬೫॥

ಸ್ವಕ್ಷಃ ಸ್ವಂಗಃ ಶತಾನಂದೋ ನಂದಿರ್ಜ್ಯೋತಿರ್ಗಣೇಶ್ವರಃ ।\\
ವಿಜಿತಾತ್ಮಾಽವಿಧೇಯಾತ್ಮಾ ಸತ್ಕೀರ್ತಿಶ್ಛಿನ್ನಸಂಶಯಃ ॥೬೬॥

ಉದೀರ್ಣಃ ಸರ್ವತಶ್ಚಕ್ಷುರನೀಶಃ ಶಾಶ್ವತಸ್ಥಿರಃ ।\\
ಭೂಶಯೋ ಭೂಷಣೋ ಭೂತಿರ್ವಿಶೋಕಃ ಶೋಕನಾಶನಃ ॥೬೭॥

ಅರ್ಚಿಷ್ಮಾನರ್ಚಿತಃ ಕುಂಭೋ ವಿಶುದ್ಧಾತ್ಮಾ ವಿಶೋಧನಃ ।\\
ಅನಿರುದ್ಧೋಽಪ್ರತಿರಥಃ ಪ್ರದ್ಯುಮ್ನೋಽಮಿತವಿಕ್ರಮಃ ॥೬೮॥

ಕಾಲನೇಮಿನಿಹಾ ವೀರಃ ಶೌರಿಃ ಶೂರಜನೇಶ್ವರಃ ।\\
ತ್ರಿಲೋಕಾತ್ಮಾ ತ್ರಿಲೋಕೇಶಃ ಕೇಶವಃ ಕೇಶಿಹಾ ಹರಿಃ ॥೬೯॥

ಕಾಮದೇವಃ ಕಾಮಪಾಲಃ ಕಾಮೀ ಕಾಂತಃ ಕೃತಾಗಮಃ ।\\
ಅನಿರ್ದೇಶ್ಯವಪುರ್ವಿಷ್ಣುರ್ವೀರೋಽನಂತೋ ಧನಂಜಯಃ ॥೭೦॥

ಬ್ರಹ್ಮಣ್ಯೋ ಬ್ರಹ್ಮಕೃದ್ ಬ್ರಹ್ಮಾ ಬ್ರಹ್ಮ ಬ್ರಹ್ಮವಿವರ್ಧನಃ ।\\
ಬ್ರಹ್ಮವಿದ್ ಬ್ರಾಹ್ಮಣೋ ಬ್ರಹ್ಮೀ ಬ್ರಹ್ಮಜ್ಞೋ ಬ್ರಾಹ್ಮಣಪ್ರಿಯಃ ॥೭೧॥

ಮಹಾಕ್ರಮೋ ಮಹಾಕರ್ಮಾ ಮಹಾತೇಜಾ ಮಹೋರಗಃ ।\\
ಮಹಾಕ್ರತುರ್ಮಹಾಯಜ್ವಾ ಮಹಾಯಜ್ಞೋ ಮಹಾಹವಿಃ ॥೭೨॥

ಸ್ತವ್ಯಃ ಸ್ತವಪ್ರಿಯಃ ಸ್ತೋತ್ರಂ ಸ್ತುತಿಃ ಸ್ತೋತಾ ರಣಪ್ರಿಯಃ ।\\
ಪೂರ್ಣಃ ಪೂರಯಿತಾ ಪುಣ್ಯಃ ಪುಣ್ಯಕೀರ್ತಿರನಾಮಯಃ ॥೭೩॥

ಮನೋಜವಸ್ತೀರ್ಥಕರೋ ವಸುರೇತಾ ವಸುಪ್ರದಃ ।\\
ವಸುಪ್ರದೋ ವಾಸುದೇವೋ ವಸುರ್ವಸುಮನಾ ಹವಿಃ ॥೭೪॥

ಸದ್ಗತಿಃ ಸತ್ಕೃತಿಃ ಸತ್ತಾ ಸದ್ಭೂತಿಃ ಸತ್ಪರಾಯಣಃ ।\\
ಶೂರಸೇನೋ ಯದುಶ್ರೇಷ್ಠಃ ಸನ್ನಿವಾಸಃ ಸುಯಾಮುನಃ ॥೭೫॥

ಭೂತಾವಾಸೋ ವಾಸುದೇವಃ ಸರ್ವಾಸುನಿಲಯೋಽನಲಃ ।\\
ದರ್ಪಹಾ ದರ್ಪದೋ ದೃಪ್ತೋ ದುರ್ಧರೋಽಥಾಪರಾಜಿತಃ ॥೭೬॥

ವಿಶ್ವಮೂರ್ತಿರ್ಮಹಾಮೂರ್ತಿರ್ದೀಪ್ತಮೂರ್ತಿರಮೂರ್ತಿಮಾನ್ ।\\
ಅನೇಕಮೂರ್ತಿರವ್ಯಕ್ತಃ ಶತಮೂರ್ತಿಃ ಶತಾನನಃ ॥೭೭॥

ಏಕೋ ನೈಕಃ ಸವಃ ಕಃ ಕಿಂ ಯತ್ ತತ್ಪದಮನುತ್ತಮಂ ।\\
ಲೋಕಬಂಧುರ್ಲೋಕನಾಥೋ ಮಾಧವೋ ಭಕ್ತವತ್ಸಲಃ ॥೭೮॥

ಸುವರ್ಣವರ್ಣೋ ಹೇಮಾಂಗೋ ವರಾಂಗಶ್ಚಂದನಾಂಗದೀ ।\\
ವೀರಹಾ ವಿಷಮಃ ಶೂನ್ಯೋ ಘೃತಾಶೀರಚಲಶ್ಚಲಃ ॥೭೯॥

ಅಮಾನೀ ಮಾನದೋ ಮಾನ್ಯೋ ಲೋಕಸ್ವಾಮೀ ತ್ರಿಲೋಕಧೃಕ್ ।\\
ಸುಮೇಧಾ ಮೇಧಜೋ ಧನ್ಯಃ ಸತ್ಯಮೇಧಾ ಧರಾಧರಃ ॥೮೦॥

ತೇಜೋವೃಷೋ ದ್ಯುತಿಧರಃ ಸರ್ವಶಸ್ತ್ರಭೃತಾಂ ವರಃ ।\\
ಪ್ರಗ್ರಹೋ ನಿಗ್ರಹೋ ವ್ಯಗ್ರೋ ನೈಕಶೃಂಗೋ ಗದಾಗ್ರಜಃ ॥೮೧॥

ಚತುರ್ಮೂರ್ತಿಶ್ಚತುರ್ಬಾಹುಶ್ಚತುರ್ವ್ಯೂಹಶ್ಚತುರ್ಗತಿಃ ।\\
ಚತುರಾತ್ಮಾ ಚತುರ್ಭಾವಶ್ಚತುರ್ವೇದವಿದೇಕಪಾತ್ ॥೮೨॥

ಸಮಾವರ್ತೋಽನಿವೃತ್ತಾತ್ಮಾ ದುರ್ಜಯೋ ದುರತಿಕ್ರಮಃ ।\\
ದುರ್ಲಭೋ ದುರ್ಗಮೋ ದುರ್ಗೋ ದುರಾವಾಸೋ ದುರಾರಿಹಾ ॥೮೩॥

ಶುಭಾಂಗೋ ಲೋಕಸಾರಂಗಃ ಸುತಂತುಸ್ತಂತುವರ್ಧನಃ ।\\
ಇಂದ್ರಕರ್ಮಾ ಮಹಾಕರ್ಮಾ ಕೃತಕರ್ಮಾ ಕೃತಾಗಮಃ ॥೮೪॥

ಉದ್ಭವಃ ಸುಂದರಃ ಸುಂದೋ ರತ್ನನಾಭಃ ಸುಲೋಚನಃ ।\\
ಅರ್ಕೋ ವಾಜಸನಃ ಶೃಂಗೀ ಜಯಂತಃ ಸರ್ವವಿಜ್ಜಯೀ ॥೮೫॥

ಸುವರ್ಣಬಿಂದುರಕ್ಷೋಭ್ಯಃ ಸರ್ವವಾಗೀಶ್ವರೇಶ್ವರಃ ।\\
ಮಹಾಹ್ರದೋ ಮಹಾಗರ್ತೋ ಮಹಾಭೂತೋ ಮಹಾನಿಧಿಃ ॥೮೬॥

ಕುಮುದಃ ಕುಂದರಃ ಕುಂದಃ ಪರ್ಜನ್ಯಃ ಪಾವನೋಽನಿಲಃ ।\\
ಅಮೃತಾಶೋಽಮೃತವಪುಃ ಸರ್ವಜ್ಞಃ ಸರ್ವತೋಮುಖಃ ॥೮೭॥

ಸುಲಭಃ ಸುವ್ರತಃ ಸಿದ್ಧಃ ಶತ್ರುಜಿಚ್ಛತ್ರುತಾಪನಃ ।\\
ನ್ಯಗ್ರೋಧೋಽದುಂಬರೋಽಶ್ವತ್ಥಶ್ಚಾಣೂರಾಂಧ್ರನಿಷೂದನಃ ॥೮೮॥

ಸಹಸ್ರಾರ್ಚಿಃ ಸಪ್ತಜಿಹ್ವಃ ಸಪ್ತೈಧಾಃ ಸಪ್ತವಾಹನಃ ।\\
ಅಮೂರ್ತಿರನಘೋಽಚಿಂತ್ಯೋ ಭಯಕೃದ್ಭಯನಾಶನಃ ॥೮೯॥

ಅಣುರ್ಬೃಹತ್ಕೃಶಃ ಸ್ಥೂಲೋ ಗುಣಭೃನ್ನಿರ್ಗುಣೋ ಮಹಾನ್ ।\\
ಅಧೃತಃ ಸ್ವಧೃತಃ ಸ್ವಾಸ್ಯಃ ಪ್ರಾಗ್ವಂಶೋ ವಂಶವರ್ಧನಃ ॥೯೦॥

ಭಾರಭೃತ್ ಕಥಿತೋ ಯೋಗೀ ಯೋಗೀಶಃ ಸರ್ವಕಾಮದಃ ।\\
ಆಶ್ರಮಃ ಶ್ರಮಣಃ ಕ್ಷಾಮಃ ಸುಪರ್ಣೋ ವಾಯುವಾಹನಃ ॥೯೧॥

ಧನುರ್ಧರೋ ಧನುರ್ವೇದೋ ದಂಡೋ ದಮಯಿತಾ ದಮಃ ।\\
ಅಪರಾಜಿತಃ ಸರ್ವಸಹೋ ನಿಯಂತಾಽನಿಯಮೋಽಯಮಃ ॥೯೨॥

ಸತ್ತ್ವವಾನ್ ಸಾತ್ತ್ವಿಕಃ ಸತ್ಯಃ ಸತ್ಯಧರ್ಮಪರಾಯಣಃ ।\\
ಅಭಿಪ್ರಾಯಃ ಪ್ರಿಯಾರ್ಹೋಽರ್ಹಃ ಪ್ರಿಯಕೃತ್ ಪ್ರೀತಿವರ್ಧನಃ ॥೯೩॥

ವಿಹಾಯಸಗತಿರ್ಜ್ಯೋತಿಃ ಸುರುಚಿರ್ಹುತಭುಗ್ವಿಭುಃ ।\\
ರವಿರ್ವಿರೋಚನಃ ಸೂರ್ಯಃ ಸವಿತಾ ರವಿಲೋಚನಃ ॥೯೪॥

ಅನಂತೋ ಹುತಭುಗ್ಭೋಕ್ತಾ ಸುಖದೋ ನೈಕಜೋಽಗ್ರಜಃ ।\\
ಅನಿರ್ವಿಣ್ಣಃ ಸದಾಮರ್ಷೀ ಲೋಕಾಧಿಷ್ಠಾನಮದ್ಭುತಃ ॥೯೫॥

ಸನಾತ್ಸನಾತನತಮಃ ಕಪಿಲಃ ಕಪಿರವ್ಯಯಃ ।\\
ಸ್ವಸ್ತಿದಃ ಸ್ವಸ್ತಿಕೃತ್ಸ್ವಸ್ತಿ ಸ್ವಸ್ತಿಭುಕ್ಸ್ವಸ್ತಿದಕ್ಷಿಣಃ ॥೯೬॥

ಅರೌದ್ರಃ ಕುಂಡಲೀ ಚಕ್ರೀ ವಿಕ್ರಮ್ಯೂರ್ಜಿತಶಾಸನಃ ।\\
ಶಬ್ದಾತಿಗಃ ಶಬ್ದಸಹಃ ಶಿಶಿರಃ ಶರ್ವರೀಕರಃ ॥೯೭॥

ಅಕ್ರೂರಃ ಪೇಶಲೋ ದಕ್ಷೋ ದಕ್ಷಿಣಃ ಕ್ಷಮಿಣಾಂವರಃ ।\\
ವಿದ್ವತ್ತಮೋ ವೀತಭಯಃ ಪುಣ್ಯಶ್ರವಣಕೀರ್ತನಃ ॥೯೮॥

ಉತ್ತಾರಣೋ ದುಷ್ಕೃತಿಹಾ ಪುಣ್ಯೋ ದುಃಸ್ವಪ್ನನಾಶನಃ ।\\
ವೀರಹಾ ರಕ್ಷಣಃ ಸಂತೋ ಜೀವನಃ ಪರ್ಯವಸ್ಥಿತಃ ॥೯೯॥

ಅನಂತರೂಪೋಽನಂತಶ್ರೀರ್ಜಿತಮನ್ಯುರ್ಭಯಾಪಹಃ ।\\
ಚತುರಶ್ರೋ ಗಭೀರಾತ್ಮಾ ವಿದಿಶೋ ವ್ಯಾದಿಶೋ ದಿಶಃ ॥೧೦೦॥

ಅನಾದಿರ್ಭೂರ್ಭುವೋ ಲಕ್ಷ್ಮೀಃ ಸುವೀರೋ ರುಚಿರಾಂಗದಃ ।\\
ಜನನೋ ಜನಜನ್ಮಾದಿರ್ಭೀಮೋ ಭೀಮಪರಾಕ್ರಮಃ ॥೧೦೧॥

ಆಧಾರನಿಲಯೋಽಧಾತಾ ಪುಷ್ಪಹಾಸಃ ಪ್ರಜಾಗರಃ ।\\
ಊರ್ಧ್ವಗಃ ಸತ್ಪಥಾಚಾರಃ ಪ್ರಾಣದಃ ಪ್ರಣವಃ ಪಣಃ ॥೧೦೨॥

ಪ್ರಮಾಣಂ ಪ್ರಾಣನಿಲಯಃ ಪ್ರಾಣಭೃತ್ಪ್ರಾಣಜೀವನಃ ।\\
ತತ್ತ್ವಂ ತತ್ತ್ವವಿದೇಕಾತ್ಮಾ ಜನ್ಮಮೃತ್ಯುಜರಾತಿಗಃ ॥೧೦೩॥

ಭೂರ್ಭುವಃಸ್ವಸ್ತರುಸ್ತಾರಃ ಸವಿತಾ ಪ್ರಪಿತಾಮಹಃ ।\\
ಯಜ್ಞೋ ಯಜ್ಞಪತಿರ್ಯಜ್ವಾ ಯಜ್ಞಾಂಗೋ ಯಜ್ಞವಾಹನಃ ॥೧೦೪॥

ಯಜ್ಞಭೃದ್ ಯಜ್ಞಕೃದ್ ಯಜ್ಞೀ ಯಜ್ಞಭುಗ್ ಯಜ್ಞಸಾಧನಃ ।\\
ಯಜ್ಞಾಂತಕೃದ್ ಯಜ್ಞಗುಹ್ಯಮನ್ನಮನ್ನಾದ ಏವ ಚ ॥೧೦೫॥

ಆತ್ಮಯೋನಿಃ ಸ್ವಯಂಜಾತೋ ವೈಖಾನಃ ಸಾಮಗಾಯನಃ ।\\
ದೇವಕೀನಂದನಃ ಸ್ರಷ್ಟಾ ಕ್ಷಿತೀಶಃ ಪಾಪನಾಶನಃ ॥೧೦೬॥

ಶಂಖಭೃನ್ನಂದಕೀ ಚಕ್ರೀ ಶಾರ್ಙ್ಗಧನ್ವಾ ಗದಾಧರಃ ।\\
ರಥಾಂಗಪಾಣಿರಕ್ಷೋಭ್ಯಃ ಸರ್ವಪ್ರಹರಣಾಯುಧಃ ॥೧೦೭॥

ಸರ್ವಪ್ರಹರಣಾಯುಧ ಓಂ ನಮ ಇತಿ ।\\
ವನಮಾಲೀ ಗದೀ ಶಾರ್ಙ್ಗೀ ಶಂಖೀ ಚಕ್ರೀ ಚ ನಂದಕೀ ।\\
ಶ್ರೀಮಾನ್ ನಾರಾಯಣೋ ವಿಷ್ಣುರ್ವಾಸುದೇವೋಽಭಿರಕ್ಷತು ॥೧೦೮॥

ಶ್ರೀ ವಾಸುದೇವೋಽಭಿರಕ್ಷತು ಓಂ ನಮ ಇತಿ ।\\
ಉತ್ತರನ್ಯಾಸಃ ।\\
ಭೀಷ್ಮ ಉವಾಚ॥
ಇತೀದಂ ಕೀರ್ತನೀಯಸ್ಯ ಕೇಶವಸ್ಯ ಮಹಾತ್ಮನಃ ।\\
ನಾಮ್ನಾಂ ಸಹಸ್ರಂ ದಿವ್ಯಾನಾಮಶೇಷೇಣ ಪ್ರಕೀರ್ತಿತಂ ॥೧॥

ಯ ಇದಂ ಶೃಣುಯಾನ್ನಿತ್ಯಂ ಯಶ್ಚಾಪಿ ಪರಿಕೀರ್ತಯೇತ್ ।\\
ನಾಶುಭಂ ಪ್ರಾಪ್ನುಯಾತ್ಕಿಂಚಿತ್ಸೋಽಮುತ್ರೇಹ ಚ ಮಾನವಃ ॥೨॥

ವೇದಾಂತಗೋ ಬ್ರಾಹ್ಮಣಃ ಸ್ಯಾತ್ಕ್ಷತ್ರಿಯೋ ವಿಜಯೀ ಭವೇತ್ ।\\
ವೈಶ್ಯೋ ಧನಸಮೃದ್ಧಃ ಸ್ಯಾಚ್ಛೂದ್ರಃ ಸುಖಮವಾಪ್ನುಯಾತ್ ॥೩॥

ಧರ್ಮಾರ್ಥೀ ಪ್ರಾಪ್ನುಯಾದ್ಧರ್ಮಮರ್ಥಾರ್ಥೀ ಚಾರ್ಥಮಾಪ್ನುಯಾತ್ ।\\
ಕಾಮಾನವಾಪ್ನುಯಾತ್ಕಾಮೀ ಪ್ರಜಾರ್ಥೀ ಪ್ರಾಪ್ನುಯಾತ್ಪ್ರಜಾಂ ॥೪॥

ಭಕ್ತಿಮಾನ್ ಯಃ ಸದೋತ್ಥಾಯ ಶುಚಿಸ್ತದ್ಗತಮಾನಸಃ ।\\
ಸಹಸ್ರಂ ವಾಸುದೇವಸ್ಯ ನಾಮ್ನಾಮೇತತ್ಪ್ರಕೀರ್ತಯೇತ್ ॥೫॥

ಯಶಃ ಪ್ರಾಪ್ನೋತಿ ವಿಪುಲಂ ಜ್ಞಾತಿಪ್ರಾಧಾನ್ಯಮೇವ ಚ ।\\
ಅಚಲಾಂ ಶ್ರಿಯಮಾಪ್ನೋತಿ ಶ್ರೇಯಃ ಪ್ರಾಪ್ನೋತ್ಯನುತ್ತಮಂ ॥೬॥

ನ ಭಯಂ ಕ್ವಚಿದಾಪ್ನೋತಿ ವೀರ್ಯಂ ತೇಜಶ್ಚ ವಿಂದತಿ ।\\
ಭವತ್ಯರೋಗೋ ದ್ಯುತಿಮಾನ್ಬಲರೂಪಗುಣಾನ್ವಿತಃ ॥೭॥

ರೋಗಾರ್ತೋ ಮುಚ್ಯತೇ ರೋಗಾದ್ಬದ್ಧೋ ಮುಚ್ಯೇತ ಬಂಧನಾತ್ ।\\
ಭಯಾನ್ಮುಚ್ಯೇತ ಭೀತಸ್ತು ಮುಚ್ಯೇತಾಪನ್ನ ಆಪದಃ ॥೮॥

ದುರ್ಗಾಣ್ಯತಿತರತ್ಯಾಶು ಪುರುಷಃ ಪುರುಷೋತ್ತಮಂ ।\\
ಸ್ತುವನ್ನಾಮಸಹಸ್ರೇಣ ನಿತ್ಯಂ ಭಕ್ತಿಸಮನ್ವಿತಃ ॥೯॥

ವಾಸುದೇವಾಶ್ರಯೋ ಮರ್ತ್ಯೋ ವಾಸುದೇವಪರಾಯಣಃ ।\\
ಸರ್ವಪಾಪವಿಶುದ್ಧಾತ್ಮಾ ಯಾತಿ ಬ್ರಹ್ಮ ಸನಾತನಂ ॥೧೦॥

ನ ವಾಸುದೇವಭಕ್ತಾನಾಮಶುಭಂ ವಿದ್ಯತೇ ಕ್ವಚಿತ್ ।\\
ಜನ್ಮಮೃತ್ಯುಜರಾವ್ಯಾಧಿಭಯಂ ನೈವೋಪಜಾಯತೇ ॥೧೧॥

ಇಮಂ ಸ್ತವಮಧೀಯಾನಃ ಶ್ರದ್ಧಾಭಕ್ತಿಸಮನ್ವಿತಃ ।\\
ಯುಜ್ಯೇತಾತ್ಮಸುಖಕ್ಷಾಂತಿಶ್ರೀಧೃತಿಸ್ಮೃತಿಕೀರ್ತಿಭಿಃ ॥೧೨॥

ನ ಕ್ರೋಧೋ ನ ಚ ಮಾತ್ಸರ್ಯಂ ನ ಲೋಭೋ ನಾಶುಭಾ ಮತಿಃ ।\\
ಭವಂತಿ ಕೃತ ಪುಣ್ಯಾನಾಂ ಭಕ್ತಾನಾಂ ಪುರುಷೋತ್ತಮೇ ॥೧೩॥

ದ್ಯೌಃ ಸಚಂದ್ರಾರ್ಕನಕ್ಷತ್ರಾ ಖಂ ದಿಶೋ ಭೂರ್ಮಹೋದಧಿಃ ।\\
ವಾಸುದೇವಸ್ಯ ವೀರ್ಯೇಣ ವಿಧೃತಾನಿ ಮಹಾತ್ಮನಃ ॥೧೪॥

ಸಸುರಾಸುರಗಂಧರ್ವಂ ಸಯಕ್ಷೋರಗರಾಕ್ಷಸಂ ।\\
ಜಗದ್ವಶೇ ವರ್ತತೇದಂ ಕೃಷ್ಣಸ್ಯ ಸಚರಾಚರಂ ॥೧೫॥

ಇಂದ್ರಿಯಾಣಿ ಮನೋ ಬುದ್ಧಿಃ ಸತ್ತ್ವಂ ತೇಜೋ ಬಲಂ ಧೃತಿಃ ।\\
ವಾಸುದೇವಾತ್ಮಕಾನ್ಯಾಹುಃ ಕ್ಷೇತ್ರಂ ಕ್ಷೇತ್ರಜ್ಞ ಏವ ಚ ॥೧೬॥

ಸರ್ವಾಗಮಾನಾಮಾಚಾರಃ ಪ್ರಥಮಂ ಪರಿಕಲ್ಪ್ಯತೇ ।\\
ಆಚಾರಪ್ರಭವೋ ಧರ್ಮೋ ಧರ್ಮಸ್ಯ ಪ್ರಭುರಚ್ಯುತಃ ॥೧೭॥

ಋಷಯಃ ಪಿತರೋ ದೇವಾ ಮಹಾಭೂತಾನಿ ಧಾತವಃ ।\\
ಜಂಗಮಾಜಂಗಮಂ ಚೇದಂ ಜಗನ್ನಾರಾಯಣೋದ್ಭವಂ ॥೧೮॥

ಯೋಗೋ ಜ್ಞಾನಂ ತಥಾ ಸಾಂಖ್ಯಂ ವಿದ್ಯಾಃ ಶಿಲ್ಪಾದಿ ಕರ್ಮ ಚ ।\\
ವೇದಾಃ ಶಾಸ್ತ್ರಾಣಿ ವಿಜ್ಞಾನಮೇತತ್ಸರ್ವಂ ಜನಾರ್ದನಾತ್ ॥೧೯॥

ಏಕೋ ವಿಷ್ಣುರ್ಮಹದ್ಭೂತಂ ಪೃಥಗ್ಭೂತಾನ್ಯನೇಕಶಃ ।\\
ತ್ರೀಂಲ್ಲೋಕಾನ್ವ್ಯಾಪ್ಯ ಭೂತಾತ್ಮಾ ಭುಂಕ್ತೇ ವಿಶ್ವಭುಗವ್ಯಯಃ ॥೨೦॥

ಇಮಂ ಸ್ತವಂ ಭಗವತೋ ವಿಷ್ಣೋರ್ವ್ಯಾಸೇನ ಕೀರ್ತಿತಂ ।\\
ಪಠೇದ್ಯ ಇಚ್ಛೇತ್ಪುರುಷಃ ಶ್ರೇಯಃ ಪ್ರಾಪ್ತುಂ ಸುಖಾನಿ ಚ ॥೨೧॥

ವಿಶ್ವೇಶ್ವರಮಜಂ ದೇವಂ ಜಗತಃ ಪ್ರಭುಮವ್ಯಯಂ ।\\
ಭಜಂತಿ ಯೇ ಪುಷ್ಕರಾಕ್ಷಂ ನ ತೇ ಯಾಂತಿ ಪರಾಭವಂ ॥೨೨॥

ನ ತೇ ಯಾಂತಿ ಪರಾಭವಂ ಓಂ ನಮ ಇತಿ ।\\
ಅರ್ಜುನ ಉವಾಚ॥
ಪದ್ಮಪತ್ರವಿಶಾಲಾಕ್ಷ ಪದ್ಮನಾಭ ಸುರೋತ್ತಮ ।\\
ಭಕ್ತಾನಾಮನುರಕ್ತಾನಾಂ ತ್ರಾತಾ ಭವ ಜನಾರ್ದನ ॥೨೩॥

ಶ್ರೀಭಗವಾನುವಾಚ॥
ಯೋ ಮಾಂ ನಾಮಸಹಸ್ರೇಣ ಸ್ತೋತುಮಿಚ್ಛತಿ ಪಾಂಡವ ।\\
ಸೋಹಽಮೇಕೇನ ಶ್ಲೋಕೇನ ಸ್ತುತ ಏವ ನ ಸಂಶಯಃ ॥೨೪॥

ಸ್ತುತ ಏವ ನ ಸಂಶಯ ಓಂ ನಮ ಇತಿ ।\\
ವ್ಯಾಸ ಉವಾಚ॥
ವಾಸನಾದ್ವಾಸುದೇವಸ್ಯ ವಾಸಿತಂ ಭುವನತ್ರಯಂ ।\\
ಸರ್ವಭೂತನಿವಾಸೋಽಸಿ ವಾಸುದೇವ ನಮೋಽಸ್ತು ತೇ ॥೨೫॥

ಶ್ರೀ ವಾಸುದೇವ ನಮೋಽಸ್ತುತ ಓಂ ನಮ ಇತಿ ।\\
ಪಾರ್ವತ್ಯುವಾಚ॥
ಕೇನೋಪಾಯೇನ ಲಘುನಾ ವಿಷ್ಣೋರ್ನಾಮಸಹಸ್ರಕಂ ।\\
ಪಠ್ಯತೇ ಪಂಡಿತೈರ್ನಿತ್ಯಂ ಶ್ರೋತುಮಿಚ್ಛಾಮ್ಯಹಂ ಪ್ರಭೋ ॥೨೬॥

ಈಶ್ವರ ಉವಾಚ॥\\
ಶ್ರೀರಾಮ ರಾಮ ರಾಮೇತಿ ರಮೇ ರಾಮೇ ಮನೋರಮೇ ।\\
ಸಹಸ್ರನಾಮ ತತ್ತುಲ್ಯಂ ರಾಮ ನಾಮ ವರಾನನೇ ॥೨೭॥

ಶ್ರೀರಾಮನಾಮ ವರಾನನ ಓಂ ನಮ ಇತಿ ।\\
ಬ್ರಹ್ಮೋವಾಚ॥\\
ನಮೋಽಸ್ತ್ವನಂತಾಯ ಸಹಸ್ರಮೂರ್ತಯ\\
ಸಹಸ್ರಪಾದಾಕ್ಷಿಶಿರೋರುಬಾಹವೇ ।\\
ಸಹಸ್ರನಾಮ್ನೇ ಪುರುಷಾಯ ಶಾಶ್ವತ\\
ಸಹಸ್ರಕೋಟಿಯುಗಧಾರಿಣೇ ನಮಃ ॥೨೮॥

ಸಹಸ್ರಕೋಟಿಯುಗಧಾರಿಣೇ ಓಂ ನಮ ಇತಿ ।\\

ಓಂ ತತ್ಸದಿತಿ ಶ್ರೀಮಹಾಭಾರತೇ ಶತಸಾಹಸ್ರ್ಯಾಂ ಸಂಹಿತಾಯಾಂ ವೈಯಾಸಿಕ್ಯಾಮಾನುಶಾಸನಿಕೇ ಪರ್ವಣಿ ಭೀಷ್ಮಯುಧಿಷ್ಠಿರಸಂವಾದೇ ಶ್ರೀವಿಷ್ಣೋರ್ದಿವ್ಯಸಹಸ್ರನಾಮಸ್ತೋತ್ರಂ॥

ಸಂಜಯ ಉವಾಚ॥\\
ಯತ್ರ ಯೋಗೇಶ್ವರಃ ಕೃಷ್ಣೋ ಯತ್ರ ಪಾರ್ಥೋ ಧನುರ್ಧರಃ ।\\
ತತ್ರ ಶ್ರೀರ್ವಿಜಯೋ ಭೂತಿರ್ಧ್ರುವಾ ನೀತಿರ್ಮತಿರ್ಮಮ ॥೨೯॥

ಶ್ರೀಭಗವಾನುವಾಚ॥\\
ಅನನ್ಯಾಶ್ಚಿಂತಯಂತೋ ಮಾಂ ಯೇ ಜನಾಃ ಪರ್ಯುಪಾಸತೇ ।\\
ತೇಷಾಂ ನಿತ್ಯಾಭಿಯುಕ್ತಾನಾಂ ಯೋಗಕ್ಷೇಮಂ ವಹಾಮ್ಯಹಂ ॥೩೦॥

ಪರಿತ್ರಾಣಾಯ ಸಾಧೂನಾಂ ವಿನಾಶಾಯ ಚ ದುಷ್ಕೃತಾಂ ।\\
ಧರ್ಮಸಂಸ್ಥಾಪನಾರ್ಥಾಯ ಸಂಭವಾಮಿ ಯುಗೇ ಯುಗೇ ॥೩೧॥

ಆರ್ತಾಃ ವಿಷಣ್ಣಾಃ ಶಿಥಿಲಾಶ್ಚ ಭೀತಾಃ ಘೋರೇಷು ಚ ವ್ಯಾಧಿಷು ವರ್ತಮಾನಾಃ ।\\
ಸಂಕೀರ್ತ್ಯ ನಾರಾಯಣಶಬ್ದಮಾತ್ರಂ ವಿಮುಕ್ತದುಃಖಾಃ ಸುಖಿನೋ ಭವಂತಿ ॥೩೨॥

ಕಾಯೇನ ವಾಚಾ ಮನಸೇಂದ್ರಿಯೈರ್ವಾ ಬುದ್ಧ್ಯಾತ್ಮನಾ ವಾ ಪ್ರಕೃತೇಃ ಸ್ವಭಾವಾತ್ ।\\
ಕರೋಮಿ ಯದ್ಯತ್ ಸಕಲಂ ಪರಸ್ಮೈ ನಾರಾಯಣಾಯೇತಿ ಸಮರ್ಪಯಾಮಿ ॥೩೩॥

ಇತಿ ಶ್ರೀವಿಷ್ಣೋರ್ದಿವ್ಯಸಹಸ್ರನಾಮಸ್ತೋತ್ರಂ ಸಂಪೂರ್ಣಂ ।\\
ಓಂ ತತ್ ಸತ್ ।\\

ಓಂ ಆಪದಾಮಪಹರ್ತಾರಂ ದಾತಾರಂ ಸರ್ವಸಂಪದಾಂ ।\\
ಲೋಕಾಭಿರಾಮಂ ಶ್ರೀರಾಮಂ ಭೂಯೋ ಭೂಯೋ ನಮಾಮ್ಯಹಂ॥

ಆರ್ತಾನಾಮಾರ್ತಿಹಂತಾರಂ ಭೀತಾನಾಂ ಭೀತಿನಾಶನಂ ।\\
ದ್ವಿಷತಾಂ ಕಾಲದಂಡಂ ತಂ ರಾಮಚಂದ್ರಂ ನಮಾಮ್ಯಹಂ॥

ನಮಃ ಕೋದಂಡಹಸ್ತಾಯ ಸಂಧೀಕೃತಶರಾಯ ಚ ।\\
ಖಂಡಿತಾಖಿಲದೈತ್ಯಾಯ ರಾಮಾಯಽಽಪನ್ನಿವಾರಿಣೇ॥

ರಾಮಾಯ ರಾಮಭದ್ರಾಯ ರಾಮಚಂದ್ರಾಯ ವೇಧಸೇ ।\\
ರಘುನಾಥಾಯ ನಾಥಾಯ ಸೀತಾಯಾಃ ಪತಯೇ ನಮಃ॥

ಅಗ್ರತಃ ಪೃಷ್ಠತಶ್ಚೈವ ಪಾರ್ಶ್ವತಶ್ಚ ಮಹಾಬಲೌ ।\\
ಆಕರ್ಣಪೂರ್ಣಧನ್ವಾನೌ ರಕ್ಷೇತಾಂ ರಾಮಲಕ್ಷ್ಮಣೌ॥

ಸನ್ನದ್ಧಃ ಕವಚೀ ಖಡ್ಗೀ ಚಾಪಬಾಣಧರೋ ಯುವಾ ।\\
ಗಚ್ಛನ್ ಮಮಾಗ್ರತೋ ನಿತ್ಯಂ ರಾಮಃ ಪಾತು ಸಲಕ್ಷ್ಮಣಃ॥

ಅಚ್ಯುತಾನಂತಗೋವಿಂದ ನಾಮೋಚ್ಚಾರಣಭೇಷಜಾತ್ ।\\
ನಶ್ಯಂತಿ ಸಕಲಾ ರೋಗಾಸ್ಸತ್ಯಂ ಸತ್ಯಂ ವದಾಮ್ಯಹಂ॥

ಸತ್ಯಂ ಸತ್ಯಂ ಪುನಸ್ಸತ್ಯಮುದ್ಧೃತ್ಯ ಭುಜಮುಚ್ಯತೇ ।\\
ವೇದಾಚ್ಛಾಸ್ತ್ರಂ ಪರಂ ನಾಸ್ತಿ ನ ದೇವಂ ಕೇಶವಾತ್ಪರಂ॥

ಶರೀರೇ ಜರ್ಝರೀಭೂತೇ ವ್ಯಾಧಿಗ್ರಸ್ತೇ ಕಳೇವರೇ ।\\
ಔಷಧಂ ಜಾಹ್ನವೀತೋಯಂ ವೈದ್ಯೋ ನಾರಾಯಣೋ ಹರಿಃ॥

ಆಲೋಡ್ಯ ಸರ್ವಶಾಸ್ತ್ರಾಣಿ ವಿಚಾರ್ಯ ಚ ಪುನಃ ಪುನಃ ।\\
ಇದಮೇಕಂ ಸುನಿಷ್ಪನ್ನಂ ಧ್ಯೇಯೋ ನಾರಾಯಣೋ ಹರಿಃ॥

ಯದಕ್ಷರಪದಭ್ರಷ್ಟಂ ಮಾತ್ರಾಹೀನಂ ತು ಯದ್ಭವೇತ್ ।\\
ತತ್ಸರ್ವಂ ಕ್ಷಮ್ಯತಾಂ ದೇವ ನಾರಾಯಣ ನಮೋಽಸ್ತು ತೇ॥

ವಿಸರ್ಗಬಿಂದುಮಾತ್ರಾಣಿ ಪದಪಾದಾಕ್ಷರಾಣಿ ಚ ।\\
ನ್ಯೂನಾನಿ ಚಾತಿರಿಕ್ತಾನಿ ಕ್ಷಮಸ್ವ ಪುರುಷೋತ್ತಮ॥

%=============================================================================================
\section{ವಿಷ್ಣೋರಷ್ಟೋತ್ತರಶತನಾಮಸ್ತೋತ್ರಂ}
\addcontentsline{toc}{section}{ವಿಷ್ಣೋರಷ್ಟೋತ್ತರಶತನಾಮಸ್ತೋತ್ರಂ}


ವಿಷ್ಣುರ್ಜಿಷ್ಣುರ್ವಷಟ್ಕಾರೋ ದೇವದೇವೋ ವೃಷಾಕಪಿಃ ।\\
ದಾಮೋದರೋ ದೀನಬಂಧುರಾದಿದೇವೋಽದಿತೇಃ ಸುತಃ ॥೧॥

ಪುಂಡರೀಕಃ ಪರಾನಂದಃ ಪರಮಾತ್ಮಾ ಪರಾತ್ಪರಃ ।\\
ಪರಶುಧಾರೀ ವಿಶ್ವಾತ್ಮಾ ಕೃಷ್ಣಃ ಕಲಿಮಲಾಪಹಃ ॥೨॥

ಕೌಸ್ತುಭೋದ್ಭಾಸಿತೋರಸ್ಕೋ ನರೋ ನಾರಾಯಣೋ ಹರಿಃ ।\\
ಹರೋ ಹರಪ್ರಿಯಃ ಸ್ವಾಮೀ ವೈಕುಂಠೋ ವಿಶ್ವತೋಮುಖಃ ॥೩॥

ಹೃಷೀಕೇಶೋಽಪ್ರಮೇಯಾತ್ಮಾ ವರಾಹೋ ಧರಣೀಧರಃ ।\\
ವಾಮನೋ ವೇದವಕ್ತಾ ಚ ವಾಸುದೇವಃ ಸನಾತನಃ ॥೪॥

ರಾಮೋ ವಿರಾಮೋ ವಿರಜೋ ರಾವಣಾರೀ ರಮಾಪತಿಃ ।\\
ವೈಕುಂಠವಾಸೀ ವಸುಮಾನ್ ಧನದೋ ಧರಣೀಧರಃ ॥೫॥

ಧರ್ಮೇಶೋ ಧರಣೀನಾಥೋ ಧ್ಯೇಯೋ ಧರ್ಮಭೃತಾಂ ವರಃ ।\\
ಸಹಸ್ರಶೀರ್ಷಾ ಪುರುಷಃ ಸಹಸ್ರಾಕ್ಷಃ ಸಹಸ್ರಪಾತ್ ॥೬॥

ಸರ್ವಗಃ ಸರ್ವವಿತ್ಸರ್ವಃ ಶರಣ್ಯಃ ಸಾಧುವಲ್ಲಭಃ ।\\
ಕೌಸಲ್ಯಾನಂದನಃ ಶ್ರೀಮಾನ್ ರಕ್ಷಃಕುಲವಿನಾಶಕಃ ॥೭॥

ಜಗತ್ಕರ್ತಾ ಜಗದ್ಧರ್ತಾ ಜಗಜ್ಜೇತಾ ಜನಾರ್ತಿಹಾ ।\\
ಜಾನಕೀವಲ್ಲಭೋ ದೇವೋ ಜಯರೂಪೋ ಜಲೇಶ್ವರಃ ॥೮॥

ಕ್ಷೀರಾಬ್ಧಿವಾಸೀ ಕ್ಷೀರಾಬ್ಧಿತನಯಾವಲ್ಲಭಸ್ತಥಾ ।\\
ಶೇಷಶಾಯೀ ಪನ್ನಗಾರಿವಾಹನೋ ವಿಷ್ಟರಶ್ರವಾಃ ॥೯॥

ಮಾಧವೋ ಮಧುರಾನಾಥೋ ಮೋಹದೋ ಮೋಹನಾಶನಃ ।\\
ದೈತ್ಯಾರಿಃ ಪುಂಡರೀಕಾಕ್ಷೋ ಹ್ಯಚ್ಯುತೋ ಮಧುಸೂದನಃ ॥೧೦॥

ಸೋಮಸೂರ್ಯಾಗ್ನಿನಯನೋ ನೃಸಿಂಹೋ ಭಕ್ತವತ್ಸಲಃ ।\\
ನಿತ್ಯೋ ನಿರಾಮಯಃ ಶುದ್ಧೋ ನರದೇವೋ ಜಗತ್ಪ್ರಭುಃ ॥೧೧॥

ಹಯಗ್ರೀವೋ ಜಿತರಿಪುರುಪೇಂದ್ರೋ ರುಕ್ಮಣೀಪತಿಃ ।\\
ಸರ್ವದೇವಮಯಃ ಶ್ರೀಶಃ ಸರ್ವಾಧಾರಃ ಸನಾತನಃ ॥೧೨॥

ಸಾಮ್ಯಃ ಸೌಮ್ಯಪ್ರದಃ ಸ್ರಷ್ಟಾ ವಿಷ್ವಕ್ಸೇನೋ ಜನಾರ್ದನಃ ।\\
ಯಶೋದಾತನಯೋ ಯೋಗೀ ಯೋಗಶಾಸ್ತ್ರಪರಾಯಣಃ ॥೧೩॥

ರುದ್ರಾತ್ಮಕೋ ರುದ್ರಮೂರ್ತಿಃ ರಾಘವೋ ಮಧುಸೂದನಃ ।\\
ಇತಿ ತೇ ಕಥಿತಂ ದಿವ್ಯಂ ನಾಮ್ನಾಮಷ್ಟೋತ್ತರಂ ಶತಂ ॥೧೪॥

ಸರ್ವಪಾಪಹರಂ ಪುಣ್ಯಂ ವಿಷ್ಣೋರಮಿತತೇಜಸಃ ।\\
ದುಃಖದಾರಿದ್ರಯದೌರ್ಭಾಗ್ಯನಾಶನಂ ಸುಖವರ್ಧನಂ ॥೧೫॥

ಸರ್ವಸಂಪತ್ಕರಂ ಸೌಮ್ಯಂ ಮಹಾಪಾತಕನಾಶನಂ ।\\
ಪ್ರಾತರುತ್ಥಾಯ ವಿಪ್ರೇಂದ್ರ ಪಠೇದೇಕಾಗ್ರಮಾನಸಃ ।\\
ತಸ್ಯ ನಶ್ಯಂತಿ ವಿಪದಾಂ ರಾಶಯಃ ಸಿದ್ಧಿಮಾಪ್ನುಯಾತ್ ॥೧೬॥

\authorline{ಇತಿ ಶಾಕ್ತಪ್ರಮೋದತಃ ವಿಷ್ಣೋಃ ಅಷ್ಟೋತ್ತರಶತನಾಮಸ್ತೋತ್ರಂ ಸಂಪೂರ್ಣಂ ।}
%===========================================
\section{ಶ್ರೀಶರಭಸಹಸ್ರನಾಮಸ್ತೋತ್ರಂ}
\addcontentsline{toc}{section}{ಶ್ರೀಶರಭಸಹಸ್ರನಾಮಸ್ತೋತ್ರಂ}


ಓಂ ಅಸ್ಯ ಶ್ರೀ ಶರಭ ಮಂತ್ರಸ್ಯ ಕಾಲಾಗ್ನಿ ರುದ್ರ ಋಷಿರ್ಜಗತೀ ಛಂದಃ । ಶ್ರೀ ಶರಭೇಶ್ವರೋ ದೇವತಾ । ಊಂ ಬೀಜಂ । ಸ್ವಾಹಾ ಶಕ್ತಿ । ಫಟ್ ಕೀಲಕಂ । ಶ್ರೀ ಶರಭೇಶ್ವರ ಪ್ರೀತ್ಯರ್ಥೇ ಜಪೇ ವಿನಿಯೋಗಃ ।\\
ಊಂ ಖೇಂ ಖಾಂ ಅಂಗುಷ್ಠಾಭ್ಯಾಂ ನಮಃ ॥\\
ಖಂ ಫಟ್ ತರ್ಜನೀಭ್ಯಾಂ ಸ್ವಾಹಾ ॥\\
ಪ್ರಾಣಗ್ರಹಾಸಿ ಪ್ರಾಣಗ್ರಹಾಸಿ ಹುಂ ಫಟ್ ಮಧ್ಯಮಾಭ್ಯಾಂ ವಷಟ್ ।\\
ಸರ್ವಶತ್ರು ಸಂಹಾರಣಾಯ ಅನಾಮಿಕಾಭ್ಯಾಂ ಹುಂ ॥\\
ಶರಭಸಾಲುವಾಯ ಕನಿಷ್ಠಿಕಾಭ್ಯಾಂ ವೌಷದ್ ॥\\
ಪಕ್ಷಿರಾಜಾಯ ಹುಂ ಫಟ್ ಸ್ವಾಹಾ ಕರತಲಕರ ಪೃಷ್ಠಾಭ್ಯಾಂ ಫಟ್ತ್ರ ॥\\
ಏವಂ ಹೃದಯಾದಿ ನ್ಯಾಸಃ॥

\as{ಚಂದ್ರಾರ್ಕೌವಹ್ನಿದೃಷ್ಟಿಃ ಕುಲಿಶವರನಖಶ್ಚಂಚಂಲೋತ್ಯುಗ್ರಜಿಹ್ವಃ \\
ಕಾಲೀ ದುರ್ಗಾ ಚ ಪಕ್ಷೌ ಹೃದಯಜಠರಗೌಭೈರವೋ ವಾಡವಾಗ್ನಿಃ ॥\\
ಊರುಸ್ಥೌ ವ್ಯಾಧಿಮೃತ್ಯು ಶರಭವರ ಖಗಶ್ಚಂಡ ವಾತಾತಿವೇಗಃ ।\\
ಸಂಹರ್ತಾ ಸರ್ವಶತ್ರೂನ್ ಸ ಜಯತಿ ಶರಭಃ ಶಾಲುವಃ ಪಕ್ಷಿರಾಜಃ ॥}\\

ಅಥ ಶ್ರೀಶರಭಸಹಸ್ರನಾಮಸ್ತೋತ್ರಂ ॥\\
ಓಂ ಸರ್ವಭೂತಾತ್ಮಭೂತಸ್ಯ ರಹಸ್ಯಮಿತ ತೇಜಸ ।\\
ಅಷ್ಟೋತ್ತರಸಹಸ್ರಂ ತು ನಾಮ್ನಾಂ ಸರ್ವಸ್ಯ ಮೇ ಶೃಣು ॥೧॥

ಯಚ್ಛ್ರುತ್ವಾ ಮನುಜ ವ್ಯಾಘ್ರ ಸರ್ವಾಂತಾಮಾನವಾಪ್ಯಸಿ ।\\
ಸ್ಥಿರಃ ಸ್ಥಾಣುಃ ಪ್ರಭುರ್ಭೀಮಃ ಪ್ರಭವೋ ವರದೋ ವರಃ ॥೨॥

ಜಟೀ ಚರ್ಮೀ ಶಿಖಂಡೀ ಚ ಸರ್ವಾಂಗಃ ಸರ್ವಭಾವನಃ ।\\
ಹರಶ್ಚ ಹರಿಣಾಕ್ಷಶ್ಚ ಸರ್ವಭೂತಹರಃ ಪ್ರಭುಃ ॥೩॥

ಪ್ರವೃತ್ತಿಶ್ಚ ನಿವೃತ್ತಿಶ್ಚ ನಿಯತಃ ಶಾಶ್ವತೋ ಧ್ರುವಃ ।\\
ಶ್ಮಶಾನವಾಸೀ ಭಗವಾನ್ ವಚಸೋಽಗೋಚರೋ ಧನಃ ॥೪॥

ಅತಿವಾಧೋ ಮಹಾಕರ್ಮಾ ತಪಸ್ವೀ ಭೂತಭಾವನಃ ।\\
ಉನ್ಮತ್ತವೃಷೋಥ ಪ್ರಚ್ಛನ್ನಃ ಸರ್ವಲೋಕಪ್ರಜಾಪತಿಃ ॥೫॥

ಮಹಾರೂಪೋ ಮಹಾಕಾಯೋ ವೃಷರೂಪೋ ಮಹಾಯಶಾಃ ।\\
ಮಹಾತ್ಮಾ ಸರ್ವಭೂತಾತ್ಮಾ ವಿಶ್ವರೂಪೋ ಮಹಾಹನುಃ ॥೬॥

ಲೋಕಪಾಲೋಽತರ್ಹಿತಾತ್ಮಾ ಪ್ರಸಾದೋ ಹಯಗರ್ದಭೀ ।\\
ಪವಿತ್ರಶ್ಚ ಮಹಾಂಶ್ಚೈವ ನಿಯಮೋ ನಿಗಮಪ್ರಿಯ ॥೭॥

ಸರ್ವಕರ್ಮಾ ಸ್ವಯಂಭೂಶ್ಚ ಆದಿಸೃಷ್ಟಿಕರೋ ನಿಧಿ ।\\
ಸಹಸ್ರಾಕ್ಷೋ ವಿರೂಪಾಕ್ಷಃ ಸೋಮೋ ನಕ್ಷತ್ರಸಾಧಕಃ ॥೮॥

ಸೂರ್ಯಚಂದ್ರಗತಿಃ ಕೇತುರ್ಗ್ರಹೋ ಗ್ರಹಪತಿರ್ವರಃ ।\\
ಅದಾರಿದ್ರಘ್ನಾಲಯ ಕರ್ತಾ ಮೃಗಬಾಣಾರ್ಪಣೋನಘಃ ॥೯॥

ಮಹಾತಪಾ ದೀರ್ಘತಪಾ ಅದೀನೋ ದೀನಸಾಧನಃ ।\\
ಸಂವತ್ಸರಕರೋ ಮಂತ್ರೀ ಪ್ರಮಾಣಂ ಪರಮಂ ತಪಃ ॥೧೦॥

ಯೋಗೀ ಯೋಗ್ಯೋ ಮಹಾಬೀಜೋ ಮಹಾರೇತಾ ಮಹಾತಪಾಃ ।\\
ಸುವರ್ಣರೇತಾಃ ಸರ್ವಜ್ಞಃ ಸುವೀಜೋ ವೃಷವಾಹನಃ ॥೧೧॥

ದಶಬಾಹುಶ್ಚ ನಿಮಿಷೋ ನೀಲಕಂಠ ಉಮಾಪತಿಃ ।\\
ಬಹುರೂಪಃ ಸ್ವಯಂಶ್ರೇಷ್ಠೋ ಬಲಿರ್ವೈರೋಚನೋ ಗಣಃ ॥೧೨॥

ಗಣಕರ್ತ್ತಾ ಗಣಪತಿರ್ದಿಗ್ವಾಸಾಃ ಕಾಮ ಏವ ಚ ।\\
ಮಂತ್ರವಿತ್ಪರಮೋಮಂತ್ರಃ ಸರ್ವಭಾವಕರೋ ಹರಃ ॥೧೩॥

ಕಮಂಡಲುಧರೋ ಧನ್ವೀ ವಾಣಹಸ್ತಃ ಕಪಾಲವಾನ್ ।\\
ಅಶಿನೀ ಶತಘ್ನೀ ಖಂಡೀ ಪಟ್ಟಿಶಶ್ಚಾಯುಧೀ ಮಹಾನ್ ॥೧೪॥

ಶ್ರುತಿಹಸ್ತಃ ಸರೂಪಶ್ಚ ತೇಜಸ್ತೇಜಸ್ಕರೋ ವಿಭುಃ ।\\
ಉಶ್ನೀಷೀ ಚ ಸುವಕ್ತ್ರಶ್ಚ ಉದಗ್ರೋ ವಿನಯಸ್ತಥಾ ॥೧೫॥

ದೀರ್ಘಶ್ಚ ಹರಿನೇತ್ರಶ್ಚ ಸುತೀರ್ಥಃ ಕೃಷ್ಣ ಏವ ಚ ।\\
ಶೃಗಾಲರೂಪಃ ಸರ್ವಾರ್ಥೋ ಮುಂಡಃ ಸರ್ವಕಮಂಡಲುಃ ॥೧೬॥

ಅಜಶ್ಚ ಮೃಗರೂಪಶ್ಚ ಗಂಧಚಾರೀ ಕಪರ್ದಿನಃ ।\\
ಊರ್ಧ್ವರೇತಾ ಉರ್ಧ್ವಲಿಂಗ ಉರ್ಧ್ವಶಾಯೀ ನಭಸ್ತಲಃ ॥೧೭॥

ತ್ರಿಜಟಶ್ಚೌರವಾಸೀ ಚ ರುದ್ರಸೇನಾಪತಿರ್ವಿಭುಃ ।\\
ನಕ್ತಂಚರೋತಿತಿಗ್ಮಶ್ಚ ಅಹಶ್ಚಾರೀ ಸುವರ್ಚಸಃ ॥೧೮॥

ಗಜಹಾ ದೈತ್ಯಹಾ ಚೈವ ಲೋಕಭ್ರಾತಾ ಗುಣಾಕರಃ ।\\
ಸಿಂಹಶಾರ್ದೂಲರೂಪಶ್ಚ ಆರ್ದ್ರಚರ್ಮಾಂವರೋ ವರಃ ॥೧೯॥

ಕಾಲಯೋಗೀ ಮಹಾಕಾಲಃ ಸರ್ವವಾಸಾಶ್ಚತುಷ್ಪಥ ।\\
ನಿಶಾಚರ ಪ್ರೇತಚಾರೀ ಭೂತಚಾರೀ ಮಹೇಶ್ವರಃ ॥೨೦॥

ಬಹುರೂಪೋ ಬಹುಧನ ಸರ್ವಾಧಾರಾ ಮನೋಗತಿಃ ।\\
ನೃತ್ಯಪ್ರಿಯೋ ನೃತ್ಯತೃಪ್ತೋ ನೃತ್ಯಕಃ ಸರ್ವಮಾಲಯಃ ॥೨೧॥

ಧೋಷೋ ಮಹಾತಪಾ ಈಶೋ ನಿತ್ಯೋ ಗಿರಿಚರೋ ನಭಃ ।\\
ಸಹಸ್ರಹಸ್ತೋ ವಿಜಯೋ ವ್ಯವಸಾಯೋಹ್ಯನಿಂದಿತಃ ॥೨೨॥

ಅಮರ್ಷಣೋ ಮಹಾಮರ್ಷೀ ಈ ಯಶಕಾಮೋ ಮನೋಮಯಃ ।\\
ದಕ್ಷಯಜ್ಞಾಪಹಾರೀ ಚ ಸುಖದೋ ಮಧ್ಯಮಸ್ತಥಾ ॥೨೩॥

ತೇಜೋಪಹಾರೀ ಬಲಹಾ ಮುದಿತೋಪ್ಯಜಿತೋ ಭವಃ ।\\
ದಂಭೀ ದ್ವೇಷೀ ಗಂಭೀರೋ ಗಂಭೀರಬಲವಾಹನಃ ॥೨೪॥

ನ್ಯಗ್ರೋಧರೂಪೋ ನ್ಯಗ್ರೋಧೋ ವೃಕ್ಷಕರ್ತ್ತಾಸ್ಗವೃದ್ವಿಭುಃ ।\\
ತೀಕ್ಷ್ಣವ್ವಾಹುಶ್ಚ ಹರ್ಷಶ್ಚ ಸಹಾಯಃ ಸರ್ವಕಾಲವಿತ್ ॥೨೫॥

ವಿಷ್ಣುಪ್ರಸಾದಿತೋ ಯಜ್ಞಃ ಸಮುದ್ರೋ ವಡವಾಮುಖಃ ।\\
ಹುತಾಶನ ಸಹಾಯಶ್ಚ ಪ್ರಶಾಂತಾತ್ಮಾ ಹುತಾಶನಃ ॥೨೬॥

ಉಗ್ರತೇಜಾ ಮಹಾತೇಜಾ ಜಯೋ (ಜಯೋ) ವಿಜಯ ಕಾಲವಿತ್ ।\\
ಜ್ಯೋತಿಷಾಮಯನಃ ಸಿದ್ಧಿಃ ಸಂಧಿವಿಗ್ರಹ ಏವ ಚ ॥೨೭॥

ಶಿಖೀ ದಂಡೀ ಜಟೀ ಜ್ವಾಲೀ ಮೃತ್ಯುಜಿದ್ದುರ್ಧರೋ ವಲೀ ।\\
ವೈಷ್ಣವೀ ಪಣವೀತಾಲೀ ಕಾಲಃ ಕಾಟಕಟಂಕರಃ ॥೨೮॥

ನಕ್ಷತ್ರವಿಗ್ರಹವಿಧಿರ್ಗುಣವೃದ್ಧಿಲಯೋಗಮಃ ।\\
ಪ್ರಜಾಪತಿ ದಿಶಾ ವಾಹು ವಿಭಾಗಃ ಸರ್ವತೋಮುಖಃ ॥೨೯॥

ವೈರೋಚನೋ ಸುರಗಣೋ ಹಿರಣ್ಯಕವಚೋದ್ಧವಃ ।\\
ಅಪ್ರಜ್ಯೋ ವಾಲಚಾರೀ ಚ ಮಹಾಚಾರೀ ಸ್ತುತಸ್ತಥಾ ॥೩೦॥

ಸರ್ವತೂರ್ಯ ನಿನಾದೀ ಚ ಸರ್ವನಾಥ ಪರಿಗ್ರಹಃ ।\\
ವ್ಯಾಲರೂಪೋ ವಿಲಾವಾಸೀ ಹೇಮಮಾಲೀ ತರಂಗವಿತ್ ॥೩೧॥

ತ್ರಿದಿಶಸ್ತ್ರಿದಿಶಾವಾಸೀ ಸರ್ವಬಂಧವಿಮೋಚನಃ ।\\
ಬಂಧನಸ್ತ್ವಸುರೇಂದ್ರಾಣಾಂ ಯುಧಿ ಶತ್ರುವಿನಾಶನಃ ॥೩೨॥

ಸಾಕ್ಷಾತ್ಪ್ರಸಾದೋ ದುರ್ವಾಸಾ ಸರ್ವಸಾಧುನಿಷೇವಿತಃ ।\\
ಪುಸ್ಕಂದನೋ ವಿಭಾವಶ್ಚ ಅತುಲ್ಯೋ ಯಜ್ಞಭಾಗವಿತ್ ॥೩೩॥

ಸರ್ವಚಾರೀ ಸರ್ವವಾಸೋ ದುರ್ವಾಸಾ ವಾಙ್ಮನೋಭವಃ ।\\
ಹೇಮೋ ಹೇಮಕರೋ ಯಜ್ಞಃ ಸರ್ವವೀರೋ ನರೋತ್ತಮಃ ॥೩೪॥

ಲೋಹಿತಾಕ್ಷೋ ಮಹೋಕ್ಷಶ್ಚ ವಿಜಯಾಖ್ಯೋ ವಿಶಾರದಃ ।\\
ಸದ್ಗ್ರಹೋ ವಿಗ್ರಹೋ ಕರ್ಮಾ ಮೋಕ್ಷಃ ಸರ್ವನಿವಾಸನಃ ॥೩೫॥

ಮುಖ್ಯೋ ಮುಕ್ತಶ್ಚ ದೇಹಶ್ಚ ದೇಹಾರ್ಥಃ ಸರ್ವಕಾಮದಃ ।\\
ಸರ್ವಕಾಲಪ್ರಸಾದಶ್ಚ ಸುವಲೋ ವಲರೂಪಧೃಕ್ ॥೩೬॥

ಆಕಾಶನಿಧಿರೂಪಶ್ಚ ನಿಷಾದೀ ಉರಗಃ ಖಗಃ ।\\
ರೌದ್ರರೂಪೀ ಪಾಂಸುರಾದೀಃ ವಸುರಗ್ನಿಃ ಸುವರ್ಚಸೀ ॥೩೭॥

ವಸುವೇಗೋ ಮಹಾವೇಗೋ ಮಹಾಯಕ್ಷೋ ನಿಶಾಕರಃ ।\\
ಸರ್ವಭಾವಪ್ರಿಯಾವಾಸೀ ಉಪದೇಶಕರೋ ಹರಃ ॥೩೮॥

ಮನುರಾತ್ಮಾ ಪತಿರ್ಲೋಕೀ ಸಂಭೋಜ್ಯಶ್ಚ ಸಹಸ್ರಶಃ ।\\
ಪಕ್ಷೀ ಚ ಪಕ್ಷಿರೂಪೀ ಚ ಅತಿದೀಪ್ತೋ ವಿಶಾಂಪತಿಃ ॥೩೯॥

ಉನ್ಮಾದೋ ಮದನಃ ಕಾಮೋಹ್ಯಾಸ್ಯೋರ್ಥಕರೋಯಶಃ ।\\
ವಾಮದೇವಶ್ಚ ರಾಮಶ್ಚ ಪ್ರಾಗ್ದಕ್ಷಿಣಶ್ಚ ವಾಮನಃ ॥೪೦॥

ಸಿದ್ಧಯೋಗೋ ಮಹರ್ಷಿಶ್ಚ ಸಿದ್ಧಾರ್ಥಃ ಸಿದ್ಧಿಸಾಧಕಃ ।\\
ವಿಷ್ಣುಶ್ಚ ಭಿಕ್ಷುರೂಪಶ್ಚ ವಿಷಧ್ನೋ ಮೃದುರವ್ಯಯಃ ॥೪೧॥

ಮಹಾಸೇನೋ ವಿಶಾಖಶ್ಚ ವೃಷ್ಟಿಭೋಗೋ ಗವಾಂ ಪತಿ ।\\
ವಜ್ರಹಸ್ತಶ್ಚ ವಿಷ್ಕುಂಭೀ ಚ ಭೂಸ್ತಂಭನ ಏವ ಚ ॥೪೨॥

ವೃತ್ತೋ ವೃತ್ತಕರಃ ಸ್ಥಣುರ್ಮಧುಮಧುಕರೋ ಧನಃ ।\\
ವಾಚಸ್ಪತ್ಯೋ ವಾಜಸೇನೋ ನಿತ್ಯಮಾಶ್ರಮಪೂಜಿತಃ ॥೪೩॥

ಬ್ರಹ್ಮಚಾರೀ ಲೋಕಚಾರೀ ಸರ್ವಚಾರೀ ವಿಚಾರವಿತ್ ।\\
ಈಶಾನ ಈಶ್ವರಃ ಕಾಲೋ ನಿಶಾಚಾರೀ ಪಿನಾಕಧೃಕ್ ॥೪೪॥

ನಿಮಿತ್ತಜ್ಞೋ (ಸ್ಥೋ) ನಿಮತ್ತಶ್ಚ ನಂದಿರ್ನಾದಕರೋ ಹರಿಃ ।\\
ನದೀಶ್ವರಶ್ಚ ನಂದೀ ಚ ನಂದಿನೋ ನಂದಿವರ್ದ್ಧನಃ ॥೪೫॥

ಭಗಹಾರೀ ನಿಹಂತಾ ಚ ಕಾಲೋ ಬ್ರಹ್ಮಾ ಪಿತಾಮಹಃ ।\\
ಚತುರ್ಮುಖೋ ಮಹಾಲಿಂಗಶ್ಚತುರ್ಲಿಂಗಸ್ಥೈವ ಚ ॥೪೬॥

ಲಿಂಗಾಧ್ಯಕ್ಷಃ ಸುರಾಧ್ಯಕ್ಷೋ ಯೋಗಾಧ್ಯಕ್ಷೋ ಯುಗಾವಹಃ ।\\
ಬೀಜಾಧ್ಯಕ್ಷೋ ಬೀಜಕರ್ತ್ತಾ ಅಧ್ಯಾತ್ಮಾನುಗತೋ ಬಲಃ ॥೪೭॥

ಇತಿಹಾಸಃ ಸಕಲ್ಪಶ್ಚ ಗೌತಮೋಥ ನಿಶಾಕರಃ ।\\
ದಭೋಹ್ಯದಭೋ ವೈದಂಭೋ ವಶ್ಯೋ ವಶ್ಯಕರಃ ಕಲಿಃ ॥೪೮॥

ಲೋಕಕರ್ತಾ ಪಶುಪತಿರ್ಮಹಾಕರ್ತಾ ಹ್ಯನೌಷಧಃ ।\\
ಅಕ್ಷರಂ ಪರಮಂ ಬ್ರಹ್ಮ ವಲಟಾಚ್ಛನ್ನ ಏವ ಚ ॥೪೯॥

ನೀತಿರ್ಹ್ಯನೀತಿಃ ಶುದ್ಧಾತ್ಮಾ ಶುದ್ಧೋ ಮಾನ್ಯೋ ಗತಾಗತಿಃ ।\\
ಬಹುಪ್ರಸಾದ ಸುಸ್ವಪ್ನೋ ದರ್ಪಣೋಥತ್ವಮಿತ್ರಜಿತ್ ॥೫೦।\\
ವೇದಕಾರೋ ಮಂತ್ರಕಾರೋ ವಿದ್ವಾನ್ಸಮರಮರ್ದನಃ ।\\
ಮಹಾಮೋಘನಿವಾಸೀ ಚ ಮಹಾಘೋರೋ ವಶೀಕರಃ ॥೫೧॥

ಅಗ್ನಿಜ್ವಾಲೋ ಮಹಾಜ್ವಾಲೋ ಅತಿಧೂಮ್ರೋ ಹುತೋ ಹವಿಃ ।\\
ವೃಷಲಃ ಶಂಕರೋ ನಿತ್ಯೋ ವರ್ಚಸೀ ಧೂಮ್ರಲೋಚನಃ ॥೫೨॥

ನೀಲಸ್ತಥಾಂಗಲುಬ್ಧಶ್ಚ ಶೋಭನೋ ನಿರವಗ್ರಹಃ ।\\
ಸ್ವಸ್ತಿದಃ ಸ್ವಸ್ತಿಭಾವಶ್ಚ ಭೋಗೀ ಭೋಗಕರೋ ಲಘುಃ ॥೫೩॥

ಉತ್ಸಂಗಶ್ಚ ಮಹಾಂಗಶ್ಚ ಮಹಾಭೋಗೋ ಪರಾಯಣಃ ।\\
ಕೃಷ್ಣವರ್ಣಃ ಸುವರ್ಣಶ್ಚ ಇಂದ್ರಿಯಂ ಸರ್ವದೇಹಿನಾಂ ॥೫೪॥

ಮಹಾಪಾದೋ ಮಹಾಹಸ್ತೋ ಮಹಾಕಾಯೋ ಮಹಾಯಶಾಃ ।\\
ಮಹಾಮೂರ್ದ್ಧಾ ಮಹಾಮಾತ್ರೋ ಮಹಾನೇತ್ರೋ ನಿಶಾಲಯಃ ॥೫೫॥

ಮಹಾಂತಕೋ ಮಹಾಕರ್ಣೋ ಮಹೋಕ್ಷಶ್ಚ ಮಹಾಹನುಃ ।\\
ಮಹಾನನೋ ಮಹಾಕಂವುರ್ಮಹಾಗ್ರೀವಃ ಶ್ಮಶಾನಭಾಕ್ ॥೫೬॥

ಮಹಾವಕ್ಷಾ ಮಹೋರಸ್ಕೋ ಹ್ಯಂತರಾಮಾ ಮೃಗಾಲಯಃ ।\\
ಲಂಬಿತೋ ಲಂಬಿತೋಷ್ಟಶ್ಚ ಮಹಾಮಾಯಾ ಪಯೋನಿಧಿ ॥೫೭॥

ಮಹಾದಂತೋ ಮಹಾದಂಷ್ಟ್ರೋ ಮಹಾಜಿಹ್ವೋ ಮಹಾಮುಖಃ ।\\
ಮಹಾನಖೋ ಮಹಾರೋಮಾ ಮಹಾಕೇಶೋ ಮಹಾಜರಃ ॥೫೮॥

ಪ್ರಸನ್ನಶ್ಚ ಪ್ರಸಾದಶ್ಚ ಪ್ರತಯೋ ಯೋಗಿಸಾಧನಃ ।\\
ಸ್ನೇಹನೋತಿಶುಭಸ್ನೇಹಃ ಅಜಿತಶ್ಚ ಮಹಾಮುನಿಃ ॥೫೯॥

ವೃಕ್ಷಕಾರೋ ವೃಕ್ಷಕೇತುಃ ಅನಲೋ ವಾಯುವಾಹನಃ ।\\
ಮಂಡಲೀ ಧಾಮಶ್ಚ ದೇವಾಧಿಪತಿರೇವ ಚ ॥೬೦॥

ಅಥರ್ವಶೀರ್ಷಃ ಸಾಮಾಸ್ಯಃ ಋಕ್ ಸಾಹಸ್ರ ಮಿತೇಕ್ಷಣಃ ।\\
ಯಜುಃ ಪಾದಭುಜಾಗುಹ್ಯಃ ಪ್ರಕಾಶೋ ಜಂಗಮಸ್ತಥಾ ॥೬೧॥

ಅಮೋಧಾರ್ಥಪ್ರಸಾದಶ್ಚ ಅತಿಗಮ್ಯಃ ಸುದರ್ಶನಃ ।\\
ಉಪಕಾರಪ್ರಿಯಃ ಸರ್ವಃ ಕನಕಃ ಕಾಂಚನಸ್ಥಿತಃ ॥೬೨॥

ನಾಭಿರ್ನದಿಕರೋ ಭಾವಃ ಪುಷ್ಕರಸ್ಯ ಪತಿಸ್ಥಿರಃ ।\\
ದ್ವಾದಶಾಸ್ತ್ರಮಸಶ್ವಾಘೋ ಯಜ್ಞೋ ಯಜ್ಞಸಮಾಹಿತಃ ॥೬೩॥

ನಕ್ತಂ ಕಲಿಶ್ಚ ಕಾಲಶ್ಚ ಕಕಾರಃ ಕಾಲಪೂಜಿತಃ ।\\
ಸವಾಣೋ ಗಣಕಾರಶ್ಚ ಭೂತವಾಹನಸಾರಥಿಃ ॥೬೪॥

ಭಸ್ಮಶಾಯೀ ಭಸ್ಮಗೋಪ್ತಾ ಭಸ್ಮಭೂತಸ್ತಮೋಗುಣಃ ।\\
ಲೋಕಪಾಲಸ್ತಥಾ ಲೋಕೋ ಮಹಾತ್ಮಾ ಸರ್ವಪೂಜಿತಃ ॥೬೫॥

ಶುಕ್ಲಸ್ತ್ರಿಶುಕ್ಲಸಂಪನ್ನಃ ಶುಚಿರ್ಭೂತನಿಷೇವಿತಃ ।\\
ಆಶ್ರಮಸ್ಥಃ ಕ್ರಿಯಾವಸ್ಥೋ ವಿಶ್ವಕರ್ಮಾ ಮತಿರ್ವರಃ ॥೬೬॥

ವಿಶಾಲಶಾಖಸ್ತಾಮ್ರೋಷ್ಟೋಹ್ಯಂಬುಜಾಲಃ ಸುನಿಶ್ಚಲಃ ।\\
ಕಪಿಲಃ ಕಪಿಲಃ ಶುಕ್ಲ ಆಯುಶ್ಚೈವ ಪರೋವರಃ ॥೬೭॥

ಗಂಧರ್ವೋ ಹ್ಯದಿತಿಸ್ತಾರ್ಕ್ಷ್ಯಃ ಸುವಿಜ್ಞೇಯಃ ಸುಶಾರದಃ ।\\
ಪರಶ್ವಧಾಯುಧೋ ದೇವ ಅಂಧಕಾರಿಃ ಸುವಾಂಧವಃ ॥೬೮॥

ತುಂಬವೀಣೋ ಮಹಾಕ್ರೋಧ ಊರ್ಧ್ವಂರೇತಾ ಜಲೇಶಯಃ ।\\
ಉಗ್ರೋ ವಂಶಕರೋ ದ್ವಂಶೋ ವಂಶನಾದೋಹ್ಯನಿಂದಿತಃ ॥೬೯॥

ಸರ್ವಾಂಗರೂಪೋ ಮಾಯಾವೀ ಸುಹ್ಯದೋಹ್ಯನಿಲೋನಲಃ ।\\
ಬಂಧನೋ ಬಂಧಕರ್ತಾ ಚ ಸುಬಧುರವಿಮೋಚನಃ ॥೭೦॥

ಮೇಷಜಾರಿಃ ಸುಕರ್ಮಾರಿರ್ಮಹಾದಂಷ್ಟ್ರಸಮೋ ಯುಧಿ ।\\
ಬಹುಸ್ವನಿರ್ಮಿತಃ ಸರ್ವಃ ಶಂಕರಃ ಶಂಕರೋ ವರಃ ॥೭೧॥

ಅಮರೇಶೋ ಮಹಾದೇವೋ ವಿಶ್ವದೇವಃ ಸುರಾರಿಹಾ ।\\
ನಿಸಂಗಶ್ಚಾಹಿರ್ಬುಧ್ಯ್ನಶ್ಚಾಕಿತಾಕ್ಷೋ ಹರಿಸ್ತಥಾ ॥೭೨॥

ಅಜೈಕಪಾಲಪಾಲೀಚ ತ್ರಿಶಂಕುರಜಿತಃ ಶಿವಃ ।\\
ಧನ್ವತರಿರ್ಧೂಮ್ರಕೇತುಃ ಸ್ಕಂದೋ ವೈಶ್ರವಣಸ್ತಥಾ ॥೭೩॥

ಧಾತಾ ಶಕ್ರಶ್ಚ ವಿಶ್ವಶ್ಚ ಮಿತ್ರಸ್ತ್ವಷ್ಠಾಧ್ರುವೋ ವಸುಃ ।\\
ಪ್ರಭಾವಃ ಸರ್ವಗೋ ವಾಯುರರ್ಯಮಾಸವಿತಾರಥಿಃ ॥೭೪॥

ಉಗ್ರದಂಷ್ಟ್ರೋ ವಿಧಾತಾ ಚ ಮಾಂಧಾತಾ ಭೂತಭಾವನಃ ।\\
ರತಿಸ್ತೀರ್ಥಶ್ಚ ವಾಗ್ಮೀ ಚ ಸರ್ವಕರ್ಮಗುಣಾವಹ ॥೭೫॥।\\

ಪದ್ಮವಕ್ರೋ ಮಹಾವಕ್ರಶ್ಚಂದ್ರವಕ್ರೋ ಮನೋರಮಃ ।\\
ವಲಯಾನ್ಯಶ್ಚ ಶಾಂತಶ್ಚ ಪುರಾಣಃ ಪುಣ್ಯವರ್ಚಸಃ ॥೭೬॥

ಕುರುಕರ್ತಾ ಕಾಲರೂಪೀ ಕುರುಭೂತೋ ಮಹೇಶ್ವರಃ ।\\
ಶರ್ವೋ ಸರ್ವೋ ದರ್ಭಶಾಯೀ ಸರ್ವೇಷಾಂ ಪ್ರಾಣಿನಾಂ ಪತಿಃ ॥೭೭॥

ದೇವದೇವ ಸುಖಾಶಕ್ತಃ ಸದಸತ್ಸಂವರರತ್ನವಿತ್ ।\\
ಕೈಲಾಸಶಿಖಿರಾವಾಸೀ ಹಿಮವದ್ಗಿರಿಸಂಶ್ರಯಃ ॥೭೮॥

ಕೂಲಹಾರೀ ಕೂಲಕರ್ತ್ತಾ ಬಹುಬೀಜೋ ಬಹುಪ್ರದಃ ।\\
ವನಿಜೋ ವರ್ದ್ಧನೋ ದಕ್ಷೋ ನಕುಲಶ್ಚದನಶ್ಛದಃ ॥೭೯॥

ಸಾರಗ್ರೀವೀ ಮಹಾಜಂತುರತ್ನಕಶ್ಚ ಮಹೌಷಧಿಃ ।\\
ಸಿದ್ಧಾರ್ಥಕಾರೀ ಸಿದ್ಧಾರ್ಥಃ ಛಂದೋ ವ್ಯಾಕರಣಾನಿ ಚ ॥೮೦॥

ಸಿಂಹನಾದಃ ಸಿಂಹದಂಷ್ಟ್ರಃ ಸಿಂಹಗಃ ಸಿಂಹವಾಹನಃ ।\\
ಪ್ರಭಾವಾತ್ಮಾ ಜರಾಸ್ತಾಲೋಲ್ಲೋಕಾಹಿತಾಂತಕಃ ॥೮೧॥

ಸಾರಗೋಽಸುಖವಕ್ರಾಂತ ಕೇತುಮಾಲೀ ಸ್ವಭಾವತಃ ।\\
ಭೂತಾಶ್ರಯೋ ಭೂತಪತಿರಹೋರಾತ್ರಮನಿಂದಕಃ ॥೮೨॥

ಆಸನಃ ಸರ್ವಭೂತಾನಾಂ ನಿಲಯಃ ವಿಭುಭೈರವಃ ।\\
ಅಮೋಘಸರ್ವಭೂಷಾಸ್ಯೋ ಯಾಜನಃ ಪ್ರಾಣಹಾರಕಃ ॥೮೩॥

ಧೃತಿಮಾನ್ ಜ್ಞಾತಿಮಾನ್ ದಕ್ಷಃ ಸತ್ಕೃತಶ್ಚ ಯುಗಾಧಿಪಃ ।\\
ಗೋಪಾಲೋ ಗೋಪತಿರ್ಗೋಪ್ತಾ ಗೋಶ್ಚವಸನೋ ಹರಃ ॥೮೪॥

ಹಿರಣ್ಯಬಾಹುಶ್ಚ ತಥಾ ಗುಹಯಕಾಲಃ ಪ್ರವೇಶಕಃ ।\\
ಪ್ರತಿಷ್ಠಾಯಾಂ ಮಹಾಹರ್ಷೋಂಜಿತಕಾಮೋ ಜಿತೇಂದ್ರಿಯಃ ॥೮೫॥

ಗಾಂಧಾರಶ್ಚ ಸುಶೀಲಶ್ಚ ತಪಃ ಕರ್ಮರತಿರ್ಧನಃ ।\\
ಮಹಾಗೀತೋ ಮಹಾಬ್ರಹ್ಮಾರ್ಹ್ಮಕ್ಷರೋ ಗಣಸೇವಿತಃ ॥೮೬॥

ಮಹಾಕೇತುಃ ಕರ್ಮಧಾತನೈಕತಾನಶ್ಚರಾಚರಃ ।\\
ಅವೇದನೀಯ ಆವೇಶಃ ಸರ್ವಗಂಧಸುಖಾವಹಃ ॥೮೭॥

ತೋರಣಾಸ್ತರಣೋ ವಾಯುಃ ಪರಿಧಾವತಿ ಚೈತಕಃ ।\\
ಸಂಯೋಗೋ ವರ್ದ್ಧನೋ ವೃದ್ಧೋ ಮಹಾವೃದ್ಧೋ ಗಣಾಧಿಪಃ ॥೮೮॥

ನಿತ್ಯೋ ಧರ್ಮಸಹಾಯಶ್ಚ ದೇವಾಸುರಪತಿಃ ಪತಿಃ ।\\
ಅಮುಕ್ತೋ ಮುಸ್ತಬಾಹುಶ್ಚ ದ್ವಿವಿಧಶ್ಚ ಸುಪರ್ವಣಃ ॥೮೯॥

ಆಷಾಢಶ್ಚ ಸುಖಾಢ್ಯಶ್ಚ ಧ್ರುವೋ ಹರಿಹಯೋ ಹರಿಃ ।\\
ವಸುರಾವರ್ತ್ತನೋ ನಿತ್ಯೋ ವಸುಶ್ರೇಷ್ಠೋ ಮಹಾಮದಃ ॥೯೦॥

ಶಿರೋಹಾರೀ ಚ ವರ್ಷೀ ಚ ಸರ್ವಲಕ್ಷಣಭೂಷಿತಃ ।\\
ಅಕ್ಷರಶ್ಚಾಕ್ಷಯೋ ಯೋಗೀ ಸರ್ವಯೋಗೀ ಮಹಾವಲಃ ॥೯೧॥

ಸಮಾಮ್ನಾಯೋಽಸಮಾಮ್ನಾಯಸ್ತೀರ್ಥದೇವೋ ಮಹಾದ್ಯುತಿಃ ।\\
ನಿರ್ಬೀಜೋ ಜೀವನೋ ಮಂತ್ರೋ ಅನಘೋ ಬಹುಕರ್ಕಶಃ ॥೯೨॥

ರಕ್ತಪ್ರಭೂತೋ ರಕ್ತಾಂಗೋ ಮಹಾರ್ಣವನಿನಾದಕೃತ್ ।\\
ಮೂಲೋ ವಿಶಾಖೋ ಯಮೃತೋಕ್ತಯಕ್ತೋವ್ಯಃ ಸನಾತನಃ ॥೯೩॥

ಆರೋಹಣೋ ನಿರಂಹಶ್ಚ ಶೈಲಹಾರೀ ಮಹಾತಪಾಃ ।\\
ಸೇನಾಕಲ್ಪೋ ಮಹಾಕಲ್ಪೋ ಯುಗೋ ಯುಗಂಕರೋ ಹರಿಃ ॥೯೪॥

ಯುಗರೂಪೋ ಮಹಾರೂಪೋ ಪವನೋ ಗಹನೋ ನಗಃ ।\\
ನ್ಯಾಯನಿರ್ವಾಪಣೋ ನಾದಃ ಪಂಡಿತೋಹ್ಯಚಲೋಪಮಃ ॥೯೫॥

ಬಹುಮಾಲೋ ಮಹಾಮಾಲಃ ಸುಮಾಲೋ ಬಹುಲೋಚನಃ ।\\
ವಿಸ್ತಾರೋ ಲವಣಃ ಕ್ರೂರಃ ಋತುಮಾಸಫಲೋದಯಃ ॥೯೬॥

ವೃಷಭೋ ವೃಷಭಾಗಾಂಗೋ ಮಣಿಬಂಧುರ್ಜಟಾಧರಃ ।\\
ಇಂದ್ರೋ ವಿಸರ್ಗಃ ಸುಮುಖಃ ಸುರಃ ಸರ್ವಾಯುಧಃ ಸಹಃ ॥೯೭॥

ನಿವೇಶನಃ ಸುಧನ್ವಾ ಚ ಪೂಗಗಂಧೋ ಮಹಾಹನುಃ ।\\
ಗಂಧಮಾಲೀ ಚ ಭಗವಾನ್ ಸಾನಂದಃ ಸರ್ವಕರ್ಮಣಾಂ ॥೯೮॥

ಮಾತ್ಮನೋ ಬಾಹುಲೋ ಬಾಹುಃ ಸಕಲಃ ಸರ್ವಲೋಚನಃ ।\\
ರುದ್ರಸ್ತಾಲೀಕರಸ್ತಾಲೀ ಊರ್ಧ್ವಸಂಹತಲೋಚನಃ ॥೯೯॥

ಛತ್ರಪದ್ಮಃ ಸುವಿಖ್ಯಾತಃ ಸರ್ವಲೋಕಾಶ್ರಯೋ ಮಹಾನ್ ।\\
ಮುಂಡೋ ವಿರೂಪೋ ಬಹುಲೋ ದಂಡೀ ಮುಂಡೋ ವಿಕುಂಡಲಃ ॥೧೦೦॥

ಹರ್ಯಕ್ಷಃ ಕಕುಭೋಕ ವಜ್ರೀ ದೀಪ್ತಾವರ್ಚಃ ಸಹಸ್ರಪಾತ್ ।\\
ಸಹಸ್ರಮೂರ್ದ್ಧಾ ದೇವೇದ್ರಃ ಸರ್ವಭೂತಮಯೋ ಹರಿ ॥೧೦೧॥

ಸಹಸ್ರಬಾಹುಃ ಸರ್ವಾಂಗಃ ಶರಣ್ಯಃ ಸರ್ವಕರ್ಮಕೃತ್ ।\\
ಪವಿತ್ರಃ ಸ್ನಿಗ್ಧಯುರ್ಮಂತ್ರಃ ಕನಿಷ್ಠಃ ಕೃಷ್ಣಪಿಂಗಲಃ ॥೧೦೨॥

ಬ್ರಹ್ಮದಂಡವಿನಿರ್ವಾತಃ ಶರಘ್ನಃ ಶರತಾಪಧೃಕ್ ।\\
ಪದ್ಮಗರ್ಭೋ ಮಹಾಗರ್ಭೋ ಪದ್ಮಗರ್ಭೋ ಜಲೋದ್ಭವ ॥೧೦೩॥

ಗಭಸ್ತಿರ್ಬ್ರಹ್ಮಕೃತ್ ಬ್ರಹ್ಮ ಬ್ರಹ್ಮಕೃದ್ ಬ್ರಾಹ್ಮಣೋ ಗತಿ ।\\
ಅನಂತರೂಪೋ ನೈಕಾತ್ಮಾ ತಗ್ಮತೇಜಾತ್ಮಸಂಭವಃ ॥೧೦೪॥

ಊರ್ಧ್ವಗಾತ್ಮಾ ಪಶುಪತಿಃ ವೀತುರಂಗಾ ಮನೋಜವಃ ।\\
ವಂದನೀ ಪದ್ಮಮಾಲೀ ಚ ಗುಣಜ್ಞೋ ಸ್ವಗುಣೋತ್ತರಃ ॥೧೦೫॥

ಕರ್ಣಿಕಾರೋ ಮಹಾಸ್ರಗ್ವೀ ನೀಲಮೌಲೀ ಪಿನಾಕಧೃಕ್ ।\\
ಉಮಾಪತಿರುಮಾಕಾಂತೋ ಜಾಹ್ನವೀಹೃದಯಂಗಮಃ ॥೧೦೬॥

ವೀರೋ ವರಾಹೋ ವರದೋ ವರೇಶಶ್ಚ ಮಹಾಮನಾ ।\\
ಮಹಾಪ್ರಭಾವಸ್ತ್ವನಘಃ ಶತ್ರುಹಾ ಶ್ವೇತಪಿಂಗಲಃ ॥೧೦೭॥

ಪ್ರೀತಾತ್ಮಾ ಪ್ರಯತ್ತಾತ್ಮಾ ಚ ಸಂಯತಾತ್ಮಾ ಪ್ರಧಾನಧೃಕ್ ।\\
ಸರ್ವಪಾರ್ಶ್ವಸ್ತುತಸ್ತಾರ್ಕ್ಷ್ಯೋ ಧರ್ಮಃ ಸಾಧಾರಣೋ ವರಃ ॥೧೦೮॥

ಚರಾಚರಾತ್ಮಾ ಸೂಕ್ಷ್ಮಾತ್ಮಾ ಗೋವೃಷೋ ಗೋವೃಷೇಶ್ಚರಃ ।\\
ಸಾಧ್ಯರ್ಷಿರ್ವಸುರಾದಿತ್ಯೋ ವಿವಸ್ವಾನ್ ಸವಿತಾ ಮೃಗಃ ॥೧೦೯॥

ವ್ಯಾಸಃ ಸರ್ವಸ್ಯ ಸಂಕ್ಷೇಪೋ ವಿಸ್ತಾರಃ ಪರ್ಯಯೋನಯಃ ।\\
ಋತುಃ ಸಂವತ್ಸರೋ ಮಾಸಃ ಪಕ್ಷಃ ಸಂಖ್ಯಾ ಪರಾಯಣಃ ॥೧೧೦॥

ಕಲಾ ಕಾಷ್ಟಾ ಲಯೋ ಮಾತ್ರಾ ಮುಹೂರ್ತಃ ಪಕ್ಷಪಾಕ್ಷಣಃ ।\\
ವಿಶ್ವಕ್ಷೇತ್ರಂ ಪ್ರಜಾಬೀಜಂ ಲಿಂಗಮಾದ್ಯಸ್ತ್ವನಿಂದಿತಃ ॥೧೧೧॥

ಸದ್ವ್ಯವತಮವ್ಯಕ್ತ ಪಿತಾ ಮಾತಾ ಪಿತಾಮಹಃ ।\\
ಸ್ವರ್ಗದ್ವಾರಂ ಪ್ರಜಾದ್ವಾರಂ ಮೋಕ್ಷದ್ವಾರಂ ತ್ರಿವಿಷ್ಟಪಂ ॥೧೧೨॥

ನಿರ್ವಾಣಂ ಜ್ಞಾನದಂ ಚೈವ ಬ್ರಹ್ಮಲೋಕ ಪರಾಗತಿಃ ।\\
ದೇವಾಸುರಗುರುರ್ದೇವೋ ದೇವಾಸುರನಮಸ್ಕೃತಃ ॥೧೧೩॥

ದೇವಾಸುರಮಹಾಮಾತ್ರೋ ದೇವಾಸುರಸಮಾಶ್ರಯ ।\\
ದೇವಾಸುರಗಣಾಧ್ಯಕ್ಷೋ ದೇವಾಸುರಗಣಾಧಿಪಃ ॥೧೧೪॥

ದೇವಾಸುರೇಶ್ವರೋ ದೇವೋ ದೇವಾಸುರಮಹೇಶ್ವರಃ ।\\
ಸರ್ವದೇವಮಯೋ ಚಿಂತ್ಯೋ ದೇವಾನಾಮಾತ್ಮಸಂಭವಃ ॥೧೧೫॥

ಉದ್ಭಿಜ್ಜಸ್ತ್ರಿಕ್ರಮೋ ವೈದ್ಯೋ ವಿರಾಜೋ ವರದೋ ವರಃ ।\\
ಈಜ್ಯೋ ಹಸ್ತಿಮುಖೋ ವ್ಯಾಘ್ರೀ ದೇವಸಿಂಹೋ ನರರ್ಷಭಃ ॥೧೧೬॥

ವಿಬುಧಾಗ್ರವರಶ್ರೇಷ್ಠಃ ಸರ್ವದೇವೋತ್ತಮೋತ್ತಮಃ ।\\
ಗುರುಃ ಕಾಂತೋ ನಿಜಃ ಸರ್ವಃ ಪವಿತ್ರಃ ಸರ್ವವಾಹನಃ ॥೧೧೭॥

ಪ್ರಯುಕ್ತಃ ಶೋಭನೋ ವಜ್ರ ಈಶಾನಃ ಪ್ರಭುರವ್ಯಯಃ ।\\
ಭೃಗೀ ಭೃಂಗಪ್ರಿಯೋ ಬಭ್ರೂ ರಾಜರಾಜೋ ನಿರಾಮಯಃ ॥೧೧೮॥

ಅವಿರಾಮಃ ಸುಶರಣೋ ವಿರಾಮಃ ಸರ್ವಸಾಧನಃ ।\\
ಲಲಾಟಾಕ್ಷೋ ವಿಶ್ವದೇಹೋ ಹಾರಿಣೋ ಬ್ರಹ್ಮವರ್ಚಸೀ ॥೧೧೯॥

ಸ್ಥಾವರಾಣಾಂ ಪತಿಶ್ಚೈವ ನಿಯಮೇಂದ್ರಿಯವರ್ದ್ಧನಃ ।\\
ಸಿದ್ಧಾರ್ಥಃ ಸರ್ವಸಿದ್ಧಾರ್ಥೋನಿತ್ಯಃ ಸತ್ಯವ್ರತಃ ಶುಚಿಃ ॥೧೨೦॥

ವ್ರತಾದಿರ್ಯತ್ಪರಂ ಬ್ರಹ್ಮ ಮುಕ್ತಾನಾಂ ಪರಮಾಗತಿಃ ।\\
ವಿಮುಕ್ತೋ ದೀರ್ಘತೇಜಾಶ್ಚ ಶ್ರೀಮಾನ್ ಶ್ರೀವರ್ದ್ಧನೋ ಜಗತ್ ॥೧೨೧॥

ಯಥಾ ಪ್ರಸಾದೋ ಭಗವಾನಿತಿ ಭಕ್ತ್ಯಾ ಸ್ತುತೋ ಮಯಾ ।\\
ಯನ್ನ ಬ್ರಹ್ಮಾದಯೋ ದೇವಾ ವಿದುರ್ಯನ್ನ ಮಹರ್ಷಯಃ ॥೧೨೨॥

ತಂಸ್ತವೀಮ್ಯಹಮಾದ್ಯಂ ಚ ಕಸ್ತೋಷ್ಯತಿ ಜಗತ್ಪ್ರಭುಂ ।\\
ಭಕ್ತಿಶ್ಚೈವ ಪುರಸ್ಕೃತ್ಯ ಮಯಾ ಯಜ್ಞಪತಿರ್ವಸುಃ ॥೧೨೩॥

ತತೋಽನುಜ್ಞಾಪಯಾಮಾಸಸ್ತುತೋ ಮತಿಮತಾಂ ಗತಿಃ ।\\
ಶಿವ ಏವಂ ಸ್ತುತೋ ದೇವೈಃ ನಾಮಭಿಃ ಪುಷ್ಟಿವರ್ದ್ಧನೈಃ ॥೧೨೪॥

ನಿತ್ಯಯುಕ್ತಃ ಶುಚಿರ್ಭೂತ್ವಾ ಪ್ರಾಪ್ನ್ಯೋತ್ಯಾತ್ಮಾನಮಾತ್ಮನಃ ।\\
ಏತದ್ಧಿಪರಮಂ ಬ್ರಹ್ಮಾ ಸ್ವಯಂಗೀತಂ ಸ್ವಯಂಭುವಾ ॥೧೨೫॥

ಋಷಯಶ್ಚೈವ ದೇವಾಶ್ಚ ಸ್ತುವಂತ್ಯೇತೇ ನು ತತ್ಪರಂ ।\\
ಸ್ತೂಯಮಾನೋ ಮಹಾದೇವಃ ಪ್ರೀಯತೇ ಚಾತ್ಮನಾಪತಿಃ ॥೧೨೬॥

ಭಕ್ತಾನುಕಂಪೀ ಭಗವಾನಾತ್ಮಸಂಸ್ಥಾನ್ ಕರೋತಿ ತಾನ್ ।\\
ತಥೈವ ಚ ಮನುಷ್ಯೇಷು ಯತ್ರ ಕುತ್ರ ಪ್ರಧಾನತಃ ॥೧೨೭॥

ಆಸ್ತಿಕಃ ಶ್ರದ್ದಧಾನಶ್ಚ ಬಹುಭಿರ್ಜನ್ಮಭಿಃ ಸ್ತವೈಃ ।\\
ಜಾಗ್ರತೋಥ ಸ್ವಪತಶ್ಚ ವ್ರಜಂತೋ ಗತಿಸಂಸ್ಥಿತಾಃ ॥೧೨೮॥

ಸ್ತುವಂತಿ ಸ್ತೂಯಮಾನೇ ಚ ಚತುಷ್ಪಥಿ ರಮಂತಿ ಚ ।\\
ಜನ್ಮಕೋಟಿಸಹಸ್ರೇಷು ನಾನಾಸಂಸಾರಯೋನಿಷು ॥೧೨೯॥

ಜಂತೋರ್ವಿಶುದ್ಧಪಾಪಸ್ಯ ಭವೇ ಭಕ್ತಿಃ ಪ್ರಜಾಯತೇ ।\\
ಉತ್ಪನ್ನಾ ಚ ಭವೇ ಭಕ್ತಿರನನ್ಯಾ ಸರ್ವಭಾವತಃ ॥೧೩೦॥

ಏತದ್ದೇವೇಷು ದುಃಪ್ರಾಪೋ ಮಾನುಷೇಷು ನ ಲಭ್ಯತೇ ।\\
ನಿರ್ವಿಘ್ನಾ ನಿಶ್ಚಲಾ ಭದ್ರೇ ಭಕ್ತಿರವ್ಯಭಿಚಾರಿಣೀ ॥೧೩೧॥

ತಸ್ಯೈವ ಚ ಪ್ರಸಾದೇನ ಭಕ್ತಿರುತ್ಪದ್ಯತೇ ನೃಪ ।\\
ಯಯಾ ಯಾತಿ ಪರಾಂ ಸಿದ್ಧಿಂ ತದ್ಭಾಗವತಮಾನಸಃ ॥೧೩೨॥

ಯೇ ಸರ್ವಭಾವೋಪಹತಾಃ ಪರತ್ವೇನಾನುಭಾವಿತಾಃ ।\\
ಪ್ರಪನ್ನವತ್ಸಲಾ ದೇವಃ ಸಂಸಾರಾತ್ತಾನ್ ಸಮುದ್ಧರೇತ್ ॥೧೩೩॥

ಏವಮನ್ಯೇಪಿ ಕುರ್ವಂತಿ ದೇವಾಃ ಸಂಸಾರಮೋಚನಂ ।\\
ಮನುಷ್ಯಾಣಾಂ ಮಹಾದೇವಾದನ್ಯತ್ರಾಪಿ ತಪೋಬಲಾತ್ ॥೧೩೪॥

ಇತಿ ತೇನೇದಂ ಕಲ್ಯಾಯಾಯ ಭಗವಾನ್ ಸದಸತ್ ಪತಿಃ ।\\
ಕೃತ್ತಿವಾಸಾ ಧುವಂ ಪೂರ್ವಂ ತಾಡಿತಾ ಶುದ್ಧಬುದ್ಧಯಃ ॥೧೩೫॥

ಸ್ತವಮೇನಂ ಭಗವತಿ ಬ್ರಹ್ಮೋ ಸ್ವಯಮಧಾರಯತ್ ।\\
ಬ್ರಹ್ಮಾ ಪ್ರೋವಾಚ ಶಕ್ರಾಯ ಶಕ್ರಃ ಪ್ರೋವಾಚ ಮೃತ್ಯವೇ ॥೧೩೬॥

ಮೃತ್ಯುಃ ಪ್ರೋವಾಚ ರುದ್ರೇಭ್ಯೋ ರುದ್ರೇಭ್ಯಸ್ತಂಡಿಮಾಗಯತ್ ।\\
ಮಹತಾ ತಪಸಾ ಪ್ರಾಪ್ತಸ್ತಂಡಿನಾ ಬ್ರಹ್ಮ ಸಮ್ಮತಿಃ ॥೧೩೭॥

ಸ್ತಂಡೀಃ ಪ್ರೋವಾಚ ಶುಕ್ರಾಯ ಗೌತಮಾಯಾಹ ಭಾರ್ಗವಃ ।\\
ವೈವಸ್ವತಾಯ ಭಗವಾನ್ ಗೌತಮಃ ಪ್ರಾಹ ಸಾಧವೇ ॥೧೩೮॥

ನಾರಾಯಣಾಯ ಸಾಧ್ಯಾಯ ಮನುರಿಷ್ಟಾಯ ಧೀಮತೇ ।\\
ಯಮಾಯ ಪ್ರಾಹ ಭಗವಾನ್ ಸಾಧ್ಯೋ ನಾರಾಯಣೋವ್ಯಯಃ ॥೧೩೯॥

ನಾಚಿಕೇತಾಯ ಭಗವಾನಾಹ ವೈವಸ್ವತೋ ಯಮಃ ।\\
ಮಾರ್ಕಂಡೇಯಾಯ ವಾರ್ಷ್ಣೇಯ ನಾಚಿಕೇತಾಭ್ಯಭಾಷತ ॥೧೪೦॥

ತಥಾಪ್ಯಹಮಮಿತ್ರಘ್ನಸ್ತಾವಂದ್ಧಾದ್ಯ ವಿಶ್ರುತಂ ।\\
ಸ್ವರ್ಗ್ಯಮಾರೋಗಯಮಾಯುಷ್ಯಂ ಧನ್ಯಂ ವೇದೈಶ್ಚ ಸಮ್ಮಿತಂ ॥೧೪೧॥

ನಾಸ್ಯ ವಿಘ್ನಾನಿ ಕುರ್ವಂತಿ ದಾನವಾ ಯಕ್ಷರಾಕ್ಷಸಾಃ ।\\
ಪಿಶಾಚಾ ಯಾತುಧಾಂತಾಶ್ಚ ಗುಹ್ಯಕಾ ಭುಜಗಾ ಅಪಿ ॥೧೪೨॥

ಯ ಪಠೇತ್ಪ್ರಯತಃ ಪ್ರಾತರ್ಬ್ರಹ್ಮಾಚಾರೀ ಜಿತೇಂದ್ರಿಯ ।\\
ಅಭಿನ್ನಯೋಗೋ ವರ್ಷಂತು ಅಶ್ವಮೇಧಫಲಂ ಲಭೇತ್ ॥೧೪೩॥

ಇಯಾಕಾಶ ಭೈರವತಂತ್ರೇ ಹರಿಹರಬ್ರಹ್ಮವಿರಚಿತೇ ।\\
ಶರಭಸಹಸ್ರನಾಮಸ್ತೋತ್ರಂ ಸಂಪೂರ್ಣಂ ॥೧೪೪॥

\authorline{॥ಇತಿ ಆಕಾಶಭೈರವತಂತ್ರೇ ಹರಿಹರವಿರಚಿತಂ ಶ್ರೀಶರಭಸಹಸ್ರನಾಮಸ್ತೋತ್ರಂ ಸಂಪೂರ್ಣಂ॥}
%=============================================================================================
\section{ಶ್ರೀಶರಭಾಷ್ಟೋತ್ತರಶತನಾಮಸ್ತೋತ್ರಮ್}
\addcontentsline{toc}{section}{ಶ್ರೀಶರಭಾಷ್ಟೋತ್ತರಶತನಾಮಸ್ತೋತ್ರಮ್}

ಅಸ್ಯ ಶ್ರೀಶರಭಾಷ್ಟೋತ್ತರಶತನಾಮಮಹಾಮಂತ್ರಸ್ಯ ಯೋಗಾನಂದ ಋಷಿಃ  ಅನುಷ್ಟುಪ್ಛಂದಃ ಶ್ರೀಮದಘೋರವೀರಶರಭೇಶ್ವರೋ ದೇವತಾ । ಖಂ ಬೀಜಂ । ಸ್ವಾಹಾಶಕ್ತಿಃ । ಫಟ್ ಕೀಲಕಂ । ಜಪೇ ವಿನಿಯೋಗಃ ।\\

ಅಷ್ಟಾಂಘ್ರಿಶ್ಚ ಸಹಸ್ರಬಾಹುರನಲಚ್ಛಾಯಾ ಶಿರೋಯುಗ್ಮಧೃಗ್\\
ಯಸ್ತ್ರ್ಯಕ್ಷೋ ದ್ವಿಖುಪುಚ್ಛಉದಿತಃ ಸಾಕ್ಷಾನ್ನೃಸಿಂಹಾಸನಃ ।\\
ಅರ್ಧೇನಾಪಿ ಮೃಗಾಕೃತಿಃ ಪುನರಥಾಪ್ಯರ್ಧೇನ ಪಕ್ಷ್ಯಾಕೃತಿಃ\\
ಶ್ರೀ ವೀರಃ ಶಲಭಃ ಸ ಪಾತು ಶಲಭಶ್ಚಿಂತ್ಯಃ ಸದಾ ಮಾಂ ಹೃದಿ ॥

ಶ್ರೀನೃಸಿಂಹ ಉವಾಚ ।\\
ಸದಾ ಶಿವೋಗ್ರರೂಪಾಯ ಪಕ್ಷವಿಕ್ಷಿಪ್ತಭೂಭೃತೇ ।\\
ನಮೋ ರುದ್ರಾಯ ಶರ್ವಾಯ ಮಹಾಗ್ರಾಸಾಯ ವಿಷ್ಣವೇ ॥

ನಮ ಉಗ್ರಾಯ ಭೀಮಾಯ ನಮಃ ಕ್ರೋಧಾಯ ಮನ್ಯವೇ ।\\
ನಮೋ ಭವಾಯ ಶರ್ವಾಯ ಶಂಕರಾಯ ಶಿವಾಯ ತೇ ॥೧॥

ಕಾಲಕಾಲಾಯ ಕಾಲಾಯ ಮಹಾಕಾಲಾಆಯ ಮೃತ್ಯವೇ ।\\
ವೀರಾಯ ವೀರಭದ್ರಾಯ ಕ್ಷಯದ್ವೀರಾಯ ಶೂಲಿನೇ ॥೨॥

ಮಹಾದೇವಾಯ ಮಹತೇ ಪಶೂನಾಂಪತಯೇ ನಮಃ ।\\
ಏಕಾಯ ನೀಲಕಂಠಾಯ ಶ್ರೀಕಂಠಾಯ ಪಿನಾಕಿನೇ ॥೩॥

ನಮೋಽನಂತಾಯ ಸೂಕ್ಷ್ಮಾಯ ನಮಸ್ತೇ ಮೃತ್ಯುಮನ್ಯವೇ ।\\
ಪರಾಯ ಪರಮೇಶಾಯ ಪರಾತ್ಪರತರಾಯ ತೇ ॥೪॥

ಪರಾತ್ಪರಾಯ ವಿಶ್ವಾಯ ನಮಸ್ತೇ ವಿಶ್ವಮೂರ್ತಯೇ ।\\
ನಮೋ ವಿಷ್ಣುಕಲತ್ರಾಯ ವಿಷ್ಣುಕ್ಷೇತ್ರಾಯ ಭಾನವೇ ॥೫॥

ಕೈವರ್ತಾಯ ಕಿರಾತಾಯ ಮಹಾವ್ಯಾಧಾಯ ಶಾಶ್ವತೇ ।\\
ಭೌರವಾಯ ಶರಣ್ಯಾಯ ಮಹಾಭೈರವರೂಪಿಣೇ ॥೬॥

ನಮೋ ನೃಸಿಂಹಸಂಹರ್ತ್ರೇ ನಮಃ ಕಾಲಪುರಾರಯೇ ।\\
ಮಹಾಪಾಶೌಘಸಂಹರ್ತ್ರೇ ವಿಷ್ಣುಮಾಯಾಂತಕಾರಿಣೇ ॥೭॥

ತ್ರ್ಯಂಬಕಾಯ ತ್ರ್ಯಕ್ಷಾಯ ಶಿಪಿವಿಷ್ಟಾಯ ಮೀಢುಷೇ ।\\
ಮೃತ್ಯುಂಜಯಾಯ ಶರ್ವಾಯ ಸರ್ವಜ್ಞಾಯ ಮಖಾರಯೇ ॥೮॥

ಮಖೇಶಾಯ ವರೇಣ್ಯಾಯ ನಮಸ್ತೇ ವಹ್ನಿರೂಪಿಣೇ ।\\
ಮಹಾಘ್ರಾಣಾಯ ಜಿಹ್ವಾಯ ಪ್ರಾಣಾಪಾನಪ್ರವರ್ತಿನೇ ॥೯॥

ತ್ರಿಗುಣಾಯ ತ್ರಿಶೂಲಾಯ ಗುಣಾತೀತಾಯ ಯೋಗಿನೇ ।\\
ಸಂಸಾರಾಯ ಪ್ರವಾಹಾಯ ಮಹಾಯಂತ್ರ ಪ್ರವರ್ತಿನೇ ॥೧೦॥

ನಮಶ್ಚಂದ್ರಾಗ್ನಿಸೂರ್ಯಾಯ ಮುಕ್ತಿವೈಚಿತ್ರ್ಯಹೇತವೇ ।\\
ವರದಾಯಾವತಾರಾಯ ಸರ್ವಕಾರಣಹೇತವೇ ॥೧೧॥

ಕಪಾಲಿನೇ ಕರಾಲಾಯ ಪತಯೇ ಪುಣ್ಯಕೀರ್ತ್ತಯೇ ।\\
ಅಮೋಘಾಯಾಗ್ನಿನೇತ್ರಾಯ ನಕುಲೀಶಾಯ ಶಂಭವೇ ॥೧೨॥

ಭಿಷಕ್ತಮಾಯ ಮುಂಡಾಯ ದಂಡಿನೇ ಯೋಗರೂಪಿಣೇ ।\\
ಮೇಘವಾಹಾಯ ದೇವಾಯ ಪಾರ್ವತೀಪತಯೇ ನಮಃ ॥೧೩॥

ಅವ್ಯಕ್ತಾಯ ವಿಶೋಕಾಯ ಸ್ಥಿರಾಯ ಸ್ಥಿರಧನ್ವಿನೇ ।\\
ಸ್ಥಾಣವೇ ಕೃತ್ತಿವಾಸಾಯ ನಮಃ ಪಂಚಾರ್ಥಹೇತವೇ ॥೧೪॥

ವರದಾಯ ಕಪರ್ದಾಯ ನಮಶ್ಚಂದ್ರಾರ್ದ್ಧಮೌಲಿನೇ ।\\
ನಮಸ್ತೇಽವರರಾಜಾಯ ವಯಸಾಂ ಪತಯೇ ನಮಃ ॥೧೫॥

ಯೋಗೀಶ್ವರಾಯ ನಿತ್ಯಾಯ ಸತ್ಯಾಯ ಪರಮೇಷ್ಠಿನೇ ।\\
ಸರ್ವಾತ್ಮನೇ ನಮಸ್ತುಭ್ಯಂ ನಮಃ ಸರ್ವೇಶ್ವರಾಯ ತೇ ॥೧೬॥

ಏಕದ್ವಿತ್ರಿಚತುಃಪಂಚ ಕೃತ್ವಾಸ್ತೇಽಸ್ತು ನಮೋ ನಮಃ ।\\
ದಶಕೃತ್ವಸ್ತು ಸಾಹಸ್ರಕೃತ್ವಸ್ತೇ ಚ ನಮೋ ನಮಃ ॥೧೭॥

ನಮೋಽಪರಿಮಿತಿಂ ಕೃತ್ವಾಽನಂತಕೃತ್ವೋ ನಮೋ ನಮಃ ।\\
ನಮೋ ನಮೋ ನಮೋ ಭೂಯಃ ಪುನರ್ಭೂಯೋ ನಮೋ ನಮಃ ॥೧೮॥

ಸೂತ ಉವಾಚ ।\\
ನಾಮ್ನಾಮಷ್ಟಶತೇನೈವ ಸ್ತುತ್ವಾಽಮೃತಯೇ ನ ತು ।\\
ಪುನಸ್ತು ಪ್ರಾರ್ಥಯಾಮಾಸ ನೃಸಿಂಹಂ ಶರಭೇಶ್ವರಂ ॥೧೯॥

ಯದಾ ಯದಾ ಮಮಾಜ್ಞಾನಮತ್ಯಹಂಕಾರದೂಷಿತಂ ।\\
ತದಾ ತದಾಽಪನೇತವ್ಯಂ ತ್ವಯೈವ ಪರಮೇಶ್ವರಃ ॥೨೦॥

ಏವಂ ವಿಜ್ಞಾಪಯನ್ಪ್ರೀತಃ ಶಂಂಕರಂ ನರಕೇಸರೀ ।\\
ನನ್ವಶಕ್ತೋ ಭವಾನ್ವಿಷ್ಣೋ! ಜೀವಿತಾಂತಂ ಪರಾಜಿತಃ ॥೨೧॥

ತದ್ವಕ್ತ್ರಶೇಷಮಾತ್ರಾಂತಂ ಕೃತ್ವಾ ಶರ್ವಸ್ಯ ವಿಗ್ರಹಂ ।\\
ಶುಕ್ತಿಶಿತ್ಯಂ ತದಾಮಂಗಂ ವೀರಭದ್ರಃ ಕ್ಷಣಾತ್ತತಃ ॥೨೨॥

\authorline{ಇತಿ ಲಿಂಗಪುರಾಣೋಕ್ತಂ ಶ್ರೀನೃಸಿಂಹಕೃತಂ ಶರಭಸ್ತೋತ್ರಂ ಸಂಪೂರ್ಣಂ ॥}
%================================================================

\section{ಶ್ರೀಗುರುಸಹಸ್ರನಾಮಸ್ತೋತ್ರಂ}
\addcontentsline{toc}{section}{ಶ್ರೀಗುರುಸಹಸ್ರನಾಮಸ್ತೋತ್ರಂ}

॥ಅಥ ಶ್ರೀಗುರುಸಹಸ್ರನಾಮಸ್ತೋತ್ರಂ॥
ಕೈಲಾಸಶಿಖರಾಸೀನಂ ಚಂದ್ರಖಂಡವಿರಾಜಿತಂ ।\\
ಪಪ್ರಚ್ಛ ವಿನಯಾದ್ಭಕ್ತ್ಯಾ ಗೌರೀ ನತ್ವಾ ವೃಷಧ್ವಜಂ ॥೧॥

॥ಶ್ರೀದೇವ್ಯುವಾಚ ॥\\
ಭಗವನ್ ಸರ್ವಧರ್ಮಜ್ಞ ಸರ್ವಶಾಸ್ತ್ರವಿಶಾರದ ।\\
ಕೇನೋಪಾಯೇನ ಚ ಕಲೌ ಲೋಕಾರ್ತಿರ್ನಾಶಮೇಷ್ಯತಿ ॥೨॥
ತನ್ಮೇ ವದ ಮಹಾದೇವ ಯದಿ ತೇಽಸ್ತಿ ದಯಾ ಮಯಿ ।\\

॥ಶ್ರೀಮಹಾದೇವ ಉವಾಚ ॥\\
ಅಸ್ತಿ ಗುಹ್ಯತಮಂ ತ್ವೇಕಂ ಜ್ಞಾನಂ ದೇವಿ ಸನಾತನಂ ॥೩॥

ಅತೀವ ಚ ಸುಗೋಪ್ಯಂ ಚ ಕಥಿತುಂ ನೈವ ಶಕ್ಯತೇ ।\\
ಅತೀವ ಮೇ ಪ್ರಿಯಾಸೀತಿ ಕಥಯಾಮಿ ತಥಾಪಿ ತೇ ॥೪॥

ಸರ್ವಂ ಬ್ರಹ್ಮಮಯಂ ಹ್ಯೇತತ್ಸಂಸಾರಂ ಸ್ಥೂಲಸೂಕ್ಷ್ಮಕಂ ।\\
ಪ್ರಕೃತ್ಯಾ ತು ವಿನಾ ನೈವ ಸಂಸಾರೋ ಹ್ಯುಪಪದ್ಯತೇ ॥೫॥

ತಸ್ಮಾತ್ತು ಪ್ರಕೃತಿರ್ಮೂಲಕಾರಣಂ ನೈವ ದೃಶ್ಯತೇ ।\\
ರೂಪಾಣಿ ಬಹುಸಂಖ್ಯಾನಿ ಪ್ರಕೃತೇಃ ಸಂತಿ ಮಾನಿನಿ ॥೬॥

ತೇಷಾಂ ಮಧ್ಯೇ ಪ್ರಧಾನಂ ತು ಗುರುರೂಪಂ ಮನೋರಮಂ ।\\
ವಿಶೇಷತಃ ಕಲಿಯುಗೇ ನರಾಣಾಂ ಭುಕ್ತಿಮುಕ್ತಿದಂ ॥೭॥

ತಸ್ಯೋಪಾಸಕಾಶ್ಚೈವ ಬ್ರಹ್ಮಾವಿಷ್ಣುಶಿವಾದಯಃ ।\\
ಸೂರ್ಯಶ್ಚಂದ್ರಶ್ಚ ವರುಣಃ ಕುಬೇರೋಽಗ್ನಿಸ್ತಥಾಪರಾಃ ॥೮॥

ದುರ್ವಾಸಾಶ್ಚ ವಸಿಷ್ಠಶ್ಚ ದತ್ತಾತ್ರೇಯೋ ಬೃಹಸ್ಪತಿಃ ।\\
ಬಹುನಾತ್ರ ಕಿಮುಕ್ತೇನ ಸರ್ವೇದೇವಾ ಉಪಾಸಕಾಃ ॥೯॥

ಗುರೂಣಾಂ ಚ ಪ್ರಸಾದೇನ ಭುಕ್ತಿಮುಕ್ತ್ಯಾದಿಭಾಗಿನಃ ।\\
ಸಂವಿತ್ಕಲ್ಪಂ ಪ್ರವಕ್ಷ್ಯಾಮಿ ಸಚ್ಚಿದಾನಂದಲಕ್ಷಣಂ ॥೧೦॥

ಯತ್ಕಲ್ಪಾರಾಧನೇನೈವ ಸ್ವಾತ್ಮಾನಂದೋ ವಿರಾಜತೇ ।\\
ಮೇರೋರುತ್ತರದೇಶೇ ತು ಶಿಲಾಹೈಮಾವತೀ ಪುರೀ ॥೧೧॥

ದಶಯೋಜನವಿಸ್ತೀರ್ಣಾ ದೀರ್ಘಷೋಡಶಯೋಜನಾ ।\\
ವರರತ್ನೈಶ್ಚ ಖಚಿತಾ ಅಮೃತಂ ಸ್ರವತೇ ಸದಾ ॥೧೨॥

ಸೋತ್ಥಿತಾ ಶಬ್ದನಿರ್ಮುಕ್ತಾ ತೃಣವೃಕ್ಷವಿವರ್ಜಿತಾ ।\\
ತಸ್ಯೋಪರಿ ವರಾರೋಹೇ ಸಂಸ್ಥಿತಾ ಸಿದ್ಧಮೂಲಿಕಾ ॥೧೩॥

ವೇದಿಕಾಜನನಿರ್ಮುಕ್ತಾ ತನ್ನದೀಜಲಸಂಸ್ಥಿತಾ ।\\
ವೇದಿಕಾಮಧ್ಯದೇಶೇ ತು ಸಂಸ್ಥಿತಂ ಚ ಶಿವಾಲಯಂ ॥೧೪॥

ಹಸ್ತಾಷ್ಟಕಸುವಿಸ್ತಾರಂ ಸಮಂತಾಚ್ಚ ತಥೈವ ಚ ।\\
ತಸ್ಯೋಪರಿ ಚ ದೇವೇಶಿ ಹ್ಯುಪವಿಷ್ಟೋ ಹ್ಯಹಂ ಪ್ರಿಯೇ ॥೧೫॥

ದಿವ್ಯಾಬ್ದವರ್ಷಪಂಚಾಶತ್ಸಮಾಧೌ ಸಂಸ್ಥಿತೋ ಹ್ಯಹಂ ।\\
ಮಹಾಗುರುಪದೇ ದೃಷ್ಟಂ ಗೂಢಂ ಕೌತುಹಲಂ ಮಯಾ ॥೧೬॥

ಓಂ ಅಸ್ಯ ಶ್ರೀಗುರುಸಹಸ್ರನಾಮಮಾಲಾಮಂತ್ರಸ್ಯ ಶ್ರೀಸದಾಶಿವಋಷಿಃ ನಾನಾವಿಧಾನಿ ಛಂದಾಂಸಿ ಶ್ರೀಗುರುರ್ದೇವತಾ ಶ್ರೀಗುರುಪ್ರೀತ್ಯರ್ಥೇ ಸಕಲಪುರುಷಾರ್ಥಸಿದ್ಧ್ಯರ್ಥೇ ಶ್ರೀಗುರುಸಹಸ್ರನಾಮ ಜಪೇ ವಿನಿಯೋಗಃ ।\\
॥ಅಥಾಂಗನ್ಯಾಸಃ॥\\
ಶ್ರೀಸದಾಶಿವಋಷಯೇ ನಮಃ ಶಿರಸಿ॥\\
ಶ್ರೀನಾನಾವಿಧಛಂದೇಭ್ಯೋ ನಮಃ ಮುಖೇ॥\\
ಶ್ರೀಗುರುದೇವತಾಯೈ ನಮಃ ಹೃದಯೇ॥\\
ಶ್ರೀ ಹಂ ಬೀಜಾಯ ನಮಃ ಗುಹ್ಯೇ॥\\
ಶ್ರೀ ಶಂ ಶಕ್ತಯೇ ನಮಃ ಪಾದಯೋಃ॥\\
ಶ್ರೀ ಕ್ರೌಂ ಕೀಲಕಾಯ ನಮಃ ಸರ್ವಾಂಗೇ॥\\

॥ಅಥ ಗುರುಗಾಯತ್ರೀಮಂತ್ರಃ॥\\
ಓಂ ಗುರುದೇವಾಯ ವಿದ್ಮಹೇ ಪರಮಗುರವೇ ಚ ಧೀಮಹಿ । ತನ್ನೋ ಪುರುಷಃ ಪ್ರಚೋದಯಾತ್ ॥

॥ಅಥ ಕರನ್ಯಾಸಃ॥\\
ಓಂ ಸದಾಶಿವಗುರವೇ ನಮಃ ಅಂಗುಷ್ಠಾಭ್ಯಾಂ ನಮಃ ।\\
ಓಂ ವಿಷ್ಣುಗುರವೇ ನಮಃ ತರ್ಜನೀಭ್ಯಾಂ ನಮಃ ।\\
ಓಂ ಬ್ರಹ್ಮಗುರವೇ ನಮಃ ಮಧ್ಯಮಾಭ್ಯಾಂ ನಮಃ ।\\
ಓಂ ಗುರು ಇಂದ್ರಾಯ ನಮಃ ಅನಾಮಿಕಾಭ್ಯಾಂ ನಮಃ ।\\
ಓಂ ಗುರುಸಕಲದೇವರೂಪಿಣೇ ನಮಃ ಕನಿಷ್ಠಿಕಾಭ್ಯಾಂ ನಮಃ ।\\
ಓಂ ಗುರುಪಂಚತತ್ತ್ವಾತ್ಮನೇ ನಮಃ ಕರತಲಕರಪೃಷ್ಠಾಭ್ಯಾಂ ನಮಃ ।\\

\as{ಹಂಸಾಭ್ಯಾಂ ಪರಿವೃತ್ತಪತ್ರಕಮಲೈರ್ದಿವ್ಯೈರ್ಜಗತ್ಕಾರಣೈ\\
ರ್ವಿಶ್ವೋತ್ಕೀರ್ಣಮನೇಕದೇಹನಿಲಯಂ ಸ್ವಚ್ಛಂದಮಾತ್ಮೇಚ್ಛಯಾ ।\\
ತತ್ತದ್ಯೋಗ್ಯತಯಾ ಸ್ವದೇಶಿಕತನುಂ ಭಾವೈಕದೀಪಾಂಕುರಂ \\
ಪ್ರತ್ಯಕ್ಷಾಕ್ಷರವಿಗ್ರಹಂ ಗುರುಪದಂ ಧ್ಯಾಯೇದ್ದ್ವಿಬಾಹುಂ ಗುರುಂ ॥೧೭॥\\
ವಿಶ್ವಂ ವ್ಯಾಪಿತಮಾದಿದೇವಮಮಲಂ ನಿತ್ಯಂ ಪರನ್ನಿಷ್ಕಲಂ\\
ನಿತ್ಯೋತ್ಫುಲ್ಲಸಹಸ್ರಪತ್ರಕಮಲೈರ್ನಿತ್ಯಾಕ್ಷರೈರ್ಮಂಡಪೈಃ ।\\
ನಿತ್ಯಾನಂದಮನಂತಪೂರ್ಣಮಖಿಲಂತದ್ಬ್ರಹ್ಮ ನಿತ್ಯಂ ಸ್ಮರೇ\\
ದಾತ್ಮಾನಂ ಸ್ವಮನುಪ್ರವಿಶ್ಯ ಕುಹರೇ ಸ್ವಚ್ಛಂದತಃ ಸರ್ವಗಂ ॥೧೮॥}\\

॥ಅಥ ಮಂತ್ರಃ ॥\\
॥ಓಂ ಐಂ ಹ್ರೀಂ ಶ್ರೀಂ ಗುರವೇ ನಮಃ ॥\\

ತ್ವಂ ಹಿ ಮಾಮನುಸಂಧೇಹಿ ಸಹಸ್ರಶಿರಸಂಪ್ರಭುಂ ।\\
ತದಾ ಮುಖೇಷು ಮೇ ನ್ಯಸ್ತಂ ಸಹಸ್ರಂ ಲಕ್ಷ್ಯತೇ ಸ್ತದಾ ॥೧೯॥

ಇದಂ ವಿಶ್ವಹಿತಾರ್ಥಾಯ ರಸನಾರಂಗಗೋಚರಂ ।\\
ಪ್ರಕಾಶಯಿತ್ವಾ ಮೇದಿನ್ಯಾಂ ಪರಮಾಗಮಸಮ್ಮತಾಂ ॥೨೦॥

ಇದಂ ಶಠಾಯ ಮೂರ್ಖಾಯ ನಾಸ್ತಿಕಾಯ ಪ್ರಕೀರ್ತನೇ ।\\
ಅಸೂಯೋಪಹತಾಯಾಪಿ ನ ಪ್ರಕಾಶ್ಯಂ ಕದಾಚನ ॥೨೧॥

ವಿವೇಕಿನೇ ವಿಶುದ್ಧಾಯ ವೇದಮಾರ್ಗಾನುಸಾರಿಣೇ ।\\
ಆಸ್ತಿಕಾಯಾತ್ಮನಿಷ್ಠಾಯ ಸ್ವಾತ್ಮನ್ಯವಿಕೃತಾಯ ಚ ॥೨೨॥

ಗುರುನಾಮಸಹಸ್ರಂ ತೇ ಕೃತಧೀರುದಿತೇ ಜಯೇ ।\\
ಭಕ್ತಿಗಮ್ಯಸ್ತ್ರಯೀಮೂರ್ತಿರ್ಭಾಸಕ್ತೋ ವಸುಧಾಧಿಪಃ ॥೨೩॥

ದೇವದೇವೋ ದಯಾಸಿಂಧುರ್ದೇವದೇವಶಿಖಾಮಣಿಃ ।\\
ಸುಖಾಭಾವಃ ಸುಖಾಚಾರಃ ಶಿವದೋ ಮುದಿತಾಶಯಃ ॥೨೪॥

ಅವಿಕ್ರಿಯಃ ಕ್ರಿಯಾಮೂರ್ತಿರಧ್ಯಾತ್ಮಾ ಚ ಸ್ವರೂಪವಾನ್ ।\\
ಸೃಷ್ಟ್ಯಾಮಲಕ್ಷ್ಯೋ ಭೂತಾತ್ಮಾ ಧರ್ಮೀ ಯಾತ್ರಾರ್ಥಚೇಷ್ಟಿತಃ ॥೨೫॥

ಅಂತರ್ಯಾಮೀ ಕಾಲರೂಪಃ ಕಾಲಾವಯವಿರೂಪಿಣಃ ।\\
ನಿರ್ಗುಣಶ್ಚ ಕೃತಾನಂದೋ ಯೋಗೀ ನಿದ್ರಾನಿಯೋಜಕಃ ॥೨೬॥

ಮಹಾಗುಣಾಂತರ್ನಿಕ್ಷಿಪ್ತಃ ಪುಣ್ಯಾರ್ಣವಪುರಾತ್ಮವಾನ್ ।\\
ನಿರವದ್ಯಃ ಕೃಪಾಮೂರ್ತಿರ್ನ್ಯಾಯವಾಕ್ಯನಿಯಾಮಕಃ ॥೨೭॥

ಅದೃಷ್ಟಚೇಷ್ಟಃ ಕೂಟಸ್ಥೋ ಧೃತಲೌಕಿಕವಿಗ್ರಹಃ ।\\
ಮಹರ್ಷಿಮಾನಸೋಲ್ಲಾಸೋ ಮಹಾಮಂಗಲದಾಯಕಃ ॥೨೮॥

ಸಂತೋಷಿತಃ ಸುರವ್ರಾತಃ ಸಾಧುಚಿತ್ತಪ್ರಸಾದಕಃ ।\\
ಶಿವಲೋಕಾಯ ನಿರ್ದೇಷ್ಟಾ ಜನಾರ್ದನಶ್ಚ ವತ್ಸಲಃ ॥೨೯॥

ಸ್ವಶಕ್ತ್ಯುದ್ಧಾಟಿತಾಶೇಷಕಪಾಟಃ ಪಿತೃವಾಹನಃ ।\\
ಶೇಷೋರಗಫಣಂಛತ್ರಃ ಶೋಷೋಕ್ತ್ಯಾಸ್ಯಸಹಸ್ರಕಃ ॥೩೦॥

ಕೃತಾತ್ಮವಿದ್ಯಾವಿನ್ಯಾಸೋ ಯೋಗಮಾಯಾಗ್ರಸಂಭವಃ ।\\
ಅಂಜನಸ್ನಿಗ್ಧನಯನಃ ಪರ್ಯಾಯಾಂಕುರಿತಸ್ಮಿತಃ ॥೩೧॥

ಲೀಲಾಕ್ಷಸ್ತರಲಾಲೋಕಸ್ತ್ರಿಪುರಾಸುರಭಂಜನಃ ।\\
ದ್ವಿಜೋದಿತಸ್ವಸ್ತ್ಯಯನೋ ಮಂತ್ರಪೂತೋ ಜಲಾಪ್ಲುತಃ ॥೩೨॥

ಪ್ರಶಸ್ತನಾಮಕರಣೋ ಜಾತುಚಂಕ್ರಮಣೋತ್ಸುಕಃ ।\\
ವ್ಯಾಲವಿಚೂಲಿಕಾರತ್ನಘೋಷೋ ಘೋಷಪ್ರಹರ್ಷಣಃ ॥೩೩॥

ಸನ್ಮುಖಃ ಪ್ರತಿಬಿಂಬಾರ್ಥೀ ಗ್ರೀವಾವ್ಯಾಘ್ರನಖೋಜ್ಜ್ವಲಃ ।\\
ಪಂಕಾನುಲೇಪರುಚಿರೋ ಮಾಂಸಲೋರುಕಟೀತಲಃ ॥೩೪॥

ದೃಷ್ಟಜಾನುಕರದ್ವಂದ್ವಃ ಪ್ರತಿಬಿಂಬಾನುಕಾರಕೃತ್ ।\\
ಅವ್ಯಕ್ತವರ್ಣವ್ಯಾವೃತ್ತಿಃ ಸ್ಮಿತಲಕ್ಷ್ಯರದೋದ್ಗಮಃ ॥೩೫॥

ಧಾತ್ರೀಕರಸಮಾಲಂಬೀ ಪ್ರಸ್ಖಲಚ್ಚಿತ್ರಚಂಕ್ರಮಃ ।\\
ಕ್ಷೇಮಣೀ ಕ್ಷೇಮಣಾಪ್ರೀತೋ ವೇಣುವಾದ್ಯವಿಶಾರದಃ ॥೩೬॥

ನಿಯುದ್ಧಲೀಲಾಸಂಹೃಷ್ಟಃ ಕಂಠಾನುಕೃತಕೋಕಿಲಃ ।\\
ಉಪಾತ್ತಹಂಸಗಮನಃ ಸರ್ವಸತ್ತ್ವರುತಾನುಕೃತ್ ॥೩೭॥

ಮನೋಜ್ಞಃ ಪಲ್ಲವೋತ್ತಂಸಃ ಪುಷ್ಪಸ್ವೇಚ್ಛಾತ್ಮಕುಂಡಲಃ ।\\
ಮಂಜುಸಂಜಿತಮಂಜೀರಪಾದಃ ಕಾಂಚನಕಂಕಣಃ ॥೩೮॥

ಅನ್ಯೋನ್ಯಸ್ಪರ್ಶನಕ್ರೀಡಾಪಟುಃ ಪರಮಕೇತನಃ ।\\
ಪ್ರತಿಧ್ವಾನಪ್ರಮುದಿತಃ ಶಾಖಾಚತುರಚಂಕ್ರಮಃ ॥೩೯॥

ಬ್ರಹ್ಮತ್ರಾಣಕರೋ ಧಾತೃಸ್ತುತಃ ಸರ್ವಾರ್ಥಸಾಧಕಃ ।\\
ಬ್ರಹ್ಮಬ್ರಹ್ಮಮಯೋಽವ್ಯಕ್ತಃ ತೇಜಾಸ್ತವ್ಯಃ ಸುಖಾತ್ಮಕಃ ॥೪೦॥

ನಿರುಕ್ತೋ ವ್ಯಾಕೃತೋ ವ್ಯಕ್ತಿರ್ನಿರಾಲಂಬವಿಭಾವನಃ ।\\
ಪ್ರಭವಿಷ್ಣುರತಂದ್ರೀಕೋ ದೇವವೃಕ್ಷಾದಿರೂಪಧೃಕ್ ॥೪೧॥

ಆಕಾಶಃ ಸರ್ವದೇವಾದಿರಣೀಯಸ್ಥೂಲರೂಪವಾನ್ ।\\
ವ್ಯಾಪ್ಯಾವ್ಯಾಪ್ಯಕೃತಾಕರ್ತಾ ವಿಚಾರಾಚಾರಸಮ್ಮತಃ ॥೪೨॥

ಛಂದೋಮಯಃ ಪ್ರಧಾನಾತ್ಮಾ ಮೂರ್ತೋ ಮೂರ್ತ್ತದ್ವಯಾಕೃತಿಃ ।\\
ಅನೇಕಮೂರ್ತಿರಕ್ರೋಧಃ ಪರಾತ್ಪರಪರಾಕ್ರಮಃ ॥೪೩॥

ಸಕಲಾವರಣಾತೀತಃ ಸರ್ವದೇವಮಹೇಶ್ವರಃ ।\\
ಅನನ್ಯವಿಭವಃ ಸತ್ಯರೂಪಃ ಸ್ವರ್ಗೇಶ್ವರಾರ್ಚಿತಃ ॥೪೪॥

ಮಹಾಪ್ರಭಾವಜ್ಞಾನಜ್ಞಃ ಪೂರ್ವಗಃ ಸಕಲಾತ್ಮಜಃ ।\\
ಸ್ಮಿತೇಕ್ಷಾಹರ್ಷಿತೋ ಬ್ರಹ್ಮಾ ಭಕ್ತವತ್ಸಲವಾಕ್ಪ್ರಿಯಃ ॥೪೫॥

ಬ್ರಹ್ಮಾನಂದೋದಧೌತಾಂಘ್ರಿಃ ಲೀಲಾವೈಚಿತ್ರ್ಯಕೋವಿದಃ ।\\
ವಿಲಾಸಸಕಲಸ್ಮೇರೋ ಗರ್ವಲೀಲಾವಿಲೋಕನಃ ॥೪೬॥

ಅಭಿವ್ಯಕ್ತದಯಾತ್ಮಾ ಚ ಸಹಜಾರ್ಧಸ್ತುತೋ ಮುನಿಃ ।\\
ಸರ್ವೇಶ್ವರಃ ಸರ್ವಗುಣಃ ಪ್ರಸಿದ್ಧಃ ಸಾತ್ವತರ್ಷಭಃ ॥೪೭॥

ಅಕುಂಠಧಾಮಾ ಚಂದ್ರಾರ್ಕಹೃಷ್ಟರಾಕಾಶನಿರ್ಮಲಃ ।\\
ಅಭಯೋ ವಿಶ್ವತಶ್ಚಕ್ಷುಸ್ತಥೋತ್ತಮಗುಣಪ್ರಭುಃ ॥೪೮॥

ಅಹಮಾತ್ಮಾ ಮರುತ್ಪ್ರಾಣಃ ಪರಮಾತ್ಮಾಽಽದ್ಯಶೀರ್ಷವಾನ್ ।\\
ದಾವಾಗ್ನಿಭೀತಸ್ಯ ಗುರೋರ್ಗೋಪ್ತಾ ದಾವಾನಿಗ್ನನಾಶನಃ ॥೪೯॥

ಮುಂಜಾಟವ್ಯಗ್ನಿಶಮನಃ ಪ್ರಾವೃಟ್ಕಾಲವಿನೋದವಾನ್ ।\\
ಶಿಲಾನ್ಯಸ್ತಾನ್ನಭುಗ್ಜಾತಸೌಹಿತ್ಯಶ್ಚಾಂಗುಲಾಶನಃ ॥೫೦॥

ಗೀತಾಸ್ಫೀತಸರಿತ್ಪೂರೋ ನಾದನರ್ತಿತಬರ್ಹಿಣಃ ।\\
ರಾಗಪಲ್ಲವಿತಸ್ಥಾಣುರ್ಗೀತಾನಮಿತಪಾದಪಃ ॥೫೧॥

ವಿಸ್ಮಾರಿತತೃಣಸ್ಯಾಗ್ರಗ್ರಾಸೀಮೃಗವಿಲೋಭನಃ ।\\
ವ್ಯಾಘ್ರಾದಿಹಿಂಸ್ರರಜಂತುವೈರಹರ್ತಾ ಸುಗಾಯನಃ ॥೫೨॥

ನಿಷ್ಯಂದಧ್ಯಾನಬ್ರಹ್ಮಾದಿವೀಕ್ಷಿತೋ ವಿಶ್ವವಂದಿತಃ ।\\
ಶಾಖೋತ್ಕೀರ್ಣಶಕುಂತೌಘಛತ್ರಾಸ್ಥಿತಬಲಾಹಕಃ ॥೫೩॥

ಅಸ್ಪಂದಃ ಪರಮಾನಂದಚಿತ್ರಾಯಿತಚರಾಚರಃ ।\\
ಮುನಿಜ್ಞಾನಪ್ರದೋ ಯಜ್ಞಸ್ತುತೋ ವಾಸಿಷ್ಠಯೋಗಕೃತ್ ॥೫೪॥

ಶತ್ರುಪ್ರೋಕ್ತಕ್ರಿಯಾರೂಪಃ ಶತ್ರುಯಜ್ಞನಿವಾರಣಃ ।\\
ಹಿರಣ್ಯಗರ್ಭಹೃದಯೋ ಮೋಹವೃತ್ತಿನಿವರ್ತಕಃ ॥೫೫॥

ಆತ್ಮಜ್ಞಾನನಿಧಿರ್ಮೇಧಾ ಕೀಶಸ್ತನ್ಮಾತ್ರರೂಪವಾನ್ ।\\
ಇಂದ್ರಾಗ್ನಿವದನಃ ಕಾಲನಾಭಃ ಸರ್ವಾಗಮಸ್ತುತಃ ॥೫೬॥

ತುರೀಯಃ ಸತ್ತ್ವಧೀಃ ಸಾಕ್ಷೀ ದ್ವಂದ್ವಾರಾಮಾತ್ಮದೂರಗಃ ।\\
ಅಜ್ಞಾತಪಾರೋ ವಿಶ್ವೇಶಃ ಅವ್ಯಾಕೃತವಿಹಾರವಾನ್ ॥೫೭॥

ಆತ್ಮಪ್ರದೀಪೋ ವಿಜ್ಞಾನಮಾತ್ರಾತ್ಮಾ ಶ್ರೀನಿಕೇತನಃ ।\\
ಪೃಥ್ವೀ ಸ್ವತಃಪ್ರಕಾಶಾತ್ಮಾ ಹೃದ್ಯೋ ಯಜ್ಞಫಲಪ್ರದಃ ॥೫೮॥

ಗುಣಗ್ರಾಹೀ ಗುಣದ್ರಷ್ಟಾ ಗೂಢಸ್ವಾತ್ಮಾನುಭೂತಿಮಾನ್ ।\\
ಕವಿರ್ಜಗದ್ರೂಪದ್ರಷ್ಟಾ ಪರಮಾಕ್ಷರವಿಗ್ರಹಃ ॥೫೯॥

ಪ್ರಪನ್ನಪಾಲನೋ ಮಾಲಾಮನುರ್ಬ್ರಹ್ಮವಿವರ್ಧನಃ ।\\
ವಾಕ್ಯವಾಚಕಶಕ್ತ್ಯಾರ್ಥಃ ಸರ್ವವ್ಯಾಪೀ ಸುಸಿದ್ಧಿದಃ ॥೬೦॥

ಸ್ವಯಂಪ್ರಭುರನಿರ್ವಿದ್ಯಃ ಸ್ವಪ್ರಕಾಶಶ್ಚಿರಂತನಃ ।\\
ನಾದಾತ್ಮಾ ಮಂತ್ರಕೋಟೀಶೋ ನಾನಾವಾದಾನುರೋಧಕಃ ॥೬೧॥

ಕಂದರ್ಪಕೋಟಿಲಾವಣ್ಯಃ ಪರಾರ್ಥೈಕಪ್ರಯೋಜಕಃ ।\\
ಅಭಯೀಕೃತದೇವೌಘಃ ಕನ್ಯಕಾಬಂಧಮೋಚನಃ ॥೬೨॥

ಕ್ರೀಡಾರತ್ನಬಲೀಹರ್ತ್ತಾ ವರುಣಚ್ಛತ್ರಶೋಭಿತಃ ।\\
ಶಕ್ರಾಭಿವಂದಿತಃ ಶಕ್ರಜನನೀಕುಂಡಲಪ್ರದಃ ॥೬೩॥

ಯಶಸ್ವೀ ನಾಭಿರಾದ್ಯಂತರಹಿತಃ ಸತ್ಕಥಾಪ್ರಿಯಃ ।\\
ಅದಿತಿಪ್ರಸ್ತುತಸ್ತೋತ್ರೋ ಬ್ರಹ್ಮಾದ್ಯುತ್ಕೃಷ್ಟಚೇಷ್ಟಿತಃ ॥೬೪॥

ಪುರಾಣಃ ಸಂಯಮೀ ಜನ್ಮ ಹ್ಯಧಿಪಃ ಶಶಕೋಽರ್ಥದಃ ।\\
ಬ್ರಹ್ಮಗರ್ಭಪರಾನಂದಃ ಪಾರಿಜಾತಾಪಹಾರಕೃತ್ ॥೬೫॥

ಪೌಂಡ್ರಿಕಪ್ರಾಣಹರಣಃ ಕಾಶೀರಾಜನಿಷೂದನಃ ।\\
ಕೃತ್ಯಾಗರ್ವಪ್ರಶಮನೋ ವಿಚಕೃತ್ಯಾಗರ್ವದರ್ಪಹಾ ॥೬೬॥

ಕಂಸವಿಧ್ವಂಸನಃ ಶಾಂತಜನಕೋಟಿಭಯಾರ್ದನಃ ।\\
ಮುನಿಗೋಪ್ತಾ ಪಿತೃವರಪ್ರದಃ ಸರ್ವಾನುದೀಕ್ಷಿತಃ ॥೬೭॥

ಕೈಲಾಸಯಾತ್ರಾಸುಮುಖೋ ಬದರ್ಯ್ಯಾಶ್ರಮಭೂಷಣಃ ।\\
ಘಂಟಾಕರ್ಣಕ್ರಿಯಾದೋಗ್ಧಾತೋಷಿತೋ ಭಕ್ತವತ್ಸಲಃ ॥೬೮॥

ಮುನಿವೃಂದಾತಿಥಿರ್ಧ್ಯೇಯೋ ಘಂಟಾಕರ್ಣವರಪ್ರದಃ ।\\
ತಪಶ್ಚರ್ಯಾ ಪಶ್ಚಿಮಾದ್ಯೋ ಶ್ವಾಸೋ ಪಿಂಗಜಟಾಧರಃ ॥೬೯॥

ಪ್ರತ್ಯಕ್ಷೀಕೃತಭೂತೇಶಃ ಶಿವಸ್ತೋತಾ ಶಿವಸ್ತುತಃ ।\\
ಗುರುಃ ಸ್ವಯಂ ವರಾಲೋಕಕೌತುಕೀ ಸರ್ವಸಮ್ಮತಃ ॥೭೦॥

ಕಲಿದೋಷನಿರಾಕರ್ತ್ತಾ ದಶನಾಮಾ ದೃಢವ್ರತಃ ।\\
ಅಮೇಯಾತ್ಮಾ ಜಗತ್ಸ್ವಾಮೀ ವಾಗ್ಮೀ ಚೈದ್ಯಶಿರೋಹರಃ ॥೭೧॥

ಗುರುಶ್ಚ ಪುಂಡರೀಕಾಕ್ಷೋ ವಿಷ್ಣುಶ್ಚ ಮಧುಸೂದನಃ ।\\
ಗುರುಮಾಧವಲೋಕೇಶೋ ಗುರುವಾಮನರೂಪಧೃಕ್ ॥೭೨॥

ವಿಹಿತೋತ್ತಮಸತ್ಕಾರೋ ವಾಸವಾಪ್ತರಿಪು ಇಷ್ಟದಃ ।\\
ಉತ್ತಂಕಹರ್ಷದಾತ್ಮಾ ಯೋ ದಿವ್ಯರೂಪಪ್ರದರ್ಶಕಃ ॥೭೩॥

ಜನಕಾವಗತಸ್ತೋತ್ರೋ ಭಾರತಃ ಸರ್ವಭಾವನಃ ।\\
ಅಸೋಢ್ಯಯಾದವೋದ್ರೇಕೋ ವಿಹಿತಾತ್ಪರಿಪೂಜಿತಃ ॥೭೪॥

ಸಮುದ್ರಕ್ಷಪಿತಾಶ್ಚರ್ಯಮುಸಲೋ ವೃಷ್ಣಿಪುಂಗವಃ ।\\
ಮುನಿಶಾರ್ದೂಲಪದ್ಮಾಂಕಃ ಸನಾದಿತ್ರಿದಶಾರ್ದಿತಃ ॥೭೫॥

ಗುರುಪ್ರತ್ಯವಹಾರೋಕ್ತಃ ಸ್ವಧಾಮಗಮನೋತ್ಸುಕಃ ।\\
ಪ್ರಭಾಸಾಲೋಕನೋದ್ಯುಕ್ತೋ ನಾನಾವಿಧನಿಮಿತ್ತಕೃತ್ ॥೭೬॥

ಸರ್ವಯಾದವಸಂಸೇವ್ಯಃ ಸರ್ವೋತ್ಕೃಷ್ಟಪರಿಚ್ಛದಃ ।\\
ವೇಲಾಕಾನನಸಂಚಾರೀ ವೇಲಾನೀಲಹತಶ್ರಮಃ ॥೭೭॥

ಕಾಲಾತ್ಮಾ ಯಾದವಾನಂತಸ್ತುತಿಸಂತುಷ್ಟಮಾನಸಃ ।\\
ದ್ವಿಜಾಲೋಕನಸಂತುಷ್ಟಃ ಪುಣ್ಯತೀರ್ಥಮಹೋತ್ಸವಃ ॥೭೮॥

ಸತ್ಕಾರಾಹ್ಲಾದಿತಾಶೇಷಭೂಸುರೋ ಭೂಸುರಪ್ರಿಯಃ ।\\
ಪುಣ್ಯತೀರ್ಥಪ್ಲುತಃ ಪುಣ್ಯಃ ಪುಣ್ಯದಸ್ತೀರ್ಥಪಾವನಃ ॥೭೯॥

ವಿಪ್ರಸಾತ್ಸ್ವಕೃತಃ ಕೋಟಿಶತಕೋಟಿಸುವರ್ಣದಃ ।\\
ಸ್ವಮಾಯಾಮೋಹಿತಾಶೇಷರುದ್ರವೀರೋ ವಿಶೇಷಜಿತ್ ॥೮೦॥

ಬ್ರಹ್ಮಣ್ಯದೇವಃ ಶ್ರುತಿಮಾನ್ ಗೋಬ್ರಾಹ್ಮಣಹಿತಾಯ ಚ ।\\
ವರಶೀಲಃ ಶಿವಾರಂಭಃ ಸ್ವಸಂವಿಜ್ಞಾತಮೂರ್ತ್ತಿಮಾನ್ ॥೮೧॥

ಸ್ವಭಾವಭದ್ರಃ ಸನ್ಮಿತ್ರಃ ಸುಶರಣ್ಯಃ ಸುಲಕ್ಷಣಃ ।\\
ಸಾಮಗಾನಪ್ರಿಯೋ ಧರ್ಮೋ ಧೇನುವರ್ಮತಮೋಽವ್ಯಯಃ ॥೮೨॥

ಚತುರ್ಯುಗಕ್ರಿಯಾಕರ್ತ್ತಾ ವಿಶ್ವರೂಪಪ್ರದರ್ಶಕಃ ।\\
ಅಕಾಲಸಂಧ್ಯಾಘಟನಃ ಚಕ್ರಾಂಕಿತಶ್ಚ ಭಾಸ್ಕರಃ ॥೮೩॥

ದುಷ್ಟಪ್ರಮಥನಃ ಪಾರ್ಥಪ್ರತಿಜ್ಞಾಪ್ರತಿಪಾಲಕಃ ।\\
ಮಹಾಧನೋ ಮಹಾವೀರೋ ವನಮಾಲಾವಿಭೂಷಣಃ ॥೮೪॥

ಸುರಃ ಸೂರ್ಯೋ ಮೃಕಂಡಶ್ಚ ಭಾಸ್ಕರೋ ವಿಶ್ವಪೂಜಿತಃ ।\\
ರವಿಸ್ತಮೋಹಾ ವಹ್ನಿಶ್ಚ ವಾಡವೋ ವಡವಾನಲಃ ॥೮೫॥

ದೈತ್ಯದರ್ಪವಿನಾಶೀ ಚ ಗರುಡೋ ಗರುಡಾಗ್ರಜಃ ।\\
ಪ್ರಪಂಚೀ ಪಂಚರೂಪಶ್ಚ ಲತಾಗುಲ್ಮಶ್ಚ ಗೋಪತಿಃ ॥೮೬॥

ಗಂಗಾ ಚ ಯಮುನಾರೂಪೀ ಗೋದಾ ವೇತ್ರಾವತೀ ತಥಾ ।\\
ಕಾವೇರೀ ನರ್ಮದಾ ತಾಪೀ ಗಂಡಕೀ ಸರಯೂ ರಜಃ ॥೮೭॥

ರಾಜಸಸ್ತಾಮಸಃ ಸಾತ್ತ್ವೀ ಸರ್ವಾಂಗೀ ಸರ್ವಲೋಚನಃ ।\\
ಮುದಾಮಯೋಽಮೃತಮಯೋ ಯೋಗಿನೀವಲ್ಲಭಃ ಶಿವಃ ॥೮೮॥

ಬುದ್ಧೋ ಬುದ್ಧಿಮತಾಂ ಶ್ರೇಷ್ಠೋ ವಿಷ್ಣುರ್ಜಿಷ್ಣುಃ ಶಚೀಪತಿಃ ।\\
ಸೃಷ್ಟಿಚಕ್ರಧರೋ ಲೋಕೋ ವಿಲೋಕೋ ಮೋಹನಾಶನಃ ॥೮೯॥

ರವೋ ರಾವೋ ರವೋ ರಾವೋ ಬಲೋ ಬಾಲಬಲಾಹಕಃ ।\\
ಶಿವರುದ್ರೋ ನಲೋ ನೀಲೋ ಲಾಂಗಲೀ ಲಾಂಗಲಾಶ್ರಯಃ ॥೯೦॥

ಪಾರಕಃ ಪಾರಕೀ ಸಾರ್ವೀ ವಟಪಿಪ್ಪಲಕಾಕೃತೀಃ ।\\
ಮ್ಲೇಚ್ಛಹಾ ಕಾಲಹರ್ತಾ ಚ ಯಶೋ ಜ್ಞಾನಂ ಚ ಏವ ಚ ॥೯೧॥

ಅಚ್ಯುತಃ ಕೇಶವೋ ವಿಷ್ಣುರ್ಹರಿಃ ಸತ್ಯೋ ಜನಾರ್ದನಃ ।\\
ಹಂಸೋ ನಾರಾಯಣೋ ಲೀಲೋ ನೀಲೋ ಭಕ್ತಪರಾಯಣಃ ॥೯೨॥

ಮಾಯಾವೀ ವಲ್ಲಭಗುರುರ್ವಿರಾಮೋ ವಿಷನಾಶನಃ ।\\
ಸಹಸ್ರಭಾನುರ್ಮಹಾಭಾನುರ್ವೀರಭಾನುರ್ಮಹೋದಧಿಃ ॥೯೩॥

ಸಮುದ್ರೋಽಬ್ಧಿರಕೂಪಾರಃ ಪಾರಾವಾರಸರಿತ್ಪತಿಃ ।\\
ಗೋಕುಲಾನಂದಕಾರೀ ಚ ಪ್ರತಿಜ್ಞಾಪ್ರತಿಪಾಲಕಃ ॥೯೪॥

ಸದಾರಾಮಃ ಕೃಪಾರಾಮೋ ಮಹಾರಾಮೋ ಧನುರ್ಧರಃ ।\\
ಪರ್ವತಃ ಪರ್ವತಾಕಾರೋ ಗಯೋ ಗೇಯೋ ದ್ವಿಜಪ್ರಿಯಃ ॥೯೫॥

ಕಮಲಾಶ್ವತರೋ ರಾಮೋ, ಭವ್ಯೋ ಯಜ್ಞಪ್ರವರ್ತ್ತಕಃ ।\\
ದ್ಯೌರ್ದಿವೌ ದಿವಓ ದಿವ್ಯೌ ಭಾವೀ ಭಾವಭಯಾಪಹಾ ॥೯೬॥

ಪಾರ್ವತೀಭಾವಸಹಿತೋ ಭರ್ತ್ತಾ ಲಕ್ಷ್ಮೀವಿಲಾಸವಾನ್ ।\\
ವಿಲಾಸೀ ಸಹಸೀ ಸರ್ವೋ ಗುರ್ವೀ ಗರ್ವಿತಲೋಚನಃ ॥೯೭॥

ಮಾಯಾಚಾರೀ ಸುಧರ್ಮಜ್ಞೋ ಜೀವನೋ ಜೀವನಾಂತಕಃ ।\\
ಯಮೋ ಯಮಾರಿರ್ಯಮನೋ ಯಾಮೀ ಯಾಮವಿಧಾಯಕಃ ॥೯೮॥

ಲಲಿತಾ ಚಂದ್ರಿಕಾಮಾಲೀ ಮಾಲೀ ಮಾಲಾಂಬುಜಾಶ್ರಯಃ ।\\
ಅಂಬುಜಾಕ್ಷೋ ಮಹಾಯಕ್ಷೋ ದಕ್ಷಶ್ಚಿಂತಾಮಣಿಃ ಪ್ರಭುಃ ॥೯೯॥

ಮೇರೋಶ್ಚೈವ ಚ ಕೇದಾರಬದರ್ಯ್ಯಾಶ್ರಮಮಾಗತಃ ।\\
ಬದರೀವನಸಂತಪ್ತೋ ವ್ಯಾಸಃ ಸತ್ಯವತೀ ಸುತಃ ॥೧೦೦॥

ಭ್ರಮರಾರಿನಿಹಂತಾ ಚ ಸುಧಾಸಿಂಧುವಿಧೂದಯಃ ।\\
ಚಂದ್ರೋ ರವಿಃ ಶಿವಃ ಶೂಲೀ ಚಕ್ರೀ ಚೈವ ಗದಾಧರಃ ॥೧೦೧॥

ಸಹಸ್ರನಾಮ ಚ ಗುರೋಃ ಪಠಿತವ್ಯಂ ಸಮಾಹಿತೈಃ ।\\
ಸ್ಮರಣಾತ್ಪಾಪರಾಶೀನಾಂ ಖಂಡನಂ ಮೃತ್ಯುನಾಶನಂ ॥೧೦೨॥

ಗುರುಭಕ್ತಪ್ರಿಯಕರಂ ಮಹಾದಾರಿದ್ರ್ಯನಾಶನಂ ।\\
ಬ್ರಹ್ಮಹತ್ಯಾ ಸುರಾಪಾನಂ ಪರಸ್ತ್ರೀಗಮನಂ ತಥಾ ॥೧೦೩॥

ಪರದ್ರವ್ಯಾಪಹರಣಂ ಪರದೋಷಸಮನ್ವಿತಂ ।\\
ಮಾನಸಂ ವಾಚಿಕಂ ಕಾಯಂ ಯತ್ಪಾಪಂ ಪಾಪಸಂಭವಂ ॥೧೦೪॥

ಸಹಸ್ರನಾಮಪಠನಾತ್ಸರ್ವಂ ನಶ್ಯತಿ ತತ್ಕ್ಷಣಾತ್ ।\\
ಮಹಾದಾರಿದ್ರ್ಯಯುಕ್ತೋ ಯೋ ಗುರುರ್ವಾ ಗುರುಭಕ್ತಿಮಾನ್ ॥೧೦೫॥

ಕಾರ್ತಿಕ್ಯಾಂ ಯಃ ಪಠೇದ್ರಾತ್ರೌ ಶತಮಷ್ಟೋತ್ತರಂ ಪಠೇತ್ ।\\
ಸುವರ್ಣಾಂಬರಧಾರೀ ಚ ಸುಗಂಧಪುಷ್ಪಚಂದನೈಃ ॥೧೦೬॥

ಪುಸ್ತಕಂ ಪೂಜಯಿತ್ವಾ ಚ ನೈವೇದ್ಯಾದಿಭಿರೇವ ಚ ।\\
ಮಹಾಮಾಯಾಂಕಿತೋ ಧೀರೋ ಪದ್ಮಮಾಲಾವಿಭೂಷಣಃ ॥೧೦೭॥

ಪ್ರಾತರಷ್ಟೋತ್ತರಂ ದೇವಿ ಪಠನ್ನಾಮ ಸಹಸ್ರಕಂ ।\\
ಚೈತ್ರಶುಕ್ಲೇ ಚ ಕೃಷ್ಣೇ ಚ ಕುಹುಸಂಕ್ರಾಂತಿವಾಸರೇ ॥೧೦೮॥

ಪಠಿತವ್ಯಂ ಪ್ರಯತ್ನೇನ ತ್ರೈಲೋಕ್ಯಂ ಮೋಹಯೇತ್ಕ್ಷಣಾತ್ ।\\
ಮುಕ್ತಾನಾಮ್ಮಾಲಯಾ ಯುಕ್ತೋ ಗುರುಭಕ್ತ್ಯಾ ಸಮನ್ವಿತಃ ॥೧೦೯॥

ರವಿವಾರೇ ಚ ಶುಕ್ರೇ ಚ ದ್ವಾದಶ್ಯಾಂ ಶ್ರಾದ್ಧವಾಸರೇ ।\\
ಬ್ರಾಹ್ಮಣಾನ್ಭೋಜಯಿತ್ವಾ ಚ ಪೂಜಯಿತ್ವಾ ವಿಧಾನತಃ ॥೧೧೦॥

ಪಠನ್ನಾಮಸಹಸ್ರಂ ಚ ತತಃ ಸಿದ್ಧಿಃ ಪ್ರಜಾಯತೇ ।\\
ಮಹಾನಿಶಾಯಾಂ ಸತತಂ ಗುರೌ ವಾ ಯಃ ಪಠೇತ್ಸದಾ ॥೧೧೧॥

ದೇಶಾಂತರಗತಾ ಲಕ್ಷ್ಮೀಃ ಸಮಾಯಾತಿ ನ ಸಂಶಯಃ ।\\
ತ್ರೈಲೋಕ್ಯೇ ಚ ಮಹಾಲಕ್ಷ್ಮೀಂ ಸುಂದರ್ಯಃ ಕಾಮಮೋಹಿತಾಃ ॥೧೧೨॥

ಮುಗ್ಧಾಃ ಸ್ವಯಂ ಸಮಾಯಾಂತಿ ಗೌರವಾಚ್ಚ ಭಜಂತಿ ತಾಃ ।\\
ರೋಗಾರ್ತ್ತೋ ಮುಚ್ಯತೇ ರೋಗಾತ್ಬದ್ಧೋ ಮುಚ್ಯೇತ ಬಂಧನಾತ್ ॥೧೧೩॥

ಗುರ್ವಿಣೀ ವಿಂದತೇ ಪುತ್ರಂ ಕನ್ಯಾ ವಿಂದತಿ ಸತ್ಪತಿಂ ।\\
ರಾಜಾನೋ ವಶತಾಂ ಯಾಂತಿ ಕಿಂಪುನಃ ಕ್ಷುದ್ರಮಾನುಷಾಃ ॥೧೧೪॥

ಸಹಸ್ರನಾಮಶ್ರವಣಾತ್ಪಠನಾತ್ಪೂಜನಾತ್ಪ್ರಿಯೇ ।\\
ಧಾರಣಾತ್ಸರ್ವಮಾಪ್ನೋತಿ ಗುರವೋ ನಾತ್ರ ಸಂಶಯಃ ॥೧೧೫॥

ಯಃ ಪಠೇದ್ಗುರುಭಕ್ತಃ ಸನ್ ಸ ಯಾತಿ ಪರಮಂ ಪದಂ ।\\
ಕೃಷ್ಣೇನೋಕ್ತಂ ಸಮಾಸಾದ್ಯ ಮಯಾ ಪ್ರೋಕ್ತಂ ಪುರಾ ಶಿವಂ ॥೧೧೬॥

ನಾರದಾಯ ಮಯಾ ಪ್ರೋಕ್ತಂ ನಾರದೇನ ಪ್ರಕಾಶಿತಂ ।\\
ಮಯಾ ತ್ವಯಿ ವರಾರೋಹೇ! ಪ್ರೋಕ್ತಮೇತತ್ಸುದುರ್ಲಭಂ ॥೧೧೭॥

ಶಠಾಯ ಪಾಪಿನೇ ಚೈವ ಲಂಪಟಾಯ ವಿಶೇಷತಃ ।\\
ನ ದಾತವ್ಯಂ ನ ದಾತವ್ಯಂ ನ ದಾತವ್ಯಂ ಕದಾಚನ ॥೧೧೮॥

ದೇಯಂ ದಾಂತಾಯ ಶಿಷ್ಯಾಯ ಗುರುಭಕ್ತಿರತಾಯ ಚ ।\\
ಗೋದಾನಂ ಬ್ರಹ್ಮಯಜ್ಞಶ್ಚ ವಾಜಪೇಯಶತಾನಿ ಚ ॥೧೧೯॥

ಅಶ್ವಮೇಧಸಹಸ್ರಸ್ಯ ಪಠತಶ್ಚ ಫಲಂ ಲಭೇತ್ ।\\
ಮೋಹನಂ ಸ್ತಂಭನಂ ಚೈವ ಮಾರಣೋಚ್ಚಾಟನಾದಿಕಂ ॥೧೨೦॥

ಯದ್ಯದ್ವಾಂಛತಿ ಚಿತ್ತೇ ತು ಪ್ರಾಪ್ನೋತಿ ಗುರುಭಕ್ತಿತಃ ।\\
ಏಕಾದಶ್ಯಾಂ ನರಃ ಸ್ನಾತ್ವಾ ಸುಗಂಧದ್ರವ್ಯಸಂಯುತಃ ॥೧೨೧॥

ಆಹಾರಂ ಬ್ರಾಹ್ಮಣೇ ದತ್ತ್ವಾ ದಕ್ಷಿಣಾಂ ಸ್ವರ್ಣಭೂಷಣಂ ।\\
ಆರಂಭಕರ್ತ್ತಾಸೌ ಸರ್ವಂ ಸರ್ವಮಾಪ್ನೋತಿ ಮಾನವಃ ॥೧೨೨॥

ಶತಾವರ್ತ್ತಂ ಸಹಸ್ರಂಚ ಯಃ ಪಠೇದ್ಗುರವೇ ಜನಾಃ ।\\
ಗುರುಸಹಸ್ರನಾಮಸ್ಯ ಪ್ರಸಾದಾತ್ಸರ್ವಮಾಪ್ನುಯಾತ್ ॥೧೨೩॥

ಯದ್ಗೇಹೇ ಪುಸ್ತಕಂ ದೇವಿ ಪೂಜಿತಂ ಚೈವ ತಿಷ್ಠತಿ॥
ನ ಮಾರೀ ನ ಚ ದುರ್ಭಿಕ್ಷಂ ನೋಪಸರ್ಗಂ ಭಯಂ ಕ್ವಚಿತ್ ॥೧೨೪॥

ಸರ್ಪಾದಿಭೂತಯಕ್ಷಾದ್ಯಾ ನಶ್ಯಂತೇ ನಾತ್ರ ಸಂಶಯಃ ।\\
ಶ್ರೀಗುರುರ್ವಾ ಮಹಾದೇವಿ! ವಸೇತ್ತಸ್ಯ ಗೃಹೇ ತಥಾ ॥೧೨೫॥

ಯತ್ರ ಗೇಹೇ ಸಹಸ್ರಂ ಚ ನಾಮ್ನಾಂ ತಿಷ್ಠತಿ ಪೂಜಿತಂ ।\\
ಶ್ರೀಗುರೋಃ ಕೃಪಯಾ ಶಿಷ್ಯೋ ಬ್ರಹ್ಮಸಾಯುಜ್ಯಮಾಪ್ನುಯಾತ್ ॥೧೨೬॥


\authorline{॥ಇತಿ ಶ್ರೀಹರಿಕೃಷ್ಣವಿನಿರ್ಮಿತೇ ಬೃಹಜ್ಜ್ಯೋತಿಷಾರ್ಣವೇಽಷ್ಟಮೇ ಧರ್ಮಸ್ಕಂಧೇ ಸಮ್ಮೋಹನತಂತ್ರೋಕ್ತಶ್ರೀಗುರುಸಹಸ್ರನಾಮಸ್ತೋತ್ರಂ॥}

%=============================================================================================
\section{ಶ್ರೀಗುರ್ವಷ್ಟೋತ್ತರಶತನಾಮಸ್ತೋತ್ರಂ}
\addcontentsline{toc}{section}{ಶ್ರೀಗುರ್ವಷ್ಟೋತ್ತರಶತನಾಮಸ್ತೋತ್ರಂ}

ಗುರುರ್ಗುಣವರೋ ಗೋಪ್ತಾ ಗೋಚರೋ ಗೋಪತಿಪ್ರಿಯಃ ।\\
ಗುಣೀ ಗುಣವತಾಂಶ್ರೇಷ್ಠೋ ಗುರೂಣಾಂಗುರುರವ್ಯಯಃ ॥೧॥

ಜೇತಾ ಜಯಂತೋ ಜಯದೋ ಜೀವೋಽನಂತೋ ಜಯಾವಹಃ ।\\
ಆಂಗೀರಸೋಽಧ್ವರಾಸಕ್ತೋ ವಿವಿಕ್ತೋಽಧ್ವರಕೃತ್ಪರಃ ॥೨॥

ವಾಚಸ್ಪತಿರ್ ವಶೀ ವಶ್ಯೋ ವರಿಷ್ಠೋ ವಾಗ್ವಿಚಕ್ಷಣಃ ।\\
ಚಿತ್ತಶುದ್ಧಿಕರಃ ಶ್ರೀಮಾನ್ ಚೈತ್ರಃ ಚಿತ್ರಶಿಖಂಡಿಜಃ ॥೩॥

ಬೃಹದ್ರಥೋ ಬೃಹದ್ಭಾನುರ್ಬೃಹಸ್ಪತಿರಭೀಷ್ಟದಃ ।\\
ಸುರಾಚಾರ್ಯಃ ಸುರಾರಾಧ್ಯಃ ಸುರಕಾರ್ಯಹಿತಂಕರಃ ॥೪॥

ಗೀರ್ವಾಣಪೋಷಕೋ ಧನ್ಯೋ ಗೀಷ್ಪತಿರ್ಗಿರಿಶೋಽನಘಃ ।\\
ಧೀವರೋ ಧಿಷಣೋ ದಿವ್ಯಭೂಷಣೋ ದೇವಪೂಜಿತಃ ॥೫॥

ಧನುರ್ಧರೋ ದೈತ್ಯಹಂತಾ ದಯಾಸಾರೋ ದಯಾಕರಃ ।\\
ದಾರಿದ್ರ್ಯನಾಶಕೋ ಧನ್ಯೋ ದಕ್ಷಿಣಾಯನಸಂಭವಃ ॥೬॥

ಧನುರ್ಮೀನಾಧಿಪೋ ದೇವೋ ಧನುರ್ಬಾಣಧರೋ ಹರಿಃ ।\\
ಆಂಗೀರಸಾಬ್ದಸಂಜಾತೋ ಆಂಗೀರಸಕುಲಸಂಭವಃ ॥೭ ॥

ಸಿಂಧುದೇಶಾಧಿಪೋ ಧೀಮಾನ್ ಸ್ವರ್ಣವರ್ಣಃ ಚತುರ್ಭುಜಃ ।\\
ಹೇಮಾಂಗದೋ ಹೇಮವಪುರ್ಹೇಮಭೂಷಣಭೂಷಿತಃ ॥೮॥

ಪುಷ್ಯನಾಥಃ ಪುಷ್ಯರಾಗಮಣಿಮಂಡಲಮಂಡಿತಃ ।\\
ಕಾಶಪುಷ್ಪಸಮಾನಾಭಃ ಕಲಿದೋಷನಿವಾರಕಃ ॥೯॥

ಇಂದ್ರಾದಿದೇವೋದೇವೇಷೋ ದೇವತಾಭೀಷ್ಟದಾಯಕಃ ।\\
ಅಸಮಾನಬಲಃ ಸತ್ತ್ವಗುಣಸಂಪದ್ವಿಭಾಸುರಃ ॥೧೦॥

ಭೂಸುರಾಭೀಷ್ಟದೋ ಭೂರಿಯಶಃ ಪುಣ್ಯವಿವರ್ಧನಃ ।\\
ಧರ್ಮರೂಪೋ ಧನಾಧ್ಯಕ್ಷೋ ಧನದೋ ಧರ್ಮಪಾಲನಃ ॥೧೧॥

ಸರ್ವವೇದಾರ್ಥತತ್ತ್ವಜ್ಞಃ ಸರ್ವಾಪದ್ವಿನಿವಾರಕಃ ।\\
ಸರ್ವಪಾಪಪ್ರಶಮನಃ ಸ್ವಮತಾನುಗತಾಮರಃ ॥೧೨॥

ಋಗ್ವೇದಪಾರಗೋ ಋಕ್ಷರಾಶಿಮಾರ್ಗಪ್ರಚಾರಕಃ ।\\
ಸದಾನಂದಃ ಸತ್ಯಸಂಧಃ ಸತ್ಯಸಂಕಲ್ಪಮಾನಸಃ ॥೧೩॥

ಸರ್ವಾಗಮಜ್ಞಃ ಸರ್ವಜ್ಞಃ ಸರ್ವವೇದಾಂತವಿದ್ವರಃ ।\\
ಬ್ರಹ್ಮಪುತ್ರೋ ಬ್ರಾಹ್ಮಣೇಶೋ ಬ್ರಹ್ಮವಿದ್ಯಾವಿಶಾರದಃ ॥೧೪॥

ಸಮಾನಾಧಿಕನಿರ್ಮುಕ್ತಃ ಸರ್ವಲೋಕವಶಂವದಃ ।\\
ಸಸುರಾಸುರಗಂಧರ್ವವಂದಿತಃ ಸತ್ಯಭಾಷಣಃ ॥೧೫॥

ನಮಃ ಸುರೇಂದ್ರವಂದ್ಯಾಯ ದೇವಾಚಾರ್ಯಾಯ ತೇ ನಮಃ ।\\
ನಮಸ್ತೇಽನಂತಸಾಮರ್ಥ್ಯ ವೇದಸಿದ್ಧಾಂತಪಾರಗಃ ॥೧೬॥

ಸದಾನಂದ ನಮಸ್ತೇಸ್ತು ನಮಃ ಪೀಡಾಹರಾಯ ಚ ।\\
ನಮೋ ವಾಚಸ್ಪತೇ ತುಭ್ಯಂ ನಮಸ್ತೇ ಪೀತವಾಸಸೇ ॥೧೭॥

ನಮೋಽದ್ವಿತೀಯರೂಪಾಯ ಲಂಬಕೂರ್ಚಾಯ ತೇ ನಮಃ ।\\
ನಮಃ ಪ್ರಕೃಷ್ಟನೇತ್ರಾಯ ವಿಪ್ರಾಣಾಂಪತಯೇ ನಮಃ ॥೧೮॥

ನಮೋ ಭಾರ್ಗವಷಿಷ್ಯಾಯ ವಿಪನ್ನಹಿತಕಾರಿಣೇ ।\\
ನಮಸ್ತೇ ಸುರಸೈನ್ಯಾನಾಂವಿಪತ್ಛಿದ್ರಾನಕೇತವೇ ॥೧೯॥

ಬೃಹಸ್ಪತಿಃ ಸುರಾಚಾರ್ಯೋ ದಯಾವಾನ್ ಶುಭಲಕ್ಷಣಃ ।\\
ಲೋಕತ್ರಯಗುರುಃ ಶ್ರೀಮಾನ್ ಸರ್ವಗಃ ಸರ್ವತೋವಿಭುಃ ॥೨೦॥

ಸರ್ವೇಶಃ ಸರ್ವದಾತುಷ್ಟಃ ಸರ್ವದಃ ಸರ್ವಪೂಜಿತಃ ।\\
ಅಕ್ರೋಧನೋ ಮುನಿಶ್ರೇಷ್ಠೋ ದೀಪ್ತಿಕರ್ತಾ ಜಗತ್ಪಿತಾ ॥೨೧॥

ವಿಶ್ವಾತ್ಮಾ ವಿಶ್ವಕರ್ತಾ ಚ ವಿಶ್ವಯೋನಿರಯೋನಿಜಃ ।\\
ಭೂರ್ಭುವೋಧನದಾಸಾಜಭಕ್ತಾಜೀವೋ ಮಹಾಬಲಃ ॥೨೨॥

ಬೃಹಸ್ಪತಿಃ ಕಾಷ್ಯಪೇಯೋ ದಯಾವಾನ್ ಷುಭಲಕ್ಷಣಃ ।\\
ಅಭೀಷ್ಟಫಲದಃ ಶ್ರೀಮಾನ್ ಸುಭದ್ಗರ ನಮೋಸ್ತು ತೇ ॥೨೩॥

ಬೃಹಸ್ಪತಿಸ್ಸುರಾಚಾರ್ಯೋ ದೇವಾಸುರಸುಪೂಜಿತಃ ।\\
ಆಚಾರ್ಯೋದಾನವಾರಿಷ್ಟ ಸುರಮಂತ್ರೀ ಪುರೋಹಿತಃ ॥೨೪॥

ಕಾಲಜ್ಞಃ ಕಾಲಋಗ್ವೇತ್ತಾ ಚಿತ್ತದಶ್ಚ ಪ್ರಜಾಪತಿಃ ।\\
ವಿಷ್ಣುಃ ಕೃಷ್ಣಃ ಸದಾಸೂಕ್ಷ್ಮಃ ಪ್ರತಿದೇವೋಜ್ಜ್ವಲಗ್ರಹಃ ॥೨೫॥

\authorline{ಇತಿ ಗುರ್ವಷ್ಟೋತ್ತರಶತನಾಮಸ್ತೋತ್ರಂ ಸಂಪೂರ್ಣಂ ॥}
%====================
\section{ ಶ್ರೀ ಲಲಿತಾಸಹಸ್ರನಾಮ ಸ್ತೋತ್ರಂ }
\addcontentsline{toc}{section}{ ಶ್ರೀ ಲಲಿತಾಸಹಸ್ರನಾಮ ಸ್ತೋತ್ರಂ }
ಅಸ್ಯ ಶ್ರೀಲಲಿತಾಸಹಸ್ರನಾಮಸ್ತೋತ್ರಮಾಲಾ ಮಂತ್ರಸ್ಯ~। ವಶಿನ್ಯಾದಿವಾಗ್ದೇವತಾ ಋಷಯಃ~। ಅನುಷ್ಟುಪ್ ಛಂದಃ~। ಶ್ರೀಲಲಿತಾಪರಮೇಶ್ವರೀ ದೇವತಾ। ಶ್ರೀಮದ್ವಾಗ್ಭವಕೂಟೇತಿ ಬೀಜಂ। ಮಧ್ಯಕೂಟೇತಿ ಶಕ್ತಿಃ। ಶಕ್ತಿಕೂಟೇತಿ ಕೀಲಕಂ। ವಾಕ್ಸಿದ್ಧ್ಯರ್ಥೇ ಜಪೇ ವಿನಿಯೋಗಃ~।\\

\dhyana{ಸಿಂದೂರಾರುಣ ವಿಗ್ರಹಾಂ ತ್ರಿನಯನಾಂ ಮಾಣಿಕ್ಯಮೌಲಿ ಸ್ಫುರತ್\\
ತಾರಾ ನಾಯಕ ಶೇಖರಾಂ ಸ್ಮಿತಮುಖೀಮಾಪೀನವಕ್ಷೋರುಹಾಂ~।\\
ಪಾಣಿಭ್ಯಾಮಲಿಪೂರ್ಣ ರತ್ನ ಚಷಕಂ ರಕ್ತೋತ್ಪಲಂ ಬಿಭ್ರತೀಂ\\
ಸೌಮ್ಯಾಂ ರತ್ನ ಘಟಸ್ಥ ರಕ್ತಚರಣಾಂ ಧ್ಯಾಯೇತ್ ಪರಾಮಂಬಿಕಾಂ ॥

ಅರುಣಾಂ ಕರುಣಾ ತರಂಗಿತಾಕ್ಷೀಂ ಧೃತ ಪಾಶಾಂಕುಶ ಪುಷ್ಪ ಬಾಣಚಾಪಾಂ~।\\
ಅಣಿಮಾದಿಭಿರಾವೃತಾಂ ಮಯೂಖೈರಹಮಿತ್ಯೇವ ವಿಭಾವಯೇ ಭವಾನೀಂ ॥

ಧ್ಯಾಯೇತ್ ಪದ್ಮಾಸನಸ್ಥಾಂ ವಿಕಸಿತವದನಾಂ ಪದ್ಮಪತ್ರಾಯತಾಕ್ಷೀಂ\\
ಹೇಮಾಭಾಂ ಪೀತವಸ್ತ್ರಾಂ ಕರಕಲಿತಲಸದ್ಧೇಮಪದ್ಮಾಂ ವರಾಂಗೀಂ~।\\
ಸರ್ವಾಲಂಕಾರಯುಕ್ತಾಂ ಸತತಮಭಯದಾಂ ಭಕ್ತನಮ್ರಾಂ ಭವಾನೀಂ\\
ಶ್ರೀವಿದ್ಯಾಂ ಶಾಂತಮೂರ್ತಿಂ ಸಕಲ ಸುರನುತಾಂ ಸರ್ವ ಸಂಪತ್ಪ್ರದಾತ್ರೀಂ ॥

ಸಕುಂಕುಮವಿಲೇಪನಾಮಲಿಕಚುಂಬಿಕಸ್ತೂರಿಕಾಂ\\
ಸಮಂದಹಸಿತೇಕ್ಷಣಾಂ ಸಶರಚಾಪಪಾಶಾಂಕುಶಾಂ~।\\
ಅಶೇಷಜನಮೋಹಿನೀಂ ಅರುಣಮಾಲ್ಯಭೂಷಾಂಬರಾಂ\\
ಜಪಾಕುಸುಮಭಾಸುರಾಂ ಜಪವಿಧೌ ಸ್ಮರೇದಂಬಿಕಾಂ ॥}

{\bfseries ಓಂ ಐಂಹ್ರೀಂಶ್ರೀಂ}\\
ಶ್ರೀಮಾತಾ ಶ್ರೀಮಹಾರಾಜ್ಞೀ ಶ್ರೀಮತ್ಸಿಂಹಾಸನೇಶ್ವರೀ~।\\
ಚಿದಗ್ನಿ-ಕುಂಡ-ಸಂಭೂತಾ ದೇವಕಾರ್ಯ-ಸಮುದ್ಯತಾ ॥೧॥

ಉದ್ಯದ್ಭಾನು-ಸಹಸ್ರಾಭಾ ಚತುರ್ಬಾಹು-ಸಮನ್ವಿತಾ~।\\
ರಾಗಸ್ವರೂಪ-ಪಾಶಾಢ್ಯಾ ಕ್ರೋಧಾಕಾರಾಂಕುಶೋಜ್ಜ್ವಲಾ ॥೨॥

ಮನೋರೂಪೇಕ್ಷು-ಕೋದಂಡಾ ಪಂಚತನ್ಮಾತ್ರ-ಸಾಯಕಾ~।\\
ನಿಜಾರುಣ-ಪ್ರಭಾಪೂರ-ಮಜ್ಜದ್‍ಬ್ರಹ್ಮಾಂಡ-ಮಂಡಲಾ ॥೩॥

ಚಂಪಕಾಶೋಕ-ಪುನ್ನಾಗ-ಸೌಗಂಧಿಕ-ಲಸತ್ಕಚಾ~।\\
ಕುರುವಿಂದಮಣಿ-ಶ್ರೇಣೀ-ಕನತ್ಕೋಟೀರ-ಮಂಡಿತಾ ॥೪॥

ಅಷ್ಟಮೀಚಂದ್ರ-ವಿಭ್ರಾಜ-ದಲಿಕಸ್ಥಲ-ಶೋಭಿತಾ~।\\
ಮುಖಚಂದ್ರ-ಕಲಂಕಾಭ-ಮೃಗನಾಭಿ-ವಿಶೇಷಕಾ ॥೫॥

ವದನಸ್ಮರ-ಮಾಂಗಲ್ಯ-ಗೃಹತೋರಣ-ಚಿಲ್ಲಿಕಾ~।\\
ವಕ್ತ್ರಲಕ್ಷ್ಮೀ-ಪರೀವಾಹ-ಚಲನ್ಮೀನಾಭ-ಲೋಚನಾ ॥೬॥

ನವಚಂಪಕ-ಪುಷ್ಪಾಭ-ನಾಸಾದಂಡ-ವಿರಾಜಿತಾ~।\\
ತಾರಾಕಾಂತಿ-ತಿರಸ್ಕಾರಿ-ನಾಸಾಭರಣ-ಭಾಸುರಾ ॥೭॥

ಕದಂಬಮಂಜರೀ-ಕ್ಲೃಪ್ತ-ಕರ್ಣಪೂರ-ಮನೋಹರಾ~।\\
ತಾಟಂಕ-ಯುಗಲೀ-ಭೂತ-ತಪನೋಡುಪ-ಮಂಡಲಾ ॥೮॥

ಪದ್ಮರಾಗಶಿಲಾದರ್ಶ-ಪರಿಭಾವಿ-ಕಪೋಲಭೂಃ~।\\
ನವವಿದ್ರುಮ-ಬಿಂಬಶ್ರೀ-ನ್ಯಕ್ಕಾರಿ-ರದನಚ್ಛದಾ ॥೯॥

ಶುದ್ಧವಿದ್ಯಾಂಕುರಾಕಾರ-ದ್ವಿಜಪಂಕ್ತಿ-ದ್ವಯೋಜ್ಜ್ವಲಾ~।\\
ಕರ್ಪೂರವೀಟಿಕಾಮೋದ-ಸಮಾಕರ್ಷದ್ದಿಗಂತರಾ ॥೧೦॥

ನಿಜ-ಸಲ್ಲಾಪ-ಮಾಧುರ್ಯ-ವಿನಿರ್ಭರ್ತ್ಸಿತ-ಕಚ್ಛಪೀ~।\\
ಮಂದಸ್ಮಿತ-ಪ್ರಭಾಪೂರ-ಮಜ್ಜತ್ಕಾಮೇಶ-ಮಾನಸಾ ॥೧೧॥

ಅನಾಕಲಿತ-ಸಾದೃಶ್ಯ-ಚುಬುಕಶ್ರೀ-ವಿರಾಜಿತಾ~।\\
ಕಾಮೇಶ-ಬದ್ಧ-ಮಾಂಗಲ್ಯ-ಸೂತ್ರ-ಶೋಭಿತ-ಕಂಧರಾ ॥೧೨॥

ಕನಕಾಂಗದ-ಕೇಯೂರ-ಕಮನೀಯ-ಭುಜಾನ್ವಿತಾ~।\\
ರತ್ನಗ್ರೈವೇಯ-ಚಿಂತಾಕ-ಲೋಲ-ಮುಕ್ತಾ-ಫಲಾನ್ವಿತಾ ॥೧೩॥

ಕಾಮೇಶ್ವರ-ಪ್ರೇಮರತ್ನ-ಮಣಿ-ಪ್ರತಿಪಣ-ಸ್ತನೀ~।\\
ನಾಭ್ಯಾಲವಾಲ-ರೋಮಾಲಿ-ಲತಾ-ಫಲ-ಕುಚದ್ವಯೀ ॥೧೪॥

ಲಕ್ಷ್ಯರೋಮ-ಲತಾಧಾರತಾ-ಸಮುನ್ನೇಯ-ಮಧ್ಯಮಾ~।\\
ಸ್ತನಭಾರ-ದಲನ್ಮಧ್ಯ-ಪಟ್ಟಬಂಧ-ವಲಿತ್ರಯಾ ॥೧೫॥

ಅರುಣಾರುಣಕೌಸುಂಭ-ವಸ್ತ್ರ-ಭಾಸ್ವತ್ಕಟೀತಟೀ~।\\
ರತ್ನ-ಕಿಂಕಿಣಿಕಾ-ರಮ್ಯ-ರಶನಾ-ದಾಮ-ಭೂಷಿತಾ ॥೧೬॥

ಕಾಮೇಶ-ಜ್ಞಾತ-ಸೌಭಾಗ್ಯ-ಮಾರ್ದವೋರು-ದ್ವಯಾನ್ವಿತಾ~।\\
ಮಾಣಿಕ್ಯ-ಮುಕುಟಾಕಾರ-ಜಾನುದ್ವಯ-ವಿರಾಜಿತಾ ॥೧೭॥

ಇಂದ್ರಗೋಪ-ಪರಿಕ್ಷಿಪ್ತಸ್ಮರತೂಣಾಭ-ಜಂಘಿಕಾ~।\\
ಗೂಢಗುಲ್ಫಾ ಕೂರ್ಮಪೃಷ್ಠ-ಜಯಿಷ್ಣು-ಪ್ರಪದಾನ್ವಿತಾ ॥೧೮॥

ನಖ-ದೀಧಿತಿ-ಸಂಛನ್ನ-ನಮಜ್ಜನ-ತಮೋಗುಣಾ~।\\
ಪದದ್ವಯ-ಪ್ರಭಾಜಾಲ-ಪರಾಕೃತ-ಸರೋರುಹಾ ॥೧೯॥

ಶಿಂಜಾನ-ಮಣಿಮಂಜೀರ-ಮಂಡಿತ-ಶ್ರೀ-ಪದಾಂಬುಜಾ~।\\
ಮರಾಲೀ-ಮಂದಗಮನಾ ಮಹಾಲಾವಣ್ಯ-ಶೇವಧಿಃ ॥೨೦॥

ಸರ್ವಾರುಣಾಽನವದ್ಯಾಂಗೀ ಸರ್ವಾಭರಣ-ಭೂಷಿತಾ~।\\
ಶಿವ-ಕಾಮೇಶ್ವರಾಂಕಸ್ಥಾ ಶಿವಾ ಸ್ವಾಧೀನ-ವಲ್ಲಭಾ ॥೨೧॥

ಸುಮೇರು-ಮಧ್ಯ-ಶೃಂಗಸ್ಥಾ ಶ್ರೀಮನ್ನಗರ-ನಾಯಿಕಾ~।\\
ಚಿಂತಾಮಣಿ-ಗೃಹಾಂತಸ್ಥಾ ಪಂಚ-ಬ್ರಹ್ಮಾಸನ-ಸ್ಥಿತಾ ॥೨೨॥

ಮಹಾಪದ್ಮಾಟವೀ-ಸಂಸ್ಥಾ ಕದಂಬವನ-ವಾಸಿನೀ~।\\
ಸುಧಾಸಾಗರ-ಮಧ್ಯಸ್ಥಾ ಕಾಮಾಕ್ಷೀ ಕಾಮದಾಯಿನೀ ॥೨೩॥

ದೇವರ್ಷಿ-ಗಣ-ಸಂಘಾತ-ಸ್ತೂಯಮಾನಾತ್ಮ-ವೈಭವಾ~।\\
ಭಂಡಾಸುರ-ವಧೋದ್ಯುಕ್ತ-ಶಕ್ತಿಸೇನಾ-ಸಮನ್ವಿತಾ ॥೨೪॥

ಸಂಪತ್ಕರೀ-ಸಮಾರೂಢ-ಸಿಂಧುರ-ವ್ರಜ-ಸೇವಿತಾ~।\\
ಅಶ್ವಾರೂಢಾಧಿಷ್ಠಿತಾಶ್ವ-ಕೋಟಿ-ಕೋಟಿಭಿರಾವೃತಾ ॥೨೫॥

ಚಕ್ರರಾಜ-ರಥಾರೂಢ-ಸರ್ವಾಯುಧ-ಪರಿಷ್ಕೃತಾ~।\\
ಗೇಯಚಕ್ರ-ರಥಾರೂಢ-ಮಂತ್ರಿಣೀ-ಪರಿಸೇವಿತಾ ॥೨೬॥

ಕಿರಿಚಕ್ರ-ರಥಾರೂಢ-ದಂಡನಾಥಾ-ಪುರಸ್ಕೃತಾ~।\\
ಜ್ವಾಲಾ-ಮಾಲಿನಿಕಾಕ್ಷಿಪ್ತ-ವಹ್ನಿಪ್ರಾಕಾರ-ಮಧ್ಯಗಾ ॥೨೭॥

ಭಂಡಸೈನ್ಯ-ವಧೋದ್ಯುಕ್ತ-ಶಕ್ತಿ-ವಿಕ್ರಮ-ಹರ್ಷಿತಾ~।\\
ನಿತ್ಯಾ-ಪರಾಕ್ರಮಾಟೋಪ-ನಿರೀಕ್ಷಣ-ಸಮುತ್ಸುಕಾ ॥೨೮॥

ಭಂಡಪುತ್ರ-ವಧೋದ್ಯುಕ್ತ-ಬಾಲಾ-ವಿಕ್ರಮ-ನಂದಿತಾ~।\\
ಮಂತ್ರಿಣ್ಯಂಬಾ-ವಿರಚಿತ-ವಿಷಂಗ-ವಧ-ತೋಷಿತಾ ॥೨೯॥

ವಿಶುಕ್ರ-ಪ್ರಾಣಹರಣ-ವಾರಾಹೀ-ವೀರ್ಯ-ನಂದಿತಾ~।\\
ಕಾಮೇಶ್ವರ-ಮುಖಾಲೋಕ-ಕಲ್ಪಿತ-ಶ್ರೀಗಣೇಶ್ವರಾ ॥೩೦॥

ಮಹಾಗಣೇಶ-ನಿರ್ಭಿನ್ನ-ವಿಘ್ನಯಂತ್ರ-ಪ್ರಹರ್ಷಿತಾ~।\\
ಭಂಡಾಸುರೇಂದ್ರ-ನಿರ್ಮುಕ್ತ-ಶಸ್ತ್ರ-ಪ್ರತ್ಯಸ್ತ್ರ-ವರ್ಷಿಣೀ ॥೩೧॥

ಕರಾಂಗುಲಿ-ನಖೋತ್ಪನ್ನ-ನಾರಾಯಣ-ದಶಾಕೃತಿಃ~।\\
ಮಹಾ-ಪಾಶುಪತಾಸ್ತ್ರಾಗ್ನಿ-ನಿರ್ದಗ್ಧಾಸುರ-ಸೈನಿಕಾ ॥೩೨॥

ಕಾಮೇಶ್ವರಾಸ್ತ್ರ-ನಿರ್ದಗ್ಧ-ಸಭಂಡಾಸುರ-ಶೂನ್ಯಕಾ~।\\
ಬ್ರಹ್ಮೋಪೇಂದ್ರ-ಮಹೇಂದ್ರಾದಿ-ದೇವ-ಸಂಸ್ತುತ-ವೈಭವಾ ॥೩೩॥

ಹರ-ನೇತ್ರಾಗ್ನಿ-ಸಂದಗ್ಧ-ಕಾಮ-ಸಂಜೀವನೌಷಧಿಃ~।\\
ಶ್ರೀಮದ್ವಾಗ್ಭವ-ಕೂಟೈಕ-ಸ್ವರೂಪ-ಮುಖ-ಪಂಕಜಾ ॥೩೪॥

ಕಂಠಾಧಃ-ಕಟಿ-ಪರ್ಯಂತ-ಮಧ್ಯಕೂಟ-ಸ್ವರೂಪಿಣೀ।\\
ಶಕ್ತಿ-ಕೂಟೈಕತಾಪನ್ನ-ಕಟ್ಯಧೋಭಾಗ ಧಾರಿಣೀ ॥೩೫॥

ಮೂಲ-ಮಂತ್ರಾತ್ಮಿಕಾ ಮೂಲಕೂಟತ್ರಯ-ಕಲೇಬರಾ~।\\
ಕುಲಾಮೃತೈಕ-ರಸಿಕಾ ಕುಲಸಂಕೇತ-ಪಾಲಿನೀ ॥೩೬॥

ಕುಲಾಂಗನಾ ಕುಲಾಂತಸ್ಥಾ ಕೌಲಿನೀ ಕುಲಯೋಗಿನೀ~।\\
ಅಕುಲಾ ಸಮಯಾಂತಸ್ಥಾ ಸಮಯಾಚಾರ-ತತ್ಪರಾ ॥೩೭॥

ಮೂಲಾಧಾರೈಕ-ನಿಲಯಾ ಬ್ರಹ್ಮಗ್ರಂಥಿ-ವಿಭೇದಿನೀ \as{(೧೦೦)}~।\\
ಮಣಿ-ಪೂರಾಂತರುದಿತಾ ವಿಷ್ಣುಗ್ರಂಥಿ-ವಿಭೇದಿನೀ ॥೩೮॥

ಆಜ್ಞಾ-ಚಕ್ರಾಂತರಾಲಸ್ಥಾ ರುದ್ರಗ್ರಂಥಿ-ವಿಭೇದಿನೀ~।\\
ಸಹಸ್ರಾರಾಂಬುಜಾರೂಢಾ ಸುಧಾ-ಸಾರಾಭಿವರ್ಷಿಣೀ ॥೩೯॥

ತಡಿಲ್ಲತಾ-ಸಮರುಚಿಃ ಷಟ್‍ಚಕ್ರೋಪರಿ-ಸಂಸ್ಥಿತಾ~।\\
ಮಹಾಸಕ್ತಿಃ ಕುಂಡಲಿನೀ ಬಿಸತಂತು-ತನೀಯಸೀ ॥೪೦॥

ಭವಾನೀ ಭಾವನಾಗಮ್ಯಾ ಭವಾರಣ್ಯ-ಕುಠಾರಿಕಾ~।\\
ಭದ್ರಪ್ರಿಯಾ ಭದ್ರಮೂರ್ತಿರ್ಭಕ್ತ-ಸೌಭಾಗ್ಯದಾಯಿನೀ ॥೪೧॥

ಭಕ್ತಿಪ್ರಿಯಾ ಭಕ್ತಿಗಮ್ಯಾ ಭಕ್ತಿವಶ್ಯಾ ಭಯಾಪಹಾ~।\\
ಶಾಂಭವೀ ಶಾರದಾರಾಧ್ಯಾ ಶರ್ವಾಣೀ ಶರ್ಮದಾಯಿನೀ ॥೪೨॥

ಶಾಂಕರೀ ಶ್ರೀಕರೀ ಸಾಧ್ವೀ ಶರಚ್ಚಂದ್ರ-ನಿಭಾನನಾ~।\\
ಶಾತೋದರೀ ಶಾಂತಿಮತೀ ನಿರಾಧಾರಾ ನಿರಂಜನಾ ॥೪೩॥

ನಿರ್ಲೇಪಾ ನಿರ್ಮಲಾ ನಿತ್ಯಾ ನಿರಾಕಾರಾ ನಿರಾಕುಲಾ~।\\
ನಿರ್ಗುಣಾ ನಿಷ್ಕಲಾ ಶಾಂತಾ ನಿಷ್ಕಾಮಾ ನಿರುಪಪ್ಲವಾ ॥೪೪॥

ನಿತ್ಯಮುಕ್ತಾ ನಿರ್ವಿಕಾರಾ ನಿಷ್ಪ್ರಪಂಚಾ ನಿರಾಶ್ರಯಾ~।\\
ನಿತ್ಯಶುದ್ಧಾ ನಿತ್ಯಬುದ್ಧಾ ನಿರವದ್ಯಾ ನಿರಂತರಾ ॥೪೫॥

ನಿಷ್ಕಾರಣಾ ನಿಷ್ಕಲಂಕಾ ನಿರುಪಾಧಿರ್ನಿರೀಶ್ವರಾ~।\\
ನೀರಾಗಾ ರಾಗಮಥನೀ ನಿರ್ಮದಾ ಮದನಾಶಿನೀ ॥೪೬॥

ನಿಶ್ಚಿಂತಾ ನಿರಹಂಕಾರಾ ನಿರ್ಮೋಹಾ ಮೋಹನಾಶಿನೀ~।\\
ನಿರ್ಮಮಾ ಮಮತಾಹಂತ್ರೀ ನಿಷ್ಪಾಪಾ ಪಾಪನಾಶಿನೀ ॥೪೭॥

ನಿಷ್ಕ್ರೋಧಾ ಕ್ರೋಧಶಮನೀ ನಿರ್ಲೋಭಾ ಲೋಭನಾಶಿನೀ~।\\
ನಿಃಸಂಶಯಾ ಸಂಶಯಘ್ನೀ ನಿರ್ಭವಾ ಭವನಾಶಿನೀ ॥೪೮॥

ನಿರ್ವಿಕಲ್ಪಾ ನಿರಾಬಾಧಾ ನಿರ್ಭೇದಾ ಭೇದನಾಶಿನೀ~।\\
ನಿರ್ನಾಶಾ ಮೃತ್ಯುಮಥಿನೀ ನಿಷ್ಕ್ರಿಯಾ ನಿಷ್ಪರಿಗ್ರಹಾ~।೪೯॥

ನಿಸ್ತುಲಾ ನೀಲಚಿಕುರಾ ನಿರಪಾಯಾ ನಿರತ್ಯಯಾ~।\\
ದುರ್ಲಭಾ ದುರ್ಗಮಾ ದುರ್ಗಾ ದುಃಖಹಂತ್ರೀ ಸುಖಪ್ರದಾ ॥೫೦॥

ದುಷ್ಟದೂರಾ ದುರಾಚಾರ-ಶಮನೀ ದೋಷವರ್ಜಿತಾ~।\\
ಸರ್ವಜ್ಞಾ ಸಾಂದ್ರಕರುಣಾ ಸಮಾನಾಧಿಕ-ವರ್ಜಿತಾ ॥೫೧॥

ಸರ್ವಶಕ್ತಿಮಯೀ ಸರ್ವ-ಮಂಗಲಾ \as{(೨೦೦)} ಸದ್ಗತಿಪ್ರದಾ~।\\
ಸರ್ವೇಶ್ವರೀ ಸರ್ವಮಯೀ ಸರ್ವಮಂತ್ರ-ಸ್ವರೂಪಿಣೀ ॥೫೨॥

ಸರ್ವ-ಯಂತ್ರಾತ್ಮಿಕಾ ಸರ್ವ-ತಂತ್ರರೂಪಾ ಮನೋನ್ಮನೀ~।\\
ಮಾಹೇಶ್ವರೀ ಮಹಾದೇವೀ ಮಹಾಲಕ್ಷ್ಮೀರ್ಮೃಡಪ್ರಿಯಾ ॥೫೩॥

ಮಹಾರೂಪಾ ಮಹಾಪೂಜ್ಯಾ ಮಹಾಪಾತಕ-ನಾಶಿನೀ~।\\
ಮಹಾಮಾಯಾ ಮಹಾಸತ್ತ್ವಾ ಮಹಾಶಕ್ತಿರ್ಮಹಾರತಿಃ ॥೫೪॥

ಮಹಾಭೋಗಾ ಮಹೈಶ್ವರ್ಯಾ ಮಹಾವೀರ್ಯಾ ಮಹಾಬಲಾ~।\\
ಮಹಾಬುದ್ಧಿರ್ಮಹಾಸಿದ್ಧಿರ್ಮಹಾಯೋಗೇಶ್ವರೇಶ್ವರೀ ॥೫೫॥

ಮಹಾತಂತ್ರಾ ಮಹಾಮಂತ್ರಾ ಮಹಾಯಂತ್ರಾ ಮಹಾಸನಾ~।\\
ಮಹಾಯಾಗ-ಕ್ರಮಾರಾಧ್ಯಾ ಮಹಾಭೈರವ-ಪೂಜಿತಾ ॥೫೬॥

ಮಹೇಶ್ವರ-ಮಹಾಕಲ್ಪ-ಮಹಾತಾಂಡವ-ಸಾಕ್ಷಿಣೀ~।\\
ಮಹಾಕಾಮೇಶ-ಮಹಿಷೀ ಮಹಾತ್ರಿಪುರ-ಸುಂದರೀ ॥೫೭॥

ಚತುಃಷಷ್ಟ್ಯುಪಚಾರಾಢ್ಯಾ ಚತುಃಷಷ್ಟಿಕಲಾಮಯೀ~।\\
ಮಹಾಚತುಃ-ಷಷ್ಟಿಕೋಟಿ-ಯೋಗಿನೀ-ಗಣಸೇವಿತಾ ॥೫೮॥

ಮನುವಿದ್ಯಾ ಚಂದ್ರವಿದ್ಯಾ ಚಂದ್ರಮಂಡಲ-ಮಧ್ಯಗಾ~।\\
ಚಾರುರೂಪಾ ಚಾರುಹಾಸಾ ಚಾರುಚಂದ್ರ-ಕಲಾಧರಾ ॥೫೯॥

ಚರಾಚರ-ಜಗನ್ನಾಥಾ ಚಕ್ರರಾಜ-ನಿಕೇತನಾ~।\\
ಪಾರ್ವತೀ ಪದ್ಮನಯನಾ ಪದ್ಮರಾಗ-ಸಮಪ್ರಭಾ ॥೬೦॥

ಪಂಚ-ಪ್ರೇತಾಸನಾಸೀನಾ ಪಂಚಬ್ರಹ್ಮ-ಸ್ವರೂಪಿಣೀ~।\\
ಚಿನ್ಮಯೀ ಪರಮಾನಂದಾ ವಿಜ್ಞಾನ-ಘನರೂಪಿಣೀ ॥೬೧॥

ಧ್ಯಾನ-ಧ್ಯಾತೃ-ಧ್ಯೇಯರೂಪಾ ಧರ್ಮಾಧರ್ಮ-ವಿವರ್ಜಿತಾ~।\\
ವಿಶ್ವರೂಪಾ ಜಾಗರಿಣೀ ಸ್ವಪಂತೀ ತೈಜಸಾತ್ಮಿಕಾ ॥೬೨॥

ಸುಪ್ತಾ ಪ್ರಾಜ್ಞಾತ್ಮಿಕಾ ತುರ್ಯಾ ಸರ್ವಾವಸ್ಥಾ-ವಿವರ್ಜಿತಾ~।\\
ಸೃಷ್ಟಿಕರ್ತ್ರೀ ಬ್ರಹ್ಮರೂಪಾ ಗೋಪ್ತ್ರೀ ಗೋವಿಂದರೂಪಿಣೀ ॥೬೩॥

ಸಂಹಾರಿಣೀ ರುದ್ರರೂಪಾ ತಿರೋಧಾನ-ಕರೀಶ್ವರೀ~।\\
ಸದಾಶಿವಾಽನುಗ್ರಹದಾ ಪಂಚಕೃತ್ಯ-ಪರಾಯಣಾ ॥೬೪॥

ಭಾನುಮಂಡಲ-ಮಧ್ಯಸ್ಥಾ ಭೈರವೀ ಭಗಮಾಲಿನೀ~।\\
ಪದ್ಮಾಸನಾ ಭಗವತೀ ಪದ್ಮನಾಭ-ಸಹೋದರೀ ॥೬೫॥

ಉನ್ಮೇಷ-ನಿಮಿಷೋತ್ಪನ್ನ-ವಿಪನ್ನ-ಭುವನಾವಲಿಃ~।\\
ಸಹಸ್ರ-ಶೀರ್ಷವದನಾ ಸಹಸ್ರಾಕ್ಷೀ ಸಹಸ್ರಪಾತ್ ॥೬೬॥

ಆಬ್ರಹ್ಮ-ಕೀಟ-ಜನನೀ ವರ್ಣಾಶ್ರಮ-ವಿಧಾಯಿನೀ~।\\
ನಿಜಾಜ್ಞಾರೂಪ-ನಿಗಮಾ ಪುಣ್ಯಾಪುಣ್ಯ-ಫಲಪ್ರದಾ ॥೬೭॥

ಶ್ರುತಿ-ಸೀಮಂತ-ಸಿಂದೂರೀ-ಕೃತ-ಪಾದಾಬ್ಜ-ಧೂಲಿಕಾ~।\\
ಸಕಲಾಗಮ-ಸಂದೋಹ-ಶುಕ್ತಿ-ಸಂಪುಟ-ಮೌಕ್ತಿಕಾ ॥೬೮॥

ಪುರುಷಾರ್ಥಪ್ರದಾ ಪೂರ್ಣಾ ಭೋಗಿನೀ ಭುವನೇಶ್ವರೀ~।\\
ಅಂಬಿಕಾಽಽನಾದಿ-ನಿಧನಾ ಹರಿಬ್ರಹ್ಮೇಂದ್ರ-ಸೇವಿತಾ ॥೬೯॥

ನಾರಾಯಣೀ ನಾದರೂಪಾ ನಾಮರೂಪ-ವಿವರ್ಜಿತಾ \as{(೩೦೦)}।\\
ಹ್ರೀಂಕಾರೀ ಹ್ರೀಂಮತೀ ಹೃದ್ಯಾ ಹೇಯೋಪಾದೇಯ-ವರ್ಜಿತಾ ॥೭೦॥

ರಾಜರಾಜಾರ್ಚಿತಾ ರಾಜ್ಞೀ ರಮ್ಯಾ ರಾಜೀವಲೋಚನಾ~।\\
ರಂಜನೀ ರಮಣೀ ರಸ್ಯಾ ರಣತ್ಕಿಂಕಿಣಿ-ಮೇಖಲಾ ॥೭೧॥

ರಮಾ ರಾಕೇಂದುವದನಾ ರತಿರೂಪಾ ರತಿಪ್ರಿಯಾ~।\\
ರಕ್ಷಾಕರೀ ರಾಕ್ಷಸಘ್ನೀ ರಾಮಾ ರಮಣಲಂಪಟಾ ॥೭೨॥

ಕಾಮ್ಯಾ ಕಾಮಕಲಾರೂಪಾ ಕದಂಬ-ಕುಸುಮ-ಪ್ರಿಯಾ~।\\
ಕಲ್ಯಾಣೀ ಜಗತೀಕಂದಾ ಕರುಣಾ-ರಸ-ಸಾಗರಾ ॥೭೩॥

ಕಲಾವತೀ ಕಲಾಲಾಪಾ ಕಾಂತಾ ಕಾದಂಬರೀಪ್ರಿಯಾ~।\\
ವರದಾ ವಾಮನಯನಾ ವಾರುಣೀ-ಮದ-ವಿಹ್ವಲಾ ॥೭೪॥

ವಿಶ್ವಾಧಿಕಾ ವೇದವೇದ್ಯಾ ವಿಂಧ್ಯಾಚಲ-ನಿವಾಸಿನೀ~।\\
ವಿಧಾತ್ರೀ ವೇದಜನನೀ ವಿಷ್ಣುಮಾಯಾ ವಿಲಾಸಿನೀ ॥೭೫॥

ಕ್ಷೇತ್ರಸ್ವರೂಪಾ ಕ್ಷೇತ್ರೇಶೀ ಕ್ಷೇತ್ರ-ಕ್ಷೇತ್ರಜ್ಞ-ಪಾಲಿನೀ~।\\
ಕ್ಷಯವೃದ್ಧಿ-ವಿನಿರ್ಮುಕ್ತಾ ಕ್ಷೇತ್ರಪಾಲ-ಸಮರ್ಚಿತಾ ॥೭೬॥

ವಿಜಯಾ ವಿಮಲಾ ವಂದ್ಯಾ ವಂದಾರು-ಜನ-ವತ್ಸಲಾ~।\\
ವಾಗ್ವಾದಿನೀ ವಾಮಕೇಶೀ ವಹ್ನಿಮಂಡಲ-ವಾಸಿನೀ ॥೭೭॥

ಭಕ್ತಿಮತ್-ಕಲ್ಪಲತಿಕಾ ಪಶುಪಾಶ-ವಿಮೋಚಿನೀ~।\\
ಸಂಹೃತಾಶೇಷ-ಪಾಷಂಡಾ ಸದಾಚಾರ-ಪ್ರವರ್ತಿಕಾ ॥೭೮॥

ತಾಪತ್ರಯಾಗ್ನಿ-ಸಂತಪ್ತ-ಸಮಾಹ್ಲಾದನ ಚಂದ್ರಿಕಾ~।\\
ತರುಣೀ ತಾಪಸಾರಾಧ್ಯಾ ತನುಮಧ್ಯಾ ತಮೋಽಪಹಾ ॥೭೯॥

ಚಿತಿಸ್ತತ್ಪದ-ಲಕ್ಷ್ಯಾರ್ಥಾ ಚಿದೇಕರಸ-ರೂಪಿಣೀ~।\\
ಸ್ವಾತ್ಮಾನಂದ-ಲವೀಭೂತ-ಬ್ರಹ್ಮಾದ್ಯಾನಂದ-ಸಂತತಿಃ ॥೮೦॥

ಪರಾ ಪ್ರತ್ಯಕ್ಚಿತೀರೂಪಾ ಪಶ್ಯಂತೀ ಪರದೇವತಾ~।\\
ಮಧ್ಯಮಾ ವೈಖರೀರೂಪಾ ಭಕ್ತ-ಮಾನಸ-ಹಂಸಿಕಾ ॥೮೧॥

ಕಾಮೇಶ್ವರ-ಪ್ರಾಣನಾಡೀ ಕೃತಜ್ಞಾ ಕಾಮಪೂಜಿತಾ~।\\
ಶೃಂಗಾರ-ರಸ-ಸಂಪೂರ್ಣಾ ಜಯಾ ಜಾಲಂಧರ-ಸ್ಥಿತಾ ॥೮೨॥

ಓಡ್ಯಾಣಪೀಠ-ನಿಲಯಾ ಬಿಂದು-ಮಂಡಲವಾಸಿನೀ~।\\
ರಹೋಯಾಗ-ಕ್ರಮಾರಾಧ್ಯಾ ರಹಸ್ತರ್ಪಣ-ತರ್ಪಿತಾ ॥೮೩॥

ಸದ್ಯಃಪ್ರಸಾದಿನೀ ವಿಶ್ವ-ಸಾಕ್ಷಿಣೀ ಸಾಕ್ಷಿವರ್ಜಿತಾ~।\\
ಷಡಂಗದೇವತಾ-ಯುಕ್ತಾ ಷಾಡ್ಗುಣ್ಯ-ಪರಿಪೂರಿತಾ ॥೮೪॥

ನಿತ್ಯಕ್ಲಿನ್ನಾ ನಿರುಪಮಾ ನಿರ್ವಾಣ-ಸುಖ-ದಾಯಿನೀ~।\\
ನಿತ್ಯಾ-ಷೋಡಶಿಕಾ-ರೂಪಾ ಶ್ರೀಕಂಠಾರ್ಧ-ಶರೀರಿಣೀ ॥೮೫॥

ಪ್ರಭಾವತೀ ಪ್ರಭಾರೂಪಾ ಪ್ರಸಿದ್ಧಾ ಪರಮೇಶ್ವರೀ~।\\
ಮೂಲಪ್ರಕೃತಿರವ್ಯಕ್ತಾ ವ್ಯಕ್ತಾವ್ಯಕ್ತ-ಸ್ವರೂಪಿಣೀ ॥೮೬॥

ವ್ಯಾಪಿನೀ \as{(೪೦೦)} ವಿವಿಧಾಕಾರಾ ವಿದ್ಯಾವಿದ್ಯಾ-ಸ್ವರೂಪಿಣೀ~।\\
ಮಹಾಕಾಮೇಶ-ನಯನ-ಕುಮುದಾಹ್ಲಾದ-ಕೌಮುದೀ ॥೮೭॥

ಭಕ್ತ-ಹಾರ್ದ-ತಮೋಭೇದ-ಭಾನುಮದ್ಭಾನು-ಸಂತತಿಃ~।\\
ಶಿವದೂತೀ ಶಿವಾರಾಧ್ಯಾ ಶಿವಮೂರ್ತಿಃ ಶಿವಂಕರೀ ॥೮೮॥

ಶಿವಪ್ರಿಯಾ ಶಿವಪರಾ ಶಿಷ್ಟೇಷ್ಟಾ ಶಿಷ್ಟಪೂಜಿತಾ~।\\
ಅಪ್ರಮೇಯಾ ಸ್ವಪ್ರಕಾಶಾ ಮನೋವಾಚಾಮಗೋಚರಾ ॥೮೯॥

ಚಿಚ್ಛಕ್ತಿಶ್ ಚೇತನಾರೂಪಾ ಜಡಶಕ್ತಿರ್ಜಡಾತ್ಮಿಕಾ~।\\
ಗಾಯತ್ರೀ ವ್ಯಾಹೃತಿಃ ಸಂಧ್ಯಾ ದ್ವಿಜಬೃಂದ-ನಿಷೇವಿತಾ ॥೯೦॥

ತತ್ತ್ವಾಸನಾ ತತ್ತ್ವಮಯೀ ಪಂಚ-ಕೋಶಾಂತರ-ಸ್ಥಿತಾ~।\\
ನಿಃಸೀಮಮಹಿಮಾ ನಿತ್ಯ-ಯೌವನಾ ಮದಶಾಲಿನೀ ॥೯೧॥

ಮದಘೂರ್ಣಿತ-ರಕ್ತಾಕ್ಷೀ ಮದಪಾಟಲ-ಗಂಡಭೂಃ~।\\
ಚಂದನ-ದ್ರವ-ದಿಗ್ಧಾಂಗೀ ಚಾಂಪೇಯ-ಕುಸುಮ-ಪ್ರಿಯಾ ॥೯೨॥

ಕುಶಲಾ ಕೋಮಲಾಕಾರಾ ಕುರುಕುಲ್ಲಾ ಕುಲೇಶ್ವರೀ~।\\
ಕುಲಕುಂಡಾಲಯಾ ಕೌಲ-ಮಾರ್ಗ-ತತ್ಪರ-ಸೇವಿತಾ ॥೯೩॥

ಕುಮಾರ-ಗಣನಾಥಾಂಬಾ ತುಷ್ಟಿಃ ಪುಷ್ಟಿರ್ಮತಿರ್ಧೃತಿಃ~।\\
ಶಾಂತಿಃ ಸ್ವಸ್ತಿಮತೀ ಕಾಂತಿರ್ನಂದಿನೀ ವಿಘ್ನನಾಶಿನೀ ॥೯೪॥

ತೇಜೋವತೀ ತ್ರಿನಯನಾ ಲೋಲಾಕ್ಷೀ-ಕಾಮರೂಪಿಣೀ~।\\
ಮಾಲಿನೀ ಹಂಸಿನೀ ಮಾತಾ ಮಲಯಾಚಲ-ವಾಸಿನೀ ॥೯೫॥

ಸುಮುಖೀ ನಲಿನೀ ಸುಭ್ರೂಃ ಶೋಭನಾ ಸುರನಾಯಿಕಾ~।\\
ಕಾಲಕಂಠೀ ಕಾಂತಿಮತೀ ಕ್ಷೋಭಿಣೀ ಸೂಕ್ಷ್ಮರೂಪಿಣೀ ॥೯೬॥

ವಜ್ರೇಶ್ವರೀ ವಾಮದೇವೀ ವಯೋಽವಸ್ಥಾ-ವಿವರ್ಜಿತಾ~।\\
ಸಿದ್ಧೇಶ್ವರೀ ಸಿದ್ಧವಿದ್ಯಾ ಸಿದ್ಧಮಾತಾ ಯಶಸ್ವಿನೀ ॥೯೭॥

ವಿಶುದ್ಧಿಚಕ್ರ-ನಿಲಯಾಽಽರಕ್ತವರ್ಣಾ ತ್ರಿಲೋಚನಾ~।\\
ಖಟ್‍ವಾಂಗಾದಿ-ಪ್ರಹರಣಾ ವದನೈಕ-ಸಮನ್ವಿತಾ ॥೯೮॥

ಪಾಯಸಾನ್ನಪ್ರಿಯಾ ತ್ವಕ್ಸ್ಥಾ ಪಶುಲೋಕ-ಭಯಂಕರೀ~।\\
ಅಮೃತಾದಿ-ಮಹಾಶಕ್ತಿ-ಸಂವೃತಾ ಡಾಕಿನೀಶ್ವರೀ ॥೯೯॥

ಅನಾಹತಾಬ್ಜ-ನಿಲಯಾ ಶ್ಯಾಮಾಭಾ ವದನದ್ವಯಾ~।\\
ದಂಷ್ಟ್ರೋಜ್ಜ್ವಲಾಽಕ್ಷ-ಮಾಲಾದಿ-ಧರಾ ರುಧಿರಸಂಸ್ಥಿತಾ ॥೧೦೦॥

ಕಾಲರಾತ್ರ್ಯಾದಿ-ಶಕ್ತ್ಯೌಘ-ವೃತಾ ಸ್ನಿಗ್ಧೌದನಪ್ರಿಯಾ~।\\
ಮಹಾವೀರೇಂದ್ರ-ವರದಾ ರಾಕಿಣ್ಯಂಬಾ-ಸ್ವರೂಪಿಣೀ ॥೧೦೧॥

ಮಣಿಪೂರಾಬ್ಜ-ನಿಲಯಾ ವದನತ್ರಯ-ಸಂಯುತಾ~।\\
ವಜ್ರಾದಿಕಾಯುಧೋಪೇತಾ ಡಾಮರ್ಯಾದಿಭಿರಾವೃತಾ ॥೧೦೨॥

ರಕ್ತವರ್ಣಾ ಮಾಂಸನಿಷ್ಠಾ \as{(೫೦೦)} ಗುಡಾನ್ನ-ಪ್ರೀತ-ಮಾನಸಾ~।\\
ಸಮಸ್ತಭಕ್ತ-ಸುಖದಾ ಲಾಕಿನ್ಯಂಬಾ-ಸ್ವರೂಪಿಣೀ ॥೧೦೩॥

ಸ್ವಾಧಿಷ್ಠಾನಾಂಬುಜ-ಗತಾ ಚತುರ್ವಕ್ತ್ರ-ಮನೋಹರಾ~।\\
ಶೂಲಾದ್ಯಾಯುಧ-ಸಂಪನ್ನಾ ಪೀತವರ್ಣಾಽತಿಗರ್ವಿತಾ ॥೧೦೪॥

ಮೇದೋನಿಷ್ಠಾ ಮಧುಪ್ರೀತಾ ಬಂದಿನ್ಯಾದಿ-ಸಮನ್ವಿತಾ~।\\
ದಧ್ಯನ್ನಾಸಕ್ತ-ಹೃದಯಾ ಕಾಕಿನೀ-ರೂಪ-ಧಾರಿಣೀ ॥೧೦೫॥

ಮೂಲಾಧಾರಾಂಬುಜಾರೂಢಾ ಪಂಚ-ವಕ್ತ್ರಾಽಸ್ಥಿ-ಸಂಸ್ಥಿತಾ~।\\
ಅಂಕುಶಾದಿ-ಪ್ರಹರಣಾ ವರದಾದಿ-ನಿಷೇವಿತಾ ॥೧೦೬॥

ಮುದ್ಗೌದನಾಸಕ್ತ-ಚಿತ್ತಾ ಸಾಕಿನ್ಯಂಬಾ-ಸ್ವರೂಪಿಣೀ~।\\
ಆಜ್ಞಾ-ಚಕ್ರಾಬ್ಜ-ನಿಲಯಾ ಶುಕ್ಲವರ್ಣಾ ಷಡಾನನಾ ॥೧೦೭॥

ಮಜ್ಜಾಸಂಸ್ಥಾ ಹಂಸವತೀ-ಮುಖ್ಯ-ಶಕ್ತಿ-ಸಮನ್ವಿತಾ~।\\
ಹರಿದ್ರಾನ್ನೈಕ-ರಸಿಕಾ ಹಾಕಿನೀ-ರೂಪ-ಧಾರಿಣೀ ॥೧೦೮॥

ಸಹಸ್ರದಲ-ಪದ್ಮಸ್ಥಾ ಸರ್ವ-ವರ್ಣೋಪ-ಶೋಭಿತಾ~।\\
ಸರ್ವಾಯುಧಧರಾ ಶುಕ್ಲ-ಸಂಸ್ಥಿತಾ ಸರ್ವತೋಮುಖೀ ॥೧೦೯॥

ಸರ್ವೌದನ-ಪ್ರೀತಚಿತ್ತಾ ಯಾಕಿನ್ಯಂಬಾ-ಸ್ವರೂಪಿಣೀ~।\\
ಸ್ವಾಹಾ ಸ್ವಧಾಽಮತಿರ್ಮೇಧಾ ಶ್ರುತಿಃ ಸ್ಮೃತಿರನುತ್ತಮಾ ॥೧೧೦॥

ಪುಣ್ಯಕೀರ್ತಿಃ ಪುಣ್ಯಲಭ್ಯಾ ಪುಣ್ಯಶ್ರವಣ-ಕೀರ್ತನಾ~।\\
ಪುಲೋಮಜಾರ್ಚಿತಾ ಬಂಧ-ಮೋಚನೀ ಬಂಧುರಾಲಕಾ ॥೧೧೧॥

ವಿಮರ್ಶರೂಪಿಣೀ ವಿದ್ಯಾ ವಿಯದಾದಿ-ಜಗತ್ಪ್ರಸೂಃ~।\\
ಸರ್ವವ್ಯಾಧಿ-ಪ್ರಶಮನೀ ಸರ್ವಮೃತ್ಯು-ನಿವಾರಿಣೀ ॥೧೧೨॥

ಅಗ್ರಗಣ್ಯಾಽಚಿಂತ್ಯರೂಪಾ ಕಲಿಕಲ್ಮಷ-ನಾಶಿನೀ~।\\
ಕಾತ್ಯಾಯನೀ ಕಾಲಹಂತ್ರೀ ಕಮಲಾಕ್ಷ-ನಿಷೇವಿತಾ ॥೧೧೩॥

ತಾಂಬೂಲ-ಪೂರಿತ-ಮುಖೀ ದಾಡಿಮೀ-ಕುಸುಮ-ಪ್ರಭಾ~।\\
ಮೃಗಾಕ್ಷೀ ಮೋಹಿನೀ ಮುಖ್ಯಾ ಮೃಡಾನೀ ಮಿತ್ರರೂಪಿಣೀ ॥೧೧೪॥

ನಿತ್ಯತೃಪ್ತಾ ಭಕ್ತನಿಧಿರ್ನಿಯಂತ್ರೀ ನಿಖಿಲೇಶ್ವರೀ~।\\
ಮೈತ್ರ್ಯಾದಿ-ವಾಸನಾಲಭ್ಯಾ ಮಹಾಪ್ರಲಯ-ಸಾಕ್ಷಿಣೀ ॥೧೧೫॥

ಪರಾ ಶಕ್ತಿಃ ಪರಾ ನಿಷ್ಠಾ ಪ್ರಜ್ಞಾನಘನ-ರೂಪಿಣೀ~।\\
ಮಾಧ್ವೀಪಾನಾಲಸಾ ಮತ್ತಾ ಮಾತೃಕಾ-ವರ್ಣ-ರೂಪಿಣೀ ॥೧೧೬॥

ಮಹಾಕೈಲಾಸ-ನಿಲಯಾ ಮೃಣಾಲ-ಮೃದು-ದೋರ್ಲತಾ~।\\
ಮಹನೀಯಾ ದಯಾಮೂರ್ತಿರ್ಮಹಾಸಾಮ್ರಾಜ್ಯ-ಶಾಲಿನೀ ॥೧೧೭॥

ಆತ್ಮವಿದ್ಯಾ ಮಹಾವಿದ್ಯಾ ಶ್ರೀವಿದ್ಯಾ ಕಾಮಸೇವಿತಾ~।\\
ಶ್ರೀ-ಷೋಡಶಾಕ್ಷರೀ-ವಿದ್ಯಾ ತ್ರಿಕೂಟಾ ಕಾಮಕೋಟಿಕಾ ॥೧೧೮॥

ಕಟಾಕ್ಷ-ಕಿಂಕರೀ-ಭೂತ-ಕಮಲಾ-ಕೋಟಿ-ಸೇವಿತಾ~।\\
ಶಿರಃಸ್ಥಿತಾ ಚಂದ್ರನಿಭಾ ಭಾಲಸ್ಥೇಂದ್ರ-ಧನುಃಪ್ರಭಾ ॥೧೧೯॥

ಹೃದಯಸ್ಥಾ ರವಿಪ್ರಖ್ಯಾ ತ್ರಿಕೋಣಾಂತರ-ದೀಪಿಕಾ~।\\
ದಾಕ್ಷಾಯಣೀ ದೈತ್ಯಹಂತ್ರೀ ದಕ್ಷಯಜ್ಞ-ವಿನಾಶಿನೀ \as{(೬೦೦)} ॥೧೨೦॥

ದರಾಂದೋಲಿತ-ದೀರ್ಘಾಕ್ಷೀ ದರ-ಹಾಸೋಜ್ಜ್ವಲನ್ಮುಖೀ~।\\
ಗುರುಮೂರ್ತಿರ್ಗುಣನಿಧಿರ್ಗೋಮಾತಾ ಗುಹಜನ್ಮಭೂಃ ॥೧೨೧॥

ದೇವೇಶೀ ದಂಡನೀತಿಸ್ಥಾ ದಹರಾಕಾಶ-ರೂಪಿಣೀ~।\\
ಪ್ರತಿಪನ್ಮುಖ್ಯ-ರಾಕಾಂತ-ತಿಥಿ-ಮಂಡಲ-ಪೂಜಿತಾ ॥೧೨೨॥

ಕಲಾತ್ಮಿಕಾ ಕಲಾನಾಥಾ ಕಾವ್ಯಾಲಾಪ-ವಿನೋದಿನೀ~।\\
ಸಚಾಮರ-ರಮಾ-ವಾಣೀ-ಸವ್ಯ-ದಕ್ಷಿಣ-ಸೇವಿತಾ ॥೧೨೩॥

ಆದಿಶಕ್ತಿರಮೇಯಾಽಽತ್ಮಾ ಪರಮಾ ಪಾವನಾಕೃತಿಃ~।\\
ಅನೇಕಕೋಟಿ-ಬ್ರಹ್ಮಾಂಡ-ಜನನೀ ದಿವ್ಯವಿಗ್ರಹಾ ॥೧೨೪॥

ಕ್ಲೀಂಕಾರೀ ಕೇವಲಾ ಗುಹ್ಯಾ ಕೈವಲ್ಯ-ಪದದಾಯಿನೀ~।\\
ತ್ರಿಪುರಾ ತ್ರಿಜಗದ್ವಂದ್ಯಾ ತ್ರಿಮೂರ್ತಿಸ್ತ್ರಿದಶೇಶ್ವರೀ ॥೧೨೫॥

ತ್ರ್ಯಕ್ಷರೀ ದಿವ್ಯ-ಗಂಧಾಢ್ಯಾ ಸಿಂದೂರ-ತಿಲಕಾಂಚಿತಾ~।\\
ಉಮಾ ಶೈಲೇಂದ್ರತನಯಾ ಗೌರೀ ಗಂಧರ್ವ-ಸೇವಿತಾ ॥೧೨೬॥

ವಿಶ್ವಗರ್ಭಾ ಸ್ವರ್ಣಗರ್ಭಾ ವರದಾ ವಾಗಧೀಶ್ವರೀ~।\\
ಧ್ಯಾನಗಮ್ಯಾಽಪರಿಚ್ಛೇದ್ಯಾ ಜ್ಞಾನದಾ ಜ್ಞಾನವಿಗ್ರಹಾ ॥೧೨೭॥

ಸರ್ವವೇದಾಂತ-ಸಂವೇದ್ಯಾ ಸತ್ಯಾನಂದ-ಸ್ವರೂಪಿಣೀ~।\\
ಲೋಪಾಮುದ್ರಾರ್ಚಿತಾ ಲೀಲಾ-ಕ್ಲೃಪ್ತ-ಬ್ರಹ್ಮಾಂಡ-ಮಂಡಲಾ ॥೧೨೮॥

ಅದೃಶ್ಯಾ ದೃಶ್ಯರಹಿತಾ ವಿಜ್ಞಾತ್ರೀ ವೇದ್ಯವರ್ಜಿತಾ~।\\
ಯೋಗಿನೀ ಯೋಗದಾ ಯೋಗ್ಯಾ ಯೋಗಾನಂದಾ ಯುಗಂಧರಾ ॥೧೨೯॥

ಇಚ್ಛಾಶಕ್ತಿ-ಜ್ಞಾನಶಕ್ತಿ-ಕ್ರಿಯಾಶಕ್ತಿ-ಸ್ವರೂಪಿಣೀ~।\\
ಸರ್ವಾಧಾರಾ ಸುಪ್ರತಿಷ್ಠಾ ಸದಸದ್ರೂಪ-ಧಾರಿಣೀ ॥೧೩೦॥

ಅಷ್ಟಮೂರ್ತಿರಜಾಜೈತ್ರೀ ಲೋಕಯಾತ್ರಾ-ವಿಧಾಯಿನೀ~।\\
ಏಕಾಕಿನೀ ಭೂಮರೂಪಾ ನಿರ್ದ್ವೈತಾ ದ್ವೈತವರ್ಜಿತಾ ॥೧೩೧॥

ಅನ್ನದಾ ವಸುದಾ ವೃದ್ಧಾ ಬ್ರಹ್ಮಾತ್ಮೈಕ್ಯ-ಸ್ವರೂಪಿಣೀ~।\\
ಬೃಹತೀ ಬ್ರಾಹ್ಮಣೀ ಬ್ರಾಹ್ಮೀ ಬ್ರಹ್ಮಾನಂದಾ ಬಲಿಪ್ರಿಯಾ ॥೧೩೨॥

ಭಾಷಾರೂಪಾ ಬೃಹತ್ಸೇನಾ ಭಾವಾಭಾವ-ವಿವರ್ಜಿತಾ~।\\
ಸುಖಾರಾಧ್ಯಾ ಶುಭಕರೀ ಶೋಭನಾ ಸುಲಭಾ ಗತಿಃ ॥೧೩೩॥

ರಾಜ-ರಾಜೇಶ್ವರೀ ರಾಜ್ಯ-ದಾಯಿನೀ ರಾಜ್ಯ-ವಲ್ಲಭಾ~।\\
ರಾಜತ್ಕೃಪಾ ರಾಜಪೀಠ-ನಿವೇಶಿತ-ನಿಜಾಶ್ರಿತಾ ॥೧೩೪॥

ರಾಜ್ಯಲಕ್ಷ್ಮೀಃ ಕೋಶನಾಥಾ ಚತುರಂಗ-ಬಲೇಶ್ವರೀ~।\\
ಸಾಮ್ರಾಜ್ಯ-ದಾಯಿನೀ ಸತ್ಯಸಂಧಾ ಸಾಗರಮೇಖಲಾ ॥೧೩೫॥

ದೀಕ್ಷಿತಾ ದೈತ್ಯಶಮನೀ ಸರ್ವಲೋಕ-ವಶಂಕರೀ~।\\
ಸರ್ವಾರ್ಥದಾತ್ರೀ ಸಾವಿತ್ರೀ ಸಚ್ಚಿದಾನಂದ-ರೂಪಿಣೀ \as{(೭೦೦)} ॥೧೩೬॥

ದೇಶ-ಕಾಲಾಪರಿಚ್ಛಿನ್ನಾ ಸರ್ವಗಾ ಸರ್ವಮೋಹಿನೀ~।\\
ಸರಸ್ವತೀ ಶಾಸ್ತ್ರಮಯೀ ಗುಹಾಂಬಾ ಗುಹ್ಯರೂಪಿಣೀ ॥೧೩೭॥

ಸರ್ವೋಪಾಧಿ-ವಿನಿರ್ಮುಕ್ತಾ ಸದಾಶಿವ-ಪತಿವ್ರತಾ~।\\
ಸಂಪ್ರದಾಯೇಶ್ವರೀ ಸಾಧ್ವೀ ಗುರುಮಂಡಲ-ರೂಪಿಣೀ ॥೧೩೮॥

ಕುಲೋತ್ತೀರ್ಣಾ ಭಗಾರಾಧ್ಯಾ ಮಾಯಾ ಮಧುಮತೀ ಮಹೀ~।\\
ಗಣಾಂಬಾ ಗುಹ್ಯಕಾರಾಧ್ಯಾ ಕೋಮಲಾಂಗೀ ಗುರುಪ್ರಿಯಾ ॥೧೩೯॥

ಸ್ವತಂತ್ರಾ ಸರ್ವತಂತ್ರೇಶೀ ದಕ್ಷಿಣಾಮೂರ್ತಿ-ರೂಪಿಣೀ~।\\
ಸನಕಾದಿ-ಸಮಾರಾಧ್ಯಾ ಶಿವಜ್ಞಾನ-ಪ್ರದಾಯಿನೀ ॥೧೪೦॥

ಚಿತ್ಕಲಾಽಽನಂದ-ಕಲಿಕಾ ಪ್ರೇಮರೂಪಾ ಪ್ರಿಯಂಕರೀ~।\\
ನಾಮಪಾರಾಯಣ-ಪ್ರೀತಾ ನಂದಿವಿದ್ಯಾ ನಟೇಶ್ವರೀ ॥೧೪೧॥

ಮಿಥ್ಯಾ-ಜಗದಧಿಷ್ಠಾನಾ ಮುಕ್ತಿದಾ ಮುಕ್ತಿರೂಪಿಣೀ~।\\
ಲಾಸ್ಯಪ್ರಿಯಾ ಲಯಕರೀ ಲಜ್ಜಾ ರಂಭಾದಿವಂದಿತಾ ॥೧೪೨॥

ಭವದಾವ-ಸುಧಾವೃಷ್ಟಿಃ ಪಾಪಾರಣ್ಯ-ದವಾನಲಾ~।\\
ದೌರ್ಭಾಗ್ಯ-ತೂಲವಾತೂಲಾ ಜರಾಧ್ವಾಂತ-ರವಿಪ್ರಭಾ ॥೧೪೩॥

ಭಾಗ್ಯಾಬ್ಧಿ-ಚಂದ್ರಿಕಾ ಭಕ್ತ-ಚಿತ್ತಕೇಕಿ-ಘನಾಘನಾ~।\\
ರೋಗಪರ್ವತ-ದಂಭೋಲಿರ್ಮೃತ್ಯುದಾರು-ಕುಠಾರಿಕಾ ॥೧೪೪॥

ಮಹೇಶ್ವರೀ ಮಹಾಕಾಲೀ ಮಹಾಗ್ರಾಸಾ ಮಹಾಶನಾ~।\\
ಅಪರ್ಣಾ ಚಂಡಿಕಾ ಚಂಡಮುಂಡಾಸುರ-ನಿಷೂದಿನೀ ॥೧೪೫॥

ಕ್ಷರಾಕ್ಷರಾತ್ಮಿಕಾ ಸರ್ವ-ಲೋಕೇಶೀ ವಿಶ್ವಧಾರಿಣೀ~।\\
ತ್ರಿವರ್ಗದಾತ್ರೀ ಸುಭಗಾ ತ್ರ್ಯಂಬಕಾ ತ್ರಿಗುಣಾತ್ಮಿಕಾ ॥೧೪೬॥

ಸ್ವರ್ಗಾಪವರ್ಗದಾ ಶುದ್ಧಾ ಜಪಾಪುಷ್ಪ-ನಿಭಾಕೃತಿಃ~।\\
ಓಜೋವತೀ ದ್ಯುತಿಧರಾ ಯಜ್ಞರೂಪಾ ಪ್ರಿಯವ್ರತಾ ॥೧೪೭॥

ದುರಾರಾಧ್ಯಾ ದುರಾಧರ್ಷಾ ಪಾಟಲೀ-ಕುಸುಮ-ಪ್ರಿಯಾ~।\\
ಮಹತೀ ಮೇರುನಿಲಯಾ ಮಂದಾರ-ಕುಸುಮ-ಪ್ರಿಯಾ ॥೧೪೮॥

ವೀರಾರಾಧ್ಯಾ ವಿರಾಡ್ರೂಪಾ ವಿರಜಾ ವಿಶ್ವತೋಮುಖೀ~।\\
ಪ್ರತ್ಯಗ್ರೂಪಾ ಪರಾಕಾಶಾ ಪ್ರಾಣದಾ ಪ್ರಾಣರೂಪಿಣೀ ॥೧೪೯॥

ಮಾರ್ತಾಂಡ-ಭೈರವಾರಾಧ್ಯಾ ಮಂತ್ರಿಣೀನ್ಯಸ್ತ-ರಾಜ್ಯಧೂಃ~।\\
ತ್ರಿಪುರೇಶೀ ಜಯತ್ಸೇನಾ ನಿಸ್ತ್ರೈಗುಣ್ಯಾ ಪರಾಪರಾ ॥೧೫೦॥

ಸತ್ಯ-ಜ್ಞಾನಾನಂದ-ರೂಪಾ ಸಾಮರಸ್ಯ-ಪರಾಯಣಾ~।\\
ಕಪರ್ದಿನೀ ಕಲಾಮಾಲಾ ಕಾಮಧುಕ್ ಕಾಮರೂಪಿಣೀ ॥೧೫೧॥

ಕಲಾನಿಧಿಃ ಕಾವ್ಯಕಲಾ ರಸಜ್ಞಾ ರಸಶೇವಧಿಃ \as{(೮೦೦)}~।\\
ಪುಷ್ಟಾ ಪುರಾತನಾ ಪೂಜ್ಯಾ ಪುಷ್ಕರಾ ಪುಷ್ಕರೇಕ್ಷಣಾ ॥೧೫೨॥

ಪರಂಜ್ಯೋತಿಃ ಪರಂಧಾಮ ಪರಮಾಣುಃ ಪರಾತ್ಪರಾ~।\\
ಪಾಶಹಸ್ತಾ ಪಾಶಹಂತ್ರೀ ಪರಮಂತ್ರ-ವಿಭೇದಿನೀ ॥೧೫೩॥

ಮೂರ್ತಾಽಮೂರ್ತಾಽನಿತ್ಯತೃಪ್ತಾ ಮುನಿಮಾನಸ-ಹಂಸಿಕಾ~।\\
ಸತ್ಯವ್ರತಾ ಸತ್ಯರೂಪಾ ಸರ್ವಾಂತರ್ಯಾಮಿನೀ ಸತೀ ॥೧೫೪॥

ಬ್ರಹ್ಮಾಣೀ ಬ್ರಹ್ಮಜನನೀ ಬಹುರೂಪಾ ಬುಧಾರ್ಚಿತಾ~।\\
ಪ್ರಸವಿತ್ರೀ ಪ್ರಚಂಡಾಽಽಜ್ಞಾ ಪ್ರತಿಷ್ಠಾ ಪ್ರಕಟಾಕೃತಿಃ ॥೧೫೫॥

ಪ್ರಾಣೇಶ್ವರೀ ಪ್ರಾಣದಾತ್ರೀ ಪಂಚಾಶತ್ಪೀಠ-ರೂಪಿಣೀ~।\\
ವಿಶೃಂಖಲಾ ವಿವಿಕ್ತಸ್ಥಾ ವೀರಮಾತಾ ವಿಯತ್ಪ್ರಸೂಃ ॥೧೫೬॥

ಮುಕುಂದಾ ಮುಕ್ತಿನಿಲಯಾ ಮೂಲವಿಗ್ರಹ-ರೂಪಿಣೀ~।\\
ಭಾವಜ್ಞಾ ಭವರೋಗಘ್ನೀ ಭವಚಕ್ರ-ಪ್ರವರ್ತಿನೀ ॥೧೫೭॥

ಛಂದಃಸಾರಾ ಶಾಸ್ತ್ರಸಾರಾ ಮಂತ್ರಸಾರಾ ತಲೋದರೀ~।\\
ಉದಾರಕೀರ್ತಿರುದ್ದಾಮವೈಭವಾ ವರ್ಣರೂಪಿಣೀ ॥೧೫೮॥

ಜನ್ಮಮೃತ್ಯು-ಜರಾತಪ್ತ-ಜನವಿಶ್ರಾಂತಿ-ದಾಯಿನೀ~।\\
ಸರ್ವೋಪನಿಷದುದ್ಘುಷ್ಟಾ ಶಾಂತ್ಯತೀತ-ಕಲಾತ್ಮಿಕಾ ॥೧೫೯॥

ಗಂಭೀರಾ ಗಗನಾಂತಸ್ಥಾ ಗರ್ವಿತಾ ಗಾನಲೋಲುಪಾ~।\\
ಕಲ್ಪನಾ-ರಹಿತಾ ಕಾಷ್ಠಾಽಕಾಂತಾ ಕಾಂತಾರ್ಧ-ವಿಗ್ರಹಾ ॥೧೬೦॥

ಕಾರ್ಯಕಾರಣ-ನಿರ್ಮುಕ್ತಾ ಕಾಮಕೇಲಿ-ತರಂಗಿತಾ~।\\
ಕನತ್ಕನಕತಾ-ಟಂಕಾ ಲೀಲಾ-ವಿಗ್ರಹ-ಧಾರಿಣೀ ॥೧೬೧॥

ಅಜಾ ಕ್ಷಯವಿನಿರ್ಮುಕ್ತಾ ಮುಗ್ಧಾ ಕ್ಷಿಪ್ರ-ಪ್ರಸಾದಿನೀ~।\\
ಅಂತರ್ಮುಖ-ಸಮಾರಾಧ್ಯಾ ಬಹಿರ್ಮುಖ-ಸುದುರ್ಲಭಾ ॥೧೬೨॥

ತ್ರಯೀ ತ್ರಿವರ್ಗನಿಲಯಾ ತ್ರಿಸ್ಥಾ ತ್ರಿಪುರಮಾಲಿನೀ~।\\
ನಿರಾಮಯಾ ನಿರಾಲಂಬಾ ಸ್ವಾತ್ಮಾರಾಮಾ ಸುಧಾಸೃತಿಃ ॥೧೬೩॥

ಸಂಸಾರಪಂಕ-ನಿರ್ಮಗ್ನ-ಸಮುದ್ಧರಣ-ಪಂಡಿತಾ~।\\
ಯಜ್ಞಪ್ರಿಯಾ ಯಜ್ಞಕರ್ತ್ರೀ ಯಜಮಾನ-ಸ್ವರೂಪಿಣೀ ॥೧೬೪॥

ಧರ್ಮಾಧಾರಾ ಧನಾಧ್ಯಕ್ಷಾ ಧನಧಾನ್ಯ-ವಿವರ್ಧಿನೀ~।\\
ವಿಪ್ರಪ್ರಿಯಾ ವಿಪ್ರರೂಪಾ ವಿಶ್ವಭ್ರಮಣ-ಕಾರಿಣೀ ॥೧೬೫॥

ವಿಶ್ವಗ್ರಾಸಾ ವಿದ್ರುಮಾಭಾ ವೈಷ್ಣವೀ ವಿಷ್ಣುರೂಪಿಣೀ~।\\
ಅಯೋನಿರ್ಯೋನಿನಿಲಯಾ ಕೂಟಸ್ಥಾ ಕುಲರೂಪಿಣೀ ॥೧೬೬॥

ವೀರಗೋಷ್ಠೀಪ್ರಿಯಾ ವೀರಾ ನೈಷ್ಕರ್ಮ್ಯಾ \as{(೯೦೦)} ನಾದರೂಪಿಣೀ~।\\
ವಿಜ್ಞಾನಕಲನಾ ಕಲ್ಯಾ ವಿದಗ್ಧಾ ಬೈಂದವಾಸನಾ ॥೧೬೭॥

ತತ್ತ್ವಾಧಿಕಾ ತತ್ತ್ವಮಯೀ ತತ್ತ್ವಮರ್ಥ-ಸ್ವರೂಪಿಣೀ~।\\
ಸಾಮಗಾನಪ್ರಿಯಾ ಸೌಮ್ಯಾ ಸದಾಶಿವ-ಕುಟುಂಬಿನೀ ॥೧೬೮॥

ಸವ್ಯಾಪಸವ್ಯ-ಮಾರ್ಗಸ್ಥಾ ಸರ್ವಾಪದ್ವಿನಿವಾರಿಣೀ~।\\
ಸ್ವಸ್ಥಾ ಸ್ವಭಾವಮಧುರಾ ಧೀರಾ ಧೀರಸಮರ್ಚಿತಾ ॥೧೬೯॥

ಚೈತನ್ಯಾರ್ಘ್ಯ-ಸಮಾರಾಧ್ಯಾ ಚೈತನ್ಯ-ಕುಸುಮಪ್ರಿಯಾ~।\\
ಸದೋದಿತಾ ಸದಾತುಷ್ಟಾ ತರುಣಾದಿತ್ಯ-ಪಾಟಲಾ ॥೧೭೦॥

ದಕ್ಷಿಣಾ-ದಕ್ಷಿಣಾರಾಧ್ಯಾ ದರಸ್ಮೇರ-ಮುಖಾಂಬುಜಾ~।\\
ಕೌಲಿನೀ-ಕೇವಲಾಽನರ್ಘ್ಯ-ಕೈವಲ್ಯ-ಪದದಾಯಿನೀ ॥೧೭೧॥

ಸ್ತೋತ್ರಪ್ರಿಯಾ ಸ್ತುತಿಮತೀ ಶ್ರುತಿ-ಸಂಸ್ತುತ-ವೈಭವಾ~।\\
ಮನಸ್ವಿನೀ ಮಾನವತೀ ಮಹೇಶೀ ಮಂಗಲಾಕೃತಿಃ ॥೧೭೨॥

ವಿಶ್ವಮಾತಾ ಜಗದ್ಧಾತ್ರೀ ವಿಶಾಲಾಕ್ಷೀ ವಿರಾಗಿಣೀ~।\\
ಪ್ರಗಲ್ಭಾ ಪರಮೋದಾರಾ ಪರಾಮೋದಾ ಮನೋಮಯೀ ॥೧೭೩॥

ವ್ಯೋಮಕೇಶೀ ವಿಮಾನಸ್ಥಾ ವಜ್ರಿಣೀ ವಾಮಕೇಶ್ವರೀ~।\\
ಪಂಚಯಜ್ಞ-ಪ್ರಿಯಾ ಪಂಚ-ಪ್ರೇತ-ಮಂಚಾಧಿಶಾಯಿನೀ ॥೧೭೪॥

ಪಂಚಮೀ ಪಂಚಭೂತೇಶೀ ಪಂಚ-ಸಂಖ್ಯೋಪಚಾರಿಣೀ~।\\
ಶಾಶ್ವತೀ ಶಾಶ್ವತೈಶ್ವರ್ಯಾ ಶರ್ಮದಾ ಶಂಭುಮೋಹಿನೀ ॥೧೭೫॥

ಧರಾ ಧರಸುತಾ ಧನ್ಯಾ ಧರ್ಮಿಣೀ ಧರ್ಮವರ್ಧಿನೀ~।\\
ಲೋಕಾತೀತಾ ಗುಣಾತೀತಾ ಸರ್ವಾತೀತಾ ಶಮಾತ್ಮಿಕಾ ॥೧೭೬॥

ಬಂಧೂಕ-ಕುಸುಮಪ್ರಖ್ಯಾ ಬಾಲಾ ಲೀಲಾವಿನೋದಿನೀ~।\\
ಸುಮಂಗಲೀ ಸುಖಕರೀ ಸುವೇಷಾಢ್ಯಾ ಸುವಾಸಿನೀ ॥೧೭೭॥

ಸುವಾಸಿನ್ಯರ್ಚನ-ಪ್ರೀತಾಽಽಶೋಭನಾ ಶುದ್ಧಮಾನಸಾ~।\\
ಬಿಂದು-ತರ್ಪಣ-ಸಂತುಷ್ಟಾ ಪೂರ್ವಜಾ ತ್ರಿಪುರಾಂಬಿಕಾ ॥೧೭೮॥

ದಶಮುದ್ರಾ-ಸಮಾರಾಧ್ಯಾ ತ್ರಿಪುರಾಶ್ರೀ-ವಶಂಕರೀ~।\\
ಜ್ಞಾನಮುದ್ರಾ ಜ್ಞಾನಗಮ್ಯಾ ಜ್ಞಾನಜ್ಞೇಯ-ಸ್ವರೂಪಿಣೀ ॥೧೭೯॥

ಯೋನಿಮುದ್ರಾ ತ್ರಿಖಂಡೇಶೀ ತ್ರಿಗುಣಾಂಬಾ ತ್ರಿಕೋಣಗಾ~।\\
ಅನಘಾಽದ್ಭುತ-ಚಾರಿತ್ರಾ ವಾಂಛಿತಾರ್ಥ-ಪ್ರದಾಯಿನೀ ॥೧೮೦॥

ಅಭ್ಯಾಸಾತಿಶಯ-ಜ್ಞಾತಾ ಷಡಧ್ವಾತೀತ-ರೂಪಿಣೀ~।\\
ಅವ್ಯಾಜ-ಕರುಣಾ-ಮೂರ್ತಿರಜ್ಞಾನ-ಧ್ವಾಂತ-ದೀಪಿಕಾ ॥೧೮೧॥

ಆಬಾಲ-ಗೋಪ-ವಿದಿತಾ ಸರ್ವಾನುಲ್ಲಂಘ್ಯ-ಶಾಸನಾ~।\\
ಶ್ರೀಚಕ್ರರಾಜ-ನಿಲಯಾ ಶ್ರೀಮತ್-ತ್ರಿಪುರಸುಂದರೀ ॥೧೮೨॥

ಶ್ರೀಶಿವಾ ಶಿವ-ಶಕ್ತ್ಯೈಕ್ಯ-ರೂಪಿಣೀ ಲಲಿತಾಂಬಿಕಾ\as{(೧೦೦೦)}।{\bfseries ಶ್ರೀಂಹ್ರೀಂಐಂ ಓಂ}\\
ಏವಂ ಶ್ರೀಲಲಿತಾ ದೇವ್ಯಾ ನಾಮ್ನಾಂ ಸಾಹಸ್ರಕಂ ಜಗುಃ ॥೧೮೩॥
%==========================================================================================
\section{ಸೌಭಾಗ್ಯಾಷ್ಟೋತ್ತರಶತನಾಮಸ್ತೋತ್ರಂ }
\addcontentsline{toc}{section}{ಸೌಭಾಗ್ಯಾಷ್ಟೋತ್ತರಶತನಾಮಸ್ತೋತ್ರಂ }
ಸೌಭಾಗ್ಯಾಷ್ಟೋತ್ತರಶತನಾಮಸ್ತೋತ್ರಸ್ಯ ಶಿವ ಋಷಿಃ । ಅನುಷ್ಟುಪ್ಛಂದಃ । ಶ್ರೀಲಲಿತಾಂಬಿಕಾ  ದೇವತಾ ॥ ಕೂಟತ್ರಯೇಣ ನ್ಯಾಸಃ॥

\as{ಓಂ ಐಂಹ್ರೀಂ ಶ್ರೀಂ}\\
ಕಾಮೇಶ್ವರೀ ಕಾಮಶಕ್ತಿಃ ಕಾಮಸೌಭಾಗ್ಯದಾಯಿನೀ।\\
ಕಾಮರೂಪಾ ಕಾಮಕಲಾ ಕಾಮಿನೀ ಕಮಲಾಸನಾ ॥೧॥

ಕಮಲಾ ಕಲ್ಪನಾಹೀನಾ ಕಮನೀಯಾ ಕಲಾವತೀ~।\\
ಕಮಲಾ ಭಾರತೀಸೇವ್ಯಾ ಕಲ್ಪಿತಾಶೇಷಸಂಸೃತಿಃ ॥೨॥

ಅನುತ್ತರಾಽನಘಾಽನಂತಾಽದ್ಭುತರೂಪಾಽನಲೋದ್ಭವಾ~।\\
ಅತಿಲೋಕಚರಿತ್ರಾಽತಿಸುಂದರ್ಯತಿಶುಭಪ್ರದಾ ॥೩॥

ಅಘಹಂತ್ರ್ಯತಿವಿಸ್ತಾರಾಽರ್ಚನತುಷ್ಟಾಽಮಿತಪ್ರಭಾ~।\\
ಏಕರೂಪೈಕವೀರೈಕನಾಥೈಕಾಂತಾಽರ್ಚನಪ್ರಿಯಾ ॥೪॥

ಏಕೈಕಭಾವತುಷ್ಟೈಕರಸೈಕಾಂತಜನಪ್ರಿಯಾ~।\\
ಏಧಮಾನಪ್ರಭಾವೈಧದ್ಭಕ್ತಪಾತಕನಾಶಿನೀ ॥೫॥

ಏಲಾಮೋದಮುಖೈನೋಽದ್ರಿಶಕ್ರಾಯುಧಸಮಸ್ಥಿತಿಃ~।\\
ಈಹಾಶೂನ್ಯೇಪ್ಸಿತೇಶಾದಿಸೇವ್ಯೇಶಾನವರಾಂಗನಾ ॥೬॥

ಈಶ್ವರಾಽಽಜ್ಞಾಪಿಕೇಕಾರಭಾವ್ಯೇಪ್ಸಿತಫಲಪ್ರದಾ~।\\
ಈಶಾನೇತಿಹರೇಕ್ಷೇಷದರುಣಾಕ್ಷೀಶ್ವರೇಶ್ವರೀ ॥೭॥

ಲಲಿತಾ ಲಲನಾರೂಪಾ ಲಯಹೀನಾ ಲಸತ್ತನುಃ~।\\
ಲಯಸರ್ವಾ ಲಯಕ್ಷೋಣಿರ್ಲಯಕರ್ಣೀ ಲಯಾತ್ಮಿಕಾ ॥೮॥

ಲಘಿಮಾ ಲಘುಮಧ್ಯಾಽಽಢ್ಯಾ ಲಲಮಾನಾ ಲಘುದ್ರುತಾ~।\\
ಹಯಾಽಽರೂಢಾ ಹತಾಽಮಿತ್ರಾ ಹರಕಾಂತಾ ಹರಿಸ್ತುತಾ ॥೯॥

ಹಯಗ್ರೀವೇಷ್ಟದಾ ಹಾಲಾಪ್ರಿಯಾ ಹರ್ಷಸಮುದ್ಧತಾ~।\\
ಹರ್ಷಣಾ ಹಲ್ಲಕಾಭಾಂಗೀ ಹಸ್ತ್ಯಂತೈಶ್ವರ್ಯದಾಯಿನೀ ॥೧೦॥

ಹಲಹಸ್ತಾಽರ್ಚಿತಪದಾ ಹವಿರ್ದಾನಪ್ರಸಾದಿನೀ~।\\
ರಾಮರಾಮಾಽರ್ಚಿತಾ ರಾಜ್ಞೀ ರಮ್ಯಾ ರವಮಯೀ ರತಿಃ ॥೧೧॥

ರಕ್ಷಿಣೀರಮಣೀರಾಕಾ ರಮಣೀಮಂಡಲಪ್ರಿಯಾ~।\\
ರಕ್ಷಿತಾಽಖಿಲಲೋಕೇಶಾ ರಕ್ಷೋಗಣನಿಷೂದಿನೀ ॥೧೨॥

ಅಂಬಾಂತಕಾರಿಣ್ಯಂಭೋಜಪ್ರಿಯಾಂತಕಭಯಂಕರೀ~।\\
ಅಂಬುರೂಪಾಂಬುಜಕರಾಂಬುಜಜಾತವರಪ್ರದಾ ॥೧೩॥

ಅಂತಃಪೂಜಾಪ್ರಿಯಾಂತಃಸ್ವರೂಪಿಣ್ಯಂತರ್ವಚೋಮಯೀ~।\\
ಅಂತಕಾರಾತಿವಾಮಾಂಕಸ್ಥಿತಾಂತಃಸುಖರೂಪಿಣೀ ॥೧೪॥

ಸರ್ವಜ್ಞಾ ಸರ್ವಗಾ ಸಾರಾ ಸಮಾ ಸಮಸುಖಾ ಸತೀ~।\\
ಸಂತತಿಃ ಸಂತತಾ ಸೋಮಾ ಸರ್ವಾ ಸಾಂಖ್ಯಾ ಸನಾತನೀ ॥೧೫॥\as{ಶ್ರೀಂಹ್ರೀಂಐಂ}
\authorline{॥ಇತಿ ಸೌಭಾಗ್ಯಾಷ್ಟೋತ್ತರಶತನಾಮಸ್ತೋತ್ರಂ ॥}
%=====================================================================================
\section{ಶ್ರೀಲಲಿತಾಷ್ಟೋತ್ತರಶತನಾಮಾವಲಿಃ}
\addcontentsline{toc}{section}{ಶ್ರೀಲಲಿತಾಷ್ಟೋತ್ತರಶತನಾಮಾವಲಿಃ}
\begin{multicols}{2}
ಓಂ ಭೂರೂಪಸಕಲಾಧಾರಾಯೈ ।\\ಬೀಜೌಷಧ್ಯನ್ನರೂಪಿಣ್ಯೈ ।\\ಜರಾಯುಜಾಂಡಜೋದ್ಭಿಜ್ಜ\\ಸ್ವೇದಜಾದಿಶರೀರಿಣ್ಯೈ ।\\ಕ್ಷೇತ್ರರೂಪಾಯೈ ।\\ತೀರ್ಥರೂಪಾಯೈ ।\\ಗಿರಿಕಾನನರೂಪಿಣ್ಯೈ ।\\ಜಲರೂಪಾಖಿಲಾಪ್ಯಾಯಾಯೈ ।\\ತೇಜಃಪುಂಜಸ್ವರೂಪಿಣ್ಯೈ ।\\ಜಗತ್ಪ್ರಕಾಶಿಕಾಯೈ ।\\ಅಜ್ಞಾನತಮೋಹೃದ್ಭಾನುರೂಪಿಣ್ಯೈ । ೧೦\\ವಾಯುರೂಪಾಯೈ ।\\ಅಖಿಲವ್ಯಾಪ್ತಾಯೈ ।\\ಉತ್ಪತ್ಯಾದಿವಿಧಾಯಿನ್ಯೈ ।\\ನಭೋರೂಪಾಯೈ ।\\ಇಂದುಸೂರ್ಯಾದಿ \\ಜ್ಯೋತಿರ್ಭೂತಾವಕಾಶದಾಯೈ ।\\ಘ್ರಾಣರೂಪಾಯೈ ।\\ಗಂಧರೂಪಾಯೈ ।\\ಗಂಧಗ್ರಹಣಕಾರಿಣ್ಯೈ ।\\ರಸನಾಯೈ ।\\ರಸರೂಪಾಯೈ । ೨೦\\ರಸಗ್ರಹಣಕಾರಿಣ್ಯೈ ।\\ಚಕ್ಷುರೂಪಾಯೈ ।\\ರೂಪರೂಪಾಯೈ ।\\ರೂಪಗ್ರಹಣಕಾರಿಣ್ಯೈ ।\\ತ್ವಗ್ರೂಪಾಯೈ ।\\ಸ್ಪರ್ಶರೂಪಾಯೈ ।\\ಸ್ಪರ್ಶಗ್ರಹಣಕಾರಿಣ್ಯೈ ।\\ಶ್ರೋತ್ರರೂಪಾಯೈ ।\\ಶಬ್ದರೂಪಾಯೈ ।\\ಶಬ್ದಗ್ರಹಣಕಾರಿಣ್ಯೈ । ೩೦\\ವಾಗಿಂದ್ರಿಯಸ್ವರೂಪಾಯೈ ।\\ವಾಚಾವೃತ್ತಿಪ್ರದಾಯಿನ್ಯೈ ।\\ಪಾಣೀಂದ್ರಿಯಸ್ವರೂಪಾಯೈ ।\\ಕ್ರಿಯಾವೃತ್ತಿಪ್ರದಾಯಿನ್ಯೈ ।\\ಪಾದೇಂದ್ರಿಯಸ್ವರೂಪಾಯೈ ।\\ಗತಿವೃತ್ತಿಪ್ರದಾಯಿನ್ಯೈ ।\\ಪಾಯ್ವಿಂದ್ರಿಯಸ್ವರೂಪಾಯೈ ।\\ವಿಸರ್ಗಾರ್ಥೈಕಕಾರಿಣ್ಯೈ ।\\ರಹಸ್ಯೇಂದ್ರಿಯರೂಪಾಯೈ ।\\ವಿಷಯಾನಂದದಾಯಿನ್ಯೈ । ೪೦\\ಮನೋರೂಪಾಯೈ ।\\ಸಂಕಲ್ಪವಿಕಲ್ಪಾದಿ ಸ್ವರೂಪಿಣ್ಯೈ ।\\ಸರ್ವೋಪಲಬ್ಧಿಹೇತವೇ ।\\ಬುದ್ಧಿನಿಶ್ಚಯರೂಪಿಣ್ಯೈ ।\\ಅಹಂಕಾರಸ್ವರೂಪಾಯೈ ।\\ಅಹಂಕರ್ತವ್ಯವೃತ್ತಿದಾಯೈ ।\\ಚೇತನಾಚಿತ್ತರೂಪಾಯೈ ।\\ಸರ್ವಚೈತನ್ಯದಾಯಿನ್ಯೈ ।\\ಗುಣವೈಷಮ್ಯರೂಪಾಢ್ಯ\\ಮಹತ್ತತ್ತ್ವಾಭಿಮಾನಿನ್ಯೈ ।\\ಗುಣಸಾಮ್ಯಾವ್ಯಕ್ತಮಾಯಾಮೂಲ\\ಪ್ರಕೃತಿಸಂಚಿಕಾಯೈ । ೫೦\\ಪಂಚೀಕೃತಮಹಾಭೂತ\\ಸೂಕ್ಷ್ಮಭೂತಸ್ವರೂಪಿಣ್ಯೈ ।\\ವಿದ್ಯಾಽವಿದ್ಯಾತ್ಮಿಕಾಯೈ ।\\ಮಾಯಾಬಂಧಮೋಚನಕಾರಿಣ್ಯೈ ।\\ಈಶ್ವರೇಚ್ಛಾರಾಗರೂಪಾಯೈ ।\\ಪ್ರಕೃತಿಕ್ಷೋಭಕಾರಿಣ್ಯೈ ।\\ಕಾಲಶಕ್ತ್ಯೈ ।\\ಕಾಲರೂಪಾಯೈ ।\\ನಿಯತ್ಯಾದಿನಿಯಾಮಿಕಾಯೈ ।\\ಧೂಮ್ರಾದಿಪಂಚವ್ಯೋಮಾಖ್ಯಾಯೈ ।\\ಯಂತ್ರಮಂತ್ರಕಲಾತ್ಮಿಕಾಯೈ । ೬೦\\ಬ್ರಹ್ಮರೂಪಾಯೈ ।\\ವಿಷ್ಣುರೂಪಾಯೈ ।\\ರುದ್ರರೂಪಾಯೈ ।\\ಮಹೇಶ್ವರ್ಯೈ ।\\ಸದಾಶಿವಸ್ವರೂಪಾಯೈ ।\\ಸರ್ವಜೀವಮಯ್ಯೈ ।\\ಶಿವಾಯೈ ।\\ಶ್ರೀವಾಣೀಲಕ್ಷ್ಮ್ಯುಮಾರೂಪಾಯೈ ।\\ಸದಾಖ್ಯಾಯೈ ।\\ಚಿತ್ಕಲಾತ್ಮಿಕಾಯೈ । ೭೦\\ಪ್ರಾಜ್ಞತೈಜಸವಿಶ್ವಾಖ್ಯ\\ವಿರಾಟ್ಸೂತ್ರೇಶ್ವರಾತ್ಮಿಕಾಯೈ ।\\ಸ್ಥೂಲದೇಹಸ್ವರೂಪಾಯೈ ।\\ಸೂಕ್ಷ್ಮದೇಹಸ್ವರೂಪಿಣ್ಯೈ ।\\ವಾಚ್ಯವಾಚಕರೂಪಾಯೈ ।\\ಜ್ಞಾನಜ್ಞೇಯಸ್ವರೂಪಿಣ್ಯೈ ।\\ಕಾರ್ಯಕಾರಣರೂಪಾಯೈ ।\\ತತ್ತತ್ತತ್ವಾಧಿದೇವತಾಯೈ ।\\ದಶನಾದಸ್ವರೂಪಾಯೈ ।\\ನಾಡೀರೂಪಾಢ್ಯಕುಂಡಲ್ಯೈ ।\\ಅಕಾರಾದಿಕ್ಷಕಾರಾಂತವೈಖರೀ\\ವಾಕ್ಸ್ವರೂಪಿಣ್ಯೈ । ೮೦\\ವೇದವೇದಾಂಗರೂಪಾಯೈ ।\\ಸೂತ್ರಶಾಸ್ತ್ರಾದಿರೂಪಿಣ್ಯೈ ।\\ಪುರಾಣರೂಪಾಯೈ ।\\ಸದ್ಧರ್ಮಶಾತ್ರರೂಪಾಯೈ ।\\ಪರಾತ್ಪರಸ್ಯೈ ।\\ಆಯುರ್ವೇದಸ್ವರೂಪಾಯೈ ।\\ಧನುರ್ವೇದಸ್ವರೂಪಿಣ್ಯೈ ।\\ಗಾಂಧರ್ವವಿದ್ಯಾರೂಪಾಯೈ ।\\ಅರ್ಥಶಾಸ್ತ್ರಾದಿರೂಪಿಣ್ಯೈ ।\\ಚತುಷ್ಷಷ್ಟಿಕಲಾರೂಪಾಯೈ । ೯೦\\ನಿಗಮಾಗಮರೂಪಿಣ್ಯೈ ।\\ಕಾವ್ಯೇತಿಹಾಸರೂಪಾಯೈ ।\\ಗಾನವಿದ್ಯಾದಿರೂಪಿಣ್ಯೈ ।\\ಪದವಾಕ್ಯಸ್ವರೂಪಾಯೈ ।\\ಸರ್ವಭಾಷಾಸ್ವರೂಪಿಣ್ಯೈ ।\\ಪದವಾಕ್ಯಸ್ಫೋಟರೂಪಾಯೈ ।\\ಜ್ಞಾನಜ್ಞೇಯಕ್ರಿಯಾತ್ಮಿಕಾಯೈ ।\\ಸರ್ವತಂತ್ರಮಯ್ಯೈ ।\\ಸರ್ವಯಂತ್ರತಂತ್ರಾದಿರೂಪಿಣ್ಯೈ ।\\ವೇದಮಾತ್ರೇ । ೧೦೦\\ಲಲಿತಾಯೈ ।\\ಮಹಾವ್ಯಾಹೃತಿರೂಪಿಣ್ಯೈ ।\\ಅವ್ಯಾಕೃತಪದಾನಾದ್ಯಚಿಂತ್ಯ ಶಕ್ತ್ಯೈ ।\\ತಮೋಮಯ್ಯೈ ।\\ಪರಸ್ಮೈ ಜ್ಯೋತಿಷೇ ।\\ಪರಬ್ರಹ್ಮಸಾಕ್ಷಾತ್ಕಾರ  ಸ್ವರೂಪಿಣ್ಯೈ ।\\ಪರಬ್ರಹ್ಮಮಯ್ಯೈ ।\\ಸತ್ಯಾಸತ್ಯಜ್ಞಾನಸುಧಾತ್ಮಿಕಾಯೈ ನಮಃ । ೧೦೮\\\end{multicols}\authorline{ಇತಿ ಶ್ರೀಲಲಿತಾಷ್ಟೋತ್ತರಶತನಾಮಾವಲಿಃ ಸಮಾಪ್ತಾ ।}
%==========================================================================================
\section{ ಶ್ರೀ ಶಿವಕಾಮಸುಂದರೀ ಸಹಸ್ರನಾಮಸ್ತೋತ್ರಂ }
\addcontentsline{toc}{section}{ ಶ್ರೀ ಶಿವಕಾಮಸುಂದರೀ ಸಹಸ್ರನಾಮಸ್ತೋತ್ರಂ }
ಪೂರ್ವಪೀಠಿಕಾ ॥
ಸಮಾಹೂಯ ಪರಂ ಕಾಂತಂ ಏಕದಾ ವಿಜನೇ ಮುದಾ।
ಪರಮಾನಂದಸಂದೋಹಮುದಿತಂ ಪ್ರಾಹ ಪಾರ್ವತೀ ॥೧॥

ಪಾರ್ವತೀ ಉವಾಚ
ಶ್ರೀಮನ್ನಾಥ ಮಹಾನಂದಕಾರಣಂ ಬ್ರೂಹಿ ಶಂಕರ।
ಯೋಗೀಂದ್ರೋಪಾಸ್ಯ ದೇವೇಶ ಪ್ರೇಮಪೂರ್ಣ ಸುಧಾನಿಧೇ ॥೨॥

ಕೃಪಾಸ್ತಿ ಮಯಿ ಚೇತ್ ಶಂಭೋ ಸುಗೋಪ್ಯಮಪಿ ಕಥ್ಯತಾಮ್।
ಶಿವಕಾಮೇಶ್ವರೀನಾಮಸಾಹಸ್ರಂ ವದ ಮೇ ಪ್ರಭೋ ॥೩॥

ಶ್ರೀಶಂಕರ ಉವಾಚ
ನಿರ್ಭರಾನಂದಸಂದೋಹಃ ಶಕ್ತಿಭಾವೇನ ಜಾಯತೇ।
ಲಾವಣ್ಯಸಿಂಧುಸ್ತನ್ನಾಪಿ ಸುಂದರೀ ರಸಕಂಧರಾ ॥೪॥

ತಾಮೇವಾನುಕ್ಷಣಂ ದೇವಿ ಚಿಂತಯಾಮಿ ತತಃ ಶಿವೇ।
ತಸ್ಯಾ ನಾಮಸಹಸ್ರಾಣಿ ಕಥಯಾಮಿ ತವ ಪ್ರಿಯೇ ॥೫॥

ಸುಗೋಪ್ಯಾನ್ಯಪಿ ರಂಭೋರು ಗಂಭೀರಸ್ನೇಹವಿಭ್ರಮಾತ್।
ತಾಮೇವ ಸ್ತುವತಾ ದೇವೀಂ ಧ್ಯಾಯತೋऽನುಕ್ಷಣಂ ಮಮ।
ಸುಖಸಂದೋಹಸಂಭಾರಭಾವನಾನಂದಕಾರಣಮ್ ॥೬॥

ಅಸ್ಯ ಶ್ರೀಶಿವಕಾಮಸುಂದರೀಸಹಸ್ರನಾಮ ಸ್ತೋತ್ರಮಹಾಮಂತ್ರಸ್ಯ~।\\
ಸದಾಶಿವ ಋಷಿಃ~। ಅನುಷ್ಟುಪ್ ಛಂದಃ~। ಶ್ರೀಮಚ್ಛಿವಕಾಮಸುಂದರೀ ದೇವತಾ~। ವಾಗ್ಭವಸ್ವರೂಪಂ ಐಂ ಬೀಜಂ~। ಚಿದಾನಂದಾತ್ಮಕಂ ಹ್ರೀಂ ಶಕ್ತಿಃ~। ಕಾಮರಾಜಾತ್ಮಕಂ ಕ್ಲೀಂ ಕೀಲಕಂ~। ಜಪೇ ವಿನಿಯೋಗಃ ॥\\
ಷೋಡಶರ್ಣಾಮೂಲೇನ ನ್ಯಾಸಃ ॥

\dhyana{ಸಿದ್ಧಸಿದ್ಧನವರತ್ನಭೂಮಿಕೇ ಕಲ್ಪವೃಕ್ಷನವವಾಟಿಸಂವೃತೇ~।\\
	ರತ್ನಸಾಲವನಸಂಭೃತೇಽನಿಶಂ ತತ್ರ ವಾಪಿಶತಕೇನ ಸಂವೃತೇ ॥೭॥

ರತ್ನವಾಟಿಮಣಿಮಂಡಪೇಽರುಣೇ ಚಂಡಭಾನುಶತಕೋಟಿಭಾಸುರೇ~।\\
ಆದಿಶೈವಮಣಿಮಂಚಕೇ ಪರೇ ಶಂಕರಾಂಕಮಣಿಪೀಠಕೋಪರಿ ॥\\
	ಕಾದಿಹಾಂತಮನುರೂಪಿಣೀಂ ಶಿವಾಂ ಸಂಸ್ಮರೇಚ್ಚ ಶಿವಕಾಮಸುಂದರೀಂ॥೮॥}

	ಶ್ರೀಶಿವಾ ಶಿವಕಾಮೀ ಚ ಸುಂದರೀ ಭುವನೇಶ್ವರೀ~।\\
	ಆನಂದಸಿಂಧುರಾನಂದಾನಂದಮೂರ್ತಿರ್ವಿನೋದಿನೀ ॥೧॥

ತ್ರೈಪುರೀ ಸುಂದರೀ ಪ್ರೇಮಪಾಥೋನಿಧಿರನುತ್ತಮಾ~।\\
ರಾಮೋಲ್ಲಾಸಾ ಪರಾ ಭೂತಿಃ ವಿಭೂತಿಶ್ಶಂಕರಪ್ರಿಯಾ ॥೨॥

	ಶೃಂಗಾರಮೂರ್ತಿರ್ವಿರತಾ ರಸಾನುಭವರೋಚನಾ~।\\
	ಪರಮಾನಂದಲಹರೀ ರತಿರಂಗವತೀ ಸತೀ ॥೩॥

ರಂಗಮಾಲಾನಂಗಕಲಾಕೇಲಿಃ ಕೈವಲ್ಯದಾ ಕಲಾ~।\\
ರಸಕಲ್ಪಾ ಕಲ್ಪಲತಾ ಕುತೂಹಲವತೀ ಗತಿಃ ॥೪॥

	ವಿನೋದದುಗ್ಧಾ ಸುಸ್ನಿಗ್ಧಾ ಮುಗ್ಧಮೂರ್ತಿರ್ಮನೋಹರಾ~।\\
	ಬಾಲಾರ್ಕಕೋಟಿಕಿರಣಾ ಚಂದ್ರಕೋಟಿಸುಶೀತಲಾ ॥೫॥

ಸ್ರವತ್ಪೀಯೂಷದಿಗ್ಧಾಂಗೀ ಸಂಗೀತ ನಟಿಕಾ ಶಿವಾ~।\\
ಕುರಂಗನಯನಾ ಕಾಂತಾ ಸುಖಸಂತತಿರಿಂದಿರಾ ॥೬॥

	ಮಂಗಲಾ ಮಧುರಾಪಾಂಗಾ ರಂಜನೀ ರಮಣೀ ರತಿಃ~।\\
	ರಾಜರಾಜೇಶ್ವರೀ ರಾಜ್ಞೀ ಮಹೇಂದ್ರಪರಿವಂದಿತಾ ॥೭॥

ಪ್ರಪಂಚಗತಿರೀಶಾನೀ ಸಾಮರಸ್ಯಪರಾಯಣಾ~।\\
ಹಂಸೋಲ್ಲಾಸಾ ಹಂಸಗತಿಃ ಶಿಂಜತ್ಕನಕನೂಪುರಾ ॥೮॥

	ಮೇರುಮಂದರವಕ್ಷೋಜಾ ಸೃಣಿಪಾಶವರಾಯುಧಾ~।\\
	ಶಂಖಕೋದಂಡಪುಸ್ತಾಬ್ಜಪಾಣಿದ್ವಯವಿರಾಜಿತಾ ॥೯॥

ಚಂದ್ರಬಿಂಬಾನನಾ ಚಾರುಮಕುಟೋತ್ತಂಸಚಂದ್ರಿಕಾ~।\\
ಸಿಂದೂರತಿಲಕಾ ಚಾರುಧಮ್ಮಿಲ್ಲಾಮಲಮಾಲಿಕಾ ॥೧೦॥

	ಮಂದಾರದಾಮಮುದಿತಾ ರಕ್ತಪುಷ್ಪವಿಭೂಷಿತಾ~।\\
	ಸುವರ್ಣಾಭರಣಪ್ರೀತಾ ಮುಕ್ತಾದಾಮಮನೋಹರಾ ॥೧೧॥

ತಾಂಬೂಲಪೂರವದನಾ ಮದನಾನಂದಮಾನಸಾ~।\\
ಸುಖಾರಾಧ್ಯಾ ತಪಸ್ಸಾರಾ ಕೃಪಾವಾರಿಧಿರೀಶ್ವರೀ ॥೧೨॥

	ವಕ್ಷಃಸ್ಥಲಲಸನ್ಮಗ್ನಾ ಪ್ರಭಾ ಮಧುರಸೋನ್ಮುಖಾ~।\\
	ಬಿಂದುನಾದಾತ್ಮಿಕಾ ಚಾರುರಸಿತಾ ತುರ್ಯರೂಪಿಣೀ ॥೧೩॥

ಕಮನೀಯಾಕೃತಿಃ ಧನ್ಯಾ ಶಂಕರಪ್ರೀತಿಮಂಜರೀ~।\\
ಕನ್ಯಾ ಕಲಾವತೀ ಮಾತಾ ಗಜೇಂದ್ರಗಮನಾ ಶುಭಾ ॥೧೪॥

	ಕುಮಾರೀ ಕರಭೋರುಶ್ರೀಃ ರೂಪಲಕ್ಷ್ಮೀಃ ಸುರಾಜಿತಾ~।\\
	ಸಂತೋಷಸೀಮಾ ಸಂಪತ್ತಿಃ ಶಾತಕುಂಭಪ್ರಿಯಾ ದ್ಯುತಿಃ ॥೧೫॥

ಪರಿಪೂರ್ಣಾ ಜಗದ್ಧಾತ್ರೀ ವಿಧಾತ್ರೀ ಬಲವರ್ಧಿನೀ~।\\
ಸಾರ್ವಭೌಮನೃಪಶ್ರೀಶ್ಚ ಸಾಮ್ರಾಜ್ಯಗತಿರಾಸಿಕಾ ॥೧೬॥

	ಸರೋಜಾಕ್ಷೀ ದೀರ್ಘದೃಷ್ಟಿಃ ಸೌಚಕ್ಷಣವಿಚಕ್ಷಣಾ~।\\
	ರಂಗಸ್ರವಂತೀ ರಸಿಕಾ ಪ್ರಧಾನರಸರೂಪಿಣೀ ॥೧೭॥

ರಸಸಿಂಧುಃ ಸುಗಾತ್ರೀ ಚ ಯುವತಿಃ ಮೈಥುನೋನ್ಮುಖೀ~।\\
ನಿರಂತರಾ ರಸಾಸಕ್ತಾ ಶಕ್ತಿಸ್ತ್ರಿಭುವನಾತ್ಮಿಕಾ ॥೧೮॥

ಕಾಮಾಕ್ಷೀ ಕಾಮನಿಷ್ಠಾ ಚ ಕಾಮೇಶೀ ಭಗಮಂಗಲಾ~।\\
ಸುಭಗಾ ಭಗಿನೀ ಭೋಗ್ಯಾ ಭಾಗ್ಯದಾ ಭಯದಾ ಭಗಾ~।\\
ಭಗಲಿಂಗಾನಂದಕಲಾ ಭಗಮಧ್ಯನಿವಾಸಿನೀ ॥೧೯॥

	ಭಗರೂಪಾ ಭಗಮಯೀ ಭಗಯಂತ್ರಾ ಭಗೋತ್ತಮಾ~।\\
	ಯೋನಿರ್ಜಯಾ ಕಾಮಕಲಾ ಕುಲಾಮೃತಪರಾಯಣಾ ॥೨೦॥

ಕುಲಕುಂಡಾಲಯಾ ಸೂಕ್ಷ್ಮಜೀವಸ್ಫುಲಿಂಗರೂಪಿಣೀ~।\\
ಮೂಲಸ್ಥಿತಾ ಕೇಲಿರತಾ ವಲಯಾಕೃತಿರೀಡಿತಾ ॥೨೧॥

	ಸುಷುಮ್ನಾ ಕಮಲಾನಂದಾ ಚಿತ್ರಾ ಕೂರ್ಮಗತಿರ್ಗಿರಿಃ~।\\
	ಸಿತಾರುಣಾ ಸಿಂಧುರೂಪಾ ಪ್ರವೇಗಾ ನಿರ್ಧನೀ ಕ್ಷಮಾ ॥೨೨॥

ಘಂಟಾಕೋಟಿರಸಾರಾವಾ ರವಿಬಿಂಬೋತ್ಥಿತಾದ್ಭುತಾ~।\\
ನಾದಾಂತಲೀನಾ ಸಂಪೂರ್ಣಾ ಪ್ರಣವಾ ಬಹುರೂಪಿಣೀ ॥೨೩॥

	ಭೃಂಗಾರಾವಾ ವಶಗತಿಃ ವಾಗೀಶೀ ಮಧುರಧ್ವನಿಃ~।\\
	ವರ್ಣಮಾಲಾ ಸಿದ್ಧಿಕಲಾ ಷಟ್ಚಕ್ರಕ್ರಮವಾಸಿನೀ ॥೨೪॥

ಮಣಿಪೂರಸ್ಥಿತಾ ಸ್ನಿಗ್ಧಾ ಕೂರ್ಮಚಕ್ರಪರಾಯಣಾ~।\\
ಮೂಲಕೇಲಿರತಾ ಸಾಧ್ವೀ ಸ್ವಾಧಿಷ್ಠಾನನಿವಾಸಿನೀ ॥೨೫॥

	ಅನಾಹತಗತಿರ್ದೀಪಾ ಶಿವಾನಂದಮಯದ್ಯುತಿಃ~।\\
	ವಿರುದ್ಧರುಧಾ ಸಂಬುದ್ಧಾ ಜೀವಭೋಕ್ತ್ರೀ ಸ್ಥಲೀರತಾ ॥೨೬॥

ಆಜ್ಞಾಚಕ್ರೋಜ್ಜ್ವಲಸ್ಫಾರಸ್ಫುರಂತೀ ನಿರ್ಗತದ್ವಿಷಾ~।\\
ಚಂದ್ರಿಕಾ ಚಂದ್ರಕೋಟೀಶೀ ಸೂರ್ಯಕೋಟಿಪ್ರಭಾಮಯೀ ॥೨೭॥

	ಪದ್ಮರಾಗಾರುಣಚ್ಛಾಯಾ ನಿತ್ಯಾಹ್ಲಾದಮಯೀಪ್ರಭಾ~।\\
	ಮಹಾಶೂನ್ಯಾಲಯಾ ಚಂದ್ರಮಂಡಲಾಮೃತನಂದಿತಾ ॥೨೮॥

ಕಾಂತಾಂಗಸಂಗಮುದಿತಾ ಸುಧಾಮಾಧುರ್ಯಸಂಭೃತಾ~।\\
ಮಹಾಚಂದ್ರಸ್ಮಿತಾಲಸಾ ಮೃತ್ಪಾತ್ರಸ್ಥಾ ಸುಧಾದ್ಯುತಿಃ ॥೨೯॥

	ಸ್ರವತ್ಪೀಯೂಷಸಂಸಕ್ತಾ ಶಶ್ವತ್ಕುಂಡಾಲಯಾ ಭವಾ~।\\
	ಶ್ರೇಯೋ ದ್ಯುತಿಃ ಪ್ರತ್ಯಗರ್ಥಾ ಸೇವಾ ಫಲವತೀ ಮಹೀ ॥೩೦॥

ಶಿವಾ ಶಿವಪ್ರಿಯಾ ಶೈವಾ ಶಂಕರೀ ಶಾಂಭವೀ ವಿಭುಃ~।\\
ಸ್ವಯಂಭೂಃ ಸ್ವಪ್ರಿಯಾ ಸ್ವೀಯಾ ಸ್ವಕೀಯಾ ಜನಮಾತೃಕಾ ॥

	ಸುರಾಮಾ ಸ್ವಪ್ರಿಯಾ ಶ್ರೇಯಃ ಸ್ವಾಧಿಕಾರಾಧಿನಾಯಿಕಾ~।\\
	ಮಂಡಲಾ ಜನನೀ ಮಾನ್ಯಾ ಸರ್ವಮಂಗಲಸಂತತಿಃ ॥೩೨॥

ಭದ್ರಾ ಭಗವತೀ ಭಾವ್ಯಾ ಕಲಿತಾರ್ಧೇಂದುಭಾಸುರಾ~।\\
ಕಲ್ಯಾಣಲಲಿತಾ ಕಾಮ್ಯಾ ಕುಕರ್ಮಕುಮತಿಪ್ರದಾ ॥೩೩॥

	ಕುರಂಗಾಕ್ಷೀ ಕ್ಷೀರನೇತ್ರಾ ಕ್ಷೀರಾ ಮಧುರಸೋನ್ಮದಾ~।\\
	ವಾರುಣೀಪಾನಮುದಿತಾ ಮದಿರಾಮುದಿತಾ ಸ್ಥಿರಾ ॥೩೪॥

ಕಾದಂಬರೀಪಾನರುಚಿಃ ವಿಪಾಶಾ ಪಶುಭಾವನಾ~।\\
ಮುದಿತಾ ಲಲಿತಾಪಾಂಗಾ ದರಾಂದೋಲಿತದೀರ್ಘದೃಕ್ ॥೩೫॥

	ದೈತ್ಯಕುಲಾನಲಶಿಖಾ ಮನೋರಥಸುಧಾದ್ಯುತಿಃ~।\\
	ಸುವಾಸಿನೀ ಪೀತಗಾತ್ರೀ ಪೀನಶ್ರೋಣಿಪಯೋಧರಾ ॥೩೬॥

ಸುಚಾರುಕಬರೀ ದಧ್ಯುದಧ್ಯುತ್ಥಿತ ಸುಮೌಕ್ತಿಕಾ~।\\
ಬಿಂಬಾಧರದ್ಯುತಿಃ ಮುಗ್ಧಾ ಪ್ರವಾಲೋತ್ತಮದೀಧಿತಿಃ ॥೩೭॥

	ತಿಲಪ್ರಸೂನನಾಸಾಗ್ರಾ ಹೇಮಮೌಕ್ತಿಕಕೋರಕಾ~।\\
	ನಿಷ್ಕಲಂಕೇಂದುವದನಾ ಬಾಲೇಂದುವದನೋಜ್ವಲಾ ॥೩೮॥

ನೃತ್ಯಂತ್ಯಂಜನನೇತ್ರಾಂತಾ ಪ್ರಸ್ಫುರತ್ಕರ್ಣಶಷ್ಕುಲೀ~।\\
ಭಾಲಚಂದ್ರಾತಪೋನ್ನದ್ಧಾ ಮಣಿಸೂರ್ಯಕಿರೀಟಿನೀ ॥೩೯॥

	ಕಚೌಘಚಂಪಕಶ್ರೇಣೀ ಮಾಲಿನೀದಾಮಮಂಡಿತಾ~।\\
	ಹೇಮಮಾಣಿಕ್ಯ ತಾಟಂಕಾ ಮಣಿಕಾಂಚನ ಕುಂಡಲಾ ॥೪೦॥

ಸುಚಾರುಚುಬುಕಾ ಕಂಬುಕಂಠೀ ಕುಂಡಾವಲೀ ರಮಾ~।\\
ಗಂಗಾತರಂಗಹಾರೋರ್ಮಿಃ ಮತ್ತಕೋಕಿಲನಿಸ್ವನಾ ॥೪೧॥

	ಮೃಣಾಲವಿಲಸದ್ಬಾಹು ಪಾಶಾಂಕುಶ ಧನುರ್ಧರಾ~।\\
	ಕೇಯೂರಕಂಕಣಶ್ರೇಣೀ ನಾನಾಮಣಿಮನೋರಮಾ ॥೪೨॥

ತಾಮ್ರಪಂಕಜಪಾಣಿಶ್ರೀಃ ನವರತ್ನಪ್ರಭಾವತೀ~।\\
ಅಂಗುಲೀಯಮಣಿಶ್ರೇಣೀ ಕಾಂತಿಮಂಗಲಸಂತತಿಃ ॥೪೩॥

	ಮಂದರದ್ವಂದ್ವಸುಕುಚಾ ರೋಮರಾಜಿಭುಜಂಗಿಕಾ~।\\
	ಗಂಭೀರನಾಭಿಸ್ತ್ರಿವಲೀಭಂಗುರಾ ಕ್ಷಣಿಮಧ್ಯಮಾ ॥೪೪॥

ರಣತ್ಕಾಂಚೀಗುಣಾನದ್ಧಾ ಪಟ್ಟಾಂಶುಕನಿತಂಬಿಕಾ~।\\
ಮೇರುಸಂಧಿನಿತಂಬಾಢ್ಯಾ ಗಜಶುಂಡೋರುಯುಗ್ಮಯುಕ್ ॥೪೫॥

	ಸುಜಾನುರ್ಮದನಾನಂದಮಯಜಂಘಾದ್ವಯಾನ್ವಿತಾ~।\\
	ಗೂಢಗುಲ್ಫಾ ಮಂಜುಶಿಂಜನ್ಮಣಿನೂಪುರಮಂಡಿತಾ ॥೪೬॥

ಪದದ್ವಂದ್ವಜಿತಾಂಭೋಜಾ ನಖಚಂದ್ರಾವಲೀಪ್ರಭಾ~।\\
ಸುಸೀಮಪ್ರಪದಾ ರಾಜಂಹಸಮತ್ತೇಭಮಂದಗಾ ॥೪೭॥

	ಯೋಗಿಧ್ಯೇಯಪದದ್ವಂದ್ವಾ ಸೌಂದರ್ಯಾಮೃತಸಾರಿಣೀ~।\\
	ಲಾವಣ್ಯಸಿಂಧುಃ ಸಿಂದೂರತಿಲಕಾ ಕುಟಿಲಾಲಕಾ ॥೪೮॥

ಸಾಧುಸೀಮಂತಿನೀ ಸಿದ್ಧಬುದ್ಧವೃಂದಾರಕೋದಯಾ~।\\
ಬಾಲಾರ್ಕಕಿರಣಶ್ರೇಣಿಶೋಣಶ್ರೀಃ ಪ್ರೇಮಕಾಮಧುಕ್ ॥೪೯॥

	ರಸಗಂಭೀರಸರಸೀ ಪದ್ಮಿನೀ ರಸಸಾರಸಾ~।\\
	ಪ್ರಸನ್ನಾಸನ್ನವರದಾ ಶಾರದಾ ಭುವಿ ಭಾಗ್ಯದಾ ॥೫೦॥

ನಟರಾಜಪ್ರಿಯಾ ವಿಶ್ವಾನಾದ್ಯಾ ನರ್ತಕನರ್ತಕೀ~।\\
ಚಿತ್ರಯಂತ್ರಾ ಚಿತ್ರತಂತ್ರಾ ಚಿತ್ರವಿದ್ಯಾವಲೀಯತಿಃ ॥೫೧॥

	ಚಿತ್ರಕೂಟಾ ತ್ರಿಕೂಟಾ ಚ ಪಂಚಕೂಟಾ ಚ ಪಂಚಮೀ~।\\
	ಚತುಷ್ಕೂಟಾ ಶಂಭುವಿದ್ಯಾ ಷಟ್ಕೂಟಾ ವಿಷ್ಣುಪೂಜಿತಾ ॥೫೨॥

ಕೂಟಷೋಡಶಸಂಪನ್ನಾ ತುರೀಯಾ ಪರಮಾ ಕಲಾ~।\\
ಷೋಡಶೀ ಮಂತ್ರಯಂತ್ರಾಣಾಂ ಈಶ್ವರೀ ಮೇರುಮಂಡಲಾ ॥೫೩॥

	ಷೋಡಶಾರ್ಣಾ ತ್ರಿವರ್ಣಾ ಚ ಬಿಂದುನಾದಸ್ವರೂಪಿಣೀ~।\\
	ವರ್ಣಾತೀತಾ ವರ್ಣಮತಾ ಶಬ್ದಬ್ರಹ್ಮಮಯೀ ಸುಖಾ ॥೫೪॥

ಸುಖಜ್ಯೋತ್ಸ್ನಾನಂದವಿದ್ಯುದಂತರಾಕಾಶದೇವತಾ~।\\
ಚೈತನ್ಯಾ ವಿಧಿಕೂಟಾತ್ಮಾ ಕಾಮೇಶೀ ಸ್ವಪ್ನದರ್ಶನಾ ॥೫೫॥

	ಸ್ವಪ್ನರೂಪಾ ಬೋಧಕರೀ ಜಾಗ್ರತೀ ಜಾಗರಾಶ್ರಯಾ~।\\
	ಸ್ವಪ್ನಾಶ್ರಯಾ ಸುಷುಪ್ತಿಸ್ಥಾ ತಂತ್ರಮೂರ್ತಿಶ್ಚ ಮಾಧವೀ ॥೫೬॥

ಲೋಪಾಮುದ್ರಾ ಕಾಮರಾಜ್ಞೀ ಮಾಧವೀ ಮಿತ್ರರೂಪಿಣೀ~।\\
ಶಾಂಕರೀ ನಂದಿವಿದ್ಯಾ ಚ ಭಾಸ್ವನ್ಮಂಡಲಮಧ್ಯಗಾ ॥೫೭॥

	ಮಾಹೇಂದ್ರಸ್ವರ್ಗಸಂಪತ್ತಿಃ ದುರ್ವಾಸಸ್ಸೇವಿತಾ ಶ್ರುತಿಃ~।\\
	ಸಾಧಕೇಂದ್ರಗತಿಸ್ಸಾಧ್ವೀ ಸುಲಿಪ್ತಾ ಸಿದ್ಧಿಕಂಧರಾ ॥೫೮॥

ಪುರತ್ರಯೇಶೀ ಪುರಕೃತ್ ಷಷ್ಠೀ ಚ ಪರದೇವತಾ~।\\
ವಿಘ್ನದೂರೀ ಭೂರಿಗುಣಾ ಪುಷ್ಟಿಃ ಪೂಜಿತಕಾಮಧುಕ್ ॥೫೯॥

	ಹೇರಂಬಮಾತಾ ಗಣಪಾ ಗುಹಾಂಬಾಽಽರ್ಯಾ ನಿತಂಬಿನೀ~।\\
	ಏಷಾ ಸೀಮಂತಿನೀ ಮೋಕ್ಷದಕ್ಷಾ ದೀಕ್ಷಿತಮಾತೃಕಾ ॥೬೦॥

ಸಾಧಕಾಂಬಾ ಸಿದ್ಧಮಾತಾ ಸಾಧಕೇಂದ್ರಮನೋರಮಾ~।\\
ಯೌವನೋನ್ಮಾದಿನೀ ತುಂಗಸ್ತನೀ ಸುಶ್ರೋಣಿಮಂಡಿತಾ ॥೬೧॥

	ಪದ್ಮರಕ್ತೋತ್ಪಲವತೀ ರಕ್ತಮಾಲ್ಯಾನುಲೇಪನಾ~।\\
	ರಕ್ತಮಾಲ್ಯರುಚಿರ್ದಕ್ಷಾ ಶಿಖಂಡಿನ್ಯತಿಸುಂದರೀ ॥೬೨॥

ಶಿಖಂಡಿನೃತ್ಯಸಂತುಷ್ಟಾ ಶಿಖಂಡಿಕುಲಪಾಲಿನೀ~।\\
ವಸುಂಧರಾ ಚ ಸುರಭಿಃ ಕಮನೀಯತನುಶ್ಶುಭಾ ॥೬೩॥

	ನಂದಿನೀ ತ್ರೀಕ್ಷಣವತೀ ವಸಿಷ್ಠಾಲಯದೇವತಾ~।\\
	ಗೋಲಕೇಶೀ ಚ ಲೋಕೇಂದ್ರಾ ನೃಲೋಕಪರಿಪಾಲಿಕಾ ॥೬೪॥

ಹವಿರ್ಧಾತ್ರೀ ದೇವಮಾತಾ ವೃಂದಾರಕಪರಾತ್ಮಯುಕ್~।\\
ರುದ್ರಮಾತಾ ರುದ್ರಪತ್ನೀ ಮದೋದ್ಗಾರಭರಾ ಕ್ಷಿತಿಃ ॥೬೫॥

	ದಕ್ಷಿಣಾ ಯಜ್ಞಸಂಪತ್ತಿಃ ಸ್ವಬಲಾ ಧೀರನಂದಿತಾ~।\\
	ಕ್ಷೀರಪೂರ್ಣಾರ್ಣವಗತಿಃ ಸುಧಾಯೋನಿಃ ಸುಲೋಚನಾ ॥೬೬॥

ರಮಾ ತುಂಗಾ ಸದಾಸೇವ್ಯಾ ಸುರಸಂಘದಯಾ ಉಮಾ~।\\
ಸುಚರಿತ್ರಾ ಚಿತ್ರವರಾ ಸುಸ್ತನೀ ವತ್ಸವತ್ಸಲಾ ॥೬೭॥

	ರಜಸ್ವಲಾ ರಜೋಯುಕ್ತಾ ರಂಜಿತಾ ರಂಗಮಾಲಿಕಾ~।\\
	ರಕ್ತಪ್ರಿಯಾ ಸುರಕ್ತಾ ಚ ರತಿರಂಗಸ್ವರೂಪಿಣೀ ॥೬೮॥

ರಜಶ್ಶುಕ್ಲಾಕ್ಷಿಕಾ ನಿಷ್ಠಾ ಋತುಸ್ನಾತಾ ರತಿಪ್ರಿಯಾ~।\\
ಭಾವ್ಯಾಭಾವ್ಯಾ ಕಾಮಕೇಲಿಃ ಸ್ಮರಭೂಃ ಸ್ಮರಜೀವಿಕಾ ॥೬೯॥

	ಸಮಾಧಿಕುಸುಮಾನಂದಾ ಸ್ವಯಂಭುಕುಸುಮಪ್ರಿಯಾ~।\\
	ಸ್ವಯಂಭುಪ್ರೇಮಸಂತುಷ್ಟಾ ಸ್ವಯಂಭೂನಿಂದಕಾಂತಕಾ ॥೭೦॥

ಸ್ವಯಂಭುಸ್ಥಾ ಶಕ್ತಿಪುಟಾ ರವಿಃ ಸರ್ವಸ್ವಪೇಟಿಕಾ~।\\
ಅತ್ಯಂತರಸಿಕಾ ದೂತಿಃ ವಿದಗ್ಧಾ ಪ್ರೀತಿಪೂಜಿತಾ ॥೭೧॥

	ತೂಲಿಕಾಯಂತ್ರನಿಲಯಾ ಯೋಗಪೀಠನಿವಾಸಿನೀ~।\\
	ಸುಲಕ್ಷಣಾ ದೃಶ್ಯರೂಪಾ ಸರ್ವ ಲಕ್ಷಣಲಕ್ಷಿತಾ ॥೭೨॥

ನಾನಾಲಂಕಾರಸುಭಗಾ ಪಂಚಕಾಮಶರಾರ್ಚಿತಾ~।\\
ಊರ್ಧ್ವತ್ರಿಕೋಣಯಂತ್ರಸ್ಥಾ ಬಾಲಾ ಕಾಮೇಶ್ವರೀ ತಥಾ ॥೭೩॥

	ಗುಣಾಧ್ಯಕ್ಷಾ ಕುಲಾಧ್ಯಕ್ಷಾ ಲಕ್ಷ್ಮೀಶ್ಚೈವ ಸರಸ್ವತೀ~।\\
	ವಸಂತಮದನೋತ್ತುಂಗ ಸ್ತನೀ ಕುಚಭರೋನ್ನತಾ ॥೭೪॥

ಕಲಾಧರಮುಖೀ ಮೂರ್ಧಪಾಥೋಧಿಶ್ಚ ಕಲಾವತೀ~।\\
ದಕ್ಷಪಾದಾದಿಶೀರ್ಷಾಂತಷೋಡಶಸ್ವರಸಂಯುತಾ ॥೭೫॥

	ಶ್ರದ್ಧಾ ಪೂರ್ತಿಃ ರತಿಶ್ಚೈವ ಭೂತಿಃ ಕಾಂತಿರ್ಮನೋರಮಾ~।\\
	ವಿಮಲಾ ಯೋಗಿನೀ ಘೋರಾ ಮದನೋನ್ಮಾದಿನೀ ಮದಾ ॥೭೬॥

ಮೋದಿನೀ ದೀಪಿನೀ ಚೈವ ಶೋಷಿಣೀ ಚ ವಶಂಕರೀ~।\\
ರಜನ್ಯಂತಾ ಕಾಮಕಲಾ ಲಸತ್ಕಮಲಧಾರಿಣೀ ॥೭೭॥

	ವಾಮಮೂರ್ಧಾದಿಪಾದಾಂತಷೋಡಶಸ್ವರಸಂಯುತಾ~।\\
	ಪೂಷರೂಪಾ ಸುಮನಸಾಂ ಸೇವ್ಯಾ ಪ್ರೀತಿಃ ದ್ಯುತಿಸ್ತಥಾ ॥೭೮॥

ಋದ್ಧಿಃ ಸೌದಾಮಿನೀ ಚಿಚ್ಚ ಹಂಸಮಾಲಾವೃತಾ ತಥಾ~।\\
ಶಶಿನೀ ಚೈವ ಚ ಸ್ವಸ್ಥಾ ಸಂಪೂರ್ಣಮಂಡಲೋದಯಾ ॥೭೯॥

	ಪುಷ್ಟಿಶ್ಚಾಮೃತಪೂರ್ಣಾ ಚ ಭಗಮಾಲಾ ಸ್ವರೂಪಿಣೀ~।\\
	ಭಗಯಂತ್ರಾಶ್ರಯಾ ಶಂಭುರೂಪಾ ಸಂಯೋಗಯೋಗಿನೀ ॥೮೦॥

ದ್ರಾವಿಣೀ ಬೀಜರೂಪಾ ಚ ಹ್ಯಕ್ಷುಬ್ಧಾ ಸಾಧಕಪ್ರಿಯಾ~।\\
ರಜಃ ಪೀಠಮಯೀ ನಾದ್ಯಾ ಸುಖದಾ ವಾಂಛಿತಪ್ರದಾ ॥೮೧॥

	ರಜಸ್ಸವಿತ್ ರಜಶ್ಶಕ್ತಿಃ ಶುಕ್ಲಬಿಂದು ಸ್ವರೂಪಿಣೀ~।\\
	ಸರ್ವಸಾಕ್ಷೀ ಸಾಮರಸ್ಯಾ ಶಿವಶಕ್ತಿಮಯೀ ಪ್ರಭಾ ॥೮೨॥

ಸಂಯೋಗಾನಂದನಿಲಯಾ ಸಂಯೋಗಪ್ರೀತಿಮಾತೃಕಾ~।\\
ಸಂಯೋಗಕುಸುಮಾನಂದಾ ಸಂಯೋಗಯೋಗಪದ್ಧತಿಃ ॥೮೩॥

	ಸಂಯೋಗಸುಖದಾವಸ್ಥಾ ಚಿದಾನಂದಾರ್ಘ್ಯಸೇವಿತಾ~।\\
	ಅರ್ಘ್ಯಪೂಜ್ಯಾ ಚ ಸಂಪತ್ತಿಃ ಅರ್ಘ್ಯದಾಭಿನ್ನರೂಪಿಣೀ ॥೮೪॥

ಸಾಮರಸ್ಯಪರಾ ಪ್ರೀತಾ ಪ್ರಿಯಸಂಗಮರಂಗಿಣೀ~।\\
ಜ್ಞಾನದೂತೀ ಜ್ಞಾನಗಮ್ಯಾ ಜ್ಞಾನಯೋನಿಶ್ಶಿವಾಲಯಾ ॥೮೫॥

	ಚಿತ್ಕಲಾ ಸತ್ಕಲಾ ಜ್ಞಾನಕಲಾ ಸಂವಿತ್ಕಲಾತ್ಮಿಕಾ~।\\
	ಕಲಾಚತುಷ್ಟಯೀ ಪದ್ಮವಾಸಿನೀ ಸೂಕ್ಷ್ಮರೂಪಿಣೀ ॥೮೬॥

ಹಂಸಕೇಲಿಸ್ಥಲಸ್ವಸ್ಥಾ ಹಂಸದ್ವಯವಿಕಾಸಿನೀ~।\\
ವಿರಾಗಿತಾ ಮೋಕ್ಷಕಲಾ ಪರಮಾತ್ಮಕಲಾವತೀ ॥೮೭॥

	ವಿದ್ಯಾಕಲಾಂತರಾತ್ಮಸ್ಥಾ ಚತುಷ್ಟಯಕಲಾವತೀ~।\\
	ವಿದ್ಯಾಸಂತೋಷಣಾ ತೃಪ್ತಿ ಪರಬ್ರಹ್ಮಪ್ರಕಾಶಿನೀ ॥೮೮॥

ಪರಮಾತ್ಮಪರಾ ವಸ್ತುಲೀನಾ ಶಕ್ತಿಚತುಷ್ಟಯೀ~।\\
ಶಾಂತಿರ್ಬೋಧಕಲಾ ವ್ಯಾಪ್ತಿಃ ಪರಜ್ಞಾನಾತ್ಮಿಕಾ ಕಲಾ ॥೮೯॥

	ಪಶ್ಯಂತೀ ಪರಮಾತ್ಮಸ್ಥಾ ಚಾಂತರಾತ್ಮಕಲಾ ಶಿವಾ~।\\
	ಮಧ್ಯಮಾ ವೈಖರೀ ಚಾತ್ಮ ಕಲಾಽಽನಂದಕಲಾವತೀ ॥೯೦॥

ತರುಣೀ ತಾರಕಾ ತಾರಾ ಶಿವಲಿಂಗಾಲಯಾತ್ಮವಿತ್~।\\
ಪರಸ್ಪರಸ್ವಭಾವಾ ಚ ಬ್ರಹ್ಮಜ್ಞಾನವಿನೋದಿನೀ ॥೯೧॥

	ರಾಮೋಲ್ಲಾಸಾ ಚ ದುರ್ಧರ್ಷಾ ಪರಮಾರ್ಘ್ಯಪ್ರಿಯಾ ರಮಾ~।\\
	ಜಾತ್ಯಾದಿರಹಿತಾ ಯೋಗಿನ್ಯಾನಂದಮಾತ್ರಪದ್ಧತಿಃ ॥೯೨॥

ಕಾಂತಾ ಶಾಂತಾ ದಾಂತಯಾತಿಃ ಕಲಿತಾ ಹೋಮಪದ್ಧತಿಃ~।\\
ದಿವ್ಯಭಾವಪ್ರದಾ ದಿವ್ಯಾ ವೀರಸೂರ್ವೀರಭಾವದಾ ॥೯೩॥

	ಪಶುದೇಹಾ ವೀರಗತಿಃ ವೀರಹಂಸಮನೋದಯಾ~।\\
	ಮೂರ್ಧಾಭಿಷಿಕ್ತಾ ರಾಜಶ್ರೀಃ ಕ್ಷತ್ರಿಯೋತ್ತಮಮಾತೃಕಾ ॥೯೪॥

ಶಸ್ತ್ರಾಸ್ತ್ರಕುಶಲಾ ಶೋಭಾ ರಥಸ್ಥಾ ಯುದ್ಧಜೀವಿಕಾ~।\\
ಅಶ್ವಾರೂಢಾ ಗಜಾರೂಢಾ ಭೂತೋಕ್ತಿಃ ಸುರಸುಶ್ರಯಾ ॥೯೫॥

	ರಾಜನೀತಿಶ್ಶಾಂತಿಕರ್ತ್ರೀ ಚತುರಂಗಬಲಾಶ್ರಯಾ~।\\
	ಪೋಷಿಣೀ ಶರಣಾ ಪದ್ಮಪಾಲಿಕಾ ಜಯಪಾಲಿಕಾ ॥೯೬॥

ವಿಜಯಾ ಯೋಗಿನೀ ಯಾತ್ರಾ ಪರಸೈನ್ಯವಿಮರ್ದಿನೀ~।\\
ಪೂರ್ಣವಿತ್ತಾ ವಿತ್ತಗಮ್ಯಾ ವಿತ್ತಸಂಚಯ ಶಾಲಿನೀ ॥೯೭॥

	ಮಹೇಶೀ ರಾಜ್ಯಭೋಗಾ ಚ ಗಣಿಕಾಗಣಭೋಗಭೃತ್~।\\
	ಉಕಾರಿಣೀ ರಮಾ ಯೋಗ್ಯಾ ಮಂದಸೇವ್ಯಾ ಪದಾತ್ಮಿಕಾ ॥

ಸೈನ್ಯಶ್ರೇಣೀ ಶೌರ್ಯರತಾ ಪತಾಕಾಧ್ವಜಮಾಲಿನೀ~।\\
ಸುಚ್ಛತ್ರ ಚಾಮರಶ್ರೇಣಿಃ ಯುವರಾಜವಿವರ್ಧಿನೀ ॥೯೯॥

	ಪೂಜಾ ಸರ್ವಸ್ವಸಂಭಾರಾ ಪೂಜಾಪಾಲನಲಾಲಸಾ~।\\
	ಪೂಜಾಭಿಪೂಜನೀಯಾ ಚ ರಾಜಕಾರ್ಯಪರಾಯಣಾ ॥೧೦೦॥

ಬ್ರಹ್ಮಕ್ಷತ್ರಮಯೀ ಸೋಮಸೂರ್ಯವಹ್ನಿಸ್ವರೂಪಿಣೀ~।\\
ಪೌರೋಹಿತ್ಯಪ್ರಿಯಾ ಸಾಧ್ವೀ ಬ್ರಹ್ಮಾಣೀ ಯಂತ್ರಸಂತತಿಃ ॥

	ಸೋಮಪಾನಜನಾಪ್ರೀತಾ ಯೋಜನಾಧ್ವಗತಿಕ್ಷಮಾ~।\\
	ಪ್ರೀತಿಗ್ರಹಾ ಪರಾ ದಾತ್ರೀ ಶ್ರೇಷ್ಠಜಾತಿಃ ಸತಾಂಗತಿಃ ॥೧೦೨॥

ಗಾಯತ್ರೀ ವೇದವಿದ್ಧ್ಯೇಯಾ ದೀಕ್ಷಾ ಸಂತೋಷತರ್ಪಣಾ~।\\
ರತ್ನದೀಧಿತಿವಿದ್ಯುತ್ಸಹಸನಾ ವೈಶ್ಯಜೀವಿಕಾ ॥೧೦೩॥

	ಕೃಷಿರ್ವಾಣಿಜ್ಯಭೂತಿಶ್ಚ ವೃದ್ಧಿದಾ ವೃದ್ಧಸೇವಿತಾ~।\\
	ತುಲಾಧಾರಾ ಸ್ವಪ್ನಕಾಮಾ ಮಾನೋನ್ಮಾನಪರಾಯಣಾ ॥೧೦೪॥

ಶ್ರದ್ಧಾ ವಿಪ್ರಗತಿಃ ಕರ್ಮಕರೀ ಕೌತುಕಪೂಜಿತಾ~।\\
ನಾನಾಭಿಚಾರಚತುರಾ ವಾರಸ್ತ್ರೀಶ್ರೀಃ ಕಲಾಮಯೀ ॥೧೦೫॥

	ಸುಕರ್ಣಧಾರಾ ನೌಪಾರಾ ಸರ್ವಾಶಾ ರತಿಮೋಹಿನೀ~।\\
	ದುರ್ಗಾ ವಿಂಧ್ಯವನಸ್ಥಾ ಚ ಕಾಲದರ್ಪನಿಷೂದಿನೀ ॥೧೦೬॥

ಭೂಭಾರಶಮನೀ ಕೃಷ್ಣಾ ರಕ್ಷೋರಾಕ್ಷಸಸಾಹಸಾ~।\\
ವಿವಿಧೋತ್ಪಾತಶಮನೀ ಸಮಯಾ ಸುರಸೇವಿತಾ ॥೧೦೭॥

	ಪಂಚಾವಯವವಾಕ್ಯಶ್ರೀಃ ಪ್ರಪಂಚೋದ್ಯಾನಚಂದ್ರಿಕಾ~।\\
	ಸಿದ್ಧಿಸಂದೋಹಸಂಸಿದ್ಧಯೋಗಿನೀವೃಂದಸೇವಿತಾ ॥೧೦೮॥

ನಿತ್ಯಾ ಷೋಡಶಿಕಾರೂಪಾ ಕಾಮೇಶೀ ಭಗಮಾಲಿನೀ~।\\
ನಿತ್ಯಕ್ಲಿನ್ನಾ ನಿರಾಧಾರಾ ವಹ್ನಿಮಂಡಲವಾಸಿನೀ ॥೧೦೯॥

	ಮಹಾವಜ್ರೇಶ್ವರೀ ನಿತ್ಯಾ ಶಿವದೂತೀತಿ ವಿಶ್ರುತಾ~।\\
	ತ್ವರಿತಾ ಪ್ರಥಿತಾ ಖ್ಯಾತಾ ವಿಖ್ಯಾತಾ ಕುಲಸುಂದರೀ ॥೧೧೦॥

ನಿತ್ಯಾ ನೀಲಪತಾಕಾ ಚ ವಿಜಯಾ ಸರ್ವಮಂಗಲಾ~।\\
ಜ್ವಾಲಾಮಾಲಾ ವಿಚಿತ್ರಾ ಚ ಮಹಾತ್ರಿಪುರಸುಂದರೀ ॥೧೧೧॥

	ಗುರುವೃಂದಾ ಪರಗುರುಃ ಪ್ರಕಾಶಾನಂದದಾಯಿನೀ~।\\
	ಶಿವಾನಂದಾ ನಾದರೂಪಾ ಶಕ್ರಾನಂದಸ್ವರೂಪಿಣೀ ॥೧೧೨॥

ದೇವ್ಯಾನಂದಾ ನಾದಮಯೀ ಕೌಲೇಶಾನಂದನಾಥಿನೀ~।\\
ಶುಕ್ಲದೇವ್ಯಾನಂದನಾಥಾ ಕುಲೇಶಾನಂದದಾಯಿನೀ ॥೧೧೩॥

	ದಿವ್ಯೌಘಸೇವಿತಾ ದಿವ್ಯಭೋಗದಾನಪರಾಯಣಾ~।\\
	ಕ್ರೀಡಾನಂದಾ ಕ್ರೀಡಮಾನಾ ಸಮಯಾನಂದದಾಯಿನೀ ॥೧೧೪॥

ವೇದಾನಂದಾ ಪಾರ್ವತೀ ಚ ಸಹಜಾನಂದದಾಯಿನೀ~।\\
ಸಿದ್ಧೌಘಗುರುರೂಪಾ ಚಾಪ್ಯಪರಾ ಗುರುರೂಪಿಣೀ ॥೧೧೫॥

	ಗಗನಾನಂದನಾಥಾ ಚ ವಿಶ್ವಾದ್ಯಾನಂದದಾಯಿನೀ~।\\
	ವಿಮಲಾನಂದನಾಥಾ ಚ ಮದನಾನಂದದಾಯಿನೀ ॥೧೧೬॥

ಭುವನಾನಂದನಾಥಾ ಚ ಲೀಲೋದ್ಯಾನಪ್ರಿಯಾ ಗತಿಃ~।\\
ಸ್ವಾತ್ಮಾನಂದವಿನೋದಾ ಚ ಪ್ರಿಯಾದ್ಯಾನಂದನಾಥಿನೀ ॥೧೧೭॥

	ಮಾನವಾದ್ಯಾ ಗುರುಶ್ರೇಷ್ಠಾ ಪರಮೇಷ್ಠಿ ಗುರುಪ್ರಭಾ~।\\
	ಪರಮಾದ್ಯಾ ಗುರುಶ್ಶಕ್ತಿಃ (ಕೀರ್ತಿತಾ)ಕಿರ್ತನಪ್ರಿಯಾ ॥೧೧೮॥

ತ್ರೈಲೋಕ್ಯಮೋಹನಾಖ್ಯಾ ಚ ಸರ್ವಾಶಾಪರಿಪೂರಕಾ~।\\
ಸರ್ವಸಂಕ್ಷೋಭಿಣೀ ಪೂರ್ವಾಮ್ನಾಯಾ ಚಕ್ರತ್ರಯಾಲಯಾ ॥೧೧೯॥

	ಸರ್ವಸೌಭಾಗ್ಯದಾತ್ರೀ ಚ ಸರ್ವಾರ್ಥಸಾಧಕಪ್ರಿಯಾ~।\\
	ಸರ್ವರಕ್ಷಾಕರೀ ಸಾಧುರ್ದಕ್ಷಿಣಾಮ್ನಾಯದೇವತಾ ॥೧೨೦॥

ಮಧ್ಯಚಕ್ರೈಕನಿಲಯಾ ಪಶ್ಚಿಮಾಮ್ನಾಯದೇವತಾ~।\\
ನವಚಕ್ರಕೃತಾವಾಸಾ ಕೌಬೇರಾಮ್ನಾಯದೇವತಾ ॥೧೨೧॥

	ಬಿಂದುಚಕ್ರಕೃತಾಯಾಸಾ ಮಧ್ಯಸಿಂಹಾಸನೇಶ್ವರೀ~।\\
	ಶ್ರೀವಿದ್ಯಾ ನವದುರ್ಗಾ ಚ ಮಹಿಷಾಸುರಮರ್ದಿನೀ ॥೧೨೨॥

ಸರ್ವಸಾಮ್ರಾಜ್ಯಲಕ್ಷ್ಮೀಶ್ಚ ಅಷ್ಟಲಕ್ಷ್ಮೀಶ್ಚ ಸಂಶ್ರುತಾ~।\\
ಶೈಲೇಂದ್ರತನಯಾ ಜ್ಯೋತಿಃ ನಿಷ್ಕಲಾ ಶಾಂಭವೀ ಉಮಾ ॥೧೨೩॥

	ಅಜಪಾ ಮಾತೃಕಾ ಚೇತಿ ಶುಕ್ಲವರ್ಣಾ ಷಡಾನನಾ~।\\
	ಪಾರಿಜಾತೇಶ್ವರೀ ಚೈವ ತ್ರಿಕೂಟಾ ಪಂಚಬಾಣದಾ ॥೧೧೪॥

ಪಂಚಕಲ್ಪಲತಾ ಚೈವ ತ್ರ್ಯಕ್ಷರೀ ಮೂಲಪೀಠಿಕಾ~।\\
ಸುಧಾಶ್ರೀರಮೃತೇಶಾನೀ ಹ್ಯನ್ನಪೂರ್ಣಾ ಚ ಕಾಮಧುಕ್ ॥೧೨೫॥

	ಪಾಶಹಸ್ತಾ ಸಿದ್ಧಲಕ್ಷ್ಮೀಃ ಮಾತಂಗೀ ಭುವನೇಶ್ವರೀ~।\\
	ವಾರಾಹೀ ನವರತ್ನಾನಾಮೀಶ್ವರೀ ಚ ಪ್ರಕೀರ್ತಿತಾ ॥೧೨೬॥

ಪರಂ ಜ್ಯೋತಿಃ ಕೋಶರೂಪಾ ಸೈಂಧವೀ ಶಿವದರ್ಶನಾ~।\\
ಪರಾಪರಾ ಸ್ವಾಮಿನೀ ಚ ಶಾಕ್ತದರ್ಶನವಿಶ್ರುತಾ ॥೧೨೭॥

	ಬ್ರಹ್ಮದರ್ಶನರೂಪಾ ಚ ಶಿವದರ್ಶನರೂಪಿಣೀ~।\\
	ವಿಷ್ಣುದರ್ಶನರೂಪಾ ಚ ಸ್ರಷ್ಟೃದರ್ಶನರೂಪಿಣೀ ॥೧೨೮॥

ಸೌರದರ್ಶನರೂಪಾ ಚ ಸ್ಥಿತಿಚಕ್ರಕೃತಾಶ್ರಯಾ~।\\
ಬೌದ್ಧದರ್ಶನರೂಪಾ ಚ ತುರೀಯಾ ಬಹುರೂಪಿಣೀ ॥೧೨೯॥

	ತತ್ವಮುದ್ರಾಸ್ವರೂಪಾ ಚ ಪ್ರಸನ್ನಾ ಜ್ಞಾನಮಾತೃಕಾ~।\\
	ಸರ್ವೋಪಚಾರಸಂತುಷ್ಟಾ ಹೃನ್ಮಯೀ ಶೀರ್ಷದೇವತಾ ॥೧೩೦॥

ಶಿಖಾಸ್ಥಿತಾ ವರ್ಮಮಯೀ ನೇತ್ರತ್ರಯವಿಲಾಸಿನೀ~।\\
ಅಸ್ತ್ರಸ್ಥಾ ಚತುರಸ್ರಸ್ಥಾ ದ್ವಾರಸ್ಥಾ ದ್ವಾರದೇವತಾ ॥೧೩೧॥

	ಅಣಿಮಾ ಪಶ್ಚಿಮಸ್ಥಾ ಚ ದಕ್ಷಿಣದ್ವಾರದೇವತಾ~।\\
	ವಶಿತ್ವಾ ವಾಯುಕೋಣಸ್ಥಾ ಪ್ರಾಕಾಮ್ಯೇಶಾನದೇವತಾ ॥೧೩೨॥

ಮಹಿಮಾಪೂರ್ವನಾಥಾ ಚ ಲಘಿಮೋತ್ತರದೇವತಾ~।\\
ಅಗ್ನಿಕೋಣಸ್ಥಗರಿಮಾ ಪ್ರಾಪ್ತಿರ್ನೈಋತಿವಾಸಿನೀ ॥೧೩೩॥

	ಈಶಿತ್ವಸಿದ್ಧಿಸುರಥಾ ಸರ್ವಕಾಮೋರ್ಧ್ವವಾಸಿನೀ~।\\
	ಬ್ರಾಹ್ಮೀ ಮಾಹೇಶ್ವರೀ ಚೈವ ಕೌಮಾರೀ ವೈಷ್ಣವೀ ತಥಾ ॥೧೩೪॥

ವಾರಾಹ್ಯೈಂದ್ರೀ ಚ ಚಾಮುಂಡಾ ವಾಮಾ ಜ್ಯೇಷ್ಠಾ ಸರಸ್ವತೀ~।\\
ಕ್ಷೋಭಿಣೀ ದ್ರಾವಿಣೀ ರೌದ್ರೀ ಕಾಲ್ಯುನ್ಮಾದನಕಾರಿಣೀ ॥೧೩೫॥

	ಖೇಚರಾ ಕಾಲಕರಣೀ ಚ ಬಲಾನಾಂ ವಿಕರಣೀ ತಥಾ~।\\
	ಮನೋನ್ಮನೀ ಸರ್ವಭೂತದಮನೀ ಸರ್ವಸಿದ್ಧಿದಾ ॥೧೩೬॥

ಬಲಪ್ರಮಥಿನೀ ಶಕ್ತಿಃ ಬುದ್ಧ್ಯಾಕರ್ಷಣರೂಪಿಣೀ~।\\
ಅಹಂಕಾರಾಕರ್ಷಣೀ ಚ ಶಬ್ದಾಕರ್ಷಣರೂಪಿಣೀ ॥೧೩೭॥

	ಸ್ಪರ್ಶಾಕರ್ಷಣರೂಪಾ ಚ ರೂಪಾಕರ್ಷಣರೂಪಿಣೀ~।\\
	ರಸಾಕರ್ಷಣರೂಪಾ ಚ ಗಂಧಾಕರ್ಷಣರೂಪಿಣೀ ॥೧೩೮॥

ಚಿತ್ರಾಕರ್ಷಣರೂಪಾ ಚ ಧೈರ್ಯಾಕರ್ಷಣರೂಪಿಣೀ~।\\
ಸ್ಮೃತ್ಯಾಕರ್ಷಣರೂಪಾ ಚ ನಾಮಾಕರ್ಷಣರೂಪಿಣೀ ॥೧೩೯॥

	ಬೀಜಾಕರ್ಷಣರೂಪಾ ಚ ಹ್ಯಾತ್ಮಾಕರ್ಷಣರೂಪಿಣೀ~।\\
	ಅಮೃತಾಕರ್ಷಣೀ ಚೈವ ಶರೀರಾಕರ್ಷಣೀ ತಥಾ ॥೧೪೦॥

ಷೋಡಶಸ್ವರಸಂಪನ್ನಾ ಸ್ರವತ್ಪೀಯೂಷಮಂಡಿತಾ~।\\
ತ್ರಿಪುರೇಶೀ ಸಿದ್ಧಿದಾತ್ರೀ ಕಲಾದರ್ಶನವಾಸಿನೀ ॥೧೪೧॥

	ಸರ್ವಸಂಕ್ಷೋಭಚಕ್ರೇಶೀ ಶಕ್ತಿರ್ಗುಹ್ಯತರಾಭಿಧಾ~।\\
	ಅನಂಗಕುಸುಮಾಶಕ್ತಿಃ ತಥೈವಾನಂಗಮೇಖಲಾ ॥೧೪೨॥

ಅನಂಗಮದನಾಽನಂಗಮದನಾತುರರೂಪಿಣೀ~।\\
ಅನಂಗರೇಖಾ ಚಾನಂಗವೇಗಾನಂಗಾಕುಶಾಭಿಧಾ ॥೧೪೩॥

	ಅನಂಗಮಾಲಿನೀ ಚೈವ ಹ್ಯಷ್ಟವರ್ಗಾಧಿಗಾಮಿನೀ~।\\
	ವಸ್ವಷ್ಟಕಕೃತಾವಾಸಾ ಶ್ರೀಮತ್ತ್ರಿಪುರಸುಂದರೀ ॥೧೪೪॥

ಸರ್ವಸಾಮ್ರಾಜ್ಯಸುಭಗಾ ಸರ್ವಭಾಗ್ಯಪ್ರದೇಶ್ವರೀ~।\\
ಸಂಪ್ರದಾಯೇಶ್ವರೀ ಸರ್ವಸಂಕ್ಷೋಭಣಕರೀ ತಥಾ ॥೧೪೫॥

	ಸರ್ವವಿದ್ರಾವಿಣೀ ಸರ್ವಾಕರ್ಷಿಣೀ ರೂಪಕಾರಿಣೀ~।\\
	ಸರ್ವಾಹ್ಲಾದನಶಕ್ತಿಶ್ಚ ಸರ್ವಸಮ್ಮೋಹಿನೀ ತಥಾ ॥೧೪೬॥

ಸರ್ವಸ್ತಂಭನಶಕ್ತಿಶ್ಚ ಸರ್ವಜೃಂಭಣಕಾರಿಣೀ~।\\
ಸರ್ವವಶ್ಯಕಶಕ್ತಿಶ್ಚ ತಥಾ ಸರ್ವಾನುರಂಜನೀ ॥೧೪೭॥

	ಸರ್ವೋನ್ಮಾದನಶಕ್ತಿಶ್ಚ ತಥಾ ಸರ್ವಾರ್ಥಸಾಧಿಕಾ~।\\
	ಸರ್ವಸಂಪತ್ತಿದಾ ಚೈವ ಸರ್ವಮಾತೃಮಯೀ ತಥಾ ॥೧೪೮॥

ಸರ್ವದ್ವಂದ್ವಕ್ಷಯಕರೀ ಸಿದ್ಧಿಸ್ತ್ರಿಪುರವಾಸಿನೀ~।\\
ಚತುರ್ದಶಾರಚಕ್ರೇಶೀ ಕುಲಯೋಗಸಮನ್ವಯಾ ॥೧೪೯॥

	ಸರ್ವಸಿದ್ಧಿಪ್ರದಾ ಚೈವ ಸರ್ವಸಂಪತ್ಪ್ರದಾ ತಥಾ~।\\
	ಸರ್ವಪ್ರಿಯಕರೀ ಚೈವ ಸರ್ವಮಂಗಲಕಾರಿಣೀ ॥೧೫೦॥

ಸರ್ವಕಾಮಪ್ರಪೂರ್ಣಾ ಚ ಸರ್ವದುಃಖವಿಮೋಚಿನೀ~।\\
ಸರ್ವಮೃತ್ಯುಪ್ರಶಮನೀ ಸರ್ವ ವಿಘ್ನವಿನಾಶಿನೀ ॥೧೫೧॥

	ಸರ್ವಾಂಗಸುಂದರೀ ಚೈವ ಸರ್ವಸೌಭಾಗ್ಯದಾಯಿನೀ~।\\
	ತ್ರಿಪುರಾ ಶ್ರೀಶ್ಚ ಸರ್ವಾರ್ಥಸಾಧಿಕಾ ದಶಕೋಣಗಾ ॥೧೫೨॥

ಸರ್ವರಕ್ಷಾಕರೀ ಚೈವ ಈಶ್ವರೀ ಯೋಗಿನೀ ತಥಾ~।\\
ಸರ್ವಜ್ಞಾ ಸರ್ವಶಕ್ತಿಶ್ಚ ಸರ್ವೈಶ್ವರ್ಯಪ್ರದಾ ತಥಾ ॥೧೫೩॥

	ಸರ್ವಜ್ಞಾನಮಯೀ ಚೈವ ಸರ್ವವ್ಯಾಧಿವಿನಾಶಿನೀ~।\\
	ಸರ್ವಾಧಾರಸ್ವರೂಪಾ ಚ ಸರ್ವಪಾಪಹರಾ ತಥಾ ॥೧೫೪॥

ಸರ್ವಾನಂದಮಯೀ ಚೈವ ಸರ್ವರಕ್ಷಾಸ್ವರೂಪಿಣೀ~।\\
ತಥೈವ ಚ ಮಹಾಶಕ್ತಿಃ ಸರ್ವೇಪ್ಸಿತಫಲಪ್ರದಾ ॥೧೫೫॥

	ಅಂತರ್ದಶಾರಚಕ್ರಸ್ಥಾ ತಥಾ ತ್ರಿಪುರಮಾಲಿನೀ~।\\
	ಸರ್ವರೋಗಹರಾ ಚೈವ ರಹಸ್ಯಯೋಗಿನೀ ತಥಾ ॥೧೫೬॥

ವಾಗ್ದೇವೀ ವಶಿನೀ ಚೈವ ತಥಾ ಕಾಮೇಶ್ವರೀ ತಥಾ~।\\
ಮೋದಿನೀ ವಿಮಲಾ ಚೈವ ಹ್ಯರುಣಾ ಜಯಿನೀ ತಥಾ ॥೧೫೭॥

	ಶಿವಕಾಮಪ್ರದಾ ದೇವೀ ಶಿವಕಾಮಸ್ಯ ಸುಂದರೀ~।\\
	ಲಲಿತಾ ಲಲಿತಾಧ್ಯಾನಫಲದಾ ಶುಭಕಾರಿಣೀ ॥೧೫೮॥

ಸರ್ವೇಶ್ವರೀ ಕೌಲಿನೀ ಚ ವಸುವಂಶಾಭಿವರ್ದ್ಧಿನೀ~।\\
ಸರ್ವಕಾಮಪ್ರದಾ ಚೈವ ಪರಾಪರರಹಸ್ಯವಿತ್ ॥೧೫೯॥

	ತ್ರಿಕೋಣಚತುರಶ್ರಸ್ಥ ಕಾಮೇಶ್ವರ್ಯಾಯುಧಾತ್ಮಿಕಾ~।\\
	ಕಾಮೇಶ್ವರೀಬಾಣರೂಪಾ ಕಾಮೇಶೀ ಚಾಪರೂಪಿಣೀ ॥೧೬೦॥

ಕಾಮೇಶೀ ಪಾಶಹಸ್ತಾ ಚ ಕಾಮೇಶ್ಯಂಕುಶರೂಪಿಣೀ~।\\
ಕಾಮೇಶ್ವರೀ ರುದ್ರಶಕ್ತಿಃ ಅಗ್ನಿಚಕ್ರಕೃತಾಲಯಾ ॥೧೬೧॥

	ಕಾಮಾಭಿಂತ್ರಾ ಕಾಮದೋಗ್ಧ್ರೀ ಕಾಮದಾ ಚ ತ್ರಿಕೋಣಗಾ~।\\
	ದಕ್ಷಕೋಣೇಶ್ವರೀ ವಿಷ್ಣುಶಕ್ತಿರ್ಜಾಲಂಧರಾಲಯಾ ॥೧೬೨॥

ಸೂರ್ಯಚಕ್ರಾಲಯಾ ವಾಮಕೋಣಗಾ ಸೋಮಚಕ್ರಗಾ~।\\
ಭಗಮಾಲಾ ಬೃಹಚ್ಛಕ್ತಿ ಪೂರ್ಣಾ ಪೂರ್ವಾಸ್ರರಾಗಿಣೀ ॥೧೬೩॥

	ಶ್ರೀಮತ್ತ್ರಿಕೋಣಭುವನಾ ತ್ರಿಪುರಾಖ್ಯಾ ಮಹೇಶ್ವರೀ~।\\
	ಸರ್ವಾನಂದಮಯೀಶಾನೀ ಬಿಂದುಗಾತಿರಹಸ್ಯಗಾ ॥೧೬೪॥

ಪರಬ್ರಹ್ಮಸ್ವರೂಪಾ ಚ ಮಹಾತ್ರಿಪುರಸುಂದರೀ~।\\
ಸರ್ವಚಕ್ರಾಂತರಸ್ಥಾ ಚ ಸರ್ವಚಕ್ರಾಧಿದೇವತಾ ॥೧೬೫॥

	ಸರ್ವಚಕ್ರೇಶ್ವರೀ ಸರ್ವಮಂತ್ರಾಣಾಮೀಶ್ವರೀ ತಥಾ~।\\
	ಸರ್ವವಿದ್ಯೇಶ್ವರೀ ಚೈವ ಸರ್ವವಾಗೀಶ್ವರೀ ತಥಾ ॥೧೬೬॥

ಸರ್ವಯೋಗೇಶ್ವರೀ ಸರ್ವಪೀಠೇಶ್ವರ್ಯಖಿಲೇಶ್ವರೀ~।\\
ಸರ್ವಕಾಮೇಶ್ವರೀ ಸರ್ವತತ್ವೇಶ್ವರ್ಯಾಗಮೇಶ್ವರೀ ॥೧೬೭॥

	ಶಕ್ತಿಃ ಶಕ್ತಿಭೃದುಲ್ಲಾಸಾ ನಿರ್ದ್ವಂದ್ವಾದ್ವೈತಗರ್ಭಿಣೀ~।\\
	ನಿಷ್ಪ್ರಪಂಚಾ ಪ್ರಪಂಚಾಭಾ ಮಹಾಮಾಯಾ ಪ್ರಪಂಚಸೂಃ ॥೧೬೮॥

ಸರ್ವವಿಶ್ವೋತ್ಪತ್ತಿಧಾತ್ರೀ ಪರಮಾನಂದಕಾರಣಾ~।\\
ಲಾವಣ್ಯಸಿಂಧುಲಹರೀ ಸುಂದರೀತೋಷಮಂದಿರಾ ॥೧೬೯॥

	ಶಿವಕಾಮಸುಂದರೀ ದೇವೀ ಸರ್ವಮಂಗಲದಾಯಿನೀ~।\\
	ಇತಿನಾಮ್ನಾಂ ಸಹಸ್ರಂ ಚ ಗದಿತಂ ಇಷ್ಟದಾಯಕಂ ॥೧೭೦॥

ಉತ್ತರಪೀಠಿಕಾ ॥\\
ಸಹಸ್ರನಾಮ ಮಂತ್ರಾಣಾಂ ಸಾರಮಾಕೃಷ್ಯ ಪಾರ್ವತಿ।\\
ರಚಿತಂ ಹಿ ಮಯಾ ಚೈತತ್ ಸಿದ್ಧಿದಂ ಪರಮೋಕ್ಷದಮ್ ॥೧॥

ಅನೇನ ಸ್ತುವತೋ ನಿತ್ಯಂ ಅರ್ಧರಾತ್ರೇ ನಿಶಾಮುಖೇ।\\
ಪ್ರಾತಃ ಕಾಲೇ ಚ ಪೂಜಾಯಾಂ ಪಠನಂ ಸರ್ವಕಾಮದಮ್ ॥೨॥

ಸರ್ವಸಾಮ್ರಾಜ್ಯಸುಖದಾ ಸುಂದರೀ ಪರಿತುಷ್ಯತಿ।\\
ರತ್ನಾನಿ ವಿವಿಧಾನ್ಯಸ್ಯ ವಿತ್ತಾನಿ ಪ್ರಚುರಾಣಿ ಚ ॥೩॥

ಮನೋರಥರಥಸ್ಥಾನಿ ದದಾತಿ ಪರಮೇಶ್ವರೀ।\\
ಪುತ್ರಪೌತ್ರಾಶ್ಚ ವರ್ಧಂತೇ ಸಂತತಿಸ್ಸಾರ್ವಕಾಲಿಕಾ ॥೪॥

ಶತ್ರವಸ್ತಸ್ಯ ನಶ್ಯಂತಿ ವರ್ಧಂತೇ ಚ ಬಲಾನಿ ಚ।\\
ವ್ಯಾಧಯಸ್ತಸ್ಯ ನಶ್ಯಂತಿ ಲಭತೇ ಚೌಷಧಾನಿ ಚ ॥೫॥

ಮಂದಿರಾಣಿ ವಿಚಿತ್ರಾಣಿ ರಾಜಂತೇ ತಸ್ಯ ಸರ್ವದಾ।\\
ಕೃಷಿಃ ಫಲವತೀ ತಸ್ಯ ಭೂಮಿಃ ಕಾಮಾಖಿಲಪ್ರದಾ ॥೬॥

ಸ್ಥಿರಂ ಜನಪದಂ ತಸ್ಯ ರಾಜ್ಯಂ ತಸ್ಯ ನಿರಂಕುಶಮ್।\\
ಮಾತಂಗಾಸ್ತುರಗಾಸ್ತುಂಗಾಃ ಸಿಂಚಿಂತೋ ಮದವಾರಿಭಿಃ ॥೭॥

ಸೈನಿಕಾಶ್ಚ ವಿರಾಜಂತೇ ತುಷ್ಟಾಃ ಪುಷ್ಟಾಸ್ತುರಂಗಮಾಃ।\\
ಪೂಜಾಃ ಶಶ್ವತ್ ವಿವರ್ಧಂತೇ ನಿರ್ವಿವಾದಾಶ್ಚ ಮಂತ್ರಿಣಃ ॥೮॥

ಜ್ಞಾತಯಸ್ತಸ್ಯ ತುಷ್ಯಂತಿ ಬಾಂಧವಾಃ ವಿಗತಜ್ವರಾಃ।\\
ಭೃತ್ಯಾಸ್ತಸ್ಯ ವಶೇ ನಿತ್ಯಂ ವರ್ತಂತೇऽಸ್ಯ ಮನೋನುಗಾಃ ॥೯॥

ಗದ್ಯಪದ್ಯಮಯೀ ವಾಣೀ ವಾಕ್ಚಾತುರ್ಯಸುಸಂಭೃತಾ।\\
ಸಮಗ್ರ ಸುಖಸಂಪತ್ತಿ ಶಾಲಿನೀ ಲಾಸ್ಯಮಾಲಿನೀ ॥೧೦॥

ನಾನಾಪದಮಯೀ ವಾಣೀ ತಸ್ಯ ಗಂಗಾಪ್ರವಾಹವತ್।\\
ಅದೃಷ್ಟಾನ್ಯಪಿ ಶಾಸ್ತ್ರಾಣಿ ಪ್ರಕಾಶಂತೇ ನಿರಂತರಮ್ ॥೧೧॥

ನಿಗ್ರಹಃ ಪರವಾಕ್ಯಾನಾಂ ಸಭಾಯಾಂ ತಸ್ಯ ಜಾಯತೇ।\\
ಸ್ತುವಂತಿ ಕೃತಿನಸ್ತಂ ವೈ ರಾಜಾನೋ ದಾಸವತ್ತಥಾ ॥೧೨॥

ಶಸ್ತ್ರಾಣ್ಯಸ್ತ್ರಾಣಿ ತದ್ಗಾತ್ರೇ ಜನಯಂತಿ ರುಜೋ ನಹಿ।\\
ಮಾತಂಗಾಃ ತಸ್ಯ ವಶಗಾಃ ಸರ್ಪವರ್ಯಾ ಭವಂತಿ ಚ ॥೧೩॥

ವಿಷಂ ನಿರ್ವಿಷತಾಂ ಯಾತಿ ಪಾನೀಯಮಮೃತಂ ಭವೇತ್।\\
ಪರಸೇನಾಸ್ತಂಭನಂ ಚ ಪ್ರತಿವಾದಿವಿಜೃಂಭಣಮ್ ॥೧೪॥

ನವರಾತ್ರೇಣ ಜಾಯಂತೇ ಸತತನ್ಯಾಸಯೋಗತಃ।\\
ಅಹೋರಾತ್ರಂ ಪಠೇದ್ಯಸ್ತು ಸ್ತೋತ್ರಂ ಸಂಯತಮಾನಸಃ ॥೧೫॥

ವಶಾಃ ತಸ್ಯೋಪಜಾಯಂತೇ ಸರ್ವೇ ಲೋಕಾಃ ಸುನಿಶ್ಚಿತಮ್।\\
ಷಣ್ಮಾಸಾಭ್ಯಾಸಯೋಗೇನ ದೇವಾ ಯಕ್ಷಾಶ್ಚ ಕಿನ್ನರಾಃ ॥೧೬॥

ಸಿದ್ಧಾ ಮಹೋರಗಾಸ್ಸರ್ವೇ ವಶಮಾಯಾಂತಿ ನಿಶ್ಚಯಮ್।\\
ನಿತ್ಯಂ ಕಾಮಕಲಾಂ ನ್ಯಸ್ಯನ್ ಯಃ ಪಠೇತ್ ಸ್ತೋತ್ರಮುತ್ತಮಮ್ ॥೧೭॥

ಮದನೋನ್ಮಾದಿನೀ ಲೀಲಾಪುರಸ್ತ್ರೀ ತದ್ವಶಾನುಗಾ।\\
ಲಾವಣ್ಯಮದನಾ ಸಾಕ್ಷಾತ್ ವಿದಗ್ಧಮುಖಚಂದ್ರಿಕಾ ॥೧೮॥

ಪ್ರೇಮಪೂರ್ಣಾಶ್ರುನಯನಾ ಸುಂದರೀ ವಶಗಾ ಭವೇತ್।\\
ಭೂರ್ಜಪತ್ರೇ ರೋಚನೇನ ಕುಂಕುಮೇನ ವರಾನನೇ ॥೧೯॥

ಧಾತುರಾಗೇಣ ವಾ ದೇವೀ ಮೂಲಮಂತ್ರಂ ವಿಲಿಖ್ಯ ಚ।\\
ರಕ್ಷಾರ್ಥಂ ಭಸ್ಮ ವಿನ್ಯಸ್ಯ ಪುಟೀಕೃತ್ಯ ಸಮಂತ್ರಕಮ್ ॥೨೦॥

ಸುವರ್ಣರೌಪ್ಯಖಚಿತೇ ಸುಷಿರೇ ಸ್ಥಾಪ್ಯ ಯತ್ನತಃ।\\
ಸಂಪೂಜ್ಯ ತತ್ರ ದೇವೇಶೀಂ ಪುನರಾದಾಯ ಭಕ್ತಿತಃ ॥೨೧॥

ಧಾರಯೇನ್ಮಸ್ತಕೇ ಕಂಠೇ ಬಾಹುಮೂಲೇ ತಥಾ ಹೃದಿ।\\
ನಾಭೌ ಚ ವಿದ್ಯುತಂ ಧನ್ಯಂ ಜಯದಂ ಸರ್ವಕಾಮದಮ್ ॥೨೨॥

ರಕ್ಷಾಕರಂ ನಾನ್ಯದಸ್ಮಾತ್ ವಿದ್ಯತೇ ಭುವನತ್ರಯೇ।\\
ಜ್ವರರೋಗನೃಪಾವಿಷ್ಟಭಯಹೃತ್ ಭೂತಿವರ್ಧನಮ್ ॥೨೩॥

ಬಲವೀರ್ಯಕರಂ ಚಾಥ ಭೂತಶತ್ರುವಿನಾಶನಮ್।\\
ಪುತ್ರಪೌತ್ರಗುಣಶ್ರೇಯೋವರ್ಧಕಂ ಧನಧಾನ್ಯಕೃತ್ ॥೨೪॥

ಯ ಇದಂ ಪಠತಿ ಸ್ತೋತ್ರಂ ಸ ಸರ್ವಂ ಲಭತೇ ನರಃ।\\
ಯದ್ಗೃಹೇ ಲಿಖಿತಂ ಸ್ತೋತ್ರಂ ತಿಷ್ಠೇದೇತದ್ ವರಾನನೇ ॥೨೫॥

ತತ್ರ ತಿಷ್ಠಾಮ್ಯಹಂ ನಿತ್ಯಂ ಹರಿಶ್ಚ ಕಮಲಾಸನಃ।\\
ವಸಂತಿ ಸರ್ವತೀರ್ಥಾನಿ ಗೌರೀ ಲಕ್ಷ್ಮೀಸ್ಸರಸ್ವತೀ ॥೨೬॥

ಶಿವಕಾಮೇಶ್ವರೀಂ ಧ್ಯಾತ್ವಾ ಪಠೇನ್ನಾಮಸಹಸ್ರಕಮ್।\\
ಅಸಕೃತ್ ಧ್ಯಾನಪಾಠೇನ ಸಾಧಕಃ ಸಿದ್ಧಿಮಾಪ್ನುಯಾತ್ ॥೨೭॥

ಶುಕ್ರವಾರೇ ಪೌರ್ಣಮಾಸ್ಯಾಂ ಪಠನ್ನಾಮಸಹಸ್ರಕಮ್।\\
ಪೂಜಾಂ ಯಃ ಕುರುತೇ ಭಕ್ತ್ಯಾ ವಾಂಛಿತಂ ಲಭತೇ ಧ್ರುವಮ್ ॥೨೮॥

ಶಿವಕಾಮೇಶ್ವರೀಮಂತ್ರಃ ಮಂತ್ರರಾಜಃ ಪ್ರಕೀರ್ತಿತಃ।\\
ತದಭ್ಯಾಸಾತ್ಸಾಧಕಶ್ಚ ಸಿದ್ಧಿಮಾಪ್ನೋತ್ಯನುತ್ತಮಾಮ್ ॥೨೯॥

ನಾಸಾಧಕಾಯ ದಾತವ್ಯಮಶ್ರದ್ಧಾಯ ಶಠಾಯ ಚ।\\
ಭಕ್ತಿಹೀನಾಯ ಮಲಿನೇ ಗುರುನಿಂದಾಪರಾಯ ಚ ॥೩೦॥

ಅಲಸಾಯಾಯತ್ನವತೇऽಶಿವಭಕ್ತಾಯ ಸುಂದರಿ।\\
ವಿಷ್ಣುಭಕ್ತಿವಿಹೀನಾಯ ನ ದಾತವ್ಯಂ ಕದಾಚನ ॥೩೧॥

ದೇಯಂ ಭಕ್ತವರಾಯೈತತ್ಭುಕ್ತಿಮುಕ್ತಿಕರಂ ಶುಭಮ್।\\
ಸಿದ್ಧಿದಂ ಭವರೋಗಘ್ನಂ ಸ್ತೋತ್ರಮೇತದ್ವರಾನನೇ ॥೩೨॥

ಲತಾಯೋಗೇ ಪಠೇದ್ಯಸ್ತು ತಸ್ಯ ಕ್ಷಿಪ್ರಂ ಫಲಂ ಭವೇತ್।\\
ಸೈವ ಕಲ್ಪಲತಾ ತಸ್ಯ ವಾಂಛಾಫಲಕರೀ ಸ್ಮೃತಾ ॥೩೩॥

ಪುಷ್ಪಿತಾಂ ಯಾಂ ಲತಾಂ ಸಮ್ಯಕ್ ದೃಷ್ಟ್ವಾ ಶ್ರೀಲಲಿತಾಂ ಸ್ಮರನ್।\\
ಅಕ್ಷುಬ್ಧಃ ಪ್ರಪಠೇದ್ಯಸ್ತು ಸ ಯಜ್ಞಕ್ರತುಪುಣ್ಯಭಾಕ್ ॥೩೪॥

ವಿಕಲ್ಪರಹಿತೋ ಯೋ ಹಿ ನಿರ್ವಿಕಲ್ಪಃ ಸ್ವಯಂ ಶಿವಃ।\\
ನೈತತ್ಪ್ರಕಾಶಯೇದ್ಭಕ್ತಃ ಕುಶಿಷ್ಯಾಯಾಲ್ಪಮೇಧಸೇ ॥೩೫॥

ಅನೇಕಜನ್ಮಪುಣ್ಯೇನ ದೀಕ್ಷಿತೋ ಜಾಯತೇ ನರಃ।\\
ತತ್ರಾಪ್ಯನೇಕಭಾಗ್ಯೇನ ಶೈವೋ ವಿಷ್ಣು ಪರಾಯಣಃ ॥೩೬॥

ತತ್ರಾಪ್ಯನೇಕಪುಣ್ಯೇನ ಶಕ್ತಿಭಾವಃ ಪ್ರಜಾಯತೇ।\\
ಮಹೋದಯೇನ ತತ್ರಾಪಿ ಸುಂದರೀಭಾವಭಾಗ್ಭವೇತ್ ॥೩೭॥

ಸಹಸ್ರನಾಮ್ನಾಂ ತತ್ರಾಪಿ ಕೀರ್ತನಂ ಚ ಸುದುರ್ಲಭಮ್।\\
ಯತ್ರ ಜನ್ಮನಿ ಸಾ ನಿತ್ಯಂ ಪೂರ್ವಪುಣ್ಯವಶಾದ್ಭವೇತ್ ॥೩೮॥

ಜೀವನ್ಮುಕ್ತೋ ಭವೇತ್ತಸ್ಯ ಕರ್ತವ್ಯಂ ನಾವಶಿಷ್ಯತೇ।\\
ಅವಧೂತತ್ವಮೇವ ಸ್ಯಾತ್ ನ ವರ್ಣಾಶ್ರಮಕಲ್ಪನಾ ॥೩೯॥

ಬ್ರಹ್ಮಾದಯೋऽಪಿ ದೇವೇಶೀಂ ಪ್ರಾರ್ಥಯಂತೇ ತದವ್ಯಯಾಮ್।\\
ಹಂಸತ್ವಂ ಭಕ್ತಿಭಾವೇನ ಪರಮಾನಂದಕಾರಣಮ್ ॥೪೦॥

ದೇವೋऽಸೌ ಸರ್ವದಾ ಶಕ್ತಿ ಭಾವಯನ್ನೇವ ಸಂಸ್ಥಿತಃ।\\
ಸ್ವಯಂ ಶಿವಸ್ತು ವಿಜ್ಞೇಯಃ ಸುಂದರೀಭಾವಲಂಪಟಃ ॥೪೧॥

ಬ್ರಹ್ಮಾನಂದಮಯೀಂ ಜ್ಯೋತ್ಸ್ನಾಂ ಸದಾಶಿವಪರಾಯಣಾಮ್।\\
ಶಿವಕಾಮೇಶ್ವರೀಂ ದೇವೀಂ ಭಾವಯನ್ ಸಿದ್ಧಿಮಾಪ್ನುಯಾತ್ ॥೪೨॥

ಆಹ್ಲಾದಃ ಸುಂದರೀಧ್ಯಾನಾತ್ ಸುಂದರೀನಾಮಕೀರ್ತನಾತ್।\\
ಸುಂದರೀದರ್ಶನಾಚ್ಚೈವ ಸದಾನಂದಃ ಪ್ರಜಾಯತೇ ॥೪೩॥
\authorline{ಇತಿ ಶ್ರೀರುದ್ರಯಾಮಲೇ ಉಮಾಮಹೇಶಸಂವಾದೇ ಶ್ರೀಶಿವಕಾಮಸುಂದರ್ಯಾಃ ಶ್ರೀಮತ್ತ್ರಿಪುರಸುಂದರ್ಯಾಃ ಷೋಡಶಾರ್ಣಾಯಾಃ ತುರೀಯಸಹಸ್ರನಾಮಸ್ತೋತ್ರಂ ಸಂಪೂರ್ಣಂ ॥}
%==============================================================================================
\section{ಶ್ರೀಬಾಲಾಸಹಸ್ರನಾಮಸ್ತೋತ್ರಂ }
\addcontentsline{toc}{section}{ಶ್ರೀಬಾಲಾಸಹಸ್ರನಾಮಸ್ತೋತ್ರಂ }
ಶೌನಕ ಉವಾಚ\\
ಕೈಲಾಸಶಿಖರೇ ರಮ್ಯೇ ನಾನಾಪುಯ್ಷ್ಪೋಪಶೋಭಿತೇ ।\\
ಕಲ್ಪಪಾದಪಮಧ್ಯಸ್ಥೇ ಗಂಧರ್ವಗಣಸೇವಿತೇ ॥ ೧॥

ಮಣಿಮಂಡಪಮಧ್ಯಸ್ಥೇ ನಾನಾರತ್ನೋಪಶೋಭಿತೇ ।\\
ತಂ ಕದಾಚಿತ್ ಸುಖಾಸೀನಂ ಭಗವಂತಂ ಜಗದ್ಗುರುಂ ॥ ೨॥

ಕಪಾಲಖಟ್ವಾಂಗಧರಂ ಚಂದ್ರಾರ್ಧಕೃತಶೇಖರಂ ।\\
ತ್ರಿಶೂಲಡಮರುಧರಂ ಮಹಾವೃಷಭವಾಹನಂ ॥ ೩॥

ಜಟಾಜೂಟಧರಂ ದೇವಂ ವಾಸುಕಿಕಂಠಭೂಷಣಂ ।\\
ವಿಭೂತಿಭೂಷಣಂ ದೇವಂ ನೀಲಕಂಠಂ ತ್ರಿಲೋಚನಂ ॥ ೪॥

ದ್ವೀಪಿಚರ್ಮಪರೀಧಾನಂ ಶುದ್ಧಸ್ಫಟಿಕಸನ್ನಿಭಂ ।\\
ಸಹಸ್ರಾದಿತ್ಯಸಂಕಾಶಂ ಗಿರಿಜಾರ್ಧಾಂಗಭೂಷಣಂ ॥ ೫॥

ಪ್ರಣಮ್ಯ ಶಿರಸಾ ನಾಥಂ ಕಾರಣಂ ವಿಶ್ವರೂಪಿಣಂ ।\\
ಕೃತಾಂಜಲಿಪುಟೋ ಭೂತ್ವಾ ಪ್ರಾಹ ತಂ ಶಿಖಿವಾಹನಃ ॥ ೬॥

ಕಾರ್ತಿಕೇಯ ಉವಾಚ\\
ದೇವದೇವ ಮಹಾದೇವ ! ಸೃಷ್ಟಿಸ್ಥಿತ್ಯಂತಕಾರಕ ! ।\\
ತ್ವಂ ಗತಿಃ ಸರ್ವದೇವಾನಾಂ ತ್ವಂ ಗತಿಃ ಸರ್ವದೇಹಿನಾಂ ॥ ೭॥

ತ್ವಂ ಗತಿಃ ಸರ್ವದೇವಾನಾಂ ಸರ್ವೇಷಾಂ ತ್ವಂ ಗತಿರ್ವಿಭೋ ! ।\\
ತ್ವಮೇವ ಜಗದಾಧಾರಸ್ತ್ವಮೇವ ವಿಶ್ವಕಾರಣಂ ॥ ೮॥

ತ್ವಮೇವ ಪೂಜ್ಯಃ ಸರ್ವೇಷಾಂ ತ್ವದನ್ಯೋ ನಾಸ್ತಿ ಮೇ ಗತಿಃ ।\\
ಕಿಂ ಗುಹ್ಯಂ ಪರಮಂ ಲೋಕೇ ಕಿಮೇಕಂ ಸರ್ವಸಿದ್ಧಿದಂ ॥ ೯॥

ಕಿಮೇಕಂ ಪರಮಂ ಸೃಷ್ಟಿಃ ಕಿಂ ಭೌಮೈಶ್ವರ್ಯಮೋಕ್ಷದಂ ।\\
ವಿನಾ ತೀರ್ಥೇನ ತಪಸಾ ವಿನಾ ವೇದೈರ್ವಿನಾ ಮಖೈಃ ॥ ೧೦॥

ವಿನಾ ಜಾಪ್ಯೇನ ಧ್ಯಾನೇನ ಕಥಂ ಸಿದ್ಧಿಮವಾಪ್ನುಯಾತ್ ।\\
ಕಸ್ಮಾದುತ್ಪದ್ಯತೇ ಸೃಷ್ಟಿಃ ಕಸ್ಮಿಂಶ್ಚ ವಿಲಯೋ ಭವೇತ್ ॥ ೧೧॥

ಕಸ್ಮಾದುತ್ತೀರ್ಯತೇ ದೇವ ! ಸಂಸಾರಾರ್ಣವಸಂಕಟಾತ್ ।\\
ತದಹಂ ಶ್ರೋತುಮಿಚ್ಛಾಮಿ ಕಥಯಸ್ವ ಮಹೇಶ್ವರ ! ॥ ೧೨॥

ಶ್ರೀಮಹಾದೇವ ಉವಾಚ\\
ಸಾಧು ಸಾಧು ತ್ವಯಾ ಪೃಷ್ಟೋಽಸ್ಮ್ಯಹಂ ಪಾರ್ವತೀನಂದನ ! ।\\
ಅಸ್ತಿ ಗುಹ್ಯತಮಂ ಪುತ್ರ ! ಕಥಯಿಷ್ಯಾಮ್ಯಸಂಶಯಂ ॥ ೧೩॥

ಸತ್ತ್ವಂ ರಜಸ್ತಮಶ್ಚೈವ ಬ್ರಹ್ಮವಿಷ್ಣುಶಿವಾದಯಃ ।\\
ಯೇ ಚಾನ್ಯೇ ಬಹವೋ ಭೂತಾಃ ಸರ್ವೇ ಪ್ರಕೃತಿಸಂಭವಾಃ ॥ ೧೪॥

ಸೈವ ದೇವೀ ಪರಾಶಕ್ತಿರ್ಮಹಾತ್ರಿಪುರಸುಂದರೀ ।\\
ಸೈವ ಸಂಹರತೇ ವಿಶ್ವಂ ಜಗದೇತಚ್ಚರಾಚರಂ ॥ ೧೫॥

ಆಧಾರಂ ಸರ್ವಭೂತಾನಾಂ ಸೈವ ರೋಗಾರ್ತಿಹಾರಿಣೀ ।\\
ಇಚ್ಛಾಶಕ್ತಿಃ ಕ್ರಿಯಾರೂಪಾ ಬ್ರಹ್ಮವಿಷ್ಣುಶಿವಾತ್ಮಿಕಾ ॥ ೧೬॥

ತ್ರಿಧಾ ಶಕ್ತಿಸ್ವರೂಪೇಣ ಸೃಷ್ಟಿಸ್ಥಿತಿವಿನಾಶಿನೀ ।\\
ಸೃಜತಿ ಬ್ರಹ್ಮರೂಪೇಣ ವಿಷ್ಣುರೂಪೇಣ ರಕ್ಷತಿ ॥ ೧೭॥

ಹರತೇ ರುದ್ರರೂಪೇಣ ಜಗದೇತಚ್ಚರಾಚರಂ ।\\
ಯಸ್ಯ ಯೋನೌ ಜಗತ್ಸರ್ವಮದ್ಯಾಪಿ ವರ್ತತೇಽಖಿಲಂ ॥ ೧೮॥

ಯಸ್ಯಾಂ ಪ್ರಲೀಯತೇ ಚಾಂತೇ ಯಸ್ಯಾಂ ಚ ಜಾಯತೇ ಪುನಃ ।\\
ಯಾಂ ಸಮಾರಾಧ್ಯ ತ್ರೈಲೋಕ್ಯೇ ಸಂಪ್ರಾಪ್ತಂ ಪದಮುತ್ತಮಂ ।\\
ತಸ್ಯಾ ನಾಮಸಹಸ್ರಂ ತೇ ಕಥಯಾಮಿ ಶೃಣುಷ್ವ ತತ್ ॥ ೧೯॥

ಓಂ ಅಸ್ಯ ಶ್ರೀಬಾಲಾಸಹಸ್ರನಾಮಸ್ತೋತ್ರಮಂತ್ರಸ್ಯ ಭಗವಾನ್ ದಕ್ಷಿಣಾಮುರ್ತಿರ್ವಾಮದೇವಃ ಋಷಿಃ । ಗಾಯತ್ರೀ ಛಂದಃ । ಪ್ರಕಟಗುಪ್ತಗುಪ್ತತರಸಂಪ್ರದಾಯ ಕುಲಕೌಲೋತ್ತೀರ್ಣಾನಿಗರ್ಭರಹಸ್ಯಾತಿರಹಸ್ಯ ಪರಾಪರರಹಸ್ಯಾ ಚಿಂತ್ಯಾ ವರ್ತಿನೀ ಬಾಲಾ ದೇವತಾ । ಆಂ ಬೀಜಂ । ಹ್ರೀಂ ಶಕ್ತಿಃ । ಕ್ಲೀಂ ಕೀಲಕಂ । ಶ್ರೀ ಬಾಲಾಪ್ರೀತ್ಯರ್ಥೇ ಪಾರಾಯಣೇ ವಿನಿಯೋಗಃ ।\\

\dhyana{ಆಧಾರೇ ತರುಣಾರ್ಕಬಿಂಬಸದೃಶಂ ಹೇಮಪ್ರಭಂ ವಾಗ್ಭವಂ\\
ಬೀಜಂ ಮಾನ್ಮಥಮಿಂದ್ರಗೋಪಸದೃಶಂ ಹೃತ್ಪಂಕಜೇ ಸಂಸ್ಥಿತಂ ।\\
ಚಕ್ರಂ ಭಾಲಮಯಂ ಶಶಾಂಕರುಚಿರಂ ಬೀಜಂ ತು ತಾರ್ತೀಯಕಂ\\
ಯೇ ಧ್ಯಾಯಂತಿ ಪದತ್ರಯಂ ತವ ಶಿವೇ ತೇ ಯಾಂತಿ ಸೂಕ್ಷ್ಮಾಂ ಗತಿಂ ॥}

ಕಲ್ಯಾಣೀ ಕಮಲಾ ಕಾಲೀ ಕರಾಲೀ ಕಾಮರೂಪಿಣೀ ।\\
ಕಾಮಾಕ್ಷಾ ಕಾಮದಾ ಕಾಮ್ಯಾ ಕಾಮನಾ ಕಾಮಚಾರಿಣೀ ॥ ೨೨॥

ಕೌಮಾರೀ ಕರುಣಾಮೂರ್ತಿಃ ಕಲಿಕಲ್ಮಷನಾಶಿನೀ ।\\
ಕಾತ್ಯಾಯನೀ ಕಲಾಧಾರಾ ಕೌಮುದೀ ಕಮಲಪ್ರಿಯಾ ॥ ೨೩॥

ಕೀರ್ತಿದಾ ಬುದ್ಧಿದಾ ಮೇಧಾ ನೀತಿಜ್ಞಾ ನೀತಿವತ್ಸಲಾ ।\\
ಮಾಹೇಶ್ವರೀ ಮಹಾಮಾಯಾ ಮಹಾತೇಜಾ ಮಹೇಶ್ವರೀ ॥ ೨೪॥

ಮಹಾಮೋಹಾಂಧಕಾರಘ್ನೀ ಮಹಾಮೋಕ್ಷಪ್ರದಯಿನೀ ।\\
ಮಹಾದಾರಿದ್ರ್ಯರಾಶಿಘ್ನೀ ಮಹಾಶತ್ರುವಿಮರ್ದಿನೀ ॥ ೨೬॥

ಮಹಾಶಕ್ತಿರ್ಮಹಾಜ್ಯೋತಿರ್ಮಹಾಸುರವಿಮರ್ದಿನೀ ।\\
ಮಹಾಕಾಯಾ ಮಹಾಬೀಜಾ ಮಹಾಪಾತಕನಾಶಿನೀ ॥ ೨೭॥

ಮಹಾಮಖಾ ಮಂತ್ರಮಯೀ ಮಣಿಪುರನಿವಾಸಿನೀ ।\\
ಮಾನಸೀ ಮಾನದಾ ಮಾನ್ಯಾ ಮನಶ್ಚಕ್ಷುರಗೋಚರಾ ॥ ೨೮॥

ಗಣಮಾತಾ ಚ ಗಾಯತ್ರೀ ಗಣಗಂಧರ್ವಸೇವಿತಾ ।\\
ಗಿರಿಜಾ ಗಿರಿಶಾ ಸಾಧ್ವೀ ಗಿರಿಸೂರ್ಗಿರಿಸಂಭವಾ ॥ ೨೯॥

ಚಂಡೇಶ್ವರೀ ಚಂದ್ರರೂಪಾ ಪ್ರಚಂಡಾ ಚಂಡಮಾಲಿನೀ ।\\
ಚರ್ಚಿಕಾ ಚರ್ಚಿತಾಕಾರಾ ಚಂಡಿಕಾ ಚಾರುರೂಪಿಣೀ ॥ ೩೦॥

ಯಜ್ಞೇಶ್ವರೀ ಯಜ್ಞರೂಪಾ ಜಪಯಜ್ಞಪರಾಯಣಾ ।\\
ಯಜ್ಞಮಾತಾ ಯಜ್ಞಗೋಪ್ತ್ರೀ ಯಜ್ಞೇಶೀ ಯಜ್ಞಸಂಭವಾ ॥ ೩೧॥

ಯಜ್ಞಸಿದ್ಧಿಃ ಕ್ರಿಯಾಸಿದ್ಧಿರ್ಯಜ್ಞಾಂಗೀ ಯಜ್ಞರಕ್ಷಕಾ ।\\
ಯಜ್ಞಪ್ರಿಯಾ ಯಜ್ಞರೂಪಾ ಯಾಜ್ಞೀ ಯಜ್ಞಕೃಪಾಲಯಾ ॥ ೩೨॥

ಜಾಲಂಧರೀ ಜಗನ್ಮಾತಾ ಜಾತವೇದಾ ಜಗತ್ಪ್ರಿಯಾ ।\\
ಜಿತೇಂದ್ರಿಯಾ ಜಿತಕ್ರೋಧಾ ಜನನೀ ಜನ್ಮದಾಯಿನೀ ॥ ೩೩॥

ಗಂಗಾ ಗೋದಾವರೀ ಗೌರೀ ಗೌತಭೀ ಚ ಶತಹ್ರದಾ ।\\
ಘುರ್ಘುರಾ ವೇದಗರ್ಭಾ ಚ ರೇವಿಕಾ ಕರಸಂಭವಾ ॥ ೩೪॥

ಸಿಂಧುರ್ಮಂದಾಕಿನೀ ಕ್ಷಿಪ್ರಾ ಯಮುನಾ ಚ ಸರಸ್ವತೀ ।\\
ಚಂದ್ರಭಾಗಾ ವಿಪಾಶಾ ಚ ಗಂಡಕೀ ವಿಂಧ್ಯವಾಸಿನೀ ॥ ೩೫॥

ನರ್ಮದಾ ಕಹ್ನಕಾವೇರೀ ವೇತ್ರವತ್ಯಾ ಚ ಕೌಶಿಕೀ ।\\
ಮಹೋನತನಯಾ ಚೈವ ಅಹಲ್ಯಾ ಚಂಪಕಾವತೀ ॥ ೩೬॥

ಅಯೋಧ್ಯಾ ಮಥುರಾ ಮಾಯಾ ಕಾಶೀ ಕಾಂಚೀ ಅವಂತಿಕಾ ।\\
ದ್ವಾರಾವತೀ ಚ ತೀರ್ಥೇಶೀ ಮಹಾಕಿಲ್ವಿಷನಾಶಿನೀ ॥ ೩೭॥

ಪದ್ಮಿನೀ ಪದ್ಮಮಧ್ಯಸ್ಥಾ ಪದ್ಮಕಿಂಜಲ್ಕವಾಸಿನೀ ।\\
ಪದ್ಮವಕ್ತ್ರಾ ಚ ಪದ್ಮಾಕ್ಷೀ ಪದ್ಮಸ್ಥಾ ಪದ್ಮಸಂಭವಾ ॥ ೩೮॥

ಹ್ರೀಂಕಾರೀ ಕುಂಡಲೀ ಧಾತ್ರೀ ಹೃತ್ಪದ್ಮಥಾ ಸುಲೋಚನಾ ।\\
ಶ್ರೀಂಕಾರೀ ಭೂಷಣಾ ಲಕ್ಷ್ಮೀಃ ಕ್ಲೀಂಕಾರೀ ಕ್ಲೇಶನಾಶಿನೀ ॥ ೩೯॥

ಹರಿಪ್ರಿಯಾ ಹರೇರ್ಮೂರ್ತಿರ್ಹರಿನೇತ್ರಕೃತಾಲಯಾ ।\\
ಹರಿವಕ್ತ್ರೋದ್ಭವಾ ಶಾಂತಾ ಹರಿವಕ್ಷಃಸ್ಥಿತಾಲಯಾ ॥ ೪೦॥

ವೈಷ್ಣವೀ ವಿಷ್ಣುರೂಪಾ ಚ ವಿಷ್ಣುಮಾತೃಸ್ವರೂಪಿಣೀ ।\\
ವಿಷ್ಣುಮಾಯಾ ವಿಶಾಲಾಕ್ಷೀ ವಿಶಾಲನಯನೋಜ್ಜ್ವಲಾ ॥ ೪೧॥

ವಿಶ್ವೇಶ್ವರೀ ಚ ವಿಶ್ವಾತ್ಮಾ ವಿಶ್ವೇಶೀ ವಿಶ್ವರೂಪಿಣೀ ।\\
ವಿಶ್ವೇಶ್ವರೀ ಶಿವಾಧಾರಾ ಶಿವನಾಥಾ ಶಿವಪ್ರಿಯಾ ॥ ೪೨॥

ಶಿವಮಾತಾ ಶಿವಾಕ್ಷೀ ಚ ಶಿವದಾ ಶಿವರೂಪಿಣೀ ।\\
ಭವೇಶ್ವರೀ ಭವಾರಾಧ್ಯಾ ಭವೇಶೀ ಭವನಾಯಿಕಾ ॥ ೪೩॥

ಭವಮಾತಾ ಭವಾಗಮ್ಯಾ ಭವಕಂಟಕನಾಶಿನೀ ।\\
ಭವಪ್ರಿಯಾ ಭವಾನಂದಾ ಭವಾನೀ ಭವಮೋಚಿನೀ ॥ ೪೪॥

ಗೀತಿರ್ವರೇಣ್ಯಾ ಸಾವಿತ್ರೀ ಬ್ರಹ್ಮಾಣೀ ಬ್ರಹ್ಮರೂಪಿಣೀ ।\\
ಬ್ರಹ್ಮೇಶೀ ಬ್ರಹ್ಮದಾ ಬ್ರಾಹ್ಮೀ ಬ್ರಹ್ಮಾಣೀ ಬ್ರಹ್ಮವಾದಿನೀ ॥ ೪೫॥

ದುರ್ಗಸ್ಥಾ ದುರ್ಗರೂಪಾ ಚ ದುರ್ಗಾ ದುರ್ಗಾರ್ತಿನಾಶಿನೀ ।\\
ತ್ರಯೀದಾ ಬ್ರಹ್ಮದಾ ಬ್ರಾಹ್ಮೀ ಬ್ರಹ್ಮಾಣೀ ಬ್ರಹ್ಮವಾದಿನೀ ॥ ೪೬॥

ತ್ವಕ್ಸ್ಥಾ ತಥಾ ಚ ತ್ವಗ್ರೂಪಾ ತ್ವಗ್ಗಾ ತ್ವಗಾರ್ತಿಹಾರಿಣೀ ।\\
ಸ್ವಗಮಾ ನಿರ್ಗಮಾ ದಾತ್ರೀ ದಾಯಾ ದೋಗ್ಧ್ರೀ ದುರಾಪಹಾ ॥ ೪೭॥

ದೂರಘ್ನೀ ಚ ದುರಾರಾಧ್ಯಾ ದೂರದುಷ್ಕೃತನಾಶಿನೀ ।\\
ಪಂಚಸ್ಥಾ ಪಂಚಮೀ ಪೂರ್ಣಾ ಪೂರ್ಣಾಪೀಠ ನಿವಾಸಿನೀ ॥ ೪೮॥

ಸತ್ತ್ವಸ್ಥಾ ಸತ್ತ್ವರೂಪಾ ಚ ಸತ್ತ್ವದಾ ಸತ್ತ್ವಸಂಭವಾ ।\\
ರಜಃಸ್ಥಾ ಚ ರಜೋತೂಪಾ ರಜೋಗುಣಸಮುದ್ಭವಾ ॥ ೪೯॥

ತಾಮಸೀ ಚ ತಮೋರೂಪಾ ತಾಮಸೀ ತಮಸಃ ಪ್ರಿಯಾ ।\\
ತಮೋಗುಣಸಮುದ್ಭೂತಾ ಸಾತ್ತ್ವಿಕೀ ರಾಜಸೀ ತಮೀ ॥ ೫೦॥

ಕಲಾ ಕಾಷ್ಠಾ ನಿಮೇಷಾ ಚ ಸ್ವಕೃತಾ ತದನಂತರಾ ।\\
ಅರ್ಧಮಾಸಾ ಚ ಮಾಸಾ ಚ ಸಂವತ್ಸರಸ್ವರೂಪಿಣೀ ॥ ೫೧॥

ಯುಗಸ್ಥಾ ಯುಗರೂಪಾ ಚ ಕಲ್ಪಸ್ಥಾ ಕಲ್ಪರೂಪಿಣೀ ।\\
ನಾನಾರತ್ನವಿಚಿತ್ರಾಂಗೀ ನಾನಾಭರಣಮಂಡಿತಾ ॥ ೫೨॥

ವಿಶ್ವಾತ್ಮಿಕಾ ವಿಶ್ವಮಾತಾ ವಿಶ್ವಪಾಶಾ ವಿಧಾಯಿನೀ ।\\
ವಿಶ್ವಾಸಕಾರಿಣೀ ವಿಶ್ವಾ ವಿಶ್ವಶಕ್ತಿರ್ವಿಚಕ್ಷಣಾ ॥ ೫೩॥

ಜಪಾಕುಸುಮಸಂಕಾಶಾ ದಾಡಿಮೀಕುಸುಮೋಪಮಾ ।\\
ಚತುರಂಗಾ ಚತುರ್ಬಾಹುಶ್ಚತುರಾ ಚಾರುಹಾಸಿನೀ ॥ ೫೪॥

ಸರ್ವೇಶೀ ಸರ್ವದಾ ಸರ್ವಾ ಸರ್ವಜ್ಞಾ ಸರ್ವದಾಯಿನೀ ।\\
ಸರ್ವೇಶ್ವರೀ ಸರ್ವವಿದ್ಯಾ ಶರ್ವಾಣೀ ಸರ್ವಮಂಗಲಾ ॥ ೫೫॥

ನಲಿನೀ ನಂದಿನೀ ನಂದಾ ಆನಂದಾನಂದವರ್ಧಿನೀ ।\\
ವ್ಯಾಪಿನೀ ಸರ್ವಭೂತೇಷು ಭವಭಾರವಿನಾಶಿನೀ ॥ ೫೬॥

ಕುಲೀನಾ ಕುಲಮಧ್ಯಸ್ಥಾ ಕುಲಧರ್ಮೋಪದೇಶಿನೀ ।\\
ಸರ್ವಶೃಂಗಾರವೇಶಾಢ್ಯಾ ಪಾಶಾಂಕುಶಕರೋದ್ಯತಾ ॥ ೫೭॥

ಸೂರ್ಯಕೋಟಿಸಹಸ್ರಭಾ ಚಂದ್ರಕೋಟಿನಿಭಾನನಾ ।\\
ಗಣೇಶಕೋಟಿಲಾವಣ್ಯಾ ವಿಷ್ಣುಕೋಟ್ಯರಿಮರ್ದಿನೀ ॥ ೫೮॥

ದಾವಾಗ್ನಿಕೋಟಿಜ್ವಲಿನೀ ರುದ್ರಕೋಟ್ಯುಗ್ರರೂಪಿಣೀ
ಸಮುದ್ರಕೋಟಿಗಂಭೀರಾ ವಾಯುಕೋಟಿಮಹಾಬಲಾ ॥ ೫೯॥

ಆಕಾಶಕೋಟಿವಿಸ್ತಾರಾ ಯಮಕೋಟಿಭಯಂಕರಾ ।\\
ಮೇರುಕೋಟಿಸಮುಂಚ್ಛ್ರಾ ಯಾ ಗುಣಕೋಟಿಸಮೃದ್ಧಿದಾ ॥ ೬೦॥

ನಿಷ್ಕಲಂಕಾ ನಿರಾಧಾರಾ ನಿರ್ಗುಣಾ ಗುಣವರ್ಜಿತಾ ।\\
ಅಶೋಕಾ ಶೋಕರಹಿತಾ ತಾಪತ್ರಯವಿವರ್ಜಿತಾ ॥ ೬೧॥

ವಿಶಿಷ್ಟಾ ವಿಶ್ವಜನನೀ ವಿಶ್ವಮೋಹವಿಧಾರಿಣೀ ।\\
ಚಿತ್ರಾ ವಿಚಿತ್ರಾ ಚಿತ್ರಾಶೀ ಹೇತುಗರ್ಭಾ ಕುಲೇಶ್ವರೀ ॥ ೬೨॥

ಇಚ್ಛಾಶಾಕ್ತಿಃ ಜ್ಞಾನಶಕ್ತಿಃ ಕ್ರಿಯಾಶಕ್ತಿಃ ಶುಚಿಸ್ಮಿತಾ ।\\
ಶ್ರುತಿಸ್ಮೃತಿಮಯೀ ಸತ್ಯಾ ಶ್ರುತಿರೂಪಾ ಶ್ರುತಿಪ್ರಿಯಾ ॥ ೬೩॥

ಶ್ರುತಿಪ್ರಜ್ಞಾ ಮಹಾಸತ್ಯಾ ಪಂಚತತ್ತ್ವೋಪರಿಸ್ಥಿತಾ ।\\
ಪಾರ್ವತೀ ಹಿಮವತ್ಪುತ್ರೀ ಪಾಶಸ್ಥಾ ಪಾಶರೂಪಿಣೀ ॥ ೬೪॥

ಜಯಂತೀ ಭದ್ರಕಾಲೀ ಚ ಅಹಲ್ಯಾ ಕುಲನಾಯಿಕಾ ।\\
ಭೂತಧಾತ್ರೀ ಚ ಭೂತೇಶೀ ಭೂತಸ್ಥಾ ಭೂತಭಾವಿನೀ ॥ ೬೫॥

ಮಹಾಕುಂಡಲಿನೀಶಕ್ತಿರ್ಮಹಾವಿಭವವರ್ಧಿನೀ ।\\
ಹಂಸಾಕ್ಷೀ ಹಂಸರೂಪಾ ಚ ಹಂಸಸ್ಥಾ ಹಂಸರೂಪಿಣೀ ॥ ೬೬॥

ಸೋಮಸೂರ್ಯಾಗ್ನಿಮಧ್ಯಸ್ಥಾ ಮಣಿಪೂರಕವಾಸಿನೀ ।\\
ಷಟ್ ಪತ್ರಾಂಭೋಜಮಧ್ಯಸ್ಥಾ ಮಣಿಪೂರನಿವಾಸಿನೀ ॥ ೬೭॥

ದ್ವಾದಶಾರಸರೋಜಸ್ಥಾ ಸೂರ್ಯಮಂಡಲವಾಸಿನೀ ।\\
ಅಕಲಂಕಾ ಶಶಾಂಕಾಭಾ ಷೋಡಶಾರನಿವಾಸಿನೀ ॥ ೬೮॥

ದ್ವಿಪತ್ರದಲಮಧ್ಯಸ್ಥಾ ಲಲಾಟತಲವಾಸಿನೀ ।\\
ಡಾಕಿನೀ ಶಾಕಿನೀ ಚೈವ ಲಾಕಿನೀ ಕಾಕಿನೀ ತಥಾ ॥ ೬೯॥

ರಾಕಿಣೀ ಹಾಕಿನೀ ಚೈವ ಷಟ್ಚಕ್ರಕ್ರಮವಾಸಿನೀ ।\\
ಸೃಷ್ಟಿಸ್ಥಿತಿವಿನಾಶಾ ಚ ಸೃಷ್ಟಿಸ್ಥಿತ್ಯಂತಕಾರಿಣೀ ॥ ೭೦॥

ಶ್ರೀಕಂಠಾ ಶ್ರೀಪ್ರಿಯಾ ಕಂಠನಾದಾಖ್ಯಾ ಬಿಂದುಮಾಲಿನೀ ।\\
ಚತುಃಷಷ್ಟಿಕಲಾಧಾರಾ ಮೇರುದಂಡಸಮಾಶ್ರಯಾ ॥ ೭೧॥

ಮಹಾಕಾಲೀ ದ್ಯುತಿರ್ಮೇಧಾ ಸ್ವಧಾ ತುಷ್ಟಿರ್ಮಹಾದ್ಯುತಿಃ ।\\
ಹಿಂಗುಲಾ ಮಂಗಲಶಿವಾ ಸುಷುಮ್ಣಾಮಧ್ಯಗಾಮಿನೀ ॥ ೭೨॥

ಪರಾ ಘೋರಾ ಕರಾಲಾಕ್ಷೀ ವಿಜಯಾ ಜಯಶಾಲಿನೀ ।\\
ಹೃತ್ಪದ್ಮನಿಲಯಾ ದೇವೀ ಭೀಮಾ ಭೈರವನಾದಿನೀ ॥ ೭೩॥

ಆಕಾಶಲಿಂಗಭೂತಾ ಚ ಭುವನೋದ್ಯಾನವಾಸಿನೀ ।\\
ಮಹಾಸೂಕ್ಷ್ಮಾಽಭಯಾ ಕಾಲೀ ಭೀಮರೂಪಾ ಮಹಾಬಲಾ ॥ ೭೪॥

ಮೇನಕಾಗರ್ಭಸಂಭೂತಾ ತಪ್ತಕಾಂಚನಸನ್ನಿಭಾ ।\\
ಅಂತಃಸ್ಥಾ ಕೂಟಬೀಜಾ ಚ ತ್ರಿಕೂಟಾಚಲವಾಸಿನೀ ॥ ೭೫॥

ವರ್ಣಾಕ್ಷಾ ವರ್ಣರಹಿತಾ ಪಂಚಾಶದ್ವರ್ಣಭೇದಿನೀ ।\\
ವಿದ್ಯಾಧರೀ ಲೋಕಧಾತ್ರೀ ಅಪ್ಸರಾ ಅಪ್ಸರಃಪ್ರಿಯಾ ॥ ೭೬॥

ದಕ್ಷಾ ದಾಕ್ಷಾಯಣೀ ದೀಕ್ಷಾ ದಕ್ಷಯಜ್ಞವಿನಾಶಿನೀ ।\\
ಯಶಸ್ವಿನೀ ಯಶಃಪೂರ್ಣಾ ಯಶೋದಾಗರ್ಭಸಂಭವಾ ॥ ೭೭॥

ದೇವಕೀ ದೇವಮಾತಾ ಚ ರಾಧಿಕಾ ಕೃಷ್ಣವಲ್ಲಭಾ ।\\
ಅರುಂಧತೀ ಶಚೀಂದ್ರಾಣೀ ಗಾಂಧಾರೀ ಗಂಧಮೋದಿನೀ ॥ ೭೮॥

ಧ್ಯಾನಾತೀತಾ ಧ್ಯಾನಗಮ್ಯಾ ಧ್ಯಾನಾ ಧ್ಯಾನಾವಧಾರಿಣೀ ।\\
ಲಂಬೋದರೀ ಚ ಲಂಬೋಷ್ಠಾ ಜಾಂಬವತೀ ಜಲೋದರೀ ॥ ೭೯॥

ಮಹೋದರೀ ಮುಕ್ತಕೇಶೀ ಮುಕ್ತಿಕಾಮಾರ್ಥಸಿದ್ಧಿದಾ ।\\
ತಪಸ್ವಿನೀ ತಪೋನಿಷ್ಠಾ ಚಾಪರ್ಣಾ ಪರ್ಣಭಕ್ಷಿಣೀ ॥ ೮೦॥

ಬಾಣಚಾಪಧರಾ ವೀರಾ ಪಾಂಚಾಲೀ ಪಂಚಮಪ್ರಿಯಾ ।\\
ಗುಹ್ಯಾ ಗಭೀರಾ ಗಹನಾ ಗುಹ್ಯತತ್ತ್ವಾ ನಿರಂಜನಾ ॥ ೮೧॥

ಅಶರೀರಾ ಶರೀರಸ್ಥಾ ಸಂಸಾರಾರ್ಣವತಾರಿಣೀ ।\\
ಅಮೃತಾ ನಿಷ್ಕಲಾ ಭದ್ರಾ ಸಕಲಾ ಕೃಷ್ಣಪಿಂಗಲಾ ॥ ೮೨॥

ಚಕ್ರೇಶ್ವರೀ ಚಕ್ರಹಸ್ತಾ ಪಾಶಚಕ್ರನಿವಾಸಿನೀ ।\\
ಪದ್ಮರಾಗಪ್ರತೀಕಾಶಾ ನಿರ್ಮಲಾಕಾಶಸನ್ನಿಭಾ ॥ ೮೩॥

ಊರ್ಧ್ವಸ್ಥಾ ಊರ್ಧ್ವರೂಪಾ ಚ ಊರ್ಧ್ವಪದ್ಮನಿವಾಸಿನೀ ।\\
ಕಾರ್ಯಕಾರಣಕರ್ತ್ರೀ ಚ ಪರ್ವಾಖ್ಯಾ ರೂಪಸಂಸ್ಥಿತಾ ॥ ೮೪॥

ರಸಜ್ಞಾ ರಸಮಧ್ಯಸ್ಥಾ ಗಂಧಜ್ಞಾ ಗಂಧರೂಪಿಣೀ ।\\
ಪರಬ್ರಹ್ಮಸ್ವರೂಪಾ ಚ ಪರಬ್ರಹ್ಮನಿವಾಸಿನೀ ॥ ೮೫॥

ಶಬ್ದಬ್ರಹ್ಮಸ್ವರೂಪಾ ಚ ಶಬ್ದಸ್ಥಾ ಶಬ್ದವರ್ಜಿತಾ ।\\
ಸಿದ್ಧಿರ್ವೃದ್ಧಿಪರಾ ವೃದ್ಧಿಃ ಸಕೀರ್ತಿರ್ದೀಪ್ತಿಸಂಸ್ಥಿತಾ ॥ ೮೬॥

ಸ್ವಗುಹ್ಯಾ ಶಾಂಭವೀಶಕ್ತಿಸ್ತತ್ತ್ವಜ್ಞಾ ತತ್ತ್ವರೂಪಿಣೀ ।\\
ಸರಸ್ವತೀ ಭೂತಮಾತಾ ಮಹಾಭೂತಾಧಿಪಪ್ರಿಯಾ ॥ ೮೭॥

ಶ್ರುತಿಪ್ರಜ್ಞಾದಿಮಾ ಸಿದ್ಧಿಃ ದಕ್ಷಕನ್ಯಾಽಪರಾಜಿತಾ ।\\
ಕಾಮಸಂದೀಪನೀ ಕಾಮಾ ಸದಾಕಾಮಾ ಕುತೂಹಲಾ ॥ ೮೮॥

ಭೋಗೋಪಚಾರಕುಶಲಾ ಅಮಲಾ ಹ್ಯಮಲಾನನಾ ।\\
ಭಕ್ತಾನುಕಂಪಿನೀ ಮೈತ್ರೀ ಶರಣಾಗತವತ್ಸಲಾ ॥ ೮೯॥

ಸಹಸ್ರಭುಜಾ ಚಿಚ್ಛಕ್ತಿಃ ಸಹಸ್ರಾಕ್ಷಾ ಶತಾನನ ।\\
ಸಿದ್ಧಲಕ್ಷ್ಮೀರ್ಮಹಾಲಕ್ಷ್ಮೀರ್ವೇದಲಕ್ಷ್ಮೀಃ ಸುಲಕ್ಷಣಾ ॥ ೯೦॥

ಯಜ್ಞಸಾರಾ ತಪಸ್ಸಾರಾ ಧರ್ಮಸಾರಾ ಜನೇಶ್ವರೀ ।\\
ವಿಶ್ವೋದರೀ ವಿಶ್ವಸೃಷ್ಟಾ ವಿಶ್ವಾಖ್ಯಾ ವಿಶ್ವತೋಮುಖೀ ॥ ೯೧॥

ವಿಶ್ವಾಸ್ಯಶ್ರವಣಘ್ರಾಣಾ ವಿಶ್ವಮಾಲಾ ಪರಾತ್ಮಿಕಾ ।\\
ತರುಣಾದಿತ್ಯಸಂಕಾಶಾ ಕರಣಾನೇಕಸಂಕುಲಾ ॥ ೯೨॥

ಕ್ಷೋಭಿಣೀ ಮೋಹಿನೀ ಚೈವ ಸ್ತಂಭಿನೀ ಜೃಂಭಿನೀ ತಥಾ ।\\
ರಥಿನೀ ಧ್ವಜಿನೀ ಸೇನಾ ಸರ್ವಮಂತ್ರಮಯೀ ತ್ರಯೀ ॥ ೯೩॥

ಜ್ಞಾನಮುದ್ರಾ ಮಹಾಮುದ್ರಾ ಜಪಮುದ್ರಾ ಮಹೋತ್ಸವಾ ।\\
ಜಟಾಜೂಟ ಧರಾ ಮುಕ್ತಾ ಸೂಕ್ಷ್ಮಶಾಂತಿರ್ವಿಭೀಷಣಾ ॥ ೯೪॥

ದ್ವೀಪಿಚರ್ಮಪರೀಧಾನಾ ಚೀರವಲ್ಕಲಧಾರಿಣೀ ।\\
ತ್ರಿಶೂಲಡಮರುಧರಾ ನರಮಾಲಾವಿಭೂಷಿಣೀ ॥ ೯೫॥

ಅತ್ಯುಗ್ರರೂಪಿಣೀ ಚೋಗ್ರಾ ಕಲ್ಪಾಂತದಹನೋಪಮಾ ।\\
ತ್ರೈಲೋಕ್ಯಸಾಧಿನೀ ಸಾಧ್ಯಾ ಸಿದ್ಧಸಾಧಕವತ್ಸಲಾ ॥ ೯೬॥

ಸರ್ವವಿದ್ಯಾಮಯೀ ಸಾರಾ ಅಸುರಾಂಬುಧಿಧಾರಿಣೀ ।\\
ಸುಭಗಾ ಸುಮುಖೀ ಸೌಮ್ಯಾ ಸುಶೂರಾ ಸೋಮಭೂಷಣಾ ॥ ೯೭॥

ಶುದ್ಧಸ್ಫಟಿಕಸಂಕಶಾ ಮಹಾವೃಷಭವಾಹಿನೀ ।\\
ಮಹಿಷೀ ಮಹಿಷಾರೂಢಾ ಮಹಿಷಾಸುರಘಾತಿನೀ ॥ ೯೮॥

ದಮಿನೀ ದಾಮಿನೀ ದಾಂತಾ ದಯಾ ದೋಗ್ಧ್ರೀ ದುರಾಪಹಾ ।\\
ಅಗ್ನಿಜಿಹ್ವಾ ಮಹಾಘೋರಾಽಘೋರಾ ಘೋರತರಾನನಾ ॥ ೯೯॥

ನಾರಾಯಣೀ ನಾರಸಿಂಹೀ ನೃಸಿಂಹಹೃದಯಸ್ಥಿತಾ ।\\
ಯೋಗೇಶ್ವರೀ ಯೋಗರೂಪಾ ಯೋಗಮಾಲಾ ಚ ಯೋಗಿನೀ ॥ ೧೦೦॥

ಖೇಚರೀ ಭೂಚರೀ ಖೇಲಾ ನಿರ್ವಾಣಪದಸಂಶ್ರಯಾ ।\\
ನಾಗಿನೀ ನಾಗಕನ್ಯಾ ಚ ಸುವೇಗಾ ನಾಗನಾಯಿಕಾ ॥ ೧೦೧॥

ವಿಷಜ್ವಾಲಾವತೀ ದೀಪ್ತಾ ಕಲಾಶತವಿಭೂಷಣಾ ।\\
ಭೀಮವಕ್ತ್ರಾ ಮಹಾವಕ್ತ್ರಾ ವಕ್ತ್ರಾಣಾಂ ಕೋಟಿಧಾರಿಣೀ ॥ ೧೦೨॥

ಮಹದಾತ್ಮಾ ಚ ಧರ್ಮಜ್ಞಾ ಧರ್ಮಾತಿಸುಖದಾಯಿನೀ ।\\
ಕೃಷ್ಣಮೂರ್ತಿರ್ಮಹಾಮೂರ್ತಿರ್ಘೋರಮೂರ್ತಿರ್ವರಾನನಾ ॥ ೧೦೩॥

ಸರ್ವೇಂದ್ರಿಯಮನೋನ್ಮತ್ತಾ ಸರ್ವೇಂದ್ರಿಯಮನೋಮಯೀ ।\\
ಸರ್ವಸಂಗ್ರಾಮಜಯದಾ ಸರ್ವಪ್ರಹರಣೋದ್ಯತಾ ॥ ೧೦೪॥

ಸರ್ವಪೀಡೋಪಶಮನೀ ಸರ್ವಾರಿಷ್ಟವಿನಾಶಿನೀ ।\\
ಸರ್ವೈಶ್ವರ್ಯಸಮುತ್ಪತ್ತಿಃ ಸರ್ವಗ್ರಹವಿನಾಶಿನೀ ॥ ೧೦೫॥

ಭೀತಿಘ್ನೀ ಭಕ್ತಿಗಮ್ಯಾ ಚ ಭಕ್ತಾನಾಮಾರ್ತಿನಾಶಿನೀ ।\\
ಮಾತಂಗೀ ಮತ್ತಮಾತಂಗೀ ಮಾತಂಗಗಣಮಂಡಿತಾ ॥ ೧೦೬॥

ಅಮೃತೋದಧಿಮಧ್ಯಸ್ಥಾ ಕಟಿಸೂತ್ರೈರಲಂಕೃತಾ ।\\
ಅಮೃತದ್ವೀಪಮಧ್ಯಸ್ಥಾ ಪ್ರಬಲಾ ವತ್ಸಲೋಜ್ಜ್ವಲಾ ॥ ೧೦೭॥

ಮಣಿಮಂಡಪಮಧ್ಯಸ್ಥಾ ರತ್ನಸಿಂಹಾಸನಸ್ಥಿತಾ ।\\
ಪರಮಾನಂದಮುದಿತಾ ಈಷತ್ಪ್ರಹಸಿತಾನನಾ ॥ ೧೦೮॥

ಕುಮುದಾ ಲಲಿತಾ ಲೋಲಾ ಲಾಕ್ಷಾ ಲೋಹಿತಲೋಚನಾ ।\\
ದಿಗ್ವಾಸಾ ದೇವದೂತೀ ಚ ದೇವದೇವಾದಿದೇವತಾ ॥ ೧೦೯॥

ಸಿಂಹೋಪರಿಸಮಾರೂಢಾ ಹಿಮಾಚಲನಿವಾಸಿನೀ ।\\
ಅಟ್ಟಾಟ್ಟಹಾಸಿನೀ ಘೋರಾ ಘೋರದೈತ್ಯವಿನಾಶಿನೀ ॥ ೧೧೦॥

ಅತ್ಯುಗ್ರಾ ರಕ್ತವಸನಾ ನಾಗಕೇಯೂರಮಂಡಿತಾ ।\\
ಮುಕ್ತಾಹಾರಸ್ತನೋಪೇತಾ ತುಂಗಪೀನಪಯೋಧರಾ ॥ ೧೧೧॥

ರಕ್ತೋತ್ಪಲದಲಾಕಾರಾ ಮದಾಘೂರ್ಣಿತಲೋಚನಾ ।\\
ಗಂಡಮಂಡಿತತಾಟಂಕಾ ಗುಂಜಾಹಾರವಿಭೂಷಣಾ ॥ ೧೧೨॥

ಸಂಗೀತರಂಗರಸನಾ ವೀಣಾವಾದ್ಯಕುತೂಹಲಾ ।\\
ಸಮಸ್ತದೇವಮೂರ್ತಿಶ್ಚ ಹ್ಯಸುರಕ್ಷಯಕಾರಿಣೀ ॥ ೧೧೩॥

ಖಡ್ಗಿನೀ ಶೂಲಹಸ್ತಾ ಚ ಚಕ್ರಿಣೀ ಚಾಕ್ಷಮಾಲಿನೀ ।\\
ಪಾಶಿನೀ ಚಕ್ರಿಣೀ ದಾಂತಾ ವಜ್ರಿಣೀ ವಜ್ರದಂಡಿನೀ ॥ ೧೧೪॥

ಆನಂದೋದಧಿಮಧ್ಯಸ್ಥಾ ಕಟಿಸೂತ್ರೈರಲಂಕೃತಾ ।\\
ನಾನಾಭರಣದೀಪ್ತಾಂಗೀ ನಾನಾಮಣಿವಿಭೂಷಣಾ ॥ ೧೧೫॥

ಜಗದಾನಂದಸಂಭೂತಿಶ್ಚಿಂತಾಮಣಿಗುಣಾಕರಾ ।\\
ತ್ರೈಲೋಕ್ಯನಮಿತಾ ಪೂಜ್ಯಾ ಚಿನ್ಮಯಾಽಽನಂದರೂಪಿಣೀ ॥ ೧೧೬॥

ತ್ರೈಲೋಕ್ಯನಂದಿನೀ ದೇವೀ ದುಃಖದುಃಸ್ವಪ್ನನಾಶಿನೀ ।\\
ಘೋರಾಗ್ನಿದಾಹಶಮನೀ ರಾಜದೈವಾದಿಶಾಲಿನೀ ॥ ೧೧೭॥

ಮಹಾಽಪರಾಧರಾಶಿಘ್ನೀ ಮಹಾವೈರಿಭಯಾಪಹಾ ।\\
ರಾಗಾದಿದೋಷರಹಿತಾ ಜರಾಮರಣವರ್ಜಿತಾ ॥ ೧೧೮॥

ಚಂದ್ರಮಂಡಲಮಧ್ಯಸ್ಥಾ ಪೀಯೂಷಾರ್ಣವಸಂಭವಾ ।\\
ಸರ್ವದೇವೈಃ ಸ್ತುತಾ ದೇವೀ ಸರ್ವಸಿದ್ಧಿನಮಸ್ಕೃತಾ ॥ ೧೧೯॥

ಅಚಿಂತ್ಯಶಕ್ತಿರೂಪಾ ಚ ಮಣಿಮಂತ್ರಮಹೌಷಧೀ ।\\
ಸ್ವಸ್ತಿಃ ಸ್ವಸ್ತಿಮತೀ ಬಾಲಾ ಮಲಯಾಚಲಸಂಸ್ಥಿತಾ ॥ ೧೨೦॥

ಧಾತ್ರೀ ವಿಧಾತ್ರೀ ಸಂಹಾರಾ ರತಿಜ್ಞಾ ರತಿದಾಯಿನೀ ।\\
ರುದ್ರಾಣೀ ರುದ್ರರೂಪಾ ಚ ರೌದ್ರೀ ರೌದ್ರಾರ್ತಿಹಾರಿಣೀ ॥ ೧೨೧॥

ಸರ್ವಜ್ಞಾ ಚೌರಧರ್ಮಜ್ಞಾ ರಸಜ್ಞಾ ದೀನವತ್ಸಲಾ ।\\
ಅನಾಹತಾ ತ್ರಿನಯನಾ ನಿರ್ಭರಾ ನಿರ್ವೃತಿಃ ಪರಾ ॥ ೧೨೨॥

ಪರಾ ಘೋರಕರಾಲಾಕ್ಷೀ ಸ್ವಮಾತಾ ಪ್ರಿಯದಾಯಿನೀ ।\\
ಮಂತ್ರಾತ್ಮಿಕಾ ಮಂತ್ರಗಮ್ಯಾ ಮಂತ್ರಮಾತಾ ಸಮಂತ್ರಿಣೀ ॥ ೧೨೩॥

ಶುದ್ಧಾನಂದಾ ಮಹಾಭದ್ರಾ ನಿರ್ದ್ವಂದ್ವಾ ನಿರ್ಗುಣಾತ್ಮಿಕಾ ।\\
ಧರಣೀ ಧಾರಿಣೀ ಪೃಥ್ವೀ ಧರಾ ಧಾತ್ರೀ ವಸುಂಧರಾ ॥ ೧೨೪॥

ಮೇರುಮಂದಿರಮಧ್ಯಸ್ಥಾ ಶಿವಾ ಶಂಕರವಲ್ಲಭಾ ।\\
ಶ್ರೀಗತಿಃ ಶ್ರೀಮತೀ ಶ್ರೇಷ್ಠಾ ಶ್ರೀಕರೀ ಶ್ರೀವಿಭಾವನೀ ॥ ೧೨೫॥

ಶ್ರೀದಾ ಶ್ರೀಮಾ ಶ್ರೀನಿವಾಸಾ ಶ್ರೀಮತೀ ಶ್ರೀಮತಾಂ ಗತಿಃ ।\\
ಉಮಾ ಶಾರಂಗಿಣೀ ಕೃಷ್ಣಾ ಕುಟಿಲಾ ಕುಟಿಲಾಲಕಾ ॥ ೧೨೬॥

ತ್ರಿಲೋಚನಾ ತ್ರಿಲೋಕಾತ್ಮಾ ಪುಣ್ಯದಾ ಪುಣ್ಯಕೀರ್ತಿದಾ ।\\
ಅಮೃತಾ ಸತ್ಯಸಂಕಲ್ಪಾ ಸತ್ಯಾಶಾ ಗ್ರಂಥಿಭೇದಿನೀ ॥ ೧೨೭॥

ಪರೇಶಾ ಪರಮಾ ವಿದ್ಯಾ ಪರಾವಿದ್ಯಾ ಪರಾತ್ಪರಾ ।\\
ಸುಂದರಾಂಗೀ ಸುವರ್ಣಾಭಾ ಸುರಾಸುರನಮಸ್ಕೃತಾ ॥ ೧೨೮॥

ಪ್ರಜಾ ಪ್ರಜಾವತೀ ಧನ್ಯಾ ಧನಧಾನ್ಯಸಮೃದ್ಧಿದಾ ।\\
ಈಶಾನೀ ಭುವನೇಶಾನೀ ಭುವನಾ ಭುವನೇಶ್ವರೀ ॥ ೧೨೯॥

ಅನಂತಾನಂತಮಹಿಮಾ ಜಗತ್ಸಾರಾ ಜಗದ್ಭವಾ ।\\
ಅಚಿಂತ್ಯಶಕ್ತಿಮಹಿಮಾ ಚಿಂತ್ಯಾಚಿಂತ್ಯಸ್ವರೂಪಿಣೀ ॥ ೧೩೦॥

ಜ್ಞಾನಗಮ್ಯಾ ಜ್ಞಾನಮೂರ್ತಿರ್ಜ್ಞಾನದಾ ಜ್ಞಾನಶಾಲಿನೀ ।\\
ಅಮಿತಾ ಘೋರರೂಪಾ ಚ ಸುಧಾಧಾರಾ ಸುಧಾವಹಾ ॥ ೧೩೧॥

ಭಾಸ್ಕರೀ ಭಾಸುರೀ ಭಾತೀ ಭಾಸ್ವದುತ್ತಾನಶಾಯಿನೀ ।\\
ಅನಸೂಯಾ ಕ್ಷಮಾ ಲಜ್ಜಾ ದುರ್ಲಭಾ ಭುವನಾಂತಿಕಾ ॥ ೧೩೨॥

ವಿಶ್ವವಂದ್ಯಾ ವಿಶ್ವಬೀಜಾ ವಿಶ್ವಧೀರ್ವಿಶ್ವಸಂಸ್ಥಿತಾ ।\\
ಶೀಲಸ್ಥಾ ಶೀಲರೂಪಾ ಚ ಶೀಲಾ ಶೀಲಪ್ರದಾಯಿನೀ ॥ ೧೩೩॥

ಬೋಧಿನೀ ಬೋಧಕುಶಲಾ ರೋಧಿನೀ ಬಾಧಿನೀ ತಥಾ ।\\
ವಿದ್ಯೋತಿನೀ ವಿಚಿತ್ರಾತ್ಮಾ ವಿದ್ಯುತ್ಪಟಲಸನ್ನಿಭಾ ॥ ೧೩೪॥

ವಿಶ್ವಯೋನಿರ್ಮಹಾಯೋನಿಃ ಕರ್ಮಯೋನಿಃ ಪ್ರಿಯಂವದಾ ।\\
ರೋಗಿಣೀ ರೋಗಶಮನೀ ಮಹಾರೋಗಭಯಾವಹಾ ॥ ೧೩೫॥

ವರದಾ ಪುಷ್ಟಿದಾ ದೇವೀ ಮಾನದಾ ಮಾನವಪ್ರಿಯಾ ।\\
ಕೃಷ್ಣಾಂಗವಾಹಿನೀ ಚೈವ ಕೃಷ್ಣಾ ಕೃಷ್ಣಸಹೋದರೀ ॥ ೧೩೬॥

ಶಾಂಭವೀ ಶಂಭುರೂಪಾ ಚ ತಥೈವ ಶಂಭುಸಂಭವಾ ।\\
ವಿಶ್ವೋದರೀ ವಿಶ್ವಮಾತಾ ಯೋಗಮುದ್ರಾ ಚ ಯೋಗಿನೀ ॥ ೧೩೭॥

ವಾಗೀಶ್ವರೀ ಯೋಗಮುದ್ರಾ ಯೋಗಿನೀಕೋಟಿಸೇವಿತಾ ।\\
ಕೌಲಿಕಾನಂದಕನ್ಯಾ ಚ ಶೃಂಗಾರಪೀಠವಾಸಿನೀ ॥ ೧೩೮॥

ಕ್ಷೇಮಂಕರೀ ಸರ್ವರೂಪಾ ದಿವ್ಯರೂಪಾ ದಿಗಂಬರಾ ।\\
ಧೂಮ್ರವಕ್ತ್ರಾ ಧೂಮ್ರನೇತ್ರಾ ಧೂಮ್ರಕೇಶೀ ಚ ಧೂಸರಾ ॥ ೧೩೯॥

ಪಿನಾಕೀ ರುದ್ರವೇತಾಲೀ ಮಹಾವೇತಾಲರೂಪಿಣೀ ।\\
ತಪನೀ ತಾಪಿನೀ ದಕ್ಷಾ ವಿಷ್ಣುವಿದ್ಯಾ ತ್ವನಾಥಿತಾ ॥ ೧೪೦॥

ಅಂಕುರಾ ಜಠರಾ ತೀವ್ರಾ ಅಗ್ನಿಜಿಹ್ವಾ ಭಯಾಪಹಾ ।\\
ಪಶುಘ್ನೀ ಪಶುರೂಪಾ ಚ ಪಶುದಾ ಪಶುವಾಹಿನೀ ॥ ೧೪೧॥

ಪಿತಾ ಮಾತಾ ಚ ಭ್ರಾತಾ ಚ ಪಶುಪಾಶವಿನಾಶಿನೀ ।\\
ಚಂದ್ರಮಾ ಚಂದ್ರರೇಖಾ ಚ ಚಂದ್ರಕಾಂತಿವಿಭೂಷಣಾ ॥ ೧೪೨॥

ಕುಂಕುಮಾಂಕಿತಸರ್ವಾಂಗೀ ಸುಧೀರ್ಬುದ್ಬುದಲೋಚನಾ ।\\
ಶುಕ್ಲಾಂಬರಧರಾ ದೇವೀ ವೀಣಾಪುಸ್ತಕಧಾರಿಣೀ ॥ ೧೪೩॥

ಶ್ವೇತವಸ್ತ್ರಧರಾ ದೇವೀ ಶ್ವೇತಪದ್ಮಾಸನಸ್ಥಿತಾ ।\\
ರಕ್ತಾಂಬರಾ ಚ ರಕ್ತಾಂಗೀ ರಕ್ತಪದ್ಮವಿಲೋಚನಾ ॥ ೧೪೪॥

ನಿಷ್ಠುರಾ ಕ್ರೂರಹೃದಯಾ ಅಕ್ರೂರಾ ಮಿತಭಾಷಿಣೀ ।\\
ಆಕಾಶಲಿಂಗಸಂಭೂತಾ ಭುವನೋದ್ಯಾನವಾಸಿನೀ ॥ ೧೪೫॥

ಮಹಾಸೂಕ್ಷ್ಮಾ ಚ ಕಂಕಾಲೀ ಭೀಮರೂಪಾ ಮಹಾಬಲಾ ।\\
ಅನೌಪಮ್ಯಗುಣೋಪೇತಾ ಸದಾ ಮಧುರಭಾಷಿಣೀ ॥ ೧೪೬॥

ವಿರೂಪಾಕ್ಷೀ ಸಹಸ್ರಾಕ್ಷೀ ಶತಾಕ್ಷೀ ಬಹುಲೋಚನಾ ।\\
ದುಸ್ತರೀ ತಾರಿಣೀ ತಾರಾ ತರುಣೀ ತಾರರೂಪಿಣೀ ॥ ೧೪೭॥

ಸುಧಾಧಾರಾ ಚ ಧರ್ಮಜ್ಞಾ ಧರ್ಮಯೋಗೋಪದೇಶಿನೀ ।\\
ಭಗೇಶ್ವರೀ ಭಗಾರಾಧ್ಯಾ ಭಗಿನೀ ಭಗಿನೀಪ್ರಿಯಾ ॥ ೧೪೮॥

ಭಗವಿಶ್ವಾ ಭಗಕ್ಲಿನ್ನಾ ಭಗಯೋನಿರ್ಭಗಪ್ರದಾ ।\\
ಭಗೇಶ್ವರೀ ಭಗರೂಪಾ ಭಗಗುಹ್ಯಾ ಭಗಾವಹಾ ॥ ೧೪೯॥

ಭಗೋದರೀ ಭಗಾನಂದಾ ಭಗಾಢ್ಯಾ ಭಗಮಾಲಿನೀ ।\\
ಸರ್ವಸಂಕ್ಷೋಭಿಣೀಶಕ್ತಿಃ ಸರ್ವವಿದ್ರಾವಿಣೀ ತಥಾ ॥ ೧೫೦॥

ಮಾಲಿನೀ ಮಾಧವೀ ಮಾಧ್ವೀ ಮದರೂಪಾ ಮದೋತ್ಕಟಾ ।\\
ಭೇರುಂಡಾ ಚಂಡಿಕಾ ಜ್ಯೋತ್ಸ್ನಾ ವಿಶ್ವಚಕ್ಷುಸ್ತಪೋವಹಾ ॥ ೧೫೧॥

ಸುಪ್ರಸನ್ನಾ ಮಹಾದೂತೀ ಯಮದೂತೀ ಭಯಂಕರೀ ।\\
ಉನ್ಮಾದಿನೀ ಮಹಾರೂಪಾ ದಿವ್ಯರೂಪಾ ಸುರಾರ್ಚಿತಾ ॥ ೧೫೨॥

ಚೈತನ್ಯರೂಪಿಣೀ ನಿತ್ಯಾ ನಿತ್ಯಕ್ಲಿನ್ನಾ ಮದೋಲ್ಲಸಾ ।\\
ಮದಿರಾನಂದಕೈವಲ್ಯಾ ಮದಿರಾಕ್ಷೀ ಮದಾಲಸಾ ॥ ೧೫೩॥

ಸಿದ್ಧೇಶ್ವರೀ ಸಿದ್ಧವಿದ್ಯಾ ಸಿದ್ಧಾದ್ಯಾ ಸಿದ್ಧವಂದಿತಾ ।\\
ಸಿದ್ಧಾರ್ಚಿತಾ ಸಿದ್ಧಮಾತಾ ಸಿದ್ಧಸರ್ವಾರ್ಥಸಾಧಿಕಾ ॥ ೧೫೪॥

ಮನೋನ್ಮನೀ ಗುಣಾತೀತಾ ಪರಂಜ್ಯೋತಿಃಸ್ವರೂಪಿಣೀ ।\\
ಪರೇಶೀ ಪಾರಗಾ ಪಾರಾ ಪಾರಸಿದ್ಧಿಃ ಪರಾ ಗತಿಃ ॥ ೧೫೫॥

ವಿಮಲಾ ಮೋಹಿನೀರೂಪಾ ಮಧುಪಾನಪರಾಯಣಾ ।\\
ವೇದವೇದಾಂಗಜನನೀ ಸರ್ವಶಾಸ್ತ್ರವಿಶಾರದಾ ॥ ೧೫೬॥

ಸರ್ವವೇದಮಯೀ ವಿದ್ಯಾ ಸರ್ವಶಾಸ್ತ್ರಮಯೀ ತಥಾ ।\\
ಸರ್ವಜ್ಞಾನಮಯೀ ದೇವೀ ಸರ್ವಧರ್ಮಮಯೀಶ್ವರೀ ॥ ೧೫೭॥

ಸರ್ವಯಜ್ಞಮಯೀ ಯಜ್ವಾ ಸರ್ವಮಂತ್ರಾಧಿಕಾರಿಣೀ ।\\
ತ್ರೈಲೋಕ್ಯಾಕರ್ಷಿಣೀ ದೇವೀ ಸರ್ವಾದ್ಯಾನಂದರೂಪಿಣೀ ॥ ೧೫೮॥

ಸರ್ವಸಂಪತ್ತ್ಯಧಿಷ್ಠಾತ್ರೀ ಸರ್ವವಿದ್ರಾವಿಣೀ ಪರಾ ।\\
ಸರ್ವಸಂಕ್ಷೋಭಿಣೀ ದೇವೀ ಸರ್ವಮಂಗಲಕಾರಿಣೀ ॥ ೧೫೯॥

ತ್ರೈಲೋಕ್ಯರಂಜನೀ ದೇವೀ ಸರ್ವಸ್ತಂಭನಕಾರಿಣೀ ।\\
ತ್ರೈಲೋಕ್ಯಜಯಿನೀ ದೇವೀ ಸರ್ವೋನ್ಮಾದಸ್ವರೂಪಿಣೀ ॥ ೧೬೦॥

ಸರ್ವಸಮ್ಮೋಹಿನೀ ದೇವೀ ಸರ್ವವಶ್ಯಂಕರೀ ತಥಾ ।\\
ಸರ್ವಾರ್ಥಸಾಧಿನೀ ದೇವೀ ಸರ್ವಸಂಪತ್ತಿದಾಯಿನೀ ॥ ೧೬೧॥

ಸರ್ವಕಾಮಪ್ರದಾ ದೇವೀ ಸರ್ವಮಂಗಲಕಾರಿಣೀ ।\\
ಸರ್ವಸಿದ್ಧಿಪ್ರದಾ ದೇವೀ ಸರ್ವದುಃಖವಿಮೋಚಿನೀ ॥ ೧೬೨॥

ಸರ್ವಮೃತ್ಯುಪ್ರಶಮನೀ ಸರ್ವವಿಘ್ನವಿನಾಶಿನೀ ।\\
ಸರ್ವಾಂಗಸುಂದರೀ ಮಾತಾ ಸರ್ವಸೌಭಾಗ್ಯದಾಯಿನೀ ॥ ೧೬೩॥

ಸರ್ವದಾ ಸರ್ವಶಕ್ತಿಶ್ಚ ಸರ್ವೈಶ್ವರ್ಯಫಲಪ್ರದಾ ।\\
ಸರ್ವಜ್ಞಾನಮಯೀ ದೇವೀ ಸರ್ವವ್ಯಾಧಿವಿನಾಶಿನೀ ॥ ೧೬೪॥

ಸರ್ವಾಧಾರಾ ಸರ್ವರೂಪಾ ಸರ್ವಪಾಪಹರಾ ತಥಾ ।\\
ಸರ್ವಾನಂದಮಯೀ ದೇವೀ ಸರ್ವರಕ್ಷಾಸ್ವರೂಪಿಣೀ ॥ ೧೬೫॥

ಸರ್ವಲಕ್ಷ್ಮೀಮಯೀ ವಿದ್ಯಾ ಸರ್ವೇಪ್ಸಿತಫಲಪ್ರದಾ ।\\
ಸರ್ವದುಃಖಪ್ರಶಮನೀ ಪರಮಾನಂದದಾಯಿನೀ ॥ ೧೬೬॥

ತ್ರಿಕೋಣನಿಲಯಾ ತ್ರೀಷ್ಟಾ ತ್ರಿಮತಾ ತ್ರಿತನುಸ್ಥಿತಾ ।\\
ತ್ರೈವಿದ್ಯಾ ಚೈವ ತ್ರಿಸ್ಮಾರಾ ತ್ರೈಲೋಕ್ಯತ್ರಿಪುರೇಶ್ವರೀ ॥ ೧೬೭॥

ತ್ರಿಕೋದರಸ್ಥಾ ತ್ರಿವಿಧಾ ತ್ರಿಪುರಾ ತ್ರಿಪುರಾತ್ಮಿಕಾ ।\\
ತ್ರಿಧಾತ್ರೀ ತ್ರಿದಶಾ ತ್ರ್ಯಕ್ಷಾ ತ್ರಿಘ್ನೀ ತ್ರಿಪುರವಾಹಿನೀ ॥ ೧೬೮॥

ತ್ರಿಪುರಾಶ್ರೀಃ ಸ್ವಜನನೀ ಬಾಲಾತ್ರಿಪುರಸುಂದರೀ ।\\
ಶ್ರೀಮತ್ತ್ರಿಪುರಸುಂದರ್ಯಾ ಮಂತ್ರನಾಮಸಹಸ್ರಕಂ ॥ ೧೬೯॥

ಗುಹ್ಯಾದ್ಗುಹ್ಯತರಂ ಪುತ್ರ ! ತವ ಪ್ರೀತ್ಯಾ ಪ್ರಕೀರ್ತಿತಂ ।\\
ಗೋಪನೀಯಂ ಪ್ರಯತ್ನೇನೇ ಪಠನೀಯಂ ಪ್ರಯತ್ನತಃ ॥ ೧೭೦॥

ನಾತಃ ಪರತರಂ ಪುಣ್ಯಂ ನಾತಃ ಪರತರಂ ಶುಭಂ ।\\
ನಾತಃ ಪರತರಂ ಸ್ತೋತ್ರಂ ನಾತಃ ಪರತರಾ ಗತಿಃ ॥ ೧೭೧॥

ಸ್ತೋತ್ರಂ ಸಹಸ್ರನಾಮಾಖ್ಯಂ ಮಮ ವಕ್ತ್ರಾದ್ವಿನಿಸ್ಸೃತಂ ।\\
ಯಃ ಪಠೇತ್ಪರಯಾ ಭಕ್ತ್ಯಾ ಶೃಣುಯಾದ್ವಾ ಸಮಾಹಿತಃ ॥ ೧೭೨॥

ಮೋಕ್ಷಾರ್ಥೀ ಲಭತೇ ಮೋಕ್ಷಂ ಸುಖಾರ್ಥೀ ಸುಖಮಾಪ್ನುಯಾತ್ ।\\
ಫಲಾರ್ಥೀ ಲಭತೇ ಕಾಮಾನ್ ಧನಾರ್ಥೀ ಲಭತೇ ಧನಂ ॥ ೧೭೩॥

ವಿದ್ಯಾರ್ಥೀ ಲಭತೇ ವಿದ್ಯಾಂ ಯಶೋಽರ್ಥೀ ಲಭತೇ ಯಶಃ ।\\
ಕನ್ಯಾರ್ಥೀ ಲಭತೇ ಕನ್ಯಾಂ ಸುತಾರ್ಥೀ ಲಭತೇ ಸುತಂ ॥ ೧೭೪॥

ಗುರ್ವಿಣೀ ಲಭತೇ ಪುತ್ರಂ ಕನ್ಯಾ ವಿಂದತಿ ಸತ್ಪತಿಂ ।\\
ಮೂರ್ಖೇಽಪಿ ಲಭತೇ ಶಾಸ್ತ್ರಂ ಚೌರೋಽಪಿ ಲಭತೇ ಗತಿಂ ॥ ೧೭೫॥

ಸಂಕ್ರಾಂತಾವಮಾವಾಸ್ಯಾಯಾಮಷ್ಟಮ್ಯಾಂ ಭೌಮವಾಸರೇ ।\\
ಪಠೇದ್ವಾ ಪಾಠಯೇದ್ವಾಪಿ ಶೃಣುಯಾದ್ವಾ ಸಮಾಹಿತಃ ॥ ೧೭೬॥

ಪೌರ್ಣಮಾಸ್ಯಾಂ ಚತುರ್ದಶ್ಯಾಂ ನವಮ್ಯಾಂ ಚ ವಿಶೇಷತಃ ।\\
ಸ ಮುಕ್ತಃ ಸರ್ವಪಾಪೇಭ್ಯಃ ಕಾಮೇಶ್ವರಸಮೋ ಭವೇತ್ ॥ ೧೭೭॥

ಲಕ್ಷ್ಮೀವಾನ್ ಸುತವಾಂಶ್ಚೈವ ವಲ್ಲಭಃ ಸರ್ವಯೋಷಿತಾಂ ।\\
ತಸ್ಯಾ ವಶ್ಯಂ ಭವೇದ್ದಾಸ್ಯೇ ತ್ರೈಲೋಕ್ಯಂ ಸಚರಾಚರಂ ॥ ೧೭೮॥

ರುದ್ರಂ ದೃಷ್ಟ್ವಾ ಯಥಾ ದೇವಾ ವಿಷ್ಣುಂ ದೃಷ್ಟ್ವಾ ಚ ದಾನವಾಃ ।\\
ಪನ್ನಗಾ ಗರುಡಂ ದೃಷ್ಟ್ವಾ ಸಿಂಹಂ ದೃಷ್ಟ್ವಾ ಯಥಾ ಮೃಗಾಃ ॥ ೧೭೯॥

ಮಂಡೂಕಾ ಭೋಗಿನಂ ದೃಷ್ಟ್ವಾ ಮಾರ್ಜಾರಂ ಮೂಷಕೋ ಯಥಾ ।\\
ಕೀಟವತ್ಪ್ರಪಲಾಯಂತೇ ತಸ್ಯ ವಕ್ತ್ರಾವಲೋಕನಾತ್ ॥ ೧೮೦॥

ಅಗ್ನಿಚೌರಭಯಂ ತಸ್ಯ ಕದಾಚಿನ್ನೈವ ಸಂಭವೇತ್ ।\\
ಪಾತಕಾ ವಿವಿಧಾಃ ಸಂತಿ ಮೇರುಮಂದರಸನ್ನಿಭಾಃ ॥ ೧೮೧॥

ಭಸ್ಮಸಾತ್ತತ್ಕ್ಷಣಂ ಕುರ್ಯಾತ್ ತೃಣಂ ವಹ್ನಿಯುತಂ ಯಥಾ ।\\
ಏಕಧಾ ಪಠನಾದೇವ ಸರ್ಪಪಾಪಕ್ಷಯೋ ಭವೇತ್ ॥ ೧೮೨॥

ದಶಧಾ ಪಠನಾದೇವ ವಾಂಛಾಸಿದ್ಧಿಃ ಪ್ರಜಾಯತೇ ।\\
ನಶ್ಯಂತಿ ಸಹಸಾ ರೋಗಾ ದಶಧಾಽಽವರ್ತನೇನ ಚ ॥ ೧೮೩॥

ಸಹಸ್ರಂ ವಾ ಪಠೇದ್ಯಸ್ತು ಖೇಚರೋ ಜಾಯತೇ ನರಃ ।\\
ಸಹಸ್ರದಶಕಂ ಯಸ್ತು ಪಠೇದ್ಭಕ್ತಿಪರಾಯಣಃ ॥ ೧೮೪॥

ಸಾ ತಸ್ಯ ಜಗತಾಂ ಧಾತ್ರೀ ಪ್ರತ್ಯಕ್ಷಾ ಭವತಿ ಧ್ರುವಂ ।\\
ಲಕ್ಷಂ ಪೂರ್ಣಂ ಯದಾ ಪುತ್ರ ! ಸ್ತವರಾಜಂ ಪಠೇತ್ಸುಧೀಃ ॥ ೧೮೫॥

ಭವಪಾಶವಿನಿರ್ಮುಕ್ತೋ ಮಮ ತುಲ್ಯೋ ನ ಸಂಶಯಃ ।\\
ಸರ್ವತೀರ್ಥೇಷು ಯತ್ಪುಣ್ಯಂ ಸರ್ವಯಜ್ಞೇಷು ಯತ್ಫಲಂ ॥ ೧೮೬॥

ಸರ್ವವೇದೇಷು ಯತ್ಪುಣ್ಯಂ ತತ್ಫಲಂ ಪರಿಕಿರ್ತಿತಂ ।\\
ತತ್ಫಲಂ ಕೋಟಿಗುಣಿತಂ ಸಕೃಜ್ಜಪ್ತ್ವಾ ಲಭೇನ್ನರಃ ॥ ೧೮೭॥

ಶ್ರುತ್ವಾ ಮಹಾಬಲಶ್ಚಾಶು ಪುತ್ರವಾನ್ ಸರ್ವಸಂಪದಃ ।\\
ದೇಹಾಂತೇ ಪರಮಂ ಸ್ಥಾನಂ ಯತ್ಸುರೈರಪಿ ದುರ್ಲಭಂ ॥ ೧೮೮॥

ಅದ್ವೈತಯೋಗಿಭಿರ್ಜ್ಞೇಯಂ ಮಾರ್ಗಗೈರಪಿ ದುರ್ಲಭಂ ।\\
ಸ ಯಾಸ್ಯತಿ ನ ಸಂದೇಹಃ ಸ್ತವರಾಜಪ್ರಕೀರ್ತನಾತ್ ॥ ೧೮೯॥

ಯಃ ಸದಾ ಪಠತೇ ಭಕ್ತೋ ಮುಕ್ತಿಸ್ತಸ್ಯ ನ ಸಂಶಯಃ ॥ ೧೯೦॥

\authorline{ಇತಿ ಶ್ರೀವಾಮಕೇಶ್ವರತಂತ್ರಾಂತರ್ಗತಂ ಶ್ರೀಬಾಲಾಸಹಸ್ರನಾಮಸ್ತೋತ್ರಂ ಸಂಪೂರ್ಣಂ ।}
%================================================================================================
\section{ಶ್ರೀಬಾಲಾಷ್ಟೋತ್ತರಶತನಾಮಸ್ತೋತ್ರಂ}
\addcontentsline{toc}{section}{ಶ್ರೀಬಾಲಾಷ್ಟೋತ್ತರಶತನಾಮಸ್ತೋತ್ರಂ}
ಅಸ್ಯ ಶ್ರೀಬಾಲಾಷ್ಟೋತ್ತರ ಶತನಾಮಸ್ತೋತ್ರ ಮಹಾಮಂತ್ರಸ್ಯ  ದಕ್ಷಿಣಾಮೂರ್ತಿಃ ಋಷಿಃ~। ಅನುಷ್ಟುಪ್ ಛಂದಃ~। ಬಾಲಾತ್ರಿಪುರಸುಂದರೀ ದೇವತಾ~। ಐಂ ಬೀಜಂ~। ಕ್ಲೀಂ ಶಕ್ತಿಃ~। ಸೌಃ ಕೀಲಕಂ~।  ಶ್ರೀಬಾಲಾಪ್ರಿತ್ಯರ್ಥೇ ನಾಮಪಾರಾಯಣೇ ವಿನಿಯೋಗಃ ॥

\dhyana{ಪಾಶಾಂಕುಶೇ ಪುಸ್ತಕಾಕ್ಷಸೂತ್ರೇ ಚ ದಧತೀ ಕರೈಃ~।\\
ರಕ್ತಾ ತ್ರ್ಯಕ್ಷಾ ಚಂದ್ರಫಾಲಾ ಪಾತು ಬಾಲಾ ಸುರಾರ್ಚಿತಾ ॥}

ಕಲ್ಯಾಣೀ ತ್ರಿಪುರಾ ಬಾಲಾ ಮಾಯಾ ತ್ರಿಪುರಸುಂದರೀ~।\\
ಸುಂದರೀ ಸೌಭಾಗ್ಯವತೀ ಕ್ಲೀಂಕಾರೀ ಸರ್ವಮಂಗಲಾ ॥೧॥

ಹ್ರೀಂಕಾರೀ ಸ್ಕಂದಜನನೀ ಪರಾ ಪಂಚದಶಾಕ್ಷರೀ~।\\
ತ್ರಿಲೋಕೀ ಮೋಹನಾಧೀಶಾ ಸರ್ವೇಶೀ ಸರ್ವರೂಪಿಣೀ ॥೨॥

ಸರ್ವಸಂಕ್ಷೋಭಿಣೀ ಪೂರ್ಣಾ ನವಮುದ್ರೇಶ್ವರೀ ಶಿವಾ~।\\
ಅನಂಗಕುಸುಮಾ ಖ್ಯಾತಾ ಅನಂಗಾ ಭುವನೇಶ್ವರೀ ॥೩॥

ಜಪ್ಯಾ ಸ್ತವ್ಯಾ ಶ್ರುತಿರ್ನಿತಾ ನಿತ್ಯಕ್ಲಿನ್ನಾಽಮೃತೋದ್ಭವಾ~।\\
ಮೋಹಿನೀ ಪರಮಾಽಽನಂದಾ ಕಾಮೇಶತರುಣಾ ಕಲಾ ॥೪॥

ಕಲಾವತೀ ಭಗವತೀ ಪದ್ಮರಾಗಕಿರೀಟಿನೀ~।\\
ಸೌಗಂಧಿನೀ ಸರಿದ್ವೇಣೀ ಮಂತ್ರಿಣೀ ಮಂತ್ರರೂಪಿಣೀ ॥೫॥

ತತ್ತ್ವತ್ರಯೀ ತತ್ತ್ವಮಯೀ ಸಿದ್ಧಾ ತ್ರಿಪುರವಾಸಿನೀ~।\\
ಶ್ರೀರ್ಮತಿಶ್ಚ ಮಹಾದೇವೀ ಕೌಲಿನೀ ಪರದೇವತಾ ॥೬॥

ಕೈವಲ್ಯರೇಖಾ ವಶಿನೀ ಸರ್ವೇಶೀ ಸರ್ವಮಾತೃಕಾ~।\\
ವಿಷ್ಣುಸ್ವಸಾ ದೇವಮಾತಾ ಸರ್ವಸಂಪತ್ಪ್ರದಾಯಿನೀ ॥೭॥

ಕಿಂಕರೀ ಮಾತಾ ಗೀರ್ವಾಣೀ ಸುರಾಪಾನಾನುಮೋದಿನೀ~।\\
ಆಧಾರಾಹಿತಪತ್ನೀಕಾ ಸ್ವಾಧಿಷ್ಠಾನಸಮಾಶ್ರಯಾ ॥೮॥

ಅನಾಹತಾಬ್ಜನಿಲಯಾ ಮಣಿಪೂರಸಮಾಶ್ರಯಾ~।\\
ಆಜ್ಞಾ ಪದ್ಮಾಸನಾಸೀನಾ ವಿಶುದ್ಧಸ್ಥಲಸಂಸ್ಥಿತಾ ॥೯॥

ಅಷ್ಟಾತ್ರಿಂಶತ್ಕಲಾಮೂರ್ತಿ ಸ್ಸುಷುಮ್ನಾ ಚಾರುಮಧ್ಯಮಾ~।\\
ಯೋಗೇಶ್ವರೀ ಮುನಿಧ್ಯೇಯಾ ಪರಬ್ರಹ್ಮಸ್ವರೂಪಿಣೀ ॥೧೦॥

ಚತುರ್ಭುಜಾ ಚಂದ್ರಚೂಡಾ ಪುರಾಣಾಗಮರೂಪಿಣೀ~।\\
ಐಂಕಾರಾದಿರ್ಮಹಾವಿದ್ಯಾ ಪಂಚಪ್ರಣವರೂಪಿಣೀ ॥೧೧॥

ಭೂತೇಶ್ವರೀ ಭೂತಮಯೀ ಪಂಚಾಶದ್ವರ್ಣರೂಪಿಣೀ~।\\
ಷೋಢಾನ್ಯಾಸ ಮಹಾಭೂಷಾ ಕಾಮಾಕ್ಷೀ ದಶಮಾತೃಕಾ ॥೧೨॥

ಆಧಾರಶಕ್ತಿಃ ತರುಣೀ ಲಕ್ಷ್ಮೀಃ ತ್ರಿಪುರಭೈರವೀ~।\\
ಶಾಂಭವೀ ಸಚ್ಚಿದಾನಂದಾ ಸಚ್ಚಿದಾನಂದರೂಪಿಣೀ ॥೧೩॥

ಮಾಂಗಲ್ಯ ದಾಯಿನೀ ಮಾನ್ಯಾ ಸರ್ವಮಂಗಲಕಾರಿಣೀ~।\\
ಯೋಗಲಕ್ಷ್ಮೀಃ ಭೋಗಲಕ್ಷ್ಮೀಃ ರಾಜ್ಯಲಕ್ಷ್ಮೀಃ ತ್ರಿಕೋಣಗಾ ॥೧೪॥

ಸರ್ವಸೌಭಾಗ್ಯಸಂಪನ್ನಾ ಸರ್ವಸಂಪತ್ತಿದಾಯಿನೀ~।\\
ನವಕೋಣಪುರಾವಾಸಾ ಬಿಂದುತ್ರಯಸಮನ್ವಿತಾ ॥೧೫॥

ನಾಮ್ನಾಮಷ್ಟೋತ್ತರಶತಂ ಪಠೇನ್ನ್ಯಾಸಸಮನ್ವಿತಂ~।\\
ಸರ್ವಸಿದ್ಧಿಮವಾಪ್ನೋತಿ ಸಾಧಕೋಽಭೀಷ್ಟಮಾಪ್ನುಯಾತ್ ॥೧೬॥
\authorline{ಇತಿ ಶ್ರೀ ರುದ್ರಯಾಮಲತಂತ್ರೇ ಉಮಾಮಹೇಶ್ವರಸಂವಾದೇ\\ ಶ್ರೀ ಬಾಲಾ ಅಷ್ಟೋತ್ತರ ಶತನಾಮಸ್ತೋತ್ರಂ ಸಂಪೂರ್ಣಂ ॥}

%===================================================================================================
\section{ಶ್ರೀರಾಮಕೃತಂ ಸೀತಾಸಹಸ್ರನಾಮಸ್ತೋತ್ರಂ }
\addcontentsline{toc}{section}{ಸೀತಾಸಹಸ್ರನಾಮಸ್ತೋತ್ರಂ }
ಬ್ರಹ್ಮಣೋ ವಚನಂ ಶ್ರುತ್ವಾ ರಾಮಃ ಕಮಲಲೋಚನಃ ।\\
ಪ್ರೋನ್ಮೀಲ್ಯ ಶನಕೈರಕ್ಷೀ ವೇಪಮಾನೋ ಮಹಾಭುಜಃ ॥೧॥

ಪ್ರಣಮ್ಯ ಶಿರಸಾ ಭೂಮೌ ತೇಜಸಾ ಚಾಪಿ ವಿಹ್ವಲಃ ।\\
ಭೀತಃ ಕೃತಾಂಜಲಿಪುಟಃ ಪ್ರೋವಾಚ ಪರಮೇಶ್ವರೀಂ ॥೨॥

ಕಾ ತ್ವಂ ದೇವಿ ವಿಶಾಲಾಕ್ಷಿ ಶಶಾಂಕಾವಯವಾಂಕಿತೇ ।\\
ನ ಜಾನೇ ತ್ವಾಂ ಮಹಾದೇವಿ ಯಥಾವದ್ಬ್ರೂಹಿ ಪೃಚ್ಛತೇ ॥೩॥

ರಾಮಸ್ಯ ವಚನಂ ಶ್ರುತ್ವಾ ತತಃ ಸಾ ಪರಮೇಶ್ವರೀ ।\\
ವ್ಯಾಜಹಾರ ರಘುವ್ಯಾಘ್ರಂ ಯೋಗಿನಾಮಭಯಪ್ರದಾ ॥೪॥

ಮಾಂ ವಿದ್ಧಿ ಪರಮಾಂ ಶಕ್ತಿಂ ಮಹೇಶ್ವರಸಮಾಶ್ರಯಾಂ ।\\
ಅನನ್ಯಾಮವ್ಯಯಾಮೇಕಾಂ ಯಾಂ ಪಶ್ಯಂತಿ ಮುಮುಕ್ಷವಃ ॥೫॥

ಅಹಂ ವೈ ಸರ್ವಭಾವಾನಾಮಾತ್ಮಾ ಸರ್ವಾಂತರಾ ಶಿವಾ ।\\
ಶಾಶ್ವತೀ ಸರ್ವವಿಜ್ಞಾನಾ ಸರ್ವಮೂರ್ತಿಪ್ರವರ್ತಿಕಾ ॥೬॥

ಅನಂತಾನಂತಮಹಿಮಾ ಸಂಸಾರಾರ್ಣವತಾರಿಣೀ ।\\
ದಿವ್ಯಂ ದದಾಮಿ ತೇ ಚಕ್ಷುಃ ಪಶ್ಯ ಮೇ ಪದಮೈಶ್ವರಂ ॥೭॥

ಇತ್ಯುಕ್ತ್ವಾ ವಿರರಾಮೈಷಾ ರಾಮೋಽಪಶ್ಯಚ್ಚ ತತ್ಪದಂ ।\\
ಕೋಟಿಸೂರ್ಯಪ್ರತೀಕಾಶಂ ವಿಶ್ವಕ್ತೇಜೋನಿರಾಕುಲಂ ॥೮॥

ಜ್ವಾಲಾವಲೀಸಹಸ್ರಾಢ್ಯಂ ಕಾಲಾನಲಶತೋಪಮಂ ।\\
ದಂಷ್ಟ್ರಾಕರಾಲಂ ದುರ್ಧರ್ಷಂ ಜಟಾಮಂಡಲಮಂಡಿತಂ ॥೯॥

ತ್ರಿಶೂಲವರಹಸ್ತಂ ಚ ಘೋರರೂಪಂ ಭಯಾವಹಂ ।\\
ಪ್ರಶಾಮ್ಯತ್ಸೌಮ್ಯವದನಮನಂತೈಶ್ವರ್ಯಸಂಯುತಂ ॥೧೦॥

ಚಂದ್ರಾವಯವಲಕ್ಷ್ಮಾಢ್ಯಂ ಚಂದ್ರಕೋಟಿಸಮಪ್ರಭಂ ।\\
ಕಿರೀಟಿನಂ ಗದಾಹಸ್ತಂ ನೂಪುರೈರುಪಶೋಭಿತಂ ॥೧೧॥

ದಿವ್ಯಮಾಲ್ಯಾಂಬರಧರಂ ದಿವ್ಯಗಂಧಾನುಲೇಪನಂ ।\\
ಶಂಖಚಕ್ರಕರಂ ಕಾಮ್ಯಂ ತ್ರಿನೇತ್ರಂ ಕೃತ್ತಿವಾಸಸಂ ॥೧೨॥

ಅಂತಃಸ್ಥಂ ಚಾಂಡಬಾಹ್ಯಸ್ಥಂ ಬಾಹ್ಯಾಭ್ಯಂತರತಃಪರಂ ।\\
ಸರ್ವಶಕ್ತಿಮಯಂ ಶಾಂತಂ ಸರ್ವಾಕಾರಂ ಸನಾತನಂ ॥೧೩॥

ಬ್ರಹ್ಮೇಂದ್ರೋಪೇಂದ್ರಯೋಗೀಂದ್ರೈರೀಡ್ಯಮಾನಪದಾಂಬುಜಂ ।\\
ಸರ್ವತಃ ಪಾಣಿಪಾದಂ ತತ್ಸರ್ವತೋಽಕ್ಷಿಶಿರೋಮುಖಂ ॥೧೪॥

ಸರ್ವಮಾವೃತ್ಯ ತಿಷ್ಠಂತಂ ದದರ್ಶ ಪದಮೈಶ್ವರಂ ।\\
ದೃಷ್ಟ್ವಾ ಚ ತಾದೃಶಂ ರೂಪಂ ದಿವ್ಯಂ ಮಾಹೇಶ್ವರಂ ಪದಂ ॥೧೫॥

ತಥೈವ ಚ ಸಮಾವಿಷ್ಟಃ ಸ ರಾಮೋ ಹೃತಮಾನಸಃ ।\\
ಆತ್ಮನ್ಯಾಧಾಯ ಚಾತ್ಮಾನಮೋಂಕಾರಂ ಸಮನುಸ್ಮರನ್ ॥೧೬॥

ನಾಮ್ನಾಮಷ್ಟಸಹಸ್ರೇಣ ತುಷ್ಟಾವ ಪರಮೇಶ್ವರೀಂ ।\\

ಓಂ ಸೀತೋಮಾ ಪರಮಾ ಶಕ್ತಿರನಂತಾ ನಿಷ್ಕಲಾಮಲಾ ॥೧೭॥

ಶಾಂತಾ ಮಾಹೇಶ್ವರೀ ಚೈವ ಶಾಶ್ವತೀ ೧೦ ಪರಮಾಕ್ಷರಾ ।\\
ಅಚಿಂತ್ಯಾ ಕೇವಲಾನಂತಾ ಶಿವಾತ್ಮಾ ಪರಮಾತ್ಮಿಕಾ ॥೧೮॥

ಅನಾದಿರವ್ಯಯಾ ಶುದ್ಧಾ ದೇವಾತ್ಮಾ ೨೦ ಸರ್ವಗೋಚರಾ ।\\
ಏಕಾನೇಕವಿಭಾಗಸ್ಥಾ ಮಾಯಾತೀತಾ ಸುನಿರ್ಮಲಾ ॥೧೯॥

ಮಹಾಮಾಹೇಶ್ವರೀ ಶಕ್ತಾ ಮಹಾದೇವೀ ನಿರಂಜನಾ ।\\
ಕಾಷ್ಠಾ ೩೦ ಸರ್ವಾಂತರಸ್ಥಾ ಚ ಚಿಚ್ಛಕ್ತಿರತಿಲಾಲಸಾ ॥೨೦॥

ಜಾನಕೀ ಮಿಥಿಲಾನಂದಾ ರಾಕ್ಷಸಾಂತವಿಧಾಯಿನೀ ।\\
ರಾವಣಾಂತರಕರೀ ರಮ್ಯಾ ರಾಮವಕ್ಷಃಸ್ಥಲಾಲಯಾ ॥೨೧॥

ಉಮಾ ಸರ್ವಾತ್ಮಿಕಾ ೪೦ ವಿದ್ಯಾ ಜ್ಯೋತಿರೂಪಾಯುತಾಕ್ಷರಾ ।\\
ಶಾಂತಿಃ ಪ್ರತಿಷ್ಠಾ ಸರ್ವೇಷಾಂ ನಿವೃತ್ತಿರಮೃತಪ್ರದಾ ॥೨೨॥

ವ್ಯೋಮಮೂರ್ತಿರ್ವ್ಯೋಮಮಯೀ ವ್ಯೋಮಧಾರಾಽಚ್ಯುತಾ ೫೧ ಲತಾ ।\\
ಅನಾದಿನಿಧನಾ ಯೋಷಾ ಕಾರಣಾತ್ಮಾ ಕಲಾಕುಲಾ ॥೨೩॥

ನಂದಪ್ರಥಮಜಾ ನಾಭಿರಮೃತಸ್ಯಾಂತಸಂಶ್ರಯಾ ।\\
ಪ್ರಾಣೇಶ್ವರಪ್ರಿಯಾ ೬೦ ಮಾತಾಮಹೀ ಮಹಿಷವಾಹನಾ ॥೨೪॥

ಪ್ರಾಣೇಶ್ವರೀ ಪ್ರಾಣರೂಪಾ ಪ್ರಧಾನಪುರುಷೇಶ್ವರೀ ।\\
ಸರ್ವಶಕ್ತಿಃ ಕಲಾ ಕಾಷ್ಠಾ ಜ್ಯೋತ್ಸ್ನೇಂದೋರ್ಮಹಿಮಾಽಽಸ್ಪದಾ ॥೨೫॥೭೨
ಸರ್ವಕಾರ್ಯನಿಯಂತ್ರೀ ಚ ಸರ್ವಭೂತೇಶ್ವರೇಶ್ವರೀ ।\\
ಅನಾದಿರವ್ಯಕ್ತಗುಣಾ ಮಹಾನಂದಾ ಸನಾತನೀ ॥೨೬॥

ಆಕಾಶಯೋನಿರ್ಯೋಗಸ್ಥಾ ಸರ್ವಯೋಗೇಶ್ವರೇಶ್ವರೀ ೮೦ ।\\
ಶವಾಸನಾ ಚಿತಾಂತಃಸ್ಥಾ ಮಹೇಶೀ ವೃಷವಾಹನಾ ॥೨೭॥

ಬಾಲಿಕಾ ತರುಣೀ ವೃದ್ಧಾ ವೃದ್ಧಮಾತಾ ಜರಾತುರಾ ।\\
ಮಹಾಮಾಯಾ ೬೦ ಸುದುಷ್ಪೂರಾ ಮೂಲಪ್ರಕೃತಿರೀಶ್ವರೀ ॥೨೮॥

ಸಂಸಾರಯೋನಿಃ ಸಕಲಾ ಸರ್ವಶಕ್ತಿಸಮುದ್ಭವಾ ।\\
ಸಂಸಾರಸಾರಾ ದುರ್ವಾರಾ ದುರ್ನಿರೀಕ್ಷ್ಯಾ ದುರಾಸದಾ ೧೦೦ ॥೨೯॥

ಪ್ರಾಣಶಕ್ತಿಃ ಪ್ರಾಣವಿದ್ಯಾ ಯೋಗಿನೀ ಪರಮಾ ಕಲಾ ।\\
ಮಹಾವಿಭೂತಿರ್ದುರ್ಧರ್ಷಾ ಮೂಲಪ್ರಕೃತಿಸಂಭವಾ ॥೩೦॥

ಅನಾದ್ಯನಂತವಿಭವಾ ಪರಾತ್ಮಾ ಪುರುಷೋ ಬಲೀ ೧೧೦ ।\\
ಸರ್ಗಸ್ಥಿತ್ಯಂತಕರಣೀ ಸುದುರ್ವಾಚ್ಯಾ ದುರತ್ಯಯಾ ॥೩೧॥

ಶಬ್ದಯೋನಿಶ್ಶಬ್ದಮಯೀ ನಾದಾಖ್ಯಾ ನಾದವಿಗ್ರಹಾ ।\\
ಪ್ರಧಾನಪುರುಷಾತೀತಾ ಪ್ರಧಾನಪುರುಷಾತ್ಮಿಕಾ ॥೩೨॥

ಪುರಾಣೀ ೧೨೦ ಚಿನ್ಮಯೀ ಪುಂಸಾಮಾದಿಃ ಪುರುಷರೂಪಿಣೀ ।\\
ಭೂತಾಂತರಾತ್ಮಾ ಕೂಟಸ್ಥಾ ಮಹಾಪುರುಷಸಂಜ್ಞಿತಾ ॥೩೩॥

ಜನ್ಮಮೃತ್ಯುಜರಾತೀತಾ ಸರ್ವಶಕ್ತಿಸಮನ್ವಿತಾ ।\\
ವ್ಯಾಪಿನೀ ಚಾನವಚ್ಛಿನ್ನಾ ೧೩೦ ಪ್ರಧಾನಾ ಸುಪ್ರವೇಶಿನೀ ॥೩೪॥

ಕ್ಷೇತ್ರಜ್ಞಾ ಶಕ್ತಿರವ್ಯಕ್ತಲಕ್ಷಣಾ ಮಲವರ್ಜಿತಾ ।\\
ಅನಾದಿಮಾಯಾಸಂಭಿನ್ನಾ ತ್ರಿತತ್ತ್ವಾ ಪ್ರಕೃತಿರ್ಗುಣಃ ೧೪೦ ॥೩೫॥

ಮಹಾಮಾಯಾ ಸಮುತ್ಪನ್ನಾ ತಾಮಸೀ ಪೌರುಷಂ ಧ್ರುವಾ ।\\
ವ್ಯಕ್ತಾವ್ಯಕ್ತಾತ್ಮಿಕಾ ಕೃಷ್ಣಾ ರಕ್ತಶುಕ್ಲಾಪ್ರಸೂತಿಕಾ ॥೩೬॥

ಸ್ವಕಾರ್ಯಾ ೧೫೦ ಕಾರ್ಯಜನನೀ ಬ್ರಹ್ಮಾಸ್ಯಾ ಬ್ರಹ್ಮಸಂಶ್ರಯಾ ।\\
ವ್ಯಕ್ತಾ ಪ್ರಥಮಜಾ ಬ್ರಾಹ್ಮೀ ಮಹತೀ ಜ್ಞಾನರೂಪಿಣೀ ॥೩೭॥

ವೈರಾಗ್ಯೈಶ್ವರ್ಯಧರ್ಮಾತ್ಮಾ ಬ್ರಹ್ಮಮೂರ್ತಿರ್ಹೃದಿಸ್ಥಿತಾ । ೧೬೧
ಜಯದಾ ಜಿತ್ವರೀ ಜೈತ್ರೀ ಜಯಶ್ರೀರ್ಜಯಶಾಲಿನೀ ॥೩೮॥

ಸುಖದಾ ಶುಭದಾ ಸತ್ಯಾ ಶುಭಾ ೧೭೦ ಸಂಕ್ಷೋಭಕಾರಿಣೀ ।\\
ಅಪಾಂ ಯೋನಿಃ ಸ್ವಯಂಭೂತಿರ್ಮಾನಸೀ ತತ್ತ್ವಸಂಭವಾ ॥೩೯॥

ಈಶ್ವರಾಣೀ ಚ ಸರ್ವಾಣೀ ಶಂಕರಾರ್ದ್ಧಶರೀರಿಣೀ ।\\
ಭವಾನೀ ಚೈವ ರುದ್ರಾಣೀ ೧೮೦ ಮಹಾಲಕ್ಷ್ಮೀರಥಾಂಬಿಕಾ ॥೪೦॥

ಮಾಹೇಶ್ವರೀ ಸಮುತ್ಪನ್ನಾ ಭುಕ್ತಿಮುಕ್ತಿಫಲಪ್ರದಾ ।\\
ಸರ್ವೇಶ್ವರೀ ಸರ್ವವರ್ಣಾ ನಿತ್ಯಾ ಮುದಿತಮಾನಸಾ ॥೪೧॥

ಬ್ರಹ್ಮೇಂದ್ರೋಪೇಂದ್ರನಮಿತಾ ಶಂಕರೇಚ್ಛಾನುವರ್ತಿನೀ ೧೯೦ ।\\
ಈಶ್ವರಾರ್ದ್ಧಾಸನಗತಾ ರಘೂತ್ತಮಪತಿವ್ರತಾ ॥೪೨॥

ಸಕೃದ್ವಿಭಾವಿತಾ ಸರ್ವಾ ಸಮುದ್ರಪರಿಶೋಷಿಣೀ ।\\
ಪಾರ್ವತೀ ಹಿಮವತ್ಪುತ್ರೀ ಪರಮಾನಂದದಾಯಿನೀ ॥೪೩॥

ಗುಣಾಢ್ಯಾ ಯೋಗದಾ ೨೦೦ ಯೋಗ್ಯಾ ಜ್ಞಾನಮೂರ್ತಿರ್ವಿಕಾಸಿನೀ ।\\
ಸಾವಿತ್ರೀ ಕಮಲಾ ಲಕ್ಷ್ಮೀ ಶ್ರೀರನಂತೋರಸಿ ಸ್ಥಿತಾ ॥೪೪॥

ಸರೋಜನಿಲಯಾ ಶುಭ್ರಾ ಯೋಗನಿದ್ರಾ ೨೧೦ ಸುದರ್ಶನಾ ।\\
ಸರಸ್ವತೀ ಸರ್ವವಿದ್ಯಾ ಜಗಜ್ಜ್ಯೇಷ್ಠಾ ಸುಮಂಗಲಾ ॥೪೫॥

ವಾಸವೀ ವರದಾ ವಾಚ್ಯಾ ಕೀರ್ತಿಃ ಸರ್ವಾರ್ಥಸಾಧಿಕಾ ೨೨೦ ।\\
ವಾಗೀಶ್ವರೀ ಸರ್ವವಿದ್ಯಾ ಮಹಾವಿದ್ಯಾ ಸುಶೋಭನಾ ॥೪೬॥

ಗುಹ್ಯವಿದ್ಯಾಽಽತ್ಮವಿದ್ಯಾ ಚ ಸರ್ವವಿದ್ಯಾಽಽತ್ಮಭಾವಿತಾ ।\\
ಸ್ವಾಹಾ ವಿಶ್ವಂಭರೀ ೨೩೦ ಸಿದ್ಧಿಃ ಸ್ವಧಾ ಮೇಧಾ ಧೃತಿಃ ಶ್ರುತಿಃ ॥೪೭॥

ನಾಭಿಃ ಸುನಾಭಿಃ ಸುಕೃತಿರ್ಮಾಧವೀ ನರವಾಹಿನೀ ೨೪೦ ।\\
ಪೂಜಾ ವಿಭಾವರೀ ಸೌಮ್ಯಾ ಭಗಿನೀ ಭೋಗದಾಯಿನೀ ॥೪೮॥

ಶೋಭಾ ವಂಶಕರೀ ಲೀಲಾ ಮಾನಿನೀ ಪರಮೇಷ್ಠಿನೀ ೨೫೦ ।\\
ತ್ರೈಲೋಕ್ಯಸುಂದರೀ ರಮ್ಯಾ ಸುಂದರೀ ಕಾಮಚಾರಿಣೀ ॥೪೯॥

ಮಹಾನುಭಾವಮಧ್ಯಸ್ಥಾ ಮಹಾಮಹಿಷಮರ್ದಿನೀ ।\\
ಪದ್ಮಮಾಲಾ ಪಾಪಹರಾ ವಿಚಿತ್ರಮುಕುಟಾನನಾ ॥೫೦॥

ಕಾಂತಾ ೨೬೦ ಚಿತ್ರಾಂಬರಧರಾ ದಿವ್ಯಾಭರಣಭೂಷಿತಾ ।\\
ಹಂಸಾಖ್ಯಾ ವ್ಯೋಮನಿಲಯಾ ಜಗತ್ಸೃಷ್ಟಿವಿವರ್ದ್ಧಿನೀ ॥೫೧॥

ನಿರ್ಯಂತ್ರಾ ಮಂತ್ರವಾಹಸ್ಥಾ ನಂದಿನೀ ಭದ್ರಕಾಲಿಕಾ ।\\
ಆದಿತ್ಯವರ್ಣಾ ೨೭೦ ಕೌಮಾರೀ ಮಯೂರವರವಾಹಿನೀ ॥೫೨॥

ವೃಷಾಸನಗತಾ ಗೌರೀ ಮಹಾಕಾಲೀ ಸುರಾರ್ಚಿತಾ ।\\
ಅದಿತಿರ್ನಿಯತಾ ರೌದ್ರೀ ಪದ್ಮಗರ್ಭಾ ೨೮೦ ವಿವಾಹನಾ ॥೫೩॥

ವಿರೂಪಾಕ್ಷೀ ಲೇಲಿಹಾನಾ ಮಹಾಸುರವಿನಾಶಿನೀ ।\\
ಮಹಾಫಲಾನವದ್ಯಾಂಗೀ ಕಾಮಪೂರಾ ವಿಭಾವರೀ ॥೫೪॥

ಕೌಶಿಕೀ ಕರ್ಷಿಣೀ ರಾತ್ರಿಸ್ತ್ರಿದಶಾರ್ತ್ತಿವಿನಾಶನೀ ॥೫೫॥

ವಿರೂಪಾ ಚ ಸರೂಪಾ ಚ ಭೀಮಾ ಮೋಕ್ಷಪ್ರದಾಯಿನೀ ।\\
ಭಕ್ತಾರ್ತ್ತಿನಾಶಿನೀ ಭವ್ಯಾ ೩೦೦ ಭವಭಾವವಿನಾಶಿನೀ ॥೫೬॥

ನಿರ್ಗುಣಾ ನಿತ್ಯವಿಭವಾ ನಿಃಸಾರಾ ನಿರಪತ್ರಪಾ ।\\
ಯಶಸ್ವಿನೀ ಸಾಮಗೀತಿರ್ಭಾವಾಂಗನಿಲಯಾಲಯಾ ॥೫೭॥

ದೀಕ್ಷಾ ೩೧೦ ವಿದ್ಯಾಧರೀ ದೀಪ್ತಾ ಮಹೇಂದ್ರವಿನಿಪಾತಿನೀ ।\\
ಸರ್ವಾತಿಶಾಯಿನೀ ವಿದ್ಯಾ ಸರ್ವಶಕ್ತಿಪ್ರದಾಯಿನೀ ॥೫೮॥

ಸರ್ವೇಶ್ವರಪ್ರಿಯಾ ತಾರ್ಕ್ಷೀ ಸಮುದ್ರಾಂತರವಾಸಿನೀ ।\\
ಅಕಲಂಕಾ ನಿರಾಧಾರಾ ೩೨೦ ನಿತ್ಯಸಿದ್ಧಾ ನಿರಾಮಯಾ ॥೫೯॥

ಕಾಮಧೇನುರ್ವೇದಗರ್ಭಾ ಧೀಮತೀ ಮೋಹನಾಶಿನೀ ।\\
ನಿಃಸಂಕಲ್ಪಾ ನಿರಾತಂಕಾ ವಿನಯಾ ವಿನಯಪ್ರದಾ ೩೨೦ ॥೬೦॥

ಜ್ವಾಲಾಮಾಲಾಸಹಸ್ರಾಢ್ಯಾ ದೇವದೇವೀ ಮನೋನ್ಮನೀ ।\\
ಉರ್ವೀ ಗುರ್ವೀ ಗುರುಃ ಶ್ರೇಷ್ಠಾ ಸಗುಣಾ ಷಡ್ಗುಣಾತ್ಮಿಕಾ ॥೬೧॥

ಮಹಾಭಗವತೀ ೩೪೦ ಭವ್ಯಾ ವಸುದೇವಸಮುದ್ಭವಾ ।\\
ಮಹೇಂದ್ರೋಪೇಂದ್ರಭಗಿನೀ ಭಕ್ತಿಗಮ್ಯಪರಾಯಣಾ ॥೬೨॥

ಜ್ಞಾನಜ್ಞೇಯಾ ಜರಾತೀತಾ ವೇದಾಂತವಿಷಯಾ ಗತಿಃ ।\\
ದಕ್ಷಿಣಾ ೩೫೦ ದಹನಾ ಬಾಹ್ಯಾ ಸರ್ವಭೂತನಮಸ್ಕೃತಾ ॥೬೩॥

ಯೋಗಮಾಯಾ ವಿಭಾವಜ್ಞಾ ಮಹಾಮೋಹಾ ಮಹೀಯಸೀ ।\\
ಸತ್ಯಾ ಸರ್ವಸಮುದ್ಭೂತಿರ್ಬ್ರಹ್ಮವೃಕ್ಷಾಶ್ರಯಾ ೩೬೦ ಮತಿಃ ॥೬೪॥

ಬೀಜಾಂಕುರಸಮುದ್ಭೂತಿರ್ಮಹಾಶಕ್ತಿರ್ಮಹಾಮತಿಃ ।\\
ಖ್ಯಾತಿಃ ಪ್ರತಿಜ್ಞಾ ಚಿತ್ಸಂವಿನ್ಮಹಾಯೋಗೇಂದ್ರಶಾಯಿನೀ ॥೬೫॥

ವಿಕೃತಿಃ ೩೭೦ ಶಂಕರೀ ಶಾಸ್ತ್ರೀ ಗಂಧರ್ವಾ ಯಕ್ಷಸೇವಿತಾ ।\\
ವೈಶ್ವಾನರೀ ಮಹಾಶಾಲಾ ದೇವಸೇನಾ ಗುಹಪ್ರಿಯಾ ॥೬೬॥

ಮಹಾರಾತ್ರೀ ಶಿವಾನಂದಾ ಶಚೀ ೩೮೦ ದುಃಸ್ವಪ್ನನಾಶಿನೀ ।\\
ಪೂಜ್ಯಾಪೂಜ್ಯಾ ಜಗದ್ಧಾತ್ರೀ ದುರ್ವಿಜ್ಞೇಯಸ್ವರೂಪಿಣೀ ॥೬೭॥

ಗುಹಾಂಬಿಕಾ ಗುಹೋತ್ಪತ್ತಿರ್ಮಹಾಪೀಠಾ ಮರುತ್ಸುತಾ ।\\
ಹವ್ಯವಾಹಾಂತರಾ ೩೬೦ ಗಾರ್ಗೀ ಹವ್ಯವಾಹಸಮುದ್ಭವಾ ॥೬೮॥

ಜಗದ್ಯೋನಿರ್ಜಗನ್ಮಾತಾ ಜಗನ್ಮೃತ್ಯುರ್ಜರಾತಿಗಾ ।\\
ಬುದ್ಧಿರ್ಮಾತಾ ಬುದ್ಧಿಮತೀ ಪುರುಷಾಂತರವಾಸಿನೀ ೪೦೦ ॥೬೯॥

ತಪಸ್ವಿನೀ ಸಮಾಧಿಸ್ಥಾ ತ್ರಿನೇತ್ರಾ ದಿವಿಸಂಸ್ಥಿತಾ ।\\
ಸರ್ವೇಂದ್ರಿಯಮನೋಮಾತಾ ಸರ್ವಭೂತಹೃದಿಸ್ಥಿತಾ ॥೭೦॥

ಬ್ರಹ್ಮಾಣೀ ಬೃಹತೀ ೪೧೦ ಬ್ರಾಹ್ಮೀ ಬ್ರಹ್ಮಭೂತಾ ಭಯಾವನೀ ॥೭೧॥

ಹಿರಣ್ಯಮಯೀ ಮಹಾರಾತ್ರಿಃ ಸಂಸಾರಪರಿವರ್ತಿಕಾ ।\\
ಸುಮಾಲಿನೀ ಸುರೂಪಾ ಚ ತಾರಿಣೀ ಭಾವಿನೀ ೪೨೦ ಪ್ರಭಾ ॥೭೨॥

ಉನ್ಮೀಲನೀ ಸರ್ವಸಹಾ ಸರ್ವಪ್ರತ್ಯಯಸಾಕ್ಷಿಣೀ ।\\
ತಪಿನೀ ತಾಪಿನೀ ವಿಶ್ವಾ ಭೋಗದಾ ಧಾರಿಣೀ ಧರಾ ೪೩೦ ॥೭೩॥

ಸುಸೌಮ್ಯಾ ಚಂದ್ರವದನಾ ತಾಂಡವಾಸಕ್ತಮಾನಸಾ ।\\
ಸತ್ತ್ವಶುದ್ಧಿಕರೀ ಶುದ್ಧಿರ್ಮಲತ್ರಯವಿನಾಶಿನೀ ॥೭೪॥

ಜಗತ್ಪ್ರಿಯಾ ಜಗನ್ಮೂರ್ತಿಸ್ತ್ರಿಮೂರ್ತಿರಮೃತಾಶ್ರಯಾ ೪೪೦ ।\\
ನಿರಾಶ್ರಯಾ ನಿರಾಹಾರಾ ನಿರಂಕುಶರಣೋದ್ಭಭವಾ ॥೭೫॥

ಚಕ್ರಹಸ್ತಾ ವಿಚಿತ್ರಾಂಗೀ ಸ್ರಗ್ವಿಣೀ ಪದ್ಮಧಾರಿಣೀ ।\\
ಪರಾಪರವಿಧಾನಜ್ಞಾ ಮಹಾಪುರುಷಪೂರ್ವಜಾ ॥೭೬॥

ವಿದ್ಯೇಶ್ವರಪ್ರಿಯಾಽವಿದ್ಯಾ ವಿದುಜ್ಜಿಹ್ವಾ ಜಿತಶ್ರಮಾ । ೪೫೩
ವಿದ್ಯಾಮಯೀ ಸಹಸ್ರಾಕ್ಷೀ ಸಹಸ್ರಶ್ರವಣಾತ್ಮಜಾ ॥೭೭॥

ಜ್ವಾಲಿನೀ ೪೬೦ ಸದ್ಮನಾ ವ್ಯಾಪ್ತಾ ತೈಜಸೀ ಪದ್ಮರೋಧಿಕಾ ॥೭೮॥

ಮಹಾದೇವಾಶ್ರಯಾ ಮಾನ್ಯಾ ಮಹಾದೇವಮನೋರಮಾ ।\\
ವ್ಯೋಮಲಕ್ಷ್ಮೀಶ್ಚ ಸಿಂಹಸ್ಥಾ ಚೇಕಿತಾನ್ಯಮಿತಪ್ರಭಾ ೪೭೦ ॥೭೯॥

ವಿಶ್ವೇಶ್ವರೀ ವಿಮಾನಸ್ಥಾ ವಿಶೋಕಾ ಶೋಕನಾಶಿನೀ ।\\
ಅನಾಹತಾ ಕುಂಡಲಿನೀ ನಲಿನೀ ಪದ್ಮವಾಸಿನೀ ॥೮೦॥

ಶತಾನಂದಾ ಸತಾಂ ಕೀರ್ತಿಃ ೪೮೦ ಸರ್ವಭೂತಾಶಯಸ್ಥಿತಾ ।\\
ವಾಗ್ದೇವತಾ ಬ್ರಹ್ಮಕಲಾ ಕಲಾತೀತಾ ಕಲಾವತೀ ॥೮೧॥

ಬ್ರಹ್ಮರ್ಷಿರ್ಬ್ರಹ್ಮಹೃದಯಾ ಬ್ರಹಾವಿಷ್ಣುಶಿವಪ್ರಿಯಾ ।\\
ವ್ಯೋಮಶಕ್ತಿಃ ಕ್ರಿಯಾಶಕ್ತಿರ್ಜನಶಕ್ತಿಃ ಪರಾಗತಿಃ (೪೯೨)॥೮೨॥

ಕ್ಷೋಭಿಕಾ ರೌದ್ರಿಕಾ ಭೇದ್ಯಾ ಭೇದಾಭೇದವಿವರ್ಜಿತಾ ।\\
ಅಭಿನ್ನಾ ಭಿನ್ನಸಂಸ್ಥಾನಾ ವಂಶಿನೀ ವಂಶಹಾರಿಣೀ ೫೦೦ ॥೮೩॥

ಗುಹ್ಯಶಕ್ತಿರ್ಗುಣಾತೀತಾ ಸರ್ವದಾ ಸರ್ವತೋಮುಖೀ ।\\
ಭಗಿನೀ ಭಗವತ್ಪತ್ನೀಂ ಸಕಲಾ ಕಾಲಕಾರಿಣೀ ॥೮೪॥

ಸರ್ವವಿತ್ಸರ್ವತೋಭದ್ರಾ ೫೧೦ ಗುಹ್ಯಾತೀತಾ ಗುಹಾಬಲಿಃ ।\\
ಪ್ರಕ್ರಿಯಾ ಯೋಗಮಾತಾ ಚ ಗಂಧಾ ವಿಶ್ವೇಶ್ವರೇಶ್ವರೀ ॥೮೫॥

ಕಪಿಲಾ ಕಪಿಲಾಕಾಂತಾ ಕನಕಾಭಾ ಕಲಾಂತರಾ ೫೨೦ ।\\
ಪುಣ್ಯಾ ಪುಷ್ಕರಿಣೀ ಭೋಕ್ತ್ರೀ ಪುರಂದರಪುರಃಸರಾ ॥೮೬॥

ಪೋಷಣೀ ಪರಮೈಶ್ವರ್ಯಭೂತಿದಾ ಭೂತಿಭೂಷಣಾ ॥

ಪಂಚಬ್ರಹ್ಮಸಮುತ್ಪತ್ತಿಃ ಪರಮಾತ್ಮಾತ್ಽಽಮವಿಗ್ರಹಾ ॥೮೭॥

ನರ್ಮೋದಯಾ ೫೩೦ ಭಾನುಮತೀ ಯೋಗಿಜ್ಞೇಯಾ ಮನೋಜವಾ ।\\
ಬೀಜರೂಪಾ ರಜೋರೂಪಾ ವಶಿನೀ ಯೋಗರೂಪಿಣೀ ॥೮೮॥

ಸುಮಂತ್ರಾ ಮಂತ್ರಿಣೀ ಪೂರ್ಣಾ ೫೪೦ ಹ್ಲಾದಿನೀ ಕ್ಲೇಶನಾಶಿನೀ ।\\
ಮನೋಹರಿರ್ಮನೋರಕ್ಷೀ ತಾಪಸೀ ವೇದರೂಪಿಣೀ ॥೮೯॥

ವೇದಶಕ್ತಿರ್ವೇದಮಾತಾ ವೇದವಿದ್ಯಾಪ್ರಕಾಶಿನೀ ।\\
ಯೋಗೇಶ್ವರೇಶ್ವರೀ ೫೫೦ ಮಾಲಾ ಮಹಾಶಕ್ತಿರ್ಮನೋಮಯೀ ॥೯೦॥

ವಿಶ್ವಾವಸ್ಥಾ ವೀರಮುಕ್ತಿರ್ವಿದ್ಯುನ್ಮಾಲಾ ವಿಹಾಯಸೀ ।\\
ಪೀವರೀ ಸುರಭೀ ವಂದ್ಯಾ ೫೬೦ ನಂದಿನೀ ನಂದವಲ್ಲಭಾ ॥೯೧॥

ಭಾರತೀ ಪರಮಾನಂದಾ ಪರಾಪರವಿಭೇದಿಕಾ ।\\
ಸರ್ವಪ್ರಹರಣೋಪೇತಾ ಕಾಮ್ಯಾ ಕಾಮೇಶ್ವರೇಶ್ವರೀ ॥೯೨॥

ಅಚಿಂತ್ಯಾಚಿಂತ್ಯಮಹಿಮಾ ೫೭೦ ದುರ್ಲೇಖಾ ಕನಕಪ್ರಭಾ ।\\
ಕೂಷ್ಮಾಂಡೀ ಧನರತ್ನಾಢ್ಯಾ ಸುಗಂಧಾ ಗಂಧದಾಯಿನೀ ॥೯೩॥

ತ್ರಿವಿಕ್ರಮಪದೋದ್ಭೂತಾ ಧನುಷ್ಪಾಣಿಃ ಶಿರೋಹಯಾ ।\\
ಸುದುರ್ಲಭಾ ೫೮೦ ಧನಾಧ್ಯಕ್ಷಾ ಧನ್ಯಾ ಪಿಂಗಲಲೋಚನಾ ॥೯೪॥

ಭ್ರಾಂತಿಃ ಪ್ರಭಾವತೀ ದೀಪ್ತಿಃ ಪಂಕಜಾಯತಲೋಚನಾ ।\\
ಆದ್ಯಾ ಹೃತ್ಕಮಲೋದ್ಭೂತಾ ಪರಾಮಾತಾ ೫೬೦ ರಣಪ್ರಿಯಾ ॥೯೫॥

ಸತ್ಕ್ರಿಯಾ ಗಿರಿಜಾ ನಿತ್ಯಶುದ್ಧಾ ಪುಷ್ಪನಿರಂತರಾ ।\\
ದುರ್ಗಾ ಕಾತ್ಯಾಯನೀ ಚಂಡೀ ಚರ್ಚಿಕಾ ಶಾಂತವಿಗ್ರಹಾ ೬೦೦ ॥೯೬॥

ಹಿರಣ್ಯವರ್ಣಾ ರಜನೀ ಜಗನ್ಮಂತ್ರಪ್ರವರ್ತಿಕಾ ।\\
ಮಂದರಾದ್ರಿನಿವಾಸಾ ಚ ಶಾರದಾ ಸ್ವರ್ಣಮಾಲಿನೀ ॥೯೭॥

ರತ್ನಮಾಲಾ ರತ್ನಗರ್ಭಾ ಪೃಥ್ವೀ ವಿಶ್ವಪ್ರಮಾಥಿನೀ ೬೧೦ ।\\
ಪದ್ಮಾಸನಾ ಪದ್ಮನಿಭಾ ನಿತ್ಯತುಷ್ಟಾಮೃತೋದ್ಭವಾ ॥೯೮॥

ಧುನ್ವತೀ ದುಷ್ಪ್ರಕಂಪಾ ಚ ಸೂರ್ಯಮಾತಾ ದೃಷದ್ವತೀ ।\\
ಮಹೇಂದ್ರಭಗಿನೀ ಮಾಯಾ ೬೨೦ ವರೇಣ್ಯಾ ವರದರ್ಪಿತಾ ॥೯೯॥

ಕಲ್ಯಾಣೀ ಕಮಲಾ ರಾಮಾ ಪಂಚಭೂತವರಪ್ರದಾ ।\\
ವಾಚ್ಯಾ ವರೇಶ್ವರೀ ನಂದ್ಯಾ ದುರ್ಜಯಾ ೬೩೦ ದುರತಿಕ್ರಮಾ ॥೧೦೦॥

ಕಾಲರಾತ್ರಿರ್ಮಹಾವೇಗಾ ವೀರಭದ್ರಹಿತಪ್ರಿಯಾ ।\\
ಭದ್ರಕಾಲೀ ಜಗನ್ಮಾತಾ ಭಕ್ತಾನಾಂ ಭದ್ರದಾಯಿನೀ ॥೧೦೧॥

ಕರಾಲಾ ಪಿಂಗಲಾಕಾರಾ ನಾಮವೇದಾ ೬೪೦ ಮಹಾನದಾ ।\\
ತಪಸ್ವಿನೀ ಯಶೋದಾ ಚ ಯಥಾಧ್ವಪರಿವರ್ತಿನೀ ॥೧೦೨॥

ಶಂಖಿನೀ ಪದ್ಮಿನೀ ಸಾಂಖ್ಯಾ ಸಾಂಖ್ಯಯೋಗಪ್ರವರ್ತಿಕಾ ।\\
ಚೈತ್ರೀ ಸಂವತ್ಸರಾ ೬೫೦ ರುದ್ರಾ ಜಗತ್ಸಂಪೂರಣೀಂದ್ರಜಾ ॥೧೦೩॥

ಶುಂಭಾರಿಃ ಖೇಚರೀ ಖಸ್ಥಾ ಕಂಬುಗ್ರೀವಾ ಕಲಿಪ್ರಿಯಾ ।\\
ಖರಧ್ವಜಾ ಖರಾರೂಢಾ ೬೬೦ ಪರಾರ್ಧ್ಯಾ ಪರಮಾಲಿನೀ ॥೧೦೪॥

ಐಶ್ವರ್ಯರತ್ನನಿಲಯಾ ವಿರಕ್ತಾ ಗರುಡಾಸನಾ ।\\
ಜಯಂತೀ ಹೃದ್ಗುಹಾ ರಮ್ಯಾ ಸತ್ತ್ವವೇಗಾ ಗಣಾಗ್ರಣೀಃ ॥೧೦೫॥

ಸಂಕಲ್ಪಸಿದ್ಧಾ ೬೭೦ ಸಾಮ್ಯಸ್ಥಾ ಸರ್ವವಿಜ್ಞಾನದಾಯಿನೀ ।\\
ಕಲಿಕಲ್ಮಷಹಂತ್ರೀ ಚ ಗುಹ್ಯೋಪನಿಷದುತ್ತಮಾ ॥೧೦೬॥

ನಿತ್ಯದೃಷ್ಟಿಃ ಸ್ಮೃತಿರ್ವ್ಯಾಪ್ತಿಃ ಪುಷ್ಟಿಸ್ತುಷ್ಟಿಃ ೬೮೦ ಕ್ರಿಯಾವತೀ ।\\
ವಿಶ್ವಾಮರೇಶ್ವರೇಶಾನಾ ಭುಕ್ತಿರ್ಮುಕ್ತಿಃ ಶಿವಾಮೃತಾ ॥೧೦೭॥

ಲೋಹಿತಾ ಸರ್ವಮಾತಾ ಚ ಭೀಷಣಾ ವನಮಾಲಿನೀ ೬೯೦ ।\\
ಅನಂತಶಯನಾನಾದ್ಯಾ ನರನಾರಾಯಣೋದ್ಭವಾ ॥೧೦೮॥

ನೃಸಿಂಹೀ ದೈತ್ಯಮಥಿನೀ ಶಂಖಚಕ್ರಗದಾಧರಾ ।\\
ಸಂಕರ್ಷಣಸಮುತ್ಪತ್ತಿರಂಬಿಕೋಪಾತ್ತಸಂಶ್ರಯಾ ॥೧೦೯॥

ಮಹಾಜ್ವಾಲಾ ಮಹಾಮೂರ್ತಿಃ ೭೦೦ ಸುಮೂರ್ತಿಃ ಸರ್ವಕಾಮಧುಕ್ ।\\
ಸುಪ್ರಭಾ ಸುತರಾಂ ಗೌರೀ ಧರ್ಮಕಾಮಾರ್ಥಮೋಕ್ಷದಾ ॥೧೧೦॥

ಭ್ರೂಮಧ್ಯನಿಲಯಾಽಪೂರ್ವಾ ಪ್ರಧಾನಪುರುಷಾ ಬಲೀ ।\\
ಮಹಾವಿಭೂತಿದಾ ೭೧೦ ಮಧ್ಯಾ ಸರೋಜನಯನಾಸನಾ ॥೧೧೧॥

ಅಷ್ಟಾದಶಭುಜಾ ನಾಟ್ಯಾ ನೀಲೋತ್ಪಲದಲಪ್ರಭಾ ।\\
ಸರ್ವಶಕ್ತಾ ಸಮಾರೂಢಾ ಧರ್ಮಾಧರ್ಮಾನುವರ್ಜಿತಾ ॥೧೧೨॥

ವೈರಾಗ್ಯಜ್ಞಾನನಿರತಾ ನಿರಾಲೋಕಾ ೭೨೦ ನಿರಿಂದ್ರಿಯಾ ।\\
ವಿಚಿತ್ರಗಹನಾ ಧೀರಾ ಶಾಶ್ವತಸ್ಥಾನವಾಸಿನೀ ॥೧೧೩॥

ಸ್ಥಾನೇಶ್ವರೀ ನಿರಾನಂದಾ ತ್ರಿಶೂಲವರಧಾರಿಣೀ ।\\
ಅಶೇಷದೇವತಾಮೂರ್ತಿದೇವತಾ ಪರದೇವತಾ ೭೩೦ ॥೧೧೪॥

ಗಣಾತ್ಮಿಕಾ ಗಿರೇಃ ಪುತ್ರೀ ನಿಶುಂಭವಿನಿಪಾತಿನಿ ।\\
ಅವರ್ಣಾ ವರ್ಣರಹಿತಾ ನಿರ್ವರ್ಣಾ ಬೀಜಸಂಭವಾ ॥೧೧೫॥

ಅನಂತವರ್ಣಾನನ್ಯಸ್ಥಾ ಶಂಕರೀ ೭೪೦ ಶಾಂತಮಾನಸಾ ।\\
ಅಗೋತ್ರಾ ಗೋಮತೀ ಗೋಪ್ತ್ರೀ ಗುಹ್ಯರೂಪಾ ಗುಣಾಂತರಾ ॥೧೧೬॥

ಗೋಶ್ರೀರ್ಗವ್ಯಪ್ರಿಯಾ ಗೌರೀ ಗಣೇಶ್ವರನಮಸ್ಕೃತಾ ।\\
ಸತ್ಯಮಾತ್ರಾ ೭೫೦ ಸತ್ಯಸಂಧಾ ತ್ರಿಸಂಧ್ಯಾ ಸಂಧಿವರ್ಜಿತಾ ॥೧೧೭॥

ಸರ್ವವಾದಾಶ್ರಯಾ ಸಾಂಖ್ಯಾ ಸಾಂಖ್ಯಯೋಗಸಮುದ್ಭವಾ ।\\
ಅಸಂಖ್ಯೇಯಾಪ್ರಮೇಯಾಖ್ಯಾ ಶೂನ್ಯಾ ಶುದ್ಧಕುಲೋದ್ಭವಾ ೭೬೦ ॥೧೧೮॥

ಬಿಂದುನಾದಸಮುತ್ಪತ್ತಿಃ ಶಂಭುವಾಮಾ ಶಶಿಪ್ರಭಾ ।\\
ವಿಸಂಗಾ ಭೇದರಹಿತಾ ಮನೋಜ್ಞಾ ಮಧುಸೂದನೀ ॥೧೧೯॥

ಮಹಾಶ್ರೀಃ ಶ್ರೀಸಮುತ್ಪತ್ತಿ ೭೭೦ ಸ್ತಮಃಪಾರೇ ಪ್ರತಿಷ್ಠಿತಾ ।\\
ತ್ರಿತತ್ತ್ವಮಾತಾ ತ್ರಿವಿಧಾ ಸುಸೂಕ್ಷ್ಮಪದಸಂಶ್ರಯಾ ॥೧೨೦॥

ಶಾಂತ್ಯಾತೀತಾ ಮಲಾತೀತಾ ನಿರ್ವಿಕಾರಾ ನಿರಾಶ್ರಯಾ ।\\
ಶಿವಾಖ್ಯಾ ಚಿತ್ರನಿಲಯಾ ೭೮೦ ಶಿವಜ್ಞಾನಸ್ವರೂಪಿಣೀ ॥೧೨೧॥

ದೈತ್ಯದಾನವನಿರ್ಮಾತ್ರೀ ಕಾಶ್ಯಪೀ ಕಾಲಕರ್ಣಿಕಾ ।\\
ಶಾಸ್ತ್ರಯೋನಿಃ ಕ್ರಿಯಾಮೂರ್ತಿಶ್ಚತುರ್ವರ್ಗಪ್ರದರ್ಶಿಕಾ ॥೧೨೨॥

ನಾರಾಯಣೀ ನವೋದ್ಭೂತಾ ಕೌಮುದೀ ೭೬೦ ಲಿಂಗಧಾರಿಣೀ ।\\
ಕಾಮುಕೀ ಲಲಿತಾ ತಾರಾ ಪರಾಪರವಿಭೂತಿದಾ ॥೧೨೩॥

ಪರಾಂತಜಾತಮಹಿಮಾ ವಾಡವಾ ವಾಮಲೋಚನಾ ।\\
ಸುಭದ್ರಾ ದೇವಕೀ ೮೦೦ ಸೀತಾ ವೇದವೇದಾಂಗಪಾರಗಾ ॥೧೨೪॥

ಮನಸ್ವಿನೀ ಮನ್ಯುಮಾತಾ ಮಹಾಮನ್ಯುಸಮುದ್ಭವಾ ॥

ಅಮೃತ್ಯುರಮೃತಾಸ್ವಾದಾ ಪುರುಹೂತಾ ಪುರುಪ್ಲುತಾ ॥೧೨೫॥

ಅಶೋಚ್ಯಾ ೮೧೦ ಭಿನ್ನವಿಷಯಾ ಹಿರಣ್ಯರಜತಪ್ರಿಯಾ ।\\
ಹಿರಣ್ಯಾ ರಾಜತೀ ಹೈಮೀ ಹೇಮಾಭರಣಭೂಷಿತಾ ॥೧೨೬॥

ವಿಭ್ರಾಜಮಾನಾ ದುರ್ಜ್ಞೇಯಾ ಜ್ಯೋತಿಷ್ಟೋಮಫಲಪ್ರದಾ ।\\
ಮಹಾನಿದ್ರಾ ೮೨೦ ಸಮುದ್ಭೂತಿರ್ಬಲೀಂದ್ರಾ ಸತ್ಯದೇವತಾ ॥೧೨೭॥

ದೀರ್ಘಾ ಕಕುದ್ಮಿನೀ ವಿದ್ಯಾ ಶಾಂತಿದಾ ಶಾಂತಿವರ್ದ್ಧಿನೀ ।\\
ಲಕ್ಷ್ಮ್ಯಾದಿಶಕ್ತಿಜನನೀ ಶಕ್ತಿಚಕ್ರಪ್ರವರ್ತಿಕಾ ॥೧೨೮॥

ತ್ರಿಶಕ್ತಿಜನನೀ ೮೩೦ ಜನ್ಯಾ ಷಡೂರ್ಮಿಪರಿವರ್ಜಿತಾ ।\\
ಸ್ವಾಹಾ ಚ ಕರ್ಮಕರಣೀ ಯುಗಾಂತದಲನಾತ್ಮಿಕಾ ॥೧೨೯॥

ಸಂಕರ್ಷಣಾ ಜಗದ್ಧಾತ್ರೀ ಕಾಮಯೋನಿಃ ಕಿರೀಟಿನೀ ।\\
ಐಂದ್ರೀ ೮೪೦ ತ್ರೈಲೋಕ್ಯನಮಿತಾ ವೈಷ್ಣವೀ ಪರಮೇಶ್ವರೀ ॥೧೩೦॥

ಪ್ರದ್ಯುಮ್ನದಯಿತಾ ದಾಂತಾ ಯುಗ್ಮದೃಷ್ಟಿಸ್ತ್ರಿಲೋಚನಾ ।\\
ಮಹೋತ್ಕಟಾ ಹಂಸಗತಿಃ ಪ್ರಚಂಡಾ ೮೫೦ ಚಂಡವಿಕ್ರಮಾ ॥೧೩೧॥

ವೃಷಾವೇಶಾ ವಿಯನ್ಮಾತ್ರಾ ವಿಂಧ್ಯಪರ್ವತವಾಸಿನೀ ।\\
ಹಿಮವನ್ಮೇರುನಿಲಯಾ ಕೈಲಾಸಗಿರಿವಾಸಿನೀ ॥೧೩೨॥

ಚಾಣೂರಹಂತ್ರೀ ತನಯಾ ನೀತಿಜ್ಞಾ ಕಾಮರೂಪಿಣೀ ೮೬೦ ।\\
ವೇದವಿದ್ಯಾ ವ್ರತರತಾ ಧರ್ಮಶೀಲಾನಿಲಾಶನಾ ॥೧೩೩॥

ಅಯೋಧ್ಯಾನಿಲಯಾ ವೀರಾ ಮಹಾಕಾಲಸಮುದ್ಭವಾ ।\\
ವಿದ್ಯಾಧರಕ್ರಿಯಾ ಸಿದ್ಧಾ ವಿದ್ಯಾಧರನಿರಾಕೃತಿಃ ॥೧೩೪॥

ಆಪ್ಯಾಯಂತೀ ೮೭೦ ವಹಂತೀ ಚ ಪಾವನೀ ಪೋಷಣೀ ಖಿಲಾ ।\\
ಮಾತೃಕಾ ಮನ್ಮಥೋದ್ಭೂತಾ ವಾರಿಜಾ ವಾಹನಪ್ರಿಯಾ ॥೧೩೫॥

ಕರೀಷಿಣೀ ಸ್ವಧಾ ವಾಣೀ ೮೮೦ ವೀಣಾವಾದನತತ್ಪರಾ ।\\
ಸೇವಿತಾ ಸೇವಿಕಾ ಸೇವಾ ಸಿನೀವಾಲೀ ಗರುತ್ಮತೀ ॥೧೩೬॥

ಅರುಂಧತೀ ಹಿರಣ್ಯಾಕ್ಷೀ ಮಣಿದಾ ಶ್ರೀವಸುಪ್ರದಾ ೮೯೦ ।\\
ವಸುಮತೀ ವಸೋರ್ಧಾರಾ ವಸುಂಧರಾಸಮುದ್ಭವಾ ॥೧೩೭॥

ವರಾರೋಹಾ ವರಾರ್ಹಾ ಚ ವಪುಃಸಂಗಸಮುದ್ಭವಾ ।\\
ಶ್ರೀಫಲೀ ಶ್ರೀಮತೀ ಶ್ರೀಶಾ ಶ್ರೀನಿವಾಸಾ ೯೦೦ ಹರಿಪ್ರಿಯಾ ॥೧೩೮॥

ಶ್ರೀಧರೀ ಶ್ರೀಕರೀ ಕಂಪಾ ಶ್ರೀಧರಾ ಈಶವೀರಣೀ ।\\
ಅನಂತದೃಷ್ಟಿರಕ್ಷುದ್ರಾ ಧಾತ್ರೀಶಾ ಧನದಪ್ರಿಯಾ ೯೧೦ ॥೧೩೯॥

ನಿಹಂತ್ರೀ ದೈತ್ಯಸಿಂಹಾನಾಂ ಸಿಂಹಿಕಾ ಸಿಂಹವಾಹಿನೀ ।\\
ಸುಸೇನಾ ಚಂದ್ರನಿಲಯಾ ಸುಕೀರ್ತಿಶ್ಛಿನ್ನಸಂಶಯಾ ॥೧೪೦॥

ಬಲಜ್ಞಾ ಬಲದಾ ವಾಮಾ ೯೨೦ ಲೇಲಿಹಾನಾಮೃತಾಶ್ರವಾ ।\\
ನಿತ್ಯೋದಿತಾ ಸ್ವಯಂಜ್ಯೋತಿರುತ್ಸುಕಾಮೃತಜೀವಿನೀ ॥೧೪೧॥

ವಜ್ರದಂಷ್ಟ್ರಾ ವಜ್ರಜಿಹ್ವಾ ವೈದೇಹೀ ವಜ್ರವಿಗ್ರಹಾ ೯೩೦ ।\\
ಮಂಗಲ್ಯಾ ಮಂಗಲಾ ಮಾಲಾ ಮಲಿನಾ ಮಲಹಾರಿಣೀ ॥೧೪೨॥

ಗಾಂಧರ್ವೀ ಗಾರುಡೀ ಚಾಂದ್ರೀ ಕಂಬಲಾಶ್ವತರಪ್ರಿಯಾ ।\\
ಸೌದಾಮಿನೀ ೯೪೦ ಜನಾನಂದಾ ಭ್ರುಕುಟೀಕುಟಿಲಾನನಾ ॥೧೪೩॥

ಕರ್ಣಿಕಾರಕರಾ ಕಕ್ಷಾ ಕಂಸಪ್ರಾಣಾಪಹಾರಿಣೀ ।\\
ಯುಗಂಧರಾ ಯುಗಾವರ್ತ್ತಾ ತ್ರಿಸಂಧ್ಯಾಹರ್ಷವರ್ಧಿನೀ ॥೧೪೪॥

ಪ್ರತ್ಯಕ್ಷದೇವತಾ ೯೫೦ ದಿವ್ಯಾ ದಿವ್ಯಗಂಧಾ ದಿವಾಪರಾ ।\\
ಶಕ್ರಾಸನಗತಾ ಶಾಕ್ರೀ ಸಾಧ್ವೀ ನಾರೀ ಶವಾಸನಾ ॥೧೪೫॥

ಇಷ್ಟಾ ವಿಶಿಷ್ಟಾ ೯೬೦ ಶಿಷ್ಟೇಷ್ಟಾ ಶಿಷ್ಟಾಶಿಷ್ಟಪ್ರಪೂಜಿತಾ ।\\
ಶತರೂಪಾ ಶತಾವರ್ತ್ತಾ ವಿನೀತಾ ಸುರಭಿಃ ಸುರಾ ॥೧೪೬॥

ಸುರೇಂದ್ರಮಾತಾ ಸುದ್ಯುಮ್ನಾ ೯೭೦ ಸುಷುಮ್ನಾ ಸೂರ್ಯಸಂಸ್ಥಿತಾ ।\\
ಸಮೀಕ್ಷಾ ಸತ್ಪ್ರತಿಷ್ಠಾ ಚ ನಿರ್ವೃತ್ತಿರ್ಜ್ಞಾನಪಾರಗಾ ॥೧೪೭॥

ಧರ್ಮಶಾಸ್ತ್ರಾರ್ಥಕುಶಲಾ ಧರ್ಮಜ್ಞಾ ಧರ್ಮವಾಹನಾ ।\\
ಧರ್ಮಾಧರ್ಮವಿನಿರ್ಮಾತ್ರೀ ೯೮೦ ಧಾರ್ಮಿಕಾಣಾಂ ಶಿವಪ್ರದಾ ॥೧೪೮॥

ಧರ್ಮಶಕ್ತಿರ್ಧರ್ಮಮಯೀ ವಿಧರ್ಮಾ ವಿಶ್ವಧರ್ಮಿಣೀ ।\\
ಧರ್ಮಾಂತರಾ ಧರ್ಮಮಧ್ಯಾ ಧರ್ಮಪೂರ್ವೀ ಧನಪ್ರಿಯಾ ॥೧೪೯॥

ಧರ್ಮೋಪದೇಶಾ ೯೯೦ ಧರ್ಮಾತ್ಮಾ ಧರ್ಮಲಭ್ಯಾ ಧರಾಧರಾ ।\\
ಕಪಾಲೀ ಶಾಕಲಾಮೂರ್ತಿಃ ಕಲಾಕಲಿತವಿಗ್ರಹಾ ॥೧೫೦॥

ಧರ್ಮಶಕ್ತಿವಿನಿರ್ಮುಕ್ತಾ ಸರ್ವಶಕ್ತ್ಯಾಶ್ರಯಾ ತಥಾ ।\\
ಸರ್ವಾ ಸರ್ವೇಶ್ವರೀ ೧೦೦೦ ಸೂಕ್ಷ್ಮಾ ಸುಸೂಕ್ಷ್ಮಜ್ಞಾನರೂಪಿಣೀ ॥೧೫೧॥

ಪ್ರಧಾನಪುರುಷೇಶಾನಾ ಮಹಾಪುರುಷಸಾಕ್ಷಿಣೀ ।\\
ಸದಾಶಿವಾ ವಿಯನ್ಮೂರ್ತಿರ್ದೇವಮೂರ್ತಿರಮೂರ್ತಿಕಾ ೧೦೦೮ ॥೧೫೨॥

ಏವಂ ನಾಮ್ನಾಂ ಸಹಸ್ರೇಣ ತುಷ್ಟಾವ ರಘುನಂದನಃ ।\\
ಕೃತಾಂಜಲಿಪುಟೋ ಭೂತ್ವಾ ಸೀತಾಂ ಹೃಷ್ಟತನೂರುಹಾಂ ॥೧೫೩॥

ಭಾರದ್ವಾಜ ಮಹಾಭಾಗ ಯಶ್ಚೈತಸ್ತೋತ್ರಮದ್ಭುತಂ ।\\
ಶೃಣುಯಾದ್ವಾ ಪಠೇದ್ವಾಪಿ ಸ ಯಾತಿ ಪರಮಂ ಪದಂ ॥೧೫೪॥

ಬ್ರಹ್ಮಕ್ಷತ್ರಿಯವಿಡ್ಯೋನಿರ್ಬ್ರಹ್ಮ ಪ್ರಾಪ್ನೋತಿ ಶಾಶ್ವತಂ ।\\
ಶೂದ್ರಃ ಸದ್ಗತಿಮಾಪ್ನೋತಿ ಧನಧಾನ್ಯವಿಭೂತಯಃ ॥೧೫೪॥

ಭವಂತಿ ಸ್ತೋತ್ರಮಹಾತ್ಮ್ಯಾದೇತತ್ಸ್ವಸ್ತ್ಯಯನಂ ಮಹತ್ ।\\
ಮಾರೀಭಯೇ ರಾಜಭಯೇ ತಥಾ ಚೋರಾಗ್ನಿಜೇ ಭಯೇ ॥೧೫೬॥

ವ್ಯಾಧೀನಾಂ ಪ್ರಭವೇ ಘೋರೇ ಶತ್ರೂತ್ಥಾನೇ ಚ ಸಂಕಟೇ ।\\
ಅನಾವೃಷ್ಟಿಭಯೇ ವಿಪ್ರ ಸರ್ವಶಾಂತಿಕರಂ ಪರಂ ॥೧೫೭॥

ಯದ್ಯದಿಷ್ಟತಮಂ ಯಸ್ಯ ತತ್ಸರ್ವಂ ಸ್ತೋತ್ರತೋ ಭವೇತ್ ।\\
ಯತ್ರೈತತ್ಪಠ್ಯತೇ ಸಮ್ಯಕ್ ಸೀತಾನಾಮಸಹಸ್ರಕಂ ॥೧೫೮॥

ರಾಮೇಣ ಸಹಿತಾ ದೇವೀ ತತ್ರ ತಿಷ್ಠತ್ಯಸಂಶಯಂ ।\\
ಮಹಾಪಾಪಾತಿಪಾಪಾನಿ ವಿಲಯಂ ಯಾಂತಿ ಸುವ್ರತ ॥೧೫೯॥
\authorline{ಇತ್ಯಾರ್ಷೇ ಶ್ರೀಮದ್ರಾಮಾಯಣೇ ವಾಲ್ಮೀಕೀಯೇ ಆದಿಕಾವ್ಯೇ ಅದ್ಭುತೋತ್ತರಕಾಂಡೇ ಸೀತಾಸಹಸ್ರನಾಮಸ್ತೋತ್ರಮ್ ॥}
%=============================================================================================
\section{ಸೀತಾಷ್ಟೋತ್ತರಶತನಾಮಸ್ತೋತ್ರಂ}
\addcontentsline{toc}{section}{ಸೀತಾಷ್ಟೋತ್ತರಶತನಾಮಸ್ತೋತ್ರಂ}
ಸೀತಾ ಸೀರಧ್ವಜಸುತಾ ಸೀಮಾತೀತಗುಣೋಜ್ಜ್ವಲಾ ।\\
ಸೌಂದರ್ಯಸಾರಸರ್ವಸ್ವಭೂತಾ ಸೌಭಾಗ್ಯದಾಯಿನೀ ॥೧॥

ದೇವೀ ದೇವಾರ್ಚಿತಪದಾ ದಿವ್ಯಾ ದಶರಥಸ್ನುಷಾ ।\\
ರಾಮಾ ರಾಮಪ್ರಿಯಾ ರಮ್ಯಾ ರಾಕೇಂದುವದನೋಜ್ವಲಾ ॥೨॥

ವೀರ್ಯಶುಲ್ಕಾ ವೀರಪತ್ನೀ ವಿಯನ್ಮಧ್ಯಾ ವರಪ್ರದಾ ।\\
ಪತಿವ್ರತಾ ಪಂಕ್ತಿಕಂಠನಾಶಿನೀ ಪಾವನಸ್ಮೃತಿಃ ॥೩॥

ವಂದಾರುವತ್ಸಲಾ ವೀರಮಾತಾ ವೃತರಘೂತ್ತಮಾ ।\\
ಸಂಪತ್ಕರೀ ಸದಾತುಷ್ಟಾ ಸಾಕ್ಷಿಣೀ ಸಾಧುಸಮ್ಮತಾ ॥೪॥

ನಿತ್ಯಾ ನಿಯತಸಂಸ್ಥಾನಾ ನಿತ್ಯಾನಂದಾ ನುತಿಪ್ರಿಯಾ ।\\
ಪೃಥ್ವೀ ಪೃಥ್ವೀಸುತಾ ಪುತ್ರದಾಯಿನೀ ಪ್ರಕೃತಿಃ ಪರಾ ॥೫॥

ಹನುಮತ್ಸ್ವಾಮಿನೀ ಹೃದ್ಯಾ ಹೃದಯಸ್ಥಾ ಹತಾಶುಭಾ ।\\
ಹಂಸಯುಕ್ತಾ ಹಂಸಗತಿಃ ಹರ್ಷಯುಕ್ತಾ ಹತಾಸುರಾ ॥೬॥

ಸಾರರೂಪಾ ಸಾರವಚಾಃ ಸಾಧ್ವೀ ಚ ಸರಮಾಪ್ರಿಯಾ ।\\
ತ್ರಿಲೋಕವಂದ್ಯಾ ತ್ರಿಜಟಾಸೇವ್ಯಾ ತ್ರಿಪಥಗಾರ್ಚಿನೀ ॥೭॥

ತ್ರಾಣಪ್ರದಾ ತ್ರಾತಕಾಕಾ ತೃಣೀಕೃತದಶಾನನಾ ।\\
ಅನಸೂಯಾಂಗರಾಗಾಂಕಾಽನಸೂಯಾ ಸುರವಂದಿತಾ ॥೮॥

ಅಶೋಕವನಿಕಾಸ್ಥಾನಾಽಶೋಕಾ ಶೋಕವಿನಾಶಿನೀ ।\\
ಸೂರ್ಯವಂಶಸ್ನುಷಾ ಸೂರ್ಯಮಂಡಲಾಂತಃಸ್ಥವಲ್ಲಭಾ ॥೯॥

ಶ್ರುತಮಾತ್ರಾಘಹರಣಾ ಶ್ರುತಿಸನ್ನಿಹಿತೇಕ್ಷಣಾ ।\\
ಪುಷ್ಪಪ್ರಿಯಾ ಪುಷ್ಪಕಸ್ಥಾ ಪುಣ್ಯಲಭ್ಯಾ ಪುರಾತನಾ ॥೧೦॥

ಪುರುಷಾರ್ಥಪ್ರದಾ ಪೂಜ್ಯಾ ಪೂತನಾಮ್ನೀ ಪರಂತಪಾ ।\\
ಪದ್ಮಪ್ರಿಯಾ ಪದ್ಮಹಸ್ತಾ ಪದ್ಮಾ ಪದ್ಮಮುಖೀ ಶುಭಾ ॥೧೧॥

ಜನಶೋಕಹರಾ ಜನ್ಮಮೃತ್ಯುಶೋಕವಿನಾಶಿನೀ ।\\
ಜಗದ್ರೂಪಾ ಜಗದ್ವಂದ್ಯಾ ಜಯದಾ ಜನಕಾತ್ಮಜಾ ॥೧೨॥

ನಾಥನೀಯಕಟಾಕಾಕ್ಷಾ ಚ ನಾಥಾ ನಾಥೈಕತತ್ಪರಾ ।\\
ನಕ್ಷತ್ರನಾಥವದನಾ ನಷ್ಟದೋಷಾ ನಯಾವಹಾ ॥೧೩॥

ವಹ್ನಿಪಾಪಹರಾ ವಹ್ನಿಶೈತ್ಯಕೃದ್ವೃದ್ಧಿದಾಯಿನೀ ।\\
ವಾಲ್ಮೀಕಿಗೀತವಿಭವಾ ವಚೋಽತೀತಾ ವರಾಂಗನಾ ॥೧೪॥

ಭಕ್ತಿಗಮ್ಯಾ ಭವ್ಯಗುಣಾ ಭಾಂತೀ ಭರತವಂದಿತಾ ।\\
ಸುವರ್ಣಾಂಗೀ ಸುಖಕರೀ ಸುಗ್ರೀವಾಂಗದಸೇವಿತಾ ॥೧೫॥

ವೈದೇಹೀ ವಿನತಾಘೌಘನಾಶಿನೀ ವಿಧಿವಂದಿತಾ ।\\
ಲೋಕಮಾತಾ ಲೋಚನಾಂತಃಸ್ಥಿತಕಾರುಣ್ಯಸಾಗರಾ ॥

ಶ್ರೀರಾಮವಲ್ಲಭಾ ಸಾ ನಃ ಪಾಯಾದಾರ್ತಾನುಪಾಶ್ರಿತಾನ್ ॥೧೬॥

ಕೃತಾಕೃತಜಗದ್ಧೇತುಃ ಕೃತರಾಜ್ಯಾಭಿಷೇಕಕಾ ।\\
ಇದಮಷ್ಟೋತ್ತರಶತಂ ಸೀತಾನಾಮ್ನಾಂ ತು ಯಾ ವಧುಃ ॥೧೭॥

ಧನಧಾನ್ಯಸಮೃದ್ಧಾ ಚ ದೀರ್ಘಸೌಭಾಗ್ಯದರ್ಶಿನೀ ।\\
ಪುಂಸಾಮಪಿ ಸ್ತೋತ್ರಮಿದಂ ಪಠನಾತ್ಸರ್ವಸಿದ್ಧಿದಂ ॥೧೮॥
\authorline{ಇತಿ ಬ್ರಹ್ಮಯಾಮಲೇ ರಾಮರಹಸ್ಯಗತಂ ಸೀತಾಷ್ಟೋತ್ತರಶತನಾಮಸ್ತೋತ್ರಂ ಸಂಪೂರ್ಣಂ ।}

%=============================================================================================
\section{ದುರ್ಗಾಸಹಸ್ರನಾಮಸ್ತೋತ್ರಂ}
\addcontentsline{toc}{section}{ದುರ್ಗಾಸಹಸ್ರನಾಮಸ್ತೋತ್ರಂ}

ನಾರದ ಉವಾಚ ॥\\
ಕುಮಾರ ಗುಣಗಂಭೀರ ದೇವಸೇನಾಪತೇ ಪ್ರಭೋ ।\\
ಸರ್ವಾಭೀಷ್ಟಪ್ರದಂ ಪುಂಸಾಂ ಸರ್ವಪಾಪಪ್ರಣಾಶನಂ ॥೧॥

ಗುಹ್ಯಾದ್ಗುಹ್ಯತರಂ ಸ್ತೋತ್ರಂ ಭಕ್ತಿವರ್ಧಕಮಂಜಸಾ ।\\
ಮಂಗಲಂ ಗ್ರಹಪೀಡಾದಿಶಾಂತಿದಂ ವಕ್ತುಮರ್ಹಸಿ ॥೨॥

ಸ್ಕಂದ ಉವಾಚ ॥\\
ಶೃಣು ನಾರದ ದೇವರ್ಷೇ ಲೋಕಾನುಗ್ರಹಕಾಮ್ಯಯಾ ।\\
ಯತ್ಪೃಚ್ಛಸಿ ಪರಂ ಪುಣ್ಯಂ ತತ್ತೇ ವಕ್ಷ್ಯಾಮಿ ಕೌತುಕಾತ್ ॥೩॥

ಮಾತಾ ಮೇ ಲೋಕಜನನೀ ಹಿಮವನ್ನಗಸತ್ತಮಾತ್ ।\\
ಮೇನಾಯಾಂ ಬ್ರಹ್ಮವಾದಿನ್ಯಾಂ ಪ್ರಾದುರ್ಭೂತಾ ಹರಪ್ರಿಯಾ ॥೪॥

ಮಹತಾ ತಪಸಾಽಽರಾಧ್ಯ ಶಂಕರಂ ಲೋಕಶಂಕರಂ ।\\
ಸ್ವಮೇವ ವಲ್ಲಭಂ ಭೇಜೇ ಕಲೇವ ಹಿ ಕಲಾನಿಧಿಂ ॥೫॥

ನಗಾನಾಮಧಿರಾಜಸ್ತು ಹಿಮವಾನ್ ವಿರಹಾತುರಃ ।\\
ಸ್ವಸುತಾಯಾಃ ಪರಿಕ್ಷೀಣೇ ವಸಿಷ್ಠೇನ ಪ್ರಬೋಧಿತಃ ॥೬॥

ತ್ರಿಲೋಕಜನನೀ ಸೇಯಂ ಪ್ರಸನ್ನಾ ತ್ವಯಿ ಪುಣ್ಯತಃ ।\\
ಪ್ರಾದುರ್ಭೂತಾ ಸುತಾತ್ವೇನ ತದ್ವಿಯೋಗಂ ಶುಭಂ ತ್ಯಜ ॥೭॥

ಬಹುರೂಪಾ ಚ ದುರ್ಗೇಯಂ ಬಹುನಾಮ್ನೀ ಸನಾತನೀ ।\\
ಸನಾತನಸ್ಯ ಜಾಯಾ ಸಾ ಪುತ್ರೀಮೋಹಂ ತ್ಯಜಾಧುನಾ ॥೮॥

ಇತಿ ಪ್ರಬೋಧಿತಃ ಶೈಲಃ ತಾಂ ತುಷ್ಟಾವ ಪರಾಂ ಶಿವಾಂ ।\\
ತದಾ ಪ್ರಸನ್ನಾ ಸಾ ದುರ್ಗಾ ಪಿತರಂ ಪ್ರಾಹ ನಂದಿನೀ ॥೯॥

ಮತ್ಪ್ರಸಾದಾತ್ಪರಂ ಸ್ತೋತ್ರಂ ಹೃದಯೇ ಪ್ರತಿಭಾಸತಾಂ ।\\
ತೇನ ನಾಮ್ನಾಂ ಸಹಸ್ರೇಣ ಪೂಜಯನ್ ಕಾಮಮಾಪ್ನುಹಿ ॥೧೦॥

ಇತ್ಯುಕ್ತ್ವಾಂತರ್ಹಿತಾಯಾಂ ತು ಹೃದಯೇ ಸ್ಫುರಿತಂ ತದಾ ।\\
ನಾಮ್ನಾಂ ಸಹಸ್ರಂ ದುರ್ಗಾಯಾಃ ಪೃಚ್ಛತೇ ಮೇ ಯದುಕ್ತವಾನ್ ॥೧೧॥

ಮಂಗಲಾನಾಂ ಮಂಗಲಂ ತದ್ ದುರ್ಗಾನಾಮ ಸಹಸ್ರಕಂ ।\\
ಸರ್ವಾಭೀಷ್ಟಪ್ರದಾಂ ಪುಂಸಾಂ ಬ್ರವೀಮ್ಯಖಿಲಕಾಮದಂ ॥೧೨॥

ದುರ್ಗಾದೇವೀ ಸಮಾಖ್ಯಾತಾ ಹಿಮವಾನೃಷಿರುಚ್ಯತೇ ।\\
ಛಂದೋನುಷ್ಟುಪ್ ಜಪೋ ದೇವ್ಯಾಃ ಪ್ರೀತಯೇ ಕ್ರಿಯತೇ ಸದಾ ॥೧೩॥

ಅಸ್ಯ ಶ್ರೀದುರ್ಗಾಸ್ತೋತ್ರಮಹಾಮಂತ್ರಸ್ಯ । ಹಿಮವಾನ್ ಋಷಿಃ । ಅನುಷ್ಟುಪ್ ಛಂದಃ । ದುರ್ಗಾಭಗವತೀ ದೇವತಾ । ಶ್ರೀದುರ್ಗಾಪ್ರಸಾದಸಿದ್ಧ್ಯರ್ಥೇ ಜಪೇ ವಿನಿಯೋಗಃ ।\\
ಕಾಲಾಭ್ರಾಭಾಂ ಕಟಾಕ್ಷೈರರಿಕುಲಭಯದಾಂ ಮೌಲಿಬದ್ಧೇಂದುರೇಖಾಂ\\
ಶಂಖಂ ಚಕ್ರಂ ಕೃಪಾಣಂ ತ್ರಿಶಿಖಮಪಿ ಕರೈರುದ್ವಹಂತೀಂ ತ್ರಿನೇತ್ರಾಂ ।\\
ಸಿಂಹಸ್ಕಂಧಾಧಿರೂಢಾಂ ತ್ರಿಭುವನಮಖಿಲಂ ತೇಜಸಾ ಪೂರಯಂತೀಂ\\
ಧ್ಯಾಯೇದ್ ದುರ್ಗಾಂ ಜಯಾಖ್ಯಾಂ ತ್ರಿದಶಪರಿವೃತಾಂ ಸೇವಿತಾಂ ಸಿದ್ಧಿಕಾಮೈಃ॥

ಶ್ರೀ ಜಯದುರ್ಗಾಯೈ ನಮಃ ।\\
ಓಂ ಶಿವಾಽಥೋಮಾ ರಮಾ ಶಕ್ತಿರನಂತಾ ನಿಷ್ಕಲಾಽಮಲಾ ।\\
ಶಾಂತಾ ಮಾಹೇಶ್ವರೀ ನಿತ್ಯಾ ಶಾಶ್ವತಾ ಪರಮಾ ಕ್ಷಮಾ ॥೧॥

ಅಚಿಂತ್ಯಾ ಕೇವಲಾನಂತಾ ಶಿವಾತ್ಮಾ ಪರಮಾತ್ಮಿಕಾ ।\\
ಅನಾದಿರವ್ಯಯಾ ಶುದ್ಧಾ ಸರ್ವಜ್ಞಾ ಸರ್ವಗಾಽಚಲಾ ॥೨॥

ಏಕಾನೇಕವಿಭಾಗಸ್ಥಾ ಮಾಯಾತೀತಾ ಸುನಿರ್ಮಲಾ ।\\
ಮಹಾಮಾಹೇಶ್ವರೀ ಸತ್ಯಾ ಮಹಾದೇವೀ ನಿರಂಜನಾ ॥೩॥

ಕಾಷ್ಠಾ ಸರ್ವಾಂತರಸ್ಥಾಽಪಿ ಚಿಚ್ಛಕ್ತಿಶ್ಚಾತ್ರಿಲಾಲಿತಾ ।\\
ಸರ್ವಾ ಸರ್ವಾತ್ಮಿಕಾ ವಿಶ್ವಾ ಜ್ಯೋತೀರೂಪಾಽಕ್ಷರಾಽಮೃತಾ ॥೪॥

ಶಾಂತಾ ಪ್ರತಿಷ್ಠಾ ಸರ್ವೇಶಾ ನಿವೃತ್ತಿರಮೃತಪ್ರದಾ ।\\
ವ್ಯೋಮಮೂರ್ತಿರ್ವ್ಯೋಮಸಂಸ್ಥಾ ವ್ಯೋಮಧಾರಾಽಚ್ಯುತಾಽತುಲಾ ॥೫॥

ಅನಾದಿನಿಧನಾಽಮೋಘಾ ಕಾರಣಾತ್ಮಕಲಾಕುಲಾ ।\\
ಋತುಪ್ರಥಮಜಾಽನಾಭಿರಮೃತಾತ್ಮಸಮಾಶ್ರಯಾ ॥೬॥

ಪ್ರಾಣೇಶ್ವರಪ್ರಿಯಾ ನಮ್ಯಾ ಮಹಾಮಹಿಷಘಾತಿನೀ ।\\
ಪ್ರಾಣೇಶ್ವರೀ ಪ್ರಾಣರೂಪಾ ಪ್ರಧಾನಪುರುಷೇಶ್ವರೀ ॥೭॥

ಸರ್ವಶಕ್ತಿಕಲಾಽಕಾಮಾ ಮಹಿಷೇಷ್ಟವಿನಾಶಿನೀ ।\\
ಸರ್ವಕಾರ್ಯನಿಯಂತ್ರೀ ಚ ಸರ್ವಭೂತೇಶ್ವರೇಶ್ವರೀ ॥೮॥

ಅಂಗದಾದಿಧರಾ ಚೈವ ತಥಾ ಮುಕುಟಧಾರಿಣೀ ।\\
ಸನಾತನೀ ಮಹಾನಂದಾಽಽಕಾಶಯೋನಿಸ್ತಥೇಚ್ಯತೇ ॥೯॥

ಚಿತ್ಪ್ರಕಾಶಸ್ವರೂಪಾ ಚ ಮಹಾಯೋಗೇಶ್ವರೇಶ್ವರೀ ।\\
ಮಹಾಮಾಯಾ ಸದುಷ್ಪಾರಾ ಮೂಲಪ್ರಕೃತಿರೀಶಿಕಾ ॥೧೦॥

ಸಂಸಾರಯೋನಿಃ ಸಕಲಾ ಸರ್ವಶಕ್ತಿಸಮುದ್ಭವಾ ।\\
ಸಂಸಾರಪಾರಾ ದುರ್ವಾರಾ ದುರ್ನಿರೀಕ್ಷಾ ದುರಾಸದಾ ॥೧೧॥

ಪ್ರಾಣಶಕ್ತಿಶ್ಚ ಸೇವ್ಯಾ ಚ ಯೋಗಿನೀ ಪರಮಾಕಲಾ ।\\
ಮಹಾವಿಭೂತಿರ್ದುರ್ದರ್ಶಾ ಮೂಲಪ್ರಕೃತಿಸಂಭವಾ ॥೧೨॥

ಅನಾದ್ಯನಂತವಿಭವಾ ಪರಾರ್ಥಾ ಪುರುಷಾರಣಿಃ ।\\
ಸರ್ಗಸ್ಥಿತ್ಯಂತಕೃಚ್ಚೈವ ಸುದುರ್ವಾಚ್ಯಾ ದುರತ್ಯಯಾ ॥೧೩॥

ಶಬ್ದಗಮ್ಯಾ ಶಬ್ದಮಾಯಾ ಶಬ್ದಾಖ್ಯಾನಂದವಿಗ್ರಹಾ ।\\
ಪ್ರಧಾನಪುರುಷಾತೀತಾ ಪ್ರಧಾನಪುರುಷಾತ್ಮಿಕಾ ॥೧೪॥

ಪುರಾಣೀ ಚಿನ್ಮಯಾ ಪುಂಸಾಮಿಷ್ಟದಾ ಪುಷ್ಟಿರೂಪಿಣೀ ।\\
ಪೂತಾಂತರಸ್ಥಾ ಕೂಟಸ್ಥಾ ಮಹಾಪುರುಷಸಂಜ್ಞಿತಾ ॥೧೫॥

ಜನ್ಮಮೃತ್ಯುಜರಾತೀತಾ ಸರ್ವಶಕ್ತಿಸ್ವರೂಪಿಣೀ ।\\
ವಾಂಛಾಪ್ರದಾಽನವಚ್ಛಿನ್ನಪ್ರಧಾನಾನುಪ್ರವೇಶಿನೀ ॥೧೬॥

ಕ್ಷೇತ್ರಜ್ಞಾಽಚಿಂತ್ಯಶಕ್ತಿಸ್ತು ಪ್ರೋಚ್ಯತೇಽವ್ಯಕ್ತಲಕ್ಷಣಾ ।\\
ಮಲಾಪವರ್ಜಿತಾಽಽನಾದಿಮಾಯಾ ತ್ರಿತಯತತ್ತ್ವಿಕಾ ॥೧೭॥

ಪ್ರೀತಿಶ್ಚ ಪ್ರಕೃತಿಶ್ಚೈವ ಗುಹಾವಾಸಾ ತಥೋಚ್ಯತೇ ।\\
ಮಹಾಮಾಯಾ ನಗೋತ್ಪನ್ನಾ ತಾಮಸೀ ಚ ಧ್ರುವಾ ತಥಾ ॥೧೮॥

ವ್ಯಕ್ತಾಽವ್ಯಕ್ತಾತ್ಮಿಕಾ ಕೃಷ್ಣಾ ರಕ್ತಾ ಶುಕ್ಲಾ ಹ್ಯಕಾರಣಾ ।\\
ಪ್ರೋಚ್ಯತೇ ಕಾರ್ಯಜನನೀ ನಿತ್ಯಪ್ರಸವಧರ್ಮಿಣೀ ॥೧೯॥

ಸರ್ಗಪ್ರಲಯಮುಕ್ತಾ ಚ ಸೃಷ್ಟಿಸ್ಥಿತ್ಯಂತಧರ್ಮಿಣೀ ।\\
ಬ್ರಹ್ಮಗರ್ಭಾ ಚತುರ್ವಿಂಶಸ್ವರೂಪಾ ಪದ್ಮವಾಸಿನೀ ॥೨೦॥

ಅಚ್ಯುತಾಹ್ಲಾದಿಕಾ ವಿದ್ಯುದ್ಬ್ರಹ್ಮಯೋನಿರ್ಮಹಾಲಯಾ ।\\
ಮಹಾಲಕ್ಷ್ಮೀ ಸಮುದ್ಭಾವಭಾವಿತಾತ್ಮಾಮಹೇಶ್ವರೀ ॥೨೧॥

ಮಹಾವಿಮಾನಮಧ್ಯಸ್ಥಾ ಮಹಾನಿದ್ರಾ ಸಕೌತುಕಾ ।\\
ಸರ್ವಾರ್ಥಧಾರಿಣೀ ಸೂಕ್ಷ್ಮಾ ಹ್ಯವಿದ್ಧಾ ಪರಮಾರ್ಥದಾ ॥೨೨॥

ಅನಂತರೂಪಾಽನಂತಾರ್ಥಾ ತಥಾ ಪುರುಷಮೋಹಿನೀ ।\\
ಅನೇಕಾನೇಕಹಸ್ತಾ ಚ ಕಾಲತ್ರಯವಿವರ್ಜಿತಾ ॥೨೩॥

ಬ್ರಹ್ಮಜನ್ಮಾ ಹರಪ್ರೀತಾ ಮತಿರ್ಬ್ರಹ್ಮಶಿವಾತ್ಮಿಕಾ ।\\
ಬ್ರಹ್ಮೇಶವಿಷ್ಣುಸಂಪೂಜ್ಯಾ ಬ್ರಹ್ಮಾಖ್ಯಾ ಬ್ರಹ್ಮಸಂಜ್ಞಿತಾ ॥೨೪॥

ವ್ಯಕ್ತಾ ಪ್ರಥಮಜಾ ಬ್ರಾಹ್ಮೀ ಮಹಾರಾತ್ರೀಃ ಪ್ರಕೀರ್ತಿತಾ ।\\
ಜ್ಞಾನಸ್ವರೂಪಾ ವೈರಾಗ್ಯರೂಪಾ ಹ್ಯೈಶ್ವರ್ಯರೂಪಿಣೀ ॥೨೫॥

ಧರ್ಮಾತ್ಮಿಕಾ ಬ್ರಹ್ಮಮೂರ್ತಿಃ ಪ್ರತಿಶ್ರುತಪುಮರ್ಥಿಕಾ ।\\
ಅಪಾಂಯೋನಿಃ ಸ್ವಯಂಭೂತಾ ಮಾನಸೀ ತತ್ತ್ವಸಂಭವಾ ॥೨೬॥

ಈಶ್ವರಸ್ಯ ಪ್ರಿಯಾ ಪ್ರೋಕ್ತಾ ಶಂಕರಾರ್ಧಶರೀರಿಣೀ ।\\
ಭವಾನೀ ಚೈವ ರುದ್ರಾಣೀ ಮಹಾಲಕ್ಷ್ಮೀಸ್ತಥಾಽಮ್ಬಿಕಾ ॥೨೭॥

ಮಹೇಶ್ವರಸಮುತ್ಪನ್ನಾ ಭುಕ್ತಿಮುಕ್ತಿ ಪ್ರದಾಯಿನೀ ।\\
ಸರ್ವೇಶ್ವರೀ ಸರ್ವವಂದ್ಯಾ ನಿತ್ಯಮುಕ್ತಾ ಸುಮಾನಸಾ ॥೨೮॥

ಮಹೇಂದ್ರೋಪೇಂದ್ರನಮಿತಾ ಶಾಂಕರೀಶಾನುವರ್ತಿನೀ ।\\
ಈಶ್ವರಾರ್ಧಾಸನಗತಾ ಮಾಹೇಶ್ವರಪತಿವ್ರತಾ ॥೨೯॥

ಸಂಸಾರಶೋಷಿಣೀ ಚೈವ ಪಾರ್ವತೀ ಹಿಮವತ್ಸುತಾ ।\\
ಪರಮಾನಂದದಾತ್ರೀ ಚ ಗುಣಾಗ್ರ್ಯಾ ಯೋಗದಾ ತಥಾ ॥೩೦॥

ಜ್ಞಾನಮೂರ್ತಿಶ್ಚ ಸಾವಿತ್ರೀ ಲಕ್ಷ್ಮೀಃ ಶ್ರೀಃ ಕಮಲಾ ತಥಾ ।\\
ಅನಂತಗುಣಗಂಭೀರಾ ಹ್ಯುರೋನೀಲಮಣಿಪ್ರಭಾ ॥೩೧॥

ಸರೋಜನಿಲಯಾ ಗಂಗಾ ಯೋಗಿಧ್ಯೇಯಾಽಸುರಾರ್ದಿನೀ ।\\
ಸರಸ್ವತೀ ಸರ್ವವಿದ್ಯಾ ಜಗಜ್ಜ್ಯೇಷ್ಠಾ ಸುಮಂಗಲಾ ॥೩೨॥

ವಾಗ್ದೇವೀ ವರದಾ ವರ್ಯಾ ಕೀರ್ತಿಃ ಸರ್ವಾರ್ಥಸಾಧಿಕಾ ।\\
ವಾಗೀಶ್ವರೀ ಬ್ರಹ್ಮವಿದ್ಯಾ ಮಹಾವಿದ್ಯಾ ಸುಶೋಭನಾ ॥೩೩॥

ಗ್ರಾಹ್ಯವಿದ್ಯಾ ವೇದವಿದ್ಯಾ ಧರ್ಮವಿದ್ಯಾಽಽತ್ಮಭಾವಿತಾ ।\\
ಸ್ವಾಹಾ ವಿಶ್ವಂಭರಾ ಸಿದ್ಧಿಃ ಸಾಧ್ಯಾ ಮೇಧಾ ಧೃತಿಃ ಕೃತಿಃ ॥೩೪॥

ಸುನೀತಿಃ ಸಂಕೃತಿಶ್ಚೈವ ಕೀರ್ತಿತಾ ನರವಾಹಿನೀ ।\\
ಪೂಜಾವಿಭಾವಿನೀ ಸೌಮ್ಯಾ ಭೋಗ್ಯಭಾಗ್ ಭೋಗದಾಯಿನೀ ॥೩೫॥

ಶೋಭಾವತೀ ಶಾಂಕರೀ ಚ ಲೋಲಾ ಮಾಲಾವಿಭೂಷಿತಾ ।\\
ಪರಮೇಷ್ಠಿಪ್ರಿಯಾ ಚೈವ ತ್ರಿಲೋಕೀಸುಂದರೀ ಮಾತಾ ॥೩೬॥

ನಂದಾ ಸಂಧ್ಯಾ ಕಾಮಧಾತ್ರೀ ಮಹಾದೇವೀ ಸುಸಾತ್ತ್ವಿಕಾ ।\\
ಮಹಾಮಹಿಷದರ್ಪಘ್ನೀ ಪದ್ಮಮಾಲಾಽಘಹಾರಿಣೀ ॥೩೭॥

ವಿಚಿತ್ರಮುಕುಟಾ ರಾಮಾ ಕಾಮದಾತಾ ಪ್ರಕೀರ್ತಿತಾ ।\\
ಪಿತಾಂಬರಧರಾ ದಿವ್ಯವಿಭೂಷಣ ವಿಭೂಷಿತಾ ॥೩೮॥

ದಿವ್ಯಾಖ್ಯಾ ಸೋಮವದನಾ ಜಗತ್ಸಂಸೃಷ್ಟಿವರ್ಜಿತಾ ।\\
ನಿರ್ಯಂತ್ರಾ ಯಂತ್ರವಾಹಸ್ಥಾ ನಂದಿನೀ ರುದ್ರಕಾಲಿಕಾ ॥೩೯॥

ಆದಿತ್ಯವರ್ಣಾ ಕೌಮಾರೀ ಮಯೂರವರವಾಹಿನೀ ।\\
ಪದ್ಮಾಸನಗತಾ ಗೌರೀ ಮಹಾಕಾಲೀ ಸುರಾರ್ಚಿತಾ ॥೪೦॥

ಅದಿತಿರ್ನಿಯತಾ ರೌದ್ರೀ ಪದ್ಮಗರ್ಭಾ ವಿವಾಹನಾ ।\\
ವಿರೂಪಾಕ್ಷಾ ಕೇಶಿವಾಹಾ ಗುಹಾಪುರನಿವಾಸಿನೀ ॥೪೧॥

ಮಹಾಫಲಾಽನವದ್ಯಾಂಗೀ ಕಾಮರೂಪಾ ಸರಿದ್ವರಾ ।\\
ಭಾಸ್ವದ್ರೂಪಾ ಮುಕ್ತಿದಾತ್ರೀ ಪ್ರಣತಕ್ಲೇಶಭಂಜನಾ ॥೪೨॥

ಕೌಶಿಕೀ ಗೋಮಿನೀ ರಾತ್ರಿಸ್ತ್ರಿದಶಾರಿವಿನಾಶಿನೀ ।\\
ಬಹುರೂಪಾ ಸುರೂಪಾ ಚ ವಿರೂಪಾ ರೂಪವರ್ಜಿತಾ ॥೪೩॥

ಭಕ್ತಾರ್ತಿಶಮನಾ ಭವ್ಯಾ ಭವಭಾವವಿನಾಶಿನೀ ।\\
ಸರ್ವಜ್ಞಾನಪರೀತಾಂಗೀ ಸರ್ವಾಸುರವಿಮರ್ದಿಕಾ ॥೪೪॥

ಪಿಕಸ್ವನೀ ಸಾಮಗೀತಾ ಭವಾಂಕನಿಲಯಾ ಪ್ರಿಯಾ ।\\
ದೀಕ್ಷಾ ವಿದ್ಯಾಧರೀ ದೀಪ್ತಾ ಮಹೇಂದ್ರಾಹಿತಪಾತಿನೀ ॥೪೫॥

ಸರ್ವದೇವಮಯಾ ದಕ್ಷಾ ಸಮುದ್ರಾಂತರವಾಸಿನೀ ।\\
ಅಕಲಂಕಾ ನಿರಾಧಾರಾ ನಿತ್ಯಸಿದ್ಧಾ ನಿರಾಮಯಾ ॥೪೬॥

ಕಾಮಧೇನುಬೃಹದ್ಗರ್ಭಾ ಧೀಮತೀ ಮೌನನಾಶಿನೀ ।\\
ನಿಃಸಂಕಲ್ಪಾ ನಿರಾತಂಕಾ ವಿನಯಾ ವಿನಯಪ್ರದಾ ॥೪೭॥

ಜ್ವಾಲಾಮಾಲಾ ಸಹಸ್ರಾಢ್ಯಾ ದೇವದೇವೀ ಮನೋಮಯಾ ।\\
ಸುಭಗಾ ಸುವಿಶುದ್ಧಾ ಚ ವಸುದೇವಸಮುದ್ಭವಾ ॥೪೮॥

ಮಹೇಂದ್ರೋಪೇಂದ್ರಭಗಿನೀ ಭಕ್ತಿಗಮ್ಯಾ ಪರಾವರಾ ।\\
ಜ್ಞಾನಜ್ಞೇಯಾ ಪರಾತೀತಾ ವೇದಾಂತವಿಷಯಾ ಮತಿಃ ॥೪೯॥

ದಕ್ಷಿಣಾ ದಾಹಿಕಾ ದಹ್ಯಾ ಸರ್ವಭೂತಹೃದಿಸ್ಥಿತಾ ।\\
ಯೋಗಮಾಯಾ ವಿಭಾಗಜ್ಞಾ ಮಹಾಮೋಹಾ ಗರೀಯಸೀ ॥೫೦॥

ಸಂಧ್ಯಾ ಸರ್ವಸಮುದ್ಭೂತಾ ಬ್ರಹ್ಮವೃಕ್ಷಾಶ್ರಿಯಾಽದಿತಿಃ ।\\
ಬೀಜಾಂಕುರಸಮುದ್ಭೂತಾ ಮಹಾಶಕ್ತಿರ್ಮಹಾಮತಿಃ ॥೫೧॥

ಖ್ಯಾತಿಃ ಪ್ರಜ್ಞಾವತೀ ಸಂಜ್ಞಾ ಮಹಾಭೋಗೀಂದ್ರಶಾಯಿನೀ ।\\
ಹೀಂಕೃತಿಃ ಶಂಕರೀ ಶಾಂತಿರ್ಗಂಧರ್ವಗಣಸೇವಿತಾ ॥೫೨॥

ವೈಶ್ವಾನರೀ ಮಹಾಶೂಲಾ ದೇವಸೇನಾ ಭವಪ್ರಿಯಾ ।\\
ಮಹಾರಾತ್ರೀ ಪರಾನಂದಾ ಶಚೀ ದುಃಸ್ವಪ್ನನಾಶಿನೀ ॥೫೩॥

ಈಡ್ಯಾ ಜಯಾ ಜಗದ್ಧಾತ್ರೀ ದುರ್ವಿಜ್ಞೇಯಾ ಸುರೂಪಿಣೀ ।\\
ಗುಹಾಂಬಿಕಾ ಗಣೋತ್ಪನ್ನಾ ಮಹಾಪೀಠಾ ಮರುತ್ಸುತಾ ॥೫೪॥

ಹವ್ಯವಾಹಾ ಭವಾನಂದಾ ಜಗದ್ಯೋನಿಃ ಪ್ರಕೀರ್ತಿತಾ ।\\
ಜಗನ್ಮಾತಾ ಜಗನ್ಮೃತ್ಯುರ್ಜರಾತೀತಾ ಚ ಬುದ್ಧಿದಾ ॥೫೫॥

ಸಿದ್ಧಿದಾತ್ರೀ ರತ್ನಗರ್ಭಾ ರತ್ನಗರ್ಭಾಶ್ರಯಾ ಪರಾ ।\\
ದೈತ್ಯಹಂತ್ರೀ ಸ್ವೇಷ್ಟದಾತ್ರೀ ಮಂಗಲೈಕಸುವಿಗ್ರಹಾ ॥೫೬॥

ಪುರುಷಾಂತರ್ಗತಾ ಚೈವ ಸಮಾಧಿಸ್ಥಾ ತಪಸ್ವಿನೀ ।\\
ದಿವಿಸ್ಥಿತಾ ತ್ರಿಣೇತ್ರಾ ಚ ಸರ್ವೇಂದ್ರಿಯಮನಾಧೃತಿಃ ॥೫೭॥

ಸರ್ವಭೂತಹೃದಿಸ್ಥಾ ಚ ತಥಾ ಸಂಸಾರತಾರಿಣೀ ।\\
ವೇದ್ಯಾ ಬ್ರಹ್ಮವಿವೇದ್ಯಾ ಚ ಮಹಾಲೀಲಾ ಪ್ರಕೀರ್ತಿತಾ ॥೫೮॥

ಬ್ರಾಹ್ಮಣಿಬೃಹತೀ ಬ್ರಾಹ್ಮೀ ಬ್ರಹ್ಮಭೂತಾಽಘಹಾರಿಣೀ ।\\
ಹಿರಣ್ಮಯೀ ಮಹಾದಾತ್ರೀ ಸಂಸಾರಪರಿವರ್ತಿಕಾ ॥೫೯॥

ಸುಮಾಲಿನೀ ಸುರೂಪಾ ಚ ಭಾಸ್ವಿನೀ ಧಾರಿಣೀ ತಥಾ ।\\
ಉನ್ಮೂಲಿನೀ ಸರ್ವಸಭಾ ಸರ್ವಪ್ರತ್ಯಯಸಾಕ್ಷಿಣೀ ॥೬೦॥

ಸುಸೌಮ್ಯಾ ಚಂದ್ರವದನಾ ತಾಂಡವಾಸಕ್ತಮಾನಸಾ ।\\
ಸತ್ತ್ವಶುದ್ಧಿಕರೀ ಶುದ್ಧಾ ಮಲತ್ರಯವಿನಾಶಿನೀ ॥೬೧॥

ಜಗತ್ತ್ತ್ರಯೀ ಜಗನ್ಮೂರ್ತಿಸ್ತ್ರಿಮೂರ್ತಿರಮೃತಾಶ್ರಯಾ ।\\
ವಿಮಾನಸ್ಥಾ ವಿಶೋಕಾ ಚ ಶೋಕನಾಶಿನ್ಯನಾಹತಾ ॥೬೨॥

ಹೇಮಕುಂಡಲಿನೀ ಕಾಲೀ ಪದ್ಮವಾಸಾ ಸನಾತನೀ ।\\
ಸದಾಕೀರ್ತಿಃ ಸರ್ವಭೂತಶಯಾ ದೇವೀ ಸತಾಂಪ್ರಿಯಾ ॥೬೩॥

ಬ್ರಹ್ಮಮೂರ್ತಿಕಲಾ ಚೈವ ಕೃತ್ತಿಕಾ ಕಂಜಮಾಲಿನೀ ।\\
ವ್ಯೋಮಕೇಶಾ ಕ್ರಿಯಾಶಕ್ತಿರಿಚ್ಛಾಶಕ್ತಿಃ ಪರಾಗತಿಃ ॥೬೪॥

ಕ್ಷೋಭಿಕಾ ಖಂಡಿಕಾಭೇದ್ಯಾ ಭೇದಾಭೇದವಿವರ್ಜಿತಾ ।\\
ಅಭಿನ್ನಾ ಭಿನ್ನಸಂಸ್ಥಾನಾ ವಶಿನೀ ವಂಶಧಾರಿಣೀ ॥೬೫॥

ಗುಹ್ಯಶಕ್ತಿರ್ಗುಹ್ಯತತ್ತ್ವಾ ಸರ್ವದಾ ಸರ್ವತೋಮುಖೀ ।\\
ಭಗಿನೀ ಚ ನಿರಾಧಾರಾ ನಿರಾಹಾರಾ ಪ್ರಕೀರ್ತಿತಾ ॥೬೬॥

ನಿರಂಕುಶಪದೋದ್ಭೂತಾ ಚಕ್ರಹಸ್ತಾ ವಿಶೋಧಿಕಾ ।\\
ಸ್ರಗ್ವಿಣೀ ಪದ್ಮಸಂಭೇದಕಾರಿಣೀ ಪರಿಕೀರ್ತಿತಾ ॥೬೭॥

ಪರಾವರವಿಧಾನಜ್ಞಾ ಮಹಾಪುರುಷಪೂರ್ವಜಾ ।\\
ಪರಾವರಜ್ಞಾ ವಿದ್ಯಾ ಚ ವಿದ್ಯುಜ್ಜಿಹ್ವಾ ಜಿತಾಶ್ರಯಾ ॥೬೮॥

ವಿದ್ಯಾಮಯೀ ಸಹಸ್ರಾಕ್ಷೀ ಸಹಸ್ರವದನಾತ್ಮಜಾ ।\\
ಸಹಸ್ರರಶ್ಮಿಃಸತ್ವಸ್ಥಾ ಮಹೇಶ್ವರಪದಾಶ್ರಯಾ ॥೬೯॥

ಜ್ವಾಲಿನೀ ಸನ್ಮಯಾ ವ್ಯಾಪ್ತಾ ಚಿನ್ಮಯಾ ಪದ್ಮಭೇದಿಕಾ ।\\
ಮಹಾಶ್ರಯಾ ಮಹಾಮಂತ್ರಾ ಮಹಾದೇವಮನೋರಮಾ ॥೭೦॥

ವ್ಯೋಮಲಕ್ಷ್ಮೀಃ ಸಿಂಹರಥಾ ಚೇಕಿತಾನಾಽಮಿತಪ್ರಭಾ ।\\
ವಿಶ್ವೇಶ್ವರೀ ಭಗವತೀ ಸಕಲಾ ಕಾಲಹಾರಿಣೀ ॥೭೧॥

ಸರ್ವವೇದ್ಯಾ ಸರ್ವಭದ್ರಾ ಗುಹ್ಯಾ ದೂಢಾ ಗುಹಾರಣೀ ।\\
ಪ್ರಲಯಾ ಯೋಗಧಾತ್ರೀ ಚ ಗಂಗಾ ವಿಶ್ವೇಶ್ವರೀ ತಥಾ ॥೭೨॥

ಕಾಮದಾ ಕನಕಾ ಕಾಂತಾ ಕಂಜಗರ್ಭಪ್ರಭಾ ತಥಾ ।\\
ಪುಣ್ಯದಾ ಕಾಲಕೇಶಾ ಚ ಭೋಕ್ತ್ತ್ರೀ ಪುಷ್ಕರಿಣೀ ತಥಾ ॥೭೩॥

ಸುರೇಶ್ವರೀ ಭೂತಿದಾತ್ರೀ ಭೂತಿಭೂಷಾ ಪ್ರಕೀರ್ತಿತಾ ।\\
ಪಂಚಬ್ರಹ್ಮಸಮುತ್ಪನ್ನಾ ಪರಮಾರ್ಥಾಽರ್ಥವಿಗ್ರಹಾ ॥೭೪॥

ವರ್ಣೋದಯಾ ಭಾನುಮೂರ್ತಿರ್ವಾಗ್ವಿಜ್ಞೇಯಾ ಮನೋಜವಾ ।\\
ಮನೋಹರಾ ಮಹೋರಸ್ಕಾ ತಾಮಸೀ ವೇದರೂಪಿಣೀ ॥೭೫॥

ವೇದಶಕ್ತಿರ್ವೇದಮಾತಾ ವೇದವಿದ್ಯಾಪ್ರಕಾಶಿನೀ ।\\
ಯೋಗೇಶ್ವರೇಶ್ವರೀ ಮಾಯಾ ಮಹಾಶಕ್ತಿರ್ಮಹಾಮಯೀ ॥೭೬॥

ವಿಶ್ವಾಂತಃಸ್ಥಾ ವಿಯನ್ಮೂರ್ತಿರ್ಭಾರ್ಗವೀ ಸುರಸುಂದರೀ ।\\
ಸುರಭಿರ್ನಂದಿನೀ ವಿದ್ಯಾ ನಂದಗೋಪತನೂದ್ಭವಾ ॥೭೭॥

ಭಾರತೀ ಪರಮಾನಂದಾ ಪರಾವರವಿಭೇದಿಕಾ ।\\
ಸರ್ವಪ್ರಹರಣೋಪೇತಾ ಕಾಮ್ಯಾ ಕಾಮೇಶ್ವರೇಶ್ವರೀ ॥೭೮॥

ಅನಂತಾನಂದವಿಭವಾ ಹೃಲ್ಲೇಖಾ ಕನಕಪ್ರಭಾ ।\\
ಕೂಷ್ಮಾಂಡಾ ಧನರತ್ನಾಢ್ಯಾ ಸುಗಂಧಾ ಗಂಧದಾಯಿನೀ ॥೭೯॥

ತ್ರಿವಿಕ್ರಮಪದೋದ್ಭೂತಾ ಚತುರಾಸ್ಯಾ ಶಿವೋದಯಾ ।\\
ಸುದುರ್ಲಭಾ ಧನಾಧ್ಯಕ್ಷಾ ಧನ್ಯಾ ಪಿಂಗಲಲೋಚನಾ ॥೮೦॥

ಶಾಂತಾ ಪ್ರಭಾಸ್ವರೂಪಾ ಚ ಪಂಕಜಾಯತಲೋಚನಾ ।\\
ಇಂದ್ರಾಕ್ಷೀ ಹೃದಯಾಂತಃಸ್ಥಾ ಶಿವಾ ಮಾತಾ ಚ ಸತ್ಕ್ರಿಯಾ ॥೮೧॥

ಗಿರಿಜಾ ಚ ಸುಗೂಢಾ ಚ ನಿತ್ಯಪುಷ್ಟಾ ನಿರಂತರಾ ।\\
ದುರ್ಗಾ ಕಾತ್ಯಾಯನೀ ಚಂಡೀ ಚಂದ್ರಿಕಾ ಕಾಂತವಿಗ್ರಹಾ ॥೮೨॥

ಹಿರಣ್ಯವರ್ಣಾ ಜಗತೀ ಜಗದ್ಯಂತ್ರಪ್ರವರ್ತಿಕಾ ।\\
ಮಂದರಾದ್ರಿನಿವಾಸಾ ಚ ಶಾರದಾ ಸ್ವರ್ಣಮಾಲಿನೀ ॥೮೩॥

ರತ್ನಮಾಲಾ ರತ್ನಗರ್ಭಾ ವ್ಯುಷ್ಟಿರ್ವಿಶ್ವಪ್ರಮಾಥಿನೀ ।\\
ಪದ್ಮಾನಂದಾ ಪದ್ಮನಿಭಾ ನಿತ್ಯಪುಷ್ಟಾ ಕೃತೋದ್ಭವಾ ॥೮೪॥

ನಾರಾಯಣೀ ದುಷ್ಟಶಿಕ್ಷಾ ಸೂರ್ಯಮಾತಾ ವೃಷಪ್ರಿಯಾ ।\\
ಮಹೇಂದ್ರಭಗಿನೀ ಸತ್ಯಾ ಸತ್ಯಭಾಷಾ ಸುಕೋಮಲಾ ॥೮೫॥

ವಾಮಾ ಚ ಪಂಚತಪಸಾಂ ವರದಾತ್ರೀ ಪ್ರಕೀರ್ತಿತಾ ।\\
ವಾಚ್ಯವರ್ಣೇಶ್ವರೀ ವಿದ್ಯಾ ದುರ್ಜಯಾ ದುರತಿಕ್ರಮಾ ॥೮೬॥

ಕಾಲರಾತ್ರಿರ್ಮಹಾವೇಗಾ ವೀರಭದ್ರಪ್ರಿಯಾ ಹಿತಾ ।\\
ಭದ್ರಕಾಲೀ ಜಗನ್ಮಾತಾ ಭಕ್ತಾನಾಂ ಭದ್ರದಾಯಿನೀ ॥೮೭॥

ಕರಾಲಾ ಪಿಂಗಲಾಕಾರಾ ಕಾಮಭೇತ್ತ್ರೀ ಮಹಾಮನಾಃ ।\\
ಯಶಸ್ವಿನೀ ಯಶೋದಾ ಚ ಷಡಧ್ವಪರಿವರ್ತಿಕಾ ॥೮೮॥

ಶಂಖಿನೀ ಪದ್ಮಿನೀ ಸಂಖ್ಯಾ ಸಾಂಖ್ಯಯೋಗಪ್ರವರ್ತಿಕಾ ।\\
ಚೈತ್ರಾದಿರ್ವತ್ಸರಾರೂಢಾ ಜಗತ್ಸಂಪೂರಣೀಂದ್ರಜಾ ॥೮೯॥

ಶುಂಭಘ್ನೀ ಖೇಚರಾರಾಧ್ಯಾ ಕಂಬುಗ್ರೀವಾ ಬಲೀಡಿತಾ ।\\
ಖಗಾರೂಢಾ ಮಹೈಶ್ವರ್ಯಾ ಸುಪದ್ಮನಿಲಯಾ ತಥಾ ॥೯೦॥

ವಿರಕ್ತಾ ಗರುಡಸ್ಥಾ ಚ ಜಗತೀಹೃದ್ಗುಹಾಶ್ರಯಾ ।\\
ಶುಂಭಾದಿಮಥನಾ ಭಕ್ತಹೃದ್ಗಹ್ವರನಿವಾಸಿನೀ ॥೯೧॥

ಜಗತ್ತ್ತ್ರಯಾರಣೀ ಸಿದ್ಧಸಂಕಲ್ಪಾ ಕಾಮದಾ ತಥಾ ।\\
ಸರ್ವವಿಜ್ಞಾನದಾತ್ರೀ ಚಾನಲ್ಪಕಲ್ಮಷಹಾರಿಣೀ ॥೯೨॥

ಸಕಲೋಪನಿಷದ್ಗಮ್ಯಾ ದುಷ್ಟದುಷ್ಪ್ರೇಕ್ಷ್ಯಸತ್ತಮಾ ।\\
ಸದ್ವೃತಾ ಲೋಕಸಂವ್ಯಾಪ್ತಾ ತುಷ್ಟಿಃ ಪುಷ್ಟಿಃ ಕ್ರಿಯಾವತೀ ॥೯೩॥

ವಿಶ್ವಾಮರೇಶ್ವರೀ ಚೈವ ಭುಕ್ತಿಮುಕ್ತಿಪ್ರದಾಯಿನೀ ।\\
ಶಿವಾಧೃತಾ ಲೋಹಿತಾಕ್ಷೀ ಸರ್ಪಮಾಲಾವಿಭೂಷಣಾ ॥೯೪॥

ನಿರಾನಂದಾ ತ್ರಿಶೂಲಾಸಿಧನುರ್ಬಾಣಾದಿಧಾರಿಣೀ ।\\
ಅಶೇಷಧ್ಯೇಯಮೂರ್ತಿಶ್ಚ ದೇವತಾನಾಂ ಚ ದೇವತಾ ॥೯೫॥

ವರಾಂಬಿಕಾ ಗಿರೇಃ ಪುತ್ರೀ ನಿಶುಂಭವಿನಿಪಾತಿನೀ ।\\
ಸುವರ್ಣಾ ಸ್ವರ್ಣಲಸಿತಾಽನಂತವರ್ಣಾ ಸದಾಧೃತಾ ॥೯೬॥

ಶಾಂಕರೀ ಶಾಂತಹೃದಯಾ ಅಹೋರಾತ್ರವಿಧಾಯಿಕಾ ।\\
ವಿಶ್ವಗೋಪ್ತ್ರೀ ಗೂಢರೂಪಾ ಗುಣಪೂರ್ಣಾ ಚ ಗಾರ್ಗ್ಯಜಾ ॥೯೭॥

ಗೌರೀ ಶಾಕಂಭರೀ ಸತ್ಯಸಂಧಾ ಸಂಧ್ಯಾತ್ರಯೀಧೃತಾ ।\\
ಸರ್ವಪಾಪವಿನಿರ್ಮುಕ್ತಾ ಸರ್ವಬಂಧವಿವರ್ಜಿತಾ ॥೯೮॥

ಸಾಂಖ್ಯಯೋಗಸಮಾಖ್ಯಾತಾ ಅಪ್ರಮೇಯಾ ಮುನೀಡಿತಾ ।\\
ವಿಶುದ್ಧಸುಕುಲೋದ್ಭೂತಾ ಬಿಂದುನಾದಸಮಾದೃತಾ ॥೯೯॥

ಶಂಭುವಾಮಾಂಕಗಾ ಚೈವ ಶಶಿತುಲ್ಯನಿಭಾನನಾ ।\\
ವನಮಾಲಾವಿರಾಜಂತೀ ಅನಂತಶಯನಾದೃತಾ ॥೧೦೦॥

ನರನಾರಾಯಣೋದ್ಭೂತಾ ನಾರಸಿಂಹೀ ಪ್ರಕೀರ್ತಿತಾ ।\\
ದೈತ್ಯಪ್ರಮಾಥಿನೀ ಶಂಖಚಕ್ರಪದ್ಮಗದಾಧರಾ ॥೧೦೧॥

ಸಂಕರ್ಷಣಸಮುತ್ಪನ್ನಾ ಅಂಬಿಕಾ ಸಜ್ಜನಾಶ್ರಯಾ ।\\
ಸುವೃತಾ ಸುಂದರೀ ಚೈವ ಧರ್ಮಕಾಮಾರ್ಥದಾಯಿನೀ ॥೧೦೨॥

ಮೋಕ್ಷದಾ ಭಕ್ತಿನಿಲಯಾ ಪುರಾಣಪುರುಷಾದೃತಾ ।\\
ಮಹಾವಿಭೂತಿದಾಽಽರಾಧ್ಯಾ ಸರೋಜನಿಲಯಾಽಸಮಾ ॥೧೦೩॥

ಅಷ್ಟಾದಶಭುಜಾಽನಾದಿರ್ನೀಲೋತ್ಪಲದಲಾಕ್ಷಿಣೀ ।\\
ಸರ್ವಶಕ್ತಿಸಮಾರೂಢಾ ಧರ್ಮಾಧರ್ಮವಿವರ್ಜಿತಾ ॥೧೦೪॥

ವೈರಾಗ್ಯಜ್ಞಾನನಿರತಾ ನಿರಾಲೋಕಾ ನಿರಿಂದ್ರಿಯಾ ।\\
ವಿಚಿತ್ರಗಹನಾಧಾರಾ ಶಾಶ್ವತಸ್ಥಾನವಾಸಿನೀ ॥೧೦೫॥

ಜ್ಞಾನೇಶ್ವರೀ ಪೀತಚೇಲಾ ವೇದವೇದಾಂಗಪಾರಗಾ ।\\
ಮನಸ್ವಿನೀ ಮನ್ಯುಮಾತಾ ಮಹಾಮನ್ಯುಸಮುದ್ಭವಾ ॥೧೦೬॥

ಅಮನ್ಯುರಮೃತಾಸ್ವಾದಾ ಪುರಂದರಪರಿಷ್ಟುತಾ ।\\
ಅಶೋಚ್ಯಾ ಭಿನ್ನವಿಷಯಾ ಹಿರಣ್ಯರಜತಪ್ರಿಯಾ ॥೧೦೭॥

ಹಿರಣ್ಯಜನನೀ ಭೀಮಾ ಹೇಮಾಭರಣಭೂಷಿತಾ ।\\
ವಿಭ್ರಾಜಮಾನಾ ದುರ್ಜ್ಞೇಯಾ ಜ್ಯೋತಿಷ್ಟೋಮಫಲಪ್ರದಾ ॥೧೦೮॥

ಮಹಾನಿದ್ರಾಸಮುತ್ಪತ್ತಿರನಿದ್ರಾ ಸತ್ಯದೇವತಾ ।\\
ದೀರ್ಘಾ ಕಕುದ್ಮಿನೀ ಪಿಂಗಜಟಾಧಾರಾ ಮನೋಜ್ಞಧೀಃ ॥೧೦೯॥

ಮಹಾಶ್ರಯಾ ರಮೋತ್ಪನ್ನಾ ತಮಃಪಾರೇ ಪ್ರತಿಷ್ಠಿತಾ ।\\
ತ್ರಿತತ್ತ್ವಮಾತಾ ತ್ರಿವಿಧಾ ಸುಸೂಕ್ಷ್ಮಾ ಪದ್ಮಸಂಶ್ರಯಾ ॥೧೧೦॥

ಶಾಂತ್ಯತೀತಕಲಾಽತೀತವಿಕಾರಾ ಶ್ವೇತಚೇಲಿಕಾ ।\\
ಚಿತ್ರಮಾಯಾ ಶಿವಜ್ಞಾನಸ್ವರೂಪಾ ದೈತ್ಯಮಾಥಿನೀ ॥೧೧೧॥

ಕಾಶ್ಯಪೀ ಕಾಲಸರ್ಪಾಭವೇಣಿಕಾ ಶಾಸ್ತ್ರಯೋನಿಕಾ ।\\
ತ್ರಯೀಮೂರ್ತಿಃ ಕ್ರಿಯಾಮೂರ್ತಿಶ್ಚತುರ್ವರ್ಗಾ ಚ ದರ್ಶಿನೀ ॥೧೧೨॥

ನಾರಾಯಣೀ ನರೋತ್ಪನ್ನಾ ಕೌಮುದೀ ಕಾಂತಿಧಾರಿಣೀ ।\\
ಕೌಶಿಕೀ ಲಲಿತಾ ಲೀಲಾ ಪರಾವರವಿಭಾವಿನೀ ॥೧೧೩॥

ವರೇಣ್ಯಾಽದ್ಭುತಮಹಾತ್ಮ್ಯಾ ವಡವಾ ವಾಮಲೋಚನಾ ।\\
ಸುಭದ್ರಾ ಚೇತನಾರಾಧ್ಯಾ ಶಾಂತಿದಾ ಶಾಂತಿವರ್ಧಿನೀ ॥೧೧೪॥

ಜಯಾದಿಶಕ್ತಿಜನನೀ ಶಕ್ತಿಚಕ್ರಪ್ರವರ್ತಿಕಾ ।\\
ತ್ರಿಶಕ್ತಿಜನನೀ ಜನ್ಯಾ ಷಟ್ಸೂತ್ರಪರಿವರ್ಣಿತಾ ॥೧೧೫॥

ಸುಧೌತಕರ್ಮಣಾಽಽರಾಧ್ಯಾ ಯುಗಾಂತದಹನಾತ್ಮಿಕಾ ।\\
ಸಂಕರ್ಷಿಣೀ ಜಗದ್ಧಾತ್ರೀ ಕಾಮಯೋನಿಃ ಕಿರೀಟಿನೀ ॥೧೧೬॥

ಐಂದ್ರೀ ತ್ರೈಲೋಕ್ಯನಮಿತಾ ವೈಷ್ಣವೀ ಪರಮೇಶ್ವರೀ ।\\
ಪ್ರದ್ಯುಮ್ನಜನನೀ ಬಿಂಬಸಮೋಷ್ಠೀ ಪದ್ಮಲೋಚನಾ ॥೧೧೭॥

ಮದೋತ್ಕಟಾ ಹಂಸಗತಿಃ ಪ್ರಚಂಡಾ ಚಂಡವಿಕ್ರಮಾ ।\\
ವೃಷಾಧೀಶಾ ಪರಾತ್ಮಾ ಚ ವಿಂಧ್ಯಾ ಪರ್ವತವಾಸಿನೀ ॥೧೧೮॥

ಹಿಮವನ್ಮೇರುನಿಲಯಾ ಕೈಲಾಸಪುರವಾಸಿನೀ ।\\
ಚಾಣೂರಹಂತ್ರೀ ನೀತಿಜ್ಞಾ ಕಾಮರೂಪಾ ತ್ರಯೀತನುಃ ॥೧೧೯॥

ವ್ರತಸ್ನಾತಾ ಧರ್ಮಶೀಲಾ ಸಿಂಹಾಸನನಿವಾಸಿನೀ ।\\
ವೀರಭದ್ರಾದೃತಾ ವೀರಾ ಮಹಾಕಾಲಸಮುದ್ಭವಾ ॥೧೨೦॥

ವಿದ್ಯಾಧರಾರ್ಚಿತಾ ಸಿದ್ಧಸಾಧ್ಯಾರಾಧಿತಪಾದುಕಾ ।\\
ಶ್ರದ್ಧಾತ್ಮಿಕಾ ಪಾವನೀ ಚ ಮೋಹಿನೀ ಅಚಲಾತ್ಮಿಕಾ ॥೧೨೧॥

ಮಹಾದ್ಭುತಾ ವಾರಿಜಾಕ್ಷೀ ಸಿಂಹವಾಹನಗಾಮಿನೀ ।\\
ಮನೀಷಿಣೀ ಸುಧಾವಾಣೀ ವೀಣಾವಾದನತತ್ಪರಾ ॥೧೨೨॥

ಶ್ವೇತವಾಹನಿಷೇವ್ಯಾ ಚ ಲಸನ್ಮತಿರರುಂಧತೀ ।\\
ಹಿರಣ್ಯಾಕ್ಷೀ ತಥಾ ಚೈವ ಮಹಾನಂದಪ್ರದಾಯಿನೀ ॥೧೨೩॥

ವಸುಪ್ರಭಾ ಸುಮಾಲ್ಯಾಪ್ತಕಂಧರಾ ಪಂಕಜಾನನಾ ।\\
ಪರಾವರಾ ವರಾರೋಹಾ ಸಹಸ್ರನಯನಾರ್ಚಿತಾ ॥೧೨೪॥

ಶ್ರೀರೂಪಾ ಶ್ರೀಮತೀ ಶ್ರೇಷ್ಠಾ ಶಿವನಾಮ್ನೀ ಶಿವಪ್ರಿಯಾ ।\\
ಶ್ರೀಪ್ರದಾ ಶ್ರಿತಕಲ್ಯಾಣಾ ಶ್ರೀಧರಾರ್ಧಶರೀರಿಣೀ ॥೧೨೫॥

ಶ್ರೀಕಲಾಽನಂತದೃಷ್ಟಿಶ್ಚ ಹ್ಯಕ್ಷುದ್ರಾಽಽರಾತಿಸೂದನೀ ।\\
ರಕ್ತಬೀಜನಿಹಂತ್ರೀ ಚ ದೈತ್ಯಸಂಗವಿಮರ್ದಿನೀ ॥೧೨೬॥

ಸಿಂಹಾರೂಢಾ ಸಿಂಹಿಕಾಸ್ಯಾ ದೈತ್ಯಶೋಣಿತಪಾಯಿನೀ ।\\
ಸುಕೀರ್ತಿಸಹಿತಾಚ್ಛಿನ್ನಸಂಶಯಾ ರಸವೇದಿನೀ ॥೧೨೭॥

ಗುಣಾಭಿರಾಮಾ ನಾಗಾರಿವಾಹನಾ ನಿರ್ಜರಾರ್ಚಿತಾ ।\\
ನಿತ್ಯೋದಿತಾ ಸ್ವಯಂಜ್ಯೋತಿಃ ಸ್ವರ್ಣಕಾಯಾ ಪ್ರಕೀರ್ತಿತಾ ॥೧೨೮॥

ವಜ್ರದಂಡಾಂಕಿತಾ ಚೈವ ತಥಾಽಮೃತಸಂಜೀವಿನೀ ।\\
ವಜ್ರಚ್ಛನ್ನಾ ದೇವದೇವೀ ವರವಜ್ರಸ್ವವಿಗ್ರಹಾ ॥೧೨೯॥

ಮಾಂಗಲ್ಯಾ ಮಂಗಲಾತ್ಮಾ ಚ ಮಾಲಿನೀ ಮಾಲ್ಯಧಾರಿಣೀ ।\\
ಗಂಧರ್ವೀ ತರುಣೀ ಚಾಂದ್ರೀ ಖಡ್ಗಾಯುಧಧರಾ ತಥಾ ॥೧೩೦॥

ಸೌದಾಮಿನೀ ಪ್ರಜಾನಂದಾ ತಥಾ ಪ್ರೋಕ್ತಾ ಭೃಗೂದ್ಭವಾ ।\\
ಏಕಾನಂಗಾ ಚ ಶಾಸ್ತ್ರಾರ್ಥಕುಶಲಾ ಧರ್ಮಚಾರಿಣೀ ॥೧೩೧॥

ಧರ್ಮಸರ್ವಸ್ವವಾಹಾ ಚ ಧರ್ಮಾಧರ್ಮವಿನಿಶ್ಚಯಾ ।\\
ಧರ್ಮಶಕ್ತಿರ್ಧರ್ಮಮಯಾ ಧಾರ್ಮಿಕಾನಾಂ ಶಿವಪ್ರದಾ ॥೧೩೨॥

ವಿಧರ್ಮಾ ವಿಶ್ವಧರ್ಮಜ್ಞಾ ಧರ್ಮಾರ್ಥಾಂತರವಿಗ್ರಹಾ ।\\
ಧರ್ಮವರ್ಷ್ಮಾ ಧರ್ಮಪೂರ್ವಾ ಧರ್ಮಪಾರಂಗತಾಂತರಾ ॥೧೩೩॥

ಧರ್ಮೋಪದೇಷ್ಟ್ರೀ ಧರ್ಮಾತ್ಮಾ ಧರ್ಮಗಮ್ಯಾ ಧರಾಧರಾ ।\\
ಕಪಾಲಿನೀ ಶಾಕಲಿನೀ ಕಲಾಕಲಿತವಿಗ್ರಹಾ ॥೧೩೪॥

ಸರ್ವಶಕ್ತಿವಿಮುಕ್ತಾ ಚ ಕರ್ಣಿಕಾರಧರಾಽಕ್ಷರಾ।\\
ಕಂಸಪ್ರಾಣಹರಾ ಚೈವ ಯುಗಧರ್ಮಧರಾ ತಥಾ ॥೧೩೫॥

ಯುಗಪ್ರವರ್ತಿಕಾ ಪ್ರೋಕ್ತಾ ತ್ರಿಸಂಧ್ಯಾ ಧ್ಯೇಯವಿಗ್ರಹಾ ।\\
ಸ್ವರ್ಗಾಪವರ್ಗದಾತ್ರೀ ಚ ತಥಾ ಪ್ರತ್ಯಕ್ಷದೇವತಾ ॥೧೩೬॥

ಆದಿತ್ಯಾ ದಿವ್ಯಗಂಧಾ ಚ ದಿವಾಕರನಿಭಪ್ರಭಾ ।\\
ಪದ್ಮಾಸನಗತಾ ಪ್ರೋಕ್ತಾ ಖಡ್ಗಬಾಣಶರಾಸನಾ ॥೧೩೭॥

ಶಿಷ್ಟಾ ವಿಶಿಷ್ಟಾ ಶಿಷ್ಟೇಷ್ಟಾ ಶಿಷ್ಟಶ್ರೇಷ್ಠಪ್ರಪೂಜಿತಾ ।\\
ಶತರೂಪಾ ಶತಾವರ್ತಾ ವಿತತಾ ರಾಸಮೋದಿನೀ ॥೧೩೮॥

ಸೂರ್ಯೇಂದುನೇತ್ರಾ ಪ್ರದ್ಯುಮ್ನಜನನೀ ಸುಷ್ಠುಮಾಯಿನೀ ।\\
ಸೂರ್ಯಾಂತರಸ್ಥಿತಾ ಚೈವ ಸತ್ಪ್ರತಿಷ್ಠತವಿಗ್ರಹಾ ॥೧೩೯॥

ನಿವೃತ್ತಾ ಪ್ರೋಚ್ಯತೇ ಜ್ಞಾನಪಾರಗಾ ಪರ್ವತಾತ್ಮಜಾ ।\\
ಕಾತ್ಯಾಯನೀ ಚಂಡಿಕಾ ಚ ಚಂಡೀ ಹೈಮವತೀ ತಥಾ ॥೧೪೦॥

ದಾಕ್ಷಾಯಣೀ ಸತೀ ಚೈವ ಭವಾನೀ ಸರ್ವಮಂಗಲಾ ।\\
ಧೂಮ್ರಲೋಚನಹಂತ್ರೀ ಚ ಚಂಡಮುಂಡವಿನಾಶಿನೀ ॥೧೪೧॥

ಯೋಗನಿದ್ರಾ ಯೋಗಭದ್ರಾ ಸಮುದ್ರತನಯಾ ತಥಾ ।\\
ದೇವಪ್ರಿಯಂಕರೀ ಶುದ್ಧಾ ಭಕ್ತಭಕ್ತಿಪ್ರವರ್ಧಿನೀ ॥೧೪೨॥

ತ್ರಿಣೇತ್ರಾ ಚಂದ್ರಮುಕುಟಾ ಪ್ರಮಥಾರ್ಚಿತಪಾದುಕಾ ।\\
ಅರ್ಜುನಾಭೀಷ್ಟದಾತ್ರೀ ಚ ಪಾಂಡವಪ್ರಿಯಕಾರಿಣೀ ॥೧೪೩॥

ಕುಮಾರಲಾಲನಾಸಕ್ತಾ ಹರಬಾಹೂಪಧಾನಿಕಾ ।\\
ವಿಘ್ನೇಶಜನನೀ ಭಕ್ತವಿಘ್ನಸ್ತೋಮಪ್ರಹಾರಿಣೀ ॥೧೪೪॥

ಸುಸ್ಮಿತೇಂದುಮುಖೀ ನಮ್ಯಾ ಜಯಾಪ್ರಿಯಸಖೀ ತಥಾ ।\\
ಅನಾದಿನಿಧನಾ ಪ್ರೇಷ್ಠಾ ಚಿತ್ರಮಾಲ್ಯಾನುಲೇಪನಾ ॥೧೪೫॥

ಕೋಟಿಚಂದ್ರಪ್ರತೀಕಾಶಾ ಕೂಟಜಾಲಪ್ರಮಾಥಿನೀ ।\\
ಕೃತ್ಯಾಪ್ರಹಾರಿಣೀ ಚೈವ ಮಾರಣೋಚ್ಚಾಟನೀ ತಥಾ ॥೧೪೬॥

ಸುರಾಸುರಪ್ರವಂದ್ಯಾಂಘ್ರಿರ್ಮೋಹಘ್ನೀ ಜ್ಞಾನದಾಯಿನೀ ।\\
ಷಡ್ವೈರಿನಿಗ್ರಹಕರೀ ವೈರಿವಿದ್ರಾವಿಣೀ ತಥಾ ॥೧೪೭॥

ಭೂತಸೇವ್ಯಾ ಭೂತದಾತ್ರೀ ಭೂತಪೀಡಾವಿಮರ್ದಿಕಾ ।\\
ನಾರದಸ್ತುತಚಾರಿತ್ರಾ ವರದೇಶಾ ವರಪ್ರದಾ ॥೧೪೮॥

ವಾಮದೇವಸ್ತುತಾ ಚೈವ ಕಾಮದಾ ಸೋಮಶೇಖರಾ ।\\
ದಿಕ್ಪಾಲಸೇವಿತಾ ಭವ್ಯಾ ಭಾಮಿನೀ ಭಾವದಾಯಿನೀ ॥೧೪೯॥

ಸ್ತ್ರೀಸೌಭಾಗ್ಯಪ್ರದಾತ್ರೀ ಚ ಭೋಗದಾ ರೋಗನಾಶಿನೀ ।\\
ವ್ಯೋಮಗಾ ಭೂಮಿಗಾ ಚೈವ ಮುನಿಪೂಜ್ಯಪದಾಂಬುಜಾ ।\\
ವನದುರ್ಗಾ ಚ ದುರ್ಬೋಧಾ ಮಹಾದುರ್ಗಾ ಪ್ರಕೀರ್ತಿತಾ ॥೧೫೦॥

ಫಲಶ್ರುತಿಃ ॥\\
ಇತೀದಂ ಕೀರ್ತಿದಂ ಭದ್ರ ದುರ್ಗಾನಾಮಸಹಸ್ರಕಂ ।\\
ತ್ರಿಸಂಧ್ಯಂ ಯಃ ಪಠೇನ್ನಿತ್ಯಂ ತಸ್ಯ ಲಕ್ಷ್ಮೀಃ ಸ್ಥಿರಾ ಭವೇತ್ ॥೧॥

ಗ್ರಹಭೂತಪಿಶಾಚಾದಿಪೀಡಾ ನಶ್ಯತ್ಯಸಂಶಯಂ ।\\
ಬಾಲಗ್ರಹಾದಿಪೀಡಾಯಾಃ ಶಾಂತಿರ್ಭವತಿ ಕೀರ್ತನಾತ್ ॥೨॥

ಮಾರಿಕಾದಿಮಹಾರೋಗೇ ಪಠತಾಂ ಸೌಖ್ಯದಂ ನೃಣಾಂ ।\\
ವ್ಯವಹಾರೇ ಚ ಜಯದಂ ಶತ್ರುಬಾಧಾನಿವಾರಕಂ ॥೩॥

ದಂಪತ್ಯೋಃ ಕಲಹೇ ಪ್ರಾಪ್ತೇ ಮಿಥಃ ಪ್ರೇಮಾಭಿವರ್ಧಕಂ ।\\
ಆಯುರಾರೋಗ್ಯದಂ ಪುಂಸಾಂ ಸರ್ವಸಂಪತ್ಪ್ರದಾಯಕಂ ॥೪॥

ವಿದ್ಯಾಭಿವರ್ಧಕಂ ನಿತ್ಯಂ ಪಠತಾಮರ್ಥಸಾಧಕಂ ।\\
ಶುಭದಂ ಶುಭಕಾರ್ಯೇಷು ಪಠತಾಂ ಶೃಣುತಾಮಪಿ ॥೫॥

ಯಃ ಪೂಜಯತಿ ದುರ್ಗಾಂ ತಾಂ ದುರ್ಗಾನಾಮಸಹಸ್ರಕೈಃ ।\\
ಪುಷ್ಪೈಃ ಕುಂಕುಮಸಮ್ಮಿಶ್ರೈಃ ಸ ತು ಯತ್ಕಾಂಕ್ಷತೇ ಹೃದಿ ॥೬॥

ತತ್ಸರ್ವಂ ಸಮವಾಪ್ನೋತಿ ನಾಸ್ತಿ ನಾಸ್ತ್ಯತ್ರ ಸಂಶಯಃ ।\\
ಯನ್ಮುಖೇ ಧ್ರಿಯತೇ ನಿತ್ಯಂ ದುರ್ಗಾನಾಮಸಹಸ್ರಕಂ ॥೭॥

ಕಿಂ ತಸ್ಯೇತರಮಂತ್ರೌಘೈಃ ಕಾರ್ಯಂ ಧನ್ಯತಮಸ್ಯ ಹಿ ।\\
ದುರ್ಗಾನಾಮಸಹಸ್ರಸ್ಯ ಪುಸ್ತಕಂ ಯದ್ಗೃಹೇ ಭವೇತ್ ॥೮॥

ನ ತತ್ರ ಗ್ರಹಭೂತಾದಿಬಾಧಾ ಸ್ಯಾನ್ಮಂಗಲಾಸ್ಪದೇ ।\\
ತದ್ಗೃಹಂ ಪುಣ್ಯದಂ ಕ್ಷೇತ್ರಂ ದೇವೀಸಾನ್ನಿಧ್ಯಕಾರಕಂ ॥೯॥

ಏತಸ್ಯ ಸ್ತೋತ್ರಮುಖ್ಯಸ್ಯ ಪಾಠಕಃ ಶ್ರೇಷ್ಠಮಂತ್ರವಿತ್ ।\\
ದೇವತಾಯಾಃ ಪ್ರಸಾದೇನ ಸರ್ವಪೂಜ್ಯಃ ಸುಖೀ ಭವೇತ್ ॥೧೦॥

ಇತ್ಯೇತನ್ನಗರಾಜೇನ ಕೀರ್ತಿತಂ ಮುನಿಸತ್ತಮ ।\\
ಗುಹ್ಯಾದ್ಗುಹ್ಯತರಂ ಸ್ತೋತ್ರಂ ತ್ವಯಿ ಸ್ನೇಹಾತ್ ಪ್ರಕೀರ್ತಿತಂ ॥೧೧॥

ಭಕ್ತಾಯ ಶ್ರದ್ಧಧಾನಾಯ ಕೇವಲಂ ಕೀರ್ತ್ಯತಾಮಿದಂ ।\\
ಹೃದಿ ಧಾರಯ ನಿತ್ಯಂ ತ್ವಂ ದೇವ್ಯನುಗ್ರಹಸಾಧಕಂ ॥೧೨॥
\authorline{॥ಇತಿ ಶ್ರೀಸ್ಕಾಂದಪುರಾಣೇ ಸ್ಕಂದನಾರದಸಂವಾದೇ ದುರ್ಗಾಸಹಸ್ರನಾಮಸ್ತೋತ್ರಂ ಸಂಪೂರ್ಣಂ॥}
%=============================================================================================
\section{ಶ್ರೀದುರ್ಗಾಷ್ಟೋತ್ತರಶತನಾಮಸ್ತೋತ್ರಂ}
\addcontentsline{toc}{section}{ಶ್ರೀದುರ್ಗಾಷ್ಟೋತ್ತರಶತನಾಮಸ್ತೋತ್ರಂ}

ಅಸ್ಯಶ್ರೀ ದುರ್ಗಾಷ್ಟೋತ್ತರಶತನಾಮಾಸ್ತೋತ್ರ ಮಾಲಾಮಂತ್ರಸ್ಯ ಬ್ರಹ್ಮವಿಷ್ಣುಮಹೇಶ್ವರಾಃ ಋಷಯಃ । ಅನುಷ್ಟುಪ್ಛಂದಃ । ಶ್ರೀದುರ್ಗಾಪರಮೇಶ್ವರೀ ದೇವತಾ । ಹ್ರಾಂ ಬೀಜಂ । ಹ್ರೀಂ ಶಕ್ತಿಃ । ಹ್ರೂಂ ಕೀಲಕಂ । ಸರ್ವಾಭೀಷ್ಟಸಿಧ್ಯರ್ಥೇ ಜಪಹೋಮಾರ್ಚನೇ ವಿನಿಯೋಗಃ ।\\

ಓಂ ಸತ್ಯಾ ಸಾಧ್ಯಾ ಭವಪ್ರೀತಾ ಭವಾನೀ ಭವಮೋಚನೀ ।\\
ಆರ್ಯಾ ದುರ್ಗಾ ಜಯಾ ಚಾದ್ಯಾ ತ್ರಿಣೇತ್ರಾಶೂಲಧಾರಿಣೀ ॥

ಪಿನಾಕಧಾರಿಣೀ ಚಿತ್ರಾ ಚಂಡಘಂಟಾ ಮಹಾತಪಾಃ ।\\
ಮನೋ ಬುದ್ಧಿ ರಹಂಕಾರಾ ಚಿದ್ರೂಪಾ ಚ ಚಿದಾಕೃತಿಃ ॥

ಅನಂತಾ ಭಾವಿನೀ ಭವ್ಯಾ ಹ್ಯಭವ್ಯಾ ಚ ಸದಾಗತಿಃ ।\\
ಶಾಂಭವೀ ದೇವಮಾತಾ ಚ ಚಿಂತಾ ರತ್ನಪ್ರಿಯಾ ತಥಾ ॥

ಸರ್ವವಿದ್ಯಾ ದಕ್ಷಕನ್ಯಾ ದಕ್ಷಯಜ್ಞವಿನಾಶಿನೀ ।\\
ಅಪರ್ಣಾಽನೇಕವರ್ಣಾ ಚ ಪಾಟಲಾ ಪಾಟಲಾವತೀ ॥

ಪಟ್ಟಾಂಬರಪರೀಧಾನಾ ಕಲಮಂಜೀರರಂಜಿನೀ ।\\
ಈಶಾನೀ ಚ ಮಹಾರಾಜ್ಞೀ ಹ್ಯಪ್ರಮೇಯಪರಾಕ್ರಮಾ ।\\
ರುದ್ರಾಣೀ ಕ್ರೂರರೂಪಾ ಚ ಸುಂದರೀ ಸುರಸುಂದರೀ ॥

ವನದುರ್ಗಾ ಚ ಮಾತಂಗೀ ಮತಂಗಮುನಿಕನ್ಯಕಾ ।\\
ಬ್ರಾಹ್ಮೀ ಮಾಹೇಶ್ವರೀ ಚೈಂದ್ರೀ ಕೌಮಾರೀ ವೈಷ್ಣವೀ ತಥಾ ॥

ಚಾಮುಂಡಾ ಚೈವ ವಾರಾಹೀ ಲಕ್ಷ್ಮೀಶ್ಚ ಪುರುಷಾಕೃತಿಃ ।\\
ವಿಮಲಾ ಜ್ಞಾನರೂಪಾ ಚ ಕ್ರಿಯಾ ನಿತ್ಯಾ ಚ ಬುದ್ಧಿದಾ ॥

ಬಹುಲಾ ಬಹುಲಪ್ರೇಮಾ ಮಹಿಷಾಸುರಮರ್ದಿನೀ ।\\
ಮಧುಕೈಟಭ ಹಂತ್ರೀ ಚ ಚಂಡಮುಂಡವಿನಾಶಿನೀ ॥

ಸರ್ವಶಾಸ್ತ್ರಮಯೀ ಚೈವ ಸರ್ವದಾನವಘಾತಿನೀ ।\\
ಅನೇಕಶಸ್ತ್ರಹಸ್ತಾ ಚ ಸರ್ವಶಸ್ತ್ರಾಸ್ತ್ರಧಾರಿಣೀ ॥

ಭದ್ರಕಾಲೀ ಸದಾಕನ್ಯಾ ಕೈಶೋರೀ ಯುವತಿರ್ಯತಿಃ ।\\
ಪ್ರೌಢಾಽಪ್ರೌಢಾ ವೃದ್ಧಮಾತಾ ಘೋರರೂಪಾ ಮಹೋದರೀ ॥

ಬಲಪ್ರದಾ ಘೋರರೂಪಾ ಮಹೋತ್ಸಾಹಾ ಮಹಾಬಲಾ ।\\
ಅಗ್ನಿಜ್ವಾಲಾ ರೌದ್ರಮುಖೀ ಕಾಲರಾತ್ರಿಃ ತಪಸ್ವಿನೀ ॥

ನಾರಾಯಣೀ ಮಹಾದೇವೀ ವಿಷ್ಣುಮಾಯಾ ಶಿವಾತ್ಮಿಕಾ ।\\
ಶಿವದೂತೀ ಕರಾಲೀ ಚ ಹ್ಯನಂತಾ ಪರಮೇಶ್ವರೀ ॥

ಕಾತ್ಯಾಯನೀ ಮಹಾವಿದ್ಯಾ ಮಹಾಮೇಧಾಸ್ವರೂಪಿಣೀ ।\\
ಗೌರೀ ಸರಸ್ವತೀ ಚೈವ ಸಾವಿತ್ರೀ ಬ್ರಹ್ಮವಾದಿನೀ ।\\
ಸರ್ವತತ್ತ್ವೈಕನಿಲಯಾ ವೇದಮಂತ್ರಸ್ವರೂಪಿಣೀ ॥

ಇದಂ ಸ್ತೋತ್ರಂ ಮಹಾದೇವ್ಯಾಃ ನಾಮ್ನಾಂ ಅಷ್ಟೋತ್ತರಂ ಶತಂ ।\\
ಯಃ ಪಠೇತ್ ಪ್ರಯತೋ ನಿತ್ಯಂ ಭಕ್ತಿಭಾವೇನ ಚೇತಸಾ ।\\
ಶತ್ರುಭ್ಯೋ ನ ಭಯಂ ತಸ್ಯ ತಸ್ಯ ಶತ್ರುಕ್ಷಯಂ ಭವೇತ್ ।\\
ಸರ್ವದುಃಖದರಿದ್ರಾಚ್ಚ ಸುಸುಖಂ ಮುಚ್ಯತೇ ಧ್ರುವಂ ॥

ವಿದ್ಯಾರ್ಥೀ ಲಭತೇ ವಿದ್ಯಾಂ ಧನಾರ್ಥೀ ಲಭತೇ ಧನಂ ।\\
ಕನ್ಯಾರ್ಥೀ ಲಭತೇ ಕನ್ಯಾಂ ಕನ್ಯಾ ಚ ಲಭತೇ ವರಂ ॥

ಋಣೀ ಋಣಾತ್ ವಿಮುಚ್ಯೇತ ಹ್ಯಪುತ್ರೋ ಲಭತೇ ಸುತಂ ।\\
ರೋಗಾದ್ವಿಮುಚ್ಯತೇ ರೋಗೀ ಸುಖಮತ್ಯಂತಮಶ್ನುತೇ ॥

ಭೂಮಿಲಾಭೋ ಭವೇತ್ತಸ್ಯ ಸರ್ವತ್ರ ವಿಜಯೀ ಭವೇತ್ ।\\
ಸರ್ವಾನ್ಕಾಮಾನವಾಪ್ನೋತಿ ಮಹಾದೇವೀಪ್ರಸಾದತಃ ॥

ಕುಂಕುಮೈಃ ಬಿಲ್ವಪತ್ರೈಶ್ಚ ಸುಗಂಧೈಃ ರಕ್ತಪುಷ್ಪಕೈಃ ।\\
ರಕ್ತಪತ್ರೈರ್ವಿಶೇಷೇಣ ಪೂಜಯನ್ಭದ್ರಮಶ್ನುತೇ ॥
%=======================================
ಗಂಗಾಷ್ಟೋತ್ತರಶತನಾಮಸ್ತೋತ್ರಂ

ಗಂಗಾ ವಿಷ್ಣುಪಾದಾಬ್ಜಸಂಭೂತಾ ಸುರನಿಮ್ನಗಾ ।\\
ಹಿಮಾಚಲೇಂದುನಯಾ ಗಿರಿಮಂಡಲಗಾಮಿನೀ ।\\।\\೧।\\।\\

ತಾರಕಾರಾತಿ ಜನನೀ ಸಗರಾತ್ಮಜತಾರಿಕಾ ।\\
ಸರಸ್ವತೀಸಮಾಯುಕ್ತಾ ಸುಘೋಷಾ ಸಿಂಧುಗಾಮಿನೀ ॥೨॥

ಭಾಗೀರಥೀ ಭಗವತೀ ಭಗೀರಥರಥಾನುಗಾ ।\\
ತ್ರಿವಿಕ್ರಮಪದೋದ್ಭೂತಾ ತ್ರಿಲೋಕಪಥಗಾಮಿನೀ ॥೩॥

ಕ್ಷೀರಶುಭ್ರಾ ಬಹುಕ್ಷೀರಾ ಕ್ಷೀರವೃಕ್ಷಸಮಾಕುಲಾ।\\
ತ್ರಿಲೋಚನಜಟಾವಾಸಾ ಋಣತ್ರಯವಿಮೋಚಿನೀ ॥೪॥

ತ್ರಿಪುರಾರಿಶಿರಶ್ಚೂಡಾ ಅವ್ಯಯಾ ಜಾಹ್ನವೀ ತಥಾ।\\
ನಿರಂಜನಾ ನಿತ್ಯಶುದ್ಧಾ ನಯನಾನಂದದಾಯಿನೀ॥೫॥

ಸಾವಿತ್ರೀ ಸಲಿಲಾವಾಸಾ ನರಭೀತಿಹರಾ ತಥಾ।\\
ಉಮಾಸಪತ್ನೀ ರಮ್ಯಾ ಚ ನೀರಜಾಲಿಪರಿಷ್ಕೃತಾ॥೬॥

ಅವ್ಯಕ್ತಾ ಬಿಂದುಸರಸಾ ಸಾಗರಾಂಬುಸಮೇಧಿನೀ।\\
ಶುಭ್ರಗಂಧೀ ಶುಭ್ರವಸ್ತ್ರಾ ಬೃಂದಾರಕಸಮಾಶ್ರಿತಾ॥೭॥

ಆಖಂಡಲವನಾವಾಸಾ ಖಂಡೇಂದುಕೃತಶೇಖರಾ।\\
ಅಮೃತಾಕಾರಸಲಿಲಾ ಲೀಲಾಲಿಂಗಿತಪರ್ವತಾ॥೮॥

ವಿರಿಂಚಿಕಲಶಾವಾಸಾ ತ್ರಿವೇಣೀ ತ್ರಿಗುಣಾತ್ಮಿಕಾ ।\\
ಸಂಗತಾಘೌಘಶಮನೀ ಶ್ರೀಮತೀಶೀಘ್ರಗಾಮಿನೀ॥೯॥

ಭೀತಿಹಂತ್ರೀ ಭಾಗ್ಯಮಾತಾ ಶಂಖದುಂದುಭಿನಿಸ್ವನಾ।\\
ಆನಂದಿನೀ ಶರಣ್ಯಾ ಚ ಭಿನ್ನಬ್ರಹ್ಮಾಂಡದರ್ಪಿಣೀ ॥೧೦॥

ಸಿದ್ಧಾ ಚ ಶಫರೀಪೂರ್ಣಾ ಅನಂತಾ ಶಶಿಶೇಖರಾ ।\\
ಭವಪ್ರಿಯಾ ಶಾಂಕರೀ ಚ ಭಗಮೂರ್ಧ್ನಿ ಕೃತಾಲಯಾ॥ ೧೧॥

ಸತ್ಯಸಂಧಪ್ರಿಯಾ ಹಂಸರೂಪಿಣೀ ಅತುಲಾ ತಥಾ ।\\
ಓಂಕಾರರೂಪಿಣೀ ಶೀತಾ ಶರಚ್ಚಂದ್ರನಿಭಾನನಾ ॥೧೨॥

ಭಗೀರಥಭೃತಾ ಪೂತಾ ಕ್ರೀಡಾಕಲ್ಲೋಲಕಾರಿಣೀ।\\
ಸ್ವರ್ಗಸೋಪಾನಸರಣಿಃ ಸರ್ವದೇವಸ್ರೂಪಿಣೀ ॥೧೩॥

ಅಂಭಃಪ್ರದಾ ದುಃಖಹಂತ್ರೀ ಶಾಂತಿಸಂತಾನಕಾರಿಣೀ ।\\
ದಾರಿದ್ರ್ಯಹಂತ್ರೀ ಶಿವದಾ ಸಂಸಾರವಿಷನಾಶಿನೀ ॥೧೪॥

ಪ್ರಯಾಗನಿಲಯಾ ಪಾಪಹಂತ್ರೀ ಪುಣ್ಯಾ ಪುರಾತನಾ ।\\
ಶರಣಾಗತದೀನಾರ್ತ ಪರಿತ್ರಾಣಪರಾಯಣಾ ॥೧೫॥

ಸುಮುಕ್ತಿದಾ ಪಾವನಾಂಗೀ ಸಿದ್ಧಯೋಗಿನಿಷೇವಿತಾ ।\\
ಜಗದ್ಧಿತಾ ಜಹ್ನುಪುತ್ರೀ ಜಂಗಮಾ ಪುಣ್ಯವಾಹಿನೀ ॥೧೬॥

ಜಯಾ ಪೂತತ್ರಿಭುವನಾ ಜಂಬೂದ್ವೀಪವಿಹಾರಿಣೀ ।\\
ಭವಪತ್ನೀ ಜಗನ್ಮಾತಾ ಅಜ್ಞಾನತಿಮಿರಾಪಹಾ ॥೧೭॥

ಭೀಷ್ಮಮಾತಾ ಪುಣ್ಯದಾ ಚ ಪೂರ್ಣಾರ್ಥಾ ನಗಪುತ್ರಿಕಾ ।\\
ಜಲರೂಪಾ ಚ ಪ್ರಥಿತಾ ತಾಪತ್ರಯ ವಿಮೋಚಿನೀ ॥೧೮॥

ಪುಲೋಮಜಾರ್ಚಿತಾ ಸಿದ್ಧಾ ರಮ್ಯರೂಪಧೃತಾ ತಥಾ ।\\
ವಿಮಲಾ ಜಂಗಮಾಧಾರಾ ಉಮಾಕರಸಮುದ್ಭವಾ ॥೧೯॥

 ॥ಇತಿ ಶ್ರೀ ಗಂಗಾಷ್ಟೋತ್ತರಶತನಾಮಸ್ತೋತ್ರಂ ಸಂಪೂರ್ಣಮ್॥
%============================================================================================
॥ ಶ್ರೀದಕ್ಷಿಣಾಮೂರ್ತಿಸಹಸ್ರನಾಮಸ್ತೋತ್ರಮ್ 2 ॥

ಶ್ರೀಗಣೇಶಾಯ ನಮಃ ।

ಆದಿದೇವೋ ದಯಾಸಿನ್ಧುರಖಿಲಾಗಮದೇಶಿಕಃ ।
ದಕ್ಷಿಣಾಮೂರ್ತಿರತುಲಃ ಶಿಕ್ಷಿತಾಸುರವಿಕ್ರಮಃ ॥ 1॥

ಕೈಲಾಸಶಿಖರೋಲ್ಲಾಸೀ ಕಮನೀಯನಿಜಾಕೃತಿಃ ।
ವೀರಾಸನಸಮಾಸೀನೋ ವೀಣಾಪುಸ್ತಲಸತ್ಕರಃ ॥ 2॥

ಅಕ್ಷಮಾಲಾಲಸತ್ಪಾಣಿಶ್ಚಿನ್ಮುದ್ರಿತಕರಾಮ್ಬುಜಃ ।
ಅಪಸ್ಮಾರೋಪರಿನ್ಯಸ್ತಸವ್ಯಪಾದಸರೋರುಹಃ ॥ 3॥

ಚಾರುಚಾಮೀಕರಾಕಾರಜಟಾಲಾರ್ಪಿತಚನ್ದ್ರಮಾಃ ।
ಅರ್ಧಚನ್ದ್ರಾಭನಿಟಿಲಪಾಟೀರತಿಲಕೋಜ್ಜ್ವಲಃ ॥ 4॥

ಕರುಣಾಲಹರೀಪೂರ್ಣ ಕರ್ಣಾನ್ತಾಯತಲೋಚನಃ ।
ಕರ್ಣದಿವ್ಯೋಲ್ಲಸದ್ದಿವ್ಯಮಣಿಕುಂಡಲಮಂಡಿತಃ ॥ 5॥

ವರವಜ್ರಶಿಲಾದರ್ಶಪರಿಭಾವಿಕಪೋಲಭೂಃ ।
ಚಾರುಚಾಮ್ಪೇಯಪುಷ್ಪಾಭನಾಸಿಕಾಪುಟರಂಜಿತಃ ॥ 6॥

ದನ್ತಾಲಿಕುಸುಮೋತ್ಕೃಷ್ಟಕೋಮಲಾಧರಪಲ್ಲವಃ ।
ಮುಗ್ಧಸ್ಮಿತಪರೀಪಾಕಪ್ರಕಾಶಿತರದಾಂಕುರಃ ॥ 7॥

ಅನಾಕಲಿತಸಾದೃಶ್ಯಚಿಬುಕಶ್ರೀವಿರಾಜಿತಃ ।
ಅನರ್ಘರತ್ನಗ್ರೈವೇಯ ವಿಲಸತ್ಕಮ್ಬುಕನ್ಧರಃ ॥ 8॥

ಮಾಣಿಕ್ಯಕಂಕಣೋಲ್ಲಾಸಿ ಕರಾಮ್ಬುಜವಿರಾಜಿತಃ ।
ಮುಕ್ತಾಹಾರಲಸತ್ತುಂಗ ವಿಪುಲೋರಸ್ಕರಾಜಿತಃ ॥ 9॥

ಆವರ್ತನಾಭಿರೋಮಾಲಿವಲಿತ್ರಯಯುತೋದರಃ ।
ವಿಶಂಕಟಕಟಿನ್ಯಸ್ತ ವಾಚಾಲ ಮಣಿಮೇಖಲಃ ॥ 10॥

ಕರಿಹಸ್ತೋಪಮೇಯೋರುರಾದರ್ಶೋಜ್ಜ್ವಲಜಾನುಕಃ ।
ಕನ್ದರ್ಪತೂಣೀಜಿಜ್ಜಂಘೋ ಗುಲ್ಪೋದಂಚಿತನೂಪುರಃ ॥ 11॥

ಮಣಿಮಂಜೀರ ಕಿರಣ ಕಿಂಜಲ್ಕಿತಪದಾಮ್ಬುಜಃ ।
ಶಾಣೋಲ್ಲೀಢಮಣಿಶ್ರೇಣೀರಮ್ಯಾಂಘ್ರಿನಖಮಂಡಲಃ ॥ 12॥

ಆಪಾದಕರ್ಣಕಾಮುಕ್ತಭೂಷಾಶತಮನೋಹರಃ ।
ಸನಕಾದಿಮಹಾಯೋಗಿಸಮಾರಾಧಿತಪಾದುಕಃ ॥ 13॥

ಯಕ್ಷಕಿನ್ನರಗನ್ಧರ್ವಸ್ತೂಯಮಾನಾತ್ಮವೈಭವಃ ।
ಬ್ರಹ್ಮಾದಿದೇವವಿನುತೋ ಯೋಗಮಾಯಾನಿಯೋಜಕಃ ॥ 14॥

ಶಿವಯೋಗೀ ಶಿವಾನನ್ದಃ ಶಿವಭಕ್ತಿಸಮುತ್ತರಃ ।
ವೇದಾನ್ತಸಾರಸನ್ದೋಹಃ ಸರ್ವಸತ್ವಾವಲಮ್ಬನಃ ॥ 15॥

ವಟಮೂಲಾಶ್ರಯೋ ವಾಗ್ಮೀ ಮಾನ್ಯೋ ಮಲಯಜಪ್ರಿಯಃ ।
ಸುಖದೋ ವಾಂಛಿತಾರ್ಥಜ್ಞಃ ಪ್ರಸನ್ನವದನೇಕ್ಷಣಃ ॥ 16॥

ಕರ್ಮಸಾಕ್ಷೀ ಕರ್ಮಮಾ(ಯಾ)ಯೀ ಸರ್ವಕರ್ಮಫಲಪ್ರದಃ ।
ಜ್ಞಾನದಾತಾ ಸದಾಚಾರಃ ಸರ್ವಪಾಪವಿಮೋಚನಃ ॥ 17॥

ಅನಾಥನಾಥೋ ಭಗವಾನ್ ಆಶ್ರಿತಾಮರಪಾದಪಃ ।
ವರಪ್ರದಃ ಪ್ರಕಾಶಾತ್ಮಾ ಸರ್ವಭೂತಹಿತೇ ರತಃ ॥ 18॥

ವ್ಯಾಘ್ರಚರ್ಮಾಸನಾಸೀನಃ ಆದಿಕರ್ತಾ ಮಹೇಶ್ವರಃ ।
ಸುವಿಕ್ರಮಃ ಸರ್ವಗತೋ ವಿಶಿಷ್ಟಜನವತ್ಸಲಃ ॥ 19॥

ಚಿನ್ತಾಶೋಕಪ್ರಶಮನೋ ಜಗದಾನನ್ದ ಕಾರಕಃ ।
ರಶ್ಮಿಮಾನ್ ಭುವನೇಶಾನೋ ದೇವಾಸುರ ಸುಪೂಜಿತಃ ॥ 20॥

ಮೃತ್ಯುಂಜಯೋ ವ್ಯೋಮಕೇಶಃ ಷಟ್ತ್ರಿಂಶತ್ತತ್ವಸಂಗ್ರಹಃ ।
ಅಜ್ಞಾತಸಮ್ಭವೋ ಭಿಕ್ಷುರದ್ವಿತೀಯೋ ದಿಗಮ್ಬರಃ ॥ 21॥

ಸಮಸ್ತದೇವತಾಮೂರ್ತಿಃ ಸೋಮಸೂರ್ಯಾಗ್ನಿಲೋಚನಃ ।
ಸರ್ವಸಾಮ್ರಾಜ್ಯನಿಪುಣೋ ಧರ್ಮಮಾರ್ಗಪ್ರವರ್ತಕಃ ॥ 22॥

ವಿಶ್ವಾಧಿಕಃ ಪಶುಪತಿಃ ಪಶುಪಾಶವಿಮೋಚಕಃ ।
ಅಷ್ಟಮೂರ್ತಿರ್ದೀಪ್ತಮೂರ್ತಿರ್ನಾಮೋಚ್ಚಾರಣಮುಕ್ತಿದಃ ॥ 23॥

ಸಹಸ್ರಾದಿತ್ಯಸಂಕಾಶಃ ಸದಾಷೋಡಶವಾರ್ಷಿಕಃ ।
ದಿವ್ಯಕೇಲೀಸಮಾಮುಕ್ತೋ ದಿವ್ಯಮಾಲ್ಯಾಮ್ಬರಾವೃತಃ ॥ 24॥

ಅನರ್ಘರತ್ನಸಮ್ಪೂರ್ಣೋ ಮಲ್ಲಿಕಾಕುಸುಮಪ್ರಿಯಃ ।
ತಪ್ತಚಾಮೀಕರಾಕಾರಃ ಕ್ರುದ್ಧದಾವಾನಲಾಕೃತಿಃ ॥ 25॥

ನಿರಂಜನೋ ನಿರ್ವಿಕಾರೋ ನಿಜಾ(ರಾ)ವಾಸೋ ನಿರಾಕೃತಿಃ ।
ಜಗದ್ಗುರುರ್ಜಗತ್ಕರ್ತಾ ಜಗದೀಶೋ ಜಗತ್ಪತಿಃ ॥ 26॥

ಕಾಮಹನ್ತಾ ಕಾಮಮೂರ್ತಿಃ ಕಲ್ಯಾಣೋ ವೃಷವಾಹನಃ ।
ಗಂಗಾಧರೋ ಮಹಾದೇವೋ ದೀನಬನ್ಧವಿಮೋಚನಃ ॥ 27॥

ಧೂರ್ಜಟಿಃ ಖಂಡಪರಶುಃಸದ್ಗುಣೋ ಗಿರಿಜಾಸಖಃ ।
ಅವ್ಯಯೋ ಭೂತಸೇನೇಶಃ ಪಾಪಘ್ನಃ ಪುಣ್ಯದಾಯಕಃ ॥ 28॥

ಉಪದೇಷ್ಟಾ ದೃಢಪ್ರಜ್ಞೋ ರುದ್ರೋ ರೋಗವಿನಾಶಕಃ ।
ನಿತ್ಯಾನನ್ದೋ ನಿರಾಧಾರೋ ಹರೋ ದೇವಶಿಖಾಮಣಿಃ ॥ 29॥

ಪ್ರಣತಾರ್ತಿಹರಃ ಸೋಮಃ ಸಾನ್ದ್ರಾನನ್ದೋ ಮಹಾಮತಿಃ ।
ಆಶ್ಚ(ಐಶ್ವ)ರ್ಯವೈಭವೋ ದೇವಃ ಸಂಸಾರಾರ್ಣವತಾರಕಃ ॥ 30॥

ಯಜ್ಞೇಶೋ ರಾಜರಾಜೇಶೋ ಭಸ್ಮರುದ್ರಾಕ್ಷಲಾಂಛನಃ ।
ಅನನ್ತಸ್ತಾರಕಃ ಸ್ಥಾಣುಃಸರ್ವವಿದ್ಯೇಶ್ವರೋ ಹರಿಃ ॥ 31॥

ವಿಶ್ವರೂಪೋ ವಿರೂಪಾಕ್ಷಃ ಪ್ರಭುಃ ಪರಿವೃಢೋ ದೃಢಃ ।
ಭವ್ಯೋ ಜಿತಾರಿಷಡ್ವರ್ಗೋ ಮಹೋದಾರೋಽಘನಾಶನಃ ॥ 32॥

ಸುಕೀರ್ತಿರಾದಿಪುರುಷೋ ಜರಾಮರಣವರ್ಜಿತಃ ।
ಪ್ರಮಾಣಭೂತೋ ದುರ್ಜ್ಞೇಯಃ ಪುಣ್ಯಃ ಪರಪುರಂಜಯಃ ॥ 33॥

ಗುಣಾಕರೋ ಗುಣಶ್ರೇಷ್ಠಃ ಸಚ್ಚಿದಾನನ್ದ ವಿಗ್ರಹಃ ।
ಸುಖದಃ ಕಾರಣಂ ಕರ್ತಾ ಭವಬನ್ಧವಿಮೋಚಕಃ ॥ 34॥

ಅನಿರ್ವಿಣ್ಣೋ ಗುಣಗ್ರಾಹೀ ನಿಷ್ಕಲಂಕಃ ಕಲಂಕಹಾ ।
ಪುರುಷಃ ಶಾಶ್ವತೋ ಯೋಗೀ ವ್ಯಕ್ತಾವ್ಯಕ್ತಃ ಸನಾತನಃ ॥ 35॥

ಚರಾಚರಾತ್ಮಾ ವಿಶ್ವಾತ್ಮಾ ವಿಶ್ವಕರ್ಮಾ ತಮೋಽಪಹೃತ್ ।
ಭುಜಂಗಭೂಷಣೋ ಭರ್ಗಸ್ತರುಣಃ ಕರುಣಾಲಯಃ ॥ 36॥

ಅಣಿಮಾದಿಗುಣೋಪೇತೋ ಲೋಕವಶ್ಯವಿಧಾಯಕಃ ।
ಯೋಗಪಟ್ಟಧರೋ ಮುಕ್ತೋ ಮುಕ್ತಾನಾಂ ಪರಮಾ ಗತಿಃ ॥ 37॥

ಗುರುರೂಪಧರಃ ಶ್ರೀಮಾನ್ ಪರಮಾನನ್ದಸಾಗರಃ ।
ಸಹಸ್ರಬಾಹುಃ ಸರ್ವೇಶಃ ಸಹಸ್ರಾವಯವಾನ್ವಿತಃ ॥ 38॥

ಸಹಸ್ರಮೂರ್ಧಾ ಸರ್ವಾತ್ಮಾ ಸಹಸ್ರಾಕ್ಷಃ ಸಹಸ್ರಪಾತ್ ।
ನಿರ್ವಿಕಲ್ಪೋ ನಿರಾಭಾಸಃ ಶಾನ್ತಃ ಸೂಕ್ಷ್ಮಃ ಪರಾತ್ಪರಃ ॥ 39॥

ಸರ್ವಾತ್ಮಕಃ ಸರ್ವಸಾಕ್ಷೀ ನಿಸ್ಸಂಗೋ ನಿರುಪದ್ರವಃ ।
ನಿರ್ಲೇಪಃ ಸಕಲಾಧ್ಯಕ್ಷಃ ಚಿನ್ಮಯಸ್ತಮಸಃ ಪರಃ ॥ 40॥

ಜ್ಞಾನವೈರಾಗ್ಯಸಮ್ಪನ್ನೋ ಯೋಗಾನನ್ದಮಯಃ ಶಿವಃ ।
ಶಾಶ್ವತೈಶ್ವರ್ಯಸಮ್ಪೂರ್ಣೋ ಮಹಾಯೋಗೀಶ್ವರೇಶ್ವರಃ ॥ 41॥

ಸಹಸ್ರಶಕ್ತಿಸಂಯುಕ್ತಃ ಪುಣ್ಯಕಾಯೋ ದುರಾಸದಃ ।
ತಾರಕಬ್ರಹ್ಮ ಸಮ್ಪೂರ್ಣಃ ತಪಸ್ವಿಜನಸಂವೃತಃ ॥ 42॥

ವಿಧೀನ್ದ್ರಾಮರಸಮ್ಪೂಜ್ಯೋ ಜ್ಯೋತಿಷಾಂ ಜ್ಯೋತಿರುತ್ತಮಃ ।
ನಿರಕ್ಷರೋ ನಿರಾಲಮ್ಬಃ ಸ್ವಾತ್ಮಾರಾಮೋ ವಿಕರ್ತನಃ ॥ 43॥

ನಿರವದ್ಯೋ ನಿರಾತಂಕೋ ಭೀಮೋ ಭೀಮಪರಾಕ್ರಮಃ ।
ವೀರಭದ್ರಃ ಪುರಾರಾತಿರ್ಜಲನ್ಧರಶಿರೋಹರಃ ॥ 44॥

ಅನ್ಧಕಾಸುರಸಂಹರ್ತಾ ಭಗನೇತ್ರಭಿದದ್ಭುತಃ ।
ವಿಶ್ವಗ್ರಾಸೋಽಧರ್ಮಶತ್ರುರ್ಬ್ರಹ್ಮಜ್ಞಾನೈ(ನನ್ದೈ)ಕಮನ್ದಿರಃ ॥ 45॥

ಅಗ್ರೇಸರಸ್ತೀರ್ಥಭೂತಃ ಸಿತಭಸ್ಮಾವಗುಂಠನಃ ।
ಅಕುಂಠಮೇಧಾಃ ಶ್ರೀಕಂಠೋ ವೈಕುಂಠಪರಮಪ್ರಿಯಃ ॥ 46॥

ಲಲಾಟೋಜ್ಜ್ವಲನೇತ್ರಾಬ್ಜಃ ತುಷಾರಕರಶೇಖರಃ ।
ಗಜಾಸುರಶಿರಶ್ಛೇತ್ತಾ ಗಂಗೋದ್ಭಾಸಿತಮೂರ್ಧಜಃ ॥ 47॥

ಕಲ್ಯಾಣಾಚಲಕೋದಂಡಃ ಕಮಲಾಪತಿಸಾಯಕಃ ।
ವಾರಾಂ ಶೇವಧಿತೂಣೀರಃ ಸರೋಜಾಸನಸಾರಥಿಃ ॥ 48॥

ತ್ರಯೀತುರಂಗಸಂಕ್ರಾನ್ತೋ ವಾಸುಕಿಜ್ಯಾವಿರಾಜಿತಃ ।
ರವೀನ್ದುಚರಣಾಚಾರಿಧರಾರಥವಿರಾಜಿತಃ ॥ 49॥

ತ್ರಯ್ಯನ್ತಪ್ರಗ್ರಹೋದಾರಃ ಉಡುಕಂಠಾರವೋಜ್ಜ್ವಲಃ ।
ಉತ್ತಾನಭಲ್ಲವಾಮಾಢಯೋ ಲೀಲಾವಿಜಿತದಾನವಃ ॥ 50॥

ಜಾತು ಪ್ರಪಂಚಜನಿತಜೀವನೋಪಾಯನೋತ್ಸುಕಃ ।
ಸಂಸಾರಾರ್ಣವಸಮ್ಮಗ್ನ ಸಮುದ್ಧರಣಪಂಡಿತಃ ॥ 51॥

ಮತ್ತದ್ವಿರದಧಿಕ್ಕಾರಿಗತಿವೈಭವಮಂಜುಲಃ ।
ಮತ್ತಕೋಕಿಲಮಾಧುರ್ಯ ರಸನಿರ್ಭರನಿಸ್ವನಃ ॥ 52॥

ಕೈವಲ್ಯೋದಿತಕಲ್ಲೋಲಲೀಲಾತಾಂಡವಪಂಡಿತಃ ।
ವಿಷ್ಣುರ್ಜಿಷ್ಣುರ್ವಾಸುದೇವಃ ಪ್ರಭವಿಷ್ಣುಃ ಪುರಾತನಃ ॥ 53॥

ವರ್ಧಿಷ್ಣುರ್ವರದೋ ವೈದ್ಯೋ ಹರಿರ್ನಾರಾಯಣೋಽಚ್ಯುತಃ ।
ಅಜ್ಞಾನವನದಾವಾಗ್ನಿಃ ಪ್ರಜ್ಞಾಪ್ರಾಸಾದಭೂಪತಿಃ ॥ 54॥

ಸರ್ವಭೂಷಿತಸರ್ವಾಂಗಃ ಕರ್ಪೂರೋಜ್ಜ್ವಲಿತಾಕೃತಿಃ ।
ಅನಾದಿಮಧ್ಯನಿಧನೋ ಗಿರಿಶೋ ಗಿರಿಜಾಪತಿಃ ॥ 55॥

ವೀತರಾಗೋ ವಿನೀತಾತ್ಮಾ ತಪಸ್ವೀ ಭೂತಭಾವನಃ ।
ದೇವಾಸುರಗುರುರ್ಧ್ಯೇಯೋ(ದೇವೋ) ದೇವಾಸುರನಮಸ್ಕೃತಿಃ ॥ 56॥

ದೇವಾದಿದೇವೋ ದೇವರ್ಷಿರ್ದೇವಾಸುರವರಪ್ರದಃ ।
ಸರ್ವದೇವಮಯೋಽಚಿನ್ತ್ಯೋ ದೇವತಾತ್ಮಾಽಽತ್ಮಸಮ್ಭವಃ ॥ 57॥

ನಿರ್ಲೇಪೋ ನಿಷ್ಪ್ರಪಂಚಾತ್ಮಾ ನಿರ್ವ್ಯಗ್ರೋ ವಿಘ್ನನಾಶನಃ ।
ಏಕಜ್ಯೋತಿರ್ನಿರಾನನ್ದೋ ವ್ಯಾಪ್ತಮೂರ್ತಿನಾಕುಲಃ ॥ 58॥

ನಿರವದ್ಯೋ ಬಹು(ಧೋ)ಪಾಯೋ ವಿದ್ಯಾರಾಶಿರಕೃತ್ರಿಮಃ ।
ನಿತ್ಯಾನನ್ದಃ ಸುರಾಧ್ಯಕ್ಷೋ ನಿಸ್ಸಂಕಲ್ಪೋ ನಿರಂಜನಃ ॥ 59॥

ನಿರಾತಂಕೋ ನಿರಾಕಾರೋ ನಿಷ್ಪ್ರಪಂಚೋ ನಿರಾಮಯಃ ।
ವಿದ್ಯಾಧರೋ ವಿಯತ್ಕೇಶೋ ಮಾರ್ಕಂಡಯೌವನಃ ಪ್ರಭುಃ ॥ 60॥

ಭೈರವೋ ಭೈರವೀನಾಥಃ ಕಾಮದಃ ಕಮಲಾಸನಃ ।
ವೇದವೇದ್ಯಃ ಸುರಾನನ್ದೋ ಲಸಜ್ಜ್ಯೋತಿಃ ಪ್ರಭಾಕರಃ ॥ 61॥

ಚೂಡಾಮಣಿಃ ಸುರಾಧೀಶೋ ಯಕ್ಷಗೇಯೋ ಹರಿಪ್ರಿಯಃ ।
ನಿರ್ಲೇಪೋ ನೀತಿಮಾನ್ ಸೂತ್ರೀ ಶ್ರೀಹಾಲಾಹಲಸುನ್ದರಃ ॥ 62॥

ಧರ್ಮರಕ್ಷೋ ಮಹಾರಾಜಃ ಕಿರೀಟೀ ವನ್ದಿತೋ ಗುಹಃ ।
ಮಾಧವೋ ಯಾಮಿನೀನಾಥಃ ಶಮ್ಬರಃ ಶಮ್ಬರೀಪ್ರಿಯಃ ॥ 63॥

ಸಂಗೀತವೇತ್ತಾ ಲೋಕಜ್ಞಃ ಶಾನ್ತಃ ಕಲಶಸಮ್ಭವಃ ।
ಬಹ್ಮಣ್ಯೋ ವರದೋ ನಿತ್ಯಃ ಶೂಲೀ ಗುರುಪರೋ ಹರಃ ॥ 64॥

ಮಾರ್ತಾಂಡಃ ಪುಂಡರೀಕಾಕ್ಷಃ ಕರ್ಮಜ್ಞೋ ಲೋಕನಾಯಕಃ ।
ತ್ರಿವಿಕ್ರಮೋ ಮುಕುನ್ದಾರ್ಚ್ಯೋ ವೈದ್ಯನಾಥಃ ಪುರನ್ದರಃ ॥ 65॥

ಭಾಷಾವಿಹೀನೋ ಭಾಷಾಜ್ಞೋ ವಿಘ್ನೇಶೋ ವಿಘ್ನನಾಶನಃ ।
ಕಿನ್ನರೇಶೋ ಬೃಹದ್ಭಾನುಃ ಶ್ರೀನಿವಾಸಃ ಕಪಾಲಭೃತ್ ॥ 66॥

ವಿಜಯೀ ಭೂತವಾಹಶ್ಚ ಭೀಮಸೇನೋ ದಿವಾಕರಃ ।
ಬಿಲ್ವಪ್ರಿಯೋ ವಸಿಷ್ಠೇಶಃ ಸರ್ವಮಾರ್ಗಪ್ರವರ್ತಕಃ ॥ 67॥

ಓಷಧೀಶೋ ವಾಮದೇವೋ ಗೋವಿನ್ದೋ ನೀಲಲೋಹಿತಃ ।
ಷಡರ್ಧನಯನಃ ಶ್ರೀಮಾನ್ ಮಹಾದೇವೋ ವೃಷಧ್ವಜಃ ॥ 68॥

ಕರ್ಪೂರವೀಟಿಕಾಲೋಲಃ ಕರ್ಪೂರವರಚರ್ಚಿತಃ ।
ಅವ್ಯಾಜಕರುಣಮೂರ್ತಿಸ್ತ್ಯಾಗರಾಜಃ ಕ್ಷಪಾಕರಃ ॥ 69॥

ಆಶ್ಚರ್ಯವಿಗ್ರಹಃ ಸೂಕ್ಷ್ಮಃ ಸಿದ್ಧೇಶಃ ಸ್ವರ್ಣಭೈರವಃ ।
ದೇವರಾಜಃ ಕೃಪಾಸಿನ್ಧುರದ್ವಯೋಽಮಿತವಿಕ್ರಮಃ ॥ 70॥

ನಿರ್ಭೇದೋ ನಿತ್ಯಸತ್ವಸ್ಥೋ ನಿರ್ಯೋಗಕ್ಷೇಮ ಆತ್ಮವಾನ್ ।
ನಿರಪಾಯೋ ನಿರಾಸಂಗೋ ನಿಃಶಬ್ದೋ ನಿರುಪಾಧಿಕಃ ॥ 71॥

ಭವಃ ಸರ್ವೇಶ್ವರಃ ಸ್ವಾಮೀ ಭವಭೀತಿವಿಭಂಜನಃ ।
ದಾರಿದ್ರಯತೃಣಕೂಟಾಗ್ನಿಃ ದಾರಿತಾಸುರಸನ್ತತಿಃ ॥ 72॥

ಮುಕ್ತಿದೋ ಮುದಿತಃ ಕುಬ್ಜೋ ಧಾರ್ಮಿಕೋ ಭಕ್ತವತ್ಸಲಃ ।
ಅಭ್ಯಾಸಾತಿಶಯಜ್ಞೇಯಶ್ಚನ್ದ್ರಮೌಲಿಃ ಕಲಾಧರಃ ॥ 73॥

ಮಹಾಬಲೋ ಮಹಾವೀರ್ಯೋ ವಿಭುಃಶ್ರೀಶಃ ಶುಭಪ್ರದಃ (ಪ್ರಿಯಃ) ।
ಸಿದ್ಧಃಪುರಾಣಪುರುಷೋ ರಣಮಂಡಲಭೈರವಃ ॥ 74॥

ಸದ್ಯೋಜಾತೋ ವಟಾರಣ್ಯವಾಸೀ ಪುರುಷವಲ್ಲಭಃ ।
ಹರಿಕೇಶೋ ಮಹಾತ್ರಾತಾ ನೀಲಗ್ರೀವಃ ಸುಮಂಗಲಃ ॥ 75॥

ಹಿರಣ್ಯಬಾಹುಸ್ತಿಗ್ಮಾಂಶುಃ ಕಾಮೇಶಃ ಸೋಮವಿಗ್ರಹಃ ।
ಸರ್ವಾತ್ಮಾ ಸರ್ವಸತ್ಕರ್ತಾ ತಾಂಡವೋ ಮುಂಡಮಾಲಿಕಃ ॥ 76॥

ಅಗ್ರಗಣ್ಯಃ ಸುಗಮ್ಭೀರೋ ದೇಶಿಕೋ ವೈದಿಕೋತ್ತಮಃ ।
ಪ್ರಸನ್ನದೇವೋ ವಾಗೀಶಃ ಚಿನ್ತಾತಿಮಿರಭಾಸ್ಕರಃ ॥ 77॥

ಗೌರೀಪತಿಸ್ತುಂಗಮೌಲಿಃ ಮಧುರಾಜೋ ಮಹಾಕವಿಃ ।
ಶ್ರೀಧರಃ ಸರ್ವಸಿದ್ಧೇಶೋ ವಿಶ್ವನಾಥೋ ದಯಾನಿಧಿಃ ॥ 78॥

ಅನ್ತರ್ಮುಖೋ ಬಹಿರ್ದೃಷ್ಟಿಃ ಸಿದ್ಧವೇಷೋ ಮನೋಹರಃ ।
ಕೃತ್ತಿವಾಸಾಃ ಕೃಪಾಸಿನ್ಧುರ್ಮನ್ತ್ರಸಿದ್ಧೋ ಮತಿಪ್ರದಃ ॥ 79॥

ಮಹೋತ್ಕೃಷ್ಟಃ ಪುಣ್ಯಕರೋ ಜಗತ್ಸಾಕ್ಷೀ ಸದಾಶಿವಃ ।
ಮಹಾಕ್ರತುರ್ಮಹಾಯಜ್ವಾ ವಿಶ್ವಕರ್ಮಾ ತಪೋನಿಧಿಃ ॥ 80॥

ಛನ್ದೋಮಯೋ ಮಹಾಜ್ಞಾನೀ ಸರ್ವಜ್ಞೋ ದೇವವನ್ದಿತಃ ।
ಸಾರ್ವಭೌಮಃ ಸದಾನನ್ದಃ ಕರುಣಾಮೃತವಾರಿಧಿಃ ॥ 81 । ।
ಕಾಲಕಾಲಃ ಕಲಿಧ್ವಂಸೀ ಜರಾಮರಣನಾಶಕಃ ।
ಶಿತಿಕಂಠಶ್ಚಿದಾನನ್ದೋ ಯೋಗಿನೀಗಣಸೇವಿತಃ ॥ 82॥

ಚಂಡೀಶಃ ಸುಖಸಂವೇದ್ಯಃ ಪುಣ್ಯಶ್ಲೋಕೋ ದಿವಸ್ಪತಿಃ ।
ಸ್ಥಾಯೀ ಸಕಲತತ್ತ್ವಾತ್ಮಾ ಸದಾ ಸೇವಕವರ್ಧಕಃ ॥ 83॥

ರೋಹಿತಾಶ್ವಃ ಕ್ಷಮಾರೂಪೀ ತಪ್ತಚಾಮೀಕರಪ್ರಭಃ ।
ತ್ರಿಯಮ್ಬಕೋ ವರರೂಚಿಃ ದೇವದೇವಶ್ಚತುರ್ಭುಜಃ ॥ 84॥

ವಿಶ್ವಮ್ಭರೋ ವಿಚಿತ್ರಾಂಗೋ ವಿಧಾತಾ ಪುರನಾಶ(ಶಾಸ)ನಃ ।
ಸುಬ್ರಹ್ಮಣ್ಯೋ ಜಗತ್ಸ್ವಾಮೀ ಲೋಹಿತಾಕ್ಷಃ ಶಿವೋತ್ತಮಃ ॥ 85॥

ನಕ್ಷತ್ರಮಾಲ್ಯಾಭರಣೋ ಭಗವಾನ್ ತಮಸಃ ಪರಃ ।
ವಿಧಿಕರ್ತಾ ವಿಧಾನಜ್ಞಃ ಪ್ರಧಾನಪುರುಷೇಶ್ವರಃ ॥ 86॥

ಚಿನ್ತಾಮಣಿಃ ಸುರಗುರುರ್ಧ್ಯೇಯೋ ನೀರಾಜನಪ್ರಿಯಃ ।
ಗೋವಿನ್ದೋ ರಾಜರಾಜೇಶೋ ಬಹುಪುಷ್ಪಾರ್ಚನಪ್ರಿಯಃ ॥ 87॥

ಸರ್ವಾನನ್ದೋ ದಯಾರೂಪೀ ಶೈಲಜಾಸುಮನೋಹರಃ ।
ಸುವಿಕ್ರಮಃ ಸರ್ವಗತೋ ಹೇತುಸಾಧನವರ್ಜಿತಃ ॥ 88॥

ವೃಷಾಂಕೋ ರಮಣೀಯಾಂಗಃ ಸತ್ಕರ್ತಾ ಸಾಮಪಾರಗಃ ।
ಚಿನ್ತಾಶೋಕಪ್ರಶಮನಃ ಸರ್ವವಿದ್ಯಾವಿಶಾರದಃ ॥ 89॥

ಭಕ್ತವಿಜ್ಞಪ್ತಿಸನ್ಧಾತಾ ವಕ್ತಾ ಗಿರಿವರಾಕೃತಿಃ ।
ಜ್ಞಾನಪ್ರದೋ ಮನೋವಾಸಃ ಕ್ಷೇಮ್ಯೋ ಮೋಹವಿನಾಶನಃ ॥ 90॥

ಸುರೋತ್ತಮಶ್ಚಿತ್ರಭಾನುಃ ಸದಾ ವೈಭವತತ್ಪರಃ ।
ಸುಹೃದಗ್ರೇಸರಃ ಸಿದ್ಧೋ ಜ್ಞಾನಮುದ್ರೋ ಗಣಾಧಿಪಃ ॥ 91॥

ಅಮರಶ್ಚರ್ಮವಸನೋ ವಾಂಛಿತಾರ್ಥಫಲಪ್ರದಃ ।
ಅಸಮಾನೋಽನ್ತರಹಿತೋ ದೇವಸಿಂಹಾಸನಾಧಿಪಃ ॥ 92॥

ವಿವಾದಹನ್ತಾ ಸರ್ವಾತ್ಮಾ ಕಾಲಃ ಕಾಲವಿವರ್ಜಿತಃ ।
ವಿಶ್ವಾತೀತೋ ವಿಶ್ವಕರ್ತಾ ವಿಶ್ವೇಶೋ ವಿಶ್ವಕಾರಣಃ ॥ 93॥

ಯೋಗಿಧ್ಯೇಯೋ ಯೋಗನಿಷ್ಠೋ ಯೋಗಾತ್ಮಾ ಯೋಗವಿತ್ತಮಃ ।
ಓಂಕಾರರೂಪೋ ಭಗವಾನ್ ಬಿನ್ದುನಾದಮಯಃ ಶಿವಃ ॥ 94॥

ಚತುರ್ಮುಖಾದಿಸಂಸ್ತುತ್ಯಶ್ಚತುರ್ವರ್ಗಫಲಪ್ರದಃ ।
ಸಹಯಾಚಲಗುಹಾವಾಸೀ ಸಾಕ್ಷಾನ್ಮೋಕ್ಷರಸಾಕೃತಿಃ ॥ 95॥

ದಕ್ಷಾಧ್ವರಸಮುಚ್ಛೇತ್ತಾ ಪಕ್ಷಪಾತವಿವರ್ಜಿತಃ ।
ಓಂಕಾರವಾಚಕಃ ಶಮ್ಭುಃ ಶಂಕರಃ ಶಶಿಶೀತಲಃ ॥ 96॥

ಪಂಕಜಾಸನಸಂಸೇವ್ಯಃ ಕಿಂಕರಾಮರವತ್ಸಲಃ ।
ನತದೌರ್ಭಾಗ್ಯತೂಲಾಗ್ನಿಃ ಕೃತಕೌತುಕವಿಭ್ರಮಃ ॥ 97॥

ತ್ರಿಲೋಕಮೋಹನಃ ಶ್ರೀಮಾನ್ ತ್ರಿಪುಂಡ್ರಾಂಕಿತಮಸ್ತಕಃ ।
ಕ್ರೌಂಚರಿಜನಕಃ ಶ್ರೀಮದ್ಗಣನಾಥಸುತಾನ್ವಿತಃ ॥ 98॥

ಅದ್ಭುತೋಽನನ್ತವರದೋಽಪರಿಚ್ಛೇದ್ಯಾತ್ಮವೈಭವಃ ।
ಇಷ್ಟಾಮೂರ್ತಪ್ರಿಯಃ ಶರ್ವ ಏಕವೀರಪ್ರಿಯಂವದಃ ॥ 99॥

ಊಹಾಪೋಹವಿನಿರ್ಮುಕ್ತ ಓಂಕಾರೇಶ್ವರಪೂಜಿತಃ ।
ಕಲಾನಿಧಿಃ ಕೀರ್ತಿನಾಥಃ ಕಾಮೇಶೀಹೃದಯಂಗಮಃ ॥ 100॥

ಕಾಮೇಶ್ವರಃ ಕಾಮರೂಪೋ ಗಣನಾಥಸಹೋದರಃ ।
ಗಾಢೋ ಗಗನಗಮ್ಭೀರೋ ಗೋಪಾಲೋ ಗೋಚರೋ ಗುರುಃ ॥ 101॥

ಗಣೇಶೋ ಗಾಯಕೋ ಗೋಪ್ತಾ ಗಾಣಾಪತ್ಯಗಣಪ್ರಿಯಃ ।
ಘಂಟಾನಿನಾದರುಚಿರಃ ಕರ್ಣಲಜ್ಜಾವಿಭಂಜನಃ ॥ 102॥

ಕೇಶವಃ ಕೇವಲಃ ಕಾನ್ತಶ್ಚಕ್ರಪಾಣಿಶ್ಚರಾಚರಃ ।
ಘನಾಘನೋ ಘೋಷಯುಕ್ತಶ್ಚಂಡೀಹೃದಯನನ್ದನಃ ॥ 103॥

ಚಿತ್ರಾರ್ಪಿತಶ್ಚಿತ್ರಮಯಃ ಚಿನ್ತಿತಾರ್ಥಪ್ರದಾಯಕಃ ।
ಛದ್ಮಚಾರೀ ಛದ್ಮಗತಿಃ ಚಿದಾಭಾಸಶ್ಚಿದಾತ್ಮಕಃ ॥ 104॥

ಛನ್ದೋಮಯಶ್ಛತ್ರಪತಿಃ ಛನ್ದಃಶಾಸ್ತ್ರವಿಶಾರದಃ ।
ಜೀವನೋ ಜೀವನಾಧಾರೋ ಜ್ಯೋತಿಃಶಾಸ್ತ್ರವಿಶಾರದಃ ॥ 105॥

ಜ್ಯೋತಿರ್ಜ್ಯೋತ್ಸ್ನಾಮಯೋ ಜೇತಾ ಜೀಮೂತವರದಾಯಕಃ ।
ಜನಾಘನಾಶನೋ ಜೀವೋ ಜೀವದೋ ಜೀವನೌಷಧಮ್ ॥ 106॥

ಜರಾಹರೋ ಜಾಡ್ಯಹರೋ ಜ್ಯೋತ್ಸ್ನಾಜಾಲಪ್ರವರ್ತಕಃ ।
ಜ್ಞಾನೇಶ್ವರೋ ಜ್ಞಾನಗಮ್ಯೋ ಜ್ಞಾನಮಾರ್ಗಪರಾಯಣಃ ॥ 107॥

ತರುಸ್ಥಸ್ತರುಮಧ್ಯಸ್ಥೋ ಡಾಮರೀಶಕ್ತಿರಂಜಕಃ ।
ತಾರಕಸ್ತಾರತಮ್ಯಾತ್ಮಾ ಟೀಪಸ್ತರ್ಪಣಕಾರಕಃ ॥ 108॥

ತುಷಾರಾಚಲಮಧ್ಯಸ್ಥಸ್ತುಷಾರಕರಭೂಷಣಃ ।
ತ್ರಿಸುಗನ್ಧಸ್ತ್ರಿಮೂರ್ತಿಶ್ಚ ತ್ರಿಮುಖಸ್ತ್ರಿಕಕುದ್ಧರಃ ॥ 109॥

ತ್ರಿಲೋಕೀಮುದ್ರಿಕಾಭೂಷಃ ತ್ರಿಕಾಲಜ್ಞಸ್ತ್ರಯೀಮಯಃ ।
ತತ್ವರೂಪಸ್ತರುಸ್ಥಾಯೀ ತನ್ತ್ರೀವಾದನತತ್ಪರಃ ॥ 110॥

ಅದ್ಭುತಾನನ್ತಸಂಗ್ರಾಮೋ ಢಕ್ಕಾವಾದನತತ್ಪರಃ (ಕೌತುಕಃ) ।
ತುಷ್ಟಸ್ತುಷ್ಟಿಮಯಃ ಸ್ತೋತ್ರಪಾಠಕೋಽತಿ(ಕಾತಿ)ಪ್ರಿಯಸ್ತವಃ ॥ 111॥

ತೀರ್ಥಪ್ರಿಯಸ್ತೀರ್ಥರತಃ ತೀರ್ಥಾಟನಪರಾಯಣಃ ।
ತೈಲದೀಪಪ್ರಿಯಸ್ತೈಲಪಕ್ಕಾನ್ನಪ್ರೀತಮಾನಸಃ ॥ 112॥

ತೈಲಾಭಿಷೇಕಸನ್ತುಷ್ಟಸ್ತಿಲಚರ್ವಣತತ್ಪರಃ ।
ದೀನಾರ್ತಿಹೃದ್ದೀನಬನ್ಧುರ್ದೀನನಾಥೋ ದಯಾಪರಃ ॥ 113॥

ದನುಜಾರಿರ್ದುಃಖಹನ್ತಾ ದುಷ್ಟಭೂತನಿಷೂದನಃ ।
ದೀನೋರುದಾಯಕೋ ದಾನ್ತೋ ದೀಪ್ತಿಮಾನ್ದಿವ್ಯಲೋಚನಃ ॥ 114॥

ದೇದೀಪ್ಯಮಾನೋ ದುರ್ಜ್ಞೇಯೋ ದೀನಸಮ್ಮಾನತೋಷಿತಃ ।
ದಕ್ಷಿಣಾಪ್ರೇಮಸನ್ತುಷ್ಟೋ ದಾರಿದ್ರಯಬಡಬಾನಲಃ ॥ 115॥

ಧರ್ಮೋ ಧರ್ಮಪ್ರದೋ ಧ್ಯೇಯೋ ಧೀಮಾನ್ಧೈರ್ಯವಿಭೂಷಿತಃ ।
ನಾನಾರೂಪಧರೋ ನಮ್ರೋ ನದೀಪುಲಿನಸಂಶ್ರಿತಃ ॥ 116॥

ನಟಪ್ರಿಯೋ ನಾಟ್ಯಕರೋ ನಾರೀಮಾನಸಮೋಹನಃ ।
ನಾರದೋ ನಾಮರಹಿತೋ ನಾನಾಮನ್ತ್ರರಹಸ್ಯವಿತ್ ॥ 117॥

ಪತಿಃ ಪಾತಿತ್ಯಸಂಹರ್ತಾ ಪರವಿದ್ಯಾವಿಕರ್ಷಕಃ ।
ಪುರಾಣಪುರುಷಃ ಪುಣ್ಯಃ ಪದ್ಯಗದ್ಯಪ್ರದಾಯಕಃ ॥ 118॥

ಪಾರ್ವತೀರಮಣಃ ಪೂರ್ಣಃ ಪುರಾಣಾಗಮಸೂಚಕಃ ।
ಪಶೂಪಹಾರರಸಿಕಃ ಪುತ್ರದಃ ಪುತ್ರಪೂಜಿತಃ ॥ 119॥

ಬ್ರಹ್ಮಾಂಡಭೇದನೋ ಬ್ರಹ್ಮಜ್ಞಾನೀ ಬ್ರಾಹ್ಮಣಪಾಲಕಃ ।
ಭೂತಾಧ್ಯಕ್ಷೋ ಭೂತಪತಿರ್ಭೂತಭೀತಿನಿವಾರಣಃ ॥ 120॥

ಭದ್ರಾಕಾರೋ ಭೀಮಗರ್ಭೋ ಭೀಮಸಂಗ್ರಾಮಲೋಲುಪಃ ।
ಭಸ್ಮಭೂಷೋ ಭಸ್ಮಸಂಸ್ಥೋ ಭೈಕ್ಷ್ಯಕರ್ಮಪರಾಯಣಃ ॥ 121॥

ಭಾನುಭೂಷೋ ಭಾನುರೂಪೋ ಭವಾನೀಪ್ರೀತಿದಾಯಕಃ ।
ಭವಪ್ರಿಯೋ ಭಾವರತೋ ಭಾವಾಭಾವವಿವರ್ಜಿತಃ ॥ 122॥

ಭ್ರಾಜಿಷ್ಣುಜೀ(ರ್ಜೀ)ವಸನ್ತುಷ್ಟೋ ಭಟ್ಟಾರೋ ಭದ್ರವಾಹನಃ ।
ಭದ್ರದೋ ಭ್ರಾನ್ತಿರಹಿತೋ ಭೀಮಚಂಡೀಪತಿರ್ಮಹಾನ್ ॥ 123॥

ಯಜುರ್ವೇದಪ್ರಿಯೋ ಯಾಜೀ ಯಮಸಂಯಮಸಂಯುತಃ ।
ರಾಮಪೂಜ್ಯೋ ರಾಮನಾಥೋ ರತ್ನದೋ ರತ್ನಹಾರಕಃ ॥ 124॥

ರಾಜ್ಯದೋ ರಾಮವರದೋ ರಂಜಕೋ ರತಿಮಾರ್ಗಧೃತ್ ।
ರಾಮಾನನ್ದಮಯೋ ರಮ್ಯೋ ರಾಜರಾಜೇಶ್ವರೋ ರಸಃ ॥ 125॥

ರತ್ನಮನ್ದಿರಮಧ್ಯಸ್ಥೋ ರತ್ನಪೂಜಾಪರಾಯಣಃ ।
ರತ್ನಾಕಾರೋ ಲಕ್ಷಣೇಶೋ ಲಕ್ಷ್ಯದೋ ಲಕ್ಷ್ಯಲಕ್ಷಣಃ ॥ 126॥

ಲೋಲಾಕ್ಷೀನಾಯಕೋ ಲೋಭೀ ಲಕ್ಷಮನ್ತ್ರಜಪಪ್ರಿಯಃ ।
ಲಮ್ಬಿಕಾಮಾರ್ಗನಿರತೋ ಲಕ್ಷ್ಯಕೋಟ್ಯಂಡನಾಯಕಃ ॥ 127॥

ವಿದ್ಯಾಪ್ರದೋ ವೀತಿಹೋತಾ ವೀರವಿದ್ಯಾವಿಕರ್ಷಕಃ ।
ವಾರಾಹೀಪಾಲಕೋ ವನ್ಯೋ ವನವಾಸೀ ವನಪ್ರಿಯಃ ॥ 128॥

ವನೇಚರೋ ವನಚರಃ ಶಕ್ತಿಪೂಜ್ಯಃ ಶಿಖಿಪ್ರಿಯಃ ।
ಶರಚ್ಚನ್ದ್ರನಿಭಃ ಶಾನ್ತಃ ಶಕ್ತಃ ಸಂಶಯವರ್ಜಿತಃ ॥ 129॥

ಶಾಪಾನುಗ್ರಹದಃ ಶಂಖಪ್ರಿಯಃ ಶತ್ರುನಿಷೂದನಃ ।
ಷಟ್ಕೃತ್ತಿಕಾಸುಸಮ್ಪೂಜ್ಯಃ ಷಟ್ಶಾಸ್ತ್ರಾರ್ಥರಹಸ್ಯವಿತ್ ॥ 130॥

ಸುಭಗಃ ಸರ್ವಜಿತ್ಸೌಮ್ಯಃ ಸಿದ್ಧಮಾರ್ಗಪ್ರವರ್ತಕಃ ।
ಸಹಜಾನನ್ದದಃ ಸೋಮಃ ಸರ್ವಶಾಸ್ತ್ರ ರಹಸ್ಯವಿತ್ ॥ 131॥

ಸರ್ವಜಿತ್ಸರ್ವವಿತ್ಸಾಧುಃ ಸರ್ವಧರ್ಮ ಸಮನ್ವಿತಃ ।
ಸರ್ವಾಧ್ಯಕ್ಷಃ ಸರ್ವದೇವೋ ಮಹರ್ಷಿರ್ಮೋಹನಾಸ್ತ್ರವಿತ್ ॥ 132 । ।
ಕ್ಷೇಮಂಕರಃ ಕ್ಷೇತ್ರಪಾಲಃ ಕ್ಷಯರೋಗಕ್ಷಯಂಕರಃ ।
ನಿಃ ಸೀಮಮಹಿಮಾ ನಿತ್ಯೋ ಲೀಲಾವಿಗ್ರಹರೂಪಧೃತ್ ॥ 133 । ।
ಚನ್ದನದ್ರವದಿಗ್ಧಾಂಗಃ ಚಾಮ್ಪೇಯಕುಸುಮಪ್ರಿಯಃ ।
ಸಮಸ್ತಭಕ್ತಸುಖದಃ ಪರಮಾಣುರ್ಮಹಾಹ್ನದಃ ॥ 134 । ।
ಆಕಾಶಗೋ ದುಷ್ಪ್ರಧರ್ಷಃ ಕಪಿಲಃ ಕಾಲಕನ್ಧರಃ ।
ಕರ್ಪೂಗೌರಃ ಕುಶಲಃ ಸತ್ಯಸನ್ಧೋ ಜಿತೇನ್ದ್ರಿಯಃ ॥ 135 । ।
ಶಾಶ್ವತೈಶ್ವರ್ಯವಿಭವಃ ಪುಷ್ಕರಃ ಸುಸಮಾಹಿತಃ ।
ಮಹರ್ಷಿಃ ಪಂಡಿತೋ ಬ್ರಹ್ಮಯೋನಿಃ ಸರ್ವೋತ್ತಮೋತ್ತಮಃ ॥ 136 । ।
ಭೂಮಿಭಾರಾರ್ತಿಸಂಹರ್ತಾ ಷಡೂರ್ಮಿರಹಿತೋ ಮೃಡಃ ।
ತ್ರಿವಿಷ್ಟಪೇಶ್ವರಃ ಸರ್ವಹೃದಯಾಮ್ಬುಜಮಧ್ಯಗಃ ॥ 137 । ।
ಸಹಸ್ರದಲಪದ್ಮಸ್ಥಃ ಸರ್ವವರ್ಣೋಪಶೋಭಿತಃ ।
ಪುಣ್ಯಮೂರ್ತಿಃ ಪುಣ್ಯಲಭ್ಯಃ ಪುಣ್ಯಶ್ರವಣಕೀರ್ತನಃ ॥ 138 । ।
ಸೂರ್ಯಮಂಡಲಮಧ್ಯಸ್ಥಶ್ಚನ್ದ್ರಮಂಡಲಮಧ್ಯಗಃ ।
ಸದ್ಭಕ್ತಧ್ಯಾನನಿಗಲಃ ಶರಣಾಗತಪಾಲಕಃ ॥ 139 । ।
ಶ್ವೇತಾತಪತ್ರರುಚಿರಃ ಶ್ವೇತಚಾಮರವೀಜಿತಃ ।
ಸರ್ವಾವಯಸಮ್ಪೂರ್ಣಃ ಸರ್ವಲಕ್ಷಣಲಕ್ಷಿತಃ ॥ 140 । ।
ಸರ್ವಮಂಗಲಾಮಾಂಗಲ್ಯಃ ಸರ್ವಕಾರಣಕಾರಣಮ್ ।
ಆಮೋದಮೋದಜನಕಃ ಸರ್ಪರಾಜೋತ್ತರೀಯಕಃ ॥ 141 । ।
ಕಪಾಲೀ ಗೋವಿನ್ದಸಿದ್ಧಃ ಕಾನ್ತಿಸಂವಲಿತಾನನಃ ।
ಸರ್ವಸದ್ಗುರುಸಂಸೇವ್ಯೋ ದಿವ್ಯಚನ್ದನಚರ್ಚಿತಃ ॥ 142 । ।
ವಿಲಾಸಿನೀಕೃತೋಲ್ಲಾಸಃ ಇಚ್ಛಾಶಕ್ತಿನಿಷೇವಿತಃ ।
ಅನನ್ತೋಽನನ್ತಸುಖದೋ ನನ್ದನಃ ಶ್ರೀನಿಕೇತನಃ ॥ 143॥

ಅಮೃತಾಬ್ಧಿಕೃತಾವಾಸೋ (ತೋಲ್ಲಾಸೀ) ನಿತ್ಯಕ್ಲಿನ್ನೋ ನಿರಾಮಯಃ ।
ಅನಪಾಯೋಽನನ್ತದೃಷ್ಟಿಃ ಅಪ್ರಮೇಯೋಽಜರೋಽಮರಃ ॥ 144॥

ಅನಾಮಯೋಽಪ್ರತಿಹತಶ್ಚಾಽಪ್ರತರ್ಕ್ಯೋಽಮೃತೋಽಕ್ಷರಃ ।
ಅಮೋಘಸಿದ್ಧಿರಾಧಾರ ಆಧಾರಾಧೇಯವರ್ಜಿತಃ ॥ 145॥

ಈಷಣಾತ್ರಯನಿರ್ಮುಕ್ತ ಈಹಾಮಾತ್ರವಿವರ್ಜಿತಃ ।
ಋಗ್ಯಜುಃಸಾಮನಯನ ಋದ್ಧಿಸಿದ್ಧಿಸಮೃದ್ಧಿದಃ ॥ 146॥

ಔದಾರ್ಯನಿಧಿರಾಪೂರ್ಣ ಐಹಿಕಾಮುಷ್ಮಿಕಪ್ರದಃ ।
ಶುದ್ಧಸನ್ಮಾತ್ರಸಂವಿತ್ತಾಸ್ವರೂಪಸು(ಮು)ಖವಿಗ್ರಹಃ ॥ 147॥

ದರ್ಶನಪ್ರಥಮಾಭಾಸೋ ದುಷ್ಟದರ್ಶನವರ್ಜಿತಃ ।
ಅಗ್ರಗಣ್ಯೋಽಚಿನ್ತ್ಯರೂಪಃ ಕಲಿಕಲ್ಮಷನಾಶನಃ ॥ 148॥

ವಿಮರ್ಶರೂಪೋ ವಿಮಲೋ ನಿತ್ಯತೃಪ್ತೋ ನಿರಾಶ್ರಯಃ ।
ನಿತ್ಯಶುದ್ಧೋ ನಿತ್ಯಬುದ್ಧೋ ನಿತ್ಯಮುಕ್ತೋ ನಿರಾವೃತಃ ॥ 149॥

ಮೈತ್ರ್ಯಾದಿವಾಸನಾಲಭ್ಯೋ ಮಹಾಪ್ರಲಯಸಂಸ್ಥಿತಃ ।
ಮಹಾಕೈಲಾಸನಿಲಯಃ ಪ್ರಜ್ಞಾನಘನವಿಗ್ರಹಃ ॥ 150॥

ಶ್ರೀಮದ್ವ್ಯಾಘ್ರಪುರಾವಾಸೋ ಭುಕ್ತಿಮುಕ್ತಿಫಲಪ್ರದಃ ।
ಜಗದ್ಯೋನಿರ್ಜಗತ್ಸಾಕ್ಷೀ ಜಗದೀಶೋ ಜಗನ್ಮಯಃ ॥ 151॥

ಜಪೋ ಜಪಪರೋ ಜಪ್ಯೋ ವಿದ್ಯಾಸಿಂಹಾಸನಪ್ರಭುಃ ।
ತತ್ತ್ವಾನಾಂ ಪ್ರಕೃತಿಸ್ತತ್ತ್ವಂ ತತ್ತ್ವಮ್ಪದನಿರೂಪಿತಃ ॥ 152॥

ದಿಕ್ಕಾಲಾಗ್ನ್ಯನವಚ್ಛಿನ್ನಃ ಸಹಜಾನನ್ದಸಾಗರಃ ।
ಪ್ರಕೃತಿಃ ಪ್ರಾಕೃತಾತೀತಃ ಪ್ರಜ್ಞಾನೈಕರಸಾಕೃತಿಃ ॥ 153॥

ನಿಃಶಂಕಮತಿದೂರಸ್ಥಃ ಚೇತ್ಯಚೇತನಚಿನ್ತಕಃ ।
ತಾರಕಾನ್ತರಸಂಸ್ಥಾನಸ್ತಾರಕಸ್ತಾರಕಾನ್ತಕಃ ॥ 154॥

ಧ್ಯಾನೈಕಪ್ರಕಟೋ ಧ್ಯೇಯೋ ಧ್ಯಾನಂ (ನೀ) ಧ್ಯಾನವಿಭೂಷಣಃ ।
ಪರಂ ವ್ಯೋಮ ಪರಂ ಧಾಮ ಪರಮಾಣುಃ ಪರಂ ಪದಮ್ ॥ 155॥

ಪೂರ್ಣಾನನ್ದಃ ಸದಾನನ್ದೋ ನಾದಮಧ್ಯಪ್ರತಿಷ್ಠಿತಃ ।
ಪ್ರಮಾವಿಪರ್ಯಯಾ(ಣಪ್ರತ್ಯಯಾ)ತೀತಃ ಪ್ರಣತಾಜ್ಞಾನನಾಶಕಃ ॥ 156॥

ಬಾಣಾರ್ಚಿತಾಂಘ್ರಿರ್ಬಹುದೋ ಬಾಲಕೇಲಿಕುತೂಹಲಃ ।
ಬೃಹತ್ತಮೋ ಬ್ರಹ್ಮಪದೋ ಬ್ರಹ್ಮವಿದ್ಬ್ರಹ್ಮವಿತ್ಪ್ರಿಯಃ ॥ 157॥

ಭ್ರೂಕ್ಷೇಪದತ್ತಲಕ್ಷ್ಮೀಕೋ ಭ್ರೂಮಧ್ಯಧ್ಯಾನಲಕ್ಷಿತಃ ।
ಯಶಸ್ಕರೋ ರತ್ನಗರ್ಭೋ ಮಹಾರಾಜ್ಯಸುಖ ಪ್ರದಃ ॥ 158॥

ಶಬ್ದಬ್ರಹ್ಮ ಶಮಪ್ರಾಪ್ಯೋ ಲಾಭಕೃಲ್ಲೋಕವಿಶ್ರುತಃ ।
ಶಾಸ್ತಾ ಶಿಖಾಗ್ರನಿಲಯಃ ಶರಣ್ಯೋ ಯಾಜಕಪ್ರಿಯಃ ॥ 159॥

ಸಂಸಾರವೇದ್ಯಃ ಸರ್ವಜ್ಞಃ ಸರ್ವಭೇಷಜಭೇಷಜಮ್ ।
ಮನೋವಾಚಾಭಿರಗ್ರಾಹ್ಯಃ ಪಂಚಕೋಶವಿಲಕ್ಷಣಃ ॥ 160॥

ಅವಸ್ಥಾತ್ರಯನಿರ್ಮುಕ್ತಸ್ತ್ವಕ್ಸ್ಥಃ ಸಾಕ್ಷೀ ತುರೀಯಕಃ ।
ಪಂಚಭೂತಾದಿದೂರಸ್ಥಃ ಪ್ರತ್ಯಗೇಕರಸೋಽವ್ಯಯಃ ॥ 161॥

ಷಟ್ಚಕ್ರಾನ್ತಃಕೃತೋಲ್ಲಾಸಃ ಷಡ್ವಿಕಾರವಿವರ್ಜಿತಃ ।
ವಿಜ್ಞಾನಘನಸಮ್ಪೂರ್ಣೋ ವೀಣಾವಾದನತತ್ಪರಃ ॥ 162॥

ನೀಹಾರಾಕಾರಗೌರಾಂಗೋ ಮಹಾಲಾವಣ್ಯವಾರಿಧಿಃ ।
ಪರಾಭಿಚಾರಶಮನಃ ಷಡಧ್ವೋಪರಿ ಸಂಸ್ಥಿತಃ ॥ 163॥

ಸುಷುಮ್ನಾಮಾರ್ಗ ಸಂಚಾರೀ ಬಿಸತನ್ತುನಿಭಾಕೃತಿಃ ।
ಪಿನಾಕೀ ಲಿಂಗರೂಪಃ ಶ್ರೀಮಂಗಲಾವಯವೋಜ್ಜ್ವಲಃ ॥ 164॥

ಕ್ಷೇತ್ರಾಧಿಪಃ ಸುಸಂವೇದ್ಯಃ ಶ್ರೀಪ್ರದೋ ವಿಭವಪ್ರದಃ ।
ಸರ್ವವಶ್ಯಕರಃ ಸರ್ವತೋಷಕಃ ಪುತ್ರಪೌತ್ರಿದಃ ।
ಆತ್ಮನಾಥಸ್ತೀರ್ಥನಾಥಃ ಸಪ್ತ(ಪ್ತಿ)ನಾಥಃ ಸದಾಶಿವಃ ॥ 165॥




ಶ್ರೀದಕ್ಷಿಣಾಮೂರ್ತಿಸಹಸ್ರನಾಮಸ್ತೋತ್ರಮ್ 1 
(ಚಿದಮ್ಬರನಟನತನ್ತ್ರತಃ)
(ದಕಾರಾದಿಥಕಾರಾನ್ತಮ್ )
ಅಸ್ಯ ಶ್ರೀದಕ್ಷಿಣಾಮೂರ್ತಿಸಹಸ್ರನಾಮಸ್ತೋತ್ರಮಹಾಮನ್ತ್ರಸ್ಯ ಗುರುರಾಟ್ ಋಷಿಃ ।
ಅನುಷ್ಟುಪ್ಛನ್ದಃ । ಶ್ರೀದಕ್ಷಿಣಾಮೂರ್ತಿಃ ಪರಮಾತ್ಮಾ ದೇವತಾ ।
ಹ್ರೀಂ ಬೀಜಂ । ಸ್ವಾಹಾ ಶಕ್ತಿಃ । ನಮಃ ಕೀಲಕಮ್ ।
ಚತುಃಷಷ್ಟಿಕಲಾವಿದ್ಯಾಜ್ಞಾನಪ್ರಾಪ್ತ್ಯೈ ನಾಮಪರಾಯಣೇ ವಿನಿಯೋಗಃ ।
ಓಂ ನಮೋ ಭಗವತೇ ದಕ್ಷಿಣಾಮೂರ್ತಯೇ ।
ಮಹ್ಯಂ ಮೇಧಾಂ ಪ್ರಜ್ಞಾಂ ಪ್ರಯಚ್ಛ ಸ್ವಾಹಾ ಇತಿ ಷಡಂಗನ್ಯಾಸಃ ।

ಓಂ ದಕ್ಷಿಣೋ ದಕ್ಷಿಣಾಮೂರ್ತಿರ್ದಯಾಲುರ್ದೀನವಲ್ಲಭಃ ।
ದೀನಾರ್ತಿಹಾ ದೀನನಾಥೋ ದೀನಬನ್ಧುರ್ದಯಾಪರಃ ॥ 1॥

ದಾರಿದ್ರ್ಯಶಮನೋಽದೀನೋ ದೀರ್ಘೋ ದಾನವನಾಶನಃ ।
ದನುಜಾರಿರ್ದುಃಖಹನ್ತಾ ದುಷ್ಟಭೂತನಿಷೂದನಃ ॥ 2॥

ದೀನಾರ್ತಿಹರಣೋ ದಾನ್ತೋ ದೀಪ್ತಿಮಾನ್ದಿವ್ಯಲೋಚನಃ ।
ದೇದೀಪ್ಯಮಾನೋ ದುರ್ಗೇಶಃ ಶ್ರೀದುರ್ಗಾವರದಾಯಕಃ ॥ 3॥

ದರಿಸಂಸ್ಥೋ ದಾನರೂಪೋ ದಾನಸನ್ಮಾನತೋಷಿತಃ ।
ದೀನೋ ದಾಡಿಮಪುಷ್ಪಾಢ್ಯೋ ದಾಡಿಮೀಪುಷ್ಪಭೂಷಿತಃ ॥ 4॥

ದೈನ್ಯಹೃದ್ದುರಿತಘ್ನಶ್ಚ ದಿಶಾವಾಸೋ ದಿಗಮ್ಬರಃ ।
ದಿಕ್ಪತಿರ್ದೀರ್ಘಸೂತ್ರೀ ಚ ದರದಮ್ಬುದಲೋಚನಃ ॥ 5॥

ದಕ್ಷಿಣಾಪ್ರೇಮಸನ್ತುಷ್ಟೋ ದಾರಿದ್ರ್ಯವಡವಾನಲಃ ।
ದಕ್ಷಿಣಾವರದೋ ದಕ್ಷೋ ದಕ್ಷಾಧ್ವರವಿನಾಶಕೃತ್ ॥ 6॥

ದಾಮೋದರಪ್ರಿಯೋ ದೀರ್ಘೋ ದೀರ್ಘಿಕಾಜನಮಧ್ಯಗಃ ।
ಧರ್ಮೋ ಧನಪ್ರದೋ ಧ್ಯೇಯೋ ಧೀಮಾನ್ಧೈರ್ಯವಿಭೂಷಿತಃ ॥ 7॥

ಧರಣೀಧಾರಕೋ ಧಾತಾ ಧನಾಧ್ಯಕ್ಷೋ ಧುರನ್ಧರಃ ।
ಧೀಧಾರಕೋ ಧಿಂಡಿಮಕೋ ನಗ್ನೋ ನಾರಾಯಣೋ ನರಃ ॥ 8॥

ನರನಾಥಪ್ರಿಯೋ ನಾಥೋ ನದೀಪುಲಿನಸಂಸ್ಥಿತಃ ।
ನಾನಾರೂಪಧರೋ ನಮೋ ನಾನ್ದೀಶ್ರಾದ್ಧಪ್ರಿಯೋ ನರಃ ॥ 9॥

ನಟಾಚಾರ್ಯೋ ನಟವರೋ ನಾರೀಮಾನಸಮೋಹನಃ ।
ನದೀಪ್ರಿಯೋ ನೀತಿಧರೋ ನಾನಾಮನ್ತ್ರರಹಸ್ಯವಿತ್ ॥ 10॥

ನಾರದೋ ನಾಮರಹಿತೋ ನೌಕಾರೂಢೋ ನಟಪ್ರಿಯಃ ।
ಪರಮಃ ಪರಮಾದಶ್ಚ ಪರವಿದ್ಯಾವಿಕರ್ಷಣಃ ॥ 11॥

ಪತಿಃ ಪಾತಿತ್ಯಸಂಹರ್ತಾ ಪರಮೇಶಃ ಪುರಾತನಃ ।
ಪುರಾಣಪುರುಷಃ ಪುಣ್ಯಃ ಪದ್ಯಗದ್ಯವಿಶಾರದಃ ॥ 12॥

ಪದ್ಯಪ್ರಿಯಃ ಪದ್ಯಹಸ್ತಃ ಪರಮಾರ್ಥಪರಾಯಣಃ ।
ಪ್ರೀತಃ ಪುರಾಣಪುರುಷಃ ಪುರಾಣಾಗಮಸೂಚಕಃ ॥ 13॥

ಪುರಾಣವೇತ್ತಾ ಪಾಪಘ್ನಃ ಪಾರ್ವತೀಶಃ ಪರಾರ್ಥವಿತ್ ।
ಪದ್ಮಾವತೀಪ್ರಿಯಃ ಪ್ರಾಣಃ ಪರಃ ಪರರಹಸ್ಯವಿತ್ ॥ 14॥

ಪಾರ್ವತೀರಮಣಃ ಪೀನಃ ಪೀತವಾಸಾಃ ಪರಾತ್ಪರಃ ।
ಪಶೂಪಹಾರರಸಿಕಃ ಪಾಶೀ ಪಾಶುಪತಃ ಪ್ರಿಯಃ ॥ 15॥

ಪಕ್ಷೀನ್ದ್ರವಾಹನಪ್ರೀತಃ ಪುತ್ರದಃ ಪುತ್ರಪೂಜಿತಃ ।
ಫಣಿನಾದಃ ಫೈಂಕೃತಿಶ್ಚ ಫಟ್ಕಾರಿಃ ಫ್ರೇಂ ಪರಾಯಣಃ ॥ 16॥

ಫ್ರೀಂ ಬೀಜಜಪಸನ್ತುಷ್ಟ ಫ್ರೀಂಕಾರಃ ಫಣಿಭೂಷಿತಃ ।
ಫಣಿವಿದ್ಯಾಮಯಃ ಫ್ರೈಂ ಫ್ರೈಂ ಫ್ರೈಂ ಫ್ರೈಂ ಶಬ್ದಪರಾಯಣಃ ॥ 17॥

ಫಡಸ್ರಜಪಸನ್ತುಷ್ಟೋ ಬಲಿಭುಗ್ ಬಾಣಭೂಷಿತಃ ।
ಬಾಣಪೂಜಾರತೋ ಬ್ಲೂಂ ಬ್ಲೂಂ ಬ್ಲೂಂ ಬೀಜನಿರತಃ ಶುಚಿಃ ॥ 18॥

ಭವಾರ್ಣವೋ ಬಾಲಮತಿಃ ಬಾಲೇಶೋ ಬಾಲಭಾವಧೃತ್  ।
ಬಾಲಪ್ರಿಯೋ ಬಾಲಗತಿಃ ಬಲಿವರದಪ್ರಿಯೋ ಬಲೀ ॥ 19॥

ಬಾಲಚನ್ದ್ರಪ್ರಿಯೋ ಬಾಲೋ ಬಾಲಶಬ್ದಪರಾಯಣಃ ।
ಬ್ರಹ್ಮಾಂಡಭೇದನೋ ಬ್ರಹ್ಮಜ್ಞಾನೀ ಬ್ರಾಹ್ಮಣಪಾಲಕಃ ॥ 20॥

ಭವಾನೀ ಭೂಪತಿರ್ಭದ್ರೋ ಭದ್ರದೋ ಭದ್ರವಾಹನಃ ।
ಭೂತಾಧ್ಯಕ್ಷೋ ಭೂತಪತಿಃ ಭೂತಭೀತಿನಿವಾರಣ ॥ 21॥

ಭದ್ರಂಕರೋ ಭೀಮಗರ್ಭೋ ಭೀಮಸಂಗಮಲೋಲುಪಃ ।
ಭೀಮೋ ಭಯಾನಕೋ ಭ್ರಾತಾ ಭ್ರಾನ್ತೋ ಭರಕಾಸುರಪ್ರಿಯಃ ॥ 22॥

ಭಸ್ಮಭೂಷೋ ಭಸ್ಮಸಂಸ್ಥೋ ಭೈಕ್ಷಕರ್ಮಪರಾಯಣಃ ।
ಭಾನುಭೂಷೋ ಭಾನುರೂಪೋ ಭವಾನೀಪ್ರೀತಿದೋ ಭವಃ ॥ 23॥

ಭಗೇದವೋ ಭರ್ಗವಾಸೋ ಭರ್ಗಪೂಜಾಪರಾಯಣಃ ।
ಭಾವವ್ರತೋ ಭಾವರತೋ ಭಾವಾಭಾವವಿವರ್ಜಿತಃ ॥ 24॥

ಭರ್ಗೋ ಭಾವಾನನ್ತಯುಕ್ತೋ ಭಾಂ ಭಿಂ ಶಬ್ದಪರಾಯಣಃ ।
ಭ್ರಾಂ ಬೀಜಜಪಸನ್ತುಷ್ಟೋ ಭಟ್ಟಾರೋ ಭದ್ರವಾಹನಃ ॥ 25॥

ಭಟ್ಟಾರಕೋ ಭೀಮಭೀಮೋ ಭೀಮಚಂಡಪತಿರ್ಭವಃ ।
ಭವಾನೀಜಪಸನ್ತುಷ್ಟೋ ಭವಾನೀಪೂಜನೋತ್ಸುಕಃ ॥ 26॥

ಭ್ರಮರೋ ಭ್ರಾಮರೀಯುಕ್ತೋ ಭ್ರಮರಾಮ್ಬಾಪ್ರಪೂಜಿತಃ ।
ಮಹಾದೇವೋ ಮಹಾಮಾನ್ಯೋ ಮಹೇಶೋ ಮಾಧವಪ್ರಿಯಃ ॥ 27॥

ಮಧುಪುಷ್ಪಪ್ರಿಯೋ ಮಾಧ್ವೀ ಮಾನಪೂಜಪರಾಯಣಃ ।
ಮಧುಪಾನಪ್ರಿಯೋ ಮೀನೋ ಮೀನಾಕ್ಷೀನಾಯಕೋ ಮಹಾನ್ ॥ 28॥

ಮಾರದೃಶೋ ಮದನಘ್ನೋ ಮಾನನೀಯೋ ಮಹೋಕ್ಷಗಃ ।
ಮಾಧವೋ ಮಾನರಹಿತೋ ಮ್ರಾಮ್ಬೀಜಜಪತೋಷಿತಃ ॥ 29॥

ಮಧುಪಾನರತೋ ಮಾನೀ ಮಹಾರ್ಹೋ ಮೋಹನಾಸ್ರವಿತ್ ।
ಮಹಾತಾಂಡವಕೃನ್ಮನ್ತ್ರೋ ಮಧುಪೂಜಾಪರಾಯಣಃ ॥ 30॥

ಮೂರ್ತಿರ್ಮುದ್ರಾಪ್ರಿಯೋ ಮಿತ್ರೋ ಮಿತ್ರಸನ್ತುಷ್ಟಮಾನಸಃ ।
ಮ್ರೀಂ ಮ್ರೀಂ ಮಧುಮತೀನಾಥೋ ಮಹಾದೇವಪ್ರಿಯೋ ಮೃಡಃ ॥ 31॥

ಯಾದೋನಿಧಿರ್ಯದುಪತಿಃ ಯತಿರ್ಯಜ್ಞಪರಾಯಣಃ ।
ಯಜ್ವಾ ಯಾಗಪ್ರಿಯೋ ಯಾಜೀ ಯಾಯೀಭಾವಪ್ರಿಯೋ ಯಮಃ ॥ 32॥

ಯಾತಾಯಾತಾದಿರಹಿತೋ ಯತಿಧರ್ಮಪರಾಯಣಃ ।
ಯತಿಸಾಧ್ಯೋ ಯಷ್ಟಿಧರೋ ಯಜಮಾನಪ್ರಿಯೋ ಯಜಃ ॥ 33॥

ಯಜುರ್ವೇದಪ್ರಿಯೋ ಯಾಯೀ ಯಮಸಂಯಮಸಂಯುತಃ ।
ಯಮಪೀಡಾಹರೋ ಯುಕ್ತಿರ್ಯೋಗೀ ಯೋಗೀಶ್ವರಾಲಯಃ ॥ 34॥

ಯಾಜ್ಞವಲ್ಕ್ಯಪ್ರಿಯೋ ಯೋನಿಃ ಯೋನಿದೋಷವಿವರ್ಜಿತಃ ।
ಯಾಮಿನೀನಾಥೋ ಯೂಷೀ ಚ ಯಮವಂಶಸಮುದ್ಭವಃ ॥ 35॥

ಯಕ್ಷೋ ಯಕ್ಷಪ್ರಿಯೋ ಯಾಮ್ಯೋ ರಾಮೋ ರಾಜೀವಲೋಚನಃ ।
ರಾತ್ರಿಂಚರೋ ರಾತ್ರಿಚರೋ ರಾಮೇಶೋ ರಾಮಪೂಜಿತಃ ॥ 36॥

ರಾಮಪೂಜ್ಯೋ ರಾಮನಾಥೋ ರತ್ನದೋ ರತ್ನಹಾರಕಃ ।
ರಾಜ್ಯದೋ ರಾಮವರದೋ ರಂಜಕೋ ರತಿಮಾರ್ಗಕೃತ್ ॥ 37॥

ರಮಣೀಯೋ ರಘುನಾಥೋ ರಘುವಂಶಪ್ರವರ್ತಕಃ ।
ರಾಮಾನನ್ದಪ್ರಿಯೋ ರಾಜಾ ರಾಜರಾಜೇಶ್ವರೋ ರಸಃ ॥ 38॥

ರತ್ನಮನ್ದಿರಮಧ್ಯಸ್ಥೋ ರತ್ನಪೂಜಾಪರಾಯಣಃ ।
ರತ್ನಾಕರೋ ಲಕ್ಷ್ಮಣೇಶೋ ಲಕ್ಷ್ಮಕೋ ಲಕ್ಷ್ಮಲಕ್ಷಣಃ ॥ 39॥

ಲಕ್ಷ್ಮೀನಾಥಪ್ರಿಯೋ ಲಾಲೀ ಲಮ್ಬಿಕಾಯೋಗಮಾರ್ಗಧೃತ್ ।
ಲಬ್ಧಲಕ್ಷ್ಯೋ ಲಬ್ಧಸಿದ್ಧಿರ್ಲಭ್ಯೋ ಲಾಕ್ಷಾರುಣೇಕ್ಷಣಃ ॥ 40॥

ಲೋಲಾಕ್ಷೀನಾಯಕೋ ಲೋಭೀ ಲೋಕನಾಥೋ ಲತಾಮಯಃ ।
ಲತಾಪೂಜಾಪರೋ ಲೀಲೋ ಲಕ್ಷಮನ್ತ್ರಜಪಪ್ರಿಯಃ ॥ 41॥

ಲಮ್ಬಿಕಾಮಾರ್ಗನಿರತೋ ಲಕ್ಷಕೋಟ್ಯಂಡನಾಯಕಃ ।
ವಾಣೀಪ್ರಿಯೋ ವಾಮಮಾರ್ಗೋ ವಾದೀ ವಾದಪರಾಯಣಃ ॥ 42॥

ವೀರಮಾರ್ಗರತೋ ವೀರೋ ವೀರಚರ್ಯಾಪರಾಯಣಃ ।
ವರೇಣ್ಯೋ ವರದೋ ವಾಮೋ ವಾಮಮಾರ್ಗಪ್ರವರ್ತಕಃ ॥ 43॥

ವಾಮದೇವೋ ವಾಗಧೀಶೋ ವೀಣಾಢ್ಯೋ ವೇಣುತತ್ಪರಃ ।
ವಿದ್ಯಾಪ್ರದೋ ವೀತಿಹೋತ್ರೋ ವೀರವಿದ್ಯಾವಿಶಾರದಃ ॥ 44॥

ವರ್ಗೋ ವರ್ಗಪ್ರಿಯೋ ವಾಯುಃ ವಾಯುವೇಗಪರಾಯಣಃ ।
ವಾರ್ತಜ್ಞಶ್ಚ ವಶೀಕಾರೀ ವರ್ಷಿಷ್ಠೋ ವಾಮಹರ್ಷಕಃ ॥ 45॥

ವಾಸಿಷ್ಠೋ ವಾಕ್ಪತಿರ್ವೇದ್ಯೋ ವಾಮನೋ ವಸುದೋ ವಿರಾಟ್ ।
ವಾರಾಹೀಪಾಲಕೋ ವಶ್ಯೋ ವನವಾಸೀ ವನಪ್ರಿಯಃ ॥ 46॥

ವನಪತಿರ್ವಾರಿಧಾರೀ ವೀರೋ ವಾರಾಂಗನಾಪ್ರಿಯಃ ।
ವನದುರ್ಗಾಪತಿರ್ವನ್ಯಃ ಶಕ್ತಿಪೂಜಾಪರಾಯಣಃ ॥ 47॥

ಶಶಾಂಕಮೌಲಿಃ ಶಾನ್ತಾತ್ಮಾ ಶಕ್ತಿಮಾರ್ಗಪರಾಯಣಃ ।
ಶರಚ್ಚನ್ದ್ರನಿಭಃ ಶಾನ್ತಃ ಶಕ್ತಿಃ ಸಂಶಯವರ್ಜಿತಃ ॥ 48॥

ಶಚೀಪತಿಃ ಶಕ್ರಪೂಜ್ಯಃ ಶರಸ್ಥಃ ಶಾಪವರ್ಜಿತಃ ।
ಶಾಪಾನುಗ್ರಾಹಕಃ ಶಂಖಪ್ರಿಯಃ ಶತ್ರುನಿಷೂದನಃ ॥ 49॥

ಶರೀರಯೋಗೀ ಶಾನ್ತಾರಿಃ ಶಕ್ತಾ ಶ್ರಮಗತಃ ಶುಭಃ ।
ಶುಕ್ರಪೂಜ್ಯಃ ಶುಕ್ರಭೋಗೀ ಶುಕ್ರಭಕ್ಷಣತತ್ಪರಃ ॥ 50॥

ಶಾರದಾನಾಯಕಃ ಶೌರಿಃ ಷಣ್ಮುಖಃ ಷಣ್ಮನಾಃ ಷಢಃ ।
ಷಂಡಃ ಷಡಂಗಃ ಷಟ್ಕಶ್ಚ ಷಡಧ್ವಂಯಾಗತತ್ಪರಃ ॥ 51॥

ಷಡಾಮ್ನಾಯರಹಸ್ಯಜ್ಞಃ ಷಷ್ಠೀಜಪಪರಾಯಣಃ ।
ಷಟ್ಚಕ್ರಭೇದನಃ ಷಷ್ಠೀನಾದಷಡ್ದರ್ಶನಪ್ರಿಯಃ ॥ 52॥

ಷಷ್ಠೀದೋಷಹರಃ ಷಟ್ಕಃ ಷಟ್ಶಾಸ್ರಾರ್ಥರಹಸ್ಯವಿತ್ ।
ಷಡ್ಭೂಮಿ ಹಿತಃ ಷಡ್ವರ್ಗಃ ಷಡೈಶ್ವರ್ಯಫಲಪ್ರದಃ ॥ 53॥

ಷಡ್ಗುಣಃ ಷಣ್ಮುಖಪ್ರೀತಃ ಷಷ್ಠಿಪಾಲಃ ಷಡಾತ್ಮಕಃ ।
ಷಟ್ಕೃತ್ತಿಕಾಸಮಾಜಸ್ಥಃ ಷಡಾಧಾರನಿವಾಸಕಃ ॥ 54॥

ಷೋಢಾನ್ಯಾಸಮಯಃ ಸಿನ್ಧುಃ ಸುನ್ದರಃ ಸುರಸುನ್ದರಃ ।
ಸುರಾಧ್ಯಕ್ಷಃ ಸುರಪತಿಃ ಸುಮುಖಃ ಸುಸಮಃ ಸುರಃ ॥ 55॥

ಸುಭಗಃ ಸರ್ವವಿತ್ಸೌಮ್ಯ ಸಿದ್ಧಮಾರ್ಗಪ್ರವರ್ತಕಃ ।
ಸಹಜಾನನ್ದಜಃ ಸಾಮ ಸರ್ವಶಾಸ್ತ್ರರಹಸ್ಯವಿತ್ ॥ 56॥

ಸಮಿದ್ಧೋಮಪ್ರಿಯಃ ಸರ್ವಃ ಸರ್ವಶಕ್ತಿಪ್ರಪೂಜಿತಃ ।
ಸುರದೇವಃ ಸುದೇವಶ್ಚ ಸನ್ಮಾರ್ಗಃ ಸಿದ್ಧದರ್ಶನಃ ॥ 57॥

ಸರ್ವವಿತ್ಸಾಧುವಿತ್ಸಾಧುಃ ಸರ್ವಧರ್ಮಸಮನ್ವಿತಃ ।
ಸರ್ವಾಧ್ಯಕ್ಷಃ ಸರ್ವವೇದ್ಯಃ ಸನ್ಮಾರ್ಗಸೂಚಕೋಽರ್ಥವಿತ್ ॥ 58॥

ಹಾರೀ ಹರಿರ್ಹರೋ ಹೃದ್ಯೋ ಹರೋ ಹರ್ಷಪ್ರದೋ ಹರಿಃ ।
ಹರಯೋಗೀ ಹೇಹರತೋ ಹರಿವಾಹೋ ಹರಿಧ್ವಜಃ ॥ 59॥

ಹ್ರಾದಿಮಾರ್ಗರತೋ ಹ್ರೀಂ ಚ ಹಾರೀತವರದಾಯಕಃ ।
ಹಾರೀತವರದೋ ಹೀನೋ ಹಿತಕೃದ್ಧುಂಕೃತಿರ್ಹವಿಃ ॥ 60॥

ಹವಿಷ್ಯಭುಗ್ ಹವಿಷ್ಯಾಶೀ ಹರಿದ್ವರ್ಣೋ ಹರಾತ್ಮಕಃ ।
ಹೈಹಯೇಶೋ ಹ್ರೀಂಕೃತಿಶ್ಚ ಹರಿಮಾನಸತೋಷಣಃ ॥ 61॥

ಹ್ರಾಂಂಕಾರಜಪಸನ್ತುಷ್ಟೋ ಹ್ರೀಂಕಾರಜಪಚಿಹ್ನಿತಃ ।
ಹಿತಕಾರೀ ಹರಿಣದೃಕ್ ಹಲಿತೋ ಹರನಾಯಕಃ ॥ 62॥

ಹಾರಪ್ರಿಯೋ ಹಾರರತೋ ಹಾಹಾಶಬ್ದಪರಾಯಣಃ ।
ಳಕಾರ ವರ್ಣಭೂಷಾಢ್ಯೋ ಳಕಾರೇಶೋ ಮಹಾಮುನಿಃ ॥ 63॥

ಳಕಾರಬೀಜನಿಲಯೋ ಳಾಂಳಿಂ ಮನ್ತ್ರಪ್ರವರ್ತಕಃ ।
ಕ್ಷೇಮಂಕರೀಪ್ರಿಯಃ ಕ್ಷಾಮ್ಯಃ ಕ್ಷಮಾಭೃತ್ಕ್ಷಣರಕ್ಷಕಃ ॥ 64॥

ಕ್ಷಾಂಕಾರಬೀಜನಿಲಯಃ ಕ್ಷೋಭಹೃತ್ ಕ್ಷೋಭವರ್ಜಿತಃ ।
ಕ್ಷೋಭಹಾರೀ ಕ್ಷೋಭಕಾರೀ ಕ್ಷ್ರೀಂ ಬೀಜ ಕ್ಷ್ರಾಂ ಸ್ವರೂಪಧೃತ್ ॥ 65॥

ಕ್ಷ್ರಾಂಕಾರಬೀಜನಿಲಯಃ ಕ್ಷೌಮಾಮ್ಬರವಿಭೂಷಿತಃ ।
ಕ್ಷೋಣೀರಥಃ ಪ್ರಿಯಕರಃ ಕ್ಷಮಾಪಾಲಃ ಕ್ಷಮಾಕರಃ ॥ 66॥

ಕ್ಷೇತ್ರಜ್ಞಃ ಕ್ಷೇತ್ರಪಾಲಶ್ಚ ಕ್ಷಯರೋಗಕ್ಷಯಂಕರಃ ।
ಕ್ಷಾಮೋದರಃ ಕ್ಷಾಮಗಾತ್ರಃ ಕ್ಷಾಮರೂಪಃ ಕ್ಷಯೋದರಃ ॥ 67॥

ಅದ್ಭುತೋಽನನ್ತವರದಃ ಅನಸೂಯುಃ ಪ್ರಿಯಂವದಃ ।
ಅತ್ರಿಪುತ್ರೋಽಗ್ನಿಗರ್ಭಶ್ಚ ಅಭೂತೋಽನನ್ತವಿಕ್ರಮಃ ॥ 68॥

ಆದಿಮಧ್ಯಾನ್ತರಹಿತಃ ಅಣಿಮಾದಿ ಗುಣಾಕರಃ ।
ಅಕ್ಷರೋಽಷ್ಟಗುಣೈಶ್ವರ್ಯಃ ಅರ್ಹೋಽನರ್ಹಃ ಸ ಉಚ್ಯತೇ ॥ 69॥

ಆದಿತ್ಯಶ್ಚಾಗುಣಶ್ಚಾತ್ಮಾ ಅಧ್ಯಾತ್ಮಪ್ರೀತಮಾನಸಃ ।
ಆದ್ಯಶ್ಚಾಮ್ರಪ್ರಿಯಶ್ಚಾಮ್ರ ಆಮ್ರಪುಷ್ಪವಿಭೂಷಿತಃ ॥ 70॥

ಆಮ್ರಪುಷ್ಪಪ್ರಿಯಃ ಪ್ರಾಣಃ ಆರ್ಷ ಆಮ್ರಾತಕೇಶ್ವರಃ ।
ಇಂಗಿತಜ್ಞಶ್ಚ ಇಷ್ಟಜ್ಞ ಇಷ್ಟಭದ್ರ ಇಷ್ಟಪ್ರದಸ್ತಥಾ ॥ 71॥

ಇಷ್ಟಾಪೂರ್ತಪ್ರಿಯಶ್ಚೇಷ್ಟ ಈಶ ಈಶ್ವರವಲ್ಲಭಃ ।
ಈಂಕಾರಶ್ಚೇಶ್ವರಾಧೀನಃ ಈಶತಟಿದಿನ್ದ್ರವಾಚಕಃ ॥ 72॥

ಉಕ್ಷಿರೂಕಾರಗರ್ಭಶ್ಚ ಊಕಾರಾಯ ನಮೋ ನಮಃ ।
ಊಹ್ಯ ಊಹವಿನಿರ್ಮುಕ್ತ ಊಷ್ಮಾ ಊಷ್ಮಮಣಿಸ್ತಥಾ ॥ 73॥

ಋದ್ಧಿಕಾರೀ ಋದ್ಧಿರೂಪೀ ಋದ್ಧಿಪ್ರಾವರ್ತಕೇಶ್ವರಃ ।
ೠಕಾರವರ್ಣಭೂಷಾಢ್ಯಃ ೠಕಾರಾಯ ನಮೋ ನಮಃ ॥ 74॥

ಌಕಾರಗರ್ಭೋ ೡಕಾರ ೡಂ ೡಂಕಾರಾಯ ತೇ ನಮಃ ।
ಏಕಾರಗರ್ಭಶ್ಚೈಕಾರಃ ಏಕಶ್ಚೈಕಪ್ರವಾಚಕಃ ॥ 75॥

ಏಕಂಕಾರಿಶ್ಚೈಕಕರ ಏಕಪ್ರಿಯತರಾಯ ತೇ ।
ಏಕವೀರ ಏಕಪತಿಃ ಏಂ ಐಂ ಶಬ್ದಪರಾಯಣಃ ॥ 76॥

ಐನ್ದ್ರಪ್ರಿಯಶ್ಚೈಕ್ಯಕಾರೀ ಐಂ ಬೀಜಜಪತತ್ಪರಃ ।
ಓಂಕಾರಶ್ಚೋಂಕಾರಬೀಜಃ ಓಂಕಾರಾಯ ನಮೋ ನಮಃ ॥ 77॥

ಓಂಕಾರಪೀಠನಿಲಯಃ ಓಂಕಾರೇಶ್ವರಪೂಜಿತಃ ।
ಅಂಕಿತೋತ್ತಮವರ್ಣಶ್ಚ ಅಂಕಿತಜ್ಞಾಯ ತೇ ನಮಃ ॥ 78॥

ಕಲಂಕಹರಃ ಕಂಕಾಲಃ ಕ್ರೂರಃ ಕುಕ್ಕುಟವಾಹನಃ ।
ಕಾಮಿನೀವಲ್ಲಭಃ ಕಾಮೀ ಕಾಮ್ಯಾರ್ಥಃ ಕಮನೀಯಕಃ ॥ 79॥

ಕಲಾನಿಧಿಃ ಕೀರ್ತಿನಾಥಃ ಕಾಮೇಶೀಹೃದಯಂಗಮಃ ।
ಕಾಮೇಶ್ವರಃ ಕಾಮರೂಪಃ ಕಾಲಃ ಕಾಲಕೃಪಾನಿಧಿಃ ॥ 80॥

ಕೃಷ್ಣಃ ಕಾಲೀಪತಿಃ ಕಾಲಿ ಕೃಶಚೂಡಾಮಣಿಃ ಕಲಃ ।
ಕೇಶವಃ ಕೇವಲಃ ಕಾನ್ತಃ ಕಾಲೀಶೋ ( ಶ) ವರದಾಯಕಃ ॥ 81॥

ಕಾಲಿಕಾಸಂಪ್ರದಾಯಜ್ಞಃ ಕಾಲಃ ಕಾಮಕಲಾತ್ಮಕಃ ।
ಖಟ್ವಾಂಗಪಾಣಿಃ ಖತಿತಃ ಖರಶೂಲಃ ಖರಾನ್ತಕೃತ್ ॥ 82॥

ಖೇಲನಃ ಖೇಟಕಃ ಖಡ್ಗಃ ಖಡ್ಗನಾಥಃ ಖಗೇಶ್ವರಃ ।
ಖೇಚರಃ ಖೇಚರನಾಥೋ ಗಣನಾಥಸಹೋದರಃ ॥ 83॥

ಗಾಢೋ ಗಹನಗಮ್ಭೀರೋ ಗೋಪಾಲೋ ಗೂರ್ಜರೋ ಗುರುಃ ।
ಗಣೇಶೋ ಗಾಯಕೋ ಗೋಪ್ತಾ ಗಾಯತ್ರೀವಲ್ಲಭೋ ಗುಣೀ ॥ 84॥

ಗೋಮನ್ತೋ ಗಾರುಡೋ ಗೌರೋ ಗೌರೀಶೋ ಗಿರಿಶೋ ಗುಹಃ ।
ಗೀರರ್ಗರ್ಯೋ ಗೋಪನೀಯೋ ಗೋಮಯೋ ಗೋಚರೋ ಗುಣಃ ॥ 85॥

ಹೇರಮ್ಬಾಯುಷ್ಯರುಚಿರೋ ಗಾಣಾಪತ್ಯಾಗಮಪ್ರಿಯಃ ।
ಘಂಟಾಕರ್ಣೋ ಘರ್ಮರಶ್ಮಿರ್ಘೃಣಿರ್ಘಂಟಾಪ್ರಿಯೋ ಘಟಃ ॥ 86॥

ಘಟಸರ್ಪೋ ಘೂರ್ಣಿತಶ್ಚ ಘೃಮಣಿರ್ಘೃತಕಮ್ಬಲಃ ।
ಘಂಟಾದಿನಾದರುಚಿರೋ ಘೃಣೀ ಲಜ್ಜಾವಿವರ್ಜಿತಃ ॥ 87॥

ಘೃಣಿಮನ್ತ್ರಜಪಪ್ರೀತೋ ಘೃತಯೋನಿರ್ಘೃತಪ್ರಿಯಃ ।
ಘರ್ಘರೋ ಘೋರನಾದಶ್ಚಾಘೋರಶಾಸ್ತ್ರಪ್ರವರ್ತಕಃ ॥ 88॥

ಘನಾಘನೋ ಘೋಷಯುಕ್ತೋ ಘೇಟಕೋ ಘೇಟಕೇಶ್ವರಃ ।
ಘನೋ ಘನರುಚಿಃ ಘ್ರಿಂ ಘ್ರಾಂ ಘ್ರಾಂ ಘ್ರಿಂ ಮನ್ತ್ರಸ್ವರೂಪಧೃತ್ ॥ 89॥

ಘನಶ್ಯಾಮೋ ಘನತರೋ ಘಟೋತ್ಕಚೋ ಘಟಾತ್ಮಜಃ ।
ಘಂಘಾದೋ ಘುರ್ಘುರೋ ಘೂಕೋ ಘಕಾರಾಯ ನಮೋ ನಮಃ ॥ 90॥

ಙಕಾರಾಖ್ಯೋ ಙಕಾರೇಶೋ ಙಕಾರಾಯ ನಮೋ ನಮಃ ।
ಙಕಾರಬೀಜನಿಲಯೋ ಙಾಂ ಙಿಂ ಮನ್ತ್ರಸ್ವರೂಪಧೃತ್ ॥ 91॥

ಚತುಷ್ಷಷ್ಟಿಕಲಾದಾಯೀ ಚತುರಶ್ಚಂಚಲಶ್ಚಲಃ ।
ಚಕ್ರೀ ಚಕ್ರಶ್ಚಕ್ರಧರಃ ಶ್ರೀಬೀಜಜಪತತ್ಪರಃ ॥ 92॥

ಚಂಡಶ್ಚಂಡೇಶ್ವರಶ್ಚಾರುಃ ಚಕ್ರಪಾಣಿಶ್ಚರಾಚರಃ ।
ಚರಾಚರಮಯಶ್ಚಿನ್ತಾಮಣಿಶ್ಚಿನ್ತಿತಸಾರಥಿಃ ॥ 93॥

ಚಂಡರಶ್ಮಿಶ್ಚನ್ದ್ರಮೌಲಿಶ್ಚಂಡೀಹೃದಯನನ್ದನಃ ।
ಚಕ್ರಾಂಕಿತಶ್ಚಂಡದೀಪ್ತಿಪ್ರಿಯಶ್ಚೂಡಾಲಶೇಖರಃ ॥ 94॥

ಚಂಡಶ್ಚಂಡಾಲದಮನಃ ಚಿನ್ತಿತಶ್ಚಿನ್ತಿತಾರ್ಥದಃ ।
ಚಿತ್ತಾರ್ಪಿತಶ್ಚಿತ್ತಮಾಯೀ ಚಿತ್ರವಿದ್ಯಾಮಯಶ್ಚ ಚಿತ್ ॥ 95॥

ಚಿಚ್ಛಕ್ತಿಶ್ಚೇತನಶ್ಚಿನ್ತ್ಯಃ ಚಿದಾಭಾಸಶ್ಚಿದಾತ್ಮಕಃ ।
ಛನ್ದಚಾರೀ ಛನ್ದಗತಿಶ್ಛಾತ್ರಶ್ಛಾತ್ರಪ್ರಿಯಶ್ಚ ಛಿತ್ ॥ 96॥

ಛೇದಕೃಚ್ಛೇದನಶ್ಛೇದಃ ಛನ್ದಃ ಶಾಸ್ತ್ರವಿಶಾರದಃ ।
ಛನ್ದೋಮಯಶ್ಚ ಛಾನ್ದೋಗ್ಯಶ್ಛನ್ದಸಾಂ ಪತಿರಿತ್ಯಪಿ ॥ 97॥

ಛನ್ದೋಭೇದಶ್ಛನ್ದನೀಯಃ ಛನ್ದಶ್ಛನ್ದೋರಹಸ್ಯವಿತ್ ।
ಛತ್ರಧಾರೀ ಛತ್ರಭೃತಶ್ಛತ್ರದಶ್ಛತ್ರಪಾಲಕಃ ॥ 98॥

ಛಿನ್ನಪ್ರಿಯಶ್ಛಿನ್ನಮಸ್ತಃ ಛಿನ್ನಮನ್ತ್ರಪ್ರಸಾದಕಃ ।
ಛಿನ್ನತಾಂಡವಸಮ್ಭೂತಃ ಛಿನ್ನಯೋಗವಿಶಾರದಃ ॥ 99॥

ಜಾಬಾಲಿಪೂಜ್ಯೋ ಜನ್ಮಾದ್ಯೋ ಜನಿತಾ ಜನ್ಮನಾಶಕಃ ।
ಜಪಾಯುಷ್ಯಪ್ರಿಯಕರೋ ಜಪಾದಾಡಿಮರಾಗಧೃತ್ ॥ 100॥

ಜಮಲೋ ಜೈನತೋ ಜನ್ಯೋ ಜನ್ಮಭೂಮಿರ್ಜನಪ್ರಿಯಃ ।
ಜನ್ಮಾದ್ಯಶ್ಚ ಪ್ರಿಯಕರೋ ಜನಿತಾ ಜಾಜಿರಾಗಧೃತ್ ॥ 101॥

ಜೈನಮಾರ್ಗರತೋ ಜೈನೋ ಜಿತಕ್ರೋಧೋ ಜಿತೇನ್ದ್ರಿಯಃ ।
ಜರ್ಜಜ್ಜಟೋ ಜರ್ಜಭೂಷೀ ಜಟಾಘಾರೋ ಜಟಾಧರಃ ॥ 102॥

ಜಗದ್ಗುರುರ್ಜಗತ್ಕಾರೀ ಜಾಮಾತೃವರದೋಽಜರಃ ।
ಜೀವನೋ ಜೀವನಾಧಾರೋ ಜ್ಯೋತಿಃಶಾಸ್ತ್ರವಿಶಾರದಃ ॥ 103॥

ಜ್ಯೋತಿರ್ಜ್ಯೋತ್ಸ್ನಾಮಯೋ ಜೇತಾ ಜಯೋ ಜನ್ಮಕೃತಾದರಃ ।
ಜಾಮಿತ್ರೋ ಜೈಮಿನೀಪುತ್ರೋ ಜ್ಯೋತಿಃಶಾಸ್ತ್ರಪ್ರವರ್ತಕಃ ॥ 104॥

ಜ್ಯೋತಿರ್ಲಿಂಗೋ ಜ್ಯೋತೀರೂಪೋ ಜೀಮೂತವರದಾಯಕಃ ।
ಜಿತೋ ಜೇತಾ ಜನ್ಮಪುತ್ರೋ ಜ್ಯೋತ್ಸ್ನಾಜಾಲಪ್ರವರ್ತಕಃ ॥ 105॥

ಜನ್ಮಾದಿನಾಶಕೋ ಜೀವೋ ಜೀವಾತುರ್ಜೀವನೌಷಧಮ್ ।
ಜರಾಹರೋ ಜಾಡ್ಯಹರೋ ಜನ್ಮಾಜನ್ಮವಿವರ್ಜಿತಃ ॥ 106॥

ಜನಕೋ ಜನನೀನಾಥೋ ಜೀಮೂತೋ ಜಾಮ್ಬವಪ್ರಿಯಃ ।
ಜಪಮೂರ್ತಿರ್ಜಗನ್ನಾಥೋ ಜಗತ್ಸ್ಥಾವರಜಂಗಮಃ ॥ 107॥

ಜಾರದೋ ಜಾರವಿದ್ಜಾರೋ ಜಠರಾಗ್ನಿಪ್ರವರ್ತಕಃ ।
ಜೀರ್ಣೋ ಜೀರ್ಣರತೋ ಜಾತಿಃ ಜಾತಿನಾಥೋ ಜಗನ್ಮಯಃ ॥ 108॥

ಜಗತ್ಪ್ರದೋ ಜಗತ್ತ್ರಾತಾ ಜಗಜ್ಜೀವನಕೌತುಕಃ ।
ಜಂಗಮೋ ಜಂಗಮಾಕಾರೋ ಜಟಿಲಶ್ಚ ಜಗದ್ಗುರುಃ ॥ 109॥

ಝೀರರ್ಝಂಝಾರಿಕೋ ಝಂಝೋ ಝಂಝಾನುರ್ಝರುಲನ್ದಕೃತ್ ।
ಝಕಾರಬೀಜನಿಲಯೋ ಝೂಂ ಝೂಂ ಝೂಂ ಮನ್ತ್ರರೂಪಧೃತ್ ॥ 110
ಜ್ಞಾನೇಶ್ವರೋ ಜ್ಞಾನಗಮ್ಯೋ ಜ್ಞಾನಮಾರ್ಗಪರಾಯಣಃ ।
ಜ್ಞಾನಕಾಂಡೀ ಜ್ಞೇಯಕಾಂಡೀ ಜ್ಞೇಯಾಜ್ಞೇಯವಿವರ್ಜಿತಃ ॥ 111॥

ಟಂಕಾಸ್ತ್ರಧಾರೀ ಟಂಕಾರಃ ಟೀಕಾಟಿಪ್ಪಣಕಾರಕಃ ।
ಟಾಂ ಟೀಂ ಟೂಂ ಜಪಸನ್ತುಷ್ಟೋ ಟಿಟ್ಟಿಭಷ್ಟಿಟ್ಟಿಭಾನನಃ ॥ 112॥

ಟಿಟ್ಟಿಭಾನನಸಹಿತಃ ಟಕಾರಾಕ್ಷರಭೂಷಿತಃ ।
ಟಂಕಾರಕಾರ್ಯಷ್ಟಸಿದ್ಧಿರಷ್ಟಮೂರ್ತ್ಯಷ್ಟಕಷ್ಟಹಾ ॥ 113॥॥

ಠಾಂಕುರಷ್ಠಕುರುಷ್ಠಷ್ಠಃ ಠಂ ಠೇ ಬೀಜಪರಾಯಣಃ ।
ಠಾಂ ಠೀಂ ಠೂಂ ಜಪಯೋಗಾಢ್ಯೋ ಡಾಮರೋ ಡಾಕಿನೀಪ್ರಿಯಃ ॥ 114॥

ಡಾಕಿನೀನಾಯಕೋ ಡಾಡಿಃ ಡೂಂ ಡೂಂ ಶಬ್ದಪರಾಯಣಃ ।
ಡಕಾರಾತ್ಮಾ ಡಾಮರಶ್ಚ ಡಾಮರೀಶಕ್ತಿರಂಜಿತಃ ॥ 115॥

ಡಾಕಾರೋ ಡಾಂಕರೋ ಡಾಂ ಡಿಂ ಡಿಂಡಿವಾದನತತ್ಪರಃ ।
ಡಕಾರಾಢ್ಯೋ ಡಂಕಹೀನೋ ಡಾಮರೀವಾದನಪ್ರಿಯಃ ॥ 116॥

ಢಾಂಕೃತಿಢಾಂ ಪತಿಃ ಢಾಂ ಢೀಂ ಢೂಂ ಢೈಂ ಢೌಂ ಶಬ್ದತತ್ಪರಃ ।
ಢೋಢಿಭೂಷಣ ಭೂಷಾಢ್ಯೋ ಢೀಂ ಢೀಂ ಪಾಲೋ ಢಪಾರಜಃ ॥ 117॥

ಣಕಾರಕುಂಡಲೋ ಣಾಡೀವರ್ಗಪ್ರಾಣೋ ಣಣಾದ್ರಿಭೂಃ ।
ಣಕಾರಪಂಜರೀಶಾಯ ಣಾಂ ಣಿಂ ಣೂಂ ಣಂ ಪ್ರವರ್ತಕಃ ॥ 118॥

ತರುಶಸ್ತರುಮಧ್ಯಸ್ಥಃ ತರ್ವನ್ತಸ್ತರುಮಧ್ಯಗಃ ।
ತಾರಕಸ್ತಾರತಮ್ಯಶ್ಚ ತಾರನಾಥಃ ಸನಾತನಃ ॥ 119॥

ತರುಣಸ್ತಾಮ್ರಚೂಡಶ್ಚ ತಮಿಸ್ರಾನಾಯಕಸ್ತಮೀ ।
ತೋತಸ್ತ್ರಿಪಥಗಸ್ತೀವ್ರಸ್ತೀವ್ರವೇಗಸ್ತ್ರಿಶಬ್ದಕೃತ್ ॥ 120॥

ತಾರಿಮತಸ್ತಾಲಧರಃ ತಪಃಶೀಲಸ್ತ್ರಪಾಕರಃ ।
ತನ್ತ್ರಮಾರ್ಗರತಸ್ತನ್ತ್ರಸ್ತಾನ್ತ್ರಿಕಸ್ತಾನ್ತ್ರಿಕೋತ್ತಮಃ ॥ 121॥

ತುಷಾರಾಚಲಮಧ್ಯಸ್ಥಃ ತುಷಾರವರಭೂಷಿತಃ ।
ತುರಸ್ತುಮ್ಬೀಫಲಪ್ರಾಣಸ್ತುಲಜಾಪುರನಾಯಕಃ ॥ 122
ತೀವ್ರಯಷ್ಟಿಕರಸ್ತೀವ್ರಸ್ತುಂಡದುರ್ಗಸಮಾಜಗಃ ।
ತ್ರಿವರ್ಗಯಜ್ಞಕೃತ್ತ್ರಯೀ ತ್ರ್ಯಮ್ಬಕಸ್ತ್ರಿಪುರಾನ್ತಕಃ ॥ 123॥

ತ್ರಿಪುರಾನ್ತಕಸಂಹಾರಸ್ತ್ರಿಧಾಮಾ ಸ್ತ್ರೀತೃತೀಯಕಃ ।
ತ್ರಿಲೋಕಮುದ್ರಿಕಾಭೂಷಃ ತ್ರಿಪಂಚನ್ಯಾಸಸಂಯುತಃ ॥ 124॥

ತ್ರಿಸುಗನ್ಧಿಸ್ತ್ರಿಮೂರ್ತಿರ್ಸ್ತ್ರಿಗುಣಸ್ತ್ರಿಗುಣಸಾರಥಿಃ ।
ತ್ರಯೀಮಯಶ್ಚ ತ್ರಿಗುಣಃ ತ್ರಿಪಾದಶ್ಚ ತ್ರಿಹಸ್ತಕಃ ॥ 125॥

ತನ್ತ್ರರೂಪಸ್ತ್ರಿಕೋಣೇಶಸ್ತ್ರಿಕಾಲಜ್ಞಸ್ತ್ರಯೀಮಯಃ ।
ತ್ರಿಸನ್ಧ್ಯಶ್ಚ ತ್ರಿಕಾಲಶ್ಚ ತಾಮ್ರಪರ್ಣೀಜಲಪ್ರಿಯಃ ॥ 126॥

ತೋಮರಸ್ತುಮುಲಃ ಸ್ಥೂಲಃ ಸ್ಥೂಲಪುರುಷರೂಪಧೃತ್ ।
ತತ್ತನ್ತ್ರೀ ತನ್ತ್ರತನ್ತ್ರೀ ತೃತೀಯಸ್ತರುಶೇಖರಃ ॥ 127॥

ತರುಣೇನ್ದುಶಿಖಸ್ತಾಲಸ್ತೀರ್ಥಸ್ನಾತಸ್ತ್ರಿಶೇಖರಃ ।
ತ್ರಿಜೋಽಜೇಶಸ್ತ್ರಿಸ್ವರೂಪಸ್ತ್ರಿತ್ರಿಶಬ್ದಪರಾಯಣಃ ॥ 128॥

ತಾರನಾಯಕಭೂಷಶ್ಚ ತರುವಾದನಚಂಚಲಃ ।
ತಿಷ್ಕಸ್ತ್ರಿರಾಶಿಕಸ್ತ್ರ್ಯಕ್ಷಃ ತರುಣಸ್ತಾಟವಾಹನಃ ॥ 129॥

ತೃತೀಯಸ್ತಾರಕಃ ಸ್ತಮ್ಭಃ ಸ್ತಮ್ಭಮಧ್ಯಗತಃ ಸ್ಥಿರಃ ।
ತತ್ತ್ವರೂಪಸ್ತಲಸ್ತಾಲಸ್ತಾನ್ತ್ರಿಕಸ್ತನ್ತ್ರಭೂಷಣಃ ॥ 130॥

ತಥ್ಯಸ್ತುತಿಮಯಃ ಸ್ಥೂಲಃ ಸ್ಥೂಲಬುದ್ಧಿಸ್ತ್ರಪಾಕರಃ ।
ತುಷ್ಟಃ ಸ್ತುತಿಮಯಃ ಸ್ತೋತಾ ಸ್ತೋತ್ರಪ್ರೀತಃ ಸ್ತುತೀಡಿತಃ ॥ 131॥

ತ್ರಿರಾಶಿಶ್ಚ ತ್ರಿಬನ್ಧುಶ್ಚ ತ್ರಿಪ್ರಸ್ತಾರಸ್ತ್ರಿಧಾಗತಿಃ ।
ತ್ರಿಕಾಲೇಶಸ್ತ್ರಿಕಾಲಜ್ಞಃ ತ್ರಿಜನ್ಮಾ ಚ ತ್ರಿಮೇಖಲಃ ॥ 132॥

ತ್ರಿದೋಷಶ್ಚ ತ್ರಿವರ್ಗಶ್ಚ ತ್ರೈರಾಶಿಕಫಲಪ್ರದಃ ।
ತನ್ತ್ರಸಿದ್ಧಸ್ತನ್ತ್ರರತಸ್ತನ್ತ್ರಸ್ತನ್ತ್ರಫಲಪ್ರದಃ ॥ 133॥

ತ್ರಿಪುರಾರಿಸ್ತ್ರಿಮಧುರಸ್ತ್ರಿಶಕ್ತಿಸ್ತ್ರಿಕತತ್ತ್ವಧೃತ್ ।
ತೀರ್ಥಪ್ರೀತಸ್ತೀರ್ಥರತಸ್ತೀರ್ಥೋದಾನಪರಾಯಣಃ ॥ 134॥

ತ್ರಯಕ್ಲೇಶಃ ತನ್ತ್ರಣೇಶಃ ತೀರ್ಥಶ್ರಾದ್ಧಫಲಪ್ರದಃ ।
ತೀರ್ಥಭೂಮಿರತಸ್ತೀರ್ಥಸ್ತಿತ್ತಿಡೀಫಲಭೋಜನಃ ॥ 135॥

ತಿತ್ತಿಡೀಫಲಭೂಷಾಢ್ಯಃ ತಾಮ್ರನೇತ್ರವಿಭೂಷಿತಃ ।
ತಕ್ಷಃ ಸ್ತೋತ್ರಪಾಠಪ್ರೀತಃ ಸ್ತೋತ್ರಮಯಃ ಸ್ತುತಿಪ್ರಿಯಃ ॥ 136॥

ಸ್ತವರಾಜಜಪಪ್ರಾಣಃ ಸ್ತವರಾಜಜಪಪ್ರಿಯಃ ।
ತೈಲಸ್ತಿಲಮನಾಸ್ತೈಲಪಕ್ವಾನ್ನಪ್ರೀತಮಾನಸಃ ॥ 137॥

ತೈಲಾಭಿಷೇಕಸನ್ತುಷ್ಟಃ ತೈಲಚರ್ವಣತತ್ಪರಃ ।
ತೈಲಾಹಾರಪ್ರಿಯಃ ಪ್ರಾಣಃ ತಿಲಮೋದಕತೋಷಣಃ ॥ 138॥

ತಿಲಪಿಷ್ಟಾನ್ನಭೋಜೀ ಚ ತಿಲಪರ್ವತರೂಪಧೃತ್ ।
ಥಕಾರ ಕೂಟನಿಲಯಃ ಥೈರಿಃ ಥೈಃ ಶಬ್ದತತ್ಪರಃ ॥ 139॥

ಥಿಮಾಥಿಮಾಥಿಮಾರೂಪಃ ಥೈ ಥೈ ಥೈ ನಾಟ್ಯನಾಯಕಃ ।
ಸ್ಥಾಣುರೂಪೋ ಮಹೇಶಾನಿ ಪ್ರೋಕ್ತನಾಮಸಹಸ್ರಕಮ್ ॥ 140॥

ಗೋಪ್ಯಾದ್ಗೋಪ್ಯಂ ಮಹೇಶಾನಿ ಸಾರಾತ್ಸಾರತರಂ ಪರಮ್ ।
ಜ್ಞಾನಕೈವಲ್ಯನಾಮಾಖ್ಯಂ ನಾಮಸಾಹಸ್ರಕಂ ಶಿವೇ ॥ 141॥

ಯಃ ಪಠೇತ್ಪ್ರಯತೋ ಭೂತ್ವಾ ಭಸ್ಮಭೂಷಿತವಿಗ್ರಹಃ ।
ರುದ್ರಾಕ್ಷಮೂಲಾಭರಣೋ ಭಕ್ತಿಮಾನ್ ಜಪತತ್ಪರಃ ॥ 142॥

ಸಹಸ್ರನಾಮ ಪ್ರಪಠೇತ್ ಜ್ಞಾನಕೈವಲ್ಯಕಾಭಿಧಮ್ ।
ಸರ್ವಸಿದ್ಧಿಮವಾಪ್ನೋತಿ ಸಾಕ್ಷಾತ್ಕಾರಂ ಚ ವಿನ್ದತಿ ॥ 143॥

ತತ್ತ್ವಮುದ್ರಾಂ ವಾಮಕರೇ ಕೃತ್ವಾ ನಾಮಸಹಸ್ರಕಮ್ ।
ಪ್ರಪಠೇತ್ಪಂಚಸಾಹಸ್ರಂ ಪುರಶ್ಚರಣಮುಚ್ಯತೇ ॥ 144॥

ಶಿವನಾಮ್ನಾ ಜಾತಭಾವೋ ವಾಙ್ಮನಃ ಕಾಯಕರ್ಮಭಿಃ ।
ಶಿವೋಽಹಮಿತಿ ವೈ ಧ್ಯಾಯನ್ನಾಮಸಾಹಸ್ರಕಂ ಪಠೇತ್ ॥ 145॥

ರೋಗಾರ್ತೋ ಮುಚ್ಯತೇ ರೋಗಾದ್ಬದ್ಧೋ ಮುಚ್ಯೇತ ಬನ್ಧನಾತ್ ।
ವಿದ್ಯಾರ್ಥೀ ಲಭತೇ ವಿದ್ಯಾಂ ಅಭೀಷ್ಟಂ ಲಭತೇ ತಥಾ ॥ 146॥

॥ ಇತಿ ಚಿದಮ್ಬರನಟನತನ್ತ್ರತಃ ಶ್ರೀದಕ್ಷಿಣಾಮೂರ್ತಿಸಹಸ್ರನಾಮಸ್ತೋತ್ರಂ ಸಮ್ಪೂರ್ಣಮ್ ॥
