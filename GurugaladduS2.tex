\fancyhead[RL]{}
\section{ ಶ್ರೀಅನ್ನಪೂರ್ಣಾಸಹಸ್ರನಾಮಸ್ತೋತ್ರಂ }
\addcontentsline{toc}{section}{ ಶ್ರೀಅನ್ನಪೂರ್ಣಾಸಹಸ್ರನಾಮಸ್ತೋತ್ರಂ }
ಓಂ ಅಸ್ಯ ಶ್ರೀಮದನ್ನಪೂರ್ಣಾಸಹಸ್ರನಾಮಸ್ತೋತ್ರಮಾಲಾಮಂತ್ರಸ್ಯ\\ ಶ್ರೀಭಗವಾನ್ ಋಷಿಃ । ಅನುಷ್ಟುಪ್ಛಂದಃ । ಪ್ರಕಟಗುಪ್ತಗುಪ್ತತರಸಂಪ್ರದಾಯಕುಲೋತ್ತೀರ್ಣನಿಗರ್ಭರಹಸ್ಯಾತಿರಹಸ್ಯಪರಾಪರಾತಿರಹಸ್ಯಾತಿಪೂರ್ವಾಚಿಂತ್ಯ\\ಪ್ರಭಾವಾ ಭಗವತೀ ಶ್ರೀಮದನ್ನಪೂರ್ಣಾ ದೇವತಾ । ಹಲೋ ಬೀಜಾನಿ । \\ಸ್ವರಾಃ ಶಕ್ತಯಃ । ಜೀವೋ ಬೀಜಂ । ಬುದ್ಧಿಃ ಶಕ್ತಿಃ । ಉದಾನೋ ಬೀಜಂ ।\\ ಸುಷುಮ್ನಾ ನಾಡೀ ಸರಸ್ವತೀ ಶಕ್ತಿಃ । ಜಪೇ ವಿನಿಯೋಗಃ ॥

\dhyana{ಅರ್ಕೋನ್ಮುಕ್ತಶಶಾಂಕಕೋಟಿವದನಾಮಾಪೀನತುಂಗಸ್ತನೀಂ\\
ಚಂದ್ರಾರ್ಧಾಂಕಿತಮಸ್ತಕಾಂ ಮಧುಮದಾಮಾಲೋಲನೇತ್ರತ್ರಯೀಂ~।\\
ಬಿಭ್ರಾಣಾಮನಿಶಂ ವರಂ ಜಪವಟೀಂ ಶೂಲಂ ಕಪಾಲಂ ಕರೈಃ\\
ಆದ್ಯಾಂ ಯೌವನಗರ್ವಿತಾಂ ಲಿಪಿತನುಂ ವಾಗೀಶ್ವರೀಮಾಶ್ರಯೇ ॥}

ಅನ್ನಪೂರ್ಣಾ ಅನ್ನದಾತ್ರೀ ಅನ್ನರಾಶಿಕೃತಾಲಯಾ~।\\
ಅನ್ನದಾ ಅನ್ನರೂಪಾ ಚ ಅನ್ನದಾನರತೋತ್ಸವಾ ॥೧॥

ಅನಂತಾ ಚ ಅನಂತಾಕ್ಷೀ ಅನಂತಗುಣಶಾಲಿನೀ~।\\
ಅಚ್ಯುತಾ ಅಚ್ಯುತಪ್ರಾಣಾ ಅಚ್ಯುತಾನಂದಕಾರಿಣೀ ॥೨॥

ಅವ್ಯಕ್ತಾಽನಂತಮಹಿಮಾ ಅನಂತಸ್ಯ ಕುಲೇಶ್ವರೀ~।\\
ಅಬ್ಧಿಸ್ಥಾ ಅಬ್ಧಿಶಯನಾ ಅಬ್ಧಿಜಾ ಅಬ್ಧಿನಂದಿನೀ ॥೩॥

ಅಬ್ಜಸ್ಥಾ ಅಬ್ಜನಿಲಯಾ ಅಬ್ಜಜಾ ಅಬ್ಜಭೂಷಣಾ~।\\
ಅಬ್ಜಾಭಾ ಅಬ್ಜಹಸ್ತಾ ಚ ಅಬ್ಜಪತ್ರಶುಭೇಕ್ಷಣಾ ॥೪॥

ಅಬ್ಜಾನನಾ ಅನಂತಾತ್ಮಾ ಅಗ್ನಿಸ್ಥಾ ಅಗ್ನಿರೂಪಿಣೀ~।\\
ಅಗ್ನಿಜಾಯಾ ಅಗ್ನಿಮುಖೀ ಅಗ್ನಿಕುಂಡಕೃತಾಲಯಾ ॥೫॥

ಅಕಾರಾ ಅಗ್ನಿಮಾತಾ ಚ ಅಜಯಾಽದಿತಿನಂದಿನೀ~।\\
ಆದ್ಯಾ ಆದಿತ್ಯಸಂಕಾಶಾ ಆತ್ಮಜ್ಞಾ ಆತ್ಮಗೋಚರಾ ॥೬॥

ಆತ್ಮಸೂರಾತ್ಮದಯಿತಾ ಆಧಾರಾ ಆತ್ಮರೂಪಿಣೀ~।\\
ಆಶಾ ಆಕಾಶಪದ್ಮಸ್ಥಾ ಅವಕಾಶಸ್ವರೂಪಿಣೀ ॥೭॥

ಆಶಾಪುರೀ ಅಗಾಧಾ ಚ ಅಣಿಮಾದಿಸುಸೇವಿತಾ~।\\
ಅಂಬಿಕಾ ಅಬಲಾ ಅಂಬಾ ಅನಾದ್ಯಾ ಚ ಅಯೋನಿಜಾ ॥೮॥

ಅನೀಶಾ ಈಶಿಕಾ ಈಶಾ ಈಶಾನೀ ಈಶ್ವರಪ್ರಿಯಾ~।\\
ಈಶ್ವರೀ ಈಶ್ವರಪ್ರಾಣಾ ಈಶ್ವರಾನಂದದಾಯಿನೀ ॥೯॥

ಇಂದ್ರಾಣೀ ಇಂದ್ರದಯಿತಾ ಇಂದ್ರಸೂರಿಂದ್ರಪಾಲಿನೀ~।\\
ಇಂದಿರಾ ಇಂದ್ರಭಗಿನೀ ಇಂದ್ರಿಯಾ ಇಂದುಭೂಷಣಾ ॥೧೦॥

ಇಂದುಮಾತಾ ಇಂದುಮುಖೀ ಇಂದ್ರಿಯಾಣಾಂ ವಶಂಕರೀ~।\\
ಉಮಾ ಉಮಾಪತೇಃ ಪ್ರಾಣಾ ಓಡ್ಯಾಣಪೀಠವಾಸಿನೀ ॥೧೧॥

ಉತ್ತರಜ್ಞಾ ಉತ್ತರಾಖ್ಯಾ ಉಕಾರಾ ಉತ್ತರಾತ್ಮಿಕಾ~।\\
ಋಮಾತಾ ಋಭವಾ ಋಸ್ಥಾ ೠಌೃಕಾರಸ್ವರೂಪಿಣೀ ॥೧೨॥

ಋಕಾರಾ ಚ ಌಕಾರಾ ಚ ಌತಕಪ್ರೀತಿದಾಯಿನೀ~।\\
ಏಕಾ ಚ ಏಕವೀರಾ ಚ ಏಕಾರೈಕಾರರೂಪಿಣೀ ॥೧೩॥

ಓಕಾರೀ ಓಘರೂಪಾ ಚ ಓಘತ್ರಯಸುಪೂಜಿತಾ~।\\
ಓಘಸ್ಥಾ ಓಘಸಂಭೂತಾ ಓಘಧಾತ್ರೀ ಚ ಓಘಸೂಃ ॥೧೪॥

ಷೋಡಶಸ್ವರಸಂಭೂತಾ ಷೋಡಶಸ್ವರರೂಪಿಣೀ~।\\
ವರ್ಣಾತ್ಮಾ ವರ್ಣನಿಲಯಾ ಶೂಲಿನೀ ವರ್ಣಮಾಲಿನೀ ॥೧೫॥

ಕಾಲರಾತ್ರಿರ್ಮಹಾರಾತ್ರಿರ್ಮೋಹರಾತ್ರಿಃ ಸುಲೋಚನಾ~।\\
ಕಾಲೀ ಕಪಾಲಿನೀ ಕೃತ್ಯಾ ಕಾಲಿಕಾ ಸಿಂಹಗಾಮಿನೀ ॥೧೬॥

ಕಾತ್ಯಾಯನೀ ಕಲಾಧಾರಾ ಕಾಲದೈತ್ಯನಿಕೃಂತನೀ~।\\
ಕಾಮಿನೀ ಕಾಮವಂದ್ಯಾ ಚ ಕಮನೀಯಾ ವಿನೋದಿನೀ ॥೧೭॥

ಕಾಮಸೂಃ ಕಾಮವನಿತಾ ಕಾಮಧುಕ್ ಕಮಲಾವತೀ~।\\
ಕಾಮದಾತ್ರೀ ಕರಾಲೀ ಚ ಕಾಮಕೇಲಿವಿನೋದಿನೀ ॥೧೮॥

ಕಾಮನಾ ಕಾಮದಾ ಕಾಮ್ಯಾ ಕಮಲಾ ಕಮಲಾರ್ಚಿತಾ~।\\
ಕಾಶ್ಮೀರಲಿಪ್ತವಕ್ಷೋಜಾ ಕಾಶ್ಮೀರದ್ರವಚರ್ಚಿತಾ ॥೧೯॥

ಕನಕಾ ಕನಕಪ್ರಾಣಾ ಕನಕಾಚಲವಾಸಿನೀ~।\\
ಕನಕಾಭಾ ಕಾನನಸ್ಥಾ ಕಾಮಾಖ್ಯಾ ಕನಕಪ್ರದಾ ॥೨೦॥

ಕಾಮಪೀಠಸ್ಥಿತಾ ನಿತ್ಯಾ ಕಾಮಧಾಮನಿವಾಸಿನೀ~।\\
ಕಂಬುಕಂಠೀ ಕರಾಲಾಕ್ಷೀ ಕಿಶೋರೀ ಚ ಕಲಾಪಿನೀ ॥೨೧॥

ಕಲಾ ಕಾಷ್ಠಾ ನಿಮೇಷಾ ಚ ಕಾಲಸ್ಥಾ ಕಾಲರೂಪಿಣೀ~।\\
ಕಾಲಜ್ಞಾ ಕಾಲಮಾತಾ ಚ ಕಾಲಧಾತ್ರೀ ಕಲಾವತೀ ॥೨೨॥

ಕಾಲದಾ ಕಾಲಹಾ ಕುಲ್ಯಾ ಕುರುಕುಲ್ಲಾ ಕುಲಾಂಗನಾ~।\\
ಕೀರ್ತಿದಾ ಕೀರ್ತಿಹಾ ಕೀರ್ತಿಃ ಕೀರ್ತಿಸ್ಥಾ ಕೀರ್ತಿವರ್ಧನೀ ॥೨೩॥

ಕೀರ್ತಿಜ್ಞಾ ಕೀರ್ತಿತಪದಾ ಕೃತ್ತಿಕಾ ಕೇಶವಪ್ರಿಯಾ~।\\
ಕೇಶಿಹಾ ಕೇಲಿಕಾರೀ ಚ ಕೇಶವಾನಂದಕಾರಿಣೀ ॥೨೪॥

ಕುಮುದಾಭಾ ಕುಮಾರೀ ಚ ಕರ್ಮದಾ ಕಮಲೇಕ್ಷಣಾ~।\\
ಕೌಮುದೀ ಕುಮುದಾನಂದಾ ಕೌಲಿನೀ ಚ ಕುಮುದ್ವತೀ ॥೨೫॥

ಕೋದಂಡಧಾರಿಣೀ ಕ್ರೋಧಾ ಕೂಟಸ್ಥಾ ಕೋಟರಾಶ್ರಯಾ~।\\
ಕಾಲಕಂಠೀ ಕರಾಲಾಂಗೀ ಕಾಲಾಂಗೀ ಕಾಲಭೂಷಣಾ ॥೨೬॥

ಕಂಕಾಲೀ ಕಾಮದಾಮಾ ಚ ಕಂಕಾಲಕೃತಭೂಷಣಾ~।\\
ಕಪಾಲಕರ್ತ್ರಿಕಕರಾ ಕರವೀರಸ್ವರೂಪಿಣೀ ॥೨೭॥

ಕಪರ್ದಿನೀ ಕೋಮಲಾಂಗೀ ಕೃಪಾಸಿಂಧುಃ ಕೃಪಾಮಯೀ~।\\
ಕುಶಾವತೀ ಕುಂಡಸಂಸ್ಥಾ ಕೌಬೇರೀ ಕೌಶಿಕೀ ತಥಾ ॥೨೮॥

ಕಾಶ್ಯಪೀ ಕದ್ರುತನಯಾ ಕಲಿಕಲ್ಮಷನಾಶಿನೀ~।\\
ಕಂಜಸ್ಥಾ ಕಂಜವದನಾ ಕಂಜಕಿಂಜಲ್ಕಚರ್ಚಿತಾ ॥೨೯॥

ಕಂಜಾಭಾ ಕಂಜಮಧ್ಯಸ್ಥಾ ಕಂಜನೇತ್ರಾ ಕಚೋದ್ಭವಾ~।\\
ಕಾಮರೂಪಾ ಚ ಹ್ರೀಂಕಾರೀ ಕಶ್ಯಪಾನ್ವಯವರ್ಧಿನೀ ॥೩೦॥

ಖರ್ವಾ ಚ ಖಂಜನದ್ವಂದ್ವಲೋಚನಾ ಖರ್ವವಾಹಿನೀ~।\\
ಖಡ್ಗಿನೀ ಖಡ್ಗಹಸ್ತಾ ಚ ಖೇಚರೀ ಖಡ್ಗರೂಪಿಣೀ ॥೩೧॥

ಖಗಸ್ಥಾ ಖಗರೂಪಾ ಚ ಖಗಗಾ ಖಗಸಂಭವಾ~।\\
ಖಗಧಾತ್ರೀ ಖಗಾನಂದಾ ಖಗಯೋನಿಸ್ವರೂಪಿಣೀ ॥೩೨॥

ಖಗೇಶೀ ಖೇಟಕಕರಾ ಖಗಾನಂದವಿವರ್ಧಿನೀ~।\\
ಖಗಮಾನ್ಯಾ ಖಗಾಧಾರಾ ಖಗಗರ್ವವಿಮೋಚಿನೀ ॥೩೩॥

ಗಂಗಾ ಗೋದಾವರೀ ಗೀತಿರ್ಗಾಯತ್ರೀ ಗಗನಾಲಯಾ~।\\
ಗೀರ್ವಾಣಸುಂದರೀ ಗೌಶ್ಚ ಗಾಧಾ ಗೀರ್ವಾಣಪೂಜಿತಾ ॥೩೪॥

ಗೀರ್ವಾಣಚರ್ಚಿತಪದಾ ಗಾಂಧಾರೀ ಗೋಮತೀ ತಥಾ~।\\
ಗರ್ವಿಣೀ ಗರ್ವಹಂತ್ರೀ ಚ ಗರ್ಭಸ್ಥಾ ಗರ್ಭಧಾರಿಣೀ ॥೩೫॥

ಗರ್ಭದಾ ಗರ್ಭಹಂತ್ರೀ ಚ ಗಂಧರ್ವಕುಲಪೂಜಿತಾ~।\\
ಗಯಾ ಗೌರೀ ಚ ಗಿರಿಜಾ ಗಿರಿಸ್ಥಾ ಗಿರಿಸಂಭವಾ ॥೩೬॥

ಗಿರಿಗಹ್ವರಮಧ್ಯಸ್ಥಾ ಕುಂಜರೇಶ್ವರಗಾಮಿನೀ~।\\
ಕಿರೀಟಿನೀ ಚ ಗದಿನೀ ಗುಂಜಾಹಾರವಿಭೂಷಣಾ ॥೩೭॥

ಗಣಪಾ ಗಣಕಾ ಗಣ್ಯಾ ಗಣಕಾನಂದಕಾರಿಣೀ~।\\
ಗಣಪೂಜ್ಯಾ ಚ ಗೀರ್ವಾಣೀ ಗಣಪಾನಂದಕಾರಿಣೀ ॥೩೮॥

ಗುರುಮಾತಾ ಗುರುರತಾ ಗುರುಭಕ್ತಿಪರಾಯಣಾ~।\\
ಗೋತ್ರಾ ಗೌಃ ಕೃಷ್ಣಭಗಿನೀ ಕೃಷ್ಣಸೂಃ ಕೃಷ್ಣನಂದಿನೀ ॥೩೯॥

ಗೋವರ್ಧನೀ ಗೋತ್ರಧರಾ ಗೋವರ್ಧನಕೃತಾಲಯಾ~।\\
ಗೋವರ್ಧನಧರಾ ಗೋದಾ ಗೌರಾಂಗೀ ಗೌತಮಾತ್ಮಜಾ ॥೪೦॥

ಘರ್ಘರಾ ಘೋರರೂಪಾ ಚ ಘೋರಾ ಘರ್ಘರನಾದಿನೀ~।\\
ಶ್ಯಾಮಾ ಘನರವಾಽಘೋರಾ ಘನಾ ಘೋರಾರ್ತಿನಾಶಿನೀ ॥೪೧॥

ಘನಸ್ಥಾ ಚ ಘನಾನಂದಾ ದಾರಿದ್ರ್ಯಘನನಾಶಿನೀ~।\\
ಚಿತ್ತಜ್ಞಾ ಚಿಂತಿತಪದಾ ಚಿತ್ತಸ್ಥಾ ಚಿತ್ತರೂಪಿಣೀ ॥೪೨॥

ಚಕ್ರಿಣೀ ಚಾರುಚಂಪಾಭಾ ಚಾರುಚಂಪಕಮಾಲಿನೀ~।\\
ಚಂದ್ರಿಕಾ ಚಂದ್ರಕಾಂತಿಶ್ಚ ಚಾಪಿನೀ ಚಂದ್ರಶೇಖರಾ ॥೪೩॥

ಚಂಡಿಕಾ ಚಂಡದೈತ್ಯಘ್ನೀ ಚಂದ್ರಶೇಖರವಲ್ಲಭಾ~।\\
ಚಾಂಡಾಲಿನೀ ಚ ಚಾಮುಂಡಾ ಚಂಡಮುಂಡವಧೋದ್ಯತಾ ॥೪೪॥

ಚೈತನ್ಯಭೈರವೀ ಚಂಡಾ ಚೈತನ್ಯಘನಗೇಹಿನೀ~।\\
ಚಿತ್ಸ್ವರೂಪಾ ಚಿದಾಧಾರಾ ಚಂಡವೇಗಾ ಚಿದಾಲಯಾ ॥೪೫॥

ಚಂದ್ರಮಂಡಲಮಧ್ಯಸ್ಥಾ ಚಂದ್ರಕೋಟಿಸುಶೀತಲಾ~।\\
ಚಪಲಾ ಚಂದ್ರಭಗಿನೀ ಚಂದ್ರಕೋಟಿನಿಭಾನನಾ ॥೪೬॥

ಚಿಂತಾಮಣಿಗುಣಾಧಾರಾ ಚಿಂತಾಮಣಿವಿಭೂಷಣಾ~।\\
ಭಕ್ತಚಿಂತಾಮಣಿಲತಾ ಚಿಂತಾಮಣಿಕೃತಾಲಯಾ ॥೪೭॥

ಚಾರುಚಂದನಲಿಪ್ತಾಂಗೀ ಚತುರಾ ಚ ಚತುರ್ಮುಖೀ~।\\
ಚೈತನ್ಯದಾ ಚಿದಾನಂದಾ ಚಾರುಚಾಮರವೀಜಿತಾ ॥೪೮॥

ಛತ್ರದಾ ಛತ್ರಧಾರೀ ಚ ಛಲಚ್ಛದ್ಮವಿನಾಶಿನೀ~।\\
ಛತ್ರಹಾ ಛತ್ರರೂಪಾ ಚ ಛತ್ರಚ್ಛಾಯಾಕೃತಾಲಯಾ ॥೪೯॥

ಜಗಜ್ಜೀವಾ ಜಗದ್ಧಾತ್ರೀ ಜಗದಾನಂದಕಾರಿಣೀ~।\\
ಯಜ್ಞಪ್ರಿಯಾ ಯಜ್ಞರತಾ ಜಪಯಜ್ಞಪರಾಯಣಾ ॥೫೦॥

ಜನನೀ ಜಾನಕೀ ಯಜ್ವಾ ಯಜ್ಞಹಾ ಯಜ್ಞನಂದಿನೀ~।\\
ಯಜ್ಞದಾ ಯಜ್ಞಫಲದಾ ಯಜ್ಞಸ್ಥಾನಕೃತಾಲಯಾ ॥೫೧॥

ಯಜ್ಞಭೋಕ್ತ್ರೀ ಯಜ್ಞರೂಪಾ ಯಜ್ಞವಿಘ್ನವಿನಾಶಿನೀ~।\\
ಜಪಾಕುಸುಮಸಂಕಾಶಾ ಜಪಾಕುಸುಮಶೋಭಿತಾ ॥೫೨॥

ಜಾಲಂಧರೀ ಜಯಾ ಜೈತ್ರೀ ಜೀಮೂತಚಯಭಾಷಿಣೀ~।\\
ಜಯದಾ ಜಯರೂಪಾ ಚ ಜಯಸ್ಥಾ ಜಯಕಾರಿಣೀ ॥೫೩॥

ಜಗದೀಶಪ್ರಿಯಾ ಜೀವಾ ಜಲಸ್ಥಾ ಜಲಜೇಕ್ಷಣಾ~।\\
ಜಲರೂಪಾ ಜಹ್ನುಕನ್ಯಾ ಯಮುನಾ ಜಲಜೋದರೀ ॥೫೪॥

ಜಲಜಾಸ್ಯಾ ಜಾಹ್ನವೀ ಚ ಜಲಜಾಭಾ ಜಲೋದರೀ~।\\
ಯದುವಂಶೋದ್ಭವಾ ಜೀವಾ ಯಾದವಾನಂದಕಾರಿಣೀ ॥೫೫॥

ಯಶೋದಾ ಯಶಸಾಂ ರಾಶಿರ್ಯಶೋದಾನಂದಕಾರಿಣೀ~।\\
ಜ್ವಲಿನೀ ಜ್ವಾಲಿನೀ ಜ್ವಾಲಾ ಜ್ವಲತ್ಪಾವಕಸನ್ನಿಭಾ ॥೫೬॥

ಜ್ವಾಲಾಮುಖೀ ಜಗನ್ಮಾತಾ ಯಮಲಾರ್ಜುನಭಂಜನೀ~।\\
ಜನ್ಮದಾ ಜನ್ಮಹಾ ಜನ್ಯಾ ಜನ್ಮಭೂರ್ಜನಕಾತ್ಮಜಾ ॥೫೭॥

ಜನಾನಂದಾ ಜಾಂಬವತೀ ಜಂಬೂದ್ವೀಪಕೃತಾಲಯಾ~।\\
ಜಾಂಬೂನದಸಮಾನಾಭಾ ಜಾಂಬೂನದವಿಭೂಷಣಾ ॥೫೮॥

ಜಂಭಹಾ ಜಾತಿದಾ ಜಾತಿರ್ಜ್ಞಾನದಾ ಜ್ಞಾನಗೋಚರಾ~।\\
ಜ್ಞಾನರೂಪಾಽಜ್ಞಾನಹಾ ಚ ಜ್ಞಾನವಿಜ್ಞಾನಶಾಲಿನೀ ॥೫೯॥

ಜಿನಜೈತ್ರೀ ಜಿನಾಧಾರಾ ಜಿನಮಾತಾ ಜಿನೇಶ್ವರೀ~।\\
ಜಿತೇಂದ್ರಿಯಾ ಜನಾಧಾರಾ ಅಜಿನಾಂಬರಧಾರಿಣೀ ॥೬೦॥

ಶಂಭುಕೋಟಿದುರಾಧರ್ಷಾ ವಿಷ್ಣುಕೋಟಿವಿಮರ್ದಿನೀ~।\\
ಸಮುದ್ರಕೋಟಿಗಂಭೀರಾ ವಾಯುಕೋಟಿಮಹಾಬಲಾ ॥೬೧॥

ಸೂರ್ಯಕೋಟಿಪ್ರತೀಕಾಶಾ ಯಮಕೋಟಿದುರಾಪಹಾ~।\\
ಕಾಮಧುಕ್ಕೋಟಿಫಲದಾ ಶಕ್ರಕೋಟಿಸುರಾಜ್ಯದಾ ॥೬೨॥

ಕಂದರ್ಪಕೋಟಿಲಾವಣ್ಯಾ ಪದ್ಮಕೋಟಿನಿಭಾನನಾ~।\\
ಪೃಥ್ವೀಕೋಟಿಜನಾಧಾರಾ ಅಗ್ನಿಕೋಟಿಭಯಂಕರೀ ॥೬೩॥

ಅಣಿಮಾ ಮಹಿಮಾ ಪ್ರಾಪ್ತಿರ್ಗರಿಮಾ ಲಘಿಮಾ ತಥಾ~।\\
ಪ್ರಾಕಾಮ್ಯದಾ ವಶಕರೀ ಈಶಿಕಾ ಸಿದ್ಧಿದಾ ತಥಾ ॥೬೪॥

ಮಹಿಮಾದಿಗುಣೋಪೇತಾ ಅಣಿಮಾದ್ಯಷ್ಟಸಿದ್ಧಿದಾ~।\\
ಜವನಘ್ನೀ ಜನಾಧೀನಾ ಜಾಮಿನೀ ಚ ಜರಾಪಹಾ ॥೬೫॥

ತಾರಿಣೀ ತಾರಿಕಾ ತಾರಾ ತೋತಲಾ ತುಲಸೀಪ್ರಿಯಾ~।\\
ತಂತ್ರಿಣೀ ತಂತ್ರರೂಪಾ ಚ ತಂತ್ರಜ್ಞಾ ತಂತ್ರಧಾರಿಣೀ ॥೬೬॥

ತಾರಹಾರಾ ಚ ತುಲಜಾ ಡಾಕಿನೀತಂತ್ರಗೋಚರಾ~।\\
ತ್ರಿಪುರಾ ತ್ರಿದಶಾ ತ್ರಿಸ್ಥಾ ತ್ರಿಪುರಾಸುರಘಾತಿನೀ ॥೬೭॥

ತ್ರಿಗುಣಾ ಚ ತ್ರಿಕೋಣಸ್ಥಾ ತ್ರಿಮಾತ್ರಾ ತ್ರಿತನುಸ್ಥಿತಾ~।\\
ತ್ರೈವಿದ್ಯಾ ಚ ತ್ರಯೀ ತ್ರಿಘ್ನೀ ತುರೀಯಾ ತ್ರಿಪುರೇಶ್ವರೀ ॥೬೮॥

ತ್ರಿಕೋದರಸ್ಥಾ ತ್ರಿವಿಧಾ ತ್ರೈಲೋಕ್ಯಾ ತ್ರಿಪುರಾತ್ಮಿಕಾ~।\\
ತ್ರಿಧಾಮ್ನೀ ತ್ರಿದಶಾರಾಧ್ಯಾ ತ್ರ್ಯಕ್ಷಾ ತ್ರಿಪುರವಾಸಿನೀ ॥೬೯॥

ತ್ರಿವರ್ಣೀ ತ್ರಿಪದೀ ತಾರಾ ತ್ರಿಮೂರ್ತಿಜನನೀ ತ್ವರಾ~।\\
ತ್ರಿದಿವಾ ತ್ರಿದಿವೇಶಾಽಽದಿದೇವೀ ತ್ರೈಲೋಕ್ಯಧಾರಿಣೀ ॥೭೦॥

ತ್ರಿಮೂರ್ತಿಶ್ಚ ತ್ರಿಜನನೀ ತ್ರೀಭೂಸ್ತ್ರೀಪುರಸುಂದರೀ~।\\
ತಪಸ್ವಿನೀ ತಪೋನಿಷ್ಠಾ ತರುಣೀ ತಾರರೂಪಿಣೀ ॥೭೧॥

ತಾಮಸೀ ತಾಪಸೀ ಚೈವ ತಾಪಘ್ನೀ ಚ ತಮೋಪಹಾ~।\\
ತರುಣಾರ್ಕಪ್ರತೀಕಾಶಾ ತಪ್ತಕಾಂಚನಸನ್ನಿಭಾ ॥೭೨॥

ಉನ್ಮಾದಿನೀ ತಂತುರೂಪಾ ತ್ರೈಲೋಕ್ಯವ್ಯಾಪಿನೀಶ್ವರೀ~।\\
ತಾರ್ಕಿಕೀ ತರ್ಕವಿದ್ಯಾ ಚ ತಾಪತ್ರಯವಿನಾಶಿನೀ ॥೭೩॥

ತ್ರಿಪುಷ್ಕರಾ ತ್ರಿಕಾಲಜ್ಞಾ ತ್ರಿಸಂಧ್ಯಾ ಚ ತ್ರಿಲೋಚನಾ~।\\
ತ್ರಿವರ್ಗಾ ಚ ತ್ರಿವರ್ಗಸ್ಥಾ ತಪಸಸ್ಸಿದ್ಧಿದಾಯಿನೀ ॥೭೪॥

ಅಧೋಕ್ಷಜಾ ಅಯೋಧ್ಯಾ ಚ ಅಪರ್ಣಾ ಚ ಅವಂತಿಕಾ~।\\
ಕಾರಿಕಾ ತೀರ್ಥರೂಪಾ ಚ ತೀರಾ ತೀರ್ಥಕರೀ ತಥಾ ॥೭೫॥

ದಾರಿದ್ರ್ಯದುಃಖದಲಿನೀ ಅದೀನಾ ದೀನವತ್ಸಲಾ~।\\
ದೀನನಾಥಪ್ರಿಯಾ ದೀರ್ಘಾ ದಯಾಪೂರ್ಣಾ ದಯಾತ್ಮಿಕಾ ॥೭೬॥

ದೇವದಾನವಸಂಪೂಜ್ಯಾ ದೇವಾನಾಂ ಪ್ರಿಯಕಾರಿಣೀ~।\\
ದಕ್ಷಪುತ್ರೀ ದಕ್ಷಮಾತಾ ದಕ್ಷಯಜ್ಞವಿನಾಶಿನೀ ॥೭೭॥

ದೇವಸೂರ್ದಕ್ಷಿಣಾ ದಕ್ಷಾ ದುರ್ಗಾ ದುರ್ಗತಿನಾಶಿನೀ~।\\
ದೇವಕೀಗರ್ಭಸಂಭೂತಾ ದುರ್ಗದೈತ್ಯವಿನಾಶಿನೀ ॥೭೮॥

ಅಟ್ಟಾಽಟ್ಟಹಾಸಿನೀ ದೋಲಾ ದೋಲಾಕರ್ಮಾಭಿನಂದಿನೀ~।\\
ದೇವಕೀ ದೇವಿಕಾ ದೇವೀ ದುರಿತಘ್ನೀ ತಟಿತ್ತಥಾ ॥೭೯॥

ಗಂಡಕೀ ಗಲ್ಲಕೀ ಕ್ಷಿಪ್ರಾ ದ್ವಾರಾ ದ್ವಾರವತೀ ತಥಾ~।\\
ಆನಂದೋದಧಿಮಧ್ಯಸ್ಥಾ ಕಟಿಸೂತ್ರೈರಲಂಕೃತಾ ॥೮೦॥

ಘೋರಾಗ್ನಿದಾಹದಮನೀ ದುಃಖದುಸ್ಸ್ವಪ್ನನಾಶಿನೀ~।\\
ಶ್ರೀಮಯೀ ಶ್ರೀಮತೀ ಶ್ರೇಷ್ಠಾ ಶ್ರೀಕರೀ ಶ್ರೀವಿಭಾವಿನೀ ॥೮೧॥

ಶ್ರೀದಾ ಶ್ರೀಶಾ ಶ್ರೀನಿವಾಸಾ ಶ್ರೀಮತೀ ಶ್ರೀರ್ಮತಿರ್ಗತಿಃ~।\\
ಧನದಾ ದಾಮಿನೀ ದಾಂತಾ ಧರ್ಮದಾ ಧನಶಾಲಿನೀ ॥೮೨॥

ದಾಡಿಮೀಪುಷ್ಪಸಂಕಾಶಾ ಧನಾಗಾರಾ ಧನಂಜಯಾ~।\\
ಧೂಮ್ರಾಭಾ ಧೂಮ್ರದೈತ್ಯಘ್ನೀ ಧವಲಾ ಧವಲಪ್ರಿಯಾ ॥೮೩॥

ಧೂಮ್ರವಕ್ತ್ರಾ ಧೂಮ್ರನೇತ್ರಾ ಧೂಮ್ರಕೇಶೀ ಚ ಧೂಸರಾ~।\\
ಧರಣೀ ಧಾರಿಣೀ ಧೈರ್ಯಾ ಧರಾ ಧಾತ್ರೀ ಚ ಧೈರ್ಯದಾ ॥೮೪॥

ದಮನೀ ಧರ್ಮಿಣೀ ಧೂಶ್ಚ ದಯಾ ದೋಗ್ಧ್ರೀ ದುರಾಸದಾ~।\\
ನಾರಾಯಣೀ ನಾರಸಿಂಹೀ ನೃಸಿಂಹಹೃದಯಾಲಯಾ ॥೮೫॥

ನಾಗಿನೀ ನಾಗಕನ್ಯಾ ಚ ನಾಗಸೂರ್ನಾಗನಾಯಿಕಾ~।\\
ನಾನಾರತ್ನವಿಚಿತ್ರಾಂಗೀ ನಾನಾಭರಣಮಂಡಿತಾ ॥೮೬॥

ದುರ್ಗಸ್ಥಾ ದುರ್ಗರೂಪಾ ಚ ದುಃಖದುಷ್ಕೃತನಾಶಿನೀ~।\\
ಹ್ರೀಂಕಾರೀ ಚೈವ ಶ್ರೀಂಕಾರೀ ಹುಂಕಾರೀ ಕ್ಲೇಶನಾಶಿನೀ ॥೮೭॥

ನಗಾತ್ಮಜಾ ನಾಗರೀ ಚ ನವೀನಾ ನೂತನಪ್ರಿಯಾ~।\\
ನೀರಜಾಸ್ಯಾ ನೀರದಾಭಾ ನವಲಾವಣ್ಯಸುಂದರೀ ॥೮೮॥

ನೀತಿಜ್ಞಾ ನೀತಿದಾ ನೀತಿರ್ನಿಮ್ನಾಭಿರ್ನಗೇಶ್ವರೀ~।\\
ನಿಷ್ಠಾ ನಿತ್ಯಾ ನಿರಾತಂಕಾ ನಾಗಯಜ್ಞೋಪವೀತಿನೀ ॥೮೯॥

ನಿಧಿದಾ ನಿಧಿರೂಪಾ ಚ ನಿರ್ಗುಣಾ ನರವಾಹಿನೀ~।\\
ನರಮಾಂಸರತಾ ನಾರೀ ನರಮುಂಡವಿಭೂಷಣಾ ॥೯೦॥

ನಿರಾಧಾರಾ ನಿರ್ವಿಕಾರಾ ನುತಿರ್ನಿರ್ವಾಣಸುಂದರೀ~।\\
ನರಾಸೃಕ್ಪಾನಮತ್ತಾ ಚ ನಿರ್ವೈರಾ ನಾಗಗಾಮಿನೀ ॥೯೧॥

ಪರಮಾ ಪ್ರಮಿತಾ ಪ್ರಾಜ್ಞಾ ಪಾರ್ವತೀ ಪರ್ವತಾತ್ಮಜಾ~।\\
ಪರ್ವಪ್ರಿಯಾ ಪರ್ವರತಾ ಪರ್ವಪಾವನಪಾವನೀ ॥೯೨॥

ಪರಾತ್ಪರತರಾ ಪೂರ್ವಾ ಪಶ್ಚಿಮಾ ಪಾಪನಾಶಿನೀ~।\\
ಪಶೂನಾಂ ಪತಿಪತ್ನೀ ಚ ಪತಿಭಕ್ತಿಪರಾಯಣಾ ॥೯೩॥

ಪರೇಶೀ ಪಾರಗಾ ಪಾರಾ ಪರಂಜ್ಯೋತಿಸ್ವರೂಪಿಣೀ~।\\
ನಿಷ್ಠುರಾ ಕ್ರೂರಹೃದಯಾ ಪರಾಸಿದ್ಧಿಃ ಪರಾಗತಿಃ ॥೯೪॥

ಪಶುಘ್ನೀ ಪಶುರೂಪಾ ಚ ಪಶುಹಾ ಪಶುವಾಹಿನೀ~।\\
ಪಿತಾ ಮಾತಾ ಚ ಯಂತ್ರೀ ಚ ಪಶುಪಾಶವಿನಾಶಿನೀ ॥೯೫॥

ಪದ್ಮಿನೀ ಪದ್ಮಹಸ್ತಾ ಚ ಪದ್ಮಕಿಂಜಲ್ಕವಾಸಿನೀ~।\\
ಪದ್ಮವಕ್ತ್ರಾ ಚ ಪದ್ಮಾಕ್ಷೀ ಪದ್ಮಸ್ಥಾ ಪದ್ಮಸಂಭವಾ ॥೯೬॥

ಪದ್ಮಾಸ್ಯಾ ಪಂಚಮೀ ಪೂರ್ಣಾ ಪೂರ್ಣಪೀಠನಿವಾಸಿನೀ~।\\
ಪದ್ಮರಾಗಪ್ರತೀಕಾಶಾ ಪಾಂಚಾಲೀ ಪಂಚಮಪ್ರಿಯಾ ॥೯೭॥

ಪರಬ್ರಹ್ಮಸ್ವರೂಪಾ ಚ ಪರಬ್ರಹ್ಮನಿವಾಸಿನೀ~।\\
ಪರಮಾನಂದಮುದಿತಾ ಪರಚಕ್ರನಿವಾಸಿನೀ ॥೯೮॥

ಪರೇಶೀ ಪರಮಾ ಪೃಥ್ವೀ ಪೀನತುಂಗಪಯೋಧರಾ~।\\
ಪರಾಪರಾ ಪರಾವಿದ್ಯಾ ಪರಮಾನಂದದಾಯಿನೀ ॥೯೯॥

ಪೂಜ್ಯಾ ಪ್ರಜ್ಞಾವತೀ ಪುಷ್ಟಿಃ ಪಿನಾಕಿಪರಿಕೀರ್ತಿತಾ~।\\
ಪ್ರಾಣಘ್ನೀ ಪ್ರಾಣರೂಪಾ ಚ ಪ್ರಾಣದಾ ಚ ಪ್ರಿಯಂವದಾ ॥೧೦೦॥

ಫಣಿಭೂಷಾ ಫಣಾವೇಶೀ ಫಕಾರಾಕುಂಠಮಾಲಿನೀ~।\\
ಫಣಿರಾಡ್ವೃತಸರ್ವಾಂಗೀ ಫಲಭಾಗನಿವಾಸಿನೀ ॥೧೦೧॥

ಬಲಭದ್ರಸ್ಯ ಭಗಿನೀ ಬಾಲಾ ಬಾಲಪ್ರದಾಯಿನೀ~।\\
ಫಲ್ಗುರುಪಾ ಪ್ರಲಂಬಧ್ನೀ ಫಲ್ಗೂತ್ಸವವಿನೋದಿನೀ ॥೧೦೨॥

ಭವಾನೀ ಭವಪತ್ನೀ ಚ ಭವಭೀತಿಹರಾ ಭವಾ~।\\
ಭವೇಶ್ವರೀ ಭವಾರಾಧ್ಯಾ ಭವೇಶೀ ಭವನಾಯಿಕಾ ॥೧೦೩॥

ಭವಮಾತಾ ಭವಾಗಮ್ಯಾ ಭವಕಂಟಕನಾಶಿನೀ~।\\
ಭವಪ್ರಿಯಾ ಭವಾನಂದಾ ಭವ್ಯಾ ಚ ಭವಮೋಚನೀ ॥೧೦೪॥

ಭಾವನೀಯಾ ಭಗವತೀ ಭವಭಾರವಿನಾಶಿನೀ~।\\
ಭೂತಧಾತ್ರೀ ಚ ಭೂತೇಶೀ ಭೂತಸ್ಥಾ ಭೂತರೂಪಿಣೀ ॥೧೦೫॥

ಭೂತಮಾತಾ ಚ ಭೂತಘ್ನೀ ಭೂತಪಂಚಕವಾಸಿನೀ~।\\
ಭೋಗೋಪಚಾರಕುಶಲಾ ಭಿಸ್ಸಾಧಾತ್ರೀ ಚ ಭೂಚರೀ ॥೧೦೬॥

ಭೀತಿಘ್ನೀ ಭಕ್ತಿಗಮ್ಯಾ ಚ ಭಕ್ತಾನಾಮಾರ್ತಿನಾಶಿನೀ~।\\
ಭಕ್ತಾನುಕಂಪಿನೀ ಭೀಮಾ ಭಗಿನೀ ಭಗನಾಯಿಕಾ ॥೧೦೭॥

ಭಗವಿದ್ಯಾ ಭಗಕ್ಲಿನ್ನಾ ಭಗಯೋನಿರ್ಭಗಪ್ರದಾ~।\\
ಭಗೇಶೀ ಭಗರೂಪಾ ಚ ಭಗಗುಹ್ಯಾ ಭಗಾಪಹಾ ॥೧೦೮॥

ಭಗೋದರೀ ಭಗಾನಂದಾ ಭಗಾದ್ಯಾ ಭಗಮಾಲಿನೀ~।\\
ಭೋಗಪ್ರದಾ ಭೋಗವಾಸಾ ಭೋಗಮೂಲಾ ಚ ಭೋಗಿನೀ ॥೧೦೯॥

ಭೇರುಂಡಾ ಭೇದಿನೀ ಭೀಮಾ ಭದ್ರಕಾಲೀ ಭಿದೋಜ್ಝಿತಾ~।\\
ಭೈರವೀ ಭುವನೇಶಾನೀ ಭುವನಾ ಭುವನೇಶ್ವರೀ ॥೧೧೦॥

ಭೀಮಾಕ್ಷೀ ಭಾರತೀ ಚೈವ ಭೈರವಾಷ್ಟಕಸೇವಿತಾ~।\\
ಭಾಸ್ವರಾ ಭಾಸ್ವತೀ ಭೀತಿರ್ಭಾಸ್ವದುತ್ಥಾನಶಾಲಿನೀ ॥೧೧೧॥

ಭಾಗೀರಥೀ ಭೋಗವತೀ ಭವಘ್ನೀ ಭುವನಾತ್ಮಿಕಾ~।\\
ಭೂತಿದಾ ಭೂತಿರೂಪಾ ಚ ಭೂತಸ್ಥಾ ಭೂತವರ್ಧಿನೀ ॥೧೧೨॥

ಮಾಹೇಶ್ವರೀ ಮಹಾಮಾಯಾ ಮಹಾತೇಜಾ ಮಹಾಸುರೀ~।\\
ಮಹಾಜಿಹ್ವಾ ಮಹಾಲೋಲಾ ಮಹಾದಂಷ್ಟ್ರಾ ಮಹಾಭುಜಾ ॥೧೧೩॥

ಮಹಾಮೋಹಾಂಧಕಾರಘ್ನೀ ಮಹಾಮೋಕ್ಷಪ್ರದಾಯಿನೀ~।\\
ಮಹಾದಾರಿದ್ರ್ಯಶಮನೀ ಮಹಾಶತ್ರುವಿಮರ್ದಿನೀ ॥೧೧೪॥

ಮಹಾಶಕ್ತಿರ್ಮಹಾಜ್ಯೋತಿರ್ಮಹಾಸುರವಿಮರ್ದಿನೀ~।\\
ಮಹಾಕಾಯಾ ಮಹಾವೀರ್ಯಾ ಮಹಾಪಾತಕನಾಶಿನೀ ॥೧೧೫॥

ಮಹಾರವಾ ಮಂತ್ರಮಯೀ ಮಣಿಪೂರನಿವಾಸಿನೀ~।\\
ಮಾನಸೀ ಮಾನದಾ ಮಾನ್ಯಾ ಮನಶ್ಚಕ್ಷುರಗೋಚರಾ ॥೧೧೬॥

ಮಾಹೇಂದ್ರೀ ಮಧುರಾ ಮಾಯಾ ಮಹಿಷಾಸುರಮರ್ದಿನೀ~।\\
ಮಹಾಕುಂಡಲಿನೀ ಶಕ್ತಿರ್ಮಹಾವಿಭವವರ್ಧಿನೀ ॥೧೧೭॥

ಮಾನಸೀ ಮಾಧವೀ ಮೇಧಾ ಮತಿದಾ ಮತಿಧಾರಿಣೀ~।\\
ಮೇನಕಾಗರ್ಭಸಂಭೂತಾ ಮೇನಕಾಭಗಿನೀ ಮತಿಃ ॥೧೧೮॥

ಮಹೋದರೀ ಮುಕ್ತಕೇಶೀ ಮುಕ್ತಿಕಾಮ್ಯಾರ್ಥಸಿದ್ಧಿದಾ~।\\
ಮಾಹೇಶೀ ಮಹಿಷಾರೂಢಾ ಮಧುದೈತ್ಯವಿಮರ್ದಿನೀ ॥೧೧೯॥

ಮಹಾವ್ರತಾ ಮಹಾಮೂರ್ಧಾ ಮಹಾಭಯವಿನಾಶಿನೀ~।\\
ಮಾತಂಗೀ ಮತ್ತಮಾತಂಗೀ ಮಾತಂಗಕುಲಮಂಡಿತಾ ॥೧೨೦॥

ಮಹಾಘೋರಾ ಮಾನನೀಯಾ ಮತ್ತಮಾತಂಗಗಾಮಿನೀ~।\\
ಮುಕ್ತಾಹಾರಲತೋಪೇತಾ ಮದಧೂರ್ಣಿತಲೋಚನಾ ॥೧೨೧॥

ಮಹಾಪರಾಧರಾಶಿಘ್ನೀ ಮಹಾಚೋರಭಯಾಪಹಾ~।\\
ಮಹಾಚಿಂತ್ಯಸ್ವರೂಪಾ ಚ ಮಣಿಮಂತ್ರಮಹೌಷಧೀ ॥೧೨೨॥

ಮಣಿಮಂಡಪಮಧ್ಯಸ್ಥಾ ಮಣಿಮಾಲಾವಿರಾಜಿತಾ~।\\
ಮಂತ್ರಾತ್ಮಿಕಾ ಮಂತ್ರಗಮ್ಯಾ ಮಂತ್ರಮಾತಾ ಸುಮಂತ್ರಿಣೀ ॥೧೨೩॥

ಮೇರುಮಂದರಮಧ್ಯಸ್ಥಾ ಮಕರಾಕೃತಿಕುಂಡಲಾ~।\\
ಮಂಥರಾ ಚ ಮಹಾಸೂಕ್ಷ್ಮಾ ಮಹಾದೂತೀ ಮಹೇಶ್ವರೀ ॥೧೨೪॥

ಮಾಲಿನೀ ಮಾನವೀ ಮಾಧ್ವೀ ಮದರೂಪಾ ಮದೋತ್ಕಟಾ~।\\
ಮದಿರಾ ಮಧುರಾ ಚೈವ ಮೋದಿನೀ ಚ ಮದೋದ್ಧತಾ ॥೧೨೫॥

ಮಂಗಲಾಂಗೀ ಮಧುಮಯೀ ಮಧುಪಾನಪರಾಯಣಾ~।\\
ಮನೋರಮಾ ರಮಾಮಾತಾ ರಾಜರಾಜೇಶ್ವರೀ ರಮಾ ॥೧೨೬॥

ರಾಜಮಾನ್ಯಾ ರಾಜಪೂಜ್ಯಾ ರಕ್ತೋತ್ಪಲವಿಭೂಷಣಾ~।\\
ರಾಜೀವಲೋಚನಾ ರಾಮಾ ರಾಧಿಕಾ ರಾಮವಲ್ಲಭಾ ॥೧೨೭॥

ಶಾಕಿನೀ ಡಾಕಿನೀ ಚೈವ ಲಾವಣ್ಯಾಂಬುಧಿವೀಚಿಕಾ~।\\
ರುದ್ರಾಣೀ ರುದ್ರರೂಪಾ ಚ ರೌದ್ರಾ ರುದ್ರಾರ್ತಿನಾಶಿನೀ ॥೧೨೮॥

ರಕ್ತಪ್ರಿಯಾ ರಕ್ತವಸ್ತ್ರಾ ರಕ್ತಾಕ್ಷೀ ರಕ್ತಲೋಚನಾ~।\\
ರಕ್ತಕೇಶೀ ರಕ್ತದಂಷ್ಟ್ರಾ ರಕ್ತಚಂದನಚರ್ಚಿತಾ ॥೧೨೯॥

ರಕ್ತಾಂಗೀ ರಕ್ತಭೂಷಾ ಚ ರಕ್ತಬೀಜನಿಪಾತಿನೀ~।\\
ರಾಗಾದಿದೋಷರಹಿತಾ ರತಿಜಾ ರತಿದಾಯಿನೀ ॥೧೩೦॥

ವಿಶ್ವೇಶ್ವರೀ ವಿಶಾಲಾಕ್ಷೀ ವಿಂಧ್ಯಪೀಠನಿವಾಸಿನೀ~।\\
ವಿಶ್ವಭೂರ್ವೀರವಿದ್ಯಾ ಚ ವೀರಸೂರ್ವೀರನಂದಿನೀ ॥೧೩೧॥

ವೀರೇಶ್ವರೀ ವಿಶಾಲಾಕ್ಷೀ ವಿಷ್ಣುಮಾಯಾ ವಿಮೋಹಿನೀ~।\\
ವಿದ್ಯಾವತೀ ವಿಷ್ಣುರೂಪಾ ವಿಶಾಲನಯನೋಜ್ಜ್ವಲಾ ॥೧೩೨॥

ವಿಷ್ಣುಮಾತಾ ಚ ವಿಶ್ವಾತ್ಮಾ ವಿಷ್ಣುಜಾಯಾಸ್ವರೂಪಿಣೀ~।\\
ವಾರಾಹೀ ವರದಾ ವಂದ್ಯಾ ವಿಖ್ಯಾತಾ ವಿಲಸಲ್ಕಚಾ ॥೧೩೩॥

ಬ್ರಹ್ಮೇಶೀ ಬ್ರಹ್ಮದಾ ಬ್ರಾಹ್ಮೀ ಬ್ರಹ್ಮಾಣೀ ಬ್ರಹ್ಮರೂಪಿಣೀ~।\\
ದ್ವಾರಕಾ ವಿಶ್ವವಂದ್ಯಾ ಚ ವಿಶ್ವಪಾಶವಿಮೋಚನೀ~।\\
ವಿಶ್ವಾಸಕಾರಿಣೀ ವಿಶ್ವಾ ವಿಶ್ವಶಕ್ತಿರ್ವಿಚಕ್ಷಣಾ ॥೧೩೪॥

ಬಾಣಚಾಪಧರಾ ವೀರಾ ಬಿಂದುಸ್ಥಾ ಬಿಂದುಮಾಲಿನೀ~।\\
ಷಟ್ಚಕ್ರಭೇದಿನೀ ಷೋಢಾ ಷೋಡಶಾರನಿವಾಸಿನೀ ॥೧೩೫॥

ಶಿತಿಕಂಠಪ್ರಿಯಾ ಶಾಂತಾ ಶಾಕಿನೀ ವಾತರೂಪಿಣೀ~।\\
ಶಾಶ್ವತೀ ಶಂಭುವನಿತಾ ಶಾಂಭವೀ ಶಿವರೂಪಿಣೀ ॥೧೩೬॥

ಶಿವಮಾತಾ ಚ ಶಿವದಾ ಶಿವಾ ಶಿವಹೃದಾಸನಾ~।\\
ಶುಕ್ಲಾಂಬರಾ ಶೀತಲಾ ಚ ಶೀಲಾ ಶೀಲಪ್ರದಾಯಿನೀ ॥೧೩೭॥

ಶಿಶುಪ್ರಿಯಾ ವೈದ್ಯವಿದ್ಯಾ ಸಾಲಗ್ರಾಮಶಿಲಾ ಶುಚಿಃ~।\\
ಹರಿಪ್ರಿಯಾ ಹರಮೂರ್ತಿರ್ಹರಿನೇತ್ರಕೃತಾಲಯಾ ॥೧೩೮॥

ಹರಿವಕ್ತ್ರೋದ್ಭವಾ ಹಾಲಾ ಹರಿವಕ್ಷಃಸ್ಥಲಸ್ಥಿತಾ~।\\
ಕ್ಷೇಮಂಕರೀ ಕ್ಷಿತಿಃ ಕ್ಷೇತ್ರಾ ಕ್ಷುಧಿತಸ್ಯ ಪ್ರಪೂರಣೀ ॥೧೩೯॥

ವೈಶ್ಯಾ ಚ ಕ್ಷತ್ರಿಯಾ ಶೂದ್ರೀ ಕ್ಷತ್ರಿಯಾಣಾಂ ಕುಲೇಶ್ವರೀ~।\\
ಹರಪತ್ನೀ ಹರಾರಾಧ್ಯಾ ಹರಸೂರ್ಹರರೂಪಿಣೀ ॥೧೪೦॥

ಸರ್ವಾನಂದಮಯೀ ಸಿದ್ಧಿಸ್ಸರ್ವರಕ್ಷಾಸ್ವರೂಪಿಣೀ~।\\
ಸರ್ವದುಷ್ಟಪ್ರಶಮನೀ ಸರ್ವೇಪ್ಸಿತಫಲಪ್ರದಾ ॥೧೪೧॥

ಸರ್ವಸಿದ್ಧೇಶ್ವರಾರಾಧ್ಯಾ ಸರ್ವಮಂಗಲಮಂಗಲಾ ॥

ಫಲಶ್ರುತಿಃ~।\\
ಪುಣ್ಯಂ ಸಹಸ್ರನಾಮೇದಂ ತವ ಪ್ರೀತ್ಯಾ ಪ್ರಕಾಶಿತಂ ॥೧೪೨॥

ಗೋಪನೀಯಂ ಪ್ರಯತ್ನೇನ ಪಠನೀಯಂ ಪ್ರಯತ್ನತಃ~।\\
ನಾತಃ ಪರತರಂ ಪುಣ್ಯಂ ನಾತಃ ಪರತರಂ ತಪಃ ॥೧೪೩॥

ನಾತಃ ಪರತರಂ ಸ್ತೋತ್ರಂ ನಾತಃ ಪರತರಾ ಗತಿಃ~।\\
ಸ್ತೋತ್ರಂ ನಾಮಸಹಸ್ರಾಖ್ಯಂ ಮಮ ವಕ್ತ್ರಾದ್ವಿನಿರ್ಗತಂ ॥೧೪೪॥

ಯಃ ಪಠೇತ್ಪರಯಾ ಭಕ್ತ್ಯಾ ಶೃಣುಯಾದ್ವಾ ಸಮಾಹಿತಃ~।\\
ಮೋಕ್ಷಾರ್ಥೀ ಲಭತೇ ಮೋಕ್ಷಂ ಸ್ವರ್ಗಾರ್ಥೀ ಸ್ವರ್ಗಮಾಪ್ನುಯಾತ್ ॥೧೪೫॥

ಕಾಮಾರ್ಥೀ ಲಭತೇ ಕಾಮಂ ಧನಾರ್ಥೀ ಲಭತೇ ಧನಂ~।\\
ವಿದ್ಯಾರ್ಥೀ ಲಭತೇ ವಿದ್ಯಾಂ ಯಶೋಽರ್ಥೀ ಲಭತೇ ಯಶಃ ॥೧೪೬॥

ಕನ್ಯಾರ್ಥೀ ಲಭತೇ ಕನ್ಯಾಂ ಸುತಾರ್ಥೀ ಲಭತೇ ಸುತಾನ್~।\\
ಮೂರ್ಖೋಽಪಿ ಲಭತೇ ಶಾಸ್ತ್ರಂ ಚೋರೋಽಪಿ ಲಭತೇ ಗತಿಂ ॥೧೪೭॥

ಗುರ್ವಿಣೀ ಜನಯೇತ್ಪುತ್ರಂ ಕನ್ಯಾ ವಿಂದತಿ ಸತ್ಪತಿಂ~।\\
ಸಂಕ್ರಾಂತ್ಯಾಂ ಚ ಚತುರ್ದಶ್ಯಾಮಷ್ಟಮ್ಯಾಂ ಚ ವಿಶೇಷತಃ ॥೧೪೮॥

ಪೌರ್ಣಮಾಸ್ಯಾಮಮಾವಸ್ಯಾಂ ನವಮ್ಯಾಂ ಭೌಮವಾಸರೇ~।\\
ಪಠೇದ್ವಾ ಪಾಠಯೇದ್ವಾಪಿ ಪೂಜಯೇದ್ವಾಪಿ ಪುಸ್ತಕಂ ॥೧೪೯॥

ಸ ಮುಕ್ತಸ್ಸರ್ವಪಾಪೇಭ್ಯಃ ಕಾಮೇಶ್ವರಸಮೋ ಭವೇತ್~।\\
ಲಕ್ಷ್ಮೀವಾನ್ ಸುತವಾಂಶ್ಚೈವ ವಲ್ಲಭಸ್ಸರ್ವಯೋಷಿತಾಂ ॥೧೫೦॥

ತಸ್ಯ ವಶ್ಯಂ ಭವೇದಾಶು ತ್ರೈಲೋಕ್ಯಂ ಸಚರಾಚರಂ~।\\
ವಿದ್ಯಾನಾಂ ಪಾರಗೋ ವಿಪ್ರಃ ಕ್ಷತ್ರಿಯೋ ವಿಜಯೀ ರಣೇ ॥೧೫೧॥

ವೈಶ್ಯೋ ಧನಸಮೃದ್ಧಸ್ಸ್ಯಾಚ್ಛೂದ್ರಸ್ಸುಖಮವಾಪ್ನುಯಾತ್~।\\
ಕ್ಷೇತ್ರೇ ಚ ಬಹುಸಸ್ಯಂ ಸ್ಯಾದ್ಗಾವಶ್ಚ ಬಹುದುಗ್ಧದಾಃ ॥೧೫೨॥

ನಾಶುಭಂ ನಾಪದಸ್ತಸ್ಯ ನ ಭಯಂ ನೃಪಶತ್ರುತಃ~।\\
ಜಾಯತೇ ನಾಶುಭಾ ಬುದ್ಧಿರ್ಲಭತೇ ಕುಲಪೂಜ್ಯತಾಂ ॥೧೫೩॥

ನ ಬಾಧಂತೇ ಗ್ರಹಾಸ್ತಸ್ಯ ನ ರಕ್ಷಾಂಸಿ ನ ಪನ್ನಗಾಃ~।\\
ನ ಪಿಶಾಚಾ ನ ಡಾಕಿನ್ಯೋ ಭೂತಭೇತಾಲಡಂಭಕಾಃ ॥೧೫೪॥

ಬಾಲಗ್ರಹಾಭಿಭೂತಾನಾಂ ಬಾಲಾನಾಂ ಶಾಂತಿಕಾರಕಂ~।\\
ದ್ವಂದ್ವಾನಾಂ ಪ್ರತಿಭೇದೇ ಚ ಮೈತ್ರೀಕರಣಮುತ್ತಮಂ ॥೧೫೫॥

ಲೋಹಪಾಶೈದೃಢೈರ್ಬದ್ಧೋ ಬಂಧೀ ವೇಶ್ಮನಿ ದುರ್ಗಮೇ~।\\
ತಿಷ್ಠಂಛೃಣ್ವನ್ಪಠನ್ಮರ್ತ್ಯೋ ಮುಚ್ಯತೇ ನಾತ್ರ ಸಂಶಯಃ ॥೧೫೬॥

ಪಶ್ಯಂತಿ ನಹಿ ತೇ ಶೋಕಂ ವಿಯೋಗಂ ಚಿರಜೀವಿನಃ~।\\
ಶೃಣ್ವತೀ ಬದ್ಧಗರ್ಭಾ ಚ ಸುಖಂ ಚೈವ ಪ್ರಸೂಯತೇ ॥೧೫೭॥

ಏಕದಾ ಪಠನಾದೇವ ಸರ್ವಪಾಪಕ್ಷಯೋ ಭವೇತ್~।\\
ನಶ್ಯಂತಿ ಚ ಮಹಾರೋಗಾ ದಶಧಾವರ್ತನೇನ ಚ ॥೧೫೮॥

ಶತಧಾವರ್ತನೇ ಚೈವ ವಾಚಾಂ ಸಿದ್ಧಿಃ ಪ್ರಜಾಯತೇ~।\\
ನವರಾತ್ರೇ ಜಿತಾಹಾರೋ ದೃಢಬುದ್ಧಿರ್ಜಿತೇಂದ್ರಿಯಃ ॥೧೫೯॥

ಅಂಬಿಕಾಯತನೇ ವಿದ್ವಾನ್ ಶುಚಿಷ್ಮಾನ್ ಮೂರ್ತಿಸನ್ನಿಧೌ~।\\
ಏಕಾಕೀ ಚ ದಶಾವರ್ತಂ ಪಠಂಧೀರಶ್ಚ ನಿರ್ಭಯಃ ॥೧೬೦॥

ಸಾಕ್ಷಾತ್ತ್ವಗವತೀ ತಸ್ಮೈ ಪ್ರಯಚ್ಛೇದೀಪ್ಸಿತಂ ಫಲಂ~।\\
ಸಿದ್ಧಪೀಠೇ ಗಿರೌ ರಮ್ಯೇ ಸಿದ್ಧಕ್ಷೇತ್ರೇ ಸುರಾಲಯೇ ॥೧೬೧॥

ಪಠನಾತ್ಸಾಧಕಸ್ಯಾಶು ಸಿದ್ಧಿರ್ಭವತಿ ವಾಂಛಿತಾ~।\\
ದಶಾವರ್ತಂ ಪಠೇನ್ನಿತ್ಯಂ ಭೂಮಿಶಾಯೀ ನರಶ್ಶುಚಿಃ ॥೧೬೨॥

ಸ್ವಪ್ನೇ ಮೂರ್ತಿಮಯೀಂ ದೇವೀಂ ವರದಾಂ ಸೋಽಪಿ ಪಶ್ಯತಿ~।\\
ಆವರ್ತನಸಹಸ್ರೈರ್ಯೇ ಜಪಂತಿ ಪುರುಷೋತ್ತಮಾಃ ॥೧೬೩॥

ತೇ ಸಿದ್ಧಾ ಸಿದ್ಧಿದಾ ಲೋಕೇ ಶಾಪಾನುಗ್ರಹಣಕ್ಷಮಾಃ~।\\
ಪ್ರಯಚ್ಛಂತಶ್ಚ ಸರ್ವಸ್ವಂ ಸೇವಂತೇ ತಾನ್ಮಹೀಶ್ವರಾಃ ॥೧೬೪॥

ಭೂರ್ಜಪತ್ರೇಽಷ್ಟಗಂಧೇನ ಲಿಖಿತ್ವಾ ತು ಶುಭೇ ದಿನೇ~।\\
ಧಾರಯೇದ್ಯಂತ್ರಿತಂ ಶೀರ್ಷೇ ಪೂಜಯಿತ್ವಾ ಕುಮಾರಿಕಾಂ ॥೧೬೫॥

ಬ್ರಾಹ್ಮಣಾನ್ ವರನಾರೀಶ್ಚ ಧೂಪೈಃ ಕುಸುಮಚಂದನೈಃ~।\\
ಕ್ಷೀರಖಂಡಾದಿಭೋಜ್ಯಾಂಶ್ಚ ಭೋಜಯಿತ್ವಾ ಸುಭಕ್ತಿತಃ ॥೧೬೬॥

ಬಧ್ನಂತಿ ಯೇ ಮಹಾರಕ್ಷಾಂ ಬಾಲಾನಾಂ ಚ ವಿಶೇಷತಃ~।\\
ರುದ್ರಂ ದೃಷ್ಟ್ವಾ ಯಥಾ ದೇವಂ ವಿಷ್ಣುಂ ದೃಷ್ಟ್ವಾ ಚ ದಾನವಾಃ ॥೧೬೭॥

ಪನ್ನಗಾ ಗರುಡಂ ದೃಷ್ಟ್ವಾ ಸಿಂಹಂ ದೃಷ್ಟ್ವಾ ಯಥಾ ಗಜಾಃ~।\\
ಮಂಡೂಕಾ ಭೋಗಿನಂ ದೃಷ್ಟ್ವಾ ಮಾರ್ಜಾರಂ ಮೂಷಿಕಾಸ್ತಥಾ ॥೧೬೮॥

ವಿಘ್ನಭೂತಾಃ ಪಲಾಯಂತೇ ತಸ್ಯ ವಕ್ತ್ರವಿಲೋಕನಾತ್~।\\
ಅಗ್ನಿಚೋರಭಯಂ ತಸ್ಯ ಕದಾಚಿನ್ನೈವ ಸಂಭವೇತ್ ॥೧೬೯॥

ಪಾತಕಾನ್ವಿವಿಧಾನ್ಸೋಽಪಿ ಮೇರುಮಂದರಸನ್ನಿಭಾನ್~।\\
ಭಸ್ಮಿತಾನ್ಕುರುತೇ ಕ್ಷಿಪ್ರಂ ತೃಣಂ ವಹ್ನಿಹುತಂ ಯಥಾ ॥೧೭೦॥

ನೃಪಾಶ್ಚ ವಶ್ಯತಾಂ ಯಾಂತಿ ನೃಪಪೂಜ್ಯಾಶ್ಚ ತೇ ನರಾಃ~।\\
ಮಹಾರ್ಣವೇ ಮಹಾನದ್ಯಾಂ ಪೋತಸ್ಥೇ ಚ ನ ಭೀಃ ಕ್ವಚಿತ್ ॥೧೭೧॥

ರಣೇ ದ್ಯೂತೇ ವಿವಾದೇ ಚ ವಿಜಯಂ ಪ್ರಾಪ್ನುವಂತಿ ತೇ~।\\
ಸರ್ವತ್ರ ಪೂಜಿತೋ ಲೋಕೈರ್ಬಹುಮಾನಪುರಸ್ಸರೈಃ ॥೧೭೨॥

ರತಿರಾಗವಿವೃದ್ಧಾಶ್ಚ ವಿಹ್ವಲಾಃ ಕಾಮಪೀಡಿತಾಃ~।\\
ಯೌವನಾಕ್ರಾಂತದೇಹಾಸ್ತಾನ್ ಶ್ರಯಂತೇ ವಾಮಲೋಚನಾಃ ॥೧೭೩॥

ಸಹಸ್ರಂ ಜಪತೇ ಯಸ್ತು ಖೇಚರೀ ಜಾಯತೇ ನರಃ~।\\
ಸಹಸ್ರದಶಕಂ ದೇವಿ ಯಃ ಪಠೇದ್ಭಕ್ತಿಮಾನ್ನರಃ ॥೧೭೪॥

ಸಾ ತಸ್ಯ ಜಗತಾಂ ಧಾತ್ರೀ ಪ್ರತ್ಯಕ್ಷಾ ಭವತಿ ಧ್ರುವಂ~।\\
ಲಕ್ಷಂ ಪೂರ್ಣಂ ಯದಾ ದೇವಿ ಸ್ತೋತ್ರರಾಜಂ ಜಪೇತ್ಸುಧೀಃ ॥೧೭೫॥

ಭವಪಾಶವಿನಿರ್ಮುಕ್ತೋ ಮಮ ತುಲ್ಯೋ ನ ಸಂಶಯಃ~।\\
ಸರ್ವತೀರ್ಥೇಷು ಯತ್ಪುಣ್ಯಂ ಸರ್ವತೀರ್ಥೇಷು ಯತ್ಫಲಂ ॥೧೭೬॥

ಸರ್ವಧರ್ಮೇಷು ಯಜ್ಞೇಷು ಸರ್ವದಾನೇಷು ಯತ್ಫಲಂ~।\\
ಸರ್ವವೇದೇಷು ಪ್ರೋಕ್ತೇಷು ಯತ್ಫಲಂ ಪರಿಕೀರ್ತಿತಂ ॥೧೭೭॥

ತತ್ಪುಣ್ಯಂ ಕೋಟಿಗುಣಿತಂ ಸಕೃಜ್ಜಪ್ತ್ವಾ ಲಭೇನ್ನರಃ~।\\
ದೇಹಾಂತೇ ಪರಮಂ ಸ್ಥಾನಂ ಯತ್ಸುರೈರಪಿ ದುರ್ಲಭಂ~।\\
ಸ ಯಾಸ್ಯತಿ ನ ಸಂದೇಹಸ್ಸ್ತವರಾಜಸ್ಯ ಕೀರ್ತನಾತ್ ॥೧೭೮॥
\authorline{॥ಇತಿ ರುದ್ರಯಾಮಲೇ ಶ್ರೀಅನ್ನಪೂರ್ಣಾಸಹಸ್ರನಾಮಸ್ತೋತ್ರಂ ಸಂಪೂರ್ಣಂ ॥}
%================================================================================================
\section{ಶ್ರೀಅನ್ನಪೂರ್ಣಾಷ್ಟೋತ್ತರಶತನಾಮಸ್ತೋತ್ರಂ }
\addcontentsline{toc}{section}{ಶ್ರೀಅನ್ನಪೂರ್ಣಾಷ್ಟೋತ್ತರಶತನಾಮಸ್ತೋತ್ರಂ }
ಅಸ್ಯ ಶ್ರೀಅನ್ನಪೂರ್ಣಾಷ್ಟೋತ್ತರ ಶತನಾಮಸ್ತೋತ್ರಮಂತ್ರಸ್ಯ ಭಗವಾನ್ ಶ್ರೀಬ್ರಹ್ಮಾ ಋಷಿಃ~। ಅನುಷ್ಟುಪ್ಛಂದಃ~। ಶ್ರೀಅನ್ನಪೂರ್ಣೇಶ್ವರೀ ದೇವತಾ~। ಸ್ವಧಾ ಬೀಜಂ~। ಸ್ವಾಹಾ ಶಕ್ತಿಃ~। ಓಂ ಕೀಲಕಂ~।\\

ಅನ್ನಪೂರ್ಣಾ ಶಿವಾ ದೇವೀ ಭೀಮಾ ಪುಷ್ಟಿಸ್ಸರಸ್ವತೀ~।\\
ಸರ್ವಜ್ಞಾ ಪಾರ್ವತೀ ದುರ್ಗಾ ಶರ್ವಾಣೀ ಶಿವವಲ್ಲಭಾ ॥೧॥

ವೇದವೇದ್ಯಾ ಮಹಾವಿದ್ಯಾ ವಿದ್ಯಾದಾತ್ರೀ ವಿಶಾರದಾ~।\\
ಕುಮಾರೀ ತ್ರಿಪುರಾ ಬಾಲಾ ಲಕ್ಷ್ಮೀಶ್ಶ್ರೀರ್ಭಯಹಾರಿಣೀ ॥೨॥

ಭವಾನೀ ವಿಷ್ಣುಜನನೀ ಬ್ರಹ್ಮಾದಿಜನನೀ ತಥಾ~।\\
ಗಣೇಶಜನನೀ ಶಕ್ತಿಃ ಕುಮಾರಜನನೀ ಶುಭಾ ॥೩॥

ಭೋಗಪ್ರದಾ ಭಗವತೀ ಭಕ್ತಾಭೀಷ್ಟಪ್ರದಾಯಿನೀ~।\\
ಭವರೋಗಹರಾ ಭವ್ಯಾ ಶುಭ್ರಾ ಪರಮಮಂಗಲಾ ॥೪॥

ಭವಾನೀ ಚಂಚಲಾ ಗೌರೀ ಚಾರುಚಂದ್ರಕಲಾಧರಾ~।\\
ವಿಶಾಲಾಕ್ಷೀ ವಿಶ್ವಮಾತಾ ವಿಶ್ವವಂದ್ಯಾ ವಿಲಾಸಿನೀ ॥೫॥

ಆರ್ಯಾ ಕಲ್ಯಾಣನಿಲಯಾ ರುದ್ರಾಣೀ ಕಮಲಾಸನಾ~।\\
ಶುಭಪ್ರದಾ ಶುಭಾವರ್ತಾ ವೃತ್ತಪೀನಪಯೋಧರಾ ॥೬॥

ಅಂಬಾ ಸಂಸಾರಮಥಿನೀ ಮೃಡಾನೀ ಸರ್ವಮಂಗಲಾ~।\\
ವಿಷ್ಣುಸಂಸೇವಿತಾ ಸಿದ್ಧಾ ಬ್ರಹ್ಮಾಣೀ ಸುರಸೇವಿತಾ ॥೭॥

ಪರಮಾನಂದದಾ ಶಾಂತಿಃ ಪರಮಾನಂದರೂಪಿಣೀ~।\\
ಪರಮಾನಂದಜನನೀ ಪರಾನಂದಪ್ರದಾಯಿನೀ ॥೮॥

ಪರೋಪಕಾರನಿರತಾ ಪರಮಾ ಭಕ್ತವತ್ಸಲಾ~।\\
ಪೂರ್ಣಚಂದ್ರಾಭವದನಾ ಪೂರ್ಣಚಂದ್ರನಿಭಾಂಶುಕಾ ॥೯॥

ಶುಭಲಕ್ಷಣಸಂಪನ್ನಾ ಶುಭಾನಂದಗುಣಾರ್ಣವಾ~।\\
ಶುಭಸೌಭಾಗ್ಯನಿಲಯಾ ಶುಭದಾ ಚ ರತಿಪ್ರಿಯಾ ॥೧೦॥

ಚಂಡಿಕಾ ಚಂಡಮಥನೀ ಚಂಡದರ್ಪನಿವಾರಿಣೀ~।\\
ಮಾರ್ತಂಡನಯನಾ ಸಾಧ್ವೀ ಚಂದ್ರಾಗ್ನಿನಯನಾ ಸತೀ ॥೧೧॥

ಪುಂಡರೀಕಕರಾ ಪೂರ್ಣಾ ಪುಣ್ಯದಾ ಪುಣ್ಯರೂಪಿಣೀ~।\\
ಮಾಯಾತೀತಾ ಶ್ರೇಷ್ಠಮಾಯಾ ಶ್ರೇಷ್ಠಧರ್ಮಾತ್ಮವಂದಿತಾ ॥೧೨॥

ಅಸೃಷ್ಟಿಸ್ಸಂಗರಹಿತಾ ಸೃಷ್ಟಿಹೇತುಃ ಕಪರ್ದಿನೀ~।\\
ವೃಷಾರೂಢಾ ಶೂಲಹಸ್ತಾ ಸ್ಥಿತಿಸಂಹಾರಕಾರಿಣೀ ॥೧೩॥

ಮಂದಸ್ಮಿತಾ ಸ್ಕಂದಮಾತಾ ಶುದ್ಧಚಿತ್ತಾ ಮುನಿಸ್ತುತಾ~।\\
ಮಹಾಭಗವತೀ ದಕ್ಷಾ ದಕ್ಷಾಧ್ವರವಿನಾಶಿನೀ ॥೧೪॥

ಸರ್ವಾರ್ಥದಾತ್ರೀ ಸಾವಿತ್ರೀ ಸದಾಶಿವಕುಟುಂಬಿನೀ~।\\
ನಿತ್ಯಸುಂದರಸರ್ವಾಂಗೀ ಸಚ್ಚಿದಾನಂದಲಕ್ಷಣಾ ॥೧೫॥

ನಾಮ್ನಾಮಷ್ಟೋತ್ತರಶತಮಂಬಾಯಾಃ ಪುಣ್ಯಕಾರಣಂ~।\\
ಸರ್ವಸೌಭಾಗ್ಯಸಿದ್ಧ್ಯರ್ಥಂ ಜಪನೀಯಂ ಪ್ರಯತ್ನತಃ ॥೧೬॥

ಏತಾನಿ ದಿವ್ಯನಾಮಾನಿ ಶ್ರುತ್ವಾ ಧ್ಯಾತ್ವಾ ನಿರಂತರಂ~।\\
ಸ್ತುತ್ವಾ ದೇವೀಂಚ ಸತತಂ ಸರ್ವಾನ್ಕಾಮಾನವಾಪ್ನುಯಾತ್ ॥೧೭॥
\authorline{॥ಇತಿ ಶ್ರೀಬ್ರಹ್ಮೋತ್ತರಖಂಡೇ ಆಗಮಪ್ರಖ್ಯಾತಿಶಿವರಹಸ್ಯೇ\\ ಶ್ರೀಅನ್ನಪೂರ್ಣಾಷ್ಟೋತ್ತರಶತನಾಮಸ್ತೋತ್ರಂ ಸಂಪೂರ್ಣಂ ॥}
%=============================================================================================



\section{ಶ್ರೀಭುವನೇಶ್ವರೀಸಹಸ್ರನಾಮಸ್ತೋತ್ರಂ}
\addcontentsline{toc}{section}{ಶ್ರೀಭುವನೇಶ್ವರೀಸಹಸ್ರನಾಮಸ್ತೋತ್ರಂ}

 ಶ್ರೀಪಾರ್ವತ್ಯುವಾಚ॥
ದೇವ ದೇವ ಮಹಾದೇವ ಸರ್ವಶಾಸ್ತ್ರವಿಶಾರದ ! ।\\
ಕಪಾಲಖಟ್ವಾಂಗಧರ ! ಚಿತಾಭಸ್ಮಾನುಲೇಪನ ! ॥೧॥

ಯಾ ಆದ್ಯಾ ಪ್ರಕೃತಿರ್ನಿತ್ಯಾ ಸರ್ವಶಾಸ್ತ್ರೇಷು ಗೋಪಿತಾ ।\\
ತಸ್ಯಾಃ ಶ್ರೀಭುವನೇಶ್ವರ್ಯಾ ನಾಮ್ನಾಂ ಪುಣ್ಯಂ ಸಹಸ್ರಕಂ ॥೨॥

ಕಥಯಸ್ವ ಮಹಾದೇವ ! ಯಥಾ ದೇವೀ ಪ್ರಸೀದತಿ ।\\

ಶ್ರೀಮಹೇಶ್ವರ ಉವಾಚ॥
ಸಾಧು ಪೃಷ್ಟಂ ಮಹಾದೇವಿ ! ಸಾಧಕಾನಾಂ ಹಿತಾಯ ವೈ ॥೩॥

ಯಾ ಆದ್ಯಾ ಪ್ರಕೃತಿರ್ನಿತ್ಯಾ ಸರ್ವಶಾಸ್ತ್ರೇಷು ಗೋಪಿತಾ ।\\
ಯಸ್ಯಾಃ ಸ್ಮರಣಮಾತ್ರೇಣ ಸರ್ವಪಾಪೈಃ ಪ್ರಮುಚ್ಯತೇ ॥೪॥

ಆರಾಧನಾದ್ಭವೇದ್ಯಸ್ಯಾ ಜೀವನ್ಮುಕ್ತೋ ನ ಸಂಶಯಃ ।\\
ತಸ್ಯಾ ನಾಮಸಹಸ್ರಂ ವೈ ಕಥಯಾಮಿ ಸಮಾಸತಃ ॥೫॥

ಅಸ್ಯ ಶ್ರೀಭುವನೇಶ್ವರ್ಯಾ ಸಹಸ್ರನಾಮಸ್ತೋತ್ರಸ್ಯ ದಕ್ಷಿಣಾಮೂರ್ತಿಋಷಿಃ । ಪಂಕ್ತಿಶ್ಛಂದಃ । ಆದ್ಯಾ ಶ್ರೀಭುವನೇಶ್ವರೀದೇವತಾ । ಹ್ರೀಂ ಬೀಜಂ । ಶ್ರೀಂ ಶಕ್ತಿಃ । ಕ್ಲೀಂ ಕೀಲಕಂ । ಮಮ ಶ್ರೀಧರ್ಮಾರ್ಥಕಾಮಮೋಕ್ಷಾರ್ಥೇ ಜಪೇ ವಿನಿಯೋಗಃ  ।\\

ಅಥ ಸಹಸ್ರನಾಮಸ್ತೋತ್ರಂ ।\\
ಆದ್ಯಾ ಮಾಯಾ ಪರಾ ಶಕ್ತಿಃ ಶ್ರೀಂ ಹ್ರೀಂ ಕ್ಲೀಂ ಭುವನೇಶ್ವರೀ ।\\
ಭುವನಾ ಭಾವನಾ ಭವ್ಯಾ ಭವಾನೀ ಭವಭಾವಿನೀ ॥೬॥

ರುದ್ರಾಣೀ ರುದ್ರಭಕ್ತಾ ಚ ತಥಾ ರುದ್ರಪ್ರಿಯಾ ಸತೀ ।\\
ಉಮಾ ಕಾತ್ಯಾಯನೀ ದುರ್ಗಾ ಮಂಗಲಾ ಸರ್ವಮಂಗಲಾ ॥೭॥

ತ್ರಿಪುರಾ ಪರಮೇಶಾನೀ ತ್ರಿಪುರಾ ಸುಂದರೀ ಪ್ರಿಯಾ ।\\
ರಮಣಾ ರಮಣೀ ರಾಮಾ ರಾಮಕಾರ್ಯಕರೀ ಶುಭಾ ॥೮॥

ಬ್ರಾಹ್ಮೀ ನಾರಾಯಣೀ ಚಂಡೀ ಚಾಮುಂಡಾ ಮುಂಡನಾಯಿಕಾ ।\\
ಮಾಹೇಶ್ವರೀ ಚ ಕೌಮಾರೀ ವಾರಾಹೀ ಚಾಪರಾಜಿತಾ ॥೯॥

ಮಹಾಮಾಯಾ ಮುಕ್ತಕೇಶೀ ಮಹಾತ್ರಿಪುರಸುಂದರೀ ।\\
ಸುಂದರೀ ಶೋಭನಾ ರಕ್ತಾ ರಕ್ತವಸ್ತ್ರಾಪಿಧಾಯಿನೀ ॥೧೦॥

ರಕ್ತಾಕ್ಷೀ ರಕ್ತವಸ್ತ್ರಾ ಚ ರಕ್ತಬೀಜಾತಿಸುಂದರೀ ।\\
ರಕ್ತಚಂದನಸಿಕ್ತಾಂಗೀ ರಕ್ತಪುಷ್ಪಸದಾಪ್ರಿಯಾ ॥೧೧॥

ಕಮಲಾ ಕಾಮಿನೀ ಕಾಂತಾ ಕಾಮದೇವಸದಾಪ್ರಿಯಾ ।\\
ಲಕ್ಷ್ಮೀ ಲೋಲಾ ಚಂಚಲಾಕ್ಷೀ ಚಂಚಲಾ ಚಪಲಾ ಪ್ರಿಯಾ ॥೧೨॥

ಭೈರವೀ ಭಯಹರ್ತ್ರೀ ಚ ಮಹಾಭಯವಿನಾಶಿನೀ ।\\
ಭಯಂಕರೀ ಮಹಾಭೀಮಾ ಭಯಹಾ ಭಯನಾಶಿನೀ ॥೧೩॥

ಶ್ಮಶಾನೇ ಪ್ರಾಂತರೇ ದುರ್ಗೇ ಸಂಸ್ಮೃತಾ ಭಯನಾಶಿನೀ ।\\
ಜಯಾ ಚ ವಿಜಯಾ ಚೈವ ಜಯಪೂರ್ಣಾ ಜಯಪ್ರದಾ ॥೧೪॥

ಯಮುನಾ ಯಾಮುನಾ ಯಾಮ್ಯಾ ಯಾಮುನಜಾ ಯಮಪ್ರಿಯಾ ।\\
ಸರ್ವೇಷಾಂ ಜನಿಕಾ ಜನ್ಯಾ ಜನಹಾ ಜನವರ್ಧಿನೀ ॥೧೫॥

ಕಾಲೀ ಕಪಾಲಿನೀ ಕುಲ್ಲಾ ಕಾಲಿಕಾ ಕಾಲರಾತ್ರಿಕಾ ।\\
ಮಹಾಕಾಲಹೃದಿಸ್ಥಾ ಚ ಕಾಲಭೈರವರೂಪಿಣೀ ॥೧೬॥

ಕಪಾಲಖಟ್ವಾಂಗಧರಾ ಪಾಶಾಂಕುಶವಿಧಾರಿಣೀ ।\\
ಅಭಯಾ ಚ ಭಯಾ ಚೈವ ತಥಾ ಚ ಭಯನಾಶಿನೀ ॥೧೭॥

ಮಹಾಭಯಪ್ರದಾತ್ರೀ ಚ ತಥಾ ಚ ವರಹಸ್ತಿನೀ ।\\
ಗೌರೀ ಗೌರಾಂಗಿನೀ ಗೌರಾ ಗೌರವರ್ಣಾ ಜಯಪ್ರದಾ ॥೧೮॥

ಉಗ್ರಾ ಉಗ್ರಪ್ರಭಾ ಶಾಂತಿಃ ಶಾಂತಿದಾಽಶಾಂತಿನಾಶಿನೀ ।\\
ಉಗ್ರತಾರಾ ತಥಾ ಚೋಗ್ರಾ ನೀಲಾ ಚೈಕಜಟಾ ತಥಾ ॥೧೯॥

ಹಾಂ ಹಾಂ ಹೂಂ ಹೂಂ ತಥಾ ತಾರಾ ತಥಾ ಚ ಸಿದ್ಧಿಕಾಲಿಕಾ ।\\
ತಾರಾ ನೀಲಾ ಚ ವಾಗೀಶೀ ತಥಾ ನೀಲಸರಸ್ವತೀ ॥೨೦॥

ಗಂಗಾ ಕಾಶೀ ಸತೀ ಸತ್ಯಾ ಸರ್ವತೀರ್ಥಮಯೀ ತಥಾ ।\\
ತೀರ್ಥರೂಪಾ ತೀರ್ಥಪುಣ್ಯಾ ತೀರ್ಥದಾ ತೀರ್ಥಸೇವಿಕಾ ॥೨೧॥

ಪುಣ್ಯದಾ ಪುಣ್ಯರೂಪಾ ಚ ಪುಣ್ಯಕೀರ್ತಿಪ್ರಕಾಶಿನೀ ।\\
ಪುಣ್ಯಕಾಲಾ ಪುಣ್ಯಸಂಸ್ಥಾ ತಥಾ ಪುಣ್ಯಜನಪ್ರಿಯಾ ॥೨೨॥

ತುಲಸೀ ತೋತುಲಾಸ್ತೋತ್ರಾ ರಾಧಿಕಾ ರಾಧನಪ್ರಿಯಾ ।\\
ಸತ್ಯಾಸತ್ಯಾ ಸತ್ಯಭಾಮಾ ರುಕ್ಮಿಣೀ ಕೃಷ್ಣವಲ್ಲಭಾ ॥೨೩॥

ದೇವಕೀ ಕೃಷ್ಣಮಾತಾ ಚ ಸುಭದ್ರಾ ಭದ್ರರೂಪಿಣೀ ।\\
ಮನೋಹರಾ ತಥಾ ಸೌಮ್ಯಾ ಶ್ಯಾಮಾಂಗೀ ಸಮದರ್ಶನಾ ॥೨೪॥

ಘೋರರೂಪಾ ಘೋರತೇಜಾ ಘೋರವತ್ಪ್ರಿಯದರ್ಶನಾ ।\\
ಕುಮಾರೀ ಬಾಲಿಕಾ ಕ್ಷುದ್ರಾ ಕುಮಾರೀರೂಪಧಾರಿಣೀ ॥೨೫॥

ಯುವತೀ ಯುವತೀರೂಪಾ ಯುವತೀರಸರಂಜಕಾ ।\\
ಪೀನಸ್ತನೀ ಕ್ಷೂದ್ರಮಧ್ಯಾ ಪ್ರೌಢಾ ಮಧ್ಯಾ ಜರಾತುರಾ ॥೨೬॥

ಅತಿವೃದ್ಧಾ ಸ್ಥಾಣುರೂಪಾ ಚಲಾಂಗೀ ಚಂಚಲಾ ಚಲಾ ।\\
ದೇವಮಾತಾ ದೇವರೂಪಾ ದೇವಕಾರ್ಯಕರೀ ಶುಭಾ ॥೨೭॥

ದೇವಮಾತಾ ದಿತಿರ್ದಕ್ಷಾ ಸರ್ವಮಾತಾ ಸನಾತನೀ ।\\
ಪಾನಪ್ರಿಯಾ ಪಾಯನೀ ಚ ಪಾಲನಾ ಪಾಲನಪ್ರಿಯಾ ॥೨೮॥

ಮತ್ಸ್ಯಾಶೀ ಮಾಂಸಭಕ್ಷ್ಯಾ ಚ ಸುಧಾಶೀ ಜನವಲ್ಲಭಾ ।\\
ತಪಸ್ವಿನೀ ತಪೀ ತಪ್ಯಾ ತಪಃಸಿದ್ಧಿಪ್ರದಾಯಿನೀ ॥೨೯॥

ಹವಿಷ್ಯಾ ಚ ಹವಿರ್ಭೋಕ್ತ್ರೀ ಹವ್ಯಕವ್ಯನಿವಾಸಿನೀ ।\\
ಯಜುರ್ವೇದಾ ವಶ್ಯಕರೀ ಯಜ್ಞಾಂಗೀ ಯಜ್ಞವಲ್ಲಭಾ ॥೩೦॥

ದಕ್ಷಾ ದಾಕ್ಷಾಯಿಣೀ ದುರ್ಗಾ ದಕ್ಷಯಜ್ಞವಿನಾಶಿನೀ ।\\
ಪಾರ್ವತೀ ಪರ್ವತಪ್ರೀತಾ ತಥಾ ಪರ್ವತವಾಸಿನೀ ॥೩೧॥

ಹೈಮೀ ಹರ್ಮ್ಯಾ ಹೇಮರೂಪಾ ಮೇನಾ ಮಾನ್ಯಾ ಮನೋರಮಾ ।\\
ಕೈಲಾಸವಾಸಿನೀ ಮುಕ್ತಾ ಶರ್ವಕ್ರೀಡಾವಿಲಾಸಿನೀ ॥೩೨॥

ಚಾರ್ವಂಗೀ ಚಾರುರೂಪಾ ಚ ಸುವಕ್ತ್ರಾ ಚ ಶುಭಾನನಾ ।\\
ಚಲತ್ಕುಂಡಲಗಂಡಶ್ರೀರ್ಲಸತ್ಕುಂಡಲಧಾರಿಣೀ ॥೩೩॥

ಮಹಾಸಿಂಹಾಸನಸ್ಥಾ ಚ ಹೇಮಭೂಷಣಭೂಷಿತಾ ।\\
ಹೇಮಾಂಗದಾ ಹೇಮಭೂಷಾ ಚ ಸೂರ್ಯಕೋಟಿಸಮಪ್ರಭಾ ॥೩೪॥

ಬಾಲಾದಿತ್ಯಸಮಾಕಾಂತಿಃ ಸಿಂದೂರಾರ್ಚಿತವಿಗ್ರಹಾ ।\\
ಯವಾ ಯಾವಕರೂಪಾ ಚ ರಕ್ತಚಂದನರೂಪಧೃಕ್ ॥೩೫॥

ಕೋಟರೀ ಕೋಟರಾಕ್ಷೀ ಚ ನಿರ್ಲಜ್ಜಾ ಚ ದಿಗಂಬರಾ ।\\
ಪೂತನಾ ಬಾಲಮಾತಾ ಚ ಶೂನ್ಯಾಲಯನಿವಾಸಿನೀ ॥೩೬॥

ಶ್ಮಶಾನವಾಸಿನೀ ಶೂನ್ಯಾ ಹೃದ್ಯಾ ಚತುರವಾಸಿನೀ ।\\
ಮಧುಕೈಟಭಹಂತ್ರೀ ಚ ಮಹಿಷಾಸುರಘಾತಿನೀ ॥೩೭॥

ನಿಶುಂಭಶುಂಭಮಥನೀ ಚಂಡಮುಂಡವಿನಾಶಿನೀ ।\\
ಶಿವಾಖ್ಯಾ ಶಿವರೂಪಾ ಚ ಶಿವದೂತೀ ಶಿವಪ್ರಿಯಾ ॥೩೮॥

ಶಿವದಾ ಶಿವವಕ್ಷಃಸ್ಥಾ ಶರ್ವಾಣೀ ಶಿವಕಾರಿಣೀ ।\\
ಇಂದ್ರಾಣೀ ಚೇಂದ್ರಕನ್ಯಾ ಚ ರಾಜಕನ್ಯಾ ಸುರಪ್ರಿಯಾ ॥೩೯॥

ಲಜ್ಜಾಶೀಲಾ ಸಾಧುಶೀಲಾ ಕುಲಸ್ತ್ರೀ ಕುಲಭೂಷಿಕಾ ।\\
ಮಹಾಕುಲೀನಾ ನಿಷ್ಕಾಮಾ ನಿರ್ಲಜ್ಜಾ ಕುಲಭೂಷಣಾ ॥೪೦॥

ಕುಲೀನಾ ಕುಲಕನ್ಯಾ ಚ ತಥಾ ಚ ಕುಲಭೂಷಿತಾ ।\\
ಅನಂತಾನಂತರೂಪಾ ಚ ಅನಂತಾಸುರನಾಶಿನೀ ॥೪೧॥

ಹಸಂತೀ ಶಿವಸಂಗೇನ ವಾಂಛಿತಾನಂದದಾಯಿನೀ ।\\
ನಾಗಾಂಗೀ ನಾಗಭೂಷಾ ಚ ನಾಗಹಾರವಿಧಾರಿಣೀ ॥೪೨॥

ಧರಿಣೀ ಧಾರಿಣೀ ಧನ್ಯಾ ಮಹಾಸಿದ್ಧಿಪ್ರದಾಯಿನೀ ।\\
ಡಾಕಿನೀ ಶಾಕಿನೀ ಚೈವ ರಾಕಿನೀ ಹಾಕಿನೀ ತಥಾ ॥೪೩॥

ಭೂತಾ ಪ್ರೇತಾ ಪಿಶಾಚೀ ಚ ಯಕ್ಷಿಣೀ ಧನದಾರ್ಚಿತಾ ।\\
ಧೃತಿಃ ಕೀರ್ತಿಃ ಸ್ಮೃತಿರ್ಮೇಧಾ ತುಷ್ಟಿಃಪುಷ್ಟಿರುಮಾ ರುಷಾ ॥೪೪॥

ಶಾಂಕರೀ ಶಾಂಭವೀ ಮೀನಾ ರತಿಃ ಪ್ರೀತಿಃ ಸ್ಮರಾತುರಾ ।\\
ಅನಂಗಮದನಾ ದೇವೀ ಅನಂಗಮದನಾತುರಾ ॥೪೫॥

ಭುವನೇಶೀ ಮಹಾಮಾಯಾ ತಥಾ ಭುವನಪಾಲಿನೀ ।\\
ಈಶ್ವರೀ ಚೇಶ್ವರೀಪ್ರೀತಾ ಚಂದ್ರಶೇಖರಭೂಷಣಾ ॥೪೬॥

ಚಿತ್ತಾನಂದಕರೀ ದೇವೀ ಚಿತ್ತಸಂಸ್ಥಾ ಜನಸ್ಯ ಚ ।\\
ಅರೂಪಾ ಬಹುರೂಪಾ ಚ ಸರ್ವರೂಪಾ ಚಿದಾತ್ಮಿಕಾ ॥೪೭॥

ಅನಂತರೂಪಿಣೀ ನಿತ್ಯಾ ತಥಾನಂತಪ್ರದಾಯಿನೀ ।\\
ನಂದಾ ಚಾನಂದರೂಪಾ ಚ ತಥಾಽನಂದಪ್ರಕಾಶಿನೀ ॥೪೮॥

ಸದಾನಂದಾ ಸದಾನಿತ್ಯಾ ಸಾಧಕಾನಂದದಾಯಿನೀ ।\\
ವನಿತಾ ತರುಣೀ ಭವ್ಯಾ ಭವಿಕಾ ಚ ವಿಭಾವಿನೀ ॥೪೯॥

ಚಂದ್ರಸೂರ್ಯಸಮಾ ದೀಪ್ತಾ ಸೂರ್ಯವತ್ಪರಿಪಾಲಿನೀ ।\\
ನಾರಸಿಂಹೀ ಹಯಗ್ರೀವಾ ಹಿರಣ್ಯಾಕ್ಷವಿನಾಶಿನೀ ॥೫೦॥

ವೈಷ್ಣವೀ ವಿಷ್ಣುಭಕ್ತಾ ಚ ಶಾಲಗ್ರಾಮನಿವಾಸಿನೀ ।\\
ಚತುರ್ಭುಜಾ ಚಾಷ್ಟಭುಜಾ ಸಹಸ್ರಭುಜಸಂಜ್ಞಿತಾ ॥೫೧॥

ಆದ್ಯಾ ಕಾತ್ಯಾಯನೀ ನಿತ್ಯಾ ಸರ್ವಾದ್ಯಾ ಸರ್ವದಾಯಿನೀ ।\\
ಸರ್ವಚಂದ್ರಮಯೀ ದೇವೀ ಸರ್ವವೇದಮಯೀ ಶುಭಾ ॥೫೨॥

ಸವದೇವಮಯೀ ದೇವೀ ಸರ್ವಲೋಕಮಯೀ ಪುರಾ ।\\
ಸರ್ವಸಮ್ಮೋಹಿನೀ ದೇವೀ ಸರ್ವಲೋಕವಶಂಕರೀ ॥೫೩॥

ರಾಜಿನೀ ರಂಜಿನೀ ರಾಗಾ ದೇಹಲಾವಣ್ಯರಂಜಿತಾ ।\\
ನಟೀ ನಟಪ್ರಿಯಾ ಧೂರ್ತಾ ತಥಾ ಧೂರ್ತಜನಾರ್ದಿನೀ ॥೫೪॥

ಮಹಾಮಾಯಾ ಮಹಾಮೋಹಾ ಮಹಾಸತ್ತ್ವವಿಮೋಹಿತಾ ।\\
ಬಲಿಪ್ರಿಯಾ ಮಾಂಸರುಚಿರ್ಮಧುಮಾಂಸಪ್ರಿಯಾ ಸದಾ ॥೫೫॥

ಮಧುಮತ್ತಾ ಮಾಧವಿಕಾ ಮಧುಮಾಧವರೂಪಿಕಾ ।\\
ದಿವಾಮಯೀ ರಾತ್ರಿಮಯೀ ಸಂಧ್ಯಾ ಸಂಧಿಸ್ವರೂಪಿಣೀ ॥೫೬॥

ಕಾಲರೂಪಾ ಸೂಕ್ಷ್ಮರೂಪಾ ಸೂಕ್ಷ್ಮಿಣೀ ಚಾತಿಸೂಕ್ಷ್ಮಿಣೀ ।\\
ತಿಥಿರೂಪಾ ವಾರರೂಪಾ ತಥಾ ನಕ್ಷತ್ರರೂಪಿಣೀ ॥೫೭॥

ಸರ್ವಭೂತಮಯೀ ದೇವೀ ಪಂಚಭೂತನಿವಾಸಿನೀ ।\\
ಶೂನ್ಯಾಕಾರಾ ಶೂನ್ಯರೂಪಾ ಶೂನ್ಯಸಂಸ್ಥಾ ಚ ಸ್ತಂಭಿನೀ ॥೫೮॥

ಆಕಾಶಗಾಮಿನೀ ದೇವೀ ಜ್ಯೋತಿಶ್ಚಕ್ರನಿವಾಸಿನೀ ।\\
ಗ್ರಹಾಣಾಂ ಸ್ಥಿತಿರೂಪಾ ಚ ರುದ್ರಾಣೀ ಚಕ್ರಸಂಭವಾ ॥೫೯॥

ಋಷೀಣಾಂ ಬ್ರಹ್ಮಪುತ್ರಾಣಾಂ ತಪಃಸಿದ್ಧಿಪ್ರದಾಯಿನೀ ।\\
ಅರುಂಧತೀ ಚ ಗಾಯತ್ರೀ ಸಾವಿತ್ರೀ ಸತ್ತ್ವರೂಪಿಣೀ ॥೬೦॥

ಚಿತಾಸಂಸ್ಥಾ ಚಿತಾರೂಪಾ ಚಿತ್ತಸಿದ್ಧಿಪ್ರದಾಯಿನೀ ।\\
ಶವಸ್ಥಾ ಶವರೂಪಾ ಚ ಶವಶತ್ರುನಿವಾಸಿನೀ ॥೬೧॥

ಯೋಗಿನೀ ಯೋಗರೂಪಾ ಚ ಯೋಗಿನಾಂ ಮಲಹಾರಿಣೀ ।\\
ಸುಪ್ರಸನ್ನಾ ಮಹಾದೇವೀ ಯಾಮುನೀ ಮುಕ್ತಿದಾಯಿನೀ ॥೬೨॥

ನಿರ್ಮಲಾ ವಿಮಲಾ ಶುದ್ಧಾ ಶುದ್ಧಸತ್ವಾ ಜಯಪ್ರದಾ ।\\
ಮಹಾವಿದ್ಯಾ ಮಹಾಮಾಯಾ ಮೋಹಿನೀ ವಿಶ್ವಮೋಹಿನೀ ॥೬೩॥

ಕಾರ್ಯಸಿದ್ಧಿಕರೀ ದೇವೀ ಸರ್ವಕಾರ್ಯನಿವಾಸಿನೀ ।\\
ಕಾರ್ಯಕಾರ್ಯಕರೀ ರೌದ್ರೀ ಮಹಾಪ್ರಲಯಕಾರಿಣೀ ॥೬೪॥

ಸ್ತ್ರೀಪುಂಭೇದಾಹ್ಯಭೇದ್ಯಾ ಚ ಭೇದಿನೀ ಭೇದನಾಶಿನೀ ।\\
ಸರ್ವರೂಪಾ ಸರ್ವಮಯೀ ಅದ್ವೈತಾನಂದರೂಪಿಣೀ ॥೬೫॥

ಪ್ರಚಂಡಾ ಚಂಡಿಕಾ ಚಂಡಾ ಚಂಡಾಸುರವಿನಾಶಿನೀ ।\\
ಸುಮಸ್ತಾ ಬಹುಮಸ್ತಾ ಚ ಛಿನ್ನಮಸ್ತಾಽಸುನಾಶಿನೀ ॥೬೬॥

ಅರೂಪಾ ಚ ವಿರೂಪಾ ಚ ಚಿತ್ರರೂಪಾ ಚಿದಾತ್ಮಿಕಾ ।\\
ಬಹುಶಸ್ತ್ರಾ ಅಶಸ್ತ್ರಾ ಚ ಸರ್ವಶಸ್ತ್ರಪ್ರಹಾರಿಣೀ ॥೬೭॥

ಶಾಸ್ತ್ರಾರ್ಥಾ ಶಾಸ್ತ್ರವಾದಾ ಚ ನಾನಾ ಶಾಸ್ತ್ರಾರ್ಥವಾದಿನೀ ।\\
ಕಾವ್ಯಶಾಸ್ತ್ರಪ್ರಮೋದಾ ಚ ಕಾವ್ಯಾಲಂಕಾರವಾಸಿನೀ ॥೬೮॥

ರಸಜ್ಞಾ ರಸನಾ ಜಿಹ್ವಾ ರಸಾಮೋದಾ ರಸಪ್ರಿಯಾ ।\\
ನಾನಾಕೌತುಕಸಂಯುಕ್ತಾ ನಾನಾರಸವಿಲಾಸಿನೀ ॥೬೯॥

ಅರೂಪಾ ಚ ಸ್ವರೂಪಾ ಚ ವಿರೂಪಾ ಚ ಸುರೂಪಿಣೀ ।\\
ರೂಪಾವಸ್ಯಾ ತಥಾ ಜೀವಾ ವೇಶ್ಯಾದ್ಯಾ ವೇಶಧಾರಿಣೀ ॥೭೦॥

ನಾನಾವೇಶಧರಾ ದೇವೀ ನಾನಾವೇಶೇಷು ಸಂಸ್ಥಿತಾ ।\\
ಕುರೂಪಾ ಕುಟಿಲಾ ಕೃಷ್ಣಾ ಕೃಷ್ಣಾರೂಪಾ ಚ ಕಾಲಿಕಾ ॥೭೧॥

ಲಕ್ಷ್ಮೀಪ್ರದಾ ಮಹಾಲಕ್ಷ್ಮೀಃ ಸರ್ವಲಕ್ಷಣಸಂಯುತಾ ।\\
ಕುಬೇರಗೃಹಸಂಸ್ಥಾ ಚ ಧನರೂಪಾ ಧನಪ್ರದಾ ॥೭೨॥

ನಾನಾರತ್ನಪ್ರದಾ ದೇವೀ ರತ್ನಖಂಡೇಷು ಸಂಸ್ಥಿತಾ ।\\
ವರ್ಣಸಂಸ್ಥಾ ವರ್ಣರೂಪಾ ಸರ್ವವರ್ಣಮಯೀ ಸದಾ ॥೭೩॥

ಓಂಕಾರರೂಪಿಣೀ ವಾಚ್ಯಾ ಆದಿತ್ಯಜ್ಯೋತೀರೂಪಿಣೀ ।\\
ಸಂಸಾರಮೋಚಿನೀ ದೇವೀ ಸಂಗ್ರಾಮೇ ಜಯದಾಯಿನೀ ॥೭೪॥

ಜಯರೂಪಾ ಜಯಾಖ್ಯಾ ಚ ಜಯಿನೀ ಜಯದಾಯಿನೀ ।\\
ಮಾನಿನೀ ಮಾನರೂಪಾ ಚ ಮಾನಭಂಗಪ್ರಣಾಶಿನೀ ॥೭೫॥

ಮಾನ್ಯಾ ಮಾನಪ್ರಿಯಾ ಮೇಧಾ ಮಾನಿನೀ ಮಾನದಾಯಿನೀ ।\\
ಸಾಧಕಾಸಾಧಕಾಸಾಧ್ಯಾ ಸಾಧಿಕಾ ಸಾಧನಪ್ರಿಯಾ ॥೭೬॥

ಸ್ಥಾವರಾ ಜಂಗಮಾ ಪ್ರೋಕ್ತಾ ಚಪಲಾ ಚಪಲಪ್ರಿಯಾ ।\\
ಋದ್ಧಿದಾ ಋದ್ಧಿರೂಪಾ ಚ ಸಿದ್ಧಿದಾ ಸಿದ್ಧಿದಾಯಿನೀ ॥೭೭॥

ಕ್ಷೇಮಂಕರೀ ಶಂಕರೀ ಚ ಸರ್ವಸಮ್ಮೋಹಕಾರಿಣೀ ।\\
ರಂಜಿತಾ ರಂಜಿನೀ ಯಾ ಚ ಸರ್ವವಾಂಛಾಪ್ರದಾಯಿನೀ ॥೭೮॥

ಭಗಲಿಂಗಪ್ರಮೋದಾ ಚ ಭಗಲಿಂಗನಿವಾಸಿನೀ ।\\
ಭಗರೂಪಾ ಭಗಾಭಾಗ್ಯಾ ಲಿಂಗರೂಪಾ ಚ ಲಿಂಗಿನೀ ॥೭೯॥

ಭಗಗೀತಿರ್ಮಹಾಪ್ರೀತಿರ್ಲಿಂಗಗೀತಿರ್ಮಹಾಸುಖಾ ।\\
ಸ್ವಯಂಭೂಃ ಕುಸುಮಾರಾಧ್ಯಾ ಸ್ವಯಂಭೂಃ ಕುಸುಮಾಕುಲಾ ॥೮೦॥

ಸ್ವಯಂಭೂಃ ಪುಷ್ಪರೂಪಾ ಚ ಸ್ವಯಂಭೂಃ ಕುಸುಮಪ್ರಿಯಾ ।\\
ಶುಕ್ರಕೂಪಾ ಮಹಾಕೂಪಾ ಶುಕ್ರಾಸವನಿವಾಸಿನೀ ॥೮೧॥

ಶುಕ್ರಸ್ಥಾ ಶುಕ್ರಿಣೀ ಶುಕ್ರಾ ಶುಕ್ರಪೂಜಕಪೂಜಿತಾ ।\\
ಕಾಮಾಕ್ಷಾ ಕಾಮರೂಪಾ ಚ ಯೋಗಿನೀ ಪೀಠವಾಸಿನೀ ॥೮೨॥

ಸರ್ವಪೀಠಮಯೀ ದೇವೀ ಪೀಠಪೂಜಾನಿವಾಸಿನೀ ।\\
ಅಕ್ಷಮಾಲಾಧರಾ ದೇವೀ ಪಾನಪಾತ್ರವಿಧಾರಿಣೀ ॥೮೩॥

ಶೂಲಿನೀ ಶೂಲಹಸ್ತಾ ಚ ಪಾಶಿನೀ ಪಾಶರೂಪಿಣೀ ।\\
ಖಡ್ಗಿನೀ ಗದಿನೀ ಚೈವ ತಥಾ ಸರ್ವಾಸ್ತ್ರಧಾರಿಣೀ ॥೮೪॥

ಭಾವ್ಯಾ ಭವ್ಯಾ ಭವಾನೀ ಸಾ ಭವಮುಕ್ತಿಪ್ರದಾಯಿನೀ ।\\
ಚತುರಾ ಚತುರಪ್ರೀತಾ ಚತುರಾನನಪೂಜಿತಾ ॥೮೫॥

ದೇವಸ್ತವ್ಯಾ ದೇವಪೂಜ್ಯಾ ಸರ್ವಪೂಜ್ಯಾ ಸುರೇಶ್ವರೀ ।\\
ಜನನೀ ಜನರೂಪಾ ಚ ಜನಾನಾಂ ಚಿತ್ತಹಾರಿಣೀ ॥೮೬॥

ಜಟಿಲಾ ಕೇಶಬದ್ಧಾ ಚ ಸುಕೇಶೀ ಕೇಶಬದ್ಧಿಕಾ ।\\
ಅಹಿಂಸಾ ದ್ವೇಷಿಕಾ ದ್ವೇಷ್ಯಾ ಸರ್ವದ್ವೇಷವಿನಾಶಿನೀ ॥೮೭॥

ಉಚ್ಚಾಟಿನೀ ದ್ವೇಷಿನೀ ಚ ಮೋಹಿನೀ ಮಧುರಾಕ್ಷರಾ ।\\
ಕ್ರೀಡಾ ಕ್ರೀಡಕಲೇಖಾಂಕಕಾರಣಾಕಾರಕಾರಿಕಾ ॥೮೮॥

ಸರ್ವಜ್ಞಾ ಸರ್ವಕಾರ್ಯಾ ಚ ಸರ್ವಭಕ್ಷಾ ಸುರಾರಿಹಾ ।\\
ಸರ್ವರೂಪಾ ಸರ್ವಶಾಂತಾ ಸರ್ವೇಷಾಂ ಪ್ರಾಣರೂಪಿಣೀ ॥೮೯॥

ಸೃಷ್ಟಿಸ್ಥಿತಿಕರೀ ದೇವೀ ತಥಾ ಪ್ರಲಯಕಾರಿಣೀ ।\\
ಮುಗ್ಧಾ ಸಾಧ್ವೀ ತಥಾ ರೌದ್ರೀ ನಾನಾಮೂರ್ತಿವಿಧಾರಿಣೀ ॥೯೦॥

ಉಕ್ತಾನಿ ಯಾನಿ ದೇವೇಶಿ ಅನುಕ್ತಾನಿ ಮಹೇಶ್ವರಿ ।\\
ಯತ್ ಕಿಂಚಿದ್ ದೃಶ್ಯತೇ ದೇವಿ ತತ್ ಸರ್ವಂ ಭುವನೇಶ್ವರೀ ॥೯೧॥

ಇತಿ ಶ್ರೀಭುವನೇಶ್ವರ್ಯಾ ನಾಮಾನಿ ಕಥಿತಾನಿ ತೇ ।\\
ಸಹಸ್ರಾಣಿ ಮಹಾದೇವಿ ಫಲಂ ತೇಷಾಂ ನಿಗದ್ಯತೇ ॥೯೨॥

ಯಃ ಪಠೇತ್ ಪ್ರಾತರುತ್ಥಾಯ ಚಾರ್ದ್ಧರಾತ್ರೇ ತಥಾ ಪ್ರಿಯೇ ।\\
ಪ್ರಾತಃಕಾಲೇ ತಥಾ ಮಧ್ಯೇ ಸಾಯಾಹ್ನೇ ಹರವಲ್ಲಭೇ ॥೯೩॥

ಯತ್ರ ತತ್ರ ಪಠಿತ್ವಾ ಚ ಭಕ್ತ್ಯಾ ಸಿದ್ಧಿರ್ನ ಸಂಶಯಃ ।\\
ಪಠೇದ್ ವಾ ಪಾಠಯೇದ್ ವಾಪಿ ಶೃಣುಯಾಚ್ಛ್ರಾವಯೇತ್ತಥಾ ॥೯೪॥

ತಸ್ಯ ಸರ್ವಂ ಭವೇತ್ ಸತ್ಯಂ ಮನಸಾ ಯಚ್ಚ ವಾಂಛಿತಂ ।\\
ಅಷ್ಟಮ್ಯಾಂ ಚ ಚತುರ್ದಶ್ಯಾಂ ನವಮ್ಯಾಂ ವಾ ವಿಶೇಷತಃ ॥೯೫॥

ಸರ್ವಮಂಗಲಸಂಯುಕ್ತೇ ಸಂಕ್ರಾತೌ ಶನಿಭೌಮಯೋಃ ।\\
ಯಃ ಪಠೇತ್ ಪರಯಾ ಭಕ್ತ್ಯಾ ದೇವ್ಯಾ ನಾಮಸಹಸ್ರಕಂ ॥೯೬॥

ತಸ್ಯ ದೇಹೇ ಚ ಸಂಸ್ಥಾನಂ ಕುರುತೇ ಭುವನೇಶ್ವರೀ ।\\
ತಸ್ಯ ಕಾರ್ಯಂ ಭವೇದ್ ದೇವಿ ಅನ್ಯಥಾ ನ ಕಥಂಚನ ॥೯೭॥

ಶ್ಮಶಾನೇ ಪ್ರಾಂತರೇ ವಾಪಿ ಶೂನ್ಯಾಗಾರೇ ಚತುಷ್ಪಥೇ ।\\
ಚತುಷ್ಪಥೇ ಚೈಕಲಿಂಗೇ ಮೇರುದೇಶೇ ತಥೈವ ಚ ॥೯೮॥

ಜಲಮಧ್ಯೇ ವಹ್ನಿಮಧ್ಯೇ ಸಂಗ್ರಾಮೇ ಗ್ರಾಮಶಾಂತಯೇ ।\\
ಜಪತ್ವಾ ಮಂತ್ರಸಹಸ್ರಂ ತು ಪಠೇನ್ನಾಮಸಹಸ್ರಕಂ ॥೯೯॥

ಧೂಪದೀಪಾದಿಭಿಶ್ಚೈವ ಬಲಿದಾನಾದಿಕೈಸ್ತಥಾ ।\\
ನಾನಾವಿಧೈಸ್ತಥಾ ದೇವಿ ನೈವೇದ್ಯೈರ್ಭುವನೇಶ್ವರೀಂ ॥೧೦೦॥

ಸಂಪೂಜ್ಯ ವಿಧಿವಜ್ಜಪ್ತ್ವಾ ಸ್ತುತ್ವಾ ನಾಮಸಹಸ್ರಕೈಃ ।\\
ಅಚಿರಾತ್ ಸಿದ್ಧಿಮಾಪ್ನೋತಿ ಸಾಧಕೋ ನಾತ್ರ ಸಂಶಯಃ ॥೧೦೧॥

ತಸ್ಯ ತುಷ್ಟಾ ಭವೇದ್ ದೇವೀ ಸರ್ವದಾ ಭುವನೇಶ್ವರೀ ।\\
ಭೂರ್ಜಪತ್ರೇ ಸಮಾಲಿಖ್ಯ ಕುಙಕುಮಾದ್ ರಕ್ತಚಂದನೈಃ ॥೧೦೨॥

ತಥಾ ಗೋರೋಚನಾದ್ಯೈಶ್ಚ ವಿಲಿಖ್ಯ ಸಾಧಕೋತ್ತಮಃ ।\\
ಸುತಿಥೌ ಶುಭನಕ್ಷತ್ರೇ ಲಿಖಿತ್ವಾ ದಕ್ಷಿಣೇ ಭುಜೇ ॥೧೦೩॥

ಧಾರಯೇತ್ ಪರಯಾ ಭಕ್ತ್ಯಾ ದೇವೀರೂಪೇಣ ಪಾರ್ವತಿ ! ।\\
ತಸ್ಯ ಸಿದ್ಧಿರ್ಮಹೇಶಾನಿ ಅಚಿರಾಚ್ಚ ಭವಿಷ್ಯತಿ ॥೧೦೪॥

ರಣೇ ರಾಜಕುಲೇ ವಾಽಪಿ ಸರ್ವತ್ರ ವಿಜಯೀ ಭವೇತ್ ।\\
ದೇವತಾ ವಶಮಾಯಾತಿ ಕಿಂ ಪುನರ್ಮಾನವಾದಯಃ ॥೧೦೫॥

ವಿದ್ಯಾಸ್ತಂಭಂ ಜಲಸ್ತಂಭಂ ಕರೋತ್ಯೇವ ನ ಸಂಶಯಃ ।\\
ಪಠೇದ್ ವಾ ಪಾಠಯೇದ್ ವಾಽಪಿ ದೇವೀಭಕ್ತ್ಯಾ ಚ ಪಾರ್ವತಿ ॥೧೦೬॥

ಇಹ ಭುಕ್ತ್ವಾ ವರಾನ್ ಭೋಗಾನ್ ಕೃತ್ವಾ ಕಾವ್ಯಾರ್ಥವಿಸ್ತರಾನ್ ।\\
ಅಂತೇ ದೇವ್ಯಾ ಗಣತ್ವಂ ಚ ಸಾಧಕೋ ಮುಕ್ತಿಮಾಪ್ನುಯಾತ್ ॥೧೦೭॥

ಪ್ರಾಪ್ನೋತಿ ದೇವದೇವೇಶಿ ಸರ್ವಾರ್ಥಾನ್ನಾತ್ರ ಸಂಶಯಃ ।\\
ಹೀನಾಂಗೇ ಚಾತಿರಿಕ್ತಾಂಗೇ ಶಠಾಯ ಪರಶಿಷ್ಯಕೇ ॥೧೦೮॥

ನ ದಾತವ್ಯಂ ಮಹೇಶಾನಿ ಪ್ರಾಣಾಂತೇಽಪಿ ಕದಾಚನ ।\\
ಶಿಷ್ಯಾಯ ಮತಿಶುದ್ಧಾಯ ವಿನೀತಾಯ ಮಹೇಶ್ವರಿ ॥೧೦೯॥

ದಾತವ್ಯಃ ಸ್ತವರಾಜಶ್ಚ ಸರ್ವಸಿದ್ಧಿಪ್ರದೋ ಭವೇತ್ ।\\
ಲಿಖಿತ್ವಾ ಧಾರಯೇದ್ ದೇಹೇ ದುಃಖಂ ತಸ್ಯ ನ ಜಾಯತೇ ॥೧೧೦॥

ಯ ಇದಂ ಭುವನೇಶ್ವರ್ಯಾಃ ಸ್ತವರಾಜಂ ಮಹೇಶ್ವರಿ ।\\
ಇತಿ ತೇ ಕಥಿತಂ ದೇವಿ ಭುವನೇಶ್ಯಾಃ ಸಹಸ್ರಕಂ ।\\
ಯಸ್ಮೈ ಕಸ್ಮೈ ನ ದಾತವ್ಯಂ ವಿನಾ ಶಿಷ್ಯಾಯ ಪಾರ್ವತಿ  ॥೧೧೧॥

ಸುರತರುವರಕಾಂತಂ ಸಿದ್ಧಿಸಾಧ್ಯೈಕಸೇವ್ಯಂ\\
ಯದಿ ಪಠತಿ ಮನುಷ್ಯೋ ನಾನ್ಯಚೇತಾಃ ಸದೈವ ।\\
ಇಹ ಹಿ ಸಕಲಭೋಗಾನ್ ಪ್ರಾಪ್ಯ ಚಾಂತೇ ಶಿವಾಯ\\
ವ್ರಜತಿ ಪರಸಮೀಪಂ ಸರ್ವದಾ ಮುಕ್ತಿಮಂತೇ ॥೧೧೨॥

\authorline{ಇತಿ ಶ್ರೀರುದ್ರಯಾಮಲೇ ತಂತ್ರೇ ಭುವನೇಶ್ವರೀಸಹಸ್ರನಾಮಾಖ್ಯಂ ಸ್ತೋತ್ರಂ ಸಂಪೂರ್ಣಂ॥}
%=============================================================================================
\section{ಶ್ರೀಭುವನೇಶ್ವರೀಶತನಾಮಸ್ತೋತ್ರಂ}
\addcontentsline{toc}{section}{ಶ್ರೀಭುವನೇಶ್ವರೀಶತನಾಮಸ್ತೋತ್ರಂ}

ಕೈಲಾಸಶಿಖರೇ ರಮ್ಯೇ ನಾನಾರತ್ನೋಪಶೋಭಿತೇ ।\\
ನರನಾರೀಹಿತಾರ್ಥಾಯ ಶಿವಂ ಪಪ್ರಚ್ಛ ಪಾರ್ವತೀ ॥೧॥

ದೇವ್ಯುವಾಚ ।\\
ಭುವನೇಶೀಮಹಾವಿದ್ಯಾನಾಮ್ನಾಮಷ್ಟೋತ್ತರಂ ಶತಂ ।\\
ಕಥಯಸ್ವ ಮಹಾದೇವ ಯದ್ಯಹಂ ತವ ವಲ್ಲಭಾ ॥೨॥

ಈಶ್ವರ ಉವಾಚ ।\\
ಶೃಣು ದೇವಿ ಮಹಾಭಾಗೇ ಸ್ತವರಾಜಮಿದಂ ಶುಭಂ ।\\
ಸಹಸ್ರನಾಮ್ನಾಮಧಿಕಂ ಸಿದ್ಧಿದಂ ಮೋಕ್ಷಹೇತುಕಂ ॥೩॥

ಶುಚಿಭಿಃ ಪ್ರಾತರುತ್ಥಾಯ ಪಠಿತವ್ಯಂ ಸಮಾಹಿತೈಃ ।\\
ತ್ರಿಕಾಲಂ ಶ್ರದ್ಧಯಾ ಯುಕ್ತೈಃ ಸರ್ವಕಾಮಫಲಪ್ರದಂ ॥೪॥

ಅಸ್ಯ ಶ್ರೀಭುವನೇಶ್ವರ್ಯಷ್ಟೋತ್ತರಶತನಾಮಸ್ತೋತ್ರಸ್ಯ ಶಕ್ತಿರೃಷಿಃ  । ಗಾಯತ್ರೀ ಛಂದಃ ।ಭುವನೇಶ್ವರೀ ದೇವತಾ । ಚತುರ್ವರ್ಗಸಾಧನೇ ಜಪೇ ವಿನಿಯೋಗಃ ।\\

ಓಂ ಮಹಾಮಾಯಾ ಮಹಾವಿದ್ಯಾ ಮಹಾಭೋಗಾ ಮಹೋತ್ಕಟಾ ।\\
ಮಾಹೇಶ್ವರೀ ಕುಮಾರೀ ಚ ಬ್ರಹ್ಮಾಣೀ ಬ್ರಹ್ಮರೂಪಿಣೀ ॥೫॥

ವಾಗೀಶ್ವರೀ ಯೋಗರೂಪಾ ಯೋಗಿನೀಕೋಟಿಸೇವಿತಾ ।\\
ಜಯಾ ಚ ವಿಜಯಾ ಚೈವ ಕೌಮಾರೀ ಸರ್ವಮಂಗಲಾ ॥೬॥

ಹಿಂಗುಲಾ ಚ ವಿಲಾಸೀ ಚ ಜ್ವಾಲಿನೀ ಜ್ವಾಲರೂಪಿಣೀ ।\\
ಈಶ್ವರೀ ಕ್ರೂರಸಂಹಾರೀ ಕುಲಮಾರ್ಗಪ್ರದಾಯಿನೀ ॥೭॥

ವೈಷ್ಣವೀ ಸುಭಗಾಕಾರಾ ಸುಕುಲ್ಯಾ ಕುಲಪೂಜಿತಾ ।\\
ವಾಮಾಂಗಾ ವಾಮಚಾರಾ ಚ ವಾಮದೇವಪ್ರಿಯಾ ತಥಾ ॥೮॥

ಡಾಕಿನೀ ಯೋಗಿನೀರೂಪಾ ಭೂತೇಶೀ ಭೂತನಾಯಿಕಾ ।\\
ಪದ್ಮಾವತೀ ಪದ್ಮನೇತ್ರಾ ಪ್ರಬುದ್ಧಾ ಚ ಸರಸ್ವತೀ ॥೯॥

ಭೂಚರೀ ಖೇಚರೀ ಮಾಯಾ ಮಾತಂಗೀ ಭುವನೇಶ್ವರೀ ।\\
ಕಾಂತಾ ಪತಿವ್ರತಾ ಸಾಕ್ಷೀ ಸುಚಕ್ಷುಃ ಕುಂಡವಾಸಿನೀ ॥೧೦॥

ಉಮಾ ಕುಮಾರೀ ಲೋಕೇಶೀ ಸುಕೇಶೀ ಪದ್ಮರಾಗಿಣೀ ।\\
ಇಂದ್ರಾಣೀ ಬ್ರಹ್ಮ ಚಾಂಡಾಲೀ ಚಂಡಿಕಾ ವಾಯುವಲ್ಲಭಾ ॥೧೧॥

ಸರ್ವಧಾತುಮಯೀಮೂರ್ತಿರ್ಜಲರೂಪಾ ಜಲೋದರೀ ।\\
ಆಕಾಶೀ ರಣಗಾ ಚೈವ ನೃಕಪಾಲವಿಭೂಷಣಾ ॥೧೨॥

ನರ್ಮದಾ ಮೋಕ್ಷದಾ ಚೈವ ಧರ್ಮಕಾಮಾರ್ಥದಾಯಿನೀ ।\\
ಗಾಯತ್ರೀ ಚಾಥ ಸಾವಿತ್ರೀ ತ್ರಿಸಂಧ್ಯಾ ತೀರ್ಥಗಾಮಿನೀ ॥೧೩॥

ಅಷ್ಟಮೀ ನವಮೀ ಚೈವ ದಶಮ್ಯೇಕಾದಶೀ ತಥಾ ।\\
ಪೌರ್ಣಮಾಸೀ ಕುಹೂರೂಪಾ ತಿಥಿಮೂರ್ತಿಸ್ವರೂಪಿಣೀ ॥೧೪॥

ಸುರಾರಿನಾಶಕಾರೀ ಚ ಉಗ್ರರೂಪಾ ಚ ವತ್ಸಲಾ ।\\
ಅನಲಾ ಅರ್ಧಮಾತ್ರಾ ಚ ಅರುಣಾ ಪೀತಲೋಚನಾ ॥೧೫॥

ಲಜ್ಜಾ ಸರಸ್ವತೀ ವಿದ್ಯಾ ಭವಾನೀ ಪಾಪನಾಶಿನೀ ।\\
ನಾಗಪಾಶಧರಾ ಮೂರ್ತಿರಗಾಧಾ ಧೃತಕುಂಡಲಾ ॥೧೬॥

ಕ್ಷತ್ರರೂಪಾ ಕ್ಷಯಕರೀ ತೇಜಸ್ವಿನೀ ಶುಚಿಸ್ಮಿತಾ ।\\
ಅವ್ಯಕ್ತಾ ವ್ಯಕ್ತಲೋಕಾ ಚ ಶಂಭುರೂಪಾ ಮನಸ್ವಿನೀ ॥೧೭॥

ಮಾತಂಗೀ ಮತ್ತಮಾತಂಗೀ ಮಹಾದೇವಪ್ರಿಯಾ ಸದಾ ।\\
ದೈತ್ಯಹಾ ಚೈವ ವಾರಾಹೀ ಸರ್ವಶಾಸ್ತ್ರಮಯೀ ಶುಭಾ ॥೧೮॥

ಯ ಇದಂ ಪಠತೇ ಭಕ್ತ್ಯಾ ಶೃಣುಯಾದ್ವಾ ಸಮಾಹಿತಃ ।\\
ಅಪುತ್ರೋ ಲಭತೇ ಪುತ್ರಂ ನಿರ್ಧನೋ ಧನವಾನ್ ಭವೇತ್ ॥೧೯॥

ಮೂರ್ಖೋಽಪಿ ಲಭತೇ ಶಾಸ್ತ್ರಂ ಚೋರೋಽಪಿ ಲಭತೇ ಗತಿಂ ।\\
ವೇದಾನಾಂ ಪಾಠಕೋ ವಿಪ್ರಃ ಕ್ಷತ್ರಿಯೋ ವಿಜಯೀ ಭವೇತ್ ॥೨೦॥

ವೈಶ್ಯಸ್ತು ಧನವಾನ್ಭೂಯಾಚ್ಛೂದ್ರಸ್ತು ಸುಖಮೇಧತೇ ।\\
ಅಷ್ಟಮ್ಯಾಂಚ ಚತುರ್ದಶ್ಯಾಂ ನವಮ್ಯಾಂ ಚೈಕಚೇತಸಃ ॥೨೧॥

ಯೇ ಪಠಂತಿ ಸದಾ ಭಕ್ತ್ಯಾ ನ ತೇ ವೈ ದುಃಖಭಾಗಿನಃ ।\\
ಏಕಕಾಲಂ ದ್ವಿಕಾಲಂ ವಾ ತ್ರಿಕಾಲಂ ವಾ ಚತುರ್ಥಕಂ ॥೨೨॥

ಯೇ ಪಠಂತಿ ಸದಾ ಭಕ್ತ್ಯಾ ಸ್ವರ್ಗಲೋಕೇ ಚ ಪೂಜಿತಾಃ ।\\
ರುದ್ರಂ ದೃಷ್ಟ್ವಾ ಯಥಾ ದೇವಾಃ ಪನ್ನಗಾ ಗರುಡಂ ಯಥಾ ॥

ಶತ್ರವಃ ಪ್ರಪಲಾಯಂತೇ ತಸ್ಯ ವಕ್ತ್ರವಿಲೋಕನಾತ್ ॥೨೩॥

\authorline{ಇತಿ ಶ್ರೀರುದ್ರಯಾಮಲೇ ದೇವೀಶಂಕರಸಂವಾದೇ ಭುವನೇಶ್ವರ್ಯಷ್ಟೋತ್ತರಶತನಾಮಸ್ತೋತ್ರಂ ॥}
%==========================================
\section{ಕಾಲೀಸಹಸ್ರನಾಮಸ್ತೋತ್ರಂ}
\addcontentsline{toc}{section}{ಕಾಲೀಸಹಸ್ರನಾಮಸ್ತೋತ್ರಂ}

ಶ್ರೀದೇವ್ಯುವಾಚ ।\\
ಪೂರ್ವಂ ಹಿ ಸೂಚಿತಂ ದೇವ ಕಾಲೀನಾಮಸಹಸ್ರಕಂ ।\\
ತದ್ವದಸ್ವ ಮಹಾದೇವ ಯದಿ ಸ್ನೇಹೋಽಸ್ತಿ ಮಾಂ ಪ್ರತಿ ॥೧॥

ಶ್ರೀಭೈರವ ಉವಾಚ ।\\
ತಂತ್ರೇಽಸ್ಮಿನ್ ಪರಮೇಶಾನಿ ಕಾಲೀನಾಮಸಹಸ್ರಕಂ ।\\
ಶೃಣುಷ್ವೈಕಮನಾ ದೇವಿ ಭಕ್ತಾನಾಂ ಪ್ರೀತಿವರ್ದ್ಧನಂ ॥೨॥

ಓಂ ಅಸ್ಯಾಃ ಶ್ರೀಕಾಲೀದೇವ್ಯಾಃ ಮಂತ್ರಸಹಸ್ರನಾಮಸ್ತೋತ್ರಸ್ಯ ಮಹಾಕಾಲಭೈರವ ಋಷಿಃ । ಅನುಷ್ಟುಪ್ ಛಂದಃ । ಶ್ರೀಕಾಲೀ ದೇವತಾ । ಕ್ರೀಂ ಬೀಜಂ । ಹೂಂ ಶಕ್ತಿಃ । ಹ್ರೀಂ ಕೀಲಕಂ । ಧರ್ಮಾರ್ಥಕಾಮಮೋಕ್ಷಾರ್ಥೇ ವಿನಿಯೋಗಃ॥

ಕಾಲಿಕಾ ಕಾಮದಾ ಕುಲ್ಲಾ ಭದ್ರಕಾಲೀ ಗಣೇಶ್ವರೀ ।\\
ಭೈರವೀ ಭೈರವಪ್ರೀತಾ ಭವಾನೀ ಭವಮೋಚಿನೀ ॥೩॥

ಕಾಲರಾತ್ರಿರ್ಮಹಾರಾತ್ರಿರ್ಮೋಹರಾತ್ರಿಶ್ಚ ಮೋಹಿನೀ ।\\
ಮಹಾಕಾಲರತಾ ಸೂಕ್ಷ್ಮಾ ಕೌಲವ್ರತಪರಾಯಣಾ ॥೪॥

ಕೋಮಲಾಂಗೀ ಕರಾಲಾಂಗೀ ಕಮನೀಯಾ ವರಾಂಗನಾ ।\\
ಗಂಧಚಂದನದಿಗ್ಧಾಂಗೀ ಸತೀ ಸಾಧ್ವೀ ಪತಿವ್ರತಾ ॥೫॥

ಕಾಕಿನೀ ವರ್ಣರೂಪಾ ಚ ಮಹಾಕಾಲಕುಟುಂಬಿನೀ ।\\
ಕಾಮಹಂತ್ರೀ ಕಾಮಕಲಾ ಕಾಮವಿಜ್ಞಾ ಮಹೋದಯಾ ॥೬॥

ಕಾಂತರೂಪಾ ಮಹಾಲಕ್ಷ್ಮೀರ್ಮಹಾಕಾಲಸ್ವರೂಪಿಣೀ ।\\
ಕುಲೀನಾ ಕುಲಸರ್ವಸ್ವಾ ಕುಲವರ್ತ್ಮಪ್ರದರ್ಶಿಕಾ ॥೭॥

ಕುಲರೂಪಾ ಚಕೋರಾಕ್ಷೀ ಶ್ರೀದುರ್ಗಾ ದುರ್ಗನಾಶಿನೀ ।\\
ಕನ್ಯಾ ಕುಮಾರೀ ಗೌರೀ ತು ಕೃಷ್ಣದೇಹಾ ಮಹಾಮನಾಃ ॥೮॥

ಕೃಷ್ಣಾಂಗೀ ನೀಲದೇಹಾ ಚ ಪಿಂಗಕೇಶೀ ಕೃಶೋದರೀ ।\\
ಪಿಂಗಾಕ್ಷೀ ಕಮಲಪ್ರೀತಾ ಕಾಲೀ ಕಾಲಪರಾಕ್ರಮಾ ॥೯॥

ಕಲಾನಾಥಪ್ರಿಯಾ ದೇವೀ ಕುಲಕಾಂತಾಽಪರಾಜಿತಾ ।\\
ಉಗ್ರತಾರಾ ಮಹೋಗ್ರಾ ಚ ತಥಾ ಚೈಕಜಟಾ ಶಿವಾ ॥೧೦॥

ನೀಲಾ ಘನಾ ಬಲಾಕಾ ಚ ಕಾಲದಾತ್ರೀ ಕಲಾತ್ಮಿಕಾ ।\\
ನಾರಾಯಣಪ್ರಿಯಾ ಸೂಕ್ಷ್ಮಾ ವರದಾ ಭಕ್ತವತ್ಸಲಾ ॥೧೧॥

ವರಾರೋಹಾ ಮಹಾಬಾಣಾ ಕಿಶೋರೀ ಯುವತೀ ಸತೀ ।\\
ದೀರ್ಘಾಂಗೀ ದೀರ್ಘಕೇಶಾ ಚ ನೃಮುಂಡಧಾರಿಣೀ ತಥಾ ॥೧೨॥

ಮಾಲಿನೀ ನರಮುಂಡಾಲೀ ಶವಮುಂಡಾಸ್ಥಿಧಾರಿಣೀ ।\\
ರಕ್ತನೇತ್ರಾ ವಿಶಾಲಾಕ್ಷೀ ಸಿಂದೂರಭೂಷಣಾ ಮಹೀ ॥೧೩॥

ಘೋರರಾತ್ರಿರ್ಮಹಾರಾತ್ರಿರ್ಘೋರಾಂತಕವಿನಾಶಿನೀ ।\\
ನಾರಸಿಂಹೀ ಮಹಾರೌದ್ರೀ ನೀಲರೂಪಾ ವೃಷಾಸನಾ ॥೧೪॥

ವಿಲೋಚನಾ ವಿರೂಪಾಕ್ಷೀ ರಕ್ತೋತ್ಪಲವಿಲೋಚನಾ ।\\
ಪೂರ್ಣೇಂದುವದನಾ ಭೀಮಾ ಪ್ರಸನ್ನವದನಾ ತಥಾ ॥೧೫॥

ಪದ್ಮನೇತ್ರಾ ವಿಶಾಲಾಕ್ಷೀ ಶರಜ್ಜ್ಯೋತ್ಸ್ನಾಸಮಾಕುಲಾ ।\\
ಪ್ರಫುಲ್ಲಪುಂಡರೀಕಾಭಲೋಚನಾ ಭಯನಾಶಿನೀ ॥೧೬॥

ಅಟ್ಟಹಾಸಾ ಮಹೋಚ್ಛ್ವಾಸಾ ಮಹಾವಿಘ್ನವಿನಾಶಿನೀ ।\\
ಕೋಟರಾಕ್ಷೀ ಕೃಶಗ್ರೀವಾ ಕುಲತೀರ್ಥಪ್ರಸಾಧಿನೀ ॥೧೭॥

ಕುಲಗರ್ತಪ್ರಸನ್ನಾಸ್ಯಾ ಮಹತೀ ಕುಲಭೂಷಿಕಾ ।\\
ಬಹುವಾಕ್ಯಾಮೃತರಸಾ ಚಂಡರೂಪಾತಿವೇಗಿನೀ ॥೧೮॥

ವೇಗದರ್ಪಾ ವಿಶಾಲೈಂದ್ರೀ ಪ್ರಚಂಡಚಂಡಿಕಾ ತಥಾ ।\\
ಚಂಡಿಕಾ ಕಾಲವದನಾ ಸುತೀಕ್ಷ್ಣನಾಸಿಕಾ ತಥಾ ॥೧೯॥

ದೀರ್ಘಕೇಶೀ ಸುಕೇಶೀ ಚ ಕಪಿಲಾಂಗೀ ಮಹಾರುಣಾ ।\\
ಪ್ರೇತಭೂಷಣಸಂಪ್ರೀತಾ ಪ್ರೇತದೋರ್ದಂಡಘಂಟಿಕಾ ॥೨೦॥

ಶಂಖಿನೀ ಶಂಖಮುದ್ರಾ ಚ ಶಂಖಧ್ವನಿನಿನಾದಿನೀ ।\\
ಶ್ಮಶಾನವಾಸಿನೀ ಪೂರ್ಣಾ ಪೂರ್ಣೇಂದುವದನಾ ಶಿವಾ ॥೨೧॥

ಶಿವಪ್ರೀತಾ ಶಿವರತಾ ಶಿವಾಸನಸಮಾಶ್ರಯಾ ।\\
ಪುಣ್ಯಾಲಯಾ ಮಹಾಪುಣ್ಯಾ ಪುಣ್ಯದಾ ಪುಣ್ಯವಲ್ಲಭಾ ॥೨೨॥

ನರಮುಂಡಧರಾ ಭೀಮಾ ಭೀಮಾಸುರವಿನಾಶಿನೀ ।\\
ದಕ್ಷಿಣಾ ದಕ್ಷಿಣಾಪ್ರೀತಾ ನಾಗಯಜ್ಞೋಪವೀತಿನೀ ॥೨೩॥

ದಿಗಂಬರೀ ಮಹಾಕಾಲೀ ಶಾಂತಾ ಪೀನೋನ್ನತಸ್ತನೀ ।\\
ಘೋರಾಸನಾ ಘೋರರೂಪಾ ಸೃಕ್ಪ್ರಾಂತೇ ರಕ್ತಧಾರಿಕಾ ॥೨೪॥

ಮಹಾಧ್ವನಿಃ ಶಿವಾಸಕ್ತಾ ಮಹಾಶಬ್ದಾ ಮಹೋದರೀ ।\\
ಕಾಮಾತುರಾ ಕಾಮಸಕ್ತಾ ಪ್ರಮತ್ತಾ ಶಕ್ತಭಾವನಾ ॥೨೫॥

ಸಮುದ್ರನಿಲಯಾ ದೇವೀ ಮಹಾಮತ್ತಜನಪ್ರಿಯಾ ।\\
ಕರ್ಷಿತಾ ಕರ್ಷಣಪ್ರೀತಾ ಸರ್ವಾಕರ್ಷಣಕಾರಿಣೀ ॥೨೬॥

ವಾದ್ಯಪ್ರೀತಾ ಮಹಾಗೀತರಕ್ತಾ ಪ್ರೇತನಿವಾಸಿನೀ ।\\
ನರಮುಂಡಸೃಜಾ ಗೀತಾ ಮಾಲಿನೀ ಮಾಲ್ಯಭೂಷಿತಾ ॥೨೭॥

ಚತುರ್ಭುಜಾ ಮಹಾರೌದ್ರೀ ದಶಹಸ್ತಾ ಪ್ರಿಯಾತುರಾ ।\\
ಜಗನ್ಮಾತಾ ಜಗದ್ಧಾತ್ರೀ ಜಗತೀ ಮುಕ್ತಿದಾ ಪರಾ ॥೨೮॥

ಜಗದ್ಧಾತ್ರೀ ಜಗತ್ತ್ರಾತ್ರೀ ಜಗದಾನಂದಕಾರಿಣೀ ।\\
ಜಗಜ್ಜೀವಮಯೀ ಹೈಮವತೀ ಮಾಯಾ ಮಹಾಕಚಾ ॥೨೯॥

ನಾಗಾಂಗೀ ಸಂಹೃತಾಂಗೀ ಚ ನಾಗಶಯ್ಯಾಸಮಾಗತಾ ।\\
ಕಾಲರಾತ್ರಿರ್ದಾರುಣಾ ಚ ಚಂದ್ರಸೂರ್ಯಪ್ರತಾಪಿನೀ ॥೩೦॥

ನಾಗೇಂದ್ರನಂದಿನೀ ದೇವಕನ್ಯಾ ಚ ಶ್ರೀಮನೋರಮಾ ।\\
ವಿದ್ಯಾಧರೀ ವೇದವಿದ್ಯಾ ಯಕ್ಷಿಣೀ ಶಿವಮೋಹಿನೀ ॥೩೧॥

ರಾಕ್ಷಸೀ ಡಾಕಿನೀ ದೇವಮಯೀ ಸರ್ವಜಗಜ್ಜಯಾ ।\\
ಶ್ರುತಿರೂಪಾ ತಥಾಗ್ನೇಯೀ ಮಹಾಮುಕ್ತಿರ್ಜನೇಶ್ವರೀ ॥೩೨॥

ಪತಿವ್ರತಾ ಪತಿರತಾ ಪತಿಭಕ್ತಿಪರಾಯಣಾ ।\\
ಸಿದ್ಧಿದಾ ಸಿದ್ಧಿಸಂದಾತ್ರೀ ತಥಾ ಸಿದ್ಧಜನಪ್ರಿಯಾ ॥೩೩॥

ಕರ್ತ್ರಿಹಸ್ತಾ ಶಿವಾರೂಢಾ ಶಿವರೂಪಾ ಶವಾಸನಾ ।\\
ತಮಿಸ್ರಾ ತಾಮಸೀ ವಿಜ್ಞಾ ಮಹಾಮೇಘಸ್ವರೂಪಿಣೀ ॥೩೪॥

ಚಾರುಚಿತ್ರಾ ಚಾರುವರ್ಣಾ ಚಾರುಕೇಶಸಮಾಕುಲಾ ।\\
ಚಾರ್ವಂಗೀ ಚಂಚಲಾ ಲೋಲಾ ಚೀನಾಚಾರಪರಾಯಣಾ ॥೩೫॥

ಚೀನಾಚಾರಪರಾ ಲಜ್ಜಾವತೀ ಜೀವಪ್ರದಾಽನಘಾ ।\\
ಸರಸ್ವತೀ ತಥಾ ಲಕ್ಷ್ಮೀರ್ಮಹಾನೀಲಸರಸ್ವತೀ ॥೩೬॥

ಗರಿಷ್ಠಾ ಧರ್ಮನಿರತಾ ಧರ್ಮಾಧರ್ಮವಿನಾಶಿನೀ ।\\
ವಿಶಿಷ್ಟಾ ಮಹತೀ ಮಾನ್ಯಾ ತಥಾ ಸೌಮ್ಯಜನಪ್ರಿಯಾ ॥೩೭॥

ಭಯದಾತ್ರೀ ಭಯರತಾ ಭಯಾನಕಜನಪ್ರಿಯಾ ।\\
ವಾಕ್ಯರೂಪಾ ಛಿನ್ನಮಸ್ತಾ ಛಿನ್ನಾಸುರಪ್ರಿಯಾ ಸದಾ ॥೩೮॥

ಋಗ್ವೇದರೂಪಾ ಸಾವಿತ್ರೀ ರಾಗಯುಕ್ತಾ ರಜಸ್ವಲಾ ।\\
ರಜಃಪ್ರೀತಾ ರಜೋರಕ್ತಾ ರಜಃಸಂಸರ್ಗವರ್ದ್ಧಿನೀ ॥೩೯॥

ರಜಃಪ್ಲುತಾ ರಜಃಸ್ಫೀತಾ ರಜಃಕುಂತಲಶೋಭಿತಾ ।\\
ಕುಂಡಲೀ ಕುಂಡಲಪ್ರೀತಾ ತಥಾ ಕುಂಡಲಶೋಭಿತಾ ॥೪೦॥

ರೇವತೀ ರೇವತಪ್ರೀತಾ ರೇವಾ ಚೈರಾವತೀ ಶುಭಾ ।\\
ಶಕ್ತಿನೀ ಚಕ್ರಿಣೀ ಪದ್ಮಾ ಮಹಾಪದ್ಮನಿವಾಸಿನೀ ॥೪೧॥

ಪದ್ಮಾಲಯಾ ಮಹಾಪದ್ಮಾ ಪದ್ಮಿನೀ ಪದ್ಮವಲ್ಲಭಾ ।\\
ಪದ್ಮಪ್ರಿಯಾ ಪದ್ಮರತಾ ಮಹಾಪದ್ಮಸುಶೋಭಿತಾ ॥೪೨॥

ಶೂಲಹಸ್ತಾ ಶೂಲರತಾ ಶೂಲಿನೀ ಶೂಲಸಂಗಿಕಾ ।\\
ಪಿನಾಕಧಾರಿಣೀ ವೀಣಾ ತಥಾ ವೀಣಾವತೀ ಮಘಾ ॥೪೩॥

ರೋಹಿಣೀ ಬಹುಲಪ್ರೀತಾ ತಥಾ ವಾಹನವರ್ದ್ಧಿತಾ ।\\
ರಣಪ್ರೀತಾ ರಣರತಾ ರಣಾಸುರವಿನಾಶಿನೀ ॥೪೪॥

ರಣಾಗ್ರವರ್ತಿನೀ ರಾಣಾ ರಣಾಗ್ರಾ ರಣಪಂಡಿತಾ ।\\
ಜಟಾಯುಕ್ತಾ ಜಟಾಪಿಂಗಾ ವಜ್ರಿಣೀ ಶೂಲಿನೀ ತಥಾ ॥೪೫॥

ರತಿಪ್ರಿಯಾ ರತಿರತಾ ರತಿಭಕ್ತಾ ರತಾತುರಾ ।\\
ರತಿಭೀತಾ ರತಿಗತಾ ಮಹಿಷಾಸುರನಾಶಿನೀ ॥೪೬॥

ರಕ್ತಪಾ ರಕ್ತಸಂಪ್ರೀತಾ ರಕ್ತಾಖ್ಯಾ ರಕ್ತಶೋಭಿತಾ ।\\
ರಕ್ತರೂಪಾ ರಕ್ತಗತಾ ರಕ್ತಖರ್ಪರಧಾರಿಣೀ ॥೪೭॥

ಗಲಚ್ಛೋಣಿತಮುಂಡಾಲೀ ಕಂಠಮಾಲಾವಿಭೂಷಿತಾ ।\\
ವೃಷಾಸನಾ ವೃಷರತಾ ವೃಷಾಸನಕೃತಾಶ್ರಯಾ ॥೪೮॥

ವ್ಯಾಘ್ರಚರ್ಮಾವೃತಾ ರೌದ್ರೀ ವ್ಯಾಘ್ರಚರ್ಮಾವಲೀ ತಥಾ ।\\
ಕಾಮಾಂಗೀ ಪರಮಾ ಪ್ರೀತಾ ಪರಾಸುರನಿವಾಸಿನೀ ॥೪೯॥

ತರುಣಾ ತರುಣಪ್ರಾಣಾ ತಥಾ ತರುಣಮರ್ದಿನೀ ।\\
ತರುಣಪ್ರೇಮದಾ ವೃದ್ಧಾ ತಥಾ ವೃದ್ಧಪ್ರಿಯಾ ಸತೀ ॥೫೦॥

ಸ್ವಪ್ನಾವತೀ ಸ್ವಪ್ನರತಾ ನಾರಸಿಂಹೀ ಮಹಾಲಯಾ ।\\
ಅಮೋಘಾ ರುಂಧತೀ ರಮ್ಯಾ ತೀಕ್ಷ್ಣಾ ಭೋಗವತೀ ಸದಾ ॥೫೧॥

ಮಂದಾಕಿನೀ ಮಂದರತಾ ಮಹಾನಂದಾ ವರಪ್ರದಾ ।\\
ಮಾನದಾ ಮಾನಿನೀ ಮಾನ್ಯಾ ಮಾನನೀಯಾ ಮದಾತುರಾ ॥೫೨॥

ಮದಿರಾ ಮದಿರೋನ್ಮಾದಾ ಮದಿರಾಕ್ಷೀ ಮದಾಲಯಾ ।\\
ಸುದೀರ್ಘಾ ಮಧ್ಯಮಾ ನಂದಾ ವಿನತಾಸುರನಿರ್ಗತಾ ॥೫೩॥

ಜಯಪ್ರದಾ ಜಯರತಾ ದುರ್ಜಯಾಸುರನಾಶಿನೀ ।\\
ದುಷ್ಟದೈತ್ಯನಿಹಂತ್ರೀ ಚ ದುಷ್ಟಾಸುರವಿನಾಶಿನೀ ॥೫೪॥

ಸುಖದಾ ಮೋಕ್ಷದಾ ಮೋಕ್ಷಾ ಮಹಾಮೋಕ್ಷಪ್ರದಾಯಿನೀ ।\\
ಕೀರ್ತಿರ್ಯಶಸ್ವಿನೀ ಭೂಷಾ ಭೂಷ್ಯಾ ಭೂತಪತಿಪ್ರಿಯಾ ॥೫೫॥

ಗುಣಾತೀತಾ ಗುಣಪ್ರೀತಾ ಗುಣರಕ್ತಾ ಗುಣಾತ್ಮಿಕಾ ।\\
ಸಗುಣಾ ನಿರ್ಗುಣಾ ಸೀತಾ ನಿಷ್ಠಾ ಕಾಷ್ಠಾ ಪ್ರತಿಷ್ಠಿತಾ ॥೫೬॥

ಧನಿಷ್ಠಾ ಧನದಾ ಧನ್ಯಾ ವಸುದಾ ಸುಪ್ರಕಾಶಿನೀ ।\\
ಗುರ್ವೀ ಗುರುತರಾ ಧೌಮ್ಯಾ ಧೌಮ್ಯಾಸುರವಿನಾಶಿನೀ ॥೫೭॥

ನಿಷ್ಕಾಮಾ ಧನದಾ ಕಾಮಾ ಸಕಾಮಾ ಕಾಮಜೀವನಾ ।\\
ಚಿಂತಾಮಣಿಃ ಕಲ್ಪಲತಾ ತಥಾ ಶಂಕರವಾಹಿನೀ ॥೫೮॥

ಶಂಕರೀ ಶಂಕರರತಾ ತಥಾ ಶಂಕರಮೋಹಿನೀ ।\\
ಭವಾನೀ ಭವದಾ ಭವ್ಯಾ ಭವಪ್ರೀತಾ ಭವಾಲಯಾ ॥೫೯॥

ಮಹಾದೇವಪ್ರಿಯಾ ರಮ್ಯಾ ರಮಣೀ ಕಾಮಸುಂದರೀ ।\\
ಕದಲೀಸ್ತಂಭಸಂರಾಮಾ ನಿರ್ಮಲಾಸನವಾಸಿನೀ ॥೬೦॥

ಮಾಥುರೀ ಮಥುರಾ ಮಾಯಾ ತಥಾ ಸುರಭಿವರ್ದ್ಧಿನೀ ।\\
ವ್ಯಕ್ತಾವ್ಯಕ್ತಾನೇಕರೂಪಾ ಸರ್ವತೀರ್ಥಾಸ್ಪದಾ ಶಿವಾ ॥೬೧॥

ತೀರ್ಥರೂಪಾ ಮಹಾರೂಪಾ ತಥಾಗಸ್ತ್ಯವಧೂರಪಿ ।\\
ಶಿವಾನೀ ಶೈವಲಪ್ರೀತಾ ತಥಾ ಶೈವಲವಾಸಿನೀ ॥೬೨॥

ಕುಂತಲಾ ಕುಂತಲಪ್ರೀತಾ ತಥಾ ಕುಂತಲಶೋಭಿತಾ ।\\
ಮಹಾಕಚಾ ಮಹಾಬುದ್ಧಿರ್ಮಹಾಮಾಯಾ ಮಹಾಗದಾ ॥೬೩॥

ಮಹಾಮೇಘಸ್ವರೂಪಾ ಚ ತಥಾ ಕಂಕಣಮೋಹಿನೀ ।\\
ದೇವಪೂಜ್ಯಾ ದೇವರತಾ ಯುವತೀ ಸರ್ವಮಂಗಲಾ ॥೬೪॥

ಸರ್ವಪ್ರಿಯಂಕರೀ ಭೋಗ್ಯಾ ಭೋಗರೂಪಾ ಭಗಾಕೃತಿಃ ।\\
ಭಗಪ್ರೀತಾ ಭಗರತಾ ಭಗಪ್ರೇಮರತಾ ಸದಾ ॥೬೫॥

ಭಗಸಂಮರ್ದನಪ್ರೀತಾ ಭಗೋಪರಿನಿವೇಶಿತಾ ।\\
ಭಗದಕ್ಷಾ ಭಗಾಕ್ರಾಂತಾ ಭಗಸೌಭಾಗ್ಯವರ್ದ್ಧಿನೀ ॥೬೬॥

ದಕ್ಷಕನ್ಯಾ ಮಹಾದಕ್ಷಾ ಸರ್ವದಕ್ಷಾ ಪ್ರಚಂಡಿಕಾ ।\\
ದಂಡಪ್ರಿಯಾ ದಂಡರತಾ ದಂಡತಾಡನತತ್ಪರಾ ॥೬೭॥

ದಂಡಭೀತಾ ದಂಡಗತಾ ದಂಡಸಂಮರ್ದನೇ ರತಾ ।\\
ಸುವೇದಿದಂಡಮಧ್ಯಸ್ಥಾ ಭೂರ್ಭುವಃಸ್ವಃಸ್ವರೂಪಿಣೀ ॥೬೮॥

ಆದ್ಯಾ ದುರ್ಗಾ ಜಯಾ ಸೂಕ್ಷ್ಮಾ ಸೂಕ್ಷ್ಮರೂಪಾ ಜಯಾಕೃತಿಃ ।\\
ಕ್ಷೇಮಂಕರೀ ಮಹಾಘೂರ್ಣಾ ಘೂರ್ಣನಾಸಾ ವಶಂಕರೀ ॥೬೯॥

ವಿಶಾಲಾವಯವಾ ಮೇಘ್ಯಾ ತ್ರಿವಲೀವಲಯಾ ಶುಭಾ ।\\
ಮದೋನ್ಮತ್ತಾ ಮದರತಾ ಮತ್ತಾಸುರವಿನಾಶಿನೀ ॥೭೦॥

ಮಧುಕೈಟಭಸಂಹಂತ್ರೀ ನಿಶುಂಭಾಸುರಮರ್ದಿನೀ ।\\
ಚಂಡರೂಪಾ ಮಹಾಚಂಡೀ ಚಂಡಿಕಾ ಚಂಡನಾಯಿಕಾ ॥೭೧॥

ಚಂಡೋಗ್ರಾ ಚಂಡವರ್ಣಾ ಪ್ರಚಂಡಾ ಚಂಡಾವತೀ ಶಿವಾ ।\\
ನೀಲಾಕಾರಾ ನೀಲವರ್ಣಾ ನೀಲೇಂದೀವರಲೋಚನಾ ॥೭೨॥

ಖಡ್ಗಹಸ್ತಾ ಚ ಮೃದ್ವಂಗೀ ತಥಾ ಖರ್ಪರಧಾರಿಣೀ ।\\
ಭೀಮಾ ಚ ಭೀಮವದನಾ ಮಹಾಭೀಮಾ ಭಯಾನಕಾ ॥೭೩॥

ಕಲ್ಯಾಣೀ ಮಂಗಲಾ ಶುದ್ಧಾ ತಥಾ ಪರಮಕೌತುಕಾ ।\\
ಪರಮೇಷ್ಠೀ ಪರರತಾ ಪರಾತ್ಪರತರಾ ಪರಾ ॥೭೪॥

ಪರಾನಂದಸ್ವರೂಪಾ ಚ ನಿತ್ಯಾನಂದಸ್ವರೂಪಿಣೀ ।\\
ನಿತ್ಯಾ ನಿತ್ಯಪ್ರಿಯಾ ತಂದ್ರೀ ಭವಾನೀ ಭವಸುಂದರೀ ॥೭೫॥

ತ್ರೈಲೋಕ್ಯಮೋಹಿನೀ ಸಿದ್ಧಾ ತಥಾ ಸಿದ್ಧಜನಪ್ರಿಯಾ ।\\
ಭೈರವೀ ಭೈರವಪ್ರೀತಾ ತಥಾ ಭೈರವಮೋಹಿನೀ ॥೭೬॥

ಮಾತಂಗೀ ಕಮಲಾ ಲಕ್ಷ್ಮೀಃ ಷೋಡಶೀ ವಿಷಯಾತುರಾ ।\\
ವಿಷಮಗ್ನಾ ವಿಷರತಾ ವಿಷರಕ್ಷಾ ಜಯದ್ರಥಾ ॥೭೭॥

ಕಾಕಪಕ್ಷಧರಾ ನಿತ್ಯಾ ಸರ್ವವಿಸ್ಮಯಕಾರಿಣೀ ।\\
ಗದಿನೀ ಕಾಮಿನೀ ಖಡ್ಗಮುಂಡಮಾಲಾವಿಭೂಷಿತಾ ॥೭೮॥

ಯೋಗೀಶ್ವರೀ ಯೋಗಮಾತಾ ಯೋಗಾನಂದಸ್ವರೂಪಿಣೀ ।\\
ಆನಂದಭೈರವೀ ನಂದಾ ತಥಾ ನಂದಜನಪ್ರಿಯಾ ॥೭೯॥

ನಲಿನೀ ಲಲನಾ ಶುಭ್ರಾ ಶುಭ್ರಾನನವಿಭೂಷಿತಾ ।\\
ಲಲಜ್ಜಿಹ್ವಾ ನೀಲಪದಾ ತಥಾ ಸುಮಖದಕ್ಷಿಣಾ ॥೮೦॥

ಬಲಿಭಕ್ತಾ ಬಲಿರತಾ ಬಲಿಭೋಗ್ಯಾ ಮಹಾರತಾ ।\\
ಫಲಭೋಗ್ಯಾ ಫಲರಸಾ ಫಲದಾ ಶ್ರೀಫಲಪ್ರಿಯಾ ॥೮೧॥

ಫಲಿನೀ ಫಲಸಂವಜ್ರಾ ಫಲಾಫಲನಿವಾರಿಣೀ ।\\
ಫಲಪ್ರೀತಾ ಫಲಗತಾ ಫಲಸಂದಾನಸಂಧಿನೀ ॥೮೨॥

ಫಲೋನ್ಮುಖೀ ಸರ್ವಸತ್ತ್ವಾ ಮಹಾಸತ್ತ್ವಾ ಚ ಸಾತ್ತ್ವಿಕೀ ।\\
ಸರ್ವರೂಪಾ ಸರ್ವರತಾ ಸರ್ವಸತ್ತ್ವನಿವಾಸಿನೀ ॥೮೩॥

ಮಹಾರೂಪಾ ಮಹಾಭಾಗಾ ಮಹಾಮೇಘಸ್ವರೂಪಿಣೀ ।\\
ಭಯನಾಸಾ ಗಣರತಾ ಗಣಪ್ರೀತಾ ಮಹಾಗತಿಃ ॥೮೪॥

ಸದ್ಗತಿಃ ಸತ್ಕೃತಿಃ ಸ್ವಕ್ಷಾ ಶವಾಸನಗತಾ ಶುಭಾ ।\\
ತ್ರೈಲೋಕ್ಯಮೋಹಿನೀ ಗಂಗಾ ಸ್ವರ್ಗಂಗಾ ಸ್ವರ್ಗವಾಸಿನೀ ॥೮೫॥

ಮಹಾನಂದಾ ಸದಾನಂದಾ ನಿತ್ಯಾನಿತ್ಯಸ್ವರೂಪಿಕಾ ।\\
ಸತ್ಯಗಂಧಾ ಸತ್ಯಗಣಾ ಸತ್ಯರೂಪಾ ಮಹಾಕೃತಿಃ ॥೮೬॥

ಶ್ಮಶಾನಭೈರವೀ ಕಾಲೀ ತಥಾ ಭಯವಿಮರ್ದಿನೀ ।\\
ತ್ರಿಪುರಾ ಪರಮೇಶಾನೀ ಸುಂದರೀ ಪುರಸುಂದರೀ ॥೮೭॥

ತ್ರಿಪುರೇಶೀ ಪಂಚದಶೀ ಪಂಚಮೀ ಪುರವಾಸಿನೀ ।\\
ಮಹಾಸಪ್ತದಶೀ ಷಷ್ಠೀ ಸಪ್ತಮೀ ಚಾಷ್ಟಮೀ ತಥಾ ॥೮೮॥

ನವಮೀ ದಶಮೀ ದೇವಪ್ರಿಯಾ ಚೈಕಾದಶೀ ಶಿವಾ ।\\
ದ್ವಾದಶೀ ಪರಮಾ ದಿವ್ಯಾ ನೀಲರೂಪಾ ತ್ರಯೋದಶೀ ॥೮೯॥

ಚತುರ್ದಶೀ ಪೌರ್ಣಮಾಸೀ ರಾಜರಾಜೇಶ್ವರೀ ತಥಾ ।\\
ತ್ರಿಪುರಾ ತ್ರಿಪುರೇಶೀ ಚ ತಥಾ ತ್ರಿಪುರಮರ್ದಿನೀ ॥೯೦॥

ಸರ್ವಾಂಗಸುಂದರೀ ರಕ್ತಾ ರಕ್ತವಸ್ತ್ರೋಪವೀತಿನೀ ।\\
ಚಾಮರೀ ಚಾಮರಪ್ರೀತಾ ಚಮರಾಸುರಮರ್ದಿನೀ ॥೯೧॥

ಮನೋಜ್ಞಾ ಸುಂದರೀ ರಮ್ಯಾ ಹಂಸೀ ಚ ಚಾರುಹಾಸಿನೀ ।\\
ನಿತಂಬಿನೀ ನಿತಂಬಾಢ್ಯಾ ನಿತಂಬಗುರುಶೋಭಿತಾ ॥೯೨॥

ಪಟ್ಟವಸ್ತ್ರಪರಿಧಾನಾ ಪಟ್ಟವಸ್ತ್ರಧರಾ ಶುಭಾ ।\\
ಕರ್ಪೂರಚಂದ್ರವದನಾ ಕುಂಕುಮದ್ರವಶೋಭಿತಾ ॥೯೩॥

ಪೃಥಿವೀ ಪೃಥುರೂಪಾ ಸಾ ಪಾರ್ಥಿವೇಂದ್ರವಿನಾಶಿನೀ ।\\
ರತ್ನವೇದಿಃ ಸುರೇಶಾ ಚ ಸುರೇಶೀ ಸುರಮೋಹಿನೀ ॥೯೪॥

ಶಿರೋಮಣಿರ್ಮಣಿಗ್ರೀವಾ ಮಣಿರತ್ನವಿಭೂಷಿತಾ ।\\
ಉರ್ವಶೀ ಶಮನೀ ಕಾಲೀ ಮಹಾಕಾಲಸ್ವರೂಪಿಣೀ ॥೯೫॥

ಸರ್ವರೂಪಾ ಮಹಾಸತ್ತ್ವಾ ರೂಪಾಂತರವಿಲಾಸಿನೀ ।\\
ಶಿವಾ ಶೈವಾ ಚ ರುದ್ರಾಣೀ ತಥಾ ಶಿವನಿನಾದಿನೀ ॥೯೬॥

ಮಾತಂಗಿನೀ ಭ್ರಾಮರೀ ಚ ತಥೈವಾಂಗನಮೇಖಲಾ ।\\
ಯೋಗಿನೀ ಡಾಕಿನೀ ಚೈವ ತಥಾ ಮಹೇಶ್ವರೀ ಪರಾ ॥೯೭॥

ಅಲಂಬುಷಾ ಭವಾನೀ ಚ ಮಹಾವಿದ್ಯೌಘಸಂಭೃತಾ ।\\
ಗೃಧ್ರರೂಪಾ ಬ್ರಹ್ಮಯೋನಿರ್ಮಹಾನಂದಾ ಮಹೋದಯಾ ॥೯೮॥

ವಿರೂಪಾಕ್ಷಾ ಮಹಾನಾದಾ ಚಂಡರೂಪಾ ಕೃತಾಕೃತಿಃ ।\\
ವರಾರೋಹಾ ಮಹಾವಲ್ಲೀ ಮಹಾತ್ರಿಪುರಸುಂದರೀ ॥೯೯॥

ಭಗಾತ್ಮಿಕಾ ಭಗಾಧಾರರೂಪಿಣೀ ಭಗಮಾಲಿನೀ ।\\
ಲಿಂಗಾಭಿಧಾಯಿನೀ ದೇವೀ ಮಹಾಮಾಯಾ ಮಹಾಸ್ಮೃತಿಃ ॥೧೦೦॥

ಮಹಾಮೇಧಾ ಮಹಾಶಾಂತಾ ಶಾಂತರೂಪಾ ವರಾನನಾ ।\\
ಲಿಂಗಮಾಲಾ ಲಿಂಗಭೂಷಾ ಭಗಮಾಲಾವಿಭೂಷಣಾ ॥೧೦೧॥

ಭಗಲಿಂಗಾಮೃತಪ್ರೀತಾ ಭಗಲಿಂಗಾಮೃತಾತ್ಮಿಕಾ ।\\
ಭಗಲಿಂಗಾರ್ಚನಪ್ರೀತಾ ಭಗಲಿಂಗಸ್ವರೂಪಿಣೀ ॥೧೦೨॥

ಸ್ವಯಂಭೂಕುಸುಮಪ್ರೀತಾ ಸ್ವಯಂಭೂಕುಸುಮಾಸನಾ ।\\
ಸ್ವಯಂಭೂಕುಸುಮರತಾ ಲತಾಲಿಂಗನತತ್ಪರಾ ॥೧೦೩॥

ಸುರಾಶನಾ ಸುರಾಪ್ರೀತಾ ಸುರಾಸವವಿಮರ್ದಿತಾ ।\\
ಸುರಾಪಾನಮಹಾತೀಕ್ಷ್ಣಾ ಸರ್ವಾಗಮವಿನಿಂದಿತಾ ॥೧೦೪॥

ಕುಂಡಗೋಲಸದಾಪ್ರೀತಾ ಗೋಲಪುಷ್ಪಸದಾರತಿಃ ।\\
ಕುಂಡಗೋಲೋದ್ಭವಪ್ರೀತಾ ಕುಂಡಗೋಲೋದ್ಭವಾತ್ಮಿಕಾ ॥೧೦೫॥

ಸ್ವಯಂಭವಾ ಶಿವಾ ಧಾತ್ರೀ ಪಾವನೀ ಲೋಕಪಾವನೀ ।\\
ಮಹಾಲಕ್ಷ್ಮೀರ್ಮಹೇಶಾನೀ ಮಹಾವಿಷ್ಣುಪ್ರಭಾವಿನೀ ॥೧೦೬॥

ವಿಷ್ಣುಪ್ರಿಯಾ ವಿಷ್ಣುರತಾ ವಿಷ್ಣುಭಕ್ತಿಪರಾಯಣಾ ।\\
ವಿಷ್ಣೋರ್ವಕ್ಷಃಸ್ಥಲಸ್ಥಾ ಚ ವಿಷ್ಣುರೂಪಾ ಚ ವೈಷ್ಣವೀ ॥೧೦೭॥

ಅಶ್ವಿನೀ ಭರಣೀ ಚೈವ ಕೃತ್ತಿಕಾ ರೋಹಿಣೀ ತಥಾ ।\\
ಧೃತಿರ್ಮೇಧಾ ತಥಾ ತುಷ್ಟಿಃ ಪುಷ್ಟಿರೂಪಾ ಚಿತಾ ಚಿತಿಃ ॥೧೦೮॥

ಚಿತಿರೂಪಾ ಚಿತ್ಸ್ವರೂಪಾ ಜ್ಞಾನರೂಪಾ ಸನಾತನೀ ।\\
ಸರ್ವವಿಜ್ಞಜಯಾ ಗೌರೀ ಗೌರವರ್ಣಾ ಶಚೀ ಶಿವಾ ॥೧೦೯॥

ಭವರೂಪಾ ಭವಪರಾ ಭವಾನೀ ಭವಮೋಚಿನೀ ।\\
ಪುನರ್ವಸುಸ್ತಥಾ ಪುಷ್ಯಾ ತೇಜಸ್ವೀ ಸಿಂಧುವಾಸಿನೀ ॥೧೧೦॥

ಶುಕ್ರಾಶನಾ ಶುಕ್ರಭೋಗಾ ಶುಕ್ರೋತ್ಸಾರಣತತ್ಪರಾ ।\\
ಶುಕ್ರಪೂಜ್ಯಾ ಶುಕ್ರವಂದ್ಯಾ ಶುಕ್ರಭೋಗ್ಯಾ ಪುಲೋಮಜಾ ॥೧೧೧॥

ಶುಕ್ರಾರ್ಚ್ಯಾ ಶುಕ್ರಸಂತುಷ್ಟಾ ಸರ್ವಶುಕ್ರವಿಮುಕ್ತಿದಾ ।\\
ಶುಕ್ರಮೂರ್ತಿಃ ಶುಕ್ರದೇಹಾ ಶುಕ್ರಾಂಗೀ ಶುಕ್ರಮೋಹಿನೀ ॥೧೧೨॥

ದೇವಪೂಜ್ಯಾ ದೇವರತಾ ಯುವತೀ ಸರ್ವಮಂಗಲಾ ।\\
ಸರ್ವಪ್ರಿಯಂಕರೀ ಭೋಗ್ಯಾ ಭೋಗರೂಪಾ ಭಗಾಕೃತಿಃ ॥೧೧೩॥

ಭಗಪ್ರೇತಾ ಭಗರತಾ ಭಗಪ್ರೇಮಪರಾ ತಥಾ ।\\
ಭಗಸಂಮರ್ದನಪ್ರೀತಾ ಭಗೋಪರಿ ನಿವೇಶಿತಾ ॥೧೧೪॥

ಭಗದಕ್ಷಾ ಭಗಾಕ್ರಾಂತಾ ಭಗಸೌಭಾಗ್ಯವರ್ದ್ಧಿನೀ ।\\
ದಕ್ಷಕನ್ಯಾ ಮಹಾದಕ್ಷಾ ಸರ್ವದಕ್ಷಾ ಪ್ರದಂತಿಕಾ ॥೧೧೫॥

ದಂಡಪ್ರಿಯಾ ದಂಡರತಾ ದಂಡತಾಡನತತ್ಪರಾ ।\\
ದಂಡಭೀತಾ ದಂಡಗತಾ ದಂಡಸಂಮರ್ದನೇ ರತಾ ॥೧೧೬॥

ವೇದಿಮಂಡಲಮಧ್ಯಸ್ಥಾ ಭೂರ್ಭುವಃಸ್ವಃಸ್ವರೂಪಿಣೀ ।\\
ಆದ್ಯಾ ದುರ್ಗಾ ಜಯಾ ಸೂಕ್ಷ್ಮಾ ಸೂಕ್ಷ್ಮರೂಪಾ ಜಯಾಕೃತಿಃ ॥೧೧೭॥

ಕ್ಷೇಮಂಕರೀ ಮಹಾಘೂರ್ಣಾ ಘೂರ್ಣನಾಸಾ ವಶಂಕರೀ ।\\
ವಿಶಾಲಾವಯವಾ ಮೇಧ್ಯಾ ತ್ರಿವಲೀವಲಯಾ ಶುಭಾ ॥೧೧೮॥

ಮದ್ಯೋನ್ಮತ್ತಾ ಮದ್ಯರತಾ ಮತ್ತಾಸುರವಿಲಾಸಿನೀ ।\\
ಮಧುಕೈಟಭಸಂಹಂತ್ರೀ ನಿಶುಂಭಾಸುರಮರ್ದಿನೀ ॥೧೧೯॥

ಚಂಡರೂಪಾ ಮಹಾಚಂಡಾ ಚಂಡಿಕಾ ಚಂಡನಾಯಿಕಾ ।\\
ಚಂಡೋಗ್ರಾ ಚ ಚತುರ್ವರ್ಗಾ ತಥಾ ಚಂಡಾವತೀ ಶಿವಾ ॥೧೨೦॥

ನೀಲದೇಹಾ ನೀಲವರ್ಣಾ ನೀಲೇಂದೀವರಲೋಚನಾ ।\\
ನಿತ್ಯಾನಿತ್ಯಪ್ರಿಯಾ ಭದ್ರಾ ಭವಾನೀ ಭವಸುಂದರೀ ॥೧೨೧॥

ಭೈರವೀ ಭೈರವಪ್ರೀತಾ ತಥಾ ಭೈರವಮೋಹಿನೀ ।\\
ಮಾತಂಗೀ ಕಮಲಾ ಲಕ್ಷ್ಮೀಃ ಷೋಡಶೀ ಭೀಷಣಾತುರಾ ॥೧೨೨॥

ವಿಷಮಗ್ನಾ ವಿಷರತಾ ವಿಷಭಕ್ಷ್ಯಾ ಜಯಾ ತಥಾ ।\\
ಕಾಕಪಕ್ಷಧರಾ ನಿತ್ಯಾ ಸರ್ವವಿಸ್ಮಯಕಾರಿಣೀ ॥೧೨೩॥

ಗದಿನೀ ಕಾಮಿನೀ ಖಡ್ಗಾ ಮುಂಡಮಾಲಾವಿಭೂಷಿತಾ ।\\
ಯೋಗೇಶ್ವರೀ ಯೋಗರತಾ ಯೋಗಾನಂದಸ್ವರೂಪಿಣೀ ॥೧೨೪॥

ಆನಂದಭೈರವೀ ನಂದಾ ತಥಾನಂದಜನಪ್ರಿಯಾ ।\\
ನಲಿನೀ ಲಲನಾ ಶುಭ್ರಾ ಶುಭಾನನವಿರಾಜಿತಾ ॥೧೨೫॥

ಲಲಜ್ಜಿಹ್ವಾ ನೀಲಪದಾ ತಥಾ ಸಂಮುಖದಕ್ಷಿಣಾ ।\\
ಬಲಿಭಕ್ತಾ ಬಲಿರತಾ ಬಲಿಭೋಗ್ಯಾ ಮಹಾರತಾ ॥೧೨೬॥

ಫಲಭೋಗ್ಯಾ ಫಲರಸಾ ಫಲದಾತ್ರೀ ಫಲಪ್ರಿಯಾ ।\\
ಫಲಿನೀ ಫಲಸಂರಕ್ತಾ ಫಲಾಫಲನಿವಾರಿಣೀ ॥೧೨೭॥

ಫಲಪ್ರೀತಾ ಫಲಗತಾ ಫಲಸಂಧಾನಸಂಧಿನೀ ।\\
ಫಲೋನ್ಮುಖೀ ಸರ್ವಸತ್ತ್ವಾ ಮಹಾಸತ್ತ್ವಾ ಚ ಸಾತ್ತ್ವಿಕಾ ॥೧೨೮॥

ಸರ್ವರೂಪಾ ಸರ್ವರತಾ ಸರ್ವಸತ್ತ್ವನಿವಾಸಿನೀ ।\\
ಮಹಾರೂಪಾ ಮಹಾಭಾಗಾ ಮಹಾಮೇಘಸ್ವರೂಪಿಣೀ ॥೧೨೯॥

ಭಯನಾಶಾ ಗಣರತಾ ಗಣಗೀತಾ ಮಹಾಗತಿಃ ।\\
ಸದ್ಗತಿಃ ಸತ್ಕೃತಿಃ ಸಾಕ್ಷಾತ್ ಸದಾಸನಗತಾ ಶುಭಾ ॥೧೩೦॥

ತ್ರೈಲೋಕ್ಯಮೋಹಿನೀ ಗಂಗಾ ಸ್ವರ್ಗಂಗಾ ಸ್ವರ್ಗವಾಸಿನೀ ।\\
ಮಹಾನಂದಾ ಸದಾನಂದಾ ನಿತ್ಯಾ ಸತ್ಯಸ್ವರೂಪಿಣೀ ॥೧೩೧॥

ಶುಕ್ರಸ್ನಾತಾ ಶುಕ್ರಕರೀ ಶುಕ್ರಸೇವ್ಯಾತಿಶುಕ್ರಿಣೀ ।\\
ಮಹಾಶುಕ್ರಾ ಶುಕ್ರರತಾ ಶುಕ್ರಸೃಷ್ಟಿವಿಧಾಯಿನೀ ॥೧೩೨॥

ಸಾರದಾ ಸಾಧಕಪ್ರಾಣಾ ಸಾಧಕಪ್ರೇಮವರ್ದ್ಧಿನೀ ।\\
ಸಾಧಕಾಭೀಷ್ಟದಾ ನಿತ್ಯಂ ಸಾಧಕಪ್ರೇಮಸೇವಿತಾ ॥೧೩೩॥

ಸಾಧಕಪ್ರೇಮಸರ್ವಸ್ವಾ ಸಾಧಕಾಭಕ್ತರಕ್ತಪಾ ।\\
ಮಲ್ಲಿಕಾ ಮಾಲತೀ ಜಾತಿಃ ಸಪ್ತವರ್ಣಾ ಮಹಾಕಚಾ ॥೧೩೪॥

ಸರ್ವಮಯೀ ಸರ್ವಶುಭ್ರಾ ಗಾಣಪತ್ಯಪ್ರದಾ ತಥಾ ।\\
ಗಗನಾ ಗಗನಪ್ರೀತಾ ತಥಾ ಗಗನವಾಸಿನೀ ॥೧೩೫॥

ಗಣನಾಥಪ್ರಿಯಾ ಭವ್ಯಾ ಭವಾರ್ಚಾ ಸರ್ವಮಂಗಲಾ ।\\
ಗುಹ್ಯಕಾಲೀ ಭದ್ರಕಾಲೀ ಶಿವರೂಪಾ ಸತಾಂಗತಿಃ ॥೧೩೬॥

ಸದ್ಭಕ್ತಾ ಸತ್ಪರಾ ಸೇತುಃ ಸರ್ವಾಂಗಸುಂದರೀ ಮಘಾ ।\\
ಕ್ಷೀಣೋದರೀ ಮಹಾವೇಗಾ ವೇಗಾನಂದಸ್ವರೂಪಿಣೀ ॥೧೩೭॥

ರುಧಿರಾ ರುಧಿರಪ್ರೀತಾ ರುಧಿರಾನಂದಶೋಭನಾ ।\\
ಪಂಚಮೀ ಪಂಚಮಪ್ರೀತಾ ತಥಾ ಪಂಚಮಭೂಷಣಾ ॥೧೩೮॥

ಪಂಚಮೀಜಪಸಂಪನ್ನಾ ಪಂಚಮೀಯಜನೇ ರತಾ ।\\
ಕಕಾರವರ್ಣರೂಪಾ ಚ ಕಕಾರಾಕ್ಷರರೂಪಿಣೀ ॥೧೩೯॥

ಮಕಾರಪಂಚಮಪ್ರೀತಾ ಮಕಾರಪಂಚಗೋಚರಾ ।\\
ಋವರ್ಣರೂಪಪ್ರಭವಾ ಋವರ್ಣಾ ಸರ್ವರೂಪಿಣೀ ॥೧೪೦॥

ಸರ್ವಾಣೀ ಸರ್ವನಿಲಯಾ ಸರ್ವಸಾರಸಮುದ್ಭವಾ ।\\
ಸರ್ವೇಶ್ವರೀ ಸರ್ವಸಾರಾ ಸರ್ವೇಚ್ಛಾ ಸರ್ವಮೋಹಿನೀ ॥೧೪೧॥

ಗಣೇಶಜನನೀ ದುರ್ಗಾ ಮಹಾಮಾಯಾ ಮಹೇಶ್ವರೀ ।\\
ಮಹೇಶಜನನೀ ಮೋಹಾ ವಿದ್ಯಾ ವಿದ್ಯೋತನೀ ವಿಭಾ ॥೧೪೨॥

ಸ್ಥಿರಾ ಚ ಸ್ಥಿರಚಿತ್ತಾ ಚ ಸುಸ್ಥಿರಾ ಧರ್ಮರಂಜಿನೀ ।\\
ಧರ್ಮರೂಪಾ ಧರ್ಮರತಾ ಧರ್ಮಾಚರಣತತ್ಪರಾ ॥೧೪೩॥

ಧರ್ಮಾನುಷ್ಠಾನಸಂದರ್ಭಾ ಸರ್ವಸಂದರ್ಭಸುಂದರೀ ।\\
ಸ್ವಧಾ ಸ್ವಾಹಾ ವಷಟ್ಕಾರಾ ಶ್ರೌಷಟ್ ವೌಷಟ್ ಸ್ವಧಾತ್ಮಿಕಾ ॥೧೪೪॥

ಬ್ರಾಹ್ಮಣೀ ಬ್ರಹ್ಮಸಂಬಂಧಾ ಬ್ರಹ್ಮಸ್ಥಾನನಿವಾಸಿನೀ ।\\
ಪದ್ಮಯೋನಿಃ ಪದ್ಮಸಂಸ್ಥಾ ಚತುರ್ವರ್ಗಫಲಪ್ರದಾ ॥೧೪೫॥

ಚತುರ್ಭುಜಾ ಶಿವಯುತಾ ಶಿವಲಿಂಗಪ್ರವೇಶಿನೀ ।\\
ಮಹಾಭೀಮಾ ಚಾರುಕೇಶೀ ಗಂಧಮಾದನಸಂಸ್ಥಿತಾ ॥೧೪೬॥

ಗಂಧರ್ವಪೂಜಿತಾ ಗಂಧಾ ಸುಗಂಧಾ ಸುರಪೂಜಿತಾ ।\\
ಗಂಧರ್ವನಿರತಾ ದೇವೀ ಸುರಭೀ ಸುಗಂಧಾ ತಥಾ ॥೧೪೭॥

ಪದ್ಮಗಂಧಾ ಮಹಾಗಂಧಾ ಗಂಧಾಮೋದಿತದಿಙ್ಮುಖಾ ।\\
ಕಾಲದಿಗ್ಧಾ ಕಾಲರತಾ ಮಹಿಷಾಸುರಮರ್ದಿನೀ ॥೧೪೮॥

ವಿದ್ಯಾ ವಿದ್ಯಾವತೀ ಚೈವ ವಿದ್ಯೇಶಾ ವಿಜ್ಞಸಂಭವಾ ।\\
ವಿದ್ಯಾಪ್ರದಾ ಮಹಾವಾಣೀ ಮಹಾಭೈರವರೂಪಿಣೀ ॥೧೪೯॥

ಭೈರವಪ್ರೇಮನಿರತಾ ಮಹಾಕಾಲರತಾ ಶುಭಾ ।\\
ಮಾಹೇಶ್ವರೀ ಗಜಾರೂಢಾ ಗಜೇಂದ್ರಗಮನಾ ತಥಾ ॥೧೫೦॥

ಯಜ್ಞೇಂದ್ರಲಲನಾ ಚಂಡೀ ಗಜಾಸನಪರಾಶ್ರಯಾ ।\\
ಗಜೇಂದ್ರಮಂದಗಮನಾ ಮಹಾವಿದ್ಯಾ ಮಹೋಜ್ಜ್ವಲಾ ॥೧೫೧॥

ಬಗಲಾ ವಾಹಿನೀ ವೃದ್ಧಾ ಬಾಲಾ ಚ ಬಾಲರೂಪಿಣೀ ।\\
ಬಾಲಕ್ರೀಡಾರತಾ ಬಾಲಾ ಬಲಾಸುರವಿನಾಶಿನೀ ॥೧೫೨॥

ಬಾಲ್ಯಸ್ಥಾ ಯೌವನಸ್ಥಾ ಚ ಮಹಾಯೌವನಸಂರತಾ ।\\
ವಿಶಿಷ್ಟಯೌವನಾ ಕಾಲೀ ಕೃಷ್ಣದುರ್ಗಾ ಸರಸ್ವತೀ ॥೧೫೩॥

ಕಾತ್ಯಾಯನೀ ಚ ಚಾಮುಂಡಾ ಚಂಡಾಸುರವಿಘಾತಿನೀ ।\\
ಚಂಡಮುಂಡಧರಾ ದೇವೀ ಮಧುಕೈಟಭನಾಶಿನೀ ॥೧೫೪॥

ಬ್ರಾಹ್ಮೀ ಮಾಹೇಶ್ವರೀ ಚೈಂದ್ರೀ ವಾರಾಹೀ ವೈಷ್ಣವೀ ತಥಾ ।\\
ರುದ್ರಕಾಲೀ ವಿಶಾಲಾಕ್ಷೀ ಭೈರವೀ ಕಾಲರೂಪಿಣೀ ॥೧೫೫॥

ಮಹಾಮಾಯಾ ಮಹೋತ್ಸಾಹಾ ಮಹಾಚಂಡವಿನಾಶಿನೀ ।\\
ಕುಲಶ್ರೀಃ ಕುಲಸಂಕೀರ್ಣಾ ಕುಲಗರ್ಭನಿವಾಸಿನೀ ॥೧೫೬॥

ಕುಲಾಂಗಾರಾ ಕುಲಯುತಾ ಕುಲಕುಂತಲಸಂಯುತಾ ।\\
ಕುಲದರ್ಭಗ್ರಹಾ ಚೈವ ಕುಲಗರ್ತಪ್ರದಾಯಿನೀ ॥೧೫೭॥

ಕುಲಪ್ರೇಮಯುತಾ ಸಾಧ್ವೀ ಶಿವಪ್ರೀತಿಃ ಶಿವಾಬಲಿಃ ।\\
ಶಿವಸಕ್ತಾ ಶಿವಪ್ರಾಣಾ ಮಹಾದೇವಕೃತಾಲಯಾ ॥೧೫೮॥

ಮಹಾದೇವಪ್ರಿಯಾ ಕಾಂತಾ ಮಹಾದೇವಮದಾತುರಾ ।\\
ಮತ್ತಾಮತ್ತಜನಪ್ರೇಮಧಾತ್ರೀ ವಿಭವವರ್ದ್ಧಿನೀ ॥೧೫೯॥

ಮದೋನ್ಮತ್ತಾ ಮಹಾಶುದ್ಧಾ ಮತ್ತಪ್ರೇಮವಿಭೂಷಿತಾ ।\\
ಮತ್ತಪ್ರಮತ್ತವದನಾ ಮತ್ತಚುಂಬನತತ್ಪರಾ ॥೧೬೦॥

ಮತ್ತಕ್ರೀಡಾತುರಾ ಭೈಮೀ ತಥಾ ಹೈಮವತೀ ಮತಿಃ ।\\
ಮದಾತುರಾ ಮದಗತಾ ವಿಪರೀತರತಾತುರಾ ॥೧೬೧॥

ವಿತ್ತಪ್ರದಾ ವಿತ್ತರತಾ ವಿತ್ತವರ್ಧನತತ್ಪರಾ ।\\
ಇತಿ ತೇ ಕಥಿತಂ ಸರ್ವಂ ಕಾಲೀನಾಮಸಹಸ್ರಕಂ ॥೧೬೨॥

ಸಾರಾತ್ಸಾರತರಂ ದಿವ್ಯಂ ಮಹಾವಿಭವವರ್ದ್ಧನಂ ।\\
ಗಾಣಪತ್ಯಪ್ರದಂ ರಾಜ್ಯಪ್ರದಂ ಷಟ್ಕರ್ಮಸಾಧಕಂ ॥೧೬೩॥

ಯಃ ಪಠೇತ್ ಸಾಧಕೋ ನಿತ್ಯಂ ಸ ಭವೇತ್ ಸಂಪದಾಂ ಪದಂ ।\\
ಯಃ ಪಠೇತ್ ಪಾಠಯೇದ್ವಾಪಿ ಶೃಣೋತಿ ಶ್ರಾವಯೇದಥ ॥೧೬೪॥

ನ ಕಿಂಚಿದ್ ದುರ್ಲಭಂ ಲೋಕೇ ಸ್ತವಸ್ಯಾಸ್ಯ ಪ್ರಸಾದತಃ ।\\
ಬ್ರಹ್ಮಹತ್ಯಾ ಸುರಾಪಾನಂ ಸುವರ್ಣಹರಣಂ ತಥಾ ॥೧೬೫॥

ಗುರುದಾರಾಭಿಗಮನಂ ಯಚ್ಚಾನ್ಯದ್ ದುಷ್ಕೃತಂ ಕೃತಂ ।\\
ಸರ್ವಮೇತತ್ಪುನಾತ್ಯೇವ ಸತ್ಯಂ ಸುರಗಣಾರ್ಚಿತೇ ॥೧೬೬॥

ರಜಸ್ವಲಾಭಗಂ ದೃಷ್ಟ್ವಾ ಪಠೇತ್ ಸ್ತೋತ್ರಮನನ್ಯಧೀಃ ।\\
ಸ ಶಿವಃ ಸತ್ಯವಾದೀ ಚ ಭವತ್ಯೇವ ನ ಸಂಶಯಃ ॥೧೬೭॥

ಪರದಾರಯುತೋ ಭೂತ್ವಾ ಪಠೇತ್ ಸ್ತೋತ್ರಂ ಸಮಾಹಿತಃ ।\\
ಸರ್ವೈಶ್ವರ್ಯಯುತೋ ಭೂತ್ವಾ ಮಹಾರಾಜತ್ವಮಾಪ್ನುಯಾತ್ ॥೧೬೮॥

ಪರನಿಂದಾಂ ಪರದ್ರೋಹಂ ಪರಹಿಂಸಾಂ ನ ಕಾರಯೇತ್ ।\\
ಶಿವಭಕ್ತಾಯ ಶಾಂತಾಯ ಪ್ರಿಯಭಕ್ತಾಯ ವಾ ಪುನಃ ॥೧೬೯॥

ಸ್ತವಂ ಚ ದರ್ಶಯೇದೇನಮನ್ಯಥಾ ಮೃತ್ಯುಮಾಪ್ನುಯಾತ್ ।\\
ಅಸ್ಮಾತ್ ಪರತರಂ ನಾಸ್ತಿ ತಂತ್ರಮಧ್ಯೇ ಸುರೇಶ್ವರಿ ॥೧೭೦॥

ಮಹಾಕಾಲೀ ಮಹಾದೇವೀ ತಥಾ ನೀಲಸರಸ್ವತೀ ।\\
ನ ಭೇದಃ ಪರಮೇಶಾನಿ ಭೇದಕೃನ್ನರಕಂ ವ್ರಜೇತ್ ॥೧೭೧॥

ಇದಂ ಸ್ತೋತ್ರಂ ಮಯಾ ದಿವ್ಯಂ ತವ ಸ್ನೇಹಾತ್ ಪ್ರಕಥ್ಯತೇ ।\\
ಉಭಯೋರೇವಮೇಕತ್ವಂ ಭೇದಬುದ್ಧ್ಯಾ ನ ತಾಂ ಭಜೇತ್ ।\\
ಸ ಯೋಗೀ ಪರಮೇಶಾನಿ ಸಮೋ ಮಾನಾಪಮಾನಯೋಃ ॥೧೭೨॥

\authorline{॥ಇತಿ ಶ್ರೀಬೃಹನ್ನೀಲತಂತ್ರೇ ಭೈರವಪಾರ್ವತೀಸಂವಾದೇ ಕಾಲೀಸಹಸ್ರನಾಮಸ್ತೋತ್ರಮ್॥}
%=============================================================================================
\section{ಶ್ರೀಕಾಲೀಶತನಾಮಸ್ತೋತ್ರಂ}
\addcontentsline{toc}{section}{ಶ್ರೀಕಾಲೀಶತನಾಮಸ್ತೋತ್ರಂ}


ಭೈರವ ಉವಾಚ ।\\
ಶತನಾಮ ಪ್ರವಕ್ಷ್ಯಾಮಿ ಕಾಲಿಕಾಯಾ ವರಾನನೇ ।\\
ಯಸ್ಯ ಪ್ರಪಠನಾದ್ವಾಗ್ಮೀ ಸರ್ವತ್ರ ವಿಜಯೀ ಭವೇತ್ ॥೧॥

ಕಾಲೀ ಕಪಾಲಿನೀ ಕಾಂತಾ ಕಾಮದಾ ಕಾಮಸುಂದರೀ ।\\
ಕಾಲರಾತ್ರಿಃ ಕಾಲಿಕಾ ಚ ಕಾಲಭೈರವಪೂಜಿತಾ ॥೨॥

ಕುರುಕುಲ್ಲಾ ಕಾಮಿನೀ ಚ ಕಮನೀಯಸ್ವಭಾವಿನೀ ।\\
ಕುಲೀನಾ ಕುಲಕರ್ತ್ರೀ ಚ ಕುಲವರ್ತ್ಮಪ್ರಕಾಶಿನೀ ॥೩॥

ಕಸ್ತೂರೀರಸನೀಲಾ ಚ ಕಾಮ್ಯಾ ಕಾಮಸ್ವರೂಪಿಣೀ ।\\
ಕಕಾರವರ್ಣನಿಲಯಾ ಕಾಮಧೇನುಃ ಕರಾಲಿಕಾ ॥೪॥

ಕುಲಕಾಂತಾ ಕರಾಲಾಸ್ಯಾ ಕಾಮಾರ್ತಾ ಚ ಕಲಾವತೀ ।\\
ಕೃಶೋದರೀ ಚ ಕಾಮಾಖ್ಯಾ ಕೌಮಾರೀ ಕುಲಪಾಲಿನೀ ॥೫॥

ಕುಲಜಾ ಕುಲಕನ್ಯಾ ಚ ಕುಲಹಾ ಕುಲಪೂಜಿತಾ ।\\
ಕಾಮೇಶ್ವರೀ ಕಾಮಕಾಂತಾ ಕುಂಜರೇಶ್ವರಗಾಮಿನೀ ॥೬॥

ಕಾಮದಾತ್ರೀ ಕಾಮಹರ್ತ್ರೀ ಕೃಷ್ಣಾ ಚೈವ ಕಪರ್ದಿನೀ ।\\
ಕುಮುದಾ ಕೃಷ್ಣದೇಹಾ ಚ ಕಾಲಿಂದೀ ಕುಲಪೂಜಿತಾ ॥೭॥

ಕಾಶ್ಯಪೀ ಕೃಷ್ಣಮಾತಾ ಚ ಕುಲಿಶಾಂಗೀ ಕಲಾ ತಥಾ ।\\
ಕ್ರೀಂರೂಪಾ ಕುಲಗಮ್ಯಾ ಚ ಕಮಲಾ ಕೃಷ್ಣಪೂಜಿತಾ ॥೮॥

ಕೃಶಾಂಗೀ ಕಿನ್ನರೀ ಕರ್ತ್ರೀ ಕಲಕಂಠೀ ಚ ಕಾರ್ತಿಕೀ ।\\
ಕಂಬುಕಂಠೀ ಕೌಲಿನೀ ಚ ಕುಮುದಾ ಕಾಮಜೀವಿನೀ ॥೯॥

ಕುಲಸ್ತ್ರೀ ಕೀರ್ತ್ತಿಕಾ ಕೃತ್ಯಾ ಕೀರ್ತಿಶ್ಚ ಕುಲಪಾಲಿಕಾ ।\\
ಕಾಮದೇವಕಲಾ ಕಲ್ಪಲತಾ ಕಾಮಾಂಗವರ್ಧಿನೀ ॥೧೦॥

ಕುಂತಾ ಚ ಕುಮುದಪ್ರೀತಾ ಕದಂಬಕುಸುಮೋತ್ಸುಕಾ ।\\
ಕಾದಂಬಿನೀ ಕಮಲಿನೀ ಕೃಷ್ಣಾನಂದಪ್ರದಾಯಿನೀ ॥೧೧॥

ಕುಮಾರೀಪೂಜನರತಾ ಕುಮಾರೀಗಣಶೋಭಿತಾ ।\\
ಕುಮಾರೀರಂಜನರತಾ ಕುಮಾರೀವ್ರತಧಾರಿಣೀ ॥೧೨॥

ಕಂಕಾಲೀ ಕಮನೀಯಾ ಚ ಕಾಮಶಾಸ್ತ್ರವಿಶಾರದಾ ।\\
ಕಪಾಲಖಟ್ವಾಂಗಧರಾ ಕಾಲಭೈರವರೂಪಿಣೀ ॥೧೩॥

ಕೋಟರೀ ಕೋಟರಾಕ್ಷೀ ಚ ಕಾಶೀ ಕೈಲಾಸವಾಸಿನೀ ।\\
ಕಾತ್ಯಾಯನೀ ಕಾರ್ಯಕರೀ ಕಾವ್ಯಶಾಸ್ತ್ರಪ್ರಮೋದಿನೀ ॥೧೪॥

ಕಾಮಾಕರ್ಷಣರೂಪಾ ಚ ಕಾಮಪೀಠನಿವಾಸಿನೀ ।\\
ಕಂಕಿನೀ ಕಾಕಿನೀ ಕ್ರೀಡಾ ಕುತ್ಸಿತಾ ಕಲಹಪ್ರಿಯಾ ॥೧೫॥

ಕುಂಡಗೋಲೋದ್ಭವಪ್ರಾಣಾ ಕೌಶಿಕೀ ಕೀರ್ತಿವರ್ಧಿನೀ ।\\
ಕುಂಭಸ್ತನೀ ಕಟಾಕ್ಷಾ ಚ ಕಾವ್ಯಾ ಕೋಕನದಪ್ರಿಯಾ ॥೧೬॥

ಕಾಂತಾರವಾಸಿನೀ ಕಾಂತಿಃ ಕಠಿನಾ ಕೃಷ್ಣವಲ್ಲಭಾ ।\\
ಇತಿ ತೇ ಕಥಿತಂ ದೇವಿ ಗುಹ್ಯಾದ್ಗುಹ್ಯತರಂ ಪರಂ ॥೧೭॥

ಪ್ರಪಠೇದ್ಯ ಇದಂ ನಿತ್ಯಂ ಕಾಲೀನಾಮಶತಾಷ್ಟಕಂ ।\\
ತ್ರಿಷು ಲೋಕೇಷು ದೇವೇಶಿ ತಸ್ಯಾಸಾಧ್ಯಂ ನ ವಿದ್ಯತೇ ॥೧೮॥

ಪ್ರಾತಃಕಾಲೇ ಚ ಮಧ್ಯಾಹ್ನೇ ಸಾಯಾಹ್ನೇ ಚ ಸದಾ ನಿಶಿ ।\\
ಯಃ ಪಠೇತ್ಪರಯಾ ಭಕ್ತ್ಯಾ ಕಾಲೀನಾಮಶತಾಷ್ಟಕಂ ॥೧೯॥

ಕಾಲಿಕಾ ತಸ್ಯ ಗೇಹೇ ಚ ಸಂಸ್ಥಾನಂ ಕುರುತೇ ಸದಾ ।\\
ಶೂನ್ಯಾಗಾರೇ ಶ್ಮಶಾನೇ ವಾ ಪ್ರಾಂತರೇ ಜಲಮಧ್ಯತಃ ॥೨೦॥

ವಹ್ನಿಮಧ್ಯೇ ಚ ಸಂಗ್ರಾಮೇ ತಥಾ ಪ್ರಾಣಸ್ಯ ಸಂಶಯೇ ।\\
ಶತಾಷ್ಟಕಂ ಜಪನ್ಮಂತ್ರೀ ಲಭತೇ ಕ್ಷೇಮಮುತ್ತಮಂ ॥೨೧॥

ಕಾಲೀಂ ಸಂಸ್ಥಾಪ್ಯ ವಿಧಿವತ್ ಸ್ತುತ್ವಾ ನಾಮಶತಾಷ್ಟಕೈಃ ।\\
ಸಾಧಕಸ್ಸಿದ್ಧಿಮಾಪ್ನೋತಿ ಕಾಲಿಕಾಯಾಃ ಪ್ರಸಾದತಃ ॥೨೨॥

\authorline{ಇತಿ ಶ್ರೀಕಾಲೀಶತನಾಮಸ್ತೋತ್ರಂ ಸಂಪೂರ್ಣಂ ॥}
%===========================================

\section{ಶ್ರೀಕುಂಡಲಿನೀಸಹಸ್ರನಾಮಸ್ತೋತ್ರಂ}
\addcontentsline{toc}{section}{ಶ್ರೀಕುಂಡಲಿನೀಸಹಸ್ರನಾಮಸ್ತೋತ್ರಂ}

ಶ್ರೀಆನಂದಭೈರವೀ ಉವಾಚ ।\\
ಅಥ ಕಾಂತ ಪ್ರವಕ್ಷ್ಯಾಮಿ ಕುಂಡಲೀಚೇತನಾದಿಕಂ ।\\
ಸಹಸ್ರನಾಮಸಕಲಂ ಕುಂಡಲಿನ್ಯಾಃ ಪ್ರಿಯಂ ಸುಖಂ ॥೧॥

ಅಷ್ಟೋತ್ತರಂ ಮಹಾಪುಣ್ಯಂ ಸಾಕ್ಷಾತ್ ಸಿದ್ಧಿಪ್ರದಾಯಕಂ ।\\
ತವ ಪ್ರೇಮವಶೇನೈವ ಕಥಯಾಮಿ ಶೃಣುಷ್ವ ತತ್ ॥೨॥

ವಿನಾ ಯಜನಯೋಗೇನ ವಿನಾ ಧ್ಯಾನೇನ ಯತ್ಫಲಂ ।\\
ತತ್ಫಲಂ ಲಭತೇ ಸದ್ಯೋ ವಿದ್ಯಾಯಾಃ ಸುಕೃಪಾ ಭವೇತ್ ॥೩॥

ಯಾ ವಿದ್ಯಾ ಭುವನೇಶಾನೀ ತ್ರೈಲೋಕ್ಯಪರಿಪೂಜಿತಾ ।\\
ಸಾ ದೇವೀ ಕುಂಡಲೀ ಮಾತಾ ತ್ರೈಲೋಕ್ಯಂ ಪಾತಿ ಸರ್ವದಾ ॥೪॥

ತಸ್ಯಾ ನಾಮ ಸಹಸ್ರಾಣಿ ಅಷ್ಟೋತ್ತರಶತಾನಿ ಚ ।\\
ಶ್ರವಣಾತ್ಪಠನಾನ್ಮಂತ್ರೀ ಮಹಾಭಕ್ತೋ ಭವೇದಿಹ ॥೫॥

ಐಹಿಕೇ ಸ ಭವೇನ್ನಾಥ ಜೀವನ್ಮುಕ್ತೋ ಮಹಾಬಲೀ ॥೬॥

ಅಸ್ಯ ಶ್ರೀಮನ್ಮಹಾಕುಂಡಲೀಸಾಷ್ಟೋತ್ತರಸಹಸ್ರನಾಮಸ್ತೋತ್ರಸ್ಯ ಬ್ರಹ್ಮರ್ಷೀರ್ಜಗತೀಚ್ಛಂದೋ ಭಗವತೀ ಶ್ರೀಮನ್ಮಹಾಕುಂಡಲೀದೇವತಾ ಸರ್ವಯೋಗಸಮೃದ್ಧಿಸಿದ್ಧ್ಯರ್ಥೇ ವಿನಿಯೋಗಃ॥

ಕುಲೇಶ್ವರೀ ಕುಲಾನಂದಾ ಕುಲೀನಾ ಕುಲಕುಂಡಲೀ ।\\
ಶ್ರೀಮನ್ಮಹಾಕುಂಡಲೀ ಚ ಕುಲಕನ್ಯಾ ಕುಲಪ್ರಿಯಾ ॥೭॥

ಕುಲಕ್ಷೇತ್ರಸ್ಥಿತಾ ಕೌಲೀ ಕುಲೀನಾರ್ಥಪ್ರಕಾಶಿನೀ ।\\
ಕುಲಾಖ್ಯಾ ಕುಲಮಾರ್ಗಸ್ಥಾ ಕುಲಶಾಸ್ತ್ರಾರ್ಥಪಾತಿನೀ ॥೮॥

ಕುಲಜ್ಞಾ ಕುಲಯೋಗ್ಯಾ ಚ ಕುಲಪುಷ್ಪಪ್ರಕಾಶಿನೀ ।\\
ಕುಲೀನಾ ಚ ಕುಲಾಧ್ಯಕ್ಷಾ ಕುಲಚಂದನಲೇಪಿತಾ ॥೯॥

ಕುಲರೂಪಾ ಕುಲೋದ್ಭೂತಾ ಕುಲಕುಂಡಲಿವಾಸಿನೀ ।\\
ಕುಲಾಭಿನ್ನಾ ಕುಲೋತ್ಪನ್ನಾ ಕುಲಾಚಾರವಿನೋದಿನೀ ॥೧೦॥

ಕುಲವೃಕ್ಷಸಮುದ್ಭೂತಾ ಕುಲಮಾಲಾ ಕುಲಪ್ರಭಾ ।\\
ಕುಲಜ್ಞಾ ಕುಲಮಧ್ಯಸ್ಥಾ ಕುಲಕಂಕಣಶೋಭಿತಾ ॥೧೧॥

ಕುಲೋತ್ತರಾ ಕೌಲಪೂಜಾ ಕುಲಾಲಾಪಾ ಕುಲಕ್ರಿಯಾ ।\\
ಕುಲಭೇದಾ ಕುಲಪ್ರಾಣಾ ಕುಲದೇವೀ ಕುಲಸ್ತುತಿಃ ॥೧೨॥

ಕೌಲಿಕಾ ಕಾಲಿಕಾ ಕಾಲ್ಯಾ ಕಲಿಭಿನ್ನಾ ಕಲಾಕಲಾ ।\\
ಕಲಿಕಲ್ಮಷಹಂತ್ರೀ ಚ ಕಲಿದೋಷವಿನಾಶಿನೀ ॥೧೩॥

ಕಂಕಾಲೀ ಕೇವಲಾನಂದಾ ಕಾಲಜ್ಞಾ ಕಾಲಧಾರಿಣೀ ।\\
ಕೌತುಕೀ ಕೌಮುದೀ ಕೇಕಾ ಕಾಕಾ ಕಾಕಲಯಾಂತರಾ ॥೧೪॥

ಕೋಮಲಾಂಗೀ ಕರಾಲಾಸ್ಯಾ ಕಂದಪೂಜ್ಯಾ ಚ ಕೋಮಲಾ ।\\
ಕೈಶೋರೀ ಕಾಕಪುಚ್ಛಸ್ಥಾ ಕಂಬಲಾಸನವಾಸಿನೀ ॥೧೫॥

ಕೈಕೇಯೀಪೂಜಿತಾ ಕೋಲಾ ಕೋಲಪುತ್ರೀ ಕಪಿಧ್ವಜಾ ।\\
ಕಮಲಾ ಕಮಲಾಕ್ಷೀ ಚ ಕಂಬಲಾಶ್ವತರಪ್ರಿಯಾ ॥೧೬॥

ಕಲಿಕಾಭಂಗದೋಷಸ್ಥಾ ಕಾಲಜ್ಞಾ ಕಾಲಕುಂಡಲೀ ।\\
ಕಾವ್ಯದಾ ಕವಿತಾ ವಾಣೀ ಕಾಲಸಂದರ್ಭಭೇದಿನೀ ॥೧೭॥

ಕುಮಾರೀ ಕರುಣಾಕಾರಾ ಕುರುಸೈನ್ಯವಿನಾಶಿನೀ ।\\
ಕಾಂತಾ ಕುಲಗತಾ ಕಾಮಾ ಕಾಮಿನೀ ಕಾಮನಾಶಿನೀ ॥೧೮॥

ಕಾಮೋದ್ಭವಾ ಕಾಮಕನ್ಯಾ ಕೇವಲಾ ಕಾಲಘಾತಿನೀ ।\\
ಕೈಲಾಸಶಿಖರಾರೂಢಾ ಕೈಲಾಸಪತಿಸೇವಿತಾ ॥೧೯॥

ಕೈಲಾಸನಾಥನಮಿತಾ ಕೇಯೂರಹಾರಮಂಡಿತಾ ।\\
ಕಂದರ್ಪಾ ಕಠಿನಾನಂದಾ ಕುಲಗಾ ಕೀಚಕೃತ್ಯಹಾ ॥೨೦॥

ಕಮಲಾಸ್ಯಾ ಕಠೋರಾ ಚ ಕೀಟರೂಪಾ ಕಟಿಸ್ಥಿತಾ ।\\
ಕಂದೇಶ್ವರೀ ಕಂದರೂಪಾ ಕೋಲಿಕಾ ಕಂದವಾಸಿನೀ ॥೨೧॥

ಕೂಟಸ್ಥಾ ಕೂಟಭಕ್ಷಾ ಚ ಕಾಲಕೂಟವಿನಾಶಿನೀ ।\\
ಕಾಮಾಖ್ಯಾ ಕಮಲಾ ಕಾಮ್ಯಾ ಕಾಮರಾಜತನೂದ್ಭವಾ ॥೨೨॥

ಕಾಮರೂಪಧರಾ ಕಮ್ರಾ ಕಮನೀಯಾ ಕವಿಪ್ರಿಯಾ ।\\
ಕಂಜಾನನಾ ಕಂಜಹಸ್ತಾ ಕಂಜಪತ್ರಾಯತೇಕ್ಷಣಾ ॥೨೩॥

ಕಾಕಿನೀ ಕಾಮರೂಪಸ್ಥಾ ಕಾಮರೂಪಪ್ರಕಾಶಿನೀ ।\\
ಕೋಲಾವಿಧ್ವಂಸಿನೀ ಕಂಕಾ ಕಲಂಕಾರ್ಕಕಲಂಕಿನೀ ॥೨೪॥

ಮಹಾಕುಲನದೀ ಕರ್ಣಾ ಕರ್ಣಕಾಂಡವಿಮೋಹಿನೀ ।\\
ಕಾಂಡಸ್ಥಾ ಕಾಂಡಕರುಣಾ ಕರ್ಮಕಸ್ಥಾ ಕುಟುಂಬಿನೀ ॥೨೫॥

ಕಮಲಾಭಾ ಭವಾ ಕಲ್ಲಾ ಕರುಣಾ ಕರುಣಾಮಯೀ ।\\
ಕರುಣೇಶೀ ಕರಾಕರ್ತ್ರೀ ಕರ್ತೃಹಸ್ತಾ ಕಲೋದಯಾ ॥೨೬॥

ಕಾರುಣ್ಯಸಾಗರೋದ್ಭೂತಾ ಕಾರುಣ್ಯಸಿಂಧುವಾಸಿನೀ ।\\
ಕಾತ್ತೀಕೇಶೀ ಕಾತ್ತೀಕಸ್ಥಾ ಕಾತ್ತೀಕಪ್ರಾಣಪಾಲನೀ ॥೨೭॥

ಕರುಣಾನಿಧಿಪೂಜ್ಯಾ ಚ ಕರಣೀಯಾ ಕ್ರಿಯಾ ಕಲಾ ।\\
ಕಲ್ಪಸ್ಥಾ ಕಲ್ಪನಿಲಯಾ ಕಲ್ಪಾತೀತಾ ಚ ಕಲ್ಪಿತಾ ॥೨೮॥

ಕುಲಯಾ ಕುಲವಿಜ್ಞಾನಾ ಕರ್ಷೀಣೀ ಕಾಲರಾತ್ರಿಕಾ ।\\
ಕೈವಲ್ಯದಾ ಕೋಕರಸ್ಥಾ ಕಲಮಂಜೀರರಂಜನೀ ॥೨೯॥

ಕಲಯಂತೀ ಕಾಲಜಿಹ್ವಾ ಕಿಂಕರಾಸನಕಾರಿಣೀ ।\\
ಕುಮುದಾ ಕುಶಲಾನಂದಾ ಕೌಶಲ್ಯಾಕಾಶವಾಸಿನೀ ॥೩೦॥

ಕಸಾಪಹಾಸಹಂತ್ರೀ ಚ ಕೈವಲ್ಯಗುಣಸಂಭವಾ ।\\
ಏಕಾಕಿನೀ ಅರ್ಕರೂಪಾ ಕುವಲಾ ಕರ್ಕಟಸ್ಥಿತಾ ॥೩೧॥

ಕರ್ಕೋಟಕಾ ಕೋಷ್ಠರೂಪಾ ಕೂಟವಹ್ನಿಕರಸ್ಥಿತಾ ।\\
ಕೂಜಂತೀ ಮಧುರಧ್ವಾನಂ ಕಾಮಯಂತೀ ಸುಲಕ್ಷಣಾಂ ॥೩೨॥

ಕೇತಕೀ ಕುಸುಮಾನಂದಾ ಕೇತಕೀಪುಣ್ಯಮಂಡಿತಾ ।\\
ಕರ್ಪೂರಪೂರರುಚಿರಾ ಕರ್ಪೂರಭಕ್ಷಣಪ್ರಿಯಾ ॥೩೩॥

ಕಪಾಲಪಾತ್ರಹಸ್ತಾ ಚ ಕಪಾಲಚಂದ್ರಧಾರಿಣೀ ।\\
ಕಾಮಧೇನುಸ್ವರೂಪಾ ಚ ಕಾಮಧೇನುಃ ಕ್ರಿಯಾನ್ವಿತಾ ॥೩೪॥

ಕಶ್ಯಪೀ ಕಾಶ್ಯಪಾ ಕುಂತೀ ಕೇಶಾಂತಾ ಕೇಶಮೋಹಿನೀ ।\\
ಕಾಲಕರ್ತ್ರೀ ಕೂಪಕರ್ತ್ರೀ ಕುಲಪಾ ಕಾಮಚಾರಿಣೀ ॥೩೫॥

ಕುಂಕುಮಾಭಾ ಕಜ್ಜಲಸ್ಥಾ ಕಮಿತಾ ಕೋಪಘಾತಿನೀ ।\\
ಕೇಲಿಸ್ಥಾ ಕೇಲಿಕಲಿತಾ ಕೋಪನಾ ಕರ್ಪಟಸ್ಥಿತಾ ॥೩೬॥

ಕಲಾತೀತಾ ಕಾಲವಿದ್ಯಾ ಕಾಲಾತ್ಮಪುರುಷೋದ್ಭವಾ ।\\
ಕಷ್ಟಸ್ಥಾ ಕಷ್ಟಕುಷ್ಠಸ್ಥಾ ಕುಷ್ಠಹಾ ಕಷ್ಟಹಾ ಕುಶಾ ॥೩೭॥

ಕಾಲಿಕಾ ಸ್ಫುಟಕರ್ತ್ರೀ ಚ ಕಾಂಬೋಜಾ ಕಾಮಲಾ ಕುಲಾ ।\\
ಕುಶಲಾಖ್ಯಾ ಕಾಕಕುಷ್ಠಾ ಕರ್ಮಸ್ಥಾ ಕೂರ್ಮಮಧ್ಯಗಾ ॥೩೮॥

ಕುಂಡಲಾಕಾರಚಕ್ರಸ್ಥಾ ಕುಂಡಗೋಲೋದ್ಭವಾ ಕಫಾ ।\\
ಕಪಿತ್ಥಾಗ್ರವಸಾಕಾಶಾ ಕಪಿತ್ಥರೋಧಕಾರಿಣೀ ॥೩೯॥

ಕಾಹೋಡ ಕಾಹಡ ಕಾಡ ಕಂಕಲಾ ಭಾಷಕಾರಿಣೀ ।\\
ಕನಕಾ ಕನಕಾಭಾ ಚ ಕನಕಾದ್ರಿನಿವಾಸಿನೀ ॥೪೦॥

ಕಾರ್ಪಾಸಯಜ್ಞಸೂತ್ರಸ್ಥಾ ಕೂಟಬ್ರಹ್ಮಾರ್ಥಸಾಧಿನೀ ।\\
ಕಲಂಜಭಕ್ಷಿಣೀ ಕ್ರೂರಾ ಕ್ರೋಧಪುಂಜಾ ಕಪಿಸ್ಥಿತಾ ॥೪೧॥

ಕಪಾಲೀ ಸಾಧನರತಾ ಕನಿಷ್ಠಾಕಾಶವಾಸಿನೀ ।\\
ಕುಂಜರೇಶೀ ಕುಂಜರಸ್ಥಾ ಕುಂಜರಾ ಕುಂಜರಾಗತಿಃ ॥೪೨॥

ಕುಂಜಸ್ಥಾ ಕುಂಜರಮಣೀ ಕುಂಜಮಂದಿರವಾಸಿನೀ ।\\
ಕುಪಿತಾ ಕೋಪಶೂನ್ಯಾ ಚ ಕೋಪಾಕೋಪವಿವರ್ಜೀತಾ ॥೪೩॥

ಕಪಿಂಜಲಸ್ಥಾ ಕಾಪಿಂಜಾ ಕಪಿಂಜಲತರೂದ್ಭವಾ ।\\
ಕುಂತೀಪ್ರೇಮಕಥಾವಿಷ್ಟಾ ಕುಂತೀಮಾನಸಪೂಜಿತಾ ॥೪೪॥

ಕುಂತಲಾ ಕುಂತಹಸ್ತಾ ಚ ಕುಲಕುಂತಲಲೋಹಿನೀ ।\\
ಕಾಂತಾಂಘ್ರಸೇವಿಕಾ ಕಾಂತಕುಶಲಾ ಕೋಶಲಾವತೀ ॥೪೫॥

ಕೇಶಿಹಂತ್ರೀ ಕಕುತ್ಸ್ಥಾ ಚ ಕಕುತ್ಸ್ಥವನವಾಸಿನೀ ।\\
ಕೈಲಾಸಶಿಖರಾನಂದಾ ಕೈಲಾಸಗಿರಿಪೂಜಿತಾ ॥೪೬॥

ಕೀಲಾಲನಿರ್ಮಲಾಕಾರಾ ಕೀಲಾಲಮುಗ್ಧಕಾರಿಣೀ ।\\
ಕುತುನಾ ಕುಟ್ಟಹೀ ಕುಟ್ಠಾ ಕೂಟನಾ ಮೋದಕಾರಿಣೀ ॥೪೭॥

ಕ್ರೌಂಕಾರೀ ಕ್ರೌಂಕರೀ ಕಾಶೀ ಕುಹುಶಬ್ದಸ್ಥಾ ಕಿರಾತಿನೀ ।\\
ಕೂಜಂತೀ ಸರ್ವವಚನಂ ಕಾರಯಂತೀ ಕೃತಾಕೃತಂ ॥೪೮॥

ಕೃಪಾನಿಧಿಸ್ವರೂಪಾ ಚ ಕೃಪಾಸಾಗರವಾಸಿನೀ ।\\
ಕೇವಲಾನಂದನಿರತಾ ಕೇವಲಾನಂದಕಾರಿಣೀ ॥೪೯॥

ಕೃಮಿಲಾ ಕೃಮಿದೋಷಘ್ನೀ ಕೃಪಾ ಕಪಟಕುಟ್ಟಿತಾ ।\\
ಕೃಶಾಂಗೀ ಕ್ರಮಭಂಗಸ್ಥಾ ಕಿಂಕರಸ್ಥಾ ಕಟಸ್ಥಿತಾ ॥೫೦॥

ಕಾಮರೂಪಾ ಕಾಂತರತಾ ಕಾಮರೂಪಸ್ಯ ಸಿದ್ಧಿದಾ ।\\
ಕಾಮರೂಪಪೀಠದೇವೀ ಕಾಮರೂಪಾಂಕುಜಾ ಕುಜಾ ॥೫೧॥

ಕಾಮರೂಪಾ ಕಾಮವಿದ್ಯಾ ಕಾಮರೂಪಾದಿಕಾಲಿಕಾ ।\\
ಕಾಮರೂಪಕಲಾ ಕಾಮ್ಯಾ ಕಾಮರೂಪಕುಲೇಶ್ವರೀ ॥೫೨॥

ಕಾಮರೂಪಜನಾನಂದಾ ಕಾಮರೂಪಕುಶಾಗ್ರಧೀಃ ।\\
ಕಾಮರೂಪಕರಾಕಾಶಾ ಕಾಮರೂಪತರುಸ್ಥಿತಾ ॥೫೩॥

ಕಾಮಾತ್ಮಜಾ ಕಾಮಕಲಾ ಕಾಮರೂಪವಿಹಾರಿಣೀ ।\\
ಕಾಮಶಾಸ್ತ್ರಾರ್ಥಮಧ್ಯಸ್ಥಾ ಕಾಮರೂಪಕ್ರಿಯಾಕಲಾ ॥೫೪॥

ಕಾಮರೂಪಮಹಾಕಾಲೀ ಕಾಮರೂಪಯಶೋಮಯೀ ।\\
ಕಾಮರೂಪಪರಮಾನಂದಾ ಕಾಮರೂಪಾದಿಕಾಮಿನೀ ॥೫೫॥

ಕೂಲಮೂಲಾ ಕಾಮರೂಪಪದ್ಮಮಧ್ಯನಿವಾಸಿನೀ ।\\
ಕೃತಾಂಜಲಿಪ್ರಿಯಾ ಕೃತ್ಯಾ ಕೃತ್ಯಾದೇವೀಸ್ಥಿತಾ ಕಟಾ ॥೫೬॥

ಕಟಕಾ ಕಾಟಕಾ ಕೋಟಿಕಟಿಘಂಟವಿನೋದಿನೀ ।\\
ಕಟಿಸ್ಥೂಲತರಾ ಕಾಷ್ಠಾ ಕಾತ್ಯಾಯನಸುಸಿದ್ಧಿದಾ ॥೫೭॥

ಕಾತ್ಯಾಯನೀ ಕಾಚಲಸ್ಥಾ ಕಾಮಚಂದ್ರಾನನಾ ಕಥಾ ।\\
ಕಾಶ್ಮೀರದೇಶನಿರತಾ ಕಾಶ್ಮೀರೀ ಕೃಷಿಕರ್ಮಜಾ ॥೫೮॥

ಕೃಷಿಕರ್ಮಸ್ಥಿತಾ ಕೌರ್ಮಾ ಕೂರ್ಮಪೃಷ್ಠನಿವಾಸಿನೀ ।\\
ಕಾಲಘಂಟಾ ನಾದರತಾ ಕಲಮಂಜೀರಮೋಹಿನೀ ॥೫೯॥

ಕಲಯಂತೀ ಶತ್ರುವರ್ಗಾನ್ ಕ್ರೋಧಯಂತೀ ಗುಣಾಗುಣಂ ।\\
ಕಾಮಯಂತೀ ಸರ್ವಕಾಮಂ ಕಾಶಯಂತೀ ಜಗತ್ತ್ರಯಂ ॥೬೦॥

ಕೌಲಕನ್ಯಾ ಕಾಲಕನ್ಯಾ ಕೌಲಕಾಲಕುಲೇಶ್ವರೀ ।\\
ಕೌಲಮಂದಿರಸಂಸ್ಥಾ ಚ ಕುಲಧರ್ಮವಿಡಂಬಿನೀ ॥೬೧॥

ಕುಲಧರ್ಮರತಾಕಾರಾ ಕುಲಧರ್ಮವಿನಾಶಿನೀ ।\\
ಕುಲಧರ್ಮಪಂಡಿತಾ ಚ ಕುಲಧರ್ಮಸಮೃದ್ಧಿದಾ ॥೬೨॥

ಕೌಲಭೋಗಮೋಕ್ಷದಾ ಚ ಕೌಲಭೋಗೇಂದ್ರಯೋಗಿನೀ ।\\
ಕೌಲಕರ್ಮಾ ನವಕುಲಾ ಶ್ವೇತಚಂಪಕಮಾಲಿನೀ ॥೬೩॥

ಕುಲಪುಷ್ಪಮಾಲ್ಯಾಕಾಂತಾ ಕುಲಪುಷ್ಪಭವೋದ್ಭವಾ ।\\
ಕೌಲಕೋಲಾಹಲಕರಾ ಕೌಲಕರ್ಮಪ್ರಿಯಾ ಪರಾ ॥೬೪॥

ಕಾಶೀಸ್ಥಿತಾ ಕಾಶಕನ್ಯಾ ಕಾಶೀ ಚಕ್ಷುಃಪ್ರಿಯಾ ಕುಥಾ ।\\
ಕಾಷ್ಠಾಸನಪ್ರಿಯಾ ಕಾಕಾ ಕಾಕಪಕ್ಷಕಪಾಲಿಕಾ ॥೬೫॥

ಕಪಾಲರಸಭೋಜ್ಯಾ ಚ ಕಪಾಲನವಮಾಲಿನೀ ।\\
ಕಪಾಲಸ್ಥಾ ಚ ಕಾಪಾಲೀ ಕಪಾಲಸಿದ್ಧಿದಾಯಿನೀ ॥೬೬॥

ಕಪಾಲಾ ಕುಲಕರ್ತ್ರೀ ಚ ಕಪಾಲಶಿಖರಸ್ಥಿತಾ ।\\
ಕಥನಾ ಕೃಪಣಶ್ರೀದಾ ಕೃಪೀ ಕೃಪಣಸೇವಿತಾ ॥೬೭॥

ಕರ್ಮಹಂತ್ರೀ ಕರ್ಮಗತಾ ಕರ್ಮಾಕರ್ಮವಿವರ್ಜೀತಾ ।\\
ಕರ್ಮಸಿದ್ಧಿರತಾ ಕಾಮೀ ಕರ್ಮಜ್ಞಾನನಿವಾಸಿನೀ ॥೬೮॥

ಕರ್ಮಧರ್ಮಸುಶೀಲಾ ಚ ಕರ್ಮಧರ್ಮವಶಂಕರೀ ।\\
ಕನಕಾಬ್ಜಸುನಿರ್ಮಾಣಮಹಾಸಿಂಹಾಸನಸ್ಥಿತಾ ॥೬೯॥

ಕನಕಗ್ರಂಥಿಮಾಲ್ಯಾಢ್ಯಾ ಕನಕಗ್ರಂಥಿಭೇದಿನೀ ।\\
ಕನಕೋದ್ಭವಕನ್ಯಾ ಚ ಕನಕಾಂಭೋಜವಾಸಿನೀ ॥೭೦॥

ಕಾಲಕೂಟಾದಿಕೂಟಸ್ಥಾ ಕಿಟಿಶಬ್ದಾಂತರಸ್ಥಿತಾ ।\\
ಕಂಕಪಕ್ಷಿನಾದಮುಖಾ ಕಾಮಧೇನೂದ್ಭವಾ ಕಲಾ ॥೭೧॥

ಕಂಕಣಾಭಾ ಧರಾ ಕರ್ದ್ದಾ ಕರ್ದ್ದಮಾ ಕರ್ದ್ದಮಸ್ಥಿತಾ ।\\
ಕರ್ದ್ದಮಸ್ಥಜಲಾಚ್ಛನ್ನಾ ಕರ್ದ್ದಮಸ್ಥಜನಪ್ರಿಯಾ ॥೭೨॥

ಕಮಠಸ್ಥಾ ಕಾರ್ಮುಕಸ್ಥಾ ಕಮ್ರಸ್ಥಾ ಕಂಸನಾಶಿನೀ ।\\
ಕಂಸಪ್ರಿಯಾ ಕಂಸಹಂತ್ರೀ ಕಂಸಾಜ್ಞಾನಕರಾಲಿನೀ ॥೭೩॥

ಕಾಂಚನಾಭಾ ಕಾಂಚನದಾ ಕಾಮದಾ ಕ್ರಮದಾ ಕದಾ ।\\
ಕಾಂತಭಿನ್ನಾ ಕಾಂತಚಿಂತಾ ಕಮಲಾಸನವಾಸಿನೀ ॥೭೪॥

ಕಮಲಾಸನಸಿದ್ಧಿಸ್ಥಾ ಕಮಲಾಸನದೇವತಾ ।\\
ಕುತ್ಸಿತಾ ಕುತ್ಸಿತರತಾ ಕುತ್ಸಾ ಶಾಪವಿವರ್ಜೀತಾ ॥೭೫॥

ಕುಪುತ್ರರಕ್ಷಿಕಾ ಕುಲ್ಲಾ ಕುಪುತ್ರಮಾನಸಾಪಹಾ ।\\
ಕುಜರಕ್ಷಕರೀ ಕೌಜೀ ಕುಬ್ಜಾಖ್ಯಾ ಕುಬ್ಜವಿಗ್ರಹಾ ॥೭೬॥

ಕುನಖೀ ಕೂಪದೀಕ್ಷುಸ್ಥಾ ಕುಕರೀ ಕುಧನೀ ಕುದಾ ।\\
ಕುಪ್ರಿಯಾ ಕೋಕಿಲಾನಂದಾ ಕೋಕಿಲಾ ಕಾಮದಾಯಿನೀ ॥೭೭॥

ಕುಕಾಮಿನಾ ಕುಬುದ್ಧಿಸ್ಥಾ ಕೂರ್ಪವಾಹನ ಮೋಹಿನೀ ।\\
ಕುಲಕಾ ಕುಲಲೋಕಸ್ಥಾ ಕುಶಾಸನಸುಸಿದ್ಧಿದಾ ॥೭೮॥

ಕೌಶಿಕೀ ದೇವತಾ ಕಸ್ಯಾ ಕನ್ನಾದನಾದಸುಪ್ರಿಯಾ ।\\
ಕುಸೌಷ್ಠವಾ ಕುಮಿತ್ರಸ್ಥಾ ಕುಮಿತ್ರಶತ್ರುಘಾತಿನೀ ॥೭೯॥

ಕುಜ್ಞಾನನಿಕರಾ ಕುಸ್ಥಾ ಕುಜಿಸ್ಥಾ ಕರ್ಜದಾಯಿನೀ ।\\
ಕಕರ್ಜಾ ಕರ್ಜ್ಜಕರಿಣೀ ಕರ್ಜವದ್ಧವಿಮೋಹಿನೀ ॥೮೦॥

ಕರ್ಜಶೋಧನಕರ್ತ್ರೀ ಚ ಕಾಲಾಸ್ತ್ರಧಾರಿಣೀ ಸದಾ ।\\
ಕುಗತಿಃ ಕಾಲಸುಗತಿಃ ಕಲಿಬುದ್ಧಿವಿನಾಶಿನೀ ॥೮೧॥

ಕಲಿಕಾಲಫಲೋತ್ಪನ್ನಾ ಕಲಿಪಾವನಕಾರಿಣೀ ।\\
ಕಲಿಪಾಪಹರಾ ಕಾಲೀ ಕಲಿಸಿದ್ಧಿಸುಸೂಕ್ಷ್ಮದಾ ॥೮೨॥

ಕಾಲಿದಾಸವಾಕ್ಯಗತಾ ಕಾಲಿದಾಸಸುಸಿದ್ಧಿದಾ ।\\
ಕಲಿಶಿಕ್ಷಾ ಕಾಲಶಿಕ್ಷಾ ಕಂದಶಿಕ್ಷಾಪರಾಯಣಾ ॥೮೩॥

ಕಮನೀಯಭಾವರತಾ ಕಮನೀಯಸುಭಕ್ತಿದಾ ।\\
ಕರಕಾಜನರೂಪಾ ಚ ಕಕ್ಷಾವಾದಕರಾ ಕರಾ ॥೮೪॥

ಕಂಚುವರ್ಣಾ ಕಾಕವರ್ಣಾ ಕ್ರೋಷ್ಟುರೂಪಾ ಕಷಾಮಲಾ ।\\
ಕ್ರೋಷ್ಟ್ರನಾದರತಾ ಕೀತಾ ಕಾತರಾ ಕಾತರಪ್ರಿಯಾ ॥೮೫॥

ಕಾತರಸ್ಥಾ ಕಾತರಾಜ್ಞಾ ಕಾತರಾನಂದಕಾರಿಣೀ ।\\
ಕಾಶಮರ್ದ್ದತರೂದ್ಭೂತಾ ಕಾಶಮರ್ದ್ದವಿಭಕ್ಷಿಣೀ ॥೮೬॥

ಕಷ್ಟಹಾನಿಃ ಕಷ್ಟದಾತ್ರೀ ಕಷ್ಟಲೋಕವಿರಕ್ತಿದಾ ।\\
ಕಾಯಾಗತಾ ಕಾಯಸಿದ್ಧಿಃ ಕಾಯಾನಂದಪ್ರಕಾಶಿನೀ ॥೮೭॥

ಕಾಯಗಂಧಹರಾ ಕುಂಭಾ ಕಾಯಕುಂಭಾ ಕಠೋರಿಣೀ ।\\
ಕಠೋರತರುಸಂಸ್ಥಾ ಚ ಕಠೋರಲೋಕನಾಶಿನೀ ॥೮೮॥

ಕುಮಾರ್ಗಸ್ಥಾಪಿತಾ ಕುಪ್ರಾ ಕಾರ್ಪಾಸತರುಸಂಭವಾ ।\\
ಕಾರ್ಪಾಸವೃಕ್ಷಸೂತ್ರಸ್ಥಾ ಕುವರ್ಗಸ್ಥಾ ಕರೋತ್ತರಾ ॥೮೯॥

ಕರ್ಣಾಟಕರ್ಣಸಂಭೂತಾ ಕಾರ್ಣಾಟೀ ಕರ್ಣಪೂಜಿತಾ ।\\
ಕರ್ಣಾಸ್ತ್ರರಕ್ಷಿಕಾ ಕರ್ಣಾ ಕರ್ಣಹಾ ಕರ್ಣಕುಂಡಲಾ ॥೯೦॥

ಕುಂತಲಾದೇಶನಮಿತಾ ಕುಟುಂಬಾ ಕುಂಭಕಾರಿಕಾ ।\\
ಕರ್ಣಾಸರಾಸನಾ ಕೃಷ್ಟಾ ಕೃಷ್ಣಹಸ್ತಾಂಬುಜಾಜೀತಾ ॥೯೧॥

ಕೃಷ್ಣಾಂಗೀ ಕೃಷ್ಣದೇಹಸ್ಥಾ ಕುದೇಶಸ್ಥಾ ಕುಮಂಗಲಾ ।\\
ಕ್ರೂರಕರ್ಮಸ್ಥಿತಾ ಕೋರಾ ಕಿರಾತ ಕುಲಕಾಮಿನೀ ॥೯೨॥

ಕಾಲವಾರಿಪ್ರಿಯಾ ಕಾಮಾ ಕಾವ್ಯವಾಕ್ಯಪ್ರಿಯಾ ಕ್ರುಧಾ ।\\
ಕಂಜಲತಾ ಕೌಮುದೀ ಚ ಕುಜ್ಯೋತ್ಸ್ನಾ ಕಲನಪ್ರಿಯಾ ॥೯೩॥

ಕಲನಾ ಸರ್ವಭೂತಾನಾಂ ಕಪಿತ್ಥವನವಾಸಿನೀ ।\\
ಕಟುನಿಂಬಸ್ಥಿತಾ ಕಾಖ್ಯಾ ಕವರ್ಗಾಖ್ಯಾ ಕವರ್ಗೀಕಾ ॥೯೪॥

ಕಿರಾತಚ್ಛೇದಿನೀ ಕಾರ್ಯಾ ಕಾರ್ಯಾಕಾರ್ಯವಿವರ್ಜೀತಾ ।\\
ಕಾತ್ಯಾಯನಾದಿಕಲ್ಪಸ್ಥಾ ಕಾತ್ಯಾಯನಸುಖೋದಯಾ ॥೯೫॥

ಕುಕ್ಷೇತ್ರಸ್ಥಾ ಕುಲಾವಿಘ್ನಾ ಕರಣಾದಿಪ್ರವೇಶಿನೀ ।\\
ಕಾಂಕಾಲೀ ಕಿಂಕಲಾ ಕಾಲಾ ಕಿಲಿತಾ ಸರ್ವಕಾಮಿನೀ ॥೯೬॥

ಕೀಲಿತಾಪೇಕ್ಷಿತಾ ಕೂಟಾ ಕೂಟಕುಂಕುಮಚರ್ಚೀತಾ ।\\
ಕುಂಕುಮಾಗಂಧನಿಲಯಾ ಕುಟುಂಬಭವನಸ್ಥಿತಾ ॥೯೭॥

ಕುಕೃಪಾ ಕರಣಾನಂದಾ ಕವಿತಾರಸಮೋಹಿನೀ ।\\
ಕಾವ್ಯಶಾಸ್ತ್ರಾನಂದರತಾ ಕಾವ್ಯಪೂಜ್ಯಾ ಕವೀಶ್ವರೀ ॥೯೮॥

ಕಟಕಾದಿಹಸ್ತಿರಥಹಯದುಂದುಭಿಶಬ್ದಿನೀ ।\\
ಕಿತವಾ ಕ್ರೂರಧೂರ್ತಸ್ಥಾ ಕೇಕಾಶಬ್ದನಿವಾಸಿನೀ ॥೯೯॥

ಕೇಂ ಕೇವಲಾಂಬಿತಾ ಕೇತಾ ಕೇತಕೀಪುಷ್ಪಮೋಹಿನೀ ।\\
ಕೈಂ ಕೈವಲ್ಯಗುಣೋದ್ವಾಸ್ಯಾ ಕೈವಲ್ಯಧನದಾಯಿನೀ ॥೧೦೦॥

ಕರೀ ಧನೀಂದ್ರಜನನೀ ಕಾಕ್ಷತಾಕ್ಷಕಲಂಕಿನೀ ।\\
ಕುಡುವಾಂತಾ ಕಾಂತಿಶಾಂತಾ ಕಾಂಕ್ಷಾ ಪಾರಮಹಂಸ್ಯಗಾ ॥೧೦೧॥

ಕರ್ತ್ರೀ ಚಿತ್ತಾ ಕಾಂತವಿತ್ತಾ ಕೃಷಣಾ ಕೃಷಿಭೋಜಿನೀ ।\\
ಕುಂಕುಮಾಶಕ್ತಹೃದಯಾ ಕೇಯೂರಹಾರಮಾಲಿನೀ ॥೧೦೨॥

ಕೀಶ್ವರೀ ಕೇಶವಾ ಕುಂಭಾ ಕೈಶೋರಜನಪೂಜಿತಾ ।\\
ಕಲಿಕಾಮಧ್ಯನಿರತಾ ಕೋಕಿಲಸ್ವರಗಾಮಿನೀ ॥೧೦೩॥

ಕುರದೇಹಹರಾ ಕುಂಬಾ ಕುಡುಂಬಾ ಕುರಭೇದಿನೀ ।\\
ಕುಂಡಲೀಶ್ವರಸಂವಾದಾ ಕುಂಡಲೀಶ್ವರಮಧ್ಯಗಾ ॥೧೦೪॥

ಕಾಲಸೂಕ್ಷ್ಮಾ ಕಾಲಯಜ್ಞಾ ಕಾಲಹಾರಕರೀ ಕಹಾ ।\\
ಕಹಲಸ್ಥಾ ಕಲಹಸ್ಥಾ ಕಲಹಾ ಕಲಹಾಂಕರೀ ॥೧೦೫॥

ಕುರಂಗೀ ಶ್ರೀಕುರಂಗಸ್ಥಾ ಕೋರಂಗೀ ಕುಂಡಲಾಪಹಾ ।\\
ಕುಕಲಂಕೀ ಕೃಷ್ಣಬುದ್ಧಿಃ ಕೃಷ್ಣಾ ಧ್ಯಾನನಿವಾಸಿನೀ ॥೧೦೬॥

ಕುತವಾ ಕಾಷ್ಠವಲತಾ ಕೃತಾರ್ಥಕರಣೀ ಕುಸೀ ।\\
ಕಲನಕಸ್ಥಾ ಕಸ್ವರಸ್ಥಾ ಕಲಿಕಾ ದೋಷಭಂಗಜಾ ॥೧೦೭॥

ಕುಸುಮಾಕಾರಕಮಲಾ ಕುಸುಮಸ್ರಗ್ವಿಭೂಷಣಾ ।\\
ಕಿಂಜಲ್ಕಾ ಕೈತವಾರ್ಕಶಾ ಕಮನೀಯಜಲೋದಯಾ ॥೧೦೮॥

ಕಕಾರಕೂಟಸರ್ವಾಂಗೀ ಕಕಾರಾಂಬರಮಾಲಿನೀ ।\\
ಕಾಲಭೇದಕರಾ ಕಾಟಾ ಕರ್ಪವಾಸಾ ಕಕುತ್ಸ್ಥಲಾ ॥೧೦೯॥

ಕುವಾಸಾ ಕಬರೀ ಕರ್ವಾ ಕೂಸವೀ ಕುರುಪಾಲನೀ ।\\
ಕುರುಪೃಷ್ಠಾ ಕುರುಶ್ರೇಷ್ಠಾ ಕುರೂಣಾಂ ಜ್ಞಾನನಾಶಿನೀ ॥೧೧೦॥

ಕುತೂಹಲರತಾ ಕಾಂತಾ ಕುವ್ಯಾಪ್ತಾ ಕಷ್ಟಬಂಧನಾ ।\\
ಕಷಾಯಣತರುಸ್ಥಾ ಚ ಕಷಾಯಣರಸೋದ್ಭವಾ ॥೧೧೧॥

ಕತಿವಿದ್ಯಾ ಕುಷ್ಠದಾತ್ರೀ ಕುಷ್ಠಿಶೋಕವಿಸರ್ಜನೀ ।\\
ಕಾಷ್ಠಾಸನಗತಾ ಕಾರ್ಯಾಶ್ರಯಾ ಕಾ ಶ್ರಯಕೌಲಿಕಾ ॥೧೧೨॥

ಕಾಲಿಕಾ ಕಾಲಿಸಂತ್ರಸ್ತಾ ಕೌಲಿಕಧ್ಯಾನವಾಸಿನೀ ।\\
ಕ್ಲೃಪ್ತಸ್ಥಾ ಕ್ಲೃಪ್ತಜನನೀ ಕ್ಲೃಪ್ತಚ್ಛನ್ನಾ ಕಲಿಧ್ವಜಾ ॥೧೧೩॥

ಕೇಶವಾ ಕೇಶವಾನಂದಾ ಕೇಶ್ಯಾದಿದಾನವಾಪಹಾ ।\\
ಕೇಶವಾಂಗಜಕನ್ಯಾ ಚ ಕೇಶವಾಂಗಜಮೋಹಿನೀ ॥೧೧೪॥

ಕಿಶೋರಾರ್ಚನಯೋಗ್ಯಾ ಚ ಕಿಶೋರದೇವದೇವತಾ ।\\
ಕಾಂತಶ್ರೀಕರಣೀ ಕುತ್ಯಾ ಕಪಟಾ ಪ್ರಿಯಘಾತಿನೀ ॥೧೧೫॥

ಕುಕಾಮಜನಿತಾ ಕೌಂಚಾ ಕೌಂಚಸ್ಥಾ ಕೌಂಚವಾಸಿನೀ ।\\
ಕೂಪಸ್ಥಾ ಕೂಪಬುದ್ಧಿಸ್ಥಾ ಕೂಪಮಾಲಾ ಮನೋರಮಾ ॥೧೧೬॥

ಕೂಪಪುಷ್ಪಾಶ್ರಯಾ ಕಾಂತಿಃ ಕ್ರಮದಾಕ್ರಮದಾಕ್ರಮಾ ।\\
ಕುವಿಕ್ರಮಾ ಕುಕ್ರಮಸ್ಥಾ ಕುಂಡಲೀಕುಂಡದೇವತಾ ॥೧೧೭॥

ಕೌಂಡಿಲ್ಯನಗರೋದ್ಭೂತಾ ಕೌಂಡಿಲ್ಯಗೋತ್ರಪೂಜಿತಾ ।\\
ಕಪಿರಾಜಸ್ಥಿತಾ ಕಾಪೀ ಕಪಿಬುದ್ಧಿಬಲೋದಯಾ ॥೧೧೮॥

ಕಪಿಧ್ಯಾನಪರಾ ಮುಖ್ಯಾ ಕುವ್ಯವಸ್ಥಾ ಕುಸಾಕ್ಷಿದಾ ।\\
ಕುಮಧ್ಯಸ್ಥಾ ಕುಕಲ್ಪಾ ಚ ಕುಲಪಂಕ್ತಿಪ್ರಕಾಶಿನೀ ॥೧೧೯॥

ಕುಲಭ್ರಮರದೇಹಸ್ಥಾ ಕುಲಭ್ರಮರನಾದಿನೀ ।\\
ಕುಲಾಸಂಗಾ ಕುಲಾಕ್ಷೀ ಚ ಕುಲಮತ್ತಾ ಕುಲಾನಿಲಾ ॥೧೨೦॥

ಕಲಿಚಿನ್ಹಾ ಕಾಲಚಿನ್ಹಾ ಕಂಠಚಿನ್ಹಾ ಕವೀಂದ್ರಜಾ ।\\
ಕರೀಂದ್ರಾ ಕಮಲೇಶಶ್ರೀಃ ಕೋಟಿಕಂದರ್ಪದರ್ಪಹಾ ॥೧೨೧॥

ಕೋಟಿತೇಜೋಮಯೀ ಕೋಟ್ಯಾ ಕೋಟೀರಪದ್ಮಮಾಲಿನೀ ।\\
ಕೋಟೀರಮೋಹಿನೀ ಕೋಟಿಃ ಕೋಟಿಕೋಟಿವಿಧೂದ್ಭವಾ ॥೧೨೨॥

ಕೋಟಿಸೂರ್ಯಸಮಾನಾಸ್ಯಾ ಕೋಟಿಕಾಲಾನಲೋಪಮಾ ।\\
ಕೋಟೀರಹಾರಲಲಿತಾ ಕೋಟಿಪರ್ವತಧಾರಿಣೀ ॥೧೨೩॥

ಕುಚಯುಗ್ಮಧರಾ ದೇವೀ ಕುಚಕಾಮಪ್ರಕಾಶಿನೀ ।\\
ಕುಚಾನಂದಾ ಕುಚಾಚ್ಛನ್ನಾ ಕುಚಕಾಠಿನ್ಯಕಾರಿಣೀ ॥೧೨೪॥

ಕುಚಯುಗ್ಮಮೋಹನಸ್ಥಾ ಕುಚಮಾಯಾತುರಾ ಕುಚಾ ।\\
ಕುಚಯೌವನಸಮ್ಮೋಹಾ ಕುಚಮರ್ದ್ದನಸೌಖ್ಯದಾ ॥೧೨೫॥

ಕಾಚಸ್ಥಾ ಕಾಚದೇಹಾ ಚ ಕಾಚಪೂರನಿವಾಸಿನೀ ।\\
ಕಾಚಗ್ರಸ್ಥಾ ಕಾಚವರ್ಣಾ ಕೀಚಕಪ್ರಾಣನಾಶಿನೀ ॥೧೨೬॥

ಕಮಲಾ ಲೋಚನಪ್ರೇಮಾ ಕೋಮಲಾಕ್ಷೀ ಮನುಪ್ರಿಯಾ ।\\
ಕಮಲಾಕ್ಷೀ ಕಮಲಜಾ ಕಮಲಾಸ್ಯಾ ಕರಾಲಜಾ ॥೧೨೭॥

ಕಮಲಾಂಘಿರದ್ವಯಾ ಕಾಮ್ಯಾ ಕರಾಖ್ಯಾ ಕರಮಾಲಿನೀ ।\\
ಕರಪದ್ಮಧರಾ ಕಂದಾ ಕಂದಬುದ್ಧಿಪ್ರದಾಯಿನೀ ॥೧೨೮॥

ಕಮಲೋದ್ಭವಪುತ್ರೀ ಚ ಕಮಲಾ ಪುತ್ರಕಾಮಿನೀ ।\\
ಕಿರಂತೀ ಕಿರಣಾಚ್ಛನ್ನಾ ಕಿರಣಪ್ರಾಣವಾಸಿನೀ ॥೧೨೯॥

ಕಾವ್ಯಪ್ರದಾ ಕಾವ್ಯಚಿತ್ತಾ ಕಾವ್ಯಸಾರಪ್ರಕಾಶಿನೀ ।\\
ಕಲಾಂಬಾ ಕಲ್ಪಜನನೀ ಕಲ್ಪಭೇದಾಸನಸ್ಥಿತಾ ॥೧೩೦॥

ಕಾಲೇಚ್ಛಾ ಕಾಲಸಾರಸ್ಥಾ ಕಾಲಮಾರಣಘಾತಿನೀ ।\\
ಕಿರಣಕ್ರಮದೀಪಸ್ಥಾ ಕರ್ಮಸ್ಥಾ ಕ್ರಮದೀಪಿಕಾ ॥೧೩೧॥

ಕಾಲಲಕ್ಷ್ಮೀಃ ಕಾಲಚಂಡಾ ಕುಲಚಂಡೇಶ್ವರಪ್ರಿಯಾ ।\\
ಕಾಕಿನೀಶಕ್ತಿದೇಹಸ್ಥಾ ಕಿತವಾ ಕಿಂತಕಾರಿಣೀ ॥೧೩೨॥

ಕರಂಚಾ ಕಂಚುಕಾ ಕ್ರೌಂಚಾ ಕಾಕಚಂಚುಪುಟಸ್ಥಿತಾ ।\\
ಕಾಕಾಖ್ಯಾ ಕಾಕಶಬ್ದಸ್ಥಾ ಕಾಲಾಗ್ನಿದಹನಾಥೀಕಾ ॥೧೩೩॥

ಕುಚಕ್ಷನಿಲಯಾ ಕುತ್ರಾ ಕುಪುತ್ರಾ ಕ್ರತುರಕ್ಷಿಕಾ ।\\
ಕನಕಪ್ರತಿಭಾಕಾರಾ ಕರಬಂಧಾಕೃತಿಸ್ಥಿತಾ ॥೧೩೪॥

ಕೃತಿರೂಪಾ ಕೃತಿಪ್ರಾಣಾ ಕೃತಿಕ್ರೋಧನಿವಾರಿಣೀ ।\\
ಕುಕ್ಷಿರಕ್ಷಾಕರಾ ಕುಕ್ಷಾ ಕುಕ್ಷಿಬ್ರಹ್ಮಾಂಡಧಾರಿಣೀ ॥೧೩೫॥

ಕುಕ್ಷಿದೇವಸ್ಥಿತಾ ಕುಕ್ಷಿಃ ಕ್ರಿಯಾದಕ್ಷಾ ಕ್ರಿಯಾತುರಾ ।\\
ಕ್ರಿಯಾನಿಷ್ಠಾ ಕ್ರಿಯಾನಂದಾ ಕ್ರತುಕರ್ಮಾ ಕ್ರಿಯಾಪ್ರಿಯಾ ॥೧೩೬॥

ಕುಶಲಾಸವಸಂಶಕ್ತಾ ಕುಶಾರಿಪ್ರಾಣವಲ್ಲಭಾ ।\\
ಕುಶಾರಿವೃಕ್ಷಮದಿರಾ ಕಾಶೀರಾಜವಶೋದ್ಯಮಾ ॥೧೩೭॥

ಕಾಶೀರಾಜಗೃಹಸ್ಥಾ ಚ ಕರ್ಣಭ್ರಾತೃಗೃಹಸ್ಥಿತಾ ।\\
ಕರ್ಣಾಭರಣಭೂಷಾಢ್ಯಾ ಕಂಠಭೂಷಾ ಚ ಕಂಠಿಕಾ ॥೧೩೮॥

ಕಂಠಸ್ಥಾನಗತಾ ಕಂಠಾ ಕಂಠಪದ್ಮನಿವಾಸಿನೀ ।\\
ಕಂಠಪ್ರಕಾಶಕಾರಿಣೀ ಕಂಠಮಾಣಿಕ್ಯಮಾಲಿನೀ ॥೧೩೯॥

ಕಂಠಪದ್ಮಸಿದ್ಧಿಕರೀ ಕಂಠಾಕಾಶನಿವಾಸಿನೀ ।\\
ಕಂಠಪದ್ಮಸಾಕಿನೀಸ್ಥಾ ಕಂಠಷೋಡಶಪತ್ರಿಕಾ ॥೧೪೦॥

ಕೃಷ್ಣಾಜಿನಧರಾ ವಿದ್ಯಾ ಕೃಷ್ಣಾಜಿನಸುವಾಸಸೀ ।\\
ಕುತಕಸ್ಥಾ ಕುಖೇಲಸ್ಥಾ ಕುಂಡವಾಲಂಕೃತಾಕೃತಾ ॥೧೪೧॥

ಕಲಗೀತಾ ಕಾಲಘಜಾ ಕಲಭಂಗಪರಾಯಣಾ ।\\
ಕಾಲೀಚಂದ್ರಾ ಕಲಾ ಕಾವ್ಯಾ ಕುಚಸ್ಥಾ ಕುಚಲಪ್ರದಾ ॥೧೪೨॥

ಕುಚೌರಘಾತಿನೀ ಕಚ್ಛಾ ಕಚ್ಛಾದಸ್ಥಾ ಕಜಾತನಾ ।\\
ಕಂಜಾಛದಮುಖೀ ಕಂಜಾ ಕಂಜತುಂಡಾ ಕಜೀವಲೀ ॥೧೪೩॥

ಕಾಮರಾಭಾರ್ಸುರವಾದ್ಯಸ್ಥಾ ಕಿಯಧಂಕಾರನಾದಿನೀ ।\\
ಕಣಾದಯಜ್ಞಸೂತ್ರಸ್ಥಾ ಕೀಲಾಲಯಜ್ಞಸಂಜ್ಞಕಾ ॥೧೪೪॥

ಕಟುಹಾಸಾ ಕಪಾಟಸ್ಥಾ ಕಟಧೂಮನಿವಾಸಿನೀ ।\\
ಕಟಿನಾದಘೋರತರಾ ಕುಟ್ಟಲಾ ಪಾಟಲಿಪ್ರಿಯಾ ॥೧೪೫॥

ಕಾಮಚಾರಾಬ್ಜನೇತ್ರಾ ಚ ಕಾಮಚೋದ್ಗಾರಸಂಕ್ರಮಾ ।\\
ಕಾಷ್ಠಪರ್ವತಸಂದಾಹಾ ಕುಷ್ಠಾಕುಷ್ಠ ನಿವಾರಿಣೀ ॥೧೪೬॥

ಕಹೋಡಮಂತ್ರಸಿದ್ಧಸ್ಥಾ ಕಾಹಲಾ ಡಿಂಡಿಮಪ್ರಿಯಾ ।\\
ಕುಲಡಿಂಡಿಮವಾದ್ಯಸ್ಥಾ ಕಾಮಡಾಮರಸಿದ್ಧಿದಾ ॥೧೪೭॥

ಕುಲಾಮರವಧ್ಯಸ್ಥಾ ಕುಲಕೇಕಾನಿನಾದಿನೀ ।\\
ಕೋಜಾಗರಢೋಲನಾದಾ ಕಾಸ್ಯವೀರರಣಸ್ಥಿತಾ ॥೧೪೮॥

ಕಾಲಾದಿಕರಣಚ್ಛಿದ್ರಾ ಕರುಣಾನಿಧಿವತ್ಸಲಾ ।\\
ಕ್ರತುಶ್ರೀದಾ ಕೃತಾರ್ಥಶ್ರೀಃ ಕಾಲತಾರಾ ಕುಲೋತ್ತರಾ ॥೧೪೯॥

ಕಥಾಪೂಜ್ಯಾ ಕಥಾನಂದಾ ಕಥನಾ ಕಥನಪ್ರಿಯಾ ।\\
ಕಾರ್ಥಚಿಂತಾ ಕಾರ್ಥವಿದ್ಯಾ ಕಾಮಮಿಥ್ಯಾಪವಾದಿನೀ ॥೧೫೦॥

ಕದಂಬಪುಷ್ಪಸಂಕಾಶಾ ಕದಂಬಪುಷ್ಪಮಾಲಿನೀ ।\\
ಕಾದಂಬರೀ ಪಾನತುಷ್ಟಾ ಕಾಯದಂಭಾ ಕದೋದ್ಯಮಾ ॥೧೫೧॥

ಕುಂಕುಲೇಪತ್ರಮಧ್ಯಸ್ಥಾ ಕುಲಾಧಾರಾ ಧರಪ್ರಿಯಾ ।\\
ಕುಲದೇವಶರೀರಾರ್ಧಾ ಕುಲಧಾಮಾ ಕಲಾಧರಾ ॥೧೫೨॥

ಕಾಮರಾಗಾ ಭೂಷಣಾಢ್ಯಾ ಕಾಮಿನೀರಗುಣಪ್ರಿಯಾ ।\\
ಕುಲೀನಾ ನಾಗಹಸ್ತಾ ಚ ಕುಲೀನನಾಗವಾಹಿನೀ ॥೧೫೩॥

ಕಾಮಪೂರಸ್ಥಿತಾ ಕೋಪಾ ಕಪಾಲೀ ಬಕುಲೋದ್ಭವಾ ।\\
ಕಾರಾಗಾರಜನಾಪಾಲ್ಯಾ ಕಾರಾಗಾರಪ್ರಪಾಲಿನೀ ॥೧೫೪॥

ಕ್ರಿಯಾಶಕ್ತಿಃ ಕಾಲಪಂಕ್ತಿಃ ಕಾರ್ಣಪಂಕ್ತಿಃ ಕಫೋದಯಾ ।\\
ಕಾಮಫುಲ್ಲಾರವಿಂದಸ್ಥಾ ಕಾಮರೂಪಫಲಾಫಲಾ ॥೧೫೫॥

ಕಾಯಫಲಾ ಕಾಯಫೇಣಾ ಕಾಂತಾ ನಾಡೀಫಲೀಶ್ವರಾ ।\\
ಕಮಫೇರುಗತಾ ಗೌರೀ ಕಾಯವಾಣೀ ಕುವೀರಗಾ ॥೧೫೬॥

ಕಬರೀಮಣಿಬಂಧಸ್ಥಾ ಕಾವೇರೀತೀರ್ಥಸಂಗಮಾ ।\\
ಕಾಮಭೀತಿಹರಾ ಕಾಂತಾ ಕಾಮವಾಕುಭ್ರಮಾತುರಾ ॥೧೫೭॥

ಕವಿಭಾವಹರಾ ಭಾಮಾ ಕಮನೀಯಭಯಾಪಹಾ ।\\
ಕಾಮಗರ್ಭದೇವಮಾತಾ ಕಾಮಕಲ್ಪಲತಾಮರಾ ॥೧೫೮॥

ಕಮಠಪ್ರಿಯಮಾಂಸಾದಾ ಕಮಠಾ ಮರ್ಕಟಪ್ರಿಯಾ ।\\
ಕಿಮಾಕಾರಾ ಕಿಮಾಧಾರಾ ಕುಂಭಕಾರಮನಸ್ಥಿತಾ ॥೧೫೯॥

ಕಾಮ್ಯಯಜ್ಞಸ್ಥಿತಾ ಚಂಡಾ ಕಾಮ್ಯಯಜ್ಞೋಪವೀತಿಕಾ ।\\
ಕಾಮಯಾಗಸಿದ್ಧಿಕರೀ ಕಾಮಮೈಥುನಯಾಮಿನೀ ॥೧೬೦॥

ಕಾಮಾಖ್ಯಾ ಯಮಲಾಶಸ್ಥಾ ಕಾಲಯಾಮಾ ಕುಯೋಗಿನೀ ।\\
ಕುರುಯಾಗಹತಾಯೋಗ್ಯಾ ಕುರುಮಾಂಸವಿಭಕ್ಷಿಣೀ ॥೧೬೧॥

ಕುರುರಕ್ತಪ್ರಿಯಾಕಾರೀ ಕಿಂಕರಪ್ರಿಯಕಾರಿಣೀ ।\\
ಕರ್ತ್ರೀಶ್ವರೀ ಕಾರಣಾತ್ಮಾ ಕವಿಭಕ್ಷಾ ಕವಿಪ್ರಿಯಾ ॥೧೬೨॥

ಕವಿಶತ್ರುಪ್ರಷ್ಠಲಗ್ನಾ ಕೈಲಾಸೋಪವನಸ್ಥಿತಾ ।\\
ಕಲಿತ್ರಿಧಾ ತ್ರಿಸಿದ್ಧಿಸ್ಥಾ ಕಲಿತ್ರಿದಿನಸಿದ್ಧಿದಾ ॥೧೬೩॥

ಕಲಂಕರಹಿತಾ ಕಾಲೀ ಕಲಿಕಲ್ಮಷಕಾಮದಾ ।\\
ಕುಲಪುಷ್ಪರಂಗ ಸೂತ್ರಮಣಿಗ್ರಂಥಿಸುಶೋಭನಾ ॥೧೬೪॥

ಕಂಬೋಜವಂಗದೇಶಸ್ಥಾ ಕುಲವಾಸುಕಿರಕ್ಷಿಕಾ ।\\
ಕುಲಶಾಸ್ತ್ರಕ್ರಿಯಾ ಶಾಂತಿಃ ಕುಲಶಾಂತಿಃ ಕುಲೇಶ್ವರೀ ॥೧೬೫॥

ಕುಶಲಪ್ರತಿಭಾ ಕಾಶೀ ಕುಲಷಟ್ಚಕ್ರಭೇದಿನೀ ।\\
ಕುಲಷಟ್ಪದ್ಮಮಧ್ಯಸ್ಥಾ ಕುಲಷಟ್ಪದ್ಮದೀಪಿನೀ ॥೧೬೬॥

ಕೃಷ್ಣಮಾರ್ಜಾರಕೋಲಸ್ಥಾ ಕೃಷ್ಣಮಾರ್ಜಾರಷಷ್ಠಿಕಾ ।\\
ಕುಲಮಾರ್ಜಾರಕುಪಿತಾ ಕುಲಮಾರ್ಜಾರಷೋಡಶೀ ॥೧೬೭॥

ಕಾಲಾಂತಕವಲೋತ್ಪನ್ನಾ ಕಪಿಲಾಂತಕಘಾತಿನೀ ।\\
ಕಲಹಾಸಾ ಕಾಲಹಶ್ರೀ ಕಲಹಾರ್ಥಾ ಕಲಾಮಲಾ ॥೧೬೮॥

ಕಕ್ಷಪಪಕ್ಷರಕ್ಷಾ ಚ ಕುಕ್ಷೇತ್ರಪಕ್ಷಸಂಕ್ಷಯಾ ।\\
ಕಾಕ್ಷರಕ್ಷಾಸಾಕ್ಷಿಣೀ ಚ ಮಹಾಮೋಕ್ಷಪ್ರತಿಷ್ಠಿತಾ ॥೧೬೯॥

ಅರ್ಕಕೋಟಿಶತಚ್ಛಾಯಾ ಆನ್ವೀಕ್ಷಿಕಿಂಕರಾಚೀತಾ ।\\
ಕಾವೇರೀತೀರಭೂಮಿಸ್ಥಾ ಆಗ್ನೇಯಾರ್ಕಾಸ್ತ್ರಧಾರಿಣೀ ॥೧೭೦॥

ಇಂ ಕಿಂ ಶ್ರೀಂ ಕಾಮಕಮಲಾ ಪಾತು ಕೈಲಾಸರಕ್ಷಿಣೀ ।\\
ಮಮ ಶ್ರೀಂ ಈಂ ಬೀಜರೂಪಾ ಪಾತು ಕಾಲೀ ಶಿರಸ್ಥಲಂ ॥೧೭೧॥

ಉರುಸ್ಥಲಾಬ್ಜಂ ಸಕಲಂ ತಮೋಲ್ಕಾ ಪಾತು ಕಾಲಿಕಾ ।\\
ಊಡುಂಬನ್ಯರ್ಕರಮಣೀ ಉಷ್ಟ್ರೇಗ್ರಾ ಕುಲಮಾತೃಕಾ ॥೧೭೨॥

ಕೃತಾಪೇಕ್ಷಾ ಕೃತಿಮತೀ ಕುಂಕಾರೀ ಕಿಂಲಿಪಿಸ್ಥಿತಾ ।\\
ಕುಂದೀರ್ಘಸ್ವರಾ ಕ್ಲೃಪ್ತಾ ಚ ಕೇಂ ಕೈಲಾಸಕರಾಚೀಕಾ ॥೧೭೩॥

ಕೈಶೋರೀ ಕೈಂ ಕರೀ ಕೈಂ ಕೇಂ ಬೀಜಾಖ್ಯಾ ನೇತ್ರಯುಗ್ಮಕಂ ।\\
ಕೋಮಾ ಮತಂಗಯಜಿತಾ ಕೌಶಲ್ಯಾದಿ ಕುಮಾರಿಕಾ ॥೧೭೪॥

ಪಾತು ಮೇ ಕರ್ಣಯುಗ್ಮಂತು ಕ್ರೌಂ ಕ್ರೌಂ ಜೀವಕರಾಲಿನೀ ।\\
ಗಂಡಯುಗ್ಮಂ ಸದಾ ಪಾತು ಕುಂಡಲೀ ಅಂಕವಾಸಿನೀ ॥೧೭೫॥

ಅರ್ಕಕೋಟಿಶತಾಭಾಸಾ ಅಕ್ಷರಾಕ್ಷರಮಾಲಿನೀ ।\\
ಆಶುತೋಷಕರೀ ಹಸ್ತಾ ಕುಲದೇವೀ ನಿರಂಜನಾ ॥೧೭೬॥

ಪಾತು ಮೇ ಕುಲಪುಷ್ಪಾಢ್ಯಾ ಪೃಷ್ಠದೇಶಂ ಸುಕೃತ್ತಮಾ ।\\
ಕುಮಾರೀ ಕಾಮನಾಪೂರ್ಣಾ ಪಾರ್ಶ್ವದೇಶಂ ಸದಾವತು ॥೧೭೭॥

ದೇವೀ ಕಾಮಾಖ್ಯಕಾ ದೇವೀ ಪಾತು ಪ್ರತ್ಯಂಗಿರಾ ಕಟಿಂ ।\\
ಕಟಿಸ್ಥದೇವತಾ ಪಾತು ಲಿಂಗಮೂಲಂ ಸದಾ ಮಮ ॥೧೭೮॥

ಗುಹ್ಯದೇಶಂ ಕಾಕಿನೀ ಮೇ ಲಿಂಗಾಧಃ ಕುಲಸಿಂಹಿಕಾ ।\\
ಕುಲನಾಗೇಶ್ವರೀ ಪಾತು ನಿತಂಬದೇಶಮುತ್ತಮಂ ॥೧೭೯॥

ಕಂಕಾಲಮಾಲಿನೀ ದೇವೀ ಮೇ ಪಾತು ಚಾರುಮೂಲಕಂ ।\\
ಜಂಘಾಯುಗ್ಮಂ ಸದಾ ಪಾತು ಕೀರ್ತೀಃ ಚಕ್ರಾಪಹಾರಿಣೀ ॥೧೮೦॥

ಪಾದಯುಗ್ಮಂ ಪಾಕಸಂಸ್ಥಾ ಪಾಕಶಾಸನರಕ್ಷಿಕಾ ।\\
ಕುಲಾಲಚಕ್ರಭ್ರಮರಾ ಪಾತು ಪಾದಾಂಗುಲೀರ್ಮಮ ॥೧೮೧॥

ನಖಾಗ್ರಾಣಿ ದಶವಿಧಾ ತಥಾ ಹಸ್ತದ್ವಯಸ್ಯ ಚ ।\\
ವಿಂಶರೂಪಾ ಕಾಲನಾಕ್ಷಾ ಸರ್ವದಾ ಪರಿರಕ್ಷತು ॥೧೮೨॥

ಕುಲಚ್ಛತ್ರಾಧಾರರೂಪಾ ಕುಲಮಂಡಲಗೋಪಿತಾ ।\\
ಕುಲಕುಂಡಲಿನೀ ಮಾತಾ ಕುಲಪಂಡಿತಮಂಡಿತಾ ॥೧೮೩॥

ಕಾಕಾನನೀ ಕಾಕತುಂಡೀ ಕಾಕಾಯುಃ ಪ್ರಖರಾರ್ಕಜಾ ।\\
ಕಾಕಜ್ವರಾ ಕಾಕಜಿಹ್ವಾ ಕಾಕಾಜಿಜ್ಞಾ ಸನಸ್ಥಿತಾ ॥೧೮೪॥

ಕಪಿಧ್ವಜಾ ಕಪಿಕ್ರೋಶಾ ಕಪಿಬಾಲಾ ಹಿಕಸ್ವರಾ ।\\
ಕಾಲಕಾಂಚೀ ವಿಂಶತಿಸ್ಥಾ ಸದಾ ವಿಂಶನಖಾಗ್ರಹಂ ॥೧೮೫॥

ಪಾತು ದೇವೀ ಕಾಲರೂಪಾ ಕಲಿಕಾಲಫಲಾಲಯಾ ।\\
ವಾತೇ ವಾ ಪರ್ವತೇ ವಾಪಿ ಶೂನ್ಯಾಗಾರೇ ಚತುಷ್ಪಥೇ ॥೧೮೬॥

ಕುಲೇಂದ್ರಸಮಯಾಚಾರಾ ಕುಲಾಚಾರಜನಪ್ರಿಯಾ ।\\
ಕುಲಪರ್ವತಸಂಸ್ಥಾ ಚ ಕುಲಕೈಲಾಸವಾಸಿನೀ ॥೧೮೭॥

ಮಹಾದಾವಾನಲೇ ಪಾತು ಕುಮಾರ್ಗೇ ಕುತ್ಸಿತಗ್ರಹೇ ।\\
ರಾಜ್ಞೋಽಪ್ರಿಯೇ ರಾಜವಶ್ಯೇ ಮಹಾಶತ್ರುವಿನಾಶನೇ ॥೧೮೮॥

ಕಲಿಕಾಲಮಹಾಲಕ್ಷ್ಮೀಃ ಕ್ರಿಯಾಲಕ್ಷ್ಮೀಃ ಕುಲಾಂಬರಾ ।\\
ಕಲೀಂದ್ರಕೀಲಿತಾ ಕೀಲಾ ಕೀಲಾಲಸ್ವರ್ಗವಾಸಿನೀ ॥೧೮೯॥

ದಶದಿಕ್ಷು ಸದಾ ಪಾತು ಇಂದ್ರಾದಿದಶಲೋಕಪಾ ।\\
ನವಚ್ಛಿನ್ನೇ ಸದಾ ಪಾತು ಸೂರ್ಯಾದಿಕನವಗ್ರಹಾ ॥೧೯೦॥

ಪಾತು ಮಾಂ ಕುಲಮಾಂಸಾಢ್ಯಾ ಕುಲಪದ್ಮನಿವಾಸಿನೀ ।\\
ಕುಲದ್ರವ್ಯಪ್ರಿಯಾ ಮಧ್ಯಾ ಷೋಡಶೀ ಭುವನೇಶ್ವರೀ ॥೧೯೧॥

ವಿದ್ಯಾವಾದೇ ವಿವಾದೇ ಚ ಮತ್ತಕಾಲೇ ಮಹಾಭಯೇ ।\\
ದುರ್ಭೀಕ್ಷಾದಿಭಯೇ ಚೈವ ವ್ಯಾಧಿಸಂಕರಪೀಡಿತೇ ॥೧೯೨॥

ಕಾಲೀಕುಲ್ಲಾ ಕಪಾಲೀ ಚ ಕಾಮಾಖ್ಯಾ ಕಾಮಚಾರಿಣೀ ।\\
ಸದಾ ಮಾಂ ಕುಲಸಂಸರ್ಗೇ ಪಾತು ಕೌಲೇ ಸುಸಂಗತಾ ॥೧೯೩॥

ಸರ್ವತ್ರ ಸರ್ವದೇಶೇ ಚ ಕುಲರೂಪಾ ಸದಾವತು ।\\
ಇತ್ಯೇತತ್ ಕಥಿತಂ ನಾಥ ಮಾತುಃ ಪ್ರಸಾದಹೇತುನಾ ॥೧೯೪॥

ಅಷ್ಟೋತ್ತರಶತಂ ನಾಮ ಸಹಸ್ರಂ ಕುಂಡಲೀಪ್ರಿಯಂ ।\\
ಕುಲಕುಂಡಲಿನೀದೇವ್ಯಾಃ ಸರ್ವಮಂತ್ರಸುಸಿದ್ಧಯೇ ॥೧೯೫॥

ಸರ್ವದೇವಮನೂನಾಂಚ ಚೈತನ್ಯಾಯ ಸುಸಿದ್ಧಯೇ ।\\
ಅಣಿಮಾದ್ಯಷ್ಟಸಿದ್ಧ್ಯರ್ಥಂ ಸಾಧಕಾನಾಂ ಹಿತಾಯ ಚ ॥೧೯೬॥

ಬ್ರಾಹ್ಮಣಾಯ ಪ್ರದಾತವ್ಯಂ ಕುಲದ್ರವ್ಯಪರಾಯ ಚ ।\\
ಅಕುಲೀನೇಽಬ್ರಾಹ್ಮಣೇ ಚ ನ ದೇಯಃ ಕುಂಡಲೀಸ್ತವಃ ।\\
ಪ್ರವೃತ್ತೇ ಕುಂಡಲೀಚಕ್ರೇ ಸರ್ವೇ ವರ್ಣಾ ದ್ವಿಜಾತಯಃ ॥೧೯೭॥

ನಿವೃತ್ತೇ ಭೈರವೀಚಕ್ರೇ ಸರ್ವೇ ವರ್ಣಾಃ ಪೃಥಕ್ಪೃಥಕ್ ।\\
ಕುಲೀನಾಯ ಪ್ರದಾತವ್ಯಂ ಸಾಧಕಾಯ ವಿಶೇಷತಃ ॥೧೯೮॥

ದಾನಾದೇವ ಹಿ ಸಿದ್ಧಿಃ ಸ್ಯಾನ್ಮಮಾಜ್ಞಾಬಲಹೇತುನಾ ।\\
ಮಮ ಕ್ರಿಯಾಯಾಂ ಯಸ್ತಿಷ್ಠೇತ್ಸ ಮೇ ಪುತ್ರೋ ನ ಸಂಶಯಃ ॥೧೯೯॥

ಸ ಆಯಾತಿ ಮಮ ಪದಂ ಜೀವನ್ಮುಕ್ತಃ ಸ ವಾಸವಃ ।\\
ಆಸವೇನ ಸಮಾಂಸೇನ ಕುಲವಹ್ನೌ ಮಹಾನಿಶಿ ॥೨೦೦॥

ನಾಮ ಪ್ರತ್ಯೇಕಮುಚ್ಚಾರ್ಯ ಜುಹುಯಾತ್ ಕಾರ್ಯಸಿದ್ಧಯೇ ।\\
ಪಂಚಾಚಾರರತೋ ಭೂತ್ತ್ವಾ ಊರ್ಧ್ವರೇತಾ ಭವೇದ್ಯತಿಃ ॥೨೦೧॥

ಸಂವತ್ಸರಾನ್ಮಮ ಸ್ಥಾನೇ ಆಯಾತಿ ನಾತ್ರ ಸಂಶಯಃ ।\\
ಐಹಿಕೇ ಕಾರ್ಯಸಿದ್ಧಿಃ ಸ್ಯಾತ್ ದೈಹಿಕೇ ಸರ್ವಸಿದ್ಧಿದಃ ॥೨೦೨॥

ವಶೀ ಭೂತ್ತ್ವಾ ತ್ರಿಮಾರ್ಗಸ್ಥಾಃ ಸ್ವರ್ಗಭೂತಲವಾಸಿನಃ ।\\
ಅಸ್ಯ ಭೃತ್ಯಾಃ ಪ್ರಭವಂತಿ ಇಂದ್ರಾದಿಲೋಕಪಾಲಕಾಃ ॥೨೦೩॥

ಸ ಏವ ಯೋಗೀ ಪರಮೋ ಯಸ್ಯಾರ್ಥೇಽಯಂ ಸುನಿಶ್ಚಲಃ ।\\
ಸ ಏವ ಖೇಚರೋ ಭಕ್ತೋ ನಾರದಾದಿಶುಕೋಪಮಃ ॥೨೦೪॥

ಯೋ ಲೋಕಃ ಪ್ರಜಪತ್ಯೇವಂ ಸ ಶಿವೋ ನ ಚ ಮಾನುಷಃ ।\\
ಸ ಸಮಾಧಿಗತೋ ನಿತ್ಯೋ ಧ್ಯಾನಸ್ಥೋ ಯೋಗಿವಲ್ಲಭಃ ॥೨೦೫॥

ಚತುರ್ವ್ಯೂಹಗತೋ ದೇವಃ ಸಹಸಾ ನಾತ್ರ ಸಂಶಯಃ ।\\
ಯಃ ಪ್ರಧಾರಯತೇ ಭಕ್ತ್ಯಾ ಕಂಠೇ ವಾ ಮಸ್ತಕೇ ಭುಜೇ ॥೨೦೬॥

ಸ ಭವೇತ್ ಕಾಲಿಕಾಪುತ್ರೋ ವಿದ್ಯಾನಾಥಃ ಸ್ವಯಂಭುವಿ ।\\
ಧನೇಶಃ ಪುತ್ರವಾನ್ ಯೋಗೀ ಯತೀಶಃ ಸರ್ವಗೋ ಭವೇತ್ ॥೨೦೭॥

ವಾಮಾ ವಾಮಕರೇ ಧೃತ್ತ್ವಾ ಸರ್ವಸಿದ್ಧೀಶ್ವರೋ ಭವೇತ್ ।\\
ಯದಿ ಪಠತಿ ಮನುಷ್ಯೋ ಮಾನುಷೀ ವಾ ಮಹತ್ಯಾ ॥೨೦೮॥

ಸಕಲಧನಜನೇಶೀ ಪುತ್ರಿಣೀ ಜೀವವತ್ಸಾ ।\\
ಕುಲಪತಿರಿಹ ಲೋಕೇ ಸ್ವರ್ಗಮೋಕ್ಷೈಕಹೇತುಃ\\
ಸ ಭವತಿ ಭವನಾಥೋ ಯೋಗಿನೀವಲ್ಲಭೇಶಃ ॥೨೦೯॥

ಪಠತಿ ಯ ಇಹ ನಿತ್ಯಂ ಭಕ್ತಿಭಾವೇನ ಮರ್ತ್ಯೋ\\
ಹರಣಮಪಿ ಕರೋತಿ ಪ್ರಾಣವಿಪ್ರಾಣಯೋಗಃ ।\\
ಸ್ತವನಪಠನಪುಣ್ಯಂ ಕೋಟಿಜನ್ಮಾಘನಾಶ\\
ಕಥಿತುಮಪಿ ನ ಶಕ್ತೋಽಹಂ ಮಹಾಮಾಂಸಭಕ್ಷಾ ॥೨೧೦॥

\authorline{॥ಇತಿ ಶ್ರೀರುದ್ರಯಾಮಲೇ ಉತ್ತರತಂತ್ರೇ ಷಟ್ಚಕ್ರಪ್ರಕಾಶೇ ಭೈರವೀಭೈರವಸಂವಾದೇ ಮಹಾಕುಲಕುಂಡಲಿನೀ ಸಹಸ್ರನಾಮಸ್ತೋತ್ರಂ ಸಂಪೂರ್ಣಂ॥}

%=============================================================================================
\section{ಶ್ರೀಛಿನ್ನಮಸ್ತಾಸಹಸ್ರನಾಮಸ್ತೋತ್ರಂ}
\addcontentsline{toc}{section}{ಶ್ರೀಛಿನ್ನಮಸ್ತಾಸಹಸ್ರನಾಮಸ್ತೋತ್ರಂ}

ಶ್ರೀದೇವ್ಯುವಾಚ ।\\
ದೇವದೇವ ಮಹಾದೇವ ಸರ್ವಶಾಸ್ತ್ರವಿದಾಂವರ ।\\
ಕೃಪಾಂ ಕುರು ಜಗನ್ನಾಥ ಕಥಯಸ್ವ ಮಮ ಪ್ರಭೋ ॥೧॥

ಪ್ರಚಂಡಚಂಡಿಕಾ ದೇವೀ ಸರ್ವಲೋಕಹಿತೈಷಿಣೀ ।\\
ತಸ್ಯಾಶ್ಚ ಕಥಿತಂ ಸರ್ವಂ ಸ್ತವಂ ಚ ಕವಚಾದಿಕಂ ॥೨॥

ಇದಾನೀಂ ಛಿನ್ನಮಸ್ತಾಯಾ ನಾಮ್ನಾಂ ಸಾಹಸ್ರಕಂ ಶುಭಂ ।\\
ತ್ವಂ ಪ್ರಕಾಶಯ ಮೇ ದೇವ ಕೃಪಯಾ ಭಕ್ತವತ್ಸಲ ॥೩॥

ಶ್ರೀಶಿವ ಉವಾಚ ।\\
ಶೃಣು ದೇವಿ ಪ್ರವಕ್ಷ್ಯಾಮಿ ಚ್ಛಿನ್ನಾಯಾಃ ಸುಮನೋಹರಂ ।\\
ಗೋಪನೀಯಂ ಪ್ರಯತ್ನೇನ ಯದೀಚ್ಛೇದಾತ್ಮನೋ ಹಿತಂ ॥೪॥

ನ ವಕ್ತವ್ಯಂ ಚ ಕುತ್ರಾಪಿ ಪ್ರಾಣೈಃ ಕಂಠಗತೈರಪಿ ।\\
ತಚ್ಛೃಣುಷ್ವ ಮಹೇಶಾನಿ ಸರ್ವಂ ತತ್ಕಥಯಾಮಿ ತೇ ॥೫॥

ವಿನಾ ಪೂಜಾಂ ವಿನಾ ಧ್ಯಾನಂ ವಿನಾ ಜಾಪ್ಯೇನ ಸಿದ್ಧ್ಯತಿ ।\\
ವಿನಾ ಧ್ಯಾನಂ ತಥಾ ದೇವಿ ವಿನಾ ಭೂತಾದಿಶೋಧನಂ ॥೬॥

ಪಠನಾದೇವ ಸಿದ್ಧಿಃ ಸ್ಯಾತ್ಸತ್ಯಂ ಸತ್ಯಂ ವರಾನನೇ ।\\
ಪುರಾ ಕೈಲಾಸಶಿಖರೇ ಸರ್ವದೇವಸಭಾಲಯೇ ।\\
ಪರಿಪಪ್ರಚ್ಛ ಕಥಿತಂ ತಥಾ ಶೃಣು ವರಾನನೇ ॥೭॥

ಓಂ ಅಸ್ಯ ಶ್ರೀಪ್ರಚಂಡಚಂಡಿಕಾಸಹಸ್ರನಾಮಸ್ತೋತ್ರಸ್ಯ ಭೈರವ ಋಷಿಃ  । ಸಮ್ರಾಟ್ ಛಂದಃ  । ಪ್ರಚಂಡಚಂಡಿಕಾ ದೇವತಾ  । ಧರ್ಮಾರ್ಥಕಾಮಮೋಕ್ಷಾರ್ಥೇ ಪಾಠೇ ವಿನಿಯೋಗಃ ॥೮॥

ಓಂ ಪ್ರಚಂಡಚಂಡಿಕಾ ಚಂಡಾ ಚಂಡದೈತ್ಯವಿನಾಶಿನೀ ।\\
ಚಾಮುಂಡಾ ಚ ಸಚಂಡಾ ಚ ಚಪಲಾ ಚಾರುದೇಹಿನೀ ॥೯॥

ಲಲಜಿಹ್ವಾ ಚಲದ್ರಕ್ತಾ ಚಾರುಚಂದ್ರನಿಭಾನನಾ ।\\
ಚಕೋರಾಕ್ಷೀ ಚಂಡನಾದಾ ಚಂಚಲಾ ಚ ಮನೋನ್ಮದಾ ॥೧೦॥

ಚೇತನಾ ಚಿತಿಸಂಸ್ಥಾ ಚ ಚಿತ್ಕಲಾ ಜ್ಞಾನರೂಪಿಣೀ ।\\
ಮಹಾಭಯಂಕರೀ ದೇವೀ ವರದಾಭಯಧಾರಿಣೀ ॥೧೧॥

ಭವಾಢ್ಯಾ ಭವರೂಪಾ ಚ ಭವಬಂಧವಿಮೋಚಿನೀ ।\\
ಭವಾನೀ ಭುವನೇಶೀ ಚ ಭವಸಂಸಾರತಾರಿಣೀ ॥೧೨॥

ಭವಾಬ್ಧಿರ್ಭವಮೋಕ್ಷಾ ಚ ಭವಬಂಧವಿಘಾತಿನೀ ।\\
ಭಾಗೀರಥೀ ಭಗಸ್ಥಾ ಚ ಭಾಗ್ಯಭೋಗಪ್ರದಾಯಿನೀ ॥೧೩॥

ಕಮಲಾ ಕಾಮದಾ ದುರ್ಗಾ ದುರ್ಗಬಂಧವಿಮೋಚಿನೀ ।\\
ದುರ್ದ್ದರ್ಶನಾ ದುರ್ಗರೂಪಾ ದುರ್ಜ್ಞೇಯಾ ದುರ್ಗನಾಶಿನೀ ॥೧೪॥

ದೀನದುಃಖಹರಾ ನಿತ್ಯಾ ನಿತ್ಯಶೋಕವಿನಾಶಿನೀ ।\\
ನಿತ್ಯಾನಂದಮಯಾ ದೇವೀ ನಿತ್ಯಂ ಕಲ್ಯಾಣಕಾರಿಣೀ ॥೧೫॥

ಸರ್ವಾರ್ಥಸಾಧನಕರೀ ಸರ್ವಸಿದ್ಧಿಸ್ವರೂಪಿಣೀ ।\\
ಸರ್ವಕ್ಷೋಭಣಶಕ್ತಿಶ್ಚ ಸರ್ವವಿದ್ರಾವಿಣೀ ಪರಾ ॥೧೬॥

ಸರ್ವರಂಜನಶಕ್ತಿಶ್ಚ ಸರ್ವೋನ್ಮಾದಸ್ವರೂಪಿಣೀ ।\\
ಸರ್ವದಾ ಸಿದ್ಧಿದಾತ್ರೀ ಚ ಸಿದ್ಧವಿದ್ಯಾಸ್ವರೂಪಿಣೀ ॥೧೭॥

ಸಕಲಾ ನಿಷ್ಕಲಾ ಸಿದ್ಧಾ ಕಲಾತೀತಾ ಕಲಾಮಯೀ ।\\
ಕುಲಜ್ಞಾ ಕುಲರೂಪಾ ಚ ಚಕ್ಷುರಾನಂದದಾಯಿನೀ ॥೧೮॥

ಕುಲೀನಾ ಸಾಮರೂಪಾ ಚ ಕಾಮರೂಪಾ ಮನೋಹರಾ ।\\
ಕಮಲಸ್ಥಾ ಕಂಜಮುಖೀ ಕುಂಜರೇಶ್ವರಗಾಮಿನೀ ॥೧೯॥

ಕುಲರೂಪಾ ಕೋಟರಾಕ್ಷೀ ಕಮಲೈಶ್ವರ್ಯದಾಯಿನೀ ।\\
ಕುಂತೀ ಕಕುದ್ಮಿನೀ ಕುಲ್ಲಾ ಕುರುಕುಲ್ಲಾ ಕರಾಲಿಕಾ ॥೨೦॥

ಕಾಮೇಶ್ವರೀ ಕಾಮಮಾತಾ ಕಾಮತಾಪವಿಮೋಚಿನೀ ।\\
ಕಾಮರೂಪಾ ಕಾಮಸತ್ವಾ ಕಾಮಕೌತುಕಕಾರಿಣೀ ॥೨೧॥

ಕಾರುಣ್ಯಹೃದಯಾ ಕ್ರೀಂಕ್ರೀಂಮಂತ್ರರೂಪಾ ಚ ಕೋಟರಾ ।\\
ಕೌಮೋದಕೀ ಕುಮುದಿನೀ ಕೈವಲ್ಯಾ ಕುಲವಾಸಿನೀ ॥೨೨॥

ಕೇಶವೀ ಕೇಶವಾರಾಧ್ಯಾ ಕೇಶಿದೈತ್ಯನಿಷೂದಿನೀ ।\\
ಕ್ಲೇಶಹಾ ಕ್ಲೇಶರಹಿತಾ ಕ್ಲೇಶಸಂಘವಿನಾಶಿನೀ ॥೨೩॥

ಕರಾಲೀ ಚ ಕರಾಲಾಸ್ಯಾ ಕರಾಲಾಸುರನಾಶಿನೀ ।\\
ಕರಾಲಚರ್ಮಾಸಿಧರಾ ಕರಾಲಕಲನಾಶಿನೀ ॥೨೪॥

ಕಂಕಿನೀ ಕಂಕನಿರತಾ ಕಪಾಲವರಧಾರಿಣೀ ।\\
ಖಡ್ಗಹಸ್ತಾ ತ್ರಿನೇತ್ರಾ ಚ ಖಂಡಮುಂಡಾಸಿಧಾರಿಣೀ ॥೨೫॥

ಖಲಹಾ ಖಲಹಂತ್ರೀ ಚ ಕ್ಷರಂತೀ ಖಗತಾ ಸದಾ ।\\
ಗಂಗಾಗೌತಮಪೂಜ್ಯಾ ಚ ಗೌರೀ ಗಂಧರ್ವವಾಸಿನೀ ॥೨೬॥

ಗಂಧರ್ವಾ ಗಗಣಾರಾಧ್ಯಾ ಗಣಾ ಗಂಧರ್ವಸೇವಿತಾ ।\\
ಗಣತ್ಕಾರಗಣಾ ದೇವೀ ನಿರ್ಗುಣಾ ಚ ಗುಣಾತ್ಮಿಕಾ ॥೨೭॥

ಗುಣತಾ ಗುಣದಾತ್ರೀ ಚ ಗುಣಗೌರವದಾಯಿನೀ ।\\
ಗಣೇಶಮಾತಾ ಗಂಭೀರಾ ಗಗಣಾ ಜ್ಯೋತಿಕಾರಿಣೀ ॥೨೮॥

ಗೌರಾಂಗೀ ಚ ಗಯಾ ಗಮ್ಯಾ ಗೌತಮಸ್ಥಾನವಾಸಿನೀ ।\\
ಗದಾಧರಪ್ರಿಯಾ ಜ್ಞೇಯಾ ಜ್ಞಾನಗಮ್ಯಾ ಗುಹೇಶ್ವರೀ ॥೨೯॥

ಗಾಯತ್ರೀ ಚ ಗುಣವತೀ ಗುಣಾತೀತಾ ಗುಣೇಶ್ವರೀ ।\\
ಗಣೇಶಜನನೀ ದೇವೀ ಗಣೇಶವರದಾಯಿನೀ ॥೩೦॥

ಗಣಾಧ್ಯಕ್ಷನುತಾ ನಿತ್ಯಾ ಗಣಾಧ್ಯಕ್ಷಪ್ರಪೂಜಿತಾ ।\\
ಗಿರೀಶರಮಣೀ ದೇವೀ ಗಿರೀಶಪರಿವಂದಿತಾ ॥೩೧॥

ಗತಿದಾ ಗತಿಹಾ ಗೀತಾ ಗೌತಮೀ ಗುರುಸೇವಿತಾ ।\\
ಗುರುಪೂಜ್ಯಾ ಗುರುಯುತಾ ಗುರುಸೇವನತತ್ಪರಾ ॥೩೨॥

ಗಂಧದ್ವಾರಾ ಚ ಗಂಧಾಢ್ಯಾ ಗಂಧಾತ್ಮಾ ಗಂಧಕಾರಿಣೀ ।\\
ಗೀರ್ವಾಣಪತಿಸಂಪೂಜ್ಯಾ ಗೀರ್ವಾಣಪತಿತುಷ್ಟಿದಾ ॥೩೩॥

ಗೀರ್ವಾಣಾಧಿಶರಮಣೀ ಗೀರ್ವಾಣಾಧಿಶವಂದಿತಾ ।\\
ಗೀರ್ವಾಣಾಧಿಶಸಂಸೇವ್ಯಾ ಗೀರ್ವಾಣಾಧಿಶಹರ್ಷದಾ ॥೩೪॥

ಗಾನಶಕ್ತಿರ್ಗಾನಗಮ್ಯಾ ಗಾನಶಕ್ತಿಪ್ರದಾಯಿನೀ ।\\
ಗಾನವಿದ್ಯಾ ಗಾನಸಿದ್ಧಾ ಗಾನಸಂತುಷ್ಟಮಾನಸಾ ॥೩೫॥

ಗಾನಾತೀತಾ ಗಾನಗೀತಾ ಗಾನಹರ್ಷಪ್ರಪೂರಿತಾ ।\\
ಗಂಧರ್ವಪತಿಸಂಹೃಷ್ಟಾ ಗಂಧರ್ವಗುಣಮಂಡಿತಾ ॥೩೬॥

ಗಂಧರ್ವಗಣಸಂಸೇವ್ಯಾ ಗಂಧರ್ವಗಣಮಧ್ಯಗಾ ।\\
ಗಂಧರ್ವಗಣಕುಶಲಾ ಗಂಧರ್ವಗಣಪೂಜಿತಾ ॥೩೭॥

ಗಂಧರ್ವಗಣನಿರತಾ ಗಂಧರ್ವಗಣಭೂಷಿತಾ ।\\
ಘರ್ಘರಾ ಘೋರರೂಪಾ ಚ ಘೋರಘುರ್ಘುರನಾದಿನೀ ॥೩೮॥

ಘರ್ಮಬಿಂದುಸಮುದ್ಭೂತಾ ಘರ್ಮಬಿಂದುಸ್ವರೂಪಿಣೀ ।\\
ಘಂಟಾರವಾ ಘನರವಾ ಘನರೂಪಾ ಘನೋದರೀ ॥೩೯॥

ಘೋರಸತ್ವಾ ಚ ಘನದಾ ಘಂಟಾನಾದವಿನೋದನೀ ।\\
ಘೋರಚಾಂಡಾಲಿನೀ ಘೋರಾ ಘೋರಚಂಡವಿನಾಶಿನೀ ॥೪೦॥

ಘೋರದಾನವದಮನೀ ಘೋರದಾನವನಾಶಿನೀ ।\\
ಘೋರಕರ್ಮಾದಿರಹಿತಾ ಘೋರಕರ್ಮನಿಷೇವಿತಾ ॥೪೧॥

ಘೋರತತ್ವಮಯೀ ದೇವೀ ಘೋರತತ್ವವಿಮೋಚನೀ ।\\
ಘೋರಕರ್ಮಾದಿರಹಿತಾ ಘೋರಕರ್ಮಾದಿಪೂರಿತಾ ॥೪೨॥

ಘೋರಕರ್ಮಾದಿನಿರತಾ ಘೋರಕರ್ಮಪ್ರವರ್ದ್ಧಿನೀ ।\\
ಘೋರಭೂತಪ್ರಮಥಿನೀ ಘೋರವೇತಾಲನಾಶಿನೀ ॥೪೩॥

ಘೋರದಾವಾಗ್ನಿದಮನೀ ಘೋರಶತ್ರುನಿಷೂದಿನೀ ।\\
ಘೋರಮಂತ್ರಯುತಾ ಚೈವ ಘೋರಮಂತ್ರಪ್ರಪೂಜಿತಾ ॥೪೪॥

ಘೋರಮಂತ್ರಮನೋಭಿಜ್ಞಾ ಘೋರಮಂತ್ರಫಲಪ್ರದಾ ।\\
ಘೋರಮಂತ್ರನಿಧಿಶ್ಚೈವ ಘೋರಮಂತ್ರಕೃತಾಸ್ಪದಾ ॥೪೫॥

ಘೋರಮಂತ್ರೇಶ್ವರೀ ದೇವೀ ಘೋರಮಂತ್ರಾರ್ಥಮಾನಸಾ ।\\
ಘೋರಮಂತ್ರಾರ್ಥತತ್ವಜ್ಞಾ ಘೋರಮಂತ್ರಾರ್ಥಪಾರಗಾ ॥೪೬॥

ಘೋರಮಂತ್ರಾರ್ಥವಿಭವಾ ಘೋರಮಂತ್ರಾರ್ಥಬೋಧಿನೀ ।\\
ಘೋರಮಂತ್ರಾರ್ಥನಿಚಯಾ ಘೋರಮಂತ್ರಾರ್ಥಜನ್ಮಭೂಃ ॥೪೭॥

ಘೋರಮಂತ್ರಜಪರತಾ ಘೋರಮಂತ್ರಜಪೋದ್ಯತಾ ।\\
ಙಕಾರವರ್ಣಾನಿಲಯಾ ಙಕಾರಾಕ್ಷರಮಂಡಿತಾ ॥೪೮॥

ಙಕಾರಾಪರರೂಪಾ ಙಕಾರಾಕ್ಷರರೂಪಿಣೀ ।\\
ಚಿತ್ರರೂಪಾ ಚಿತ್ರನಾಡೀ ಚಾರುಕೇಶೀ ಚಯಪ್ರಭಾ ॥೪೯॥

ಚಂಚಲಾ ಚಂಚಲಾಕಾರಾ ಚಾರುರೂಪಾ ಚ ಚಂಡಿಕಾ ।\\
ಚತುರ್ವೇದಮಯೀ ಚಂಡಾ ಚಂಡಾಲಗಣಮಂಡಿತಾ ॥೫೦॥

ಚಾಂಡಾಲಚ್ಛೇದಿನೀ ಚಂಡತಪೋನಿರ್ಮೂಲಕಾರಿಣೀ ।\\
ಚತುರ್ಭುಜಾ ಚಂಡರೂಪಾ ಚಂಡಮುಂಡವಿನಾಶಿನೀ ॥೫೧॥

ಚಂದ್ರಿಕಾ ಚಂದ್ರಕೀರ್ತಿಶ್ಚ ಚಂದ್ರಕಾಂತಿಸ್ತಥೈವ ಚ ।\\
ಚಂದ್ರಾಸ್ಯಾ ಚಂದ್ರರೂಪಾ ಚ ಚಂದ್ರಮೌಲಿಸ್ವರೂಪಿಣೀ ॥೫೨॥

ಚಂದ್ರಮೌಲಿಪ್ರಿಯಾ ಚಂದ್ರಮೌಲಿಸಂತುಷ್ಟಮಾನಸಾ ।\\
ಚಕೋರಬಂಧುರಮಣೀ ಚಕೋರಬಂಧುಪೂಜಿತಾ ॥೫೩॥

ಚಕ್ರರೂಪಾ ಚಕ್ರಮಯೀ ಚಕ್ರಾಕಾರಸ್ವರೂಪಿಣೀ ।\\
ಚಕ್ರಪಾಣಿಪ್ರಿಯಾ ಚಕ್ರಪಾಣಿಪ್ರೀತಿದಾಯಿನೀ ॥೫೪॥

ಚಕ್ರಪಾಣಿರಸಾಭಿಜ್ಞಾ ಚಕ್ರಪಾಣಿವರಪ್ರದಾ ।\\
ಚಕ್ರಪಾಣಿವರೋನ್ಮತ್ತಾ ಚಕ್ರಪಾಣಿಸ್ವರೂಪಿಣೀ ॥೫೫॥

ಚಕ್ರಪಾಣಿಶ್ವರೀ ನಿತ್ಯಂ ಚಕ್ರಪಾಣಿನಮಸ್ಕೃತಾ ।\\
ಚಕ್ರಪಾಣಿಸಮುದ್ಭೂತಾ ಚಕ್ರಪಾಣಿಗುಣಾಸ್ಪದಾ ॥೫೬॥

ಚಂದ್ರಾವಲೀ ಚಂದ್ರವತೀ ಚಂದ್ರಕೋಟಿಸಮಪ್ರಭಾ ।\\
ಚಂದನಾರ್ಚಿತಪಾದಾಬ್ಜಾ ಚಂದನಾನ್ವಿತಮಸ್ತಕಾ ॥೫೭॥

ಚಾರುಕೀರ್ತಿಶ್ಚಾರುನೇತ್ರಾ ಚಾರುಚಂದ್ರವಿಭೂಷಣಾ ।\\
ಚಾರುಭೂಷಾ ಚಾರುವೇಷಾ ಚಾರುವೇಷಪ್ರದಾಯಿನೀ ॥೫೮॥

ಚಾರುಭೂಷಾಭೂಷಿತಾಂಗೀ ಚತುರ್ವಕ್ತ್ರವರಪ್ರದಾ ।\\
ಚತುರ್ವಕ್ತ್ರಸಮಾರಾಧ್ಯಾ ಚತುರ್ವಕ್ತ್ರಸಮಾಶ್ರಿತಾ ॥೫೯॥

ಚತುರ್ವಕ್ತ್ರಚತುರ್ವಾಹಾ ಚತುರ್ಥೀ ಚ ಚತುರ್ದಶೀ ।\\
ಚಿತ್ರಾ ಚರ್ಮಣ್ವತೀ ಚೈತ್ರೀ ಚಂದ್ರಭಾಗಾ ಚ ಚಂಪಕಾ ॥೬೦॥

ಚತುರ್ದ್ದಶಯಮಾಕಾರಾ ಚತುರ್ದಶಯಮಾನುಗಾ ।\\
ಚತುರ್ದಶಯಮಪ್ರೀತಾ ಚತುರ್ದಶಯಮಪ್ರಿಯಾ ॥೬೧॥

ಛಲಸ್ಥಾ ಚ್ಛಿದ್ರರೂಪಾ ಚ ಚ್ಛದ್ಮದಾ ಚ್ಛದ್ಮರಾಜಿಕಾ ।\\
ಛಿನ್ನಮಸ್ತಾ ತಥಾ ಚ್ಛಿನ್ನಾ ಚ್ಛಿನ್ನಮುಂಡವಿಧಾರಿಣೀ ॥೬೨॥

ಜಯದಾ ಜಯರೂಪಾ ಚ ಜಯಂತೀ ಜಯಮೋಹಿನೀ ।\\
ಜಯಾ ಜೀವನಸಂಸ್ಥಾ ಚ ಜಾಲಂಧರನಿವಾಸಿನೀ ॥೬೩॥

ಜ್ವಾಲಾಮುಖೀ ಜ್ವಾಲದಾತ್ರೀ ಜಾಜ್ವಲ್ಯದಹನೋಪಮಾ ।\\
ಜಗದ್ವಂದ್ಯಾ ಜಗತ್ಪೂಜ್ಯಾ ಜಗತ್ತ್ರಾಣಪರಾಯಣಾ ॥೬೪॥

ಜಗತೀ ಜಗತಾಧಾರಾ ಜನ್ಮಮೃತ್ಯುಜರಾಪಹಾ ।\\
ಜನನೀ ಜನ್ಮಭೂಮಿಶ್ಚಜನ್ಮದಾ ಜಯಶಾಲಿನೀ ॥೬೫॥

ಜ್ವರರೋಗಹರಾ ಜ್ವಾಲಾ ಜ್ವಾಲಾಮಾಲಾಪ್ರಪೂರಿತಾ ।\\
ಜಂಭಾರಾತೀಶ್ವರೀ ಜಂಭಾರಾತಿವೈಭವಕಾರಿಣೀ ॥೬೬॥

ಜಂಭಾರಾತಿಸ್ತುತಾ ಜಂಭಾರಾತಿಶತ್ರುನಿಷೂದಿನೀ ।\\
ಜಯದುರ್ಗಾ ಜಯಾರಾಧ್ಯಾ ಜಯಕಾಲೀ ಜಯೇಶ್ವರೀ ॥೬೭॥

ಜಯತಾರಾ ಜಯಾತೀತಾ ಜಯಶಂಕರವಲ್ಲಭಾ ।\\
ಜಯದಾ ಜಹ್ನುತನಯಾ ಜಲಧಿತ್ರಾಸಕಾರಿಣೀ ॥೬೮॥

ಜಲಧಿವ್ಯಾಧಿದಮನೀ ಜಲಧಿಜ್ವರನಾಶಿನೀ ।\\
ಜಂಗಮೇಶೀ ಜಾಡ್ಯಹರಾ ಜಾಡ್ಯಸಂಘನಿವಾರಿಣೀ ॥೬೯॥

ಜಾಡ್ಯಗ್ರಸ್ತಜನಾತೀತಾ ಜಾಡ್ಯರೋಗನಿವಾರಿಣೀ ।\\
ಜನ್ಮದಾತ್ರೀ ಜನ್ಮಹರ್ತ್ರೀ ಜಯಘೋಷಸಮನ್ವಿತಾ ॥೭೦॥

ಜಪಯೋಗಸಮಾಯುಕ್ತಾ ಜಪಯೋಗವಿನೋದಿನೀ ।\\
ಜಪಯೋಗಪ್ರಿಯಾ ಜಾಪ್ಯಾ ಜಪಾತೀತಾ ಜಯಸ್ವನಾ ॥೭೧॥

ಜಾಯಾಭಾವಸ್ಥಿತಾ ಜಾಯಾ ಜಾಯಾಭಾವಪ್ರಪೂರಣೀ ।\\
ಜಪಾಕುಸುಮಸಂಕಾಶಾ ಜಪಾಕುಸುಮಪೂಜಿತಾ ॥೭೨॥

ಜಪಾಕುಸುಮಸಂಪ್ರೀತಾ ಜಪಾಕುಸುಮಮಂಡಿತಾ ।\\
ಜಪಾಕುಸುಮವದ್ಭಾಸಾ ಜಪಾಕುಸುಮರೂಪಿಣೀ ॥೭೩॥

ಜಮದಗ್ನಿಸ್ವರೂಪಾ ಚ ಜಾನಕೀ ಜನಕಾತ್ಮಜಾ ।\\
ಝಂಝಾವಾತಪ್ರಮುಕ್ತಾಂಗೀ ಝೋರಝಂಕಾರವಾಸಿನೀ ॥೭೪॥

ಝಂಕಾರಕಾರಿಣೀ ಝಂಝಾವಾತರೂಪಾ ಚ ಝಂಕರೀ ।\\
ಞಕಾರಾಣುಸ್ವರೂಪಾ ಚ ಟನಟಂಕಾರನಾದಿನೀ ॥೭೫॥

ಟಂಕಾರೀ ಟಕುವಾಣೀ ಚ ಠಕಾರಾಕ್ಷರರೂಪಿಣೀ ।\\
ಡಿಂಡಿಮಾ ಚ ತಥಾ ಡಿಂಭಾ ಡಿಂಡುಡಿಂಡಿಮನಾದಿನೀ ॥೭೬॥

ಢಕ್ಕಾಮಯೀ ಢಿಲಮಯೀ ನೃತ್ಯಶಬ್ದಾ ವಿಲಾಸಿನೀ ।\\
ಢಕ್ಕಾ ಢಕ್ಕೇಶ್ವರೀ ಢಕ್ಕಾಶಬ್ದರೂಪಾ ತಥೈವ ಚ ॥೭೭॥

ಢಕ್ಕಾನಾದಪ್ರಿಯಾ ಢಕ್ಕಾನಾದಸಂತುಷ್ಟಮಾನಸಾ ।\\
ಣಂಕಾರಾ ಣಾಕ್ಷರಮಯೀ ಣಾಕ್ಷರಾದಿಸ್ವರೂಪಿಣೀ ॥೭೮॥

ತ್ರಿಪುರಾ ತ್ರಿಪುರಮಯೀ ಚೈವ ತ್ರಿಶಕ್ತಿಸ್ತ್ರಿಗುಣಾತ್ಮಿಕಾ ।\\
ತಾಮಸೀ ಚ ತ್ರಿಲೋಕೇಶೀ ತ್ರಿಪುರಾ ಚ ತ್ರಯೀಶ್ವರೀ ॥೭೯॥

ತ್ರಿವಿದ್ಯಾ ಚ ತ್ರಿರೂಪಾ ಚ ತ್ರಿನೇತ್ರಾ ಚ ತ್ರಿರೂಪಿಣೀ ।\\
ತಾರಿಣೀ ತರಲಾ ತಾರಾ ತಾರಕಾರಿಪ್ರಪೂಜಿತಾ ॥೮೦॥

ತಾರಕಾರಿಸಮಾರಾಧ್ಯಾ ತಾರಕಾರಿವರಪ್ರದಾ ।\\
ತಾರಕಾರಿಪ್ರಸೂಸ್ತನ್ವೀ ತರುಣೀ ತರಲಪ್ರಭಾ ॥೮೧॥

ತ್ರಿರೂಪಾ ಚ ತ್ರಿಪುರಗಾ ತ್ರಿಶೂಲವರಧಾರಿಣೀ ।\\
ತ್ರಿಶೂಲಿನೀ ತಂತ್ರಮಯೀ ತಂತ್ರಶಾಸ್ತ್ರವಿಶಾರದಾ ॥೮೨॥

ತಂತ್ರರೂಪಾ ತಪೋಮೂರ್ತಿಸ್ತಂತ್ರಮಂತ್ರಸ್ವರೂಪಿಣೀ ।\\
ತಡಿತ್ತಡಿಲ್ಲತಾಕಾರಾ ತತ್ವಜ್ಞಾನಪ್ರದಾಯಿನೀ ॥೮೩॥

ತತ್ವಜ್ಞಾನೇಶ್ವರೀ ದೇವೀ ತತ್ವಜ್ಞಾನಪ್ರಬೋಧಿನೀ ।\\
ತ್ರಯೀಮಯೀ ತ್ರಯೀಸೇವ್ಯಾ ತ್ರ್ಯಕ್ಷರೀ ತ್ರ್ಯಕ್ಷರೇಶ್ವರೀ ॥೮೪॥

ತಾಪವಿಧ್ವಂಸಿನೀ ತಾಪಸಂಘನಿರ್ಮೂಲಕಾರಿಣೀ ।\\
ತ್ರಾಸಕರ್ತ್ರೀ ತ್ರಾಸಹರ್ತ್ರೀ ತ್ರಾಸದಾತ್ರೀ ಚ ತ್ರಾಸಹಾ ॥೮೫॥

ತಿಥೀಶಾ ತಿಥಿರೂಪಾ ಚ ತಿಥಿಸ್ಥಾ ತಿಥಿಪೂಜಿತಾ ।\\
ತಿಲೋತ್ತಮಾ ಚ ತಿಲದಾ ತಿಲಪ್ರಿತಾ ತಿಲೇಶ್ವರೀ ॥೮೬॥

ತ್ರಿಗುಣಾ ತ್ರಿಗುಣಾಕಾರಾ ತ್ರಿಪುರೀ ತ್ರಿಪುರಾತ್ಮಿಕಾ ।\\
ತ್ರಿಕುಟಾ ತ್ರಿಕುಟಾಕಾರಾ ತ್ರಿಕುಟಾಚಲಮಧ್ಯಗಾ ॥೮೭॥

ತ್ರಿಜಟಾ ಚ ತ್ರಿನೇತ್ರಾ ಚ ತ್ರಿನೇತ್ರವರಸುಂದರೀ ।\\
ತೃತೀಯಾ ಚ ತ್ರಿವರ್ಷಾ ಚ ತ್ರಿವಿಧಾ ತ್ರಿಮತೇಶ್ವರೀ ॥೮೮॥

ತ್ರಿಕೋಣಸ್ಥಾ ತ್ರಿಕೋಣೇಶೀ ತ್ರಿಕೋಣಯಂತ್ರಮಧ್ಯಗಾ ।\\
ತ್ರಿಸಂಧ್ಯಾ ಚ ತ್ರಿಸಂಧ್ಯಾರ್ಚ್ಯಾ ತ್ರಿಪದಾ ತ್ರಿಪದಾಸ್ಪದಾ ॥೮೯॥

ಸ್ಥಾನಸ್ಥಿತಾ ಸ್ಥಲಸ್ಥಾ ಚ ಧನ್ಯಸ್ಥಲನಿವಾಸಿನೀ ।\\
ಥಕಾರಾಕ್ಷರರೂಪಾ ಚ ಸ್ಥಲರೂಪಾ ತಥೈವ ಚ ॥೯೦॥

ಸ್ಥೂಲಹಸ್ತಾ ತಥಾ ಸ್ಥೂಲಾ ಸ್ಥೈರ್ಯರೂಪಪ್ರಕಾಶಿನೀ ।\\
ದುರ್ಗಾ ದುರ್ಗಾರ್ತಿಹಂತ್ರೀ ಚ ದುರ್ಗಬಂಧವಿಮೋಚಿನೀ ॥೯೧॥

ದೇವೀ ದಾನವಸಂಹಂತ್ರೀ ದನುಜ್ಯೇಷ್ಠನಿಷೂದಿನೀ ।\\
ದಾರಾಪತ್ಯಪ್ರದಾ ನಿತ್ಯಾ ಶಂಕರಾರ್ದ್ಧಾಂಗಧಾರಿಣೀ ॥೯೨॥

ದಿವ್ಯಾಂಗೀ ದೇವಮಾತಾ ಚ ದೇವದುಷ್ಟವಿನಾಶಿನೀ ।\\
ದೀನದುಃಖಹರಾ ದೀನತಾಪನಿರ್ಮೂಲಕಾರಿಣೀ ॥೯೩॥

ದೀನಮಾತಾ ದೀನಸೇವ್ಯಾ ದೀನದಂಭವಿನಾಶಿನೀ ।\\
ದನುಜಧ್ವಂಸಿನೀ ದೇವೀ ದೇವಕೀ ದೇವವಲ್ಲಭಾ ॥೯೪॥

ದಾನವಾರಿಪ್ರಿಯಾ ದೀರ್ಘಾ ದಾನವಾರಿಪ್ರಪೂಜಿತಾ ।\\
ದೀರ್ಘಸ್ವರಾ ದೀರ್ಘತನುರ್ದ್ದೀರ್ಘದುರ್ಗತಿನಾಶಿನೀ ॥೯೫॥

ದೀರ್ಘನೇತ್ರಾ ದೀರ್ಘಚಕ್ಷುರ್ದ್ದೀರ್ಘಕೇಶೀ ದಿಗಂಬರಾ ।\\
ದಿಗಂಬರಪ್ರಿಯಾ ದಾಂತಾ ದಿಗಂಬರಸ್ವರೂಪಿಣೀ ॥೯೬॥

ದುಃಖಹೀನಾ ದುಃಖಹರಾ ದುಃಖಸಾಗರತಾರಿಣೀ ।\\
ದುಃಖದಾರಿದ್ರ್ಯಶಮನೀ ದುಃಖದಾರಿದ್ರ್ಯಕಾರಿಣೀ ॥೯೭॥

ದುಃಖದಾ ದುಸ್ಸಹಾ ದುಷ್ಟಖಂಡನೈಕಸ್ವರೂಪಿಣೀ ।\\
ದೇವವಾಮಾ ದೇವಸೇವ್ಯಾ ದೇವಶಕ್ತಿಪ್ರದಾಯಿನೀ ॥೯೮॥

ದಾಮಿನೀ ದಾಮಿನೀಪ್ರೀತಾ ದಾಮಿನೀಶತಸುಂದರೀ ।\\
ದಾಮಿನೀಶತಸಂಸೇವ್ಯಾ ದಾಮಿನೀದಾಮಭೂಷಿತಾ ॥೯೯॥

ದೇವತಾಭಾವಸಂತುಷ್ಟಾ ದೇವತಾಶತಮಧ್ಯಗಾ ।\\
ದಯಾರ್ದ್ದರಾ ಚ ದಯಾರೂಪಾ ದಯಾದಾನಪರಾಯಣಾ ॥೧೦೦॥

ದಯಾಶೀಲಾ ದಯಾಸಾರಾ ದಯಾಸಾಗರಸಂಸ್ಥಿತಾ ।\\
ದಶವಿದ್ಯಾತ್ಮಿಕಾ ದೇವೀ ದಶವಿದ್ಯಾಸ್ವರೂಪಿಣೀ ॥೧೦೧॥

ಧರಣೀ ಧನದಾ ಧಾತ್ರೀ ಧನ್ಯಾ ಧನ್ಯಪರಾ ಶಿವಾ ।\\
ಧರ್ಮರೂಪಾ ಧನಿಷ್ಠಾ ಚ ಧೇಯಾ ಚ ಧೀರಗೋಚರಾ ॥೧೦೨॥

ಧರ್ಮರಾಜೇಶ್ವರೀ ಧರ್ಮಕರ್ಮರೂಪಾ ಧನೇಶ್ವರೀ ।\\
ಧನುರ್ವಿದ್ಯಾ ಧನುರ್ಗಮ್ಯಾ ಧನುರ್ದ್ಧರವರಪ್ರದಾ ॥೧೦೩॥

ಧರ್ಮಶೀಲಾ ಧರ್ಮಲೀಲಾ ಧರ್ಮಕರ್ಮವಿವರ್ಜಿತಾ ।\\
ಧರ್ಮದಾ ಧರ್ಮನಿರತಾ ಧರ್ಮಪಾಖಂಡಖಂಡಿನೀ ॥೧೦೪॥

ಧರ್ಮೇಶೀ ಧರ್ಮರೂಪಾ ಚ ಧರ್ಮರಾಜವರಪ್ರದಾ ।\\
ಧರ್ಮಿಣೀ ಧರ್ಮಗೇಹಸ್ಥಾ ಧರ್ಮಾಧರ್ಮಸ್ವರೂಪಿಣೀ ॥೧೦೫॥

ಧನದಾ ಧನದಪ್ರೀತಾ ಧನಧಾನ್ಯಸಮೃದ್ಧಿದಾ ।\\
ಧನಧಾನ್ಯಸಮೃದ್ಧಿಸ್ಥಾ ಧನಧಾನ್ಯವಿನಾಶಿನೀ ॥೧೦೬॥

ಧರ್ಮನಿಷ್ಠಾ ಧರ್ಮಧೀರಾ ಧರ್ಮಮಾರ್ಗರತಾ ಸದಾ ।\\
ಧರ್ಮಬೀಜಕೃತಸ್ಥಾನಾ ಧರ್ಮಬೀಜಸುರಕ್ಷಿಣೀ ॥೧೦೭॥

ಧರ್ಮಬೀಜೇಶ್ವರೀ ಧರ್ಮಬೀಜರೂಪಾ ಚ ಧರ್ಮಗಾ ।\\
ಧರ್ಮಬೀಜಸಮುದ್ಭೂತಾ ಧರ್ಮಬೀಜಸಮಾಶ್ರಿತಾ ॥೧೦೮॥

ಧರಾಧರಪತಿಪ್ರಾಣಾ ಧರಾಧರಪತಿಸ್ತುತಾ ।\\
ಧರಾಧರೇಂದ್ರತನುಜಾ ಧರಾಧರೇಂದ್ರವಂದಿತಾ ॥೧೦೯॥

ಧರಾಧರೇಂದ್ರಗೇಹಸ್ಥಾ ಧರಾಧರೇಂದ್ರಪಾಲಿನೀ ।\\
ಧರಾಧರೇಂದ್ರಸರ್ವಾರ್ತಿನಾಶಿನೀ ಧರ್ಮಪಾಲಿನೀ ॥೧೧೦॥

ನವೀನಾ ನಿರ್ಮ್ಮಲಾ ನಿತ್ಯಾ ನಾಗರಾಜಪ್ರಪೂಜಿತಾ ।\\
ನಾಗೇಶ್ವರೀ ನಾಗಮಾತಾ ನಾಗಕನ್ಯಾ ಚ ನಗ್ನಿಕಾ ॥೧೧೧॥

ನಿರ್ಲೇಪಾ ನಿರ್ವಿಕಲ್ಪಾ ಚ ನಿರ್ಲೋಮಾ ನಿರುಪದ್ರವಾ ।\\
ನಿರಾಹಾರಾ ನಿರಾಕಾರಾ ನಿರಂಜನಸ್ವರೂಪಿಣೀ ॥೧೧೨॥

ನಾಗಿನೀ ನಾಗವಿಭವಾ ನಾಗರಾಜಪರಿಸ್ತುತಾ ।\\
ನಾಗರಾಜಗುಣಜ್ಞಾ ಚ ನಾಗರಾಜಸುಖಪ್ರದಾ ॥೧೧೩॥

ನಾಗಲೋಕಗತಾ ನಿತ್ಯಂ ನಾಗಲೋಕನಿವಾಸಿನೀ ।\\
ನಾಗಲೋಕೇಶ್ವರೀ ನಾಗಭಾಗಿನೀ ನಾಗಪೂಜಿತಾ ॥೧೧೪॥

ನಾಗಮಧ್ಯಸ್ಥಿತಾ ನಾಗಮೋಹಸಂಕ್ಷೋಭದಾಯಿನೀ ।\\
ನೃತ್ಯಪ್ರಿಯಾ ನೃತ್ಯವತೀ ನೃತ್ಯಗೀತಪರಾಯಣಾ ॥೧೧೫॥

ನೃತ್ಯೇಶ್ವರೀ ನರ್ತಕೀ ಚ ನೃತ್ಯರೂಪಾ ನಿರಾಶ್ರಯಾ ।\\
ನಾರಾಯಣೀ ನರೇಂದ್ರಸ್ಥಾ ನರಮುಂಡಾಸ್ಥಿಮಾಲಿನೀ ॥೧೧೬॥

ನರಮಾಂಸಪ್ರಿಯಾ ನಿತ್ಯಾ ನರರಕ್ತಪ್ರಿಯಾ ಸದಾ ।\\
ನರರಾಜೇಶ್ವರೀ ನಾರೀರೂಪಾ ನಾರೀಸ್ವರೂಪಿಣೀ ॥೧೧೭॥

ನಾರೀಗಣಾರ್ಚಿತಾ ನಾರೀಮಧ್ಯಗಾ ನೂತನಾಂಬರಾ ।\\
ನರ್ಮದಾ ಚ ನದೀರೂಪಾ ನದೀಸಂಗಮಸಂಸ್ಥಿತಾ ॥೧೧೮॥

ನರ್ಮದೇಶ್ವರಸಂಪ್ರೀತಾ ನರ್ಮದೇಶ್ವರರೂಪಿಣೀ ।\\
ಪದ್ಮಾವತೀ ಪದ್ಮಮುಖೀ ಪದ್ಮಕಿಂಜಲ್ಕವಾಸಿನೀ ॥೧೧೯॥

ಪಟ್ಟವಸ್ತ್ರಪರೀಧಾನಾ ಪದ್ಮರಾಗವಿಭೂಷಿತಾ ।\\
ಪರಮಾ ಪ್ರೀತಿದಾ ನಿತ್ಯಂ ಪ್ರೇತಾಸನನಿವಾಸಿನೀ ॥೧೨೦॥

ಪರಿಪೂರ್ಣರಸೋನ್ಮತ್ತಾ ಪ್ರೇಮವಿಹ್ವಲವಲ್ಲಭಾ ।\\
ಪವಿತ್ರಾಸವನಿಷ್ಪೂತಾ ಪ್ರೇಯಸೀ ಪರಮಾತ್ಮಿಕಾ ॥೧೨೧॥

ಪ್ರಿಯವ್ರತಪರಾ ನಿತ್ಯಂ ಪರಮಪ್ರೇಮದಾಯಿನೀ ।\\
ಪುಷ್ಪಪ್ರಿಯಾ ಪದ್ಮಕೋಶಾ ಪದ್ಮಧರ್ಮನಿವಾಸಿನೀ ॥೧೨೨॥

ಫೇತ್ಕಾರಿಣೀ ತಂತ್ರರೂಪಾ ಫೇರುಫೇರವನಾದಿನೀ ।\\
ವಂಶಿನೀ ವಂಶರೂಪಾ ಚ ಬಗಲಾ ವಾಮರೂಪಿಣೀ ॥೧೨೩॥

ವಾಙ್ಮಯೀ ವಸುಧಾ ಧೃಷ್ಯಾ ವಾಗ್ಭವಾಖ್ಯಾ ವರಾ ನರಾ ।\\
ಬುದ್ಧಿದಾ ಬುದ್ಧಿರೂಪಾ ಚ ವಿದ್ಯಾ ವಾದಸ್ವರೂಪಿಣೀ ॥೧೨೪॥

ಬಾಲಾ ವೃದ್ಧಮಯೀರೂಪಾ ವಾಣೀ ವಾಕ್ಯನಿವಾಸಿನೀ ।\\
ವರುಣಾ ವಾಗ್ವತೀ ವೀರಾ ವೀರಭೂಷಣಭೂಷಿತಾ ॥೧೨೫॥

ವೀರಭದ್ರಾರ್ಚಿತಪದಾ ವೀರಭದ್ರಪ್ರಸೂರಪಿ ।\\
ವೇದಮಾರ್ಗರತಾ ವೇದಮಂತ್ರರೂಪಾ ವಷಟ್ ಪ್ರಿಯಾ ॥೧೨೬॥

ವೀಣಾವಾದ್ಯಸಮಾಯುಕ್ತಾ ವೀಣಾವಾದ್ಯಪರಾಯಣಾ ।\\
ವೀಣಾರವಾ ತಥಾ ವೀಣಾಶಬ್ದರೂಪಾ ಚ ವೈಷ್ಣವೀ ॥೧೨೭॥

ವೈಷ್ಣವಾಚಾರನಿರತಾ ವೈಷ್ಣವಾಚಾರತತ್ಪರಾ ।\\
ವಿಷ್ಣುಸೇವ್ಯಾ ವಿಷ್ಣುಪತ್ನೀ ವಿಷ್ಣುರೂಪಾ ವರಾನನಾ ॥೧೨೮॥

ವಿಶ್ವೇಶ್ವರೀ ವಿಶ್ವಮಾತಾ ವಿಶ್ವನಿರ್ಮಾಣಕಾರಿಣೀ ।\\
ವಿಶ್ವರೂಪಾ ಚ ವಿಶ್ವೇಶೀ ವಿಶ್ವಸಂಹಾರಕಾರಿಣೀ ॥೧೨೯॥

ಭೈರವೀ ಭೈರವಾರಾಧ್ಯಾ ಭೂತಭೈರವಸೇವಿತಾ ।\\
ಭೈರವೇಶೀ ತಥಾ ಭೀಮಾ ಭೈರವೇಶ್ವರತುಷ್ಟಿದಾ ॥೧೩೦॥

ಭೈರವಾಧಿಶರಮಣೀ ಭೈರವಾಧಿಶಪಾಲಿನೀ ।\\
ಭೀಮೇಶ್ವರೀ ಭೀಮಮಾತಾ ಭೀಮಶಬ್ದಪರಾಯಣಾ ॥೧೩೧॥

ಭೀಮರೂಪಾ ಚ ಭೀಮೇಶೀ ಭೀಮಾ ಭೀಮವರಪ್ರದಾ ।\\
ಭೀಮಪೂಜಿತಪಾದಾಬ್ಜಾ ಭೀಮಭೈರವಪಾಲಿನೀ ॥೧೩೨॥

ಭೀಮಾಸುರಧ್ವಂಸಕರೀ ಭೀಮದುಷ್ಟವಿನಾಶಿನೀ ।\\
ಭುವನಾ ಭುವನಾರಾಧ್ಯಾ ಭವಾನೀ ಭೂತಿದಾ ಸದಾ ॥೧೩೩॥

ಭಯದಾ ಭಯಹಂತ್ರೀ ಚ ಅಭಯಾ ಭಯರೂಪಿಣೀ ।\\
ಭೀಮನಾದಾ ವಿಹ್ವಲಾ ಚ ಭಯಭೀತಿವಿನಾಶಿನೀ ॥೧೩೪॥

ಮತ್ತಾ ಪ್ರಮತ್ತರೂಪಾ ಚ ಮದೋನ್ಮತ್ತಸ್ವರೂಪಿಣೀ ।\\
ಮಾನ್ಯಾ ಮನೋಜ್ಞಾ ಮಾನಾ ಚ ಮಂಗಲಾ ಚ ಮನೋಹರಾ ॥೧೩೫॥

ಮಾನನೀಯಾ ಮಹಾಪೂಜ್ಯಾ ಮಹಾಮಹಿಷಮರ್ದ್ದಿನೀ ।\\
ಮಹಿಷಾಸುರಹಂತ್ರೀ ಚ ಮಾತಂಗೀ ಮಯವಾಸಿನೀ ॥೧೩೬॥

ಮಾಧ್ವೀ ಮಧುಮಯೀ ಮುದ್ರಾ ಮುದ್ರಿಕಾ ಮಂತ್ರರೂಪಿಣೀ ।\\
ಮಹಾವಿಶ್ವೇಶ್ವರೀ ದೂತೀ ಮೌಲಿಚಂದ್ರಪ್ರಕಾಶಿನೀ ॥೧೩೭॥

ಯಶಃಸ್ವರೂಪಿಣೀ ದೇವೀ ಯೋಗಮಾರ್ಗಪ್ರದಾಯಿನೀ ।\\
ಯೋಗಿನೀ ಯೋಗಗಮ್ಯಾ ಚ ಯಾಮ್ಯೇಶೀ ಯೋಗರೂಪಿಣೀ ॥೧೩೮॥

ಯಜ್ಞಾಂಗೀ ಚ ಯೋಗಮಯೀ ಜಪರೂಪಾ ಜಪಾತ್ಮಿಕಾ ।\\
ಯುಗಾಖ್ಯಾ ಚ ಯುಗಾಂತಾ ಚ ಯೋನಿಮಂಡಲವಾಸಿನೀ ॥೧೩೯॥

ಅಯೋನಿಜಾ ಯೋಗನಿದ್ರಾ ಯೋಗಾನಂದಪ್ರದಾಯಿನೀ ।\\
ರಮಾ ರತಿಪ್ರಿಯಾ ನಿತ್ಯಂ ರತಿರಾಗವಿವರ್ದ್ಧಿನೀ ॥೧೪೦॥

ರಮಣೀ ರಾಸಸಂಭೂತಾ ರಮ್ಯಾ ರಾಸಪ್ರಿಯಾ ರಸಾ ।\\
ರಣೋತ್ಕಂಠಾ ರಣಸ್ಥಾ ಚ ವರಾ ರಂಗಪ್ರದಾಯಿನೀ ॥೧೪೧॥

ರೇವತೀ ರಣಜೈತ್ರೀ ಚ ರಸೋದ್ಭೂತಾ ರಣೋತ್ಸವಾ ।\\
ಲತಾ ಲಾವಣ್ಯರೂಪಾ ಚ ಲವಣಾಬ್ಧಿಸ್ವರೂಪಿಣೀ ॥೧೪೨॥

ಲವಂಗಕುಸುಮಾರಾಧ್ಯಾ ಲೋಲಜಿಹ್ವಾ ಚ ಲೇಲಿಹಾ ।\\
ವಶಿನೀ ವನಸಂಸ್ಥಾ ಚ ವನಪುಷ್ಪಪ್ರಿಯಾ ವರಾ ॥೧೪೩॥

ಪ್ರಾಣೇಶ್ವರೀ ಬುದ್ಧಿರೂಪಾ ಬುದ್ಧಿದಾತ್ರೀ ಬುಧಾತ್ಮಿಕಾ ।\\
ಶಮನೀ ಶ್ವೇತವರ್ಣಾ ಚ ಶಾಂಕರೀ ಶಿವಭಾಷಿಣೀ ॥೧೪೪॥

ಶ್ಯಾಮ್ಯರೂಪಾ ಶಕ್ತಿರೂಪಾ ಶಕ್ತಿಬಿಂದುನಿವಾಸಿನೀ ।\\
ಸರ್ವೇಶ್ವರೀ ಸರ್ವದಾತ್ರೀ ಸರ್ವಮಾತಾ ಚ ಶರ್ವರೀ ॥೧೪೫॥

ಶಾಂಭವೀ ಸಿದ್ಧಿದಾ ಸಿದ್ಧಾ ಸುಷುಮ್ನಾ ಸುರಭಾಸಿನೀ ।\\
ಸಹಸ್ರದಲಮಧ್ಯಸ್ಥಾ ಸಹಸ್ರದಲವರ್ತ್ತಿನೀ ॥೧೪೬॥

ಹರಪ್ರಿಯಾ ಹರಧ್ಯೇಯಾ ಹೂಁಕಾರಬೀಜರೂಪಿಣೀ ।\\
ಲಂಕೇಶ್ವರೀ ಚ ತರಲಾ ಲೋಮಮಾಂಸಪ್ರಪೂಜಿತಾ ॥೧೪೭॥

ಕ್ಷೇಮ್ಯಾ ಕ್ಷೇಮಕರೀ ಕ್ಷಾಮಾ ಕ್ಷೀರಬಿಂದುಸ್ವರೂಪಿಣೀ ।\\
ಕ್ಷಿಪ್ತಚಿತ್ತಪ್ರದಾ ನಿತ್ಯಂ ಕ್ಷೌಮವಸ್ತ್ರವಿಲಾಸಿನೀ ॥೧೪೮॥

ಛಿನ್ನಾ ಚ ಚ್ಛಿನ್ನರೂಪಾ ಚ ಕ್ಷುಧಾ ಕ್ಷೌತ್ಕಾರರೂಪಿಣೀ ।\\
ಸರ್ವವರ್ಣಮಯೀ ದೇವೀ ಸರ್ವಸಂಪತ್ಪ್ರದಾಯಿನೀ ॥೧೪೯॥

ಸರ್ವಸಂಪತ್ಪ್ರದಾತ್ರೀ ಚ ಸಂಪದಾಪದ್ವಿಭೂಷಿತಾ ।\\
ಸತ್ತ್ವರೂಪಾ ಚ ಸರ್ವಾರ್ಥಾ ಸರ್ವದೇವಪ್ರಪೂಜಿತಾ ॥೧೫೦॥

ಸರ್ವೇಶ್ವರೀ ಸರ್ವಮಾತಾ ಸರ್ವಜ್ಞಾ ಸುರಸೃತ್ಮಿಕಾ ।\\
ಸಿಂಧುರ್ಮಂದಾಕಿನೀ ಗಂಗಾ ನದೀಸಾಗರರೂಪಿಣೀ ॥೧೫೧॥

ಸುಕೇಶೀ ಮುಕ್ತಕೇಶೀ ಚ ಡಾಕಿನೀ ವರವರ್ಣಿನೀ ।\\
ಜ್ಞಾನದಾ ಜ್ಞಾನಗಗನಾ ಸೋಮಮಂಡಲವಾಸಿನೀ ॥೧೫೨॥

ಆಕಾಶನಿಲಯಾ ನಿತ್ಯಾ ಪರಮಾಕಾಶರೂಪಿಣೀ ।\\
ಅನ್ನಪೂರ್ಣಾ ಮಹಾನಿತ್ಯಾ ಮಹಾದೇವರಸೋದ್ಭವಾ ॥೧೫೩॥

ಮಂಗಲಾ ಕಾಲಿಕಾ ಚಂಡಾ ಚಂಡನಾದಾತಿಭೀಷಣಾ ।\\
ಚಂಡಾಸುರಸ್ಯ ಮಥಿನೀ ಚಾಮುಂಡಾ ಚಪಲಾತ್ಮಿಕಾ ॥೧೫೪॥

ಚಂಡೀ ಚಾಮರಕೇಶೀ ಚ ಚಲತ್ಕುಂಡಲಧಾರಿಣೀ ।\\
ಮುಂಡಮಾಲಾಧರಾ ನಿತ್ಯಾ ಖಂಡಮುಂಡವಿಲಾಸಿನೀ ॥೧೫೫॥

ಖಡ್ಗಹಸ್ತಾ ಮುಂಡಹಸ್ತಾ ವರಹಸ್ತಾ ವರಪ್ರದಾ ।\\
ಅಸಿಚರ್ಮಧರಾ ನಿತ್ಯಾ ಪಾಶಾಂಕುಶಧರಾ ಪರಾ ॥೧೫೬॥

ಶೂಲಹಸ್ತಾ ಶಿವಹಸ್ತಾ ಘಂಟಾನಾದವಿಲಾಸಿನೀ ।\\
ಧನುರ್ಬಾಣಧರಾಽಽದಿತ್ಯಾ ನಾಗಹಸ್ತಾ ನಗಾತ್ಮಜಾ ॥೧೫೭॥

ಮಹಿಷಾಸುರಹಂತ್ರೀ ಚ ರಕ್ತಬೀಜವಿನಾಶಿನೀ ।\\
ರಕ್ತರೂಪಾ ರಕ್ತಗಾ ಚ ರಕ್ತಹಸ್ತಾ ಭಯಪ್ರದಾ ॥೧೫೮॥

ಅಸಿತಾ ಚ ಧರ್ಮಧರಾ ಪಾಶಾಂಕುಶಧರಾ ಪರಾ ।\\
ಧನುರ್ಬಾಣಧರಾ ನಿತ್ಯಾ ಧೂಮ್ರಲೋಚನನಾಶಿನೀ ॥೧೫೯॥

ಪರಸ್ಥಾ ದೇವತಾಮೂರ್ತಿಃ ಶರ್ವಾಣೀ ಶಾರದಾ ಪರಾ ।\\
ನಾನಾವರ್ಣವಿಭೂಷಾಂಗೀ ನಾನಾರಾಗಸಮಾಪಿನೀ ॥೧೬೦॥

ಪಶುವಸ್ತ್ರಪರೀಧಾನಾ ಪುಷ್ಪಾಯುಧಧರಾ ಪರಾ ।\\
ಮುಕ್ತರಂಜಿತಮಾಲಾಢ್ಯಾ ಮುಕ್ತಾಹಾರವಿಲಾಸಿನೀ ॥೧೬೧॥

ಸ್ವರ್ಣಕುಂಡಲಭೂಷಾ ಚ ಸ್ವರ್ಣಸಿಂಹಾಸನಸ್ಥಿತಾ ।\\
ಸುಂದರಾಂಗೀ ಸುವರ್ಣಾಭಾ ಶಾಂಭವೀ ಶಕಟಾತ್ಮಿಕಾ ॥೧೬೨॥

ಸರ್ವಲೋಕೇಶವಿದ್ಯಾ ಚ ಮೋಹಸಮ್ಮೋಹಕಾರಿಣೀ ।\\
ಶ್ರೇಯಸೀ ಸೃಷ್ಟಿರೂಪಾ ಚ ಚ್ಛಿನ್ನಚ್ಛದ್ಮಮಯೀ ಚ್ಛಲಾ ॥೧೬೩॥

ಛಿನ್ನಮುಂಡಧರಾ ನಿತ್ಯಾ ನಿತ್ಯಾನಂದವಿಧಾಯಿನೀ ।\\
ನಂದಾ ಪೂರ್ಣಾ ಚ ರಿಕ್ತಾ ಚ ತಿಥಯಃ ಪೂರ್ಣಷೋಡಶೀ ॥೧೬೪॥

ಕುಹೂಃ ಸಂಕ್ರಾಂತಿರೂಪಾ ಚ ಪಂಚಪರ್ವವಿಲಾಸಿನೀ ।\\
ಪಂಚಬಾಣಧರಾ ನಿತ್ಯಾ ಪಂಚಮಪ್ರೀತಿದಾ ಪರಾ ॥೧೬೫॥

ಪಂಚಪತ್ರಾಭಿಲಾಷಾ ಚ ಪಂಚಾಮೃತವಿಲಾಸಿನೀ ।\\
ಪಂಚಾಲೀ ಪಂಚಮೀ ದೇವೀ ಪಂಚರಕ್ತಪ್ರಸಾರಿಣೀ ॥೧೬೬॥

ಪಂಚಬಾಣಧರಾ ನಿತ್ಯಾ ನಿತ್ಯದಾತ್ರೀ ದಯಾಪರಾ ।\\
ಪಲಲಾದಿಪ್ರಿಯಾ ನಿತ್ಯಾಽಪಶುಗಮ್ಯಾ ಪರೇಶಿತಾ ॥೧೬೭॥

ಪರಾ ಪರರಹಸ್ಯಾ ಚ ಪರಮಪ್ರೇಮವಿಹ್ವಲಾ ।\\
ಕುಲಿನಾ ಕೇಶಿಮಾರ್ಗಸ್ಥಾ ಕುಲಮಾರ್ಗಪ್ರಕಾಶಿನೀ ॥೧೬೮॥

ಕುಲಾಕುಲಸ್ವರೂಪಾ ಚ ಕುಲಾರ್ಣವಮಯೀ ಕುಲಾ ।\\
ರುಕ್ಮಾ ಚ ಕಾಲರೂಪಾ ಚ ಕಾಲಕಂಪನಕಾರಿಣೀ ॥೧೬೯॥

ವಿಲಾಸರೂಪಿಣೀ ಭದ್ರಾ ಕುಲಾಕುಲನಮಸ್ಕೃತಾ ।\\
ಕುಬೇರವಿತ್ತಧಾತ್ರೀ ಚ ಕುಮಾರಜನನೀ ಪರಾ ॥೧೭೦॥

ಕುಮಾರೀರೂಪಸಂಸ್ಥಾ ಚ ಕುಮಾರೀಪೂಜನಾಂಬಿಕಾ ।\\
ಕುರಂಗನಯನಾ ದೇವೀ ದಿನೇಶಾಸ್ಯಾಽಪರಾಜಿತಾ ॥೧೭೧॥

ಕುಂಡಲೀಕದಲೀ ಸೇನಾ ಕುಮಾರ್ಗರಹಿತಾ ವರಾ ।\\
ಅನತರೂಪಾಽನಂತಸ್ಥಾ ಆನಂದಸಿಂಧುವಾಸಿನೀ ॥೧೭೨॥

ಇಲಾಸ್ವರೂಪಿಣೀ ದೇವೀ ಇಈಭೇದಭಯಂಕರೀ ।\\
ಇಡಾ ಚ ಪಿಂಗಲಾ ನಾಡೀ ಇಕಾರಾಕ್ಷರರೂಪಿಣೀ ॥೧೭೩॥

ಉಮಾ ಚೋತ್ಪತ್ತಿರೂಪಾ ಚ ಉಚ್ಚಭಾವವಿನಾಶಿನೀ ।\\
ಋಗ್ವೇದಾ ಚ ನಿರಾರಾಧ್ಯಾ ಯಜುರ್ವೇದಪ್ರಪೂಜಿತಾ ॥೧೭೪॥

ಸಾಮವೇದೇನ ಸಂಗೀತಾ ಅಥರ್ವವೇದಭಾಷಿಣೀ ।\\
ಋಕಾರರೂಪಿಣೀ ಋಕ್ಷಾ ನಿರಕ್ಷರಸ್ವರೂಪಿಣೀ ॥೧೭೫॥

ಅಹಿದುರ್ಗಾಸಮಾಚಾರಾ ಇಕಾರಾರ್ಣಸ್ವರೂಪಿಣೀ ।\\
ಓಂಕಾರಾ ಪ್ರಣವಸ್ಥಾ ಚ ಓಂಕಾರಾದಿಸ್ವರೂಪಿಣೀ ॥೧೭೬॥

ಅನುಲೋಮವಿಲೋಮಸ್ಥಾ ಥಕಾರವರ್ಣಸಂಭವಾ ।\\
ಪಂಚಾಶದ್ವರ್ಣಬೀಜಾಢ್ಯಾ ಪಂಚಾಶನ್ಮುಂಡಮಾಲಿಕಾ ॥೧೭೭॥

ಪ್ರತ್ಯೇಕಾ ದಶಸಂಖ್ಯಾ ಚ ಷೋಡಶೀ ಚ್ಛಿನ್ನಮಸ್ತಕಾ ।\\
ಷಡಂಗಯುವತೀಪೂಜ್ಯಾ ಷಡಂಗರೂಪವರ್ಜಿತಾ ॥೧೭೮॥

ಷಡ್ವಕ್ತ್ರಸಂಶ್ರಿತಾ ನಿತ್ಯಾ ವಿಶ್ವೇಶೀ ಖಡ್ಗದಾಲಯಾ ।\\
ಮಾಲಾಮಂತ್ರಮಯೀ ಮಂತ್ರಜಪಮಾತಾ ಮದಾಲಸಾ ॥೧೭೯॥

ಸರ್ವವಿಶ್ವೇಶ್ವರೀ ಶಕ್ತಿಃ ಸರ್ವಾನಂದಪ್ರದಾಯಿನೀ ।\\
ಇತಿ ಶ್ರೀಚ್ಛಿನ್ನಮಸ್ತಾಯಾ ನಾಮಸಹಸ್ರಮುತ್ತಮಂ ॥೧೮೦॥

ಪೂಜಾಕ್ರಮೇಣ ಕಥಿತಂ ಸಾಧಕಾನಾಂ ಸುಖಾವಹಂ ।\\
ಗೋಪನೀಯಂ ಗೋಪನೀಯಂ ಗೋಪನೀಯಂ ನ ಸಂಶಯಃ ॥೧೮೧॥

ಅರ್ದ್ಧರಾತ್ರೇ ಮುಕ್ತಕೇಶೋ ಭಕ್ತಿಯುಕ್ತೋ ಭವೇನ್ನರಃ ।\\
ಜಪಿತ್ವಾ ಪೂಜಯಿತ್ವಾ ಚ ಪಠೇನ್ನಾಮಸಹಸ್ರಕಂ ॥೧೮೨॥

ವಿದ್ಯಾಸಿದ್ಧಿರ್ಭವೇತ್ತಸ್ಯ ಷಣ್ಮಾಸಾಭ್ಯಾಸಯೋಗತಃ ।\\
ಯೇನ ಕೇನ ಪ್ರಕಾರೇಣ ದೇವೀಭಕ್ತಿಪರೋ ಭವೇತ್ ॥೧೮೩॥

ಅಖಿಲಾನ್ಸ್ತಂಭಯೇಲ್ಲೋಕಾಂರಾಜ್ಞೋಽಪಿ ಮೋಹಯೇತ್ಸದಾ ।\\
ಆಕರ್ಷಯೇದ್ದೇವಶಕ್ತಿಂ ಮಾರಯೇದ್ದೇವಿ ವಿದ್ವಿಷಂ ॥೧೮೪॥

ಶತ್ರವೋ ದಾಸತಾಂ ಯಾಂತಿ ಯಾಂತಿ ಪಾಪಾನಿ ಸಂಕ್ಷಯಂ ।\\
ಮೃತ್ಯುಶ್ಚ ಕ್ಷಯತಾಂ ಯಾತಿ ಪಠನಾದ್ಭಾಷಣಾತ್ಪ್ರಿಯೇ ॥೧೮೫॥

ಪ್ರಶಸ್ತಾಯಾಃ ಪ್ರಸಾದೇನ ಕಿಂ ನ ಸಿದ್ಧ್ಯತಿ ಭೂತಲೇ ।\\
ಇದಂ ರಹಸ್ಯಂ ಪರಮಂ ಪರಂ ಸ್ವಸ್ತ್ಯಯನಂ ಮಹತ್ ॥೧೮೬॥

ಧೃತ್ವಾ ಬಾಹೌ ಮಹಾಸಿದ್ಧಿಃ ಪ್ರಾಪ್ಯತೇ ನಾತ್ರ ಸಂಶಯಃ ।\\
ಅನಯಾ ಸದೃಶೀ ವಿದ್ಯಾ ವಿದ್ಯತೇ ನ ಮಹೇಶ್ವರಿ ॥೧೮೭॥

ವಾರಮೇಕಂ ತು ಯೋಽಧೀತೇ ಸರ್ವಸಿದ್ಧೀಶ್ವರೋ ಭವೇತ್ ।\\
ಕುಲವಾರೇ ಕುಲಾಷ್ಟಮ್ಯಾಂ ಕುಹೂಸಂಕ್ರಾಂತಿಪರ್ವಸು ॥೧೮೮॥

ಯಶ್ಚೇಮಂ ಪಠತೇ ವಿದ್ಯಾಂ ತಸ್ಯ ಸಮ್ಯಕ್ಫಲಂ ಶೃಣು ।\\
ಅಷ್ಟೋತ್ತರಶತಂ ಜಪ್ತ್ವಾ ಪಠೇನ್ನಾಮಸಹಸ್ರಕಂ ॥೧೮೯॥

ಭಕ್ತ್ಯಾ ಸ್ತುತ್ವಾ ಮಹಾದೇವಿ ಸರ್ವಪಾಪಾತ್ಪ್ರಮುಚ್ಯತೇ ।\\
ಸರ್ವಪಾಪೈರ್ವಿನಿರ್ಮುಕ್ತಃ ಸರ್ವಸಿದ್ಧೀಶ್ವರೋ ಭವೇತ್ ॥೧೯೦॥

ಅಷ್ಟಮ್ಯಾಂ ವಾ ನಿಶೀಥೇ ಚ ಚತುಷ್ಪಥಗತೋ ನರಃ ।\\
ಮಾಷಭಕ್ತಬಲಿಂ ದತ್ವಾ ಪಠೇನ್ನಾಮಸಹಸ್ರಕಂ ॥೧೯೧॥

ಸುದರ್ಶವಾಮವೇದ್ಯಾಂ ತು ಮಾಸತ್ರಯವಿಧಾನತಃ ।\\
ದುರ್ಜಯಃ ಕಾಮರೂಪಶ್ಚ ಮಹಾಬಲಪರಾಕ್ರಮಃ ॥೧೯೨॥

ಕುಮಾರೀಪೂಜನಂ ನಾಮ ಮಂತ್ರಮಾತ್ರಂ ಪಠೇನ್ನರಃ ।\\
ಏತನ್ಮಂತ್ರಸ್ಯ ಪಠನಾತ್ಸರ್ವಸಿದ್ಧೀಶ್ವರೋ ಭವೇತ್ ॥೧೯೩॥

ಇತಿ ತೇ ಕಥಿತಂ ದೇವಿ ಸರ್ವಸಿದ್ಧಿಪರಂ ನರಃ ।\\
ಜಪ್ತ್ವಾ ಸ್ತುತ್ವಾ ಮಹಾದೇವೀಂ ಸರ್ವಪಾಪೈಃ ಪ್ರಮುಚ್ಯತೇ ॥೧೯೪॥

ನ ಪ್ರಕಾಶ್ಯಮಿದಂ ದೇವಿ ಸರ್ವದೇವನಮಸ್ಕೃತಂ ।\\
ಇದಂ ರಹಸ್ಯಂ ಪರಮಂ ಗೋಪ್ತವ್ಯಂ ಪಶುಸಂಕಟೇ ॥೧೯೫॥

ಇತಿ ಸಕಲವಿಭೂತೇರ್ಹೇತುಭೂತಂ ಪ್ರಶಸ್ತಂ ಪಠತಿ\\
ಯ ಇಹ ಮರ್ತ್ತ್ಯಶ್ಛಿನ್ನಮಸ್ತಾಸ್ತವಂ ಚ ।\\
ಧನದ ಇವ ಧನಾಢ್ಯೋ ಮಾನನೀಯೋ ನೃಪಾಣಾಂ\\
ಸ ಭವತಿ ಚ ಜನಾನಾಮಾಶ್ರಯಃ ಸಿದ್ಧಿವೇತ್ತಾ ॥೧೯೬॥

\authorline{॥ಇತಿ ಶ್ರೀವಿಶ್ವಸಾರತಂತ್ರೇ ಶಿವಪಾರ್ವತೀಸಂವಾದೇ ಶ್ರೀಚ್ಛಿನ್ನಮಸ್ತಾಸಹಸ್ರನಾಮಸ್ತೋತ್ರಂ ಸಂಪೂರ್ಣಂ॥}

%=============================================================================================

\section{ಶ್ರೀಛಿನ್ನಮಸ್ತಾಷ್ಟೋತ್ತರಶತನಾಮಸ್ತೋತ್ರಂ}
\addcontentsline{toc}{section}{ಶ್ರೀಛಿನ್ನಮಸ್ತಾಷ್ಟೋತ್ತರಶತನಾಮಸ್ತೋತ್ರಂ}


ಶ್ರೀಪಾರ್ವತ್ಯುವಾಚ ।\\
ನಾಮ್ನಾಂ ಸಹಸ್ರಮಂ ಪರಮಂ ಛಿನ್ನಮಸ್ತಾಪ್ರಿಯಂ ಶುಭಂ ।\\
ಕಥಿತಂ ಭವತಾ ಶಂಭೋ ಸದ್ಯಃ ಶತ್ರುನಿಕೃಂತನಂ ॥೧॥

ಪುನಃ ಪೃಚ್ಛಾಮ್ಯಹಂ ದೇವ ಕೃಪಾಂ ಕುರು ಮಮೋಪರಿ ।\\
ಸಹಸ್ರನಾಮಪಾಠೇ ಚ ಅಶಕ್ತೋ ಯಃ ಪುಮಾನ್ ಭವೇತ್ ॥೨॥

ತೇನ ಕಿಂ ಪಠ್ಯತೇ ನಾಥ ತನ್ಮೇ ಬ್ರೂಹಿ ಕೃಪಾಮಯ ।\\
ಶ್ರೀ ಸದಾಶಿವ ಉವಾಚ ।\\
ಅಷ್ಟೋತ್ತರಶತಂ ನಾಮ್ನಾಂ ಪಠ್ಯತೇ ತೇನ ಸರ್ವದಾ ॥೩॥

ಸಹಸ್ರ್ನಾಮಪಾಠಸ್ಯ ಫಲಂ ಪ್ರಾಪ್ನೋತಿ ನಿಶ್ಚಿತಂ ।\\
ಓಂ ಅಸ್ಯ ಶ್ರೀಛಿನ್ನಮಸ್ತಾಷ್ಟೋತ್ತರಶತನಾಮಸ್ತೋತ್ರಸ್ಯ ಸದಾಶಿವಋಷಿರನುಷ್ಟುಪ್ ಛಂದಃ ಶ್ರೀಛಿನ್ನಮಸ್ತಾ ದೇವತಾ । ಮಮಸಕಲಸಿದ್ಧಿಪ್ರಾಪ್ತಯೇ ಜಪೇ ವಿನಿಯೋಗಃ ॥

ಓಂ ಛಿನ್ನಮಸ್ತಾ ಮಹಾವಿದ್ಯಾ ಮಹಾಭೀಮಾ ಮಹೋದರೀ ।\\
ಚಂಡೇಶ್ವರೀ ಚಂಡಮಾತಾ ಚಂಡಮುಂಡ್ಪ್ರಭಂಜಿನೀ ॥೪॥

ಮಹಾಚಂಡಾ ಚಂಡರೂಪಾ ಚಂಡಿಕಾ ಚಂಡಖಂಡಿನೀ ।\\
ಕ್ರೋಧಿನೀ ಕ್ರೋಧಜನನೀ ಕ್ರೋಧರೂಪಾ ಕುಹೂ ಕಲಾ ॥೫॥

ಕೋಪಾತುರಾ ಕೋಪಯುತಾ ಜೋಪಸಂಹಾರಕಾರಿಣೀ ।\\
ವಜ್ರವೈರೋಚನೀ ವಜ್ರಾ ವಜ್ರಕಲ್ಪಾ ಚ ಡಾಕಿನೀ ॥೬॥

ಡಾಕಿನೀ ಕರ್ಮನಿರತಾ ಡಾಕಿನೀ ಕರ್ಮಪೂಜಿತಾ ।\\
ಡಾಕಿನೀ ಸಂಗನಿರತಾ ಡಾಕಿನೀ ಪ್ರೇಮಪೂರಿತಾ ॥೭॥

ಖಟ್ವಾಂಗಧಾರಿಣೀ ಖರ್ವಾ ಖಡ್ಗಖಪ್ಪರಧಾರಿಣೀ ।\\
ಪ್ರೇತಾಸನಾ ಪ್ರೇತಯುತಾ ಪ್ರೇತಸಂಗವಿಹಾರಿಣೀ ॥೮॥

ಛಿನ್ನಮುಂಡಧರಾ ಛಿನ್ನಚಂಡವಿದ್ಯಾ ಚ ಚಿತ್ರಿಣೀ ।\\
ಘೋರರೂಪಾ ಘೋರದೃಷ್ಟರ್ಘೋರರಾವಾ ಘನೋವರೀ ॥೯॥

ಯೋಗಿನೀ ಯೋಗನಿರತಾ ಜಪಯಜ್ಞಪರಾಯಣಾ ।\\
ಯೋನಿಚಕ್ರಮಯೀ ಯೋನಿರ್ಯೋನಿಚಕ್ರಪ್ರವರ್ತಿನೀ ॥೧೦॥

ಯೋನಿಮುದ್ರಾಯೋನಿಗಮ್ಯಾ ಯೋನಿಯಂತ್ರನಿವಾಸಿನೀ ।\\
ಯಂತ್ರರೂಪಾ ಯಂತ್ರಮಯೀ ಯಂತ್ರೇಶೀ ಯಂತ್ರಪೂಜಿತಾ ॥೧೧॥

ಕೀರ್ತ್ಯಾ ಕರ್ಪಾದನೀ ಕಾಲೀ ಕಂಕಾಲೀ ಕಲಕಾರಿಣೀ ।\\
ಆರಕ್ತಾ ರಕ್ತನಯನಾ ರಕ್ತಪಾನಪರಾಯಣಾ ॥೧೨॥

ಭವಾನೀ ಭೂತಿದಾ ಭೂತಿರ್ಭೂತಿದಾತ್ರೀ ಚ ಭೈರವೀ ।\\
ಭೈರವಾಚಾರನಿರತಾ ಭೂತಭೈರವಸೇವಿತಾ ॥೧೩॥

ಭೀಮಾ ಭೀಮೇಶ್ವರೀ ದೇವೀ ಭೀಮನಾದಪರಾಯಣಾ ।\\
ಭವಾರಾಧ್ಯಾ ಭವನುತಾ ಭವಸಾಗರತಾರಿಣೀ ॥೧೪॥

ಭದ್ರಕಾಲೀ ಭದ್ರತನುರ್ಭದ್ರರೂಪಾ ಚ ಭದ್ರಿಕಾ ।\\
ಭದ್ರರೂಪಾ ಮಹಾಭದ್ರಾ ಸುಭದ್ರಾ ಭದ್ರಪಾಲಿನೀ ॥೧೫॥

ಸುಭವ್ಯಾ ಭವ್ಯವದನಾ ಸುಮುಖೀ ಸಿದ್ಧಸೇವಿತಾ ।\\
ಸಿದ್ಧಿದಾ ಸಿದ್ಧಿನಿವಹಾ ಸಿದ್ಧಾಸಿದ್ಧನಿಷೇವಿತಾ ॥೧೬॥

ಶುಭದಾ ಶುಭಫ಼್ಗಾ ಶುದ್ಧಾ ಶುದ್ಧಸತ್ವಾಶುಭಾವಹಾ ।\\
ಶ್ರೇಷ್ಠಾ ದೃಷ್ಠಿಮಯೀ ದೇವೀ ದೃಷ್ಠಿಸಂಹಾರಕಾರಿಣೀ ॥೧೭॥

ಶರ್ವಾಣೀ ಸರ್ವಗಾ ಸರ್ವಾ ಸರ್ವಮಂಗಲಕಾರಿಣೀ ।\\
ಶಿವಾ ಶಾಂತಾ ಶಾಂತಿರೂಪಾ ಮೃಡಾನೀ ಮದಾನತುರಾ ॥೧೮॥

ಇತಿ ತೇ ಕಥಿತಂ ದೇವಿ ಸ್ತೋತ್ರಂ ಪರಮದುರ್ಲಭಮಂ ।\\
ಗುಹ್ಯಾದ್ಗುಹ್ಯತರಂ ಗೋಪ್ಯಂ ಗೋಪನಿಯಂ ಪ್ರಯತ್ನತಃ ॥೧೯॥

ಕಿಮತ್ರ ಬಹುನೋಕ್ತೇನ ತ್ವದಗ್ರಂ ಪ್ರಾಣವಲ್ಲಭೇ ।\\
ಮಾರಣಂ ಮೋಹನಂ ದೇವಿ ಹ್ಯುಚ್ಚಾಟನಮತಃ ಪರಮಂ ॥೨೦॥

ಸ್ತಂಭನಾದಿಕಕರ್ಮಾಣಿ ಋದ್ಧಯಃ ಸಿದ್ಧಯೋಽಪಿ ಚ ।\\
ತ್ರಿಕಾಲಪಠನಾದಸ್ಯ ಸರ್ವೇ ಸಿಧ್ಯಂತ್ಯಸಂಶಯಃ ॥೨೧॥

ಮಹೋತ್ತಮಂ ಸ್ತೋತ್ರಮಿದಂ ವರಾನನೇ ಮಯೇರಿತಂ ನಿತ್ಯ ಮನನ್ಯಬುದ್ಧಯಃ ।\\
ಪಠಂತಿ ಯೇ ಭಕ್ತಿಯುತಾ ನರೋತ್ತಮಾ ಭವೇನ್ನ ತೇಷಾಂ ರಿಪುಭಿಃ ಪರಾಜಯಃ ॥೨೨॥

\authorline{ಇತಿ ಶ್ರೀಛಿನ್ನಮಸ್ತಾಷ್ಟೋತ್ತರಶತನಾಮ ಸ್ತೋತ್ರಂ ॥}
%===================================================
\section{ಶ್ರೀತ್ರಿಪುರಭೈರವೀಸಹಸ್ರನಾಮಸ್ತೋತ್ರಂ}
\addcontentsline{toc}{section}{ಶ್ರೀತ್ರಿಪುರಭೈರವೀಸಹಸ್ರನಾಮಸ್ತೋತ್ರಂ}

ಮಹಾಕಾಲಭೈರವ ಉವಾಚ ।\\
ಅಥ ವಕ್ಷ್ಯೇ ಮಹೇಶಾನಿ ದೇವ್ಯಾ ನಾಮಸಹಸ್ರಕಂ ।\\
ಯತ್ಪ್ರಸಾದಾನ್ಮಹಾದೇವಿ ಚತುರ್ವರ್ಗಫಲಲ್ಲಭೇತ್ ॥೧॥

ಸರ್ವರೋಗಪ್ರಶಮನಂ ಸರ್ವಮೃತ್ಯುವಿನಾಶನಂ ।\\
ಸರ್ವಸಿದ್ಧಿಕರಂ ಸ್ತೋತ್ರನ್ನಾತಃ ಪರತಃ ಸ್ತವಃ ॥೨॥

ನಾತಃ ಪರತರಾ ವಿದ್ಯಾ ತೀರ್ತ್ಥನ್ನಾತಃ ಪರಂ ಸ್ಮೃತಂ ।\\
ಯಸ್ಯಾಂ ಸರ್ವಂ ಸಮುತ್ಪನ್ನಯ್ಯಸ್ಯಾಮದ್ಯಾಪಿ ತಿಷ್ಠತಿ ॥೩॥

ಕ್ಷಯಮೇಷ್ಯತಿ ತತ್ಸರ್ವಂ ಲಯಕಾಲೇ ಮಹೇಶ್ವರಿ ।\\
ನಮಾಮಿ ತ್ರಿಪುರಾಂದೇವೀಂಭೈರವೀಂ ಭಯಮೋಚಿನೀಂ ।\\
ಸರ್ವಸಿದ್ಧಿಕರೀಂ ಸಾಕ್ಷಾನ್ಮಹಾಪಾತಕನಾಶಿನೀಂ ॥೪॥

ಅಸ್ಯ ಶ್ರೀತ್ರಿಪುರಭೈರವೀಸಹಸ್ರನಾಮಸ್ತೋತ್ರಸ್ಯ ಭಗವಾನ್ ಋಷಿಃ । ಪಂಕ್ತಿಶ್ಛಂದಃ । ಆದ್ಯಾ ಶಕ್ತಿಃ । ಭಗವತೀ ತ್ರಿಪುರಭೈರವೀ ದೇವತಾ । ಸರ್ವಕಾಮಾರ್ತ್ಥಸಿದ್ಧ್ಯರ್ತ್ಥೇ ಜಪೇ ವಿನಿಯೋಗಃ॥

ಓಂ ತ್ರಿಪುರಾ ಪರಮೇಶಾನೀ ಯೋಗಸಿದ್ಧಿನಿವಾಸಿನೀ ।\\
ಸರ್ವಮಂತ್ರಮಯೀ ದೇವೀ ಸರ್ವಸಿದ್ಧಿಪ್ರವರ್ತ್ತಿನೀ॥

ಸರ್ವಾಧಾರಮಯೀ ದೇವೀ ಸರ್ವಸಂಪತ್ಪ್ರದಾ ಶುಭಾ ।\\
ಯೋಗಿನೀ ಯೋಗಮಾತಾ ಚ ಯೋಗಸಿದ್ಧಿಪ್ರವರ್ತ್ತಿನೀ॥

ಯೋಗಿಧ್ಯೇಯಾ ಯೋಗಮಯೀ ಯೋಗಯೋಗನಿವಾಸಿನೀ ।\\
ಹೇಲಾ ಲೀಲಾ ತಥಾ ಕ್ರೀಡಾ ಕಾಲರೂಪಪ್ರವರ್ತ್ತಿನೀ॥

ಕಾಲಮಾತಾ ಕಾಲರಾತ್ರಿಃ ಕಾಲೀ ಕಾಮಲವಾಸಿನೀ ।\\
ಕಮಲಾ ಕಾಂತಿರೂಪಾ ಚ ಕಾಮರಾಜೇಶ್ವರೀ ಕ್ರಿಯಾ॥

ಕಟುಃ ಕಪಟಕೇಶಾ ಚ ಕಪಟಾ ಕುಲಟಾಕೃತಿಃ ।\\
ಕುಮುದಾ ಚರ್ಚ್ಚಿಕಾ ಕಾಂತಿಃ ಕಾಲರಾತ್ರಿಪ್ರಿಯಾ ಸದಾ॥

ಘೋರಾಕಾರಾ ಘೋರತರಾ ಧರ್ಮಾಧರ್ಮಪ್ರದಾ ಮತಿಃ ।\\
ಘಂಟಾ ಘರ್ಗ್ಘರದಾ ಘಂಟಾ ಘಂಟಾನಾದಪ್ರಿಯಾ ಸದಾ॥

ಸೂಕ್ಷ್ಮಾ ಸೂಕ್ಷ್ಮತರಾ ಸ್ಥೂಲಾ ಅತಿಸ್ಥೂಲಾ ಸದಾ ಮತಿಃ ।\\
ಅತಿಸತ್ಯಾ ಸತ್ಯವತೀ ಸತ್ಯಸಂಕೇತವಾಸಿನೀ॥

ಕ್ಷಮಾ ಭೀಮಾ ತಥಾಽಭೀಮಾ ಭೀಮನಾದಪ್ರವರ್ತ್ತಿನೀ ।\\
ಭ್ರಮರೂಪಾ ಭಯಹರಾ ಭಯದಾ ಭಯನಾಶಿನೀ॥

ಶ್ಮಶಾನವಾಸಿನೀ ದೇವೀ ಶ್ಮಶಾನಾಲಯವಾಸಿನೀ ।\\
ಶವಾಸನಾ ಶವಾಹಾರಾ ಶವದೇಹಾ ಶಿವಾಶಿವಾ॥

ಕಂಠದೇಶಶವಾಹಾರಾ ಶವಕಂಕಣಧಾರಿಣೀ ।\\
ದಂತುರಾ ಸುದತೀ ಸತ್ಯಾ ಸತ್ಯಸಂಕೇತವಾಸಿನೀ॥

ಸತ್ಯದೇಹಾ ಸತ್ಯಹಾರಾ ಸತ್ಯವಾದಿನಿವಾಸಿನೀ ।\\
ಸತ್ಯಾಲಯಾ ಸತ್ಯಸಂಗಾ ಸತ್ಯಸಂಗರಕಾರಿಣೀ॥

ಅಸಂಗಾ ಸಾಂಗರಹಿತಾ ಸುಸಂಗಾ ಸಂಗಮೋಹಿನೀ ।\\
ಮಾಯಾಮತಿರ್ಮಹಾಮಾಯಾ ಮಹಾಮಖವಿಲಾಸಿನೀ॥

ಗಲದ್ರುಧಿರಧಾರಾ ಚ ಮುಖದ್ವಯನಿವಾಸಿನೀ ।\\
ಸತ್ಯಾಯಾಸಾ ಸತ್ಯಸಂಗಾ ಸತ್ಯಸಂಗತಿಕಾರಿಣೀ॥

ಅಸಂಗಾ ಸಂಗನಿರತಾ ಸುಸಂಗಾ ಸಂಗವಾಸಿನೀ ।\\
ಸದಾಸತ್ಯಾ ಮಹಾಸತ್ಯಾ ಮಾಂಸಪಾಶಾ ಸುಮಾಂಸಕಾ॥

ಮಾಂಸಾಹಾರಾ ಮಾಂಸಧರಾ ಮಾಂಸಾಶೀ ಮಾಂಸಭಕ್ಷಕಾ ।\\
ರಕ್ತಪಾನಾ ರಕ್ತರುಚಿರಾ ರಕ್ತಾ ರಕ್ತವಲ್ಲಭಾ॥

ರಕ್ತಾಹಾರಾ ರಕ್ತಪ್ರಿಯಾ ರಕ್ತನಿಂದಕನಾಶಿನೀ ।\\
ರಕ್ತಪಾನಪ್ರಿಯಾ ಬಾಲಾ ರಕ್ತದೇಶಾ ಸುರಕ್ತಿಕಾ॥

ಸ್ವಯಂಭೂಕುಸುಮಸ್ಥಾ ಚ ಸ್ವಯಂಭೂಕುಸುಮೋತ್ಸುಕಾ ।\\
ಸ್ವಯಂಭೂಕುಸುಮಾಹಾರಾ ಸ್ವಯಂಭೂನಿಂದಕಾಸನಾ॥

ಸ್ವಯಂಭೂಪುಷ್ಪಕಪ್ರೀತಾ ಸ್ವಯಂಭೂಪುಷ್ಪಸಂಭವಾ ।\\
ಸ್ವಯಂಭೂಪುಷ್ಪಹಾರಾಢ್ಯಾ ಸ್ವಯಂಭೂನಿಂದಕಾಂತಕಾ॥

ಕುಂಡಗೋಲವಿಲಾಸೀ ಚ ಕುಂಡಗೋಲಸದಾಮತಿಃ ।\\
ಕುಂಡಗೋಲಪ್ರಿಯಕರೀ ಕುಂಡಗೋಲಸಮುದ್ಭವಾ॥

ಶುಕ್ರಾತ್ಮಿಕಾ ಶುಕ್ರಕರಾ ಸುಶುಕ್ರಾ ಚ ಸುಶುಕ್ತಿಕಾ ।\\
ಶುಕ್ರಪೂಜಕಪೂಜ್ಯಾ ಚ ಶುಕ್ರನಿಂದಕನಿಂದಕಾ॥

ರಕ್ತಮಾಲ್ಯಾ ರಕ್ತಪುಷ್ಪಾ ರಕ್ತಪುಷ್ಪಕಪುಷ್ಪಕಾ ।\\
ರಕ್ತಚಂದನಸಿಕ್ತಾಂಗೀ ರಕ್ತಚಂದನನಿಂದಕಾ॥

ಮತ್ಸ್ಯಾ ಮತ್ಸ್ಯಪ್ರಿಯಾ ಮಾನ್ಯಾ ಮತ್ಸ್ಯಭಕ್ಷಾ ಮಹೋದಯಾ ।\\
ಮತ್ಸ್ಯಾಹಾರಾ ಮತ್ಸ್ಯಕಾಮಾ ಮತ್ಸ್ಯನಿಂದಕನಾಶಿನೀ॥

ಕೇಕರಾಕ್ಷೀ ತಥಾ ಕ್ರೂರಾ ಕ್ರೂರಸೈನ್ಯವಿನಾಶಿನೀ ।\\
ಕ್ರೂರಾಂಗೀ ಕುಲಿಶಾಂಗೀ ಚ ಚಕ್ರಾಂಗೀ ಚಕ್ರಸಂಭವಾ॥

ಚಕ್ರದೇಹಾ ಚಕ್ರಹಾರಾ ಚಕ್ರಕಂಕಾಲವಾಸಿನೀ ।\\
ನಿಮ್ನನಾಭೀ ಭೀತಿಹರಾ ಭಯದಾ ಭಯಹಾರಿಕಾ॥

ಭಯಪ್ರದಾ ಭಯಭೀತಾ ಅಭೀಮಾ ಭೀಮನಾದಿನೀ ।\\
ಸುಂದರೀ ಶೋಭನಾ ಸತ್ಯಾ ಕ್ಷೇಮ್ಯಾ ಕ್ಷೇಮಕರೀ ತಥಾ॥

ಸಿಂದೂರಾಂಚಿತಸಿಂದೂರಾ ಸಿಂದೂರಸದೃಶಾಕೃತಿಃ ।\\
ರಕ್ತಾರಂಜಿತನಾಸಾ ಚ ಸುನಾಸಾ ನಿಮ್ನನಾಸಿಕಾ॥

ಖರ್ವಾ ಲಂಬೋದರೀ ದೀರ್ಗ್ಘಾ ದೀರ್ಗ್ಘಘೋಣಾ ಮಹಾಕುಚಾ ।\\
ಕುಟಿಲಾ ಚಂಚಲಾ ಚಂಡೀ ಚಂಡನಾದಪ್ರಚಂಡಿಕಾ॥

ಅತಿಚಂಡಾ ಮಹಾಚಂಡಾ ಶ್ರೀಚಂಡಾಚಂಡವೇಗಿನೀ ।\\
ಚಾಂಡಾಲೀ ಚಂಡಿಕಾ ಚಂಡಶಬ್ದರೂಪಾ ಚ ಚಂಚಲಾ॥

ಚಂಪಾ ಚಂಪಾವತೀ ಚೋಸ್ತಾ ತೀಕ್ಷ್ಣಾ ತೀಕ್ಷ್ಣಪ್ರಿಯಾ ಕ್ಷತಿಃ ।\\
ಜಲದಾ ಜಯದಾ ಯೋಗಾ ಜಗದಾನಂದಕಾರಿಣೀ॥

ಜಗದ್ವಂದ್ಯಾ ಜಗನ್ಮಾತಾ ಜಗತೀ ಜಗತಕ್ಷಮಾ ।\\
ಜನ್ಯಾ ಜಯಜನೇತ್ರೀ ಚ ಜಯಿನೀ ಜಯದಾ ತಥಾ॥

ಜನನೀ ಚ ಜಗದ್ಧಾತ್ರೀ ಜಯಾಖ್ಯಾ ಜಯರೂಪಿಣೀ ।\\
ಜಗನ್ಮಾತಾ ಜಗನ್ಮಾನ್ಯಾ ಜಯಶ್ರೀರ್ಜ್ಜಯಕಾರಿಣೀ॥

ಜಯಿನೀ ಜಯಮಾತಾ ಚ ಜಯಾ ಚ ವಿಜಯಾ ತಥಾ ।\\
ಖಡ್ಗಿನೀ ಖಡ್ಗರೂಪಾ ಚ ಸುಖಡ್ಗಾ ಖಡ್ಗಧಾರಿಣೀ॥

ಖಡ್ಗರೂಪಾ ಖಡ್ಗಕರಾ ಖಡ್ಗಿನೀ ಖಡ್ಗವಲ್ಲಭಾ ।\\
ಖಡ್ಗದಾ ಖಡ್ಗಭಾವಾ ಚ ಖಡ್ಗದೇಹಸಮುದ್ಭವಾ॥

ಖಡ್ಗಾ ಖಡ್ಗಧರಾ ಖೇಲಾ ಖಡ್ಗಿನೀ ಖಡ್ಗಮಂಡಿನೀ ।\\
ಶಂಖಿನೀ ಚಾಪಿನೀ ದೇವೀ ವಜ್ರಿಣೀ ಶುಲಿನೀ ಮತಿಃ॥

ಬಲಿನೀ ಭಿಂದಿಪಾಲೀ ಚ ಪಾಶೀ ಚ ಅಂಕುಶೀ ಶರೀ ।\\
ಧನುಷೀ ಚಟಕೀ ಚರ್ಮಾ ದಂತೀ ಚ ಕರ್ಣನಾಲಿಕೀ॥

ಮುಸಲೀ ಹಲರೂಪಾ ಚ ತೂಣೀರಗಣವಾಸಿನೀ ।\\
ತೂಣಾಲಯಾ ತೂಣಹರಾ ತೂಣಸಂಭವರೂಪಿಣೀ॥

ಸುತೂಣೀ ತೂಣಖೇದಾ ಚ ತೂಣಾಂಗೀ ತೂಣವಲ್ಲಭಾ ।\\
ನಾನಾಸ್ತ್ರಧಾರಿಣೀ ದೇವೀ ನಾನಾಶಸ್ತ್ರಸಮುದ್ಭವಾ॥

ಲಾಕ್ಷಾ ಲಕ್ಷಹರಾ ಲಾಭಾ ಸುಲಾಭಾ ಲಾಭನಾಶಿನೀ ।\\
ಲಾಭಹಾರಾ ಲಾಭಕರಾ ಲಾಭಿನೀ ಲಾಭರೂಪಿಣೀ॥

ಧರಿತ್ರೀ ಧನದಾ ಧಾನ್ಯಾ ಧನ್ಯರೂಪಾ ಧರಾ ಧನುಃ ।\\
ಧುರಶಬ್ದಾ ಧುರಾಮಾನ್ಯಾ ಧರಾಂಗೀ ಧನನಾಶಿನೀ॥

ಧನಹಾ ಧನಲಾಭಾ ಚ ಧನಲಭ್ಯಾ ಮಹಾಧನುಃ ।\\
ಅಶಾಂತಾ ಶಾಂತಿರೂಪಾ ಚ ಶ್ವಾಸಮಾರ್ಗನಿವಾಸಿನೀ॥

ಗಗಣಾ ಗಣಸೇವ್ಯಾ ಚ ಗಣಾಂಗಾವಾಗವಲ್ಲಭಾ ।\\
ಗಣದಾ ಗಣಹಾ ಗಮ್ಯಾ ಗಮನಾಗಮಸುಂದರೀ॥

ಗಮ್ಯದಾ ಗಣನಾಶೀ ಚ ಗದಹಾ ಗದವರ್ದ್ಧಿನೀ ।\\
ಸ್ಥೈರ್ಯಾ ಚ ಸ್ಥೈರ್ಯನಾಶಾ ಚ ಸ್ಥೈರ್ಯಾಂತಕರಣೀ ಕುಲಾ॥

ದಾತ್ರೀ ಕರ್ತ್ರೀ ಪ್ರಿಯಾ ಪ್ರೇಮಾ ಪ್ರಿಯದಾ ಪ್ರಿಯವರ್ದ್ಧಿನೀ ।\\
ಪ್ರಿಯಹಾ ಪ್ರಿಯಭವ್ಯಾ ಚ ಪ್ರಿಯಪ್ರೇಮಾಂಘ್ರಿಪಾತನುಃ॥

ಪ್ರಿಯಜಾ ಪ್ರಿಯಭವ್ಯಾ ಚ ಪ್ರಿಯಸ್ಥಾ ಭವನಸ್ಥಿತಾ ।\\
ಸುಸ್ಥಿರಾ ಸ್ಥಿರರೂಪಾ ಚ ಸ್ಥಿರದಾ ಸ್ಥೈರ್ಯಬರ್ಹಿಣೀ॥

ಚಂಚಲಾ ಚಪಲಾ ಚೋಲಾ ಚಪಲಾಂಗನಿವಾಸಿನೀ ।\\
ಗೌರೀ ಕಾಲೀ ತಥಾ ಛಿನ್ನಾ ಮಾಯಾ ಮಾನ್ಯಾ ಹರಪ್ರಿಯಾ॥

ಸುಂದರೀ ತ್ರಿಪುರಾ ಭವ್ಯಾ ತ್ರಿಪುರೇಶ್ವರವಾಸಿನೀ ।\\
ತ್ರಿಪುರನಾಶಿನೀ ದೇವೀ ತ್ರಿಪುರಪ್ರಾಣಹಾರಿಣೀ॥

ಭೈರವೀ ಭೈರವಸ್ಥಾ ಚ ಭೈರವಸ್ಯ ಪ್ರಿಯಾ ತನುಃ ।\\
ಭವಾಂಗೀ ಭೈರವಾಕಾರಾ ಭೈರವಪ್ರಿಯವಲ್ಲಭಾ॥

ಕಾಲದಾ ಕಾಲರಾತ್ರಿಶ್ಚ ಕಾಮಾ ಕಾತ್ಯಾಯನೀ ಕ್ರಿಯಾ ।\\
ಕ್ರಿಯದಾ ಕ್ರಿಯಹಾ ಕ್ಲೈಬ್ಯಾ ಪ್ರಿಯಪ್ರಾಣಕ್ರಿಯಾ ತಥಾ॥

ಕ್ರೀಂಕಾರೀ ಕಮಲಾ ಲಕ್ಷ್ಮೀಃ ಶಕ್ತಿಃ ಸ್ವಾಹಾ ವಿಭುಃ ಪ್ರಭುಃ ।\\
ಪ್ರಕೃತಿಃ ಪುರುಷಶ್ಚೈವ ಪುರುಷಾಪುರುಷಾಕೃತಿಃ॥

ಪರಮಃ ಪುರುಷಶ್ಚೈವ ಮಾಯಾ ನಾರಾಯಣೀ ಮತಿಃ ।\\
ಬ್ರಾಹ್ಮೀ ಮಾಹೇಶ್ವರೀ ಚೈವ ಕೌಮಾರೀ ವೈಷ್ಣವೀ ತಥಾ॥

ವಾರಾಹೀ ಚೈವ ಚಾಮುಂಡಾ ಇಂದ್ರಾಣೀ ಹರವಲ್ಲಭಾ ।\\
ಭರ್ಗ್ಗೀ ಮಾಹೇಶ್ವರೀ ಕೃಷ್ಣಾ ಕಾತ್ಯಾಯನ್ಯಪಿ ಪೂತನಾ॥

ರಾಕ್ಷಸೀ ಡಾಕಿನೀ ಚಿತ್ರಾ ವಿಚಿತ್ರಾ ವಿಭ್ರಮಾ ತಥಾ ।\\
ಹಾಕಿನೀ ರಾಕಿನೀ ಭೀತಾ ಗಂಧರ್ವಾ ಗಂಧವಾಹಿನೀ॥

ಕೇಕರೀ ಕೋಟರಾಕ್ಷೀ ಚ ನಿರ್ಮಾಂಸಾಲೂಕಮಾಂಸಿಕಾ ।\\
ಲಲಜ್ಜಿಹ್ವಾ ಸುಜಿಹ್ವಾ ಚ ಬಾಲದಾ ಬಾಲದಾಯಿನೀ॥

ಚಂದ್ರಾ ಚಂದ್ರಪ್ರಭಾ ಚಾಂದ್ರೀ ಚಂದ್ರಕಾಂತಿಷು ತತ್ಪರಾ ।\\
ಅಮೃತಾ ಮಾನದಾ ಪೂಷಾ ತುಷ್ಟಿಃ ಪುಷ್ಟೀ ರತಿರ್ಧೃತಿಃ॥

ಶಶಿನೀ ಚಂದ್ರಿಕಾ ಕಾಂತಿರ್ಜ್ಜ್ಯೋತ್ಸ್ನಾ ಶ್ರೀಃ ಪ್ರೀತಿರಂಗದಾ ।\\
ಪೂರ್ಣಾ ಪೂರ್ಣಾಮೃತಾ ಕಲ್ಪಲತಿಕಾ ಕಲ್ಪದಾನದಾ॥

ಸುಕಲ್ಪಾ ಕಲ್ಪಹಸ್ತಾ ಚ ಕಲ್ಪವೃಕ್ಷಕರೀ ಹನುಃ ।\\
ಕಲ್ಪಾಖ್ಯಾ ಕಲ್ಪಭವ್ಯಾ ಚ ಕಲ್ಪಾನಂದಕವಂದಿತಾ॥

ಸೂಚೀಮುಖೀ ಪ್ರೇತಮುಖೀ ಉಲ್ಕಾಮುಖೀ ಮಹಾಸುಖೀ ।\\
ಉಗ್ರಮುಖೀ ಚ ಸುಮುಖೀ ಕಾಕಾಸ್ಯಾ ವಿಕಟಾನನಾ॥

ಕೃಕಲಾಸ್ಯಾ ಚ ಸಂಧ್ಯಾಸ್ಯಾ ಮುಕುಲೀಶಾ ರಮಾಕೃತಿಃ ।\\
ನಾನಾಮುಖೀ ಚ ನಾನಾಸ್ಯಾ ನಾನಾರೂಪಪ್ರಧಾರಿಣೀ॥

ವಿಶ್ವಾರ್ಚ್ಯಾ ವಿಶ್ವಮಾತಾ ಚ ವಿಶ್ವಾಖ್ಯಾ ವಿಶ್ವಭಾವಿನೀ ।\\
ಸೂರ್ಯಾ ಸುರ್ಯಪ್ರಭಾ ಶೋಭಾ ಸೂರ್ಯಮಂಡಲಸಂಸ್ಥಿತಾ॥

ಸೂರ್ಯಕಾಂತಿಃ ಸೂರ್ಯಕರಾ ಸೂರ್ಯಾಖ್ಯಾ ಸೂರ್ಯಭಾವನಾ ।\\
ತಪಿನೀ ತಾಪಿನೀ ಧೂಮ್ರಾ ಮರೀಚಿರ್ಜ್ಜ್ವಾಲಿನೀ ರುಚಿಃ॥

ಸುರದಾ ಭೋಗದಾ ವಿಶ್ವಾ ಬೋಧಿನೀ ಧಾರಿಣೀ ಕ್ಷಮಾ ।\\
ಯುಗದಾ ಯೋಗಹಾ ಯೋಗ್ಯಾ ಯೋಗ್ಯಹಾ ಯೋಗವರ್ದ್ಧಿನೀ॥

ವಹ್ನಿಮಂಡಲಸಂಸ್ಥಾ ಚ ವಹ್ನಿಮಂಡಲಮಧ್ಯಗಾ ।\\
ವಹ್ನಿಮಂಡಲರೂಪಾ ಚ ವಹ್ನಿಮಂಡಲಸಂಜ್ಞಕಾ॥

ವಹ್ನಿತೇಜಾ ವಹ್ನಿರಾಗಾ ವಹ್ನಿದಾ ವಹ್ನಿನಾಶಿನೀ ।\\
ವಹ್ನಿಕ್ರಿಯಾ ವಹ್ನಿಭುಜಾ ಕಲಾ ವಹ್ನೌ ಸ್ಥಿತಾ ಸದಾ॥

ಧೂಮ್ರಾರ್ಚಿತಾ ಚೋಜ್ಜ್ವಲಿನೀ ತಥಾ ಚ ವಿಸ್ಫುಲಿಂಗಿನೀ ।\\
ಶೂಲಿನೀ ಚ ಸುರೂಪಾ ಚ ಕಪಿಲಾ ಹವ್ಯವಾಹಿನೀ॥

ನಾನಾತೇಜಸ್ವಿನೀ ದೇವೀ ಪರಬ್ರಹ್ಮಕುಟುಂಬಿನೀ ।\\
ಜ್ಯೋತಿರ್ಬ್ರಹ್ಮಮಯೀ ದೇವೀ ಪ್ರಬ್ರಹ್ಮಸ್ವರೂಪಿಣೀ॥

ಪರಮಾತ್ಮಾ ಪರಾ ಪುಣ್ಯಾ ಪುಣ್ಯದಾ ಪುಣ್ಯವರ್ದ್ಧಿನೀ ।\\
ಪುಣ್ಯದಾ ಪುಣ್ಯನಾಮ್ನೀ ಚ ಪುಣ್ಯಗಂಧಾ ಪ್ರಿಯಾತನುಃ॥

ಪುಣ್ಯದೇಹಾ ಪುಣ್ಯಕರಾ ಪುಣ್ಯನಿಂದಕನಿಂದಕಾ ।\\
ಪುಣ್ಯಕಾಲಕರಾ ಪುಣ್ಯಾ ಸುಪುಣ್ಯಾ ಪುಣ್ಯಮಾಲಿಕಾ॥

ಪುಣ್ಯಖೇಲಾ ಪುಣ್ಯಕೇಲೀ ಪುಣ್ಯನಾಮಸಮಾ ಪುರಾ ।\\
ಪುಣ್ಯಸೇವ್ಯಾ ಪುಣ್ಯಖೇಲ್ಯಾ ಪುರಾಣಪುಣ್ಯವಲ್ಲಭಾ॥

ಪುರುಷಾ ಪುರುಷಪ್ರಾಣಾ ಪುರುಷಾತ್ಮಸ್ವರೂಪಿಣೀ ।\\
ಪುರುಷಾಂಗೀ ಚ ಪುರುಷೀ ಪುರುಷಸ್ಯ ಕಲಾ ಸದಾ॥

ಸುಪುಷ್ಪಾ ಪುಷ್ಪಕಪ್ರಾಣಾ ಪುಷ್ಪಹಾ ಪುಷ್ಪವಲ್ಲಭಾ ।\\
ಪುಷ್ಪಪ್ರಿಯಾ ಪುಷ್ಪಹಾರಾ ಪುಷ್ಪವಂದಕವಂದಕಾ॥

ಪುಷ್ಪಹಾ ಪುಷ್ಪಮಾಲಾ ಚ ಪುಷ್ಪನಿಂದಕನಾಶಿನೀ ।\\
ನಕ್ಷತ್ರಪ್ರಾಣಹಂತ್ರೀ ಚ ನಕ್ಷತ್ರಾಲಕ್ಷವಂದಕಾ॥

ಲಕ್ಷ್ಯಮಾಲ್ಯಾ ಲಕ್ಷಹಾರಾ ಲಕ್ಷಾ ಲಕ್ಷಸ್ವರೂಪಿಣೀ ।\\
ನಕ್ಷತ್ರಾಣೀ ಸುನಕ್ಷತ್ರಾ ನಕ್ಷತ್ರಾಹಾ ಮಹೋದಯಾ॥

ಮಹಾಮಾಲ್ಯಾ ಮಹಾಮಾನ್ಯಾ ಮಹತೀ ಮಾತೃಪೂಜಿತಾ ।\\
ಮಹಾಮಹಾಕನೀಯಾ ಚ ಮಹಾಕಾಲೇಶ್ವರೀ ಮಹಾ॥

ಮಹಾಸ್ಯಾ ವಂದನೀಯಾ ಚ ಮಹಾಶಬ್ದನಿವಾಸಿನೀ ।\\
ಮಹಾಶಂಖೇಶ್ವರೀ ಮೀನಾ ಮತ್ಸ್ಯಗಂಧಾ ಮಹೋದರೀ॥

ಲಂಬೋದರೀ ಚ ಲಂಬೋಷ್ಠೀ ಲಂಬನಿಮ್ನತನೂದರೀ ।\\
ಲಂಬೋಷ್ಠೀ ಲಂಬನಾಸಾ ಚ ಲಂಬಘೋಣಾ ಲಲತ್ಸುಕಾ॥

ಅತಿಲಂಬಾ ಮಹಾಲಂಬಾ ಸುಲಂಬಾ ಲಂಬವಾಹಿನೀ ।\\
ಲಂಬಾರ್ಹಾ ಲಂಬಶಕ್ತಿಶ್ಚ ಲಂಬಸ್ಥಾ ಲಂಬಪೂರ್ವಿಕಾ॥

ಚತುರ್ಘಂಟಾ ಮಹಾಘಂಟಾ ಘಂಟಾನಾದಪ್ರಿಯಾ ಸದಾ ।\\
ವಾದ್ಯಪ್ರಿಯಾ ವಾದ್ಯರತಾ ಸುವಾದ್ಯಾ ವಾದ್ಯನಾಶಿನೀ॥

ರಮಾ ರಾಮಾ ಸುಬಾಲಾ ಚ ರಮಣೀಯಸ್ವಭಾವಿನೀ ।\\
ಸುರಮ್ಯಾ ರಮ್ಯದಾ ರಂಭಾ ರಂಭೋರೂ ರಾಮವಲ್ಲಭಾ॥

ಕಾಮಪ್ರಿಯಾ ಕಾಮಕರಾ ಕಾಮಾಂಗೀ ರಮಣೀ ರತಿಃ ।\\
ರತಿಪ್ರಿಯಾ ರತಿ ರತೀ ರತಿಸೇವ್ಯಾ ರತಿಪ್ರಿಯಾ॥

ಸುರಭಿಃ ಸುರಭೀ ಶೋಭಾ ದಿಕ್ಷೋಭಾಽಶುಭನಾಶಿನೀ ।\\
ಸುಶೋಭಾ ಚ ಮಹಾಶೋಭಾಽತಿಶೋಭಾ ಪ್ರೇತತಾಪಿನೀ॥

ಲೋಭಿನೀ ಚ ಮಹಾಲೋಭಾ ಸುಲೋಭಾ ಲೋಭವರ್ದ್ಧಿನೀ ।\\
ಲೋಭಾಂಗೀ ಲೋಭವಂದ್ಯಾ ಚ ಲೋಭಾಹೀ ಲೋಭಭಾಸಕಾ॥

ಲೋಭಪ್ರಿಯಾ ಮಹಾಲೋಭಾ ಲೋಭನಿಂದಕನಿಂದಕಾ ।\\
ಲೋಭಾಂಗವಾಸಿನೀ ಗಂಧವಿಗಂಧಾ ಗಂಧನಾಶಿನೀ॥

ಗಂಧಾಂಗೀ ಗಂಧಪುಷ್ಟಾ ಚ ಸುಗಂಧಾ ಪ್ರೇಮಗಂಧಿಕಾ ।\\
ದುರ್ಗಂಧಾ ಪೂತಿಗಂಧಾ ಚ ವಿಗಂಧಾ ಅತಿಗಂಧಿಕಾ॥

ಪದ್ಮಾಂತಿಕಾ ಪದ್ಮವಹಾ ಪದ್ಮಪ್ರಿಯಪ್ರಿಯಂಕರೀ ।\\
ಪದ್ಮನಿಂದಕನಿಂದಾ ಚ ಪದ್ಮಸಂತೋಷವಾಹನಾ॥

ರಕ್ತೋತ್ಪಲವರಾ ದೇವೀ ರಕ್ತೋತ್ಪಲಪ್ರಿಯಾ ಸದಾ ।\\
ರಕ್ತೋತ್ಪಲಸುಗಂಧಾ ಚ ರಕ್ತೋತ್ಪಲನಿವಾಸಿನೀ॥

ರಕ್ತೋತ್ಪಲಗ್ರಹಾಮಾಲಾ ರಕ್ತೋತ್ಪಲಮನೋಹರಾ ।\\
ರಕ್ತೋತ್ಪಲಸುನೇತ್ರಾ ಚ ರಕ್ತೋತ್ಪಲಸ್ವರೂಪಧೃಕ್॥

ವೈಷ್ಣವೀ ವಿಷ್ಣುಪೂಜ್ಯಾ ಚ ವೈಷ್ಣವಾಂಗನಿವಾಸಿನೀ ।\\
ವಿಷ್ಣುಪೂಜಕಪೂಜ್ಯಾ ಚ ವೈಷ್ಣವೇ ಸಂಸ್ಥಿತಾ ತನುಃ॥

ನಾರಾಯಣಸ್ಯ ದೇಹಸ್ಥಾ ನಾರಾಯಣಮನೋಹರಾ ।\\
ನಾರಾಯಣಸ್ವರೂಪಾ ಚ ನಾರಾಯಣಮನಃಸ್ಥಿತಾ॥

ನಾರಾಯಣಾಂಗಸಂಭೂತಾ ನಾರಾಯಣಪ್ರಿಯಾತನುಃ ।\\
ನಾರೀ ನಾರಾಯಣೀಗಣ್ಯಾ ನಾರಾಯಣಗೃಹಪ್ರಿಯಾ॥

ಹರಪೂಜ್ಯಾ ಹರಶ್ರೇಷ್ಠಾ ಹರಸ್ಯ ವಲ್ಲಭಾ ಕ್ಷಮಾ ।\\
ಸಂಹಾರೀ ಹರದೇಹಸ್ಥಾ ಹರಪೂಜನತತ್ಪರಾ॥

ಹರದೇಹಸಮುದ್ಭೂತಾ ಹರಾಂಗವಾಸಿನೀಕುಹೂಃ ।\\
ಹರಪೂಜಕಪೂಜ್ಯಾ ಚ ಹರವಂದಕತತ್ಪರಾ॥

ಹರದೇಹಸಮುತ್ಪನ್ನಾ ಹರಕ್ರೀಡಾಸದಾಗತಿಃ ।\\
ಸುಗಣಾಸಂಗರಹಿತಾ ಅಸಂಗಾಸಂಗನಾಶಿನೀ॥

ನಿರ್ಜನಾ ವಿಜನಾ ದುರ್ಗಾ ದುರ್ಗಕ್ಲೇಶನಿವಾರಿಣೀ ।\\
ದುರ್ಗದೇಹಾಂತಕಾ ದುರ್ಗಾರೂಪಿಣೀ ದುರ್ಗತಸ್ಥಿಕಾ॥

ಪ್ರೇತಕರಾ ಪ್ರೇತಪ್ರಿಯಾ ಪ್ರೇತದೇಹಸಮುದ್ಭವಾ ।\\
ಪ್ರೇತಾಂಗವಾಸಿನೀ ಪ್ರೇತಾ ಪ್ರೇತದೇಹವಿಮರ್ದ್ದಕಾ॥

ಡಾಕಿನೀ ಯೋಗಿನೀ ಕಾಲರಾತ್ರಿಃ ಕಾಲಪ್ರಿಯಾ ಸದಾ ।\\
ಕಾಲರಾತ್ರಿಹರಾ ಕಾಲಾ ಕೃಷ್ಣದೇಹಾ ಮಹಾತನುಃ॥

ಕೃಷ್ಣಾಂಗೀ ಕುಟಿಲಾಂಗೀ ಚ ವಜ್ರಾಂಗೀ ವಜ್ರರೂಪಧೃಕ್ ।\\
ನಾನಾದೇಹಧರಾ ಧನ್ಯಾ ಷಟ್ಚಕ್ರಕ್ರಮವಾಸಿನೀ॥

ಮೂಲಾಧಾರನಿವಾಸೀ ಚ ಮೂಲಾಧಾರಸ್ಥಿತಾ ಸದಾ ।\\
ವಾಯುರೂಪಾ ಮಹಾರೂಪಾ ವಾಯುಮಾರ್ಗನಿವಾಸಿನೀ॥

ವಾಯುಯುಕ್ತಾ ವಾಯುಕರಾ ವಾಯುಪೂರಕಪೂರಕಾ ।\\
ವಾಯುರೂಪಧರಾ ದೇವೀ ಸುಷುಮ್ನಾಮಾರ್ಗಗಾಮಿನೀ॥

ದೇಹಸ್ಥಾ ದೇಹರೂಪಾ ಚ ದೇಹಧ್ಯೇಯಾ ಸುದೇಹಿಕಾ ।\\
ನಾಡೀರೂಪಾ ಮಹೀರೂಪಾ ನಾಡೀಸ್ಥಾನನಿವಾಸಿನೀ॥

ಇಂಗಲಾ ಪಿಂಗಲಾ ಚೈವ ಸುಷುಮ್ನಾಮಧ್ಯವಾಸಿನೀ ।\\
ಸದಾಶಿವಪ್ರಿಯಕರೀ ಮೂಲಪ್ರಕೃತಿರೂಪಧೃಕ್॥

ಅಮೃತೇಶೀ ಮಹಾಶಾಲೀ ಶೃಂಗಾರಾಂಗನಿವಾಸಿನೀ ।\\
ಉಪತ್ತಿಸ್ಥಿತಿಸಂಹಂತ್ರೀ ಪ್ರಲಯಾಪದವಾಸಿನೀ॥

ಮಹಾಪ್ರಲಯಯುಕ್ತಾ ಚ ಸೃಷ್ಟಿಸಂಹಾರಕಾರಿಣೀ ।\\
ಸ್ವಧಾ ಸ್ವಾಹಾ ಹವ್ಯವಾಹಾ ಹವ್ಯಾ ಹವ್ಯಪ್ರಿಯಾ ಸದಾ॥

ಹವ್ಯಸ್ಥಾ ಹವ್ಯಭಕ್ಷಾ ಚ ಹವ್ಯದೇಹಸಮುದ್ಭವಾ ।\\
ಹವ್ಯಕ್ರೀಡಾ ಕಾಮಧೇನುಸ್ವರೂಪಾ ರೂಪಸಂಭವಾ॥

ಸುರಭೀ ನಂದನೀ ಪುಣ್ಯಾ ಯಜ್ಞಾಂಗೀ ಯಜ್ಞಸಂಭವಾ ।\\
ಯಜ್ಞಸ್ಥಾ ಯಜ್ಞದೇಹಾ ಚ ಯೋನಿಜಾ ಯೋನಿವಾಸಿನೀ॥

ಅಯೋನಿಜಾ ಸತೀ ಸತ್ಯಾ ಅಸತೀ ಕುಟಿಲಾತನುಃ ।\\
ಅಹಲ್ಯಾ ಗೌತಮೀ ಗಮ್ಯಾ ವಿದೇಹಾ ದೇಹನಾಶಿನೀ॥

ಗಾಂಧಾರೀ ದ್ರೌಪದೀ ದೂತೀ ಶಿವಪ್ರಿಯಾ ತ್ರಯೋದಶೀ ।\\
ಪಂಚದಶೀ ಪೌರ್ಣಮಾಸೀ ಚತುರ್ದ್ದಶೀ ಚ ಪಂಚಮೀ॥

ಷಷ್ಠೀ ಚ ನವಮೀ ಚೈವ ಅಷ್ಟಮೀ ದಶಮೀ ತಥಾ ।\\
ಏಕಾದಶೀ ದ್ವಾದಶೀ ಚ ದ್ವಾರರೂಪೀಭಯಪ್ರದಾ॥

ಸಂಕ್ರಾಂತ್ಯಾ ಸಾಮರೂಪಾ ಚ ಕುಲೀನಾ ಕುಲನಾಶಿನೀ ।\\
ಕುಲಕಾಂತಾ ಕೃಶಾ ಕುಂಭಾ ಕುಂಭದೇಹವಿವರ್ದ್ಧಿನೀ॥

ವಿನೀತಾ ಕುಲವತ್ಯರ್ತ್ಥೀ ಅಂತರೀ ಚಾನುಗಾಪ್ಯುಷಾ ।\\
ನದೀಸಾಗರದಾ ಶಾಂತಿಃ ಶಾಂತಿರೂಪಾ ಸುಶಾಂತಿಕಾ॥

ಆಶಾ ತೃಷ್ಣಾ ಕ್ಷುಧಾ ಕ್ಷೋಭ್ಯಾ ಕ್ಷೋಭರೂಪನಿವಾಸಿನೀ ।\\
ಗಂಗಾಸಾಗರಗಾ ಕಾಂತಿಃ ಶ್ರುತಿಃ ಸ್ಮೃತಿರ್ದ್ಧೃತಿರ್ಮಹೀ॥

ದಿವಾರಾತ್ರಿಃ ಪಂಚಭೂತದೇಹಾ ಚೈವ ಸುದೇಹಕಾ ।\\
ತಂಡುಲಾ ಚ್ಛಿನ್ನಮಸ್ತಾ ಚ ನಾಗಯಜ್ಞೋಪವೀತಿನೀ॥

ವರ್ಣಿನೀ ಡಾಕಿನೀ ಶಕ್ತಿಃ ಕುರುಕುಲ್ಲಾ ಸುಕುಲ್ಲಕಾ ।\\
ಪ್ರತ್ಯಂಗಿರಾಽಪರಾ ದೇವೀ ಅಜಿತಾ ಜಯದಾಯಿನೀ॥

ಜಯಾ ಚ ವಿಜಯಾ ಚೈವ ಮಹಿಷಾಸುರಘಾತಿನೀ ।\\
ಮಧುಕೈಟಭಹಂತ್ರೀ ಚ ಚಂಡಮುಂಡವಿನಾಶಿನೀ॥

ನಿಶುಂಭಶುಂಭಹನನೀ ರಕ್ತಬೀಜಕ್ಷಯಂಕರೀ ।\\
ಕಾಶೀ ಕಾಶೀನಿವಾಸೀ ಚ ಮಧುರಾ ಪಾರ್ವತೀ ಪರಾ॥

ಅಪರ್ಣಾ ಚಂಡಿಕಾ ದೇವೀ ಮೃಡಾನೀ ಚಾಂಬಿಕಾ ಕಲಾ ।\\
ಶುಕ್ಲಾ ಕೃಷ್ಣಾ ವರ್ಣವರ್ಣಾ ಶರದಿಂದುಕಲಾಕೃತಿಃ॥

ರುಕ್ಮಿಣೀ ರಾಧಿಕಾ ಚೈವ ಭೈರವ್ಯಾಃ ಪರಿಕೀರ್ತ್ತಿತಂ ।\\
ಅಷ್ಟಾಧಿಕಸಹಸ್ರಂತು ದೇವ್ಯಾ ನಾಮಾನುಕೀರ್ತ್ತನಾತ್॥

ಮಹಾಪಾತಕಯುಕ್ತೋಽಪಿ ಮುಚ್ಯತೇ ನಾತ್ರ ಸಂಶಯಃ ।\\
ಬ್ರಹ್ಮಹತ್ಯಾ ಸುರಾಪಾನಂ ಸ್ತೇಯಂಗುರ್ವಂಗನಾಗಮಃ॥

ಮಹಾಪಾತಕಕೋಟ್ಯಸ್ತು ತಥಾ ಚೈವೋಪಪಾತಕಾಃ ।\\
ಸ್ತೋತ್ರೇಣ ಭೈರವೋಕ್ತೇನ ಸರ್ವನ್ನಶ್ಯತಿ ತತ್ಕ್ಷಣಾತ್॥

ಸರ್ವವ್ವಾ ಶ್ಲೋಕಮೇಕವ್ವಾ ಪಠನಾತ್ಸ್ಮರಣಾದಪಿ ।\\
ಪಠೇದ್ವಾ ಪಾಠಯೇದ್ವಾಪಿ ಸದ್ಯೋ ಮುಚ್ಯೇತ ಬಂಧನಾತ್॥

ರಾಜದ್ವಾರೇ ರಣೇ ದುರ್ಗೇ ಸಂಕಟೇ ಗಿರಿದುರ್ಗ್ಗಮೇ ।\\
ಪ್ರಾಂತರೇ ಪರ್ವತೇ ವಾಪಿ ನೌಕಾಯಾವ್ವಾ ಮಹೇಶ್ವರಿ॥

ವಹ್ನಿದುರ್ಗಭಯೇ ಪ್ರಾಪ್ತೇ ಸಿಂಹವ್ಯಾಘ್ರಭ್ಯಾಕುಲೇ ।\\
ಪಠನಾತ್ಸ್ಮರಣಾನ್ಮರ್ತ್ತ್ಯೋ ಮುಚ್ಯತೇ ಸರ್ವಸಂಕಟಾತ್॥

ಅಪುತ್ರೋ ಲಭತೇ ಪುತ್ರಂದರಿದ್ರೋ ಧನವಾನ್ಭವೇತ್ ।\\
ಸರ್ವಶಾಸ್ತ್ರಪರೋ ವಿಪ್ರಃ ಸರ್ವಯಜ್ಞಫಲಲ್ಲಭೇತ್॥

ಅಗ್ನಿವಾಯುಜಲಸ್ತಂಭಂಗತಿಸ್ತಂಭವಿವಸ್ವತಃ ।\\
ಮಾರಣೇ ದ್ವೇಷಣೇ ಚೈವ ತಥೋಚ್ಚಾಟೇ ಮಹೇಶ್ವರಿ॥

ಗೋರೋಚನಾಕುಂಕುಮೇನ ಲಿಖೇತ್ಸ್ತೋತ್ರಮನನ್ಯಧೀಃ ।\\
ಗುರುಣಾ ವೈಷ್ಣವೈರ್ವಾಪಿ ಸರ್ವಯಜ್ಞಫಲಲ್ಲಭೇತ್॥

ವಶೀಕರಣಮತ್ರೈವ ಜಾಯಂತೇ ಸರ್ವಸಿದ್ಧಯಃ ।\\
ಪ್ರಾತಃಕಾಲೇ ಶುಚಿರ್ಬ್ಭೂತ್ವಾ ಮಧ್ಯಾಹ್ನೇ ಚ ನಿಶಾಮುಖೇ॥

ಪಠೇದ್ವಾ ಪಾಠಯೇದ್ವಾಪಿ ಸರ್ವಯಜ್ಞಫಲಲ್ಲಭೇತ್ ।\\
ವಾದೀ ಮೂಕೋ ಭವೇದ್ದುಷ್ಟೋ ರಾಜಾ ಚ ಸೇವಕೋ ಯಥಾ॥

ಆದಿತ್ಯಮಂಗಲದಿನೇ ಗುರೌ ವಾಪಿ ಮಹೇಶ್ವರಿ ।\\
ಗೋರೋಚನಾಕುಂಕುಮೇನ ಲಿಖೇತ್ಸ್ತೋತ್ರಮನನ್ಯಧೀಃ॥

ಗುರುಣಾ ವೈಷ್ಣವೈರ್ವಾಪಿ ಸರ್ವಯಜ್ಞಫಲಲ್ಲಭೇತ್ ।\\
ಧೃತ್ವಾ ಸುವರ್ಣಮಧ್ಯಸ್ಥಂ ಸರ್ವಾನ್ಕಾಮಾನವಾಪ್ನುಯಾತ್॥

ಸ್ತ್ರೀಣಾವ್ವಾಮಕರೇ ಧಾರ್ಯಂಪುಮಾಂದಕ್ಷಕರೇ ತಥಾ ।\\
ಆದಿತ್ಯಮಂಗಲದಿನೇ ಗುರೌ ವಾಪಿ ಮಹೇಶ್ವರಿ॥

ಶನೈಶ್ಚರೇ ಲಿಖೇದ್ವಾಪಿ ಸರ್ವಸಿದ್ಧಿಂ ಲಭೇದ್ಧ್ರುವಂ ।\\
ಪ್ರಾಂತರೇ ವಾ ಶ್ಮಶಾನೇ ವಾ ನಿಶಾಯಾಮರ್ದ್ಧರಾತ್ರಕೇ॥

ಶೂನ್ಯಾಗಾರೇ ಚ ದೇವೇಶಿ ಲಿಖೇದ್ಯತ್ನೇನ ಸಾಧಕಃ ।\\
ಸಿಂಹರಾಶೌ ಗುರುಗತೇ ಕರ್ಕ್ಕಟಸ್ಥೇ ದಿವಾಕರೇ॥

ಮೀನರಾಶೌ ಗುರುಗತೇ ಲಿಖೇದ್ಯತ್ನೇನ ಸಾಧಕಃ ।\\
ರಜಸ್ವಲಾಭಗಂದೃಷ್ಟ್ವಾ ತತ್ರಸ್ಥೋ ವಿಲಿಖೇತ್ಸದಾ॥

ಸುಗಂಧಿಕುಸುಮೈಃ ಶುಕ್ರೈಃ ಸುಗಂಧಿಗಂಧಚಂದನೈಃ ।\\
ಮೃಗನಾಭಿಮೃಗಮದೈರ್ವಿಲಿಖೇದ್ಯತ್ನಪೂರ್ವಕಂ॥

ಲಿಖಿತ್ವಾ ಚ ಪಠಿತ್ವಾ ಚ ಧಾರಯೇಚ್ಚಾಪ್ಯನನ್ಯಧೀಃ ।\\
ಕುಮಾರೀಂಪೂಜಯಿತ್ವಾ ಚ ನಾರೀಶ್ಚಾಪಿ ಪ್ರಪೂಜಯೇತ್॥

ಪೂಜಯಿತ್ವಾ ಚ ಕುಸುಮೈರ್ಗ್ಗಂಧಚಂದನವಸ್ತ್ರಕೈಃ ।\\
ಸಿಂದೂರರಕ್ತಕುಸುಮೈಃ ಪೂಜಯೇದ್ಭಕ್ತಿಯೋಗತಃ॥

ಅಥವಾ ಪೂಜಯೇದ್ದೇವಿ ಕುಮಾರೀರ್ದ್ದಶಮಾವಧೀಃ ।\\
ಸರ್ವಾಭೀಷ್ಟಫಲಂತತ್ರ ಲಭತೇ ತತ್ಕ್ಷಣಾದಪಿ॥

ನಾತ್ರ ಸಿದ್ಧಾದ್ಯಪೇಕ್ಷಾಸ್ತಿ ನ ವಾ ಮಿತ್ರಾರಿದೂಷಣಂ ।\\
ನ ವಿಚಾರ್ಯಂಚ ದೇವೇಶಿ ಜಪಮಾತ್ರೇಣ ಸಿದ್ಧಿದಂ॥

ಸರ್ವದಾ ಸರ್ವಕಾರ್ಯೇಷು ಷಟ್ಸಾಹಸ್ರಪ್ರಮಾಣತಃ ।\\
ಬಲಿಂದತ್ತ್ವಾ ವಿಧಾನೇನ ಪ್ರತ್ಯಹಂಪೂಜಯೇಚ್ಛಿವಾಂ॥

ಸ್ವಯಂಭೂಕುಸುಮೈಃ ಪುಷ್ಪೈರ್ಬ್ಬಲಿದಾನಂದಿವಾನಿಶಂ ।\\
ಪೂಜಯೇತ್ಪಾರ್ವತೀಂದೇವೀಂಭೈರವೀಂತ್ರಿಪುರಾತ್ಮಿಕಾಂ॥

ಬ್ರಾಹ್ಮಣಾನ್ಭೋಜಯೇನ್ನಿತ್ಯಂದಶಕಂದ್ವಾದಶಂತಥಾ ।\\
ಅನೇನ ವಿಧಿನಾ ದೇವಿ ಬಾಲಾನ್ನಿತ್ಯಂಪ್ರಪೂಜಯೇತ್॥

ಮಾಸಮೇಕಂಪಠೇದ್ಯಸ್ತು ತ್ರಿಸಂಧ್ಯವ್ವಿಧಿನಾಮುನಾ ।\\
ಅಪುತ್ರೋ ಲಭತೇ ಪುತ್ರನ್ನಿರ್ದ್ಧನೋ ಧನವಾನ್ಭವೇತ್॥

ಸದಾ ಚಾನೇನ ವಿಧಿನಾ ತಥಾ ಮಾಸತ್ರಯೇಣ ಚ ।\\
ಕೃತಕಾರ್ಯಂ ಭವೇದ್ದೇವಿ ತಥಾ ಮಾಸಚತುಷ್ಟಯೇ॥

ದೀರ್ಗ್ಘರೋಗಾತ್ಪ್ರಮುಚ್ಯೇತ ಪಂಚಮೇ ಕವಿರಾಡ್ಭವೇತ್ ।\\
ಸರ್ವೈಶ್ವರ್ಯಂ ಲಭೇದ್ದೇವಿ ಮಾಸಷಟ್ಕೇ ತಥೈವ ಚ॥

ಸಪ್ತಮೇ ಖೇಚರತ್ವಂಚ ಅಷ್ಟಮೇ ಚ ವೃಹದ್ದ್ಯುತಿಃ ।\\
ನವಮೇ ಸರ್ವಸಿದ್ಧಿಃ ಸ್ಯಾದ್ದಶಮೇ ಲೋಕಪೂಜಿತಃ॥

ಏಕಾದಶೇ ರಾಜವಶ್ಯೋ ದ್ವಾದಶೇ ತು ಪುರಂದರಃ ।\\
ವಾರಮೇಕಂಪಠೇದ್ಯಸ್ತು ಪ್ರಾಪ್ನೋತಿ ಪೂಜನೇ ಫಲಂ॥

ಸಮಗ್ರಂ ಶ್ಲೋಕಮೇಕವ್ವಾ ಯಃ ಪಠೇತ್ಪ್ರಯತಃ ಶುಚಿಃ ।\\
ಸ ಪೂಜಾಫಲಮಾಪ್ನೋತಿ ಭೈರವೇಣ ಚ ಭಾಷಿತಂ॥

ಆಯುಷ್ಮತ್ಪ್ರೀತಿಯೋಗೇ ಚ ಬ್ರಾಹ್ಮೈಂದ್ರೇ ಚ ವಿಶೇಷತಃ ।\\
ಪಂಚಮ್ಯಾಂಚ ತಥಾ ಷಷ್ಠ್ಯಾಯ್ಯತ್ರ ಕುತ್ರಾಪಿ ತಿಷ್ಠತಿ॥

ಶಂಕಾ ನ ವಿದ್ಯತೇ ತತ್ರ ನ ಚ ಮಾಯಾದಿದೂಷಣಂ ।\\
ವಾರಮೇಕಂ ಪಠೇನ್ಮರ್ತ್ತ್ಯೋ ಮುಚ್ಯತೇ ಸರ್ವಸಂಕಟಾತ್ ।\\
ಕಿಮನ್ಯದ್ಬಹುನಾ ದೇವಿ ಸರ್ವಾಭೀಷ್ಟಫಲಲ್ಲಭೇತ್॥

\authorline{॥ಇತಿ ಶ್ರೀವಿಶ್ವಸಾರೇ ಮಹಾಭೈರವವಿರಚಿತಂ ಶ್ರೀಮತ್ತ್ರಿಪುರಭೈರವೀಸಹಸ್ರನಾಮಸ್ತೋತ್ರಂ ಸಮಾಪ್ತಂ॥}
%=============================================================================================
\section{ಶ್ರೀತ್ರಿಪುರಭೈರವೀ ಅಷ್ಟೋತ್ತರಶತನಾಮಸ್ತೋತ್ರಂ}
\addcontentsline{toc}{section}{ಶ್ರೀತ್ರಿಪುರಭೈರವೀ ಅಷ್ಟೋತ್ತರಶತನಾಮಸ್ತೋತ್ರಂ}


ಶ್ರೀದೇವ್ಯುವಾಚ ।\\
ಕೈಲಾಸವಾಸಿನ್ಭಗವನ್ಪ್ರಾಣೇಶ್ವರ ಕೃಪಾನಿಧೇ ।\\
ಭಕ್ತವತ್ಸಲ ಭೈರವ್ಯಾ ನಾಮ್ನಾಮಷ್ಟೋತ್ತರಂ ಶತಂ ॥೧॥

ನ ಶ್ರುತಂ ದೇವದೇವೇಶ ವದ ಮಾಂ ದೀನವತ್ಸಲ ।\\
ಶ್ರೀಶಿವ ಉವಾಚ ।\\
ಶೃಣು ಪ್ರಿಯೇ ಮಹಾಗೋಪ್ಯಂ ನಾಮ್ನಾಮಷ್ಟೋತ್ತರಂ ಶತಂ ॥೨॥

ಭೈರವ್ಯಾಃ ಶುಭದಂ ಸೇವ್ಯಂ ಸರ್ವಸಂಪತ್ಪ್ರದಾಯಕಂ ।\\
ಯಸ್ಯಾನುಷ್ಠಾನಮಾತ್ರೇಣ ಕಿಂ ನ ಸಿದ್ಧ್ಯತಿ ಭೂತಲೇ ॥೩॥

ಓಂ ಭೈರವೀ ಭೈರವಾರಾಧ್ಯಾ ಭೂತಿದಾ ಭೂತಭಾವನಾ ।\\
ಕಾರ್ಯ್ಯಾ ಬ್ರಾಹ್ಮೀ ಕಾಮಧೇನುಃ ಸರ್ವಸಂಪತ್ಪ್ರದಾಯಿನೀ ॥೪॥

ತ್ರೈಲೋಕ್ಯವಂದಿತಾ ದೇವೀ ಮಹಿಷಾಸುರಮರ್ದ್ದಿನೀ ।\\
ಮೋಹಘ್ನೀ ಮಾಲತೀಮಾಲಾ ಮಹಾಪಾತಕನಾಶಿನೀ ॥೫॥

ಕ್ರೋಧಿನೀ ಕ್ರೋಧನಿಲಯಾ ಕ್ರೋಧರಕ್ತೇಕ್ಷಣಾ ಕುಹೂಃ ।\\
ತ್ರಿಪುರಾ ತ್ರಿಪುರಾಧಾರಾ ತ್ರಿನೇತ್ರಾ ಭೀಮಭೈರವೀ ॥೬॥

ದೇವಕೀ ದೇವಮಾತಾ ಚ ದೇವದುಷ್ಟವಿನಾಶಿನೀ ।\\
ದಾಮೋದರಪ್ರಿಯಾ ದೀರ್ಘಾ ದುರ್ಗಾ ದುರ್ಗತಿನಾಶಿನೀ ॥೭॥

ಲಂಬೋದರೀ ಲಂಬಕರ್ಣಾ ಪ್ರಲಂಬಿತಪಯೋಧರಾ ।\\
ಪ್ರತ್ಯಂಗಿರಾ ಪ್ರತಿಪದಾ ಪ್ರಣತಕ್ಲೇಶನಾಶಿನೀ ॥೮॥

ಪ್ರಭಾವತೀ ಗುಣವತೀ ಗಣಮಾತಾ ಗುಹೇಶ್ವರೀ ।\\
ಕ್ಷೀರಾಬ್ಧಿತನಯಾ ಕ್ಷೇಮ್ಯಾ ಜಗತ್ತ್ರಾಣವಿಧಾಯಿನೀ ॥೯॥

ಮಹಾಮಾರೀ ಮಹಾಮೋಹಾ ಮಹಾಕ್ರೋಧಾ ಮಹಾನದೀ ।\\
ಮಹಾಪಾತಕಸಂಹರ್ತ್ರೀ ಮಹಾಮೋಹಪ್ರದಾಯಿನೀ ॥೧೦॥

ವಿಕರಾಲಾ ಮಹಾಕಾಲಾ ಕಾಲರೂಪಾ ಕಲಾವತೀ ।\\
ಕಪಾಲಖಟ್ವಾಂಗಧರಾ ಖಡ್ಗಖರ್ಪ್ಪರಧಾರಿಣೀ ॥೧೧॥

ಕುಮಾರೀ ಕುಂಕುಮಪ್ರೀತಾ ಕುಂಕುಮಾರುಣರಂಜಿತಾ ।\\
ಕೌಮೋದಕೀ ಕುಮುದಿನೀ ಕೀರ್ತ್ತ್ಯಾ ಕೀರ್ತ್ತಿಪ್ರದಾಯಿನೀ ॥೧೨॥

ನವೀನಾ ನೀರದಾ ನಿತ್ಯಾ ನಂದಿಕೇಶ್ವರಪಾಲಿನೀ ।\\
ಘರ್ಘರಾ ಘರ್ಘರಾರಾವಾ ಘೋರಾ ಘೋರಸ್ವರೂಪಿಣೀ ॥೧೩॥

ಕಲಿಘ್ನೀ ಕಲಿಧರ್ಮಘ್ನೀ ಕಲಿಕೌತುಕನಾಶಿನೀ ।\\
ಕಿಶೋರೀ ಕೇಶವಪ್ರೀತಾ ಕ್ಲೇಶಸಂಘನಿವಾರಿಣೀ ॥೧೪॥

ಮಹೋತ್ತಮಾ ಮಹಾಮತ್ತಾ ಮಹಾವಿದ್ಯಾ ಮಹೀಮಯೀ ।\\
ಮಹಾಯಜ್ಞಾ ಮಹಾವಾಣೀ ಮಹಾಮಂದರಧಾರಿಣೀ ॥೧೫॥

ಮೋಕ್ಷದಾ ಮೋಹದಾ ಮೋಹಾ ಭುಕ್ತಿಮುಕ್ತಿಪ್ರದಾಯಿನೀ ।\\
ಅಟ್ಟಾಟ್ಟಹಾಸನಿರತಾ ಕಂಕಣನ್ನೂಪುರಧಾರಿಣೀ ॥೧೬॥

ದೀರ್ಘದಂಷ್ಟ್ರಾ ದೀರ್ಘಮುಖೀ ದೀರ್ಘಘೋಣಾ ಚ ದೀರ್ಘಿಕಾ ।\\
ದನುಜಾಂತಕರೀ ದುಷ್ಟಾ ದುಃಖದಾರಿದ್ರ್ಯಭಂಜಿನೀ ॥೧೭॥

ದುರಾಚಾರಾ ಚ ದೋಷಘ್ನೀ ದಮಪತ್ನೀ ದಯಾಪರಾ ।\\
ಮನೋಭವಾ ಮನುಮಯೀ ಮನುವಂಶಪ್ರವರ್ದ್ಧಿನೀ ॥೧೮॥

ಶ್ಯಾಮಾ ಶ್ಯಾಮತನುಃ ಶೋಭಾ ಸೌಮ್ಯಾ ಶಂಭುವಿಲಾಸಿನೀ ।\\
ಇತಿ ತೇ ಕಥಿತಂ ದಿವ್ಯಂ ನಾಮ್ನಾಮಷ್ಟೋತ್ತರಂ ಶತಂ ॥೧೯॥

ಭೈರವ್ಯಾ ದೇವದೇವೇಶ್ಯಾಸ್ತವ ಪ್ರೀತ್ಯೈ ಸುರೇಶ್ವರಿ ।\\
ಅಪ್ರಕಾಶ್ಯಮಿದಂ ಗೋಪ್ಯಂ ಪಠನೀಯಂ ಪ್ರಯತ್ನತಃ ॥೨೦॥

ದೇವೀಂ ಧ್ಯಾತ್ವಾ ಸುರಾಂ ಪೀತ್ವಾ ಮಕಾರಪಂಚಕೈಃ ಪ್ರಿಯೇ ।\\
ಪೂಜಯೇತ್ಸತತಂ ಭಕ್ತ್ಯಾ ಪಠೇತ್ಸ್ತೋತ್ರಮಿದಂ ಶುಭಂ ॥೨೧॥

ಷಣ್ಮಾಸಾಭ್ಯಂತರೇ ಸೋಽಪಿ ಗಣನಾಥಸಮೋ ಭವೇತ್ ।\\
ಕಿಮತ್ರ ಬಹುನೋಕ್ತೇನ ತ್ವದಗ್ರೇ ಪ್ರಾಣವಲ್ಲಭೇ ॥೨೨ ॥

ಸರ್ವಂ ಜಾನಾಸಿ ಸರ್ವಜ್ಞೇ ಪುನರ್ಮಾಂ ಪರಿಪೃಚ್ಛಸಿ ।\\
ನ ದೇಯಂ ಪರಶಿಷ್ಯೇಭ್ಯೋ ನಿಂದಕೇಭ್ಯೋ ವಿಶೇಷತಃ ॥೨೩॥

\authorline{॥ಇತಿ ಶ್ರೀಭೈರವ್ಯಷ್ಟೋತ್ತರಶತನಾಮಸ್ತೋತ್ರಂ ಸಂಪೂರ್ಣಂ ॥}
%======================================================
\section{ಶ್ರೀಧೂಮಾವತೀಸಹಸ್ರನಾಮಸ್ತೋತ್ರಂ}
\addcontentsline{toc}{section}{ಶ್ರೀಧೂಮಾವತೀಸಹಸ್ರನಾಮಸ್ತೋತ್ರಂ}

ಶ್ರೀಭೈರವ್ಯುವಾಚ ।\\
ಧೂಮಾವತ್ಯಾ ಧರ್ಮರಾತ್ರ್ಯಾಃ ಕಥಯಸ್ವ ಮಹೇಶ್ವರ ।\\
ಸಹಸ್ರನಾಮಸ್ತೋತ್ರಮ್ಮೇ ಸರ್ವಸಿದ್ಧಿಪ್ರದಾಯಕಂ ॥೧॥

ಶ್ರೀಭೈರವ ಉವಾಚ।\\
ಶೃಣು ದೇವಿ ಮಹಾಮಾಯೇ ಪ್ರಿಯೇ ಪ್ರಾಣಸ್ವರೂಪಿಣಿ ।\\
ಸಹಸ್ರನಾಮಸ್ತೋತ್ರಮ್ಮೇ ಭವಶತ್ರುವಿನಾಶಂ ॥೨॥

ಓಂ ಅಸ್ಯ ಶ್ರೀಧೂಮಾವತೀಸಹಸ್ರನಾಮಸ್ತೋತ್ರಸ್ಯ ಪಿಪ್ಪಲಾದ ಋಷಿಃ ಪಂಕ್ತಿಶ್ಛಂದೋ ಧೂಮಾವತೀ ದೇವತಾ ಶತ್ರುವಿನಿಗ್ರಹೇ ಪಾಠೇ ವಿನಿಯೋಗಃ॥

ಧುಮಾ ಧೂಮವತೀ ಧೂಮಾ ಧೂಮಪಾನಪರಾಯಣಾ ।\\
ಧೌತಾ ಧೌತಗಿರಾ ಧಾಮ್ನೀ ಧೂಮೇಶ್ವರನಿವಾಸಿನೀ ॥೩॥

ಅನಂತಾಽನಂತರೂಪಾ ಚ ಅಕಾರಾಕಾರರೂಪಿಣೀ ।\\
ಆದ್ಯಾ ಆನಂದದಾನಂದಾ ಇಕಾರಾ ಇಂದ್ರರೂಪಿಣೀ ॥೪॥

ಧನಧಾನ್ಯಾರ್ತ್ಥವಾಣೀದಾ ಯಶೋಧರ್ಮಪ್ರಿಯೇಷ್ಟದಾ ।\\
ಭಾಗ್ಯಸೌಭಾಗ್ಯಭಕ್ತಿಸ್ಥಾ ಗೃಹಪರ್ವತವಾಸಿನೀ ॥೫॥

ರಾಮರಾವಣಸುಗ್ರೀವಮೋಹದಾ ಹನುಮತ್ಪ್ರಿಯಾ ।\\
ವೇದಶಾಸ್ತ್ರಪುರಾಣಜ್ಞಾ ಜ್ಯೋತಿಶ್ಛಂದಃಸ್ವರೂಪಿಣೀ ॥೬॥

ಚಾತುರ್ಯಚಾರುರುಚಿರಾ ರಂಜನಪ್ರೇಮತೋಷದಾ ।\\
ಕಮಲಾಸನಸುಧಾವಕ್ತ್ರಾ ಚಂದ್ರಹಾಸಾ ಸ್ಮಿತಾನನಾ ॥೭॥

ಚತುರಾ ಚಾರುಕೇಶೀ ಚ ಚತುರ್ವರ್ಗಪ್ರದಾ ಮುದಾ ।\\
ಕಲಾ ಕಾಲಧರಾ ಧೀರಾ ಧಾರಿಣೀ ವಸುನೀರದಾ ॥೮॥

ಹೀರಾ ಹೀರಕವರ್ಣಾಭಾ ಹರಿಣಾಯತಲೋಚನಾ ।\\
ದಂಭಮೋಹಕ್ರೋಧಲೋಭಸ್ನೇಹದ್ವೇಷಹರಾ ಪರಾ ॥೯॥

ನಾರದೇವಕರೀ ರಾಮಾ ರಾಮಾನಂದಮನೋಹರಾ ।\\
ಯೋಗಭೋಗಕ್ರೋಧಲೋಭಹರಾ ಹರನಮಸ್ಕೃತಾ ॥೧೦॥

ದಾನಮಾನಜ್ಞಾನಮಾನಪಾನಗಾನಸುಖಪ್ರದಾ ।\\
ಗಜಗೋಶ್ವಪದಾಗಂಜಾ ಭೂತಿದಾ ಭೂತನಾಶಿನೀ ॥೧೧॥

ಭವಭಾವಾ ತಥಾ ಬಾಲಾ ವರದಾ ಹರವಲ್ಲಭಾ ।\\
ಭಗಭಂಗಭಯಾ ಮಾಲಾ ಮಾಲತೀ ತಾಲನಾಹೃದಾ ॥೧೨॥

ಜಾಲವಾಲಹಾಲಕಾಲಕಪಾಲಪ್ರಿಯವಾದಿನೀ ।\\
ಕರಂಜಶೀಲಗುಂಜಾಢ್ಯಾ ಚೂತಾಂಕುರನಿವಾಸಿನೀ ॥೧೩॥

ಪನಸಸ್ಥಾ ಪಾನಸಕ್ತಾ ಪನಸೇಶಕುಟುಂಬಿನೀ ।\\
ಪಾವನೀ ಪಾವನಾಧಾರಾ ಪೂರ್ಣಾ ಪೂರ್ಣಮನೋರಥಾ ॥೧೪॥

ಪೂತಾ ಪೂತಕಲಾ ಪೌರಾ ಪುರಾಣಸುರಸುಂದರೀ ।\\
ಪರೇಶೀ ಪರದಾ ಪಾರಾ ಪರಾತ್ಮಾ ಪರಮೋಹಿನೀ ॥೧೫॥

ಜಗನ್ಮಾಯಾ ಜಗತ್ಕರ್ತ್ತ್ರೀ ಜಗತ್ಕೀರ್ತ್ತಿರ್ಜಗನ್ಮಯೀ ।\\
ಜನನೀ ಜಯಿನೀ ಜಾಯಾ ಜಿತಾ ಜಿನಜಯಪ್ರದಾ ॥೧೬॥

ಕೀರ್ತ್ತಿರ್ಜ್ಞಾನಧ್ಯಾನಮಾನದಾಯಿನೀ ದಾನವೇಶ್ವರೀ ।\\
ಕಾವ್ಯವ್ಯಾಕರಣಜ್ಞಾನಾ ಪ್ರಜ್ಞಾಪ್ರಜ್ಞಾನದಾಯಿನೀ ॥೧೭॥

ವಿಜ್ಞಾಜ್ಞಾ ವಿಜ್ಞಜಯದಾ ವಿಜ್ಞಾ ವಿಜ್ಞಪ್ರಪೂಜಿತಾ ।\\
ಪರಾವರೇಜ್ಯಾ ವರದಾ ಪಾರದಾ ಶಾರದಾ ದರಾ ॥೧೮॥

ದಾರಿಣೀ ದೇವದೂತೀ ಚ ಮದನಾ ಮದನಾಮದಾ ।\\
ಪರಮಜ್ಞಾನಗಮ್ಯಾ ಚ ಷರೇಶೀ ಪರಗಾ ಪರಾ ॥೧೯॥

ಯಜ್ಞಾ ಯಜ್ಞಪ್ರದಾ ಯಜ್ಞಜ್ಞಾನಕಾರ್ಯಕರೀ ಶುಭಾ ।\\
ಶೋಭಿನೀ ಶುಭ್ರಮಥಿನೀ ನಿಶುಂಭಾಸುರಮರ್ದ್ದಿನೀ ॥೨೦॥

ಶಾಂಭವೀ ಶಂಭುಪತ್ನೀ ಚ ಶಂಭುಜಾಯಾ ಶುಭಾನನಾ ।\\
ಶಾಂಕರೀ ಶಂಕರಾರಾಧ್ಯಾ ಸಂಧ್ಯಾ ಸಂಧ್ಯಾಸುಧರ್ಮಿಣೀ ॥೨೧॥

ಶತ್ರುಘ್ನೀ ಶತ್ರುಹಾ ಶತ್ರುಪ್ರದಾ ಶಾತ್ರವನಾಶಿನೀ ।\\
ಶೈವೀ ಶಿವಲಯಾ ಶೈಲಾ ಶೈಲರಾಜಪ್ರಿಯಾ ಸದಾ ॥೨೨॥

ಶರ್ವರೀ ಶವರೀ ಶಂಭುಃ ಸುಧಾಢ್ಯಾ ಸೌಧವಾಸಿನೀ ।\\
ಸಗುಣಾ ಗುಣರೂಪಾ ಚ ಗೌರವೀ ಭೈರವೀರವಾ ॥೨೩॥

ಗೌರಾಂಗೀ ಗೌರದೇಹಾ ಚ ಗೌರೀ ಗುರುಮತೀ ಗುರುಃ ।\\
ಗೌರ್ಗ್ಗೌರ್ಗವ್ಯಸ್ವರೂಪಾ ಚ ಗುಣಾನಂದಸ್ವರೂಪಿಣೀ ॥೨೪॥

ಗಣೇಶಗಣದಾ ಗುಣ್ಯಾ ಗುಣಾ ಗೌರವವಾಂಛಿತಾ ।\\
ಗಣಮಾತಾ ಗಣಾರಾಧ್ಯಾ ಗಣಕೋಟಿವಿನಾಶಿನೀ ॥೨೫॥

ದುರ್ಗಾ ದುರ್ಜ್ಜನಹಂತ್ರೀ ಚ ದುರ್ಜ್ಜನಪ್ರೀತಿದಾಯಿನೀ ।\\
ಸ್ವರ್ಗಾಪವರ್ಗದಾ ದಾತ್ರೀ ದೀನಾ ದೀನದಯಾವತೀ ॥೨೬॥

ದುರ್ನ್ನಿರೀಕ್ಷ್ಯಾ ದುರಾದುಃಸ್ಥಾ ದೌಃಸ್ಥಭಂಜನಕಾರಿಣೀ ।\\
ಶ್ವೇತಪಾಂಡುರಕೃಷ್ಣಾಭಾ ಕಾಲದಾ ಕಾಲನಾಶಿನೀ ॥೨೭॥

ಕರ್ಮನರ್ಮಕರೀ ನರ್ಮಾ ಧರ್ಮಾಧರ್ಮವಿನಾಶಿನೀ ।\\
ಗೌರೀ ಗೌರವದಾ ಗೋದಾ ಗಣದಾ ಗಾಯನಪ್ರಿಯಾ ॥೨೮॥

ಗಂಗಾ ಭಾಗೀರಥೀ ಭಂಗಾ ಭಗಾ ಭಾಗ್ಯವಿವರ್ದ್ಧಿನೀ ।\\
ಭವಾನೀ ಭವಹಂತ್ರೀ ಚ ಭೈರವೀ ಭೈರವೀಸಮಾ ॥೨೯॥

ಭೀಮಾ ಭೀಮರವಾ ಭೈಮೀ ಭೀಮಾನಂದಪ್ರದಾಯಿನೀ ।\\
ಶರಣ್ಯಾ ಶರಣಾ ಶಮ್ಯಾ ಶಶಿನೀ ಶಂಖನಾಶಿನೀ ॥೩೦॥

ಗುಣಾ ಗುಣಕರೀ ಗೌಣೀ ಪ್ರಿಯಾಪ್ರೀತಿಪ್ರದಾಯಿನೀ ।\\
ಜನಮೋಹನಕರ್ತ್ತ್ರೀ ಚ ಜಗದಾನಂದದಾಯಿನೀ ॥೩೧॥

ಜಿತಾ ಜಾಯಾ ಚ ವಿಜಯಾ ವಿಜಯಾ ಜಯದಾಯಿನೀ ।\\
ಕಾಮಾ ಕಾಲೀ ಕರಾಲಾಸ್ಯಾ ಖರ್ವಾ ಖಂಜಾ ಖರಾ ಗದಾ ॥೩೨॥

ಗರ್ವಾ ಗರುತ್ಮತೀ ಧರ್ಮಾ ಘರ್ಗ್ಘರಾ ಘೋರನಾದಿನೀ ।\\
ಚರಾಚರೀ ಚರಾರಾಧ್ಯಾ ಛಿನಾ ಛಿನ್ನಮನೋರಥಾ ॥೩೩॥

ಛಿನ್ನಮಸ್ತಾ ಜಯಾ ಜಾಪ್ಯಾ ಜಗಜ್ಜಾಯಾ ಚ ಝರ್ಜ್ಝರೀ ।\\
ಝಕಾರಾ ಝೀಷ್ಕೃತಿಷ್ಟೀಕಾ ಟಂಕಾ ಟಂಕಾರನಾದಿನೀ ॥೩೪॥

ಠೀಕಾ ಠಕ್ಕುರಠಕ್ಕಾಂಗೀ ಠಠಠಾಂಕಾರಢುಂಢುರಾ ।\\
ಢುಂಢೀತಾರಾಜತೀರ್ಣಾ ಚ ತಾಲಸ್ಥಾಭ್ರಮನಾಶಿನೀ ॥೩೫॥

ಥಕಾರಾ ಥಕರಾ ದಾತ್ರೀ ದೀಪಾ ದೀಪವಿನಾಶಿನೀ ।\\
ಧನ್ಯಾ ಧನಾ ಧನವತೀ ನರ್ಮದಾ ನರ್ಮಮೋದಿನೀ ॥೩೬॥

ಪದ್ಮಾ ಪದ್ಮಾವತೀ ಪೀತಾ ಸ್ಫಾಂತಾ ಫೂತ್ಕಾರಕಾರಿಣೀ ।\\
ಫುಲ್ಲಾ ಬ್ರಹ್ಮಮಯೀ ಬ್ರಾಹ್ಮೀ ಬ್ರಹ್ಮಾನಂದಪ್ರದಾಯಿನೀ ॥೩೭॥

ಭವಾರಾಧ್ಯಾ ಭವಾಧ್ಯಕ್ಷಾ ಭಗಾಲೀ ಮಂದಗಾಮಿನೀ ।\\
ಮದಿರಾ ಮದಿರೇಕ್ಷಾ ಚ ಯಶೋದಾ ಯಮಪೂಜಿತಾ ॥೩೮॥

ಯಾಮ್ಯಾ ರಾಮ್ಯಾ ರಾಮರೂಪಾ ರಮಣೀ ಲಲಿತಾ ಲತಾ ।\\
ಲಂಕೇಶ್ವರೀ ವಾಕ್ಪ್ರದಾ ವಾಚ್ಯಾ ಸದಾಶ್ರಮವಾಸಿನೀ ॥೩೯॥

ಶ್ರಾಂತಾ ಶಕಾರರೂಪಾ ಚ ಷಕಾರಖರವಾಹನಾ ।\\
ಸಹ್ಯಾದ್ರಿರೂಪಾ ಸಾನಂದಾ ಹರಿಣೀ ಹರಿರೂಪಿಣೀ ॥೪೦॥

ಹರಾರಾಧ್ಯಾ ವಾಲವಾಚಲವಂಗಪ್ರೇಮತೋಷಿತಾ ।\\
ಕ್ಷಪಾ ಕ್ಷಯಪ್ರದಾ ಕ್ಷೀರಾ ಅಕಾರಾದಿಸ್ವರೂಪಿಣೀ ॥೪೧॥

ಕಾಲಿಕಾ ಕಾಲಮೂರ್ತ್ತಿಶ್ಚ ಕಲಹಾ ಕಲಹಪ್ರಿಯಾ ।\\
ಶಿವಾ ಶಂದಾಯಿನೀ ಸೌಮ್ಯಾ ಶತ್ರುನಿಗ್ರಹಕಾರಿಣೀ ॥೪೨॥

ಭವಾನೀ ಭವಮೂರ್ತ್ತಿಶ್ಚ ಶರ್ವಾಣೀ ಸರ್ವಮಂಗಲಾ ।\\
ಶತ್ರುವಿದ್ದ್ರಾವಿಣೀ ಶೈವೀ ಶುಂಭಾಸುರವಿನಾಶಿನೀ ॥೪೩॥

ಧಕಾರಮಂತ್ರರೂಪಾ ಚ ಧೂಂಬೀಜಪರಿತೋಷಿತಾ ।\\
ಧನಾಧ್ಯಕ್ಷಸ್ತುತಾ ಧೀರಾ ಧರಾರೂಪಾ ಧರಾವತೀ ॥೪೪॥

ಚರ್ವಿಣೀ ಚಂದ್ರಪೂಜ್ಯಾ ಚ ಚ್ಛಂದೋರೂಪಾ ಛಟಾವತೀ ।\\
ಛಾಯಾ ಛಾಯಾವತೀ ಸ್ವಚ್ಛಾ ಛೇದಿನೀ ಮೇದಿನೀ ಕ್ಷಮಾ ॥೪೫॥

ವಲ್ಗಿನೀ ವರ್ದ್ಧಿನೀ ವಂದ್ಯಾ ವೇದಮಾತಾ ಬುಧಸ್ತುತಾ ।\\
ಧಾರಾ ಧಾರಾವತೀ ಧನ್ಯಾ ಧರ್ಮದಾನಪರಾಯಣಾ ॥೪೬॥

ಗರ್ವಿಣೀ ಗುರುಪೂಜ್ಯಾ ಚ ಜ್ಞಾನದಾತ್ರೀ ಗುಣಾನ್ವಿತಾ ।\\
ಧರ್ಮಿಣೀ ಧರ್ಮರೂಪಾ ಚ ಘಂಟಾನಾದಪರಾಯಣಾ ।॥೪೭॥

ಘಂಟಾನಿನಾದಿನೀ ಘೂರ್ಣಾ ಘೂರ್ಣಿತಾ ಘೋರರೂಪಿಣೀ ।\\
ಕಲಿಘ್ನೀ ಕಲಿದೂತೀ ಚ ಕಲಿಪೂಜ್ಯಾ ಕಲಿಪ್ರಿಯಾ ॥೪೮॥

ಕಾಲನಿರ್ಣಾಶಿನೀ ಕಾಲ್ಯಾ ಕಾವ್ಯದಾ ಕಾಲರೂಪಿಣೀ ।\\
ವರ್ಷಿಣೀ ವೃಷ್ಟಿದಾ ವೃಷ್ಟಿರ್ಮಹಾವೃಷ್ಟಿನಿವಾರಿಣೀ ॥೪೯॥

ಘಾತಿನೀ ಘಾಟಿನೀ ಘೋಂಟಾ ಘಾತಕೀ ಘನರೂಪಿಣೀ ।\\
ಧೂಂಬೀಜಾ ಧೂಂಜಪಾನಂದಾ ಧೂಂಬೀಜಜಪತೋಷಿತಾ ॥೫೦॥

ಧೂಂಧೂಂಬೀಜಜಪಾಸಕ್ತಾ ಧೂಂಧೂಂಬೀಜಪರಾಯಣಾ ।\\
ಧೂಂಕಾರಹರ್ಷಿಣೀ ಧೂಮಾ ಧನದಾ ಧನಗರ್ವಿತಾ ॥೫೧॥

ಪದ್ಮಾವತೀ ಪದ್ಮಮಾಲಾ ಪದ್ಮಯೋನಿಪ್ರಪೂಜಿತಾ ।\\
ಅಪಾರಾ ಪೂರಣೀ ಪೂರ್ಣಾ ಪೂರ್ಣಿಮಾಪರಿವಂದಿತಾ ॥೫೨॥

ಫಲದಾ ಫಲಭೋಕ್ತ್ರೀ ಚ ಫಲಿನೀ ಫಲದಾಯಿನೀ ।\\
ಫೂತ್ಕಾರಿಣೀ ಫಲಾವಾಪ್ತ್ರೀ ಫಲಭೋಕ್ತ್ರೀ ಫಲಾನ್ವಿತಾ ॥೫೩॥

ವಾರಿಣೀ ವರಣಪ್ರೀತಾ ವಾರಿಪಾಥೋಧಿಪಾರಗಾ ।\\
ವಿವರ್ಣಾ ಧೂಮ್ರನಯನಾ ಧೂಮ್ರಾಕ್ಷೀ ಧೂಮ್ರರೂಪಿಣೀ ॥೫೪॥

ನೀತಿರ್ನೀತಿಸ್ವರೂಪಾ ಚ ನೀತಿಜ್ಞಾ ನಯಕೋವಿದಾ ।\\
ತಾರಿಣೀ ತಾರರೂಪಾ ಚ ತತ್ತ್ವಜ್ಞಾನಪರಾಯಣಾ ॥೫೫॥

ಸ್ಥೂಲಾ ಸ್ಥೂಲಾಧರಾ ಸ್ಥಾತ್ರೀ ಉತ್ತಮಸ್ಥಾನವಾಸಿನೀ ।\\
ಸ್ಥೂಲಾ ಪದ್ಮಪದಸ್ಥಾನಾ ಸ್ಥಾನಭ್ರಷ್ಟಾ ಸ್ಥಲಸ್ಥಿತಾ ॥೫೬॥

ಶೋಷಿಣೀ ಶೋಭಿನೀ ಶೀತಾ ಶೀತಪಾನೀಯಪಾಯಿನೀ ।\\
ಶಾರಿಣೀ ಶಾಂಖಿನೀ ಶುದ್ಧಾ ಶಂಖಾಸುರವಿನಾಶಿನೀ ॥೫೭॥

ಶರ್ವರೀ ಶರ್ವರೀಪೂಜ್ಯಾ ಶರ್ವರೀಶಪ್ರಪೂಜಿತಾ ।\\
ಶರ್ವರೀಜಾಗ್ರಿತಾ ಯೋಗ್ಯಾ ಯೋಗಿನೀ ಯೋಗಿವಂದಿತಾ ॥೫೮॥

ಯೋಗಿನೀಗಣಸಂಸೇವ್ಯಾ ಯೋಗಿನೀ ಯೋಗಭಾವಿತಾ ।\\
ಯೋಗಮಾರ್ಗರತಾಯುಕ್ತಾ ಯೋಗಮಾರ್ಗಾನುಸಾರಿಣೀ ॥೫೯॥

ಯೋಗಭಾವಾ ಯೋಗಯುಕ್ತಾ ಯಾಮಿನೀಪತಿವಂದಿತಾ ।\\
ಅಯೋಗ್ಯಾ ಯೋಘಿನೀ ಯೋದ್ಧ್ರೀ ಯುದ್ಧಕರ್ಮವಿಶಾರದಾ ॥೬೦॥

ಯುದ್ಧಮಾರ್ಗರತಾನಾಂತಾ ಯುದ್ಧಸ್ಥಾನನಿವಾಸಿನೀ ।\\
ಸಿದ್ಧಾ ಸಿದ್ಧೇಶ್ವರೀ ಸಿದ್ಧಿಃ ಸಿದ್ಧಿಗೇಹನಿವಾಸಿನೀ ॥೬೧॥

ಸಿದ್ಧರೀತಿಸ್ಸಿದ್ಧಪ್ರೀತಿಃ ಸಿದ್ಧಾ ಸಿದ್ಧಾಂತಕಾರಿಣೀ ।\\
ಸಿದ್ಧಗಮ್ಯಾ ಸಿದ್ಧಪೂಜ್ಯಾ ಸಿದ್ಧಬಂದ್ಯಾ ಸುಸಿದ್ಧಿದಾ ॥೬೨॥

ಸಾಧಿನೀ ಸಾಧನಪ್ರೀತಾ ಸಾಧ್ಯಾ ಸಾಧನಕಾರಿಣೀ ।\\
ಸಾಧನೀಯಾ ಸಾಧ್ಯಸಾಧ್ಯಾ ಸಾಧ್ಯಸಂಘಸುಶೋಭಿನೀ ॥೬೩॥

ಸಾಧ್ವೀ ಸಾಧುಸ್ವಭಾವಾ ಸಾ ಸಾಧುಸಂತತಿದಾಯಿನೀ ।\\
ಸಾಧುಪೂಜ್ಯಾ ಸಾಧುವಂದ್ಯಾ ಸಾಧುಸಂದರ್ಶನೋದ್ಯತಾ ॥೬೪॥

ಸಾಧುದೃಷ್ಟಾ ಸಾಧುಪೃಷ್ಠಾ ಸಾಧುಪೋಷಣತತ್ಪರಾ ।\\
ಸಾತ್ತ್ವಿಕೀ ಸತ್ತ್ವಸಂಸಿದ್ಧಾ ಸತ್ತ್ವಸೇವ್ಯಾ ಸುಖೋದಯಾ ॥೬೫॥

ಸತ್ತ್ವವೃದ್ಧಿಕರೀ ಶಾಂತಾ ಸತ್ತ್ವಸಂಹರ್ಷಮಾನಸಾ ।\\
ಸತ್ತ್ವಜ್ಞಾನಾ ಸತ್ತ್ವವಿದ್ಯಾ ಸತ್ತ್ವಸಿದ್ಧಾಂತಕಾರಿಣೀ ॥೬೬॥

ಸತ್ತ್ವವೃದ್ಧಿಸ್ಸತ್ತ್ವಸಿದ್ಧಿಸ್ಸತ್ತ್ವಸಂಪನ್ನಮಾನಸಾ ।\\
ಚಾರುರೂಪಾ ಚಾರುದೇಹಾ ಚಾರುಚಂಚಲಲೋಚನಾ ॥೬೭॥

ಛದ್ಮಿನೀ ಛದ್ಮಸಂಕಲ್ಪಾ ಛದ್ಮವಾರ್ತ್ತಾ ಕ್ಷಮಾಪ್ರಿಯಾ ।\\
ಹಠಿನೀ ಹಠಸಂಪ್ರೀತಿರ್ಹಠವಾರ್ತ್ತಾ ಹಠೋದ್ಯಮಾ ॥೬೮॥

ಹಠಕಾರ್ಯಾ ಹಠಧರ್ಮಾ ಹಠಕರ್ಮಪರಾಯಣಾ ।\\
ಹಠಸಂಭೋಗನಿರತಾ ಹಠಾತ್ಕಾರರತಿಪ್ರಿಯಾ ॥೬೯॥

ಹಠಸಂಭೇದಿನೀ ಹೃದ್ಯಾ ಹೃದ್ಯವಾರ್ತ್ತಾ ಹರಿಪ್ರಿಯಾ ।\\
ಹರಿಣೀ ಹರಿಣೀದೃಷ್ಟಿರ್ಹರಿಣೀಮಾಂಸಭಕ್ಷಣಾ ॥೭೦॥

ಹರಿಣಾಕ್ಷೀ ಹರಿಣಪಾ ಹರಿಣೀಗಣಹರ್ಷದಾ ।\\
ಹರಿಣೀಗಣಸಂಹರ್ತ್ರೀ ಹರಿಣೀಪರಿಪೋಷಿಕಾ ॥೭೧॥

ಹರಿಣೀಮೃಗಯಾಸಕ್ತಾ ಹರಿಣೀಮಾನಪುರಸ್ಸರಾ ।\\
ದೀನಾ ದೀನಾಕೃತಿರ್ದೂನಾ ದ್ರಾವಿಣೀ ದ್ರವಿಣಪ್ರದಾ ॥೭೨॥

ದ್ರವಿಣಾಚಲಸಂವ್ವಾಸಾ ದ್ರವಿತಾ ದ್ರವ್ಯಸಂಯ್ಯುತಾ ।\\
ದೀರ್ಗ್ಘಾ ದೀರ್ಗ್ಘಪದಾ ದೃಶ್ಯಾ ದರ್ಶನೀಯಾ ದೃಢಾಕೃತಿಃ ॥೭೩॥

ದೃಢಾ ದ್ವಿಷ್ಟಮತಿರ್ದ್ದುಷ್ಟಾ ದ್ವೇಷಿಣೀ ದ್ವೇಷಿಭಂಜಿನೀ ।\\
ದೋಷಿಣೀ ದೋಷಸಂಯ್ಯುಕ್ತಾ ದುಷ್ಟಶತ್ರುವಿನಾಶಿನೀ ॥೭೪॥

ದೇವತಾರ್ತ್ತಿಹರಾ ದುಷ್ಟದೈತ್ಯಸಂಘವಿದಾರಿಣೀ ।\\
ದುಷ್ಟದಾನವಹಂತ್ರೀ ಚ ದುಷ್ಟದೈತ್ಯನಿಷೂದಿನೀ ॥೭೫॥

ದೇವತಾಪ್ರಾಣದಾ ದೇವೀ ದೇವದುರ್ಗತಿನಾಶಿನೀ ।\\
ನಟನಾಯಕಸಂಸೇವ್ಯಾ ನರ್ತ್ತಕೀ ನರ್ತ್ತಕಪ್ರಿಯಾ ॥೭೬॥

ನಾಟ್ಯವಿದ್ಯಾ ನಾಟ್ಯಕರ್ತ್ರೀ ನಾದಿನೀ ನಾದಕಾರಿಣೀ ।\\
ನವೀನನೂತನಾ ನವ್ಯಾ ನವೀನವಸ್ತ್ರಧಾರಿಣೀ ॥೭೭॥

ನವ್ಯಭೂಷಾ ನವ್ಯಮಾಲ್ಯಾ ನವ್ಯಾಲಂಕಾರಶೋಭಿತಾ ।\\
ನಕಾರವಾದಿನೀ ನಮ್ಯಾ ನವಭೂಷಣಭೂಷಿತಾ ॥೭೮॥

ನೀಚಮಾರ್ಗಾ ನೀಚಭೂಮಿರ್ನೀಚಮಾರ್ಗಗತಿರ್ಗತಿಃ ।\\
ನಾಥಸೇವ್ಯಾ ನಾಥಭಕ್ತಾ ನಾಥಾನಂದಪ್ರದಾಯಿನೀ ॥೭೯॥

ನಮ್ರಾ ನಮ್ರಗತಿರ್ನ್ನೇತ್ರೀ ನಿದಾನವಾಕ್ಯವಾದಿನೀ ।\\
ನಾರೀಮಧ್ಯಸ್ಥಿತಾ ನಾರೀ ನಾರೀಮಧ್ಯಗತಾಽನಘಾ ॥೮೦॥

ನಾರೀಪ್ರೀತಿ ನರಾರಾಧ್ಯಾ ನರನಾಮಪ್ರಕಾಶಿನೀ ।\\
ರತೀ ರತಿಪ್ರಿಯಾ ರಮ್ಯಾ ರತಿಪ್ರೇಮಾ ರತಿಪ್ರದಾ ॥೮೧॥

ರತಿಸ್ಥಾನಸ್ಥಿತಾರಾಧ್ಯಾ ರತಿಹರ್ಷಪ್ರದಾಯಿನೀ ।\\
ರತಿರೂಪಾ ರತಿಧ್ಯಾನಾ ರತಿರೀತಿಸುಧಾರಿಣೀ ॥೮೨॥

ರತಿರಾಸಮಹೋಲ್ಲಾಸಾ ರತಿರಾಸವಿಹಾರಿಣೀ ।\\
ರತಿಕಾಂತಸ್ತುತಾ ರಾಶೀ ರಾಶಿರಕ್ಷಣಕಾರಿಣೀ ॥೮೩॥

ಅರೂಪಾ ಶುದ್ಧರೂಪಾ ಚ ಸುರೂಪಾ ರೂಪಗರ್ವಿತಾ ।\\
ರೂಪಯೌವನಸಂಪನ್ನಾ ರೂಪರಾಶೀ ರಮಾವತೀ ॥೮೪॥

ರೋಧಿನೀ ರೋಷಿಣೀ ರುಷ್ಟಾ ರೋಷಿರುದ್ಧಾ ರಸಪ್ರದಾ ।\\
ಮಾದಿನೀ ಮದನಪ್ರೀತಾ ಮಧುಮತ್ತಾ ಮಧುಪ್ರದಾ ॥೮೫॥

ಮದ್ಯಪಾ ಮದ್ಯಪಧ್ಯೇಯಾ ಮದ್ಯಪಪ್ರಾಣರಕ್ಷಿಣೀ ।\\
ಮದ್ಯಪಾನಂದಸಂದಾತ್ರೀ ಮದ್ಯಪಪ್ರೇಮತೋಷಿತಾ ॥೮೬॥

ಮದ್ಯಪಾನರತಾ ಮತ್ತಾ ಮದ್ಯಪಾನವಿಹಾರಿಣೀ ।\\
ಮದಿರಾ ಮದಿರಾರಕ್ತಾ ಮದಿರಾಪಾನಹರ್ಷಿಣೀ ॥೮೭॥

ಮದಿರಾಪಾನಸಂತುಷ್ಟಾ ಮದಿರಾಪಾನಮೋಹಿನೀ ।\\
ಮದಿರಾಮಾನಸಾಮುಗ್ಧಾ ಮಾಧ್ವೀಪಾ ಮದಿರಾಪ್ರದಾ ॥೮೮॥

ಮಾಧ್ವೀದಾನಸದಾನಂದಾ ಮಾಧ್ವೀಪಾನರತಾ ಮದಾ ।\\
ಮೋದಿನೀ ಮೋದಸಂದಾತ್ರೀ ಮುದಿತಾ ಮೋದಮಾನಸಾ ॥೮೯॥

ಮೋದಕರ್ತ್ರೀ ಮೋದದಾತ್ರೀ ಮೋದಮಂಗಲಕಾರಿಣೀ ।\\
ಮೋದಕಾದಾನಸಂತುಷ್ಟಾ ಮೋದಕಗ್ರಹಣಕ್ಷಮಾ ॥೯೦॥

ಮೋದಕಾಲಬ್ಧಿಸಂಕ್ರುದ್ಧಾ ಮೋದಕಪ್ರಾಪ್ತಿತೋಷಿಣೀ ।\\
ಮಾಂಸಾದಾ ಮಾಂಸಸಂಭಕ್ಷಾ ಮಾಂಸಭಕ್ಷಣಹರ್ಷಿಣೀ ॥೯೧॥

ಮಾಂಸಪಾಕಪರಪ್ರೇಮಾ ಮಾಂಸಪಾಕಾಲಯಸ್ಥಿತಾ ।\\
ಮತ್ಸ್ಯಮಾಂಸಕೃತಾಸ್ವಾದಾ ಮಕಾರಪಂಚಕಾನ್ವಿತಾ ॥೯೨॥

ಮುದ್ರಾ ಮುದ್ರಾನ್ವಿತಾ ಮಾತಾ ಮಹಾಮೋಹಾ ಮನಸ್ವಿನೀ ।\\
ಮುದ್ರಿಕಾ ಮುದ್ರಿಕಾಯುಕ್ತಾ ಮುದ್ರಿಕಾಕೃತಲಕ್ಷಣಾ ॥೯೩॥

ಮುದ್ರಿಕಾಲಂಕೃತಾ ಮಾದ್ರೀ ಮಂದರಾಚಲವಾಸಿನೀ ।\\
ಮಂದರಾಚಲಸಂಸೇವ್ಯಾ ಮಂದರಾಚಲವಾಸಿನೀ ॥೯೪॥

ಮಂದರಧ್ಯೇಯಪಾದಾಬ್ಜಾ ಮಂದರಾರಣ್ಯವಾಸಿನೀ ।\\
ಮಂದುರಾವಾಸಿನೀ ಮಂದಾ ಮಾರಿಣೀ ಮಾರಿಕಾಮಿತಾ ॥೯೫॥

ಮಹಾಮಾರೀ ಮಹಾಮಾರೀಶಮಿನೀ ಶವಸಂಸ್ಥಿತಾ ।\\
ಶವಮಾಂಸಕೃತಾಹಾರಾ ಶ್ಮಶಾನಾಲಯವಾಸಿನೀ ॥೯೬॥

ಶ್ಮಶಾನಸಿದ್ಧಿಸಂಹೃಷ್ಟಾ ಶ್ಮಶಾನಭವನಸ್ಥಿತಾ ।\\
ಶ್ಮಶಾನಶಯನಾಗಾರಾ ಶ್ಮಶಾನಭಸ್ಮಲೇಪಿತಾ ॥೯೭॥

ಶ್ಮಶಾನಭಸ್ಮಭೀಮಾಂಗೀ ಶ್ಮಶಾನಾವಾಸಕಾರಿಣೀ ।\\
ಶಾಮಿನೀ ಶಮನಾರಾಧ್ಯಾ ಶಮನಸ್ತುತಿವಂದಿತಾ ॥೯೮॥

ಶಮನಾಚಾರಸಂತುಷ್ಟಾ ಶಮನಾಗಾರವಾಸಿನೀ ।\\
ಶಮನಸ್ವಾಮಿನೀ ಶಾಂತಿಃ ಶಾಂತಸಜ್ಜನಪೂಜಿತಾ ॥೯೯॥

ಶಾಂತಪೂಜಾಪರಾ ಶಾಂತಾ ಶಾಂತಾಗಾರಪ್ರಭೋಜಿನೀ ।\\
ಶಾಂತಪೂಜ್ಯಾ ಶಾಂತವಂದ್ಯಾ ಶಾಂತಗ್ರಹಸುಧಾರಿಣೀ ॥೧೦೦॥

ಶಾಂತರೂಪಾ ಶಾಂತಿಯುಕ್ತಾ ಶಾಂತಚಂದ್ರಪ್ರಭಾಽಮಲಾ ।\\
ಅಮಲಾ ವಿಮಲಾ ಮ್ಲಾನಾ ಮಾಲತೀ ಕುಂಜವಾಸಿನೀ ॥೧೦೧॥

ಮಾಲತೀಪುಷ್ಪಸಂಪ್ರೀತಾ ಮಾಲತೀಪುಷ್ಪಪೂಜಿತಾ ।\\
ಮಹೋಗ್ರಾ ಮಹತೀ ಮಧ್ಯಾ ಮಧ್ಯದೇಶನಿವಾಸಿನೀ ॥೧೦೨॥

ಮಧ್ಯಮಧ್ವನಿಸಂಪ್ರೀತಾ ಮಧ್ಯಮಧ್ವನಿಕಾರಿಣೀ ।\\
ಮಧ್ಯಮಾ ಮಧ್ಯಮಪ್ರೀತಿರ್ಮಧ್ಯಮಪ್ರೇಮಪೂರಿತಾ ॥೧೦೩॥

ಮಧ್ಯಾಂಗಚಿತ್ರವಸನಾ ಮಧ್ಯಖಿನ್ನಾ ಮಹೋದ್ಧತಾ ।\\
ಮಹೇಂದ್ರಕೃತಸಂಪೂಜಾ ಮಹೇಂದ್ರಪರಿವಂದಿತಾ ॥೧೦೪॥

ಮಹೇಂದ್ರಜಾಲಸಂಯ್ಯುಕ್ತಾ ಮಹೇಂದ್ರಜಾಲಕಾರಿಣೀ ।\\
ಮಹೇಂದ್ರಮಾನಿತಾಽಮಾನಾ ಮಾನಿನೀಗಣಮಧ್ಯಗಾ ॥೧೦೫॥

ಮಾನಿನೀಮಾನಸಂಪ್ರೀತಾ ಮಾನವಿಧ್ವಂಸಕಾರಿಣೀ ।\\
ಮಾನಿನ್ಯಾಕರ್ಷಿಣೀ ಮುಕ್ತಿರ್ಮುಕ್ತಿದಾತ್ರೀ ಸುಮುಕ್ತಿದಾ ॥೧೦೬॥

ಮುಕ್ತಿದ್ವೇಷಕರೀ ಮೂಲ್ಯಕಾರಿಣೀ ಮೂಲ್ಯಹಾರಿಣೀ ।\\
ನಿರ್ಮಲಾ ಮೂಲಸಂಯ್ಯುಕ್ತಾ ಮೂಲಿನೀ ಮೂಲಮಂತ್ರಿಣೀ ॥೧೦೭॥

ಮೂಲಮಂತ್ರಕೃತಾರ್ಹಾದ್ಯಾ ಮೂಲಮಂತ್ರಾರ್ಗ್ಘ್ಯಹರ್ಷಿಣೀ ।\\
ಮೂಲಮಂತ್ರಪ್ರತಿಷ್ಠಾತ್ರೀ ಮೂಲಮಂತ್ರಪ್ರಹರ್ಷಿಣೀ ॥೧೦೮॥

ಮೂಲಮಂತ್ರಪ್ರಸನ್ನಾಸ್ಯಾ ಮೂಲಮಂತ್ರಪ್ರಪೂಜಿತಾ ।\\
ಮೂಲಮಂತ್ರಪ್ರಣೇತ್ರೀ ಚ ಮೂಲಮಂತ್ರಕೃತಾರ್ಚ್ಚನಾ ॥೧೦೯॥

ಮೂಲಮಂತ್ರಪ್ರಹೃಷ್ಟಾತ್ಮಾ ಮೂಲವಿದ್ಯಾ ಮಲಾಪಹಾ ।\\
ವಿದ್ಯಾಽವಿದ್ಯಾ ವಟಸ್ಥಾ ಚ ವಟವೃಕ್ಷನಿವಾಸಿನೀ ॥೧೧೦॥

ವಟವೃಕ್ಷಕೃತಸ್ಥಾನಾ ವಟಪೂಜಾಪರಾಯಣಾ ।\\
ವಟಪೂಜಾಪರಿಪ್ರೀತಾ ವಟದರ್ಶನಲಾಲಸಾ ॥೧೧೧॥

ವಟಪೂಜಾ ಕೃತಾ ಹ್ಲಾದಾ ವಟಪೂಜಾವಿವರ್ದ್ಧಿನೀ ।\\
ವಶಿನೀ ವಿವಶಾರಾಧ್ಯಾ ವಶೀಕರಣಮಂತ್ರಿಣೀ ॥೧೧೨॥

ವಶೀಕರಣಸಂಪ್ರೀತಾ ವಶೀಕಾರಕಸಿದ್ಧಿದಾ ।\\
ಬಟುಕಾ ಬಟುಕಾರಾಧ್ಯಾ ಬಟುಕಾಹಾರದಾಯಿನೀ ॥೧೧೩॥

ಬಟುಕಾರ್ಚ್ಚಾಪರಾ ಪೂಜ್ಯಾ ಬಟುಕಾರ್ಚ್ಚಾವಿವರ್ದ್ಧಿನೀ ।\\
ಬಟುಕಾನಂದಕರ್ತ್ತ್ರೀ ಚ ಬಟುಕಪ್ರಾಣರಕ್ಷಿಣೀ ॥೧೧೪॥

ಬಟುಕೇಜ್ಯಾಪ್ರದಾಽಪಾರಾ ಪಾರಿಣೀ ಪಾರ್ವತೀಪ್ರಿಯಾ ।\\
ಪರ್ವತಾಗ್ರಕೃತಾವಾಸಾ ಪರ್ವತೇಂದ್ರಪ್ರಪೂಜಿತಾ ॥೧೧೫॥

ಪಾರ್ವತೀಪತಿಪೂಜ್ಯಾ ಚ ಪಾರ್ವತೀಪತಿಹರ್ಷದಾ ।\\
ಪಾರ್ವತೀಪತಿಬುದ್ಧಿಸ್ಥಾ ಪಾರ್ವತೀಪತಿಮೋಹಿನೀ ॥೧೧೬॥

ಪಾರ್ವತೀಯದ್ದ್ವಿಜಾರಾಧ್ಯಾ ಪರ್ವತಸ್ಥಾ ಪ್ರತಾರಿಣೀ ।\\
ಪದ್ಮಲಾ ಪದ್ಮಿನೀ ಪದ್ಮಾ ಪದ್ಮಮಾಲಾವಿಭೂಷಿತಾ ॥೧೧೭॥

ಪದ್ಮಜೇಡ್ಯಪದಾ ಪದ್ಮಮಾಲಾಲಂಕೃತಮಸ್ತಕಾ ।\\
ಪದ್ಮಾರ್ಚ್ಚಿತಪದದ್ವಂದ್ವಾ ಪದ್ಮಹಸ್ತಪಯೋಧಿಜಾ ॥೧೧೮॥

ಪಯೋಧಿಪಾರಗಂತ್ರೀ ಚ ಪಾಥೋಧಿಪರಿಕೀರ್ತ್ತಿತಾ ।\\
ಪಾಥೋಧಿಪಾರಗಾಪೂತಾ ಪಲ್ವಲಾಂಬುಪ್ರತರ್ಪಿತಾ ॥೧೧೯॥

ಪಲ್ವಲಾಂತಃ ಪಯೋಮಗ್ನಾ ಪವಮಾನಗತಿರ್ಗತಿಃ ।\\
ಪಯಃ ಪಾನಾ ಪಯೋದಾತ್ರೀ ಪಾನೀಯಪರಿಕಾಂಕ್ಷಿಣೀ ॥೧೨೦॥

ಪಯೋಜಮಾಲಾಭರಣಾ ಮುಂಡಮಾಲಾವಿಭೂಷಣಾ ।\\
ಮುಂಡಿನೀ ಮುಂಡಹಂತ್ರೀ ಚ ಮುಂಡಿತಾ ಮುಂಡಶೋಭಿತಾ ॥೧೨೧॥

ಮಣಿಭೂಷಾ ಮಣಿಗ್ರೀವಾ ಮಣಿಮಾಲಾವಿರಾಜಿತಾ ।\\
ಮಹಾಮೋಹಾ ಮಹಾಮರ್ಷಾ ಮಹಾಮಾಯಾ ಮಹಾಹವಾ ॥೧೨೨॥

ಮಾನವೀ ಮಾನವೀಪೂಜ್ಯಾ ಮನುವಂಶವಿವರ್ದ್ಧಿನೀ ।\\
ಮಠಿನೀ ಮಠಸಂಹಂತ್ರೀ ಮಠಸಂಪತ್ತಿಹಾರಿಣೀ ॥೧೨೩॥

ಮಹಾಕ್ರೋಧವತೀ ಮೂಢಾ ಮೂಢಶತ್ರುವಿನಾಶಿನೀ ।\\
ಪಾಠೀನಭೋಜಿನೀ ಪೂರ್ಣಾ ಪೂರ್ಣಹಾರವಿಹಾರಿಣೀ ॥೧೨೪॥

ಪ್ರಲಯಾನಲತುಲ್ಯಾಭಾ ಪ್ರಲಯಾನಲರೂಪಿಣೀ ।\\
ಪ್ರಲಯಾರ್ಣವಸಮ್ಮಗ್ನಾ ಪ್ರಲಯಾಬ್ಧಿವಿಹಾರಿಣೀ ॥೧೨೫॥

ಮಹಾಪ್ರಲಯಸಂಭೂತಾ ಮಹಾಪ್ರಲಯಕಾರಿಣೀ ।\\
ಮಹಾಪ್ರಲಯಸಂಪ್ರೀತಾ ಮಹಾಪ್ರಲಯಸಾಧಿನೀ ॥೧೨೬॥

ಮಹಾಮಹಾಪ್ರಲಯೇಜ್ಯಾ ಮಹಾಪ್ರಲಯಮೋದಿನೀ ।\\
ಛೇದಿನೀ ಛಿನ್ನಮುಂಡೋಗ್ರಾ ಛಿನ್ನಾ ಛಿನ್ನರುಹಾರ್ತ್ಥಿನೀ ॥೧೨೭॥

ಶತ್ರುಸಂಛೇದಿನೀ ಛನ್ನಾ ಕ್ಷೋದಿನೀ ಕ್ಷೋದಕಾರಿಣೀ ।\\
ಲಕ್ಷಿಣೀ ಲಕ್ಷಸಂಪೂಜ್ಯಾ ಲಕ್ಷಿತಾ ಲಕ್ಷಣಾನ್ವಿತಾ ॥೧೨೮॥

ಲಕ್ಷಶಸ್ತ್ರಸಮಾಯುಕ್ತಾ ಲಕ್ಷಬಾಣಪ್ರಮೋಚಿನೀ ।\\
ಲಕ್ಷಪೂಜಾಪರಾಽಲಕ್ಷ್ಯಾ ಲಕ್ಷಕೋದಂಡಖಂಡಿನೀ ॥೧೨೯॥

ಲಕ್ಷಕೋದಂಡಸಂಯ್ಯುಕ್ತಾ ಲಕ್ಷಕೋದಂಡಧಾರಿಣೀ ।\\
ಲಕ್ಷಲೀಲಾಲಯಾಲಭ್ಯಾ ಲಾಕ್ಷಾಗಾರನಿವಾಸಿನೀ ॥೧೩೦॥

ಲಕ್ಷಲೋಭಪರಾ ಲೋಲಾ ಲಕ್ಷಭಕ್ತಪ್ರಪೂಜಿತಾ ।\\
ಲೋಕಿನೀ ಲೋಕಸಂಪೂಜ್ಯಾ ಲೋಕರಕ್ಷಣಕಾರಿಣೀ ॥೧೩೧॥

ಲೋಕವಂದಿತಪಾದಾಬ್ಜಾ ಲೋಕಮೋಹನಕಾರಿಣೀ ।\\
ಲಲಿತಾ ಲಾಲಿತಾಲೀನಾ ಲೋಕಸಂಹಾರಕಾರಿಣೀ ॥೧೩೨॥

ಲೋಕಲೀಲಾಕರೀ ಲೋಕ್ಯಾಲೋಕಸಂಭವಕಾರಿಣೀ ।\\
ಭೂತಶುದ್ಧಿಕರೀ ಭೂತರಕ್ಷಿಣೀ ಭೂತತೋಷಿಣೀ ॥೧೩೩॥

ಭೂತವೇತಾಲಸಂಯ್ಯುಕ್ತಾ ಭೂತಸೇನಾಸಮಾವೃತಾ ।\\
ಭೂತಪ್ರೇತಪಿಶಾಚಾದಿಸ್ವಾಮಿನೀ ಭೂತಪೂಜಿತಾ ॥೧೩೪॥

ಡಾಕಿನೀ ಶಾಕಿನೀ ಡೇಯಾ ಡಿಂಡಿಮಾರಾವಕಾರಿಣೀ ।\\
ಡಮರೂವಾದ್ಯಸಂತುಷ್ಟಾ ಡಮರೂವಾದ್ಯಕಾರಿಣೀ ॥೧೩೫॥

ಹುಂಕಾರಕಾರಿಣೀ ಹೋತ್ರೀ ಹಾವಿನೀ ಹಾವನಾರ್ತ್ಥಿನೀ ।\\
ಹಾಸಿನೀ ಹ್ವಾಸಿನೀ ಹಾಸ್ಯಹರ್ಷಿಣೀ ಹಠವಾದಿನೀ ॥೧೩೬॥

ಅಟ್ಟಾಟ್ಟಹಾಸಿನೀ ಟೀಕಾ ಟೀಕಾನಿರ್ಮಾಣಕಾರಿಣೀ ।\\
ಟಂಕಿನೀ ಟಂಕಿತಾ ಟಂಕಾ ಟಂಕಮಾತ್ರಸುವರ್ಣದಾ ॥೧೩೭॥

ಟಂಕಾರಿಣೀ ಟಕಾರಾಢ್ಯಾ ಶತ್ರುತ್ರೋಟನಕಾರಿಣೀ ।\\
ತ್ರುಟಿತಾ ತ್ರುಟಿರೂಪಾ ಚ ತ್ರುಟಿಸಂದೇಹಕಾರಿಣೀ ॥೧೩೮॥

ತರ್ಷಿಣ ತೃಟ್ಪರಿಕ್ಲಾಂತಾ ಕ್ಷುತ್ಕ್ಷಾಮಾ ಕ್ಷುತ್ಪರಿಪ್ಲುತಾ ।\\
ಅಕ್ಷಿಣೀ ತಕ್ಷಿಣೀ ಭಿಕ್ಷಾಪ್ರಾರ್ತ್ಥಿನೀ ಶತ್ರುಭಕ್ಷಿಣೀ ॥೧೩೯॥

ಕಾಂಕ್ಷಿಣೀ ಕುಟ್ಟನೀ ಕ್ರೂರಾ ಕುಟ್ಟನೀವೇಶ್ಮವಾಸಿನೀ ।\\
ಕುಟ್ಟನೀಕೋಟಿಸಂಪೂಜ್ಯಾ ಕುಟ್ಟನೀಕುಲಮಾರ್ಗಿಣೀ ॥೧೪೦॥

ಕುಟ್ಟನೀಕುಲಸಂರಕ್ಷಾ ಕುಟ್ಟನೀಕುಲರಕ್ಷಿಣೀ ।\\
ಕಾಲಪಾಶಾವೃತಾ ಕನ್ಯಾ ಕುಮಾರೀಪೂಜನಪ್ರಿಯಾ ॥೧೪೧॥

ಕೌಮುದೀ ಕೌಮುದೀಹೃಷ್ಟಾ ಕರುಣಾದೃಷ್ಟಿಸಂಯ್ಯುತಾ ।\\
ಕೌತುಕಾಚಾರನಿಪುಣಾ ಕೌತುಕಾಗಾರವಾಸಿನೀ ॥೧೪೨॥

ಕಾಕಪಕ್ಷಧರಾ ಕಾಕರಕ್ಷಿಣೀ ಕಾಕಸಂವೃತಾ ।\\
ಕಾಕಾಂಕರಥಸಂಸ್ಥಾನಾ ಕಾಕಾಂಕಸ್ಯಂದನಾಸ್ಥಿತಾ ॥೧೪೩॥

ಕಾಕಿನೀ ಕಾಕದೃಷ್ಟಿಶ್ಚ ಕಾಕಭಕ್ಷಣದಾಯಿನೀ ।\\
ಕಾಕಮಾತಾ ಕಾಕಯೋನಿಃ ಕಾಕಮಂಡಲಮಂಡಿತಾ ॥೧೪೪॥

ಕಾಕದರ್ಶನಸಂಶೀಲಾ ಕಾಕಸಂಕೀರ್ಣಮಂದಿರಾ ।\\
ಕಾಕಧ್ಯಾನಸ್ಥದೇಹಾದಿಧ್ಯಾನಗಮ್ಯಾ ಧಮಾವೃತಾ ॥೧೪೫॥

ಧನಿನೀ ಧನಿಸಂಸೇವ್ಯಾ ಧನಚ್ಛೇದನಕಾರಿಣೀ ।\\
ಧುಂಧುರಾ ಧುಂಧುರಾಕಾರಾ ಧೂಮ್ರಲೋಚನಘಾತಿನೀ ॥೧೪೬॥

ಧೂಂಕಾರಿಣೀ ಚ ಧೂಮ್ಮಂತ್ರಪೂಜಿತಾ ಧರ್ಮನಾಶಿನೀ ।\\
ಧೂಮ್ರವರ್ಣಿನೀ ಧೂಮ್ರಾಕ್ಷೀ ಧೂಮ್ರಾಕ್ಷಾಸುರಘಾತಿನೀ ॥೧೪೭॥

ಧೂಂಬೀಜಜಪಸಂತುಷ್ಟಾ ಧೂಂಬೀಜಜಪಮಾನಸಾ ।\\
ಧೂಂಬೀಜಜಪಪೂಜಾರ್ಹಾ ಧೂಂಬೀಜಜಪಕಾರಿಣೀ ॥೧೪೮॥

ಧೂಂಬೀಜಾಕರ್ಷಿತಾ ಧೃಷ್ಯಾ ಧರ್ಷಿಣೀ ಧೃಷ್ಟಮಾನಸಾ ।\\
ಧೂಲೀಪ್ರಕ್ಷೇಪಿಣೀ ಧೂಲೀವ್ಯಾಪ್ತಧಮ್ಮಿಲ್ಲಧಾರಿಣೀ ॥೧೪೯॥

ಧೂಂಬೀಜಜಪಮಾಲಾಢ್ಯಾ ಧೂಂಬೀಜನಿಂದಕಾಂತಕಾ ।\\
ಧರ್ಮವಿದ್ವೇಷಿಣೀ ಧರ್ಮರಕ್ಷಿಣೀ ಧರ್ಮತೋಷಿತಾ ॥೧೫೦॥

ಧಾರಾಸ್ತಂಭಕರೀ ಧೂರ್ತಾ ಧಾರಾವಾರಿವಿಲಾಸಿನೀ ।\\
ಧಾಂಧೀಂಧೂಂಧೈಮ್ಮಂತ್ರವರ್ಣಾ ಧೌಂಧಃಸ್ವಾಹಾಸ್ವರೂಪಿಣೀ ॥೧೫೧॥

ಧರಿತ್ರೀಪೂಜಿತಾ ಧೂರ್ವಾ ಧಾನ್ಯಚ್ಛೇದನಕಾರಿಣೀ ।\\
ಧಿಕ್ಕಾರಿಣೀ ಸುಧೀಪೂಜ್ಯಾ ಧಾಮೋದ್ಯಾನನಿವಾಸಿನೀ ॥೧೫೨॥

ಧಾಮೋದ್ಯಾನಪಯೋದಾತ್ರೀ ಧಾಮಧೂಲೀಪ್ರಧೂಲಿತಾ ।\\
ಮಹಾಧ್ವನಿಮತೀ ಧೂಪ್ಯಾ ಧೂಪಾಮೋದಪ್ರಹರ್ಷಿಣೀ ॥೧೫೩॥

ಧೂಪಾದಾನಮತಿಪ್ರೀತಾ ಧೂಪದಾನವಿನೋದಿನೀ ।\\
ಧೀವರೀಗಣಸಂಪೂಜ್ಯಾ ಧೀವರೀವರದಾಯಿನೀ ॥೧೫೪॥

ಧೀವರೀಗಣಮಧ್ಯಸ್ಥಾ ಧೀವರೀಧಾಮವಾಸಿನೀ ।\\
ಧೀವರೀಗಣಗೋಪ್ತ್ರೀ ಚ ಧೀವರೀಗಣತೋಷಿತಾ ॥೧೫೫॥

ಧೀವರೀಧನದಾತ್ರೀ ಚ ಧೀವರೀಪ್ರಾಣರಕ್ಷಿಣೀ ।\\
ಧಾತ್ರೀಶಾ ಧಾತೃಸಂಪೂಜ್ಯಾ ಧಾತ್ರೀವೃಕ್ಷಸಮಾಶ್ರಯಾ ॥೧೫೬॥

ಧಾತ್ರೀಪೂಜನಕರ್ತ್ರೀ ಚ ಧಾತ್ರೀರೋಪಣಕಾರಿಣೀ ।\\
ಧೂಮ್ರಪಾನರತಾಸಕ್ತಾ ಧೂಮ್ರಪಾನರತೇಷ್ಟದಾ ॥೧೫೭॥

ಧೂಮ್ರಪಾನಕರಾನಂದಾ ಧೂಮ್ರವರ್ಷಣಕಾರಿಣೀ ।\\
ಧನ್ಯಶಬ್ದಶ್ರುತಿಪ್ರೀತಾ ಧುಂಧುಕಾರೀಜನಚ್ಛಿದಾ ॥೧೫೮॥

ಧುಂಧುಕಾರೀಷ್ಟಸಂದಾತ್ರೀ ಥುಂಧುಕಾರಿಸುಮುಕ್ತಿದಾ ।\\
ಧುಂಧುಕಾರ್ಯಾರಾಧ್ಯರೂಪಾ ಧುಂಧುಕಾರಿಮನಸ್ಸ್ಥಿತಾ ॥೧೫೯॥

ಧುಂಧುಕಾರಿಹಿತಾಕಾಂಕ್ಷಾ ಧುಂಧುಕಾರಿಹಿತೈಷಿಣೀ ।\\
ಧಿಂಧಿಮಾರಾವಿಣೀ ಧ್ಯಾತ್ರೀ ಧ್ಯಾನಗಮ್ಯಾ ಧನಾರ್ಥಿನೀ ॥೧೬೦॥

ಧೋರಿಣೀ ಧೋರಣಪ್ರೀತಾ ಧಾರಿಣೀ ಘೋರರೂಪಿಣೀ ।\\
ಧರಿತ್ರೀರಕ್ಷಿಣೀ ದೇವೀ ಧರಾಪ್ರಲಯಕಾರಿಣೀ ॥೧೬೧॥

ಧರಾಧರಸುತಾಽಶೇಷಧಾರಾಧರಸಮದ್ಯುತಿಃ ।\\
ಧನಾಧ್ಯಕ್ಷಾ ಧನಪ್ರಾಪ್ತಿರ್ದ್ಧನಧಾನ್ಯವಿವರ್ದ್ಧಿನೀ ॥೧೬೨॥

ಧನಾಕರ್ಷಣಕರ್ತ್ತ್ರೀ ಚ ಧನಾಹರಣಕಾರಿಣೀ ।\\
ಧನಚ್ಛೇದನಕರ್ತ್ರೀ ಚ ಧನಹೀನಾ ಧನಪ್ರಿಯಾ ॥೧೬೩॥

ಧನಸಂವೃದ್ಧಿಸಂಪನ್ನಾ ಧನದಾನಪರಾಯಣಾ ॥೧೬೪॥

ಧನಹೃಷ್ಟಾ ಧನಪುಷ್ಟಾ ದಾನಾಧ್ಯಯನಕಾರಿಣೀ ।\\
ಧನರಕ್ಷಾ ಧನಪ್ರಾಣಾ ಧನಾನಂದಕರೀ ಸದಾ ॥೧೬೫॥

ಶತ್ರುಹಂತ್ರೀ ಶವಾರೂಢಾ ಶತ್ರುಸಂಹಾರಕಾರಿಣೀ ।\\
ಶತ್ರುಪಕ್ಷಕ್ಷತಿಪ್ರೀತಾ ಶತ್ರುಪಕ್ಷನಿಷೂದಿನೀ ॥೧೬೬॥

ಶತ್ರುಗ್ರೀವಾಚ್ಛಿದಾಛಾಯಾ ಶತ್ರುಪದ್ಧತಿಖಂಡಿನೀ ।\\
ಶತ್ರುಪ್ರಾಣಹರಾಹಾರ್ಯಾ ಶತ್ರೂನ್ಮೂಲನಕಾರಿಣೀ ॥೧೬೭॥

ಶತ್ರುಕಾರ್ಯವಿಹಂತ್ರೀ ಚ ಸಾಂಗಶತ್ರುವಿನಾಶಿನೀ ।\\
ಸಾಂಗಶತ್ರುಕುಲಚ್ಛೇತ್ರೀ ಶತ್ರುಸದ್ಮಪ್ರದಾಯಿನೀ ॥೧೬೮॥

ಸಾಂಗಸಾಯುಧಸರ್ವಾರಿಸರ್ವಸಂಪತ್ತಿನಾಶಿನೀ ।\\
ಸಾಂಗಸಾಯುಧಸರ್ವಾರಿದೇಹಗೇಹಪ್ರದಾಹಿನೀ ॥೧೬೯॥

ಇತೀದಂಧೂಮರೂಪಿಣ್ಯಾಸ್ಸ್ತೋತ್ರನ್ನಾಮ ಸಹಸ್ರಕಂ ।\\
ಯಃ ಪಠೇಚ್ಛೂನ್ಯಭವನೇ ಸಧ್ವಾಂತೇ ಯತಮಾನಸಃ ॥೧೭೦॥

ಮದಿರಾಮೋದಯುಕ್ತೋ ವೈ ದೇವೀಧ್ಯಾನಪರಾಯಣಃ ।\\
ತಸ್ಯ ಶತ್ರುಃ ಕ್ಷಯಂ ಯಾತಿ ಯದಿ ಶಕ್ರಸಮೋಽಪಿ ವೈ ॥೧೭೧॥

ಭವಪಾಶಹರಂಪುಣ್ಯಂಧೂಮಾವತ್ಯಾಃ ಪ್ರಿಯಮ್ಮಹತ್ ।\\
ಸ್ತೋತ್ರಂ ಸಹಸ್ರನಾಮಾಖ್ಯಮ್ಮಮ ವಕ್ತ್ರಾದ್ವಿನಿರ್ಗತಂ ॥೧೭೨॥

ಪಠೇದ್ವಾ ಶೃಣುಯಾದ್ವಾಪಿ ಶತ್ರುಘಾತಕರೋ ಭವೇತ್ ।\\
ನ ದೇಯಂಪರಶಿಷ್ಯಾಯಾಽಭಕ್ತಾಯ ಪ್ರಾಣವಲ್ಲಭೇ ॥೧೭೩॥

ದೇಯಂ ಶಿಷ್ಯಾಯ ಭಕ್ತಾಯ ದೇವೀಭಕ್ತಿಪರಾಯ ಚ ।\\
ಇದಂ ರಹಸ್ಯಂಪರಮಂದುರ್ಲ್ಲಭಂದುಷ್ಟಚೇತಸಾಂ ॥೧೭೪॥

ಇತಿ ಧೂಮಾವತೀಸಹಸ್ರನಾಮಸ್ತೋತ್ರಂ ಸಂಪೂರ್ಣಂ॥


%=============================================================================================
\section{ಶ್ರೀಧೂಮಾವತ್ಯಷ್ಟೋತ್ತರಶತನಾಮಸ್ತೋತ್ರಂ}
\addcontentsline{toc}{section}{ಶ್ರೀಧೂಮಾವತ್ಯಷ್ಟೋತ್ತರಶತನಾಮಸ್ತೋತ್ರಂ}


ಈಶ್ವರ ಉವಾಚ ।\\
ಧೂಮಾವತೀ ಧೂಮ್ರವರ್ಣಾ ಧೂಮ್ರಪಾನಪರಾಯಣಾ ।\\
ಧೂಮ್ರಾಕ್ಷಮಥಿನೀ ಧನ್ಯಾ ಧನ್ಯಸ್ಥಾನನಿವಾಸಿನೀ ॥೧॥

ಅಘೋರಾಚಾರಸಂತುಷ್ಟಾ ಅಘೋರಾಚಾರಮಂಡಿತಾ ।\\
ಅಘೋರಮಂತ್ರಸಂಪ್ರೀತಾ ಅಘೋರಮಂತ್ರಪೂಜಿತಾ ॥೨॥

ಅಟ್ಟಾಟ್ಟಹಾಸನಿರತಾ ಮಲಿನಾಂಬರಧಾರಿಣೀ ।\\
ವೃದ್ಧಾ ವಿರೂಪಾ ವಿಧವಾ ವಿದ್ಯಾ ಚ ವಿರಲದ್ವಿಜಾ ॥೩॥

ಪ್ರವೃದ್ಧಘೋಣಾ ಕುಮುಖೀ ಕುಟಿಲಾ ಕುಟಿಲೇಕ್ಷಣಾ ।\\
ಕರಾಲೀ ಚ ಕರಾಲಾಸ್ಯಾ ಕಂಕಾಲೀ ಶೂರ್ಪಧಾರಿಣೀ ॥೪॥

ಕಾಕಧ್ವಜರಥಾರೂಢಾ ಕೇವಲಾ ಕಠಿನಾ ಕುಹೂಃ ।\\
ಕ್ಷುತ್ಪಿಪಾಸಾರ್ದಿತಾ ನಿತ್ಯಾ ಲಲಜ್ಜಿಹ್ವಾ ದಿಗಂಬರೀ ॥೫॥

ದೀರ್ಘೋದರೀ ದೀರ್ಘರವಾ ದೀರ್ಘಾಂಗೀ ದೀರ್ಘಮಸ್ತಕಾ ।\\
ವಿಮುಕ್ತಕುಂತಲಾ ಕೀರ್ತ್ಯಾ ಕೈಲಾಸಸ್ಥಾನವಾಸಿನೀ ॥೬॥

ಕ್ರೂರಾ ಕಾಲಸ್ವರೂಪಾ ಚ ಕಾಲಚಕ್ರಪ್ರವರ್ತಿನೀ ।\\
ವಿವರ್ಣಾ ಚಂಚಲಾ ದುಷ್ಟಾ ದುಷ್ಟವಿಧ್ವಂಸಕಾರಿಣೀ ॥೭॥

ಚಂಡೀ ಚಂಡಸ್ವರೂಪಾ ಚ ಚಾಮುಂಡಾ ಚಂಡನಿಸ್ವನಾ ।\\
ಚಂಡವೇಗಾ ಚಂಡಗತಿಶ್ಚಂಡಮುಂಡವಿನಾಶಿನೀ ॥೮॥

ಚಾಂಡಾಲಿನೀ ಚಿತ್ರರೇಖಾ ಚಿತ್ರಾಂಗೀ ಚಿತ್ರರೂಪಿಣೀ ।\\
ಕೃಷ್ಣಾ ಕಪರ್ದಿನೀ ಕುಲ್ಲಾ ಕೃಷ್ಣಾರೂಪಾ ಕ್ರಿಯಾವತೀ ॥೯॥

ಕುಂಭಸ್ತನೀ ಮಹೋನ್ಮತ್ತಾ ಮದಿರಾಪಾನವಿಹ್ವಲಾ ।\\
ಚತುರ್ಭುಜಾ ಲಲಜ್ಜಿಹ್ವಾ ಶತ್ರುಸಂಹಾರಕಾರಿಣೀ ॥೧೦॥

ಶವಾರೂಢಾ ಶವಗತಾ ಶ್ಮಶಾನಸ್ಥಾನವಾಸಿನೀ ।\\
ದುರಾರಾಧ್ಯಾ ದುರಾಚಾರಾ ದುರ್ಜನಪ್ರೀತಿದಾಯಿನೀ ॥೧೧॥

ನಿರ್ಮಾಂಸಾ ಚ ನಿರಾಕಾರಾ ಧೂತಹಸ್ತಾ ವರಾನ್ವಿತಾ ।\\
ಕಲಹಾ ಚ ಕಲಿಪ್ರೀತಾ ಕಲಿಕಲ್ಮಷನಾಶಿನೀ ॥೧೨॥

ಮಹಾಕಾಲಸ್ವರೂಪಾ ಚ ಮಹಾಕಾಲಪ್ರಪೂಜಿತಾ ।\\
ಮಹಾದೇವಪ್ರಿಯಾ ಮೇಧಾ ಮಹಾಸಂಕಟನಾಶಿನೀ ॥೧೩॥

ಭಕ್ತಪ್ರಿಯಾ ಭಕ್ತಗತಿರ್ಭಕ್ತಶತ್ರುವಿನಾಶಿನೀ ।\\
ಭೈರವೀ ಭುವನಾ ಭೀಮಾ ಭಾರತೀ ಭುವನಾತ್ಮಿಕಾ ॥೧೪॥

ಭೇರುಂಡಾ ಭೀಮನಯನಾ ತ್ರಿನೇತ್ರಾ ಬಹುರೂಪಿಣೀ ।\\
ತ್ರಿಲೋಕೇಶೀ ತ್ರಿಕಾಲಜ್ಞಾ ತ್ರಿಸ್ವರೂಪಾ ತ್ರಯೀತನುಃ ॥೧೫॥

ತ್ರಿಮೂರ್ತಿಶ್ಚ ತಥಾ ತನ್ವೀ ತ್ರಿಶಕ್ತಿಶ್ಚ ತ್ರಿಶೂಲಿನೀ ।\\
ಇತಿ ಧೂಮಾಮಹತ್ಸ್ತೋತ್ರಂ ನಾಮ್ನಾಮಷ್ಟೋತ್ತರಾತ್ಮಕಂ ॥೧೬॥

ಮಯಾ ತೇ ಕಥಿತಂ ದೇವಿ ಶತ್ರುಸಂಘವಿನಾಶನಂ ।\\
ಕಾರಾಗಾರೇ ರಿಪುಗ್ರಸ್ತೇ ಮಹೋತ್ಪಾತೇ ಮಹಾಭಯೇ ॥೧೭॥

ಇದಂ ಸ್ತೋತ್ರಂ ಪಠೇನ್ಮರ್ತ್ಯೋ ಮುಚ್ಯತೇ ಸರ್ವಸಂಕಟೈಃ ।\\
ಗುಹ್ಯಾದ್ಗುಹ್ಯತರಂ ಗುಹ್ಯಂ ಗೋಪನೀಯಂ ಪ್ರಯತ್ನತಃ ॥೧೮॥

ಚತುಷ್ಪದಾರ್ಥದಂ ನೄಣಾಂ ಸರ್ವಸಂಪತ್ಪ್ರದಾಯಕಂ ॥೧೯॥

\authorline{ಇತಿ ಶ್ರೀಧೂಮಾವತ್ಯಷ್ಟೋತ್ತರಶತನಾಮಸ್ತೋತ್ರಂ ಸಂಪೂರ್ಣಂ ॥}
%======================================================

\section{ಕಮಲಾಸಹಸ್ರನಾಮಸ್ತೋತ್ರಂ}
\addcontentsline{toc}{section}{ಕಮಲಾಸಹಸ್ರನಾಮಸ್ತೋತ್ರಂ}


ಓಂ ತಾಮಾಹ್ವಯಾಮಿ ಸುಭಗಾಂ ಲಕ್ಷ್ಮೀಂ ತ್ರೈಲೋಕ್ಯಪೂಜಿತಾಂ ।\\
ಏಹ್ಯೇಹಿ ದೇವಿ ಪದ್ಮಾಕ್ಷಿ ಪದ್ಮಾಕರಕೃತಾಲಯೇ ॥೧॥

ಆಗಚ್ಛಾಗಚ್ಛ ವರದೇ ಪಶ್ಯ ಮಾಂ ಸ್ವೇನ ಚಕ್ಷುಷಾ ।\\
ಆಯಾಹ್ಯಾಯಾಹಿ ಧರ್ಮಾರ್ಥಕಾಮಮೋಕ್ಷಮಯೇ ಶುಭೇ ॥೨॥

ಏವಂವಿಧೈಃ ಸ್ತುತಿಪದೈಃ ಸತ್ಯೈಃ ಸತ್ಯಾರ್ಥಸಂಸ್ತುತಾ ।\\
ಕನೀಯಸೀ ಮಹಾಭಾಗಾ ಚಂದ್ರೇಣ ಪರಮಾತ್ಮನಾ ॥೩॥

ನಿಶಾಕರಶ್ಚ ಸಾ ದೇವೀ ಭ್ರಾತರೌ ದ್ವೌ ಪಯೋನಿಧೇಃ ।\\
ಉತ್ಪನ್ನಮಾತ್ರೌ ತಾವಾಸ್ತಾಂ ಶಿವಕೇಶವಸಂಶ್ರಿತೌ ॥೪॥

ಸನತ್ಕುಮಾರಸ್ತಮೃಷಿಂ ಸಮಾಭಾಷ್ಯ ಪುರಾತನಂ ।\\
ಪ್ರೋಕ್ತವಾನಿತಿಹಾಸಂ ತು ಲಕ್ಷ್ಮ್ಯಾಃ ಸ್ತೋತ್ರಮನುತ್ತಮಂ ॥೫॥

ಅಥೇದೃಶಾನ್ಮಹಾಘೋರಾದ್ ದಾರಿದ್ರ್ಯಾನ್ನರಕಾತ್ಕಥಂ ।\\
ಮುಕ್ತಿರ್ಭವತಿ ಲೋಕೇಽಸ್ಮಿನ್ ದಾರಿದ್ರ್ಯಂ ಯಾತಿ ಭಸ್ಮತಾಂ ॥೬॥

ಸನತ್ಕುಮಾರ ಉವಾಚ ।\\
ಪೂರ್ವಂ ಕೃತಯುಗೇ ಬ್ರಹ್ಮಾ ಭಗವಾನ್ ಸರ್ವಲೋಕಕೃತ್ ।\\
ಸೃಷ್ಟಿಂ ನಾನಾವಿಧಾಂ ಕೃತ್ವಾ ಪಶ್ಚಾಚ್ಚಿ ನ್ತಾಮುಪೇಯಿವಾನ್ ॥೭॥

ಕಿಮಾಹಾರಾಃ ಪ್ರಜಾಸ್ತ್ವೇತಾಃ ಸಂಭವಿಷ್ಯಂತಿ ಭೂತಲೇ ।\\
ತಥೈವ ಚಾಸಾಂ ದಾರಿದ್ರ್ಯಾತ್ಕಥಮುತ್ತರಣಂ ಭವೇತ್ ॥೮॥

ದಾರಿದ್ರ್ಯಾನ್ಮರಣಂ ಶ್ರೇಯಸ್ತಿ್ವತಿ ಸಂಚಿಂತ್ಯ ಚೇತಸಿ ।\\
ಕ್ಷೀರೋದಸ್ಯೋತ್ತರೇ ಕೂಲೇ ಜಗಾಮ ಕಮಲೋದ್ಭವಃ ॥೯॥

ತತ್ರ ತೀವ್ರಂ ತಪಸ್ತಪ್ತ್ವಾ ಕದಾಚಿತ್ಪರಮೇಶ್ವರಂ ।\\
ದದರ್ಶ ಪುಂಡರೀಕಾಕ್ಷಂ ವಾಸುದೇವಂ ಜಗದ್ಗುರುಂ ॥೧೦॥

ಸರ್ವಜ್ಞಂ ಸರ್ವಶಕ್ತೀನಾಂ ಸರ್ವಾವಾಸಂ ಸನಾತನಂ ।\\
ಸರ್ವೇಶ್ವರಂ ವಾಸುದೇವಂ ವಿಷ್ಣುಂ ಲಕ್ಷ್ಮೀಪತಿಂ ಪ್ರಭುಂ ॥೧೧॥

ಸೋಮಕೋಟಿಪ್ರತೀಕಾಶಂ ಕ್ಷೀರೋದ ವಿಮಲೇ ಜಲೇ ।\\
ಅನಂತಭೋಗಶಯನಂ ವಿಶ್ರಾಂತಂ ಶ್ರೀನಿಕೇತನಂ ॥೧೨॥

ಕೋಟಿಸೂರ್ಯಪ್ರತೀಕಾಶಂ ಮಹಾಯೋಗೇಶ್ವರೇಶ್ವರಂ ।\\
ಯೋಗನಿದ್ರಾರತಂ ಶ್ರೀಶಂ ಸರ್ವಾವಾಸಂ ಸುರೇಶ್ವರಂ ॥೧೩॥

ಜಗದುತ್ಪತ್ತಿಸಂಹಾರಸ್ಥಿತಿಕಾರಣಕಾರಣಂ ।\\
ಲಕ್ಷ್ಮ್ಯಾದಿ ಶಕ್ತಿಕರಣಜಾತಮಂಡಲಮಂಡಿತಂ ॥೧೪॥

ಆಯುಧೈರ್ದೇಹವದ್ಭಿಶ್ಚ ಚಕ್ರಾದ್ಯೈಃ ಪರಿವಾರಿತಂ ।\\
ದುರ್ನಿರೀಕ್ಷ್ಯಂ ಸುರೈಃ ಸಿದ್ಧಃ ಮಹಾಯೋನಿಶತೈರಪಿ ॥೧೫॥

ಆಧಾರಂ ಸರ್ವಶಕ್ತೀನಾಂ ಪರಂ ತೇಜಃ ಸುದುಸ್ಸಹಂ ।\\
ಪ್ರಬುದ್ಧ ಂ ದೇವಮೀಶಾನಂ ದೃಷ್ಟ್ವಾ ಕಮಲಸಂಭವಃ ॥೧೬॥

ಶಿರಸ್ಯಂಜಲಿಮಾಧಾಯ ಸ್ತೋತ್ರಂ ಪೂರ್ವಮುವಾಚ ಹ ।\\
ಮನೋವಾಂಛಿತಸಿದ್ಧಿ ಂ ತ್ವಂ ಪೂರಯಸ್ವ ಮಹೇಶ್ವರ ॥೧೭॥

ಜಿತಂ ತೇ ಪುಂಡರೀಕ್ಷ ನಮಸ್ತೇ ವಿಶ್ವಭಾವನ ।\\
ನಮಸ್ತೇಽಸ್ತು ಹೃಷೀಕೇಶ ಮಹಾಪುರುಷಪೂರ್ವಜ ॥೧೮॥

ಸರ್ವೇಶ್ವರ ಜಯಾನಂದ ಸರ್ವಾವಾಸ ಪರಾತ್ಪರ ।\\
ಪ್ರಸೀದ ಮಮ ಭಕ್ತಸ್ಯ ಛಿಂಧಿ ಸಂದೇಹಜಂ ತಮಃ ॥೧೯॥

ಏವಂ ಸ್ತುತಃ ಸ ಭಗವಾನ್ ಬ್ರಹ್ಮ ಣಾಽವ್ಯಕ್ತಜನ್ಮನಾ ।\\
ಪ್ರಸಾದಾಭಿಮುಖಃ ಪ್ರಾಹ ಹರಿರ್ವಿಶ್ರಾಂತಲೋಚನಃ ॥೨೦॥

ಶ್ರೀಭಗವಾನುವಾಚ ।\\
ಹಿರಣ್ಯಗರ್ಭ ತುಷ್ಟೋಽಸ್ಮಿ ಬ್ರೂಹಿ ಯತ್ತೇಽಭಿವಾಂಛಿತಂ ।\\
ತದ್ವಕ್ಷ್ಯಾಮಿ ನ ಸಂದೇಹೋ ಭಕ್ತೋಽಸಿ ಮಮ ಸುವ್ರತ ॥೨೧॥

ಕೇಶವಾದ್ವಚನಂ ಶ್ರುತ್ವಾ ಕರುಣಾವಿಷ್ಟಚೇತನಃ ।\\
ಪ್ರತ್ಯುವಾಚ ಮಹಾಬುದ್ಧಿರ್ಭಗವಂತಂ ಜನಾರ್ದನಂ ॥೨೨॥

ಚತುರ್ವಿಧಂ ಭವಸ್ಯಾಸ್ಯ ಭೂತಸರ್ಗಸ್ಯ ಕೇಶವ ।\\
ಪರಿತ್ರಾಣಾಯ ಮೇ ಬ್ರೂಹಿ ರಹಸ್ಯಂ ಪರಮಾದ್ಭುತಂ ॥೨೩॥

ದಾರಿದ್ರ್ಯಶಮನಂ ಧನ್ಯಂ ಮನೋಜ್ಞಂ ಪಾವನಂ ಪರಂ ।\\
ಸರ್ವೇಶ್ವರ ಮಹಾಬುದ್ಧ ಸ್ವರೂಪಂ ಭೈರವಂ ಮಹತ್ ॥೨೪॥

ಶ್ರಿಯಃ ಸರ್ವಾತಿಶಾಯಿನ್ಯಾಸ್ತಥಾ ಜ್ಞಾನಂ ಚ ಶಾಶ್ವತಂ ।\\
ನಾಮಾನಿ ಚೈವ ಮುಖ್ಯಾನಿ ಯಾನಿ ಗೌಣಾನಿ ಚಾಚ್ಯುತ ॥೨೫॥

ತ್ವದ್ವಕ್ತ್ರಕಮಲೋತ್ಥಾನಿ ಶ್ರೇತುಮಿಚ್ಛಾಮಿ ತತ್ತ್ವತಃ ।\\
ಇತಿ ತಸ್ಯ ವಚಃ ಶ್ರುತ್ವಾ ಪ್ರತಿವಾಕ್ಯಮುವಾಚ ಸಃ ॥೨೬॥

ಶ್ರೀಭಗವಾನುವಾಚ ।\\
ಮಹಾವಿಭೂತಿಸಂಯುಕ್ತಾ ಷಾಡ್ಗುಣ್ಯವಪುಷಃ ಪ್ರಭೋ ।\\
ಭಗವದ್ವಾಸುದೇವಸ್ಯ ನಿತ್ಯಂ ಚೈಷಾಽನಪಾಯಿನೀ ॥೨೭॥

ಏಕೈವ ವರ್ತತೇಽಭಿನ್ನಾ ಜ್ಯೋತ್ಸ್ನೇವ ಹಿಮದೀಧಿತೇಃ ।\\
ಸರ್ವಶಕ್ತ್ಯಾತ್ಮಿಕಾ ಚೈವ ವಿಶ್ವಂ ವ್ಯಾಪ್ಯ ವ್ಯವಸ್ಥಿತಾ ॥೨೮॥

ಸರ್ವೈಶ್ವರ್ಯಗುಣೋಪೇತಾ ನಿತ್ಯಶುದ್ಧಸ್ವರೂಪಿಣೀ ।\\
ಪ್ರಾಣಶಕ್ತಿಃ ಪರಾ ಹ್ಯೇಷಾ ಸರ್ವೇಷಾಂ ಪ್ರಾಣಿನಾಂ ಭುವಿ ॥೨೯॥

ಶಕ್ತೀನಾಂ ಚೈವ ಸರ್ವಾಸಾಂ ಯೋನಿಭೂತಾ ಪರಾ ಕಲಾ ।\\
ಅಹಂ ತಸ್ಯಾಃ ಪರಂ ನಾಮ್ನಾಂ ಸಹಸ್ರಮಿದಮುತ್ತಮಂ ॥೩೦॥

ಶೃಣುಷ್ವಾವಹಿತೋ ಭೂತ್ವಾ ಪರಮೈಶ್ವರ್ಯಭೂತಿದಂ ।\\
ದೇವ್ಯಾಖ್ಯಾಸ್ಮೃತಿಮಾತ್ರೇಣ ದಾರಿದ್ರ್ಯಂ ಯಾತಿ ಭಸ್ಮತಾಂ ॥೩೧॥

ಕಮಲಾಸಹಸ್ರನಾಮಸ್ತೋತ್ರಂ ।

ಶ್ರೀಃ ಪದ್ಮಾ ಪ್ರಕೃತಿಃ ಸತ್ತ್ವಾ ಶಾಂತಾ ಚಿಚ್ಛಕ್ತಿರವ್ಯಯಾ ।\\
ಕೇವಲಾ ನಿಷ್ಕಲಾ ಶುದ್ಧಾ ವ್ಯಾಪಿನೀ ವ್ಯೋಮವಿಗ್ರಹಾ ॥೧॥

ವ್ಯೋಮಪದ್ಮಕೃತಾಧಾರಾ ಪರಾ ವ್ಯೋಮಾಮೃತೋದ್ಭವಾ ।\\
ನಿರ್ವ್ಯೋಮಾ ವ್ಯೋಮಮಧ್ಯಸ್ಥಾ ಪಂಚವ್ಯೋಮಪದಾಶ್ರಿತಾ ॥೨॥

ಅಚ್ಯುತಾ ವ್ಯೋಮನಿಲಯಾ ಪರಮಾನಂದರೂಪಿಣೀ ।\\
ನಿತ್ಯಶುದ್ಧಾ ನಿತ್ಯತೃಪ್ತಾ ನಿರ್ವಿಕಾರಾ ನಿರೀಕ್ಷಣಾ ॥೩॥

ಜ್ಞಾನಶಕ್ತಿಃ ಕರ್ತೃಶಕ್ತಿರ್ಭೋಕ್ತೃಶಕ್ತಿಃ ಶಿಖಾವಹಾ ।\\
ಸ್ನೇಹಾಭಾಸಾ ನಿರಾನಂದಾ ವಿಭೂತಿರ್ವಿಮಲಾಚಲಾ ॥೪॥

ಅನಂತಾ ವೈಷ್ಣವೀ ವ್ಯಕ್ತಾ ವಿಶ್ವಾನಂದಾ ವಿಕಾಸಿನೀ ।\\
ಶಕ್ತಿರ್ವಿಭಿನ್ನಸರ್ವಾರ್ತಿಃ ಸಮುದ್ರಪರಿತೋಷಿಣೀ ॥೫॥

ಮೂರ್ತಿಃ ಸನಾತನೀ ಹಾರ್ದೀ ನಿಸ್ತರಂಗಾ ನಿರಾಮಯಾ ।\\
ಜ್ಞಾನಜ್ಞೇಯಾ ಜ್ಞಾನಗಮ್ಯಾ ಜ್ಞಾನಜ್ಞೇಯವಿಕಾಸಿನೀ ॥೬॥

ಸ್ವಚ್ಛಂದಶಕ್ತಿರ್ಗಹನಾ ನಿಷ್ಕಂಪಾರ್ಚಿಃ ಸುನಿರ್ಮಲಾ ।\\
ಸ್ವರೂಪಾ ಸರ್ವಗಾ ಪಾರಾ ಬೃಂಹಿಣೀ ಸುಗುಣೋರ್ಜಿತಾ ॥೭॥

ಅಕಲಂಕಾ ನಿರಾಧಾರಾ ನಿಃಸಂಕಲ್ಪಾ ನಿರಾಶ್ರಯಾ ।\\
ಅಸಂಕೀರ್ಣಾ ಸುಶಾಂತಾ ಚ ಶಾಶ್ವತೀ ಭಾಸುರೀ ಸ್ಥಿರಾ ॥೮॥

ಅನೌಪಮ್ಯಾ ನಿರ್ವಿಕಲ್ಪಾ ನಿಯಂತ್ರೀ ಯಂತ್ರವಾಹಿನೀ ।\\
ಅಭೇದ್ಯಾ ಭೇದಿನೀ ಭಿನ್ನಾ ಭಾರತೀ ವೈಖರೀ ಖಗಾ ॥೯॥

ಅಗ್ರಾಹ್ಯಾ ಗ್ರಾಹಿಕಾ ಗೂಢಾ ಗಂಭೀರಾ ವಿಶ್ವಗೋಪಿನೀ ।\\
ಅನಿರ್ದೇಶ್ಯಾ ಪ್ರತಿಹತಾ ನಿರ್ಬೀಜಾ ಪಾವನೀ ಪರಾ ॥೧೦॥

ಅಪ್ರತರ್ಕ್ಯಾ ಪರಿಮಿತಾ ಭವಭ್ರಾಂತಿವಿನಾಶಿನೀ ।\\
ಏಕಾ ದ್ವಿರೂಪಾ ತ್ರಿವಿಧಾ ಅಸಂಖ್ಯಾತಾ ಸುರೇಶ್ವರೀ ॥೧೧॥

ಸುಪ್ರತಿಷ್ಠಾ ಮಹಾಧಾತ್ರೀ ಸ್ಥಿತಿರ್ವೃದ್ಧಿರ್ಧ್ರುವಾ ಗತಿಃ ।\\
ಈಶ್ವರೀ ಮಹಿಮಾ ಋದ್ಧಿಃ ಪ್ರಮೋದಾ ಉಜ್ಜ್ವಲೋದ್ಯಮಾ ॥೧೨॥

ಅಕ್ಷಯಾ ವರ್ದ್ಧಮಾನಾ ಚ ಸುಪ್ರಕಾಶಾ ವಿಹಂಗಮಾ ।\\
ನೀರಜಾ ಜನನೀ ನಿತ್ಯಾ ಜಯಾ ರೋಚಿಷ್ಮತೀ ಶುಭಾ ॥೧೩॥

ತಪೋನುದಾ ಚ ಜ್ವಾಲಾ ಚ ಸುದೀಪ್ತಿಶ್ಚಾಂಶುಮಾಲಿನೀ ।\\
ಅಪ್ರಮೇಯಾ ತ್ರಿಧಾ ಸೂಕ್ಷ್ಮಾ ಪರಾ ನಿರ್ವಾಣದಾಯಿನೀ ॥೧೪॥

ಅವದಾತಾ ಸುಶುದ್ಧಾ ಚ ಅಮೋಘಾಖ್ಯಾ ಪರಂಪರಾ ।\\
ಸಂಧಾನಕೀ ಶುದ್ಧವಿದ್ಯಾ ಸರ್ವಭೂತಮಹೇಶ್ವರೀ ॥೧೫॥

ಲಕ್ಷ್ಮೀಸ್ತುಷ್ಟಿರ್ಮಹಾಧೀರಾ ಶಾಂತಿರಾಪೂರಣಾನವಾ ।\\
ಅನುಗ್ರಹಾ ಶಕ್ತಿರಾದ್ಯಾ ಜಗಜ್ಜ್ಯೇಷ್ಠಾ ಜಗದ್ವಿಧಿಃ ॥೧೬॥

ಸತ್ಯಾ ಪ್ರಹ್ವಾ ಕ್ರಿಯಾ ಯೋಗ್ಯಾ ಅಪರ್ಣಾ ಹ್ಲಾದಿನೀ ಶಿವಾ ।\\
ಸಂಪೂರ್ಣಾಹ್ಲಾದಿನೀ ಶುದ್ಧಾ ಜ್ಯೋತಿಷ್ಮತ್ಯಮೃತಾವಹಾ ॥೧೭॥

ರಜೋವತ್ಯರ್ಕಪ್ರತಿಭಾಽಽಕರ್ಷಿಣೀ ಕರ್ಷಿಣೀ ರಸಾ ।\\
ಪರಾ ವಸುಮತೀ ದೇವೀ ಕಾಂತಿಃ ಶಾಂತಿರ್ಮತಿಃ ಕಲಾ ॥೧೮॥

ಕಲಾ ಕಲಂಕರಹಿತಾ ವಿಶಾಲೋದ್ದೀಪನೀ ರತಿಃ ।\\
ಸಂಬೋಧಿನೀ ಹಾರಿಣೀ ಚ ಪ್ರಭಾವಾ ಭವಭೂತಿದಾ ॥೧೯॥

ಅಮೃತಸ್ಯಂದಿನೀ ಜೀವಾ ಜನನೀ ಖಂಡಿಕಾ ಸ್ಥಿರಾ ।\\
ಧೂಮಾ ಕಲಾವತೀ ಪೂರ್ಣಾ ಭಾಸುರಾ ಸುಮತೀರಸಾ ॥೨೦॥

ಶುದ್ಧಾ ಧ್ವನಿಃ ಸೃತಿಃ ಸೃಷ್ಟಿರ್ವಿಕೃತಿಃ ಕೃಷ್ಟಿರೇವ ಚ ।\\
ಪ್ರಾಪಣೀ ಪ್ರಾಣದಾ ಪ್ರಹ್ವಾ ವಿಶ್ವಾ ಪಾಂಡುರವಾಸಿನೀ ॥೨೧॥

ಅವನಿರ್ವಜ್ರನಲಿಕಾ ಚಿತ್ರಾ ಬ್ರಹ್ಮಾಂಡವಾಸಿನೀ ।\\
ಅನಂತರೂಪಾನಂತಾತ್ಮಾನಂತಸ್ಥಾನಂತಸಂಭವಾ ॥೨೨॥

ಮಹಾಶಕ್ತಿಃ ಪ್ರಾಣಶಕ್ತಿಃ ಪ್ರಾಣದಾತ್ರೀ ಋತಂಭರಾ ।\\
ಮಹಾಸಮೂಹಾ ನಿಖಿಲಾ ಇಚ್ಛಾಧಾರಾ ಸುಖಾವಹಾ ॥೨೩॥

ಪ್ರತ್ಯಕ್ಷಲಕ್ಷ್ಮೀರ್ನಿಷ್ಕಂಪಾ ಪ್ರರೋಹಾಬುದ್ಧಿಗೋಚರಾ ।\\
ನಾನಾದೇಹಾ ಮಹಾವರ್ತಾ ಬಹುದೇಹವಿಕಾಸಿನೀ ॥೨೪॥

ಸಹಸ್ರಾಣೀ ಪ್ರಧಾನಾ ಚ ನ್ಯಾಯವಸ್ತುಪ್ರಕಾಶಿಕಾ ।\\
ಸರ್ವಾಭಿಲಾಷಪೂರ್ಣೇಚ್ಛಾ ಸರ್ವಾ ಸರ್ವಾರ್ಥಭಾಷಿಣೀ ॥೨೫॥

ನಾನಾಸ್ವರೂಪಚಿದ್ಧಾತ್ರೀ ಶಬ್ದಪೂರ್ವಾ ಪುರಾತನೀ ।\\
ವ್ಯಕ್ತಾವ್ಯಕ್ತಾ ಜೀವಕೇಶಾ ಸರ್ವೇಚ್ಛಾಪರಿಪೂರಿತಾ ॥೨೬॥

ಸಂಕಲ್ಪಸಿದ್ಧಾ ಸಾಂಖ್ಯೇಯಾ ತತ್ತ್ವಗರ್ಭಾ ಧರಾವಹಾ ।\\
ಭೂತರೂಪಾ ಚಿತ್ಸ್ವರೂಪಾ ತ್ರಿಗುಣಾ ಗುಣಗರ್ವಿತಾ ॥೨೭॥

ಪ್ರಜಾಪತೀಶ್ವರೀ ರೌದ್ರೀ ಸರ್ವಾಧಾರಾ ಸುಖಾವಹಾ ।\\
ಕಲ್ಯಾಣವಾಹಿಕಾ ಕಲ್ಯಾ ಕಲಿಕಲ್ಮಷನಾಶಿನೀ ॥೨೮॥

ನೀರೂಪೋದ್ಭಿನ್ನಸಂತಾನಾ ಸುಯಂತ್ರಾ ತ್ರಿಗುಣಾಲಯಾ ।\\
ಮಹಾಮಾಯಾ ಯೋಗಮಾಯಾ ಮಹಾಯೋಗೇಶ್ವರೀ ಪ್ರಿಯಾ ॥೨೯॥

ಮಹಾಸ್ತ್ರೀ ವಿಮಲಾ ಕೀರ್ತಿರ್ಜಯಾ ಲಕ್ಷ್ಮೀರ್ನಿರಂಜನಾ ।\\
ಪ್ರಕೃತಿರ್ಭಗವನ್ಮಾಯಾ ಶಕ್ತಿರ್ನಿದ್ರಾ ಯಶಸ್ಕರೀ ॥೩೦॥

ಚಿಂತಾ ಬುದ್ಧಿರ್ಯಶಃ ಪ್ರಜ್ಞಾ ಶಾಂತಿಃ ಸುಪ್ರೀತಿವರ್ದ್ಧಿನೀ ।\\
ಪ್ರದ್ಯುಮ್ನಮಾತಾ ಸಾಧ್ವೀ ಚ ಸುಖಸೌಭಾಗ್ಯಸಿದ್ಧಿದಾ ॥೩೧॥

ಕಾಷ್ಠಾ ನಿಷ್ಠಾ ಪ್ರತಿಷ್ಠಾ ಚ ಜ್ಯೇಷ್ಠಾ ಶ್ರೇಷ್ಠಾ ಜಯಾವಹಾ ।\\
ಸರ್ವಾತಿಶಾಯಿನೀ ಪ್ರೀತಿರ್ವಿಶ್ವಶಕ್ತಿರ್ಮಹಾಬಲಾ ॥೩೨॥

ವರಿಷ್ಠಾ ವಿಜಯಾ ವೀರಾ ಜಯಂತೀ ವಿಜಯಪ್ರದಾ ।\\
ಹೃದ್ಗೃಹಾ ಗೋಪಿನೀ ಗುಹ್ಯಾ ಗಣಗಂಧರ್ವಸೇವಿತಾ ॥೩೩॥

ಯೋಗೀಶ್ವರೀ ಯೋಗಮಾಯಾ ಯೋಗಿನೀ ಯೋಗಸಿದ್ಧಿದಾ ।\\
ಮಹಾಯೋಗೇಶ್ವರವೃತಾ ಯೋಗಾ ಯೋಗೇಶ್ವರಪ್ರಿಯಾ ॥೩೪॥

ಬ್ರಹ್ಮೇಂದ್ರರುದ್ರನಮಿತಾ ಸುರಾಸುರವರಪ್ರದಾ ।\\
ತ್ರಿವರ್ತ್ಮಗಾ ತ್ರಿಲೋಕಸ್ಥಾ ತ್ರಿವಿಕ್ರಮಪದೋದ್ಭವಾ ॥೩೫॥

ಸುತಾರಾ ತಾರಿಣೀ ತಾರಾ ದುರ್ಗಾ ಸಂತಾರಿಣೀ ಪರಾ ।\\
ಸುತಾರಿಣೀ ತಾರಯಂತೀ ಭೂರಿತಾರೇಶ್ವರಪ್ರಭಾ ॥೩೬॥

ಗುಹ್ಯವಿದ್ಯಾ ಯಜ್ಞವಿದ್ಯಾ ಮಹಾವಿದ್ಯಾ ಸುಶೋಭಿತಾ ।\\
ಅಧ್ಯಾತ್ಮವಿದ್ಯಾ ವಿಘ್ನೇಶೀ ಪದ್ಮಸ್ಥಾ ಪರಮೇಷ್ಠಿನೀ ॥೩೭॥

ಆನ್ವೀಕ್ಷಿಕೀ ತ್ರಯೀ ವಾರ್ತಾ ದಂಡನೀತಿರ್ನಯಾತ್ಮಿಕಾ ।\\
ಗೌರೀ ವಾಗೀಶ್ವರೀ ಗೋಪ್ತ್ರೀ ಗಾಯತ್ರೀ ಕಮಲೋದ್ಭವಾ ॥೩೮॥

ವಿಶ್ವಂಭರಾ ವಿಶ್ವರೂಪಾ ವಿಶ್ವಮಾತಾ ವಸುಪ್ರದಾ ।\\
ಸಿದ್ಧಿಃ ಸ್ವಾಹಾ ಸ್ವಧಾ ಸ್ವಸ್ತಿಃ ಸುಧಾ ಸರ್ವಾರ್ಥಸಾಧಿನೀ ॥೩೯॥

ಇಚ್ಛಾ ಸೃಷ್ಟಿರ್ದ್ಯುತಿರ್ಭೂತಿಃ ಕೀರ್ತಿಃ ಶ್ರದ್ಧಾ ದಯಾಮತಿಃ ।\\
ಶ್ರುತಿರ್ಮೇಧಾ ಧೃತಿರ್ಹ್ರೀಃ ಶ್ರೀರ್ವಿದ್ಯಾ ವಿಬುಧವಂದಿತಾ ॥೪೦॥

ಅನಸೂಯಾ ಘೃಣಾ ನೀತಿರ್ನಿರ್ವೃತಿಃ ಕಾಮಧುಕ್ಕರಾ ।\\
ಪ್ರತಿಜ್ಞಾ ಸಂತತಿರ್ಭೂತಿರ್ದ್ಯೌಃ ಪ್ರಜ್ಞಾ ವಿಶ್ವಮಾನಿನೀ ॥೪೧॥

ಸ್ಮೃತಿರ್ವಾಗ್ವಿಶ್ವಜನನೀ ಪಶ್ಯಂತೀ ಮಧ್ಯಮಾ ಸಮಾ ।\\
ಸಂಧ್ಯಾ ಮೇಧಾ ಪ್ರಭಾ ಭೀಮಾ ಸರ್ವಾಕಾರಾ ಸರಸ್ವತೀ ॥೪೨॥

ಕಾಂಕ್ಷಾ ಮಾಯಾ ಮಹಾಮಾಯಾ ಮೋಹಿನೀ ಮಾಧವಪ್ರಿಯಾ ।\\
ಸೌಮ್ಯಾಭೋಗಾ ಮಹಾಭೋಗಾ ಭೋಗಿನೀ ಭೋಗದಾಯಿನೀ ॥೪೩॥

ಸುಧೌತಕನಕಪ್ರಖ್ಯಾ ಸುವರ್ಣಕಮಲಾಸನಾ ।\\
ಹಿರಣ್ಯಗರ್ಭಾ ಸುಶ್ರೋಣೀ ಹಾರಿಣೀ ರಮಣೀ ರಮಾ ॥೪೪॥

ಚಂದ್ರಾ ಹಿರಣ್ಮಯೀ ಜ್ಯೋತ್ಸ್ನಾ ರಮ್ಯಾ ಶೋಭಾ ಶುಭಾವಹಾ ।\\
ತ್ರೈಲೋಕ್ಯಮಂಡನಾ ನಾರೀ ನರೇಶ್ವರವರಾರ್ಚಿತಾ ॥೪೫॥

ತ್ರೈಲೋಕ್ಯಸುಂದರೀ ರಾಮಾ ಮಹಾವಿಭವವಾಹಿನೀ ।\\
ಪದ್ಮಸ್ಥಾ ಪದ್ಮನಿಲಯಾ ಪದ್ಮಮಾಲಾವಿಭೂಷಿತಾ ॥೪೬॥

ಪದ್ಮಯುಗ್ಮಧರಾ ಕಾಂತಾ ದಿವ್ಯಾಭರಣಭೂಷಿತಾ ।\\
ವಿಚಿತ್ರರತ್ನಮುಕುಟಾ ವಿಚಿತ್ರಾಂಬರಭೂಷಣಾ ॥೪೭॥

ವಿಚಿತ್ರಮಾಲ್ಯಗಂಧಾಢ್ಯಾ ವಿಚಿತ್ರಾಯುಧವಾಹನಾ ।\\
ಮಹಾನಾರಾಯಣೀ ದೇವೀ ವೈಷ್ಣವೀ ವೀರವಂದಿತಾ ॥೪೮॥

ಕಾಲಸಂಕರ್ಷಿಣೀ ಘೋರಾ ತತ್ತ್ವಸಂಕರ್ಷಿಣೀಕಲಾ ।\\
ಜಗತ್ಸಂಪೂರಣೀ ವಿಶ್ವಾ ಮಹಾವಿಭವಭೂಷಣಾ ॥೪೯॥

ವಾರುಣೀ ವರದಾ ವ್ಯಾಖ್ಯಾ ಘಂಟಾಕರ್ಣವಿರಾಜಿತಾ ।\\
ನೃಸಿಂಹೀ ಭೈರವೀ ಬ್ರಾಹ್ಮೀ ಭಾಸ್ಕರೀ ವ್ಯೋಮಚಾರಿಣೀ ॥೫೦॥

ಐಂದ್ರೀ ಕಾಮಧೇನುಃ ಸೃಷ್ಟಿಃ ಕಾಮಯೋನಿರ್ಮಹಾಪ್ರಭಾ ।\\
ದೃಷ್ಟಾ ಕಾಮ್ಯಾ ವಿಶ್ವಶಕ್ತಿರ್ಬೀಜಗತ್ಯಾತ್ಮದರ್ಶನಾ ॥೫೧॥

ಗರುಡಾರೂಢಹೃದಯಾ ಚಾಂದ್ರೀ ಶ್ರೀರ್ಮಧುರಾನನಾ ।\\
ಮಹೋಗ್ರರೂಪಾ ವಾರಾಹೀ ನಾರಸಿಂಹೀ ಹತಾಸುರಾ ॥೫೨॥

ಯುಗಾಂತಹುತಭುಗ್ಜ್ವಾಲಾ ಕರಾಲಾ ಪಿಂಗಲಾಕಲಾ ।\\
ತ್ರೈಲೋಕ್ಯಭೂಷಣಾ ಭೀಮಾ ಶ್ಯಾಮಾ ತ್ರೈಲೋಕ್ಯಮೋಹಿನೀ ॥೫೩॥

ಮಹೋತ್ಕಟಾ ಮಹಾರಕ್ತಾ ಮಹಾಚಂಡಾ ಮಹಾಸನಾ ।\\
ಶಂಖಿನೀ ಲೇಖಿನೀ ಸ್ವಸ್ಥಾ ಲಿಖಿತಾ ಖೇಚರೇಶ್ವರೀ ॥೫೪॥

ಭದ್ರಕಾಲೀ ಚೈಕವೀರಾ ಕೌಮಾರೀ ಭವಮಾಲಿನೀ ।\\
ಕಲ್ಯಾಣೀ ಕಾಮಧುಗ್ಜ್ವಾಲಾಮುಖೀ ಚೋತ್ಪಲಮಾಲಿಕಾ ॥೫೫॥

ಬಾಲಿಕಾ ಧನದಾ ಸೂರ್ಯಾ ಹೃದಯೋತ್ಪಲಮಾಲಿಕಾ ।\\
ಅಜಿತಾ ವರ್ಷಿಣೀ ರೀತಿರ್ಭರುಂಡಾ ಗರುಡಾಸನಾ ॥೫೬॥

ವೈಶ್ವಾನರೀ ಮಹಾಮಾಯಾ ಮಹಾಕಾಲೀ ವಿಭೀಷಣಾ ।\\
ಮಹಾಮಂದಾರವಿಭವಾ ಶಿವಾನಂದಾ ರತಿಪ್ರಿಯಾ ॥೫೭॥

ಉದ್ರೀತಿಃ ಪದ್ಮಮಾಲಾ ಚ ಧರ್ಮವೇಗಾ ವಿಭಾವನೀ ।\\
ಸತ್ಕ್ರಿಯಾ ದೇವಸೇನಾ ಚ ಹಿರಣ್ಯರಜತಾಶ್ರಯಾ ॥೫೮॥

ಸಹಸಾವರ್ತಮಾನಾ ಚ ಹಸ್ತಿನಾದಪ್ರಬೋಧಿನೀ ।\\
ಹಿರಣ್ಯಪದ್ಮವರ್ಣಾ ಚ ಹರಿಭದ್ರಾ ಸುದುರ್ದ್ಧರಾ ॥೫೯॥

ಸೂರ್ಯಾ ಹಿರಣ್ಯಪ್ರಕಟಸದೃಶೀ ಹೇಮಮಾಲಿನೀ ।\\
ಪದ್ಮಾನನಾ ನಿತ್ಯಪುಷ್ಟಾ ದೇವಮಾತಾ ಮೃತೋದ್ಭವಾ ॥೬೦॥

ಮಹಾಧನಾ ಚ ಯಾ ಶೃಂಗೀ ಕರ್ದ್ದಮೀ ಕಂಬುಕಂಧರಾ ।\\
ಆದಿತ್ಯವರ್ಣಾ ಚಂದ್ರಾಭಾ ಗಂಧದ್ವಾರಾ ದುರಾಸದಾ ॥೬೧॥

ವರಾಚಿತಾ ವರಾರೋಹಾ ವರೇಣ್ಯಾ ವಿಷ್ಣುವಲ್ಲಭಾ ।\\
ಕಲ್ಯಾಣೀ ವರದಾ ವಾಮಾ ವಾಮೇಶೀ ವಿಂಧ್ಯವಾಸಿನೀ ॥೬೨॥

ಯೋಗನಿದ್ರಾ ಯೋಗರತಾ ದೇವಕೀ ಕಾಮರೂಪಿಣೀ ।\\
ಕಂಸವಿದ್ರಾವಿಣೀ ದುರ್ಗಾ ಕೌಮಾರೀ ಕೌಶಿಕೀ ಕ್ಷಮಾ ॥೬೩॥

ಕಾತ್ಯಾಯನೀ ಕಾಲರಾತ್ರಿರ್ನಿಶಿತೃಪ್ತಾ ಸುದುರ್ಜಯಾ ।\\
ವಿರೂಪಾಕ್ಷೀ ವಿಶಾಲಾಕ್ಷೀ ಭಕ್ತಾನಾಂಪರಿರಕ್ಷಿಣೀ ॥೬೪॥

ಬಹುರೂಪಾ ಸ್ವರೂಪಾ ಚ ವಿರೂಪಾ ರೂಪವರ್ಜಿತಾ ।\\
ಘಂಟಾನಿನಾದಬಹುಲಾ ಜೀಮೂತಧ್ವನಿನಿಃಸ್ವನಾ ॥೬೫॥

ಮಹಾದೇವೇಂದ್ರಮಥಿನೀ ಭ್ರುಕುಟೀಕುಟಿಲಾನನಾ ।\\
ಸತ್ಯೋಪಯಾಚಿತಾ ಚೈಕಾ ಕೌಬೇರೀ ಬ್ರಹ್ಮಚಾರಿಣೀ ॥೬೬॥

ಆರ್ಯಾ ಯಶೋದಾ ಸುತದಾ ಧರ್ಮಕಾಮಾರ್ಥಮೋಕ್ಷದಾ ।\\
ದಾರಿದ್ರ್ಯದುಃಖಶಮನೀ ಘೋರದುರ್ಗಾರ್ತಿನಾಶಿನೀ ॥೬೭॥

ಭಕ್ತಾರ್ತಿಶಮನೀ ಭವ್ಯಾ ಭವಭರ್ಗಾಪಹಾರಿಣೀ ।\\
ಕ್ಷೀರಾಬ್ಧಿತನಯಾ ಪದ್ಮಾ ಕಮಲಾ ಧರಣೀಧರಾ ॥೬೮॥

ರುಕ್ಮಿಣೀ ರೋಹಿಣೀ ಸೀತಾ ಸತ್ಯಭಾಮಾ ಯಶಸ್ವಿನೀ ।\\
ಪ್ರಜ್ಞಾಧಾರಾಮಿತಪ್ರಜ್ಞಾ ವೇದಮಾತಾ ಯಶೋವತೀ ॥೬೯॥

ಸಮಾಧಿರ್ಭಾವನಾ ಮೈತ್ರೀ ಕರುಣಾ ಭಕ್ತವತ್ಸಲಾ ।\\
ಅಂತರ್ವೇದೀ ದಕ್ಷಿಣಾ ಚ ಬ್ರಹ್ಮಚರ್ಯಪರಾಗತಿಃ ॥೭೦॥

ದೀಕ್ಷಾ ವೀಕ್ಷಾ ಪರೀಕ್ಷಾ ಚ ಸಮೀಕ್ಷಾ ವೀರವತ್ಸಲಾ ।\\
ಅಂಬಿಕಾ ಸುರಭಿಃ ಸಿದ್ಧಾ ಸಿದ್ಧವಿದ್ಯಾಧರಾರ್ಚಿತಾ ॥೭೧॥

ಸುದೀಕ್ಷಾ ಲೇಲಿಹಾನಾ ಚ ಕರಾಲಾ ವಿಶ್ವಪೂರಕಾ ।\\
ವಿಶ್ವಸಂಧಾರಿಣೀ ದೀಪ್ತಿಸ್ತಾಪನೀ ತಾಂಡವಪ್ರಿಯಾ ॥೭೨॥

ಉದ್ಭವಾ ವಿರಜಾ ರಾಜ್ಞೀ ತಾಪನೀ ಬಿಂದುಮಾಲಿನೀ ।\\
ಕ್ಷೀರಧಾರಾಸುಪ್ರಭಾವಾ ಲೋಕಮಾತಾ ಸುವರ್ಚಸಾ ॥೭೩॥

ಹವ್ಯಗರ್ಭಾ ಚಾಜ್ಯಗರ್ಭಾ ಜುಹ್ವತೋಯಜ್ಞಸಂಭವಾ ।\\
ಆಪ್ಯಾಯನೀ ಪಾವನೀ ಚ ದಹನೀ ದಹನಾಶ್ರಯಾ ॥೭೪॥

ಮಾತೃಕಾ ಮಾಧವೀ ಮುಖ್ಯಾ ಮೋಕ್ಷಲಕ್ಷ್ಮೀರ್ಮಹರ್ದ್ಧಿದಾ ।\\
ಸರ್ವಕಾಮಪ್ರದಾ ಭದ್ರಾ ಸುಭದ್ರಾ ಸರ್ವಮಂಗಲಾ ॥೭೫॥

ಶ್ವೇತಾ ಸುಶುಕ್ಲವಸನಾ ಶುಕ್ಲಮಾಲ್ಯಾನುಲೇಪನಾ ।\\
ಹಂಸಾ ಹೀನಕರೀ ಹಂಸೀ ಹೃದ್ಯಾ ಹೃತ್ಕಮಲಾಲಯಾ ॥೭೬॥

ಸಿತಾತಪತ್ರಾ ಸುಶ್ರೋಣೀ ಪದ್ಮಪತ್ರಾಯತೇಕ್ಷಣಾ ।\\
ಸಾವಿತ್ರೀ ಸತ್ಯಸಂಕಲ್ಪಾ ಕಾಮದಾ ಕಾಮಕಾಮಿನೀ ॥೭೭॥

ದರ್ಶನೀಯಾ ದೃಶಾ ದೃಶ್ಯಾ ಸ್ಪೃಶ್ಯಾ ಸೇವ್ಯಾ ವರಾಂಗನಾ ।\\
ಭೋಗಪ್ರಿಯಾ ಭೋಗವತೀ ಭೋಗೀಂದ್ರಶಯನಾಸನಾ ॥೭೮॥

ಆರ್ದ್ರಾ ಪುಷ್ಕರಿಣೀ ಪುಣ್ಯಾ ಪಾವನೀ ಪಾಪಸೂದನೀ ।\\
ಶ್ರೀಮತೀ ಚ ಶುಭಾಕಾರಾ ಪರಮೈಶ್ವರ್ಯಭೂತಿದಾ ॥೭೯॥

ಅಚಿಂತ್ಯಾನಂತವಿಭವಾ ಭವಭಾವವಿಭಾವನೀ ।\\
ನಿಶ್ರೇಣಿಃ ಸರ್ವದೇಹಸ್ಥಾ ಸರ್ವಭೂತನಮಸ್ಕೃತಾ ॥೮೦॥

ಬಲಾ ಬಲಾಧಿಕಾ ದೇವೀ ಗೌತಮೀ ಗೋಕುಲಾಲಯಾ ।\\
ತೋಷಿಣೀ ಪೂರ್ಣಚಂದ್ರಾಭಾ ಏಕಾನಂದಾ ಶತಾನನಾ ॥೮೧॥

ಉದ್ಯಾನನಗರದ್ವಾರಹರ್ಮ್ಯೋಪವನವಾಸಿನೀ ।\\
ಕೂಷ್ಮಾಂಡಾ ದಾರುಣಾ ಚಂಡಾ ಕಿರಾತೀ ನಂದನಾಲಯಾ ॥೮೨॥

ಕಾಲಾಯನಾ ಕಾಲಗಮ್ಯಾ ಭಯದಾ ಭಯನಾಶಿನೀ ।\\
ಸೌದಾಮನೀ ಮೇಘರವಾ ದೈತ್ಯದಾನವಮರ್ದಿನೀ ॥೮೩॥

ಜಗನ್ಮಾತಾ ಭಯಕರೀ ಭೂತಧಾತ್ರೀ ಸುದುರ್ಲಭಾ ।\\
ಕಾಶ್ಯಪೀ ಶುಭದಾತಾ ಚ ವನಮಾಲಾ ಶುಭಾವರಾ ॥೮೪॥

ಧನ್ಯಾ ಧನ್ಯೇಶ್ವರೀ ಧನ್ಯಾ ರತ್ನದಾ ವಸುವರ್ದ್ಧಿನೀ ।\\
ಗಾಂಧರ್ವೀ ರೇವತೀ ಗಂಗಾ ಶಕುನೀ ವಿಮಲಾನನಾ ॥೮೫॥

ಇಡಾ ಶಾಂತಿಕರೀ ಚೈವ ತಾಮಸೀ ಕಮಲಾಲಯಾ ।\\
ಆಜ್ಯಪಾ ವಜ್ರಕೌಮಾರೀ ಸೋಮಪಾ ಕುಸುಮಾಶ್ರಯಾ ॥೮೬॥

ಜಗತ್ಪ್ರಿಯಾ ಚ ಸರಥಾ ದುರ್ಜಯಾ ಖಗವಾಹನಾ ।\\
ಮನೋಭವಾ ಕಾಮಚಾರಾ ಸಿದ್ಧಚಾರಣಸೇವಿತಾ ॥೮೭॥

ವ್ಯೋಮಲಕ್ಷ್ಮೀರ್ಮಹಾಲಕ್ಷ್ಮೀಸ್ತೇಜೋಲಕ್ಷ್ಮೀಃ ಸುಜಾಜ್ವಲಾ ।\\
ರಸಲಕ್ಷ್ಮೀರ್ಜಗದ್ಯೋನಿರ್ಗಂಧಲಕ್ಷ್ಮೀರ್ವನಾಶ್ರಯಾ ॥೮೮॥

ಶ್ರವಣಾ ಶ್ರಾವಣೀ ನೇತ್ರೀ ರಸನಾಪ್ರಾಣಚಾರಿಣೀ ।\\
ವಿರಿಂಚಿಮಾತಾ ವಿಭವಾ ವರವಾರಿಜವಾಹನಾ ॥೮೯॥

ವೀರ್ಯಾ ವೀರೇಶ್ವರೀ ವಂದ್ಯಾ ವಿಶೋಕಾ ವಸುವರ್ದ್ಧಿನೀ ।\\
ಅನಾಹತಾ ಕುಂಡಲಿನೀ ನಲಿನೀ ವನವಾಸಿನೀ ॥೯೦॥

ಗಾಂಧಾರಿಣೀಂದ್ರನಮಿತಾ ಸುರೇಂದ್ರನಮಿತಾ ಸತೀ ।\\
ಸರ್ವಮಂಗಲ್ಯಮಾಂಗಲ್ಯಾ ಸರ್ವಕಾಮಸಮೃದ್ಧಿದಾ ॥೯೧॥

ಸರ್ವಾನಂದಾ ಮಹಾನಂದಾ ಸತ್ಕೀರ್ತಿಃ ಸಿದ್ಧಸೇವಿತಾ ।\\
ಸಿನೀವಾಲೀ ಕುಹೂ ರಾಕಾ ಅಮಾ ಚಾನುಮತಿರ್ದ್ಯುತಿಃ ॥೯೨॥

ಅರುಂಧತೀ ವಸುಮತೀ ಭಾರ್ಗವೀ ವಾಸ್ತುದೇವತಾ ।\\
ಮಾಯೂರೀ ವಜ್ರವೇತಾಲೀ ವಜ್ರಹಸ್ತಾ ವರಾನನಾ ॥೯೩॥

ಅನಘಾ ಧರಣಿರ್ಧೀರಾ ಧಮನೀ ಮಣಿಭೂಷಣಾ ।\\
ರಾಜಶ್ರೀ ರೂಪಸಹಿತಾ ಬ್ರಹ್ಮಶ್ರೀರ್ಬ್ರಹ್ಮವಂದಿತಾ ॥೯೪॥

ಜಯಶ್ರೀರ್ಜಯದಾ ಜ್ಞೇಯಾ ಸರ್ಗಶ್ರೀಃ ಸ್ವರ್ಗತಿಃ ಸತಾಂ ।\\
ಸುಪುಷ್ಪಾ ಪುಷ್ಪನಿಲಯಾ ಫಲಶ್ರೀರ್ನಿಷ್ಕಲಪ್ರಿಯಾ ॥೯೫॥

ಧನುರ್ಲಕ್ಷ್ಮೀಸ್ತ್ವಮಿಲಿತಾ ಪರಕ್ರೋಧನಿವಾರಿಣೀ ।\\
ಕದ್ರೂರ್ದ್ಧನಾಯುಃ ಕಪಿಲಾ ಸುರಸಾ ಸುರಮೋಹಿನೀ ॥೯೬॥

ಮಹಾಶ್ವೇತಾ ಮಹಾನೀಲಾ ಮಹಾಮೂರ್ತಿರ್ವಿಷಾಪಹಾ ।\\
ಸುಪ್ರಭಾ ಜ್ವಾಲಿನೀ ದೀಪ್ತಿಸ್ತೃಪ್ತಿರ್ವ್ಯಾಪ್ತಿಃ ಪ್ರಭಾಕರೀ ॥೯೭॥

ತೇಜೋವತೀ ಪದ್ಮಬೋಧಾ ಮದಲೇಖಾರುಣಾವತೀ ।\\
ರತ್ನಾ ರತ್ನಾವಲೀ ಭೂತಾ ಶತಧಾಮಾ ಶತಾಪಹಾ ॥೯೮॥

ತ್ರಿಗುಣಾ ಘೋಷಿಣೀ ರಕ್ಷ್ಯಾ ನರ್ದ್ದಿನೀ ಘೋಷವರ್ಜಿತಾ ।\\
ಸಾಧ್ಯಾ ದಿತಿರ್ದಿತಿದೇವೀ ಮೃಗವಾಹಾ ಮೃಗಾಂಕಗಾ ॥೯೯॥

ಚಿತ್ರನೀಲೋತ್ಪಲಗತಾ ವೃಷರತ್ನಕರಾಶ್ರಯಾ ।\\
ಹಿರಣ್ಯರಜತದ್ವಂದ್ವಾ ಶಂಖಭದ್ರಾಸನಾಸ್ಥಿತಾ ॥೧೦೦॥

ಗೋಮೂತ್ರಗೋಮಯಕ್ಷೀರದಧಿಸರ್ಪಿರ್ಜಲಾಶ್ರಯಾ ।\\
ಮರೀಚಿಶ್ಚೀರವಸನಾ ಪೂರ್ಣಾ ಚಂದ್ರಾರ್ಕವಿಷ್ಟರಾ ॥೧೦೧॥

ಸುಸೂಕ್ಷ್ಮಾ ನಿರ್ವೃತಿಃ ಸ್ಥೂಲಾ ನಿವೃತ್ತಾರಾತಿರೇವ ಚ ।\\
ಮರೀಚಿಜ್ವಾಲಿನೀ ಧೂಮ್ರಾ ಹವ್ಯವಾಹಾ ಹಿರಣ್ಯದಾ ॥೧೦೨॥

ದಾಯಿನೀ ಕಾಲಿನೀ ಸಿದ್ಧಿಃ ಶೋಷಿಣೀ ಸಂಪ್ರಬೋಧಿನೀ ।\\
ಭಾಸ್ವರಾ ಸಂಹತಿಸ್ತೀಕ್ಷ್ಣಾ ಪ್ರಚಂಡಜ್ವಲನೋಜ್ಜ್ವಲಾ ॥೧೦೩॥

ಸಾಂಗಾ ಪ್ರಚಂಡಾ ದೀಪ್ತಾ ಚ ವೈದ್ಯುತಿಃ ಸುಮಹಾದ್ಯುತಿಃ ।\\
ಕಪಿಲಾ ನೀಲರಕ್ತಾ ಚ ಸುಷುಮ್ಣಾ ವಿಸ್ಫುಲಿಂಗಿನೀ ॥೧೦೪॥

ಅರ್ಚಿಷ್ಮತೀ ರಿಪುಹರಾ ದೀರ್ಘಾ ಧೂಮಾವಲೀ ಜರಾ ।\\
ಸಂಪೂರ್ಣಮಂಡಲಾ ಪೂಷಾ ಸ್ರಂಸಿನೀ ಸುಮನೋಹರಾ ॥೧೦೫॥

ಜಯಾ ಪುಷ್ಟಿಕರೀಚ್ಛಾಯಾ ಮಾನಸಾ ಹೃದಯೋಜ್ಜ್ವಲಾ ।\\
ಸುವರ್ಣಕರಣೀ ಶ್ರೇಷ್ಠಾ ಮೃತಸಂಜೀವಿನೀರಣೇ ॥೧೦೬॥

ವಿಶಲ್ಯಕರಣೀ ಶುಭ್ರಾ ಸಂಧಿನೀ ಪರಮೌಷಧಿಃ ।\\
ಬ್ರಹ್ಮಿಷ್ಠಾ ಬ್ರಹ್ಮಸಹಿತಾ ಐಂದವೀ ರತ್ನಸಂಭವಾ ॥೧೦೭॥

ವಿದ್ಯುತ್ಪ್ರಭಾ ಬಿಂದುಮತೀ ತ್ರಿಸ್ವಭಾವಗುಣಾಂಬಿಕಾ ।\\
ನಿತ್ಯೋದಿತಾ ನಿತ್ಯಹೃಷ್ಟಾ ನಿತ್ಯಕಾಮಕರೀಷಿಣೀ ॥೧೦೮॥

ಪದ್ಮಾಂಕಾ ವಜ್ರಚಿಹ್ನಾ ಚ ವಕ್ರದಂಡವಿಭಾಸಿನೀ ।\\
ವಿದೇಹಪೂಜಿತಾ ಕನ್ಯಾ ಮಾಯಾ ವಿಜಯವಾಹಿನೀ ॥೧೦೯॥

ಮಾನಿನೀ ಮಂಗಲಾ ಮಾನ್ಯಾ ಮಾಲಿನೀ ಮಾನದಾಯಿನೀ ।\\
ವಿಶ್ವೇಶ್ವರೀ ಗಣವತೀ ಮಂಡಲಾ ಮಂಡಲೇಶ್ವರೀ ॥೧೧೦॥

ಹರಿಪ್ರಿಯಾ ಭೌಮಸುತಾ ಮನೋಜ್ಞಾ ಮತಿದಾಯಿನೀ ।\\
ಪ್ರತ್ಯಂಗಿರಾ ಸೋಮಗುಪ್ತಾ ಮನೋಽಭಿಜ್ಞಾ ವದನ್ಮತಿಃ ॥೧೧೧॥

ಯಶೋಧರಾ ರತ್ನಮಾಲಾ ಕೃಷ್ಣಾ ತ್ರೈಲೋಕ್ಯಬಂಧನೀ ।\\
ಅಮೃತಾ ಧಾರಿಣೀ ಹರ್ಷಾ ವಿನತಾ ವಲ್ಲಕೀ ಶಚೀ ॥೧೧೨॥

ಸಂಕಲ್ಪಾ ಭಾಮಿನೀ ಮಿಶ್ರಾ ಕಾದಂಬರ್ಯಮೃತಪ್ರಭಾ ।\\
ಅಗತಾ ನಿರ್ಗತಾ ವಜ್ರಾ ಸುಹಿತಾ ಸಂಹಿತಾಕ್ಷತಾ ॥೧೧೩॥

ಸರ್ವಾರ್ಥಸಾಧನಕರೀ ಧಾತುರ್ಧಾರಣಿಕಾಮಲಾ ।\\
ಕರುಣಾಧಾರಸಂಭೂತಾ ಕಮಲಾಕ್ಷೀ ಶಶಿಪ್ರಿಯಾ ॥೧೧೪॥

ಸೌಮ್ಯರೂಪಾ ಮಹಾದೀಪ್ತಾ ಮಹಾಜ್ವಾಲಾ ವಿಕಾಶಿನೀ ।\\
ಮಾಲಾ ಕಾಂಚನಮಾಲಾ ಚ ಸದ್ವಜ್ರಾ ಕನಕಪ್ರಭಾ ॥೧೧೫॥

ಪ್ರಕ್ರಿಯಾ ಪರಮಾ ಯೋಕ್ತ್ರೀ ಕ್ಷೋಭಿಕಾ ಚ ಸುಖೋದಯಾ ।\\
ವಿಜೃಂಭಣಾ ಚ ವಜ್ರಾಖ್ಯಾ ಶೃಂಖಲಾ ಕಮಲೇಕ್ಷಣಾ ॥೧೧೬॥

ಜಯಂಕರೀ ಮಧುಮತೀ ಹರಿತಾ ಶಶಿನೀ ಶಿವಾ ।\\
ಮೂಲಪ್ರಕೃತಿರೀಶಾನೀ ಯೋಗಮಾತಾ ಮನೋಜವಾ ॥೧೧೭॥

ಧರ್ಮೋದಯಾ ಭಾನುಮತೀ ಸರ್ವಾಭಾಸಾ ಸುಖಾವಹಾ ।\\
ಧುರಂಧರಾ ಚ ಬಾಲಾ ಚ ಧರ್ಮಸೇವ್ಯಾ ತಥಾಗತಾ ॥೧೧೮॥

ಸುಕುಮಾರಾ ಸೌಮ್ಯಮುಖೀ ಸೌಮ್ಯಸಂಬೋಧನೋತ್ತಮಾ ।\\
ಸುಮುಖೀ ಸರ್ವತೋಭದ್ರಾ ಗುಹ್ಯಶಕ್ತಿರ್ಗುಹಾಲಯಾ ॥೧೧೯॥

ಹಲಾಯುಧಾ ಚೈಕವೀರಾ ಸರ್ವಶಸ್ತ್ರಸುಧಾರಿಣೀ ।\\
ವ್ಯೋಮಶಕ್ತಿರ್ಮಹಾದೇಹಾ ವ್ಯೋಮಗಾ ಮಧುಮನ್ಮಯೀ ॥೧೨೦॥

ಗಂಗಾ ವಿತಸ್ತಾ ಯಮುನಾ ಚಂದ್ರಭಾಗಾ ಸರಸ್ವತೀ ।\\
ತಿಲೋತ್ತಮೋರ್ವಶೀ ರಂಭಾ ಸ್ವಾಮಿನೀ ಸುರಸುಂದರೀ ॥೧೨೧॥

ಬಾಣಪ್ರಹರಣಾವಾಲಾ ಬಿಂಬೋಷ್ಠೀ ಚಾರುಹಾಸಿನೀ ।\\
ಕಕುದ್ಮಿನೀ ಚಾರುಪೃಷ್ಠಾ ದೃಷ್ಟಾದೃಷ್ಟಫಲಪ್ರದಾ ॥೧೨೨॥

ಕಾಮ್ಯಾಚರೀ ಚ ಕಾಮ್ಯಾ ಚ ಕಾಮಾಚಾರವಿಹಾರಿಣೀ ।\\
ಹಿಮಶೈಲೇಂದ್ರಸಂಕಾಶಾ ಗಜೇಂದ್ರವರವಾಹನಾ ॥೧೨೩॥

ಅಶೇಷಸುಖಸೌಭಾಗ್ಯಸಂಪದಾ ಯೋನಿರುತ್ತಮಾ ।\\
ಸರ್ವೋತ್ಕೃಷ್ಟಾ ಸರ್ವಮಯೀ ಸರ್ವಾ ಸರ್ವೇಶ್ವರಪ್ರಿಯಾ ॥೧೨೪॥

ಸರ್ವಾಂಗಯೋನಿಃ ಸಾವ್ಯಕ್ತಾ ಸಂಪ್ರಧಾನೇಶ್ವರೇಶ್ವರೀ ।\\
ವಿಷ್ಣುವಕ್ಷಃಸ್ಥಲಗತಾ ಕಿಮತಃ ಪರಮುಚ್ಯತೇ ॥೧೨೫॥

ಪರಾ ನಿರ್ಮಹಿಮಾ ದೇವೀ ಹರಿವಕ್ಷಃಸ್ಥಲಾಶ್ರಯಾ ।\\
ಸಾ ದೇವೀ ಪಾಪಹಂತ್ರೀ ಚ ಸಾನ್ನಿಧ್ಯಂ ಕುರುತಾನ್ಮಮ ॥೧೨೬॥

ಇತಿ ನಾಮ್ನಾಂ ಸಹಸ್ರಂ ತು ಲಕ್ಷ್ಮ್ಯಾಃ ಪ್ರೋಕ್ತಂ ಶುಭಾವಹಂ ।\\
ಪರಾವರೇಣ ಭೇದೇನ ಮುಖ್ಯಗೌಣೇನ ಭಾಗತಃ ॥೧೨೭॥

ಯಶ್ಚೈತತ್ ಕೀರ್ತಯೇನ್ನಿತ್ಯಂ ಶೃಣುಯಾದ್ ವಾಪಿ ಪದ್ಮಜ ।\\
ಶುಚಿಃ ಸಮಾಹಿತೋ ಭೂತ್ವಾ ಭಕ್ತಿಶ್ರದ್ಧಾಸಮನ್ವಿತಃ ॥೧೨೮॥

ಶ್ರೀನಿವಾಸಂ ಸಮಭ್ಯರ್ಚ್ಯ ಪುಷ್ಪಧೂಪಾನುಲೇಪನೈಃ ।\\
ಭೋಗೈಶ್ಚ ಮಧುಪರ್ಕಾದ್ಯೈರ್ಯಥಾಶಕ್ತಿ ಜಗದ್ಗುರುಂ ॥೧೨೯॥

ತತ್ಪಾರ್ಶ್ವಸ್ಥಾಂ ಶ್ರಿಯಂ ದೇವೀಂ ಸಂಪೂಜ್ಯ ಶ್ರೀಧರಪ್ರಿಯಾಂ ।\\
ತತೋ ನಾಮಸಹಸ್ರೋಣ ತೋಷಯೇತ್ ಪರಮೇಶ್ವರೀಂ ॥೧೩೦॥

ನಾಮರತ್ನಾವಲೀಸ್ತೋತ್ರಮಿದಂ ಯಃ ಸತತಂ ಪಠೇತ್ ।\\
ಪ್ರಸಾದಾಭಿಮುಖೀಲಕ್ಷ್ಮೀಃ ಸರ್ವಂ ತಸ್ಮೈ ಪ್ರಯಚ್ಛತಿ ॥೧೩೧॥

ಯಸ್ಯಾ ಲಕ್ಷ್ಮ್ಯಾಶ್ಚ ಸಂಭೂತಾಃ ಶಕ್ತಯೋ ವಿಶ್ವಗಾಃ ಸದಾ ।\\
ಕಾರಣತ್ವೇ ನ ತಿಷ್ಠಂತಿ ಜಗತ್ಯಸ್ಮಿಂಶ್ಚರಾಚರೇ ॥೧೩೨॥

ತಸ್ಮಾತ್ ಪ್ರೀತಾ ಜಗನ್ಮಾತಾ ಶ್ರೀರ್ಯಸ್ಯಾಚ್ಯುತವಲ್ಲಭಾ ।\\
ಸುಪ್ರೀತಾಃ ಶಕ್ತಯಸ್ತಸ್ಯ ಸಿದ್ಧಿಮಿಷ್ಟಾಂ ದಿಶಂತಿ ಹಿ ॥೧೩೩॥

ಏಕ ಏವ ಜಗತ್ಸ್ವಾಮೀ ಶಕ್ತಿಮಾನಚ್ಯುತಃ ಪ್ರಭುಃ ।\\
ತದಂಶಶಕ್ತಿಮಂತೋಽನ್ಯೇ ಬ್ರಹ್ಮೇಶಾನಾದಯೋ ಯಥಾ ॥೧೩೪॥

ತಥೈವೈಕಾ ಪರಾ ಶಕ್ತಿಃ ಶ್ರೀಸ್ತಸ್ಯ ಕರುಣಾಶ್ರಯಾ ।\\
ಜ್ಞಾನಾದಿಷಾಂಗುಣ್ಯಮಯೀ ಯಾ ಪ್ರೋಕ್ತಾ ಪ್ರಕೃತಿಃ ಪರಾ ॥೧೩೫॥

ಏಕೈವ ಶಕ್ತಿಃ ಶ್ರೀಸ್ತಸ್ಯಾ ದ್ವಿತೀಯಾತ್ಮನಿ ವರ್ತತೇ ।\\
ಪರಾ ಪರೇಶೀ ಸರ್ವೇಶೀ ಸರ್ವಾಕಾರಾ ಸನಾತನೀ ॥೧೩೬॥

ಅನಂತನಾಮಧೇಯಾ ಚ ಶಕ್ತಿಚಕ್ರಸ್ಯ ನಾಯಿಕಾ ।\\
ಜಗಚ್ಚರಾಚರಮಿದಂ ಸರ್ವಂ ವ್ಯಾಪ್ಯ ವ್ಯವಸ್ಥಿತಾ ॥೧೩೭॥

ತಸ್ಮಾದೇಕೈವ ಪರಮಾ ಶ್ರೀರ್ಜ್ಞೇಯಾ ವಿಶ್ವರೂಪಿಣೀ ।\\
ಸೌಮ್ಯಾ ಸೌಮ್ಯೇನ ರೂಪೇಣ ಸಂಸ್ಥಿತಾ ನಟಜೀವವತ್ ॥೧೩೮॥

ಯೋ ಯೋ ಜಗತಿ ಪುಂಭಾವಃ ಸ ವಿಷ್ಣುರಿತಿ ನಿಶ್ಚಯಃ ।\\
ಯಾ ಯಾ ತು ನಾರೀಭಾವಸ್ಥಾ ತತ್ರ ಲಕ್ಷ್ಮೀರ್ವ್ಯವಸ್ಥಿತಾ ॥೧೩೯॥

ಪ್ರಕೃತೇಃ ಪುರುಷಾಚ್ಚಾನ್ಯಸ್ತೃತೀಯೋ ನೈವ ವಿದ್ಯತೇ ।\\
ಅಥ ಕಿಂ ಬಹುನೋಕ್ತೇನ ನರನಾರೀಮಯೋ ಹರಿಃ ॥೧೪೦॥

ಅನೇಕಭೇದಭಿನ್ನಸ್ತು ಕ್ರಿಯತೇ ಪರಮೇಶ್ವರಃ ।\\
ಮಹಾವಿಭೂತಿಂ ದಯಿತಾಂ ಯೇ ಸ್ತುವಂತ್ಯಚ್ಯುತಪ್ರಿಯಾಂ ॥೧೪೧॥

ತೇ ಪ್ರಾಪ್ನುವಂತಿ ಪರಮಾಂ ಲಕ್ಷ್ಮೀಂ ಸಂಶುದ್ಧಚೇತಸಃ ।\\
ಪದ್ಮಯೋನಿರಿದಂ ಪ್ರಾಪ್ಯ ಪಠನ್ ಸ್ತೋತ್ರಮಿದಂ ಕ್ರಮಾತ್ ॥೧೪೨॥

ದಿವ್ಯಮಷ್ಟಗುಣೈಶ್ವರ್ಯಂ ತತ್ಪ್ರಸಾದಾಚ್ಚ ಲಬ್ಧವಾನ್ ।\\
ಸಕಾಮಾನಾಂ ಚ ಫಲದಾಮಕಾಮಾನಾಂ ಚ ಮೋಕ್ಷದಾಂ ॥೧೪೩॥

ಪುಸ್ತಕಾಖ್ಯಾಂ ಭಯತ್ರಾತ್ರೀಂ ಸಿತವಸ್ತ್ರಾಂ ತ್ರಿಲೋಚನಾಂ ।\\
ಮಹಾಪದ್ಮನಿಷಣ್ಣಾಂ ತಾಂ ಲಕ್ಷ್ಮೀಮಜರತಾಂ ನಮಃ ॥೧೪೪॥

ಕರಯುಗಲಗೃಹೀತಂ ಪೂರ್ಣಕುಂಭಂ ದಧಾನಾ\\
ಕ್ವಚಿದಮಲಗತಸ್ಥಾ ಶಂಖಪದ್ಮಾಕ್ಷಪಾಣಿಃ ।\\
ಕ್ವಚಿದಪಿ ದಯಿತಾಂಗೇ ಚಾಮರವ್ಯಗ್ರಹಸ್ತಾ\\
ಕ್ವಚಿದಪಿ ಸೃಣಿಪಾಶಂ ಬಿಭ್ರತೀ ಹೇಮಕಾಂತಿಃ ॥೧೪೫॥

\authorline{ ಇತ್ಯಾದಿಪದ್ಮಪುರಾಣೇ ಕಾಶ್ಮೀರವರ್ಣನೇ ಹಿರಣ್ಯಗರ್ಭಹೃದಯೇ\\ ಸರ್ವಕಾಮಪ್ರದಾಯಕಂ ಪುರುಷೋತ್ತಮಪ್ರೋಕ್ತಂ\\ ಶ್ರೀಲಕ್ಷ್ಮೀಸಹಸ್ರನಾಮಸ್ತೋತ್ರಂ ಸಮಾಪ್ತಂ॥}
%=============================================================================================
\section{ಶ್ರೀಕಮಲಾಷ್ಟೋತ್ತರಶತನಾಮಸ್ತೋತ್ರಂ}
\addcontentsline{toc}{section}{ಶ್ರೀಕಮಲಾಷ್ಟೋತ್ತರಶತನಾಮಸ್ತೋತ್ರಂ}


ಶ್ರೀಶಿವ ಉವಾಚ ।\\
ಶತಮಷ್ಟೋತ್ತರಂ ನಾಮ್ನಾಂ ಕಮಲಾಯಾ ವರಾನನೇ ।\\
ಪ್ರವಕ್ಷ್ಯಾಮ್ಯತಿಗುಹ್ಯಂ ಹಿ ನ ಕದಾಪಿ ಪ್ರಕಾಶಯೇತ್ ॥೧॥

ಮಹಾಮಾಯಾ ಮಹಾಲಕ್ಷ್ಮೀರ್ಮಹಾವಾಣೀ ಮಹೇಶ್ವರೀ ।\\
ಮಹಾದೇವೀ ಮಹಾರಾತ್ರಿರ್ಮಹಿಷಾಸುರಮರ್ದಿ ನೀ ॥೨॥

ಕಾಲರಾತ್ರಿಃ ಕುಹೂಃ ಪೂರ್ಣಾ ನಂದಾಽಽದ್ಯಾ ಭದ್ರಿಕಾ ನಿಶಾ ।\\
ಜಯಾ ರಿಕ್ತಾ ಮಹಾಶಕ್ತಿರ್ದೇವಮಾತಾ ಕೃಶೋದರೀ ॥೩॥

ಶಚೀಂದ್ರಾಣೀ ಶಕ್ರನುತಾ ಶಂಕರಪ್ರಿಯವಲ್ಲಭಾ ।\\
ಮಹಾವರಾಹಜನನೀ ಮದನೋನ್ಮಥಿನೀ ಮಹೀ ॥೪॥

ವೈಕುಂಠನಾಥರಮಣೀ ವಿಷ್ಣುವಕ್ಷಃಸ್ಥಲಸ್ಥಿತಾ ।\\
ವಿಶ್ವೇಶ್ವರೀ ವಿಶ್ವಮಾತಾ ವರದಾಽಭಯದಾ ಶಿವಾ ॥೫॥

ಶೂಲಿನೀ ಚಕ್ರಿಣೀ ಮಾ ಚ ಪಾಶಿನೀ ಶಂಖಧಾರಿಣೀ ।\\
ಗದಿನೀ ಮುಂಡಮಾಲಾ ಚ ಕಮಲಾ ಕರುಣಾಲಯಾ ॥೬॥

ಪದ್ಮಾಕ್ಷಧಾರಿಣೀ ಹ್ಯಂಬಾ ಮಹಾವಿಷ್ಣುಪ್ರಿಯಂಕರೀ ।\\
ಗೋಲೋಕನಾಥರಮಣೀ ಗೋಲೋಕೇಶ್ವರಪೂಜಿತಾ ॥೭॥

ಗಯಾ ಗಂಗಾ ಚ ಯಮುನಾ ಗೋಮತೀ ಗರುಡಾಸನಾ ।\\
ಗಂಡಕೀ ಸರಯೂಸ್ತಾಪೀ ರೇವಾ ಚೈವ ಪಯಸ್ವಿನೀ ॥೮॥

ನರ್ಮದಾ ಚೈವ ಕಾವೇರೀ ಕೇದಾರಸ್ಥಲವಾಸಿನೀ ।\\
ಕಿಶೋರೀ ಕೇಶವನುತಾ ಮಹೇಂದ್ರಪರಿವಂದಿತಾ ॥೯॥

ಬ್ರಹ್ಮಾದಿದೇವನಿರ್ಮಾಣಕಾರಿಣೀ ವೇದಪೂಜಿತಾ ।\\
ಕೋಟಿಬ್ರಹ್ಮಾಂಡಮಧ್ಯಸ್ಥಾ ಕೋಟಿಬ್ರಹ್ಮಾಂಡಕಾರಿಣೀ ॥೧೦॥

ಶ್ರುತಿರೂಪಾ ಶ್ರುತಿಕರೀ ಶ್ರುತಿಸ್ಮೃತಿಪರಾಯಣಾ ।\\
ಇಂದಿರಾ ಸಿಂಧುತನಯಾ ಮಾತಂಗೀ ಲೋಕಮಾತೃಕಾ ॥೧೧॥

ತ್ರಿಲೋಕಜನನೀ ತಂತ್ರಾ ತಂತ್ರಮಂತ್ರಸ್ವರೂಪಿಣೀ ।\\
ತರುಣೀ ಚ ತಮೋಹಂತ್ರೀ ಮಂಗಲಾ ಮಂಗಲಾಯನಾ ॥೧೨॥

ಮಧುಕೈಟಭಮಥನೀ ಶುಂಭಾಸುರವಿನಾಶಿನೀ ।\\
ನಿಶುಂಭಾದಿ ಹರಾ ಮಾತಾ ಹರಿಶಂಕರಪೂಜಿತಾ ॥೧೩॥

ಸರ್ವದೇವಮಯೀ ಸರ್ವಾ ಶರಣಾಗತಪಾಲಿನೀ ।\\
ಶರಣ್ಯಾ ಶಂಭುವನಿತಾ ಸಿಂಧುತೀರನಿವಾಸಿನೀ ॥೧೪॥

ಗಂಧರ್ವಗಾನರಸಿಕಾ ಗೀತಾ ಗೋವಿಂದವಲ್ಲಭಾ ।\\
ತ್ರೈಲೋಕ್ಯಪಾಲಿನೀ ತತ್ತ್ವರೂಪಾ ತಾರುಣ್ಯಪೂರಿತಾ ॥೧೫॥

ಚಂದ್ರಾವಲೀ ಚಂದ್ರಮುಖೀ ಚಂದ್ರಿಕಾ ಚಂದ್ರಪೂಜಿತಾ ।\\
ಚಂದ್ರಾ ಶಶಾಂಕಭಗಿನೀ ಗೀತವಾದ್ಯಪರಾಯಣಾ ॥೧೬॥

ಸೃಷ್ಟಿರೂಪಾ ಸೃಷ್ಟಿಕರೀ ಸೃಷ್ಟಿಸಂಹಾರಕಾರಿಣೀ ।\\
ಇತಿ ತೇ ಕಥಿತಂ ದೇವಿ ರಮಾನಾಮಶತಾಷ್ಟಕಂ ॥೧೭॥

ತ್ರಿಸಂಧ್ಯಂ ಪ್ರಯತೋ ಭೂತ್ವಾ ಪಠೇದೇತತ್ಸಮಾಹಿತಃ ।\\
ಯಂ ಯಂ ಕಾಮಯತೇ ಕಾಮಂ ತಂ ತಂ ಪ್ರಾಪ್ನೋತ್ಯಸಂಶಯಂ ॥೧೮॥

ಇಮಂ ಸ್ತವಂ ಯಃ ಪಠತೀಹ ಮರ್ತ್ಯೋ\\ ವೈಕುಂಠಪತ್ನ್ಯಾಃ ಪರಸಾದರೇಣ ।\\
ಧನಾಧಿಪಾದ್ಯೈಃ ಪರಿವಂದಿತಃ ಸ್ಯಾತ್\\ ಪ್ರಯಾಸ್ಯತಿ ಶ್ರೀಪದಮಂತಕಾಲೇ ॥೧೯॥

\authorline{ಇತಿ ಶ್ರೀಕಮಲಾಷ್ಟೋತ್ತರಶತನಾಮಸ್ತೋತ್ರಂ ಸಂಪೂರ್ಣಂ ॥}
%======================================================
\section{ಶ್ರೀಶಾಕಂಭರೀ ಸಹಸ್ರನಾಮಸ್ತೋತ್ರಂ}
\addcontentsline{toc}{section}{ಶ್ರೀಶಾಕಂಭರೀ ಸಹಸ್ರನಾಮಸ್ತೋತ್ರಂ}


ಶಾಂತಾ ಶಾರದಚಂದ್ರಸುಂದರಮುಖೀ ಶಾಲ್ಯನ್ನಭೋಜ್ಯಪ್ರಿಯಾ\\
ಶಾಕೈಃ ಪಾಲಿತವಿಷ್ಟಪಾ ಶತದೃಶಾ ಶಾಕೋಲ್ಲಸದ್ವಿಗ್ರಹಾ ।\\
ಶ್ಯಾಮಾಂಗೀ ಶರಣಾಗತಾರ್ತಿಶಮನೀ ಶಕ್ರಾದಿಭಿಃ ಶಂಸಿತಾ\\
ಶಂಕರ್ಯಷ್ಟಫಲಪ್ರದಾ ಭಗವತೀ ಶಾಕಂಭರೀ ಪಾತು ಮಾಂ॥

ಶೂಲಂ ಪಾಶಕಪಾಲಚಾಪಕುಲಿಶಾನ್ಬಾಣಾನ್ಸೃಣಿಂ ಖೇಟಕಾಂ\\
ಶಂಕಂ ಚಕ್ರಗದಾಹಿಖಡ್ಗಮಭಯಂ ಖಟ್ವಾಂಗದಂಡಾಂಧರಾಂ ।\\
ವರ್ಷಾಭಾವವಶಾದ್ಧತಾನ್ಮುನಿಗಣಾನ್ಶಾಕೇನ ಯಾ ರಕ್ಷತಿ\\
ಲೋಕಾನಾಂ ಜನನೀಂ ಮಹೇಶದಯಿತಾಂ ತಾಂ ನೌಮಿ ಶಾಕಂಭರೀಂ ॥೧॥

ಕೈಲಾಸಶಿಖರಾಸೀನಂ ಸ್ಕಂದಂ ಮುನಿ ಗಣಾನ್ವಿತಂ ।\\
ಪ್ರಣಮ್ಯ ಶಕ್ರಃ ಪಪ್ರಚ್ಛ ಲೋಕಾನಾಂ ಹಿತಕಾಮ್ಯಯಾ ॥೨॥

ಶಕ್ರ ಉವಾಚ ।\\
ಸ್ಕಂದ ಸ್ಕಂದ ಮಹಾಬಾಹೋ ಸರ್ವಜ್ಞ ಶಿವನಂದನ ।\\
ನಾಮ್ನಾಂ ಸಹಸ್ರಮಾಚಕ್ಷ್ವ ಶಾಕಂಭರ್ಯಾಃ ಸುಸಿದ್ಧಿದಂ ॥೩॥

ಸ್ಕಂದ ಉವಾಚ ।\\
ಯಾ ದೇವೀ ಶತವಾರ್ಷಿಕ್ಯಾಮನಾವೃಷ್ಟ್ಯಾಂ ಸ್ವದೇಹಜೈಃ ।\\
ಶಾಕೈರಬೀಭರತ್ಸರ್ವಾನೃಷೀನ್ ಶಕ್ರ ಶತಂ ಸಮಾಃ ॥೪॥

ಮಹಾಸರಸ್ವತೀ ಸೈವ ಜಾತಾ ಶಾಕಂಭರೀ ಶಿವಾ ।\\
ನಾಮ್ನಾಂ ಸಹಸ್ರಂ ತಸ್ಯಾಸ್ತೇ ವಕ್ಷ್ಯಾಮಿ ಶ್ರುಣುಭಕ್ತಿತಃ ॥೫॥

ಓಂ ಅಸ್ಯ ಶ್ರೀಶಾಕಂಭರೀಸಹಸ್ರನಾಮಮಾಲಾಮಂತ್ರಸ್ಯ ಮಹಾದೇವಃ ಋಷಿಃ । ಅನುಷ್ಟುಪ್ ಛಂದಃ । ಶಾಕಂಭರೀ ದೇವತಾ । ಸೌಃ ಬೀಜಂ । ಕ್ಲೀಂ ಶಕ್ತಿಃ । ಹ್ರೀಂ ಕೀಲಕಂ । ಮಮ ಶ್ರೀ ಶಾಕಂಭರೀಪ್ರಸಾದಸಿದ್ಧ್ಯರ್ಥೇ ತತ್ಸಹಸ್ರನಾಮಪಾರಾಯಣೇ ವಿನಿಯೋಗಃ ।\\

ಓಂ ಸೌಃ ಚಾಮುಂಡಾಯೈ ಅಂಗುಷ್ಠಾಭ್ಯಾಂ ನಮಃ ।\\
ಓಂ ಕ್ಲೀಂ ಶತಾಕ್ಷ್ಯೈ ತರ್ಜನೀಭ್ಯಾಂ ನಮಃ ।\\
ಓಂ ಹ್ರೀಂ ಶಾಕಂಭರ್ಯೈ ಮಧ್ಯಮಾಭ್ಯಾಂ ನಮಃ ।\\
ಓಂ ಸೌಃ ಚಾಮುಂಡಾಯೈ ಅನಾಮಿಕಾಭ್ಯಾಂ ನಮಃ ।\\
ಓಂ ಕ್ಲೀಂ ಶತಾಕ್ಷ್ಯೇ ಕನಿಷ್ಠಿಕಾಭ್ಯಾಂ ನಮಃ ।\\
ಓಂ ಹ್ರೀಂ ಶಾಕಂಭರ್ಯೈ ಕರತಲಕರಪೃಷ್ಠಾಭ್ಯಾಂ ನಮಃ ।\\
ಏವಂ ಹೃದಯಾದಿನ್ಯಾಸಃ ।\\


\as{ಸೌವರ್ಣಸಿಂಹಾಸನಸಂಸ್ಥಿತಾಂ ಶಿವಾಂ ತ್ರಿಲೋಚನಾ ಚಂದ್ರಕಲಾವತಂಸಿಕಾಂ ।\\
ಶೂಲಂ ಕಪಾಲಂ ನಿಜವಾಮಹಸ್ತಯೋಸ್ತದನ್ಯಯೋಃ ಖಡ್ಗಮಭೀತಿಮುದ್ರಿಕಾಂ॥

ಪಾಣ್ಯೋರ್ದಧಾನಾಂ ಮಣಿಭೂಷಣಾಜ್ಜ್ವಲಾಂ ಸುವಾಸಸಂ ಮಾಲ್ಯವಿಲೇಪನಾಂಚಿತಾಂ ।\\
ಪ್ರಸನ್ನಾವಕ್ತ್ರಾಂ ಪರದೇವತಾ ಮುದಾ ಧ್ಯಾಯಾಮಿ ಭಕ್ತ್ಯಾ ವನಶಂಕರೀಂ ಹೃದಿ॥}

ಓಂ ಸೌಃ ಕ್ಲೀಂ ಹ್ರೀಂ ಓಂ ।\\
ಓಂ ಶಾಕಂಭರ್ಯೈ ನಮಃ ।\\
ಹ್ರೀಂ ಶಾಕಂಭರ್ಯೈ ನಮಃ ।\\
ಓಂ ಹ್ರೀಂ ಶಾಕಂಭರ್ಯೈ ನಮಃ ।\\
ಸೌಃ ಕ್ಲೀಂ ಹ್ರೀಂ ಶಾಕಂಭರ್ಯೈ ನಮಃ ।\\
ಓಂ ಸೌಃ ಕ್ಲೀಂ ಹ್ರೀಂ ಶಾಕಂಭರ್ಯೈ ನಮಃ ।\\
ಓಂ ಸೌಃ ಕ್ಲೀಂ ಹ್ರೀಂ ಶಾಕಂಭರ್ಯೈ ನಮಃ ಓಂ ।\\
ಇತ್ಯೇಷು ಗುರೂಪದಿಷ್ಟಂ ಮಂತ್ರಂ ಜಪೇತ್ ।\\

॥ಅಥ ಸಹಸ್ರನಾಮಸ್ತೋತ್ರಂ॥

ಓಂ ಶಾಕಂಭರೀ ಶತಾಕ್ಷೀ ಚ ಚಾಮುಂಡಾ ರಕ್ತದಂತಿಕಾ ।\\
ಮಹಾಕಾಲೀ ಮಹಾಶಕ್ತಿರ್ಮಧುಕೈಟಭನಾಶಿನೀ ॥೧॥

ಬ್ರಹ್ಮಾದಿತೇಜಃ ಸಂಭೂತಾ ಮಹಾಲಕ್ಷ್ಮೀರ್ವರಾನನಾ ।\\
ಅಷ್ಟಾದಶಭುಜಾ ಸಮ್ರಾಣ್ಮಹಿಷಾಸುರಮರ್ದಿನೀ ॥೨॥

ಮಹಾಮಾಯಾ ಮಹಾದೇವೀ ಸೃಷ್ಟಿಸ್ಥಿತ್ಯಂತಕಾರಿಣೀ ।\\
ಕಾಂತಿಃ ಕಾಮಪ್ರದಾ ಕಾಮ್ಯಾ ಕಲ್ಯಾಣೀ ಕರುಣಾನಿಧಿಃ॥

ಸಿಂಹಸ್ಥಿತಾ ನಾರಸಿಂಹೀ ವೈಷ್ಣವೀ ವಿಷ್ಣುವಲ್ಲಭಾ ।\\
ಭ್ರಾಮರೀ ರಕ್ತಚಾಮುಂಡಾ ರಕ್ತಾಕ್ಷೀ ರಕ್ತಪಾಯಿನೀ ॥೪॥

ರಕ್ತಪ್ರಿಯಾ ಸುರಕ್ತೋಷ್ಠೀ ರಕ್ತಬೀಜವಿನಾಶಿನೀ ।\\
ಸುರಕ್ತವಸನಾ ರಕ್ತಮಾಲ್ಯಾ ರಕ್ತವಿಭೂಷಣಾ ॥೫॥

ರಕ್ತಪಾಣಿತಲಾ ರಕ್ತನಖೀ ರಕ್ತೋತ್ಪಲಾಂಘ್ರಿಕಾ ।\\
ರಕ್ತಚಂದನಲಿಪ್ತಾಂಗೀ ರಮಣೀ ರತಿದಾಯಿನೀ ॥೬॥

ಸುರಭಿಃ ಸುಂದರೀ ಬಾಲಾ ಬಗಲಾ ಭೈರವೀ ಸಮಿತ್ ।\\
ಚಂದ್ರಲಾಂಬಾ ಸುಮಂಗಲ್ಯಾ ಭೀಮಾ ಭಯನಿವಾರಿಣೀ ॥೭॥

ಜಾಗೃತಿಃ ಸ್ವಪ್ನರೂಪಾ ಚ ಸುಷುಪ್ತಿಸುಖರೂಪಿಣೀ ।\\
ತುರ್ಯೋಂನ್ಮನೀ ತ್ರಿಮಾತ್ರಾ ಚ ತ್ರಯೀ ತ್ರೇತಾ ತ್ರಿಮೂರ್ತಿಕಾ ॥೮॥

ವಿಷ್ಣುಮಾಯಾ ವಿಷ್ಣುಶಕ್ತಿರ್ವಿಷ್ಣುಜಿಹ್ವಾ ವಿನೋದಿನೀ ।\\
ಛಾಯಾ ಶಾಂತಿಃ ಕ್ಷಮಾ ಕ್ಷುತ್ತೃಟ್ತುಷ್ಟಿಃ ಪುಷ್ಟಿರ್ಧೃತಿರ್ಭೃತಿಃ ॥೯॥

ಮತಿರ್ಮಿತಿರ್ನತಿರ್ನೀತಿಃ ಸಂಯತಿರ್ನಿಯತಿ ಕೃತಿಃ ।\\
ಸ್ಫೂರ್ತಿಃ ಕೀರ್ತಿಃ ಸ್ತುತಿರ್ಜೂತಿಃ ಪೂರ್ತಿರ್ಮೂರ್ತಿರ್ನಿಜಪ್ರದಾ ॥೧೦॥

ತ್ರಿಶೂಲಧಾರಿಣೀ ತೀಕ್ಷ್ಣಗದಿನೀ ಖಡ್ಗಧಾರಿಣೀ ।\\
ಪಾಶಿನೀ ತ್ರಾಸಿನೀ ವಾಮಾ ವಾಮದೇವೀ ವರಾನನಾ ॥೧೧॥

ವಾಮಾಕ್ಷೀ ವಾರುಣೀಮತ್ತಾ ವಾಮೋರುರ್ವಾಸವಸ್ತುತಾ ।\\
ಬ್ರಹ್ಮವಿದ್ಯಾ ಮಹಾವಿದ್ಯಾ ಯೋಗಿನೀ ಯೋಗಪೂಜಿತಾ ॥೧೨॥

ತ್ರಿಕೂಟನಿಲಯಾ ನಿತ್ಯಾ ಕಲ್ಪಾತೀತಾ ಚ ಕಲ್ಪನಾ ।\\
ಕಾಮೇಶ್ವರೀ ಕಾಮದಾತ್ರೀ ಕಾಮಾಂತಕಕುಟುಂಬಿನೀ ॥೧೩॥

ಕಾಲರಾತ್ರಿರ್ಮಹಾರಾತ್ರಿರ್ಮೋಹರಾತ್ರಿಶ್ಚ ದಾರುಣಾ ।\\
ನಾನಾದ್ರುಮಲತಾಕೀರ್ಣಗಿರಿಮಧ್ಯನಿವಾಸಿನೀ ॥೧೪॥

ಶಾಕಪೋಷಿತಸರ್ವರ್ಷಿಃ ಪಾಕಶಾಸನಪೂಜಿತಾ ।\\
ಕ್ಲೇದಿನೀ ಭೇದಿನೀ ಭ್ರಾಂತಿರ್ಭೀತಿದಾ ಭ್ರಾಂತಿನಾಶಿನೀ ॥೧೫॥

ಕ್ರಾಂತಿಃ ಸಂಕ್ರಾಂತಿರುತ್ಕ್ರಾಂತಿರ್ವಿಕ್ರಾಂತಿಃ ಕ್ರಾಂತಿವರ್ಜಿತಾ ।\\
ಡಿಂಡಿಮಧ್ವನಿಸಂಹೃಷ್ಟಾ ಭೇರೀನಾದವಿನೋದಿನೀ ॥೧೬॥

ಸುತಂತ್ರೀವಾದನರತಾ ಸ್ವರಭೇದವಿಚಕ್ಷಣಾ ।\\
ಗಾಂಧರ್ವಶಾಸ್ತ್ರನಿಪೂಣಾ ನಾಟ್ಯಶಾಸ್ತ್ರವಿಶಾರದಾ ॥೧೭॥

ಹಾವಭಾವಪ್ರಮಾಣಜ್ಞಾ ಚತುಃಷಷ್ಟಿಕಲಾನ್ವಿತಾ ।\\
ಪದವಾಕ್ಯಪ್ರಮಾಣಾಬ್ಧಿಪಾರೀಣಾ ವಾದಿಭಂಗಿನೀ ॥೧೮॥

ಸರ್ವತಂತ್ರಸ್ವತಂತ್ರಾ ಚ ಮಂತ್ರಶಾಸ್ತ್ರಾಬ್ಧಿಪಾರಗಾ ।\\
ನಾನಾಯಂತ್ರ ವಿಧಾನಜ್ಞಾ ಸೌಗತಾಗಮಕೋವಿದಾ ॥೧೯॥

ವೈಖಾನಸಾದ್ಯಾಗಮಜ್ಞಾ ಶೈವಾಗಮ ವಿಚಕ್ಷಣಾ ।\\
ವಾಮದಕ್ಷಿಣಮಾರ್ಗಜ್ಞಾ ಭೈರವಾಗಮಭೇದವಿತ್ ॥೨೦॥

ಪಂಚಾಯತನತತ್ತ್ವಜ್ಞಾ ಪಂಚಾಮ್ನಾಯಪ್ರಪಂಚವಿತ್ ।\\
ರಾಜರಾಜೇಶ್ವರೀ ಭಟ್ಟಾರಿಕಾ ತ್ರಿಪುರಸುಂದರೀ ॥೨೧॥

ಮಹಾದಕ್ಷಿಣಕಾಲೀ ಚ ಮಾತಂಗೀ ಮದ್ಯಮಾಂಸಭುಕ್ ।\\
ಲುಲಾಯಬಲಿಸಂತುಷ್ಟಾ ಮೇಷಚ್ಛಾಗಬಲಿಪ್ರಿಯಾ ॥೨೨॥

ದೀಪಿಕಾಕ್ರೀಡನರತಾ ಡಮಡ್ಡಮರುನಾದಿನೀ ।\\
ಮಹಾಪ್ರದೀಪಿಕಾದೇಹಲೇಹನಾದ್ಭುತಶಕ್ತಿದಾ ॥೨೩॥

ನಿಜಾವೇಶವದುನ್ಮತ್ತಭಕ್ತೋಕ್ತ ಫಲದಾಯಿನೀ ।\\
ಉದೋಉದೋಮಹಾಧ್ವಾನ ಶ್ರವಣಾಸಕ್ತಮಾನಸಾ ॥೨೪॥

ಚೌಡೇಶ್ವರೀ ಚೌಡವಾದ್ಯಪ್ರವಾದನಪರಾಯಣಾ ।\\
ಕಪರ್ದಮಾಲಾಭರಣಾ ಕಪರ್ದಿಪ್ರಾಣವಲ್ಲಭಾ ॥೨೫॥

ನಗ್ರಸ್ತ್ರೀವೀಕ್ಷಣರತಾ ನಗ್ನಿಕೋದ್ಭಿನ್ನಯೌವನಾ ।\\
ಬಾಲಾ ಬದರವಕ್ಷೋಜಾ ಮಧ್ಯಾ ಬಿಲ್ವಫಲಸ್ತನೀ ॥೨೬॥

ನಾರಿಕೇಲಸ್ತನೀ ಪ್ರೌಢಾ ಪ್ರಗಲ್ಭಾ ಚ ಘಟಸ್ತನೀ ।\\
ಕಾತ್ಯಾಯನೀ ನಮ್ರಕುಚಾ ಪ್ರೌಢಾದ್ಭುತಪರಾಕ್ರಮಾ ।\\
ಕದಲೀವನಮಧ್ಯಸ್ಥಾ ಕದಂಬವನವಾಸಿನೀ ॥೨೭॥

ನಿಜೋಪಕಂಠಸಂಪ್ರಾಪ್ತನದೀಮಾಹಾತ್ಮ್ಯವರ್ಧಿನೀ ।\\
ಕಾವೇರೀ ತಾಮ್ರಪರ್ಣೀ ಚ ಕಾಮಾಕ್ಷೀ ಕಾಮಿತಾರ್ಥದಾ ॥೨೮॥

ಮೂಕಾಂಬಿಕಾ ಮಹಾಶಕ್ತಿಃ ಶ್ರೀಶೈಲಭ್ರಮರಾಂಬಿಕಾ ।\\
ಜೋಗಲಾಂಬಾ ಜಗನ್ಮಾತಾ ಮಾತಾಪುರನಿವಾಸಿನೀ ॥೨೯॥

ಜಮದಗ್ನಿಪ್ರಿಯಾ ಸಾಧ್ವೀ ತಾಪಸೀವೇಷಧಾರಿಣೀ ।\\
ಕಾರ್ತವೀರ್ಯಮುಖಾನೇಕಕ್ಷತ್ರವಿಧ್ವಂಸಕಾರಿಣೀ ॥೩೦॥

ನಾನಾವತಾರಸಂಪನ್ನಾ ನಾನಾವಿಧಚರಿತ್ರಕೃತ್ ।\\
ಕರವೀರಮಹಾಲಕ್ಷ್ಮೀಃ ಕೋಲ್ಹಾಸುರವಿನಾಶಿನೀ ॥೩೧॥

ನಾಗಲಿಂಗಭಗಾಂಕಾಢ್ಯಮೌಲಿಃ ಶಕ್ರಾದಿಸಂಸ್ತುತಾ ।\\
ದುರ್ಗಮಾಸುರಸಂಹರ್ತ್ರೀ ದುರ್ಗಾಽನರ್ಗಲಶಾಸನಾ ॥೩೨॥

ನಾನಾವಿಧಭಯತ್ರಾತ್ರೀ ಸುತ್ರಾಮಾದಿಸುಧಾಶನಾ ।\\
ನಾಗಪರ್ಯಂಕಶಯನಾ ನಾಗೀ ನಾಗಾಂಗನಾರ್ಚಿತಾ ॥೩೩॥

ಹಾಟಕಾಲಂಕೃತಿಮತೀ ಹಾಟಕೇಶಸಭಾಜಿತಾ ।\\
ವಾಗ್ದೇವತಾಸ್ಫೂರ್ತಿದಾತ್ರೀ ತ್ರಾತವಾಗ್ಬೀಜಜಾಪಕಾ ॥೩೪॥

ಜಪ್ಯಾ ಜಪವಿಧಿಜ್ಞಾ ಚ ಜಪಸಿದ್ಧಿಪ್ರದಾಯಿನೀ ।\\
ಮಂತ್ರಶಕ್ತಿರ್ಮಂತ್ರವಿದ್ಯಾ ಸುಮಂತ್ರಾ ಮಾಂತ್ರಿಕಂಪ್ರಿಯಾ ॥೩೫॥

ಇಂದುಮಂಡಲಪೀಠಸ್ಥಾ ಸೂರ್ಯಮಂಡಲಪೀಠಗಾ ।\\
ಚಂದ್ರಶಕ್ತಿಃ ಸೂರ್ಯಶಕ್ತಿರ್ಗ್ರಹಶಕ್ತಿರ್ಗ್ರಹಾರ್ತಿಹಾ ॥೩೬॥

ಗ್ರಹಪೀಡಾಹರೀ ಸೌಮ್ಯಾ ಶುಭಗ್ರಹಫಲಪ್ರದಾ ।\\
ಜ್ಯೋತಿರ್ಮಂಡಲಸಂಸ್ಥಾನಾ ಗ್ರಹತಾರಾಧಿದೇವತಾ ॥೩೭॥

ಸಪ್ತವಿಂಶತಿಯೋಗೇಶೀ ಬವಾದಿಕರಣೇಶ್ವರೀ ।\\
ಪ್ರಭವಾದ್ಯಬ್ದಶಕ್ತಿಶ್ಚ ಕಾಲಚಕ್ರಪ್ರವರ್ತಿನೀ ॥೩೮॥

ಸೋಹಂಮಂತ್ರಜಪಾಧಾರಾ ಊರ್ಧ್ವಷಟ್ಚಕ್ರದೇವತಾ ।\\
ಇಡಾಖ್ಯಾ ಪಿಂಗಲಾಖ್ಯಾ ಚ ಸುಷುಮ್ನಾ ಬ್ರಹ್ಮರಂಧ್ರಗಾ ॥೩೯॥

ಶಿವಶಕ್ತಿಃ ಕುಂಡಲಿನೀ ನಾಭಿಮಂಡಲನಿದ್ರಿತಾ ।\\
ಯೋಗೋದ್ಬುದ್ಧಾ ಮುಕ್ತಿದಾತ್ರೀ ಸಹಸ್ರಾರಾಬ್ಜಪೀಠಗಾ ॥೪೦॥

ಪೀಯೂಷವರ್ಷಿಣೀ ಜೀವಶಿವಭೇದತಿನಾಶಿನೀ ।\\
ನಾಭೌ ಪರಾ ಚ ಪಶ್ಯಂತೀ ಹೃದಯೇ ಮಧ್ಯಮಾ ಗಲೇ ॥೪೧॥

ಜಿಹ್ವಾಗ್ರೇ ವೈಖರೀವಾಣೀ ಪಂಚಾಶನ್ಮಾತೃಕಾತ್ಮಿಕಾ ।\\
ಓಂಕಾರರೂಪಿಣೀ ಶಬ್ದಸೂಷ್ಟಿರೂಪಾಽರ್ಥರೂಪಿಣೀ ॥೪೨॥

ಮೌನಶಕ್ತಿರ್ಮುನಿಧ್ಯೇಯಾ ಮುನಿಮಾನಸಸಂಸ್ಥಿತಾ ।\\
ವ್ಯಷ್ಟಿಃ ಸಮಷ್ಟಿಸ್ತ್ರಿಪುಟೀ ತಾಪತ್ರಯವಿನಾಶಿನೀ ॥೪೩॥

ಗಾಯತ್ರೀ ವ್ಯಾಹೃತಿಃ ಸಂಧ್ಯಾ ಸಾವಿತ್ರೀ ಚ ಪಿತೃಪ್ರಸೂಃ ।\\
ನಂದಾ ಭದ್ರಾ ಜಯಾ ರಿಕ್ತಾ ಪೂರ್ಣಾ ವಿಷ್ಟಿಶ್ಚ ವೈಧೃತಿಃ ॥೪೪॥

ಶ್ರುತಿಃ ಸ್ಮೃತಿಶ್ಚ ಮೀಮಾಂಸಾ ವಿದ್ಯಾಽವಿದ್ಯಾ ಪರಾವರಾ ।\\
ಸುಮೇರುಶೃಂಗನಿಲಯಾ ಲೋಕಾಲೋಕನಿವಾಸಿನೀ ॥೪೫॥

ಮಾನಸೋತ್ತರಗೋತ್ರಸ್ಥಾ ಪುಷ್ಕರದ್ವೀಪದೇವತಾ ।\\
ಮಂದರಾದ್ರಿಕೃತಕ್ರೀಡಾ ನೀಪೋಪವನವಾಸಿನೀ ॥೪೬॥

ಮಣಿದ್ವೀಪಕೃತಾವಾಸಾ ಪೀತವಾಸಃ ಸುಪೂಜಿತಾ ।\\
ಪ್ಲಕ್ಷಾದಿದ್ವೀಪಗೋತ್ರಸ್ಥಾ ತತ್ರತ್ಯಜನಪೂಜಿತಾ ॥೪೭॥

ಸುರಾಬ್ಧಿ ದ್ವೀಪನಿಲಯಾ ಸುರಾಪಾನಪರಾಯಣ ।\\
ಏಕಪಾದೇಕಹಸ್ತೈಕದೃಗೇಕಶ್ರುತಿಪಾರ್ಶ್ವಿಕಾ ॥೪೮॥

ಅರ್ಧನಾರೀಶ್ವರಾರ್ಧಾಂಗೀ ಸ್ಯೂತಾಲೋಕೋತ್ತರಾಕೃತಿಃ ।\\
ಭಕ್ತೈಕಭಕ್ತಿ ಸಂಸಾಧ್ಯಾ ಧ್ಯಾನಾಧಾರಾ ಪರಾರ್ಹಣಾ ॥೪೯॥

ಪಂಚಕೋಶಾಂತರಗತಾ ಪಂಚಕೋಶವಿವರ್ಜಿತಾ ।\\
ಪಂಚಭೂತಾತರಾಲಸ್ಥಾ ಪ್ರಪಂಚಾತೀತವೈಭವಾ ॥೫೦॥

ಪಂಚೀಕೃತಮಹಾಭೂತನಿರ್ಮಿತಾನೇಕಭೌತಿಕಾ ।\\
ಸರ್ವಾಂತರ್ಯಾಮಿಣೀ ಶಂಭುಕಾಮಿನೀ ಸಿಂಹಗಾಮಿನೀ ॥೫೧॥

ಯಾಮಿನೀಕೃತಸಂಚಾರಾ ಶಾಕಿನ್ಯಾದಿಗಣೇಶ್ವರೀ ।\\
ಖಟ್ವಾಂಗಿನೀ ಖೇಟಕಿನೀ ಕುಂತಿನೀ ಭಿಂದಿಪಾಲಿನೀ ॥೫೨॥

ವರ್ಮಿಣೀ ಚರ್ಮಿಣೀ ಚಂಡೀ ಚಂಡಮುಂಡಪ್ರಮಾಥಿನೀ ।\\
ಅನೀಕಿನೀ ಚ ಧ್ವಜಿನೀ ಮೋಹಿನ್ಯೇಜತ್ಪತಾಕಿನೀ ॥೫೩॥

ಅಶ್ವಿನೀ ಗಜಿನೀ ಚಾಟ್ಟಹಾಸಿನೀ ದೈತ್ಯನಾಶಿನೀ ।\\
ಶುಂಭಘ್ನೀ ಚ ನಿಶುಂಭಘ್ನೀ ಧೂಮ್ರಲೋಚನನಾಶಿನೀ ॥೫೪॥

ಬಹುಶೀರ್ಷಾ ಬಹುಕುಕ್ಷಿರ್ವ್ಯಾದಿತಾಸ್ಯಾಽಶಿತಾಸುರಾ ।\\
ದಂಷ್ಟ್ರಾಸಂಕಟಸಂಲಗ್ನದೈತ್ಯಸಾಸ್ರಾಂತ್ರ ಮಾಲಿನೀ ॥೫೫॥

ದೈತ್ಯಾಸೃಙ್ಮಾಂಸಸಂತೃಪ್ತಾ ಕ್ರವ್ಯಾದಕೃತವಂದನಾ ।\\
ಲಂಬಕೇಶೀ ಪ್ರಲಂಬೋಷ್ಠೀ ಲಂಬಕುಕ್ಷಿರ್ಮಹಾಕಟೀ ॥೫೬॥

ಲಂಬಸ್ತನೀ ಲಂಬಜಿಹ್ವಾ ಲಂಬಪಾಣ್ಯಂಘ್ರಿಜಂಘಿಕಾ ।\\
ಲಂಬೋರುರ್ಲಂಬಜಘನಾ ಕಾಲಿಕಾ ಕರ್ಕಶಾತ್ಕೃತಿಃ ॥೫೭॥

ಭಿನ್ನಭೇರೀಖರರವಾ ವಾರಣಗ್ರಾಸಕಾರಿಣೀ ।\\
ಪ್ರೇತದೇಹಪರೀಧಾನಾ ರುಂಡಕುಂಡಲಮಂಡಿತಾ ॥೫೮॥

ಗಂಡಶೈಲಸ್ಪರ್ಧಿಗಂಡಾ ಶೈಲಕಂದುಕಧಾರಿಣೀ ।\\
ಶಿವದೂತೀ ಘೋರರೂಪಾ ಶಿವಾಶತನಿನಾದಿನೀ ॥೫೯॥

ನಾರಾಯಣೀ ಜಗದ್ಧಾತ್ರೀ ಜಗತ್ಪಾತ್ರೀ ಜಗನ್ಮಯೀ ।\\
ಅಲ್ಲಾಂಂಬಾಕ್ಕಾ ಕಾಮದುಘಾಽನಲ್ಪದಾ ಕಲ್ಪವಲ್ಲಿಕಾ ॥೬೦॥

ಮಲ್ಲೀಮತಲ್ಲಿಕಾ ಗುಂಜಾಲಂಕೃತಿಃ ಶಿವಮೋಹಿನೀ ।\\
ಗಾಂಧರ್ವಗಾನರಸಿಕಾ ಕೇಕಾವಾಕ್ಕೀರಪಾಲಿನೀ ॥೬೧॥

ಸಿನೀವಾಲೀ ಕುಹೂ ರಾಕಾಽನುಮತಿಃ ಕೌಮುದೀ ಕಕುಪ್ ।\\
ಬ್ರಹ್ಮಾಂಡಮಂಡಪಸ್ಥೂಣಾ ಬ್ರಹ್ಮಾಂಡಗೃಹದೇವತಾ ॥೬೨॥

ಮಹಾಗೃಹಸ್ಥಮಹಿಷೀ ಪಶುಪಾಶವಿಮೋಚಿನೀ ।\\
ವೀರಸ್ಥಾನಗತಾ ವೀರಾ ವೀರಾಸನಪರಿಗ್ರಹಾ ॥೬೩॥

ದೀಪಸ್ಥಾನಗತಾ ದೀಪ್ತಾ ದೀಪೋತ್ಸವಕುತೂಹಲಾ ।\\
ತೀರ್ಥಪಾತ್ರಪ್ರದಾ ತೀರ್ಥಕುಂಭಪೂಜನಸಂಭ್ರಮಾ ॥೬೪॥

ಮಾಹೇಶ್ವರಜನಪ್ರೀತಾ ಪಶುಲೋಕಪರಾಙ್ಮುಖೀ ।\\
ಚತುಃಷಷ್ಟಿಮಹಾತಂತ್ರಪ್ರತಿಪಾದ್ಯಾಗಮಾಧ್ವಗಾ ॥೬೫॥

ಶುದ್ಧಾಚಾರಾ ಶುದ್ಧಪೂಜ್ಯಾ ಶುದ್ಧಪೂಜಾ ಜನಾಶ್ರಿತಾ ।\\
ಅಷ್ಟಾದಶಮಹಾಪೀಠಾ ಶ್ರೀಚಕ್ರಪರದೇವತಾ ॥೬೬॥

ಯೋಗಿನೀಪೂಜನಪ್ರೀತಾ ಯೋಗಿನೀಚಕ್ರವಂದಿತಾ ।\\
ರಣತ್ಕಾಂಚೀಕ್ಷುದ್ರಘಂಟಾ ಘಂಟಾಧ್ವನಿವಿನೋದಿನೀ ॥೬೭॥

ತೌರ್ಯತ್ರಿಕಕಲಾಭಿಜ್ಞಾ ಮನೋಜ್ಞಾ ಮಂಜುಭಾಷಿಣೀ ।\\
ಶಿವವಾಮಾಂಕಲಸಿತಾ ಸುಸ್ಮಿತಾ ಲಲಿತಾಲಸಾ ॥೬೮॥

ಲಾವಣ್ಯಭೂಮಿಸ್ತಲ್ಲೇಶನಿರ್ಮಿತಾಮರಸುಂದರೀ ।\\
ಶ್ರೀರ್ಧೀರ್ಭೀರ್ಹ್ರೀರ್ನತಿರ್ಜಾತಿರೀಜ್ಯಾಜ್ಯಾ ಪೂಜ್ಯಪಾದುಕಾ ॥೬೯॥

ಸುರಸ್ರವಂತೀ ಯಮುನಾ ತಥಾ ಗುಪ್ತಸರಸ್ವತೀ ।\\
ಗೋಮತೀ ಗಂಡಕೀ ತಾಪೀ ಶತದ್ರುಶ್ಚ ವಿಪಾಶಿಕಾ ॥೭೦॥

ಸರಯೂರ್ನರ್ಮದಾ ಗೋದಾ ಪಯೋಷ್ಣೀ ಚ ಪುನಃ ಪುನಾ ।\\
ಭೀಮಾ ಕೃಷ್ಣಾ ತುಂಗಭದ್ರಾ ನಾನಾತೀರ್ಥಸ್ವರೂಪಿಣೀ ॥೭೧॥

ದ್ವಾರಕಾ ಮಧುರಾ ಮಾಯಾ ಕಾಶ್ಯಯೋಧ್ಯಾ ತ್ವವಂತಿಕಾ ।\\
ಗಯಾ ಕಾಂಚೀ ವಿಶಾಲಾ ಚ ಮುಕ್ತಿಕ್ಷೇತ್ರಸ್ವರೂಪಿಣೀ ॥೭೨॥

ಮಂತ್ರದೀಕ್ಷಾ ಯಾಗದೀಕ್ಷಾ ಯೋಗದೀಕ್ಷಾಽಕ್ಷತವ್ರತಾ ।\\
ಅಕ್ಷಮಾಲಾವಿಭೇದಜ್ಞಾಽಽಸನಭೇದವಿಚಕ್ಷಣಾ ॥೭೩॥

ಮಾತೃಕಾನ್ಯಾಸಕುಶಲಾ ಮಂತ್ರನ್ಯಾಸವಿಶಾರದಾ ।\\
ನಾನಾಮುದ್ರಾಪ್ರಭೇದಜ್ಞಾ ಪಂಚೋಪಾಸ್ತಿಪ್ರಭೇದವಿತ್ ॥೭೪॥

ಸರ್ವಮಂತ್ರೋಪದೇಷ್ಟ್ರೀ ಚ ವ್ಯಾಖ್ಯಾತ್ರೀ ದೇಶಿಕೋತ್ತರಾ ।\\
ವಾಗ್ವಾದಿನೀ ದುರ್ವಿವಾದಿಪ್ರೌಢವಾಕ್ಸ್ತಂಭಕಾರಿಣೀ ॥೭೫॥

ಸ್ವತಂತ್ರಯಂತ್ರಣಾಶಕ್ತಿಸ್ತದ್ಯಂತ್ರಿತಜಗತ್ತ್ರಯೀ ।\\
ಬ್ರಹ್ಮಾದ್ಯಾಕರ್ಷಿಣೀ ಶಂಭುಮೋಹಿನ್ಯುಚ್ಚಾಟಿನೀ ದ್ವಿಷಾಂ ॥೭೬॥

ಸುರಾಸುರಾಣಾಂ ಪ್ರದ್ವೇಷಕಾರಿಣೀ ದೈತ್ಯಮಾರಿಣೀ ।\\
ಪೂರಿಣೀ ಭಕ್ತಕಾಮಾನಾಂ ಶ್ರಿತಪ್ರತ್ಯೂಹವಾರಿಣೀ ॥೭೭॥

ಸಾಧಾರಣೀ ಧಾರಿಣೀ ಚ ಪ್ರೌಢವಾಗ್ಧಾರಿಣೀ ಪ್ರಧೂಃ ।\\
ಪ್ರಭೂರ್ವಿಭೂಃ ಸ್ವಯಂಭೂಶ್ಚ ನಿಗ್ರಹಾನುಗ್ರಹಕ್ಷಮಾ ॥೭೮॥

ಕ್ಷಮಾಽಕ್ಷಮಾ ಕ್ಷಮಾಧಾರಾ ಧಾರಾಧರನಿಭದ್ಯುತಿಃ ।\\
ಕಾದಂಬಿನೀ ಕಾಲಶಕ್ತಿಃ ಕರ್ಷಿಣೀ ವರ್ಷಿಣೀರ್ಷಿಣೀ ॥೭೯॥

ಅದೇವಮಾತೃಕಾ ದೇವಮಾತೃಕಾ ಚೋರ್ವರಾ ಕೃಷಿಃ ।\\
ಕೃಷ್ಟಪಚ್ಯಾಽಕೃಷ್ಟಪಚ್ಯಾಽನೂಷರಾಽಧಿತ್ಯಕಾ ಗುಹಾ ॥೮೦॥

ಉಪತ್ಯಕಾ ದರೀ ವನ್ಯಾಽರಣ್ಯಾನೀ ಶೈಲನಿಮ್ನಗಾ ।\\
ಕೇದಾರಭೂಮಿರ್ವ್ರೀಹಿಲಾ ಕಮಲರ್ಧಿರ್ಮಹಾಫಲಾ ॥೮೧॥

ಇಕ್ಷುಮತ್ಯೂರ್ಜಿತಾ ಜಂಬೂಪನಸಾಮ್ರಾದಿಶಾಲಿನೀ ।\\
ಅಷ್ಟಾಪದಖನೀ ರೌಪ್ಯಖನೀ ರತ್ನಖನಿಃ ಖನಿಃ ॥೮೨॥

ರುಚಾಂ ಭಾವಖನಿರ್ಜೀವಖನಿಃ ಸೌಭಾಗ್ಯಸತ್ಖನಿಃ ।\\
ಲಾವಣ್ಯಸ್ಯ ಖನಿರ್ಧೈರ್ಯಖನಿಃ ಶೌರ್ಯಖನಿಃ ಖನಿಃ ॥೮೩॥

ಗಾಂಭೀರ್ಯಸ್ಯ ವಿಲಾಸಸ್ಯ ಖನಿಃ ಸಾಹಿತ್ಯಸತ್ಖನಿಃ ।\\
ಪಕ್ಷಮಾಸರ್ತುವರ್ಷಾಣಾಂ ಖನಿಃ ಖನಿರನೇಹಸಾಂ ॥೮೪॥

ಖನಿಶ್ಚ ಯುಗಕಲ್ಪಾನಾಂ ಸೂರ್ಯಚಂದ್ರಮಸೋಃ ಖನಿಃ ।\\
ಖನಿರ್ಮನೂನಾಮಿಂದ್ರಾಣಾಂ ಖನಿಃ ಕೌತುಕಸತ್ಖನಿಃ ॥೮೫॥

ಮಷೀ ಲೇಖನಿಕಾ ಪಾತ್ರೀ ವರ್ಣಪಂಕ್ತಿರ್ಲಿಪಿಃ ಕಥಾ ।\\
ಕವಿತಾ ಕಾವ್ಯಕರ್ತ್ರೀ ಚ ದೇಶಭಾಷಾ ಜನಶ್ರುತಿಃ ॥೮೬॥

ರಚನಾ ಕಲ್ಪನಾಽಽಚಾರಭಟೀ ಧಾಟೀ ಪಟೀಯಸೀ ।\\
ಅಪಸವ್ಯಲಿಪಿರ್ದೇವಲಿಪೀ ರಕ್ಷೋಲಿಪಿರ್ಲಿಪಿಃ ॥೮೭॥

ತುಲಜಾ ರಾಮವರದಾ ಶಬರೀ ಬರ್ಬರಾಲಕಾ ।\\
ಜ್ಯೋತಿರ್ಲಿಂಗಮಯೀ ಲಿಂಗಮಸ್ತಕಾ ಲಿಂಗಧಾರಿಣೀ ॥೮೮॥

ರುದ್ರಾಕ್ಷಧಾರಿಣೀ ಭೂತಿಧಾರಿಣೀ ಲೈಂಗಿಕವ್ರತಾ ।\\
ವಿಷ್ಣುವ್ರತಪರಾ ವಿಷ್ಣುಮುದ್ರಿಕಾ ಪ್ರಿಯವೈಷ್ಣವೀ ॥೮೯॥

ಜೈನೀ ದೈಗಂಬರೀ ನಾನಾವಿಧವೈದಿಕ ಮಾರ್ಗಗಾ ।\\
ಪಂಚದ್ರವಿಡಸಂಸೇವ್ಯಾ ಪಂಚಗೌಡಸಮರ್ಚಿತಾ ॥೯೦॥

ಹಿಂಗುಲಾ ಶಾರದಾ ಜ್ವಾಲಾಮುಖೀ ಗಂಜಾಧಿದೇವತಾ ।\\
ಮಂದುರಾದೇವತಾಽಽಲಾನದೇವತಾ ಗೋಷ್ಠದೇವತಾ ॥೯೧॥

ಗೃಹಾದಿದೇವತಾಸೌಧದೇವತೋದ್ಯಾನದೇವತಾ ।\\
ಶೃಂಗಾರದೇವತಾ ಗ್ರಾಮದೇವತಾ ಚೈತ್ಯದೇವತಾ ॥೯೨॥

ಪೂರ್ದೇವತಾ ರಾಜಸಭಾದೇವತಾಽಶೋಕದೇವತಾ ।\\
ಚತುರ್ದಶಾನಾಂ ಲೋಕಾನಾಂ ದೇವತಾ ಪರದೇವತಾ ॥೯೩॥

ವಾಪೀಕೂಪತಡಾಗಾದಿದೇವತಾ ವನದೇವತಾ ।\\
ಸೃಷ್ಟಿಖೇಲಾ ಕ್ಷೇಮಖೇಲಾ ಪ್ರತಿಸಂಹಾರಖೇಲನಾ ॥೯೪॥

ಪುಷ್ಪಪರ್ಯಂಕಶಯನಾ ಪುಷ್ಪವತ್ಕುಂಡಲದ್ವಯೀ ।\\
ತಾಂಬೂಲಚರ್ವಣಪ್ರೀತಾ ಗೌರೀ ಪ್ರಥಮಪುಷ್ಪಿಣೀ ॥೯೫॥

ಕಲ್ಯಾಣೀಯುಗಸಂಪನ್ನಾ ಕಾಮಸಂಜೀವನೀ ಕಲಾ ।\\
ಕಲಾಕಲಾಪಕುಶಲಾ ಕಲಾತೀತಾ ಕಲಾತ್ಮಿಕಾ ॥೯೬॥

ಮಹಾಜಾಂಗಲಕೀ ಕಾಲಭುಜಂಗವಿಷನಾಶಿನೀ ।\\
ಚಿಕಿತ್ಸಾ ವೈದ್ಯವಿದ್ಯಾ ಚ ನಾನಾಮಯನಿದಾನವಿತ್ ॥೯೭॥

ಪಿತ್ತಪ್ರಕೋಪಶಮನೀ ವಾತಘ್ನೀ ಕಫನಾಶಿನೀ ।\\
ಆಮಜ್ವರಪ್ರಕೋಪಘ್ನೀ ನವಜ್ವರನಿವಾರಿಣೀ ॥೯೮॥

ಅರ್ಶೋಘ್ನೀ ಶೂಲಶಮನೀ ಗುಲ್ಮವ್ಯಾಧಿನಿವಾರಿಣೀ ।\\
ಗ್ರಹಣೀನಿಗ್ರಹಕರೀ ರಾಜಯಕ್ಷ್ಮವಿನಾಶಿನೀ ॥೯೯॥

ಮೇಹಘ್ನೀ ಪಾಂಡುರೋಗಘ್ನೀ ಕ್ಷಯಾಪಸ್ಮಾರನಾಶಿನೀ ।\\
ಉಪದಂಶಹರೀ ಶ್ವಾಸಕಾಸಚ್ಛರ್ದಿನಿವಾರಿಣೀ ॥೧೦೦॥

ಪ್ಲೀಹಪ್ರಕೋಪಸಂಹರ್ತ್ರೀ ಪಾಮಾಕಂಡೂವಿನಾಶಿನೀ ।\\
ದದ್ರುಕುಷ್ಠಾದಿರೋಗಘ್ನೀ ನಾನಾರೋಗವಿನಾಶಿನೀ ॥೧೦೧॥

ಯಕ್ಷರಾಕ್ಷಸವೇತಾಲಕೂಷ್ಮಾಡಗ್ರಹಭೇದಿನೀ ।\\
ಬಾಲಗ್ರಹಘ್ನೀ ಚಂಡಾಲಗ್ರಹಚಂಡಗ್ರಹಾರ್ದಿನೀ ॥೧೦೨॥

ಭೂತಪ್ರೇತಪಿಶಾಚಘ್ನೀ ನಾನಾಗ್ರಹವಿಮರ್ದಿನೀ ।\\
ಶಾಕಿನೀಡಾಕಿನೀಲಾಮದಸ್ತ್ರೀಗ್ರಹನಿಷೂದಿನೀ ॥೧೦೩॥

ಅಭಿಚಾರಕದುರ್ಬಾಧೋನ್ಮಥಿನ್ಯುನ್ಮಾದನಾಶಿನೀ ।\\
ನಾನಾವಿಷಾರ್ತಿಸಂಹರ್ತ್ರೀ ದುಷ್ಟದೃಷ್ಟ್ಯಾರ್ತಿನಾಶಿನೀ ॥೧೦೪॥

ದುಷ್ಟಸ್ಥಾನಸ್ಥಿತಾರ್ಕಾದಿಗ್ರಹಪೀಡಾವಿನಾಶಿನೀ ।\\
ಯೋಗಕ್ಷೇಮಕರೀ ಪುಷ್ಟಿಕರೀ ತುಷ್ಟಿಕರೀಷ್ಟದಾ ॥೧೦೫॥

ಧನದಾ ಧಾನ್ಯದಾ ಗೋದಾ ವಾಸೋದಾ ಬಹುಮಾನದಾ ।\\
ಕಾಮಿತಾರ್ಥಪ್ರದಾ ಶಂದಾ ಚತುರ್ವಿಧತುಮರ್ಥದಾ ॥೧೦೬॥

ಬ್ರಹ್ಮಮಾನ್ಯಾ ವಿಷ್ಣುಮಾನ್ಯಾ ಶಿವಮಾನ್ಯೇಂದ್ರಮಾನಿತಾ ।\\
ದೇವರ್ಷಿಮಾನ್ಯಾ ಬ್ರಹ್ಮರ್ಷಿಮಾನ್ಯಾ ರಾಜರ್ಷಿಮಾನಿತಾ ॥೧೦೭॥

ಜಗನ್ಮಾನ್ಯಾ ಜಗದ್ಧಾತ್ರೀ ತ್ರ್ಯಂಬಕಾ ತ್ರ್ಯಂಬಕಾಂಗನಾ ।\\
ಗುಹೇಭವಕ್ತ್ರಜನನೀ ಭದ್ರಕಾಲ್ಯಮಯಂಕರೀ ॥೧೦೮॥

ಸತೀ ದಾಕ್ಷಾಯಣೀ ದಕ್ಷಾ ದಕ್ಷಾಧ್ವರನಿಷೂದಿನೀ ।\\
ಉಮಾ ಹಿಮಾದ್ರಿಜಾಽಪರ್ಣಾ ಕರ್ಣಪೂರಾಂಚಿತಾನನಾ ॥೧೦೯॥

ಕ್ಷೋಣೀಧರನಿತಂಬಾಂಬಾ ಬಿಂಬಾಭರದನಚ್ಛದಾ ।\\
ಕುಂದಕೋರಕನೀಕಾಶರದಪಂಕ್ತಿಃ ಸುನಾಸಿಕಾ ॥೧೧೦॥

ನಾಸಿಕಾಮೌಕ್ತಿಕಮಣಿಪ್ರಭಾಮಲರದಚ್ಛದಾ ।\\
ಸುಮಂದಹಸಿತಾಲೋಕವಿಮುಹ್ಯಚ್ಛಂಭು ಮಾನಸಾ ॥೧೧೧॥

ಮಾನಸೌಕೋಗತಿಃ ಸಿಂಜನ್ಮಂಜೀರಾದಿವಿಭೂಷಣಾ ।\\
ಕಾಂಚೀಕ್ಕಣತ್ಕ್ಷುದ್ರಘಂಟಾ ಕಲಧೌತಘಟಸ್ತನೀ ॥೧೧೨॥

ಮುಕ್ತಾವಿದ್ರುಮಹಾರಶ್ರೀರಾಜನ್ಕುಚತಟಸ್ಥಲೀ ।\\
ತಾಟಂಕಯುಗಸತ್ಕಾಂತಿವಿಲಸದ್ಗಂಡದರ್ಪಣಾ ॥೧೧೩॥

ಧಮ್ಮಿಲ್ಲಪ್ರೋತ ಸೌವರ್ಣಕೇತಕೀದಲಮಂಡಿತಾ ।\\
ಶಿವಾರ್ಪಿತಸ್ವಲಾವಣ್ಯತಾರುಣ್ಯಕಿಲಿಕಿಂಚಿತಾ ॥೧೧೪॥

ಸಗ್ರೈವೇಯಕಚಿಂತಾಕವಿಭ್ರಾಜತ್ಕಂಬುಕಂಧರಾ ।\\
ಸಖೀಸ್ಕಂಧಾಸಕ್ತಬಾಹುಲತಾಸುಲಲಿತಾಂಗಿಕಾ ॥೧೧೫॥

ನಿತಂಬಕುಚಭಾರಾರ್ತಕೃಶಮಧ್ಯಸುಮಧ್ಯಮಾ ।\\
ನಾಭೀಸರೋವರೋದ್ಭೂತರೋಮಾವಲಿವಿರಾಜಿತಾ ॥೧೧೬॥

ತ್ರಿವಲೀರಾಜಲಲಿತಗೌರವರ್ಣತಲೋದರೀ ।\\
ಚಾಮರಗ್ರಾಹಿಣೀವೀಜ್ಯಮಾನೇಂದುದ್ಯುತಿಚಾಮರಾ ॥೧೧೭॥

ಮಹಾಕುರಬಕಾಶೋಕಪುಷ್ಪವಚಯಲಾಲಸಾ ।\\
ಸ್ವತಪಃಸುಫಲೀಭೂತವರೋತ್ತಮಮಹೇಶ್ವರೀ ॥೧೧೮॥

ಹಿಮಾಚಲತಪಃಪುಣ್ಯಫಲಭೂತಸುತಾಕೃತಿಃ ।\\
ತ್ರೈಲೋಕ್ಯರಮಣೀರತ್ನಾ ಶಿವಚಿಚ್ಚಂದ್ರಚಂದ್ರಿಕಾ ॥೧೧೯॥

ಕೀರ್ತಿಜ್ಯೋತ್ಸ್ನಾಧವಲಿತಾನೇಕಬ್ರಹ್ಮಾಂಡಗೋಲಿಕಾ ।\\
ಶಿವಜೀವಾತುಗುಲಿಕಾಲಸದಂಗುಲಿಮುದ್ರಿಕಾ ॥೧೨೦॥

ಪ್ರೀತಿಲಾಲಿತಸತ್ಪುತ್ರಗಜಾನನಷಡಾನನಾ ।\\
ಗಂಗಾಸಾಪತ್ನ್ಯಜನಿತೇರ್ಯಾಯತೇಕ್ಷಣಶೋಣಿಮಾ ॥೧೨೧॥

ಶಿವವಾಮಾಂಕಪರ್ಯಂಂಕಕೃತಾಸನಪರಿಗ್ರಹಾ ।\\
ಶಿವದೃಷ್ಟಿಚಕೋರೀಷ್ಟಮುಖಪೂರ್ಣೇಂದುಮಂಡಲಾ ॥೧೨೨॥

ಉರೋಜಶೈಲಯುಗಲಭ್ರಮಚ್ಛಿವಮನೋಮೃಗಾ ।\\
ಬ್ರಹ್ಮವಿಷ್ಣುಮುಖಾಶೇಷವೃಂದಾರಕನಭಸ್ಕೃತಾ ॥೧೨೩॥

ಅಪಾಂಗಾಂಗಣಸಂತಿಷ್ಠನ್ನಿಗ್ರಹಾನುಗ್ರಹದ್ವಯೀ ।\\
ಸನತ್ಕುಮಾರದುರ್ವಾಸಃ ಪ್ರಮುಖೋಪಾಸಕಾರ್ಥಿತಾ ॥೧೨೪॥

ಹಾದಿಕೂಟತ್ರಯೋಪಾಸ್ಯಾ ಕಾದಿಕೂಟತ್ರಯಾರ್ಚಿತಾ ।\\
ಮೃದ್ವೀಕಮಧುಪಾನಾದ್ಯುದ್ಬೋಧಧೂರ್ಣಿತಲೋಚನಾ ॥೧೨೫॥

ಅವ್ಯಕ್ತಭಾಷಣಾ ಚೇಟೀದತ್ತವೀಟೀಗ್ರಹಾಲಸಾ ।\\
ಗಾಯಂತೀ ವಿಲಸಂತೀ ಚ ಲಿಖಂತೀ ಪ್ರಿಯಲೇಖನಾ ॥೧೨೬॥

ಉಲ್ಲಸಂತೀ ಲಸಂತೀ ಚ ದೋಲಾಂದೋಲನತುಷ್ಟಿಭಾಕ್ ।\\
ವಿಪಂಚೀವಾದನರತಾ ಸಪ್ತಸ್ವರವಿಭೇದವಿತ್ ॥೧೨೭॥

ಗಂಧರ್ವೀಕಿನ್ನರೀವಿದ್ಯಾಧರೀಯಕ್ಷಸುರೀನುತಾ ।\\
ಅಮರೀಕಬರೀಭೃಂಗೀಸ್ಥಗಿತಾಂಘ್ರಿಸರೋರುಹಾ ॥೧೨೮॥

ಗಂಧರ್ವಗಾನಶ್ರವಣಾನಂದಾಂದೋಲಿತಮಸ್ತಕಾ ।\\
ತಿಲೋತ್ತಮೋರ್ವಶೀರಂಭಾನೃತ್ಯದರ್ಶನಜಾತಮುತ್ ॥೧೨೯॥

ಅಚಿಂತ್ಯಮಹಿಮಾಽಚಿಂತ್ಯಗರಿಮಾಽಚಿಂತ್ಯಲಾಧವಾ ।\\
ಅಚಿಂತ್ಯವಿಭವಾಽಚಿಂತ್ಯವಿಕ್ರಮಾಽಚಿಂತ್ಯಸದ್ಗುಣಾ ॥೧೩೦॥

ಸ್ತುತಿಃ ಸ್ತವ್ಯಾ ನತಿರ್ನಮ್ಯಾ ಗತಿರ್ಗಮ್ಯಾ ಮಹಾಸತೀ ।\\
ಅನಸೂಯಾಽರುಂಧತೀ ಚ ಲೋಪಾಮುದ್ರಾಽದಿತಿರ್ದಿತಿಃ ॥೧೩೧॥

ಸಪ್ತಸಂಸ್ಥಾಸ್ವರೂಪಾ ಚಾರಣಿಃ ಸ್ರುಗ್ವೇದಿಕಾ ಧ್ರುವಾ ।\\
ಇಡಾ ಪ್ರಣೀತಾ ಪಾತ್ರೀ ಚ ಸ್ವಧಾಸ್ವಾಹಾಽಽಹುತಿರ್ವಪಾ ॥೧೩೨॥

ಕವ್ಯರೂಪಾ ಹವ್ಯರೂಪಾ ಯಜ್ಞಪಾತ್ರಸ್ವರೂಪಿಣೀ ।\\
ಶಂಕರಾಹೋಪುರುಷಿಕಾ ಗಾಯತ್ರೀಗರ್ಭವಲ್ಲಭಾ ॥೧೩೩॥

ಚತುರ್ವಿಂಶತ್ಯಕ್ಷರಾತ್ಮಪರೋರಜಸಿಸಾವದೋಂ ।\\
ವೇದಪ್ರಸೂರ್ವೇದಗರ್ಭಾ ವಿಶ್ವಾಮಿತ್ರರ್ಷಿಪೂಜಿತಾ ॥೧೩೪॥

ಋಗ್ಯಜುಃ ಸಾಮತ್ರಿಪದಾ ಮಹಾಸವಿತೃದೇವತಾ ।\\
ದ್ವಿಜತ್ವಸಿದ್ಧಿದಾತ್ರೀ ಚ ಸಾಂಖ್ಯಾಯನಸಗೋತ್ರಿಕಾ ॥೧೩೫॥

ಗಾತೃದುರ್ಗತಿಸಂಹರ್ತ್ರೀ ಚ ಗಾತೃಸ್ವರ್ಗಾಪವರ್ಗದಾ ।\\
ಭೃಗುವಲ್ಲೀ ಬ್ರಹ್ಮವಲ್ಲೀ ಕಠವಲ್ಲೀ ಪರಾತ್ಪರಾ ॥೧೩೬॥

ಕೈವಲ್ಪೋಪನಿಷದ್ಬ್ರಹ್ಮೋಪನಿಷನ್ಮುಂಡಕೋಪನಿಷತ್ ।\\
ಛಾಂದೋಗ್ಯೋಪನಿಷಚ್ಚೈವ ಸರ್ವೋಪನಿಷದಾತ್ಮಿಕಾ ॥೧೩೭॥

ತತ್ತ್ವಮಾದಿಮಹಾವಾಕ್ಯರೂಪಾಖಂಡಾರ್ಥಬೋಧಕೃತ್ ।\\
ಉಪಕ್ರಮಾದಿಷಡ್ಲಿಂಗಮಹಾತಾತ್ಪರ್ಯಬೋಧಿನೀ ॥೧೩೮॥

ಜಹತ್ಸ್ವಾರ್ಥಾಜಹತ್ಸ್ವಾರ್ಥಭಾಗತ್ಯಾಗಾಖ್ಯಲಕ್ಷಣಾ ।\\
ಅಸತ್ಖ್ಯಾತಿಶ್ಚ ಸತ್ಖ್ಯಾತಿರ್ಮಿಥ್ಯಾಖ್ಯಾತಿಃ ಪ್ರಭಾಽಪ್ರಭಾ ॥೧೩೯॥

ಅಸ್ತಿಭಾತಿಪ್ರಿಯಾತ್ಮಾ ಚ ನಾಮರೂಪಸ್ವರೂಪಿಣೀ ।\\
ಷಣ್ಮತಸ್ಥಾಪನಾಚಾರ್ಯಮಹಾದೇವಕುಟುಂಬಿನೀ ॥೧೪೦॥

ಭ್ರಾಂತಿರ್ಭಾಂತಿಹರೀ ಭ್ರಾಂತಿದಾಯಿನೀ ಭ್ರಾಂತಿಕಾರಿಣೀ ।\\
ವಿಕೃತಿರ್ನಿಕೃತಿಶ್ಚಾಪಿ ಕೃತಿಶ್ಚೋಪಕೃತಿಸ್ತಥಾ ॥೧೪೧॥

ಸಕೃತಿಃ ಸತ್ಕೃತಿಃ ಪಾಪಕೃತಿಃ ಸುಕೃತಿರುತ್ಕೃತಿಃ ।\\
ಆಕೃತಿರ್ವ್ಯಾಕೃತಿಃ ಪ್ರಾಯಶ್ಚಿತ್ತಿರ್ವಿತ್ತಿಃ ಸ್ಥಿತಿರ್ಗತಿಃ ॥೧೪೨॥

ನವಕುಂಡೀ ಪಂಚಕುಂಡೀ ಚತುಷ್ಕುಂಡ್ಯೇಕಕುಂಡಿಕಾ ।\\
ದೇವಸೇನಾದೈತ್ಯಸೇನಾರಕ್ಷಃ ಸೇನಾದಿಭೇದಭಾಕ್ ॥೧೪೩॥

ಮಹಿಷಾಸ್ಯೋದ್ಗತೋದ್ದಂಡದೈತ್ಯವೇತಂಡರುಂಡಹೃತ್ ।\\
ದೈತ್ಯಸೇನಾತೃಣಾರಣ್ಯಜ್ವಾಲಾ ಜ್ಯೋತಿಃಸ್ವರೂಪಿಣೀ ॥೧೪೪॥

ಚಂಡಮುಂಡ ಮಹಾದೈತ್ಯರುಂಡಡಕಂದುಕಖೇಲಕೃತ್ ।\\
ಪಶುದೇಹಾಸೃಕ್ಪಲಲತೃಪ್ತಾ ದಕ್ಷಿಣಕಾಲಿಕಾ ॥೧೪೫॥

ತದಂಚತ್ಪ್ರೇತತದ್ರುಂಡಮಾಲಾ ವ್ಯಾಲವಿಭೂಷಣಾ ।\\
ಹಲದಂಷ್ಟ್ರಾ ಶಂಕುರದಾ ಪೀತಾನೇಕಸುರಾಘಟಾ ॥೧೪೬॥

ನೇತ್ರಭ್ರಮಿಪರಾಭೂತರಥಚಕ್ರದ್ವಯಭ್ರಮಿಃ ।\\
ವ್ಯಾಯಾಮಾಗ್ರಾಹ್ಯವಕ್ಷೋಜನ್ಯಕ್ಕೃತೇಭೇಂದ್ರಕುಂಭಿಕಾ ॥೧೪೭॥

ಮಾತೃಮಂಡಲಮಧ್ಯಸ್ಥಾ ಮಾತೃಮಂಡಲಪೂಜಿತಾ ।\\
ಘಂಟಾಘಣಘಣಧ್ವಾನಪ್ರೀತಾ ಪ್ರೇತಾಸನಸ್ಥಿತಾ ॥೧೪೮॥

ಮಹಾಲಸಾ ಚ ಮಾರ್ತಂಡಭೈರವಪ್ರಾಣವಲ್ಲಭಾ ।\\
ರಾಜವಿದ್ಯಾ ರಾಜಸತೀ ರಾಜಸ್ತ್ರೀ ರಾಜಸುಂದರೀ ॥೧೪೯॥

ಗಜರಾಜಾದಿರಲಕಾಪುರೀಸ್ಥಾ ಯಕ್ಷದೇವತಾ ।\\
ಮತ್ಸ್ಯಮೂರ್ತಿಃ ಕೂರ್ಮಮೂರ್ತಿಃ ಸ್ವೀಕೃತಕ್ರೋಡವಿಗ್ರಹಾ ॥೧೫೦॥

ನೃಸಿಂಹಮೂರ್ತಿರ್ರತ್ಯುಗ್ರಾ ಧೃತವಾಮನವಿಗ್ರಹಾ ।\\
ಶ್ರೀಜಾಮದಗ್ನ್ಯಮೂರ್ತಿಶ್ಚ ಶ್ರೀರಾಮಮೂರ್ತಿರ್ಹತಾಸ್ರಪಾ ॥೧೫೧॥

ಕೃಷ್ಣಮೂರ್ತಿರ್ಬುದ್ಧಮೂರ್ತಿಃ ಕಲ್ಕಿಮೂರ್ತಿರಮೂರ್ತಿಕಾ ।\\
ವಿರಾಣ್ಮೂರ್ತಿರ್ಜಗನ್ಮೂರ್ತಿರ್ಜಗಜ್ಜನ್ಮಾದಿಕಾರಿಣೀ ॥೧೫೨॥

ಆಧಾರಾಧೇಯಸಂಬಂಧಹೀನಾ ತದುಭಯಾತ್ಮಿಕಾ ।\\
ನಿರ್ಗುಣಾ ನಿಷ್ಕ್ರಿಯಾಽಸಂಗಾ ಧರ್ಮಾಧರ್ಮವಿವರ್ಜಿತಾ ॥೧೫೩॥

ಮಾಯಾಸಂಬಂಧರಹಿತಾಸಚ್ಚಿದಾನಂದವಿಗ್ರಹಾ ।\\
ಜಗತ್ತರಂಗಜಲಧಿರೂಪಾ ಚೀತ್ಕೃತಿಭೇದಭಾಕ್ ॥೧೫೪॥

ಮಹಾತತ್ತ್ವಾತ್ಮಿಕಾ ವೈಕಾರಾಹಂಕಾರಸ್ವರೂಪಿಣೀ ।\\
ರಜೋಹಂಕಾರರೂಪಾ ಚ ತಮೋಹಂಕಾರರೂಪಿಣೀ ॥೧೫೫॥

ಆಕಾಶರೂಪಿಣೀ ವಾಯುರೂಪಿಣ್ಯಗ್ನಿಸ್ವರೂಪಿಣೀ ।\\
ಅಂಬಾತ್ಮಿಕಾ ಭೂಸ್ವರೂಪಾ ಪಂಚಜ್ಞಾನೇಂದ್ರಿಯಾತ್ಮಿಕಾ ॥೧೫೬॥

ಕರ್ಮೇಂದ್ರಿಯಾತ್ಮಿಕಾ ಪ್ರಾಣಾಪಾನವ್ಯಾನಾದಿರೂಪಿಣೀ ।\\
ನಾಗಕೂರ್ಮಾದಿರೂಪಾ ಚ ಸರ್ವನಾಡೀವಿಹಾರಿಣೀ ॥೧೫೭॥

ಆಧಾರಚಕ್ರಾಧಿಷ್ಠಾತ್ರೀ ಸ್ವಾಧಿಷ್ಠಾನಪ್ರತಿಷ್ಠಿತಾ ।\\
ಮಣಿಪೂರಕಸಂಸ್ಥಾನಾಽನಾಹತಾಬ್ಜಾಧಿದೇವತಾ ॥೧೫೮॥

ವಿಶುದ್ಧಚಕ್ರಪೀಠಸ್ಥಾಽಽಜ್ಞಾಚಕ್ರಪರಮೇಶ್ವರೀ ।\\
ದ್ವಿಪತ್ರೀ ಷೋಡಶದಲೀ ತಥಾ ದ್ವಾದಶಪತ್ರಿಕಾ ॥೧೫೯॥

ಪ್ರದಲದ್ದಶಪತ್ರೀ ಚ ಷಡ್ದಲೀ ಚ ಚತುರ್ದಲೀ ।\\
ವಾಸಾಂತಮಾತೃಕಾ ಬಾದಿಲಾಂತವರ್ಣಾಧಿದೇವತಾ ॥೧೬೦॥

ಡಾದಿಫಾಂತಾಕ್ಷರವತೀ ಕಾದಿಠಾಂತಾಕ್ಷರೇಶ್ವರೀ ।\\
ಷೋಡಶಸ್ವರಬೀಜೇಶೀ ಸ್ಪರ್ಶೋಷ್ಮಾಂತಃಸ್ಥದೇವತಾ ॥೧೬೧॥

ಹಕ್ಷಾಕ್ಷರದ್ವಯೀರೂಪಾ ಪಂಚಾಶನ್ಮಾತೃಕೇಶ್ವರೀ ।\\
ಸಹಸ್ರಾರಾಬ್ಜಪೀಠಸ್ಥಾ ಶಿವಶಕ್ತಿರ್ವಿಮುಕ್ತಿದಾ ॥೧೬೨॥

ಪಂಚಾಮ್ನಾಯಶಿವಪ್ರೋಕ್ತಾ ಮಂತ್ರಬೀಜಾಧಿದೇವತಾ ।\\
ಸೌಃಕ್ಲೀಂಹ್ರೀಂಬೀಜಫಲದಾ ಮಹಾಪ್ಲಕ್ಷಸರಸ್ವತೀ ॥೧೬೩॥

ನವಾರ್ಣವಾ ಸಪ್ತಶತೀ ಮಾಲಾಮಂತ್ರಸ್ವರೂಪಿಣೀ ।\\
ಅರ್ಗಲೇಶೀ ಕೀಲಕೇಶೀ ಕವಚೇಶೀ ತ್ರಿಮೂರ್ತಿಕಾ ॥೧೬೪॥

ಸಕಾರಾದಿಹಕಾರಾಂತಮಹಮಂತ್ರಾಧಿದೇವತಾ ।\\
ಸಕೃತ್ಸಪ್ತಶತೀಪಾಠಪ್ರೀತಾ ಪ್ರೋಕ್ತಫಲಪ್ರದಾ ॥೧೬೫॥

ಕುಮಾರೀಪೂಜನೋದ್ಯನ್ಮುಚ್ಚಿರಂಟೀಪೂಜನೋತ್ಸುಕಾ ।\\
ವಿಪ್ರಪೂಜನಸಂತುಷ್ಟಾ ನಿತ್ಯಶ್ರೀರ್ನಿತ್ಯಮಂಗಲಾ ॥೧೬೬॥

ಜಯದಾದಿಮಚಾರಿತ್ರಾ ಶ್ರೀದ ಮಧ್ಯಚರಿತ್ರಿಕಾ ।\\
ವಿದ್ಯಾದೋತ್ತಮಚಾರಿತ್ರಾ ಕಾಮಿತಾರ್ಥಪ್ರದಾಯಿಕಾ ॥೧೬೭॥

ಇಷ್ಟಕೃಷ್ಣಾಷ್ಟಮೀ ಚೇಷ್ಟನವಮೀ ಭೂತಪೂರ್ಣಿಮಾ ।\\
ಇಷ್ಟಶುಕ್ರಾರದಿವಸಾ ಧೂತದೀಪದ್ವಯೋತ್ಸುಕಾ ॥೧೬೮॥

ನವರಾತ್ರೋತ್ಸವಾಸಕ್ತಾ ಪೂಜಾಹೋಮಬಲಿಪ್ರಿಯಾ ।\\
ಇಷ್ಟೇಕ್ಷುಕೂಷ್ಮಾಂಡಫಲಾಹೂತಿರಿಷ್ಟಫಲಾಹುತಿಃ ॥೧೬೯॥

ಪ್ರತಿಮಾಸೂತ್ಕೃಷ್ಟಪುಣ್ಯಾ ಚ ಮಹಾಯಂತ್ರಾರ್ಚನಾವಿಧಿಃ ।\\
ಕಾತ್ಯಾಯನೀ ಕಾಮದೋಗ್ಧ್ರೀ ಖೇಚರೀ ಖಡ್ಗಚರ್ಮಧೃಕ್ ॥೧೭೦॥

ಗಜಾಸ್ಯಮಾತಾ ಘಟಿಕಾ ಚಂಡಿಕಾ ಚಕ್ರಧಾರಿಣೀ ।\\
ಛಾಯಾ ಛವಿಮಯೀ ಛನ್ನಾ ಜರಾಮೃತ್ಯುವಿವರ್ಜಿತಾ ॥೧೭೧॥

ಝಲ್ಲೀಝಂಕಾರಮುದಿತಾ ಝಂಝಾವಾತಝಣತ್ಕೃತಿಃ ।\\
ಟಂಕಹಸ್ತಾ ಟಣಚ್ಚಾಪಠದ್ವಯೀ ಪಲ್ಲವೋನ್ಮನುಃ ॥೧೭೨॥

ಡಮರುಧ್ವಾನಮುದಿತಾ ಡಾಕಿನೀಶಾಕಿನೀಶ್ವರೀ ।\\
ಢುಂಢಿರಾಜಸ್ಯ ಜನನೀ ಢಕಾವಾದ್ಯವಿಲಾಸಿನೀ ॥೧೭೩॥

ತರಿಕಾ ತಾರಿಕಾ ತಾರಾ ತನ್ವಂಗೀ ತನುಮಧ್ಯಮಾ ।\\
ಧೂಪೀಕೃತಾಸುರಾ ದೀರ್ಘವೇಣೀ ದೃಪ್ತಾ ಸುರಾರ್ತಿಹೃತ್ ॥೧೭೪॥

ಧೂಮ್ರವರ್ಣಾ ಧೂಮ್ರಕೇಶೀ ಧೂಮ್ರಾಕ್ಷಪ್ರಾಣಹಾರಿಣೀ ।\\
ನಗೇಶತನಯಾ ನಾರೀ ಮತಲ್ಲೀ ಪಟ್ಟಿಹೇತಿಕಾ ॥೧೭೫॥

ಪಾತಾಲಲೋಕಾಧಿಷ್ಠಾತ್ರೀ ಫೇರೂಕೃತಮಹಾಸುರಾ ।\\
ಫಣೀಂದ್ರಶಯನಾ ಬೋಧದಾಯಿನೀ ಬಹುರೂಪಿಣೀ ॥೧೭೬॥

ಭಾಮಿನೀ ಭಾಸಿನೀ ಭ್ರಾಂತಿಕರೀ ಭ್ರಾಂತಿವಿನಾಶಿನೀ ।\\
ಮಾತಂಗೀ ಮದಿರಾಮತ್ತಾ ಮಾಧವೀ ಮಾಧವಪ್ರಿಯಾ ॥೧೭೭॥

ಯಾಯಜೂಕಾರ್ಚಿತಾ ಯೋಗಿಧ್ಯೇಯಾ ಯೋಗೀಶವಲ್ಲಭಾ ।\\
ರಾಕಾಚಂದ್ರಮುಖೀ ರಾಮಾ ರೇಣುಕಾ ರೇಣುಕಾತ್ಮಜಾ ॥೧೭೮॥

ಲೋಕಾಕ್ಷೀ ಲೋಹಿತಾ ಲಜ್ಜಾ ವಾಮಾಕ್ಷೀ ವಾಸ್ತುಶಾಂತಿದಾ ।\\
ಶಾತೋದರೀ ಶಾಶ್ವತಿಕಾ ಶಾತಕುಂಭವಿಭೂಷಣಾ ॥೧೭೯॥

ಷಡಾಸ್ಯಮಾತಾ ಷಟ್ಚಕ್ರವಾಸಿನೀ ಸರ್ವಮಂಗಲಾ ।\\
ಸ್ಮೇರಾನನಾ ಸುಪ್ರಸನ್ನಾ ಹರವಾಮಾಂಕಸಂಸ್ಥಿತಾ ॥೧೮೦॥

ಹಾರಿವಂದಿತಪಾದಾಬ್ಜಾ ಹ್ರೀಂಬೀಜಭುವನೇಶ್ವರೀ ।\\
ಕ್ಷೌಮಾಂಬರೇಂದುಗೋಕ್ಷೀರಧವಲಾ ವನಶಂಕರೀ ॥೧೮೧॥

(॥ಷಡಂಗನ್ಯಾಸಧ್ಯಾನ ಮಾನಸೋಪಚಾರ ಪೂಜನಂ ॥)

ಅಥ ಫಲಶ್ರುತಿಃ ।\\
ಇತೀದಂ ವನಶಂಕರ್ಯಾಃ ಪ್ರೋಕ್ತಂ ನಾಮಸಹಸ್ರಕಂ ।\\
ಶಾಂತಿದಂ ಪುಷ್ಟಿದಂ ಪುಣ್ಯಂ ಮಹಾವಿಪತ್ತಿನಾಶನಂ ॥೧॥

ದಾರಿದ್ರ್ಯದುಃಖಶಮನಂ ನಾನಾರೋಗನಿವಾರಣಂ ।\\
ಇದಂ ಸ್ತೋತ್ರಂ ಪಠೇದ್ಭಕ್ತ್ಯಾ ಯಸ್ತ್ರಿಸಂಧ್ಯಂ ನರಃ ಶುಚಿಃ ॥೨॥

ನಾರೀ ವಾ ನಿಯತಾ ಭಕ್ತ್ಯಾ ಸರ್ವಾನ್ಕಾಮಾನವಾಪ್ನುಯಾತ್ ।\\
ಅಸ್ಯ ಸ್ತೋತ್ರಸ್ಯ ಸತತಂ ಯತ್ರ ಪಾಠಃ ಪ್ರವರ್ತತೇ ॥೩॥

ತಸ್ಮಿನ್ನಗೃಹೇ ಮಹಾಲಕ್ಷ್ಮೀಃ ಸ್ವಭರ್ತ್ರಾ ಸಹ ಮೋದತೇ ।\\
ನಾನೇನ ಸದೃಶಂ ಸ್ತೋತ್ರಂ ಭುಕ್ತಿಮುಕ್ತಿಪ್ರದಾಯಕಂ ॥೪॥

ವಿದ್ಯತೇಽನ್ಯಸತ್ಯಮೇತತ್ ಶಕ್ರ ವಚ್ಮಿ ಪುನಃ ಪುನಃ ।\\
ಬಕವಂಧ್ಯಾ ಕಾಕವಂಧ್ಯಾ ನಾರೀಸ್ತೋತ್ರಮಿದಂ ಪಠೇತ್ ॥೫॥

ವಷೇಣೈಕೇನ ಸಾ ಪುತ್ರಂ ಲಭೇನ್ನಾತ್ರಾಸ್ತಿ ಸಂಶಯಃ ।\\
ಸಾಮಾನ್ಯಕಾರ್ಯಸಿದ್ಧಿಸ್ತು ಶತಾವರ್ತನತೋ ಭವೇತ್ ॥೬॥

ಮಹಾಕಾರ್ಯಸ್ಯ ಸಿದ್ಧಿಸ್ತು ಸಹಸ್ರಾವರ್ತನಾದ್ಧ್ರುವಂ ।\\
ನ ತತ್ರ ಪೀಡಾ ಜಾಯೇತ ಭೂತಪ್ರೇತಗ್ರಹೋದ್ಭವಾ ॥೭॥

ಲೋಕವಶ್ಯಂ ರಾಜವಶ್ಯಂ ಸ್ತೋತ್ರಸ್ಯ ಪಠನಾದ್ಭವೇತ್ ।\\
ಶತ್ರವಃ ಸಂಕ್ಷಯಂ ಯಾಂತಿ ದಸ್ಯವಃ ಪಿಶುನಾಸ್ತಥಾ ॥೮॥

ನೈವ ಶಾಕಂಭರೀಭಕ್ತಾಃ ಸೀದಂತಿ ಬಲಸೂದನ ।\\
ಶಾಕಂಭರೀ ಸ್ವಭಕ್ತಾನಾಮವಿತ್ರೀ ಜನನೀ ಯಥಾ ॥೯॥

ಯದ್ಯತ್ಕಾರ್ಯಂ ಸಮುದ್ದಿಶ್ಯ ಧ್ಯಾಯನ್ ಶಾಕಂಭರೀಂ ಹೃದಿ ।\\
ಸ್ತೋತ್ರಮೇತತ್ಪಠೇತ್ತಸ್ಯ ತತ್ಕಾರ್ಯಂ ಚ ಪ್ರಸಿದ್ಧ್ಯತಿ ॥೧೦॥

ಮುಚ್ಯತೇ ಬಂಧನಾದ್ಬದ್ಧೋ ರೋಗೀ ಮುಚ್ಯೇತ ರೋಗತಃ ।\\
ಋಣವಾನೃಣತೋ ಮುಚ್ಯೇನ್ನಾತ್ರ ಕಾರ್ಯಾ ವಿಚಾರಣಾ ॥೧೧॥

ಪೌಷೇ ಮಾಸಿ ಸಿತೇ ಪಕ್ಷೇ ಪ್ರಾರಭ್ಯ ತಿಥಿಮಷ್ಟಮೀಂ ।\\
ದೇವ್ಯಾಃ ಆರಾಧನಂ ಕುರ್ಯಾದನ್ವಹಂ ಪೂರ್ಣಿಮಾವಧಿ ॥೧೨॥

ಸಶಾಕೈರುತ್ತಮಾನ್ನೈಶ್ಚ ಬ್ರಾಹ್ಮಣಾಂಶ್ಚ ಸುವಾಸಿನೀಃ ।\\
ಸಂತರ್ಪಯೇದ್ಯಥಾಶಕ್ತಿ ದೇವೀಂ ಸಂಪೂಜ್ಯ ಭಕ್ತಿತಃ ॥೧೩॥

ವಿತ್ತಶಾಠ್ಯಂ ನ ಕುರ್ವೀತ ಶಾಕಂಭರ್ಯಾಃ ಸಮರ್ಚನೇ ।\\
ತಸ್ಮೈ ಪ್ರಸನ್ನಾ ಭಕ್ತಾಯ ದದ್ಯಾತ್ಕಾಮಾನಭೀಪ್ಸಿತಾನ್ ॥೧೪॥

ವಿದ್ಯಾರ್ಥೀ ಪ್ರಾಪ್ನುಯಾದ್ವಿದ್ಯಾಂ ಧನಾರ್ಥೀ ಚಾಪ್ನುಯಾದ್ಧನಂ ।\\
ದಾರಾರ್ಥೀ ಪ್ರಾಪ್ನುಯಾದ್ದಾರಾನಪತ್ಯಾರ್ಥೀ ತದಾಪ್ನುಯಾತ್ ॥೧೫॥

ಆಪ್ನುಯಾನ್ಮಂದಬುದ್ಧಿಸ್ತು ಗ್ರಂಥಧಾರಣಪಾಟವಂ ।\\
ಅಸ್ಯ ಸ್ತೋತ್ರಸ್ಯ ಪಾಠೇನ ಪ್ರಾಪ್ನುಯಾದ್ಯದ್ಯದೀಪ್ಸಿತಂ ॥೧೬॥

ಅಯುತಾವರ್ತನಾದಸ್ಯ ಸ್ತೋತ್ರಸ್ಯ ಬಲಸೂದನ ।\\
ಪಶ್ಯೇಚ್ಛಾಕಂಭರೀಂ ಸಾಕ್ಷಾತ್ತದ್ಭಕ್ತೋ ನಾತ್ರ ಸಂಶಯಃ ॥೧೭॥

ಹೋಮಂ ಚ ಕುರ್ಯಾದ್ವಿಧಿವತ್ಪಾಯಸೇನ ಸಸರ್ಪಿಷಾ ।\\
ಸೌಭಾಗ್ಯದ್ರವ್ಯಯುಕ್ತೇನ ಸೇಕ್ಷುಕೂಷ್ಮಾಂಡಕೇನ ಚ ॥೧೮॥

ಸುವಾಸಿನೀಃ ಕುಮಾರೀಶ್ಚ ಬ್ರಾಹ್ಮಣಾಂಶ್ಚ ದಿನೇ ದಿನೇ ।\\
ಸಂಭೋಜಯೇತ್ಸದನ್ನೇನ ಸಸಿತಾಮಧುಸರ್ಪಿಷಾ ॥೧೯॥

ದದ್ಯಾತ್ತೇಭ್ಯಶ್ಚ ತಾಭ್ಯಶ್ಚ ವಸ್ತ್ರಾಲಂಕಾರದಕ್ಷಿಣಾಃ ।\\
ತಸ್ಮೈ ಶಾಕಂಭರೀ ದದ್ಯಾತ್ಪುರುಷಾರ್ಥಚತುಷ್ಟಯಂ ॥೨೦॥

ಆಚಂದ್ರಾರ್ಕಂ ತಸ್ಯ ವಶಃ ಸ್ಥಾಸ್ಯತ್ಯತ್ರ ಗುಣೀ ಸುಖೀ ।\\
ಶಾಕಂಭರ್ಯೈ ನಮ ಇತಿ ಯಸ್ತು ಮಂತ್ರಂ ಷಡಕ್ಷರಂ ॥೨೧॥

ಭಕ್ತ್ಯಾ ಜಪೇನ್ನರಸ್ತಸ್ಯ ಸರ್ವತ್ರ ಜಯಮಂಗಲಂ ।\\
ಶಾಕಂಭರ್ಯಾ ಇದಂ ಶಕ್ರ ದಿವ್ಯಂ ನಾಮಸಹಸ್ರಕಂ ॥೨೨॥

ಗುರುಭಕ್ತಾಯ ಶಾಂತಾಯ ದೇಯಂ ಶ್ರದ್ಧಾಲವೇ ತ್ವಯಾ ।\\
ತ್ವಮಪ್ಯಾಖಂಡಲ ಸದಾ ಪಠೇದಂ ಸ್ತೋತ್ರಮುತ್ತಮಂ ॥೨೩॥

ಶತ್ರೂನ್ ಜೇಷ್ಯಸಿ ಸಂಗ್ರಾಮೇ ಸರ್ವಾನ್ಕಾಮಾನವಾಪ್ಸ್ಯಸಿ ।\\
ಇತಿ ಶ್ರುತ್ವಾ ಸ್ಕಂದವಾಕ್ಯಂ ಶಕ್ರಃ ಸಂತುಷ್ಟಮಾನಸಃ ॥೨೪॥

ಪ್ರಣಮ್ಯ ಗುಹಮಾಪೃಷ್ಟ್ವಾ ಸಗಣಃ ಸ್ವದಿವಂ ಯಯೌ ॥೨೫॥

ಸೂತ ಉವಾಚ ।\\
ದುರ್ಭಿಕ್ಷೇ ಋಷಿಪೋಷಿಣ್ಯಾಃ ಶಾಕಂಭರ್ಯಾಃ ಪ್ರಕೀರ್ತಿತಂ ।\\
ಇದಂ ನಾಮಸಹಸ್ರಂ ವಃ ಕಿಂ ಭೂಯಃ ಶ್ರೋತುಮಿಚ್ಛಥ ॥೨೬॥

\authorline{॥ಇತಿ ಶ್ರೀಸ್ಕಂದಪುರಾಣೇ ಶಾಕಂಭರೀ ತಥಾ ವನಶಂಕರೀ ಸಹಸ್ರನಾಮಸ್ತೋತ್ರಂ ಸಂಪೂರ್ಣಂ॥}
%=============================================================================================
\section{ಶ್ರೀಷೋಡಶೀಸಹಸ್ರನಾಮಸ್ತೋತ್ರಂ}
\addcontentsline{toc}{section}{ಶ್ರೀಷೋಡಶೀಸಹಸ್ರನಾಮಸ್ತೋತ್ರಂ}


॥ಪೂರ್ವ ಪೀಠಿಕಾ॥

ಕೈಲಾಸಶಿಖರೇ ರಮ್ಯೇ ನಾನಾರತ್ನೋಪಶೋಭಿತೇ।\\
ಕಲ್ಪಪಾದಪಮಧ್ಯಸ್ಥೇ ನಾನಾಪುಷ್ಪೋಪಶೋಭಿತೇ ॥೧॥

ಮಣಿಮಂಡಪಮಧ್ಯಸ್ಥೇ ಮುನಿಗಂಧರ್ವಸೇವಿತೇ ।\\
ಕದಾಚಿತ್ಸುಖಮಾಸೀನಂ ಭಗವಂತಂ ಜಗದ್ಗುರುಂ ॥೨॥

ಕಪಾಲಖಟ್ವಾಂಗಧರಂ ಚಂದ್ರಾರ್ಧಕೃತಶೇಖರಂ ।\\
ಹಸ್ತತ್ರಿಶೂಲಡಮರುಂ ಮಹಾವೃಷಭವಹನಂ ॥೩॥

ಜಟಾಜೂಟಧರಂದೇವಂ ಕಂಠಭೂಷಣವಾಸುಕಿಂ ।\\
ವಿಭೂತಿಭೂಷಣಂದೇವಂ ನೀಲಕಂಠಂತ್ರಿಲೋಚನಂ ॥೪॥

ದ್ವೀಪಿಚರ್ಮಪರೀಧಾನಂ ಶುದ್ಧಸ್ಫಟಿಕಸನ್ನಿಭಂ ।\\
ಸಹಸ್ರಾದಿತ್ಯಸಂಕಾಶಂ ಗಿರಿಜಾರ್ದ್ಧಾಂಗಭೂಷಣಂ ॥೫॥

ಪ್ರಣಮ್ಯ ಶಿರಸಾ ನಾಥಂ ಕಾರಣಂ ವಿಶ್ವರೂಪಿಣಂ ।\\
ಕೃತಾಂಜಲಿ ಪುಟೋ ಭೂತ್ವಾ ಪ್ರಾಹೈನಂ ಶಿಖವಾಹನಃ ॥೬॥

॥ಕಾರ್ತಿಕೇಯ ಉವಾಚ॥

ದೇವದೇವ ಜಗನ್ನಾಥ! ಸೃಷ್ಟಿಸ್ಥಿತಿಲಯಾತ್ಮಕ ।\\
ತ್ವಮೇವ ಪರಮಾತ್ಮಾ ಚ ತ್ವಂ ಗತಿಃ ಸರ್ವದೇಹಿನಾಂ ॥೭॥

ತ್ವಂಗತಿಃ ಸರ್ವಲೋಕಾನಾಂ ದೀನಾನಾಂ ಚ ತ್ವಮೇವ ಹಿ ।\\
ತ್ವಮೇವ ಜಗದಾಧಾರಸ್ತ್ವಮೇವ ವಿಶ್ವಕಾರಣಂ ॥೮॥

ತ್ವಮೇವ ಪೂಜ್ಯಃ ಸರ್ವೇಷಾಂ ತ್ವದನ್ಯೋ ನಾಸ್ತಿ ಮೇ ಗತಿಃ ।\\
ಕಿಂ ಗುಹ್ಯಂಪರಮಂ ಲೋಕೇ ಕಿಮೇಕಂ ಸರ್ವಸಿದ್ಧಿದಂ ॥೯॥

ಕಿಮೇಕಂ ಪರಮಂ ಶ್ರೇಷ್ಠಂ ಕೋ ಯೋಗ: ಸ್ವರ್ಗಮೋಕ್ಷದ: ।\\
ವಿನಾ ತೀರ್ಥೇನ ತಪಸಾ ವಿನಾ ದಾನೈರ್ವಿನಾ ಮಖೈ: ॥೧೦॥

ವಿನಾ ಲಯೇನ ಧ್ಯಾನೇನ ನರಃ ಸಿದ್ಧಿಮವಾಪ್ನುಯಾತ್ ।\\
ಕಸ್ಮಾದುತ್ಪದ್ಯತೇ ಸೃಷ್ಟಿ: ಕಸ್ಮಿಂಶ್ಚ ಪ್ರಲಯೋ ಭವೇತ್ ॥೧೧॥

ಕಸ್ಮಾದುತ್ತೀರ್ಯತೇ ದೇವ ! ಸಂಸಾರಾರ್ಣವಸಂಕಟಾತ್ ।\\
ತದಹಂ ಶ್ರೋತುಮಿಚ್ಛಾಮಿ ಕಥಯಸ್ವ ಮಹೇಶ್ವರ ! ॥೧೨॥

॥ಈಶ್ವರ ಉವಾಚ॥

ಸಾಧು ಸಾಧು ತ್ವಯಾ ಪೃಷ್ಟಂ ಪಾರ್ವತೀಪ್ರಿಯನಂದನ ।\\
ಅಸ್ತಿ ಗುಹ್ಯತಮಂಪುತ್ರ! ಕಥಯಿಷ್ಯಾಮ್ಯಸಂಶಯಂ ॥೧೩॥

ಸತ್ವಂ ರಜಸ್ತಮಶ್ಚೈವ ಯೇ ಚಾನ್ಯೇ ಮಹದಾದಯಃ ।\\
ಯೇ ಚಾನ್ಯೇ ಬಹವೋ ಭೂತಾಃ ಸರ್ವೇ ಪ್ರಕೃತಿಸಂಭವಾಃ ॥೧೪॥

ಸೈವ ದೇವೀ ಪರಾಶಕ್ತಿಃ ಮಹಾತ್ರಿಪುರಸುಂದರೀ ।\\
ಸೈವ ಪ್ರಸೂಯತೇ ವಿಶ್ವಂ ವಿಶ್ವಂ ಸೈವ ಪ್ರಪಾಸ್ಯತಿ ॥೧೫॥

ಸೈವ ಸಂಹರತೇ ವಿಶ್ವಂ ಜಗದೇತಚ್ಚರಾಚರಂ ।\\
ಆಧಾರಃ ಸರ್ವಭೂತಾನಾಂ ಸೈವ ರೋಗಾರ್ತಿಹಾರಿಣೀ ॥೧೬॥

ಇಚ್ಛಾಜ್ಞಾನಕ್ರಿಯಾಶಕ್ತಿರಬ್ರಹ್ಮವಿಷ್ಣುಶಿವಾತ್ಮಿಕಾ ।\\
ತ್ರಿಧಾ ಶಕ್ತಿಸ್ವರೂಪೇಣ ಸೃಷ್ಟಿಸ್ಥಿತಿವಿನಾಶಿನೀ ॥೧೭॥

ಸೃಜ್ಯತೇ ಬ್ರಹ್ಮರೂಪೇಣ ವಿಷ್ಣುರೂಪೇಣ ಪಾಲ್ಯತೇ ।\\
ಹ್ರಿಯತೇ ರುದ್ರರೂಪೇಣ ಜಗದೇತಚ್ಚರಾಚರಂ ॥೧೮॥

ಯಸ್ಯಾ ಯೋನೌ ಜಗತ್ಸರ್ವಮದ್ಯಾಪಿ ಪರಿವರ್ತತೇ ।\\
ಯಸ್ಯಾಂ ಪ್ರಲೀಯತೇ ಚಾಂತೇ ಯಸ್ಯಾಂ ಚ ಜಾಯತೇ ಪುನಃ ॥೧೯॥

ಯಾಂ ಸಮಾರಾಧ್ಯ ತ್ರೈಲೋಕ್ಯೇ ಸಂಪ್ರಾಪ್ಯಂ ಪದಮುತ್ತಮಂ ।\\
ತಸ್ಯಾ ನಾಮಸಹಸ್ರಂ ತು ಕಥಯಾಮಿ ಶೃಣುಷ್ವ ತತ್ ॥೨೦॥

॥ವಿನಿಯೋಗಃ॥

ಓಂ ಅಸ್ಯ ಶ್ರೀಮಹಾತ್ರಿಪುರಸುಂದರೀಸಹಸ್ರನಾಮಸ್ತೋತ್ರಮಂತ್ರಸ್ಯ ಶ್ರೀಭಗವಾನ್ ದಕ್ಷಿಣಾಮೂರ್ತಿಃ ಋಷಿಃ । ಜಗತೀಛಂದಃ । ಸಮಸ್ತ ಪ್ರಕಟ ಗುಪ್ತ ಸಂಪ್ರದಾಯ ಕುಲ ಕೌಲೋತ್ತೀರ್ಣ ನಿರ್ಗರ್ಭ ರಹಸ್ಯಾ ಚಿಂತ್ಯಪ್ರಭಾವತೀ ದೇವತಾ । ಓಂ ಬೀಜಂ । ಹ್ರೀಂ ಶಕ್ತಿಃ । ಕ್ಲೀಂ ಕೀಲಕಂ । ಧರ್ಮಾರ್ಥಕಾಮಮೋಕ್ಷಾರ್ಥೇ ಜಪೇ ವಿನಿಯೋಗಃ  ॥

\as{ಓಂ ಆಧಾರೇ ತರುಣಾರ್ಕಬಿಂಬರುಚಿರಂ ಹೇಮಪ್ರಭಂ ವಾಗ್ಭವಂ ।\\
ಬೀಜಂ ಮನ್ಮಥಮಿಂದ್ರಗೋಪಸದೃಶಂ ಹೃತ್ಪಂಕಜೇ ಸಂಸ್ಥಿತಂ॥

ವಿಷ್ಣುಬ್ರಹ್ಮಪದಸ್ಥಶಕ್ತಿಕಲಿತಂ ಸೋಮಪ್ರಭಾಭಾಸುರಂ ।\\
ಯೇ ಧ್ಯಾಯಂತಿ ಪದತ್ರಯಂ ತವ ಶಿವೇ ! ತೇ ಯಾಂತಿ ಸೌಖ್ಯಂ ಪದಂ॥}

॥ಮೂಲಪಾಠಃ॥

ಕಲ್ಯಾಣೀ ಕಮಲಾ ಕಾಲೀ ಕರಾಲೀ ಕಾಮರೂಪಿಣಿ ।\\
ಕಾಮಾಖ್ಯಾ ಕಾಮದಾ ಕಾಮ್ಯಾ ಕಾಮನಾ ಕಾಮಚಾರಿಣೀ ॥೧॥

ಕಾಲರಾತ್ರಿರ್ಮಹಾರಾತ್ರಿಃ ಕಪಾಲೀ ಕಾಮರೂಪಿಣೀ ।\\
ಕೌಮಾರೀ ಕರುಣಾ ಮುಕ್ತಿಃ ಕಲಿಕಲ್ಮಷನಾಶಿನೀ ॥೨॥

ಕಾತ್ಯಾಯನೀ ಕರಾಧಾರಾ ಕೌಮುದೀ ಕಮಲಪ್ರಿಯಾ ।\\
ಕಿರ್ತಿದಾ ಬುದ್ಧಿದಾ ಮೇಧಾ ನೀತಿಜ್ಞಾ ನೀತಿವತ್ಸಲಾ ॥೩॥

ಮಾಹೇಶ್ವರೀ ಮಹಾಮಾಯಾ ಮಹಾತೇಜಾ ಮಹೇಶ್ವರೀ ।\\
ಮಹಾಜಿಹ್ವಾ ಮಹಾಘೋರಾ ಮಹಾದಂಷ್ಟ್ರಾ ಮಹಾಭುಜಾ ॥೪॥

ಮಹಾಮೋಹಾಂಧಕಾರಘ್ನೀ ಮಹಾಮೋಕ್ಷಪ್ರದಾಯಿನೀ ।\\
ಮಹಾದಾರಿದ್ರ್ಯನಾಶಾ ಚ ಮಹಾಶತ್ರುವಿಮರ್ದಿನೀ ॥೫॥

ಮಹಾಮಾಯಾ ಮಹಾವೀರ್ಯಾ ಮಹಾಪಾತಕನಾಶಿನೀ ।\\
ಮಹಾಮಖಾ ಮಂತ್ರಮಯೀ ಮಣಿಪೂರಕವಾಸಿನೀ ॥೬॥

ಮಾನಸೀ ಮಾನದಾ ಮಾನ್ಯಾ ಮನಶ್ಚಕ್ಷೂರಣೇಚರಾ ।\\
ಗಣಮಾತಾ ಚ ಗಾಯತ್ರೀ ಗಣಗಂಧರ್ವಸೇವಿತಾ ॥೭॥

ಗಿರಿಜಾ ಗಿರಿಶಾ ಸಾಧ್ವೀ ಗಿರಿಸ್ಥಾ ಗಿರಿವಲ್ಲಭಾ ।\\
ಚಂಡೇಶ್ವರೀ ಚಂಡರೂಪಾ ಪ್ರಚಂಡಾ ಚಂಡಮಾಲಿನೀ ॥೮॥

ಚರ್ವಿಕಾ ಚರ್ಚಿಕಾಕಾರಾ ಚಂಡಿಕಾ ಚಾರುರೂಪಿಣೀ ।\\
ಯಜ್ಞೇಶ್ವರೀ ಯಜ್ಞರೂಪಾ ಜಪಯಜ್ಞಪರಾಯಣಾ ॥೯॥

ಯಜ್ಞಮಾತಾ ಯಜ್ಞಭೋಕ್ತ್ರೀ ಯಜ್ಞೇಶೀ ಯಜ್ಞಸಂಭವಾ ।\\
ಸಿದ್ಧಯಜ್ಞಾ ಕ್ರಿಯಾಸಿದ್ಧಿರ್ಯಜ್ಞಾಂಗೀ ಯಜ್ಞರಕ್ಷಿಕಾ ॥೧೦॥

ಯಜ್ಞಕ್ರಿಯಾ ಚ ಯಜ್ಞಾ ಚ ಯಜ್ಞಾಯಜ್ಞಕ್ರಿಯಾಲಯಾ ।\\
ಜಾಲಂಧರೀ ಜಗನ್ಮಾತಾ ಜಾತವೇದಾ ಜಗತ್ಪ್ರಿಯಾ ॥೧೧॥

ಜಿತೇಂದ್ರಿಯಾ ಜಿತಕ್ರೋಧಾ ಜನನೀ ಜನ್ಮದಾಯಿನೀ ।\\
ಗಂಗಾ ಗೋದಾವರೀ ಚೈವ ಗೋಮತೀ ಚ ಶತದ್ರುಕಾ ॥೧೨॥

ಘರ್ಘರಾ ವೇದಗರ್ಭಾ ಚ ರೇಚಿಕಾ ಸಮವಾಸಿನೀ ।\\
ಸಿಂಧುರ್ಮಂದಾಕಿನೀ ಕ್ಷಿಪ್ರಾ ಯಮುನಾ ಚ ಸರಸ್ವತೀ ॥೧೩॥

ಭದ್ರಾ ರಾಗಾ ವಿಪಾಶಾ ಚ ಗಂಡಕೀ ವಿಂಧಯವಾಸಿನೀ ।\\
ನರ್ಮದಾ ತಾಪ್ತೀ ಕಾವೇರೀ ವೇತ್ರವತೀ ಸುಕೌಶಿಕೀ ॥೧೪॥

ಮಹೇಂದ್ರತನಯಾ ಚೈವ ಅಹಲ್ಯಾ ಚರ್ಮಕಾವತೀ ।\\
ಅಯೋಧ್ಯಾ ಮಥುರಾ ಮಾಯಾ ಕಾಶೀ ಕಾಂಚೀ ಅವಂತಿಕಾ ॥೧೫॥

ಪುರೀ ದ್ವಾರಾವತೀ ತೀರ್ಥಾ ಮಹಾಕಿಲ್ವಿಷನಾಶಿನೀ ।\\
ಪದ್ಮಿನೀ ಪದ್ಮಮಧ್ಯಸ್ಥಾ ಪದ್ಮಕಿಂಜಲ್ಕವಾಸಿನೀ ॥೧೬॥

ಪದ್ಮವಕ್ತ್ರಾ ಚಕೋರಾಕ್ಷೀ ಪದ್ಮಸ್ಥಾ ಪದ್ಮಸಂಭವಾ ।\\
ಹ್ರೀಂಕಾರೀ ಕುಂಡಲಾಧಾರಾ ಹೃತ್ಪದ್ಮಸ್ಥಾ ಸುಲೋಚನಾ ॥೧೭॥

ಶ್ರೀಂಕಾರೀ ಭೂಷಣಾ ಲಕ್ಷ್ಮೀಃ ಕ್ಲೀಂಕಾರೀ ಕ್ಲೇಶನಾಶಿನೀ ।\\
ಹರಿವಕ್ತ್ರೋದ್ಭವಾ ಶಾಂತಾ ಹರಿವಕ್ತ್ರಕೃತಾಲಯಾ ॥೧೮॥

ಹರಿವಕ್ತ್ರೋಪಮಾ ಹಾಲಾ ಹರಿವಕ್ಷ:ಸ್ಥಲಾಸ್ಥಿತಾ ।\\
ವೈಷ್ಣವೀ ವಿಷ್ಣುರೂಪಾ ಚ ವಿಷ್ಣುಮಾತೃಸ್ವರೂಪಿಣೀ ॥೧೯॥

ವಿಷ್ಣುಮಾಯಾ ವಿಶಾಲಾಕ್ಷೀ ವಿಶಾಲನಯನೋಜ್ಜ್ವಲಾ ।\\
ವಿಂಶ್ವೇಶ್ವರೀ ಚ ವಿಶ್ವಾತ್ಮಾ ವಿಶ್ವೇಶೀ ವಿಶ್ವರೂಪಿಣೀ ॥೨೦॥

ವಿಶ್ವನಾಥಾ ಶಿವಾರಾಧ್ಯಾ ಶಿವನಾಥಾ ಶಿವಪ್ರಿಯಾ ।\\
ಶಿವಮಾತಾ ಶಿವಾಖ್ಯಾ ಚ ಶಿವದಾ ಶಿವರೂಪಿಣೀ ॥೨೧॥

ಭವೇಶ್ವರೀ ಭವಾರಾಧ್ಯಾ ಭವೇಶೀ ಭವನಾಯಿಕಾ ।\\
ಭವಮಾತಾ ಭವಗಮ್ಯಾ ಭವಕಂಟಕನಾಶಿನೀ ॥೨೨॥

ಭವಪ್ರಿಯಾ ಭವಾನಂದಾ ಭವಾನೀ ಭವಮೋಹಿನೀ ।\\
ಗಾಯತ್ರೀ ಚೈವ ಸಾವಿತ್ರೀ ಬ್ರಹ್ಮಾಣೀ ಬ್ರಹ್ಮರೂಪಿಣೀ ॥೨೩॥

ಬ್ರಹ್ಮೇಶೀ ಬ್ರಹ್ಮದಾ ಬ್ರಹ್ಮಾ ಬ್ರಹ್ಮಾಣೀ ಬ್ರಹ್ಮವಾದಿನೀ ।\\
ದುರ್ಗಸ್ಥಾ ದುರ್ಗರೂಪಾ ಚ ದುರ್ಗಾ ದುರ್ಗಾರ್ತಿನಾಶಿನೀ ॥೨೪॥

ಸುಗಮಾ ದುರ್ಗಮಾ ದಾಂತಾ ದಯಾ ದೋಗ್ಧ್ರೀ ದುರಾಪಹಾ ।\\
ದುರಿತಘ್ನೀ ದುರಾಧ್ಯಕ್ಷಾ ದುರಾ ದುಷ್ಕೃತನಾಶಿನೀ ॥೨೫॥

ಪಂಚಾಸ್ಯಾ ಪಂಚಮೀ ಪೂರ್ಣಾ ಪೂರ್ಣಪೀಠನಿವಾಸಿನೀ ।\\
ಸತ್ತ್ವಸ್ಥಾ ಸತ್ತ್ವರೂಪಾ ಚ ಸತ್ತ್ವಗಾ ಸತ್ತ್ವಸಂಭವಾ ॥೨೬॥

ರಜಸ್ಥಾ ಚ ರಜೋರೂಪಾ ರಜೋಗುಣಸಮುದ್ಭವಾ ।\\
ತಮಸ್ಥಾ ಚ ತಮೋರೂಪಾ ತಾಮಸೀ ತಾಮಸಪ್ರಿಯಾ ॥೨೭॥

ತಮೋಗುಣಸಮುದ್ಭೂತಾ ಸಾತ್ವಿಕೀ ರಾಜಸೀ ಕಲಾ ।\\
ಕಾಷ್ಠಾ ಮುಹೂರ್ತಾ ನಿಮಿಷಾ ಅನಿಮೇಷಾ ತತಃ ಪರಂ ॥೨೮॥

ಅರ್ಧಮಾಸಾ ಚ ಮಾಸಾ ಚ ಸಂವತ್ಸರಸ್ವರೂಪಿಣೀ ।\\
ಯೋಗಸ್ಥಾ ಯೋಗರೂಪಾ ಚ ಕಲ್ಪಸ್ಥಾ ಕಲ್ಪರೂಪಿಣೀ ॥೨೯॥

ನಾನಾರತ್ನವಿಚಿತ್ರಾಂಗೀ ನಾನಾಽಽಭರಣಮಂಡಿತಾ ।\\
ವಿಶ್ವಾತ್ಮಿಕಾ ವಿಶ್ವಮಾತಾ ವಿಶ್ವಪಾಶವಿನಾಶಿನೀ ॥೩೦॥

ವಿಶ್ವಾಸಕಾರಿಣೀ ವಿಶ್ವಾ ವಿಶ್ವಶಕ್ತಿವಿಚಾರಣಾ ।\\
ಜಪಾಕುಸುಮಸಂಕಾಶಾ ದಾಡಿಮೀಕುಸುಮೋಪಮಾ ॥೩೧॥

ಚತುರಂಗೀ ಚತುರ್ಬಾಹುಶ್ಚತುರಾಚಾರವಾಸಿನೀ ।\\
ಸರ್ವೇಶೀ ಸರ್ವದಾ ಸರ್ವಾ ಸರ್ವದಾಸರ್ವದಾಯಿನೀ ॥೩೨॥

ಮಾಹೇಶ್ವರೀ ಚ ಸರ್ವಾದ್ಯಾ ಶರ್ವಾಣೀ ಸರ್ವಮಂಗಲಾ ।\\
ನಲಿನೀ ನಂದಿನೀ ನಂದಾ ಆನಂದಾನಂದವರ್ದ್ಧಿನೀ ॥೩೩॥

ವ್ಯಾಪಿನೀ ಸರ್ವಭುತೇಷು ಭವಭಾರವಿನಾಶಿನೀ ।\\
ಸರ್ವಶೃಂಗಾರವೇಷಾಢ್ಯಾ ಪಾಶಾಂಕುಶಕರೋದ್ಯತಾ ॥೩೪॥

ಸೂರ್ಯಕೋಟಿಸಹಸ್ರಾಭಾ ಚಂದ್ರಕೋಟಿನಿಭಾನನಾ ।\\
ಗಣೇಶಕೋಟಿಲಾವಣ್ಯಾ ವಿಷ್ಣುಕೋಟ್ಯರಿಮರ್ದಿನೀ ॥೩೫॥

ದಾವಾಗ್ನಿಕೋಟಿದಲಿನೀ ರುದ್ರಕೋಟ್ಯುಗ್ರರೂಪಿಣೀ ।\\
ಸಮುದ್ರಕೋಟಿಗಂಭೀರಾ ವಾಯುಕೋಟಿಮಹಾಬಲಾ ॥೩೬॥

ಆಕಾಶಕೋಟಿವಿಸ್ತಾರಾ ಯಮಕೋಟಿಭಯಂಕರೀ ।\\
ಮೇರುಕೋಟಿಸಮುಛ್ರಾಯಾ ಗಣಕೋಟಿಸಮೃದ್ಧಿದಾ ॥೩೭॥

ನಿಷ್ಕಸ್ತೋಕಾ ನಿರಾಧರಾ ನಿರ್ಗುಣಾ ಗುಣವರ್ಜಿತಾ ।\\
ಅಶೋಕಾ ಶೋಕರಹಿತಾ ತಾಪತ್ರಯವಿವರ್ಜಿತಾ ॥೩೮॥

ವಸಿಷ್ಠಾ ವಿಶ್ವಜನನೀ ವಿಶ್ವಾಖ್ಯಾ ವಿಶ್ವವರ್ದ್ಧಿನೀ ।\\
ಚಿತ್ರಾ ವಿಚಿತ್ರಾ ಚಿತ್ರಾಂಗೀ ಹೇತುಗರ್ಭಾಕುಲೇಶ್ವರೀ ॥೩೯॥

ಇಚ್ಛಾಶಕ್ತಿಃ ಜ್ಞಾನಶಕ್ತಿಃ ಕ್ರಿಯಾಶಕ್ತಿಃ ಶುಚಿಸ್ಮಿತಾ ।\\
ಶುಚಿಃ ಸ್ಮೃತಿಮಯೀ ಸತ್ಯಾ ಶ್ರುತಿರೂಪಾ ಶ್ರುತಿಪ್ರಿಯಾ ॥೪೦॥

ಮಹಾಸತ್ವಮಯೀ ಸತ್ವಾ ಪಂಚತತ್ತ್ವೋಪರಿಸ್ಥಿತಾ ।\\
ಪಾರ್ವತೀ ಹಿಮವತ್ಪುತ್ರೀ ಪಾರಸ್ಥಾ ಪಾರರೂಪಿಣೀ ॥೪೧॥

ಜಯಂತೀ ಭದ್ರಕಾಲೀ ಚ ಅಹಲ್ಯಾ ಕುಲನಾಯಿಕಾ ।\\
ಭೂತಧಾತ್ರೀ ಚ ಭೂತೇಶೀ ಭೂತಸ್ಥಾ ಭೂತಭಾವಿನೀ ॥೪೨॥

ಮಹಾಕುಂಡಲಿನೀಶಕ್ತಿರ್ಮಹಾವಿಭವರ್ದ್ಧಿನೀ ।\\
ಹಂಸಾಕ್ಷೀ ಹಂಸರೂಪಾ ಚ ಹಂಸಸ್ಥಾ ಹಂಸರೂಪಿಣೀ ॥೪೩॥

ಸೋಮಸೂರ್ಯಾಗ್ನಿಮಧ್ಯಸ್ಥಾ ಮಣಿಮಂಡಲವಾಸಿನೀ ।\\
ದ್ವಾದಶಾರಸರೋಜಸ್ಥಾ ಸೂರ್ಯಮಂಡಲವಾಸಿನೀ ॥೪೪॥

ಅಕಲಂಕಾ ಶಶಾಂಕಾಭಾ ಷೋಡಶಾರನಿವಾಸಿನೀ ।\\
ಡಾಕಿನೀ ರಾಕಿನೀ ಚೈವ ಲಾಕಿನೀ ಕಾಕಿನೀ ತಥಾ ॥೪೫॥

ಶಾಕಿನೀ ಹಾಕಿನೀ ಚೈವ ಷಟ್ ಚಕ್ರೇಷು ನಿವಾಸಿನೀ ।\\
ಸೃಷ್ಟಿ ಸ್ಥಿತಿವಿನಾಶಿನೀ ಸೃಷ್ಟ್ಯಂತಾ ಸೃಷ್ಟಿಕಾರಿಣೀ ॥೪೬॥

ಶ್ರೀಕಂಠಪ್ರಿಯಾ ಹೃತಕಂಠಾ ನಂದಾಖ್ಯಾ ವಿಂದುಮಾಲಿನೀ ।\\
ಚತುಷ್ಷಷ್ಟಿ ಕಲಾಧಾರಾ ದೇಹದಂಡಸಮಾಶ್ರಿತಾ ॥೪೭॥

ಮಾಯಾ ಕಾಲೀ ಧೃತಿರ್ಮೇಧಾ ಕ್ಷುಧಾ ತುಷ್ಟಿರ್ಮಹಾದ್ಯುತಿಃ ।\\
ಹಿಂಗುಲಾ ಮಂಗಲಾ ಸೀತಾ ಸುಷುಮ್ನಾಮಧ್ಯಗಾಮಿನೀ ॥೪೮॥

ಪರಘೋರಾ ಕರಾಲಾಕ್ಷೀ ವಿಜಯಾ ಜಯದಾಯಿನೀ ।\\
ಹೃತಪದ್ಮನಿಲಯಾ ಭೀಮಾ ಮಹಾಭೈರವನಾದಿನೀ ॥೪೯॥

ಆಕಾಶಲಿಂಗಸಂಭೂತಾ ಭುವನೋದ್ಯಾನವಾಸಿನೀ ।\\
ಮಹತ್ಸೂಕ್ಷ್ಮಾ ಚ ಕಂಕಾಲೀ ಭೀಮರೂಪಾ ಮಹಾಬಲಾ ॥೫೦॥

ಮೇನಕಾಗರ್ಭಸಂಭೂತಾ ತಪ್ತಕಾಂಚನಸನ್ನಿಭಾ ।\\
ಅಂತರಸ್ಥಾ ಚ ಕೂಟಬೀಜಾ ಚಿತ್ರಕೂಟಾಚಲವಾಸಿನೀ ॥೫೧॥

ವರ್ಣಾಖ್ಯಾ ವರ್ಣರಹಿತಾ ಪಂಚಾಶದ್ವರ್ಣಭೇದಿನೀ ।\\
ವಿದ್ಯಾಧರೀ ಲೋಕಧಾತ್ರೀ ಅಪ್ಸರಾ ಅಪ್ಸರಃ ಪ್ರಿಯಾ ॥೫೨॥

ದೀಕ್ಷಾ ದಾಕ್ಷಾಯಣೀ ದಕ್ಷಾ ದಕ್ಷಯಜ್ಞವಿನಾಶಿನೀ ।\\
ಯಶಃಪೂರ್ಣಾ ಯಶೋದಾ ಚ ಯಶೋದಾಗರ್ಭಸಂಭವಾ ॥೫೩॥

ದೇವಕೀ ದೇವಮಾತಾ ಚ ರಾಧಿಕಾ ಕೃಷ್ಣವಲ್ಲಭಾ ।\\
ಅರುಂಧತೀ ಶಚೀಂದ್ರಾಣೀ ಗಾಂಧಾರೀ ಗಂಧಮಾಲಿನೀ ॥೫೪॥

ಧ್ಯಾನಾತೀತಾ ಧ್ಯಾನಗಮ್ಯಾ ಧ್ಯಾನಜ್ಞಾ ಧ್ಯಾನಧಾರಿಣೀ ।\\
ಲಂಬೋದರೀ ಚ ಲಂಬೋಷ್ಠೀ ಜಾಂಬವಂತೀ ಜಲೋದರೀ ॥೫೫॥

ಮಹೋದರೀ ಮುಕ್ತಕೇಶೀ ಮುಕ್ತಕಾಮಾರ್ಥಸಿದ್ಧಿದಾ ।\\
ತಪಸ್ವಿನೀ ತಪೋನಿಷ್ಠಾ ಸುಪರ್ಣಾ ಧರ್ಮವಾಸಿನೀ ॥೫೬॥

ಬಾಣಚಾಪಧರಾ ಧೀರಾ ಪಾಂಚಾಲೀ ಪಂಚಮಪ್ರಿಯಾ ।\\
ಗುಹ್ಯಾಂಗೀ ಚ ಸುಭೀಮಾಂಗೀ ಗುಹ್ಯತತ್ತ್ವಾ ನಿರಂಜನಾ ॥೫೭॥

ಅಶರೀರಾ ಶರೀರಸ್ಥಾ ಸಂಸಾರಾರ್ಣವತಾರಿಣೀ ।\\
ಅಮೃತಾ ನಿಷ್ಕಲಾ ಭದ್ರಾ ಸಕಲಾ ಕೃಷ್ಣಪಿಂಗಲಾ ॥೫೮॥

ಚಕ್ರಪ್ರಿಯಾ ಚ ಚಕ್ರಾಹ್ವಾ ಪಂಚಚಕ್ರಾದಿದಿರಿಣೀ ।\\
ಪದ್ಮರಾಗಪ್ರತೀಕಾಶಾ ನಿರ್ಮಲಾಕಾಶ ಸನ್ನಿಭಾ ॥೫೯॥

ಅಧಃಸ್ಥಾ ಊರ್ಧ್ವರೂಪಾ ಚ ಊರಧ್ವಪದ್ಮನಿವಾಸಿನೀ ।\\
ಕಾರ್ಯಕಾರಣಕರ್ತೃತ್ವೇ ಶಶ್ವದ್ರೂಪೇಷುಸಂಸ್ಥಿತಾ ॥೬೦॥

ರಸಜ್ಞಾ ರಸಮಧ್ಯಸ್ಥಾ ಗಂಧಸ್ಥಾ ಗಂಧರೂಪಿಣೀ ।\\
ಪರಬ್ರಹ್ಮಸ್ವರೂಪಾ ಚ ಪರಬ್ರಹ್ಮನಿವಾಸಿನೀ ॥೬೧॥

ಶಬ್ದಬ್ರಹ್ಮಸ್ವರೂಪಾ ಚ ಶಬ್ದಸ್ಥಾ ಶಬ್ದವರ್ಜಿತಾ ।\\
ಸಿದ್ಧಿರ್ಬುದ್ಧಿರ್ಪರಾಬುದ್ಧಿಃ ಸಂದೀಪ್ತಿರ್ಮಧ್ಯಸಂಸ್ಥಿತಾ ॥೬೨॥

ಸ್ವಗುಹ್ಯಾ ಶಾಂಭವೀಶಕ್ತಿಃ ತತ್ತ್ವಸ್ಥಾ ತತ್ತ್ವರೂಪಿಣೀ ।\\
ಶಾಶ್ವತೀ ಭೂತಮಾತಾ ಚ ಮಹಾಭೂತಾಧಿಪಪ್ರಿಯಾ ॥೬೩॥

ಶುಚಿಪ್ರೇತಾ ಧರ್ಮಸಿದ್ಧಿಃ ಧರ್ಮವೃದ್ಧಿಃ ಪರಾಜಿತಾ ।\\
ಕಾಮಸಂದೀಪನೀ ಕಾಮಾ ಸದಾಕೌತೂಹಲಪ್ರಿಯಾ ॥೬೪॥

ಜಟಾಜೂಟಧರಾ ಮುಕ್ತಾ ಸೂಕ್ಷ್ಮಾ ಶಕ್ತಿವಿಭೂಷಣಾ ।\\
ದ್ವೀಪಿಚರ್ಮಪರೀಧಾನಾ ಚೀರವಲ್ಕಲಧಾರಿಣೀ ॥೬೫॥

ತ್ರಿಶೂಲಡಮರೂಧರಾ ನರಮಾಲಾವಿಭೂಷಣಾ ।\\
ಅತ್ಯುಗ್ರರೂಪಿಣೀ ಚೋಗ್ರಾ ಕಲ್ಪಾಂತದಹನೋಪಮಾ ॥೬೬॥

ತ್ರೈಲೋಕ್ಯಸಾಧಿನೀ ಸಾಧ್ಯಾ ಸಿದ್ಧಿಸಾಧಕವತ್ಸಲಾ ।\\
ಸರ್ವವಿದ್ಯಾಮಯೀ ಸಾರಾ ಚಾಸುರಾಣಾಂ ವಿನಾಶಿನೀ ॥೬೭॥

ದಮನೀ ದಾಮಿನೀ ದಾಂತಾ ದಯಾ ದೋಗ್ಘ್ರೀ ದುರಾಪಹಾ ।\\
ಅಗ್ನಿಜಿಹ್ವೋಪಮಾ ಘೋರಾಘೋರ ಘೋರ ತರಾನನಾ॥೬೮॥

ನಾರಾಯಣೀ ನಾರಸಿಂಹೀ ನೃಸಿಂಹಹೃದಯೇಸ್ಥಿತಾ ।\\
ಯೋಗೇಶ್ವರೀ ಯೋಗರೂಪಾ ಯೋಗಮಾತಾ ಚ ಯೋಗಿನೀ ॥೬೯॥

ಖೇಚರೀ ಖಚರೀ ಖೇಲಾ ನಿರ್ವಾಣಪದಸಂಶ್ರಯಾ ।\\
ನಾಗಿನೀ ನಾಗಕನ್ಯಾ ಚ ಸುವೇಶಾ ನಾಗನಾಯಿಕಾ ॥೭೦॥

ವಿಷಜ್ವಾಲಾವತೀ ದೀಪ್ತಾ ಕಲಾಶತವಿಭೂಷಣಾ ।\\
ತೀವ್ರವಕ್ತ್ರಾ ಮಹಾವಕ್ತ್ರಾ ನಾಗಕೋಟಿತ್ವಧಾರಿಣೀ (೫೦೦)॥೭೧॥

ಮಹಾಸತ್ವಾ ಚ ಧರ್ಮಜ್ಞಾ ಧರ್ಮಾತಿಸುಖದಾಯಿನೀ ।\\
ಕೃಷ್ಣಮೂರ್ದ್ಧಾ ಮಹಾಮೂರ್ದ್ಧಾ ಘೋರಮೂರ್ದ್ಧಾ ವರಾನನಾ ॥೭೨॥

ಸರ್ವೇಂದ್ರಿಯಮನೋನ್ಮತ್ತಾ ಸರ್ವೇಂದ್ರಿಯಮನೋಮಯೀ ।\\
ಸರ್ವಸಂಗ್ರಾಮಜಯದಾ ಸರ್ವಪ್ರಹರಣೋದ್ಯತಾ ॥೭೩॥

ಸರ್ವಪೀಡೋಪಶಮನೀ ಸರ್ವಾರಿಷ್ಟನಿವಾರಿಣೀ ।\\
ಸರ್ವೈಶ್ವರ್ಯಸಮುತ್ಪನ್ನಾ ಸರ್ವಗ್ರಹವಿನಾಶಿನೀ ॥೭೪॥

ಮಾತಂಗೀ ಮತ್ತಮಾತಂಗೀ ಮಾತಂಗೀಪ್ರಿಯಮಂಡಲಾ ।\\
ಅಮೃತೋದಧಿಮಧ್ಯಸ್ಥಾ ಕಟಿಸೂತ್ರೈರಲಂಕೃತಾ ॥೭೫॥

ಅಮೃತೋದಧಿಮಧ್ಯಸ್ಥಾ ಪ್ರವಾಲವಸನಾಂಬುಜಾ ।\\
ಮಣಿಮಂಡಲಮಧ್ಯಸ್ಥಾ ಈಷತ್ಪ್ರಹಸಿತಾನನಾ ॥೭೬॥

ಕುಮುದಾ ಲಲಿತಾ ಲೋಲಾ ಲಾಕ್ಷಾಲೋಹಿತಲೋಚನಾ ।\\
ದಿಗ್ವಾಸಾ ದೇವದೂತೀ ಚ ದೇವದೇವಾಧಿದೇವತಾ ॥೭೭॥

ಸಿಂಹೋಪರಿಸಮಾರೂಢಾ ಹಿಮಾಚಲನಿವಾಸಿನೀ ।\\
ಅಟ್ಟಾಟ್ಟಹಾಸಿನೀ ಘೋರಾ ಘೋರದೈತ್ಯವಿನಾಶಿನೀ ॥೭೮॥

ಅತ್ಯಗ್ರರಕ್ತವಸ್ತ್ರಾಭಾ ನಾಗಕೇಯೂರಮಂಡಿತಾ ।\\
ಮುಕ್ತಾಹಾರಲತೋಪೇತಾ ತುಂಗಪೀನಪಯೋಧರಾ ॥೭೯॥

ರಕ್ತೋತ್ಪಲದಲಾಕಾರಾ ಮದಾಘೂರ್ಣಿತಲೋಚನಾ ।\\
ಸಮಸ್ತದೇವತಾಮೂರ್ತಿಃ ಸುರಾರಿಕ್ಷಯಕಾರಿಣೀ ॥೮೦॥

ಖಡ್ಗಿನೀ ಶೂಲಹಸ್ತಾ ಚ ಚಕ್ರಿಣೀ ಚಕ್ರಮಾಲಿನೀ ।\\
ಶಂಖಿನೀ ಚಾಪಿನೀ ಬಾಣಾ ವಜ್ರಣೀ ವಜ್ರದಂಡಿನೀ ॥೮೧॥

ಆನ್ನದೋದಧತಿಮಧ್ಯಸ್ಥಾ ಕಟಿಸೂತ್ರಧಾರಾಪರಾ ।\\
ನಾನಾಭರಣದೀಪ್ತಾಂಗಾ ನಾನಮಣಿವಿಭೂಷಿತಾ ॥೮೨॥

ಜಗದಾನಂದಸಂಭೂತಾ ಚಿಂತಾಮಣಿಗುಣಾನ್ವಿತಾ ।\\
ತ್ರೈಲೋಕ್ಯನಮಿತಾ ತುರ್ಯಾ ಚಿನ್ಮಯಾನಂದರೂಪಿಣೀ ॥೮೩॥

ತ್ರೈಲೋಕ್ಯನಂದಿನೀದೇವೀ ದುಃಖ ದುಃಸ್ವಪ್ನನಾಶಿನೀ ।\\
ಘೋರಾಗ್ನಿದಾಹಶಮನೀ ರಾಜ್ಯದೇವಾರ್ಥಸಾಧಿನೀ ॥೮೪॥

ಮಹಾಽಪರಾಧರಾಶಿಘ್ನೀ ಮಹಾಚೌರಭಯಾಪಹಾ ।\\
ರಾಗಾದಿ ದೋಷರಹಿತಾ ಜರಾಮರಣವರ್ಜಿತಾ ॥೮೫॥

ಚಂದ್ರಮಂಡಲಮಧ್ಯಸ್ಥಾ ಪೀಯೂಷಾರ್ಣವಸಂಭವಾ ।\\
ಸರ್ವದೇವೈಃಸ್ತುತಾದೇವೀ ಸರ್ವಸಿದ್ಧೈರ್ನಮಸ್ಕೃತಾ ॥೮೬॥

ಅಚಿಂತ್ಯಶಕ್ತಿರೂಪಾ ಚ ಮಣಿಮಂತ್ರಮಹೌಷಧಿ ।\\
ಅಸ್ತಿಸ್ವಸ್ತಿಮಯೀಬಾಲಾ ಮಲಯಾಚಲವಾಸಿನೀ ॥೮೭॥

ಧಾತ್ರೀ ವಿಧಾತ್ರೀ ಸಂಹಾರೀ ರತಿಜ್ಞಾ ರತಿದಾಯಿನೀ ।\\
ರುದ್ರಾಣೀ ರುದ್ರರೂಪಾ ಚ ರುದ್ರರೌದ್ರಾರ್ತಿನಾಶಿನೀ ॥೮೮॥

ಸರ್ವಜ್ಞಾಚೈವಧರ್ಮಜ್ಞಾ ರಸಜ್ಞಾ ದೀನವತ್ಸಲಾ ।\\
ಅನಾಹತಾ ತ್ರಿನಯನಾ ನಿರ್ಭಾರಾ ನಿರ್ವೃತಿಃಪರಾ ॥೮೯॥

ಪರಾಽಘೋರಾ ಕರಾಲಾಕ್ಷೀ ಸುಮತೀ ಶ್ರೇಷ್ಠದಾಯಿನೀ ।\\
ಮಂತ್ರಾಲಿಕಾ ಮಂತ್ರಗಮ್ಯಾ ಮಂತ್ರಮಾಲಾ ಸುಮಂತ್ರಿಣೀ ॥೯೦॥

ಶ್ರದ್ಧಾನಂದಾ ಮಹಾಭದ್ರಾ ನಿರ್ದ್ವಂದ್ವಾ ನಿರ್ಗುಣಾತ್ಮಿಕಾ ।\\
ಧರಿಣೀ ಧಾರಿಣೀ ಪೃಥ್ವೀ ಧರಾ ಧಾತ್ರೀ ವಸುಂಧರಾ ॥೯೧॥

ಮೇರೂಮಂದರಮಧ್ಯಸ್ಥಾ ಸ್ಥಿತಿಃ ಶಂಕರವಲ್ಲಭಾ ।\\
ಶ್ರೀಮತೀ ಶ್ರೀಮಯೀ ಶ್ರೇಷ್ಠಾ ಶ್ರೀಕರೀ ಭಾವಭಾವಿನೀ ॥೯೨॥

ಶ್ರೀದಾ ಶ್ರೀಮಾ ಶ್ರೀನಿವಾಸಾ ಶ್ರೀವತೀ ಶ್ರೀಮತಾಂಗತಿಃ ।\\
ಉಮಾ ಸಾರಂಗಿಣೀ ಕೃಷ್ಣಾ ಕುಟಿಲಾ ಕುಟಿಲಾಲಿಕಾ ॥೯೩॥

ತ್ರಿಲೋಚನಾ ತ್ರಿಲೋಕಾತ್ಮಾ ಪುಣ್ಯಾಪುಣ್ಯಪ್ರಕೀರ್ತಿತಾ  ।\\
ಅಮೃತಾ ಸತ್ಯಸಂಕಲ್ಪಾ ಸಾ ಸತ್ಯಾ ಗ್ರಂಥಿಭೇದಿನೀ ॥೯೪॥

ಪರೇಶೀ ಪರಮಾಸಾಧ್ಯಾ ಪರಾವಿದ್ಯಾ ಪರಾತ್ಪರಾ ।\\
ಸುಂದರಾಂಗೀ ಸುವರ್ಣಾಭಾ ಸುರಾಸುರನಮಸ್ಕೃತಾ ॥೯೫॥

ಪ್ರಜಾ ಪ್ರಜಾವತೀ ಧಾನ್ಯಾ ಧನಧಾನ್ಯಸಮೃದ್ಧಿದಾ ।\\
ಈಶಾನೀ ಭುವನೇಶಾನೀ ಭವಾನೀ ಭುವನೇಶ್ವರೀ ॥೯೬॥

ಅನಂತಾನಂತಮಹಿತಾ ಜಗತ್ಸಾರಾ ಜಗದ್ಭವಾ ।\\
ಅಚಿಂತ್ಯಾತ್ಮಾಚಿಂತ್ಯಶಕ್ತಿಃ ಚಿಂತ್ಯಾಚಿಂತ್ಯಸ್ವರೂಪಿಣೀ ॥೯೭॥

ಜ್ಞಾನಗಮ್ಯಾ ಜ್ಞಾನಮೂರ್ತಿಃ ಜ್ಞಾನಿನೀ ಜ್ಞಾನಶಾಲಿನೀ ।\\
ಅಸಿತಾ ಘೋರರೂಪಾ ಚ ಸುಧಾಧಾರಾ ಸುಧಾವಹಾ ॥೯೮॥

ಭಾಸ್ಕರೀ ಭಾಸ್ವರೀ ಭೀತಿರ್ಭಾಸ್ವದಕ್ಷಾನುಶಾಯಿನೀ ।\\
ಅನಸೂಯಾ ಕ್ಷಮಾ ಲಜ್ಜಾ ದುರ್ಲಭಾಭರಣಾತ್ಮಿಕಾ ॥೯೯॥

ವಿಶ್ವಧ್ನೀ ವಿಶ್ವವೀರಾ ವ ವಿಶ್ವಾಶಾ ವಿಶ್ವಸಂಸ್ಥಿತಾ ।\\
ಶೀಲಸ್ಥಾ ಶೀಲರೂಪಾ ಚ ಶೀಲಾ ಶೀಲಪ್ರದಾಯಿನೀ ॥೧೦೦॥

ಬೋಧಿನೀ ಬೋಧಕುಶಲಾ ರೋಧಿನೀಬೋಧಿನೀ ತಥಾ ।\\
ವಿದ್ಯೋತಿನೀ ವಿಚಿತ್ರಾತ್ಮಾ ವಿದ್ಯುತ್ಪಟಲಸನ್ನಿಭಾ ॥೧೦೧॥

ವಿಶ್ವಯೋನಿರ್ಮಹಾಯೋನಿಃ ಕರ್ಮಯೋನಿಃ ಪ್ರಿಯಾತ್ಮಿಕಾ ।\\

ರೋಹಿಣೀ ರೋಗಶಮನೀ ಮಹಾರೋಗಜ್ವರಾಪಹಾ ॥೧೦೨॥

ರಸದಾ ಪುಷ್ಟಿದಾ ಪುಷ್ಟಿರ್ಮಾನದಾ ಮಾನವಪ್ರಿಯಾ ।\\
ಕೃಷ್ಣಾಂಗವಾಹಿನೀ ಕೃಷ್ಣಾಽಕಲಾ ಕೃಷ್ಣಸಹೋದರಾ ॥೧೦೩॥

ಶಾಂಭವೀ ಶಂಭುರೂಪಾ ಚ ಶಂಭುಸ್ಥಾ ಶಂಭುಸಂಭವಾ ।\\
ವಿಶ್ವೋದರೀ ಯೋಗಮಾತಾ ಯೋಗಮುದ್ರಾ ಸುಯೋಗಿನೀ ॥೧೦೪॥

ವಾಗೀಶ್ವರೀ ಯೋಗನಿದ್ರಾ ಯೋಗಿನೀಕೋಟಿಸೇವಿತಾ ।\\
ಕೌಲಿಕಾ ನಂದಕನ್ಯಾ ಚ ಶೃಂಗಾರಪೀಠವಾಸಿನೀ ॥೧೦೫॥

ಕ್ಷೇಮಂಕರೀ ಸರ್ವರೂಪಾ ದಿವ್ಯರೂಪಾ ದಿಗಂಬರೀ ।\\
ಧೂಮ್ರವಕ್ತ್ರಾ ಧೂಮ್ರನೇತ್ರಾ ಧೂಮ್ರಕೇಶೀ ಚ ಧೂಸರಾ ॥೧೦೬॥

ಪಿನಾಕೀ ರುದ್ರವೇತಾಲೀ ಮಹಾವೇತಾಲರೂಪಿಣೀ ।\\
ತಪಿನೀ ತಾಪಿನೀ ದೀಕ್ಷಾ ವಿಷ್ಣುವಿದ್ಯಾತ್ಮನಾಶ್ರಿತಾ ॥೧೦೭॥

ಮಂಥರಾ ಜಠರಾ ತೀವ್ರಾಽಗ್ನಿಜಿಹ್ವಾ ಚ ಭಯಾಪಹಾ ।\\
ಪಶುಘ್ನೀ ಪಶುಪಾಲಾ ಚ ಪಶುಹಾ ಪಶುವಾಹಿನೀ ॥೧೦೮॥

ಪಿತಾಮಾತಾ ಚ ಧೀರಾ ಚ ಪಶುಪಾಶವಿನಾಶಿನೀ ।\\
ಚಂದ್ರಪ್ರಭಾ ಚಂದ್ರರೇಖಾ ಚಂದ್ರಕಾಂತಿವಿಭೂಷಿಣೀ ॥೧೦೯॥

ಕುಂಕಮಾಂಕಿತಸರ್ವಾಂಗೀ ಸುಧಾಸದ್ಗುರುಲೋಚನಾ ।\\
ಶುಕ್ಲಾಂಬರಧರಾದೇವೀ ವೀಣಾಪುಸ್ತಕಧಾರಿಣೀ ॥೧೧೦॥

ಐರಾವತಪದ್ಮಧರಾ ಶ್ವೇತಪದ್ಮಾಸನಸ್ಥಿತಾ ।\\
ರಕ್ತಾಂಬರಧರಾದೇವೀ ರಕ್ತಪದ್ಮವಿಲೋಚನಾ ॥೧೧೧॥

ದುಸ್ತರಾ ತಾರಿಣೀ ತಾರಾ ತರುಣೀ ತಾರರೂಪಿಣೀ ।\\
ಸುಧಾಧಾರಾ ಚ ಧರ್ಮಜ್ಞಾ ಧರ್ಮಸಂಘೋಪದೇಶಿನೀ ॥೧೧೨॥

ಭಗೇಶ್ವರೀ ಭಗಾರಾಧ್ಯಾ ಭಗಿನೀ ಭಗನಾಯಿಕಾ ।\\
ಭಗಬಿಂಬಾ ಭಗಕ್ಲಿನ್ನಾ ಭಗಯೋನಿರ್ಭಗಪ್ರದಾ ॥೧೧೩॥

ಭಗೇಶೀ ಭಗರೂಪಾ ಚ ಭಗಗುಹ್ಯಾ ಭಗಾವಹಾ ।\\
ಭಗೋದರೀ ಭಗಾನಂದಾ ಭಗಸ್ಥಾ ಭಗಶಾಲಿನೀ ॥೧೧೪॥

ಸರ್ವಸಂಕ್ಷೋಭಿಣೀಶಕ್ತಿಃ ಸರ್ವವಿದ್ರಾವಿಣೀ ತಥಾ ।\\
ಮಾಲಿನೀ ಮಾಧವೀ ಮಾಧ್ವೀ ಮಧುರೂಪಾ ಮಹೋತ್ಕಟಾ ॥೧೧೫॥

ಭೇರುಂಡಾ ಚಂದ್ರಿಕಾ ಜ್ಯೋತ್ಸ್ನಾ ವಿಶ್ವಚಕ್ಷುಸ್ತಮೋಽಪಹಾ ।\\
ಸುಪ್ರಸನ್ನಾ ಮಹಾದೂತೀ ಯಮದೂತೀ ಭಯಂಕರೀ ॥೧೧೬॥

ಉನ್ಮಾದಿನೀ ಮಹಾರೂಪಾ ದಿವ್ಯರೂಪಾ ಸುರಾರ್ಚಿತಾ ।\\
ಚೈತನ್ಯರೂಪಿಣೀ ನಿತ್ಯಾ ಕ್ಲಿನ್ನಾ ಕಾಮಮದೋದ್ಧತಾ ॥೧೧೭॥

ಮದಿರಾನಂದಕೈವಲ್ಯಾ ಮದಿರಾಕ್ಷೀ ಮದಾಲಸಾ ।\\
ಸಿದ್ಧೇಶ್ವರೀ ಸಿದ್ಧವಿದ್ಯಾ ಸಿದ್ಧಾದ್ಯಾ ಸಿದ್ಧಸಂಭವಾ ॥೧೧೮॥

ಸಿದ್ಧರ್ದ್ಧಿಃ ಸಿದ್ಧಮಾತಾ ಚ ಸಿದ್ಧಃಸರ್ವಾರ್ಥಸಿದ್ಧಿದಾ ।\\
ಮನೋಮಯೀ ಗುಣಾತೀತಾ ಪರಂಜ್ಯೋತಿಃಸ್ವರೂಪಿಣೀ ॥೧೧೯॥

ಪರೇಶೀ ಪರಗಾಪಾರಾ ಪರಾಸಿದ್ಧಿಃ ಪರಾಗತಿಃ ।\\
ವಿಮಲಾ ಮೋಹಿನೀ ಆದ್ಯಾ ಮಧುಪಾನಪರಾಯಣಾ ॥೧೨೦॥

ವೇದವೇದಾಂಗಜನನೀ ಸರ್ವಶಾಸ್ತ್ರವಿಶಾರದಾ ।\\
ಸರ್ವದೇವಮಯೀವಿದ್ಯಾ ಸರ್ವಶಾಸ್ತ್ರಮಯೀ ತಥಾ ॥೧೨೧॥

ಸರ್ವಜ್ಞಾನಮಯೀದೇವೀ ಸರ್ವಧರ್ಮಮಯೀಶ್ವರೀ ।\\
ಸರ್ವಯಜ್ಞಮಯೀ ಯಜ್ಞಾ ಸರ್ವಮಂತ್ರಾಧಿಕಾರಿಣೀ ॥೧೨೨॥

ಸರ್ವಸಂಪತಪ್ರತಿಷ್ಠಾತ್ರೀ ಸರ್ವವಿದ್ರಾವಿಣೀ ಪರಾ ।\\
ಸರ್ವಸಂಕ್ಷೋಭಿಣೀದೇವೀ ಸರ್ವಮಂಗಲಕಾರಿಣೀ ॥೧೨೩॥

ತ್ರೈಲೋಕ್ಯಾಕರ್ಷಿಣೀ ದೇವೀ ಸರ್ವಾಹ್ಲಾದನಕಾರಿಣೀ ।\\
ಸರ್ವಸಮ್ಮೋಹಿನೀದೇವೀ ಸರ್ವಸ್ತಂಭನಕಾರಿಣೀ॥೧೨೪॥

ತ್ರೈಲೋಕ್ಯಜೃಂಭಿಣೀ ದೇವೀ ತಥಾ ಸರ್ವವಶಂಕರೀ ।\\
ತ್ರೈಲೋಕ್ಯರಂಜನೀದೇವೀ ಸರ್ವಸಂಪತ್ತಿದಾಯಿನೀ ॥೧೨೫॥

ಸರ್ವಮಂತ್ರಮಯಿದೇವೀ ಸರ್ವದ್ವಂದ್ವಕ್ಷಯಂಕರೀ ।\\
ಸರ್ವಸಿದ್ಧಿಪ್ರದಾದೇವೀ ಸರ್ವಸಂಪತ್ಪ್ರದಾಯಿನೀ ॥೧೨೬॥

ಸರ್ವಪ್ರಿಯಂಕರೀದೇವೀ ಸರ್ವಮಂಗಲಕಾರಿಣೀ ।\\
ಸರ್ವಕಾಮಪ್ರದಾದೇವೀ ಸರ್ವದುಃಖವಿಮೋಚಿನೀ ॥೧೨೭॥

ಸರ್ವಮೃತ್ಯುಪ್ರಶಮನೀ ಸರ್ವವಿಘ್ನವಿನಾಶಿನೀ ।\\
ಸರ್ವಾಂಗಸುಂದರೀಮಾತಾ ಸರ್ವಸೌಭಾಗ್ಯದಾಯಿನೀ ॥೧೨೮॥

ಸರ್ವಜ್ಞಾ ಸರ್ವಶಕ್ತಿಶ್ಚ ಸರ್ವೈಶ್ವರ್ಯಫಲಪ್ರದಾ ।\\
ಸರ್ವಜ್ಞಾನಮಯೀದೇವೀ ಸರ್ವವ್ಯಾಧಿವಿನಾಶಿನೀ ॥೧೨೯॥

ಸರ್ವಾಧಾರಸ್ವರೂಪಾ ಚ ಸರ್ವಪಾಪಹರಾ ತಥಾ ।\\
ಸರ್ವಾನಂದಮಯೀದೇವೀ ಸರ್ವೇಚ್ಛಾಯಾ:ಸ್ವರೂಪಿಣೀ ॥೧೩೦॥

ಸರ್ವಲಕ್ಷ್ಮೀಮಯೀವಿದ್ಯಾ ಸರ್ವೇಪ್ಸಿತಫಲಪ್ರದಾ ।\\
ಸರ್ವಾರಿಷ್ಟಪ್ರಶಮನೀ ಪರಮಾನಂದದಾಯಿನೀ ॥೧೩೧॥

ತ್ರಿಕೋಣನಿಲಯಾ ತ್ರಿಸ್ಥಾ ತ್ರಿಮಾತಾ ತ್ರಿತನುಸ್ಥಿತಾ।\\
ತ್ರಿವೇಣೀ ತ್ರಿಪಥಾ ಗುಣ್ಯಾ ತ್ರಿಮೂರ್ತಿಃ ತ್ರಿಪುರೇಶ್ವರೀ ॥೧೩೨॥

ತ್ರಿಧಾಮ್ನೀ ತ್ರಿದಶಾಧ್ಯಕ್ಷಾ ತ್ರಿವಿತ್ತ್ರಿಪುರವಾಸಿನೀ ।\\
ತ್ರಯೀವಿದ್ಯಾ ಚ ತ್ರಿಶಿರಾ ತ್ರೈಲೋಕ್ಯಾ ಚ ತ್ರಿಪುಷ್ಕರಾ ॥೧೩೩॥

ತ್ರಿಕೋಟರಸ್ಥಾ ತ್ರಿವಿಧಾ ತ್ರಿಪುರಾ ತ್ರಿಪುರಾತ್ಮಿಕಾ ।\\
ತ್ರಿಪುರಾಶ್ರೀ ತ್ರಿಜನನೀ ತ್ರಿಪುರಾತ್ರಿಪುರಸುಂದರೀ ॥೧೩೪॥

ಮಹಾಮಾಯಾ ಮಹಾಮೇಧಾ ಮಹಾಚಕ್ಷುಃ ಮಹೋಕ್ಷಜಾ ।\\
ಮಹಾವೇಧಾ ಪರಾಶಕ್ತಿಃ ಪರಾಪ್ರಜ್ಞಾ ಪರಂಪರಾ ॥೧೩೫॥

ಮಹಾಲಕ್ಷ್ಯಾ ಮಹಾಭಕ್ಷ್ಯಾ ಮಹಾಕಕ್ಷ್ಯಾಽಕಲೇಶ್ವರೀ ।\\
ಕಲೇಶ್ವರೀ ಕಲಾನಂದಾ ಕಲೇಶೀ ಕಲಸುಂದರೀ ॥೧೩೬॥

ಕಲಶಾ ಕಲಶೇಶೀ ಚ ಕುಂಭಮುದ್ರಾ ಕೃಶೋದರೀ ।\\
ಕುಂಭಪಾ ಕುಂಭಮಧ್ಯೇಶೀ ಕುಂಭಾನಂದಪ್ರದಾಯಿನೀ॥೧೩೭॥

ಕುಂಭಜಾನಂದನಾಥಾ ವ ಕುಂಭಜಾನಂದವರ್ದ್ಧಿನೀ ।\\
ಕುಂಭಜಾನಂದಸಂತೋಷಾ ಕುಂಭಜತರ್ಪಿಣೀಮುದಾ ॥೧೩೮॥

ವೃತ್ತಿಃ ವೃತ್ತೀಶ್ವರೀಽಮೋಘಾ ವಿಶ್ವವೃತ್ತ್ಯಂತತರ್ಪಿಣೀ।\\
ವಿಶ್ವಶಾಂತಿ ವಿಶಾಲಾಕ್ಷೀ ಮೀನಾಕ್ಷೀ ಮೀನವರ್ಣದಾ ॥೧೩೯॥

ವಿಶ್ವಾಕ್ಷೀ ದುರ್ಧರಾ ಧೂಮಾ ಇಂದ್ರಾಕ್ಷೀ ವಿಷ್ಣುಸೇವಿತಾ ।\\
ವಿರಂಚಿಸೇವಿತಾ ವಿಶ್ವಾ ಈಶಾನಾ ಈಶವಂದಿತಾ ॥೧೪೦॥

ಮಹಾಶೋಭಾ ಮಹಾಲೋಭಾ ಮಹಾಮೋಹಾ ಮಹೇಶ್ವರೀ ।\\
ಮಹಾಭೀಮಾ ಮಹಾಕ್ರೋಧಾ ಮನ್ಮಥಾ ಮದನೇಶ್ವರೀ ॥೧೪೧॥

ಮಹಾನಲಾ ಮಹಾಕ್ರೋಧಾ ವಿಶ್ವಸಂಹಾರತಾಂಡವಾ ।\\
ಸರ್ವಸಂಹಾರವರ್ಣೇಶೀ ಸರ್ವಪಾಲನತತ್ಪರಾ ॥೧೪೨॥

ಸರ್ವಾದಿಃ ಸೃಷ್ಟಿಕರ್ತ್ರೀ ಚ ಶಿವಾದ್ಯಾ ಶಂಭುಸ್ವಾಮಿನೀ ।\\
ಮಹಾನಂದೇಶ್ವರೀ ಮೃತ್ಯುರ್ಮಹಾಸ್ಪಂದೇಶ್ವರೀ ಸುಧಾ ॥೧೪೩॥

ಪರ್ಣಾಪರ್ಣ ಪರಾವರ್ಣಾಽಪರ್ಣೇಶೀ ಪರ್ಣಮಾನಸಾ ।\\
ವರಾಹೀ ತುಂಡದಾ ತುಂಡಾ ಗಣೇಶೀ ಗಣನಾಯಿಕಾ ॥೧೪೪॥

ವಟುಕಾ ವಟುಕೇಶೀ ಚ ಕ್ರೌಚದಾರಣಜನ್ಮದಾ ।\\
ಕ ಏ ಈ ಲ ಮಹಾಮಾಯಾ ಹ ಸ ಕ ಹ ಲ ಮಾಯಯಾ ॥೧೪೫॥

ದಿವಯಾನಾಮಾ ಸದಾಕಾಮಾ ಶ್ಯಾಮಾ ರಾಮಾ ರಮಾ ರಸಾ ।\\
ಸ ಕ ಲ ಹ್ರೀಂ ತತ್ಸ್ವರೂಪಾ ಶ್ರೀಂ ಹ್ರೀಂ ನಾಮಾದಿ ರೂಪಿಣೀ ॥೧೪೬॥

ಕಾಲಜ್ಞಾ ಕಾಲಹಾಮೂರ್ತಿಃ ಸರ್ವಸೌಭಾಗ್ಯದಾ ಮುದಾ ।\\
ಉರ್ವಾ ಉರ್ವೇಶ್ವರೀ ಖರ್ವಾ ಖರ್ವಪರ್ವಾ ಖಗೇಶ್ವರೀ ॥೧೪೭॥

ಗರುಡಾ ಗಾರುಡೀಮಾತಾ ಗರುಡೇಶ್ವರಪೂಜಿತಾ ।\\
ಅಂತರಿಕ್ಷಾಂತರಪದಾ ಪ್ರಜ್ಞಾ ಪ್ರಜ್ಞಾನದಾ ಪರಾ ॥೧೪೮॥

ವಿಜ್ಞಾನಾ ವಿಶ್ವವಿಜ್ಞಾನಾ ಅಂತರಾಕ್ಷಾ ವಿಶಾರದಾ ।\\
ಅಂತರ್ಜ್ಞಾನಮಯೀ ಸೌಮ್ಯಾ ಮೋಕ್ಷಾನಂದವಿವರ್ದ್ಧಿನೀ ॥೧೪೯॥

ಶಿವಶಕ್ತಿಮಯೀಶಕ್ತಿಃ ಏಕಾನಂದಪ್ರವರ್ತಿನೀ ।\\
ಶ್ರೀಮಾತಾ ಶ್ರೀಪರಾವಿದ್ಯಾ ಸಿದ್ಧಾಶ್ರೀ ಸಿದ್ಧಸಾಗರಾ ।\\
ಸಿದ್ಧಲಕ್ಷ್ಮೀ ಸಿದ್ಧವಿದ್ಯಾ ಸಿದ್ಧಾ ಸಿದ್ಧೇಶ್ವರೀ ಸುಧಾ ॥೧೫೦॥

॥ಫಲಶ್ರುತಿಃ॥

ಇದಂ ತ್ರಿಪುರಾಸುಂದರ್ಯಾಃ ಸ್ತೋತ್ರನಾಮಸಹಸ್ರಕಂ ।\\
ಗುಹ್ಯಾದ್ಗುಹ್ಯತರಂ ಪುತ್ರ! ತವ ಪ್ರೀತ್ಯೈ ಪ್ರಕೀರ್ತಿತಂ ॥೧॥

ಗೋಪನೀಯಂ ಪ್ರಯತ್ನೇನ ಪಠನೀಯಂ ಪ್ರಯತ್ನತಃ ।\\
ನಾತಃ ಪರತರಂ ಪುಣ್ಯಂ ನಾತಃಪರತರಂ ತಪಃ ॥೨॥

ನಾತಃ ಪರತರಂ ಸ್ತೋತ್ರಂ ನಾತಃ ಪರತರಂ ಗತಿಃ ।\\
ಸ್ತೋತ್ರಂ ಸಹಸ್ರನಾಮಾಖ್ಯಂ ಮಮ ವಕ್ತ್ರಾದ್ವಿನಿರ್ಗತಂ ॥೩॥

ಯಃ ಪಠೇತ್ಪ್ರಯತೋ ಭಕ್ತ್ಯಾ ಶೃಣುಯಾದ್ವಾ ಸಮಾಹಿತಃ ।\\
ಮೋಕ್ಷಾರ್ಥೀಂ ಲಭತೇ ಮೋಕ್ಷಂ ಸ್ವರ್ಗಾರ್ಥೀ ಸ್ವರ್ಗಮಾಪ್ನುಯಾತ್ ॥೪॥

ಕಾಮಾಂಶ್ಚ ಪ್ರಾಪ್ನುಯಾತಕಾಮೀ ಧನಾರ್ಥೀ ಚ ಲಭೇದ್ಧನಂ ।\\
ವಿದ್ಯಾರ್ಥೀ ಲಭತೇ ವಿದ್ಯಾಂ ಯಶೋಽರ್ಥೀ ಲಭತೇ ಯಶಃ ॥೫॥

ಕನ್ಯಾರ್ಥೀ ಲಭತೇ ಕನ್ಯಾಂ ಸುತಾರ್ಥೀ ಲಭತೇ ಸುತಂ ।\\
ಗುರ್ವಿಣೀ ಜನಯೇತ್ಪುತ್ರಂ ಕನ್ಯಾ ವಿಂದತಿ ಸತ್ಪತಿಂ ॥೬॥

ಮೂರ್ಖೋಽಪಿ ಲಭತೇ ಶಾಸ್ತ್ರಂ ಹೀನೋಽಪಿ ಲಭತೇ ಗತಿಂ ।\\
ಸಂಕ್ರಾಂತ್ಯಾಂ ವಾರ್ಕಾಮಾವಸ್ಯಾಂ ಅಷ್ಟಮ್ಯಾಂ ಚ ವಿಶೇಷತಃ ॥೭॥

ಪೌರ್ಣಮಾಸ್ಯಾಂ ಚತುರ್ದಶ್ಯಾಂ ನವಮ್ಯಾಂ ಭೌಮವಾಸರೇ ।\\
ಪಠೇದ್ವಾ ಪಾಠಯೇದ್ವಾಪಿ ಶೃಣುಯಾದ್ವಾ ಸಮಾಹಿತಃ ॥೮॥

ಸ ಮುಕ್ತೋ ಸರ್ವಪಾಪೇಭ್ಯಃ ಕಾಮೇಶ್ವರಸಮೋ ಭವೇತ್ ।\\
ಲಕ್ಷ್ಮೀವಾನ್ ಧರ್ಮವಾಂಶ್ಚೈವ ವಲ್ಲಭಸ್ಸರ್ವಯೋಷಿತಾಂ ॥೯॥

ತಸ್ಯ ವಶ್ಯಂ ಭವೇದಾಶು ತ್ರೈಲೋಕ್ಯಂ ಸಚರಾಚರಂ ।\\
ರುದ್ರಂ ದೃಷ್ಟವಾ ಯಥಾ ದೇವಾ ವಿಷ್ಣುಂ ದೃಷ್ಟ್ವಾ ಚ ದಾನವಾಃ ॥೧೦॥

ಯಥಾಹಿರ್ಗರುಡಂ ದೃಷ್ಟ್ವಾ ಸಿಂಹ ದೃಷ್ಟ್ವಾ ಯಥಾ ಗಜಾಃ ।\\
ಕೀಟವತ್ಪ್ರಪಲಾಯಂತೇ ತಸ್ಯ ವಕ್ತ್ರಾವಲೋಕನಾತ್ ॥೧೧॥

ಅಗ್ನಿಚೌರಭಯಂ ತಸ್ಯ ಕದಾಚಿನ್ನೈವ ಸಂಭವೇತ್ ।\\
ಪಾತಕಾ ವಿವಿಧಾಃ ಶಂತಿರ್ಮೇರುಪರ್ವತಸನ್ನಿಭಾಃ ॥೧೨॥

ಯಸ್ಮಾತ್ತಚ್ಛೃಣುಯಾದ್ವಿಘ್ನಾಂಸ್ತೃಣಂ ವಹ್ನಿಹುತಂ ಯಥಾ ।\\
ಏಕದಾ ಪಠನಾದೇವ ಸರ್ವಪಾಪಕ್ಷಯೋ ಭವೇತ್ ॥೧೩॥

ದಶಧಾ ಪಠನಾದೇವ ವಾಚಾ ಸಿದ್ಧಃ ಪ್ರಜಾಯತೇ ।\\
ಶತಧಾ ಪಠನಾದ್ವಾಪಿ ಖೇಚರೋ ಜಾಯತೇ ನರಃ ॥೧೪॥

ಸಹಸ್ರದಶಸಂಖ್ಯಾತಂ ಯಃ ಪಠೇದ್ಭಕ್ತಿಮಾನಸಃ ।\\
ಮಾತಾಽಸ್ಯ ಜಗತಾಂ ಧಾತ್ರೀ ಪ್ರತ್ಯಕ್ಷಾ ಭವತಿ ಧ್ರುವಂ ॥೧೫॥

ಲಕ್ಷಪೂರ್ಣೇ ಯಥಾ ಪುತ್ರ! ಸ್ತೋತ್ರರಾಜಂ ಪಠೇತ್ಸುಧೀಃ ।\\
ಭವಪಾಶವಿನಿರ್ಮುಕ್ತೋ ಮಮ ತುಲ್ಯೋ ನ ಸಂಶಯಃ ॥೧೬॥

ಸರ್ವತೀರ್ಥೇಷು ಯತ್ಪುಣ್ಯಂ ಸಕೃಜ್ಜಪ್ತ್ವಾ ಲಭೇನ್ನರಃ ।\\
ಸರ್ವವೇದೇಷು ಯತ್ಪ್ರೋಕ್ತಂ ತತ್ಫಲಂ ಪರಿಕೀರ್ತಿತಂ ॥೧೭॥

ಭೂತ್ವಾ ಚ ಬಲವಾನ್ ಪುತ್ರ ಧನವಾನ್ಸರ್ವಸಂಪದಃ ।\\
ದೇಹಾಂತೇ ಪರಮಂ ಸ್ಥಾನಂ ಯತ್ಸುರೈರಪಿ ದುರ್ಲಭಂ ॥೧೮॥

ಸ ಯಾಸ್ಯತಿ ನ ಸಂದೇಹಃ ಸ್ತವರಾಜಸ್ಯ ಕಿರ್ತ್ತನಾತ್ ॥೧೯॥

\authorline{॥ಇತಿ ಶ್ರೀವಾಮಕೇಶ್ವರತಂತ್ರೇ ಷೋಡಶ್ಯಾಃ ಸಹಸ್ರನಾಮಸ್ತೋತ್ರಂ ಸಂಪೂರ್ಣಂ॥}
%=============================================================================================

\section{ಶ್ರೀಷೋಡಶೀಶತನಾಮಸ್ತೋತ್ರಂ}
\addcontentsline{toc}{section}{ಶ್ರೀಷೋಡಶೀಶತನಾಮಸ್ತೋತ್ರಂ}


ಭೃಗುರುವಾಚ ।\\
ಚತುರ್ವಕ್ತ್ರ ಜಗನ್ನಾಥ ಸ್ತೋತ್ರಂ ವದ ಮಯಿ ಪ್ರಭೋ ।\\
ಯಸ್ಯಾನುಷ್ಠಾನಮಾತ್ರೇಣ ನರೋ ಭಕ್ತಿಮವಾಪ್ನುಯಾತ್ ॥೧॥

ಬ್ರಹ್ಮೋವಾಚ ।\\
ಸಹಸ್ರನಾಮ್ನಾಮಾಕೃಷ್ಯ ನಾಮ್ನಾಮಷ್ಟೋತ್ತರಂ ಶತಂ ।\\
ಗುಹ್ಯಾದ್ಗುಹ್ಯತರಂ ಗುಹ್ಯಂ ಸುಂದರ್ಯಾಃ ಪರಿಕೀರ್ತಿತಂ ॥೨॥

ಅಸ್ಯ ಶ್ರೀಷೋಡಶ್ಯಷ್ಟೋತ್ತರಶತನಾಮಸ್ತೋತ್ರಸ್ಯ ಶಂಭುರೃಷಿಃ । ಅನುಷ್ಟುಪ್ ಛಂದಃ । ಶ್ರೀಷೋಡಶೀ ದೇವತಾ । ಧರ್ಮಾರ್ಥಕಾಮಮೋಕ್ಷಸಿದ್ಧಯೇ ವಿನಿಯೋಗಃ ॥

ಓಂ ತ್ರಿಪುರಾ ಷೋಡಶೀ ಮಾತಾ ತ್ರ್ಯಕ್ಷರಾ ತ್ರಿತಯಾ ತ್ರಯೀ ।\\
ಸುಂದರೀ ಸುಮುಖೀ ಸೇವ್ಯಾ ಸಾಮವೇದಪರಾಯಣಾ ॥೩॥

ಶಾರದಾ ಶಬ್ದನಿಲಯಾ ಸಾಗರಾ ಸರಿದಂಬರಾ ।\\
ಶುದ್ಧಾ ಶುದ್ಧತನುಸ್ಸಾಧ್ವೀ ಶಿವಧ್ಯಾನಪರಾಯಣಾ ॥೪॥

ಸ್ವಾಮಿನೀ ಶಂಭುವನಿತಾ ಶಾಂಭವೀ ಚ ಸರಸ್ವತೀ ।\\
ಸಮುದ್ರಮಥಿನೀ ಶೀಘ್ರಗಾಮಿನೀ ಶೀಘ್ರಸಿದ್ಧಿದಾ ॥೫॥

ಸಾಧುಸೇವ್ಯಾ ಸಾಧುಗಮ್ಯಾ ಸಾಧುಸಂತುಷ್ಟಮಾನಸಾ ।\\
ಖಟ್ವಾಂಗಧಾರಿಣೀ ಖರ್ವಾ ಖಡ್ಗಖರ್ಪರಧಾರಿಣೀ ॥೬॥

ಷಡ್ವರ್ಗಭಾವರಹಿತಾ ಷಡ್ವರ್ಗಪರಿಚಾರಿಕಾ ।\\
ಷಡ್ವರ್ಗಾ ಚ ಷಡಂಗಾ ಚ ಷೋಢಾ ಷೋಡಶವಾರ್ಷಿಕೀ ॥೭॥

ಕ್ರತುರೂಪಾ ಕ್ರತುಮಯೀ ಋಭುಕ್ಷಕ್ರತುಮಂಡಿತಾ ।\\
ಕವರ್ಗಾದಿ ಪವರ್ಗಾಂತಾ ಅಂತಸ್ಥಾನಂತರೂಪಿಣೀ ॥೮॥

ಅಕಾರಾಕಾರರಹಿತಾ ಕಾಲಮೃತ್ಯುಜರಾಪಹಾ ।\\
ತನ್ವೀ ತತ್ತ್ವೇಶ್ವರೀ ತಾರಾ ತ್ರಿವರ್ಷಾ ಜ್ಞಾನರೂಪಿಣೀ ॥೯॥

ಕಾಲೀ ಕರಾಲೀ ಕಾಮೇಶೀ ಛಾಯಾ ಸಂಜ್ಞಾಪ್ಯರುಂಧತೀ ।\\
ನಿರ್ವಿಕಲ್ಪಾ ಮಹಾವೇಗಾ ಮಹೋತ್ಸಾಹಾ ಮಹೋದರೀ ॥೧೦॥

ಮೇಘಾ ಬಲಾಕಾ ವಿಮಲಾ ವಿಮಲಜ್ಞಾನದಾಯಿನೀ ।\\
ಗೌರೀ ವಸುಂಧರಾ ಗೋಪ್ತ್ರೀ ಗವಾಂಪತಿನಿಷೇವಿತಾ ॥೧೧॥

ಭಗಾಂಗಾ ಭಗರೂಪಾ ಚ ಭಕ್ತಿಭಾವಪರಾಯಣಾ ।\\
ಛಿನ್ನಮಸ್ತಾ ಮಹಾಧೂಮಾ ತಥಾ ಧೂಮ್ರವಿಭೂಷಣಾ ॥೧೨॥

ಧರ್ಮಕರ್ಮಾದಿ ರಹಿತಾ ಧರ್ಮಕರ್ಮಪರಾಯಣಾ ।\\
ಸೀತಾ ಮಾತಂಗಿನೀ ಮೇಧಾ ಮಧುದೈತ್ಯವಿನಾಶಿನೀ ॥೧೩॥

ಭೈರವೀ ಭುವನಾ ಮಾತಾಽಭಯದಾ ಭವಸುಂದರೀ ।\\
ಭಾವುಕಾ ಬಗಲಾ ಕೃತ್ಯಾ ಬಾಲಾ ತ್ರಿಪುರಸುಂದರೀ ॥೧೪॥

ರೋಹಿಣೀ ರೇವತೀ ರಮ್ಯಾ ರಂಭಾ ರಾವಣವಂದಿತಾ ।\\
ಶತಯಜ್ಞಮಯೀ ಸತ್ತ್ವಾ ಶತಕ್ರತುವರಪ್ರದಾ ॥೧೫॥

ಶತಚಂದ್ರಾನನಾ ದೇವೀ ಸಹಸ್ರಾದಿತ್ಯಸನ್ನಿಭಾ ।\\
ಸೋಮಸೂರ್ಯಾಗ್ನಿನಯನಾ ವ್ಯಾಘ್ರಚರ್ಮಾಂಬರಾವೃತಾ ॥೧೬॥

ಅರ್ಧೇಂದುಧಾರಿಣೀ ಮತ್ತಾ ಮದಿರಾ ಮದಿರೇಕ್ಷಣಾ ।\\
ಇತಿ ತೇ ಕಥಿತಂ ಗೋಪ್ಯಂ ನಾಮ್ನಾಮಷ್ಟೋತ್ತರಂ ಶತಂ ॥೧೭॥

ಸುಂದರ್ಯಾಃ ಸರ್ವದಂ ಸೇವ್ಯಂ ಮಹಾಪಾತಕನಾಶನಂ ।\\
ಗೋಪನೀಯಂ ಗೋಪನೀಯಂ ಗೋಪನೀಯಂ ಕಲೌ ಯುಗೇ ॥೧೮॥

ಸಹಸ್ರನಾಮಪಾಠಸ್ಯ ಫಲಂ ಯದ್ವೈ ಪ್ರಕೀರ್ತಿತಂ ।\\
ತಸ್ಮಾತ್ಕೋಟಿಗುಣಂ ಪುಣ್ಯಂ ಸ್ತವಸ್ಯಾಸ್ಯ ಪ್ರಕೀರ್ತನಾತ್ ॥೧೯॥

ಪಠೇತ್ಸದಾ ಭಕ್ತಿಯುತೋ ನರೋ ಯೋ\\ ನಿಶೀಥಕಾಲೇಽಪ್ಯರುಣೋದಯೇ ವಾ ।\\
ಪ್ರದೋಷಕಾಲೇ ನವಮೀದಿನೇಽಥವಾ\\ ಲಭೇತ ಭೋಗಾನ್ಪರಮಾದ್ಭುತಾನ್ಪ್ರಿಯಾನ್ ॥೨೦॥

\authorline{ಇತಿ ಬ್ರಹ್ಮಯಾಮಲೇ ಪೂರ್ವಖಂಡೇ ಷೋಡಶ್ಯಷ್ಟೋತ್ತರಶತನಾಮಸ್ತೋತ್ರಂ ಸಮಾಪ್ತಂ ॥}
%==================================================
\section{ತಾರಾಸಹಸ್ರನಾಮಸ್ತೋತ್ರಂ}
\addcontentsline{toc}{section}{ತಾರಾಸಹಸ್ರನಾಮಸ್ತೋತ್ರಂ}


ಶ್ರೀದೇವ್ಯುವಾಚ ।\\
ದೇವ ದೇವ ಮಹಾದೇವ ಸೃಷ್ಟಿಸ್ಥಿತ್ಯಂತಕಾರಕ ।\\
ಪ್ರಸಂಗೇನ ಮಹಾದೇವ್ಯಾ ವಿಸ್ತರಂ ಕಥಿತಂ ಮಯಿ ॥೧॥

ದೇವ್ಯಾ ನೀಲಸರಸ್ವತ್ಯಾಃ ಸಹಸ್ರಂ ಪರಮೇಶ್ವರ ।\\
ನಾಮ್ನಾಂ ಶ್ರೋತುಂ ಮಹೇಶಾನ ಪ್ರಸಾದಃ ಕ್ರಿಯತಾಂ ಮಯಿ ।\\
ಕಥಯಸ್ವ ಮಹಾದೇವ ಯದ್ಯಹಂ ತವ ವಲ್ಲಭಾ ॥೨॥

ಶ್ರೀಭೈರವ ಉವಾಚ ।\\
ಸಾಧು ಪೃಷ್ಟಂ ಮಹಾದೇವಿ ಸರ್ವತಂತ್ರೇಷು ಗೋಪಿತಂ ।\\
ನಾಮ್ನಾಂ ಸಹಸ್ರಂ ತಾರಾಯಾಃ ಕಥಿತುಂ ನೈವ ಶಕ್ಯತೇ ॥೩॥

ಪ್ರಕಾಶಾತ್ ಸಿದ್ಧಿಹಾನಿಃ ಸ್ಯಾತ್ ಶ್ರಿಯಾ ಚ ಪರಿಹೀಯತೇ ।\\
ಪ್ರಕಾಶಯತಿ ಯೋ ಮೋಹಾತ್ ಷಣ್ಮಾಸಾದ್ ಮೃತ್ಯುಮಾಪ್ನುಯಾತ್ ॥೪॥

ಅಕಥ್ಯಂ ಪರಮೇಶಾನಿ ಅಕಥ್ಯಂ ಚೈವ ಸುಂದರಿ ।\\
ಕ್ಷಮಸ್ವ ವರದೇ ದೇವಿ ಯದಿ ಸ್ನೇಹೋಽಸ್ತಿ ಮಾಂ ಪ್ರತಿ ॥೫॥

ಸರ್ವಸ್ವಂ ಶೃಣು ಹೇ ದೇವಿ ಸರ್ವಾಗಮವಿದಾಂ ವರೇ ।\\
ಧನಸಾರಂ ಮಹಾದೇವಿ ಗೋಪ್ತವ್ಯಂ ಪರಮೇಶ್ವರಿ ॥೬॥

ಆಯುರ್ಗೋಪ್ಯಂ ಗೃಹಚ್ಛಿದ್ರಂ ಗೋಪ್ಯಂ ನ ಪಾಪಭಾಗ್ ಭವೇತ್ ।\\
ಸುಗೋಪ್ಯಂ ಪರಮೇಶಾನಿ ಗೋಪನಾತ್ ಸಿದ್ಧಿಮಶ್ನುತೇ ॥೭॥

ಪ್ರಕಾಶಾತ್ ಕಾರ್ಯಹಾನಿಶ್ಚ ಪ್ರಕಾಶಾತ್ ಪ್ರಲಯಂ ಭವೇತ್ ।\\
ತಸ್ಮಾದ್ ಭದ್ರೇ ಮಹೇಶಾನಿ ನ ಪ್ರಕಾಶ್ಯಂ ಕದಾಚನ ॥೮॥

ಇತಿ ದೇವವಚಃ ಶ್ರುತ್ವಾ ದೇವೀ ಪರಮಸುಂದರೀ ।\\
ವಿಸ್ಮಿತಾ ಪರಮೇಶಾನೀ ವಿಷಣಾ ತತ್ರ ಜಾಯತೇ ॥೯॥

ಶೃಣು ಹೇ ಪರಮೇಶಾನ ಕೃಪಾಸಾಗರಪಾರಗ ।\\
ತವ ಸ್ನೇಹೋ ಮಹಾದೇವ ಮಯಿ ನಾಸ್ತ್ಯತ್ರ ನಿಶ್ಚಿತಂ ॥೧೦॥

ಭದ್ರಂ ಭದ್ರಂ ಮಹಾದೇವ ಇತಿ ಕೃತ್ವಾ ಮಹೇಶ್ವರೀ ।\\
ವಿಮುಖೀಭೂಯ ದೇವೇಶೀ ತತ್ರಾಸ್ತೇ ಶೈಲಜಾ ಶುಭಾ ॥೧೧॥

ವಿಲೋಕ್ಯ ವಿಮುಖೀಂ ದೇವೀಂ ಮಹಾದೇವೋ ಮಹೇಶ್ವರಃ ।\\
ಪ್ರಹಸ್ಯ ಪರಮೇಶಾನೀಂ ಪರಿಷ್ವಜ್ಯ ಪ್ರಿಯಾಂ ಕಥಾಂ ॥೧೨॥

ಕಥಯಾಮಾಸ ತತ್ರೈವ ಮಹಾದೇವ್ಯೈ ಮಹೇಶ್ವರಿ ।\\
ಮಮ ಸರ್ವಸ್ವರೂಪಾ ತ್ವಂ ಜಾನೀಹಿ ನಗನಂದಿನಿ ॥೧೩॥

ತ್ವಾಂ ವಿನಾಹಂ ಮಹಾದೇವಿ ಪೂರ್ವೋಕ್ತಶವರೂಪವಾನ್ ।\\
ಕ್ಷಮಸ್ವ ಪರಮಾನಂದೇ ಕ್ಷಮಸ್ವ ನಗನಂದಿನಿ ॥೧೪॥

ಯಥಾ ಪ್ರಾಣೋ ಮಹೇಶಾನಿ ದೇಹೇ ತಿಷ್ಠತಿ ಸುಂದರಿ ।\\
ತಥಾ ತ್ವಂ ಜಗತಾಮಾದ್ಯೇ ಚರಣೇ ಪತಿತೋಽಸ್ಮ್ಯಹಂ ॥೧೫॥

ಇತಿ ಮತ್ವಾ ಮಹಾದೇವಿ ರಕ್ಷ ಮಾಂ ತವ ಕಿಂಕರಂ ।\\
ತತೋ ದೇವೀ ಮಹೇಶಾನೀ ತ್ರೈಲೋಕ್ಯಮೋಹಿನೀ ಶಿವಾ ॥೧೬॥

ಮಹಾದೇವಂ ಪರಿಷ್ವಜ್ಯ ಪ್ರಾಹ ಗದ್ಗದಯಾ ಗಿರಾ ।\\
ಸದಾ ದೇಹಸ್ವರೂಪಾಹಂ ದೇಹೀ ತ್ವಂ ಪರಮೇಶ್ವರ ॥೧೭॥

ತಥಾಪಿ ವಂಚನಾಂ ಕರ್ತುಂ ಮಾಮಿತ್ಥಂ ವದಸಿ ಪ್ರಿಯಂ ।\\
ಮಹಾದೇವಃ ಪುನಃ ಪ್ರಾಹ ಭೈರವಿ ಪ್ರಾಣವಲ್ಲಭೇ ॥೧೮॥

ನಾಮ್ನಾಂ ಸಹಸ್ರಂ ತಾರಾಯಾಃ ಶ್ರೋತುಮಿಚ್ಛಸ್ಯಶೇಷತಃ ।\\

ಶ್ರೀದೇವ್ಯುವಾಚ ।\\
ನ ಶ್ರುತಂ ಪರಮೇಶಾನ ತಾರಾನಾಮಸಹಸ್ರಕಂ ।\\
ಕಥಯಸ್ವ ಮಹಾಭಾಗ ಸತ್ಯಂ ಪರಮಸುಂದರಂ ॥೧೯॥

ಶ್ರೀಪಾರ್ವತ್ಯುವಾಚ ।\\
ಕಥಮೀಶಾನ ಸರ್ವಜ್ಞ ಲಭಂತೇ ಸಿದ್ಧಿಮುತ್ತಮಾಂ ।\\
ಸಾಧಕಾಃ ಸರ್ವದಾ ಯೇನ ತನ್ಮೇ ಕಥಯ ಸುಂದರ ॥೨೦॥

ಯಸ್ಮಾತ್ ಪರತರಂ ನಾಸ್ತಿ ಸ್ತೋತ್ರಂ ತಂತ್ರೇಷು ನಿಶ್ಚಿತಂ ।\\
ಸರ್ವಪಾಪಹರಂ ದಿವ್ಯಂ ಸರ್ವಾಪದ್ವಿನಿವಾರಕಂ ॥೨೧॥

ಸರ್ವಜ್ಞಾನಕರಂ ಪುಣ್ಯಂ ಸರ್ವಮಂಗಲಸಂಯುತಂ ।\\
ಪುರಶ್ಚರ್ಯಾಶತೈಸ್ತುಲ್ಯಂ ಸ್ತೋತ್ರಂ ಸರ್ವಪ್ರಿಯಂಕರಂ ॥೨೨॥

ವಶ್ಯಪ್ರದಂ ಮಾರಣದಮುಚ್ಚಾಟನಪ್ರದಂ ಮಹತ್ ।\\
ನಾಮ್ನಾಂ ಸಹಸ್ರಂ ತಾರಾಯಾಃ ಕಥಯಸ್ವ ಸುರೇಶ್ವರ ॥೨೩॥

ಶ್ರೀಮಹಾದೇವ ಉವಾಚ ।\\
ನಾಮ್ನಾಂ ಸಹಸ್ರಂ ತಾರಾಯಾಃ ಸ್ತೋತ್ರಪಾಠಾದ್ ಭವಿಷ್ಯತಿ ।\\
ನಾಮ್ನಾಂ ಸಹಸ್ರಂ ತಾರಾಯಾಃ ಕಥಯಿಷ್ಯಾಮ್ಯಶೇಷತಃ ॥೨೪॥

ಶೃಣು ದೇವಿ ಸದಾ ಭಕ್ತ್ಯಾ ಭಕ್ತಾನಾಂ ಪರಮಂ ಹಿತಂ ।\\
ವಿನಾ ಪೂಜೋಪಹಾರೇಣ ವಿನಾ ಜಾ(ಪ್ಯೇನ ಯತ್ ಫಲಂ ॥೨೫॥

ತತ್ ಫಲಂ ಸಕಲಂ ದೇವಿ ಕಥಯಿಷ್ಯಾಮಿ ತಚ್ಛೃಣು ।\\

ಓಂ ಅಸ್ಯ ಶ್ರೀತಾರಾಸಹಸ್ರನಾಮಸ್ತೋತ್ರಮಹಾಮಂತ್ರಸ್ಯ,
ಅಕ್ಷೋಭ್ಯ ಋಷಿಃ, ಬೃಹತೀಉಷ್ಣಿಕ್ ಛಂದಃ,
ಶ್ರೀ ಉಗ್ರತಾರಾ ಶ್ರೀಮದೇಕಜಟಾ ಶ್ರೀನೀಲಸರಸ್ವತೀ ದೇವತಾ,
ಪುರುಷಾರ್ಥಚತುಷ್ಟಯಸಿದ್ಧ್ಯರ್ಥೇ ವಿನಿಯೋಗಃ॥

ತಾರಾ ರಾತ್ರಿರ್ಮಹಾರಾತ್ರಿರ್ಕಾಲರಾತ್ರಿರ್ಮಹಾಮತಿಃ ।\\
ಕಾಲಿಕಾ ಕಾಮದಾ ಮಾಯಾ ಮಹಾಮಾಯಾ ಮಹಾಸ್ಮೃತಿಃ ॥೨೬॥

ಮಹಾದಾನರತಾ ಯಜ್ಞಾ ಯಜ್ಞೋತ್ಸವವಿಭೂಷಿತಾ ।\\
ಚಂದ್ರವ್ವಜ್ರಾ ಚಕೋರಾಕ್ಷೀ ಚಾರುನೇತ್ರಾ ಸುಲೋಚನಾ ॥೨೭॥

ತ್ರಿನೇತ್ರಾ ಮಹತೀ ದೇವೀ ಕುರಂಗಾಕ್ಷೀ ಮನೋರಮಾ ।\\
ಬ್ರಾಹ್ಮೀ ನಾರಾಯಣೀ ಜ್ಯೋತ್ಸ್ನಾ ಚಾರುಕೇಶೀ ಸುಮೂರ್ಧಜಾ ॥೨೮॥

ವಾರಾಹೀ ವಾರುಣೀ ವಿದ್ಯಾ ಮಹಾವಿದ್ಯಾ ಮಹೇಶ್ವರೀ ।\\
ಸಿದ್ಧಾ ಕುಂಚಿತಕೇಶಾ ಚ ಮಹಾಯಜ್ಞಸ್ವರೂಪಿಣೀ ॥೨೯॥

ಗೌರೀ ಚಂಪಕವರ್ಣಾ ಚ ಕೃಶಾಂಗೀ ಶಿವಮೋಹಿನೀ ।\\
ಸರ್ವಾನಂದಸ್ವರೂಪಾ ಚ ಸರ್ವಶಂಕೈಕತಾರಿಣೀ ॥೩೦॥

ವಿದ್ಯಾನಂದಮಯೀ ನಂದಾ ಭದ್ರಕಾಲೀಸ್ವರೂಪಿಣೀ ।\\
ಗಾಯತ್ರೀ ಸುಚರಿತ್ರಾ ಚ ಕೌಲವ್ರತಪರಾಯಣಾ ॥೩೧॥

ಹಿರಣ್ಯಗರ್ಭಾ ಭೂಗರ್ಭಾ ಮಹಾಗರ್ಭಾ ಸುಲೋಚನೀ ।\\
ಹಿಮವತ್ತನಯಾ ದಿವ್ಯಾ ಮಹಾಮೇಘಸ್ವರೂಪಿಣೀ ॥೩೨॥

ಜಗನ್ಮಾತಾ ಜಗದ್ಧಾತ್ರೀ ಜಗತಾಮುಪಕಾರಿಣೀ ।\\
ಐಂದ್ರೀ ಸೌಮ್ಯಾ ತಥಾ ಘೋರಾ ವಾರುಣೀ ಮಾಧವೀ ತಥಾ ॥೩೩॥

ಆಗ್ನೇಯೀ ನೈರೃತೀ ಚೈವ ಐಶಾನೀ ಚಂಡಿಕಾತ್ಮಿಕಾ ।\\
ಸುಮೇರುತನಯಾ ನಿತ್ಯಾ ಸರ್ವೇಷಾಮುಪಕಾರಿಣೀ ॥೩೪॥

ಲಲಜ್ಜಿಹ್ವಾ ಸರೋಜಾಕ್ಷೀ ಮುಂಡಸ್ರಕ್ಪರಿಭೂಷಿತಾ ।\\
ಸರ್ವಾನಂದಮಯೀ ಸರ್ವಾ ಸರ್ವಾನಂದಸ್ವರೂಪಿಣೀ ॥೩೫॥

ಧೃತಿರ್ಮೇಧಾ ತಥಾ ಲಕ್ಷ್ಮೀಃ ಶ್ರದ್ಧಾ ಪನ್ನಗಗಾಮಿನೀ ।\\
ರುಕ್ಮಿಣೀ ಜಾನಕೀ ದುರ್ಗಾಂಬಿಕಾ ಸತ್ಯವತೀ ರತಿಃ ॥೩೬॥

ಕಾಮಾಖ್ಯಾ ಕಾಮದಾ ನಂದಾ ನಾರಸಿಂಹೀ ಸರಸ್ವತೀ ।\\
ಮಹಾದೇವರತಾ ಚಂಡೀ ಚಂಡದೋರ್ದಂಡಖಂಡಿನೀ ॥೩೭॥

ದೀರ್ಘಕೇಶೀ ಸುಕೇಶೀ ಚ ಪಿಂಗಕೇಶೀ ಮಹಾಕಚಾ ।\\
ಭವಾನೀ ಭವಪತ್ನೀ ಚ ಭವಭೀತಿಹರಾ ಸತೀ ॥೩೮॥

ಪೌರಂದರೀ ತಥಾ ವಿಷ್ಣೋರ್ಜಾಯಾ ಮಾಹೇಶ್ವರೀ ತಥಾ ।\\
ಸರ್ವೇಷಾಂ ಜನನೀ ವಿದ್ಯಾ ಚಾರ್ವಂಗೀ ದೈತ್ಯನಾಶಿನೀ ॥೩೯॥

ಸರ್ವರೂಪಾ ಮಹೇಶಾನಿ ಕಾಮಿನೀ ವರವರ್ಣಿನೀ ।\\
ಮಹಾವಿದ್ಯಾ ಮಹಾಮಾಯಾ ಮಹಾಮೇಧಾ ಮಹೋತ್ಸವಾ ॥೪೦॥

ವಿರೂಪಾ ವಿಶ್ವರೂಪಾ ಚ ಮೃಡಾನೀ ಮೃಡವಲ್ಲಭಾ ।\\
ಕೋಟಿಚಂದ್ರಪ್ರತೀಕಾಶಾ ಶತಸೂರ್ಯಪ್ರಕಾಶಿನೀ ॥೪೧॥

ಜಹ್ನುಕನ್ಯಾ ಮಹೋಗ್ರಾ ಚ ಪಾರ್ವತೀ ವಿಶ್ವಮೋಹಿನೀ ।\\
ಕಾಮರೂಪಾ ಮಹೇಶಾನೀ ನಿತ್ಯೋತ್ಸಾಹಾ ಮನಸ್ವಿನೀ ॥೪೨॥

ವೈಕುಂಠನಾಥಪತ್ನೀ ಚ ತಥಾ ಶಂಕರಮೋಹಿನೀ ।\\
ಕಾಶ್ಯಪೀ ಕಮಲಾ ಕೃಷ್ಣಾ ಕೃಷ್ಣರೂಪಾ ಚ ಕಾಲಿನೀ ॥೪೩॥

ಮಾಹೇಶ್ವರೀ ವೃಷಾರೂಢಾ ಸರ್ವವಿಸ್ಮಯಕಾರಿಣೀ ।\\
ಮಾನ್ಯಾ ಮಾನವತೀ ಶುದ್ಧಾ ಕನ್ಯಾ ಹಿಮಗಿರೇಸ್ತಥಾ ॥೪೪॥

ಅಪರ್ಣಾ ಪದ್ಮಪತ್ರಾಕ್ಷೀ ನಾಗಯಜ್ಞೋಪವೀತಿನೀ ।\\
ಮಹಾಶಂಖಧರಾ ಕಾಂತಾ ಕಮನೀಯಾ ನಗಾತ್ಮಜಾ ॥೪೫॥

ಬ್ರಹ್ಮಾಣೀ ವೈಷ್ಣವೀ ಶಂಭೋರ್ಜಾಯಾ ಗಂಗಾ ಜಲೇಶ್ವರೀ ।\\
ಭಾಗೀರಥೀ ಮನೋಬುದ್ಧಿರ್ನಿತ್ಯಾ ವಿದ್ಯಾಮಯೀ ತಥಾ ॥೪೬॥

ಹರಪ್ರಿಯಾ ಗಿರಿಸುತಾ ಹರಪತ್ನೀ ತಪಸ್ವಿನೀ ।\\
ಮಹಾವ್ಯಾಧಿಹರಾ ದೇವೀ ಮಹಾಘೋರಸ್ವರೂಪಿಣೀ ॥೪೭॥

ಮಹಾಪುಣ್ಯಪ್ರಭಾ ಭೀಮಾ ಮಧುಕೈಟಭನಾಶಿನೀ ।\\
ಶಂಖಿನೀ ವಜ್ರಿಣೀ ಧಾತ್ರೀ ತಥಾ ಪುಸ್ತಕಧಾರಿಣೀ ॥೪೮॥

ಚಾಮುಂಡಾ ಚಪಲಾ ತುಂಗಾ ಶುಂಬದೈತ್ಯನಿಕೃಂತನೀ ।\\
ಶಾಂತಿರ್ನಿದ್ರಾ ಮಹಾನಿದ್ರಾ ಪೂರ್ಣನಿದ್ರಾ ಚ ರೇಣುಕಾ ॥೪೯॥

ಕೌಮಾರೀ ಕುಲಜಾ ಕಾಂತೀ ಕೌಲವ್ರತಪರಾಯಣಾ ।\\
ವನದುರ್ಗಾ ಸದಾಚಾರಾ ದ್ರೌಪದೀ ದ್ರುಪದಾತ್ಮಜಾ ॥೫೦॥

ಯಶಸ್ವಿನೀ ಯಶಸ್ಯಾ ಚ ಯಶೋಧಾತ್ರೀ ಯಶಃಪ್ರದಾ ।\\
ಸೃಷ್ಟಿರೂಪಾ ಮಹಾಗೌರೀ ನಿಶುಂಬಪ್ರಾಣನಾಶಿನೀ ॥೫೧॥

ಪದ್ಮಿನೀ ವಸುಧಾ ಪೃಥ್ವೀ ರೋಹಿಣೀ ವಿಂಧ್ಯವಾಸಿನೀ ।\\
ಶಿವಶಕ್ತಿರ್ಮಹಾಶಕ್ತಿಃ ಶಂಖಿನೀ ಶಕ್ತಿನಿರ್ಗತಾ ॥೫೨॥

ದೈತ್ಯಪ್ರಾಣಹರಾ ದೇವೀ ಸರ್ವರಕ್ಷಣಕಾರಿಣೀ ।\\
ಕ್ಷಾಂತಿಃ ಕ್ಷೇಮಂಕರೀ ಚೈವ ಬುದ್ಧಿರೂಪಾ ಮಹಾಧನಾ ॥೫೩॥

ಶ್ರೀವಿದ್ಯಾ ಭೈರವಿ ಭವ್ಯಾ ಭವಾನೀ ಭವನಾಶಿನೀ ।\\
ತಾಪಿನೀ ಭಾವಿನೀ ಸೀತಾ ತೀಕ್ಷ್ಣತೇಜಃಸ್ವರೂಪಿಣೀ ॥೫೪॥

ದಾತ್ರೀ ದಾನಪರಾ ಕಾಲೀ ದುರ್ಗಾ ದೈತ್ಯವಿಭೂಷಣಾ ।\\
ಮಹಾಪುಣ್ಯಪ್ರದಾ ಭೀಮಾ ಮಧುಕೈಟಭನಾಶಿನೀ ॥೫೫॥

ಪದ್ಮಾ ಪದ್ಮಾವತೀ ಕೃಷ್ಣಾ ತುಷ್ಟಾ ಪುಷ್ಟಾ ತಥೋರ್ವಶೀ ।\\
ವಜ್ರಿಣೀ ವಜ್ರಹಸ್ತಾ ಚ ತಥಾ ನಾರಾಯಣೀ ಶಿವಾ ॥೫೬॥

ಖಡ್ಗಿನೀ ಖಡ್ಗಹಸ್ತಾ ಚ ಖಡ್ಗಖರ್ಪರಧಾರಿಣೀ ।\\
ದೇವಾಂಗನಾ ದೇವಕನ್ಯಾ ದೇವಮಾತಾ ಪುಲೋಮಜಾ ॥೫೭॥

ಸುಖಿನೀ ಸ್ವರ್ಗದಾತ್ರೀ ಚ ಸರ್ವಸೌಖ್ಯವಿವರ್ಧಿನೀ ।\\
ಶೀಲಾ ಶೀಲಾವತೀ ಸೂಕ್ಷ್ಮಾ ಸೂಕ್ಷ್ಮಾಕಾರಾ ವರಪ್ರದಾ ॥೫೮॥

ವರೇಣ್ಯಾ ವರದಾ ವಾಣೀ ಜ್ಞಾನಿನೀ ಜ್ಞಾನದಾ ಸದಾ ।\\
ಉಗ್ರಕಾಲೀ ಮಹಾಕಾಲೀ ಭದ್ರಕಾಲೀ ಚ ದಕ್ಷಿಣಾ ॥೫೯॥

ಭೃಗುವಂಶಸಮುದ್ಭೂತಾ ಭಾರ್ಗವೀ ಭೃಗುವಲ್ಲಭಾ ।\\
ಶೂಲಿನೀ ಶೂಲಹಸ್ತಾ ಚ ಕರ್ತ್ರೀಖರ್ಪರಧಾರಿಣೀ ॥೬೦॥

ಮಹಾವಂಶಸಮುದ್ಭೂತಾ ಮಯೂರವರವಾಹನಾ ।\\
ಮಹಾಶಂಖರತಾ ರಕ್ತಾ ರಕ್ತಖರ್ಪರಧಾರಿಣೀ ॥೬೧॥

ರಕ್ತಾಂಬರಧರಾ ರಾಮಾ ರಮಣೀ ಸುರನಾಯಿಕಾ ।\\
ಮೋಕ್ಷದಾ ಶಿವದಾ ಶ್ಯಾಮಾ ಮದವಿಭ್ರಮಮಂಥರಾ ॥೬೨॥

ಪರಮಾನಂದದಾ ಜ್ಯೇಷ್ಠಾ ಯೋಗಿನೀ ಗಣಸೇವಿತಾ ।\\
ಸಾರಾ ಜಾಂಬವತೀ ಚೈವ ಸತ್ಯಭಾಮಾ ನಗಾತ್ಮಜಾ ॥೬೩॥

ರೌದ್ರಾ ರೌದ್ರಬಲಾ ಘೋರಾ ರುದ್ರಸಾರಾರುಣಾತ್ಮಿಕಾ ।\\
ರುದ್ರರೂಪಾ ಮಹಾರೌದ್ರೀ ರೌದ್ರದೈತ್ಯವಿನಾಶಿನೀ ॥೬೪॥

ಕೌಮಾರೀ ಕೌಶಿಕೀ ಚಂಡಾ ಕಾಲದೈತ್ಯವಿನಾಶಿನೀ ।\\
ಶಂಭುಪತ್ನೀ ಶಂಭುರತಾ ಶಂಬುಜಾಯಾ ಮಹೋದರೀ ॥೬೫॥

ಶಿವಪತ್ನೀ ಶಿವರತಾ ಶಿವಜಾಯಾ ಶಿವಪ್ರಿಯಾ ।\\
ಹರಪತ್ನೀ ಹರರತಾ ಹರಜಾಯಾ ಹರಪ್ರಿಯಾ ॥೬೬॥

ಮದನಾಂತಕಕಾಂತಾ ಚ ಮದನಾಂತಕವಲ್ಲಭಾ ।\\
ಗಿರಿಜಾ ಗಿರಿಕನ್ಯಾ ಚ ಗಿರೀಶಸ್ಯ ಚ ವಲ್ಲಭಾ ॥೬೭॥

ಭೂತಾ ಭವ್ಯಾ ಭವಾ ಸ್ಪಷ್ಟಾ ಪಾವನೀ ಪರಪಾಲಿನೀ ।\\
ಅದೃಶ್ಯಾ ಚ ವ್ಯಕ್ತರೂಪಾ ಇಷ್ಟಾನಿಷ್ಟಪ್ರವರ್ದ್ಧಿನೀ ॥೬೮॥

ಅಚ್ಯುತಾ ಪ್ರಚ್ಯುತಪ್ರಾಣಾ ಪ್ರಮದಾ ವಾಸವೇಶ್ವರೀ ।\\
ಅಪಾಂನಿಧಿಸಮುದ್ಭೂತಾ ಧಾರಿಣೀ ಚ ಪ್ರತಿಷ್ಠಿತಾ ॥೬೯॥

ಉದ್ಭವಾ ಕ್ಷೋಭಣಾ ಕ್ಷೇಮಾ ಶ್ರೀಗರ್ಭಾ ಪರಮೇಶ್ವರೀ ।\\
ಕಮಲಾ ಪುಷ್ಪದೇಹಾ ಚ ಕಾಮಿನೀ ಕಂಜಲೋಚನಾ ॥೭೦॥

ಶರಣ್ಯಾ ಕಮಲಾ ಪ್ರೀತಿರ್ವಿಮಲಾನಂದವರ್ಧಿನೀ ।\\
ಕಪರ್ದಿನೀ ಕರಾಲಾ ಚ ನಿರ್ಮಲಾ ದೇವರೂಪಿಣೀ ॥೭೧॥

ಉದೀರ್ಣಭೂಷಣಾ ಭವ್ಯಾ ಸುರಸೇನಾ ಮಹೋದರೀ ।\\
ಶ್ರೀಮತೀ ಶಿಶಿರಾ ನವ್ಯಾ ಶಿಶಿರಾಚಲಕನ್ಯಕಾ ॥೭೨॥

ಸುರಮಾನ್ಯಾ ಸುರಶ್ರೇಷ್ಠಾ ಜ್ಯೇಷ್ಠಾ ಪ್ರಾಣೇಶ್ವರೀ ಸ್ಥಿರಾ ।\\
ತಮೋಘ್ನೀ ಧ್ವಾಂತಸಂಹಂತ್ರೀ ಪ್ರಯತಾತ್ಮಾ ಪತಿವ್ರತಾ ॥೭೩॥

ಪ್ರದ್ಯೋತಿನೀ ರಥಾರೂಢಾ ಸರ್ವಲೋಕಪ್ರಕಾಶಿನೀ ।\\
ಮೇಧಾವಿನೀ ಮಹಾವೀರ್ಯಾ ಹಂಸೀ ಸಂಸಾರತಾರಿಣೀ ॥೭೪॥

ಪ್ರಣತಪ್ರಾಣಿನಾಮಾರ್ತಿಹಾರಿಣೀ ದೈತ್ಯನಾಶಿನೀ ।\\
ಡಾಕಿನೀ ಶಾಕಿನೀದೇವೀ ವರಖಟ್ವಾಂಗಧಾರಿಣೀ ॥೭೫॥

ಕೌಮುದೀ ಕುಮುದಾ ಕುಂದಾ ಕೌಲಿಕಾ ಕುಲಜಾಮರಾ ।\\
ಗರ್ವಿತಾ ಗುಣಸಂಪನ್ನಾ ನಗಜಾ ಖಗವಾಹಿನೀ ॥೭೬॥

ಚಂದ್ರಾನನಾ ಮಹೋಗ್ರಾ ಚ ಚಾರುಮೂರ್ಧಜಶೋಭನಾ ।\\
ಮನೋಜ್ಞಾ ಮಾಧವೀ ಮಾನ್ಯಾ ಮಾನನೀಯಾ ಸತಾಂ ಸುಹೃತ್ ॥೭೭॥

ಜ್ಯೇಷ್ಠಾ ಶ್ರೇಷ್ಠಾ ಮಘಾ ಪುಷ್ಯಾ ಧನಿಷ್ಠಾ ಪೂರ್ವಫಾಲ್ಗುನೀ ।\\
ರಕ್ತಬೀಜನಿಹಂತ್ರೀ ಚ ರಕ್ತಬೀಜವಿನಾಶಿನೀ ॥೭೮॥

ಚಂಡಮುಂಡನಿಹಂತ್ರೀ ಚ ಚಂಡಮುಂಡವಿನಾಶಿನೀ ।\\
ಕರ್ತ್ರೀ ಹರ್ತ್ರೀ ಸುಕರ್ತ್ರೀ ಚ ವಿಮಲಾಮಲವಾಹಿನೀ ॥೭೯॥

ವಿಮಲಾ ಭಾಸ್ಕರೀ ವೀಣಾ ಮಹಿಷಾಸುರಘಾತಿನೀ ।\\
ಕಾಲಿಂದೀ ಯಮುನಾ ವೃದ್ಧಾ ಸುರಭಿಃ ಬಾಲಿಕಾ ಸತೀ ॥೮೦॥

ಕೌಶಲ್ಯಾ ಕೌಮುದೀ ಮೈತ್ರೀರೂಪಿಣೀ ಚಾಪ್ಯರುಂಧತೀ ।\\
ಪುರಾರಿಗೃಹಿಣೀ ಪೂರ್ಣಾ ಪೂರ್ಣಾನಂದಸ್ವರೂಪಿಣೀ ॥೮೧॥

ಪುಂಡರೀಕಾಕ್ಷಪತ್ನೀ ಚ ಪುಂಡರೀಕಾಕ್ಷವಲ್ಲಭಾ ।\\
ಸಂಪೂರ್ಣಚಂದ್ರವದನಾ ಬಾಲಚಂದ್ರಸಮಪ್ರಭಾ ॥೮೨॥

ರೇವತೀ ರಮಣೀ ಚಿತ್ರಾ ಚಿತ್ರಾಂಬರವಿಭೂಷಣಾಂ ।\\
ಸೀತಾ ವೀಣಾವತೀ ಚೈವ ಯಶೋದಾ ವಿಜಯಾ ಪ್ರಿಯಾ ॥೮೩॥

ನವಪುಷ್ಪಸಮುದ್ಭೂತಾ ನವಪುಷ್ಪೋತ್ಸವೋತ್ಸವಾ ।\\
ನವಪುಷ್ಪಸ್ರಜಾಮಾಲಾ ಮಾಲ್ಯಭೂಷಣಭೂಷಿತಾ ॥೮೪॥

ನವಪುಷ್ಪಸಮಪ್ರಾಣಾ ನವಪುಷ್ಪೋತ್ಸವಪ್ರಿಯಾ ।\\
ಪ್ರೇತಮಂಡಲಮಧ್ಯಸ್ತಾ ಸರ್ವಾಂಗಸುಂದರೀ ಶಿವಾ ॥೮೫॥

ನವಪುಷ್ಪಾತ್ಮಿಕಾ ಷಷ್ಠೀ ಪುಷ್ಪಸ್ತವಕಮಂಡಲಾ ।\\
ನವಪುಷ್ಪಗುಣೋಪೇತಾ ಶ್ಮಶಾನಭೈರವಪ್ರಿಯಾ ॥೮೬॥

ಕುಲಶಾಸ್ತ್ರಪ್ರದೀಪಾ ಚ ಕುಲಮಾರ್ಗಪ್ರವರ್ದ್ಧಿನೀ ।\\
ಶ್ಮಶಾನಭೈರವೀ ಕಾಲೀ ಭೈರವೀ ಭೈರವಪ್ರಿಯಾ ॥೮೭॥

ಆನಂದಭೈರವೀ ಧ್ಯೇಯಾ ಭೈರವೀ ಕುರುಭೈರವೀ ।\\
ಮಹಾಭೈರವಸಂಪ್ರೀತಾ ಭೈರವೀಕುಲಮೋಹಿನೀ ॥೮೮॥

ಶ್ರೀವಿದ್ಯಾಭೈರವೀ ನೀತಿಭೈರವೀ ಗುಣಭೈರವೀ ।\\
ಸಮ್ಮೋಹಭೈರವೀ ಪುಷ್ಟಿಭೈರವೀ ತುಷ್ಟಿಭೈರವೀ ॥೮೯॥

ಸಂಹಾರಭೈರವೀ ಸೃಷ್ಟಿಭೈರವೀ ಸ್ಥಿತಿಭೈರವೀ ।\\
ಆನಂದಭೈರವೀ ವೀರಾ ಸುಂದರೀ ಸ್ಥಿತಿಸುಂದರೀ ॥೯೦॥

ಗುಣಾನಂದಸ್ವರೂಪಾ ಚ ಸುಂದರೀ ಕಾಲರೂಪಿಣೀ ।\\
ಶ್ರೀಮಾಯಾಸುಂದರೀ ಸೌಮ್ಯಸುಂದರೀ ಲೋಕಸುಂದರೀ ॥೯೧॥

ಶ್ರೀವಿದ್ಯಾಮೋಹಿನೀ ಬುದ್ಧಿರ್ಮಹಾಬುದ್ಧಿಸ್ವರೂಪಿಣೀ ।\\
ಮಲ್ಲಿಕಾ ಹಾರರಸಿಕಾ ಹಾರಾಲಂಬನಸುಂದರೀ ॥೯೨॥

ನೀಲಪಂಕಜವರ್ಣಾ ಚ ನಾಗಕೇಸರಭೂಷಿತಾ ।\\
ಜಪಾಕುಸುಮಸಂಕಾಶಾ ಜಪಾಕುಸುಮಶೋಭಿತಾ ॥೯೩॥

ಪ್ರಿಯಾ ಪ್ರಿಯಂಕರೀ ವಿಷ್ಣೋರ್ದಾನವೇಂದ್ರವಿನಾಶಿನೀ ।\\
ಜ್ಞಾನೇಶ್ವರೀ ಜ್ಞಾನದಾತ್ರೀ ಜ್ಞಾನಾನಂದಪ್ರದಾಯಿನೀ ॥೯೪॥

ಗುಣಗೌರವಸಂಪನ್ನಾ ಗುಣಶೀಲಸಮನ್ವಿತಾ ।\\
ರೂಪಯೌವನಸಂಪನ್ನಾ ರೂಪಯೌವನಶೋಭಿತಾ ॥೯೫॥

ಗುಣಾಶ್ರಯಾ ಗುಣರತಾ ಗುಣಗೌರವಸುಂದರೀ ।\\
ಮದಿರಾಮೋದಮತ್ತಾ ಚ ತಾಟಂಕದ್ವಯಶೋಭಿತಾ ॥೯೬॥

ವೃಕ್ಷಮೂಲಸ್ಥಿತಾ ದೇವೀ ವೃಕ್ಷಶಾಖೋಪರಿಸ್ಥಿತಾ ।\\
ತಾಲಮಧ್ಯಾಗ್ರನಿಲಯಾ ವೃಕ್ಷಮಧ್ಯನಿವಾಸಿನೀ ॥೯೭॥

ಸ್ವಯಂಭೂಪುಷ್ಪಸಂಕಾಶಾ ಸ್ವಯಂಭೂಪುಷ್ಪಧಾರಿಣೀ ।\\
ಸ್ವಯಂಭೂಕುಸುಮಪ್ರೀತಾ ಸ್ವಯಂಭೂಪುಷ್ಪಶೋಭಿನೀ ॥೯೮॥

ಸ್ವಯಂಭೂಪುಷ್ಪರಸಿಕಾ ನಗ್ನಾ ಧ್ಯಾನವತೀ ಸುಧಾ ।\\
ಶುಕ್ರಪ್ರಿಯಾ ಶುಕ್ರರತಾ ಶುಕ್ರಮಜ್ಜನತತ್ಪರಾ ॥೯೯॥

ಪೂರ್ಣಪರ್ಣಾ ಸುಪರ್ಣಾ ಚ ನಿಷ್ಪರ್ಣಾ ಪಾಪನಾಶಿನೀ ।\\
ಮದಿರಾಮೋದಸಂಪನ್ನಾ ಮದಿರಾಮೋದಧಾರಿಣೀ ॥೧೦೦॥

ಸರ್ವಾಶ್ರಯಾ ಸರ್ವಗುಣಾ ನಂದನಂದನಧಾರಿಣೀ ।\\
ನಾರೀಪುಷ್ಪಸಮುದ್ಭೂತಾ ನಾರೀಪುಷ್ಪೋತ್ಸವೋತ್ಸವಾ ॥೧೦೧॥

ನಾರೀಪುಷ್ಪಸಮಪ್ರಾಣಾ ನಾರೀಪುಷ್ಪರತಾ ಮೃಗೀ ।\\
ಸರ್ವಕಾಲೋದ್ಭವಪ್ರೀತಾ ಸರ್ವಕಾಲೋದ್ಭವೋತ್ಸವಾ ॥೧೦೨॥

ಚತುರ್ಭುಜಾ ದಶಭುಜಾ ಅಷ್ಟಾದಶಭುಜಾ ತಥಾ ।\\
ದ್ವಿಭುಜಾ ಷಡ್ಭುಜಾ ಪ್ರೀತಾ ರಕ್ತಪಂಕಜಶೋಭಿತಾ ॥೧೦೩॥

ಕೌಬೇರೀ ಕೌರವೀ ಕೌರ್ಯಾ ಕುರುಕುಲ್ಲಾ ಕಪಾಲಿನೀ ।\\
ಸುದೀರ್ಘಕದಲೀಜಂಘಾ ರಂಭೋರೂ ರಾಮವಲ್ಲಭಾ ॥೧೦೪॥

ನಿಶಾಚರೀ ನಿಶಾಮೂರ್ತಿರ್ನಿಶಾಚಂದ್ರಸಮಪ್ರಭಾ ।\\
ಚಾಂದ್ರೀ ಚಾಂದ್ರಕಲಾ ಚಂದ್ರಾ ಚಾರುಚಂದ್ರನಿಭಾನನಾ ॥೧೦೫॥

ಸ್ರೋತಸ್ವತೀ ಸ್ರುತಿಮತೀ ಸರ್ವದುರ್ಗತಿನಾಶಿನೀ ।\\
ಸರ್ವಾಧಾರಾ ಸರ್ವಮಯೀ ಸರ್ವಾನಂದಸ್ವರೂಪಿಣೀ ॥೧೦೬॥

ಸರ್ವಚಕ್ರೇಶ್ವರೀ ಸರ್ವಾ ಸರ್ವಮಂತ್ರಮಯೀ ಶುಭಾ ।\\
ಸಹಸ್ರನಯನಪ್ರಾಣಾ ಸಹಸ್ರನಯನಪ್ರಿಯಾ ॥೧೦೭॥

ಸಹಸ್ರಶೀರ್ಷಾ ಸುಷಮಾ ಸದಂಭಾ ಸರ್ವಭಕ್ಷಿಕಾ ।\\
ಯಷ್ಟಿಕಾ ಯಷ್ಟಿಚಕ್ರಸ್ಥಾ ಷದ್ವರ್ಗಫಲದಾಯಿನೀ ॥೧೦೮॥

ಷಡ್ವಿಂಶಪದ್ಮಮಧ್ಯಸ್ಥಾ ಷಡ್ವಿಂಶಕುಲಮಧ್ಯಗಾ ।\\
ಹೂಁಕಾರವರ್ಣನಿಲಯಾ ಹೂಁಕಾರಾಕ್ಷರಭೂಷಣಾ ॥೧೦೯॥

ಹಕಾರವರ್ಣನಿಲಯಾ ಹಕಾರಾಕ್ಷರಭೂಷಣಾ ।\\
ಹಾರಿಣೀ ಹಾರವಲಿತಾ ಹಾರಹೀರಕಭೂಷಣಾ ॥೧೧೦॥

ಹ್ರೀಂಕಾರಬೀಜಸಹಿತಾ ಹ್ರೀಂಕಾರೈರುಪಶೋಭಿತಾ ।\\
ಕಂದರ್ಪಸ್ಯ ಕಲಾ ಕುಂದಾ ಕೌಲಿನೀ ಕುಲದರ್ಪಿತಾ ॥೧೧೧॥

ಕೇತಕೀಕುಸುಮಪ್ರಾಣಾ ಕೇತಕೀಕೃತಭೂಷಣಾ ।\\
ಕೇತಕೀಕುಸುಮಾಸಕ್ತಾ ಕೇತಕೀಪರಿಭೂಷಿತಾ ॥೧೧೨॥

ಕರ್ಪೂರಪೂರ್ಣವದನಾ ಮಹಾಮಾಯಾ ಮಹೇಶ್ವರೀ ।\\
ಕಲಾ ಕೇಲಿಃ ಕ್ರಿಯಾ ಕೀರ್ಣಾ ಕದಂಬಕುಸುಮೋತ್ಸುಕಾ ॥೧೧೩॥

ಕಾದಂಬಿನೀ ಕರಿಶುಂಡಾ ಕುಂಜರೇಶ್ವರಗಾಮಿನೀ ।\\
ಖರ್ವಾ ಸುಖಂಜನಯನಾ ಖಂಜನದ್ವಂದ್ವಭೂಷಣಾ ॥೧೧೪॥

ಖದ್ಯೋತ ಇವ ದುರ್ಲಕ್ಷಾ ಖದ್ಯೋತ ಇವ ಚಂಚಲಾ ।\\
ಮಹಾಮಾಯಾ ಜ್ಗದ್ಧಾತ್ರೀ ಗೀತವಾದ್ಯಪ್ರಿಯಾ ರತಿಃ ॥೧೧೫॥

ಗಣೇಶ್ವರೀ ಗಣೇಜ್ಯಾ ಚ ಗುಣಪೂಜ್ಯಾ ಗುಣಪ್ರದಾ ।\\
ಗುಣಾಢ್ಯಾ ಗುಣಸಂಪನ್ನಾ ಗುಣದಾತ್ರೀ ಗುಣಾತ್ಮಿಕಾ ॥೧೧೬॥

ಗುರ್ವೀ ಗುರುತರಾ ಗೌರೀ ಗಾಣಪತ್ಯಫಲಪ್ರದಾ ।\\
ಮಹಾವಿದ್ಯಾ ಮಹಾಮೇಧಾ ತುಲಿನೀ ಗಣಮೋಹಿನೀ ॥೧೧೭॥

ಭವ್ಯಾ ಭವಪ್ರಿಯಾ ಭಾವ್ಯಾ ಭಾವನೀಯಾ ಭವಾತ್ಮಿಕಾ ।\\
ಘರ್ಘರಾ ಘೋರವದನಾ ಘೋರದೈತ್ಯವಿನಾಶಿನೀ ॥೧೧೮॥

ಘೋರಾ ಘೋರವತೀ ಘೋಷಾ ಘೋರಪುತ್ರೀ ಘನಾಚಲಾ ।\\
ಚರ್ಚರೀ ಚಾರುನಯನಾ ಚಾರುವಕ್ತ್ರಾ ಚತುರ್ಗುಣಾ ॥೧೧೯॥

ಚತುರ್ವೇದಮಯೀ ಚಂಡೀ ಚಂದ್ರಾಸ್ಯಾ ಚತುರಾನನಾ ।\\
ಚಲಚ್ಚಕೋರನಯನಾ ಚಲತ್ಖಂಜನಲೋಚನಾ ॥೧೨೦॥

ಚಲದಂಭೋಜನಿಲಯಾ ಚಲದಂಭೋಜಲೋಚನಾ ।\\
ಛತ್ರೀ ಛತ್ರಪ್ರಿಯಾ ಛತ್ರಾ ಛತ್ರಚಾಮರಶೋಭಿತಾ ॥೧೨೧॥

ಛಿನ್ನಛದಾ ಛಿನ್ನಶಿರಾಶ್ಛಿನ್ನನಾಸಾ ಛಲಾತ್ಮಿಕಾ ।\\
ಛಲಾಢ್ಯಾ ಛಲಸಂತ್ರಸ್ತಾ ಛಲರೂಪಾ ಛಲಸ್ಥಿರಾ ॥೧೨೨॥

ಛಕಾರವರ್ಣನಿಲಯಾ ಛಕಾರಾಢ್ಯಾ ಛಲಪ್ರಿಯಾ ।\\
ಛದ್ಮಿನೀ ಛದ್ಮನಿರತಾ ಛದ್ಮಚ್ಛದ್ಮನಿವಾಸಿನೀ ॥೧೨೩॥

ಜಗನ್ನಾಥಪ್ರಿಯಾ ಜೀವಾ ಜಗನ್ಮುಕ್ತಿಕರೀ ಮತಾ ।\\
ಜೀರ್ಣಾ ಜೀಮೂತವನಿತಾ ಜೀಮೂತೈರುಪಶೋಭಿತಾ ॥೧೨೪॥

ಜಾಮಾತೃವರದಾ ಜಂಭಾ ಜಮಲಾರ್ಜುನಭಂಜಿನೀ ।\\
ಝರ್ಝರೀ ಝಾಕೃತಿರ್ಝಲ್ಲೀ ಝರೀ ಝರ್ಝರಿಕಾ ತಥಾ ॥೧೨೫॥

ಟಂಕಾರಕಾರಿಣೀ ಟೀಕಾ ಸರ್ವಟಂಕಾರಕಾರಿಣೀ ।\\
ಠಂಕರಾಂಗೀ ಡಮರುಕಾ ಡಾಕಾರಾ ಡಮರುಪ್ರಿಯಾ ॥೧೨೬॥

ಢಕ್ಕಾರಾವರತಾ ನಿತ್ಯಾ ತುಲಸೀ ಮಣಿಭೂಷಿತಾ ।\\
ತುಲಾ ಚ ತೋಲಿಕಾ ತೀರ್ಣಾ ತಾರಾ ತಾರಣಿಕಾ ತಥಾ ॥೧೨೭॥

ತಂತ್ರವಿಜ್ಞಾ ತಂತ್ರರತಾ ತಂತ್ರವಿದ್ಯಾ ಚ ತಂತ್ರದಾ ।\\
ತಾಂತ್ರಿಕೀ ತಂತ್ರಯೋಗ್ಯಾ ಚ ತಂತ್ರಸಾರಾ ಚ ತಂತ್ರಿಕಾ ॥೧೨೮॥

ತಂತ್ರಧಾರೀ ತಂತ್ರಕರೀ ಸರ್ವತಂತ್ರಸ್ವರೂಪಿಣೀ ।\\
ತುಹಿನಾಂಶುಸಮಾನಾಸ್ಯಾ ತುಹಿನಾಂಶುಸಮಪ್ರಭಾ ॥೧೨೯॥

ತುಷಾರಾಕರತುಲ್ಯಾಂಗೀ ತುಷಾರಾಧಾರಸುಂದರೀ ।\\
ತಂತ್ರಸಾರಾ ತಂತ್ರಕರೋ ತಂತ್ರಸಾರಸ್ವರೂಪಿಣೀ ॥೧೩೦॥

ತುಷಾರಧಾಮತುಲ್ಯಾಸ್ಯಾ ತುಷಾರಾಂಶುಸಮಪ್ರಭಾ ।\\
ತುಷಾರಾದ್ರಿಸುತಾ ತಾರ್ಕ್ಷ್ಯಾ ತಾರಾಂಗೀ ತಾಲಸುಂದರೀ ॥೧೩೧॥

ತಾರಸ್ವರೇಣ ಸಹಿತಾ ತಾರಸ್ವರವಿಭೂಷಿತಾ ।\\
ಥಕಾರಕೂಟನಿಲಯಾ ಥಕಾರಾಕ್ಷರಮಾಲಿನೀ ॥೧೩೨॥

ದಯಾವತೀ ದೀನರತಾ ದುಃಖದಾರಿದ್ರ್ಯನಾಶಿನೀ ।\\
ದೌರ್ಭಾಗ್ಯದುಃಖದಲಿನೀ ದೌರ್ಭಾಗ್ಯಪದನಾಶಿನೀ ॥೧೩೩॥

ದುಹಿತಾ ದೀನಬಂಧುಶ್ಚ ದಾನವೇಂದ್ರವಿನಾಶಿನೀ ।\\
ದಾನಪಾತ್ರೀ ದಾನರತಾ ದಾನಸಮ್ಮಾನತೋಷಿತಾ ॥೧೩೪॥

ದಾಂತ್ಯಾದಿಸೇವಿತಾ ದಾಂತಾ ದಯಾ ದಾಮೋದರಪ್ರಿಯಾ ।\\
ದಧೀಚಿವರದಾ ತುಷ್ಟಾ ದಾನವೇಂದ್ರವಿಮರ್ದಿನೀ ॥೧೩೫॥

ದೀರ್ಘನೇತ್ರಾ ದೀರ್ಘಕಚಾ ದೀರ್ಘನಾಸಾ ಚ ದೀರ್ಘಿಕಾ ।\\
ದಾರಿದ್ರ್ಯದುಃಖಸಂನಾಶಾ ದಾರಿದ್ರ್ಯದುಃಖನಾಶಿನೀ ॥೧೩೬॥

ದಾಂಭಿಕಾ ದಂತುರಾ ದಂಭಾ ದಂಭಾಸುರವರಪ್ರದಾ ।\\
ಧನಧಾನ್ಯಪ್ರದಾ ಧನ್ಯಾ ಧನೇಶ್ವರಧನಪ್ರದಾ ॥೧೩೭॥

ಧರ್ಮಪತ್ನೀ ಧರ್ಮರತಾ ಧರ್ಮಾಧರ್ಮವಿವಿವರ್ದ್ಧಿನೀ ।\\
ಧರ್ಮಿಣೀ ಧರ್ಮಿಕಾ ಧರ್ಮ್ಯಾ ಧರ್ಮಾಧರ್ಮವಿವರ್ದ್ಧಿನೀ ॥೧೩೮॥

ಧನೇಶ್ವರೀ ಧರ್ಮರತಾ ಧರ್ಮಾನಂದಪ್ರವರ್ದ್ಧಿನೀ ।\\
ಧನಾಧ್ಯಕ್ಷಾ ಧನಪ್ರೀತಾ ಧನಾಢ್ಯಾ ಧನತೋಷಿತಾ ॥೧೩೯॥

ಧೀರಾ ಧೈರ್ಯವತೀ ಧಿಷ್ಣ್ಯಾ ಧವಲಾಂಭೋಜಸಂನಿಭಾ ।\\
ಧರಿಣೀ ಧಾರಿಣೀ ಧಾತ್ರೀ ಧೂರಣೀ ಧರಣೀ ಧರಾ ॥೧೪೦॥

ಧಾರ್ಮಿಕಾ ಧರ್ಮಸಹಿತಾ ಧರ್ಮನಿಂದಕವರ್ಜಿತಾ ।\\
ನವೀನಾ ನಗಜಾ ನಿಮ್ನಾ ನಿಮ್ನನಾಭಿರ್ನಗೇಶ್ವರೀ ॥೧೪೧॥

ನೂತನಾಂಭೋಜನಯನಾ ನವೀನಾಂಭೋಜಸುಂದರೀ ।\\
ನಾಗರೀ ನಗರಜ್ಯೇಷ್ಠಾ ನಗರಾಜಸುತಾ ನಗಾ ॥೧೪೨॥

ನಾಗರಾಜಕೃತತೋಷಾ ನಾಗರಾಜವಿಭೂಷಿತಾ ।\\
ನಾಗೇಶ್ವರೀ ನಾಗರೂಢಾ ನಾಗರಾಜಕುಲೇಶ್ವರೀ ॥೧೪೩॥

ನವೀನೇಂದುಕಲಾ ನಾಂದೀ ನಂದಿಕೇಶ್ವರವಲ್ಲಭಾ ।\\
ನೀರಜಾ ನೀರಜಾಕ್ಷೀ ಚ ನೀರಜದ್ವಂದ್ವಲೋಚನಾ ॥೧೪೪॥

ನೀರಾ ನೀರಭವಾ ವಾಣೀ ನೀರನಿರ್ಮಲದೇಹಿನೀ ।\\
ನಾಗಯಜ್ಞೋಪವೀತಾಢ್ಯಾ ನಾಗಯಜ್ಞೋಪವೀತಿಕಾ ॥೧೪೫॥

ನಾಗಕೇಸರಸಂತುಷ್ಟಾ ನಾಗಕೇಸರಮಾಲಿನೀ ।\\
ನವೀನಕೇತಕೀಕುಂದ ಮಲ್ಲಿಕಾಂಭೋಜಭೂಷಿತಾ ॥೧೪೬॥

ನಾಯಿಕಾ ನಾಯಕಪ್ರೀತಾ ನಾಯಕಪ್ರೇಮಭೂಷಿತಾ ।\\
ನಾಯಕಪ್ರೇಮಸಹಿತಾ ನಾಯಕಪ್ರೇಮಭಾವಿತಾ ॥೧೪೭॥

ನಾಯಕಾನಂದನಿಲಯಾ ನಾಯಕಾನಂದಕಾರಿಣೀ ।\\
ನರ್ಮಕರ್ಮರತಾ ನಿತ್ಯಂ ನರ್ಮಕರ್ಮಫಲಪ್ರದಾ ॥೧೪೮॥

ನರ್ಮಕರ್ಮಪ್ರಿಯಾ ನರ್ಮಾ ನರ್ಮಕರ್ಮಕೃತಾಲಯಾ ।\\
ನರ್ಮಪ್ರೀತಾ ನರ್ಮರತಾ ನರ್ಮಧ್ಯಾನಪರಾಯಣಾ ॥೧೪೯॥

ಪೌಷ್ಣಪ್ರಿಯಾ ಚ ಪೌಷ್ಪೇಜ್ಯಾ ಪುಷ್ಪದಾಮವಿಭೂಷಿತಾ ।\\
ಪುಣ್ಯದಾ ಪೂರ್ಣಿಮಾ ಪೂರ್ಣಾ ಕೋಟಿಪುಣ್ಯಫಲಪ್ರದಾ ॥೧೫೦॥

ಪುರಾಣಾಗಮಗೋಪ್ಯಾ ಚ ಪುರಾಣಾಗಮಗೋಪಿತಾ ।\\
ಪುರಾಣಗೋಚರಾ ಪೂರ್ಣಾ ಪೂರ್ವಾ ಪ್ರೌಢಾ ವಿಲಾಸಿನೀ ॥೧೫೧॥

ಪ್ರಹ್ಲಾದಹೃದಯಾಹ್ಲಾದಗೇಹಿನೀ ಪುಣ್ಯಚಾರಿಣೀ ।\\
ಫಾಲ್ಗುನೀ ಫಾಲ್ಗುನಪ್ರೀತಾ ಫಾಲ್ಗುನಪ್ರೇಧಾರಿಣೀ ॥೧೫೨॥

ಫಾಲ್ಗುನಪ್ರೇಮದಾ ಚೈವ ಫಣಿರಾಜವಿಭೂಷಿತಾ ।\\
ಫಣಿಕಾಂಚೀ ಫಣಿಪ್ರೀತಾ ಫಣಿಹಾರವಿಭೂಷಿತಾ ॥೧೫೩॥

ಫಣೀಶಕೃತಸರ್ವಾಂಗಭೂಷಣಾ ಫಣಿಹಾರಿಣೀ ।\\
ಫಣಿಪ್ರೀತಾ ಫಣಿರತಾ ಫಣಿಕಂಕಣಧಾರಿಣೀ ॥೧೫೪॥

ಫಲದಾ ತ್ರಿಫಲಾ ಶಕ್ತಾ ಫಲಾಭರಣಭೂಷಿತಾ ।\\
ಫಕಾರಕೂಟಸರ್ವಾಂಗೀ ಫಾಲ್ಗುನಾನಂದವರ್ದ್ಧಿನೀ ॥೧೫೫॥

ವಾಸುದೇವರತಾ ವಿಜ್ಞಾ ವಿಜ್ಞವಿಜ್ಞಾನಕಾರಿಣೀ ।\\
ವೀಣಾವತೀ ಬಲಾಕೀರ್ಣಾ ಬಾಲಪೀಯೂಷರೋಚಿಕಾ ॥೧೫೬॥

ಬಾಲಾವಸುಮತೀ ವಿದ್ಯಾ ವಿದ್ಯಾಹಾರವಿಭೂಷಿತಾ ।\\
ವಿದ್ಯಾವತೀ ವೈದ್ಯಪದಪ್ರೀತಾ ವೈವಸ್ವತೀ ಬಲಿಃ ॥೧೫೭॥

ಬಲಿವಿಧ್ವಂಸಿನೀ ಚೈವ ವರಾಂಗಸ್ಥಾ ವರಾನನಾ ।\\
ವಿಷ್ಣೋರ್ವಕ್ಷಃಸ್ಥಲಸ್ಥಾ ಚ ವಾಗ್ವತೀ ವಿಂಧ್ಯವಾಸಿನೀ ॥೧೫೮॥

ಭೀತಿದಾ ಭಯದಾ ಭಾನೋರಂಶುಜಾಲಸಮಪ್ರಭಾ ।\\
ಭಾರ್ಗವೇಜ್ಯಾ ಭೃಗೋಃ ಪೂಜ್ಯಾ ಭರದ್ವಾರನಮಸ್ಕೃತಾ ॥೧೫೯॥

ಭೀತಿದಾ ಭಯಸಂಹಂತ್ರೀ ಭೀಮಾಕಾರಾ ಚ ಸುಂದರೀ ।\\
ಮಾಯಾವತೀ ಮಾನರತಾ ಮಾನಸಮ್ಮಾನತತ್ಪರಾ ॥೧೬೦॥

ಮಾಧವಾನಂದದಾ ಮಾಧ್ವೀ ಮದಿರಾಮುದಿತೇಕ್ಷಣಾ ।\\
ಮಹೋತ್ಸವಗುಣೋಪೇತಾ ಮಹತೀ ಚ ಮಹದ್ಗುಣಾ ॥೧೬೧॥

ಮದಿರಾಮೋದನಿರತಾ ಮದಿರಾಮಜ್ಜನೇ ರತಾ ।\\
ಯಶೋಧರೀ ಯಶೋವಿದ್ಯಾ ಯಶೋದಾನಂದವರ್ದ್ಧಿನೀ ॥೧೬೨॥

ಯಶಃಕರ್ಪೂರಧವಲಾ ಯಶೋದಾಮವಿಭೂಷಿತಾ ।\\
ಯಮರಾಜಪ್ರಿಯಾ ಯೋಗಮಾರ್ಗಾನಂದಪ್ರವರ್ದ್ಧಿನೀ ॥೧೬೩॥

ಯಮಸ್ವಸಾ ಚ ಯಮುನಾ ಯೋಗಮಾರ್ಗಪ್ರವರ್ದ್ಧಿನೀ ।\\
ಯಾದವಾನಂದಕರ್ತ್ರೀ ಚ ಯಾದವಾನಂದವರ್ದ್ಧಿನೀ ॥೧೬೪॥

ಯಜ್ಞಪ್ರೀತಾ ಯಜ್ಞಮಯೀ ಯಜ್ಞಕರ್ಮವಿಭೂಷಿತಾ ।\\
ರಾಮಪ್ರೀತಾ ರಾಮರತಾ ರಾಮತೋಷಣತತ್ಪರಾ ॥೧೬೫॥

ರಾಜ್ಞೀ ರಾಜಕುಲೇಜ್ಯಾ ಚ ರಾಜರಾಜೇಶ್ವರೀ ರಮಾ ।\\
ರಮಣೀ ರಾಮಣೀ ರಮ್ಯಾ ರಾಮಾನಂದಪ್ರದಾಯಿನೀ ॥೧೬೬॥

ರಜನೀಕರಪೂರ್ಣಾಸ್ಯಾ ರಕ್ತೋತ್ಪಲವಿಲೋಚನಾ ।\\
ಲಾಂಗಲಿಪ್ರೇಮಸಂತುಷ್ಟಾ ಲಾಂಗಲಿಪ್ರಣಯಪ್ರಿಯಾ ॥೧೬೭॥

ಲಾಕ್ಷಾರುಣಾ ಚ ಲಲನಾ ಲೀಲಾ ಲೀಲಾವತೀ ಲಯಾ ।\\
ಲಂಕೇಶ್ವರಗುಣಪ್ರೀತಾ ಲಂಕೇಶವರದಾಯಿನೀ ॥೧೬೮॥

ಲವಂಗೀಕುಸುಮಪ್ರೀತಾ ಲವಂಗಕುಸುಮಸ್ರಜಾ ।\\
ಧಾತಾ ವಿವಸ್ವದ್ಗೃಹಿಣೀ ವಿವಸ್ವತ್ಪ್ರೇಮಧಾರಿಣೀ ॥೧೬೯॥

ಶವೋಪರಿಸಮಾಸೀನಾ ಶವವಕ್ಷಃಸ್ಥಲಸ್ಥಿತಾ ।\\
ಶರಣಾಗತರಕ್ಷಿತ್ರೀ ಶರಣ್ಯಾ ಶ್ರೀಃ ಶರದ್ಗುಣಾ ॥೧೭೦॥

ಷಟ್ಕೋಣಚಕ್ರಮಧ್ಯಸ್ಥಾ ಸಂಪದಾರ್ಥನಿಷೇವಿತಾ ।\\
ಹೂಂಕಾರಾಕಾರಿಣೀ ದೇವೀ ಹೂಂಕಾರರೂಪಶೋಭಿತಾ ॥೧೭೧॥

ಕ್ಷೇಮಂಕರೀ ತಥಾ ಕ್ಷೇಮಾ ಕ್ಷೇಮಧಾಮವಿವರ್ದ್ಧಿನೀ ।\\
ಕ್ಷೇಮಾಮ್ನಾಯಾ ತಥಾಜ್ಞಾ ಚ ಇಡಾ ಇಶ್ವರವಲ್ಲಭಾ ॥೧೭೨॥

ಉಗ್ರದಕ್ಷಾ ತಥಾ ಚೋಗ್ರಾ ಅಕಾರಾದಿಸ್ವರೋದ್ಭವಾ ।\\
ಋಕಾರವರ್ಣಕೂಟಸ್ಥಾ ೠಕಾರಸ್ವರಭೂಷಿತಾ ॥೧೭೩॥

ಏಕಾರಾ ಚ ತಥಾ ಚೈಕಾ ಏಕಾರಾಕ್ಷರವಾಸಿತಾ ।\\
ಐಷ್ಟಾ ಚೈಷಾ ತಥಾ ಚೌಷಾ ಔಕಾರಾಕ್ಷರಧಾರಿಣೀ ॥೧೭೪॥

ಅಂ ಅಃಕಾರಸ್ವರೂಪಾ ಚ ಸರ್ವಾಗಮಸುಗೋಪಿತಾ ।\\
ಇತ್ಯೇತತ್ ಕಥಿತಂ ದೇವಿ ತಾರಾನಾಮಸಹಸ್ರಕಂ ॥೧೭೫॥

ಯ ಇದಂ ಪಠತಿ ಸ್ತೋತ್ರಂ ಪ್ರತ್ಯಹಂ ಭಕ್ತಿಭಾವತಃ ।\\
ದಿವಾ ವಾ ಯದಿ ವಾ ರಾತ್ರೌ ಸಂಧ್ಯಯೋರುಭಯೋರಪಿ ॥೧೭೬॥

ಸ್ತವರಾಜಸ್ಯ ಪಾಠೇನ ರಾಜಾ ಭವತಿ ಕಿಂಕರಃ ।\\
ಸರ್ವಾಗಮೇಷು ಪೂಜ್ಯಃ ಸ್ಯಾತ್ ಸರ್ವತಂತ್ರೇ ಸ್ವಯಂ ಹರಃ ॥೧೭೭॥

ಶಿವಸ್ಥಾನೇ ಶ್ಮಶಾನೇ ಚ ಶೂನ್ಯಾಗಾರೇ ಚತುಷ್ಪಥೇ ।\\
ಯ ಪಠೇಚ್ಛೃಣುಯಾದ್ ವಾಪಿ ಸ ಯೋಗೀ ನಾತ್ರ ಸಂಶಯಃ ॥೧೭೮॥

ಯಾನಿ ನಾಮಾನಿ ಸಂತ್ಯಸ್ಮಿನ್ ಪ್ರಸಂಗಾದ್ ಮುರವೈರಿಣಃ ।\\
ಗ್ರಾಹ್ಯಾಣಿ ತಾನಿ ಕಲ್ಯಾಣಿ ನಾನ್ಯಾನ್ಯಪಿ ಕದಾಚನ ॥೧೭೯॥

ಹರೇರ್ನಾಮ ನ ಗೃಹ್ಣೀಯಾದ್ ನ ಸ್ಪೃಶೇತ್ ತುಲಸೀದಲಂ ।\\
ನಾನ್ಯಚಿಂತಾ ಪ್ರಕರ್ತವ್ಯಾ ನಾನ್ಯನಿಂದಾ ಕದಾಚನ ॥೧೮೦॥

ಸಿಂದೂರಕರವೀರಾದ್ಯೈಃ ಪುಷ್ಪೈರ್ಲೋಹಿತಕೈಸ್ತಥಾ ।\\
ಯೋಽರ್ಚಯೇದ್ ಭಕ್ತಿಭಾವೇನ ತಸ್ಯಾಸಾಧ್ಯಂ ನ ಕಿಂಚನ ॥೧೮೧॥

ವಾತಸ್ತಂಭಂ ಜಲಸ್ತಂಭಂ ಗತಿಸ್ತಂಭಂ ವಿವಸ್ವತಃ ।\\
ವಹ್ನೇಃ ಸ್ತಂಭಂ ಕರೋತ್ಯೇವ ಸ್ತವಸ್ಯಾಸ್ಯ ಪ್ರಕೀರ್ತನಾತ್॥
೧೮೨॥

ಶ್ರಿಯಮಾಕರ್ಷಯೇತ್ ತೂರ್ಣಮಾನೃಣ್ಯಂ ಜಾಯತೇ ಹಠಾತ್ ।\\
ಯಥಾ ತೃಣಂ ದಹೇದ್ ವಹ್ನಿಸ್ತಥಾರೀನ್ ಮರ್ದಯೇತ್ ಕ್ಷಣಾತ್ ॥೧೮೩॥

ಮೋಹಯೇದ್ ರಾಜಪತ್ನೀಶ್ಚ ದೇವಾನಪಿ ವಶಂ ನಯೇತ್ ।\\
ಯಃ ಪಠೇತ್ ಶೃಣುಯಾದ್ ವಾಪಿ ಏಕಚಿತ್ತೇನ ಸರ್ವದಾ ॥೧೮೪॥

ದೀರ್ಘಾಯುಶ್ಚ ಸುಖೀ ವಾಗ್ಮೀ ವಾಣೀ ತಸ್ಯ ವಶಂಕರೀ ।\\
ಸರ್ವತೀರ್ಥಾಭಿಷೇಕೇಣ ಗಯಾಶ್ರಾದ್ಧೇನ ಯತ್ ಫಲಂ ॥೧೮೫॥

ತತ್ಫಲಂ ಲಭತೇ ಸತ್ಯಂ ಯಃ ಪಠೇದೇಕಚಿತ್ತತಃ ।\\
ಯೇಷಾಮಾರಾಧನೇ ಶ್ರದ್ಧಾ ಯೇ ತು ಸಾಧಿತುಮುದ್ಯತಾಃ ॥೧೮೬॥

ತೇಷಾಂ ಕೃತಿತ್ವಂ ಸರ್ವಂ ಸ್ಯಾದ್ ಗತಿರ್ದೇವಿ ಪರಾ ಚ ಸಾ ।\\
ಋತುಯುಕ್ತಲತಾಗಾರೇ ಸ್ಥಿತ್ವಾ ದಂಡೇನ ತಾಡಯೇತ್ ॥೧೮೭॥

ಜಪ್ತ್ವಾ ಸ್ತುತ್ವಾ ಚ ಭಕ್ತ್ಯಾ ಚ ಗಚ್ಛೇದ್ ವೈ ತಾರಿಣೀಪದಂ ।\\
ಅಷ್ಟಮ್ಯಾಂ ಚ ಚತುರ್ದಶ್ಯಾಂ ನವಮ್ಯಾಂ ಶನಿವಾಸರೇ ॥೧೮೮॥

ಸಂಕ್ರಾಂತ್ಯಾಂ ಮಂಡಲೇ ರಾತ್ರೌ ಅಮಾವಾಸ್ಯಾಂ ಚ ಯೋಽರ್ಚಯೇತ್ ।\\
ವರ್ಷಂ ವ್ಯಾಪ್ಯ ಚ ದೇವೇಶಿ ತಸ್ಯಾಧೀನಾಶ್ಚ ಸಿದ್ಧಯಃ ॥೧೮೯॥

ಸುತಹೀನಾ ಚ ಯಾ ನಾರೀ ದೌರ್ಭಾಗ್ಯಾಮಯಪೀಡಿತಾ ।\\
ವಂಧ್ಯಾ ವಾ ಕಾಕವಂಧ್ಯಾ ವಾ ಮೃತಗರ್ಭಾ ಚ ಯಾಂಗನಾ ॥೧೯೦॥

ಧನಧಾನ್ಯವಿಹೀನಾ ಚ ರೋಗಶೋಕಾಕುಲಾ ಚ ಯಾ ।\\
ಸಾಪಿ ಚೈತದ್ ಮಹಾದೇವಿ ಭೂರ್ಜಪತ್ರೇ ಲಿಖೇತ್ತತಃ ॥೧೯೧॥

ಸವ್ಯೇ ಭುಜೇ ಚ ಬಧ್ನೀಯಾತ್ ಸರ್ವಸೌಖ್ಯವತೀ ಭವೇತ್ ।\\
ಏವಂ ಪುಮಾನಪಿ ಪ್ರಾಯೋ ದುಃಖೇನ ಪರಿಪೀಡಿತಃ ॥೧೯೨॥

ಸಭಾಯಾಂ ವ್ಯಸನೇ ಘೋರೇ ವಿವಾದೇ ಶತ್ರುಸಂಕಟೇ ।\\
ಚತುರಂಗೇ ಚ ತಥಾ ಯುದ್ಧೇ ಸರ್ವತ್ರಾರಿಪ್ರಪೀಡಿತೇ ॥೧೯೩॥

ಸ್ಮರಣಾದೇವ ಕಲ್ಯಾಣಿ ಸಂಕ್ಷಯಂ ಯಾಂತಿ ದೂರತಃ ।\\
ಪೂಜನೀಯಂ ಪ್ರಯತ್ನೇನ ಶೂನ್ಯಾಗಾರೇ ಶಿವಾಲಯೇ ॥೧೯೪॥

ಬಿಲ್ವಮೂಲೇ ಶ್ಮಶಾನೇ ಚ ತಟೇ ವಾ ಕುಲಮಂಡಲೇ ।\\
ಶರ್ಕರಾಸವಸಂಯುಕ್ತೈರ್ಭಕ್ತೈರ್ದುಗ್ಧೈಃ ಸಪಾಯಸೈಃ ॥೧೯೫॥

ಅಪೂಪಾಪಿಷ್ಟಸಂಯುಕ್ತೈರ್ನೈವೇದ್ಯೈಶ್ಚ ಯಥೋಚಿತೈಃ ।\\
ನಿವೇದಿತಂ ಚ ಯದ್ದ್ರವ್ಯಂ ಭೋಕ್ತವ್ಯಂ ಚ ವಿಧಾನತಃ ॥೧೯೬॥

ತನ್ನ ಚೇದ್ ಭುಜ್ಯತೇ ಮೋಹಾದ್ ಭೋಕ್ತುಂ ನೇಚ್ಛಂತಿ ದೇವತಾಃ ।\\
ಅನೇನೈವ ವಿಧಾನೇನ ಯೋಽರ್ಚಯೇತ್ ಪರಮೇಶ್ವರೀಂ ॥೧೯೭॥

ಸ ಭೂಮಿವಲಯೇ ದೇವಿ ಸಾಕ್ಷಾದೀಶೋ ನ ಸಂಶಯಃ ।\\
ಮಹಾಶಂಖೇನ ದೇವೇಶಿ ಸರ್ವಂ ಕಾರ್ಯಂ ಜಪಾದಿಕಂ ॥೧೯೮॥

ಕುಲಸರ್ವಸ್ವಕಸ್ಯೈವಂ ಪ್ರಭಾವೋ ವರ್ಣಿತೋ ಮಯಾ ।\\
ನ ಶಕ್ಯತೇ ಸಮಾಖ್ಯಾತುಂ ವರ್ಷಕೋಟಿಶತೈರಪಿ ॥೧೯೯॥

ಕಿಂಚಿದ್ ಮಯಾ ಚ ಚಾಪಲ್ಯಾತ್ ಕಥಿತಂ ಪರಮೇಶ್ವರಿ ।\\
ಜನ್ಮಾಂತರಸಹಸ್ರೇಣ ವರ್ಣಿತುಂ ನೈವ ಶಕ್ಯತೇ ॥೨೦೦॥

ಕುಲೀನಾಯ ಪ್ರದಾತವ್ಯಂ ತಾರಾಭಕ್ತಿಪರಾಯ ಚ ।\\
ಅನ್ಯಭಕ್ತಾಯ ನೋ ದೇಯಂ ವೈಷ್ಣವಾಯ ವಿಶೇಷತಃ ॥೨೦೧॥

ಕುಲೀನಾಯ ಮಹೇಚ್ಛಾಯ ಭಕ್ತಿಶ್ರದ್ಧಾಪರಾಯ ಚ ।\\
ಮಹಾತ್ಮನೇ ಸದಾ ದೇಯಂ ಪರೀಕ್ಷಿತಗುಣಾಯ ಚ ॥೨೦೨॥

ನಾಭಕ್ತಾಯ ಪ್ರದಾತವ್ಯಂ ಪಥ್ಯಂತರಪರಾಯ ಚ ।\\
ನ ದೇಯಂ ದೇವದೇವೇಶಿ ಗೋಪ್ಯಂ ಸರ್ವಾಗಮೇಷು ಚ ॥೨೦೩॥

ಪೂಜಾಜಪವಿಹೀನಾಯ ಸ್ತ್ರೀಸುರಾನಿಂದಕಾಯ ಚ ।\\
ನ ಸ್ತವಂ ದರ್ಶಯೇತ್ ಕ್ವಾಪಿ ಸಂದರ್ಶ್ಯ ಶಿವಹಾ ಭವೇತ್ ।\\
ಪಠನೀಯಂ ಸದಾ ದೇವಿ ಸರ್ವಾವಸ್ಥಾಸು ಸರ್ವದಾ ॥೨೦೪॥

ಯಃ ಸ್ತೋತ್ರಂ ಕುಲನಾಯಿಕೇ ಪ್ರತಿದಿನಂ ಭಕ್ತ್ಯಾ ಪಠೇದ್ ಮಾನವಃ\\
ಸ ಸ್ಯಾದ್ವಿತ್ತಚಯೈರ್ಧನೇಶ್ವರಸಮೋ ವಿದ್ಯಾಮದೈರ್ವಾಕ್ಪತಿಃ ।\\
ಸೌಂದರ್ಯೇಣ ಚ ಮೂರ್ತಿಮಾನ್ ಮನಸಿಜಃ ಕೀರ್ತ್ಯಾ ಚ ನಾರಾಯಣಃ\\
ಶಕ್ತ್ಯಾ ಶಂಕರ ಏವ ಸೌಖ್ಯವಿಭವೈರ್ಭೂಮೇಃ ಪತಿರ್ನಾನ್ಯಥಾ ॥೨೦೫॥

ಇತಿ ತೇ ಕಥಿತಂ ಗುಹ್ಯಂ ತಾರಾನಾಮಸಹಸ್ರಕಂ ।\\
ಅಸ್ಮಾತ್ ಪರತರಂ ಸ್ತೋತ್ರಂ ನಾಸ್ತಿ ತಂತ್ರೇಷು ನಿಶ್ಚಯಃ ॥೨೦೬॥

\authorline{ಇತಿ ಶ್ರೀಬೃಹನ್ನೀಲತಂತ್ರೇ ಭೈರವಭೈರವೀಸಂವಾದೇ ತಾರಾಸಹಸ್ರನಾಮಸ್ತೋತ್ರಮ್॥}

%=======================================================================

\section{ಶ್ರೀತಾರಾಶತನಾಮಸ್ತೋತ್ರಂ}
\addcontentsline{toc}{section}{ಶ್ರೀತಾರಾಶತನಾಮಸ್ತೋತ್ರಂ}


ಶ್ರೀಶಿವ ಉವಾಚ ॥

ತಾರಿಣೀ ತರಲಾ ತನ್ವೀ ತಾರಾ ತರುಣವಲ್ಲರೀ ।\\
ತೀರರೂಪಾ ತರೀ ಶ್ಯಾಮಾ ತನುಕ್ಷೀಣಪಯೋಧರಾ ॥೧॥

ತುರೀಯಾ ತರಲಾ ತೀವ್ರಗಮನಾ ನೀಲವಾಹಿನೀ ।\\
ಉಗ್ರತಾರಾ ಜಯಾ ಚಂಡೀ ಶ್ರೀಮದೇಕಜಟಾಶಿರಾಃ ॥೨॥

ತರುಣೀ ಶಾಂಭವೀಛಿನ್ನಭಾಲಾ ಚ ಭದ್ರತಾರಿಣೀ ।\\
ಉಗ್ರಾ ಚೋಗ್ರಪ್ರಭಾ ನೀಲಾ ಕೃಷ್ಣಾ ನೀಲಸರಸ್ವತೀ ॥೩॥

ದ್ವಿತೀಯಾ ಶೋಭನಾ ನಿತ್ಯಾ ನವೀನಾ ನಿತ್ಯನೂತನಾ ।\\
ಚಂಡಿಕಾ ವಿಜಯಾರಾಧ್ಯಾ ದೇವೀ ಗಗನವಾಹಿನೀ ॥೪॥

ಅಟ್ಟಹಾಸ್ಯಾ ಕರಾಲಾಸ್ಯಾ ಚರಾಸ್ಯಾ ದಿತಿಪೂಜಿತಾ ।\\
ಸಗುಣಾ ಸಗುಣಾರಾಧ್ಯಾ ಹರೀಂದ್ರದೇವಪೂಜಿತಾ ॥೫॥

ರಕ್ತಪ್ರಿಯಾ ಚ ರಕ್ತಾಕ್ಷೀ ರುಧಿರಾಸ್ಯವಿಭೂಷಿತಾ ।\\
ಬಲಿಪ್ರಿಯಾ ಬಲಿರತಾ ದುರ್ಗಾ ಬಲವತೀ ಬಲಾ ॥೬॥

ಬಲಪ್ರಿಯಾ ಬಲರತಾ ಬಲರಾಮಪ್ರಪೂಜಿತಾ ।\\
ಅರ್ಧಕೇಶೇಶ್ವರೀ ಕೇಶಾ ಕೇಶವಾಸವಿಭೂಷಿತಾ ॥೭॥

ಪದ್ಮಮಾಲಾ ಚ ಪದ್ಮಾಕ್ಷೀ ಕಾಮಾಖ್ಯಾ ಗಿರಿನಂದಿನೀ ।\\
ದಕ್ಷಿಣಾ ಚೈವ ದಕ್ಷಾ ಚ ದಕ್ಷಜಾ ದಕ್ಷಿಣೇ ರತಾ ॥೮॥

ವಜ್ರಪುಷ್ಪಪ್ರಿಯಾ ರಕ್ತಪ್ರಿಯಾ ಕುಸುಮಭೂಷಿತಾ ।\\
ಮಾಹೇಶ್ವರೀ ಮಹಾದೇವಪ್ರಿಯಾ ಪಂಚವಿಭೂಷಿತಾ ॥೯॥

ಇಡಾ ಚ ಪಿಂಗಲಾ ಚೈವ ಸುಷುಮ್ನಾ ಪ್ರಾಣರೂಪಿಣೀ ।\\
ಗಾಂಧಾರೀ ಪಂಚಮೀ ಪಂಚಾನನಾದಿ ಪರಿಪೂಜಿತಾ ॥೧೦॥

ತಥ್ಯವಿದ್ಯಾ ತಥ್ಯರೂಪಾ ತಥ್ಯಮಾರ್ಗಾನುಸಾರಿಣೀ ।\\
ತತ್ತ್ವಪ್ರಿಯಾ ತತ್ತ್ವರೂಪಾ ತತ್ತ್ವಜ್ಞಾನಾತ್ಮಿಕಾಽನಘಾ ॥೧೧॥

ತಾಂಡವಾಚಾರಸಂತುಷ್ಟಾ ತಾಂಡವಪ್ರಿಯಕಾರಿಣೀ ।\\
ತಾಲದಾನರತಾ ಕ್ರೂರತಾಪಿನೀ ತರಣಿಪ್ರಭಾ ॥೧೨॥

ತ್ರಪಾಯುಕ್ತಾ ತ್ರಪಾಮುಕ್ತಾ ತರ್ಪಿತಾ ತೃಪ್ತಿಕಾರಿಣೀ ।\\
ತಾರುಣ್ಯಭಾವಸಂತುಷ್ಟಾ ಶಕ್ತಿರ್ಭಕ್ತಾನುರಾಗಿಣೀ ॥೧೩॥

ಶಿವಾಸಕ್ತಾ ಶಿವರತಿಃ ಶಿವಭಕ್ತಿಪರಾಯಣಾ ।\\
ತಾಮ್ರದ್ಯುತಿಸ್ತಾಮ್ರರಾಗಾ ತಾಮ್ರಪಾತ್ರಪ್ರಭೋಜಿನೀ ॥೧೪॥

ಬಲಭದ್ರಪ್ರೇಮರತಾ ಬಲಿಭುಗ್ಬಲಿಕಲ್ಪಿನೀ ।\\
ರಾಮರೂಪಾ ರಾಮಶಕ್ತೀ ರಾಮರೂಪಾನುಕಾರಿಣೀ ॥೧೫॥

ಇತ್ಯೇತತ್ಕಥಿತಂ ದೇವಿ ರಹಸ್ಯಂ ಪರಮಾದ್ಭುತಂ ।\\
ಶ್ರುತ್ವಾ ಮೋಕ್ಷಮವಾಪ್ನೋತಿ ತಾರಾದೇವ್ಯಾಃ ಪ್ರಸಾದತಃ ॥೧೬॥

ಯ ಇದಂ ಪಠತಿ ಸ್ತೋತ್ರಂ ತಾರಾಸ್ತುತಿರಹಸ್ಯಕಂ ।\\
ಸರ್ವಸಿದ್ಧಿಯುತೋ ಭೂತ್ವಾ ವಿಹರೇತ್ ಕ್ಷಿತಿಮಂಡಲೇ ॥೧೭॥

ತಸ್ಯೈವ ಮಂತ್ರಸಿದ್ಧಿಃ ಸ್ಯಾನ್ಮಮಸಿದ್ಧಿರನುತ್ತಮಾ ।\\
ಭವತ್ಯೇವ ಮಹಾಮಾಯೇ ಸತ್ಯಂ ಸತ್ಯಂ ನ ಸಂಶಯಃ ॥೧೮॥

ಮಂದೇ ಮಂಗಲವಾರೇ ಚ ಯಃ ಪಠೇನ್ನಿಶಿ ಸಂಯತಃ ।\\
ತಸ್ಯೈವ ಮಂತ್ರಸಿದ್ಧಿಸ್ಸ್ಯಾದ್ಗಾಣಪತ್ಯಂ ಲಭೇತ ಸಃ ॥೧೯॥

ಶ್ರದ್ಧಯಾಽಶ್ರದ್ಧಯಾ ವಾಪಿ ಪಠೇತ್ತಾರಾರಹಸ್ಯಕಂ ।\\
ಸೋಽಚಿರೇಣೈವ ಕಾಲೇನ ಜೀವನ್ಮುಕ್ತಃ ಶಿವೋ ಭವೇತ್ ॥೨೦॥

ಸಹಸ್ರಾವರ್ತನಾದ್ದೇವಿ ಪುರಶ್ಚರ್ಯಾಫಲಂ ಲಭೇತ್ ।\\
ಏವಂ ಸತತಯುಕ್ತಾ ಯೇ ಧ್ಯಾಯಂತಸ್ತ್ವಾಮುಪಾಸತೇ ।\\
ತೇ ಕೃತಾರ್ಥಾ ಮಹೇಶಾನಿ ಮೃತ್ಯುಸಂಸಾರವರ್ತ್ಮನಃ ॥೨೧॥
\authorline{ಇತಿ ಸ್ವರ್ಣಮಾಲಾತಂತ್ರೇ ತಾರಾಶತನಾಮಸ್ತೋತ್ರಂ ಸಮಾಪ್ತಂ ॥}
