‌\chapter*{\center ಗುರುಗಣಪತಿಪೂಜಾ}
\thispagestyle{empty}
ಭವಸಂಚಿತಪಾಪೌಘವಿಧ್ವಂಸನವಿಚಕ್ಷಣಂ~।\\
ವಿಘ್ನಾಂಧಕಾರಭಾಸ್ವಂತಂ ವಿಘ್ನರಾಜಮಹಂ ಭಜೇ ॥ಧ್ಯಾನಮ್॥

ಅತ್ರಾಗಚ್ಛ ಜಗದ್ವಂದ್ಯ ಸುರರಾಜಾರ್ಚಿತೇಶ್ವರ~।\\
ಅನಾಥನಾಥ ಸರ್ವಜ್ಞ ಗೌರೀಗರ್ಭಸಮುದ್ಭವ ॥ಆವಾಹನಮ್॥

ಮೌಕ್ತಿಕೈಃ ಪುಷ್ಯರಾಗೈಶ್ಚ ನಾನಾರತ್ನವಿರಾಜಿತಂ~।\\
ರತ್ನಸಿಂಹಾಸನಂ ಚಾರು ಪ್ರೀತ್ಯರ್ಥಂ ಪ್ರತಿಗೃಹ್ಯತಾಂ ॥ಆಸನಮ್॥

ಗಜವಕ್ತ್ರ ನಮಸ್ತೇಽಸ್ತು ಸರ್ವಾಭೀಷ್ಟಪ್ರದಾಯಕ~।\\
ಭಕ್ತ್ಯಾ ಪಾದ್ಯಂ ಮಯಾ ದತ್ತಂ ಗೃಹಾಣ ದ್ವಿರದಾನನ ॥ಪಾದ್ಯಮ್॥

ಗೌರೀಪುತ್ರ ನಮಸ್ತೇಽಸ್ತು ಶಂಕರಪ್ರಿಯನಂದನ~।\\
ಗೃಹಾಣಾರ್ಘ್ಯಂ ಮಯಾ ದತ್ತಂ ಗಂಧಪುಷ್ಪಾಕ್ಷತೈರ್ಯುತಂ ॥ಅರ್ಘ್ಯಮ್॥

ಅನಾಥನಾಥ ಸರ್ವಜ್ಞ ಗೀರ್ವಾಣವರಪೂಜಿತ~।\\
ಗೃಹಾಣಾಚಮನಂ ದೇವ ತುಭ್ಯಂ ದತ್ತಂ ಮಯಾ ಪ್ರಭೋ ॥ಆಚಮನಮ್॥

ದಧಿಕ್ಷೀರಸಮಾಯುಕ್ತಂ ಮಧ್ವಾಜ್ಯೇನ ಸಮನ್ವಿತಂ~।\\
ಮಧುಪರ್ಕಂ ಗೃಹಾಣೇಮಂ ಗಜವಕ್ತ್ರ ನಮೋಽಸ್ತು ತೇ ॥ಮಧುಪರ್ಕಃ॥

ಗಂಗಾ ಸರಸ್ವತೀ ರೇವಾ ಪಯೋಷ್ಣೀ ಯಮುನಾಜಲೈಃ~।\\
ಸ್ನಪಯಾಮಿ ಗಣೇಶಾನ ತಥಾ ಶಾಂತಂ ಕುರುಷ್ವ ಮಾಂ॥ಸ್ನಾನಮ್॥

ದಧಿ ಕ್ಷೀರ ಘೃತೈರ್ಯುಕ್ತಂ ಶರ್ಕರಾ ಮಧುಮಿಶ್ರಿತಂ ।\\
ಪಂಚಾಮೃತಂ ಗೃಹಾಣ ತ್ವಂ ಕೃಪಯಾ ಪುರುಷೋತ್ತಮ ॥\\ಪಂಚಾಮೃತಸ್ನಾನಂ॥

\section{ಭಾವನೋಪನಿಷತ್}
ಓಂ ಭದ್ರಂ ಕರ್ಣೇಭಿಃ ಇತಿ ಶಾಂತಿಃ ॥
ಓಂ ಶ್ರೀಗುರುಃ ಸರ್ವಕಾರಣಭೂತಾ ಶಕ್ತಿಃ~। ತೇನ ನವರಂಧ್ರರೂಪೋ ದೇಹಃ~। ನವಚಕ್ರರೂಪಂ ಶ್ರೀಚಕ್ರಂ~। ವಾರಾಹೀ ಪಿತೃರೂಪಾ~। ಕುರುಕುಲ್ಲಾ ಬಲಿದೇವತಾ ಮಾತಾ~। ಪುರುಷಾರ್ಥಾಃ ಸಾಗರಾಃ~। ದೇಹೋ ನವರತ್ನದ್ವೀಪಃ~। ತ್ವಗಾದಿಸಪ್ತಧಾತುರೋಮ ಸಂಯುಕ್ತಃ। ಸಂಕಲ್ಪಾಃ ಕಲ್ಪತರವಸ್ತೇಜಃ ಕಲ್ಪಕೋ\-ದ್ಯಾನಂ~। ರಸನಯಾ ಭಾವ್ಯಮಾನಾ ಮಧುರಾಮ್ಲತಿಕ್ತಕಟುಕಷಾಯಲವಣರಸಾಃ ಷಡೃತವಃ~। ಜ್ಞಾನಮರ್ಘ್ಯಂ। ಜ್ಞೇಯಂ ಹವಿಃ~। ಜ್ಞಾತಾ ಹೋತಾ~। ಜ್ಞಾತೃಜ್ಞಾನಜ್ಞೇಯಾನಾಮಭೇದಭಾವನಂ ಶ್ರೀಚಕ್ರಪೂಜನಂ~। ನಿಯತಿಃ ಶೃಂಗಾರಾದಯೋ ರಸಾ ಅಣಿಮಾದಯಃ~। ಕಾಮಕ್ರೋಧ ಲೋಭಮೋಹ ಮದ ಮಾತ್ಸರ್ಯ ಪುಣ್ಯ ಪಾಪಮಯಾ ಬ್ರಾಹ್ಮ್ಯಾದ್ಯಷ್ಟಶಕ್ತಯಃ~। ಆಧಾರನವಕಂ ಮುದ್ರಾಶಕ್ತಯಃ~। ಪೃಥಿವ್ಯಪ್ತೇಜೋವಾಯ್ವಾಕಾಶ ಶ್ರೋತ್ರತ್ವಕ್ಚಕ್ಷುರ್ಜಿಹ್ವಾ ಘ್ರಾಣ ವಾಕ್ಪಾಣಿ ಪಾದ ಪಾಯೂಪಸ್ಥಾನಿ ಮನೋವಿಕಾರಾಃ ಕಾಮಾಕರ್ಷಣ್ಯಾದಿ ಷೋಡಶ ಶಕ್ತಯಃ~। ವಚನಾದಾನಗಮನವಿಸರ್ಗಾನಂದ ಹಾನೋಪಾದಾನೋಪೇಕ್ಷಾಖ್ಯ ಬುದ್ಧಯೋನಂಗಕುಸುಮಾದ್ಯಷ್ಟೌ~। ಅಲಂಬುಸಾ ಕುಹೂರ್ವಿಶ್ವೋದರಾ ವಾರಣಾ ಹಸ್ತಿಜಿಹ್ವಾ ಯಶೋವತೀ ಪಯಸ್ವಿನೀ ಗಾಂಧಾರೀ ಪೂಷಾ ಶಂಖಿನೀ ಸರಸ್ವತೀ ಇಡಾ ಪಿಂಗಲಾ ಸುಷುಮ್ನಾ ಚೇತಿ ಚತುರ್ದಶ ನಾಡ್ಯಃ ಸರ್ವಸಂಕ್ಷೋಭಿಣ್ಯಾದಿ ಚತುರ್ದಶ ಶಕ್ತಯಃ~। ಪ್ರಾಣಾಪಾನ ವ್ಯಾನೋದಾನ ಸಮಾನ ನಾಗ ಕೂರ್ಮ ಕೃಕರ ದೇವದತ್ತ ಧನಂಜಯಾ ಇತಿ ದಶ ವಾಯವಃ ಸರ್ವಸಿದ್ಧಿಪ್ರದಾದಿ ಬಹಿರ್ದಶಾರಗಾ ದೇವತಾಃ~। ಏತದ್ವಾಯುಸಂಸರ್ಗ ಕೋಪಾಧಿ ಭೇಧೇನ ರೇಚಕಃ ಪಾಚಕಃ ಶೋಷಕೋ ದಾಹಕಃ ಪ್ಲಾವಕ ಇತಿ ಪ್ರಾಣಮುಖ್ಯಶ್ರೀತ್ವೇನ ಪಂಚಧಾ ಜಠರಾಗ್ನಿರ್ಭವತಿ~। ಕ್ಷಾರಕ ಉದ್ಗಾರಕಃ ಕ್ಷೋಭಕೋ ಜೃಂಭಕೋ ಮೋಹಕ ಇತಿ ನಾಗಪ್ರಾಧಾನ್ಯೇನ ಪಂಚವಿಧಾಸ್ತೇ ಮನುಷ್ಯಾಣಾಂ ದೇಹಗಾ ಭಕ್ಷ್ಯಭೋಜ್ಯ ಚೋಷ್ಯ ಲೇಹ್ಯ ಪೇಯಾತ್ಮಕ ಪಂಚವಿಧಮನ್ನಂ ಪಾಚಯಂತಿ। ಏತಾ ದಶ ವಹ್ನಿಕಲಾಃ ಸರ್ವಜ್ಞಾದ್ಯಾ ಅಂತರ್ದಶಾರಗಾ ದೇವತಾಃ~। ಶೀತೋಷ್ಣಸುಖದುಃಖೇಚ್ಛಾಃ ಸತ್ತ್ವಂ ರಜಸ್ತಮೋ ವಶಿನ್ಯಾದಿಶಕ್ತಯೋಽಷ್ಟೌ~। ಶಬ್ದಾದಿ ತನ್ಮಾತ್ರಾಃ ಪಂಚಪುಷ್ಪಬಾಣಾಃ। ಮನ ಇಕ್ಷುಧನುಃ~। ರಾಗಃ ಪಾಶಃ~। ದ್ವೇಷೋಂಕುಶಃ~। ಅವ್ಯಕ್ತಮಹದಹಂಕಾರಾಃ ಕಾಮೇಶ್ವರೀವಜ್ರೇಶ್ವರೀ ಭಗಮಾಲಿನ್ಯೋಂತಸ್ತ್ರಿಕೋಣಗಾ ದೇವತಾಃ~। ನಿರುಪಾಧಿಕ ಸಂವಿದೇವ ಕಾಮೇಶ್ವರಃ। ಸದಾನಂದಪೂರ್ಣ ಸ್ವಾತ್ಮೈವ ಪರದೇವತಾ ಲಲಿತಾ~। ಲೌಹಿತ್ಯಮೇತಸ್ಯ ಸರ್ವಸ್ಯ ವಿಮರ್ಶಃ~। ಅನನ್ಯಚಿತ್ತತ್ವೇನ ಚ ಸಿದ್ಧಿಃ~। ಭಾವನಾಯಾಃ ಕ್ರಿಯಾ ಉಪಚಾರಃ~। ಅಹಂತ್ವಮಸ್ತಿನಾಸ್ತಿಕರ್ತವ್ಯಮಕರ್ತವ್ಯ ಮುಪಾಸಿತವ್ಯಮಿತಿ ವಿಕಲ್ಪಾನಾ ಮಾತ್ಮನಿ ವಿಲಾಪನಂ ಹೋಮಃ~। ಭಾವನಾವಿಷಯಾಣಾ ಮಭೇದಭಾವನಾ ತರ್ಪಣಂ~। ಪಂಚದಶತಿಥಿರೂಪೇಣ ಕಾಲಸ್ಯ ಪರಿಣಾಮಾವಲೋಕನಂ ಪಂಚದಶನಿತ್ಯಾಃ~। ಏವಂ ಮುಹೂರ್ತತ್ರಿತಯಂ ಮುಹೂರ್ತದ್ವಿತಯಂ ಮುಹೂರ್ತಮಾತ್ರಂ ವಾ ಭಾವನಾಪರೋ ಜೀವನ್ಮುಕ್ತೋ ಭವತಿ~। ಸ ಏವ ಶಿವಯೋಗೀತಿ ಗದ್ಯತೇ~। ಕಾದಿಮತೇನಾಂತಶ್ಚಕ್ರ ಭಾವನಾಃ ಪ್ರತಿಪಾದಿತಾಃ। ಯ ಏವಂ ವೇದ। ಸೋಽಥರ್ವಶಿರ್ಷೋಽಧೀತೇ ॥ ಓಂ ಭದ್ರಂ ಕರ್ಣೇಭಿಃ ಇತಿ ಶಾಂತಿಃ ॥

ರಕ್ತವಸ್ತ್ರದ್ವಯಂ ಚಾರು ದೇವಯೋಗ್ಯಂ ಚ ಮಂಗಳಂ~।\\
ಶುಭಪ್ರದ ಗೃಹಾಣ ತ್ವಂ ಲಂಬೋದರ ಹರಾತ್ಮಜ ॥ವಸ್ತ್ರಮ್॥

ರಾಜತಂ ಬ್ರಹ್ಮಸೂತ್ರಂ ಚ ಕಾಂಚನಂ ಚೋತ್ತರೀಯಕಂ~।\\
ಗೃಹಾಣ ಚಾರು ಸರ್ವಜ್ಞ ಭಕ್ತಾನಾಂ ವರದೋ ಭವ ॥ಉಪವೀತಮ್॥

ತತಃ ಸಿಂದೂರಕಂ ದೇವ ಗೃಹಾಣ ಗಣನಾಯಕ~।\\
ಅಂಗಲೇಪನಭಾವಾರ್ಥಂ ಸದಾನಂದವಿವರ್ಧನಂ ॥ಸಿಂದೂರಮ್॥

ನಾನಾವಿಧಾನಿ ದಿವ್ಯಾನಿ ನಾನಾರತ್ನೋಜ್ವಲಾನಿ ಚ~।\\
ಭೂಷಣಾನಿ ಗೃಹಾಣೇಶ ಪಾರ್ವತೀಪ್ರಿಯನಂದನ ॥ಆಭರಣಂ॥

ಚಂದನಾಗರುಕಸ್ತೂರೀ ಕರ್ಪೂರಕುಂಕುಮಾನ್ವಿತಂ~।\\
ವಿಲೇಪನಂ ಸುರಶ್ರೇಷ್ಠ ಪ್ರೀತ್ಯರ್ಥಂ ಪ್ರತಿಗೃಹ್ಯತಾಂ ॥ಗಂಧಃ॥

ರಕ್ತಾಕ್ಷತಾಂಶ್ಚ ದೇವೇಶ ಗೃಹಾಣ ದ್ವಿರದಾನನ~।\\
ಲಲಾಟಫಲಕೇ ಚಂದ್ರಸ್ತಸ್ಯೋಪರ್ಯವಧಾರಯ ॥ಅಕ್ಷತಾಃ॥

ಮಾಲ್ಯಾದೀನಿ ಸುಗಂಧೀನಿ ಮಾಲತ್ಯಾದೀನಿ ಚ ಪ್ರಭೋ~।\\
ಮಯಾಹೃತಾನಿ ಪುಷ್ಪಾಣಿ ಸ್ವೀಕುರುಷ್ವ ಗಣಾಧಿಪ ॥ಪುಷ್ಪಾಣಿ॥

ಗೌರೀಪುತ್ರ ಮಹಾಕಾಯ ಗೀರ್ವಾಣಗಣಪೂಜಿತ~।\\
ದೂರ್ವಾಯುಗ್ಮಂ ಪ್ರದಾಸ್ಯಾಮಿ ಸ್ವೀಕುರುಷ್ವ ಗಣಾಧಿಪ ॥ದೂರ್ವಾ॥

ಭಕ್ತಾನಾಂ ಸುಖದಾತಾ ತ್ವಂ ಸರ್ವಮಂಗಳಕಾರಕ~।\\
ಹರಿದ್ರಾಂ ತೇ ಪ್ರಯಚ್ಛಾಮಿ ಗೃಹಾಣ ಗಣನಾಯಕ ॥ಹರಿದ್ರಾ॥

ಕುಂಕುಮಂ ಕಾಂತಿದಂ ದಿವ್ಯಂ ಸರ್ವಕಾಮಫಲಪ್ರದಂ~।\\
ಸರ್ವದೇವೈಶ್ಚ ಸಂಪೂಜ್ಯಂ ಗೃಹಾಣ ವರದಾಯಕ ॥ಕುಂಕುಮಮ್॥
\section{ದಕ್ಷಿಣಾಮೂರ್ತಿ ಅಂಗಪೂಜಾ ॥}
ಓಂ ಪಾಪನಾಶಾಯ ನಮಃ~। ಪಾದೌ ಪೂಜಯಾಮಿ॥\\
ಓಂ ಗುರವೇ ನಮಃ~। ಗುಲ್ಫೌ ಪೂಜಯಾಮಿ॥\\
ಓಂ ಜ್ಞಾನೇಶ್ವರಾಯ ನಮಃ~। ಜಂಘೇ ಪೂಜಯಾಮಿ॥\\
ಓಂ ಜಾಹ್ನವೀಪತಯೇ ನಮಃ~। ಜಾನುನೀ ಪೂಜಯಾಮಿ॥\\
ಓಂ ಉತ್ತಮೋತ್ತಮಾಯ ನಮಃ~। ಊರೂ ಪೂಜಯಾಮಿ॥\\
ಓಂ ಕಂದರ್ಪನಾಶಾಯ ನಮಃ~। ಕಟಿಂ ಪೂಜಯಾಮಿ॥\\
ಓಂ ಗೌರೀಶಾಯ ನಮಃ~। ಗುಹ್ಯಂ ಪೂಜಯಾಮಿ॥\\
ಓಂ ನಿರಂಜನಾಯ ನಮಃ~। ನಾಭಿಂ ಪೂಜಯಾಮಿ॥\\
ಓಂ ಸ್ಕಂದಗುರವೇ ನಮಃ~। ಸ್ಕಂಧೌ ಪೂಜಯಾಮಿ॥\\
ಓಂ ಹಿರಣ್ಯಬಾಹವೇ ನಮಃ~। ಬಾಹೂ ಪೂಜಯಾಮಿ॥\\
ಓಂ ಹರಾಯ ನಮಃ~। ಹಸ್ತೌ ಪೂಜಯಾಮಿ॥\\
ಓಂ ನೀಲಕಂಠಾಯ ನಮಃ~। ಕಂಠಂ ಪೂಜಯಾಮಿ॥\\
ಓಂ ಕಮಲವದನಾಯ ನಮಃ~। ಮುಖಂ ಪೂಜಯಾಮಿ॥\\
ಓಂ ಉಗ್ರಾಯ ನಮಃ~। ನಾಸಿಕಾಂ ಪೂಜಯಾಮಿ॥\\
ಓಂ ನಳಿನಾಕ್ಷಾಯ ನಮಃ~। ನೇತ್ರೇ ಪೂಜಯಾಮಿ॥\\
ಓಂ ಅಗ್ನಿತಿಲಕಾಯ ನಮಃ~। ತೃತೀಯನೇತ್ರಂ ಪೂಜಯಾಮಿ ।\\
ಓಂ ಭಸಿತಭೂಷಣಾಯ ನಮಃ~। ಲಲಾಟಂ ಪೂಜಯಾಮಿ॥\\
ಓಂ ಚಂದ್ರಮೌಲಯೇ ನಮಃ~। ಮೌಲಿಂ ಪೂಜಯಾಮಿ॥\\
ಓಂ ಗಂಗಾಧರಾಯ ನಮಃ~। ಶಿರಃ ಪೂಜಯಾಮಿ॥\\
ಓಂ ದಕ್ಷಿಣಾಮೂರ್ತಯೇ ನಮಃ~। ಸರ್ವಾಂಗಾನಿ ಪೂಜಯಾಮಿ॥
\section{ಪತ್ರ ಪೂಜಾ ॥}
ಓಂ ಬಿಲ್ವೇಶಾಯ ನಮಃ~। ಬಿಲ್ವಪತ್ರಂ ಸಮರ್ಪಯಾಮಿ॥\\
ಓಂ ತುಷ್ಟಾಯ ನಮಃ~। ತುಲಸೀಪತ್ರಂ ಸಮರ್ಪಯಾಮಿ॥\\
ಓಂ ಆನಂದಾಯ ನಮಃ~। ಅರ್ಕಪತ್ರಂ ಸಮರ್ಪಯಾಮಿ॥\\
ಓಂ ಜಗದ್ಗುರವೇ ನಮಃ~। ಜಂಬೀರಪತ್ರಂ ಸಮರ್ಪಯಾಮಿ॥\\
ಓಂ ನಿರ್ಗುಣಾಯ ನಮಃ~। ನಿರ್ಗುಂಡೀಪತ್ರಂ ಸಮರ್ಪಯಾಮಿ॥\\
ಓಂ ದುರಿತಶಮನಾಯ ನಮಃ~। ದೂರ್ವಾಪತ್ರಂ ಸಮರ್ಪಯಾಮಿ॥\\
ಓಂ ಕುಮುದನೇತ್ರಾಯ ನಮಃ~। ಕುಶಪತ್ರಂ ಸಮರ್ಪಯಾಮಿ॥\\
ಓಂ ಮೃಗಧರಾಯ ನಮಃ~। ಮರುಗಪತ್ರಂ ಸಮರ್ಪಯಾಮಿ॥\\
ಓಂ ಕಾಮಾರಯೇ ನಮಃ~। ಕಾಮಕಸ್ತೂರಿಕಾಪತ್ರಂ ಸಮರ್ಪಯಾಮಿ॥\\
ಓಂ ಗಿರಿಜಾಪತಯೇ ನಮಃ~। ಗಿರಿಕರ್ಣಿಕಾಪತ್ರಂ ಸಮರ್ಪಯಾಮಿ॥\\
ಓಂ ಮಹಾದೇವಾಯ ನಮಃ~। ಮಾಚೀಪತ್ರಂ ಸಮರ್ಪಯಾಮಿ॥\\
ಓಂ ಭಕ್ತಜನಪ್ರಿಯಾಯ ನಮಃ~। ಧಾತ್ರೀಪತ್ರಂ ಸಮರ್ಪಯಾಮಿ॥\\
ಓಂ ವಿಷ್ಣುಪ್ರಿಯಾಯ ನಮಃ~। ವಿಷ್ಣುಕ್ರಾಂತಿಪತ್ರಂ ಸಮರ್ಪಯಾಮಿ॥\\
ಓಂ ದಾಂತಾಯ ನಮಃ~। ದ್ರೋಣಪತ್ರಂ ಸಮರ್ಪಯಾಮಿ॥\\
ಓಂ ಧೂತಕಲ್ಮಶಾಯ ನಮಃ~। ಧತ್ತೂರಪತ್ರಂ ಸಮರ್ಪಯಾಮಿ॥\\
ಓಂ ಶಮಪ್ರಾಪ್ತಾಯ ನಮಃ~। ಶಮೀಪತ್ರಂ ಸಮರ್ಪಯಾಮಿ॥\\
ಓಂ ಸಾಂಬಶಿವಾಯ ನಮಃ~। ಸೇವಂತಿಕಾಪತ್ರಂ ಸಮರ್ಪಯಾಮಿ॥\\
ಓಂ ಚರ್ಮವಾಸಸೇ ನಮಃ~। ಚಂಪಕಪತ್ರಂ ಸಮರ್ಪಯಾಮಿ॥\\
ಓಂ ಕರುಣಾಕರಾಯ ನಮಃ~। ಕರವೀರಪತ್ರಂ ಸಮರ್ಪಯಾಮಿ॥\\ 
ಓಂ ಅಪರಾಜಿತಾಯ ನಮಃ~। ಅಶೋಕಪತ್ರಂ ಸಮರ್ಪಯಾಮಿ॥\\
ಓಂ ಪುಣ್ಯಮೂರ್ತಯೇ ನಮಃ~। ಪುನ್ನಾಗಪತ್ರಂ ಸಮರ್ಪಯಾಮಿ॥\\
ಓಂ ದಕ್ಷಿಣಾಮೂರ್ತಯೇ ನಮಃ~। ಸರ್ವಪತ್ರಂ ಸಮರ್ಪಯಾಮಿ॥
\section{ಪುಷ್ಪಪೂಜಾ॥}
ಓಂ ಆತ್ಮಸಂಭವಾಯ ನಮಃ~। ಅರ್ಕಪುಷ್ಪಂ ಸಮರ್ಪಯಾಮಿ॥\\
ಓಂ ದ್ರವಿಣಪ್ರದಾಯ ನಮಃ~। ದ್ರೋಣಪುಷ್ಪಂ ಸಮರ್ಪಯಾಮಿ॥\\
ಓಂ ಧರ್ಮಸೇತವೇ ನಮಃ~। ಧತ್ತೂರಪುಷ್ಪಂ ಸಮರ್ಪಯಾಮಿ॥\\
ಓಂ ಬೃಹದ್ಗರ್ಭಾಯ ನಮಃ~। ಬೃಹತೀಪುಷ್ಪಂ ಸಮರ್ಪಯಾಮಿ॥\\
ಓಂ ಉಮಾಪತಯೇ ನಮಃ~। ಬಕುಲಪುಷ್ಪಂ ಸಮರ್ಪಯಾಮಿ॥\\
ಓಂ ಜಗತ್ಪತಯೇ ನಮಃ~। ಜಾತೀಪುಷ್ಪಂ ಸಮರ್ಪಯಾಮಿ॥\\
ಓಂ ಕಾಲಾಂತಕಾಯ ನಮಃ~। ಕರವೀರಪುಷ್ಪಂ ಸಮರ್ಪಯಾಮಿ॥\\
ಓಂ ನೀಲಕಂಠಾಯ ನಮಃ~। ನೀಲೋತ್ಪಲಪುಷ್ಪಂ ಸಮರ್ಪಯಾಮಿ॥\\
ಓಂ ಪರಮೇಶ್ವರಾಯ ನಮಃ~। ಪುನ್ನಾಗಪುಷ್ಪಂ ಸಮರ್ಪಯಾಮಿ॥\\
ಓಂ ವೃಷಧ್ವಜಾಯ ನಮಃ~। ವೈಜಯಂತಿಕಾಪುಷ್ಪಂ ಸಮರ್ಪಯಾಮಿ॥\\
ಓಂ ಗಿರಿಧನ್ವನೇ ನಮಃ~। ಗಿರಿಕರ್ಣಿಕಾಪುಷ್ಪಂ ಸಮರ್ಪಯಾಮಿ॥\\
ಓಂ ಚಂದ್ರಚೂಡಾಯ ನಮಃ~। ಚಂಪಕಪುಷ್ಪಂ ಸಮರ್ಪಯಾಮಿ॥\\
ಓಂ ಸೋಮೇಶಾಯ ನಮಃ~। ಸೇವಂತಿಕಾಪುಷ್ಪಂ ಸಮರ್ಪಯಾಮಿ॥\\
ಓಂ ಮಹೇಶ್ವರಾಯ ನಮಃ~। ಮಲ್ಲಿಕಾಪುಷ್ಪಂ ಸಮರ್ಪಯಾಮಿ॥\\
ಓಂ ಜಟಾಧರಾಯ ನಮಃ~। ಜಪಾಪುಷ್ಪಂ ಸಮರ್ಪಯಾಮಿ॥\\
ಓಂ ದಕ್ಷಿಣಾಮೂರ್ತಯೇ ನಮಃ~। ಸರ್ವಾಣಿ ಪುಷ್ಪಾಣಿ ಸಮರ್ಪಯಾಮಿ॥
\section{ಗಣೇಶ ಅಂಗಪೂಜಾ ॥}
ಓಂ ಗಣೇಶಾಯ ನಮಃ~। ಪಾದೌ ಪೂಜಯಾಮಿ ॥\\
ಓಂ ಗೌರೀಪುತ್ರಾಯ ನಮಃ~। ಗುಲ್ಫೌ ಪೂಜಯಾಮಿ ॥\\
ಓಂ ಅಘನಾಶನಾಯ ನಮಃ~। ಜಾನುನೀ ಪೂಜಯಾಮಿ ॥\\
ಓಂ ವಿಘ್ನರಾಜಾಯ ನಮಃ~। ಜಂಘೇ ಪೂಜಯಾಮಿ ॥\\
ಓಂ ಆಖುವಾಹನಾಯ ನಮಃ~। ಊರೂ ಪೂಜಯಾಮಿ ॥\\
ಓಂ ಹೇರಂಬಾಯ ನಮಃ~। ಕಟಿಂ ಪೂಜಯಾಮಿ ॥\\
ಓಂ ಲಂಬೋದರಾಯ ನಮಃ~। ಉದರಂ ಪೂಜಯಾಮಿ ॥\\
ಓಂ ವಕ್ರತುಂಡಾಯ ನಮಃ~। ನಾಭಿಂ ಪೂಜಯಾಮಿ ॥\\
ಓಂ ಗಣನಾಥಾಯ ನಮಃ~। ಹೃದಯಂ ಪೂಜಯಾಮಿ ॥\\
ಓಂ ಸ್ಥೂಲಕಂಠಾಯ ನಮಃ~। ಕಂಠಂ ಪೂಜಯಾಮಿ ॥\\
ಓಂ ಸ್ಕಂದಾಗ್ರಜಾಯ ನಮಃ~। ಸ್ಕಂಧೌ ಪೂಜಯಾಮಿ ॥\\
ಓಂ ಪರಶುಹಸ್ತಾಯ ನಮಃ~। ಹಸ್ತಾನ್ ಪೂಜಯಾಮಿ ॥\\
ಓಂ ಗಜವಕ್ತ್ರಾಯ ನಮಃ~। ವಕ್ತ್ರಂ ಪೂಜಯಾಮಿ ॥\\
ಓಂ ವಿಘ್ನಹರ್ತ್ರೇ ನಮಃ~। ನೇತ್ರೇ ಪೂಜಯಾಮಿ ॥\\
ಓಂ ಶೂರ್ಪಕರ್ಣಾಯ ನಮಃ~। ಕರ್ಣೌ ಪೂಜಯಾಮಿ ॥\\
ಓಂ ಏಕದಂತಾಯ ನಮಃ~। ದಂತಂ ಪೂಜಯಾಮಿ ॥\\
ಓಂ ಫಾಲಚಂದ್ರಾಯ ನಮಃ~। ಲಲಾಟಂ ಪೂಜಯಾಮಿ ॥\\
ಓಂ ಸರ್ವೇಶ್ವರಾಯ ನಮಃ~। ಶಿರಃ ಪೂಜಯಾಮಿ ॥\\
ಅಂಗಪೂಜಾಂ ಸಮರ್ಪಯಾಮಿ ॥
\section{ಪತ್ರ ಪೂಜಾ ॥}
ಓಂ ಸುಮುಖಾಯ ನಮಃ~। ಮಾಚೀಪತ್ರಂ ಸಮರ್ಪಯಾಮಿ ॥\\
ಓಂ ಗಣಾಧಿಪಾಯ ನಮಃ~। ಭೃಂಗರಾಜಪತ್ರಂ ಸಮರ್ಪಯಾಮಿ ॥\\
ಓಂ ಉಮಾಪುತ್ರಾಯ ನಮಃ~। ಬಿಲ್ವಪತ್ರಂ ಸಮರ್ಪಯಾಮಿ ॥\\
ಓಂ ಗಜಾನನಾಯ ನಮಃ~। ಶ್ವೇತದೂರ್ವಾಯುಗ್ಮಂ ಸಮರ್ಪಯಾಮಿ ॥\\
ಓಂ ಲಂಬೋದರಾಯ ನಮಃ~। ಬದರೀಪತ್ರಂ ಸಮರ್ಪಯಾಮಿ ॥\\
ಓಂ ಹರಸೂನವೇ ನಮಃ~। ಧತ್ತೂರಪತ್ರಂ ಸಮರ್ಪಯಾಮಿ ॥\\
ಓಂ ಗಜಕರ್ಣಕಾಯ ನಮಃ~। ಕೇತಕೀಪತ್ರಂ ಸಮರ್ಪಯಾಮಿ ॥\\
ಓಂ ಗುಹಾಗ್ರಜಾಯ ನಮಃ~। ಅಪಾಮಾರ್ಗಪತ್ರಂ ಸಮರ್ಪಯಾಮಿ ॥\\
ಓಂ ಏಕದಂತಾಯ ನಮಃ~। ಚೂತಪತ್ರಂ ಸಮರ್ಪಯಾಮಿ ॥\\
ಓಂ ಇಭವಕ್ತ್ರಾಯ ನಮಃ~। ಶಮೀಪತ್ರಂ ಸಮರ್ಪಯಾಮಿ ॥\\
ಓಂ ವಿನಾಯಕಾಯ ನಮಃ~। ಕರವೀರಪತ್ರಂ ಸಮರ್ಪಯಾಮಿ ॥\\
ಓಂ ಕಪಿಲಾಯ ನಮಃ~। ಅರ್ಕಪತ್ರಂ ಸಮರ್ಪಯಾಮಿ ॥\\
ಓಂ ವರಪ್ರದಾಯ ನಮಃ~। ಅರ್ಜುನಪತ್ರಂ ಸಮರ್ಪಯಾಮಿ ॥\\
ಓಂ ವಕ್ರತುಂಡಾಯ ನಮಃ~। ಆಮಲಕಪತ್ರಂ ಸಮರ್ಪಯಾಮಿ ॥\\
ಓಂ ವಿಘ್ನರಾಜಾಯ ನಮಃ~। ವಿಷ್ಣುಕ್ರಾಂತಿಪತ್ರಂ ಸಮರ್ಪಯಾಮಿ ॥\\
ಓಂ ವಟವೇ ನಮಃ~। ದಾಡಿಮೀಪತ್ರಂ ಸಮರ್ಪಯಾಮಿ ॥\\
ಓಂ ಸುರಸೇವಿತಾಯ ನಮಃ~। ದೇವದಾರುಪತ್ರಂ ಸಮರ್ಪಯಾಮಿ ॥\\
ಓಂ ಫಾಲಚಂದ್ರಾಯ ನಮಃ~। ಮರುಗಪತ್ರಂ ಸಮರ್ಪಯಾಮಿ ॥\\
ಓಂ ಹೇರಂಬಾಯ ನಮಃ~। ಸಿಂದೂರಪತ್ರಂ ಸಮರ್ಪಯಾಮಿ ॥\\
ಓಂ ಶೂರ್ಪಕರ್ಣಾಯ ನಮಃ~। ಜಾಜೀಪತ್ರಂ ಸಮರ್ಪಯಾಮಿ ॥\\
ಓಂ ಸರ್ವೇಶ್ವರಾಯ ನಮಃ~। ಅಗಸ್ತ್ಯಪತ್ರಂ ಸಮರ್ಪಯಾಮಿ ॥\\
ಪತ್ರಪೂಜಾಂ ಸಮರ್ಪಯಾಮಿ ॥
\section{ಪುಷ್ಪಪೂಜಾ ॥}
ಓಂ ಸುಮುಖಾಯ ನಮಃ~। ದಾಡಿಮೀ ಪುಷ್ಪಂ ಸಮರ್ಪಯಾಮಿ ॥\\
ಓಂ ಏಕದಂತಾಯ ನಮಃ~। ಯೂಥಿಕಾ ಪುಷ್ಪಂ ಸಮರ್ಪಯಾಮಿ ॥\\
ಓಂ ಕಪಿಲಾಯ ನಮಃ~। ಮಲ್ಲಿಕಾ ಪುಷ್ಪಂ ಸಮರ್ಪಯಾಮಿ ॥\\
ಓಂ ಗಜಕರ್ಣಕಾಯ ನಮಃ~। ಚಂಪಕ ಪುಷ್ಪಂ ಸಮರ್ಪಯಾಮಿ ॥\\
ಓಂ ಲಂಬೋದರಾಯ ನಮಃ~। ಕಲ್ಹಾರ ಪುಷ್ಪಂ ಸಮರ್ಪಯಾಮಿ ॥\\
ಓಂ ವಿಕಟಾಯ ನಮಃ~। ಕೇತಕೀ ಪುಷ್ಪಂ ಸಮರ್ಪಯಾಮಿ ॥\\
ಓಂ ವಿಘ್ನನಾಶಿನೇ ನಮಃ~। ಬಕುಲ ಪುಷ್ಪಂ ಸಮರ್ಪಯಾಮಿ ॥\\
ಓಂ ಧೂಮ್ರಕೇತವೇ ನಮಃ~। ಶತಪತ್ರ ಪುಷ್ಪಂ ಸಮರ್ಪಯಾಮಿ ॥\\
ಓಂ ಗಣಾಧ್ಯಕ್ಷಾಯ ನಮಃ~। ಚೂತ ಪುಷ್ಪಂ ಸಮರ್ಪಯಾಮಿ ॥\\
ಓಂ ಫಾಲಚಂದ್ರಾಯ ನಮಃ~। ನಂದ್ಯಾವರ್ತ ಪುಷ್ಪಂ ಸಮರ್ಪಯಾಮಿ ॥\\
ಓಂ ಗಜಾನನಾಯ ನಮಃ~। ಕುಮುದ ಪುಷ್ಪಂ ಸಮರ್ಪಯಾಮಿ ॥\\
ಓಂ ಗುಹಾಗ್ರಜಾಯ ನಮಃ~। ಅಶೋಕ ಪುಷ್ಪಂ ಸಮರ್ಪಯಾಮಿ ॥\\
ಓಂ ಭಕ್ತಪ್ರಿಯಾಯ ನಮಃ~। ಜಾತೀ ಪುಷ್ಪಂ ಸಮರ್ಪಯಾಮಿ ॥\\
ಓಂ ಗೌರೀಪುತ್ರಾಯ ನಮಃ~। ಆಮಲಕ ಪುಷ್ಪಂ ಸಮರ್ಪಯಾಮಿ ॥\\
ಓಂ ಗಣಂಜಯಾಯ ನಮಃ~। ಅರ್ಕ ಪುಷ್ಪಂ ಸಮರ್ಪಯಾಮಿ ॥\\
ಓಂ ಹೇರಂಬಾಯ ನಮಃ~। ಬದರೀ ಪುಷ್ಪಂ ಸಮರ್ಪಯಾಮಿ ॥\\
ಓಂ ದ್ವೈಮಾತುರಾಯ ನಮಃ~। ಪಾರಿಜಾತ ಪುಷ್ಪಂ ಸಮರ್ಪಯಾಮಿ ॥\\
ಓಂ ಕುಮಾರಗುರವೇ ನಮಃ~। ಜಂಬೀರ ಪುಷ್ಪಂ ಸಮರ್ಪಯಾಮಿ ॥\\
ಓಂ ಅವ್ಯಕ್ತಾಯ ನಮಃ~। ಕಮಲ ಪುಷ್ಪಂ ಸಮರ್ಪಯಾಮಿ ॥\\
ಓಂ ಬೀಜಾಪೂರಾಯ ನಮಃ~। ನಾಗಲಿಂಗ ಪುಷ್ಪಂ ಸಮರ್ಪಯಾಮಿ ॥\\
ಓಂ ಸಿದ್ಧಿವಿನಾಯಕಾಯ ನಮಃ~। ಕರ್ಣಿಕಾರ ಪುಷ್ಪಂ ಸಮರ್ಪಯಾಮಿ ॥\\
ಪುಷ್ಪಪೂಜಾಂ ಸಮರ್ಪಯಾಮಿ ॥

\section{ಶ್ರೀದಕ್ಷಿಣಾಮೂರ್ತಿ ಅಷ್ಟೋತ್ತರ ಶತನಾಮಸ್ತೋತ್ರಂ }
\dhyana{ವಟವೃಕ್ಷ ತಟಾಸೀನಂ ಯೋಗಿ ಧ್ಯೇಯಾಂಘ್ರಿ ಪಂಕಜಂ।\\
ಶರಚ್ಚಂದ್ರ ನಿಭಂ ಪೂಜ್ಯಂ ಜಟಾಮುಕುಟ ಮಂಡಿತಂ ॥

ಗಂಗಾಧರಂ ಲಲಾಟಾಕ್ಷಂ ವ್ಯಾಘ್ರ ಚರ್ಮಾಂಬರಾವೃತಂ।\\
ನಾಗಭೂಷಂ ಪರಂ ಬ್ರಹ್ಮ ದ್ವಿಜರಾಜಾವತಂಸಕಂ ॥

ಅಕ್ಷಮಾಲಾ ಜ್ಞಾನಮುದ್ರಾ ವೀಣಾ ಪುಸ್ತಕ ಶೋಭಿತಂ।\\
ಶುಕಾದಿ ವೃದ್ಧ ಶಿಷ್ಯಾಢ್ಯಂ ವೇದ ವೇದಾಂತಗೋಚರಂ ।\\
ಯುವಾನಂ ಮನ್ಮಥಾರಾತಿಂ ದಕ್ಷಿಣಾಮೂರ್ತಿಮಾಶ್ರಯೇ॥}

ಓಂ ವಿದ್ಯಾರೂಪೀ ಮಹಾಯೋಗೀ ಶುದ್ಧ ಜ್ಞಾನೀ ಪಿನಾಕಧೃತ್~।\\
ರತ್ನಾಲಂಕೃತ ಸರ್ವಾಂಗೀ ರತ್ನಮೌಳಿರ್ಜಟಾಧರಃ ॥೧॥

ಗಂಗಾಧಾರ್ಯಚಲಾವಾಸೀ ಮಹಾಜ್ಞಾನೀ ಸಮಾಧಿಕೃತ್।\\
ಅಪ್ರಮೇಯೋ ಯೋಗನಿಧಿಸ್ತಾರಕೋ ಭಕ್ತವತ್ಸಲಃ॥೨॥

ಬ್ರಹ್ಮರೂಪೀ ಜಗದ್ವ್ಯಾಪೀ ವಿಷ್ಣುಮೂರ್ತಿಃ ಪುರಾತನಃ~।\\
ಉಕ್ಷವಾಹಶ್ಚರ್ಮವಾಸಾಃ ಪೀತಾಂಬರ ವಿಭೂಷಣಃ॥೩॥

ಮೋಕ್ಷದಾಯೀ ಮೋಕ್ಷ ನಿಧಿಶ್ಚಾಂಧಕಾರಿರ್ಜಗತ್ಪತಿಃ।\\
ವಿದ್ಯಾಧಾರೀ ಶುಕ್ಲ ತನುಃ ವಿದ್ಯಾದಾಯೀ ಗಣಾಧಿಪಃ॥೪॥

ಪ್ರೌಢಾಪಸ್ಮೃತಿ ಸಂಹರ್ತಾ ಶಶಿಮೌಳಿರ್ಮಹಾಸ್ವನಃ~।\\
ಸಾಮ ಪ್ರಿಯೋಽವ್ಯಯಃ ಸಾಧುಃ ಸರ್ವ ವೇದೈರಲಂಕೃತಃ ॥೫॥

ಹಸ್ತೇ ವಹ್ನಿಧರಃ ಶ್ರೀಮಾನ್ ಮೃಗಧಾರೀ ವಶಂಕರಃ~।\\
ಯಜ್ಞನಾಥಃ ಕ್ರತುಧ್ವಂಸೀ ಯಜ್ಞಭೋಕ್ತಾ ಯಮಾಂತಕಃ॥೬॥

ಭಕ್ತಾನುಗ್ರಹ ಮೂರ್ತಿಶ್ಚ ಭಕ್ತಸೇವ್ಯೋ ವೃಷಧ್ವಜಃ~।\\
ಭಸ್ಮೋದ್ಧೂಲಿತ ಸರ್ವಾಂಗಃ ಚಾಕ್ಷಮಾಲಾಧರೋ ಮಹಾನ್ ॥೭॥
\newpage
ತ್ರಯೀಮೂರ್ತಿಃ ಪರಂಬ್ರಹ್ಮ ನಾಗರಾಜೈರಲಂಕೃತಃ~।\\
ಶಾಂತರೂಪೋ ಮಹಾಜ್ಞಾನೀ ಸರ್ವ ಲೋಕ ವಿಭೂಷಣಃ ॥೮॥

ಅರ್ಧನಾರೀಶ್ವರೋ ದೇವೋ ಮುನಿಸೇವ್ಯಸ್ಸುರೋತ್ತಮಃ~।\\
ವ್ಯಾಖ್ಯಾನದೇವೋ ಭಗವಾನ್ ರವಿ ಚಂದ್ರಾಗ್ನಿ ಲೋಚನಃ ॥೯॥

ಜಗದ್ಗುರುರ್ಮಹಾದೇವೋ ಮಹಾನಂದ ಪರಾಯಣಃ~।\\
ಜಟಾಧಾರೀ ಮಹಾಯೋಗೀ ಜ್ಞಾನಮಾಲೈರಲಂಕೃತಃ ॥೧೦॥

ವ್ಯೋಮಗಂಗಾ ಜಲ ಸ್ಥಾನಃ ವಿಶುದ್ಧೋ ಯತಿರೂರ್ಜಿತಃ~।\\
ತತ್ತ್ವಮೂರ್ತಿರ್ಮಹಾಯೋಗೀ ಮಹಾಸಾರಸ್ವತಪ್ರದಃ ॥೧೧॥

ವ್ಯೋಮಮೂರ್ತಿಶ್ಚ ಭಕ್ತಾನಾಂ ಇಷ್ಟಃ ಕಾಮಫಲಪ್ರದಃ~।\\
ಪರಮೂರ್ತಿಃ ಚಿತ್ಸ್ವರೂಪೀ ತೇಜೋಮೂರ್ತಿರನಾಮಯಃ ॥೧೨॥

ವೇದವೇದಾಂಗ ತತ್ತ್ವಜ್ಞಃ ಚತುಃಷಷ್ಟಿ ಕಲಾನಿಧಿಃ~।\\
ಭವರೋಗ ಭಯಧ್ವಂಸೀ ಭಕ್ತಾನಾಮಭಯಪ್ರದಃ ॥೧೩॥

ನೀಲಗ್ರೀವೋ ಲಲಾಟಾಕ್ಷೋ ಗಜ ಚರ್ಮಾಗತಿಪ್ರದಃ~।\\
ಅರಾಗೀ ಕಾಮದಶ್ಚಾಥ ತಪಸ್ವೀ ವಿಷ್ಣುವಲ್ಲಭಃ ॥೧೪॥

ಬ್ರಹ್ಮಚಾರೀ ಚ ಸನ್ಯಾಸೀ ಗೃಹಸ್ಥಾಶ್ರಮ ಕಾರಣಃ~।\\
ದಾಂತಃ ಶಮವತಾಂ ಶ್ರೇಷ್ಠೋ ಸತ್ಯರೂಪೋ ದಯಾಪರಃ ॥೧೫॥

ಯೋಗಪಟ್ಟಾಭಿರಾಮಶ್ಚ ವೀಣಾಧಾರೀ ವಿಚೇತನಃ~।\\
ಮತಿಪ್ರಜ್ಞಾ ಸುಧಾಧಾರೀ ಮುದ್ರಾಪುಸ್ತಕ ಧಾರಣಃ ॥೧೬॥

ವೇತಾಲಾದಿ ಪಿಶಾಚೌಘ ರಾಕ್ಷಸೌಘ ವಿನಾಶನಃ~।\\
ರಾಜ ಯಕ್ಷ್ಮಾದಿ ರೋಗಾಣಾಂ ವಿನಿಹಂತಾ ಸುರೇಶ್ವರಃ ॥೧೭॥
%\authorline{॥ಇತಿ ಶ್ರೀ ದಕ್ಷಿಣಾಮೂರ್ತಿ ಅಷ್ಟೋತ್ತರ ಶತನಾಮ ಸ್ತೋತ್ರಂ ಸಂಪೂರ್ಣಂ ॥}
\section{ಶ್ರೀವಿಘ್ನೇಶ್ವರಾಷ್ಟೋತ್ತರ ಶತನಾಮಸ್ತೋತ್ರಂ}
\dhyana{ಗಜವದನಮಚಿಂತ್ಯಂ ತೀಕ್ಷ್ಣದಂಷ್ಟ್ರಂ ತ್ರಿಣೇತ್ರಂ\\
ಬೃಹದುದರಮಶೇಷಂ ಭೂತಿರೂಪಂ ಪುರಾಣಮ್~।\\
ಅಮರವರಸುಪೂಜ್ಯಂ ರಕ್ತವರ್ಣಂ ಸುರೇಶಂ\\
 ಪಶುಪತಿಸುತಮೀಶಂ ವಿಘ್ನರಾಜಂ ನಮಾಮಿ ॥}
  
ವಿನಾಯಕೋ ವಿಘ್ನರಾಜೋ ಗೌರೀಪುತ್ರೋ ಗಣೇಶ್ವರಃ~।\\
ಸ್ಕಂದಾಗ್ರಜೋಽವ್ಯಯಃ ಪೂತೋ ದಕ್ಷೋಽಧ್ಯಕ್ಷೋ ದ್ವಿಜಪ್ರಿಯಃ ॥೧॥

ಅಗ್ನಿಗರ್ವಚ್ಛಿದಿಂದ್ರಶ್ರೀಪ್ರದೋ ವಾಣೀಬಲಪ್ರದಃ~।\\
ಸರ್ವಸಿದ್ಧಿಪ್ರದಶ್ಶರ್ವತನಯಃ ಶರ್ವರೀಪ್ರಿಯಃ ॥೨॥

ಸರ್ವಾತ್ಮಕಃ ಸೃಷ್ಟಿಕರ್ತಾ ದೇವಾನೀಕಾರ್ಚಿತಶ್ಶಿವಃ~।\\
ಶುದ್ಧೋ ಬುದ್ಧಿಪ್ರಿಯಶ್ಶಾಂತೋ ಬ್ರಹ್ಮಚಾರೀ ಗಜಾನನಃ ॥೩॥

ದ್ವೈಮಾತ್ರೇಯೋ ಮುನಿಸ್ತುತ್ಯೋ ಭಕ್ತವಿಘ್ನವಿನಾಶನಃ~।\\
ಏಕದಂತಶ್ಚತುರ್ಬಾಹುಶ್ಚತುರಶ್ಶಕ್ತಿಸಂಯುತಃ ॥೪॥

ಲಂಬೋದರಶ್ಶೂರ್ಪಕರ್ಣೋ ಹರಿರ್ಬ್ರಹ್ಮ ವಿದುತ್ತಮಃ~।\\
ಕಾಲೋ ಗ್ರಹಪತಿಃ ಕಾಮೀ ಸೋಮಸೂರ್ಯಾಗ್ನಿಲೋಚನಃ ॥೫॥

ಪಾಶಾಂಕುಶಧರಶ್ಚಂಡೋ ಗುಣಾತೀತೋ ನಿರಂಜನಃ~।\\
ಅಕಲ್ಮಷಸ್ಸ್ವಯಂಸಿದ್ಧಸ್ಸಿದ್ಧಾರ್ಚಿತಪದಾಂಬುಜಃ ॥೬॥

ಬೀಜಪೂರಫಲಾಸಕ್ತೋ ವರದಶ್ಶಾಶ್ವತಃ ಕೃತೀ~।\\
ದ್ವಿಜಪ್ರಿಯೋ ವೀತಭಯೋ ಗದೀ ಚಕ್ರೀಕ್ಷುಚಾಪಧೃತ್ ॥೭॥

ಶ್ರೀದೋಽಜ ಉತ್ಪಲಕರಃ ಶ್ರೀಪತಿಃ ಸ್ತುತಿಹರ್ಷಿತಃ~।\\
ಕುಲಾದ್ರಿಭೇತ್ತಾ ಜಟಿಲಃ ಕಲಿಕಲ್ಮಷನಾಶನಃ ॥೮॥

ಚಂದ್ರಚೂಡಾಮಣಿಃ ಕಾಂತಃ ಪಾಪಹಾರೀ ಸಮಾಹಿತಃ~।\\
ಆಶ್ರಿತಶ್ಶ್ರೀಕರಸ್ಸೌಮ್ಯೋ ಭಕ್ತವಾಂಛಿತದಾಯಕಃ ॥೯॥

ಶಾಂತಃ ಕೈವಲ್ಯಸುಖದಸ್ಸಚ್ಚಿದಾನಂದವಿಗ್ರಹಃ~।\\
ಜ್ಞಾನೀ ದಯಾಯುತೋ ದಾಂತೋ ಬ್ರಹ್ಮ ದ್ವೇಷವಿವರ್ಜಿತಃ ॥೧೦॥

ಪ್ರಮತ್ತದೈತ್ಯಭಯದಃ ಶ್ರೀಕಂಠೋ ವಿಬುಧೇಶ್ವರಃ~।\\
ರಮಾರ್ಚಿತೋವಿಧಿರ್ನಾಗರಾಜಯಜ್ಞೋಪವೀತಕಃ ॥೧೧॥

ಸ್ಥೂಲಕಂಠಃ ಸ್ವಯಂಕರ್ತಾ ಸಾಮಘೋಷಪ್ರಿಯಃ ಪರಃ~।\\
ಸ್ಥೂಲತುಂಡೋಽಗ್ರಣೀರ್ಧೀರೋ ವಾಗೀಶಸ್ಸಿದ್ಧಿದಾಯಕಃ ॥೧೨॥

ದೂರ್ವಾಬಿಲ್ವಪ್ರಿಯೋಽವ್ಯಕ್ತಮೂರ್ತಿರದ್ಭುತಮೂರ್ತಿಮಾನ್~।\\
ಶೈಲೇಂದ್ರತನುಜೋತ್ಸಂಗಖೇಲನೋತ್ಸುಕಮಾನಸಃ ॥೧೩॥

ಸ್ವಲಾವಣ್ಯಸುಧಾಸಾರೋ ಜಿತಮನ್ಮಥವಿಗ್ರಹಃ~।\\
ಸಮಸ್ತಜಗದಾಧಾರೋ ಮಾಯೀ ಮೂಷಕವಾಹನಃ ॥೧೪॥

ಹೃಷ್ಟಸ್ತುಷ್ಟಃ ಪ್ರಸನ್ನಾತ್ಮಾ ಸರ್ವಸಿದ್ಧಿಪ್ರದಾಯಕಃ~।\\
ಅಷ್ಟೋತ್ತರಶತೇನೈವಂ ನಾಮ್ನಾಂ ವಿಘ್ನೇಶ್ವರಂ ವಿಭುಂ ॥೧೫॥

ತುಷ್ಟಾವ ಶಂಕರಃ ಪುತ್ರಂ ತ್ರಿಪುರಂ ಹಂತುಮುದ್ಯತಃ~।\\
ಯಃ ಪೂಜಯೇದನೇನೈವ ಭಕ್ತ್ಯಾ ಸಿದ್ಧಿವಿನಾಯಕಂ ॥೧೬॥

ದೂರ್ವಾದಲೈರ್ಬಿಲ್ವಪತ್ರೈಃ ಪುಷ್ಪೈರ್ವಾ ಚಂದನಾಕ್ಷತೈಃ~।\\
ಸರ್ವಾನ್ಕಾಮಾನವಾಪ್ನೋತಿ ಸರ್ವವಿಘ್ನೈಃ ಪ್ರಮುಚ್ಯತೇ ॥
\authorline{ಇತಿ ಶ್ರೀವಿಘ್ನೇಶ್ವರಾಷ್ಟೋತ್ತರ ಶತನಾಮಸ್ತೋತ್ರಂ}
\newpage
ದಶಾಂಗಂ ಗುಗ್ಗುಲಂ ಧೂಪಂ ಸುಗಂಧಂ ಚ ಮನೋಹರಂ~।\\
ಕಪಿಲಾಘೃತಸಂಯುಕ್ತಂ ಗೃಹಾಣ ಗಣನಾಯಕ॥ಧೂಪಃ॥

ಭಕ್ತ್ಯಾ ದೀಪಂ ಪ್ರಯಚ್ಛಾಮಿ ದೇವಾಯ ಪರಮಾತ್ಮನೇ~।\\
ತ್ರಾಹಿ ಮಾಂ ನರಕಾತ್ ಘೋರಾತ್ ವಿಘ್ನರಾಜ ನಮೋಽಸ್ತು ತೇ ॥ದೀಪಃ॥

ನೈವೇದ್ಯಂ ಷಡ್ರಸೋಪೇತಂ ಪಂಚಭಕ್ಷ್ಯಸಮನ್ವಿತಮ್~।\\
ಚೂತಾದಿ ಫಲಸಂಯುಕ್ತಂ ಗೃಹ್ಯತಾಂ ಗಣನಾಯಕ ॥ನೈವೇದ್ಯಮ್॥

ಪೂಗೀಫಲಸಮಾಯುಕ್ತಂ ನಾಗವಲ್ಲೀ ದಲೈರ್ಯುತಂ~।\\
ಕರ್ಪೂರಚೂರ್ಣಸಂಯುಕ್ತಂ ತಾಂಬೂಲಂ ಪ್ರತಿಗೃಹ್ಯತಾಂ ॥ತಾಂಬೂಲಮ್॥

ಇದಂ ಫಲಂ ಮಯಾದೇವ ಸ್ಥಾಪಿತಂ ಪುರತಸ್ತವ~।\\
ತೇನ ಮೇ ಸುಫಲಾವಾಪ್ತಿಃ ಭವೇಜ್ಜನ್ಮನಿ ಜನ್ಮನಿ ॥ಪೂರ್ಣಫಲಮ್॥

ಹಿರಣ್ಯಗರ್ಭಗರ್ಭಸ್ಥಂ ಹೇಮಬೀಜಂ ವಿಭಾವಸೋಃ~।\\
ಅನಂತಪುಣ್ಯಫಲದಂ ಅತಃ ಶಾಂತಿಂ ಪ್ರಯಚ್ಛ ಮೇ ॥ಸುವರ್ಣಪುಷ್ಪದಕ್ಷಿಣಾ॥

ಘೃತವರ್ತಿ ಸಮಾಯುಕ್ತಂ ಕರ್ಪೂರಶಕಲೈಸ್ತಥಾ~।\\
ನೀರಾಜನಂ ಮಯಾ ದತ್ತಂ ಗೃಹಾಣ ದ್ವಿರದಾನನ ॥ನೀರಾಜನಮ್॥

ಗಣಾಧಿಪನಮಸ್ತೇಽಸ್ತು ಉಮಾಪುತ್ರಾಘನಾಶನ~।\\
ವಿನಾಯಕೇಶತನಯ ಸರ್ವಸಿದ್ಧಿಪ್ರದಾಯಕ ॥

ಏಕದಂತೇಭವದನ ತಥಾ ಮೂಷಕವಾಹನ~।\\
ಕುಮಾರಗುರವೇ ತುಭ್ಯಂ ಅರ್ಪಯಾಮಿ ಸುಮಾಂಜಲಿಂ ॥ಮಂತ್ರಪುಷ್ಪಂ॥

ಪ್ರದಕ್ಷಿಣಂ ಕರಿಷ್ಯಾಮಿ ಸತತಂ ಮೋದಕಪ್ರಿಯ~।\\
ನಮಸ್ತೇ ವಿಘ್ನರಾಜಾಯ ನಮಸ್ತೇ ವಿಘ್ನನಾಶನ ॥ಪ್ರದಕ್ಷಿಣಂ॥

ನಮಸ್ತುಭ್ಯಂ ಗಣೇಶಾಯ ನಮಸ್ತೇ ವಿಘ್ನನಾಶನ~।\\
ಈಪ್ಸಿತಂ ಮೇ ವರಂ ದೇಹಿ ಪರತ್ರ ಚ ಪರಾಂ ಗತಿಂ ॥ನಮಸ್ಕಾರಾಃ ॥

ಅರ್ಘ್ಯಂ ಗೃಹಾಣ ಹೇರಂಬ ಸರ್ವಭದ್ರಪ್ರದಾಯಕ~।\\
ಗಂಧಪುಷ್ಪಾಕ್ಷತೈರ್ಯುಕ್ತಂ ಪಾತ್ರಸ್ಥಂ ಪಾಪನಾಶನ ॥ಪ್ರಸನ್ನಾರ್ಘ್ಯಮ್॥

ನಾರೋಗ್ಯಮದ್ಯ ಕಲಯೇ ನ ಚ ಭೋಗ್ಯಜಾತಂ\\
ಭಾಗ್ಯಂ ನ ವಾ ವಿಷಯಲಂಪಟತಾನಿದಾನಂ~।\\
ವೈರಾಗ್ಯಮೇವ ಸುತರಾಮುಪಲಂಭಯನ್ ಮಾಂ\\
ಸ್ವಾಮಿನ್ ಸಮುದ್ಧರ ಭವಾಂಬುನಿಧೇರ್ದುರಂತಾತ್ ॥ಪ್ರಾರ್ಥನಾ॥

\dhyana{ಶ್ರೀಸಿದ್ಧಮಾನವಮುಖಾ ಗುರವಃ ಸ್ವರೂಪಂ\\
ಸಂಸಾರದಾಹಶಮನಂ ದ್ವಿಭುಜಂ ತ್ರಿನೇತ್ರಂ ।\\
ವಾಮಾಂಗಶಕ್ತಿಸಕಲಾಭರಣೈರ್ವಿಭೂಷಂ\\
ಧ್ಯಾಯೇಜ್ಜಪೇತ್ ಸಕಲಸಿದ್ಧಿಫಲಪ್ರದಂ ಚ ॥}

ಓಂ ನಮಃ ಪ್ರಕಾಶಾನಂದನಾಥಃ ತು ಶಿಖಾಯಾಂ ಪಾತು ಮೇ ಸದಾ ।\\
ಪರಶಿವಾನಂದನಾಥಃ ಶಿರೋ ಮೇ ರಕ್ಷಯೇತ್ ಸದಾ ॥೧॥

ಪರಶಕ್ತಿದಿವ್ಯಾನಂದನಾಥೋ ಭಾಲೇ ಚ ರಕ್ಷತು ।\\
ಕಾಮೇಶ್ವರಾನಂದನಾಥೋ ಮುಖಂ ರಕ್ಷತು ಸರ್ವಧೃಕ್ ॥೨॥

ದಿವ್ಯೌಘೋ ಮಸ್ತಕಂ ದೇವಿ ಪಾತು ಸರ್ವಶಿರಃ ಸದಾ ।\\
ಕಂಠಾದಿನಾಭಿಪರ್ಯಂತಂ ಸಿದ್ಧೌಘಾ ಗುರವಃ ಪ್ರಿಯೇ ॥೩॥

ಭೋಗಾನಂದನಾಥ ಗುರುಃ ಪಾತು ದಕ್ಷಿಣಬಾಹುಕಂ ।\\
ಸಮಯಾನಂದನಾಥಶ್ಚ ಸಂತತಂ ಹೃದಯೇಽವತು ॥೪॥
\newpage
ಸಹಜಾನಂದನಾಥಶ್ಚ ಕಟಿಂ ನಾಭಿಂ ಚ ರಕ್ಷತು ।\\
ಏಷು ಸ್ಥಾನೇಷು ಸಿದ್ಧೌಘಾಃ ರಕ್ಷಂತು ಗುರವಃ ಸದಾ ॥೫॥

ಅಧರೇ ಮಾನವೌಘಾಶ್ಚ ಗುರವಃ ಕುಲನಾಯಿಕೇ ।\\
ಗಗನಾನಂದನಾಥಶ್ಚ ಗುಲ್ಫಯೋಃ ಪಾತು ಸರ್ವದಾ ॥೬॥

ನೀಲೌಘಾನಂದನಾಥಶ್ಚ ರಕ್ಷಯೇತ್ ಪಾದಪೃಷ್ಠತಃ ।\\
ಸ್ವಾತ್ಮಾನಂದನಾಥಗುರುಃ ಪಾದಾಂಗುಲೀಶ್ಚ ರಕ್ಷತು ॥೭॥

ಕಂದೋಲಾನಂದನಾಥಶ್ಚ ರಕ್ಷೇತ್ ಪಾದತಲೇ ಸದಾ ।\\
ಇತ್ಯೇವಂ ಮಾನವೌಘಾಶ್ಚ ನ್ಯಸೇನ್ನಾಭ್ಯಾದಿಪಾದಯೋಃ ॥೮॥

ಗುರುರ್ಮೇ ರಕ್ಷಯೇದುರ್ವ್ಯಾಂ ಸಲಿಲೇ ಪರಮೋ ಗುರುಃ ।\\
ಪರಾಪರಗುರುರ್ವಹ್ನೌ ರಕ್ಷಯೇತ್ ಶಿವವಲ್ಲಭೇ ॥೯॥

ಪರಮೇಷ್ಠೀಗುರುಶ್ಚೈವ ರಕ್ಷಯೇತ್ ವಾಯುಮಂಡಲೇ ।\\
ಶಿವಾದಿಗುರವಃ ಸಾಕ್ಷಾತ್ ಆಕಾಶೇ ರಕ್ಷಯೇತ್ ಸದಾ ॥೧೦॥

ಇಂದ್ರೋ ಗುರುಃ ಪಾತು ಪೂರ್ವೇ ಆಗ್ನೇಯ್ಯಾಂ ಗುರುರಗ್ನಯಃ ।\\
ದಕ್ಷೇ ಯಮೋ ಗುರುಃ ಪಾತು ನೈಋತ್ಯಾಂ ನಿಋತಿರ್ಗುರುಃ ॥೧೧॥

ವರುಣೋ ಗುರುಃ ಪಶ್ಚಿಮೇ ವಾಯವ್ಯಾಂ ಮಾರುತೋ ಗುರುಃ ।\\
ಉತ್ತರೇ ಧನದಃ ಪಾತು ಐಶಾನ್ಯಾಮೀಶ್ವರೋ ಗುರುಃ ॥೧೨॥

ಊರ್ಧ್ವಂ ಪಾತು ಗುರುರ್ಬ್ರಹ್ಮಾ ಅನಂತೋ ಗುರುರಪ್ಯಧಃ ।\\
ಏವಂ ದಶದಿಶಃ ಪಾಂತು ಇಂದ್ರಾದಿಗುರವಃ ಕ್ರಮಾತ್ ॥೧೩॥

ಶಿರಸಃ ಪಾದಪರ್ಯಂತಂ ಪಾಂತು ದಿವ್ಯೌಘಸಿದ್ಧಯಃ ।\\
ಮಾನವೌಘಾಶ್ಚ ಗುರವೋ ವ್ಯಾಪಕಂ ಪಾಂತು ಸರ್ವದಾ ॥೧೪॥

ಸರ್ವತ್ರ ಗುರುರೂಪೇಣ ಸಂರಕ್ಷೇತ್ ಸಾಧಕೋತ್ತಮಂ ।\\
ಆತ್ಮಾನಂ ಗುರುರೂಪಂ ಚ ಧ್ಯಾಯೇನ್ ಮಂತ್ರಂ ಸದಾ ಬುಧಃ ॥೧೫॥
\newpage
ಮನುಶ್ಚಂದ್ರಃ ಕುಬೇರಶ್ಚ ಲೋಪಾಮುದ್ರಾ ಚ ಮನ್ಮಥಃ ।\\
ಅಗಸ್ತಿರಗ್ನಿಃ ಸೂರ್ಯಶ್ಚ ಇಂದ್ರಃ ಸ್ಕಂದಃ ಶಿವಸ್ತಥಾ ॥

ಕ್ರೋಧಭಟ್ಟಾರಕೋ ದೇವ್ಯಾ ದ್ವಾದಶಾಮೀ ಉಪಾಸಕಾಃ ।\\
ಏತಾನ್ ಪ್ರಾತಃ ಸ್ಮರನ್ನಿತ್ಯಂ ಸರ್ವಪಾಪೈಃ ಪ್ರಮುಚ್ಯತೇ ॥

ಕಾಂಚೀಪುರೇ ತು ಕಾಮಾಕ್ಷೀ ಮಲಯೇ ಭ್ರಾಮರೀ ತಥಾ ।\\
ಕೇರಲೇ ತು ಕುಮಾರೀ ಸಾ ಅಂಬಾಽಽನರ್ತೇಷು ಸಂಸ್ಥಿತಾ ॥

ಕರವೀರೇ ಮಹಾಲಕ್ಷ್ಮೀಃ ಕಾಲಿಕಾ ಮಾಲವೇಷು ಸಾ ।\\
ಪ್ರಯಾಗೇ ಲಲಿತಾ ದೇವೀ ವಿಂಧ್ಯೇ ವಿಂಧ್ಯನಿವಾಸಿನೀ ॥

ವಾರಾಣಸ್ಯಾಂ ವಿಶಾಲಾಕ್ಷೀ ಗಯಾಯಾಂ ಮಂಗಲಾವತೀ ।\\
ವಂಗೇಷು ಸುಂದರೀ ದೇವೀ ನೇಪಾಲೇ ಗುಹ್ಯಕೇಶ್ವರೀ ॥

ಇತಿ ದ್ವಾದಶರೂಪೇಣ ಸಂಸ್ಥಿತಾ ಭಾರತೇ ಶಿವಾ ।\\
ಏತಾಸಾಂ ದರ್ಶನಾದೇವ ಸರ್ವಪಾಪೈಃ ಪ್ರಮುಚ್ಯತೇ ॥

ಅಶಕ್ತೋ ದರ್ಶನೇ ನಿತ್ಯಂ ಸ್ಮರೇತ್ ಪ್ರಾತಃ ಸಮಾಹಿತಃ ।\\
ತಥಾಪ್ಯುಪಾಸಕಃ ಸರ್ವೈರಪರಾಧೈರ್ವಿಮುಚ್ಯತೇ ॥
\newpage
\thispagestyle{empty}
\section{ ಚಕ್ರಾರ್ಚನಮ್\\ಅಥ ನ್ಯಾಸಾಃ\\ಲಘುನ್ಯಾಸಃ}
\addcontentsline{toc}{section}{ಲಘುನ್ಯಾಸಃ}
ಓಂ ಅಥಾತ್ಮಾನಂ ಶಿವಾತ್ಮಾನಂ ಶ್ರೀರುದ್ರರೂಪಂ ಧ್ಯಾಯೇತ್ ॥
\dhyana{ಶುದ್ಧಸ್ಫಟಿಕಸಂಕಾಶಂ ತ್ರಿಣೇತ್ರಂ ಪಂಚವಕ್ತ್ರಕಂ ।\\
ಗಂಗಾಧರಂ ದಶಭುಜಂ ಸರ್ವಾಭರಣಭೂಷಿತಂ ॥

ನೀಲಗ್ರೀವಂ ಶಶಾಂಕಾಂಕಂ ನಾಗಯಜ್ಞೋಪವೀತಿನಂ ।\\
ವ್ಯಾಘ್ರಚರ್ಮೋತ್ತರೀಯಂ ಚ ವರೇಣ್ಯಮಭಯಪ್ರದಂ ॥

ಕಮಂಡಲ್ವಕ್ಷಸೂತ್ರಾಭ್ಯಾಮನ್ವಿತಂ ಶೂಲಪಾಣಿನಂ ।\\
ಜ್ವಲಂತಂ ಪಿಂಗಳಜಟಾಶಿಖಾಮುದ್ದ್ಯೋತಧಾರಿಣಂ ॥

ವೃಷಸ್ಕಂಧಸಮಾರೂಢಂ ಉಮಾದೇಹಾರ್ಧಧಾರಿಣಂ ।\\
ಅಮೃತೇನಾಪ್ಲುತಂ ಶಾಂತಂ ದಿವ್ಯಭೋಗಸಮನ್ವಿತಂ ॥

ದಿಗ್ದೇವತಾಸಮಾಯುಕ್ತಂ ಸುರಾಸುರನಮಸ್ಕೃತಂ ।\\
ನಿತ್ಯಂ ಚ ಶಾಶ್ವತಂ ಶುದ್ಧಂ ಧ್ರುವಮಕ್ಷರಮವ್ಯಯಂ ॥

ಸರ್ವವ್ಯಾಪಿನಮೀಶಾನಂ ರುದ್ರಂ ವೈ ವಿಶ್ವರೂಪಿಣಂ ।\\
ಏವಂ ಧ್ಯಾತ್ವಾ ದ್ವಿಜಃ ಸಮ್ಯಕ್ ತತೋ ಯಜನಮಾರಭೇತ್ ॥}

ಅಥಾತ್ಮನಿ ದೇವತಾಃ ಸ್ಥಾಪಯೇತ್ ॥
\newpage
ಪ್ರಜನನೇ ಬ್ರಹ್ಮಾ ತಿಷ್ಠತು । ಪಾದಯೋರ್ವಿಷ್ಣುಸ್ತಿಷ್ಠತು । ಹಸ್ತಯೋರ್ಹರಸ್ತಿಷ್ಠತು । ಬಾಹ್ವೋರಿಂದ್ರಸ್ತಿಷ್ಠತು । ಜಠರೇಽಅಗ್ನಿಸ್ತಿಷ್ಠತು । ಹೃದಯೇ ಶಿವಸ್ತಿಷ್ಠತು । ಕಂಠೇ ವಸವಸ್ತಿಷ್ಠಂತು । ವಕ್ತ್ರೇ ಸರಸ್ವತೀ ತಿಷ್ಠತು । ನಾಸಿಕಯೋರ್ವಾಯುಸ್ತಿಷ್ಠತು । ನಯನಯೋಶ್ಚಂದ್ರಾದಿತ್ಯೌ ತಿಷ್ಠೇತಾಂ । ಕರ್ಣಯೋರಶ್ವಿನೌ ತಿಷ್ಠೇತಾಂ । ಲಲಾಟೇ ರುದ್ರಾಸ್ತಿಷ್ಠಂತು । ಮೂರ್ಧ್ನ್ಯಾದಿತ್ಯಾಸ್ತಿಷ್ಠಂತು । ಶಿರಸಿ ಮಹಾದೇವಸ್ತಿಷ್ಠತು । ಶಿಖಾಯಾಂ ವಾಮದೇವಾಸ್ತಿಷ್ಠತು । ಪೃಷ್ಠೇ ಪಿನಾಕೀ ತಿಷ್ಠತು । ಪುರತಃ ಶೂಲೀ ತಿಷ್ಠತು । ಪಾರ್ಶ್ವಯೋಃ ಶಿವಾಶಂಕರೌ ತಿಷ್ಠೇತಾಂ । ಸರ್ವತೋ ವಾಯುಸ್ತಿಷ್ಠತು । ತತೋ ಬಹಿಃ ಸರ್ವತೋಽಗ್ನಿರ್ಜ್ವಾಲಾಮಾಲಾಪರಿವೃತಸ್ತಿಷ್ಠತು । ಸರ್ವೇಷ್ವಂಗೇಷು ಸರ್ವಾ ದೇವತಾ ಯಥಾಸ್ಥಾನಂ ತಿಷ್ಠಂತು। ಮಾಂ ರಕ್ಷಂತು।

\as{ಅ॒ಗ್ನಿರ್ಮೇ॑} ವಾ॒ಚಿ ಶ್ರಿ॒ತಃ । ವಾಗ್ಘೃದ॑ಯೇ । ಹೃದ॑ಯಂ॒ ಮಯಿ॑ । ಅ॒ಹಮ॒ಮೃತೇ᳚ । ಅ॒ಮೃತಂ॒ ಬ್ರಹ್ಮ॑ಣಿ । \as{ವಾ॒ಯುರ್ಮೇ᳚} ಪ್ರಾ॒ಣೇ ಶ್ರಿ॒ತಃ । ಪ್ರಾ॒ಣೋ ಹೃದ॑ಯೇ । ಹೃದ॑ಯಂ॒ ಮಯಿ॑ । ಅ॒ಹಮ॒ಮೃತೇ᳚ । ಅ॒ಮೃತಂ॒ ಬ್ರಹ್ಮ॑ಣಿ । \as{ಸೂರ್ಯೋ॑} ಮೇ॒ ಚಕ್ಷು॑ಷಿ ಶ್ರಿ॒ತಃ । ಚ॒ಕ್ಷುರ್ಹೃದ॑ಯೇ । ಹೃದ॑ಯಂ॒ ಮಯಿ॑ । ಅ॒ಹಮ॒ಮೃತೇ᳚ । ಅ॒ಮೃತಂ॒ ಬ್ರಹ್ಮ॑ಣಿ । \as{ಚಂ॒ದ್ರ॒ಮಾ} ಮೇ॒ ಮನ॑ಸಿ ಶ್ರಿ॒ತಃ । ಮನೋ॒ ಹೃದ॑ಯೇ । ಹೃದ॑ಯಂ॒ ಮಯಿ॑ । ಅ॒ಹಮ॒ಮೃತೇ᳚ । ಅ॒ಮೃತಂ॒ ಬ್ರಹ್ಮ॑ಣಿ । \as{ದಿಶೋ॑} ಮೇ॒ ಶ್ರೋತ್ರೇ᳚ ಶ್ರಿ॒ತಾಃ । ಶ್ರೋತ್ರ॒ಗ್ಂ॒ ಹೃದ॑ಯೇ । ಹೃದ॑ಯಂ॒ ಮಯಿ॑ । ಅ॒ಹಮ॒ಮೃತೇ᳚ । ಅ॒ಮೃತಂ॒ ಬ್ರಹ್ಮ॑ಣಿ । \as{ಆಪೋ॑} ಮೇ॒ ರೇತ॑ಸಿ ಶ್ರಿ॒ತಾಃ । ರೇ॒ತೋ ಹೃದ॑ಯೇ । ಹೃದ॑ಯಂ॒ ಮಯಿ॑ । ಅ॒ಹಮ॒ಮೃತೇ᳚ । ಅ॒ಮೃತಂ॒ ಬ್ರಹ್ಮ॑ಣಿ । \as{ಪೃ॒ಥಿ॒ವೀ} ಮೇ ಶರೀ॑ರೇ ಶ್ರಿ॒ತಾ । ಶರೀ॑ರ॒ಗ್ಂ ಹೃದ॑ಯೇ । ಹೃದ॑ಯಂ॒ ಮಯಿ॑ । ಅ॒ಹಮ॒ಮೃತೇ᳚ । ಅ॒ಮೃತಂ॒ ಬ್ರಹ್ಮ॑ಣಿ । \as{ಓ॒ಷ॒ಧಿ॒ ವ॒ನ॒ಸ್ಪ॒ತಯೋ॑} ಮೇ॒ ಲೋಮ॑ಸು ಶ್ರಿ॒ತಾಃ । ಲೋಮಾ॑ನಿ॒ ಹೃದ॑ಯೇ । ಹೃದ॑ಯಂ॒ ಮಯಿ॑ । ಅ॒ಹಮ॒ಮೃತೇ᳚ । ಅ॒ಮೃತಂ॒ ಬ್ರಹ್ಮ॑ಣಿ । \as{ಇಂದ್ರೋ॑} ಮೇ॒ ಬಲೇ᳚ ಶ್ರಿ॒ತಃ । ಬಲ॒ಗ್ಂ॒ ಹೃದ॑ಯೇ । ಹೃದ॑ಯಂ॒ ಮಯಿ॑ । ಅ॒ಹಮ॒ಮೃತೇ᳚ । ಅ॒ಮೃತಂ॒ ಬ್ರಹ್ಮ॑ಣಿ । \as{ಪ॒ರ್ಜನ್ಯೋ॑} ಮೇ ಮೂ॒ರ್ಧ್ನಿ ಶ್ರಿ॒ತಃ । ಮೂ॒ರ್ಧಾ ಹೃದ॑ಯೇ । ಹೃದ॑ಯಂ॒ ಮಯಿ॑ । ಅ॒ಹಮ॒ಮೃತೇ᳚ । ಅ॒ಮೃತಂ॒ ಬ್ರಹ್ಮ॑ಣಿ । \as{ಈಶಾ॑ನೋ} ಮೇ ಮ॒ನ್ಯೌ ಶ್ರಿ॒ತಃ । ಮ॒ನ್ಯುರ್-ಹೃದ॑ಯೇ । ಹೃದ॑ಯಂ॒ ಮಯಿ॑ । ಅ॒ಹಮ॒ಮೃತೇ᳚ । ಅ॒ಮೃತಂ॒ ಬ್ರಹ್ಮ॑ಣಿ । \as{ಆ॒ತ್ಮಾ} ಮ॑ ಆ॒ತ್ಮನಿ॑ ಶ್ರಿ॒ತಃ । ಆ॒ತ್ಮಾ ಹೃದ॑ಯೇ । ಹೃದ॑ಯಂ॒ ಮಯಿ॑ । ಅ॒ಹಮ॒ಮೃತೇ᳚ । ಅ॒ಮೃತಂ॒ ಬ್ರಹ್ಮ॑ಣಿ । ಪುನ॑ರ್ಮ ಆ॒ತ್ಮಾ ಪುನ॒ರಾಯು॒ರಾಗಾ᳚ತ್ । ಪುನಃ॑ ಪ್ರಾ॒ಣಃ ಪು॑ನ॒ರಾಕೂ॑ತ॒ಮಾಗಾ᳚ತ್ । ವೈ॒ಶ್ವಾ॒ನ॒ರೋ ರ॒ಶ್ಮಿಭಿ॑ರ್ವಾವೃಧಾ॒ನಃ । ಅಂ॒ತಸ್ತಿ॑ಷ್ಠತ್ವ॒ಮೃತ॑ಸ್ಯ ಗೋ॒ಪಾಃ ॥

ಏವಂ ಯಥಾಲಿಂಗಮಂಗಾನಿ ಸಂಮೃಜ್ಯ, ದೇವಮಾತ್ಮಾನಂ ಚ ಪ್ರತ್ಯಾರಾಧಯೇತ್ ॥\\
\dhyana{ಆರಾಧಿತೋ ಮನುಷ್ಯೈಸ್ತ್ವಂ ಸಿದ್ಧೈರ್ದೇವಾಸುರಾದಿಭಿಃ ।\\
ಆರಾಧಯಾಮಿ ಭಕ್ತ್ಯಾ ತ್ವಾಂ ಮಾಂ ಗೃಹಾಣ ಮಹೇಶ್ವರ ॥}

ಆ ತ್ವಾ॑ ವಹಂತು॒ ಹರ॑ಯಃ॒ ಸಚೇ॑ತಸಃ ಶ್ವೇ॒ತೈರಶ್ವೈಃ ᳚  ಸ॒ಹಕೇ॑ತು॒ಮದ್ಭಿಃ॑~। ವಾತಾ॑ಜಿರೈ॒ರ್ಬಲ॑ವದ್ಭಿ॑ರ್ಮನೋ॑ಜವೈ॒ರಾಯಾ॑ಹಿ ಶೀ॒ಘ್ರಂ ಮಮ॑ ಹ॒ವ್ಯಾಯ ಶ॒ರ್ವೋಮ್ । ಈಶಾನಮಾವಾಹಯಾಮೀತ್ಯಾವಾಹ್ಯ\\
\dhyana{ಶಂಕರಸ್ಯ ಚರಿತಂ ಕಥಾಮೃತಂ ಚಂದ್ರಶೇಖರ ಗುಣಾನುಕೀರ್ತನಮ್ ।\\
ನೀಲಕಂಠ ತವ ಪಾದಸೇವನಂ ಸಂಭವಂತು ಮಮ ಜನ್ಮಜನ್ಮನಿ ॥\\
ಸ್ವಾಮಿನ್ ಸರ್ವಜಗನ್ನಾಥ ಯಾವತ್ಪೂಜಾವಸಾನಕಮ್ ।\\
ತಾವತ್ತ್ವಂ ಪ್ರೀತಿ ಭಾವೇನ ಬಿಂಬೇಽಸ್ಮಿನ್ ಸನ್ನಿಧಿಂ ಕುರು ॥}
\section{ಮಾತೃಕಾಸರಸ್ವತೀನ್ಯಾಸಃ}
\addcontentsline{toc}{section}{ಮಾತೃಕಾಸರಸ್ವತೀನ್ಯಾಸಃ}
ಅಸ್ಯ ಶ್ರೀಮಾತೃಕಾಸರಸ್ವತೀ ನ್ಯಾಸಮಂತ್ರಸ್ಯ ಬ್ರಹ್ಮಣೇ ಋಷಯೇ ನಮಃ (ಶಿರಸಿ) ಗಾಯತ್ರೀ ಛಂದಸೇ ನಮಃ(ಮುಖೇ)~। ಮಾತೃಕಾ ಸರಸ್ವತೀ ದೇವತಾಯೈ ನಮಃ (ಹೃದಯೇ)~। ಹಲ್ಭ್ಯೋ ಬೀಜೇಭ್ಯೋ ನಮಃ (ಗುಹ್ಯೇ)~। ಸ್ವರೇಭ್ಯಃ ಶಕ್ತಿಭ್ಯೋ ನಮಃ (ಪಾದಯೋ)~। ಬಿಂದುಭ್ಯಃ ಕೀಲಕೇಭ್ಯೋ ನಮಃ (ನಾಭೌ)~। ಶ್ರೀವಿದ್ಯಾಂಗತ್ವೇನ ನ್ಯಾಸೇ ವಿನಿಯೋಗಾಯ ನಮಃ(ಸರ್ವಾಂಗೇ)~॥

ಅಂ ಆಂ ಇಂ ಈಂ ಉಂ ಊಂ ಋಂ ೠಂ ಲೃಂ ಲೄಂ ಏಂ ಐಂ ಓಂ ಔಂ ಅಂ ಅಃ ಕಂಖಂಗಂಘಂಙಂ ಚಂಛಂಜಂಝಂಞಂ ಟಂಠಂಡಂಢಂಣಂ ತಂಥಂದಂಧಂನಂ ಪಂಫಂಬಂಭಂಮಂ ಯಂರಂಲಂವಂ ಶಂಷಂಸಂಹಂಳಂಕ್ಷಂ~। (ಇತಿ ಅಂಜಲಿನಾ ತ್ರಿಃ ವ್ಯಾಪಕಂ ನ್ಯಸ್ಯ)

ಓಂ ಐಂಹ್ರೀಂಶ್ರೀಂ ಐಂಕ್ಲೀಂಸೌಃ\\ಅಂ ಕಂಖಂಗಂಘಂಙಂ ಆಂ ಅಂಗುಷ್ಠಾಭ್ಯಾಂ ನಮಃ ।\\
೭ ಇಂ ಚಂಛಂಜಂಝಂಞಂ ಈಂ ತರ್ಜನೀಭ್ಯಾಂ ನಮಃ ।\\
೭ ಉಂ ಟಂಠಂಡಂಢಂಣಂ ಊಂ ಮಧ್ಯಮಾಭ್ಯಾಂ ನಮಃ ।\\
೭ ಏಂ ತಂಥಂದಂಧಂನಂ ಐಂ ಅನಾಮಿಕಾಭ್ಯಾಂ ನಮಃ ।\\
೭ ಓಂ ಪಂಫಂಬಂಭಂಮಂ ಔಂ ಕನಿಷ್ಠಿಕಾಭ್ಯಾಂ ನಮಃ ।\\
೭ ಅಂ ಯಂರಂಲಂವಂಶಂಷಂಸಂಹಂಳಂಕ್ಷಂ ಅಃ ಕರತಲಕರಪೃಷ್ಠಾಭ್ಯಾಂ ನಮಃ।\\
ಏವಮೇವಾಂಗನ್ಯಾಸಂ ವಿಧಾಯ ಧ್ಯಾಯೇತ್ \\
\dhyana{ಪಂಚಾಶದ್ವರ್ಣಭೇದೈರ್ವಿಹಿತವದನದೋಃಪಾದಯುಕ್ಕುಕ್ಷಿವಕ್ಷೋ\\
ದೇಶಾಂ ಭಾಸ್ವತ್ಕಪರ್ದಾಕಲಿತಶಶಿಕಲಾಮಿಂದುಕುಂದಾವದಾತಾಂ~।\\
ಅಕ್ಷಸ್ರಕ್ಕುಂಭಚಿಂತಾಲಿಖಿತವರಕರಾಂ ತ್ರೀಕ್ಷಣಾಂ ಪದ್ಮಸಂಸ್ಥಾಂ\\
ಅಚ್ಛಾಕಲ್ಪಾಮತುಚ್ಛಸ್ತನಜಘನಭರಾಂ ಭಾರತೀಂ ತಾಂ ನಮಾಮಿ ॥}\\
ಲಮಿತ್ಯಾದಿನಾ ಪಂಚೋಪಚಾರ ಪೂಜಾ~॥\\
\as{ಓಂ ಐಂಹ್ರೀಂಶ್ರೀಂ ಐಂಕ್ಲೀಂಸೌಃ ಅಂ} ನಮಃ ಹಂಸಃ~।(ಶಿರಸಿ)\\
\as{೭ ಆಂ} ನಮಃ ಹಂಸಃ~।(ಮುಖವೃತ್ತೇ)\\
\as{೭ ಇಂ} ನಮಃ ಹಂಸಃ~।(ದಕ್ಷನೇತ್ರೇ)\\
\as{೭ ಈಂ} ನಮಃ ಹಂಸಃ~।(ವಾಮನೇತ್ರೇ)\\
\as{೭ ಉಂ} ನಮಃ ಹಂಸಃ~।(ದಕ್ಷಕರ್ಣೇ)\\
\as{೭ ಊಂ} ನಮಃ ಹಂಸಃ~।(ವಾಮಕರ್ಣೇ)\\
\as{೭ ಋಂ} ನಮಃ ಹಂಸಃ~।(ದಕ್ಷನಾಸಾಯಾಂ)\\
\as{೭ ೠಂ} ನಮಃ ಹಂಸಃ~।(ವಾಮನಾಸಾಯಾಂ)\\
\as{೭ ಲೃಂ} ನಮಃ ಹಂಸಃ~।(ದಕ್ಷಗಂಡೇ)\\
\as{೭ ಲೄಂ} ನಮಃ ಹಂಸಃ~।(ವಾಮಗಂಡೇ)\\
\as{೭ ಏಂ} ನಮಃ ಹಂಸಃ~।(ಊರ್ಧ್ವೋಷ್ಠೇ)\\
\as{೭ ಐಂ} ನಮಃ ಹಂಸಃ~।(ಅಧರೋಷ್ಠೇ)\\
\as{೭ ಓಂ} ನಮಃ ಹಂಸಃ~।(ಊರ್ಧ್ವದಂತಪಂಕ್ತೌ)\\
\as{೭ ಔಂ} ನಮಃ ಹಂಸಃ~।(ಅಧೋದಂತಪಂಕ್ತೌ)\\
\as{೭ ಅಂ} ನಮಃ ಹಂಸಃ~।(ಜಿಹ್ವಾಯಾಂ)\\
\as{೭ ಅಃ} ನಮಃ ಹಂಸಃ~।(ಕಂಠೇ)\\
\as{೭ ಕಂ} ನಮಃ ಹಂಸಃ~।(ದಕ್ಷ ಬಾಹುಮೂಲೇ)\\
\as{೭ ಖಂ} ನಮಃ ಹಂಸಃ~।(ದಕ್ಷಕೂರ್ಪರೇ)\\
\as{೭ ಗಂ} ನಮಃ ಹಂಸಃ~।(ದಕ್ಷಮಣಿಬಂಧೇ)\\
\as{೭ ಘಂ} ನಮಃ ಹಂಸಃ~।(ದಕ್ಷಕರಾಂಗುಲಿಮೂಲೇ)\\
\as{೭ ಙಂ} ನಮಃ ಹಂಸಃ~।(ದಕ್ಷಕರಾಂಗುಲ್ಯಗ್ರೇ)\\
\as{೭ ಚಂ} ನಮಃ ಹಂಸಃ~।(ವಾಮಬಾಹುಮೂಲೇ)\\
\as{೭ ಛಂ} ನಮಃ ಹಂಸಃ~।(ವಾಮಕೂರ್ಪರೇ)\\
\as{೭ ಜಂ} ನಮಃ ಹಂಸಃ~।(ವಾಮಮಣಿಬಂಧೇ)\\
\as{೭ ಝಂ} ನಮಃ ಹಂಸಃ~।(ವಾಮಕರಾಂಗುಲಿಮೂಲೇ)\\
\as{೭ ಞಂ} ನಮಃ ಹಂಸಃ~।(ವಾಮಕರಾಂಗುಲ್ಯಗ್ರೇ)\\
\as{೭ ಟಂ} ನಮಃ ಹಂಸಃ~।(ದಕ್ಷೋರುಮೂಲೇ)\\
\as{೭ ಠಂ} ನಮಃ ಹಂಸಃ~।(ದಕ್ಷಜಾನುನಿ)\\
\as{೭ ಡಂ} ನಮಃ ಹಂಸಃ~।(ದಕ್ಷಗುಲ್ಫೇ)\\
\as{೭ ಢಂ} ನಮಃ ಹಂಸಃ~।(ದಕ್ಷಪಾದಾಂಗುಲಿಮೂಲೇ)\\
\as{೭ ಣಂ} ನಮಃ ಹಂಸಃ~।(ದಕ್ಷಪಾದಾಂಗುಲ್ಯಗ್ರೇ)\\
\as{೭ ತಂ} ನಮಃ ಹಂಸಃ~।(ವಾಮೋರುಮೂಲೇ)\\
\as{೭ ಥಂ} ನಮಃ ಹಂಸಃ~।(ವಾಮಜಾನುನಿ)\\
\as{೭ ದಂ} ನಮಃ ಹಂಸಃ~।(ವಾಮಗುಲ್ಫೇ)\\
\as{೭ ಧಂ} ನಮಃ ಹಂಸಃ~।(ವಾಮಪಾದಾಂಗುಲಿಮೂಲೇ)\\
\as{೭ ನಂ} ನಮಃ ಹಂಸಃ~।(ವಾಮಪಾದಾಂಗುಲ್ಯಗ್ರೇ)\\
\as{೭ ಪಂ} ನಮಃ ಹಂಸಃ~।(ದಕ್ಷಪಾರ್ಶ್ವೇ)\\
\as{೭ ಫಂ} ನಮಃ ಹಂಸಃ~।(ವಾಮಪಾರ್ಶ್ವೇ)\\
\as{೭ ಬಂ} ನಮಃ ಹಂಸಃ~।(ಪೃಷ್ಠೇ)\\
\as{೭ ಭಂ} ನಮಃ ಹಂಸಃ~।(ನಾಭೌ)\\
\as{೭ ಮಂ} ನಮಃ ಹಂಸಃ~।(ಜಠರೇ)\\
\as{೭ ಯಂ} ನಮಃ ಹಂಸಃ~।(ಹೃದಿ )\\
\as{೭ ರಂ} ನಮಃ ಹಂಸಃ~।(ದಕ್ಷಾಂಸೇ)\\
\as{೭ ಲಂ} ನಮಃ ಹಂಸಃ~।(ಕಕುದಿ)\\
\as{೭ ವಂ} ನಮಃ ಹಂಸಃ~।(ವಾಮಾಂಸೇ)\\
\as{೭ ಶಂ} ನಮಃ ಹಂಸಃ~।(ಹೃದಯಾದಿ ದಕ್ಷ ಕರಾಂಗುಲ್ಯಂತಂ )\\
\as{೭ ಷಂ} ನಮಃ ಹಂಸಃ~।(ಹೃದಯಾದಿ ವಾಮ ಕರಾಂಗುಲ್ಯಂತಂ )\\
\as{೭ ಸಂ} ನಮಃ ಹಂಸಃ~।(ಹೃದಯಾದಿ ದಕ್ಷ ಪಾದಾಂತಂ )\\
\as{೭ ಹಂ} ನಮಃ ಹಂಸಃ~।(ಹೃದಯಾದಿ ವಾಮ ಪಾದಾಂತಂ )\\
\as{೭ ಳಂ} ನಮಃ ಹಂಸಃ~।(ಕಟ್ಯಾದಿ ಪಾದಪರ್ಯಂತಂ)\\
\as{೭ ಕ್ಷಂ} ನಮಃ ಹಂಸಃ~।(ಕಟ್ಯಾದಿ ಶಿರಃಪರ್ಯಂತಂ )\\
\dhyana{ಆಧಾರೇ ಲಿಂಗನಾಭೌ ಹೃದಯಸರಸಿಜೇ ತಾಲುಮೂಲೇ ಲಲಾಟೇ\\
ದ್ವೇ ಪತ್ರೇ ಷೋಡಶಾರೇ ದ್ವಿದಶದಶದಲೇ ದ್ವಾದಶಾರ್ಧೇ ಚತುಷ್ಕೇ~।\\
ವಾಸಾಂತೇ ಬಾಲಮಧ್ಯೇ ಡಫಕಠಸಹಿತೇ ಕಂಠದೇಶೇ ಸ್ವರಾಣಾಂ\\
ಹಂ ಕ್ಷಂ ತತ್ವಾರ್ಥಯುಕ್ತಂ ಸಕಲದಲಗತಂ ವರ್ಣರೂಪಂ ನಮಾಮಿ ॥}

\as{೭ ಅಂ} ನಮಃ ಹಂಸಃ, \as{ಆಂ} ನಮಃ ಹಂಸಃ+ + +\as{ಅಃ} ನಮಃ ಹಂಸಃ॥ಕಂಠೇ\\
\as{೭ ಕಂ} ನಮಃ ಹಂಸಃ + + \as{ಠಂ} ನಮಃ ಹಂಸಃ ॥ಹೃದಯೇ\\
\as{೭ ಡಂ} ನಮಃ ಹಂಸಃ+ + \as{ಫಂ} ನಮಃ ಹಂಸಃ ॥ನಾಭೌ\\
\as{೭ ಬಂ} ನಮಃ ಹಂಸಃ + + \as{ಲಂ} ನಮಃ ಹಂಸಃ ॥ಗುಹ್ಯೇ\\
\as{೭ ವಂ} ನಮಃ ಹಂಸಃ + + \as{ಸಂ} ನಮಃ ಹಂಸಃ ॥ಆಧಾರೇ\\
\as{೭ ಹಂ} ನಮಃ ಹಂಸಃ, \as{ಕ್ಷಂ} ನಮಃ ಹಂಸಃ ॥ಭ್ರೂಮಧ್ಯೇ\\
\as{೭ ಅಂ} ನಮಃ ಹಂಸಃ + +\as{ಕ್ಷಂ} ನಮಃ ಹಂಸಃ॥ಸಹಸ್ರದಲಕಮಲೇ\\
 ಪೂರ್ವವತ್ ಉತ್ತರನ್ಯಾಸಃ ॥
\section{ಕರಶುದ್ಧಿನ್ಯಾಸಃ}
\addcontentsline{toc}{section}{ಕರಶುದ್ಧಿನ್ಯಾಸಃ}
\as{೪ ಅಂ} ನಮಃ ।(ದಕ್ಷಕರಮಧ್ಯೇ)\\
\as{೪ ಆಂ} ನಮಃ ।(ದಕ್ಷಕರಪೃಷ್ಠೇ)\\
\as{೪ ಸೌಃ} ನಮಃ ।(ದಕ್ಷಕರಪಾರ್ಶ್ವಯೋಃ)\\
\as{೪ ಅಂ} ನಮಃ ।(ವಾಮಕರಮಧ್ಯೇ)\\
\as{೪ ಆಂ} ನಮಃ ।(ವಾಮಕರಪೃಷ್ಠೇ)\\
\as{೪ ಸೌಃ} ನಮಃ ।(ವಾಮಕರಪಾರ್ಶ್ವಯೋಃ)\\
\as{೪ ಅಂ} ನಮಃ ।(ಮಧ್ಯಮಯೋಃ)\\
\as{೪ ಆಂ} ನಮಃ ।(ಅನಾಮಿಕಯೋಃ)\\
\as{೪ ಸೌಃ} ನಮಃ ।(ಕನಿಷ್ಠಿಕಯೋಃ)\\
\as{೪ ಅಂ} ನಮಃ ।(ಅಂಗುಷ್ಠಯೋಃ)\\
\as{೪ ಆಂ} ನಮಃ ।(ತರ್ಜನ್ಯೋಃ)\\
\as{೪ ಸೌಃ} ನಮಃ ।(ಉಭಯ ಕರತಲ ಕರಪೃಷ್ಠಯೋಃ)

\section{ಆತ್ಮರಕ್ಷಾನ್ಯಾಸಃ}
\addcontentsline{toc}{section}{ಆತ್ಮರಕ್ಷಾನ್ಯಾಸಃ}
\as{ಓಂ ಐಂಹ್ರೀಂಶ್ರೀಂ ಐಂಕ್ಲೀಂಸೌಃ ।} ಶ್ರೀಮಹಾತ್ರಿಪುರಸುಂದರಿ ಆತ್ಮಾನಂ ರಕ್ಷರಕ್ಷ॥\\ಇತಿ ಹೃದಿ ಅಂಜಲಿಸಮರ್ಪಣಂ~।
\section{ಬಾಲಾಷಡಂಗನ್ಯಾಸಃ}
\addcontentsline{toc}{section}{ಬಾಲಾಷಡಂಗನ್ಯಾಸಃ}
\as{೪ ಐಂ ।} ಹೃದಯಾಯ ನಮಃ\\
\as{೪ ಕ್ಲೀಂ ।} ಶಿರಸೇ ಸ್ವಾಹಾ\\
\as{೪ ಸೌಃ  ।}ಶಿಖಾಯೈ ವಷಟ್\\
\as{೪ ಐಂ ।} ಕವಚಾಯ ಹುಂ\\
\as{೪ ಕ್ಲೀಂ । }ನೇತ್ರತ್ರಯಾಯ ವೌಷಟ್\\
\as{೪ ಸೌಃ ।} ಅಸ್ತ್ರಾಯ ಫಟ್ ॥
\section{ಚತುರಾಸನನ್ಯಾಸಃ}
\addcontentsline{toc}{section}{ಚತುರಾಸನನ್ಯಾಸಃ}
{\bfseries ೪ ಹ್ರೀಂಕ್ಲೀಂಸೌಃ ।}ದೇವ್ಯಾತ್ಮಾಸನಾಯ ನಮಃ। ಪಾದಯೋಃ\\
{\bfseries ೪ ಹೈಂ ಹ್‌ಕ್ಲೀಂ ಹ್ಸೌಃ ।} ಶ್ರೀಚಕ್ರಾಸನಾಯ ನಮಃ। ಜಾನುನೋಃ\\
{\bfseries ೪ ಹ್‌ಸೈಂ ಹ್‌ಸ್‌ಕ್ಲೀಂ ಹ್‌ಸ್ಸೌಃ ।} ಸರ್ವಮಂತ್ರಾಸನಾಯ ನಮಃ ।\\ ಊರುಮೂಲಯೋಃ\\
{\bfseries ೪ ಹ್ರೀಂಕ್ಲೀಂಬ್ಲೇಂ ।} ಸಾಧ್ಯಸಿದ್ಧಾಸನಾಯ ನಮಃ। ಆಧಾರೇ
\section{ವಶಿನ್ಯಾದಿವಾಗ್ದೇವತಾನ್ಯಾಸಃ}
\addcontentsline{toc}{section}{ವಶಿನ್ಯಾದಿವಾಗ್ದೇವತಾನ್ಯಾಸಃ}
\as{ಓಂ ಐಂಹ್ರೀಂಶ್ರೀಂ ಅಂ ಆಂ ಇಂ ಈಂ+ + ಅಂ ಅಃ~। ರ್ಬ್ಲೂಂ॥}\\ ವಶಿನೀ ವಾಗ್ದೇವತಾಯೈ ನಮಃ। ಶಿರಸಿ\\
\as{೪ ಕಂ ಖಂ ಗಂ ಘಂ ಙಂ। ಕ್‌ಲ್‌ಹ್ರೀಂ ॥}\\ ಕಾಮೇಶ್ವರೀ ವಾಗ್ದೇವತಾಯೈ ನಮಃ। ಲಲಾಟೇ\\
\as{೪ ಚಂ ಛಂ ಜಂ ಝಂ ಞಂ। ನ್‌ವ್ಲೀಂ ॥}\\ ಮೋದಿನೀ ವಾಗ್ದೇವತಾಯೈ ನಮಃ। ಭ್ರೂಮಧ್ಯೇ\\
\as{೪ ಟಂ ಠಂ ಡಂ ಢಂ ಣಂ। ಯ್ಲೂಂ ॥}\\ ವಿಮಲಾ ವಾಗ್ದೇವತಾಯೈ ನಮಃ । ಕಂಠೇ\\
\as{೪ ತಂ ಥಂ ದಂ ಧಂ ನಂ~। ಜ್‌ಮ್ರೀಂ ॥}\\ ಅರುಣಾ ವಾಗ್ದೇವತಾಯೈ ನಮಃ। ಹೃದಯೇ\\
\as{೪ ಪಂ ಫಂ ಬಂ ಭಂ ಮಂ। ಹ್‌ಸ್‌ಲ್‌ವ್ಯೂಂ ॥}\\ ಜಯಿನೀ ವಾಗ್ದೇವತಾಯೈ ನಮಃ। ನಾಭೌ\\
\as{೪ ಯಂ ರಂ ಲಂ ವಂ~। ಝ್‌ಮ್‌ರ್ಯೂಂ ॥}\\ ಸರ್ವೇಶ್ವರೀ ವಾಗ್ದೇವತಾಯೈ ನಮಃ। ಗುಹ್ಯೇ\\
\as{೪ ಶಂ ಷಂ ಸಂ ಹಂ ಳಂ ಕ್ಷಂ~। ಕ್ಷ್‌ಮ್ರೀಂ ॥}\\ ಕೌಲಿನೀ ವಾಗ್ದೇವತಾಯೈ ನಮಃ। ಆಧಾರೇ
\section{ಬಹಿಶ್ಚಕ್ರನ್ಯಾಸಃ}
\addcontentsline{toc}{section}{ಬಹಿಶ್ಚಕ್ರನ್ಯಾಸಃ}
{\bfseries ೪ ಅಂಆಂಸೌಃ ।} ಚತುರಶ್ರ ತ್ರಯಾತ್ಮಕ ತ್ರೈಲೋಕ್ಯಮೋಹನ ಚಕ್ರಾಧಿಷ್ಠಾತ್ರ್ಯೈ ಅಣಿಮಾದ್ಯಷ್ಟಾವಿಂಶತಿ ಶಕ್ತಿಸಹಿತ ಪ್ರಕಟಯೋಗಿನೀ ರೂಪಾಯೈ  ತ್ರಿಪುರಾದೇವ್ಯೈ ನಮಃ॥(ಪಾದಯೋಃ)\\
{\bfseries ೪ ಐಂಕ್ಲೀಂಸೌಃ ।} ಷೋಡಶದಲ ಪದ್ಮಾತ್ಮಕ ಸರ್ವಾಶಾ ಪರಿಪೂರಕ ಚಕ್ರಾಧಿಷ್ಠಾತ್ರ್ಯೈ ಕಾಮಾಕರ್ಷಣ್ಯಾದಿ ಷೋಡಶಶಕ್ತಿ ಸಹಿತ ಗುಪ್ತಯೋಗಿನೀ ರೂಪಾಯೈ ತ್ರಿಪುರೇಶೀ ದೇವ್ಯೈ ನಮಃ॥(ಜಾನುನೋಃ)\\
{\bfseries ೪ ಹ್ರೀಂಕ್ಲೀಂಸೌಃ ।} ಅಷ್ಟದಲ ಪದ್ಮಾತ್ಮಕ ಸರ್ವ ಸಂಕ್ಷೋಭಣ ಚಕ್ರಾಧಿಷ್ಠಾತ್ರ್ಯೈ ಅನಂಗ ಕುಸುಮಾದ್ಯಷ್ಟಶಕ್ತಿ ಸಹಿತ ಗುಪ್ತತರ ಯೋಗಿನೀ ರೂಪಾಯೈ ತ್ರಿಪುರಸುಂದರೀ ದೇವ್ಯೈ ನಮಃ॥(ಊರುಮೂಲಯೋಃ)\\
{\bfseries ೪ ಹೈಂಹ್‌ಕ್ಲೀಂಹ್ಸೌಃ ।} ಚತುರ್ದಶಾರಾತ್ಮಕ ಸರ್ವಸೌಭಾಗ್ಯದಾಯಕ ಚಕ್ರಾಧಿಷ್ಠಾತ್ರ್ಯೈ ಸರ್ವ ಸಂಕ್ಷೋಭಿಣ್ಯಾದಿ ಚತುರ್ದಶ ಶಕ್ತಿಸಹಿತ ಸಂಪ್ರದಾಯ ಯೋಗಿನೀ ರೂಪಾಯೈ  ತ್ರಿಪುರವಾಸಿನೀ ದೇವ್ಯೈ ನಮಃ~॥(ನಾಭೌ)\\
{\bfseries ೪ ಹ್‌ಸೈಂಹ್‌ಸ್‌ಕ್ಲೀಂಹ್‌ಸ್ಸೌಃ ।} ಬಹಿರ್ದಶಾರಾತ್ಮಕ ಸರ್ವಾರ್ಥ ಸಾಧಕ ಚಕ್ರಾಧಿಷ್ಠಾತ್ರ್ಯೈ ಸರ್ವಸಿದ್ಧಿಪ್ರದಾದಿ ದಶಶಕ್ತಿ ಸಹಿತ ಕುಲೋತ್ತೀರ್ಣ ಯೋಗಿನೀ ರೂಪಾಯೈ  ತ್ರಿಪುರಾಶ್ರೀ ದೇವ್ಯೈ ನಮಃ॥(ಹೃದಯೇ)\\
{\bfseries ೪ ಹ್ರೀಂ ಕ್ಲೀಂ ಬ್ಲೇಂ ।} ಅಂತರ್ದಶಾರಾತ್ಮಕ ಸರ್ವರಕ್ಷಾಕರ ಚಕ್ರಾಧಿಷ್ಠಾತ್ರ್ಯೈ ಸರ್ವಜ್ಞಾದಿ ದಶಶಕ್ತಿ ಸಹಿತ ನಿಗರ್ಭ ಯೋಗಿನೀ ರೂಪಾಯೈ  ತ್ರಿಪುರಮಾಲಿನೀ ದೇವ್ಯೈ ನಮಃ॥(ಕಂಠೇ)\\
{\bfseries ೪ ಹ್ರೀಂ ಶ್ರೀಂ ಸೌಃ ।} ಅಷ್ಟಾರಾತ್ಮಕ ಸರ್ವರೋಗಹರ ಚಕ್ರಾಧಿಷ್ಠಾತ್ರ್ಯೈ ವಶಿನ್ಯಾದ್ಯಷ್ಟ ಶಕ್ತಿಸಹಿತ ರಹಸ್ಯ ಯೋಗಿನೀ ರೂಪಾಯೈ  ತ್ರಿಪುರಾಸಿದ್ಧಾ ದೇವ್ಯೈ ನಮಃ॥(ಮುಖೇ)\\
{\bfseries ೪ ಹ್‌ಸ್‌ರೈಂ ಹ್‌ಸ್‌ಕ್ಲ್ರೀಂ ಹ್‌ಸ್‌ರ್ಸೌಃ ।} ತ್ರಿಕೋಣಾತ್ಮಕ ಸರ್ವಸಿದ್ಧಿ ಪ್ರದಚಕ್ರಾಧಿಷ್ಠಾತ್ರ್ಯೈ ಕಾಮೇಶ್ವರ್ಯಾದಿ ತ್ರಿಶಕ್ತಿ ಸಹಿತ ಅತಿರಹಸ್ಯಯೋಗಿನೀ ರೂಪಾಯೈ  ತ್ರಿಪುರಾಂಬಾ ದೇವ್ಯೈ ನಮಃ~॥(ನೇತ್ರಯೋಃ)\\
{\bfseries ೪ ೧೫॥} ಬಿಂದ್ವಾತ್ಮಕ ಸರ್ವಾನಂದಮಯ ಚಕ್ರಾಧಿಷ್ಠಾತ್ರ್ಯೈ ಷಡಂಗಾಯುಧ ದಶಶಕ್ತಿ ಸಹಿತ ಪರಾಪರಾತಿ ರಹಸ್ಯಯೋಗಿನೀ ರೂಪಾಯೈ ಮಹಾ ತ್ರಿಪುರಸುಂದರೀ ದೇವ್ಯೈ ನಮಃ~॥(ಶಿರಸಿ)
\section{ಅಂತಶ್ಚಕ್ರನ್ಯಾಸಃ}
\addcontentsline{toc}{section}{ಅಂತಶ್ಚಕ್ರನ್ಯಾಸಃ}
{\bfseries ೪ ಅಂಆಂಸೌಃ।} ಚತುರಶ್ರ ತ್ರಯಾತ್ಮಕ ತ್ರೈಲೋಕ್ಯಮೋಹನ ಚಕ್ರಾಧಿಷ್ಠಾತ್ರ್ಯೈ ಅಣಿಮಾದ್ಯಷ್ಟಾವಿಂಶತಿ ಶಕ್ತಿ ಸಹಿತ ಪ್ರಕಟಯೋಗಿನೀ ರೂಪಾಯೈ  ತ್ರಿಪುರಾ ದೇವ್ಯೈ ನಮಃ॥(ಅಕುಲಸಹಸ್ರಾರೇ)\\
{\bfseries ೪ ಐಂಕ್ಲೀಂಸೌಃ ।} ಷೋಡಶ ದಲ ಪದ್ಮಾತ್ಮಕ ಸರ್ವಾಶಾಪರಿಪೂರಕ ಚಕ್ರಾಧಿಷ್ಠಾತ್ರ್ಯೈ ಕಾಮಾಕರ್ಷಣ್ಯಾದಿ ಷೋಡಶ ಶಕ್ತಿ ಸಹಿತ ಗುಪ್ತಯೋಗಿನೀ ರೂಪಾಯೈ  ತ್ರಿಪುರೇಶೀ ದೇವ್ಯೈ ನಮಃ॥(ವಿಷುಚಕ್ರೇ)\\
{\bfseries ೪ ಹ್ರೀಂ ಕ್ಲೀಂ ಸೌಃ ।} ಅಷ್ಟದಲ ಪದ್ಮಾತ್ಮಕ ಸರ್ವಸಂಕ್ಷೋಭಣ ಚಕ್ರಾಧಿಷ್ಠಾತ್ರ್ಯೈ ಅನಂಗ ಕುಸುಮಾದ್ಯಷ್ಟ ಶಕ್ತಿಸಹಿತ ಗುಪ್ತತರ ಯೋಗಿನೀ ರೂಪಾಯೈ  ತ್ರಿಪುರಸುಂದರೀ ದೇವ್ಯೈ ನಮಃ॥(ಮೂಲಾಧಾರೇ)\\
{\bfseries ೪ ಹೈಂ ಹ್‌ಕ್ಲೀಂ ಹ್ಸೌಃ ।} ಚತುರ್ದಶಾರಾತ್ಮಕ ಸರ್ವಸೌಭಾಗ್ಯದಾಯಕ ಚಕ್ರಾಧಿಷ್ಠಾತ್ರ್ಯೈ ಸರ್ವಸಂಕ್ಷೋಭಿಣ್ಯಾದಿ ಚತುರ್ದಶ ಶಕ್ತಿಸಹಿತ ಸಂಪ್ರದಾಯ ಯೋಗಿನೀ ರೂಪಾಯೈ  ತ್ರಿಪುರವಾಸಿನೀ ದೇವ್ಯೈ ನಮಃ॥(ಸ್ವಾಧಿಷ್ಠಾನೇ)\\
{\bfseries ೪ ಹ್‌ಸೈಂ ಹ್‌ಸ್‌ಕ್ಲೀಂ ಹ್‌ಸ್ಸೌಃ।} ಬಹಿರ್ದಶಾರಾತ್ಮಕ ಸರ್ವಾರ್ಥ ಸಾಧಕ ಚಕ್ರಾಧಿಷ್ಠಾತ್ರ್ಯೈ ಸರ್ವಸಿದ್ಧಿ ಪ್ರದಾದಿ ದಶ ಶಕ್ತಿ ಸಹಿತ ಕುಲೋತ್ತೀರ್ಣ ಯೋಗಿನೀ ರೂಪಾಯೈ  ತ್ರಿಪುರಾಶ್ರೀ ದೇವ್ಯೈ ನಮಃ॥(ಮಣಿಪೂರೇ)\\
{\bfseries ೪ ಹ್ರೀಂ ಕ್ಲೀಂ ಬ್ಲೇಂ ।} ಅಂತರ್ದಶಾರಾತ್ಮಕ ಸರ್ವರಕ್ಷಾಕರ ಚಕ್ರಾಧಿಷ್ಠಾತ್ರ್ಯೈ ಸರ್ವಜ್ಞಾದಿ ದಶಶಕ್ತಿ ಸಹಿತ ನಿಗರ್ಭಯೋಗಿನೀ ರೂಪಾಯೈ  ತ್ರಿಪುರಮಾಲಿನೀ ದೇವ್ಯೈ ನಮಃ॥(ಅನಾಹತೇ)\\
{\bfseries ೪ ಹ್ರೀಂ ಶ್ರೀಂ ಸೌಃ ।} ಅಷ್ಟಾರಾತ್ಮಕ ಸರ್ವರೋಗಹರ ಚಕ್ರಾಧಿಷ್ಠಾತ್ರ್ಯೈ ವಶಿನ್ಯಾದ್ಯಷ್ಟ ಶಕ್ತಿಸಹಿತ ರಹಸ್ಯ ಯೋಗಿನೀ ರೂಪಾಯೈ  ತ್ರಿಪುರಾಸಿದ್ಧಾ ದೇವ್ಯೈ ನಮಃ॥(ವಿಶುದ್ಧೌ)\\
{\bfseries ೪ ಹ್‌ಸ್‌ರೈಂ ಹ್‌ಸ್‌ಕ್ಲ್ರೀಂ ಹ್‌ಸ್‌ರ್ಸೌಃ ।} ತ್ರಿಕೋಣಾತ್ಮಕ ಸರ್ವಸಿದ್ಧಿಪ್ರದ ಚಕ್ರಾಧಿಷ್ಠಾತ್ರ್ಯೈ ಕಾಮೇಶ್ವರ್ಯಾದಿ ತ್ರಿಶಕ್ತಿಸಹಿತ ಅತಿರಹಸ್ಯಯೋಗಿನೀ ರೂಪಾಯೈ  ತ್ರಿಪುರಾಂಬಾ ದೇವ್ಯೈ ನಮಃ॥(ಲಂಬಿಕಾಗ್ರೇ)\\
{\bfseries ೪ ೧೫~॥}ಬಿಂದ್ವಾತ್ಮಕ ಸರ್ವಾನಂದಮಯ ಚಕ್ರಾಧಿಷ್ಠಾತ್ರ್ಯೈ ಷಡಂಗಾಯುಧ ದಶ ಶಕ್ತಿಸಹಿತ ಪರಾಪರಾತಿರಹಸ್ಯ ಯೋಗಿನೀ ರೂಪಾಯೈ ಮಹಾ ತ್ರಿಪುರಸುಂದರೀ ದೇವ್ಯೈ ನಮಃ॥(ಆಜ್ಞಾಯಾಂ)\\
{\bfseries ೪ ಅಂ ಆಂ ಸೌಃ} ನಮಃ(ಬಿಂದೌ)\\
{\bfseries ೪ ಐಂ ಕ್ಲೀಂ ಸೌಃ} ನಮಃ(ಅರ್ಧಚಂದ್ರೇ)\\
{\bfseries ೪ ಹ್ರೀಂ ಕ್ಲೀಂ ಸೌಃ} ನಮಃ(ರೋಧಿನ್ಯಾಂ)\\
{\bfseries ೪ ಹೈಂ ಹ್‌ಕ್ಲೀಂ ಹ್ಸೌಃ} ನಮಃ(ನಾದೇ)\\
{\bfseries ೪ ಹ್‌ಸೈಂ ಹ್‌ಸ್‌ಕ್ಲೀಂ ಹ್ಸ್ಸೌಃ} ನಮಃ(ನಾದಾಂತೇ)\\
{\bfseries ೪ ಹ್ರೀಂ ಕ್ಲೀಂ ಬ್ಲೇಂ} ನಮಃ (ಶಕ್ತೌ)\\
{\bfseries ೪ ಹ್ರೀಂ ಶ್ರೀಂ ಸೌಃ} ನಮಃ (ವ್ಯಾಪಿಕಾಯಾಂ)\\
{\bfseries ೪ ಹ್‌ಸ್‌ರೈಂ ಹ್‌ಸ್‌ಕ್ಲ್ರೀಂ ಹ್‌ಸ್‌ರ್ಸೌಃ} ನಮಃ(ಸಮನಾಯಾಂ)\\
{\bfseries ೪ (ಪಂಚದಶೀ)}ನಮಃ(ಉನ್ಮನಾಯಾಂ)\\
{\bfseries ೪ (ಷೋಡಶೀ)}ನಮಃ(ಬ್ರಹ್ಮರಂಧ್ರೇ)
\section{ಕಾಮೇಶ್ವರ್ಯಾದಿ ನ್ಯಾಸಃ}
\addcontentsline{toc}{section}{ಕಾಮೇಶ್ವರ್ಯಾದಿ ನ್ಯಾಸಃ}
{\bfseries ೪ ಐಂ ೫॥} ಸೂರ್ಯಚಕ್ರೇ ಕಾಮಗಿರಿಪೀಠೇ ಮಿತ್ರೀಶನಾಥ ನವಯೋನಿ ಚಕ್ರಾತ್ಮಕ ಆತ್ಮತತ್ತ್ವ  ಸಂಹಾರಕೃತ್ಯ ಜಾಗ್ರದ್ದಶಾಧಿಷ್ಠಾಯಕ ಇಚ್ಛಾಶಕ್ತಿ ವಾಗ್ಭವಾತ್ಮಕ ಪರಾಪರಶಕ್ತಿ ಸ್ವರೂಪ ಮಹಾಕಾಮೇಶ್ವರೀ ರುದ್ರಾತ್ಮಶಕ್ತಿ  ಶ್ರೀಪಾದುಕಾಂ ಪೂಜಯಾಮಿ ನಮಃ॥(ಆಧಾರೇ)\\
{\bfseries೪ ಕ್ಲೀಂ ೬॥} ಸೋಮಚಕ್ರೇ ಪೂರ್ಣಗಿರಿಪೀಠೇ ಉಡ್ಡೀಶನಾಥ  ದಶಾರದ್ವಯ ಚತುರ್ದಶಾರ ಚಕ್ರಾತ್ಮಕ ವಿದ್ಯಾತತ್ವ ಸ್ಥಿತಿಕೃತ್ಯ ಸ್ವಪ್ನದಶಾಧಿಷ್ಠಾಯಕ ಜ್ಞಾನಶಕ್ತಿ ಕಾಮರಾಜಾತ್ಮಕ ಕಾಮಕಲಾ ಸ್ವರೂಪ ಮಹಾವಜ್ರೇಶ್ವರೀ ವಿಷ್ಣ್ವಾತ್ಮಶಕ್ತಿ  ಶ್ರೀಪಾದುಕಾಂ ಪೂಜಯಾಮಿ ನಮಃ॥(ಅನಾಹತೇ)\\
{\bfseries೪ ಸೌಃ ೪॥} ಅಗ್ನಿಚಕ್ರೇ ಜಾಲಂಧರಪೀಠೇ  ಷಷ್ಠೀಶನಾಥ ಅಷ್ಟದಳ ಷೋಡಶದಳ ಚತುರಸ್ರ ಚಕ್ರಾತ್ಮಕ ಶಿವತತ್ತ್ವ  ಸೃಷ್ಟಿಕೃತ್ಯ ಸುಷುಪ್ತಿದಶಾಧಿಷ್ಠಾಯಕ ಕ್ರಿಯಾಶಕ್ತಿ ಶಕ್ತಿಬೀಜಾತ್ಮಕ ವಾಗೀಶ್ವರೀ ಸ್ವರೂಪ ಮಹಾಭಗಮಾಲಿನೀ ಬ್ರಹ್ಮಾತ್ಮಶಕ್ತಿ  ಶ್ರೀಪಾದುಕಾಂ ಪೂಜಯಾಮಿ ನಮಃ॥(ಆಜ್ಞಾಯಾಮ್)\\
{\bfseries ೪ ಐಂ೫ ಕ್ಲೀಂ೬ ಸೌಃ೪} ಪರಬ್ರಹ್ಮಚಕ್ರೇ ಮಹೋಡ್ಯಾಣಪೀಠೇ ಚರ್ಯಾನಂದನಾಥ ಸಮಸ್ತಚಕ್ರಾತ್ಮಕ ಸಪರಿವಾರ ಪರಮತತ್ವ ಸೃಷ್ಟಿ ಸ್ಥಿತಿ ಸಂಹಾರಕೃತ್ಯ ತುರೀಯ ದಶಾಧಿಷ್ಠಾಯಕ ಇಚ್ಛಾ ಜ್ಞಾನ ಕ್ರಿಯಾ ಶಾಂತಶಕ್ತಿ ವಾಗ್ಭವ ಕಾಮರಾಜ ಶಕ್ತಿ ಬೀಜಾತ್ಮಕ ಪರಮಶಕ್ತಿ ಸ್ವರೂಪ ಶ್ರೀ ಮಹಾತ್ರಿಪುರಸುಂದರೀ ಪರಬ್ರಹ್ಮಾತ್ಮಶಕ್ತಿ ಶ್ರೀಪಾದುಕಾಂ ಪೂಜಯಾಮಿ ನಮಃ~।(ಬ್ರಹ್ಮರಂಧ್ರೇ)
\newpage
\section{ಮೂಲವಿದ್ಯಾನ್ಯಾಸಃ}
\addcontentsline{toc}{section}{ಮೂಲವಿದ್ಯಾನ್ಯಾಸಃ}
ಅಸ್ಯ ಶ್ರೀ ಮೂಲವಿದ್ಯಾನ್ಯಾಸಮಹಾಮಂತ್ರಸ್ಯ ದಕ್ಷಿಣಾಮೂರ್ತಿಃ ಋಷಿಃ~। ಪಂಕ್ತಿಶ್ಛಂದಃ~। ಶ್ರೀಮಹಾತ್ರಿಪುರಸುಂದರೀ ದೇವತಾ~। ಐಂ ಕಏಈಲಹ್ರೀಂ ಇತಿ ಬೀಜಂ~। ಕ್ಲೀಂ ಹಸಕಹಲಹ್ರೀಂ ಇತಿ ಶಕ್ತಿಃ~। ಸೌಃ ಸಕಲಹ್ರೀಂ ಇತಿ ಕೀಲಕಂ~। ಪೂಜಾಯಾಂ ವಿನಿಯೋಗಃ~।\\
ಕೂಟತ್ರಯೇಣ ನ್ಯಾಸಂ ಧ್ಯಾನಂ ಚ ವಿಧಾಯ~।\\
\as{೪ ಕಂ} ನಮಃ (ಶಿರಸಿ)\\
\as{೪ ಏಂ} ನಮಃ (ಆಧಾರೇ)\\
\as{೪ ಈಂ} ನಮಃ (ಹೃದಯೇ)\\
\as{೪ ಲಂ} ನಮಃ (ದಕ್ಷನೇತ್ರೇ)\\
\as{೪ ಹ್ರೀಂ} ನಮಃ (ವಾಮನೇತ್ರೇ)\\
\as{೪ ಹಂ} ನಮಃ (ಭ್ರೂಮಧ್ಯೇ)\\
\as{೪ ಸಂ} ನಮಃ (ದಕ್ಷಕರ್ಣೇ)\\
\as{೪ ಕಂ} ನಮಃ (ವಾಮಕರ್ಣೇ)\\
\as{೪ ಹಂ} ನಮಃ (ಮುಖೇ)\\
\as{೪ ಲಂ} ನಮಃ (ದಕ್ಷಾಂಸೇ)\\
\as{೪ ಹ್ರೀಂ} ನಮಃ (ವಾಮಾಂಸೇ)\\
\as{೪ ಸಂ} ನಮಃ (ಪೃಷ್ಠೇ)\\
\as{೪ ಕಂ} ನಮಃ (ದಕ್ಷಜಾನುನಿ)\\
\as{೪ ಲಂ} ನಮಃ (ವಾಮಜಾನುನಿ)\\
\as{೪ ಹ್ರೀಂ} ನಮಃ (ನಾಭೌ)\\
\as{೪ ೧೫} ನಮಃ (ಸರ್ವಾಂಗೇ)\\
ಪ್ರಾಗ್ವದುತ್ತರನ್ಯಾಸಃ
\section{ಷೋಡಶಾಕ್ಷರೀ ನ್ಯಾಸಃ}
\addcontentsline{toc}{section}{ಷೋಡಶಾಕ್ಷರೀ ನ್ಯಾಸಃ}
ಅಸ್ಯ ಶ್ರೀ ಷೋಡಶಾಕ್ಷರೀನ್ಯಾಸಮಹಾಮಂತ್ರಸ್ಯ ಆನಂದಭೈರವ ಋಷಿಃ~। ಅನುಷ್ಟುಪ್ಛಂದಃ~। ಶ್ರೀಮಹಾತ್ರಿಪುರಸುಂದರೀ ದೇವತಾ~। ಐಂ ಬೀಜಂ~। ಕ್ಲೀಂ ಶಕ್ತಿಃ~। ಸೌಃ ಕೀಲಕಂ~। ಪೂಜಾಯಾಂ ವಿನಿಯೋಗಃ~। ಮೂಲಾಕ್ಷರೈಃ ನ್ಯಾಸಃ~।\\
\as{೪ (ಮೂಲಮಂತ್ರ)} ನಮಃ ॥ದಕ್ಷಕರಾಂಗುಷ್ಠಾನಾಮಿಕಾಭ್ಯಾಂ ಶಿರಸಿ\\
\as{೪ (ಮೂಲಮಂತ್ರ)} ನಮಃ ॥ಮಹಾಸೌಭಾಗ್ಯಂ ಮೇ ದೇಹಿ॥ಶಿರ ಆದಿ\\ ಪಾದಾಂತಂ ಶರೀರವಾಮಭಾಗೇ\\
\as{೪ (ಮೂಲಮಂತ್ರ)} ನಮಃ ॥ ಮಮ ಶತ್ರೂನ್ನಿಗೃಹ್ಣಾಮಿ ॥\\ರಿಪುಜಿಹ್ವಾಗ್ರಮುದ್ರಯಾ ವಾಮಪಾದಸ್ಯ ಅಧಃ\\
\as{೪ (ಮೂಲಮಂತ್ರ)} ನಮಃ ॥ ತ್ರೈಲೋಕ್ಯಸ್ಯಾಹಂ ಕರ್ತಾ॥\\ತ್ರಿಖಂಡಯಾ ಫಾಲೇ\\
\as{೪ (ಮೂಲಮಂತ್ರ)} ನಮಃ ॥ ತ್ರಿಖಂಡಯಾ ಮುಖೇ\\
\as{೪ (ಮೂಲಮಂತ್ರ)} ನಮಃ ॥ ತ್ರಿಖಂಡಯಾ ದಕ್ಷಕರ್ಣಾದಿ \\ವಾಮಕರ್ಣಪರ್ಯಂತಂ\\
\as{೪ (ಮೂಲಮಂತ್ರ)} ನಮಃ ॥ ತ್ರಿಖಂಡಯಾ ಗಲಾದಿ ಶಿರಃಪರ್ಯಂತಂ\\
\as{೪ (ಮೂಲಮಂತ್ರ)} ನಮಃ ॥ ತ್ರಿಖಂಡಯಾ ಶಿರ ಆದಿ ಪಾದಪರ್ಯಂತಂ ಪಾದಾದಿ ಶಿರಃಪರ್ಯಂತಂ\\
\as{೪ (ಮೂಲಮಂತ್ರ)} ನಮಃ ॥ ಯೋನಿಮುದ್ರಯಾ ಮುಖೇ\\
\as{೪ (ಮೂಲಮಂತ್ರ)} ನಮಃ ॥ ಯೋನಿಮುದ್ರಯಾ ಲಲಾಟೇ
\section{ಸಂಮೋಹನನ್ಯಾಸಃ}
\addcontentsline{toc}{section}{ಸಂಮೋಹನನ್ಯಾಸಃ}
\as{೪ (ಮೂಲಮಂತ್ರ)} ನಮಃ ॥ಅನಾಮಿಕಯಾ ತ್ರಿವಾರಂ ಶಿರಸಿ ಪರಿಭ್ರಾಮ್ಯ\\
\as{೪ (ಮೂಲಮಂತ್ರ)} ನಮಃ ॥ಅಂಗುಷ್ಠಾನಾಮಿಕಾಭ್ಯಾಂ ಬ್ರಹ್ಮರಂಧ್ರೇ\\
\as{೪ (ಮೂಲಮಂತ್ರ)} ನಮಃ ॥ಮಣಿಬಂಧಯೋಃ\\
\as{೪ (ಮೂಲಮಂತ್ರ)} ನಮಃ ॥ಲಲಾಟೇ\\
\as{೪ (ಮೂಲಮಂತ್ರ)} ನಮಃ ॥ಇತಿ ಶಾಕ್ತತಿಲಕಂ ಪ್ರಕಲ್ಪಯೇತ್
\section{ಸಂಹಾರನ್ಯಾಸಃ}
\addcontentsline{toc}{section}{ಸಂಹಾರನ್ಯಾಸಃ}
\as{೪ ಶ್ರೀಂ} ನಮಃ (ಪಾದಯೋಃ)\\
\as{೪ ಹ್ರೀಂ} ನಮಃ (ಜಂಘಯೋಃ)\\
\as{೪ ಕ್ಲೀಂ} ನಮಃ (ಜಾನುನೋಃ)\\
\as{೪ ಐಂ} ನಮಃ (ಸ್ಫಿಚೋಃ)\\
\as{೪ ಸೌಃ} ನಮಃ (ಪೃಷ್ಠೇ)\\
\as{೪ ಓಂ} ನಮಃ (ಲಿಂಗೇ)\\
\as{೪ ಹ್ರೀಂ} ನಮಃ (ನಾಭೌ)\\
\as{೪ ಶ್ರೀಂ} ನಮಃ (ಪಾರ್ಶ್ವಯೋಃ)\\
\as{೪ ಐಂ ಕಏಈಲಹ್ರೀಂ} ನಮಃ (ಸ್ತನಯೋಃ)\\
\as{೪ ಕ್ಲೀಂ ಹಸಕಹಲಹ್ರೀಂ} ನಮಃ (ಅಂಸಯೋಃ)\\
\as{೪ ಸೌಃ ಸಕಲಹ್ರೀಂ} ನಮಃ (ಕರ್ಣಯೋಃ)\\
\as{೪ ಸೌಃ} ನಮಃ (ಮೂರ್ಧ್ನಿ)\\
\as{೪ ಐಂ} ನಮಃ (ಮುಖೇ)\\
\as{೪ ಕ್ಲೀಂ} ನಮಃ (ನೇತ್ರಯೋಃ)\\
\as{೪ ಹ್ರೀಂ} ನಮಃ (ಉಪಕರ್ಣಯೋಃ)\\
\as{೪ ಶ್ರೀಂ} ನಮಃ (ಕರ್ಣಯೋಃ)
\section{ಸೃಷ್ಟಿನ್ಯಾಸಃ}
\addcontentsline{toc}{section}{ಸೃಷ್ಟಿನ್ಯಾಸಃ}
\as{೪ ಶ್ರೀಂ} ನಮಃ (ಬ್ರಹ್ಮರಂಧ್ರೇ)\\
\as{೪ ಹ್ರೀಂ} ನಮಃ (ಲಲಾಟೇ)\\
\as{೪ ಕ್ಲೀಂ} ನಮಃ (ನೇತ್ರಯೋಃ)\\
\as{೪ ಐಂ} ನಮಃ (ಕರ್ಣಯೋಃ)\\
\as{೪ ಸೌಃ} ನಮಃ (ನಾಸಿಕಯೋಃ)\\
\as{೪ ಓಂ} ನಮಃ (ಗಂಡಯೋಃ)\\
\as{೪ ಹ್ರೀಂ} ನಮಃ (ದಂತಪಂಕ್ತೌ)\\
\as{೪ ಶ್ರೀಂ} ನಮಃ (ಓಷ್ಠಯೋಃ)\\
\as{೪ ಐಂ ಕಏಈಲಹ್ರೀಂ} ನಮಃ (ಜಿಹ್ವಾಯಾಂ)\\
\as{೪ ಕ್ಲೀಂ ಹಸಕಹಲಹ್ರೀಂ} ನಮಃ (ಕಂಠೇ)\\
\as{೪ ಸೌಃ ಸಕಲಹ್ರೀಂ} ನಮಃ (ಪೃಷ್ಠೇ)\\
\as{೪ ಸೌಃ} ನಮಃ (ಸರ್ವಾಂಗೇ)\\
\as{೪ ಐಂ} ನಮಃ (ಹೃದಯೇ)\\
\as{೪ ಕ್ಲೀಂ} ನಮಃ (ಸ್ತನಯೋಃ)\\
\as{೪ ಹ್ರೀಂ} ನಮಃ (ಉದರೇ)\\
\as{೪ ಶ್ರೀಂ} ನಮಃ (ಲಿಂಗೇ)
\section{ಸ್ಥಿತಿನ್ಯಾಸಃ}
\addcontentsline{toc}{section}{ಸ್ಥಿತಿನ್ಯಾಸಃ}
\as{೪ ಶ್ರೀಂ} ನಮಃ (ಅಂಗುಷ್ಠಯೋಃ)\\
\as{೪ ಹ್ರೀಂ} ನಮಃ (ತರ್ಜನ್ಯೋಃ)\\
\as{೪ ಕ್ಲೀಂ} ನಮಃ (ಮಧ್ಯಮಯೋಃ)\\
\as{೪ ಐಂ} ನಮಃ (ಅನಾಮಿಕಯೋಃ)\\
\as{೪ ಸೌಃ} ನಮಃ (ಕನಿಷ್ಠಿಕಯೋಃ)\\
\as{೪ ಓಂ} ನಮಃ (ಮೂರ್ಧ್ನಿ)\\
\as{೪ ಹ್ರೀಂ} ನಮಃ (ಮುಖೇ)\\
\as{೪ ಶ್ರೀಂ} ನಮಃ (ಹೃದಯೇ)\\
\as{೪ ಐಂ ಕಏಈಲಹ್ರೀಂ} ನಮಃ (ಪಾದಾದಿನಾಭಿಪರ್ಯಂತಂ)\\
\as{೪ ಕ್ಲೀಂ ಹಸಕಹಲಹ್ರೀಂ} ನಮಃ (ನಾಭೇರ್ವಿಶುದ್ಧಿಪರ್ಯಂತಂ)\\
\as{೪ ಸೌಃ ಸಕಲಹ್ರೀಂ} ನಮಃ (ವಿಶುದ್ಧೇರ್ಬ್ರಹ್ಮರಂಧ್ರಾಂತಂ)\\
\as{೪ ಸೌಃ} ನಮಃ (ಪಾದಾಂಗುಷ್ಠಯೋಃ)\\
\as{೪ ಐಂ} ನಮಃ (ಪಾದತರ್ಜನ್ಯೋಃ)\\
\as{೪ ಕ್ಲೀಂ} ನಮಃ (ಪಾದಮಧ್ಯಮಯೋಃ)\\
\as{೪ ಹ್ರೀಂ} ನಮಃ (ಪಾದಾನಾಮಿಕಯೋಃ)\\
\as{೪ ಶ್ರೀಂ} ನಮಃ (ಪಾದಕನಿಷ್ಠಿಕಯೋಃ)\\
\as{೪ ೧೬} ನಮಃ (ಸರ್ವಾಂಗೇ)\\
ಪ್ರಾಗ್ವದುತ್ತರನ್ಯಾಸಃ

\section{ಶ್ರೀಕಂಠಮಾತೃಕಾ ನ್ಯಾಸಃ}
ಅಸ್ಯ ಶ್ರೀ ಶ್ರೀಕಂಠಮಾತೃಕಾ ನ್ಯಾಸಸ್ಯ ದಕ್ಷಿಣಾಮೂರ್ತಿಃ ಋಷಿಃ । ಗಾಯತ್ರೀ ಚ್ಛಂದಃ । ಅರ್ಧನಾರೀಶ್ವರೋ ದೇವತಾ ॥

ಓಂ ಹ್ರೀಂ ನಮೋ ಭಗವತಿ ಬ್ರಾಹ್ಮಿ ರಕ್ಷ ರಕ್ಷ ಪದ್ಮಹಸ್ತೇ ಮಾಂ\\ ರಕ್ಷ ರಕ್ಷ  ಅಂ ಕಂಖಂಗಂಘಂಙಂ ಆಂ ಹ್ರೀಂ ಹ್ಸೌಃ \\ಸಂಜೀವನಿ ಸಂಜೀವನಿ  \as{ಹೃದಯಾಯ ನಮಃ ॥}

ಓಂ ಹ್ರೀಂ ನಮೋ ಭಗವತಿ ಮಾಹೇಶ್ವರಿ ರಕ್ಷ ರಕ್ಷ ತ್ರಿಶೂಲಹಸ್ತೇ ಮಾಂ \\ರಕ್ಷ ರಕ್ಷ ಇಂ ಚಂಛಂಜಂಝಂಞಂ ಈಂ  ಹ್ರೀಂ ಹ್ಸೌಃ\\ ಊರ್ಧ್ವಕೇಶಿನಿ \as{ಶಿರಸೇ ಸ್ವಾಹಾ ॥}

ಓಂ ಹ್ರೀಂ ನಮೋ ಭಗವತಿ ಕೌಮಾರಿ ರಕ್ಷ ರಕ್ಷ ಶಕ್ತಿಹಸ್ತೇ ಮಾಂ\\ ರಕ್ಷ ರಕ್ಷ ಉಂ ಟಂಠಂಡಂಢಂಣಂ ಊಂ  ಹ್ರೀಂ ಹ್ಸೌಃ\\ ಜಟಿಲಕೇಶಿನಿ \as{ಶಿಖಾಯೈ ವೌಷಟ್ ॥}

ಓಂ ಹ್ರೀಂ ನಮೋ ಭಗವತಿ ವೈಷ್ಣವಿ ರಕ್ಷ ರಕ್ಷ ಚಕ್ರಹಸ್ತೇ ಮಾಂ\\ ರಕ್ಷ ರಕ್ಷ ಏಂ ತಂಥಂದಂಧಂನಂ ಐಂ ಹ್ರೀಂ ಹ್ಸೌಃ\\ ಮಾಯಾತ್ರೈಲೋಕ್ಯರೂಪಿಣಿ ಸಹಸ್ರ ಪರಿವರ್ತಿನಿ \as{ಕವಚಾಯ ಹುಂ ॥}

ಓಂ ಹ್ರೀಂ ನಮೋ ಭಗವತಿ ವಾರಾಹಿ ರಕ್ಷ ರಕ್ಷ ದಂಷ್ಟ್ರಹಸ್ತೇ ಮಾಂ\\ ರಕ್ಷ ರಕ್ಷ ಓಂ ಪಂಫಂಬಂಭಂಮಂ ಔಂ ಹ್ರೀಂ ಹ್ಸೌಃ\\ ತಾರಕಾಕ್ಷಿಣಿ \as{ನೇತ್ರತ್ರಯಾಯ ವೌಷಟ್ ॥}

ಓಂ ಹ್ರೀಂ ನಮೋ ಭಗವತಿ ಐಂದ್ರಿ ರಕ್ಷ ರಕ್ಷ ವಜ್ರಹಸ್ತೇ ಮಾಂ\\ ರಕ್ಷ ರಕ್ಷ ಅಂ ಯಂರಂಲಂವಂಶಂಷಂಸಂಹಂಳಂಕ್ಷಂ ಅಃ  ಹ್ರೀಂ ಹ್ಸೌಃ \\ಮಾರಯ ಮಾರಯ \as{ಅಸ್ತ್ರಾಯ ಫಟ್ ॥}

ಓಂ ಹ್ರೀಂ ನಮೋ ಭಗವತಿ ಚಾಮುಂಡೇ ರಕ್ಷ ರಕ್ಷ ಪಾಶ ಹಸ್ತೇ ಮಾಂ\\ ರಕ್ಷ ರಕ್ಷ  ಅಂಇಂಉಂಋಂಲೃಂಏಂಓಂಅಂ ಕಂಖಂ++ಳಂಕ್ಷಂ ಆಂಈಂಊಂೠಂಲೄಂಐಂಔಂಅಃ  ಹ್ರೀಂ ಹ್ಸೌಃ ಮಮ ಸರ್ವಾಂಗಂ ರಕ್ಷ ರಕ್ಷ ಸರ್ವಾಂಗವ್ಯಾಪಿನಿ ಸ್ವಾಹಾ \as{॥ ಸರ್ವಾಂಗೇ ॥}

\dhyana{ಬಂಧೂಕಕಾಂಚನನಿಭಂ ರುಚಿರಾಕ್ಷಮಾಲಾಂ\\
ಪಾಶಾಂಕುಶೌ ಚ ವರದಂ ನಿಜಬಾಹುದಂಡೈಃ ।\\
ಬಿಭ್ರಾಣಮಿಂದುಶಕಲಾಭರಣಂ ತ್ರಿಣೇತ್ರಮ್\\
ಅರ್ಧಾಂಬಿಕೇಶಮನಿಶಂ ವಪುರಾಶ್ರಯಾಮಃ ॥}
\newpage
\as{೪ ಅಂ}  ಶ್ರೀಕಂಠೇಶ ಪೂರ್ಣೋದರೀಭ್ಯಾಂ ನಮಃ ~।(ಶಿರಸಿ)\\
\as{೪ ಆಂ}  ಅನಂತೇಶ ವಿರಜಾಭ್ಯಾಂ ನಮಃ  ~।(ಮುಖವೃತ್ತೇ)\\
\as{೪ ಇಂ}  ಸೂಕ್ಷ್ಮೇಶ ಶಾಲ್ಮಲೀಭ್ಯಾಂ ನಮಃ  ~।(ದಕ್ಷನೇತ್ರೇ)\\
\as{೪ ಈಂ}  ತ್ರಿಮೂರ್ತೀಶ ಲೋಲಾಕ್ಷೀಭ್ಯಾಂ ನಮಃ  ~।(ವಾಮನೇತ್ರೇ)\\
\as{೪ ಉಂ}  ಅಮರೇಶ ವರ್ತುಲಾಕ್ಷೀಭ್ಯಾಂ ನಮಃ  ~।(ದಕ್ಷಕರ್ಣೇ)\\
\as{೪ ಊಂ}  ಅರ್ಘೀಶ ದೀರ್ಘಘೋಣಾಭ್ಯಾಂ ನಮಃ  ~।(ವಾಮಕರ್ಣೇ)\\
\as{೪ ಋಂ}  ಭಾವಭೂತೀಶ ದೀರ್ಘಮುಖೀಭ್ಯಾಂ ನಮಃ  ~।(ದಕ್ಷನಾಸಾಯಾಂ)\\
\as{೪ ೠಂ}  ತಿಥೀಶ ಗೋಮುಖೀಭ್ಯಾಂ ನಮಃ  ~।(ವಾಮನಾಸಾಯಾಂ)\\
\as{೪ ಲೃಂ}  ಸ್ಥಾಣ್ವೀಶ ದೀರ್ಘಜಿಹ್ವಾಭ್ಯಾಂ ನಮಃ  ~।(ದಕ್ಷಗಂಡೇ)\\
\as{೪ ಲೄಂ}  ಹರೇಶ ಕುಂಡೋದರೀಭ್ಯಾಂ ನಮಃ  ~।(ವಾಮಗಂಡೇ)\\
\as{೪ ಏಂ}  ಝಿಂಟೀಶ ಊರ್ಧ್ವಕೇಶೀಭ್ಯಾಂ ನಮಃ  ~।(ಊರ್ಧ್ವೋಷ್ಠೇ)\\
\as{೪ ಐಂ}  ಭೌತಿಕೇಶ ವಿಕೃತಮುಖೀಭ್ಯಾಂ ನಮಃ  ~।(ಅಧರೋಷ್ಠೇ)\\
\as{೪ ಓಂ}  ಸದ್ಯೋಜಾತೇಶ ಜ್ವಾಲಾಮುಖೀಭ್ಯಾಂ ನಮಃ  ~।\\(ಊರ್ಧ್ವದಂತಪಂಕ್ತೌ)\\
\as{೪ ಔಂ}  ಅನುಗ್ರಹೇಶ ಉಲ್ಕಾಮುಖೀಭ್ಯಾಂ ನಮಃ  ~।(ಅಧೋದಂತಪಂಕ್ತೌ)\\
\as{೪ ಅಂ}  ಅಕ್ರೂರೇಶ ಶ್ರೀಮುಖೀಭ್ಯಾಂ ನಮಃ  ~।(ಜಿಹ್ವಾಯಾಂ)\\
\as{೪ ಅಃ}  ಮಹಾಸೇನೇಶ ವಿದ್ಯಾಮುಖೀಭ್ಯಾಂ ನಮಃ  ~।(ಕಂಠೇ)\\
\as{೪ ಕಂ}  ಕ್ರೋಧೀಶ ಮಹಾಕಾಲೀಭ್ಯಾಂ ನಮಃ  ~।(ದಕ್ಷ ಬಾಹುಮೂಲೇ)\\
\as{೪ ಖಂ}  ಚಂಡೇಶ ಸರಸ್ವತೀಭ್ಯಾಂ ನಮಃ  ~।(ದಕ್ಷಕೂರ್ಪರೇ)\\
\as{೪ ಗಂ}  ಪಂಚಾಂತಕೇಶ ಸರ್ವಸಿದ್ಧಿಗೌರೀಭ್ಯಾಂ ನಮಃ  ~।(ದಕ್ಷಮಣಿಬಂಧೇ)\\
\as{೪ ಘಂ}  ಶಿವೋತ್ತಮೇಶ ತ್ರೈಲೋಕ್ಯವಿದ್ಯಾಭ್ಯಾಂ ನಮಃ  ~।\\(ದಕ್ಷಕರಾಂಗುಲಿಮೂಲೇ)\\
\as{೪ ಙಂ}  ಏಕರುದ್ರೇಶ ಮಂತ್ರಶಕ್ತಿಭ್ಯಾಂ ನಮಃ  ~।(ದಕ್ಷಕರಾಂಗುಲ್ಯಗ್ರೇ)\\
\as{೪ ಚಂ}  ಕೂರ್ಮೇಶ ಆತ್ಮಶಕ್ತಿಭ್ಯಾಂ ನಮಃ  ~।(ವಾಮಬಾಹುಮೂಲೇ)\\
\as{೪ ಛಂ}  ಏಕನೇತ್ರೇಶ ಭೂತಮಾತೃಭ್ಯಾಂ ನಮಃ  ~।(ವಾಮಕೂರ್ಪರೇ)\\
\as{೪ ಜಂ}  ಚತುರಾನನೇಶ ಲಂಬೋದರೀಭ್ಯಾಂ ನಮಃ  ~।(ವಾಮಮಣಿಬಂಧೇ)\\
\as{೪ ಝಂ}  ಅಜೇಶ ದ್ರಾವಿಣೀಭ್ಯಾಂ ನಮಃ  ~।(ವಾಮಕರಾಂಗುಲಿಮೂಲೇ)\\
\as{೪ ಞಂ}  ಶರ್ವೇಶ ನಾಗರೀಭ್ಯಾಂ ನಮಃ  ~।(ವಾಮಕರಾಂಗುಲ್ಯಗ್ರೇ)\\
\as{೪ ಟಂ}  ಸೋಮೇಶ ವೈಖರೀಭ್ಯಾಂ ನಮಃ  ~।(ದಕ್ಷೋರುಮೂಲೇ)\\
\as{೪ ಠಂ}  ಲಾಂಗಲೀಶ ಮಂಜರೀಭ್ಯಾಂ ನಮಃ  ~।(ದಕ್ಷಜಾನುನಿ)\\
\as{೪ ಡಂ}  ದಾರುಕೇಶ ರೂಪಿಣೀಭ್ಯಾಂ ನಮಃ  ~।(ದಕ್ಷಗುಲ್ಫೇ)\\
\as{೪ ಢಂ}  ಅರ್ಧನಾರೀಶ ವೀರಿಣೀಭ್ಯಾಂ ನಮಃ  ~।(ದಕ್ಷಪಾದಾಂಗುಲಿಮೂಲೇ)\\
\as{೪ ಣಂ}  ಉಮಾಕಾಂತೇಶ ಕೋಟರೀಭ್ಯಾಂ ನಮಃ  ~।(ದಕ್ಷಪಾದಾಂಗುಲ್ಯಗ್ರೇ)\\
\as{೪ ತಂ}  ಆಷಾಢೀಶ ಪೂತನಾಭ್ಯಾಂ ನಮಃ  ~।(ವಾಮೋರುಮೂಲೇ)\\
\as{೪ ಥಂ}  ದಂಡೀಶ ಭದ್ರಕಾಲೀಭ್ಯಾಂ ನಮಃ  ~।(ವಾಮಜಾನುನಿ)\\
\as{೪ ದಂ}  ಅದ್ರೀಶ ಯೋಗಿನೀಭ್ಯಾಂ ನಮಃ  ~।(ವಾಮಗುಲ್ಫೇ)\\
\as{೪ ಧಂ}  ಮೀನೇಶ ಶಂಖಿನೀಭ್ಯಾಂ ನಮಃ  ~।(ವಾಮಪಾದಾಂಗುಲಿಮೂಲೇ)\\
\as{೪ ನಂ}  ಮೇಷೇಶ ಗರ್ಜಿನೀಭ್ಯಾಂ ನಮಃ  ~।(ವಾಮಪಾದಾಂಗುಲ್ಯಗ್ರೇ)\\
\as{೪ ಪಂ}  ಲೋಹಿತೇಶ ಕಾಲರಾತ್ರಿಭ್ಯಾಂ ನಮಃ  ~।(ದಕ್ಷಪಾರ್ಶ್ವೇ)\\
\as{೪ ಫಂ}  ಶಿಖೀಶ ಕುರ್ದಿನೀಭ್ಯಾಂ ನಮಃ  ~।(ವಾಮಪಾರ್ಶ್ವೇ)\\
\as{೪ ಬಂ}  ಛಗಲಂಡೇಶ ಕಪರ್ದಿನೀಭ್ಯಾಂ ನಮಃ  ~।(ಪೃಷ್ಠೇ)\\
\as{೪ ಭಂ}  ದ್ವಿರಂಡೇಶ ವಜ್ರಿಣೀಭ್ಯಾಂ ನಮಃ  ~।(ನಾಭೌ)\\
\as{೪ ಮಂ}  ಮಹಾಕಾಲೇಶ ಜಯಾಭ್ಯಾಂ ನಮಃ  ~।(ಜಠರೇ)\\
\as{೪ ಯಂ}  ತ್ವಗಾತ್ಮಕ ಕಪಾಲೀಶ ಸುಮುಖೇಶ್ವರೀಭ್ಯಾಂ ನಮಃ  ~।(ಹೃದಿ )\\
\as{೪ ರಂ}  ಅಸೃಗಾತ್ಮಕ ಭುಜಂಗೇಶ ರೇವತೀಭ್ಯಾಂ ನಮಃ  ~।(ದಕ್ಷಾಂಸೇ)\\
\as{೪ ಲಂ}  ಮಾಂಸಾತ್ಮಕ ಪಿನಾಕೀಶ ಮಾಧವೀಭ್ಯಾಂ ನಮಃ  ~।(ಕಕುದಿ)\\
\as{೪ ವಂ}  ಮೇದ ಆತ್ಮಕ ಖಡ್ಗೀಶ ವಾರುಣೀಭ್ಯಾಂ ನಮಃ  ~।(ವಾಮಾಂಸೇ)\\
\as{೪ ಶಂ}  ಅಸ್ಥ್ಯಾತ್ಮಕ ಬಕೇಶ ವಾಯವೀಭ್ಯಾಂ ನಮಃ  ~।\\(ಹೃದಯಾದಿ ದಕ್ಷ ಕರಾಂಗುಲ್ಯಂತಂ )\\
\as{೪ ಷಂ}  ಮಜ್ಜಾತ್ಮಕ ಶ್ವೇತೇಶ ರಕ್ಷೋಪಧಾರಿಣೀಭ್ಯಾಂ ನಮಃ  ~।\\(ಹೃದಯಾದಿ ವಾಮ ಕರಾಂಗುಲ್ಯಂತಂ )\\
\as{೪ ಸಂ}  ಶುಕ್ರಾತ್ಮಕ ಭೃಗ್ವೀಶ ಸಹಜಾಭ್ಯಾಂ ನಮಃ  ~।\\(ಹೃದಯಾದಿ ದಕ್ಷ ಪಾದಾಂತಂ )\\
\as{೪ ಹಂ}  ಪ್ರಾಣಾತ್ಮಕ ನಕುಲೀಶ ಮಹಾಲಕ್ಷ್ಮೀಭ್ಯಾಂ ನಮಃ  ~।\\(ಹೃದಯಾದಿ ವಾಮ ಪಾದಾಂತಂ )\\
\as{೪ ಳಂ}  ಶಕ್ತ್ಯಾತ್ಮಕ ಶಿವೇಶ ವ್ಯಾಪಿನೀಭ್ಯಾಂ ನಮಃ  ~।\\(ಕಟ್ಯಾದಿ ಪಾದಪರ್ಯಂತಂ)\\
\as{೪ ಕ್ಷಂ} ಶಿವಾತ್ಮಕ ಸಂವರ್ತಕೇಶ ಮಹಾಮಾಯಾಭ್ಯಾಂ ನಮಃ  ~।\\(ಕಟ್ಯಾದಿ ಶಿರಃಪರ್ಯಂತಂ )\\
ಪ್ರಾಗ್ವದುತ್ತರನ್ಯಾಸಃ
\section{ಪೀಠಪೂಜಾ}ಓಂ ಐಂಹ್ರೀಂಶ್ರೀಂ ಅಮೃತಾಂಭೋನಿಧಯೇ ನಮಃ\\೪ ರತ್ನದ್ವೀಪಾಯ ನಮಃ\\೪ ನಾನಾವೃಕ್ಷಮಹೋದ್ಯಾನಾಯ ನಮಃ\\೪ ಕಲ್ಪವೃಕ್ಷವಾಟಿಕಾಯೈ ನಮಃ\\೪ ಸಂತಾನವಾಟಿಕಾಯೈ ನಮಃ\\೪ ಹರಿಚಂದನವಾಟಿಕಾಯೈ ನಮಃ\\೪ ಮಂದಾರವಾಟಿಕಾಯೈ ನಮಃ\\೪ ಪಾರಿಜಾತವಾಟಿಕಾಯೈ ನಮಃ\\೪ ಕದಂಬವಾಟಿಕಾಯೈ ನಮಃ\\೪ ಪುಷ್ಯರಾಗರತ್ನಪ್ರಾಕಾರಾಯ ನಮಃ\\೪ ಪದ್ಮರಾಗರತ್ನಪ್ರಾಕಾರಾಯ ನಮಃ\\೪ ಗೋಮೇಧಕರತ್ನಪ್ರಾಕಾರಾಯ ನಮಃ\\೪ ವಜ್ರರತ್ನಪ್ರಾಕಾರಾಯ ನಮಃ\\೪ ವೈಡೂರ್ಯರತ್ನಪ್ರಾಕಾರಾಯ ನಮಃ\\೪ ಇಂದ್ರನೀಲರತ್ನಪ್ರಾಕಾರಾಯ ನಮಃ\\೪ ಮುಕ್ತಾರತ್ನಪ್ರಾಕಾರಾಯ ನಮಃ\\೪ ಮರಕತರತ್ನಪ್ರಾಕಾರಾಯ ನಮಃ\\೪ ವಿದ್ರುಮರತ್ನಪ್ರಾಕಾರಾಯ ನಮಃ\\೪ ಮಾಣಿಕ್ಯಮಂಡಪಾಯ ನಮಃ\\೪ ಸಹಸ್ರಸ್ತಂಭಮಂಡಪಾಯ ನಮಃ\\೪ ಅಮೃತವಾಪಿಕಾಯೈ ನಮಃ\\೪ ಆನಂದವಾಪಿಕಾಯೈ ನಮಃ\\೪ ವಿಮರ್ಶವಾಪಿಕಾಯೈ ನಮಃ\\೪ ಬಾಲಾತಪೋದ್ಗಾರಕಕ್ಷಾಯ ನಮಃ\\೪ ಚಂದ್ರಿಕೋದ್ಗಾರಕಕ್ಷಾಯ ನಮಃ\\೪ ಮಹಾಶೃಂಗಾರಪರಿಘಾಯೈ ನಮಃ\\೪ ಮಹಾಪದ್ಮಾಟವ್ಯೈ ನಮಃ\\೪ ಚಿಂತಾಮಣಿಮಯಗೃಹರಾಜಾಯ ನಮಃ\\೪ ಪೂರ್ವಾಮ್ನಾಯಮಯಪೂರ್ವದ್ವಾರಾಯ ನಮಃ\\೪ ದಕ್ಷಿಣಾಮ್ನಾಯಮಯದಕ್ಷಿಣದ್ವಾರಾಯ ನಮಃ\\೪ ಪಶ್ಚಿಮಾಮ್ನಾಯಮಯಪಶ್ಚಿಮದ್ವಾರಾಯ ನಮಃ\\೪ ಉತ್ತರಾಮ್ನಾಯಮಯೋತ್ತರದ್ವಾರಾಯ ನಮಃ\\೪ ರತ್ನಪ್ರದೀಪವಲಯಾಯ ನಮಃ\\೪ ಮಣಿಮಯಮಹಾಸಿಂಹಾಸನಾಯ ನಮಃ\\೪ ಬ್ರಹ್ಮಮಯೈಕಮಂಚಪಾದಾಯ ನಮಃ(ಆಗ್ನೇಯ್ಯಾಂ)\\೪ ವಿಷ್ಣುಮಯೈಕಮಂಚಪಾದಾಯ ನಮಃ(ನೈರೃತ್ಯಾಂ)\\೪ ರುದ್ರಮಯೈಕಮಂಚಪಾದಾಯ ನಮಃ(ವಾಯವ್ಯಾಂ)\\೪ ಈಶ್ವರಮಯೈಕಮಂಚಪಾದಾಯ ನಮಃ(ಐಶಾನ್ಯಾಂ)\\೪ ಸದಾಶಿವಮಯೈಕಮಂಚಫಲಕಾಯ ನಮಃ\\೪ ಹಂಸತೂಲಿಕಾತಲ್ಪಾಯ ನಮಃ\\೪ ಹಂಸತೂಲಿಕಾತಲ್ಪಮಹೋಪಧಾನಾಯ ನಮಃ\\೪ ಕೌಸುಂಭಾಸ್ತರಣಾಯ ನಮಃ\\೪ ಮಹಾವಿತಾನಕಾಯ ನಮಃ\\೪ ಮಹಾಮಾಯಾಜವನಿಕಾಯೈ ನಮಃ
\section{ಅಥ ನಿತ್ಯಾ ಮಂತ್ರಾಃ\\೧। ಶ್ರೀಕಾಮೇಶ್ವರೀನಿತ್ಯಾ}
ಅಸ್ಯ ಶ್ರೀಕಾಮೇಶ್ವರೀನಿತ್ಯಾಮಹಾಮಂತ್ರಸ್ಯ ಸಮ್ಮೋಹನ ಋಷಿಃ~। ಗಾಯತ್ರೀ ಛಂದಃ~। ಶ್ರೀಕಾಮೇಶ್ವರೀದೇವತಾ~। ಕಂ ಬೀಜಂ~। ಇಂ ಶಕ್ತಿಃ। ಲಂ ಕೀಲಕಂ~।\\
\as{ನ್ಯಾಸಃ :}೧.ಓಂ ಐಂ ೨.ಓಂ ಸಕಲಹ್ರೀಂ ೩.ಓಂ ನಿತ್ಯ  ೪.ಓಂ ಕ್ಲಿನ್ನೇ ೫.ಓಂ ಮದದ್ರವೇ ೬.ಓಂ ಸೌಃ \\
\newpage
{\bfseries ದೇವೀಂ ಧ್ಯಾಯೇಜ್ಜಗದ್ಧಾತ್ರೀಂ ಜಪಾಕುಸುಮಸನ್ನಿಭಾಂ~।\\
ಬಾಲಭಾನುಪ್ರತೀಕಾಶಾಂ ಶಾತಕುಂಭಸಮಪ್ರಭಾಂ ॥\\
ರಕ್ತವಸ್ತ್ರಪರೀಧಾನಾಂ ಸಂಪದ್ವಿದ್ಯಾವಶಂಕರೀಂ~।\\
ನಮಾಮಿ ವರದಾಂ ದೇವೀಂ ಕಾಮೇಶೀಮಭಯಪ್ರದಾಂ ॥\\}
ಮನುಃ :{\bfseries  ಐಂ ಸಕಲಹ್ರೀಂ ನಿತ್ಯಕ್ಲಿನ್ನೇ ಮದದ್ರವೇ ಸೌಃ~॥}
\section{೨। ಭಗಮಾಲಿನೀನಿತ್ಯಾ}
ಅಸ್ಯ ಶ್ರೀಭಗಮಾಲಿನೀನಿತ್ಯಾಮಹಾಮಂತ್ರಸ್ಯ ಸುಭಗಋಷಿಃ~। ಗಾಯತ್ರೀ ಛಂದಃ~। ಶ್ರೀಭಗಮಾಲಿನೀ ದೇವತಾ~। ಹ್‌ರ್‌ಬ್ಲೇಂ ಬೀಜಂ~।  ಶ್ರೀಂ ಶಕ್ತಿಃ~। ಕ್ಲೀಂ ಕೀಲಕಂ~।\\
\as{ನ್ಯಾಸಃ :}೧.ಓಂ ಐಂ  ೨.ಓಂ ಭಗಭುಗೇ ೩.ಓಂ ಭಗಿನಿ  ೪.ಓಂ ಭಗೋದರಿ  ೫.ಓಂ ಭಗಮಾಲೇ ೬.ಓಂ ಭಗಾವಹೇ\\
{\bfseries ಭಗರೂಪಾಂ ಭಗಮಯಾಂ ದುಕೂಲವಸನಾಂ ಶಿವಾಂ~।\\
ಸರ್ವಾಲಂಕಾರಸಂಯುಕ್ತಾಂ ಸರ್ವಲೋಕವಶಂಕರೀಂ ॥\\
ಭಗೋದರೀಂ ಮಹಾದೇವೀಂ ರಕ್ತೋತ್ಪಲಸಮಪ್ರಭಾಂ~।\\
ಕಾಮೇಶ್ವರಾಂಕನಿಲಯಾಂ ವಂದೇ ಶ್ರೀಭಗಮಾಲಿನೀಂ ॥\\}
ಮನುಃ :{\bfseries  ಐಂ ಭಗಭುಗೇ ಭಗಿನಿ ಭಗೋದರಿ ಭಗಮಾಲೇ ಭಗಾವಹೇ ಭಗಗುಹ್ಯೇ ಭಗಯೋನಿ ಭಗನಿಪಾತನಿ ಸರ್ವಭಗವಶಂಕರಿ ಭಗರೂಪೇ ನಿತ್ಯಕ್ಲಿನ್ನೇ ಭಗಸ್ವರೂಪೇ ಸರ್ವಾಣಿ ಭಗಾನಿ ಮೇ ಹ್ಯಾನಯ ವರದೇ ರೇತೇ ಸುರೇತೇ ಭಗಕ್ಲಿನ್ನೇ ಕ್ಲಿನ್ನದ್ರವೇ ಕ್ಲೇದಯ ದ್ರಾವಯ ಅಮೋಘೇ ಭಗವಿಚ್ಚೇ ಕ್ಷುಭ ಕ್ಷೋಭಯ ಸರ್ವಸತ್ವಾನ್ ಭಗೇಶ್ವರಿ ಐಂ ಬ್ಲೂಂ ಜಂ ಬ್ಲೂಂ ಭೇಂ ಬ್ಲೂಂ ಮೋಂ ಬ್ಲೂಂ ಹೇಂ ಬ್ಲೂಂ ಹೇಂ ಕ್ಲಿನ್ನೇ ಸರ್ವಾಣಿ ಭಗಾನಿ ಮೇ ವಶಮಾನಯ ಸ್ತ್ರೀಂ ಹ್‌ರ್‌ಬ್ಲೇಂ ಹ್ರೀಂ ॥}
\section{೩।ನಿತ್ಯಕ್ಲಿನ್ನಾ}
ಅಸ್ಯ ಶ್ರೀನಿತ್ಯಕ್ಲಿನ್ನಾಮಹಾಮಂತ್ರಸ್ಯ ಬ್ರಹ್ಮಾ ಋಷಿಃ। ವಿರಾಟ್ಛಂದಃ~।\\ ಶ್ರೀನಿತ್ಯಕ್ಲಿನ್ನಾನಿತ್ಯಾದೇವತಾ। ಹ್ರೀಂ ಬೀಜಂ। ಸ್ವಾಹಾ ಶಕ್ತಿಃ । ನ್ನೇ ಕೀಲಕಂ~।\\
\as{ನ್ಯಾಸಃ :}೧.ಓಂ ಹ್ರೀಂ ೨.ಓಂ ನಿತ್ಯ  ೩.ಓಂ ಕ್ಲಿನ್ನೇ ೪.ಓಂ ಮದ ೫.ಓಂ ದ್ರವೇ ೬.ಓಂ ಸ್ವಾಹಾ\\
{\bfseries ಪದ್ಮರಾಗಮಣಿಪ್ರಖ್ಯಾಂ ಹೇಮತಾಟಂಕಸಂಯುತಾಂ~।\\
ರಕ್ತವಸ್ತ್ರಧರಾಂ ದೇವೀಂ ರಕ್ತಮಾಲ್ಯಾನುಲೇಪನಾಂ~॥\\
ಅಂಜನಾಂಚಿತನೇತ್ರಾಂ ತಾಂ ಪದ್ಮಪತ್ರನಿಭೇಕ್ಷಣಾಂ~।\\
ನಿತ್ಯಕ್ಲಿನ್ನಾಂ ನಮಸ್ಯಾಮಿ ಚತುರ್ಭುಜವಿರಾಜಿತಾಂ~॥\\}
ಮನುಃ :{\bfseries  ಹ್ರೀಂ ನಿತ್ಯಕ್ಲಿನ್ನೇ ಮದದ್ರವೇ ಸ್ವಾಹಾ ॥}
\section{೪।ಭೇರುಂಡಾ}
ಶ್ರೀಭೇರುಂಡಾನಿತ್ಯಾಮಹಾಮಂತ್ರಸ್ಯ ಮಹಾವಿಷ್ಣುಃ ಋಷಿಃ। \\ಗಾಯತ್ರೀಛಂದಃ~। ಭೇರುಂಡಾನಿತ್ಯಾ ದೇವತಾ। ಭ್ರೋಂ ಬೀಜಂ।\\ ಸ್ವಾಹಾ ಶಕ್ತಿಃ। ಕ್ರೋಂ ಕೀಲಕಂ~।\\
\as{ನ್ಯಾಸಃ :}೧.ಓಂ ಕ್ರೋಂ  ೨.ಓಂ ಭ್ರೋಂ ೩.ಓಂ ಕ್ರೋಂ  ೪.ಓಂ ಝ್ರೋಂ ೫.ಓಂ ಛ್ರೋಂ  ೬.ಓಂ ಜ್ರೋಂ \\
{\bfseries ಶುದ್ಧಸ್ಫಟಿಕಸಂಕಾಶಾಂ ಪದ್ಮಪತ್ರಸಮಪ್ರಭಾಂ~।\\
ಮಧ್ಯಾಹ್ನಾದಿತ್ಯಸಂಕಾಶಾಂ ಶುಭ್ರವಸ್ತ್ರಸಮನ್ವಿತಾಂ~॥\\
ಶ್ವೇತಚಂದನಲಿಪ್ತಾಂಗೀಂ ಶುಭ್ರಮಾಲ್ಯವಿಭೂಷಿತಾಂ~।\\	
ಬಿಭ್ರತೀಂ ಚಿನ್ಮಯೀಂ ಮುದ್ರಾಮಕ್ಷಮಾಲಾಂ ಚ ಪುಸ್ತಕಂ~॥\\
ಸಹಸ್ರಪದ್ಮಕಮಲೇ ಸಮಾಸೀನಾಂ ಶುಚಿಸ್ಮಿತಾಂ~।\\
ಸರ್ವವಿದ್ಯಾಪ್ರದಾಂ ದೇವೀಂ ಭೇರುಂಡಾಂ ಪ್ರಣಮಾಮ್ಯಹಂ~॥\\}
ಮನುಃ :{\bfseries  ಕ್ರೋಂ ಭ್ರೋಂ ಕ್ರೋಂ ಝ್ರೋಂ ಛ್ರೋಂ ಜ್ರೋಂ ಸ್ವಾಹಾ ॥}
\section{೫।ವಹ್ನಿವಾಸಿನೀನಿತ್ಯಾ}
ಶ್ರೀವಹ್ನಿವಾಸಿನೀನಿತ್ಯಾ ಮಂತ್ರಸ್ಯ ವಸಿಷ್ಠ ಋಷಿಃ। ಗಾಯತ್ರೀಛಂದಃ~।\\ ಶ್ರೀವಹ್ನಿವಾಸಿನೀನಿತ್ಯಾದೇವತಾ। ಹ್ರೀಂ ಬೀಜಂ। ನಮಃ ಶಕ್ತಿಃ।\\ ವಹ್ನಿವಾಸಿನ್ಯೈ ಕೀಲಕಂ~। ಹ್ರಾಂ ಇತ್ಯಾದಿನ್ಯಾಸಃ~।\\
{\bfseries ವಹ್ನಿಕೋಟಿಪ್ರತೀಕಾಶಾಂ ಸೂರ್ಯಕೋಟಿಸಮಪ್ರಭಾಂ~।\\
ಅಗ್ನಿಜ್ವಾಲಾಸಮಾಕೀರ್ಣಾಂ ಸರ್ವರೋಗಾಪಹಾರಿಣೀಂ~॥\\
ಕಾಲಮೃತ್ಯುಪ್ರಶಮನೀಂ ಭಯಮೃತ್ಯುನಿವಾರಿಣೀಂ~।\\
ಪರಮಾಯುಷ್ಯದಾಂ ವಂದೇ ನಿತ್ಯಾಂ ಶ್ರೀವಹ್ನಿವಾಸಿನೀಂ ॥\\}
ಮನುಃ :{\bfseries ಓಂ ಹ್ರೀಂ ವಹ್ನಿವಾಸಿನ್ಯೈ ನಮಃ~॥}
\section{೬। ಮಹಾವಜ್ರೇಶ್ವರೀ}
ಶ್ರೀಮಹಾವಜ್ರೇಶ್ವರೀನಿತ್ಯಾ ಮಂತ್ರಸ್ಯ ಬ್ರಹ್ಮಾ ಋಷಿಃ। ಗಾಯತ್ರೀಛಂದಃ~। ಶ್ರೀಮಹಾವಜ್ರೇಶ್ವರೀನಿತ್ಯಾದೇವತಾ। ಹ್ರೀಂ ಬೀಜಂ। ಹ್ರೀಂ ಶಕ್ತಿಃ।\\
ಐಂ ಕೀಲಕಂ~।\\
\as{ನ್ಯಾಸಃ :}೧.ಓಂ ಹ್ರೀಂ ಕ್ಲಿನ್ನೇ ಹ್ರೀಂ  ೨.ಓಂ ಹ್ರೀಂ ಐಂ ಹ್ರೀಂ  ೩.ಓಂ ಹ್ರೀಂ ಕ್ರೋಂ ಹ್ರೀಂ ೪.ಓಂ ಹ್ರೀಂ ನಿತ್ಯ ಹ್ರೀಂ  ೫.ಓಂ ಹ್ರೀಂ ಮದ ಹ್ರೀಂ ೬.ಓಂ ಹ್ರೀಂ ದ್ರವೇ ಹ್ರೀಂ \\
{\bfseries ತಪ್ತಕಾಂಚನಸಂಕಾಶಾಂ ಕನಕಾಭರಣಾನ್ವಿತಾಂ~।\\
ಹೇಮತಾಟಂಕಸಂಯುಕ್ತಾಂ ಕಸ್ತೂರೀತಿಲಕಾನ್ವಿತಾಂ ॥\\
ಹೇಮಚಿಂತಾಕಸಂಯುಕ್ತಾಂ ಪೂರ್ಣಚಂದ್ರಮುಖಾಂಬುಜಾಂ~।\\
ಪೀತಾಂಬರಸಮೋಪೇತಾಂ ಪುಣ್ಯಮಾಲ್ಯವಿಭೂಷಿತಾಂ ॥\\
ಮುಕ್ತಾಹಾರಸಮೋಪೇತಾಂ ಮುಕುಟೇನ ವಿರಾಜಿತಾಂ~।\\
ಮಹಾವಜ್ರೇಶ್ವರೀಂ ವಂದೇ ಸರ್ವೈಶ್ವರ್ಯಫಲಪ್ರದಾಂ ॥\\}
ಮನುಃ :{\bfseries  ಓಂ ಹ್ರೀಂ ಕ್ಲಿನ್ನೇ ಐಂ ಕ್ರೋಂ ನಿತ್ಯಮದದ್ರವೇ ಹ್ರೀಂ~॥}
\section{೭।ಶಿವಾದೂತೀನಿತ್ಯಾ}
ಶ್ರೀಶಿವಾದೂತೀನಿತ್ಯಾ ಮಂತ್ರಸ್ಯ ರುದ್ರಋಷಿಃ~। ಗಾಯತ್ರೀ ಛಂದಃ~।\\ ಶ್ರೀಶಿವಾದೂತೀನಿತ್ಯಾ ದೇವತಾ~। ಹ್ರೀಂ ಬೀಜಂ~। ನಮಃ ಶಕ್ತಿಃ~।\\ ಶಿವಾದೂತ್ಯೈ ಕೀಲಕಂ~। ಹ್ರಾಂ ಇತ್ಯಾದಿನಾ ನ್ಯಾಸಃ~।
{\bfseries ಬಾಲಸೂರ್ಯಪ್ರತೀಕಾಶಾಂ ಬಂಧೂಕಪ್ರಸವಾರುಣಾಂ~।\\
ವಿಧಿವಿಷ್ಣುಶಿವಸ್ತುತ್ಯಾಂ ದೇವಗಂಧರ್ವಸೇವಿತಾಂ ॥\\
ರಕ್ತಾರವಿಂದಸಂಕಾಶಾಂ ಸರ್ವಾಭರಣಭೂಷಿತಾಂ~।\\
ಶಿವದೂತೀಂ ನಮಸ್ಯಾಮಿ ರತ್ನಸಿಂಹಾಸನಸ್ಥಿತಾಂ ॥\\}
ಮನುಃ :{\bfseries  ಓಂ ಹ್ರೀಂ ಶಿವಾದೂತ್ಯೈ ನಮಃ~॥}
\section{೮। ತ್ವರಿತಾ ನಿತ್ಯಾ}
ಅಸ್ಯ ಶ್ರೀತ್ವರಿತಾನಿತ್ಯಾ ಮಹಾಮಂತ್ರಸ್ಯ ಈಶ್ವರ ಋಷಿಃ~। ವಿರಾಟ್ ಛಂದಃ~। ತ್ವರಿತಾನಿತ್ಯಾ ದೇವತಾ। ಹೂಂ ಬೀಜಂ~। ಸ್ತ್ರೀಂ ಶಕ್ತಿಃ~। ಹ್ರೀಂ ಕೀಲಕಂ~।\\
ಹ್ರಾಂ ಇತ್ಯಾದಿನ್ಯಾಸಃ ।\\
{\bfseries ರಕ್ತಾರವಿಂದಸಂಕಾಶಾಮುದ್ಯತ್ಸೂರ್ಯಸಮಪ್ರಭಾಂ~।\\
ದಧತೀಮಂಕುಶಂ ಪಾಶಂ ಬಾಣಾನ್ ಚಾಪಂ ಮನೋಹರಂ ॥\\
ಚತುರ್ಭುಜಾಂ ಮಹಾದೇವೀಮಪ್ಸರೋಗಣಸಂಕುಲಾಂ~।\\
ನಮಾಮಿ ತ್ವರಿತಾಂ ನಿತ್ಯಾಂ ಭಕ್ತಾನಾಮಭಯಪ್ರದಾಂ ॥\\}
ಮನುಃ :{\bfseries  ಓಂ ಹ್ರೀಂ ಹೂಂ ಖೇ ಚ ಛೇ ಕ್ಷಃ ಸ್ತ್ರೀಂ ಹೂಂ ಕ್ಷೇ ಹ್ರೀಂ ಫಟ್~॥}
\newpage
\section{೯। ಕುಲಸುಂದರೀನಿತ್ಯಾ}
ಅಸ್ಯ ಶ್ರೀ ಕುಲಸುಂದರೀ ನಿತ್ಯಾ ಮಹಾಮಂತ್ರಸ್ಯ ದಕ್ಷಿಣಾಮೂರ್ತಿಃ ಋಷಿಃ~। ಪಂಕ್ತಿಶ್ಛಂದಃ~। ಶ್ರೀ ಕುಲಸುಂದರೀ ನಿತ್ಯಾ ದೇವತಾ~। ಐಂ ಬೀಜಂ~। ಸೌಃ ಶಕ್ತಿಃ~। ಕ್ಲೀಂ ಕೀಲಕಂ~। ಆಂ , ಈಂ , ಇತ್ಯಾದಿನ್ಯಾಸಃ~।\\
{\bfseries ಅರುಣಕಿರಣಜಾಲೈ ರಂಜಿತಾಶಾವಕಾಶಾ\\
ವಿಧೃತಜಪವಟೀಕಾ ಪುಸ್ತಕಾಭೀತಿಹಸ್ತಾ ॥\\
ಇತರಕರವರಾಢ್ಯಾ ಫುಲ್ಲಕಹ್ಲಾರಸಂಸ್ಥಾ\\
ನಿವಸತು ಹೃದಿ ಬಾಲಾ ನಿತ್ಯಕಲ್ಯಾಣಶೀಲಾ ॥\\}
ಮನುಃ :{\bfseries  ಐಂ ಕ್ಲೀಂ ಸೌಃ~॥}
\section{೧೦।ನಿತ್ಯಾನಿತ್ಯಾ}
ಅಸ್ಯ ಶ್ರೀನಿತ್ಯಾನಿತ್ಯಾಮಹಾಮಂತ್ರಸ್ಯ ದಕ್ಷಿಣಾಮೂರ್ತಿಃ ಋಷಿಃ।\\ ಪಂಕ್ತಿಃ ಛಂದಃ~। ಶ್ರೀನಿತ್ಯಾನಿತ್ಯಾದೇವತಾ। ಐಂ ಬೀಜಂ। ಔಃ ಶಕ್ತಿಃ~।\\ ಈಂ ಕೀಲಕಂ~। ಹ್ರೀಂ ಹ್ಸಾಂ, ಹ್ರೀಂ ಹ್ಸೀಂ ಇತ್ಯಾದಿನಾ ನ್ಯಾಸಃ~।\\
{\bfseries ಉದ್ಯತ್ಪ್ರದ್ಯೋತನನಿಭಾಂ ಜಪಾಕುಸುಮಸನ್ನಿಭಾಂ~।\\
ಹರಿಚಂದನಲಿಪ್ತಾಂಗೀಂ ರಕ್ತಮಾಲ್ಯ ವಿಭೂಷಿತಾಂ ॥\\
ರತ್ನಾಭರಣಭೂಷಾಂಗೀಂ ರಕ್ತವಸ್ತ್ರಸುಶೋಭಿತಾಂ~।\\
ಜಗದಂಬಾಂ ನಮಸ್ಯಾಮಿ ನಿತ್ಯಾಂ ಶ್ರೀಪರಮೇಶ್ವರೀಂ ॥\\}
ಮನುಃ :{\bfseries  ಓಂ ಹಸಕಲರಡೈಂ ಹಸಕಲರಡೀಂ ಹಸಕಲರಡೌಃ ॥}
\section{೧೧।ಶ್ರೀನೀಲಪತಾಕಾನಿತ್ಯಾ}
ಅಸ್ಯ ಶ್ರೀನೀಲಪತಾಕಾನಿತ್ಯಾ ಮಹಾಮಂತ್ರಸ್ಯ ಸಮ್ಮೋಹನ ಋಷಿಃ~। \\ಗಾಯತ್ರೀ ಛಂದಃ~। ಶ್ರೀನೀಲಪತಾಕಾನಿತ್ಯಾ ದೇವತಾ~। ಹ್ರೀಂ ಬೀಜಂ~।\\ ಹ್ರೀಂ ಶಕ್ತಿಃ~। ಕ್ಲೀಂ ಕೀಲಕಂ~।\\
\as{ನ್ಯಾಸಃ :}೧.ಓಂ ಓಂ ಹ್ರೀಂ ಫ್ರೇಂ ೨.ಓಂ ಸ್ರೂಂ ಓಂ ಆಂ ಕ್ಲೀಂ ೩.ಓಂ ಐಂ ಬ್ಲೂಂ ನಿತ್ಯಮದ ೪.ಓಂ ದ್ರ ೫.ಓಂ ವೇ ೬.ಓಂ ಹುಂ \\
{\bfseries ಪಂಚವಕ್ತ್ರಾಂ ತ್ರಿಣಯನಾಮರುಣಾಂಶುಕಧಾರಿಣೀಂ~।\\
ದಶಹಸ್ತಾಂ ಲಸನ್ಮುಕ್ತಾಪ್ರಾಯಾಭರಣಮಂಡಿತಾಂ ॥\\
ನೀಲಮೇಘಸಮಪ್ರಖ್ಯಾಂ ಧೂಮ್ರಾರ್ಚಿಸ್ಸದೃಶಪ್ರಭಾಂ~।\\
ನೀಲಪುಷ್ಪಸ್ರಜೋಪೇತಾಂ ಧ್ಯಾಯೇನ್ನೀಲಪತಾಕಿನೀಂ ॥\\}
ಮನುಃ :{\bfseries ಓಂ ಹ್ರೀಂ ಫ್ರೇಂ ಸ್ರೂಂ ಓಂ ಆಂ ಕ್ಲೀಂ ಐಂ ಬ್ಲೂಂ ನಿತ್ಯಮದದ್ರವೇ ಹುಂ ಫ್ರೇಂ ಹ್ರೀಂ~॥}
\section{೧೨। ವಿಜಯಾ ನಿತ್ಯಾ}
ಅಸ್ಯ ಶ್ರೀ ವಿಜಯಾನಿತ್ಯಾಮಹಾಮಂತ್ರಸ್ಯ ಅಹಿರ್ಋಷಿಃ~। ಗಾಯತ್ರೀಛಂದಃ~। ಶ್ರೀವಿಜಯಾನಿತ್ಯಾ ದೇವತಾ~।\\
\as{ನ್ಯಾಸಃ :}೧.ಓಂ ಭಾಂ ೨.ಓಂ ಮೀಂ ೩.ಓಂ ರೂಂ ೪.ಓಂ ಯೈಂ ೫.ಓಂ ಉಂ ೬.ಓಂ ಔಂ \\
{\bfseries ಉದ್ಯದರ್ಕಸಹಸ್ರಾಭಾಂ ದಾಡಿಮೀಪುಷ್ಪಸನ್ನಿಭಾಂ~।\\
ರಕ್ತಕಂಕಣಕೇಯೂರಕಿರೀಟಾಂಗದಸಂಯುತಾಂ ॥\\
ದೇವಗಂಧರ್ವಯೋಗೀಶಮುನಿಸಿದ್ಧನಿಷೇವಿತಾಂ~।\\
ನಮಾಮಿ ವಿಜಯಾಂ ನಿತ್ಯಾಂ ಸಿಂಹೋಪರಿ ಕೃತಾಸನಾಂ ॥\\}
ಮನುಃ :{\bfseries  ಭ ಮ ರ ಯ ಉ ಔಂ ॥}
\section{೧೩। ಸರ್ವಮಂಗಲಾನಿತ್ಯಾ}
ಅಸ್ಯ ಶ್ರೀಸರ್ವಮಂಗಲಾನಿತ್ಯಾ ಮಹಾಮಂತ್ರಸ್ಯ ಚಂದ್ರ ಋಷಿಃ~।\\ ಗಾಯತ್ರೀ ಛಂದಃ~। ಸರ್ವಮಂಗಲಾನಿತ್ಯಾ ದೇವತಾ~।\\
ಸ್ವಾಂ , ಸ್ವೀಂ ಇತ್ಯಾದಿನಾ ನ್ಯಾಸಃ~।\\
{\bfseries ರಕ್ತೋತ್ಪಲಸಮಪ್ರಖ್ಯಾಂ ಪದ್ಮಪತ್ರನಿಭೇಕ್ಷಣಾಂ~।\\
ಇಕ್ಷುಕಾರ್ಮುಕಪುಷ್ಪೌಘಪಾಶಾಂಕುಶಸಮನ್ವಿತಾಂ ॥\\
ಸುಪ್ರಸನ್ನಾಂ ಶಶಿಮುಖೀಂ ನಾನಾರತ್ನವಿಭೂಷಿತಾಂ~।\\
ಶುಭ್ರಪದ್ಮಾಸನಸ್ಥಾಂ ತಾಂ ಭಜಾಮಿ ಸರ್ವಮಂಗಲಾಂ ॥\\}
ಮನುಃ :{\bfseries  ಸ್ವೌಂ ॥}
\section{೧೪। ಜ್ವಾಲಾಮಾಲಿನೀ}
ಅಸ್ಯ ಶ್ರೀ ಜ್ವಾಲಾಮಾಲಿನೀನಿತ್ಯಾಮಹಾಮಂತ್ರಸ್ಯ ಕಶ್ಯಪ ಋಷಿಃ~। ಗಾಯತ್ರೀ ಛಂದಃ~। ಜ್ವಾಲಾಮಾಲಿನೀನಿತ್ಯಾ ದೇವತಾ~। ರಂ ಬೀಜಂ~। ಫಟ್ ಶಕ್ತಿಃ~। ಹುಂ ಕೀಲಕಂ~।\\
\as{ನ್ಯಾಸಃ :}೧.ಓಂ ಓಂ ೨.ಓಂ ನಮಃ ೩.ಓಂ ಭಗವತಿ ೪.ಓಂ ಜ್ವಾಲಾಮಾಲಿನಿ ೫.ಓಂ ದೇವದೇವಿ ೬.ಓಂ ಸರ್ವಭೂತಸಂಹಾರಕಾರಿಕೇ \\
{\bfseries ಅಗ್ನಿಜ್ವಾಲಾಸಮಾಭಾಕ್ಷೀಂ ನೀಲವಕ್ತ್ರಾಂ ಚತುರ್ಭುಜಾಂ~।\\ನೀಲನೀರದಸಂಕಾಶಾಂ ನೀಲಕೇಶೀಂ ತನೂದರೀಂ ॥\\
ಖಡ್ಗಂ ತ್ರಿಶೂಲಂ ಬಿಭ್ರಾಣಾಂ ವರಂ ಸಾಭಯಮೇವ ಚ~।\\ಸಿಂಹಪೃಷ್ಠಸಮಾರೂಢಾಂ ಧ್ಯಾಯೇಜ್ಜ್ವಾಲಾದ್ಯಮಾಲಿನೀಂ ॥\\}
ಮನುಃ :{\bfseries  ಓಂ ನಮೋ ಭಗವತಿ ಜ್ವಾಲಾಮಾಲಿನಿ ದೇವದೇವಿ ಸರ್ವಭೂತಸಂಹಾರಕಾರಿಕೇ ಜಾತವೇದಸಿ ಜ್ವಲಂತಿ ಜ್ವಲ ಜ್ವಲ ಪ್ರಜ್ವಲ ಪ್ರಜ್ವಲ ಹ್ರಾಂ ಹ್ರೀಂ ಹ್ರೂಂ ರರ ರರ ರರರ ಹುಂ ಫಟ್ ಸ್ವಾಹಾ~॥}
\section{೧೫। ಚಿತ್ರಾನಿತ್ಯಾ}
ಚಿತ್ರಾನಿತ್ಯಾಮಹಾಮಂತ್ರಸ್ಯ ಬ್ರಹ್ಮಾ ಋಷಿಃ~। ಗಾಯತ್ರೀ ಛಂದಃ~। \\ಚಿತ್ರಾ ನಿತ್ಯಾ ದೇವತಾ~। ಚಾಂ ಚೀಂ ಇತ್ಯಾದಿನಾ ನ್ಯಾಸಃ~।\\
\newpage
{\bfseries ಶುದ್ಧಸ್ಫಟಿಕಸಂಕಾಶಾಂ ಪಲಾಶಕುಸುಮಪ್ರಭಾಂ~।\\
ನೀಲಮೇಘಪ್ರತೀಕಾಶಾಂ ಚತುರ್ಹಸ್ತಾಂ ತ್ರಿಲೋಚನಾಂ ॥\\
ಸರ್ವಾಲಂಕಾರಸಂಯುಕ್ತಾಂ ಪುಷ್ಪಬಾಣೇಕ್ಷುಚಾಪಿನೀಂ~।\\
ಪಾಶಾಂಕುಶಸಮೋಪೇತಾಂ ಧ್ಯಾಯೇಚ್ಚಿತ್ರಾಂ ಮಹೇಶ್ವರೀಂ ॥\\}
ಮನುಃ :{\bfseries  ಚ್ಕೌಂ~॥}

ಹ್ಸಾಂ ಹ್ಸೀಂ ಇತ್ಯಾದಿನಾ ಕರಾಂಗನ್ಯಾಸಂ ವಿಧಾಯ
\section{ಧ್ಯಾನಂ}
ಪಂಚವಕ್ತ್ರಂ ಚತುರ್ಬಾಹುಂ ಸರ್ವಾಭರಣ ಭೂಷಿತಮ್ ।\\
ಚಂದ್ರಸೂರ್ಯಸಹಸ್ರಾಭಂ ಶಿವಶಕ್ತ್ಯಾತ್ಮಕಂ ಭಜೇ ॥

ಅಮೃತಾರ್ಣವಮಧ್ಯಸ್ಥ ರತ್ನದ್ವೀಪೇ ಮನೋರಮೇ~।\\
ಕಲ್ಪವೃಕ್ಷವನಾಂತಸ್ಥೇ ನವಮಾಣಿಕ್ಯಮಂಡಪೇ ॥

ನವರತ್ನಮಯ ಶ್ರೀಮತ್ ಸಿಂಹಾಸನಾಗತಾಂಬುಜೇ~।\\
ತ್ರಿಕೋಣಾಂತಃ ಸಮಾಸೀನಂ ಚಂದ್ರಸೂರ್ಯಾಯುತಪ್ರಭಂ ॥

ಅರ್ಧಾಂಬಿಕಾಸಮಾಯುಕ್ತಂ ಪ್ರವಿಭಕ್ತವಿಭೂಷಣಂ~।\\
ಕೋಟಿಕಂದರ್ಪಲಾವಣ್ಯಂ ಸದಾ ಷೋಡಶವಾರ್ಷಿಕಂ ॥

ಮಂದಸ್ಮಿತಮುಖಾಂಭೋಜಂ ತ್ರಿನೇತ್ರಂ ಚಂದ್ರಶೇಖರಂ~।\\
ದಿವ್ಯಾಂಬರಸ್ರಗಾಲೇಪಂ ದಿವ್ಯಾಭರಣಭೂಷಿತಂ ॥

ಪಾನಪಾತ್ರಂ ಚ ಚಿನ್ಮುದ್ರಾಂ ತ್ರಿಶೂಲಂ ಪುಸ್ತಕಂ ಕರೈಃ~।\\
ವಿದ್ಯಾ ಸಂಸದಿ ಬಿಭ್ರಾಣಂ ಸದಾನಂದಮುಖೇಕ್ಷಣಂ ॥
\newpage
ಮಹಾಷೋಢೋದಿತಾಶೇಷದೇವತಾಗಣಸೇವಿತಂ~।\\
ಏವಂ ಚಿತ್ತಾಂಬುಜೇ ಧ್ಯಾಯೇದರ್ಧನಾರೀಶ್ವರಂ ಶಿವಂ ॥

ಪುರುಷಂ ವಾ ಸ್ಮರೇದ್ ದೇವಿ ಸ್ತ್ರೀರೂಪಂ ವಾ ವಿಚಿಂತಯೇತ್~।\\
ಅಥವಾ ನಿಷ್ಕಲಂ ಧ್ಯಾಯೇತ್ ಸಚ್ಚಿದಾನಂದವಿಗ್ರಹಂ~।\\
ಸರ್ವತೇಜೋಮಯಂ ದಿವ್ಯಂ ಸಚರಾಚರವಿಗ್ರಹಂ ॥

\dhyana{ಸಹಸ್ರದಲಪಂಕಜೇ ಸಕಲಶೀತರಶ್ಮಿಪ್ರಭಂ\\
ವರಾಭಯಕರಾಂಬುಜಂ ವಿಮಲಗಂಧಪುಷ್ಪಾಂಬರಂ~।\\
ಪ್ರಸನ್ನವದನೇಕ್ಷಣಂ ಸಕಲದೇವತಾರೂಪಿಣಂ\\
ಸ್ಮರೇಚ್ಛಿರಸಿ ಹಂಸಗಂ ತದಭಿಧಾನಪೂರ್ವಂ ಗುರುಂ ॥}\\
ಓಂ ಐಂಹ್ರೀಂಶ್ರೀಂ ಐಂಕ್ಲೀಂಸೌಃ ಹ್‌ಸ್‌ಖ್‌ಫ್ರೇಂ ಹಸಕ್ಷಮಲವರಯೂಂ ಹ್ಸೌಃ ಸಹಕ್ಷಮಲವರಯೀಂ ಸ್ಹೌಃ ॥ ಇತಿ ಗುರುಂ ಧ್ಯಾತ್ವಾ

ಶಾಂತಂ ಪದ್ಮಾಸನಸ್ಥಂ ಶಶಧರಮುಕುಟಂ ಪಂಚವಕ್ತ್ರಂ ತ್ರಿನೇತ್ರಂ\\
ಶೂಲಂ ವಜ್ರಂ ಚ ಖಡ್ಗಂ ಪರಶುಮಭಯದಂ ದಕ್ಷಭಾಗೇ ವಹಂತಂ~।\\
ನಾಗಂ ಪಾಶಂ ಚ ಘಂಟಾಂ ಪ್ರಲಯಹುತವಹಂ ಚಾಂಕುಶಂ ವಾಮಭಾಗೇ\\
ನಾನಾಲಂಕಾರಯುಕ್ತಂ ಸ್ಫಟಿಕಮಣಿನಿಭಂ ಪಾರ್ವತೀಶಂ ನಮಾಮಿ ॥

ಅಥಾಹಂ ಬೈಂದವೇ ಚಕ್ರೇ ಸರ್ವಾನಂದಮಯಾತ್ಮಕೇ~।\\
ರತ್ನಸಿಂಹಾಸನೇ ರಮ್ಯೇ ಸಮಾಸೀನಾಂ ಶಿವಪ್ರಿಯಾಂ ॥

ಉದ್ಯದ್ಭಾನುಸಹಸ್ರಾಭಾಂ ಜಪಾಪುಷ್ಪಸಮಪ್ರಭಾಂ~।\\
ನವರತ್ನಪ್ರಭಾಯುಕ್ತಮಕುಟೇನ ವಿರಾಜಿತಾಂ ॥

ಚಂದ್ರರೇಖಾಸಮೋಪೇತಾಂ ಕಸ್ತೂರೀತಿಲಕಾಂಕಿತಾಂ~।\\
ಕಾಮಕೋದಂಡಸೌಂದರ್ಯನಿರ್ಜಿತ ಭ್ರೂಲತಾಯುತಾಂ ॥

ಅಂಜನಾಂಚಿತನೇತ್ರಾಂ ತಾಂ ಪದ್ಮಪತ್ರ ನಿಭೇಕ್ಷಣಾಂ~।\\
ಮಣಿಕುಂಡಲಸಂಯುಕ್ತ ಕರ್ಣದ್ವಯ ವಿರಾಜಿತಾಂ ॥

ಮುಕ್ತಾಮಾಣಿಕ್ಯಖಚಿತ ನಾಸಿಕಾಭರಣಾನ್ವಿತಾಂ~।\\
ಶುದ್ಧಮುಕ್ತಾವಲೀಪ್ರಖ್ಯ ದಂತಪಂಕ್ತಿವಿರಾಜಿತಾಂ ॥

ಪಕ್ವಬಿಂಬ ಫಲಾಭಾಸಾಧರದ್ವಯ ರಾಜಿತಾಂ~।\\
ತಾಂಬೂಲಪೂರಿತಮುಖೀಂ ಸುಸ್ಮಿತಾಸ್ಯವಿರಾಜಿತಾಂ ॥

ಆದ್ಯಭೂಷಣಸಂಯುಕ್ತಾಂ ಹೇಮಚಿಂತಾಕಸಂಯುತಾಂ~।\\
ಪದಕೇನಸಮಾಯುಕ್ತಾಂ ಮಹಾಪದಕಸಂಯುತಾಂ ॥

ಮುಕ್ತಾವಲಿಸಮೋಪೇತಾಮೇಕಾವಲಿಸಮನ್ವಿತಾಂ~।\\
ಕೇಯೂರಾಂಗದ ಸಂಯುಕ್ತಾಂ ಚತುರ್ಬಾಹುವಿರಾಜಿತಾಂ॥

ಅಷ್ಟಗಂಧಸಮಾಯುಕ್ತ ಶ್ರೀಚಂದನವಿಲೇಪನಾಂ~।\\
ಹೇಮಕುಂಭಸಮಪ್ರಖ್ಯ ಸ್ತನದ್ವಯವಿರಾಜಿತಾಂ ॥

ರಕ್ತವಸ್ತ್ರಪರೀಧಾನಾಂ ರಕ್ತಕಂಚುಕಸಂಯುತಾಂ~।\\
ಸೂಕ್ಷ್ಮರೋಮಾವಲೀಯುಕ್ತ ತನುಮಧ್ಯವಿರಾಜಿತಾಂ ॥

ಮುಕ್ತಾಮಾಣಿಕ್ಯಖಚಿತ ಕಾಂಚೀಯುತನಿತಂಬಿನೀಂ~।\\
ಸದಾಶಿವಾಂಕಸ್ಥಪೃಥು ಮಹಾಜಘನಮಂಡಲಾಂ ॥

ಕದಲೀಸ್ತಂಭಸಂಕಾಶಸಕ್ಥಿ ದ್ವಯವಿರಾಜಿತಾಂ~।\\
ಕಾಹಲೀಕಾಂತಿವಿಲಸಜ್ಜಂಘಾ ಯುಗಲಶೋಭಿತಾಂ ॥

ಗೂಢಗುಲ್ಫದ್ವಯೋಪೇತಾಂ ರಕ್ತಪಾದದ್ವಯಾನ್ವಿತಾಂ~।\\
ಬ್ರಹ್ಮವಿಷ್ಣುಮಹಾದೇವಶಿರೋಮುಕುಟಜಾತಯಾ ॥

ಕಾಂತ್ಯಾ ವಿರಾಜಿತಪದಾಂ ಭಕ್ತತ್ರಾಣ ಪರಾಯಣಾಂ~।\\
ಇಕ್ಷುಕಾರ್ಮುಕ ಪುಷ್ಪೇಷು ಪಾಶಾಂಕುಶ ಧರಾಂ ಪರಾಂ~।\\
ಸಂವಿತ್ಸ್ವರೂಪಿಣೀಂ ದೇವೀಂ ಧ್ಯಾಯಾಮಿ ಪರಮೇಶ್ವರೀಂ ॥

೪ ಹ್ರೀಂಶ್ರೀಂಸೌಃ ಲಲಿತಾಯಾ ಅಮೃತಚೈತನ್ಯಮೂರ್ತಿಂ ಕಲ್ಪಯಾಮಿ ನಮಃ\\
೪ ಹ್‌ಸ್‌ರೈಂ ಹ್‌ಸ್‌ಕ್ಲ್ರೀಂ ಹ್‌ಸ್‌ರ್ಸೌಃ\\
\dhyana{ಮಹಾಪದ್ಮವನಾಂತಸ್ಥೇ ಕಾರಣಾನಂದವಿಗ್ರಹೇ~।\\
ಸರ್ವಭೂತಹಿತೇ ಮಾತಃ ಏಹ್ಯೇಹಿ ಪರಮೇಶ್ವರಿ ॥\\
ದೇವೇಶಿ ಭಕ್ತಸುಲಭೇ ಸರ್ವಾವರಣಸಂಯುತೇ~।\\
ಯಾವತ್ತ್ವಾಂ ಪೂಜಯಿಷ್ಯಾಮಿ ತಾವತ್ತ್ವಂ ಸುಸ್ಥಿರಾ ಭವ ॥}

ಸುಭಗೇ ನಮಃ~। ಆಂ ಸೋಹಂ~। ಆಂ ಹ್ರೀಂ ಕ್ರೋಂ ಯರಲವಶಷಸಹೋಂ ಶ್ರೀಲಲಿತಾಮಹಾತ್ರಿಪುರಸುಂದರ್ಯಾಃ ಪ್ರಾಣಾ ಇಹ ಪ್ರಾಣಾಃ~।\\ ಶ್ರೀಲಲಿತಾಮಹಾತ್ರಿಪುರಸುಂದರ್ಯಾಃ ಜೀವ ಇಹ ಸ್ಥಿತಃ~।\\ ಶ್ರೀಲಲಿತಾಮಹಾತ್ರಿಪುರಸುಂದರ್ಯಾಃ ಸರ್ವೇಂದ್ರಿಯಾಣಿ~।\\ಶ್ರೀಲಲಿತಾಮಹಾತ್ರಿಪುರಸುಂದರ್ಯಾಃ ವಾಙ್ಮನಸ್ತ್ವಕ್ಚಕ್ಷುಃ ಶ್ರೋತ್ರ\\ಜಿಹ್ವಾಘ್ರಾಣಪ್ರಾಣಾ ಇಹೈವಾಗತ್ಯ ಸುಖಂ ಚಿರಂ ತಿಷ್ಠಂತು ಸ್ವಾಹಾ ॥\\ಆವಾಹಿತಾ ಭವ~। ಸಂಸ್ಥಾಪಿತಾ ಭವ~। ಸನ್ನಿಹಿತಾ ಭವ~। ಸನ್ನಿರುದ್ಧಾ ಭವ~। ಸಮ್ಮುಖಾ ಭವ~। ಅವಗುಂಠಿತಾ ಭವ~। ವ್ಯಾಪ್ತಾ ಭವ~। ಸುಪ್ರಸನ್ನಾ ಭವ~।\\ ವರದಾ ಭವ ॥

\dhyana{ಸುಖಸ್ವರೂ-ಪಂ ನಿಗಮೈಕವೇದ್ಯಂ ಉಮಾಸಹಾಯಂ ರವಿಚಂದ್ರನೇತ್ರಮ್ ।\\
ಕಾಲಾಗ್ನಿಫಾಲಂ ಭಸಿತಾಂಗಭೂಷಂ ಧ್ಯಾಯಾಮಿ ಹಾಲಾಹಲನೀಲಕಂಠಮ್ ॥}

ಆವಾಹಯಾಮಿ ದೇವೇಶಂ ನಂದಿ ಭೃಂಗ್ಯಾದಿಸೇವಿತಮ್ ।\\
ಸಪರ್ಯಾರ್ಥಂ ಸುಖಾತ್ಮಾನಂ ಶಂಕರಂ ಗಿರಿಜಾಪತಿಮ್ ॥ಆವಾಹನಂ॥

ಸರ್ವಾಂತರ್ಯಾಮಿಣಂ ದೇವಂ ಸರ್ವಬೀಜಮಜಂ ಶುಭಂ ।\\
ಸ್ವಾತ್ಮನಿ ಸ್ಥಾಪಯೇ ಶುದ್ಧಮಾಸನಂ ಕಲ್ಪಯಾಮ್ಯಹಂ ॥ಆಸನಂ॥

ಯದ್ಭಕ್ತಿಲೇಶ ಸಂಪರ್ಕಾತ್ಪರಮಾನಂದ ಸಂಭವಃ ।\\
ತಸ್ಮೈ ತೇ ಚರಣಾಬ್ಜಾಯ ಪಾದ್ಯಂ ಶುದ್ಧಾಯ ಕಲ್ಪಯೇ ॥ಪಾದ್ಯಂ॥

ತಾಪತ್ರಯಹರಂ ದಿವ್ಯಂ ಪರಮಾನಂದ ಲಕ್ಷಣಂ ।\\
ತಾಪತ್ರಯ ವಿನಿರ್ಮುಕ್ತಂ ತವಾರ್ಘ್ಯಂ ಕಲ್ಪಯಾಮ್ಯಹಂ॥ಅರ್ಘ್ಯಂ॥

ವೇದಾನಾಮಪ್ಯವೇದ್ಯಾಯ ದೇವಾನಾಂ ದೇವತಾತ್ಮನೇ ।\\
ಆಚಮ್ಯಂ ಕಲ್ಪಯಾಮೀಶ ಶುದ್ಧಾನಾಂ ಶುದ್ಧಿಹೇತವೇ ॥ಆಚಮನೀಯಂ॥

ಸರ್ವಕಾಲುಷ್ಯಹೀನಾಯ ಪರಿಪೂರ್ಣ ಸುಖಾತ್ಮನೇ ।\\
ಮಧುಪರ್ಕಮಿಮಂ ದೇವ ಕಲ್ಪಯಾಮಿ ಪ್ರಸೀದ ಮೇ ॥ಮಧುಪರ್ಕಃ॥

ಉಚ್ಛಿಷ್ಟೋಪ್ಯಶುಚಿರ್ವಾಪಿ ಯಸ್ಯ ಸ್ಮರಣ ಮಾತ್ರತಃ ।\\
ಶುದ್ಧಿಮಾಪ್ನೋತಿ ತಸ್ಮೈ ತೇ ಪುನರಾಚಮನೀಯಕಂ  ॥ಆಚಮನೀಯಂ॥

ಸ್ನೇಹಂ ಗೃಹಾಣ ಸ್ನೇಹೇನ ಲೋಕನಾಥ ಮಹಾಶಯ ।\\
ಸರ್ವಲೋಕೇಷು ಶುದ್ಧಾತ್ಮನ್ ದದಾಮಿ ಸ್ನೇಹಮುತ್ತಮಮ್ ॥ತೈಲಾಭ್ಯಂಗಃ॥

ಗಂಗಾ ಸರಸ್ವತೀ ರೇವಾ ಪಯೋಷ್ಣೀ ಯಮುನಾಜಲೈಃ~।\\
ಸ್ನಪಯಾಮಿ ಮಹಾದೇವ ತಥಾ ಶಾಂತಂ ಕುರುಷ್ವ ಮಾಂ॥ಸ್ನಾನಮ್॥

ದಧಿ ಕ್ಷೀರ ಘೃತೈರ್ಯುಕ್ತಂ ಶರ್ಕರಾ ಮಧುಮಿಶ್ರಿತಂ ।\\
ಪಂಚಾಮೃತಂ ಗೃಹಾಣೇಶ ಕೃಪಯಾ ಪರಮೇಶ್ವರ  ॥ಪಂಚಾಮೃತಸ್ನಾನಂ॥

ಪರಮಾನಂದಬೋಧಾಬ್ಧಿ ನಿಮಗ್ನ ನಿಜಮೂರ್ತಯೇ ।\\
ಸಾಂಗೋಪಾಂಗಮಿದಂ ಸ್ನಾನಂ ಕಲ್ಪಯಾಮಿ ಜಗತ್ಪತೇ ॥ಶುದ್ಧೋದಕಸ್ನಾನಂ॥

ಕರ್ಪೂರೋಶೀರಸುರಭಿ ಶೀತಲಂ ನಿರ್ಮಲಂ ಜಲಂ ।\\
ಗಂಗಾಯಾಸ್ತು ಸಮಾನೀತಂ ಗೃಹಾಣಾಚಮನೀಯಕಮ್ ॥

ಯಮಾಶ್ರಿತ್ಯ ಮಹಾಮಾಯಾ ಜಗತ್ಸಂಮೋಹಿನೀ ಸದಾ ।\\
ತಸ್ಮೈ ತೇ ಪರಮೇಶಾಯ ಕಲ್ಪಯಾಮ್ಯುತ್ತರೀಯಕಂ ॥ಉತ್ತರೀಯಂ॥

ಮಾಯಾ ಚಿತ್ರಪಟಚ್ಛನ್ನ ನಿಜಗುಹ್ಯೋರು ತೇಜಸೇ ।\\
ನಿರಾವರಣವಿಜ್ಞಾನ ವಾಸಸ್ತೇ ಕಲ್ಪಯಾಮ್ಯಹಂ ॥ವಸ್ತ್ರಂ॥

ಯಸ್ಯ ಶಕ್ತಿತ್ರಯೇಣೇದಂ ಸಂಪ್ರೋತಮಖಿಲಂ ಜಗತ್ ।\\
ಯಜ್ಞಸೂತ್ರಾಯ ತಸ್ಮೈ ತೇ ಯಜ್ಞಸೂತ್ರಂ ಪ್ರಕಲ್ಪಯೇ ॥ಉಪವೀತಮ್॥

ವಿಭೂತಿಭೂಷಣ ಸ್ವಾಮಿನ್ ವಿಭೂತಿಭಿರುಪಾಸಿತ ।\\
ವಿಭೂತಿಂ ತೇ ಪ್ರಯಚ್ಛಾಮಿ ಭಗವನ್ ವರದೋ ಭವ ॥ಭಸ್ಮ॥

ಸ್ವಭಾವ ಸುಂದರಾಂಗಾಯ ನಾನಾ ಶಕ್ತ್ಯಾಶ್ರಯಾಯ ತೇ ।\\
ಭೂಷಣಾನಿ ವಿಚಿತ್ರಾಣಿ ಕಲ್ಪಯಾಮ್ಯಮರಾರ್ಚಿತ ॥ಆಭರಣಂ॥

ಪರಮಾನಂದಸೌಭಾಗ್ಯ ಪರಿಪೂರ್ಣ ದಿಗಂತರ ।\\
ಗೃಹಾಣ ಪರಮಂ ಗಂಧಂ ಕೃಪಯಾ ಪರಮೇಶ್ವರ ॥ಗಂಧಃ॥

ಅಖಂಡತಂಡುಲಾನ್ ಮೃಷ್ಟಾನ್ ಹರಿದ್ರಾ ಕುಂಕುಮಾಂಕಿತಾನ್ ।\\
ಅಕ್ಷತಾಂಶ್ಚ ಮಯಾನೀತಾನ್ ಗೃಹಾಣ ಜಗದೀಶ್ವರ ॥ಅಕ್ಷತಾಃ॥

ತುರೀಯವನಸಂಭೂತಂ ನಾನಾ ಗಣಮನೋಹರಂ।\\
ಆನಂದ ಸೌರಭಂ ಪುಷ್ಪಂ ಗೃಹ್ಯತಾಮಿದಮುತ್ತಮಮ್ ॥ಪುಷ್ಪಾಣಿ॥%ತುಲಸೀಬಿಲ್ವಪತ್ರಾಬ್ಜ ಜಾಜೀ ರಕ್ತೋತ್ಪಲಾಸಿತೈಃ ।\\ ಚಂಪಕಾ ಮಲ್ಲಿಕಾಭಿಶ್ಚ ಯೂಥಿಕಾ ಕರವೀರಕೈಃ ॥\\ಸುಗಂಧಯುಕ್ತ ಕುಸುಮೈರ್ವನಮಾಲಾದಿಕಾನ್ ಯಜೇತ್ ।\\ಸ್ವಸ್ವಮೂಲಂ ಸಮುಚ್ಚಾರ್ಯ ಕುಸುಮಾಂಜಲಿಮರ್ಪಯೇತ್ ।\\ಪುಷ್ಪಾಂಜಲಿಂ ವಿಧಾಯೇತ್ಥಂ ಕುರ್ಯಾದಾವರಣಾರ್ಚನಂ॥
\newpage
\section{ಶಿವ ಅಂಗಪೂಜಾ ॥}
ಓಂ ಪಾಪನಾಶನಾಯ ನಮಃ~। ಪಾದೌ ಪೂಜಯಾಮಿ॥\\
ಓಂ ಗುರವೇ ನಮಃ~। ಗುಲ್ಫೌ ಪೂಜಯಾಮಿ॥\\
ಓಂ ಜ್ಞಾನಪ್ರದಾಯ ನಮಃ~। ಜಂಘೇ ಪೂಜಯಾಮಿ॥\\
ಓಂ ಜಾಹ್ನವೀಪತಯೇ ನಮಃ~। ಜಾನುನೀ ಪೂಜಯಾಮಿ॥\\
ಓಂ ಉತ್ತಮೋತ್ತಮಾಯ ನಮಃ~। ಊರೂ ಪೂಜಯಾಮಿ॥\\
ಓಂ ಕಂದರ್ಪನಾಶಾಯ ನಮಃ~। ಕಟಿಂ ಪೂಜಯಾಮಿ॥\\
ಓಂ ಗುಹೇಶ್ವರಾಯ ನಮಃ~। ಗುಹ್ಯಂ ಪೂಜಯಾಮಿ॥\\
ಓಂ ನಂದಿಸೇವ್ಯಾಯ ನಮಃ~। ನಾಭಿಂ ಪೂಜಯಾಮಿ॥\\
ಓಂ ಸ್ಕಂದಗುರವೇ ನಮಃ~। ಸ್ಕಂಧೌ ಪೂಜಯಾಮಿ॥\\
ಓಂ ಹಿರಣ್ಯಬಾಹವೇ ನಮಃ~। ಬಾಹೂನ್ ಪೂಜಯಾಮಿ॥\\
ಓಂ ಹರಾಯ ನಮಃ~। ಹಸ್ತಾನ್ ಪೂಜಯಾಮಿ॥\\
ಓಂ ನೀಲಕಂಠಾಯ ನಮಃ~। ಕಂಠಂ ಪೂಜಯಾಮಿ॥\\
ಓಂ ವೇದಮೂರ್ತಯೇ ನಮಃ~। ಮುಖಂ ಪೂಜಯಾಮಿ॥\\
ಓಂ ನಾಗಹಾರಾಯ ನಮಃ~। ನಾಸಿಕಾಂ ಪೂಜಯಾಮಿ॥\\
ಓಂ ತ್ರಿಣೇತ್ರಾಯ ನಮಃ~। ನೇತ್ರಾಣಿ ಪೂಜಯಾಮಿ॥\\
ಓಂ ಭಸಿತಾಭಾಸಾಯ ನಮಃ~। ಲಲಾಟಂ ಪೂಜಯಾಮಿ॥\\
ಓಂ ಇಂದುಮೌಲಯೇ ನಮಃ~। ಮೌಲಿಂ ಪೂಜಯಾಮಿ॥\\
ಓಂ ಶರ್ವಾಯ ನಮಃ~। ಶಿರಃ ಪೂಜಯಾಮಿ॥\\
ಓಂ ಸರ್ವಾತ್ಮನೇ ನಮಃ~। ಸರ್ವಾಂಗಂ ಪೂಜಯಾಮಿ॥
\newpage
\section{ಪತ್ರ ಪೂಜಾ ॥}
ಓಂ ಉಮಾಪತಯೇ ನಮಃ~। ಬಿಲ್ವಪತ್ರಂ ಸಮರ್ಪಯಾಮಿ॥\\
ಓಂ ಜಗದ್ಗುರವೇ ನಮಃ~। ತುಲಸೀಪತ್ರಂ ಸಮರ್ಪಯಾಮಿ॥\\
ಓಂ ಆನಂದಾಯ ನಮಃ~। ಅರ್ಕಪತ್ರಂ ಸಮರ್ಪಯಾಮಿ॥\\
ಓಂ ಸರ್ವಬಂಧವಿಮೋಚನಾಯ ನಮಃ~। ಜಂಬೀರಪತ್ರಂ ಸಮರ್ಪಯಾಮಿ॥\\
ಓಂ ಲೋಕನಾಥಾಯ ನಮಃ~। ನಿರ್ಗುಂಡೀಪತ್ರಂ ಸಮರ್ಪಯಾಮಿ॥\\
ಓಂ ಜಗತ್ಕಾರಣಾಯ ನಮಃ~। ದೂರ್ವಾಪತ್ರಂ ಸಮರ್ಪಯಾಮಿ॥\\
ಓಂ ನಾಗಭೂಷಣಾಯ ನಮಃ~। ಕುಶಪತ್ರಂ ಸಮರ್ಪಯಾಮಿ॥\\
ಓಂ ಮೃಗಧರಾಯ ನಮಃ~। ಮರುಗಪತ್ರಂ ಸಮರ್ಪಯಾಮಿ॥\\
ಓಂ ಪಶುಪತಯೇ ನಮಃ~। ಕಾಮಕಸ್ತೂರಿಕಾಪತ್ರಂ ಸಮರ್ಪಯಾಮಿ॥\\
ಓಂ ಮುಕುಂದಪ್ರಿಯಾಯ ನಮಃ~। ಗಿರಿಕರ್ಣಿಕಾಪತ್ರಂ ಸಮರ್ಪಯಾಮಿ॥\\
ಓಂ ತ್ರ್ಯಂಬಕಾಯ ನಮಃ~। ಮಾಚೀಪತ್ರಂ ಸಮರ್ಪಯಾಮಿ॥\\
ಓಂ ಭಕ್ತಜನಪ್ರಿಯಾಯ ನಮಃ~। ಧಾತ್ರೀಪತ್ರಂ ಸಮರ್ಪಯಾಮಿ॥\\
ಓಂ ವರದಾಯ ನಮಃ~। ವಿಷ್ಣುಕ್ರಾಂತಿಪತ್ರಂ ಸಮರ್ಪಯಾಮಿ॥\\
ಓಂ ಶಿವಾಯ ನಮಃ~। ದ್ರೋಣಪತ್ರಂ ಸಮರ್ಪಯಾಮಿ॥\\
ಓಂ ಶಂಕರಾಯ ನಮಃ~। ಧತ್ತೂರಪತ್ರಂ ಸಮರ್ಪಯಾಮಿ॥\\
ಓಂ ಶಮಪ್ರಾಪ್ತಾಯ ನಮಃ~। ಶಮೀಪತ್ರಂ ಸಮರ್ಪಯಾಮಿ॥\\
ಓಂ ಸಾಂಬಶಿವಾಯ ನಮಃ~। ಸೇವಂತಿಕಾಪತ್ರಂ ಸಮರ್ಪಯಾಮಿ॥\\
ಓಂ ಚರ್ಮವಾಸಸೇ ನಮಃ~। ಚಂಪಕಪತ್ರಂ ಸಮರ್ಪಯಾಮಿ॥\\
ಓಂ ಬ್ರಾಹ್ಮಣಪ್ರಿಯಾಯ ನಮಃ~। ಕರವೀರಪತ್ರಂ ಸಮರ್ಪಯಾಮಿ॥\\
ಓಂ ಗಂಗಾಧರಾಯ ನಮಃ~। ಅಶೋಕಪತ್ರಂ ಸಮರ್ಪಯಾಮಿ॥\\
ಓಂ ಪುಣ್ಯಮೂರ್ತಯೇ ನಮಃ~। ಪುನ್ನಾಗಪತ್ರಂ ಸಮರ್ಪಯಾಮಿ॥\\
ಓಂ ಉಮಾಮಹೇಶ್ವರಾಯ ನಮಃ~। ಸರ್ವಾಣಿ ಪತ್ರಾಣಿ ಸಮರ್ಪಯಾಮಿ॥
\section{ಪುಷ್ಪಪೂಜಾ}
ಓಂ ರುದ್ರಾಯ ನಮಃ~। ದ್ರೋಣಪುಷ್ಪಂ ಸಮರ್ಪಯಾಮಿ॥\\
ಓಂ ಪಶುಪತಯೇ ನಮಃ~। ಧತ್ತೂರಪುಷ್ಪಂ ಸಮರ್ಪಯಾಮಿ॥\\
ಓಂ ಸ್ಥಾಣವೇ ನಮಃ~। ಬೃಹತೀಪುಷ್ಪಂ ಸಮರ್ಪಯಾಮಿ॥\\
ಓಂ ನೀಲಕಂಠಾಯ ನಮಃ~। ಅರ್ಕಪುಷ್ಪಂ ಸಮರ್ಪಯಾಮಿ॥\\
ಓಂ ಉಮಾಪತಯೇ ನಮಃ~। ಬಕುಲಪುಷ್ಪಂ ಸಮರ್ಪಯಾಮಿ॥\\
ಓಂ ಕಾಲಕಾಲಾಯ ನಮಃ~। ಜಾತೀಪುಷ್ಪಂ ಸಮರ್ಪಯಾಮಿ॥\\
ಓಂ ಕಾಲಮೂರ್ತಯೇ ನಮಃ~। ಕರವೀರಪುಷ್ಪಂ ಸಮರ್ಪಯಾಮಿ॥\\
ಓಂ ದೇವದೇವಾಯ ನಮಃ~। ಪಂಕಜಪುಷ್ಪಂ ಸಮರ್ಪಯಾಮಿ॥\\
ಓಂ ವಿಶ್ವಪ್ರಿಯಾಯ ನಮಃ~। ಪುನ್ನಾಗಪುಷ್ಪಂ ಸಮರ್ಪಯಾಮಿ॥\\
ಓಂ ವೃಷಭಧ್ವಜಾಯ ನಮಃ~। ವೈಜಯಂತಿಕಾಪುಷ್ಪಂ ಸಮರ್ಪಯಾಮಿ॥\\
ಓಂ ಸದಾಶಿವಾಯ ನಮಃ~। ಗಿರಿಕರ್ಣಿಕಾಪುಷ್ಪಂ ಸಮರ್ಪಯಾಮಿ॥\\
ಓಂ ಶೂಲಿನೇ ನಮಃ~। ಚಂಪಕಪುಷ್ಪಂ ಸಮರ್ಪಯಾಮಿ॥\\
ಓಂ ಸುರೇಶಾಯ ನಮಃ~। ಸೇವಂತಿಕಾಪುಷ್ಪಂ ಸಮರ್ಪಯಾಮಿ॥\\
ಓಂ ನಿರಹಂಕಾರಾಯ ನಮಃ~। ಮಲ್ಲಿಕಾಪುಷ್ಪಂ ಸಮರ್ಪಯಾಮಿ॥\\
ಓಂ ಸತ್ಯವ್ರತಾಯ ನಮಃ~। ಜಪಾಪುಷ್ಪಂ ಸಮರ್ಪಯಾಮಿ॥\\
ಓಂ ಉಮಾಮಹೇಶ್ವರಾಯ ನಮಃ~। ಸರ್ವಾಣಿ ಪುಷ್ಪಾಣಿ ಸಮರ್ಪಯಾಮಿ॥
\section{ಲಲಿತಾ ಅಂಗಪೂಜಾ}
೪ ಶ್ರೀಮಾತ್ರೇ ನಮಃ~। ಪಾದೌ ಪೂಜಯಾಮಿ~।\\
೪ ಭಾವನಾಯೈ ನಮಃ~। ಗುಲ್ಫೌ ಪೂಜಯಾಮಿ~।\\
೪ ಭಾವನಾಗಮ್ಯಾಯೈ ನಮಃ~। ಜಂಘೇ ಪೂಜಯಾಮಿ~।\\
೪ ಭವಾರಣ್ಯಕುಠಾರಿಕಾಯೈ ನಮಃ~। ಜಾನುನೀ ಪೂಜಯಾಮಿ~।\\
೪ ಭದ್ರಪ್ರಿಯಾಯೈ ನಮಃ~। ಊರೂ ಪೂಜಯಾಮಿ~।\\
೪ ಭದ್ರಮೂರ್ತ್ಯೈ ನಮಃ~। ಕಟಿಂ ಪೂಜಯಾಮಿ~।\\
೪ ಭಕ್ತಸೌಭಾಗ್ಯದಾಯಿನ್ಯೈ ನಮಃ~। ನಾಭಿಂ ಪೂಜಯಾಮಿ~।\\
೪ ಭಕ್ತಿಪ್ರಿಯಾಯೈ ನಮಃ~। ಉದರಂ ಪೂಜಯಾಮಿ~।\\
೪ ಭಕ್ತಿಗಮ್ಯಾಯೈ ನಮಃ~। ಸ್ತನೌ ಪೂಜಯಾಮಿ~।\\
೪ ಭಕ್ತಿವಶ್ಯಾಯೈ ನಮಃ~। ವಕ್ಷಸ್ಥಲಂ ಪೂಜಯಾಮಿ~।\\
೪ ಕಲ್ಯಾಣ್ಯೈ ನಮಃ~। ಬಾಹೂನ್ ಪೂಜಯಾಮಿ~।\\
೪ ಚಕ್ರಿಣ್ಯೈ ನಮಃ~। ಹಸ್ತಾನ್ ಪೂಜಯಾಮಿ~।\\
೪ ವಿಶ್ವಾತೀತಾಯೈ ನಮಃ~। ಕಂಠಂ ಪೂಜಯಾಮಿ~।\\
೪ ತ್ರಿಪುರಾಯೈ ನಮಃ~। ಜಿಹ್ವಾಂ ಪೂಜಯಾಮಿ~।\\
೪ ವಿರಾಗಿಣ್ಯೈ ನಮಃ~। ಮುಖಂ ಪೂಜಯಾಮಿ~।\\
೪ ಮಹೀಯಸ್ಯೈ ನಮಃ~। ನೇತ್ರೇ ಪೂಜಯಾಮಿ~।\\
೪ ಮನುವಿದ್ಯಾಯೈ ನಮಃ~। ಕರ್ಣೌ ಪೂಜಯಾಮಿ~।\\
೪ ಶಿವಾಯೈ ನಮಃ~। ಲಲಾಟಂ ಪೂಜಯಾಮಿ~।\\
೪ ಶಿವಶಕ್ತ್ಯೈ ನಮಃ~। ಶಿರಃ ಪೂಜಯಾಮಿ~।\\
೪ ಲಲಿತಾಂಬಿಕಾಯೈ ನಮಃ~। ಸರ್ವಾಂಗಂ ಪೂಜಯಾಮಿ~।
\section{ಪತ್ರಪೂಜಾ}
೪ ಕಲ್ಯಾಣ್ಯೈ ನಮಃ~। ಮಾಚೀಪತ್ರಂ ಸಮರ್ಪಯಾಮಿ~।\\
೪ ಕಮಲಾಕ್ಷ್ಯೈ ನಮಃ~। ಸೇವಂತಿಕಾಪತ್ರಂ ಸಮರ್ಪಯಾಮಿ~।\\
೪ ಏಕಾರರೂಪಾಯೈ ನಮಃ~। ಬಿಲ್ವಪತ್ರಂ ಸಮರ್ಪಯಾಮಿ~।\\
೪ ಏಕಭೋಗಾಯೈ ನಮಃ~। ತುಲಸೀಪತ್ರಂ ಸಮರ್ಪಯಾಮಿ~।\\
೪ ಏಕರಸಾಯೈ ನಮಃ~। ಕಸ್ತೂರಿಕಾಪತ್ರಂ ಸಮರ್ಪಯಾಮಿ~।\\
೪ ಈಶಿತ್ರ್ಯೈ ನಮಃ~। ಮರುವಕಪತ್ರಂ ಸಮರ್ಪಯಾಮಿ~।\\
೪ ಈಶಶಕ್ತ್ಯೈ ನಮಃ~। ಗಿರಿಕರ್ಣಿಕಾಪತ್ರಂ ಸಮರ್ಪಯಾಮಿ~।\\
೪ ಹ್ರೀಂಮತ್ಯೈ ನಮಃ~। ಕರವೀರಪತ್ರಂ ಸಮರ್ಪಯಾಮಿ~।\\
೪ ಕಾಮದಾಯೈ ನಮಃ~। ದಾಡಿಮೀಪತ್ರಂ ಸಮರ್ಪಯಾಮಿ~।\\
೪ ವಿಶ್ವತೋಮುಖ್ಯೈ ನಮಃ~। ವಿಷ್ಣುಕ್ರಾಂತಿಪತ್ರಂ ಸಮರ್ಪಯಾಮಿ~।\\
೪ ಭುವನೇಶ್ವರ್ಯೈ ನಮಃ~। ಜಂಬೀರಪತ್ರಂ ಸಮರ್ಪಯಾಮಿ~।\\
೪ ಶ್ರಿಯೈ ನಮಃ~। ಪದ್ಮಪತ್ರಂ ಸಮರ್ಪಯಾಮಿ~।\\
೪ ರಮಾಯೈ ನಮಃ~। ಚಂಪಕಪತ್ರಂ ಸಮರ್ಪಯಾಮಿ~।\\
೪ ಪುಷ್ಟ್ಯೈ ನಮಃ~। ಅಪಾಮಾರ್ಗಪತ್ರಂ ಸಮರ್ಪಯಾಮಿ~।\\
೪ ಗೌರ್ಯೈ ನಮಃ~। ಮಾಲತೀಪತ್ರಂ ಸಮರ್ಪಯಾಮಿ~।\\
೪ ಸರಸ್ವತ್ಯೈ ನಮಃ~। ಬದರೀಪತ್ರಂ ಸಮರ್ಪಯಾಮಿ~।\\
೪ ದುರ್ಗಾಯೈ ನಮಃ~। ಪಾರಿಜಾತಪತ್ರಂ ಸಮರ್ಪಯಾಮಿ~।\\
೪ ರುದ್ರಾಣ್ಯೈ ನಮಃ~। ಶಮೀಪತ್ರಂ ಸಮರ್ಪಯಾಮಿ~।\\
೪ ಮಾತೃಕಾಯೈ ನಮಃ~। ಧತ್ತೂರಪತ್ರಂ ಸಮರ್ಪಯಾಮಿ~।\\
೪ ಜಗದಂಬಾಯೈ ನಮಃ~। ನಿರ್ಗುಂಡೀಪತ್ರಂ ಸಮರ್ಪಯಾಮಿ~।\\
೪ ಲಲಿತಾಂಬಿಕಾಯೈ ನಮಃ~। ಕದಂಬಪತ್ರಂ ಸಮರ್ಪಯಾಮಿ~।\\
೪ ಶ್ರೀಚಕ್ರವಾಸಿನ್ಯೈ ನಮಃ~। ಸರ್ವಾಣಿ ಪತ್ರಾಣಿ ಸಮರ್ಪಯಾಮಿ~।
\section{ಪುಷ್ಪಪೂಜಾ}
೪ ಕಾಲಶಕ್ತ್ಯೈ ನಮಃ~। ಜಾಜೀಪುಷ್ಪಂ ಸಮರ್ಪಯಾಮಿ~।\\
೪ ಬ್ರಹ್ಮರೂಪಿಣ್ಯೈ ನಮಃ~। ಕೇತಕೀಪುಷ್ಪಂ ಸಮರ್ಪಯಾಮಿ~।\\
೪ ವಿಷ್ಣುರೂಪಿಣ್ಯೈ ನಮಃ~। ಪುನ್ನಾಗಪುಷ್ಪಂ ಸಮರ್ಪಯಾಮಿ~।\\
೪ ರುದ್ರರೂಪಿಣ್ಯೈ ನಮಃ~। ಕಮಲಪುಷ್ಪಂ ಸಮರ್ಪಯಾಮಿ~।\\
೪ ಸದಾಶಿವರೂಪಿಣ್ಯೈ ನಮಃ~। ಮಲ್ಲಿಕಾಪುಷ್ಪಂ ಸಮರ್ಪಯಾಮಿ~।\\
೪ ಚಿತ್ಕಲಾಯೈ ನಮಃ~। ಸೇವಂತಿಕಾಪುಷ್ಪಂ ಸಮರ್ಪಯಾಮಿ~।\\
೪ ವೇದಮಾತ್ರೇ ನಮಃ~। ಬಕುಲಪುಷ್ಪಂ ಸಮರ್ಪಯಾಮಿ~।\\
೪ ದುರ್ಗಾಯೈ ನಮಃ~। ಜಪಾಪುಷ್ಪಂ ಸಮರ್ಪಯಾಮಿ~।\\
೪ ಮಹಾಲಕ್ಷ್ಮ್ಯೈ ನಮಃ~। ಪಾರಿಜಾತಪುಷ್ಪಂ ಸಮರ್ಪಯಾಮಿ~।\\
೪ ಮಹಾಸರಸ್ವತ್ಯೈ ನಮಃ~। ಮಾಲತೀಪುಷ್ಪಂ ಸಮರ್ಪಯಾಮಿ~।\\
೪ ಚಂಪಕವಾಹಿನ್ಯೈ ನಮಃ~। ಚಂಪಕಪುಷ್ಪಂ ಸಮರ್ಪಯಾಮಿ~।\\
೪ ಕಾಮೇಶ್ವರ್ಯೈ ನಮಃ~। ಕದಂಬಪುಷ್ಪಂ ಸಮರ್ಪಯಾಮಿ~।\\
೪ ಕಾತ್ಯಾಯನ್ಯೈ ನಮಃ~। ಕರವೀರಪುಷ್ಪಂ ಸಮರ್ಪಯಾಮಿ~।\\
೪ ಸರ್ವೇಶ್ವರ್ಯೈ ನಮಃ~। ಅಶೋಕಪುಷ್ಪಂ ಸಮರ್ಪಯಾಮಿ~।\\
೪ ತ್ರಿಪುರಸುಂದರ್ಯೈ ನಮಃ~। ಗಿರಿಕರ್ಣಿಕಾಪುಷ್ಪಂ ಸಮರ್ಪಯಾಮಿ~।\\
೪ ರಾಜರಾಜೇಶ್ವರ್ಯೈ ನಮಃ~। ಬದರೀಪುಷ್ಪಂ ಸಮರ್ಪಯಾಮಿ~।\\
೪ ಶ್ರೀಚಕ್ರವಾಸಿನ್ಯೈ ನಮಃ~। ಸರ್ವಾಣಿ ಪುಷ್ಪಾಣಿ ಸಮರ್ಪಯಾಮಿ~।

\section{ ಶ್ರೀಶಂಕರಭಗವತ್ಪಾದಾಷ್ಟೋತ್ತರಶತನಾಮಸ್ತೋತ್ರಂ}
\dhyana{ಚಿನ್ಮುದ್ರಾಂ ದಕ್ಷಹಸ್ತೇ ಪ್ರಣತಜನಮಹಾಬೋಧದಾತ್ರೀಂ ದಧಾನಂ\\
ವಾಮೇ ನಮ್ರೇಷ್ಟದಾನಪ್ರಕಟನಚತುರಂ ಚಿಹ್ನಮಪ್ಯಾದಧಾನಮ್।\\
ಕಾರುಣ್ಯಾಪಾರವಾರ್ಧಿಂ ಯತಿವರವಪುಷಂ ಶಂಕರಂ ಶಂಕರಾಂಶಂ\\
ಚಂದ್ರಾಹಂಕಾರಹುಂಕೃತ್ ಸ್ಮಿತಲಸಿತಮುಖಂ ಭಾವಯಾಮ್ಯಂತರಂಗೇ ॥}

ಶ್ರೀಶಂಕರಾಚಾರ್ಯವರ್ಯೋ  ಬ್ರಹ್ಮಾನಂದಪ್ರದಾಯಕಃ~।\\
ಅಜ್ಞಾನತಿಮಿರಾದಿತ್ಯಃ  ಸುಜ್ಞಾನಾಂಬುಧಿಚಂದ್ರಮಾಃ~॥೧॥

ವರ್ಣಾಶ್ರಮಪ್ರತಿಷ್ಠಾತಾ  ಶ್ರೀಮಾನ್ಮುಕ್ತಿಪ್ರದಾಯಕಃ ।\\
ಶಿಷ್ಯೋಪದೇಶನಿರತೋ ಭಕ್ತಾಭೀಷ್ಟಪ್ರದಾಯಕಃ ॥೨॥

ಸೂಕ್ಷ್ಮತತ್ತ್ವರಹಸ್ಯಜ್ಞಃ ಕಾರ್ಯಾಕಾರ್ಯಪ್ರಬೋಧಕಃ ।\\
ಜ್ಞಾನಮುದ್ರಾಂಚಿತಕರಃ  ಶಿಷ್ಯಹೃತ್ತಾಪಹಾರಕಃ ॥೩॥

ಪರಿವ್ರಾಜಾಶ್ರಮೋದ್ಧರ್ತಾ  ಸರ್ವತಂತ್ರಸ್ವತಂತ್ರಧೀಃ।\\
ಅದ್ವೈತಸ್ಥಾಪನಾಚಾರ್ಯಃ ಸಾಕ್ಷಾಚ್ಛಂಕರರೂಪಧೃತ್~॥೪॥
 
ಷಣ್ಮತಸ್ಥಾಪನಾಚಾರ್ಯಃ  ತ್ರಯೀಮಾರ್ಗಪ್ರಕಾಶಕಃ~।\\
ವೇದವೇದಾಂತತತ್ತ್ವಜ್ಞೋ ದುರ್ವಾದಿಮತಖಂಡನಃ ॥೫॥

ವೈರಾಗ್ಯನಿರತಃ ಶಾಂತಃ ಸಂಸಾರಾರ್ಣವತಾರಕಃ~।\\
ಪ್ರಸನ್ನವದನಾಂಭೋಜಃ ಪರಮಾರ್ಥಪ್ರಕಾಶಕಃ~॥೬॥

ಪುರಾಣಸ್ಮೃತಿಸಾರಜ್ಞೋ ನಿತ್ಯತೃಪ್ತೋ ಮಹಾನ್ ಶುಚಿಃ।\\
ನಿತ್ಯಾನಂದೋ ನಿರಾತಂಕೋ ನಿಃಸಂಗೋ ನಿರ್ಮಲಾತ್ಮಕಃ ॥೭॥

ನಿರ್ಮಮೋ ನಿರಹಂಕಾರೋ ವಿಶ್ವವಂದ್ಯಪದಾಂಬುಜಃ ।\\
ಸತ್ತ್ವಪ್ರಧಾನಃ ಸದ್ಭಾವಃ ಸಂಖ್ಯಾತೀತಗುಣೋಜ್ವಲಃ ॥೮॥

ಅನಘಃ ಸಾರಹೃದಯಃ ಸುಧೀಃ ಸಾರಸ್ವತಪ್ರದಃ।\\
ಸತ್ಯಾತ್ಮಾ ಪುಣ್ಯಶೀಲಶ್ಚ ಸಾಂಖ್ಯಯೋಗವಿಚಕ್ಷಣಃ ॥೯॥

ತಪೋರಾಶಿರ್ಮಹಾತೇಜಾ ಗುಣತ್ರಯವಿಭಾಗವಿತ್~।\\
ಕಲಿಘ್ನಃ ಕಾಲಕರ್ಮಜ್ಞಃ ತಮೋಗುಣನಿವಾರಕಃ॥೧೦॥

ಭಗವಾನ್ ಭಾರತೀಜೇತಾ ಶಾರದಾಹ್ವಾನಪಂಡಿತಃ ।\\
ಧರ್ಮಾಧರ್ಮವಿಭಾಗಜ್ಞೋ ಲಕ್ಷ್ಯಭೇದಪ್ರದರ್ಶಕಃ ॥೧೧॥

ನಾದಬಿಂದುಕಲಾಭಿಜ್ಞೋ ಯೋಗಿಹೃತ್ಪದ್ಮಭಾಸ್ಕರಃ।\\
ಅತೀಂದ್ರಿಯಜ್ಞಾನನಿಧಿಃ ನಿತ್ಯಾನಿತ್ಯವಿವೇಕವಾನ್~॥೧೨॥

ಚಿದಾನಂದಃ ಚಿನ್ಮಯಾತ್ಮಾ ಪರಕಾಯಪ್ರವೇಶಕೃತ್~।\\
ಅಮಾನುಷಚರಿತ್ರಾಢ್ಯಃ ಕ್ಷೇಮದಾಯೀ ಕ್ಷಮಾಕರಃ॥೧೩॥

ಭವ್ಯೋ ಭದ್ರಪ್ರದೋ ಭೂರಿಮಹಿಮಾ ವಿಶ್ವರಂಜಕಃ ।\\
ಸ್ವಪ್ರಕಾಶಃ ಸದಾಧಾರೋ  ವಿಶ್ವಬಂಧುಃ ಶುಭೋದಯಃ ॥೧೪॥

ವಿಶಾಲಕೀರ್ತಿಃ ವಾಗೀಶಃ ಸರ್ವಲೋಕಹಿತೋತ್ಸುಕಃ।\\
ಕೈಲಾಸಯಾತ್ರಾಸಂಪ್ರಾಪ್ತ ಚಂದ್ರಮೌಲಿಪ್ರಪೂಜಕಃ॥೧೫॥
 
ಕಾಂಚ್ಯಾಂ ಶ್ರೀಚಕ್ರರಾಜಾಖ್ಯ ಯಂತ್ರಸ್ಥಾಪನದೀಕ್ಷಿತಃ।\\
 ಶ್ರೀಚಕ್ರಾತ್ಮಕತಾಟಂಕ ತೋಷಿತಾಂಬಾಮನೋರಥಃ॥೧೬॥

 ಶ್ರೀಬ್ರಹ್ಮಸೂತ್ರೋಪನಿಷದ್ಭಾಷ್ಯಾದಿ ಗ್ರಂಥಕಲ್ಪಕಃ।\\
ಚತುರ್ದಿಕ್ಚತುರಾಮ್ನಾಯ ಪ್ರತಿಷ್ಠಾತಾ ಮಹಾಮತಿಃ॥೧೭॥

ದ್ವಿಸಪ್ತತಿಮತೋಚ್ಛೇತ್ತಾ ಸರ್ವದಿಗ್ವಿಜಯಪ್ರಭುಃ~।\\
ಕಾಷಾಯವಸನೋಪೇತೋ ಭಸ್ಮೋದ್ಧೂಲಿತವಿಗ್ರಹಃ॥೧೮॥

ಜ್ಞಾನಾತ್ಮಕೈಕದಂಡಾಢ್ಯಃ ಕಮಂಡಲುಲಸತ್ಕರಃ।\\
ಗುರುಭೂಮಂಡಲಾಚಾರ್ಯೋ ಭಗವತ್ಪಾದಸಂಜ್ಞಕಃ॥೧೯॥

ವ್ಯಾಸಸಂದರ್ಶನಪ್ರೀತಃ ಋಷ್ಯಶೃಂಗಪುರೇಶ್ವರಃ।\\
ಸೌಂದರ್ಯಲಹರೀಮುಖ್ಯ ಬಹುಸ್ತೋತ್ರವಿಧಾಯಕಃ॥೨೦॥

ಚತುಷ್ಷಷ್ಟಿಕಲಾಭಿಜ್ಞೋ ಬ್ರಹ್ಮರಾಕ್ಷಸಮೋಕ್ಷದಃ।\\
 ಶ್ರೀಮನ್ಮಂಡನಮಿಶ್ರಾಖ್ಯ ಸ್ವಯಂಭೂಜಯಸನ್ನುತಃ॥೨೧॥

 ತೋಟಕಾಚಾರ್ಯಸಂಪೂಜ್ಯಃ  ಪದ್ಮಪಾದಾರ್ಚಿತಾಂಘ್ರಿಕಃ।\\
 ಹಸ್ತಾಮಲಕಯೋಗೀಂದ್ರ ಬ್ರಹ್ಮಜ್ಞಾನಪ್ರದಾಯಕಃ॥೨೨॥

 ಸುರೇಶ್ವರಾಖ್ಯಸಚ್ಛಿಷ್ಯ ಸನ್ನ್ಯಾಸಾಶ್ರಮದಾಯಕಃ।\\
 ನೃಸಿಂಹಭಕ್ತಃ  ಸದ್ರತ್ನಗರ್ಭಹೇರಂಬಪೂಜಕಃ ।\\
 ವ್ಯಾಖ್ಯಾಸಿಂಹಾಸನಾಧೀಶೋ ಜಗತ್ಪೂಜ್ಯೋ ಜಗದ್ಗುರುಃ~॥೨೩॥
\authorline{ಇತಿ ಶ್ರೀಶಂಕರಭಗವತ್ಪಾದಾಷ್ಟೋತ್ತರಶತನಾಮಸ್ತೋತ್ರಂ ॥}
\section{ಶ್ರೀದತ್ತಾತ್ರೇಯಾಷ್ಟೋತ್ತರಶತನಾಮಸ್ತೋತ್ರಂ}
ಅಸ್ಯ ದತ್ತಾತ್ರೇಯಾಷ್ಟೋತ್ತರ ಶತನಾಮಸ್ತೋತ್ರ ಮಹಾಮಂತ್ರಸ್ಯ, ಬ್ರಹ್ಮವಿಷ್ಣುಮಹೇಶ್ವರಾ ಋಷಯಃ~। ಶ್ರೀದತ್ತಾತ್ರೇಯೋ ದೇವತಾ~। ಅನುಷ್ಟುಪ್ಛಂದಃ~। ಶ್ರೀದತ್ತಾತ್ರೇಯಪ್ರೀತ್ಯರ್ಥೇ ಜಪೇ ವಿನಿಯೋಗಃ~।\\
\dhyana{ಬಾಲಾರ್ಕಪ್ರಭಮಿಂದ್ರನೀಲಜಟಿಲಂ ಭಸ್ಮಾಂಗರಾಗೋಜ್ಜ್ವಲಂ\\
ಶಾಂತಂ ನಾದವಿಲೀನಚಿತ್ತಪವನಂ ಶಾರ್ದೂಲಚರ್ಮಾಂಬರಂ ।\\
ಬ್ರಹ್ಮಜ್ಞೈಃ ಸನಕಾದಿಭಿಃ ಪರಿವೃತಂ ಸಿದ್ಧೈರ್ಮಹಾಯೋಗಿಭಿಃ\\
ದತ್ತಾತ್ರೇಯಮುಪಾಸ್ಮಹೇ ಹೃದಿ ಮುದಾ ಧ್ಯೇಯಂ ಸದಾ ಯೋಗಿನಾಂ ॥}

ಓಂ ಅನಸೂಯಾಸುತೋ ದತ್ತೋ ಹ್ಯತ್ರಿಪುತ್ರೋ ಮಹಾಮುನಿಃ~।\\
ಯೋಗೀಂದ್ರಃ ಪುಣ್ಯಪುರುಷೋ ದೇವೇಶೋ ಜಗದೀಶ್ವರಃ ॥೧॥

ಪರಮಾತ್ಮಾ ಪರಂ ಬ್ರಹ್ಮ ಸದಾನಂದೋ ಜಗದ್ಗುರುಃ~।\\
ನಿತ್ಯತೃಪ್ತೋ ನಿರ್ವಿಕಾರೋ ನಿರ್ವಿಕಲ್ಪೋ ನಿರಂಜನಃ ॥೨॥

ಗುಣಾತ್ಮಕೋ ಗುಣಾತೀತೋ ಬ್ರಹ್ಮವಿಷ್ಣುಶಿವಾತ್ಮಕಃ~।\\
ನಾನಾರೂಪಧರೋ ನಿತ್ಯಃ ಶಾಂತೋ ದಾಂತಃ ಕೃಪಾನಿಧಿಃ ॥೩॥

ಭಕ್ತಪ್ರಿಯೋ ಭವಹರೋ ಭಗವಾನ್ಭವನಾಶನಃ~।\\
ಆದಿದೇವೋ ಮಹಾದೇವಃ ಸರ್ವೇಶೋ ಭುವನೇಶ್ವರಃ ॥೪॥

ವೇದಾಂತವೇದ್ಯೋ ವರದೋ ವಿಶ್ವರೂಪೋಽವ್ಯಯೋ ಹರಿಃ~।\\
ಸಚ್ಚಿದಾನಂದಃ ಸರ್ವೇಶೋ ಯೋಗೀಶೋ ಭಕ್ತವತ್ಸಲಃ ॥೫॥

ದಿಗಂಬರೋ ದಿವ್ಯಮೂರ್ತಿರ್ದಿವ್ಯಭೂತಿವಿಭೂಷಣಃ~।\\
ಅನಾದಿಃ ಸಿದ್ಧಸುಲಭೋ ಭಕ್ತವಾಂಛಿತದಾಯಕಃ ॥೬॥

ಏಕೋಽನೇಕೋ ಹ್ಯದ್ವಿತೀಯೋ ನಿಗಮಾಗಮಪಂಡಿತಃ~।\\
ಭುಕ್ತಿಮುಕ್ತಿಪ್ರದಾತಾ ಚ ಕಾರ್ತವೀರ್ಯವರಪ್ರದಃ ॥೭॥

ಶಾಶ್ವತಾಂಗೋ ವಿಶುದ್ಧಾತ್ಮಾ ವಿಶ್ವಾತ್ಮಾ ವಿಶ್ವತೋ ಮುಖಃ~।\\
ಸರ್ವೇಶ್ವರಃ ಸದಾತುಷ್ಟಃ ಸರ್ವಮಂಗಲದಾಯಕಃ ॥೮॥

ನಿಷ್ಕಲಂಕೋ ನಿರಾಭಾಸೋ ನಿರ್ವಿಕಲ್ಪೋ ನಿರಾಶ್ರಯಃ~।\\
ಪುರುಷೋತ್ತಮೋ ಲೋಕನಾಥಃ ಪುರಾಣಪುರುಷೋಽನಘಃ ॥೯॥

ಅಪಾರಮಹಿಮಾಽನಂತೋ ಹ್ಯಾದ್ಯಂತರಹಿತಾಕೃತಿಃ~।\\
ಸಂಸಾರವನದಾವಾಗ್ನಿರ್ಭವಸಾಗರತಾರಕಃ ॥೧೦॥

ಶ್ರೀನಿವಾಸೋ ವಿಶಾಲಾಕ್ಷಃ ಕ್ಷೀರಾಬ್ಧಿಶಯನೋಽಚ್ಯುತಃ~।\\
ಸರ್ವಪಾಪಕ್ಷಯಕರಸ್ತಾಪತ್ರಯನಿವಾರಣಃ ॥೧೧॥

ಲೋಕೇಶಃ ಸರ್ವಭೂತೇಶೋ ವ್ಯಾಪಕಃ ಕರುಣಾಮಯಃ~।\\
ಬ್ರಹ್ಮಾದಿವಂದಿತಪದೋ ಮುನಿವಂದ್ಯಃ ಸ್ತುತಿಪ್ರಿಯಃ ॥೧೨॥

ನಾಮರೂಪಕ್ರಿಯಾತೀತೋ ನಿಃಸ್ಪೃಹೋ ನಿರ್ಮಲಾತ್ಮಕಃ~।\\
ಮಾಯಾಧೀಶೋ ಮಹಾತ್ಮಾ ಚ ಮಹಾದೇವೋ ಮಹೇಶ್ವರಃ ॥೧೩॥

ವ್ಯಾಘ್ರಚರ್ಮಾಂಬರಧರೋ ನಾಗಕುಂಡಲಭೂಷಣಃ~।\\
ಸರ್ವಲಕ್ಷಣಸಂಪೂರ್ಣಃ ಸರ್ವಸಿದ್ಧಿಪ್ರದಾಯಕಃ ॥೧೪॥

ಸರ್ವಜ್ಞಃ ಕರುಣಾಸಿಂಧುಃ ಸರ್ಪಹಾರಃ ಸದಾಶಿವಃ~।\\
ಸಹ್ಯಾದ್ರಿವಾಸಃ ಸರ್ವಾತ್ಮಾ ಭವಬಂಧವಿಮೋಚನಃ ॥೧೫॥

ವಿಶ್ವಂಭರೋ ವಿಶ್ವನಾಥೋ ಜಗನ್ನಾಥೋ ಜಗತ್ಪ್ರಭುಃ~।\\
ನಿತ್ಯಂ ಪಠತಿ ಯೋ ಭಕ್ತ್ಯಾ ಸರ್ವಪಾಪೈಃ ಪ್ರಮುಚ್ಯತೇ ॥೧೬॥

ಸರ್ವದುಃಖಪ್ರಶಮನಂ ಸರ್ವಾರಿಷ್ಟನಿವಾರಣಂ~।\\
ಭೋಗಮೋಕ್ಷಪ್ರದಂ ನೄಣಾಂ ದತ್ತಸಾಯುಜ್ಯದಾಯಕಂ~।\\
ಪಠಂತಿ ಯೇ ಪ್ರಯತ್ನೇನ ಸತ್ಯಂ ಸತ್ಯಂ ವದಾಮ್ಯಹಂ ॥೧೭॥

\authorline{॥ಇತಿ ಬ್ರಹ್ಮಾಂಡಪುರಾಣೇ ಬ್ರಹ್ಮನಾರದಸಂವಾದೇ ಶ್ರೀದತ್ತಾತ್ರೇಯಾಷ್ಟೋತ್ತರ ಶತನಾಮಸ್ತೋತ್ರಂ ॥}
\section{ಲಕ್ಷ್ಮೀನರಸಿಂಹ ಅಷ್ಟೋತ್ತರಶತನಾಮಸ್ತೋತ್ರಂ}
\dhyana{ಮಾಣಿಕ್ಯಾದ್ರಿಸಮಪ್ರಭಂ ನಿಜರುಚಾ ಸಂತ್ರಸ್ತರಕ್ಷೋಗಣಂ\\
ಜಾನುನ್ಯಸ್ತ ಕರಾಂಬುಜಂ ತ್ರಿಣಯನಂ ರತ್ನೋಲ್ಲಸದ್ಭೂಷಣಮ್ ।\\
ಬಾಹುಭ್ಯಾಂ ಧೃತಶಂಖಚಕ್ರಮನಿಶಂ ದಂಷ್ಟ್ರಾಗ್ರವಕ್ತ್ರೋಲ್ಲಸದ್\\
ಜ್ವಾಲಾಜಿಹ್ವಮುದಗ್ರ ಕೇಶನಿಚಯಂ ವಂದೇ ನೃಸಿಂಹಂ ವಿಭುಂ॥ }

ನಾರಸಿಂಹೋ ಮಹಾಸಿಂಹೋ ದಿವ್ಯಸಿಂಹೋ ಮಹಾಬಲಃ~।\\
ಉಗ್ರಸಿಂಹೋ ಮಹಾದೇವಃ ಸ್ತಂಭಜಶ್ಚೋಗ್ರಲೋಚನಃ ॥೧ ॥

ರೌದ್ರಃ ಸರ್ವಾದ್ಭುತಃ ಶ್ರೀಮಾನ್ ಯೋಗಾನಂದ ಸ್ತ್ರಿವಿಕ್ರಮಃ।\\
ಹರಿಃ ಕೋಲಾಹಲಃ ಚಕ್ರೀ ವಿಜಯೋ ಜಯವರ್ಧನಃ ॥೨ ॥

ಪಂಚಾನನಃ ಪರಂಬ್ರಹ್ಮ ಚಾಘೋರೋ ಘೋರವಿಕ್ರಮಃ~।\\
ಜ್ವಲನ್ಮುಖೋ ಜ್ವಾಲಮಾಲೀ ಮಹಾಜ್ವಾಲೋ ಮಹಾಪ್ರಭುಃ ॥೩ ॥

ನಿಟಿಲಾಕ್ಷಃ ಸಹಸ್ರಾಕ್ಷೋ ದುರ್ನಿರೀಕ್ಷಃ ಪ್ರತಾಪನಃ~।\\
ಮಹಾದಂಷ್ಟ್ರಾಯುಧಃ ಪ್ರಾಜ್ಞಶ್ಚಂಡಕೋಪೀ ಸದಾಶಿವಃ ॥೪ ॥

ಹಿರಣ್ಯಕಶಿಪುಧ್ವಂಸೀ ದೈತ್ಯದಾನವ ಭಂಜನಃ~।\\
ಗುಣಭದ್ರೋ ಮಹಾಭದ್ರೋ ಬಲಭದ್ರಃ ಸುಭದ್ರಕಃ ॥೫ ॥

ಕರಾಳೋ ವಿಕರಾಳಶ್ಚ ವಿಕರ್ತಾ ಸರ್ವಕರ್ತೃಕಃ~।\\
ಶಿಂಶುಮಾರಸ್ತ್ರಿಲೋಕಾತ್ಮಾ ಈಶಃ ಸರ್ವೇಶ್ವರೋ ವಿಭುಃ ॥೬ ॥

ಭೈರವಾಡಂಬರೋ ದಿವ್ಯಶ್ಚಾಚ್ಯುತಃ ಕವಿ ಮಾಧವಃ~।\\
ಅಧೋಕ್ಷಜೋಽಕ್ಷರಃ ಸರ್ವೋ ವನಮಾಲೀ ವರಪ್ರದಃ ॥೭ ॥

ವಿಶ್ವಂಭರೋದ್ಭುತೋ ಭವ್ಯಃ ಶ್ರೀವಿಷ್ಣುಃ ಪುರುಷೋತ್ತಮಃ~।\\
ಅನಘಾಸ್ತ್ರೋ ನಖಾಸ್ತ್ರಶ್ಚ ಸೂರ್ಯ ಜ್ಯೋತಿಃ ಸುರೇಶ್ವರಃ ॥೮ ॥

ಸಹಸ್ರಬಾಹುಃ ಸರ್ವಜ್ಞಃ ಸರ್ವಸಿದ್ಧಿ ಪ್ರದಾಯಕಃ~।\\
ವಜ್ರದಂಷ್ಟ್ರೋ ವಜ್ರನಖೋ ಮಹಾನಂದಃ ಪರಂತಪಃ ॥೯ ॥

ಸರ್ವಮಂತ್ರೈಕ ರೂಪಶ್ಚ ಸರ್ವಯಂತ್ರ ವಿದಾರಣಃ~।\\
ಸರ್ವತಂತ್ರಾತ್ಮಕೋ ಽವ್ಯಕ್ತಃ ಸುವ್ಯಕ್ತೋ ಭಕ್ತವತ್ಸಲಃ ॥೧೦ ॥

ವೈಶಾಖಶುಕ್ಲ ಭೂತೋತ್ಥಃ ಶರಣಾಗತವತ್ಸಲಃ~।\\
ಉದಾರಕೀರ್ತಿಃ ಪುಣ್ಯಾತ್ಮಾ ಮಹಾತ್ಮಾ ಚಂಡ ವಿಕ್ರಮಃ ॥೧೧ ॥

ವೇದತ್ರಯ ಪ್ರಪೂಜ್ಯಶ್ಚ ಭಗವಾನ್ ಪರಮೇಶ್ವರಃ~।\\
ಶ್ರೀವತ್ಸಾಂಕಃ ಶ್ರೀನಿವಾಸೋ ಜಗದ್ವ್ಯಾಪೀ ಜಗನ್ಮಯಃ ॥೧೨ ॥

ಜಗತ್ಪಾಲೋ ಜಗನ್ನಾಥೋ ಮಹಾಕಾಯೋ ದ್ವಿರೂಪಭೃತ್~।\\
ಪರಮಾತ್ಮಾ ಪರಂಜ್ಯೋತಿಃ ನಿರ್ಗುಣಶ್ಚ ನೃಕೇಸರೀ ॥೧೩ ॥

ಪರತತ್ತ್ವಂ ಪರಂಧಾಮ ಸಚ್ಚಿದಾನಂದವಿಗ್ರಹಃ~।\\
ಲಕ್ಷ್ಮೀನೃಸಿಂಹಃ ಸರ್ವಾತ್ಮಾ ಧೀರಃ ಪ್ರಹ್ಲಾದಪಾಲಕಃ ॥೧೪ ॥

ಇದಂ ಶ್ರೀಮನ್ನೃಸಿಂಹಸ್ಯ ನಾಮ್ನಾಮಷ್ಟೋತ್ತರಂ ಶತಂ~।\\
ತ್ರಿಸಂಧ್ಯಂ ಯಃಪಠೇತ್ ಭಕ್ತ್ಯಾ ಸರ್ವಾಭೀಷ್ಟಮವಾಪ್ನುಯಾತ್ ॥೧೫ ॥

\authorline{॥ಇತಿ ಲಕ್ಷ್ಮೀನರಸಿಂಹ ಅಷ್ಟೋತ್ತರ ಶತನಾಮಸ್ತೋತ್ರಂ॥}
\section{ಶಿವ ಆವರಣಪೂಜಾ}
(ಲಿಂಗ, ಯೋನಿ, ತ್ರಿಶೂಲ, ಅಕ್ಷಮಾಲಾ, ಅಭಯ, ವರ, ಮೃಗ, ಖಟ್ವಾಂಗ, ಕಪಾಲ, ಡಮರು ಮುದ್ರಾಃ ಪ್ರದರ್ಶ್ಯ)
\subsection{ಪ್ರಥಮಾವರಣಮ್}
ಓಂ ಓಂ ಹೃದಯಾಂಗ ದೇವತಾಭ್ಯೋ ನಮಃ । ಓಂ ನಂ ಶಿರೋಽಂಗ \\ದೇವತಾಭ್ಯೋ ನಮಃ । ಓಂ ಮಂ ಶಿಖಾಂಗ ದೇವತಾಭ್ಯೋ ನಮಃ । ಓಂ ಶಿಂ ಕವಚಾಂಗ ದೇವತಾಭ್ಯೋ ನಮಃ । ಓಂ ವಾಂ ನೇತ್ರಾಂಗ ದೇವತಾಭ್ಯೋ ನಮಃ। ಓಂ ಯಂ ಅಸ್ತ್ರಾಂಗ ದೇವತಾಭ್ಯೋ ನಮಃ ॥\\
ಅಭೀಷ್ಟಸಿದ್ಧಿಂ ಮೇ ದೇಹಿ ಶರಣಾಗತ ವತ್ಸಲ~।\\
ಭಕ್ತ್ಯಾ ಸಮರ್ಪಯೇ ತುಭ್ಯಂ ಪ್ರಥಮಾವರಣಾರ್ಚನಂ ॥
\subsection{ದ್ವಿತೀಯಾವರಣಮ್}
ಓಂ ಗಗನಾ ತತ್ಪುರುಷಾಭ್ಯಾಂ ನಮಃ । ಓಂ  ಕರಾಲಿಕಾ ವಾಮದೇವಾಭ್ಯಾಂ ನಮಃ । ಓಂ  ರಕ್ತಾಘೋರಾಭ್ಯಾಂ ನಮಃ । ಓಂ  ಮಹೋಚ್ಛುಷ್ಮಾ \\ಸದ್ಯೋಜಾತಾಭ್ಯಾಂ ನಮಃ । ಓಂ  ಹೃಲ್ಲೇಖೇಶಾನಾಭ್ಯಾಂ ನಮಃ ॥\\
ಅಭೀಷ್ಟಸಿದ್ಧಿಂ******ದ್ವಿತೀಯಾವರಣಾರ್ಚನಂ ॥
\subsection{ತೃತೀಯಾವರಣಮ್}
ಓಂ ಗಾಯತ್ರ್ಯೈ ನಮಃ । ಓಂ  ಸಾವಿತ್ರ್ಯೈ ನಮಃ । ಓಂ ಸರಸ್ವತ್ಯೈ ನಮಃ । ಓಂ  ವಸುಧಾ ಶಂಖನಿಧಿಭ್ಯಾಂ ನಮಃ । ಓಂ  ವಸುಮತೀ ಪುಷ್ಪನಿಧಿಭ್ಯಾಂ ನಮಃ ॥\\
ಅಭೀಷ್ಟಸಿದ್ಧಿಂ******ತೃತೀಯಾವರಣಾರ್ಚನಂ ॥
\subsection{ಚತುರ್ಥಾವರಣಮ್}
ಓಂ ಅನಂತಾಯ ನಮಃ । ಓಂ  ಸೂಕ್ಷ್ಮಾಯ ನಮಃ । ಓಂ  ಶಿವೋತ್ತಮಾಯ ನಮಃ। ಓಂ  ಏಕನೇತ್ರಾಯ ನಮಃ । ಓಂ  ಏಕರುದ್ರಾಯ ನಮಃ । \\ಓಂ   ತ್ರಿಮೂರ್ತಯೇ ನಮಃ । ಓಂ  ಶ್ರೀಕಂಠಾಯ ನಮಃ ।\\ ಓಂ  ಶಿಖಂಡಿನೇ ನಮಃ ॥\\
ಅಭೀಷ್ಟಸಿದ್ಧಿಂ******ತುರೀಯಾವರಣಾರ್ಚನಂ ॥
\subsection{ಪಂಚಮಾವರಣಮ್}
ಓಂ ಉಮಾಯೈ ನಮಃ । ಓಂ  ಚಂಡೇಶ್ವರಾಯ ನಮಃ । ಓಂ  ನಂದಿನೇ ನಮಃ । ಓಂ  ಮಹಾಕಾಲಾಯ ನಮಃ । ಓಂ  ಗಣೇಶ್ವರಾಯ ನಮಃ । ಓಂ  ಸ್ಕಂದಾಯ ನಮಃ । ಓಂ  ವೀರಭದ್ರಾಯ ನಮಃ । ಓಂ  ಭೃಂಗಿರಿಟಾಯ ನಮಃ ॥\\
ಅಭೀಷ್ಟಸಿದ್ಧಿಂ******ಪಂಚಮಾವರಣಾರ್ಚನಂ ॥
\subsection{ಷಷ್ಠಾವರಣಮ್}
ಓಂ ಬ್ರಾಹ್ಮ್ಯೈ ನಮಃ । ಓಂ ಮಾಹೇಶ್ವರ್ಯೈ ನಮಃ । ಓಂ ಕೌಮಾರ್ಯೈ ನಮಃ । ಓಂ ವೈಷ್ಣವ್ಯೈ ನಮಃ । ಓಂ ವಾರಾಹ್ಯೈ ನಮಃ । ಓಂ ಮಾಹೇಂದ್ರ್ಯೈ ನಮಃ । ಓಂ ಚಾಮುಂಡಾಯೈ ನಮಃ । ಓಂ ಮಹಾಲಕ್ಷ್ಮ್ಯೈ ನಮಃ ॥\\
ಅಭೀಷ್ಟಸಿದ್ಧಿಂ******ಷಷ್ಠಾಖ್ಯಾವರಣಾರ್ಚನಂ ॥
\newpage
\subsection{ಸಪ್ತಮಾವರಣಮ್}
ಓಂ ಅಸಿತಾಂಗಭೈರವಾಯ ನಮಃ । ಓಂ ರುರುಭೈರವಾಯ ನಮಃ । ಓಂ ಚಂಡಭೈರವಾಯ ನಮಃ । ಓಂ ಕ್ರೋಧಭೈರವಾಯ ನಮಃ । ಓಂ ಉನ್ಮತ್ತಭೈರವಾಯ ನಮಃ । ಓಂ ಕಪಾಲಭೈರವಾಯ ನಮಃ । ಓಂ ಭೀಷಣಭೈರವಾಯ ನಮಃ । ಓಂ ಸಂಹಾರಭೈರವಾಯ ನಮಃ ॥\\%। ಓಂ ವರಪ್ರದಭೈರವಾಯ ನಮಃ
ಅಭೀಷ್ಟಸಿದ್ಧಿಂ******ಸಪ್ತಮಾವರಣಾರ್ಚನಂ ॥
\subsection{ಅಷ್ಟಮಾವರಣಮ್}
ಓಂ ಲಂ ಇಂದ್ರಾಯ ನಮಃ । ಓಂ ರಂ ಅಗ್ನಯೇ ನಮಃ । ಓಂ ಮಂ ಯಮಾಯ ನಮಃ । ಓಂ ಕ್ಷಂ ನಿರ್ಋತಯೇ ನಮಃ ।ಓಂ ವಂ ವರುಣಾಯ ನಮಃ ।ಓಂ ಯಂ ವಾಯವೇ ನಮಃ । ಓಂ ಕುಂ ಕುಬೇರಾಯ ನಮಃ । ಓಂ ಹಂ ಈಶಾನಾಯ ನಮಃ । ಓಂ ಆಂ ಬ್ರಹ್ಮಣೇ ನಮಃ । ಓಂ ಹ್ರೀಂ ಅನಂತಾಯ ನಮಃ । ಓಂ ನಿಯತ್ಯೈ ನಮಃ । ಓಂ ಕಾಲಾಯ ॥\\
ಅಭೀಷ್ಟಸಿದ್ಧಿಂ******ಅಷ್ಟಮಾವರಣಾರ್ಚನಂ ॥
\subsection{ನವಮಾವರಣಮ್}
ಓಂ ವಂ ವಜ್ರಾಯ ನಮಃ । ಓಂ ಶಂ ಶಕ್ತ್ಯೈ ನಮಃ । ಓಂ ದಂ ದಂಡಾಯ ನಮಃ । ಓಂ ಖಂ ಖಡ್ಗಾಯ ನಮಃ । ಓಂ ಪಾಂ ಪಾಶಾಯ ನಮಃ । ಓಂ ಅಂ ಅಂಕುಶಾಯ ನಮಃ । ಓಂ ಗಂ ಗದಾಯೈ ನಮಃ । ಓಂ  ತ್ರಿಂ  ತ್ರಿಶೂಲಾಯ ನಮಃ । ಓಂ ಪಂ ಪದ್ಮಾಯ ನಮಃ । ಓಂ ಚಂ ಚಕ್ರಾಯ ನಮಃ ॥\\
ಅಭೀಷ್ಟಸಿದ್ಧಿಂ******ನವಮಾವರಣಾರ್ಚನಂ ॥
\newpage
\section{ಅಥ ಶಿವಾಷ್ಟೋತ್ತರಶತನಾಮಸ್ತೋತ್ರಮ್ }
\dhyana{ವಾಮಾಂಕ ನ್ಯಸ್ತ ವಾಮೇತರಕರಕಮಲಾಯಾಸ್ತಥಾ ವಾಮಬಾಹು\\ ನ್ಯಸ್ತಾರಕ್ತೋತ್ಪಲಾಯಾಃ ಸ್ತನವಿಧೃತಿಲಸದ್ವಾಮಹಸ್ತಃ ಪ್ರಿಯಾಯಾಃ ।\\
ನಾನಾಕಲ್ಪಾಭಿರಾಮೋ ಧೃತಪರಶುಮೃಗಾಭೀಷ್ಟದಃ ಕಾಂಚನಾಭಃ\\
ಧ್ಯೇಯಃ ಪದ್ಮಾಸನಸ್ಥಃ ಸ್ಮರಲಲಿತವಪುಃ ಸಂಪದೇ ಪಾರ್ವತೀಶಃ ॥}

ಶಿವೋ ಮಹೇಶ್ವರಃ ಶಂಭುಃ ಪಿನಾಕೀ ಶಶಿಶೇಖರಃ~।\\
ವಾಮದೇವೋ ವಿರೂಪಾಕ್ಷಃ ಕಪರ್ದೀ ನೀಲಲೋಹಿತಃ ॥೧ ॥

ಶಂಕರಃ ಶೂಲಪಾಣಿಶ್ಚ ಖಟ್ವಾಂಗೀ ವಿಷ್ಣುವಲ್ಲಭಃ~।\\
ಶಿಪಿವಿಷ್ಟೋಂಬಿಕಾನಾಥಃ ಶ್ರೀಕಂಠೋ ಭಕ್ತವತ್ಸಲಃ ॥೨ ॥

ಭವಃ ಶರ್ವಸ್ತ್ರಿಲೋಕೇಶಃ ಶಿತಿಕಂಠಃ ಶಿವಾಪ್ರಿಯಃ।\\
ಉಗ್ರಃ ಕಪಾಲೀ ಕಾಮಾರಿರಂಧಕಾಸುರಸೂದನಃ ॥೩ ॥

ಗಂಗಾಧರೋ ಲಲಾಟಾಕ್ಷಃ ಕಾಲಕಾಲಃ ಕೃಪಾನಿಧಿಃ~।\\
ಭೀಮಃ ಪರಶುಹಸ್ತಶ್ಚ ಮೃಗಪಾಣಿರ್ಜಟಾಧರಃ ॥೪ ॥

ಕೈಲಾಸವಾಸೀ ಕವಚೀ ಕಠೋರಸ್ತ್ರಿಪುರಾಂತಕಃ।\\
ವೃಷಾಂಕೋ ವೃಷಭಾರೂಢೋ ಭಸ್ಮೋದ್ಧೂಲಿತವಿಗ್ರಹಃ ॥೫ ॥

ಸಾಮಪ್ರಿಯಃ ಸ್ವರಮಯಸ್ತ್ರಯೀಮೂರ್ತಿರನೀಶ್ವರಃ~।\\
ಸರ್ವಜ್ಞಃ ಪರಮಾತ್ಮಾ ಚ ಸೋಮಸೂರ್ಯಾಗ್ನಿಲೋಚನಃ ॥೬॥

ಹವಿರ್ಯಜ್ಞಮಯಃ ಸೋಮಃ ಪಂಚವಕ್ತ್ರಃ ಸದಾಶಿವಃ।\\
ವಿಶ್ವೇಶ್ವರೋ ವೀರಭದ್ರೋ ಗಣನಾಥಃ ಪ್ರಜಾಪತಿಃ ॥೭॥

ಹಿರಣ್ಯರೇತಾ ದುರ್ಧರ್ಷೋ ಗಿರೀಶೋ ಗಿರಿಶೋನಘಃ।\\
ಭುಜಂಗಭೂಷಣೋ ಭರ್ಗೋ ಗಿರಿಧನ್ವಾ ಗಿರಿಪ್ರಿಯಃ ॥೮॥

ಕೃತ್ತಿವಾಸಾಃ ಪುರಾರಾತಿರ್ಭಗವಾನ್ ಪ್ರಮಥಾಧಿಪಃ।\\
ಮೃತ್ಯುಂಜಯಃ ಸೂಕ್ಷ್ಮತನುರ್ಜಗದ್ವ್ಯಾಪೀ ಜಗದ್ಗುರುಃ ॥೯॥

ವ್ಯೋಮಕೇಶೋ ಮಹಾಸೇನಜನಕಶ್ಚಾರುವಿಕ್ರಮಃ।\\
ರುದ್ರೋ ಭೂತಪತಿಃ ಸ್ಥಾಣುರಹಿರ್ಬುಧ್ನ್ಯೋ ದಿಗಂಬರಃ ॥೧೦॥

ಅಷ್ಟಮೂರ್ತಿರನೇಕಾತ್ಮಾ ಸಾತ್ವಿಕಃ ಶುದ್ಧವಿಗ್ರಹಃ।\\
ಶಾಶ್ವತಃ ಖಂಡಪರಶುರಜಃ ಪಾಶವಿಮೋಚನಃ ॥೧೧॥

ಮೃಡಃ ಪಶುಪತಿರ್ದೇವೋ ಮಹಾದೇವೋಽವ್ಯಯೋ ಹರಿಃ।\\
ಪೂಷದಂತಭಿದವ್ಯಗ್ರೋ ದಕ್ಷಾಧ್ವರಹರೋ ಹರಃ ॥೧೨॥

ಭಗನೇತ್ರಭಿದವ್ಯಕ್ತಃ ಸಹಸ್ರಾಕ್ಷಃ ಸಹಸ್ರಪಾತ್।\\
ಅಪವರ್ಗಪ್ರದೋಽನಂತಸ್ತಾರಕಃ ಪರಮೇಶ್ವರಃ ॥೧೩॥

\authorline{॥ಇತಿ ಶಿವಾಷ್ಟೋತ್ತರ ಶತನಾಮಸ್ತೋತ್ರಂ ಸಂಪೂರ್ಣಂ॥}
\section{ಶ್ರೀಬಾಲಾಷ್ಟೋತ್ತರಶತನಾಮಸ್ತೋತ್ರಂ}
ಅಸ್ಯ ಶ್ರೀಬಾಲಾಷ್ಟೋತ್ತರ ಶತನಾಮಸ್ತೋತ್ರ ಮಹಾಮಂತ್ರಸ್ಯ  ದಕ್ಷಿಣಾಮೂರ್ತಿಃ ಋಷಿಃ~। ಅನುಷ್ಟುಪ್ ಛಂದಃ~। ಬಾಲಾತ್ರಿಪುರಸುಂದರೀ ದೇವತಾ~। ಐಂ ಬೀಜಂ~। ಕ್ಲೀಂ ಶಕ್ತಿಃ~। ಸೌಃ ಕೀಲಕಂ~।  ಶ್ರೀಬಾಲಾಪ್ರಿತ್ಯರ್ಥೇ ನಾಮಪಾರಾಯಣೇ ವಿನಿಯೋಗಃ ॥

\dhyana{ಪಾಶಾಂಕುಶೇ ಪುಸ್ತಕಾಕ್ಷಸೂತ್ರೇ ಚ ದಧತೀ ಕರೈಃ~।\\
ರಕ್ತಾ ತ್ರ್ಯಕ್ಷಾ ಚಂದ್ರಫಾಲಾ ಪಾತು ಬಾಲಾ ಸುರಾರ್ಚಿತಾ ॥}

ಕಲ್ಯಾಣೀ ತ್ರಿಪುರಾ ಬಾಲಾ ಮಾಯಾ ತ್ರಿಪುರಸುಂದರೀ~।\\
ಸುಂದರೀ ಸೌಭಾಗ್ಯವತೀ ಕ್ಲೀಂಕಾರೀ ಸರ್ವಮಂಗಲಾ ॥೧॥

ಹ್ರೀಂಕಾರೀ ಸ್ಕಂದಜನನೀ ಪರಾ ಪಂಚದಶಾಕ್ಷರೀ~।\\
ತ್ರಿಲೋಕೀ ಮೋಹನಾಧೀಶಾ ಸರ್ವೇಶೀ ಸರ್ವರೂಪಿಣೀ ॥೨॥

ಸರ್ವಸಂಕ್ಷೋಭಿಣೀ ಪೂರ್ಣಾ ನವಮುದ್ರೇಶ್ವರೀ ಶಿವಾ~।\\
ಅನಂಗಕುಸುಮಾ ಖ್ಯಾತಾ ಅನಂಗಾ ಭುವನೇಶ್ವರೀ ॥೩॥

ಜಪ್ಯಾ ಸ್ತವ್ಯಾ ಶ್ರುತಿರ್ನಿತ್ಯಾ ನಿತ್ಯಕ್ಲಿನ್ನಾಽಮೃತೋದ್ಭವಾ~।\\
ಮೋಹಿನೀ ಪರಮಾಽಽನಂದಾ ಕಾಮೇಶತರುಣಾ ಕಲಾ ॥೪॥

ಕಲಾವತೀ ಭಗವತೀ ಪದ್ಮರಾಗಕಿರೀಟಿನೀ~।\\
ಸೌಗಂಧಿನೀ ಸರಿದ್ವೇಣೀ ಮಂತ್ರಿಣೀ ಮಂತ್ರರೂಪಿಣೀ ॥೫॥

ತತ್ತ್ವತ್ರಯೀ ತತ್ತ್ವಮಯೀ ಸಿದ್ಧಾ ತ್ರಿಪುರವಾಸಿನೀ~।\\
ಶ್ರೀರ್ಮತಿಶ್ಚ ಮಹಾದೇವೀ ಕೌಲಿನೀ ಪರದೇವತಾ ॥೬॥

ಕೈವಲ್ಯರೇಖಾ ವಶಿನೀ ಸರ್ವೇಶೀ ಸರ್ವಮಾತೃಕಾ~।\\
ವಿಷ್ಣುಸ್ವಸಾ ದೇವಮಾತಾ ಸರ್ವಸಂಪತ್ಪ್ರದಾಯಿನೀ ॥೭॥

ಕಿಂಕರೀ ಮಾತಾ ಗೀರ್ವಾಣೀ ಸುರಾಪಾನಾನುಮೋದಿನೀ~।\\
ಆಧಾರಾಹಿತಪತ್ನೀಕಾ ಸ್ವಾಧಿಷ್ಠಾನಸಮಾಶ್ರಯಾ ॥೮॥

ಅನಾಹತಾಬ್ಜನಿಲಯಾ ಮಣಿಪೂರಸಮಾಶ್ರಯಾ~।\\
ಆಜ್ಞಾ ಪದ್ಮಾಸನಾಸೀನಾ ವಿಶುದ್ಧಸ್ಥಲಸಂಸ್ಥಿತಾ ॥೯॥

ಅಷ್ಟಾತ್ರಿಂಶತ್ಕಲಾಮೂರ್ತಿ ಸ್ಸುಷುಮ್ನಾ ಚಾರುಮಧ್ಯಮಾ~।\\
ಯೋಗೇಶ್ವರೀ ಮುನಿಧ್ಯೇಯಾ ಪರಬ್ರಹ್ಮಸ್ವರೂಪಿಣೀ ॥೧೦॥

ಚತುರ್ಭುಜಾ ಚಂದ್ರಚೂಡಾ ಪುರಾಣಾಗಮರೂಪಿಣೀ~।\\
ಐಂಕಾರಾದಿರ್ಮಹಾವಿದ್ಯಾ ಪಂಚಪ್ರಣವರೂಪಿಣೀ ॥೧೧॥

ಭೂತೇಶ್ವರೀ ಭೂತಮಯೀ ಪಂಚಾಶದ್ವರ್ಣರೂಪಿಣೀ~।\\
ಷೋಢಾನ್ಯಾಸ ಮಹಾಭೂಷಾ ಕಾಮಾಕ್ಷೀ ದಶಮಾತೃಕಾ ॥೧೨॥

ಆಧಾರಶಕ್ತಿಃ ತರುಣೀ ಲಕ್ಷ್ಮೀಃ ತ್ರಿಪುರಭೈರವೀ~।\\
ಶಾಂಭವೀ ಸಚ್ಚಿದಾನಂದಾ ಸಚ್ಚಿದಾನಂದರೂಪಿಣೀ ॥೧೩॥

ಮಾಂಗಲ್ಯ ದಾಯಿನೀ ಮಾನ್ಯಾ ಸರ್ವಮಂಗಲಕಾರಿಣೀ~।\\
ಯೋಗಲಕ್ಷ್ಮೀಃ ಭೋಗಲಕ್ಷ್ಮೀಃ ರಾಜ್ಯಲಕ್ಷ್ಮೀಃ ತ್ರಿಕೋಣಗಾ ॥೧೪॥

ಸರ್ವಸೌಭಾಗ್ಯಸಂಪನ್ನಾ ಸರ್ವಸಂಪತ್ತಿದಾಯಿನೀ~।\\
ನವಕೋಣಪುರಾವಾಸಾ ಬಿಂದುತ್ರಯಸಮನ್ವಿತಾ ॥೧೫॥

ನಾಮ್ನಾಮಷ್ಟೋತ್ತರಶತಂ ಪಠೇನ್ನ್ಯಾಸಸಮನ್ವಿತಂ~।\\
ಸರ್ವಸಿದ್ಧಿಮವಾಪ್ನೋತಿ ಸಾಧಕೋಽಭೀಷ್ಟಮಾಪ್ನುಯಾತ್ ॥೧೬॥
\authorline{ಇತಿ ಶ್ರೀ ರುದ್ರಯಾಮಲತಂತ್ರೇ ಉಮಾಮಹೇಶ್ವರಸಂವಾದೇ\\ ಶ್ರೀ ಬಾಲಾ ಅಷ್ಟೋತ್ತರ ಶತನಾಮಸ್ತೋತ್ರಂ ಸಂಪೂರ್ಣಂ ॥}
\section{ಶ್ರೀಅನ್ನಪೂರ್ಣಾಷ್ಟೋತ್ತರಶತನಾಮಸ್ತೋತ್ರಂ }
ಅಸ್ಯ ಶ್ರೀಅನ್ನಪೂರ್ಣಾಷ್ಟೋತ್ತರ ಶತನಾಮಸ್ತೋತ್ರಮಂತ್ರಸ್ಯ ಭಗವಾನ್ ಶ್ರೀಬ್ರಹ್ಮಾ ಋಷಿಃ~। ಅನುಷ್ಟುಪ್ಛಂದಃ~। ಶ್ರೀಅನ್ನಪೂರ್ಣೇಶ್ವರೀ ದೇವತಾ~। ಸ್ವಧಾ ಬೀಜಂ~। ಸ್ವಾಹಾ ಶಕ್ತಿಃ~। ಓಂ ಕೀಲಕಂ~।

ಅನ್ನಪೂರ್ಣಾ ಶಿವಾ ದೇವೀ ಭೀಮಾ ಪುಷ್ಟಿಸ್ಸರಸ್ವತೀ~।\\
ಸರ್ವಜ್ಞಾ ಪಾರ್ವತೀ ದುರ್ಗಾ ಶರ್ವಾಣೀ ಶಿವವಲ್ಲಭಾ ॥೧॥

ವೇದವೇದ್ಯಾ ಮಹಾವಿದ್ಯಾ ವಿದ್ಯಾದಾತ್ರೀ ವಿಶಾರದಾ~।\\
ಕುಮಾರೀ ತ್ರಿಪುರಾ ಬಾಲಾ ಲಕ್ಷ್ಮೀಶ್ಶ್ರೀರ್ಭಯಹಾರಿಣೀ ॥೨॥

ಭವಾನೀ ವಿಷ್ಣುಜನನೀ ಬ್ರಹ್ಮಾದಿಜನನೀ ತಥಾ~।\\
ಗಣೇಶಜನನೀ ಶಕ್ತಿಃ ಕುಮಾರಜನನೀ ಶುಭಾ ॥೩॥

ಭೋಗಪ್ರದಾ ಭಗವತೀ ಭಕ್ತಾಭೀಷ್ಟಪ್ರದಾಯಿನೀ~।\\
ಭವರೋಗಹರಾ ಭವ್ಯಾ ಶುಭ್ರಾ ಪರಮಮಂಗಲಾ ॥೪॥

ಭವಾನೀ ಚಂಚಲಾ ಗೌರೀ ಚಾರುಚಂದ್ರಕಲಾಧರಾ~।\\
ವಿಶಾಲಾಕ್ಷೀ ವಿಶ್ವಮಾತಾ ವಿಶ್ವವಂದ್ಯಾ ವಿಲಾಸಿನೀ ॥೫॥

ಆರ್ಯಾ ಕಲ್ಯಾಣನಿಲಯಾ ರುದ್ರಾಣೀ ಕಮಲಾಸನಾ~।\\
ಶುಭಪ್ರದಾ ಶುಭಾವರ್ತಾ ವೃತ್ತಪೀನಪಯೋಧರಾ ॥೬॥

ಅಂಬಾ ಸಂಸಾರಮಥಿನೀ ಮೃಡಾನೀ ಸರ್ವಮಂಗಲಾ~।\\
ವಿಷ್ಣುಸಂಸೇವಿತಾ ಸಿದ್ಧಾ ಬ್ರಹ್ಮಾಣೀ ಸುರಸೇವಿತಾ ॥೭॥

ಪರಮಾನಂದದಾ ಶಾಂತಿಃ ಪರಮಾನಂದರೂಪಿಣೀ~।\\
ಪರಮಾನಂದಜನನೀ ಪರಾನಂದಪ್ರದಾಯಿನೀ ॥೮॥

ಪರೋಪಕಾರನಿರತಾ ಪರಮಾ ಭಕ್ತವತ್ಸಲಾ~।\\
ಪೂರ್ಣಚಂದ್ರಾಭವದನಾ ಪೂರ್ಣಚಂದ್ರನಿಭಾಂಶುಕಾ ॥೯॥

ಶುಭಲಕ್ಷಣಸಂಪನ್ನಾ ಶುಭಾನಂದಗುಣಾರ್ಣವಾ~।\\
ಶುಭಸೌಭಾಗ್ಯನಿಲಯಾ ಶುಭದಾ ಚ ರತಿಪ್ರಿಯಾ ॥೧೦॥

ಚಂಡಿಕಾ ಚಂಡಮಥನೀ ಚಂಡದರ್ಪನಿವಾರಿಣೀ~।\\
ಮಾರ್ತಂಡನಯನಾ ಸಾಧ್ವೀ ಚಂದ್ರಾಗ್ನಿನಯನಾ ಸತೀ ॥೧೧॥

ಪುಂಡರೀಕಕರಾ ಪೂರ್ಣಾ ಪುಣ್ಯದಾ ಪುಣ್ಯರೂಪಿಣೀ~।\\
ಮಾಯಾತೀತಾ ಶ್ರೇಷ್ಠಮಾಯಾ ಶ್ರೇಷ್ಠಧರ್ಮಾತ್ಮವಂದಿತಾ ॥೧೨॥

ಅಸೃಷ್ಟಿಸ್ಸಂಗರಹಿತಾ ಸೃಷ್ಟಿಹೇತುಃ ಕಪರ್ದಿನೀ~।\\
ವೃಷಾರೂಢಾ ಶೂಲಹಸ್ತಾ ಸ್ಥಿತಿಸಂಹಾರಕಾರಿಣೀ ॥೧೩॥

ಮಂದಸ್ಮಿತಾ ಸ್ಕಂದಮಾತಾ ಶುದ್ಧಚಿತ್ತಾ ಮುನಿಸ್ತುತಾ~।\\
ಮಹಾಭಗವತೀ ದಕ್ಷಾ ದಕ್ಷಾಧ್ವರವಿನಾಶಿನೀ ॥೧೪॥

ಸರ್ವಾರ್ಥದಾತ್ರೀ ಸಾವಿತ್ರೀ ಸದಾಶಿವಕುಟುಂಬಿನೀ~।\\
ನಿತ್ಯಸುಂದರಸರ್ವಾಂಗೀ ಸಚ್ಚಿದಾನಂದಲಕ್ಷಣಾ ॥೧೫॥

ನಾಮ್ನಾಮಷ್ಟೋತ್ತರಶತಮಂಬಾಯಾಃ ಪುಣ್ಯಕಾರಣಂ~।\\
ಸರ್ವಸೌಭಾಗ್ಯಸಿದ್ಧ್ಯರ್ಥಂ ಜಪನೀಯಂ ಪ್ರಯತ್ನತಃ ॥೧೬॥

ಏತಾನಿ ದಿವ್ಯನಾಮಾನಿ ಶ್ರುತ್ವಾ ಧ್ಯಾತ್ವಾ ನಿರಂತರಂ~।\\
ಸ್ತುತ್ವಾ ದೇವೀಂಚ ಸತತಂ ಸರ್ವಾನ್ಕಾಮಾನವಾಪ್ನುಯಾತ್ ॥೧೭॥
\authorline{॥ಇತಿ ಶ್ರೀಬ್ರಹ್ಮೋತ್ತರಖಂಡೇ ಆಗಮಪ್ರಖ್ಯಾತಿಶಿವರಹಸ್ಯೇ\\
ಶ್ರೀಅನ್ನಪೂರ್ಣಾಷ್ಟೋತ್ತರಶತನಾಮಸ್ತೋತ್ರಂ ಸಂಪೂರ್ಣಂ ॥}
\section{ಸೌಭಾಗ್ಯಾಷ್ಟೋತ್ತರಶತನಾಮಸ್ತೋತ್ರಂ }
ಸೌಭಾಗ್ಯಾಷ್ಟೋತ್ತರಶತನಾಮಸ್ತೋತ್ರಸ್ಯ ಶಿವ ಋಷಿಃ । ಅನುಷ್ಟುಪ್ಛಂದಃ । ಶ್ರೀಲಲಿತಾಂಬಿಕಾ  ದೇವತಾ ॥
ಕೂಟತ್ರಯೇಣ ನ್ಯಾಸಃ॥

\as{ಓಂ ಐಂಹ್ರೀಂ ಶ್ರೀಂ}\\
ಕಾಮೇಶ್ವರೀ ಕಾಮಶಕ್ತಿಃ ಕಾಮಸೌಭಾಗ್ಯದಾಯಿನೀ।\\
ಕಾಮರೂಪಾ ಕಾಮಕಲಾ ಕಾಮಿನೀ ಕಮಲಾಸನಾ ॥೧॥

ಕಮಲಾ ಕಲ್ಪನಾಹೀನಾ ಕಮನೀಯಕಲಾವತೀ~।\\
ಕಮಲಾ ಭಾರತೀಸೇವ್ಯಾ ಕಲ್ಪಿತಾಶೇಷಸಂಸೃತಿಃ ॥೨॥

ಅನುತ್ತರಾಽನಘಾಽನಂತಾಽದ್ಭುತರೂಪಾಽನಲೋದ್ಭವಾ~।\\
ಅತಿಲೋಕಚರಿತ್ರಾಽತಿಸುಂದರ್ಯತಿಶುಭಪ್ರದಾ ॥೩॥

ಅಘಹಂತ್ರ್ಯತಿವಿಸ್ತಾರಾಽರ್ಚನತುಷ್ಟಾಽಮಿತಪ್ರಭಾ~।\\
ಏಕರೂಪೈಕವೀರೈಕನಾಥೈಕಾಂತಾಽರ್ಚನಪ್ರಿಯಾ ॥೪॥

ಏಕೈಕಭಾವತುಷ್ಟೈಕರಸೈಕಾಂತಜನಪ್ರಿಯಾ~।\\
ಏಧಮಾನಪ್ರಭಾವೈಧದ್ಭಕ್ತಪಾತಕನಾಶಿನೀ ॥೫॥

ಏಲಾಮೋದಮುಖೈನೋಽದ್ರಿಶಕ್ರಾಯುಧಸಮಸ್ಥಿತಿಃ~।\\
ಈಹಾಶೂನ್ಯೇಪ್ಸಿತೇಶಾದಿಸೇವ್ಯೇಶಾನವರಾಂಗನಾ ॥೬॥

ಈಶ್ವರಾಽಽಜ್ಞಾಪಿಕೇಕಾರಭಾವ್ಯೇಪ್ಸಿತಫಲಪ್ರದಾ~।\\
ಈಶಾನೇತಿಹರೇಕ್ಷೇಷದರುಣಾಕ್ಷೀಶ್ವರೇಶ್ವರೀ ॥೭॥

ಲಲಿತಾ ಲಲನಾರೂಪಾ ಲಯಹೀನಾ ಲಸತ್ತನುಃ~।\\
ಲಯಸರ್ವಾ ಲಯಕ್ಷೋಣಿರ್ಲಯಕರ್ಣೀ ಲಯಾತ್ಮಿಕಾ ॥೮॥

ಲಘಿಮಾ ಲಘುಮಧ್ಯಾಽಽಢ್ಯಾ ಲಲಮಾನಾ ಲಘುದ್ರುತಾ~।\\
ಹಯಾಽಽರೂಢಾ ಹತಾಽಮಿತ್ರಾ ಹರಕಾಂತಾ ಹರಿಸ್ತುತಾ ॥೯॥

ಹಯಗ್ರೀವೇಷ್ಟದಾ ಹಾಲಾಪ್ರಿಯಾ ಹರ್ಷಸಮುದ್ಧತಾ~।\\
ಹರ್ಷಣಾ ಹಲ್ಲಕಾಭಾಂಗೀ ಹಸ್ತ್ಯಂತೈಶ್ವರ್ಯದಾಯಿನೀ ॥೧೦॥

ಹಲಹಸ್ತಾಽರ್ಚಿತಪದಾ ಹವಿರ್ದಾನಪ್ರಸಾದಿನೀ~।\\
ರಾಮರಾಮಾಽರ್ಚಿತಾ ರಾಜ್ಞೀ ರಮ್ಯಾ ರವಮಯೀ ರತಿಃ ॥೧೧॥

ರಕ್ಷಿಣೀರಮಣೀರಾಕಾ ರಮಣೀಮಂಡಲಪ್ರಿಯಾ~।\\
ರಕ್ಷಿತಾಽಖಿಲಲೋಕೇಶಾ ರಕ್ಷೋಗಣನಿಷೂದಿನೀ ॥೧೨॥

ಅಂಬಾಂತಕಾರಿಣ್ಯಂಭೋಜಪ್ರಿಯಾಂತಕಭಯಂಕರೀ~।\\
ಅಂಬುರೂಪಾಂಬುಜಕರಾಂಬುಜಜಾತವರಪ್ರದಾ ॥೧೩॥

ಅಂತಃಪೂಜಾಪ್ರಿಯಾಂತಃಸ್ವರೂಪಿಣ್ಯಂತರ್ವಚೋಮಯೀ~।\\
ಅಂತಕಾರಾತಿವಾಮಾಂಕಸ್ಥಿತಾಂತಃಸುಖರೂಪಿಣೀ ॥೧೪॥

ಸರ್ವಜ್ಞಾ ಸರ್ವಗಾ ಸಾರಾ ಸಮಾ ಸಮಸುಖಾ ಸತೀ~।\\
ಸಂತತಿಃ ಸಂತತಾ ಸೋಮಾ ಸರ್ವಾ ಸಾಂಖ್ಯಾ ಸನಾತನೀ ॥೧೫॥\as{ಶ್ರೀಂಹ್ರೀಂಐಂ}
\authorline{॥ಇತಿ ಸೌಭಾಗ್ಯಾಷ್ಟೋತ್ತರಶತನಾಮಸ್ತೋತ್ರಂ ॥}
\section{ಅಥ ಚತುಃಷಷ್ಟಿ ಯೋಗಿನೀಪೂಜಾ}
\begin{multicols}{2} ಓಂ ಬ್ರಹ್ಮಾಣ್ಯೈ~।\\ ಚಂಡಿಕಾಯೈ~।\\ ರೌದ್ರ್ಯೈ~।\\ ಗೌರ್ಯೈ~।\\ ಇಂದ್ರಾಣ್ಯೈ~।\\ ಕೌಮಾರ್ಯೈ~।\\ ವೈಷ್ಣವ್ಯೈ~।\\ ದುರ್ಗಾಯೈ~।\\ ನಾರಸಿಂಹ್ಯೈ~।\\ ಕಾಲಿಕಾಯೈ~।\\ ಚಾಮುಂಡಾಯೈ~।\\ ಶಿವದೂತ್ಯೈ~।\\ ವಾರಾಹ್ಯೈ~।\\ ಕೌಶಿಕ್ಯೈ~।\\ ಮಾಹೇಶ್ವರ್ಯೈ~।\\ ಶಾಂಕರ್ಯೈ~।\\ ಜಯಂತ್ಯೈ~।\\ ಸರ್ವಮಂಗಳಾಯೈ~।\\ ಕಾಳ್ಯೈ~।\\ ಕಪಾಲಿನ್ಯೈ~।\\ ಮೇಧಾಯೈ~।\\ ಶಿವಾಯೈ~।\\ ಶಾಕಂಭರ್ಯೈ~।\\ ಭೀಮಾಯೈ~।\\ ಶಾಂತಾಯೈ~।\\ ಭ್ರಾಮರ್ಯೈ~।\\ ರುದ್ರಾಣ್ಯೈ~।\\ ಅಂಬಿಕಾಯೈ~।\\ ಕ್ಷಮಾಯೈ~।\\ ಧಾತ್ರ್ಯೈ~।\\ ಸ್ವಾಹಾಯೈ~।\\ ಸ್ವಧಾಯೈ~।\\ ಅಪರ್ಣಾಯೈ~।\\ ಮಹೋದರ್ಯೈ~।\\ ಘೋರರೂಪಾಯೈ~।\\ ಮಹಾಕಾಳ್ಯೈ~।\\ ಭದ್ರಕಾಳ್ಯೈ~।\\ ಭಯಂಕರ್ಯೈ~।\\ ಕ್ಷೇಮಂಕರ್ಯೈ~।\\ ಉಗ್ರದಂಡಾಯೈ~।\\ ಚಂಡನಾಯಿಕಾಯೈ~।\\ ಚಂಡಾಯೈ~।\\ ಚಂಡವತ್ಯೈ~।\\ ಚಂಡ್ಯೈ~।\\ ಮಹಾಮೋಹಾಯೈ~।\\ ಪ್ರಿಯಂಕರ್ಯೈ~।\\ ಕಲವಿಕರಿಣ್ಯೈ~।\\ ದೇವ್ಯೈ~।\\ ಬಲಪ್ರಮಥಿನ್ಯೈ~।\\ ಮದನೋನ್ಮಥಿನ್ಯೈ~।\\ ಸರ್ವಭೂತದಮನಾಯೈ~।\\ ಉಮಾಯೈ~।\\ ತಾರಾಯೈ~।\\ ಮಹಾನಿದ್ರಾಯೈ~।\\ ವಿಜಯಾಯೈ~।\\ ಜಯಾಯೈ~।\\ ಶೈಲಪುತ್ರ್ಯೈ~।\\ ಚಂಡಘಂಟಾಯೈ~।\\ ಸ್ಕಂದಮಾತ್ರೇ~।\\ ಕಾಲರಾತ್ರ್ಯೈ~।\\ ಚಂಡಿಕಾಯೈ~।\\ ಕೂಷ್ಮಾಂಡಿನ್ಯೈ~।\\ ಕಾತ್ಯಾಯನ್ಯೈ~।\\ ಮಹಾಗೌರ್ಯೈ ನಮಃ~॥
\end{multicols}\authorline{ಇತಿ ಚತುಃಷಷ್ಟಿಯೋಗಿನೀಪೂಜಾ}
\newpage
\section{ ಯೋಗಿನೀಪೂಜಾ}
\begin{multicols}{2} \as{ಅಂ} ಅಮೃತಾಯೈ ನಮಃ~।\\ \as{ಆಂ} ಆಕರ್ಷಿಣ್ಯೈ ನಮಃ~।\\ \as{ಇಂ} ಇಂದ್ರಾಣ್ಯೈ ನಮಃ~।\\ \as{ಈಂ} ಈಶಾನ್ಯೈ ನಮಃ~।\\ \as{ಉಂ} ಉಮಾಯೈ ನಮಃ~।\\ \as{ಊಂ} ಊರ್ಧ್ವಕೇಶ್ಯೈ ನಮಃ~।\\ \as{ಋಂ} ಋದ್ಧಿದಾಯೈ ನಮಃ~।\\ \as{ೠಂ} ೠಕಾರಾಯೈ ನಮಃ~।\\ \as{ಲೃಂ} ಲೃಕಾರಾಯೈ ನಮಃ~।\\ \as{ಲೄಂ} ಲೄಕಾರಾಯೈ ನಮಃ~।\\ \as{ಏಂ} ಏಕಪದಾಯೈ ನಮಃ~।\\ \as{ಐಂ} ಐಶ್ವರ್ಯಾತ್ಮಿಕಾಯೈ ನಮಃ~।\\ \as{ಓಂ} ಓಂಕಾರಾಯೈ ನಮಃ~।\\ \as{ಔಂ} ಔಷಧ್ಯೈ ನಮಃ~।\\ \as{ಅಂ} ಅಂಬಿಕಾಯೈ ನಮಃ~।\\ \as{ಅಃ} ಅಕ್ಷರಾಯೈ ನಮಃ~।\\ \as{ಕಂ} ಕಾಲರಾತ್ರ್ಯೈ ನಮಃ~।\\ \as{ಖಂ} ಖಂಡಿತಾಯೈ ನಮಃ~।\\ \as{ಗಂ} ಗಾಯತ್ರ್ಯೈ ನಮಃ~।\\ \as{ಘಂ} ಘಂಟಾಕರ್ಷಿಣ್ಯೈ ನಮಃ~।\\ \as{ಙಂ} ಙಾರ್ಣಾಯೈ ನಮಃ~।\\ \as{ಚಂ} ಚಂಡಾಯೈ ನಮಃ~।\\ \as{ಛಂ} ಛಾಯಾಯೈ ನಮಃ~।\\ \as{ಜಂ} ಜಯಾಯೈ ನಮಃ~।\\ \as{ಝಂ} ಝಂಕಾರಿಣ್ಯೈ ನಮಃ~।\\ \as{ಞಂ} ಜ್ಞಾನರೂಪಾಯೈ ನಮಃ~।\\ \as{ಟಂ} ಟಂಕಹಸ್ತಾಯೈ ನಮಃ~।\\ \as{ಠಂ} ಠಂಕಾರಿಣ್ಯೈ ನಮಃ~।\\ \as{ಡಂ} ಡಾಮರ್ಯೈ ನಮಃ~।\\ \as{ಢಂ} ಢಂಕಾರಿಣ್ಯೈ ನಮಃ~।\\ \as{ಣಂ} ಣಾರ್ಣಾಯೈ ನಮಃ~।\\ \as{ತಂ} ತಾಮಸ್ಯೈ ನಮಃ~।\\ \as{ಥಂ} ಸ್ಥಾಣ್ವ್ಯೈ ನಮಃ~।\\ \as{ದಂ} ದಾಕ್ಷಾಯಣ್ಯೈ ನಮಃ~।\\ \as{ಧಂ} ಧಾತ್ರ್ಯೈ ನಮಃ~।\\ \as{ನಂ} ನಾರ್ಯೈ ನಮಃ~।\\ \as{ಪಂ} ಪಾರ್ವತ್ಯೈ ನಮಃ~।\\ \as{ಫಂ} ಫಟ್ಕಾರಿಣ್ಯೈ ನಮಃ~।\\ \as{ಬಂ} ಬಂಧಿನ್ಯೈ ನಮಃ ~।\\ \as{ಭಂ} ಭದ್ರಕಾಲ್ಯೈ ನಮಃ ~।\\ \as{ಮಂ} ಮಹಾಮಾಯಾಯೈ ನಮಃ~।\\ \as{ಯಂ} ಯಶಸ್ವಿನ್ಯೈ ನಮಃ~।\\ \as{ರಂ} ರಕ್ತಾಯೈ ನಮಃ~।\\ \as{ಲಂ} ಲಂಬೋಷ್ಠ್ಯೈ ನಮಃ~।\\ \as{ವಂ} ವರದಾಯೈ ನಮಃ~।\\ \as{ಶಂ} ಶ್ರಿಯೈ ನಮಃ~।\\ \as{ಷಂ} ಷಂಡಾಯೈ ನಮಃ~।\\ \as{ಸಂ} ಸರಸ್ವತ್ಯೈ ನಮಃ~।\\ \as{ಹಂ} ಹಂಸವತ್ಯೈ ನಮಃ~।\\ \as{ಕ್ಷಂ} ಕ್ಷಮಾವತ್ಯೈ ನಮಃ ॥
\end{multicols}
\thispagestyle{empty}
\section{ಅಥ ನಿತ್ಯಾ ಯಜನವಿಧಿಃ\\೧। ಶ್ರೀಕಾಮೇಶ್ವರೀನಿತ್ಯಾ}
ಅಸ್ಯ ಶ್ರೀಕಾಮೇಶ್ವರೀನಿತ್ಯಾಮಹಾಮಂತ್ರಸ್ಯ ಸಮ್ಮೋಹನ ಋಷಿಃ~। ಗಾಯತ್ರೀ ಛಂದಃ~। ಶ್ರೀಕಾಮೇಶ್ವರೀದೇವತಾ~। ಕಂ ಬೀಜಂ~। ಇಂ ಶಕ್ತಿಃ। ಲಂ ಕೀಲಕಂ~।\\
\as{ನ್ಯಾಸಃ :}೧.ಓಂ ಐಂ ೨.ಓಂ ಸಕಲಹ್ರೀಂ ೩.ಓಂ ನಿತ್ಯ  ೪.ಓಂ ಕ್ಲಿನ್ನೇ ೫.ಓಂ ಮದದ್ರವೇ ೬.ಓಂ ಸೌಃ \\
{\bfseries ದೇವೀಂ ಧ್ಯಾಯೇಜ್ಜಗದ್ಧಾತ್ರೀಂ ಜಪಾಕುಸುಮಸನ್ನಿಭಾಂ~।\\
ಬಾಲಭಾನುಪ್ರತೀಕಾಶಾಂ ಶಾತಕುಂಭಸಮಪ್ರಭಾಂ ॥\\
ರಕ್ತವಸ್ತ್ರಪರೀಧಾನಾಂ ಸಂಪದ್ವಿದ್ಯಾವಶಂಕರೀಂ~।\\
ನಮಾಮಿ ವರದಾಂ ದೇವೀಂ ಕಾಮೇಶೀಮಭಯಪ್ರದಾಂ ॥\\}
ಮನುಃ :{\bfseries  ಐಂ ಸಕಲಹ್ರೀಂ ನಿತ್ಯಕ್ಲಿನ್ನೇ ಮದದ್ರವೇ ಸೌಃ~॥}

ಓಂ ದ್ರಾಂ ಮದನಬಾಣಾಯ ।  ದ್ರೀಂ ಉನ್ಮಾದನಬಾಣಾಯ ।  ಕ್ಲೀಂ ದೀಪನಬಾಣಾಯ ।  ಬ್ಲೂಂ ಮೋಹನಬಾಣಾಯ ।  ಸಃ ಶೋಷಣಬಾಣಾಯ ನಮಃ~॥

ಓಂ ಅನಂಗಕುಸುಮಾ~।  ಅನಂಗಮೇಖಲಾ~।  ಅನಂಗಮದನಾ~।  ಅನಂಗಮದನಾತುರಾ~।  ಅನಂಗಮದವೇಗಿನೀ~।  ಅನಂಗಭುವನಪಾಲಾ~।  ಅನಂಗಶಶಿರೇಖಾ~।  ಅನಂಗಗಗನರೇಖಾ~॥ 

ಓಂ ಅಂ ಶ್ರದ್ಧಾ~।  ಆಂ ಪ್ರೀತಿ~।  ಇಂ ರತಿ~।  ಈಂ ಧೃತಿ~।  ಉಂ ಕಾಂತಿ~।  ಊಂ ಮನೋರಮಾ~।  ಋಂ ಮನೋಹರಾ~।  ೠಂ ಮನೋರಥಾ~।  ಲೃಂ ಮದನಾ~।  ಲೄಂ ಉನ್ಮಾದಿನೀ~।  ಏಂ ಮೋಹಿನೀ~।  ಐಂ ಶಂಖಿನೀ~। ಓಂ  ಶೋಷಿಣೀ~।  ಔಂ ವಶಂಕರೀ~।  ಅಂ ಶಿಂಜಿನೀ~।  ಅಃ ಸುಭಗಾ~॥ 

ಓಂ ಅಂ ಪೂಷಾ~।  ಆಂ ಇದ್ಧಾ~।  ಇಂ ಸುಮನಸಾ~।  ಈಂ ರತಿ~।  ಉಂ ಪ್ರೀತಿ~।  ಊಂ ಧೃತಿ~।  ಋಂ ಋದ್ಧಿ~।  ೠಂ ಸೌಮ್ಯಾ~।  ಲೃಂ ಮರೀಚಿ~।  ಲೄಂ ಅಂಶುಮಾಲಿನೀ~।  ಏಂ ಶಶಿನೀ~।  ಐಂ ಅಂಗಿರಾ~। ಓಂ ಛಾಯಾ~।  ಔಂ ಸಂಪೂರ್ಣಮಂಡಲಾ~।  ಅಂ ತುಷ್ಟಿ~।  ಅಃ ಅಮೃತಾ~॥ 

ಓಂ ಡಾಕಿನೀ~।  ರಾಕಿಣೀ~।  ಲಾಕಿನೀ~।  ಕಾಕಿನೀ~।  ಸಾಕಿನೀ~।  ಹಾಕಿನೀ~॥ 

ಓಂ ವಂ ವಟುಕ~।  ಗಂ ಗಣಪತಿ~।  ದುಂ ದುರ್ಗಾ~।  ಕ್ಷಂ ಕ್ಷೇತ್ರಪಾಲ~॥ 

ಓಂ ಲಂ ಇಂದ್ರಶಕ್ತಿ~।  ರಂ ಅಗ್ನಿಶಕ್ತಿ~।  ಮಂ ಯಮಶಕ್ತಿ~।  ಕ್ಷಂ ನಿರ್ಋತಿಶಕ್ತಿ~।  ವಂ ವರುಣಶಕ್ತಿ~।  ಯಂ ವಾಯುಶಕ್ತಿ~।  ಕುಂ ಕುಬೇರಶಕ್ತಿ~।  ಹಂ ಈಶಾನಶಕ್ತಿ~।  ಆಂ ಬ್ರಹ್ಮಶಕ್ತಿ~।  ಹ್ರೀಂ ಅನಂತಶಕ್ತಿ~॥ 

ಓಂ ವಂ ವಜ್ರಶಕ್ತಿ~।  ಶಂ ಶಕ್ತಿಶಕ್ತಿ~।  ದಂ ದಂಡಶಕ್ತಿ~।  ಖಂ ಖಡ್ಗಶಕ್ತಿ~।  ಪಾಂ ಪಾಶಶಕ್ತಿ~।  ಅಂ ಅಂಕುಶಶಕ್ತಿ~।  ಗಂ ಗದಾಶಕ್ತಿ~।  ತ್ರಿಂ ತ್ರಿಶೂಲಶಕ್ತಿ~।  ಪಂ ಪದ್ಮಶಕ್ತಿ~।  ಚಂ ಚಕ್ರಶಕ್ತಿ~।\\ ಕಾಮೇಶ್ವರ್ಯೈ ವಿದ್ಮಹೇ ನಿತ್ಯಕ್ಲಿನ್ನಾಯೈ ಧೀಮಹಿ ।\\ತನ್ನೋ ನಿತ್ಯಾ ಪ್ರಚೋದಯಾತ್~॥\\
ಇತಿ ಕಾಮೇಶ್ವರೀಆವರಣಪೂಜಾ~।
\section{೨। ಭಗಮಾಲಿನೀನಿತ್ಯಾ}
ಅಸ್ಯ ಶ್ರೀಭಗಮಾಲಿನೀನಿತ್ಯಾಮಹಾಮಂತ್ರಸ್ಯ ಸುಭಗಋಷಿಃ~। ಗಾಯತ್ರೀ ಛಂದಃ~। ಶ್ರೀಭಗಮಾಲಿನೀ ದೇವತಾ~। ಹ್‌ರ್‌ಬ್ಲೇಂ ಬೀಜಂ~।  ಶ್ರೀಂ ಶಕ್ತಿಃ~। ಕ್ಲೀಂ ಕೀಲಕಂ~।\\
\as{ನ್ಯಾಸಃ :}೧.ಓಂ ಐಂ  ೨.ಓಂ ಭಗಭುಗೇ ೩.ಓಂ ಭಗಿನಿ  ೪.ಓಂ ಭಗೋದರಿ  ೫.ಓಂ ಭಗಮಾಲೇ ೬.ಓಂ ಭಗಾವಹೇ\\
{\bfseries ಭಗರೂಪಾಂ ಭಗಮಯಾಂ ದುಕೂಲವಸನಾಂ ಶಿವಾಂ~।\\
ಸರ್ವಾಲಂಕಾರಸಂಯುಕ್ತಾಂ ಸರ್ವಲೋಕವಶಂಕರೀಂ ॥\\
ಭಗೋದರೀಂ ಮಹಾದೇವೀಂ ರಕ್ತೋತ್ಪಲಸಮಪ್ರಭಾಂ~।\\
ಕಾಮೇಶ್ವರಾಂಕನಿಲಯಾಂ ವಂದೇ ಶ್ರೀಭಗಮಾಲಿನೀಂ ॥\\}
ಮನುಃ :{\bfseries  ಐಂ ಭಗಭುಗೇ ಭಗಿನಿ ಭಗೋದರಿ ಭಗಮಾಲೇ ಭಗಾವಹೇ ಭಗಗುಹ್ಯೇ ಭಗಯೋನಿ ಭಗನಿಪಾತನಿ ಸರ್ವಭಗವಶಂಕರಿ ಭಗರೂಪೇ ನಿತ್ಯಕ್ಲಿನ್ನೇ ಭಗಸ್ವರೂಪೇ ಸರ್ವಾಣಿ ಭಗಾನಿ ಮೇ ಹ್ಯಾನಯ ವರದೇ ರೇತೇ ಸುರೇತೇ ಭಗಕ್ಲಿನ್ನೇ ಕ್ಲಿನ್ನದ್ರವೇ ಕ್ಲೇದಯ ದ್ರಾವಯ ಅಮೋಘೇ ಭಗವಿಚ್ಚೇ ಕ್ಷುಭ ಕ್ಷೋಭಯ ಸರ್ವಸತ್ವಾನ್ ಭಗೇಶ್ವರಿ ಐಂ ಬ್ಲೂಂ ಜಂ ಬ್ಲೂಂ ಭೇಂ ಬ್ಲೂಂ ಮೋಂ ಬ್ಲೂಂ ಹೇಂ ಬ್ಲೂಂ ಹೇಂ ಕ್ಲಿನ್ನೇ ಸರ್ವಾಣಿ ಭಗಾನಿ ಮೇ ವಶಮಾನಯ ಸ್ತ್ರೀಂ ಹ್‌ರ್‌ಬ್ಲೇಂ ಹ್ರೀಂ ॥}

ಓಂ ರಾಗಶಕ್ತಿ~।  ದ್ವೇಷಶಕ್ತಿ~।  ಮದನಾ~।  ಮೋಹಿನೀ~।  ಲೋಲಾ~।  ಜಂಭಿನೀ~।  ಉದ್ಯಮಾ~।  ಶುಭಾ~।  ಹ್ಲಾದಿನೀ~।  ದ್ರಾವಿಣೀ~।  ಪ್ರೀತಿ~।  ರತಿ~।  ರಕ್ತಾ~।  ಮನೋರಮಾ~।  ಸರ್ವೋನ್ಮಾದಾ~।  ಸರ್ವಸುಖಾ~।  ಅನಂಗಾ~।  ಅಭಿತೋದ್ಯಮಾ~। ಅನಲ್ಪಾ~।  ವ್ಯಕ್ತವಿಭವಾ~।  ವಿವಿಧವಿಗ್ರಹಾ~।  ಕ್ಷೋಭವಿಗ್ರಹಾ~॥ 

ಓಂ ಇಕ್ಷುಕೋದಂಡ~।  ಪಾಶ~।  ಕಹ್ಲಾರ~।  ಪದ್ಮ~।  ಅಂಕುಶ~।  ಪುಷ್ಪಸಾಯಕ~॥ 

ಓಂ ಲಂ ಇಂದ್ರಶಕ್ತಿ~।  ರಂ ಅಗ್ನಿಶಕ್ತಿ~।  ಮಂ ಯಮಶಕ್ತಿ~।  ಕ್ಷಂ ನಿರ್ಋತಿಶಕ್ತಿ~।  ವಂ ವರುಣಶಕ್ತಿ~।  ಯಂ ವಾಯುಶಕ್ತಿ~।  ಕುಂ ಕುಬೇರಶಕ್ತಿ~।  ಹಂ ಈಶಾನಶಕ್ತಿ~।  ಆಂ ಬ್ರಹ್ಮಶಕ್ತಿ~।  ಹ್ರೀಂ ಅನಂತಶಕ್ತಿ~।  ನಿಯತಿಶಕ್ತಿ~।  ಕಾಲಶಕ್ತಿ~॥ 

ಓಂ ವಂ ವಜ್ರಶಕ್ತಿ~।  ಶಂ ಶಕ್ತಿಶಕ್ತಿ~।  ದಂ ದಂಡಶಕ್ತಿ~।  ಖಂ ಖಡ್ಗಶಕ್ತಿ~।  ಪಾಂ ಪಾಶಶಕ್ತಿ~।  ಅಂ ಅಂಕುಶಶಕ್ತಿ~।  ಗಂ ಗದಾಶಕ್ತಿ~।  ತ್ರಿಂ ತ್ರಿಶೂಲಶಕ್ತಿ~।  ಪಂ ಪದ್ಮಶಕ್ತಿ~।  ಚಂ ಚಕ್ರಶಕ್ತಿ~॥ 

ಭಗಮಾಲಿನ್ಯೈ ವಿದ್ಮಹೇ ಸರ್ವವಶಂಕರ್ಯೈ ಧೀಮಹಿ ।\\ತನ್ನೋ ನಿತ್ಯಾ ಪ್ರಚೋದಯಾತ್~॥\\
ಇತಿ ಭಗಮಾಲಿನೀ ಆವರಣಪೂಜಾ
\section{೩।ನಿತ್ಯಕ್ಲಿನ್ನಾ}
ಅಸ್ಯ ಶ್ರೀನಿತ್ಯಕ್ಲಿನ್ನಾಮಹಾಮಂತ್ರಸ್ಯ ಬ್ರಹ್ಮಾ ಋಷಿಃ। ವಿರಾಟ್ಛಂದಃ~।\\ ಶ್ರೀನಿತ್ಯಕ್ಲಿನ್ನಾನಿತ್ಯಾದೇವತಾ। ಹ್ರೀಂ ಬೀಜಂ। ಸ್ವಾಹಾ ಶಕ್ತಿಃ । ನ್ನೇ ಕೀಲಕಂ~।\\
\as{ನ್ಯಾಸಃ :}೧.ಓಂ ಹ್ರೀಂ ೨.ಓಂ ನಿತ್ಯ  ೩.ಓಂ ಕ್ಲಿನ್ನೇ ೪.ಓಂ ಮದ ೫.ಓಂ ದ್ರವೇ ೬.ಓಂ ಸ್ವಾಹಾ\\
{\bfseries ಪದ್ಮರಾಗಮಣಿಪ್ರಖ್ಯಾಂ ಹೇಮತಾಟಂಕಸಂಯುತಾಂ~।\\
ರಕ್ತವಸ್ತ್ರಧರಾಂ ದೇವೀಂ ರಕ್ತಮಾಲ್ಯಾನುಲೇಪನಾಂ~॥\\
ಅಂಜನಾಂಚಿತನೇತ್ರಾಂ ತಾಂ ಪದ್ಮಪತ್ರನಿಭೇಕ್ಷಣಾಂ~।\\
ನಿತ್ಯಕ್ಲಿನ್ನಾಂ ನಮಸ್ಯಾಮಿ ಚತುರ್ಭುಜವಿರಾಜಿತಾಂ~॥\\}
ಮನುಃ :{\bfseries  ಹ್ರೀಂ ನಿತ್ಯಕ್ಲಿನ್ನೇ ಮದದ್ರವೇ ಸ್ವಾಹಾ ॥}

ಓಂ ಮದಾವಿಲಾ~।  ಮಂಗಲಾ~।  ಮನ್ಮಥಾರ್ತಾ~।  ಮನಸ್ವಿನೀ~।  ಮೋಹಾ~।  ಆಮೋದಾ~।  ಮಾನಮಯೀ~।  ಮಾಯಾ~।  ಮಂದಾ~।  ಮನೋವತೀ~॥ 

ಓಂ ನಿತ್ಯಾ~।  ನಿರಂಜನಾ~।  ಕ್ಲಿನ್ನಾ~।  ಕ್ಲೇದಿನೀ~।  ಮದನಾತುರಾ~।  ಮದದ್ರವಾ~।  ದ್ರಾವಿಣೀ~।  ದ್ರವಿಣಾ~॥ 

ಓಂ ಕ್ಷೋಭಿಣೀ~।  ಮೋಹಿನೀ~।  ಲೋಲಾ~॥

ಓಂ ಲಂ ಇಂದ್ರಶಕ್ತಿ~।  ರಂ ಅಗ್ನಿಶಕ್ತಿ~।  ಮಂ ಯಮಶಕ್ತಿ~।  ಕ್ಷಂ ನಿರ್ಋತಿಶಕ್ತಿ~।  ವಂ ವರುಣಶಕ್ತಿ~।  ಯಂ ವಾಯುಶಕ್ತಿ~।  ಕುಂ ಕುಬೇರಶಕ್ತಿ~।  ಹಂ ಈಶಾನಶಕ್ತಿ~।  ಆಂ ಬ್ರಹ್ಮಶಕ್ತಿ~।  ಹ್ರೀಂ ಅನಂತಶಕ್ತಿ~।  ನಿಯತಿಶಕ್ತಿ~।  ಕಾಲಶಕ್ತಿ~॥ 

ಓಂ ವಂ ವಜ್ರಶಕ್ತಿ~।  ಶಂ ಶಕ್ತಿಶಕ್ತಿ~।  ದಂ ದಂಡಶಕ್ತಿ~।  ಖಂ ಖಡ್ಗಶಕ್ತಿ~।  ಪಾಂ ಪಾಶಶಕ್ತಿ~।  ಅಂ ಅಂಕುಶಶಕ್ತಿ~।  ಗಂ ಗದಾಶಕ್ತಿ~।  ತ್ರಿಂ ತ್ರಿಶೂಲಶಕ್ತಿ~।  ಪಂ ಪದ್ಮಶಕ್ತಿ~।  ಚಂ ಚಕ್ರಶಕ್ತಿ~॥ 

ನಿತ್ಯಕ್ಲಿನ್ನಾಯೈ ವಿದ್ಮಹೇ ನಿತ್ಯಮದದ್ರವಾಯೈ ಧೀಮಹಿ ।\\ತನ್ನೋ ನಿತ್ಯಾ ಪ್ರಚೋದಯಾತ್ ॥\\
ಇತಿ ನಿತ್ಯಕ್ಲಿನ್ನಾ ಆವರಣಪೂಜಾ
\section{೪।ಭೇರುಂಡಾ}
ಶ್ರೀಭೇರುಂಡಾನಿತ್ಯಾಮಹಾಮಂತ್ರಸ್ಯ ಮಹಾವಿಷ್ಣುಃ ಋಷಿಃ। \\ಗಾಯತ್ರೀಛಂದಃ~। ಭೇರುಂಡಾನಿತ್ಯಾ ದೇವತಾ। ಭ್ರೋಂ ಬೀಜಂ।\\ ಸ್ವಾಹಾ ಶಕ್ತಿಃ। ಕ್ರೋಂ ಕೀಲಕಂ~।\\
\as{ನ್ಯಾಸಃ :}೧.ಓಂ ಕ್ರೋಂ  ೨.ಓಂ ಭ್ರೋಂ ೩.ಓಂ ಕ್ರೋಂ  ೪.ಓಂ ಝ್ರೋಂ ೫.ಓಂ ಛ್ರೋಂ  ೬.ಓಂ ಜ್ರೋಂ \\
{\bfseries ಶುದ್ಧಸ್ಫಟಿಕಸಂಕಾಶಾಂ ಪದ್ಮಪತ್ರಸಮಪ್ರಭಾಂ~।\\
ಮಧ್ಯಾಹ್ನಾದಿತ್ಯಸಂಕಾಶಾಂ ಶುಭ್ರವಸ್ತ್ರಸಮನ್ವಿತಾಂ~॥\\
ಶ್ವೇತಚಂದನಲಿಪ್ತಾಂಗೀಂ ಶುಭ್ರಮಾಲ್ಯವಿಭೂಷಿತಾಂ~।\\	
ಬಿಭ್ರತೀಂ ಚಿನ್ಮಯೀಂ ಮುದ್ರಾಮಕ್ಷಮಾಲಾಂ ಚ ಪುಸ್ತಕಂ~॥\\
ಸಹಸ್ರಪದ್ಮಕಮಲೇ ಸಮಾಸೀನಾಂ ಶುಚಿಸ್ಮಿತಾಂ~।\\
ಸರ್ವವಿದ್ಯಾಪ್ರದಾಂ ದೇವೀಂ ಭೇರುಂಡಾಂ ಪ್ರಣಮಾಮ್ಯಹಂ~॥\\}
ಮನುಃ :{\bfseries  ಕ್ರೋಂ ಭ್ರೋಂ ಕ್ರೋಂ ಝ್ರೋಂ ಛ್ರೋಂ ಜ್ರೋಂ ಸ್ವಾಹಾ ॥}

ಓಂ ಬ್ರಾಹ್ಮೀ~।  ಮಾಹೇಶ್ವರೀ~।  ಕೌಮಾರೀ~।  ವೈಷ್ಣವೀ~। ವಾರಾಹೀ~।\\  ಇಂದ್ರಾಣೀ~।  ಚಾಮುಂಡಾ~।  ಮಹಾಲಕ್ಷ್ಮೀ~॥ 

ಓಂ ಕೃತಯುಗಶಕ್ತಿ~।  ತ್ರೇತಾಯುಗಶಕ್ತಿ~।  ದ್ವಾಪರಯುಗಶಕ್ತಿ~।  ಕಲಿಯುಗಶಕ್ತಿ~॥ 

ಓಂ ವಿಜಯಾ~।  ವಿಮಲಾ~।  ಶುಭಾ~।  ವಿಶ್ವಾ~।  ವಿಭೂತಿ~।  ವಿನತಾ~।  ವಿವಿಧಾ~।  ವಿಮಲಾ~॥ 

ಓಂ ಕಮಲಾ~।  ಕಾಮಿನೀ~।  ಕಿರಾತೀ~।  ಕೀರ್ತಿ~।  ಕುರ್ದಿನೀ~।  ಕುಲಸುಂದರೀ~।  ಕಲ್ಯಾಣೀ~।  ಕಾಲಕೋಲಾ~।  

ಡಾಕಿನೀ~।  ರಾಕಿಣೀ~।  ಲಾಕಿನೀ~।  ಕಾಕಿನೀ~।  ಸಾಕಿನೀ~।  ಹಾಕಿನೀ~॥ 

ಓಂ ಇಚ್ಛಾಶಕ್ತಿ~।  ಜ್ಞಾನಶಕ್ತಿ~।  ಕ್ರಿಯಾಶಕ್ತಿ~॥ 

ಓಂ ಶರೇಭ್ಯೋ ~।  ಖಡ್ಗಾಯ ~।  ಅಂಕುಶಾಯ ~। ಪಾಶಾಯ ~।  ಗದಾಯೈ ~। ಚರ್ಮಣೇ ~।  ಧನುಷೇ ~।  ವಜ್ರಾಯ  ನಮಃ~॥

ಓಂ ಲಂ ಇಂದ್ರಶಕ್ತಿ~।  ರಂ ಅಗ್ನಿಶಕ್ತಿ~।  ಮಂ ಯಮಶಕ್ತಿ~।  ಕ್ಷಂ ನಿರ್ಋತಿಶಕ್ತಿ~।  ವಂ ವರುಣಶಕ್ತಿ~।  ಯಂ ವಾಯುಶಕ್ತಿ~।  ಕುಂ ಕುಬೇರಶಕ್ತಿ~।  ಹಂ ಈಶಾನಶಕ್ತಿ~।  ಆಂ ಬ್ರಹ್ಮಶಕ್ತಿ~।  ಹ್ರೀಂ ಅನಂತಶಕ್ತಿ~।  ನಿಯತಿಶಕ್ತಿ~।  ಕಾಲಶಕ್ತಿ~॥ 

ಓಂ ವಂ ವಜ್ರಶಕ್ತಿ~।  ಶಂ ಶಕ್ತಿಶಕ್ತಿ~।  ದಂ ದಂಡಶಕ್ತಿ~।  ಖಂ ಖಡ್ಗಶಕ್ತಿ~।  ಪಾಂ ಪಾಶಶಕ್ತಿ~।  ಅಂ ಅಂಕುಶಶಕ್ತಿ~।  ಗಂ ಗದಾಶಕ್ತಿ~।  ತ್ರಿಂ ತ್ರಿಶೂಲಶಕ್ತಿ~।  ಪಂ ಪದ್ಮಶಕ್ತಿ~।  ಚಂ ಚಕ್ರಶಕ್ತಿ~॥ 

ಭೇರುಂಡಾಯೈ ವಿದ್ಮಹೇ ವಿಷಹರಾಯೈ ಧೀಮಹಿ ।\\ತನ್ನೋ ನಿತ್ಯಾ ಪ್ರಚೋದಯಾತ್~।\\
ಇತಿ ಭೇರುಂಡಾಆವರಣಪೂಜಾ~।
\section{೫।ವಹ್ನಿವಾಸಿನೀನಿತ್ಯಾ}
ಶ್ರೀವಹ್ನಿವಾಸಿನೀನಿತ್ಯಾ ಮಂತ್ರಸ್ಯ ವಸಿಷ್ಠ ಋಷಿಃ। ಗಾಯತ್ರೀಛಂದಃ~।\\ ಶ್ರೀವಹ್ನಿವಾಸಿನೀನಿತ್ಯಾದೇವತಾ। ಹ್ರೀಂ ಬೀಜಂ। ನಮಃ ಶಕ್ತಿಃ।\\ ವಹ್ನಿವಾಸಿನ್ಯೈ ಕೀಲಕಂ~। ಹ್ರಾಂ ಇತ್ಯಾದಿನ್ಯಾಸಃ~।\\
{\bfseries ವಹ್ನಿಕೋಟಿಪ್ರತೀಕಾಶಾಂ ಸೂರ್ಯಕೋಟಿಸಮಪ್ರಭಾಂ~।\\
ಅಗ್ನಿಜ್ವಾಲಾಸಮಾಕೀರ್ಣಾಂ ಸರ್ವರೋಗಾಪಹಾರಿಣೀಂ~॥\\
ಕಾಲಮೃತ್ಯುಪ್ರಶಮನೀಂ ಭಯಮೃತ್ಯುನಿವಾರಿಣೀಂ~।\\
ಪರಮಾಯುಷ್ಯದಾಂ ವಂದೇ ನಿತ್ಯಾಂ ಶ್ರೀವಹ್ನಿವಾಸಿನೀಂ ॥\\}
ಮನುಃ :{\bfseries ಓಂ ಹ್ರೀಂ ವಹ್ನಿವಾಸಿನ್ಯೈ ನಮಃ~॥}

ಜ್ವಾಲಿನೀ~। ವಿಸ್ಫುಲಿಂಗಿನೀ~। ಮಂಗಲಾ~। ಮನೋಹರಾ~। ಕನಕಾ~। ಕಿತವಾ~। ವಿಶ್ವಾ~। ವಿವಿಧಾ~॥

ಮೇಷಾ~। ವೃಷಾ~। ಮಿಥುನಾ~। ಕರ್ಕಟಾ~। ಸಿಂಹಾ~। ಕನ್ಯಾ~। ತುಲಾ~। ಕೀಟಾ~। ಚಾಪಾ~। ಮಕರಾ~। ಕುಂಭಾ~। ಮೀನಾ~॥ 

ಘಸ್ಮರಾ~। ಸರ್ವಭಕ್ಷಾ~। ವಿಶ್ವಾ~। ವಿವಿಧೋದ್ಭವಾ~। ಚಿತ್ರಾ~। ನಿಃಸಪತ್ನಾ~।\\ ಪಾವನೀ~। ರಕ್ತಾ~। ನಿರಾತಂಕಾ~। ಅಚಿಂತ್ಯವೈಭವಾ~॥

ಓಂ ಲಂ ಇಂದ್ರಶಕ್ತಿ~।  ರಂ ಅಗ್ನಿಶಕ್ತಿ~।  ಮಂ ಯಮಶಕ್ತಿ~।  ಕ್ಷಂ ನಿರ್ಋತಿಶಕ್ತಿ~।  ವಂ ವರುಣಶಕ್ತಿ~।  ಯಂ ವಾಯುಶಕ್ತಿ~।  ಕುಂ ಕುಬೇರಶಕ್ತಿ~।  ಹಂ ಈಶಾನಶಕ್ತಿ~।  ಆಂ ಬ್ರಹ್ಮಶಕ್ತಿ~।  ಹ್ರೀಂ ಅನಂತಶಕ್ತಿ~।  ನಿಯತಿಶಕ್ತಿ~।  ಕಾಲಶಕ್ತಿ~॥ 

ಓಂ ವಂ ವಜ್ರಶಕ್ತಿ~।  ಶಂ ಶಕ್ತಿಶಕ್ತಿ~।  ದಂ ದಂಡಶಕ್ತಿ~।  ಖಂ ಖಡ್ಗಶಕ್ತಿ~।  ಪಾಂ ಪಾಶಶಕ್ತಿ~।  ಅಂ ಅಂಕುಶಶಕ್ತಿ~।  ಗಂ ಗದಾಶಕ್ತಿ~।  ತ್ರಿಂ ತ್ರಿಶೂಲಶಕ್ತಿ~।  ಪಂ ಪದ್ಮಶಕ್ತಿ~।  ಚಂ ಚಕ್ರಶಕ್ತಿ~॥ 

ವಹ್ನಿವಾಸಿನ್ಯೈ ವಿದ್ಮಹೇ ಸಿದ್ಧಿಪ್ರದಾಯೈ ಧೀಮಹಿ ।\\ತನ್ನೋ ನಿತ್ಯಾ ಪ್ರಚೋದಯಾತ್~।\\
ಇತಿ ವಹ್ನಿವಾಸಿನೀನಿತ್ಯಾ ಆವರಣಪೂಜಾ
\section{೬। ಮಹಾವಜ್ರೇಶ್ವರೀ}
ಶ್ರೀಮಹಾವಜ್ರೇಶ್ವರೀನಿತ್ಯಾ ಮಂತ್ರಸ್ಯ ಬ್ರಹ್ಮಾ ಋಷಿಃ। ಗಾಯತ್ರೀಛಂದಃ~। ಶ್ರೀಮಹಾವಜ್ರೇಶ್ವರೀನಿತ್ಯಾದೇವತಾ। ಹ್ರೀಂ ಬೀಜಂ। ಹ್ರೀಂ ಶಕ್ತಿಃ।\\
ಐಂ ಕೀಲಕಂ~।\\
\as{ನ್ಯಾಸಃ :}೧.ಓಂ ಹ್ರೀಂ ಕ್ಲಿನ್ನೇ ಹ್ರೀಂ  ೨.ಓಂ ಹ್ರೀಂ ಐಂ ಹ್ರೀಂ  ೩.ಓಂ ಹ್ರೀಂ ಕ್ರೋಂ ಹ್ರೀಂ ೪.ಓಂ ಹ್ರೀಂ ನಿತ್ಯ ಹ್ರೀಂ  ೫.ಓಂ ಹ್ರೀಂ ಮದ ಹ್ರೀಂ ೬.ಓಂ ಹ್ರೀಂ ದ್ರವೇ ಹ್ರೀಂ \\
{\bfseries ತಪ್ತಕಾಂಚನಸಂಕಾಶಾಂ ಕನಕಾಭರಣಾನ್ವಿತಾಂ~।\\
ಹೇಮತಾಟಂಕಸಂಯುಕ್ತಾಂ ಕಸ್ತೂರೀತಿಲಕಾನ್ವಿತಾಂ ॥\\
ಹೇಮಚಿಂತಾಕಸಂಯುಕ್ತಾಂ ಪೂರ್ಣಚಂದ್ರಮುಖಾಂಬುಜಾಂ~।\\
ಪೀತಾಂಬರಸಮೋಪೇತಾಂ ಪುಣ್ಯಮಾಲ್ಯವಿಭೂಷಿತಾಂ ॥\\
ಮುಕ್ತಾಹಾರಸಮೋಪೇತಾಂ ಮುಕುಟೇನ ವಿರಾಜಿತಾಂ~।\\
ಮಹಾವಜ್ರೇಶ್ವರೀಂ ವಂದೇ ಸರ್ವೈಶ್ವರ್ಯಫಲಪ್ರದಾಂ ॥\\}
ಮನುಃ :{\bfseries  ಓಂ ಹ್ರೀಂ ಕ್ಲಿನ್ನೇ ಐಂ ಕ್ರೋಂ ನಿತ್ಯಮದದ್ರವೇ ಹ್ರೀಂ~॥}

ಶೋಣಸಮುದ್ರಾಯ ನಮಃ~। ಕನಕಪೋತಾಯ ನಮಃ~। ರತ್ನಸಿಂಹಾಸನಾಯ ನಮಃ~॥

ಹೃಲ್ಲೇಖಾ~। ಕ್ಲೇದಿನೀ~। ಕ್ಲಿನ್ನಾ~। ಕ್ಷೋಭಿಣೀ~। ಮದನಾ~। ಮದನಾತುರಾ~। \\ನಿರಂಜನಾ।ರಾಗವತೀ।ಮದನಾವತೀ।ಮೇಖಲಾ। ದ್ರಾವಿಣೀ। ವೇಗವತೀ॥ 

ಕಮಲಾ। ಕಲ್ಪಾ। ಕಲಾ। ಕಲಿತಾ। ಕೌತುಕಾ। ಕಿರಾತಾ। ಕಾಲಾ। ಕದನಾ। ಕೌಶಿಕೀ~। ಕಂಬುವಾಹಿನೀ~। ಕಾತರಾ~। ಕಪಟಾ~। ಕೀರ್ತಿ~। ಕುಮಾರೀ~। ಕುಂಕುಮಾ~। ಭಂಜಿನೀ~। ವೇಗಿನೀ~। ಭೋಗಾ~। ಚಪಲಾ~। ಪೇಶಲಾ~। ಸತೀ~। ರತಿ~। ಶ್ರದ್ಧಾ~। ಭೋಗಲೋಲಾ~। ಮದಾ~। ಉನ್ಮತ್ತಾ~। ಮನಸ್ವಿನೀ~॥ 

ಓಂ ಲಂ ಇಂದ್ರಶಕ್ತಿ~।  ರಂ ಅಗ್ನಿಶಕ್ತಿ~।  ಮಂ ಯಮಶಕ್ತಿ~।  ಕ್ಷಂ ನಿರ್ಋತಿಶಕ್ತಿ~।  ವಂ ವರುಣಶಕ್ತಿ~।  ಯಂ ವಾಯುಶಕ್ತಿ~।  ಕುಂ ಕುಬೇರಶಕ್ತಿ~।  ಹಂ ಈಶಾನಶಕ್ತಿ~।  ಆಂ ಬ್ರಹ್ಮಶಕ್ತಿ~।  ಹ್ರೀಂ ಅನಂತಶಕ್ತಿ~।  ನಿಯತಿಶಕ್ತಿ~।  ಕಾಲಶಕ್ತಿ~॥

ಓಂ ವಂ ವಜ್ರಶಕ್ತಿ~।  ಶಂ ಶಕ್ತಿಶಕ್ತಿ~।  ದಂ ದಂಡಶಕ್ತಿ~।  ಖಂ ಖಡ್ಗಶಕ್ತಿ~।  ಪಾಂ ಪಾಶಶಕ್ತಿ~।  ಅಂ ಅಂಕುಶಶಕ್ತಿ~।  ಗಂ ಗದಾಶಕ್ತಿ~।  ತ್ರಿಂ ತ್ರಿಶೂಲಶಕ್ತಿ~।  ಪಂ ಪದ್ಮಶಕ್ತಿ~।  ಚಂ ಚಕ್ರಶಕ್ತಿ~॥

ಮಹಾವಜ್ರೇಶ್ವರ್ಯೈ ವಿದ್ಮಹೇ ವಜ್ರನಿತ್ಯಾಯೈ ಧೀಮಹಿ ।\\ತನ್ನೋ ನಿತ್ಯಾ ಪ್ರಚೋದಯಾತ್~॥\\
ಇತಿ ಮಹಾವಜ್ರೇಶ್ವರೀನಿತ್ಯಾ
\section{೭।ಶಿವಾದೂತೀನಿತ್ಯಾ}
ಶ್ರೀಶಿವಾದೂತೀನಿತ್ಯಾ ಮಂತ್ರಸ್ಯ ರುದ್ರಋಷಿಃ~। ಗಾಯತ್ರೀ ಛಂದಃ~।\\ ಶ್ರೀಶಿವಾದೂತೀನಿತ್ಯಾ ದೇವತಾ~। ಹ್ರೀಂ ಬೀಜಂ~। ನಮಃ ಶಕ್ತಿಃ~।\\ ಶಿವಾದೂತ್ಯೈ ಕೀಲಕಂ~। ಹ್ರಾಂ ಇತ್ಯಾದಿನಾ ನ್ಯಾಸಃ~।
{\bfseries ಬಾಲಸೂರ್ಯಪ್ರತೀಕಾಶಾಂ ಬಂಧೂಕಪ್ರಸವಾರುಣಾಂ~।\\
ವಿಧಿವಿಷ್ಣುಶಿವಸ್ತುತ್ಯಾಂ ದೇವಗಂಧರ್ವಸೇವಿತಾಂ ॥\\
ರಕ್ತಾರವಿಂದಸಂಕಾಶಾಂ ಸರ್ವಾಭರಣಭೂಷಿತಾಂ~।\\
ಶಿವದೂತೀಂ ನಮಸ್ಯಾಮಿ ರತ್ನಸಿಂಹಾಸನಸ್ಥಿತಾಂ ॥\\}
ಮನುಃ :{\bfseries  ಓಂ ಹ್ರೀಂ ಶಿವಾದೂತ್ಯೈ ನಮಃ~॥}

ಓಂ ವಿಹ್ವಲಾ~।  ಆಕರ್ಷಿಣೀ~।  ಲೋಲಾ~।  ನಿತ್ಯಾ~।  ಮದನಾ~।  ಮಾಲಿನೀ~। \\ವಿನೋದಾ~।  ಕೌತುಕಾ~।  ಪುಣ್ಯಾ~।  ಪುರಾಣಾ~॥ 

ಓಂ ವಾಗೀಶಾ~।  ವರದಾ~।  ವಿಶ್ವಾ~।  ವಿಭವಾ~।  ವಿಘ್ನಕಾರಿಣೀ~।  ವೀರಾ~। \\ವಿಘ್ನಹರಾ~।  ವಿದ್ಯಾ~॥ 

ಓಂ ಸುಮುಖೀ~।  ಸುಂದರೀ~।  ಸಾರಾ~।  ಸಮರಾ~।  ಸರಸ್ವತೀ~।  ಸಮಯಾ~।  ಸರ್ವಗಾ~।  ಸಿದ್ಧಾ~॥ 

ಓಂ ಡಾಕಿನೀ~।  ರಾಕಿಣೀ~।  ಲಾಕಿನೀ~।  ಕಾಕಿನೀ~।  ಸಾಕಿನೀ~।  ಹಾಕಿನೀ~॥ 

ಓಂ ಶಿವಾ~। ವಾಣೀ।ದೂರಸಿದ್ಧಾ।ತ್ಯೈವಿಗ್ರಹವತೀ~। ನಾದಾ।  ಮನೋನ್ಮನೀ॥ 

ಓಂ ಇಚ್ಛಾಶಕ್ತಿ~।  ಜ್ಞಾನಶಕ್ತಿ~।  ಕ್ರಿಯಾಶಕ್ತಿ~॥ 

ಓಂ ಕಮಲಶಕ್ತಿ~।  ಕುಠಾರಶಕ್ತಿ~।  ಖಡ್ಗ ಶಕ್ತಿ~।  ಅಂಕುಶ ಶಕ್ತಿ~।  ರತ್ನಚಷಕ ಶಕ್ತಿ~।\\ಗದಾಶಕ್ತಿ~।  ಖೇಟಶಕ್ತಿ~।  ಪಾಶಶಕ್ತಿ~॥ 

ಓಂ ಲಂ ಇಂದ್ರಶಕ್ತಿ~।  ರಂ ಅಗ್ನಿಶಕ್ತಿ~।  ಮಂ ಯಮಶಕ್ತಿ~।  ಕ್ಷಂ ನಿರ್ಋತಿಶಕ್ತಿ~।  ವಂ ವರುಣಶಕ್ತಿ~।  ಯಂ ವಾಯುಶಕ್ತಿ~।  ಕುಂ ಕುಬೇರಶಕ್ತಿ~।  ಹಂ ಈಶಾನಶಕ್ತಿ~।  ಆಂ ಬ್ರಹ್ಮಶಕ್ತಿ~।  ಹ್ರೀಂ ಅನಂತಶಕ್ತಿ~।  ನಿಯತಿಶಕ್ತಿ~।  ಕಾಲಶಕ್ತಿ~॥

ಓಂ ವಂ ವಜ್ರಶಕ್ತಿ~।  ಶಂ ಶಕ್ತಿಶಕ್ತಿ~।  ದಂ ದಂಡಶಕ್ತಿ~।  ಖಂ ಖಡ್ಗಶಕ್ತಿ~।  ಪಾಂ ಪಾಶಶಕ್ತಿ~।  ಅಂ ಅಂಕುಶಶಕ್ತಿ~।  ಗಂ ಗದಾಶಕ್ತಿ~।  ತ್ರಿಂ ತ್ರಿಶೂಲಶಕ್ತಿ~।  ಪಂ ಪದ್ಮಶಕ್ತಿ~।  ಚಂ ಚಕ್ರಶಕ್ತಿ~॥ 

ಶಿವಾದೂತ್ಯೈ ವಿದ್ಮಹೇ ಶಿವಂಕರ್ಯೈ ಧೀಮಹಿ~। ತನ್ನೋ ನಿತ್ಯಾ ಪ್ರಚೋದಯಾತ್~॥\\
ಇತಿ ಶಿವಾದೂತೀನಿತ್ಯಾ ಆವರಣಪೂಜಾ
\section{೮। ತ್ವರಿತಾ ನಿತ್ಯಾ}
ಅಸ್ಯ ಶ್ರೀತ್ವರಿತಾನಿತ್ಯಾ ಮಹಾಮಂತ್ರಸ್ಯ ಈಶ್ವರ ಋಷಿಃ~। ವಿರಾಟ್ ಛಂದಃ~। ತ್ವರಿತಾನಿತ್ಯಾ ದೇವತಾ। ಹೂಂ ಬೀಜಂ~। ಸ್ತ್ರೀಂ ಶಕ್ತಿಃ~। ಹ್ರೀಂ ಕೀಲಕಂ~।\\
ಹ್ರಾಂ ಇತ್ಯಾದಿನ್ಯಾಸಃ ।\\
{\bfseries ರಕ್ತಾರವಿಂದಸಂಕಾಶಾಮುದ್ಯತ್ಸೂರ್ಯಸಮಪ್ರಭಾಂ~।\\
ದಧತೀಮಂಕುಶಂ ಪಾಶಂ ಬಾಣಾನ್ ಚಾಪಂ ಮನೋಹರಂ ॥\\
ಚತುರ್ಭುಜಾಂ ಮಹಾದೇವೀಮಪ್ಸರೋಗಣಸಂಕುಲಾಂ~।\\
ನಮಾಮಿ ತ್ವರಿತಾಂ ನಿತ್ಯಾಂ ಭಕ್ತಾನಾಮಭಯಪ್ರದಾಂ ॥\\}
ಮನುಃ :{\bfseries  ಓಂ ಹ್ರೀಂ ಹೂಂ ಖೇ ಚ ಛೇ ಕ್ಷಃ ಸ್ತ್ರೀಂ ಹೂಂ ಕ್ಷೇ ಹ್ರೀಂ ಫಟ್~॥}

ಓಂ ಫಟ್ಕಾರೀ~। ಕಿಂಕರ~। ಜಯಾ~। ವಿಜಯಾ~। ಹುಂಕಾರೀ~। ಖೇಚರೀ~। ಚಂಡಾ~। ಛೇದಿನೀ~। ಕ್ಷೇದಿನೀ~। ಕ್ಷೇಪಿಣೀ~। ಸ್ತ್ರೀಕಾರೀ~। ಹುಂಕಾರೀ~। ಕ್ಷೇಮಕರೀ~॥

ಓಂ ಲಂ ಇಂದ್ರಶಕ್ತಿ~।  ರಂ ಅಗ್ನಿಶಕ್ತಿ~।  ಮಂ ಯಮಶಕ್ತಿ~।  ಕ್ಷಂ ನಿರ್ಋತಿಶಕ್ತಿ~।  ವಂ ವರುಣಶಕ್ತಿ~।  ಯಂ ವಾಯುಶಕ್ತಿ~।  ಕುಂ ಕುಬೇರಶಕ್ತಿ~।  ಹಂ ಈಶಾನಶಕ್ತಿ~।  ಆಂ ಬ್ರಹ್ಮಶಕ್ತಿ~।  ಹ್ರೀಂ ಅನಂತಶಕ್ತಿ~।  ನಿಯತಿಶಕ್ತಿ~।  ಕಾಲಶಕ್ತಿ~॥ 

ಓಂ ವಂ ವಜ್ರಶಕ್ತಿ~।  ಶಂ ಶಕ್ತಿಶಕ್ತಿ~।  ದಂ ದಂಡಶಕ್ತಿ~।  ಖಂ ಖಡ್ಗಶಕ್ತಿ~।  ಪಾಂ ಪಾಶಶಕ್ತಿ~।  ಅಂ ಅಂಕುಶಶಕ್ತಿ~।  ಗಂ ಗದಾಶಕ್ತಿ~।  ತ್ರಿಂ ತ್ರಿಶೂಲಶಕ್ತಿ~।  ಪಂ ಪದ್ಮಶಕ್ತಿ~।  ಚಂ ಚಕ್ರಶಕ್ತಿ~॥ 

ಓಂ ತ್ವರಿತಾಯೈ ವಿದ್ಮಹೇ ಮಹಾನಿತ್ಯಾಯೈ ಧೀಮಹಿ~। ತನ್ನೋ ನಿತ್ಯಾ ಪ್ರಚೋದಯಾತ್~॥\\
ಇತಿ ತ್ವರಿತಾ ಆವರಣಪೂಜಾ
\section{೯। ಕುಲಸುಂದರೀನಿತ್ಯಾ}
ಅಸ್ಯ ಶ್ರೀ ಕುಲಸುಂದರೀ ನಿತ್ಯಾ ಮಹಾಮಂತ್ರಸ್ಯ ದಕ್ಷಿಣಾಮೂರ್ತಿಃ ಋಷಿಃ~। ಪಂಕ್ತಿಶ್ಛಂದಃ~। ಶ್ರೀ ಕುಲಸುಂದರೀ ನಿತ್ಯಾ ದೇವತಾ~। ಐಂ ಬೀಜಂ~। ಸೌಃ ಶಕ್ತಿಃ~। ಕ್ಲೀಂ ಕೀಲಕಂ~। ಆಂ , ಈಂ , ಇತ್ಯಾದಿನ್ಯಾಸಃ~।\\
{\bfseries ಅರುಣಕಿರಣಜಾಲೈ ರಂಜಿತಾಶಾವಕಾಶಾ\\
ವಿಧೃತಜಪವಟೀಕಾ ಪುಸ್ತಕಾಭೀತಿಹಸ್ತಾ ॥\\
ಇತರಕರವರಾಢ್ಯಾ ಫುಲ್ಲಕಹ್ಲಾರಸಂಸ್ಥಾ\\
ನಿವಸತು ಹೃದಿ ಬಾಲಾ ನಿತ್ಯಕಲ್ಯಾಣಶೀಲಾ ॥\\}
ಮನುಃ :{\bfseries  ಐಂ ಕ್ಲೀಂ ಸೌಃ~॥}

ಓಂ ವಾಮಾಯೈ~। ಜ್ಯೇಷ್ಠಾಯೈ~। ರೌದ್ರ್ಯೈ~। ಅಂಬಿಕಾಯೈ~। ಇಚ್ಛಾಯೈ~। \\ಜ್ಞಾನಾಯೈ~। ಕ್ರಿಯಾಯೈ~। ಕುಲಿಕಾಯೈ~। ಚಿತ್ರಾಯೈ~। ವಿಷಘ್ನ್ಯೈ~। ದೂತ್ಯೈ~। ಆನಂದಾಯೈ ನಮಃ~॥

ಓಂ ಭಾಷಾಪಾದುಕಾಂ ಪೂಜಯಾಮಿ~।  ಸರಸ್ವತೀ ಪಾದುಕಾಂ ಪೂಜಯಾಮಿ॥ 

ಓಂ ವಾಣೀ~।  ಸಂಸ್ಕೃತಾ~।  ಪ್ರಾಕೃತಾ~।  ಪರಾ~।  ಬಹುರೂಪಾ~।  ಚಿತ್ರರೂಪಾ~।  ರಮ್ಯಾ~।  ಆನಂದಾ~।  ಕೌತುಕಾ~॥ 

ಓಂ ಬ್ರಾಹ್ಮೀ~।  ಮಾಹೇಶ್ವರೀ~।  ಕೌಮಾರೀ~।  ವೈಷ್ಣವೀ~।  ವಾರಾಹೀ~। \\ ಮಾಹೇಂದ್ರೀ~।  ಚಾಮುಂಡಾ~।  ಮಹಾಲಕ್ಷ್ಮೀ~॥ 

ಓಂ ಲಂ ಇಂದ್ರಶಕ್ತಿ~।  ರಂ ಅಗ್ನಿಶಕ್ತಿ~।  ಮಂ ಯಮಶಕ್ತಿ~।  ಕ್ಷಂ ನಿರ್ಋತಿಶಕ್ತಿ~।  ವಂ ವರುಣಶಕ್ತಿ~।  ಯಂ ವಾಯುಶಕ್ತಿ~।  ಕುಂ ಕುಬೇರಶಕ್ತಿ~।  ಹಂ ಈಶಾನಶಕ್ತಿ~।  ಆಂ ಬ್ರಹ್ಮಶಕ್ತಿ~।  ಹ್ರೀಂ ಅನಂತಶಕ್ತಿ~।  ನಿಯತಿಶಕ್ತಿ~।  ಕಾಲಶಕ್ತಿ~॥ 

ಓಂ ವಂ ವಜ್ರಶಕ್ತಿ~।  ಶಂ ಶಕ್ತಿಶಕ್ತಿ~।  ದಂ ದಂಡಶಕ್ತಿ~।  ಖಂ ಖಡ್ಗಶಕ್ತಿ~।  ಪಾಂ ಪಾಶಶಕ್ತಿ~।  ಅಂ ಅಂಕುಶಶಕ್ತಿ~।  ಗಂ ಗದಾಶಕ್ತಿ~।  ತ್ರಿಂ ತ್ರಿಶೂಲಶಕ್ತಿ~।  ಪಂ ಪದ್ಮಶಕ್ತಿ~।  ಚಂ ಚಕ್ರಶಕ್ತಿ~॥ 

ಕುಲಸುಂದರ್ಯೈ ವಿದ್ಮಹೇ ಕಾಮೇಶ್ವರ್ಯೈ ಧೀಮಹಿ ತನ್ನೋ ನಿತ್ಯಾ ಪ್ರಚೋದಯಾತ್~॥\\
ಇತಿ ಕುಲಸುಂದರೀ ಆವರಣಪೂಜಾ
\section{೧೦।ನಿತ್ಯಾನಿತ್ಯಾ}
ಅಸ್ಯ ಶ್ರೀನಿತ್ಯಾನಿತ್ಯಾಮಹಾಮಂತ್ರಸ್ಯ ದಕ್ಷಿಣಾಮೂರ್ತಿಃ ಋಷಿಃ।\\ ಪಂಕ್ತಿಃ ಛಂದಃ~। ಶ್ರೀನಿತ್ಯಾನಿತ್ಯಾದೇವತಾ। ಐಂ ಬೀಜಂ। ಔಃ ಶಕ್ತಿಃ~।\\ ಈಂ ಕೀಲಕಂ~। ಹ್ರೀಂ ಹ್ಸಾಂ, ಹ್ರೀಂ ಹ್ಸೀಂ ಇತ್ಯಾದಿನಾ ನ್ಯಾಸಃ~।

{\bfseries ಉದ್ಯತ್ಪ್ರದ್ಯೋತನನಿಭಾಂ ಜಪಾಕುಸುಮಸನ್ನಿಭಾಂ~।\\
ಹರಿಚಂದನಲಿಪ್ತಾಂಗೀಂ ರಕ್ತಮಾಲ್ಯ ವಿಭೂಷಿತಾಂ ॥\\
ರತ್ನಾಭರಣಭೂಷಾಂಗೀಂ ರಕ್ತವಸ್ತ್ರಸುಶೋಭಿತಾಂ~।\\
ಜಗದಂಬಾಂ ನಮಸ್ಯಾಮಿ ನಿತ್ಯಾಂ ಶ್ರೀಪರಮೇಶ್ವರೀಂ ॥\\}
ಮನುಃ :{\bfseries  ಓಂ ಹಸಕಲರಡೈಂ ಹಸಕಲರಡೀಂ ಹಸಕಲರಡೌಃ ॥}

ಓಂ ಡಾಕಿನೀ~।  ರಾಕಿಣೀ~।  ಲಾಕಿನೀ~।  ಕಾಕಿನೀ~।  ಸಾಕಿನೀ~।  ಹಾಕಿನೀ~॥ 

ಓಂ ಅ ಶಕ್ತಿ~।  ಆ ಶಕ್ತಿ~।  ಇ ಶಕ್ತಿ~।  ಈ ಶಕ್ತಿ~।  ಉ ಶಕ್ತಿ~।  ಊ ಶಕ್ತಿ~।  ಋ ಶಕ್ತಿ *********************ಳ ಶಕ್ತಿ~।  ಕ್ಷ ಶಕ್ತಿ~॥ 

ಓಂ ಲಂ ಇಂದ್ರಶಕ್ತಿ~।  ರಂ ಅಗ್ನಿಶಕ್ತಿ~।  ಮಂ ಯಮಶಕ್ತಿ~।  ಕ್ಷಂ ನಿರ್ಋತಿಶಕ್ತಿ~।  ವಂ ವರುಣಶಕ್ತಿ~।  ಯಂ ವಾಯುಶಕ್ತಿ~।  ಕುಂ ಕುಬೇರಶಕ್ತಿ~।  ಹಂ ಈಶಾನಶಕ್ತಿ~।  ಆಂ ಬ್ರಹ್ಮಶಕ್ತಿ~।  ಹ್ರೀಂ ಅನಂತಶಕ್ತಿ~।  ನಿಯತಿಶಕ್ತಿ~।  ಕಾಲಶಕ್ತಿ~॥ 

ಓಂ ವಂ ವಜ್ರಶಕ್ತಿ~।  ಶಂ ಶಕ್ತಿಶಕ್ತಿ~।  ದಂ ದಂಡಶಕ್ತಿ~।  ಖಂ ಖಡ್ಗಶಕ್ತಿ~।  ಪಾಂ ಪಾಶಶಕ್ತಿ~।  ಅಂ ಅಂಕುಶಶಕ್ತಿ~।  ಗಂ ಗದಾಶಕ್ತಿ~।  ತ್ರಿಂ ತ್ರಿಶೂಲಶಕ್ತಿ~।  ಪಂ ಪದ್ಮಶಕ್ತಿ~।  ಚಂ ಚಕ್ರಶಕ್ತಿ~॥ 

ಓಂ ಅಭಯಶಕ್ತಿ~।  ಖಡ್ಗಶಕ್ತಿ। ಪುಸ್ತಕಶಕ್ತಿ। ಪಾಶಶಕ್ತಿ।ಇಕ್ಷುಚಾಪಶಕ್ತಿ। ತ್ರಿಶೂಲಶಕ್ತಿ~।  ಕಪಾಲಶಕ್ತಿ~।  ಇಷುಶಕ್ತಿ~।  ಅಂಕುಶಶಕ್ತಿ~।  ಅಕ್ಷಗುಣಶಕ್ತಿ~।  ಖೇಟಕಶಕ್ತಿ~।  ವರಶಕ್ತಿ~॥

ನಿತ್ಯಭೈರವ್ಯೈ ವಿದ್ಮಹೇ ನಿತ್ಯಾನಿತ್ಯಾಯೈ ಧೀಮಹಿ ।\\ತನ್ನೋ ನಿತ್ಯಾ ಪ್ರಚೋದಯಾತ್ ॥\\
ಇತಿ ನಿತ್ಯಾನಿತ್ಯಾ ಆವರಣಪೂಜಾ~।
\section{೧೧।ಶ್ರೀನೀಲಪತಾಕಾನಿತ್ಯಾ}
ಅಸ್ಯ ಶ್ರೀನೀಲಪತಾಕಾನಿತ್ಯಾ ಮಹಾಮಂತ್ರಸ್ಯ ಸಮ್ಮೋಹನ ಋಷಿಃ~। \\ಗಾಯತ್ರೀ ಛಂದಃ~। ಶ್ರೀನೀಲಪತಾಕಾನಿತ್ಯಾ ದೇವತಾ~। ಹ್ರೀಂ ಬೀಜಂ~।\\ ಹ್ರೀಂ ಶಕ್ತಿಃ~। ಕ್ಲೀಂ ಕೀಲಕಂ~।\\
\as{ನ್ಯಾಸಃ :}೧.ಓಂ ಓಂ ಹ್ರೀಂ ಫ್ರೇಂ ೨.ಓಂ ಸ್ರೂಂ ಓಂ ಆಂ ಕ್ಲೀಂ ೩.ಓಂ ಐಂ ಬ್ಲೂಂ ನಿತ್ಯಮದ ೪.ಓಂ ದ್ರ ೫.ಓಂ ವೇ ೬.ಓಂ ಹುಂ \\
{\bfseries ಪಂಚವಕ್ತ್ರಾಂ ತ್ರಿಣಯನಾಮರುಣಾಂಶುಕಧಾರಿಣೀಂ~।\\
ದಶಹಸ್ತಾಂ ಲಸನ್ಮುಕ್ತಾಪ್ರಾಯಾಭರಣಮಂಡಿತಾಂ ॥\\
ನೀಲಮೇಘಸಮಪ್ರಖ್ಯಾಂ ಧೂಮ್ರಾರ್ಚಿಸ್ಸದೃಶಪ್ರಭಾಂ~।\\
ನೀಲಪುಷ್ಪಸ್ರಜೋಪೇತಾಂ ಧ್ಯಾಯೇನ್ನೀಲಪತಾಕಿನೀಂ ॥\\}
ಮನುಃ :{\bfseries ಓಂ ಹ್ರೀಂ ಫ್ರೇಂ ಸ್ರೂಂ ಓಂ ಆಂ ಕ್ಲೀಂ ಐಂ ಬ್ಲೂಂ ನಿತ್ಯಮದದ್ರವೇ ಹುಂ ಫ್ರೇಂ ಹ್ರೀಂ~॥}

ಓಂ ಅಭಯಾಯ ನಮಃ~।  ಬಾಣಾಯ ನಮಃ~।  ಖಡ್ಗಾಯ ನಮಃ~।  ಶಕ್ತಯೇ ನಮಃ~।  ಅಂಕುಶಾಯ ನಮಃ~।  ಪಾಶಾಯ ನಮಃ~।  ಪತಾಕಾಯ ನಮಃ~। \\ ಚರ್ಮಣೇ ನಮಃ~।  ಶಾರ್ಙ್ಗಚಾಪಾಯ ನಮಃ~।  ವರಾಯ ನಮಃ~॥

ಓಂ ಇಚ್ಛಾಶಕ್ತಿ~।  ಜ್ಞಾನಶಕ್ತಿ~।  ಕ್ರಿಯಾಶಕ್ತಿ~॥ 

ಓಂ ಡಾಕಿನೀ~।  ರಾಕಿಣೀ~।  ಲಾಕಿನೀ~।  ಕಾಕಿನೀ~।  ಸಾಕಿನೀ~।  ಹಾಕಿನೀ~॥ 

ಓಂ ಬ್ರಾಹ್ಮೀ~।  ಮಾಹೇಶ್ವರೀ~।  ಕೌಮಾರೀ~।  ವೈಷ್ಣವೀ~।  ವಾರಾಹೀ~। \\ ಮಾಹೇಂದ್ರೀ~।  ಚಾಮುಂಡಾ~॥ 

ಓಂ ಸುಮುಖೀ~।  ಸುಂದರೀ~।  ಸಾರಾ~।  ಸುಮನಾ~।  ಸರಸ್ವತೀ~।  ಸಮಯಾ~।  ಸರ್ವಗಾ~।  ಸಿದ್ಧಾ~।  ವಿಹ್ವಲಾ~।  ಲೋಲಾ~।  ಮದನಾ~।  ವಿನೋದಾ~।  ಪುಣ್ಯಾ~।  ಆಕರ್ಷಿಣೀ~।  ನಿತ್ಯಾ~।  ಮಾಲಿನೀ~।  ಕೌತುಕಾ~।  ಪುರಾಣಾ~॥ 

ಓಂ ಲಂ ಇಂದ್ರಶಕ್ತಿ~।  ರಂ ಅಗ್ನಿಶಕ್ತಿ~।  ಮಂ ಯಮಶಕ್ತಿ~।  ಕ್ಷಂ ನಿರ್ಋತಿಶಕ್ತಿ~।  ವಂ ವರುಣಶಕ್ತಿ~।  ಯಂ ವಾಯುಶಕ್ತಿ~।  ಕುಂ ಕುಬೇರಶಕ್ತಿ~।  ಹಂ ಈಶಾನಶಕ್ತಿ~।  ಆಂ ಬ್ರಹ್ಮಶಕ್ತಿ~।  ಹ್ರೀಂ ಅನಂತಶಕ್ತಿ~।  ನಿಯತಿಶಕ್ತಿ~।  ಕಾಲಶಕ್ತಿ~॥ 

ಓಂ ವಂ ವಜ್ರಶಕ್ತಿ~।  ಶಂ ಶಕ್ತಿಶಕ್ತಿ~।  ದಂ ದಂಡಶಕ್ತಿ~।  ಖಂ ಖಡ್ಗಶಕ್ತಿ~।  ಪಾಂ ಪಾಶಶಕ್ತಿ~।  ಅಂ ಅಂಕುಶಶಕ್ತಿ~।  ಗಂ ಗದಾಶಕ್ತಿ~।  ತ್ರಿಂ ತ್ರಿಶೂಲಶಕ್ತಿ~। ಪಂ ಪದ್ಮಶಕ್ತಿ~।  ಚಂ ಚಕ್ರಶಕ್ತಿ~॥

ನೀಲಪತಾಕಾಯೈ ವಿದ್ಮಹೇ ಮಹಾನಿತ್ಯಾಯೈ ಧೀಮಹಿ ।\\ತನ್ನೋ ನಿತ್ಯಾ ಪ್ರಚೋದಯಾತ್~॥\\
ಇತಿ ನೀಲಪತಾಕಾನಿತ್ಯಾ ಆವರಣಂ
\newpage
\section{೧೨। ವಿಜಯಾ ನಿತ್ಯಾ}
ಅಸ್ಯ ಶ್ರೀ ವಿಜಯಾನಿತ್ಯಾಮಹಾಮಂತ್ರಸ್ಯ ಅಹಿರ್ಋಷಿಃ~। ಗಾಯತ್ರೀಛಂದಃ~। ಶ್ರೀವಿಜಯಾನಿತ್ಯಾ ದೇವತಾ~।\\
\as{ನ್ಯಾಸಃ :}೧.ಓಂ ಭಾಂ ೨.ಓಂ ಮೀಂ ೩.ಓಂ ರೂಂ ೪.ಓಂ ಯೈಂ ೫.ಓಂ ಉಂ ೬.ಓಂ ಔಂ \\
{\bfseries ಉದ್ಯದರ್ಕಸಹಸ್ರಾಭಾಂ ದಾಡಿಮೀಪುಷ್ಪಸನ್ನಿಭಾಂ~।\\
ರಕ್ತಕಂಕಣಕೇಯೂರಕಿರೀಟಾಂಗದಸಂಯುತಾಂ ॥\\
ದೇವಗಂಧರ್ವಯೋಗೀಶಮುನಿಸಿದ್ಧನಿಷೇವಿತಾಂ~।\\
ನಮಾಮಿ ವಿಜಯಾಂ ನಿತ್ಯಾಂ ಸಿಂಹೋಪರಿ ಕೃತಾಸನಾಂ ॥\\}
ಮನುಃ :{\bfseries  ಭ ಮ ರ ಯ ಉ ಔಂ ॥}

ಓಂ ಲಂ ಇಂದ್ರಶಕ್ತಿ~।  ರಂ ಅಗ್ನಿಶಕ್ತಿ~।  ಮಂ ಯಮಶಕ್ತಿ~।  ಕ್ಷಂ ನಿರ್ಋತಿಶಕ್ತಿ~।  ವಂ ವರುಣಶಕ್ತಿ~।  ಯಂ ವಾಯುಶಕ್ತಿ~।  ಕುಂ ಕುಬೇರಶಕ್ತಿ~।  ಹಂ ಈಶಾನಶಕ್ತಿ~।  ಆಂ ಬ್ರಹ್ಮಶಕ್ತಿ~।  ಹ್ರೀಂ ಅನಂತಶಕ್ತಿ~।  ನಿಯತಿಶಕ್ತಿ~।  ಕಾಲಶಕ್ತಿ~॥ 

ಓಂ ವಂ ವಜ್ರಶಕ್ತಿ~।  ಶಂ ಶಕ್ತಿಶಕ್ತಿ~।  ದಂ ದಂಡಶಕ್ತಿ~।  ಖಂ ಖಡ್ಗಶಕ್ತಿ~।  ಪಾಂ ಪಾಶಶಕ್ತಿ~।  ಅಂ ಅಂಕುಶಶಕ್ತಿ~।  ಗಂ ಗದಾಶಕ್ತಿ~।  ತ್ರಿಂ ತ್ರಿಶೂಲಶಕ್ತಿ~।  ಪಂ ಪದ್ಮಶಕ್ತಿ~।  ಚಂ ಚಕ್ರಶಕ್ತಿ~॥ 

ಓಂ ಜಯಾ~।  ವಿಜಯಾ~।  ದುರ್ಗಾ~।  ಭದ್ರಾ~।  ಭದ್ರಕರೀ~।  ಕ್ಷೇಮಾ~।  \\ಕ್ಷೇಮಕರೀ~।  ನಿತ್ಯಾ~॥ 

ಓಂ ವಿದಾರಿಕಾ~।  ವಿಶ್ವಮಯೀ~।  ವಿಶ್ವಾ।  ವಿಶ್ವವಿಭಂಜಿಕಾ।  ವೀರಾ। ವಿಕ್ಷೋಭಿಣೀ।  ವಿದ್ಯಾ~।  ವಿನೋದಾ~।  ಅಂಚಿತವಿಗ್ರಹಾ~।  ವೀತಶೋಕಾ~।  ವಿಷಗ್ರೀವಾ~।  ವಿಪುಲಾ~।  ವಿಜಯಪ್ರದಾ~।  ವಿಭವಾ~।  ವಿವಿಧಾ~।  ವಿಪ್ರಾ~॥ 

ಓಂ ಮನೋಹರಾ~।  ಮಂಗಲಾ~।  ಮದೋತ್ಸಿಕ್ತಾ~।  ಮನಸ್ವಿನೀ~।  ಮಾನಿನೀ~।  ಮಧುರಾ~।  ಮಾಯಾ~।  ಮೋಹಿನೀ~॥ 

ಓಂ ಮಾತುಲುಂಗಾಯ~।ಓಂ ಸಾಯಕೇಭ್ಯಃ~।ಓಂ ಖಡ್ಗಾಯ~।ಓಂ ಅಂಕುಶಾಯ~।\\ಓಂ ಚಕ್ರಾಯ~।ಓಂ ಶಂಖಾಯ~।ಓಂ ಪಾಶಾಯ~।ಓಂ ಖೇಟಾಯ~।\\ಓಂ ಚಾಪಾಯ~।  ಕಹ್ಲಾರಾಯ ನಮಃ~॥

ವಿಜಯಾದೇವ್ಯೈ ವಿದ್ಮಹೇ ಮಹಾನಿತ್ಯಾಯೈ ಧೀಮಹಿ ತನ್ನೋ ನಿತ್ಯಾ ಪ್ರಚೋದಯಾತ್~॥\\
ಇತಿ ವಿಜಯಾನಿತ್ಯಾ ಆವರಣಪೂಜಾ~।
\section{೧೩। ಸರ್ವಮಂಗಲಾನಿತ್ಯಾ}
ಅಸ್ಯ ಶ್ರೀಸರ್ವಮಂಗಲಾನಿತ್ಯಾ ಮಹಾಮಂತ್ರಸ್ಯ ಚಂದ್ರ ಋಷಿಃ~।\\ ಗಾಯತ್ರೀ ಛಂದಃ~। ಸರ್ವಮಂಗಲಾನಿತ್ಯಾ ದೇವತಾ~।\\
ಸ್ವಾಂ , ಸ್ವೀಂ ಇತ್ಯಾದಿನಾ ನ್ಯಾಸಃ~।\\
{\bfseries ರಕ್ತೋತ್ಪಲಸಮಪ್ರಖ್ಯಾಂ ಪದ್ಮಪತ್ರನಿಭೇಕ್ಷಣಾಂ~।\\
ಇಕ್ಷುಕಾರ್ಮುಕಪುಷ್ಪೌಘಪಾಶಾಂಕುಶಸಮನ್ವಿತಾಂ ॥\\
ಸುಪ್ರಸನ್ನಾಂ ಶಶಿಮುಖೀಂ ನಾನಾರತ್ನವಿಭೂಷಿತಾಂ~।\\
ಶುಭ್ರಪದ್ಮಾಸನಸ್ಥಾಂ ತಾಂ ಭಜಾಮಿ ಸರ್ವಮಂಗಲಾಂ ॥\\}
ಮನುಃ :{\bfseries  ಸ್ವೌಂ ॥}

ಓಂ ಭದ್ರಾ~।  ಭವಾನೀ~।  ಭವ್ಯಾ~।  ವಿಶಾಲಾಕ್ಷೀ~।  ಶುಚಿಸ್ಮಿತಾ~।  ಕುಂಕುಮಾ~।  ಕಮಲಾ~।  ಕಲ್ಪಾ~॥ 

ಓಂ ಕಲಾ~।  ಪೂರಣೀ~।  ನಿತ್ಯಾ~।  ಅಮೃತಾ~।  ಜೀವಿತಾ~।  ದಯಾ~।  ಅಶೋಕಾ~।  ಅಮಲಾ~।  ಪೂರ್ಣಾ~।  ಪುಣ್ಯಾ~।  ಭಾಗ್ಯಾ~।  ಉದ್ಯತಾ~।  ವಿವೇಕಾ~।  ವಿಭವಾ~।  ವಿಶ್ವಾ~।  ವಿನತಾ~॥ 

ಓಂ ಕಾಮಿನೀ~।  ಖೇಚರೀ~।  ಆರ್ಯಾ~।  ಪುರಾಣಾ~।  ಪರಮೇಶ್ವರೀ~।  ಗೌರೀ~।  ಶಿವಾ~।  ಅಮೇಯಾ~।  ವಿಮಲಾ~।  ವಿಜಯಾ~।  ಪರಾ~।  ಪವಿತ್ರಾ~।  ಪದ್ಮಿನೀ~।  ವಿದ್ಯಾ~।  ವಿಶ್ವೇಶೀ। ಶಿವವಲ್ಲಭಾ।  ಅಶೇಷರೂಪಾ।  ಆನಂದಾ।  ಅಂಬುಜಾಕ್ಷೀ~।  ಅನಿಂದಿತಾ~।  ವರದಾ~।  ವಾಕ್ಪ್ರದಾ~।  ವಾಣೀ।  ವಿವಿಧಾ।  ವೇದವಿಗ್ರಹಾ~।  ವಂದ್ಯಾ~।  ವಾಗೀಶ್ವರೀ~।  ಸತ್ಯಾ~।  ಸಂಯತಾ~।  ಸರಸ್ವತೀ~।  ನಿರ್ಮಲಾ~।  ನಾದರೂಪಾ~॥ 

ಓಂ ಲಂ ಇಂದ್ರಶಕ್ತಿ~।  ರಂ ಅಗ್ನಿಶಕ್ತಿ~।  ಮಂ ಯಮಶಕ್ತಿ~।  ಕ್ಷಂ ನಿರ್ಋತಿಶಕ್ತಿ~।  ವಂ ವರುಣಶಕ್ತಿ~।  ಯಂ ವಾಯುಶಕ್ತಿ~।  ಕುಂ ಕುಬೇರಶಕ್ತಿ~।  ಹಂ ಈಶಾನಶಕ್ತಿ~।  ಆಂ ಬ್ರಹ್ಮಶಕ್ತಿ~।  ಹ್ರೀಂ ಅನಂತಶಕ್ತಿ~।  ನಿಯತಿಶಕ್ತಿ~।  ಕಾಲಶಕ್ತಿ~॥ 

ಓಂ ವಂ ವಜ್ರಶಕ್ತಿ~।  ಶಂ ಶಕ್ತಿಶಕ್ತಿ~।  ದಂ ದಂಡಶಕ್ತಿ~।  ಖಂ ಖಡ್ಗಶಕ್ತಿ~।  ಪಾಂ ಪಾಶಶಕ್ತಿ~।  ಅಂ ಅಂಕುಶಶಕ್ತಿ~।  ಗಂ ಗದಾಶಕ್ತಿ~।  ತ್ರಿಂ ತ್ರಿಶೂಲಶಕ್ತಿ~।  ಪಂ ಪದ್ಮಶಕ್ತಿ~।  ಚಂ ಚಕ್ರಶಕ್ತಿ~॥

ಸರ್ವಮಂಗಲಾಯೈ ವಿದ್ಮಹೇ ಚಂದ್ರಾತ್ಮಿಕಾಯೈ ಧೀಮಹಿ ।\\ತನ್ನೋ ನಿತ್ಯಾ ಪ್ರಚೋದಯಾತ್~॥\\
ಇತಿ ಸರ್ವಮಂಗಲಾ ಆವರಣಪೂಜಾ~।
\section{೧೪। ಜ್ವಾಲಾಮಾಲಿನೀ}
ಅಸ್ಯ ಶ್ರೀ ಜ್ವಾಲಾಮಾಲಿನೀನಿತ್ಯಾಮಹಾಮಂತ್ರಸ್ಯ ಕಶ್ಯಪ ಋಷಿಃ~। ಗಾಯತ್ರೀ ಛಂದಃ~। ಜ್ವಾಲಾಮಾಲಿನೀನಿತ್ಯಾ ದೇವತಾ~। ರಂ ಬೀಜಂ~। ಫಟ್ ಶಕ್ತಿಃ~। ಹುಂ ಕೀಲಕಂ~।\\
\as{ನ್ಯಾಸಃ :}೧.ಓಂ ಓಂ ೨.ಓಂ ನಮಃ ೩.ಓಂ ಭಗವತಿ ೪.ಓಂ ಜ್ವಾಲಾಮಾಲಿನಿ ೫.ಓಂ ದೇವದೇವಿ ೬.ಓಂ ಸರ್ವಭೂತಸಂಹಾರಕಾರಿಕೇ \\
{\bfseries ಅಗ್ನಿಜ್ವಾಲಾಸಮಾಭಾಕ್ಷೀಂ ನೀಲವಕ್ತ್ರಾಂ ಚತುರ್ಭುಜಾಂ~।\\ನೀಲನೀರದಸಂಕಾಶಾಂ ನೀಲಕೇಶೀಂ ತನೂದರೀಂ ॥\\
ಖಡ್ಗಂ ತ್ರಿಶೂಲಂ ಬಿಭ್ರಾಣಾಂ ವರಂ ಸಾಭಯಮೇವ ಚ~।\\ಸಿಂಹಪೃಷ್ಠಸಮಾರೂಢಾಂ ಧ್ಯಾಯೇಜ್ಜ್ವಾಲಾದ್ಯಮಾಲಿನೀಂ ॥\\}
ಮನುಃ :{\bfseries  ಓಂ ನಮೋ ಭಗವತಿ ಜ್ವಾಲಾಮಾಲಿನಿ ದೇವದೇವಿ ಸರ್ವಭೂತಸಂಹಾರಕಾರಿಕೇ ಜಾತವೇದಸಿ ಜ್ವಲಂತಿ ಜ್ವಲ ಜ್ವಲ ಪ್ರಜ್ವಲ ಪ್ರಜ್ವಲ ಹ್ರಾಂ ಹ್ರೀಂ ಹ್ರೂಂ ರರ ರರ ರರರ ಹುಂ ಫಟ್ ಸ್ವಾಹಾ~॥}

ಓಂ ಅಭೀತ್ಯೈ ನಮಃ~।  ವಹ್ನಯೇ ನಮಃ~।  ಶಂಖಾಯ ನಮಃ~।  ಬಾಣೇಭ್ಯೋ ನಮಃ~।  ಖಡ್ಗಾಯ ನಮಃ~।  ಅಂಕುಶಾಯ ನಮಃ~।  ಪಾಶಾಯ ನಮಃ~। \\ಖೇಟಾಯ ನಮಃ~।  ಚಾಪಾಯ ನಮಃ~।  ಗದಾಯೈ ನಮಃ~।  ಶೂಲಾಯ ನಮಃ~।  ವರಾಯ ನಮಃ~॥

ಓಂ ಇಚ್ಛಾಶಕ್ತಿ~।  ಜ್ಞಾನಶಕ್ತಿ~।  ಕ್ರಿಯಾಶಕ್ತಿ~॥ 

ಓಂ ಡಾಕಿನೀ~।  ರಾಕಿಣೀ~।  ಲಾಕಿನೀ~।  ಕಾಕಿನೀ~।  ಸಾಕಿನೀ~।  ಹಾಕಿನೀ~॥ 

ಓಂ ಘಸ್ಮರಾ~।  ವಿಶ್ವಕವಲಾ~।  ಲೋಲಾಕ್ಷೀ~।  ಲೋಲಜಿಹ್ವಿಕಾ~।  ಸರ್ವಭಕ್ಷಾ~। \\ ಸಹಸ್ರಾಕ್ಷೀ~।  ನಿಃಸಂಗಾ~।  ಸಂಹೃತಿಪ್ರಿಯಾ~॥ 

ಓಂ ಅಚಿಂತ್ಯಾ~।  ಅಪ್ರಮೇಯಾ~।  ಪೂರ್ಣರೂಪಾ~।  ದುರಾಸದಾ~।  ಸರ್ವಾ~।  ಸಂಸಿದ್ಧಿರೂಪಾ~।  ಪಾವನಾ~।  ಏಕರೂಪಿಣೀ~॥ 

ಓಂ ಬ್ರಾಹ್ಮೀ~।  ಮಾಹೇಶ್ವರೀ~।  ಕೌಮಾರೀ~।  ವೈಷ್ಣವೀ~।  ವಾರಾಹೀ~।\\ ಇಂದ್ರಾಣೀ~।  ಚಾಮುಂಡಾ~।  ಮಹಾಲಕ್ಷ್ಮೀ~॥ 

ಓಂ ಲಂ ಇಂದ್ರಶಕ್ತಿ~।  ರಂ ಅಗ್ನಿಶಕ್ತಿ~।  ಮಂ ಯಮಶಕ್ತಿ~।  ಕ್ಷಂ ನಿರ್ಋತಿಶಕ್ತಿ~।  ವಂ ವರುಣಶಕ್ತಿ~।  ಯಂ ವಾಯುಶಕ್ತಿ~।  ಕುಂ ಕುಬೇರಶಕ್ತಿ~।  ಹಂ ಈಶಾನಶಕ್ತಿ~।  ಆಂ ಬ್ರಹ್ಮಶಕ್ತಿ~।  ಹ್ರೀಂ ಅನಂತಶಕ್ತಿ~।  ನಿಯತಿಶಕ್ತಿ~।  ಕಾಲಶಕ್ತಿ~॥

ಓಂ ವಂ ವಜ್ರಶಕ್ತಿ~।  ಶಂ ಶಕ್ತಿಶಕ್ತಿ~।  ದಂ ದಂಡಶಕ್ತಿ~।  ಖಂ ಖಡ್ಗಶಕ್ತಿ~।  ಪಾಂ ಪಾಶಶಕ್ತಿ~।  ಅಂ ಅಂಕುಶಶಕ್ತಿ~।  ಗಂ ಗದಾಶಕ್ತಿ~।  ತ್ರಿಂ ತ್ರಿಶೂಲಶಕ್ತಿ~।  ಪಂ ಪದ್ಮಶಕ್ತಿ~।  ಚಂ ಚಕ್ರಶಕ್ತಿ~॥

ಜ್ವಾಲಾಮಾಲಿನ್ಯೈ ವಿದ್ಮಹೇ ಮಹಾಜ್ವಾಲಾಯೈ ಧೀಮಹಿ ।\\ತನ್ನೋ ನಿತ್ಯಾ ಪ್ರಚೋದಯಾತ್~॥\\
ಇತಿ ಜ್ವಾಲಾಮಾಲಿನೀನಿತ್ಯಾ ಆವರಣಪೂಜಾ।
\section{೧೫। ಚಿತ್ರಾನಿತ್ಯಾ}
ಚಿತ್ರಾನಿತ್ಯಾಮಹಾಮಂತ್ರಸ್ಯ ಬ್ರಹ್ಮಾ ಋಷಿಃ~। ಗಾಯತ್ರೀ ಛಂದಃ~। \\ಚಿತ್ರಾ ನಿತ್ಯಾ ದೇವತಾ~। ಚಾಂ ಚೀಂ ಇತ್ಯಾದಿನಾ ನ್ಯಾಸಃ~।\\
{\bfseries ಶುದ್ಧಸ್ಫಟಿಕಸಂಕಾಶಾಂ ಪಲಾಶಕುಸುಮಪ್ರಭಾಂ~।\\
ನೀಲಮೇಘಪ್ರತೀಕಾಶಾಂ ಚತುರ್ಹಸ್ತಾಂ ತ್ರಿಲೋಚನಾಂ ॥\\
ಸರ್ವಾಲಂಕಾರಸಂಯುಕ್ತಾಂ ಪುಷ್ಪಬಾಣೇಕ್ಷುಚಾಪಿನೀಂ~।\\
ಪಾಶಾಂಕುಶಸಮೋಪೇತಾಂ ಧ್ಯಾಯೇಚ್ಚಿತ್ರಾಂ ಮಹೇಶ್ವರೀಂ ॥\\}
ಮನುಃ :{\bfseries  ಚ್ಕೌಂ~॥}

ಓಂ ಅಭಯಾಯ ನಮಃ~।  ಅಂಕುಶಾಯ ನಮಃ।  ಪಾಶಾಯ ನಮಃ।  ವರಾಯ ನಮಃ।  ಇಚ್ಛಾಶಕ್ತಿ~।  ಜ್ಞಾನಶಕ್ತಿ~।  ಕ್ರಿಯಾಶಕ್ತಿ~॥ 

ಓಂ ಅಂ ಆಂ ಇಂ ಈಂ ಉಂ ಊಂ ಋಂ ೠಂ \\ಲೃಂ ಲೄಂ ಏಂ ಐಂ  ಔಂ ಅಂ ಅಃ ಬ್ರಾಹ್ಮೀ~।\\  ಕಂ ಖಂ ಗಂ ಘಂ ಙಂ ಮಾಹೇಶ್ವರೀ~।\\  ಚಂ ಛಂ ಜಂ ಝಂ ಞಂ ಕೌಮಾರೀ~।\\  ಟಂ ಠಂ ಡಂ ಢಂ ಣಂ ವೈಷ್ಣವೀ~। \\ತಂ ಥಂ ದಂ ಧಂ ನಂ ವಾರಾಹೀ~। \\ಪಂ ಫಂ ಬಂ ಭಂ ಮಂ ಇಂದ್ರಾಣೀ~। \\ ಯಂ ರಂ ಲಂ ವಂ ಶಂ ಚಾಮುಂಡಾ~। \\ ಷಂ ಸಂ ಹಂ ಳಂ ಕ್ಷಂ ಮಹಾಲಕ್ಷ್ಮೀ~॥ 

ಓಂ ಭದ್ರಾ~।  ಭವಾನೀ~।  ಭವ್ಯಾ~।  ವಿಶಾಲಾಕ್ಷೀ~।  ಶುಚಿಸ್ಮಿತಾ~।  ಕುಂಕುಮಾ~।  ಕಮಲಾ~।  ಕಲ್ಪಾ~।  ಕಲಾ~।  ಪೂರಣೀ~।  ನಿತ್ಯಾ~।  ಅಮೃತಾ~।  ಜೀವಿತಾ~।  ದಯಾ~।  ಅಶೋಕಾ~।  ಅಮಲಾ~।  ಪೂರ್ಣಾ~।  ಪುಣ್ಯಾ~।  ಭಾಗ್ಯಾ~।  ಉದ್ಯತಾ~।  ವಿವೇಕಾ~।  ವಿಭವಾ~।  ವಿಶ್ವಾ~।  ವಿನತಾ~॥ 

ಓಂ ಕಾಮಿನೀ~।  ಖೇಚರೀ~।  ಆರ್ಯಾ~।  ಪುರಾಣಾ~।  ಪರಮೇಶ್ವರೀ~।  ಗೌರೀ~।  ಶಿವಾ~।  ಅಮೇಯಾ~।  ವಿಮಲಾ~।  ವಿಜಯಾ~।  ಪರಾ~।  ಪವಿತ್ರಾ~।  ಪದ್ಮಿನೀ~।  ವಿದ್ಯಾ~।  ವಿಶ್ವೇಶೀ। ಶಿವವಲ್ಲಭಾ। ಅಶೇಷರೂಪಾ।ಆನಂದಾ~।  ಅಂಬುಜಾಕ್ಷೀ~।  ಅನಿಂದಿತಾ~।  ವರದಾ। ವಾಕ್ಪ್ರದಾ। ವಾಣೀ।  ವಿವಿಧಾ~।  ವೇದವಿಗ್ರಹಾ~।  ವಂದ್ಯಾ~।  ವಾಗೀಶ್ವರೀ~।  ಸತ್ಯಾ~।  ಸಂಯತಾ~।  ಸರಸ್ವತೀ~।  ನಿರ್ಮಲಾ~।  ನಾದರೂಪಾ~॥

ಓಂ ಲಂ ಇಂದ್ರಶಕ್ತಿ~।  ರಂ ಅಗ್ನಿಶಕ್ತಿ~।  ಮಂ ಯಮಶಕ್ತಿ~।  ಕ್ಷಂ ನಿರ್ಋತಿಶಕ್ತಿ~।  ವಂ ವರುಣಶಕ್ತಿ~।  ಯಂ ವಾಯುಶಕ್ತಿ~।  ಕುಂ ಕುಬೇರಶಕ್ತಿ~।  ಹಂ ಈಶಾನಶಕ್ತಿ~।  ಆಂ ಬ್ರಹ್ಮಶಕ್ತಿ~।  ಹ್ರೀಂ ಅನಂತಶಕ್ತಿ~।  ನಿಯತಿಶಕ್ತಿ~।  ಕಾಲಶಕ್ತಿ~॥

ಓಂ ವಂ ವಜ್ರಶಕ್ತಿ~।  ಶಂ ಶಕ್ತಿಶಕ್ತಿ~।  ದಂ ದಂಡಶಕ್ತಿ~।  ಖಂ ಖಡ್ಗಶಕ್ತಿ~।  ಪಾಂ ಪಾಶಶಕ್ತಿ~।  ಅಂ ಅಂಕುಶಶಕ್ತಿ~।  ಗಂ ಗದಾಶಕ್ತಿ~।  ತ್ರಿಂ ತ್ರಿಶೂಲಶಕ್ತಿ~।  ಪಂ ಪದ್ಮಶಕ್ತಿ~।  ಚಂ ಚಕ್ರಶಕ್ತಿ~॥

ಓಂ ವಿಚಿತ್ರಾಯೈ ವಿದ್ಮಹೇ ಮಹಾನಿತ್ಯಾಯೈ ಧೀಮಹಿ~। ತನ್ನೋ ನಿತ್ಯಾ ಪ್ರಚೋದಯಾತ್ ॥\\
ಇತಿಚಿತ್ರಾನಿತ್ಯಾ
\authorline{॥ಇತಿ ನಿತ್ಯಾ ಯಜನವಿಧಿಃ ಸಂಪೂರ್ಣಃ ॥}

\section{ಶ್ರೀಚಕ್ರಆವರಣಪೂಜಾ}
\as{ಓಂ ಐಂಹ್ರೀಂಶ್ರೀಂ} ತ್ರಿಪುರಸುಂದರ್ಯೈ ನಮಃ\\
\as{೪} ಹೃದಯದೇವ್ಯೈ ನಮಃ\\
\as{೪} ಶಿರೋದೇವ್ಯೈ ನಮಃ\\
\as{೪} ಶಿಖಾದೇವ್ಯೈ ನಮಃ\\
\as{೪} ಕವಚದೇವ್ಯೈ ನಮಃ\\
\as{೪} ನೇತ್ರದೇವ್ಯೈ ನಮಃ\\
\as{೪} ಅಸ್ತ್ರದೇವ್ಯೈ ನಮಃ

%\as{೪} ಕಾಮೇಶ್ವರ್ಯೈ ನಮಃ\\x\as{೪} ಭಗಮಾಲಿನ್ಯೈ ನಮಃ\\x\as{೪} ನಿತ್ಯಕ್ಲಿನ್ನಾಯೈ ನಮಃ\\x\as{೪} ಭೇರುಂಡಾಯೈ ನಮಃ\\x\as{೪} ವಹ್ನಿವಾಸಿನ್ಯೈ ನಮಃ\\x\as{೪} ಮಹಾವಜ್ರೇಶ್ವರ್ಯೈ ನಮಃ\\x\as{೪} ಶಿವಾದೂತ್ಯೈ ನಮಃ\\x\as{೪} ತ್ವರಿತಾಯೈ ನಮಃ\\x\as{೪} ಕುಲಸುಂದರ್ಯೈ ನಮಃ\\x\as{೪} ನಿತ್ಯಾಯೈ ನಮಃ\\x\as{೪} ನೀಲಪತಾಕಾಯೈ ನಮಃ\\x\as{೪} ವಿಜಯಾಯೈ ನಮಃ\\x\as{೪} ಸರ್ವಮಂಗಳಾಯೈ ನಮಃ\\x\as{೪} ಜ್ವಾಲಾಮಾಲಿನ್ಯೈ ನಮಃ\\x\as{೪} ಚಿತ್ರಾಯೈ ನಮಃ\\x\as{೪} ಮಹಾನಿತ್ಯಾಯೈ ನಮಃ
\as{೪ ಅಂ} ಐಂ ಸಕಲಹ್ರೀಂ ನಿತ್ಯಕ್ಲಿನ್ನೇ ಮದದ್ರವೇ ಸೌಃ \as{ಅಂ~॥} ಕಾಮೇಶ್ವರ್ಯೈ ನಮಃ ॥೧\\
\as{೪ ಆಂ} ಐಂ ಭಗಭುಗೇ ಭಗಿನಿ ಭಗೋದರಿ ಭಗಮಾಲೇ ಭಗಾವಹೇ ಭಗಗುಹ್ಯೇ ಭಗಯೋನಿ ಭಗನಿಪಾತನಿ ಸರ್ವಭಗವಶಂಕರಿ ಭಗರೂಪೇ ನಿತ್ಯಕ್ಲಿನ್ನೇ ಭಗಸ್ವರೂಪೇ ಸರ್ವಾಣಿ ಭಗಾನಿ ಮೇ ಹ್ಯಾನಯ ವರದೇ ರೇತೇ ಸುರೇತೇ ಭಗಕ್ಲಿನ್ನೇ ಕ್ಲಿನ್ನದ್ರವೇ ಕ್ಲೇದಯ ದ್ರಾವಯ ಅಮೋಘೇ ಭಗವಿಚ್ಚೇ ಕ್ಷುಭ ಕ್ಷೋಭಯ ಸರ್ವಸತ್ವಾನ್ ಭಗೇಶ್ವರಿ ಐಂ ಬ್ಲೂಂ ಜಂ ಬ್ಲೂಂ ಭೇಂ ಬ್ಲೂಂ ಮೋಂ ಬ್ಲೂಂ ಹೇಂ ಬ್ಲೂಂ ಹೇಂ ಕ್ಲಿನ್ನೇ ಸರ್ವಾಣಿ ಭಗಾನಿ ಮೇ ವಶಮಾನಯ ಸ್ತ್ರೀಂ ಹ್‌ರ್‌ಬ್ಲೇಂ ಹ್ರೀಂ \as{ಆಂ ॥} ಭಗಮಾಲಿನ್ಯೈ ನಮಃ ॥೨\\
\as{೪ ಇಂ} ಓಂ ಹ್ರೀಂ ನಿತ್ಯಕ್ಲಿನ್ನೇ ಮದದ್ರವೇ ಸ್ವಾಹಾ \as{ಇಂ ॥} ನಿತ್ಯಕ್ಲಿನ್ನಾಯೈ ನಮಃ ॥೩\\
\as{೪ ಈಂ} ಓಂ ಕ್ರೋಂಭ್ರೋಂಕ್ರೋಂಝ್ರೋಂಛ್ರೋಂಜ್ರೋಂ ಸ್ವಾಹಾ \as{ಈಂ ॥} ಭೇರುಂಡಾಯೈ ನಮಃ ॥೪\\
\as{೪ ಉಂ} ಓಂ ಹ್ರೀಂ ವಹ್ನಿವಾಸಿನ್ಯೈ ನಮಃ \as{ಉಂ~॥} ವಹ್ನಿವಾಸಿನ್ಯೈ ನಮಃ ॥೫\\
\as{೪ ಊಂ} ಹ್ರೀಂ ಕ್ಲಿನ್ನೇ ಐಂ ಕ್ರೋಂ ನಿತ್ಯಮದದ್ರವೇ ಹ್ರೀಂ \as{ಊಂ~॥} ಮಹಾವಜ್ರೇಶ್ವರ್ಯೈ ನಮಃ ॥೬\\
\as{೪ ಋಂ} ಹ್ರೀಂ ಶಿವಾದೂತ್ಯೈ ನಮಃ \as{ಋಂ~॥} ಶಿವಾದೂತ್ಯೈ ನಮಃ ॥೭\\
\as{೪ ೠಂ} ಓಂ ಹ್ರೀಂ ಹೂಂಖೇಚಛೇಕ್ಷಃಸ್ತ್ರೀಂಹೂಂಕ್ಷೇ ಹ್ರೀಂ ಫಟ್ \as{ೠಂ~॥} ತ್ವರಿತಾಯೈ ನಮಃ ॥೮\\
\as{೪ ಲೃಂ} ಐಂಕ್ಲೀಂಸೌಃ \as{ಲೃಂ~॥} ಕುಲಸುಂದರ್ಯೈ ನಮಃ ॥೯\\
\as{೪ ಲೄಂ} ಹಸಕಲರಡೈಂ ಹಸಕಲರಡೀಂ ಹಸಕಲರಡೌಃ \as{ಲೄಂ ॥} ನಿತ್ಯಾಯೈ ನಮಃ ॥೧೦\\
\as{೪ ಏಂ} ಹ್ರೀಂ ಫ್ರೇಂಸ್ರೂಂಓಂಆಂಕ್ಲೀಂಐಂಬ್ಲೂಂ ನಿತ್ಯಮದದ್ರವೇ ಹುಂಫ್ರೇಂ ಹ್ರೀಂ \as{ಏಂ~॥} ನೀಲಪತಾಕಾಯೈ ನಮಃ ॥೧೧\\
\as{೪ ಐಂ} ಭಮರಯಉಔಂ \as{ಐಂ ॥} ವಿಜಯಾಯೈ ನಮಃ ॥೧೨\\
\as{೪ ಓಂ} ಸ್ವೌಂ \as{ಓಂ ॥} ಸರ್ವಮಂಗಳಾಯೈ ನಮಃ ॥೧೩\\
\as{೪ ಔಂ} ಓಂ ನಮೋ ಭಗವತಿ ಜ್ವಾಲಾಮಾಲಿನಿ ದೇವದೇವಿ ಸರ್ವಭೂತಸಂಹಾರಕಾರಿಕೇ ಜಾತವೇದಸಿ ಜ್ವಲಂತಿ ಜ್ವಲ ಜ್ವಲ ಪ್ರಜ್ವಲ ಪ್ರಜ್ವಲ ಹ್ರಾಂ ಹ್ರೀಂ ಹ್ರೂಂ ರರ ರರ ರರರ ಹುಂ ಫಟ್ ಸ್ವಾಹಾ \as{ಔಂ~॥} ಜ್ವಾಲಾಮಾಲಿನ್ಯೈ ನಮಃ ॥೧೪\\
\as{೪ ಅಂ} ಚ್ಕೌಂ \as{ಅಂ~॥} ಚಿತ್ರಾಯೈ ನಮಃ ॥೧೫\\
\as{೪ ಅಃ} ೧೫ \as{ಅಃ~॥} ಲಲಿತಾ ಮಹಾನಿತ್ಯಾಯೈ ನಮಃ ॥೧೬

\as{೪ ದಿವ್ಯೌಘಸಿದ್ಧೌಘಮಾನವೌಘೇಭ್ಯೋ ನಮಃ}\\
\as{೪} ಪರಪ್ರಕಾಶಾನಂದನಾಥಾಯ ನಮಃ\\
\as{೪} ಪರಶಿವಾನಂದನಾಥಾಯ ನಮಃ\\
\as{೪} ಪರಾಶಕ್ತ್ಯಂಬಾಯೈ ನಮಃ\\
\as{೪} ಕೌಲೇಶ್ವರಾನಂದನಾಥಾಯ ನಮಃ\\
\as{೪} ಶುಕ್ಲದೇವ್ಯಂಬಾಯೈ ನಮಃ\\
\as{೪} ಕುಲೇಶ್ವರಾನಂದನಾಥಾಯ ನಮಃ\\
\as{೪} ಕಾಮೇಶ್ವರ್ಯಂಬಾಯೈ ನಮಃ

\as{೪} ಭೋಗಾನಂದನಾಥಾಯ ನಮಃ\\
\as{೪} ಕ್ಲಿನ್ನಾನಂದನಾಥಾಯ ನಮಃ\\
\as{೪} ಸಮಯಾನಂದನಾಥಾಯ ನಮಃ\\
\as{೪} ಸಹಜಾನಂದನಾಥಾಯ ನಮಃ

\as{೪} ಗಗನಾನಂದನಾಥಾಯ ನಮಃ\\
\as{೪} ವಿಶ್ವಾನಂದನಾಥಾಯ ನಮಃ\\
\as{೪} ವಿಮಲಾನಂದನಾಥಾಯ ನಮಃ\\
\as{೪} ಮದನಾನಂದನಾಥಾಯ ನಮಃ\\
\as{೪} ಭುವನಾನಂದನಾಥಾಯ ನಮಃ\\
\as{೪} ಲೀಲಾನಂದನಾಥಾಯ ನಮಃ\\
\as{೪} ಸ್ವಾತ್ಮಾನಂದನಾಥಾಯ ನಮಃ\\
\as{೪} ಪ್ರಿಯಾನಂದನಾಥಾಯ ನಮಃ
\subsection{(ಷೋಡಶ್ಯುಪಾಸಕಾನಾಂ ಕೃತೇ ವಿದ್ಯಾರ್ಣವತಂತ್ರೋಕ್ತ ಗುರುಪರಂಪರಾ)}
\as{೪} ವ್ಯೋಮಾತೀತಾಂಬಾಯೈ ನಮಃ\\
\as{೪} ವ್ಯೋಮೇಶ್ವರ್ಯಂಬಾಯೈ ನಮಃ\\
\as{೪} ವ್ಯೋಮಗಾಂಬಾಯೈ ನಮಃ\\
\as{೪} ವ್ಯೋಮಚಾರಿಣ್ಯಂಬಾಯೈ ನಮಃ\\
\as{೪} ವ್ಯೋಮಸ್ಥಾಂಬಾಯೈ ನಮಃ

\as{೪} ಉನ್ಮನಾಕಾಶಾನಂದನಾಥಾಯ ನಮಃ\\
\as{೪} ಸಮನಾಕಾಶಾನಂದನಾಥಾಯ ನಮಃ\\
\as{೪} ವ್ಯಾಪಕಾಕಾಶಾನಂದನಾಥಾಯ ನಮಃ\\
\as{೪} ಶಕ್ತ್ಯಾಕಾಶಾನಂದನಾಥಾಯ ನಮಃ\\
\as{೪} ಧ್ವನ್ಯಾಕಾಶಾನಂದನಾಥಾಯ ನಮಃ\\
\as{೪} ಧ್ವನಿಮಾತ್ರಾಕಾಶಾನಂದನಾಥಾಯ ನಮಃ\\
\as{೪} ಅನಾಹತಾಕಾಶನಂದನಾಥಾಯ ನಮಃ\\
\as{೪} ಬಿಂದ್ವಾಕಾಶಾನಂದನಾಥಾಯ ನಮಃ\\
\as{೪} ಇಂದ್ವಾಕಾಶಾನಂದನಾಥಾಯ ನಮಃ

\as{೪} ಪರಮಾತ್ಮಾನಂದನಾಥಾಯ ನಮಃ\\
\as{೪} ಶಾಂಭವಾನಂದನಾಥಾಯ ನಮಃ\\
\as{೪} ಚಿನ್ಮುದ್ರಾನಂದನಾಥಾಯ ನಮಃ\\
\as{೪} ವಾಗ್ಭವಾನಂದನಾಥಾಯ ನಮಃ\\
\as{೪} ಲೀಲಾನಂದನಾಥಾಯ ನಮಃ\\
\as{೪} ಸಂಭ್ರಮಾನಂದನಾಥಾಯ ನಮಃ\\
\as{೪} ಚಿದಾನಂದನಾಥಾಯ ನಮಃ\\
\as{೪} ಪ್ರಸನ್ನಾನಂದನಾಥಾಯ ನಮಃ\\
\as{೪} ವಿಶ್ವಾನಂದನಾಥಾಯ ನಮಃ

\as{೪} ಪರಮೇಶ್ವರ ಪರಮೇಶ್ವರ್ಯೈ ನಮಃ\\
\as{೪} ಮಿತ್ರೀಶಮಯ್ಯೈ ನಮಃ\\
\as{೪} ಷಷ್ಠೀಶಮಯ್ಯೈ ನಮಃ\\
\as{೪} ಉಡ್ಡೀಶಮಯ್ಯೈ ನಮಃ\\
\as{೪} ಚರ್ಯಾನಾಥಮಯ್ಯೈ ನಮಃ\\
\as{೪} ಲೋಪಾಮುದ್ರಾಮಯ್ಯೈ ನಮಃ\\
\as{೪} ಅಗಸ್ತ್ಯಮಯ್ಯೈ ನಮಃ

\as{೪} ಕಾಲತಾಪನಮಯ್ಯೈ ನಮಃ\\
\as{೪} ಧರ್ಮಾಚಾರಮಯ್ಯೈ ನಮಃ\\
\as{೪} ಮುಕ್ತಕೇಶೀಶ್ವರಮಯ್ಯೈ ನಮಃ\\
\as{೪} ದೀಪಕಲಾನಾಥಮಯ್ಯೈ ನಮಃ

\as{೪} ವಿಷ್ಣುದೇವಮಯ್ಯೈ ನಮಃ\\
\as{೪} ಪ್ರಭಾಕರದೇವಮಯ್ಯೈ ನಮಃ\\
\as{೪} ತೇಜೋದೇವಮಯ್ಯೈ ನಮಃ\\
\as{೪} ಮನೋಜದೇವಮಯ್ಯೈ ನಮಃ\\
\as{೪} ಕಲ್ಯಾಣದೇವಮಯ್ಯೈ ನಮಃ\\
\as{೪} ರತ್ನದೇವಮಯ್ಯೈ ನಮಃ\\
\as{೪} ವಾಸುದೇವಮಯ್ಯೈ ನಮಃ\\
\as{೪} ಶ್ರೀರಾಮಾನಂದಮಯ್ಯೈ ನಮಃ

\as{೪} ಐಂಕ್ಲೀಂಸೌಃ ಹಂಸಃ ಶಿವಃ ಸೋಽಹಂ ಹಂಸಃ, ಹ್‌ಸ್‌ಖ್‌ಫ್ರೇಂ ಹಸಕ್ಷಮಲವರಯೂಂ ಹ್‌ಸೌಃ ಸಹಕ್ಷಮಲವರಯೀಂ ಸ್‌ಹೌಃ ,ಐಂ ಕಏಈಲಹ್ರೀಂ ಕ್ಲೀಂ ಹಸಕಹಲಹ್ರೀಂ ಸೌಃ  ಸಕಲಹ್ರೀಂ, ಹಸಕಲಹ್ರೀಂ ಹಸಕಹಲಹ್ರೀಂ ಸಕಲಹ್ರೀಂ, ಹಸಕಲ ಹಸಕಹಲ ಸಕಲಹ್ರೀಂ, ಪ್ರಜ್ಞಾನಂ ಬ್ರಹ್ಮ, ಅಹಂ ಬ್ರಹ್ಮಾಸ್ಮಿ, ತತ್ವಮಸಿ, ಅಯಮಾತ್ಮಾ ಬ್ರಹ್ಮ, ಹಂಸಃ ಶಿವಃ ಸೋಽಹಂ ಹಂಸಃ॥ ಸ್ವಾತ್ಮಾರಾಮ ಪರಮಾನಂದ ಪಂಜರ ವಿಲೀನ ತೇಜಸೇ ಪರಮೇಷ್ಠಿಗುರವೇ ನಮಃ~। ಶ್ರೀಪಾದುಕಾಂ ಪೂಜಯಾಮಿ ನಮಃ॥\\
\as{೪} ಐಂಕ್ಲೀಂಸೌಃ ಸೋಽಹಂ ಹಂಸಃ ಶಿವಃ, ಹ್‌ಸ್‌ಖ್‌ಫ್ರೇಂ ಹಸಕ್ಷಮಲವರಯೂಂ ಹ್‌ಸೌಃ ಸಹಕ್ಷಮಲವರಯೀಂ ಸ್‌ಹೌಃ,ಐಂ ಕಏಈಲಹ್ರೀಂ ಕ್ಲೀಂ ಹಸಕಹಲಹ್ರೀಂ ಸೌಃ  ಸಕಲಹ್ರೀಂ, ಹಸಕಲಹ್ರೀಂ ಹಸಕಹಲಹ್ರೀಂ ಸಕಲಹ್ರೀಂ, ಹಸಕಲ ಹಸಕಹಲ ಸಕಲಹ್ರೀಂ, ಪ್ರಜ್ಞಾನಂ ಬ್ರಹ್ಮ, ಅಹಂ ಬ್ರಹ್ಮಾಸ್ಮಿ, ತತ್ವಮಸಿ, ಅಯಮಾತ್ಮಾ ಬ್ರಹ್ಮ,  ಸೋಽಹಂ ಹಂಸಃ ಶಿವಃ ॥ ಸ್ವಚ್ಛಪ್ರಕಾಶ ವಿಮರ್ಶಹೇತವೇ ಪರಮಗುರವೇ ನಮಃ ।ಶ್ರೀಪಾದುಕಾಂ ಪೂಜಯಾಮಿ ನಮಃ ॥\\
\as{೪} ಐಂಕ್ಲೀಂಸೌಃ ಹಂಸಃ ಶಿವಃ ಸೋಽಹಂ, ಹ್‌ಸ್‌ಖ್‌ಫ್ರೇಂ ಹಸಕ್ಷಮಲವರಯೂಂ ಹ್‌ಸೌಃ ಸಹಕ್ಷಮಲವರಯೀಂ ಸ್‌ಹೌಃ , ಐಂ ಕಏಈಲಹ್ರೀಂ ಕ್ಲೀಂ ಹಸಕಹಲಹ್ರೀಂ ಸೌಃ  ಸಕಲಹ್ರೀಂ, ಹಸಕಲಹ್ರೀಂ ಹಸಕಹಲಹ್ರೀಂ ಸಕಲಹ್ರೀಂ, ಹಸಕಲ ಹಸಕಹಲ ಸಕಲಹ್ರೀಂ, ಪ್ರಜ್ಞಾನಂ ಬ್ರಹ್ಮ, ಅಹಂ ಬ್ರಹ್ಮಾಸ್ಮಿ, ತತ್ವಮಸಿ, ಅಯಮಾತ್ಮಾ ಬ್ರಹ್ಮ, ಹಂಸಃ ಶಿವಃ ಸೋಽಹಂ~॥ ಸ್ವರೂಪ ನಿರೂಪಣ ಹೇತವೇ ಶ್ರೀಗುರವೇ ನಮಃ~। ಶ್ರೀಪಾದುಕಾಂ ಪೂಜಯಾಮಿ ನಮಃ ॥

\as{ ಸಂವಿನ್ಮಯೇ ಪರೇ ದೇವಿ ಪರಾಮೃತರುಚಿಪ್ರಿಯೇ ।\\
 ಅನುಜ್ಞಾಂ ತ್ರಿಪುರೇ ದೇಹಿ ಪರಿವಾರಾರ್ಚನಾಯ ಮೇ ॥}
 
\subsection{ಶ್ರೀಚಕ್ರ ಪ್ರಥಮಾವರಣದೇವತಾಃ}
{\bfseries ೪ ಅಂ ಆಂ ಸೌಃ ತ್ರೈಲೋಕ್ಯಮೋಹನ ಚಕ್ರಾಯ ನಮಃ }\\
\as{೪ ಅಂ} ಅಣಿಮಾಸಿದ್ಧ್ಯೈ ನಮಃ ।೧\\
\as{೪ ಲಂ} ಲಘಿಮಾಸಿದ್ಧ್ಯೈ ನಮಃ ।೨\\
\as{೪ ಮಂ} ಮಹಿಮಾಸಿದ್ಧ್ಯೈ ನಮಃ ।೩\\
\as{೪ ಈಂ} ಈಶಿತ್ವಸಿದ್ಧ್ಯೈ ನಮಃ ।೪\\
\as{೪ ವಂ} ವಶಿತ್ವಸಿದ್ಧ್ಯೈ ನಮಃ ।೫\\
\as{೪ ಪಂ} ಪ್ರಾಕಾಮ್ಯಸಿದ್ಧ್ಯೈ ನಮಃ ।೬\\
\as{೪ ಭುಂ} ಭುಕ್ತಿಸಿದ್ಧ್ಯೈ ನಮಃ ।೭\\
\as{೪ ಇಂ} ಇಚ್ಛಾಸಿದ್ಧ್ಯೈ ನಮಃ ।೮\\
\as{೪ ಪಂ} ಪ್ರಾಪ್ತಿಸಿದ್ಧ್ಯೈ ನಮಃ ।೯\\
\as{೪ ಸಂ} ಸರ್ವಕಾಮಸಿದ್ಧ್ಯೈ ನಮಃ ।೧೦

\as{೪ ಆಂ} ಬ್ರಾಹ್ಮ್ಯೈ ನಮಃ ।೧\\
\as{೪ ಈಂ} ಮಾಹೇಶ್ವರ್ಯೈ ನಮಃ ।೨\\
\as{೪ ಊಂ} ಕೌಮಾರ್ಯೈ ನಮಃ ।೩\\
\as{೪ ೠಂ} ವೈಷ್ಣವ್ಯೈ ನಮಃ ।೪\\
\as{೪ ಲೄಂ} ವಾರಾಹ್ಯೈ ನಮಃ ।೫\\
\as{೪ ಐಂ} ಮಾಹೇಂದ್ರ್ಯೈ ನಮಃ ।೬\\
\as{೪ ಔಂ} ಚಾಮುಂಡಾಯೈ ನಮಃ ।೭\\
\as{೪ ಅಃ} ಮಹಾಲಕ್ಷ್ಮ್ಯೈ ನಮಃ ।೮

\as{೪ ದ್ರಾಂ} ಸರ್ವಸಂಕ್ಷೋಭಿಣ್ಯೈ ನಮಃ ।೧\\
\as{೪ ದ್ರೀಂ} ಸರ್ವವಿದ್ರಾವಿಣ್ಯೈ ನಮಃ ।೨\\
\as{೪ ಕ್ಲೀಂ} ಸರ್ವಾಕರ್ಷಿಣ್ಯೈ ನಮಃ ।೩\\
\as{೪ ಬ್ಲೂಂ} ಸರ್ವವಶಂಕರ್ಯೈ ನಮಃ ।೪\\
\as{೪ ಸಃ} ಸರ್ವೋನ್ಮಾದಿನ್ಯೈ ನಮಃ ।೫\\
\as{೪ ಕ್ರೋಂ} ಸರ್ವಮಹಾಂಕುಶಾಯೈ ನಮಃ ।೬\\
\as{೪ ಹ್‌ಸ್‌ಖ್‌ಫ್ರೇಂ} ಸರ್ವಖೇಚರ್ಯೈ ನಮಃ ।೭\\
\as{೪ ಹ್ಸೌಃ} ಸರ್ವಬೀಜಾಯೈ ನಮಃ ।೮\\
\as{೪ ಐಂ} ಸರ್ವಯೋನ್ಯೈ ನಮಃ ।೯\\
\as{೪ ಹ್‌ಸ್‌ರೈಂ ಹ್‌ಸ್‌ಕ್ಲ್ರೀಂ ಹ್‌ಸ್‌ರ್ಸೌಃ} ಸರ್ವತ್ರಿಖಂಡಾಯೈ ನಮಃ ।೧೦\\
\as{೪ ಅಂ ಆಂ ಸೌಃ॥} ತ್ರೈಲೋಕ್ಯಮೋಹನ ಚಕ್ರಸ್ವಾಮಿನ್ಯೈ ನಮಃ\\
ಪ್ರಕಟಯೋಗಿನ್ಯೈ ನಮಃ

\as{೪ ಅಂ ಆಂ ಸೌಃ॥} ತ್ರಿಪುರಾಚಕ್ರೇಶ್ವರೀ ಶ್ರೀಪಾದುಕಾಂ ಪೂಜಯಾಮಿ ತರ್ಪಯಾಮಿ ನಮಃ ॥\\
\as{೪ ಅಂ} ಅಣಿಮಾಸಿದ್ಧಿ ಶ್ರೀಪಾದುಕಾಂ ಪೂಜಯಾಮಿ ತರ್ಪಯಾಮಿ ನಮಃ ॥\\
\as{೪ ದ್ರಾಂ} ಸರ್ವಸಂಕ್ಷೋಭಿಣೀಮುದ್ರಾಶಕ್ತಿ ಶ್ರೀಪಾದುಕಾಂ ಪೂಜಯಾಮಿ ತರ್ಪಯಾಮಿ ನಮಃ ॥\\
\as{೪ ದ್ರಾಂ ॥}\\
ಮೂಲೇನ ತ್ರಿಃ ಸಂತರ್ಪ್ಯ\\
\as{೪} ಅಭೀಷ್ಟಸಿದ್ಧಿಂ ಮೇ ದೇಹಿ ಶರಣಾಗತ ವತ್ಸಲೇ~।\\
ಭಕ್ತ್ಯಾ ಸಮರ್ಪಯೇ ತುಭ್ಯಂ ಪ್ರಥಮಾವರಣಾರ್ಚನಂ ॥
\subsection{ಶ್ರೀಚಕ್ರ ದ್ವಿತೀಯಾವರಣದೇವತಾಃ}
{\bfseries ೪ ಐಂ ಕ್ಲೀಂ ಸೌಃ ಸರ್ವಾಶಾಪರಿಪೂರಕ ಚಕ್ರಾಯ ನಮಃ}\\
\as{೪ ಅಂ} ಕಾಮಾಕರ್ಷಣ್ಯೈ ನಮಃ ।೧\\
\as{೪ ಆಂ} ಬುದ್ಧ್ಯಾಕರ್ಷಣ್ಯೈ ನಮಃ ।೨\\
\as{೪ ಇಂ} ಅಹಂಕಾರಾಕರ್ಷಣ್ಯೈ ನಮಃ ।೩\\
\as{೪ ಈಂ} ಶಬ್ದಾಕರ್ಷಣ್ಯೈ ನಮಃ ।೪\\
\as{೪ ಉಂ} ಸ್ಪರ್ಶಾಕರ್ಷಣ್ಯೈ ನಮಃ ।೫\\
\as{೪ ಊಂ} ರೂಪಾಕರ್ಷಣ್ಯೈ ನಮಃ ।೬\\
\as{೪ ಋಂ} ರಸಾಕರ್ಷಣ್ಯೈ ನಮಃ ।೭\\
\as{೪ ೠಂ} ಗಂಧಾಕರ್ಷಣ್ಯೈ ನಮಃ ।೮\\
\as{೪ ಲೃಂ} ಚಿತ್ತಾಕರ್ಷಣ್ಯೈ ನಮಃ ।೯\\
\as{೪ ಲೄಂ} ಧೈರ್ಯಾಕರ್ಷಣ್ಯೈ ನಮಃ ।೧೦\\
\as{೪ ಏಂ} ಸ್ಮೃತ್ಯಾಕರ್ಷಣ್ಯೈ ನಮಃ ।೧೧\\
\as{೪ ಐಂ} ನಾಮಾಕರ್ಷಣ್ಯೈ ನಮಃ ।೧೨\\
\as{೪ ಓಂ} ಬೀಜಾಕರ್ಷಣ್ಯೈ ನಮಃ ।೧೩\\
\as{೪ ಔಂ} ಆತ್ಮಾಕರ್ಷಣ್ಯೈ ನಮಃ ।೧೪\\
\as{೪ ಅಂ} ಅಮೃತಾಕರ್ಷಣ್ಯೈ ನಮಃ ।೧೫\\
\as{೪ ಅಃ} ಶರೀರಾಕರ್ಷಣ್ಯೈ ನಮಃ ।೧೬\\
\as{೪ ಐಂ ಕ್ಲೀಂ ಸೌಃ॥} ಸರ್ವಾಶಾಪರಿಪೂರಕ ಚಕ್ರಸ್ವಾಮಿನ್ಯೈ ನಮಃ\\
ಗುಪ್ತಯೋಗಿನ್ಯೈ ನಮಃ

\as{೪ ಐಂ ಕ್ಲೀಂ ಸೌಃ॥} ತ್ರಿಪುರೇಶೀಚಕ್ರೇಶ್ವರೀ ಶ್ರೀಪಾದುಕಾಂ ಪೂ।ತ।ನಮಃ॥\\
\as{೪ ಲಂ} ಲಘಿಮಾಸಿದ್ಧಿ ಶ್ರೀಪಾದುಕಾಂ ಪೂ।ತ।ನಮಃ॥\\
\as{೪ ದ್ರೀಂ} ಸರ್ವವಿದ್ರಾವಿಣೀಮುದ್ರಾಶಕ್ತಿ ಶ್ರೀಪಾದುಕಾಂ ಪೂ।ತ।ನಮಃ॥\\
\as{೪ ದ್ರೀಂ ॥}\\
ಮೂಲೇನ ತ್ರಿಃ ಸಂತರ್ಪ್ಯ\\
\as{೪} ಅಭೀಷ್ಟಸಿದ್ಧಿಂ******ದ್ವಿತೀಯಾವರಣಾರ್ಚನಂ ॥
\subsection{ಶ್ರೀಚಕ್ರ ತೃತೀಯಾವರಣದೇವತಾಃ}
{\bfseries ೪ ಹ್ರೀಂ ಕ್ಲೀಂ ಸೌಃ ಸರ್ವಸಂಕ್ಷೋಭಣಚಕ್ರಾಯ ನಮಃ}\\
\as{೪ ಕಂಖಂಗಂಘಂಙಂ} ಅನಂಗಕುಸುಮಾಯೈ ನಮಃ ।೧\\
\as{೪ ಚಂಛಂಜಂಝಂಞಂ} ಅನಂಗಮೇಖಲಾಯೈ ನಮಃ ।೨\\
\as{೪ ಟಂಠಂಡಂಢಂಣಂ} ಅನಂಗಮದನಾಯೈ ನಮಃ ।೩\\
\as{೪ ತಂಥಂದಂಧಂನಂ} ಅನಂಗಮದನಾತುರಾಯೈ ನಮಃ ।೪\\
\as{೪ ಪಂಫಂಬಂಭಂಮಂ} ಅನಂಗರೇಖಾಯೈ ನಮಃ ।೫\\
\as{೪ ಯಂರಂಲಂವಂ} ಅನಂಗವೇಗಿನ್ಯೈ ನಮಃ ।೬\\
\as{೪ ಶಂಷಂಸಂಹಂ} ಅನಂಗಾಂಕುಶಾಯೈ ನಮಃ ।೭\\
\as{೪ ಳಂಕ್ಷಂ} ಅನಂಗಮಾಲಿನ್ಯೈ ನಮಃ ।೮\\
\as{೪ ಹ್ರೀಂ ಕ್ಲೀಂ ಸೌಃ॥} ಸರ್ವಸಂಕ್ಷೋಭಣ ಚಕ್ರಸ್ವಾಮಿನ್ಯೈ ನಮಃ
ಗುಪ್ತತರಯೋಗಿನ್ಯೈ ನಮಃ

\as{೪ ಹ್ರೀಂ ಕ್ಲೀಂ ಸೌಃ॥}  ತ್ರಿಪುರಸುಂದರೀಚಕ್ರೇಶ್ವರೀ ಶ್ರೀಪಾದುಕಾಂ ಪೂ।ತ।ನಮಃ॥\\
\as{೪ ಮಂ} ಮಹಿಮಾಸಿದ್ಧಿ ಶ್ರೀಪಾದುಕಾಂ ಪೂ।ತ।ನಮಃ॥\\
\as{೪ ಕ್ಲೀಂ} ಸರ್ವಾಕರ್ಷಿಣೀಮುದ್ರಾಶಕ್ತಿ ಶ್ರೀಪಾದುಕಾಂ ಪೂ।ತ।ನಮಃ॥\\
\as{೪ ಕ್ಲೀಂ ॥}\\
ಮೂಲೇನ ತ್ರಿಃ ಸಂತರ್ಪ್ಯ\\
\as{೪} ಅಭೀಷ್ಟಸಿದ್ಧಿಂ******ತೃತೀಯಾವರಣಾರ್ಚನಂ ॥
\subsection{ಶ್ರೀಚಕ್ರ ಚತುರ್ಥಾವರಣದೇವತಾಃ}
{\bfseries ೪ ಹೈಂ ಹ್‌ಕ್ಲೀಂ ಹ್‌ಸೌಃ ಸರ್ವಸೌಭಾಗ್ಯದಾಯಕ ಚಕ್ರಾಯ ನಮಃ}\\
\as{೪ ಕಂ} ಸರ್ವಸಂಕ್ಷೋಭಿಣ್ಯೈ ನಮಃ ।೧\\
\as{೪ ಖಂ} ಸರ್ವವಿದ್ರಾವಿಣ್ಯೈ ನಮಃ ।೨\\
\as{೪ ಗಂ} ಸರ್ವಾಕರ್ಷಿಣ್ಯೈ ನಮಃ ।೩\\
\as{೪ ಘಂ} ಸರ್ವಾಹ್ಲಾದಿನ್ಯೈ ನಮಃ ।೪\\
\as{೪ ಙಂ} ಸರ್ವಸಮ್ಮೋಹಿನ್ಯೈ ನಮಃ ।೫\\
\as{೪ ಚಂ} ಸರ್ವಸ್ತಂಭಿನ್ಯೈ ನಮಃ ।೬\\
\as{೪ ಛಂ} ಸರ್ವಜೃಂಭಿಣ್ಯೈ ನಮಃ ।೭\\
\as{೪ ಜಂ} ಸರ್ವವಶಂಕರ್ಯೈ ನಮಃ ।೮\\
\as{೪ ಝಂ} ಸರ್ವರಂಜನ್ಯೈ ನಮಃ ।೯\\
\as{೪ ಞಂ} ಸರ್ವೋನ್ಮಾದಿನ್ಯೈ ನಮಃ ।೧೦\\
\as{೪ ಟಂ} ಸರ್ವಾರ್ಥಸಾಧಿನ್ಯೈ ನಮಃ ।೧೧\\
\as{೪ ಠಂ} ಸರ್ವಸಂಪತ್ತಿಪೂರಣ್ಯೈ ನಮಃ ।೧೨\\
\as{೪ ಡಂ} ಸರ್ವಮಂತ್ರಮಯ್ಯೈ ನಮಃ ।೧೩\\
\as{೪ ಢಂ} ಸರ್ವದ್ವಂದ್ವಕ್ಷಯಂಕರ್ಯೈ ನಮಃ ।೧೪\\
\as{೪ ಹೈಂ ಹ್‌ಕ್ಲೀಂ ಹ್‌ಸೌಃ॥}ಸರ್ವಸೌಭಾಗ್ಯದಾಯಕ ಚಕ್ರಸ್ವಾಮಿನ್ಯೈ ನಮಃ\\
ಸಂಪ್ರದಾಯಯೋಗಿನ್ಯೈ ನಮಃ

\as{೪ ಹೈಂ ಹ್‌ಕ್ಲೀಂ ಹ್‌ಸೌಃ॥} ತ್ರಿಪುರವಾಸಿನೀಚಕ್ರೇಶ್ವರೀ ಶ್ರೀಪಾದುಕಾಂ ಪೂ।ತ।ನಮಃ॥\\
\as{೪ ಈಂ} ಈಶಿತ್ವಸಿದ್ಧಿ ಶ್ರೀಪಾದುಕಾಂ ಪೂ।ತ।ನಮಃ॥\\
\as{೪ ಬ್ಲೂಂ} ಸರ್ವವಶಂಕರೀಮುದ್ರಾಶಕ್ತಿ ಶ್ರೀಪಾದುಕಾಂ ಪೂ।ತ।ನಮಃ॥\\
\as{೪ ಬ್ಲೂಂ ॥}\\
ಮೂಲೇನ ತ್ರಿಃ ಸಂತರ್ಪ್ಯ\\
\as{೪} ಅಭೀಷ್ಟಸಿದ್ಧಿಂ******ತುರೀಯಾವರಣಾರ್ಚನಂ ॥
\newpage
\subsection{ಶ್ರೀಚಕ್ರ ಪಂಚಮಾವರಣದೇವತಾಃ}
{\bfseries ೪ ಹ್‌ಸೈಂ ಹ್‌ಸ್‌ಕ್ಲೀಂ ಹ್‌ಸ್ಸೌಃ ಸರ್ವಾರ್ಥಸಾಧಕ ಚಕ್ರಾಯ ನಮಃ}\\
\as{೪ ಣಂ} ಸರ್ವಸಿದ್ಧಿಪ್ರದಾಯೈ ನಮಃ ।೧\\
\as{೪ ತಂ} ಸರ್ವಸಂಪತ್ಪ್ರದಾಯೈ ನಮಃ ।೨\\
\as{೪ ಥಂ} ಸರ್ವಪ್ರಿಯಂಕರ್ಯೈ ನಮಃ ।೩\\
\as{೪ ದಂ} ಸರ್ವಮಂಗಳಕಾರಿಣ್ಯೈ ನಮಃ ।೪\\
\as{೪ ಧಂ} ಸರ್ವಕಾಮಪ್ರದಾಯೈ ನಮಃ ।೫\\
\as{೪ ನಂ} ಸರ್ವದುಃಖವಿಮೋಚನ್ಯೈ ನಮಃ ।೬\\
\as{೪ ಪಂ} ಸರ್ವಮೃತ್ಯುಪ್ರಶಮನ್ಯೈ ನಮಃ ।೭\\
\as{೪ ಫಂ} ಸರ್ವವಿಘ್ನನಿವಾರಿಣ್ಯೈ ನಮಃ ।೮\\
\as{೪ ಬಂ} ಸರ್ವಾಂಗಸುಂದರ್ಯೈ ನಮಃ ।೯\\
\as{೪ ಭಂ} ಸರ್ವಸೌಭಾಗ್ಯದಾಯಿನ್ಯೈ ನಮಃ ।೧೦\\
\as{೪ ಹ್‌ಸೈಂ ಹ್‌ಸ್‌ಕ್ಲೀಂ ಹ್‌ಸ್ಸೌಃ ॥} ಸರ್ವಾರ್ಥಸಾಧಕ ಚಕ್ರಸ್ವಾಮಿನ್ಯೈ ನಮಃ\\
ಕುಲೋತ್ತೀರ್ಣಯೋಗಿನ್ಯೈ ನಮಃ

\as{೪ ಹ್‌ಸೈಂ ಹ್‌ಸ್‌ಕ್ಲೀಂ ಹ್‌ಸ್ಸೌಃ ॥} ತ್ರಿಪುರಾಶ್ರೀಚಕ್ರೇಶ್ವರೀ ಶ್ರೀಪಾದುಕಾಂ ಪೂ।ತ।ನಮಃ॥\\
\as{೪ ವಂ} ವಶಿತ್ವಸಿದ್ಧಿ ಶ್ರೀಪಾದುಕಾಂ ಪೂ।ತ।ನಮಃ॥\\
\as{೪ ಸಃ }ಸರ್ವೋನ್ಮಾದಿನೀಮುದ್ರಾಶಕ್ತಿ ಶ್ರೀಪಾದುಕಾಂ ಪೂ।ತ।ನಮಃ॥\\
\as{೪ ಸಃ॥}\\
ಮೂಲೇನ ತ್ರಿಃ ಸಂತರ್ಪ್ಯ\\
\as{೪} ಅಭೀಷ್ಟಸಿದ್ಧಿಂ******ಪಂಚಮಾವರಣಾರ್ಚನಂ ॥
\newpage
\subsection{ಶ್ರೀಚಕ್ರ ಷಷ್ಠಾವರಣದೇವತಾಃ}
{\bfseries ೪ ಹ್ರೀಂ ಕ್ಲೀಂ ಬ್ಲೇಂ ಸರ್ವರಕ್ಷಾಕರ ಚಕ್ರಾಯ ನಮಃ}\\
\as{೪ ಮಂ} ಸರ್ವಜ್ಞಾಯೈ ನಮಃ ।೧\\
\as{೪ ಯಂ} ಸರ್ವಶಕ್ತ್ಯೈ ನಮಃ ।೨\\
\as{೪ ರಂ} ಸರ್ವೈಶ್ವರ್ಯಪ್ರದಾಯೈ ನಮಃ ।೩\\
\as{೪ ಲಂ} ಸರ್ವಜ್ಞಾನಮಯ್ಯೈ ನಮಃ ।೪\\
\as{೪ ವಂ} ಸರ್ವವ್ಯಾಧಿವಿನಾಶಿನ್ಯೈ ನಮಃ ।೫\\
\as{೪ ಶಂ} ಸರ್ವಾಧಾರಸ್ವರೂಪಾಯೈ ನಮಃ ।೬\\
\as{೪ ಷಂ} ಸರ್ವಪಾಪಹರಾಯೈ ನಮಃ ।೭\\
\as{೪ ಸಂ} ಸರ್ವಾನಂದಮಯ್ಯೈ ನಮಃ ।೮\\
\as{೪ ಹಂ} ಸರ್ವರಕ್ಷಾಸ್ವರೂಪಿಣ್ಯೈ ನಮಃ ।೯\\
\as{೪ ಕ್ಷಂ} ಸರ್ವೇಪ್ಸಿತಫಲಪ್ರದಾಯೈ ನಮಃ ।೧೦\\
\as{೪ ಹ್ರೀಂ ಕ್ಲೀಂ ಬ್ಲೇಂ॥} ಸರ್ವರಕ್ಷಾಕರಚಕ್ರಸ್ವಾಮಿನ್ಯೈ ನಮಃ\\
ನಿಗರ್ಭಯೋಗಿನ್ಯೈ ನಮಃ

\as{೪ ಹ್ರೀಂ ಕ್ಲೀಂ ಬ್ಲೇಂ॥} ತ್ರಿಪುರಮಾಲಿನೀಚಕ್ರೇಶ್ವರೀ ಶ್ರೀಪಾದುಕಾಂ ಪೂ।ತ।ನಮಃ॥\\
\as{೪ ಪಂ} ಪ್ರಾಕಾಮ್ಯಸಿದ್ಧಿ ಶ್ರೀಪಾದುಕಾಂ ಪೂ।ತ।ನಮಃ॥\\
\as{೪ ಕ್ರೋಂ} ಸರ್ವಮಹಾಂಕುಶಾಮುದ್ರಾಶಕ್ತಿ ಶ್ರೀಪಾದುಕಾಂ ಪೂ।ತ।ನಮಃ॥\\
\as{೪ ಕ್ರೋಂ॥}\\
ಮೂಲೇನ ತ್ರಿಃ ಸಂತರ್ಪ್ಯ\\
\as{೪} ಅಭೀಷ್ಟಸಿದ್ಧಿಂ*******ಷಷ್ಠಾಖ್ಯಾವರಣಾರ್ಚನಂ ॥
\newpage
\subsection{ಶ್ರೀಚಕ್ರ ಸಪ್ತಮಾವರಣದೇವತಾಃ}
{\bfseries ೪ ಹ್ರೀಂ ಶ್ರೀಂ ಸೌಃ ಸರ್ವರೋಗಹರ ಚಕ್ರಾಯ ನಮಃ}\\
\as{೪ ಅಂಆಂ++ಅಃ । ರ್ಬ್ಲೂಂ} ವಶಿನ್ಯೈ ನಮಃ ।೧\\
\as{೪ ಕಂಖಂಗಂಘಂಙಂ । ಕ್‌ಲ್‌ಹ್ರೀಂ} ಕಾಮೇಶ್ವರ್ಯೈ ನಮಃ ।೨\\
\as{೪ ಚಂಛಂಜಂಝಂಞಂ । ನ್‌ವ್ಲೀಂ} ಮೋದಿನ್ಯೈ ನಮಃ ।೩\\
\as{೪ ಟಂಠಂಡಂಢಂಣಂ । ಯ್ಲೂಂ} ವಿಮಲಾಯೈ ನಮಃ ।೪\\
\as{೪ ತಂಥಂದಂಧಂನಂ । ಜ್‌ಮ್ರೀಂ} ಅರುಣಾಯೈ ನಮಃ ।೫\\
\as{೪ ಪಂಫಂಬಂಭಂಮಂ । ಹ್‌ಸ್‌ಲ್‌ವ್ಯೂಂ} ಜಯಿನ್ಯೈ ನಮಃ ।೬\\
\as{೪ ಯಂರಂಲಂವಂ । ಝ್‌ಮ್‌ರ್ಯೂಂ} ಸರ್ವೇಶ್ವರ್ಯೈ ನಮಃ ।೭\\
\as{೪ ಶಂಷಂಸಂಹಂಳಂಕ್ಷಂ । ಕ್ಷ್‌ಮ್ರೀಂ} ಕೌಳಿನ್ಯೈ ನಮಃ ।೮\\
\as{೪ ಹ್ರೀಂ ಶ್ರೀಂ ಸೌಃ॥} ಸರ್ವರೋಗಹರಚಕ್ರಸ್ವಾಮಿನ್ಯೈ ನಮಃ\\
ರಹಸ್ಯಯೋಗಿನ್ಯೈ ನಮಃ

\as{೪ ಹ್ರೀಂ ಶ್ರೀಂ ಸೌಃ॥} ತ್ರಿಪುರಾಸಿದ್ಧಾಚಕ್ರೇಶ್ವರೀ ಶ್ರೀಪಾದುಕಾಂ ಪೂ।ತ।ನಮಃ॥\\
\as{೪ ಭುಂ} ಭುಕ್ತಿಸಿದ್ಧಿ ಶ್ರೀಪಾದುಕಾಂ ಪೂ।ತ।ನಮಃ॥\\
\as{೪ ಹ್‌ಸ್‌ಖ್‌ಫ್ರೇಂ} ಸರ್ವಖೇಚರೀಮುದ್ರಾಶಕ್ತಿ ಶ್ರೀಪಾದುಕಾಂ ಪೂ।ತ।ನಮಃ॥\\
\as{೪ ಹ್‌ಸ್‌ಖ್‌ಫ್ರೇಂ ॥}\\
ಮೂಲೇನ ತ್ರಿಃ ಸಂತರ್ಪ್ಯ\\
\as{೪} ಅಭೀಷ್ಟಸಿದ್ಧಿಂ********ಸಪ್ತಮಾವರಣಾರ್ಚನಂ ॥
\subsection{ಶ್ರೀಚಕ್ರ ಅಷ್ಟಮಾವರಣದೇವತಾಃ}
\as{೪ ಯಾಂರಾಂಲಾಂವಾಂಸಾಂ ದ್ರಾಂದ್ರೀಂಕ್ಲೀಂಬ್ಲೂಂಸಃ} ಬಾಣಿನ್ಯೈ ನಮಃ ।೧\\
\as{೪ ಥಂಧಂ} ಚಾಪಿನ್ಯೈ ನಮಃ ।೨\\
\as{೪ ಹ್ರೀಂಆಂ} ಪಾಶಿನ್ಯೈ ನಮಃ ।೩\\
\as{೪ ಕ್ರೋಂಕ್ರೋಂ} ಅಂಕುಶಿನ್ಯೈ ನಮಃ ।೪

\as{೪ ಹ್‌ಸ್‌ರೈಂ ಹ್‌ಸ್‌ಕ್ಲ್ರೀಂ ಹ್‌ಸ್‌ರ್ಸೌಃ ಸರ್ವಸಿದ್ಧಿಪ್ರದ ಚಕ್ರಾಯ ನಮಃ}\\
\as{೪ ಐಂ೫} ಮಹಾಕಾಮೇಶ್ವರ್ಯೈ ನಮಃ ।೧\\
\as{೪ ಕ್ಲೀಂ೬} ಮಹಾವಜ್ರೇಶ್ವರ್ಯೈ ನಮಃ ।೨\\
\as{೪ ಸೌಃ೪} ಮಹಾಭಗಮಾಲಿನ್ಯೈ ನಮಃ ।೩\\
\as{೪ ೧೫} ಮಹಾಶ್ರೀಸುಂದರ್ಯೈ ನಮಃ । (ಬಿಂದೌ)\\
\as{೪ ಹ್‌ಸ್‌ರೈಂ ಹ್‌ಸ್‌ಕ್ಲ್ರೀಂ ಹ್‌ಸ್‌ರ್ಸೌಃ ॥} ಸರ್ವಸಿದ್ಧಿಪ್ರದಚಕ್ರಸ್ವಾಮಿನ್ಯೈ ನಮಃ\\
ಅತಿರಹಸ್ಯಯೋಗಿನ್ಯೈ ನಮಃ

{\bfseries ೪ ಐಂ ೫॥} ಸೂರ್ಯಚಕ್ರೇ ಕಾಮಗಿರಿಪೀಠೇ ಮಿತ್ರೀಶನಾಥ ನವಯೋನಿ ಚಕ್ರಾತ್ಮಕ ಆತ್ಮತತ್ವ ಸಂಹಾರಕೃತ್ಯ ಜಾಗ್ರದ್ದಶಾಧಿಷ್ಠಾಯಕ ಇಚ್ಛಾಶಕ್ತಿ ವಾಗ್ಭವಾತ್ಮಕ ಪರಾಪರಶಕ್ತಿ ಸ್ವರೂಪ ರುದ್ರಾತ್ಮಶಕ್ತಿ ಮಹಾಕಾಮೇಶ್ವರೀ ಶ್ರೀಪಾದುಕಾಂ ಪೂ।ತ।ನಮಃ॥೧

{\bfseries೪ ಕ್ಲೀಂ ೬॥} ಸೋಮಚಕ್ರೇ ಪೂರ್ಣಗಿರಿಪೀಠೇ ಉಡ್ಡೀಶನಾಥ  ದಶಾರದ್ವಯ ಚತುರ್ದಶಾರ ಚಕ್ರಾತ್ಮಕ ವಿದ್ಯಾತತ್ವ ಸ್ಥಿತಿಕೃತ್ಯ ಸ್ವಪ್ನದಶಾಧಿಷ್ಠಾಯಕ ಜ್ಞಾನಶಕ್ತಿ ಕಾಮರಾಜಾತ್ಮಕ ಕಾಮಕಲಾ ಸ್ವರೂಪ ವಿಷ್ಣ್ವಾತ್ಮಶಕ್ತಿ ಮಹಾವಜ್ರೇಶ್ವರೀ ಶ್ರೀಪಾದುಕಾಂ ಪೂ।ತ।ನಮಃ॥೨

{\bfseries೪ ಸೌಃ ೪॥} ಅಗ್ನಿಚಕ್ರೇ ಜಾಲಂಧರಪೀಠೇ  ಷಷ್ಠೀಶನಾಥ ಅಷ್ಟದಳ ಷೋಡಶದಳ ಚತುರಸ್ರ ಚಕ್ರಾತ್ಮಕ ಶಿವತತ್ವ ಸೃಷ್ಟಿಕೃತ್ಯ ಸುಷುಪ್ತಿದಶಾಧಿಷ್ಠಾಯಕ ಕ್ರಿಯಾಶಕ್ತಿ ಶಕ್ತಿಬೀಜಾತ್ಮಕ ವಾಗೀಶ್ವರೀ ಸ್ವರೂಪ ಬ್ರಹ್ಮಾತ್ಮಶಕ್ತಿ  ಮಹಾಭಗಮಾಲಿನೀ ಶ್ರೀಪಾದುಕಾಂ ಪೂ।ತ।ನಮಃ॥೩

{\bfseries೪ ೧೫॥} ಪರಬ್ರಹ್ಮಚಕ್ರೇ ಮಹೋಡ್ಯಾಣಪೀಠೇ ಚರ್ಯಾನಂದನಾಥ ಸಮಸ್ತಚಕ್ರಾತ್ಮಕ ಸಪರಿವಾರ ಪರಮತತ್ವ ಸೃಷ್ಟಿಸ್ಥಿತಿಸಂಹಾರಕೃತ್ಯ ತುರೀಯದಶಾಧಿಷ್ಠಾಯಕ ಇಚ್ಛಾಜ್ಞಾನಕ್ರಿಯಾಶಾಂತಶಕ್ತಿ ವಾಗ್ಭವ ಕಾಮರಾಜ ಶಕ್ತಿಬೀಜಾತ್ಮಕ ಪರಮಶಕ್ತಿಸ್ವರೂಪ ಪರಬ್ರಹ್ಮಾತ್ಮಶಕ್ತಿ ಶ್ರೀಮಹಾತ್ರಿಪುರಸುಂದರೀ ಶ್ರೀಪಾದುಕಾಂ ಪೂ।ತ।ನಮಃ॥(ಬಿಂದೌ)

\as{೪ ಹ್‌ಸ್‌ರೈಂ ಹ್‌ಸ್‌ಕ್ಲ್ರೀಂ ಹ್‌ಸ್‌ರ್ಸೌಃ ॥} ತ್ರಿಪುರಾಂಬಾ ಚಕ್ರೇಶ್ವರೀ ಶ್ರೀಪಾದುಕಾಂ ಪೂ।ತ।ನಮಃ॥\\
\as{೪ ಇಂ} ಇಚ್ಛಾಸಿದ್ಧಿ ಶ್ರೀಪಾದುಕಾಂ ಪೂ।ತ।ನಮಃ॥\\
\as{೪ ಹ್ಸೌಃ} ಸರ್ವಬೀಜಮುದ್ರಾಶಕ್ತಿ ಶ್ರೀಪಾದುಕಾಂ ಪೂ।ತ।ನಮಃ॥\\
\as{೪ ಹ್ಸೌಃ॥}\\
ಮೂಲೇನ ತ್ರಿಃ ಸಂತರ್ಪ್ಯ\\
\as{೪} ಅಭೀಷ್ಟಸಿದ್ಧಿಂ*******ಅಷ್ಟಮಾವರಣಾರ್ಚನಂ ॥
\subsection{ಶ್ರೀಚಕ್ರ ನವಮಾವರಣದೇವತಾ}
{\bfseries ೪ (೧೫) ಸರ್ವಾನಂದಮಯ ಚಕ್ರಾಯ ನಮಃ}\\
\as{೪ ೧೫} ಶ್ರೀ ಶ್ರೀ ಮಹಾಭಟ್ಟಾರಿಕಾಯೈ ನಮಃ ।(ಬಿಂದೌ)\\
\as{೪ ೧೫} ಸರ್ವಾನಂದಮಯಚಕ್ರಸ್ವಾಮಿನ್ಯೈ ನಮಃ ।\\
ಪರಾಪರರಹಸ್ಯಯೋಗಿನ್ಯೈ ನಮಃ ।

%\as{೪ ಅಂ ಆಂ ಸೌಃ} ತ್ರಿಪುರಾಯೈ ನಮಃ\\ \as{೪ ಐಂ ಕ್ಲೀಂ ಸೌಃ} ತ್ರಿಪುರೇಶ್ಯೈ ನಮಃ\\ \as{೪ ಹ್ರೀಂ ಕ್ಲೀಂ ಸೌಃ} ತ್ರಿಪುರಸುಂದರ್ಯೈ ನಮಃ\\ \as{೪ ಹೈಂ ಹ್‌ಕ್ಲೀಂ ಹ್‌ಸೌಃ} ತ್ರಿಪುರವಾಸಿನ್ಯೈ ನಮಃ\\ \as{೪ ಹ್‌ಸೈಂ ಹ್‌ಸ್‌ಕ್ಲೀಂ ಹ್‌ಸ್ಸೌಃ} ತ್ರಿಪುರಾಶ್ರಿಯೈ ನಮಃ\\ \as{೪ ಹ್ರೀಂ ಕ್ಲೀಂ ಬ್ಲೇಂ} ತ್ರಿಪುರಮಾಲಿನ್ಯೈ ನಮಃ\\ \as{೪ ಹ್ರೀಂ ಶ್ರೀಂ ಸೌಃ} ತ್ರಿಪುರಾಸಿದ್ಧಾಯೈ ನಮಃ\\ \as{೪ ಹ್‌ಸ್‌ರೈಂ ಹ್‌ಸ್‌ಕ್ಲ್ರೀಂ ಹ್‌ಸ್‌ರ್ಸೌಃ} ತ್ರಿಪುರಾಂಬಾಯೈ ನಮಃ\\ \as{೪ ೧೫} ಮಹಾತ್ರಿಪುರಸುಂದರ್ಯೈ ನಮಃ

\as{೪} ಮಹಾಮಹೇಶ್ವರ್ಯೈ ನಮಃ ।\\
\as{೪} ಮಹಾಮಹಾರಾಜ್ಞ್ಯೈ ನಮಃ ।\\
\as{೪} ಮಹಾಮಹಾಶಕ್ತ್ಯೈ ನಮಃ ।\\
\as{೪} ಮಹಾಮಹಾಗುಪ್ತಾಯೈ ನಮಃ ।\\
\as{೪} ಮಹಾಮಹಾಜ್ಞಪ್ತ್ಯೈ ನಮಃ ।\\
\as{೪} ಮಹಾಮಹಾನಂದಾಯೈ ನಮಃ ।\\
\as{೪} ಮಹಾಮಹಾಸ್ಪಂದಾಯೈ ನಮಃ ।\\
\as{೪} ಮಹಾಮಹಾಶಯಾಯೈ ನಮಃ ।\\
\as{೪} ಮಹಾಮಹಾ ಶ್ರೀಚಕ್ರನಗರಸಾಮ್ರಾಜ್ಞ್ಯೈ ನಮಃ ॥

{\bfseries ೪ (೧೫)} ಲಲಿತಾ ಮಹಾತ್ರಿಪುರಸುಂದರೀ ಪರಾಭಟ್ಟಾರಿಕಾ ಶ್ರೀಪಾದುಕಾಂ ಪೂ।ತ।ನಮಃ॥ಇತಿ ತ್ರಿಃ ಸಂತರ್ಪ್ಯ\\
\as{೪ (೧೫) }ಮಹಾತ್ರಿಪುರಸುಂದರೀ ಚಕ್ರೇಶ್ವರೀ ಶ್ರೀಪಾದುಕಾಂ ಪೂ।ತ।ನಮಃ॥\\
\as{೪ ಪಂ} ಪ್ರಾಪ್ತಿಸಿದ್ಧಿ ಶ್ರೀಪಾದುಕಾಂ ಪೂ।ತ।ನಮಃ॥\\
\as{೪ ಐಂ} ಸರ್ವಯೋನಿಮುದ್ರಾಶಕ್ತಿ ಶ್ರೀಪಾದುಕಾಂ ಪೂ।ತ।ನಮಃ॥\\
\as{೪ ಐಂ ॥}\\
ಮೂಲೇನ ತ್ರಿಃ ಸಂತರ್ಪ್ಯ

 [\,ಷೋಡಶೀ ಉಪಾಸಕರು :\\ 
\as{೪ ಹಸಕಲ ಹಸಕಹಲ ಸಕಲಹ್ರೀಂ ॥} ತುರೀಯಾಂಬಾ ಶ್ರೀಪಾದುಕಾಂ ಪೂ।ತ।ನಮಃ॥ (ಇತಿ ತ್ರಿಃ ಸಂತರ್ಪ್ಯ)\\
\as{೪} ಸರ್ವಾನಂದಮಯೇ ಚಕ್ರೇ ಮಹೋಡ್ಯಾಣಪೀಠೇ ಚರ್ಯಾನಂದನಾಥಾತ್ಮಕ ತುರೀಯಾತೀತದಶಾಧಿಷ್ಠಾಯಕ ಶಾಂತ್ಯತೀತಕಲಾತ್ಮಕ ಪ್ರಕಾಶ ವಿಮರ್ಶ ಸಾಮರಸ್ಯಾತ್ಮಕ ಪರಬ್ರಹ್ಮಸ್ವರೂಪಿಣೀ ಪರಾಮೃತಶಕ್ತಿಃ ಸರ್ವ ಮಂತ್ರೇಶ್ವರೀ ಸರ್ವಪೀಠೇಶ್ವರೀ ಸರ್ವವೀರೇಶ್ವರೀ ಸಕಲಜಗದುತ್ಪತ್ತಿ ಮಾತೃಕಾ ಸಚಕ್ರಾ ಸದೇವತಾ ಸಾಸನಾ ಸಾಯುಧಾ ಸಶಕ್ತಿಃ ಸವಾಹನಾ ಸಪರಿವಾರಾ ಸಚಕ್ರೇಶೀಕಾ ಪರಯಾ ಅಪರಯಾ ಪರಾಪರಯಾ ಸಪರ್ಯಯಾ ಸರ್ವೋಪಚಾರೈಃ ಸಂಪೂಜಿತಾ ಸಂತರ್ಪಿತಾ ಸಂತುಷ್ಟಾ ಅಸ್ತು ನಮಃ ॥(ಇತಿ ಸಮಷ್ಟ್ಯಂಜಲಿಃ)

\as{೪ ಸಂ} ಸರ್ವಕಾಮಸಿದ್ಧಿ ಶ್ರೀಪಾದುಕಾಂ ಪೂ।ತ।ನಮಃ॥\\
\as{೪ ಹ್‌ಸ್‌ರೈಂ ಹ್‌ಸ್‌ಕ್ಲ್ರೀಂ ಹ್‌ಸ್‌ರ್ಸೌಃ} ಸರ್ವತ್ರಿಖಂಡಾ ಮುದ್ರಾಶಕ್ತಿ ಶ್ರೀಪಾದುಕಾಂ ಪೂ।ತ।ನಮಃ॥\\
\as{೪ ಹ್‌ಸ್‌ರೈಂ ಹ್‌ಸ್‌ಕ್ಲ್ರೀಂ ಹ್‌ಸ್‌ರ್ಸೌಃ ॥}\\
\as{೪ (೧೬)} ಮಹಾತ್ರಿಪುರಸುಂದರೀ ಶ್ರೀಪಾದುಕಾಂ ಪೂ।ತ।ನಮಃ॥(ಇತಿ ತ್ರಿಃ)]\,\\
\as{೪} ಅಭೀಷ್ಟಸಿದ್ಧಿಂ******ನವಮಾವರಣಾರ್ಚನಂ ॥\\
ಇತ್ಯಾವರಣಾರ್ಚನಂ
\section{ಪಂಚಪಂಚಿಕಾಪೂಜಾ}
\subsection{ಪಂಚಲಕ್ಷ್ಮ್ಯಂಬಾಃ}
{\bfseries ೪ ೧೫ ॥} ಶ್ರೀವಿದ್ಯಾ ಲಕ್ಷ್ಮ್ಯಂಬಾ ಶ್ರೀಪಾದುಕಾಂ ಪೂ।ತ।ನಮಃ॥\\
\as{೪ ಶ್ರೀಂ ॥}ಲಕ್ಷ್ಮೀಲಕ್ಷ್ಮ್ಯಂಬಾ ಶ್ರೀಪಾದುಕಾಂ ಪೂ।ತ।ನಮಃ॥೧\\
\as{೪ ಓಂಶ್ರೀಂ ಹ್ರೀಂ ಶ್ರೀಂ ಕಮಲೇ ಕಮಲಾಲಯೇ ಪ್ರಸೀದ ಪ್ರಸೀದ ಶ್ರೀಂ ಹ್ರೀಂ ಶ್ರೀಂ ಓಂ ಮಹಾಲಕ್ಷ್ಮ್ಯೈ ನಮಃ ॥}\\ಮಹಾಲಕ್ಷ್ಮೀಲಕ್ಷ್ಮ್ಯಂಬಾ ಶ್ರೀಪಾದುಕಾಂ ಪೂ।ತ।ನಮಃ॥೨\\
\as{೪ ಶ್ರೀಂ ಹ್ರೀಂ ಕ್ಲೀಂ ॥}ತ್ರಿಶಕ್ತಿಲಕ್ಷ್ಮ್ಯಂಬಾ ಶ್ರೀಪಾದುಕಾಂ ಪೂ।ತ।ನಮಃ॥೩\\
\as{೪ ಶ್ರೀಂ ಸಹಕಲ ಹ್ರೀಂ ಶ್ರೀಂ ॥} \\ಸರ್ವಸಾಮ್ರಾಜ್ಯಲಕ್ಷ್ಮ್ಯಂಬಾ ಶ್ರೀಪಾದುಕಾಂ ಪೂ।ತ।ನಮಃ॥೪
\subsection{ಪಂಚಕೋಶಾಂಬಾಃ}
{\bfseries ೪ ೧೫ ॥} ಶ್ರೀವಿದ್ಯಾ ಕೋಶಾಂಬಾ ಶ್ರೀಪಾದುಕಾಂ ಪೂ।ತ।ನಮಃ॥\\
\as{೪ ಓಂ ಹ್ರೀಂ ಹಂಸಃ ಸೋಹಂ ಸ್ವಾಹಾ ॥}\\ ಪರಂಜ್ಯೋತಿಃಕೋಶಾಂಬಾ ಶ್ರೀಪಾದುಕಾಂ ಪೂ।ತ।ನಮಃ॥೧\\
\as{೪ ಓಂ ಹಂಸಃ ॥} ಪರಾನಿಷ್ಕಲಾ ಕೋಶಾಂಬಾ ಶ್ರೀಪಾದುಕಾಂ ಪೂ।ತ।ನಮಃ॥೨\\
\as{೪ ಹಂಸಃ ॥}ಅಜಪಾಕೋಶಾಂಬಾ ಶ್ರೀಪಾದುಕಾಂ ಪೂ।ತ।ನಮಃ॥೩\\
\as{೪ ಅಂ ಆಂ ಇಂ ಈಂ++++ಳಂ ಕ್ಷಂ ॥} ಮಾತೃಕಾಕೋಶಾಂಬಾ ಶ್ರೀಪಾದುಕಾಂ ಪೂ।ತ।ನಮಃ॥೪
\subsection{ಪಂಚಕಲ್ಪಲತಾಂಬಾಃ}
{\bfseries ೪ ೧೫ ॥} ಶ್ರೀವಿದ್ಯಾ ಕಲ್ಪಲತಾಂಬಾ ಶ್ರೀಪಾದುಕಾಂ ಪೂ।ತ।ನಮಃ॥\\
\as{೪ ಹ್ರೀಂ ಕ್ಲೀಂ ಐಂ ಬ್ಲೂಂ ಸ್ತ್ರೀಂ ॥} ತ್ವರಿತಾ ಕಲ್ಪಲತಾಂಬಾ ಶ್ರೀಪಾದುಕಾಂ ಪೂ।ತ।ನಮಃ॥೧\\
\as{೪ ಓಂ ಹ್ರೀಂ ಹ್ರಾಂ ಹಸಕಲಹ್ರೀಂ ಓಂ ಸರಸ್ವತ್ಯೈ ನಮಃ ಹ್‌ಸ್ರೈಂ ॥}\\ಪಾರಿಜಾತೇಶ್ವರೀ ಕಲ್ಪಲತಾಂಬಾ ಶ್ರೀಪಾದುಕಾಂ ಪೂ।ತ।ನಮಃ॥೨\\
\as{೪ ಶ್ರೀಂ ಹ್ರೀಂ ಕ್ಲೀಂ ಐಂ ಕ್ಲೀಂ ಸೌಃ ॥}\\ ತ್ರಿಪುಟಾ ಕಲ್ಪಲತಾಂಬಾ ಶ್ರೀಪಾದುಕಾಂ ಪೂ।ತ।ನಮಃ॥೩\\
\as{೪ ದ್ರಾಂ ದ್ರೀಂ ಕ್ಲೀಂ ಬ್ಲೂಂ ಸಃ ॥}\\ ಪಂಚಬಾಣೇಶ್ವರೀ ಕಲ್ಪಲತಾಂಬಾ ಶ್ರೀಪಾದುಕಾಂ ಪೂ।ತ।ನಮಃ॥೪
\subsection{ಪಂಚಕಾಮದುಘಾಂಬಾಃ}
{\bfseries ೪ ೧೫ ॥} ಶ್ರೀವಿದ್ಯಾ ಕಾಮದುಘಾಂಬಾ ಶ್ರೀಪಾದುಕಾಂ ಪೂ।ತ।ನಮಃ॥\\
\as{೪ ಓಂ ಹ್ರೀಂ ಹಂಸಃ ಜುಂ ಸಂಜೀವನಿ ಜೀವಂ ಪ್ರಾಣಗ್ರಂಥಿಸ್ಥಂ ಕುರು ಕುರು ಸ್ವಾಹಾ ॥} ಅಮೃತಪೀಠೇಶ್ವರೀ ಕಾಮದುಘಾಂಬಾ ಶ್ರೀಪಾದುಕಾಂ ಪೂ।ತ।ನಮಃ॥೧\\
\as{೪ ಐಂ ವದ ವದ ವಾಗ್ವಾದಿನಿ ಹ್‌ಸ್ರೈಂ ಕ್ಲೀಂ ಕ್ಲಿನ್ನೇ ಕ್ಲೇದಿನಿ ಮಹಾಕ್ಷೋಭಂ ಕುರು ಕುರು ಹ್‌ಸ್‌ಕ್ಲ್ರೀಂ ಸೌಃ ಓಂ ಮೋಕ್ಷಂ ಕುರು ಕುರು ಹ್‌ಸ್‌ರ್ಸೌಃ~।}\\ಸುಧಾಸೂಕಾಮದುಘಾಂಬಾ ಶ್ರೀಪಾದುಕಾಂ ಪೂ।ತ।ನಮಃ॥೨\\
\as{೪ ಐಂ ಬ್ಲೂಂ ಝ್ರೌಂ ಜುಂ ಸಃ ಅಮೃತೇ ಅಮೃತೋದ್ಭವೇ ಅಮೃತೇಶ್ವರಿ \\ಅಮೃತವರ್ಷಿಣಿ ಅಮೃತಂ ಸ್ರಾವಯ ಸ್ರಾವಯ ಸ್ವಾಹಾ ॥}\\ ಅಮೃತೇಶ್ವರೀ ಕಾಮದುಘಾಂಬಾ ಶ್ರೀಪಾದುಕಾಂ ಪೂ।ತ।ನಮಃ॥೩\\
\as{೪ ಓಂ ಹ್ರೀಂಶ್ರೀಂಕ್ಲೀಂ ನಮೋ ಭಗವತಿ ಮಾಹೇಶ್ವರಿ ಅನ್ನಪೂರ್ಣೇ ಸ್ವಾಹಾ ॥}\\ ಅನ್ನಪೂರ್ಣಾಕಾಮದುಘಾಂಬಾ ಶ್ರೀಪಾದುಕಾಂ ಪೂ।ತ।ನಮಃ॥೪
\subsection{ಪಂಚರತ್ನಾಂಬಾಃ}
{\bfseries ೪ ೧೫ ॥} ಶ್ರೀವಿದ್ಯಾ ರತ್ನಾಂಬಾ ಶ್ರೀಪಾದುಕಾಂ ಪೂ।ತ।ನಮಃ॥\\
\as{೪ ಜ್‌ಝ್ರೀಂ ಮಹಾಚಂಡೇ ತೇಜಃಕರ್ಷಿಣಿ ಕಾಲಮಂಥಾನೇ ಹಃ ॥}\\ಸಿದ್ಧಲಕ್ಷ್ಮೀರತ್ನಾಂಬಾ ಶ್ರೀಪಾದುಕಾಂ ಪೂ।ತ।ನಮಃ॥೧\\
\as{೪ ಐಂಹ್ರೀಂಶ್ರೀಂ ಐಂಕ್ಲೀಂಸೌಃ ಓಂ ನಮೋ ಭಗವತಿ ರಾಜಮಾತಂಗೀಶ್ವರಿ ಸರ್ವಜನಮನೋಹರಿ ಸರ್ವಮುಖರಂಜನಿ ಕ್ಲೀಂಹ್ರೀಂಶ್ರೀಂ ಸರ್ವರಾಜವಶಂಕರಿ ಸರ್ವಸ್ತ್ರೀಪುರುಷ ವಶಂಕರಿ ಸರ್ವ ದುಷ್ಟಮೃಗವಶಂಕರಿ ಸರ್ವ ಸತ್ವವಶಂಕರಿ ಸರ್ವ ಲೋಕವಶಂಕರಿ ತ್ರೈಲೋಕ್ಯಂ ಮೇ ವಶಮಾನಯ ಸ್ವಾಹಾ ಸೌಃಕ್ಲೀಂಐಂ ಶ್ರೀಂಹ್ರೀಂಐಂ॥}ರಾಜಮಾತಂಗೀಶ್ವರೀರತ್ನಾಂಬಾ ಶ್ರೀಪಾದುಕಾಂ ಪೂ।ತ।ನಮಃ॥೨\\
\as{೪ ಶ್ರೀಂ ಹ್ರೀಂ ಶ್ರೀಂ ॥}ಭುವನೇಶ್ವರೀರತ್ನಾಂಬಾ ಶ್ರೀಪಾದುಕಾಂ ಪೂ।ತ।ನಮಃ॥೩\\
\as{೪ ಐಂ ಗ್ಲೌಂ ಐಂ ನಮೋ ಭಗವತಿ ವಾರ್ತಾಲಿ ವಾರ್ತಾಲಿ ವಾರಾಹಿ ವಾರಾಹಿ ವರಾಹಮುಖಿ ವರಾಹಮುಖಿ ಅಂಧೇ ಅಂಧಿನಿ ನಮಃ ರುಂಧೇ ರುಂಧಿನಿ ನಮಃ ಜಂಭೇ ಜಂಭಿನಿ ನಮಃ ಮೋಹೇ ಮೋಹಿನಿ ನಮಃ ಸ್ತಂಭೇ ಸ್ತಂಭಿನಿ ನಮಃ ಸರ್ವದುಷ್ಟಪ್ರದುಷ್ಟಾನಾಂ ಸರ್ವೇಷಾಂ ಸರ್ವವಾಕ್ಚಿತ್ತ ಚಕ್ಷುರ್ಮುಖಗತಿ ಜಿಹ್ವಾ  ಸ್ತಂಭನಂ ಕುರು ಕುರು ಶೀಘ್ರಂ ವಶ್ಯಂ ಐಂ ಗ್ಲೌಂ ಐಂ ಠಃಠಃಠಃಠಃ ಹುಂ ಫಟ್ ಸ್ವಾಹಾ ॥}ವಾರಾಹೀರತ್ನಾಂಬಾ ಶ್ರೀಪಾದುಕಾಂ ಪೂ।ತ।ನಮಃ॥೪
\newpage
\section{ಆಮ್ನಾಯಸಮಷ್ಟಿಪೂಜಾ}
\as{೪ ಹ್‌ಸ್‌ರೈಂ ಹ್‌ಸ್‌ಕ್ಲ್ರೀಂ ಹ್‌ಸ್‌ರ್ಸೌಃ }॥ ಪೂರ್ವಾಮ್ನಾಯ ಸಮಯವಿದ್ಯೇಶ್ವರೀ ಉನ್ಮೋದಿನೀ ದೇವ್ಯಂಬಾ ಶ್ರೀಪಾ।ಪೂ।ತ।ನಮಃ॥\\
\as{೪ ೧೫॥} ಪೂರ್ವಾಮ್ನಾಯ ಸಮಷ್ಟಿರೂಪಿಣೀ ಮಹಾತ್ರಿಪುರಸುಂದರೀ ಶ್ರೀಪಾ।ಪೂ।ತ।ನಮಃ॥\\
\as{೪ ಓಂ ಹ್ರೀಂ ಐಂ ಕ್ಲಿನ್ನೇ ಕ್ಲಿನ್ನಮದದ್ರವೇ ಕುಲೇ ಹ್ಸೌಃ }॥ ದಕ್ಷಿಣಾಮ್ನಾಯ ಸಮಯವಿದ್ಯೇಶ್ವರೀ ಭೋಗಿನೀ ದೇವ್ಯಂಬಾ ಶ್ರೀಪಾ।ಪೂ।ತ।ನಮಃ॥\\
\as{೪ ೧೫॥} ದಕ್ಷಿಣಾಮ್ನಾಯ ಸಮಷ್ಟಿರೂಪಿಣೀ ಮಹಾತ್ರಿಪುರಸುಂದರೀ ಶ್ರೀಪಾ।ಪೂ।ತ।ನಮಃ॥\\
\as{೪ ಹ್ಸ್ರೈಂ ಹ್ಸ್ರೀಂ ಹ್ಸ್ರೌಃ ಹ್‌ಸ್‌ಖ್‌ಫ್ರೇಂ ಭಗವತ್ಯಂಬೇ ಹಸಕ್ಷಮಲವರಯೂಂ ಹ್‌ಸ್‌ಖ್‌ಫ್ರೇಂ ಅಘೋರಮುಖಿ ಛ್ರಾಂ ಛ್ರೀಂ ಕಿಣಿ ಕಿಣಿ ವಿಚ್ಚೇ ಹ್‌ಸ್ರೌಃ ಹ್‌ಸ್‌ಖ್‌ಫ್ರೇಂ ಹ್‌ಸ್ರೌಃ }॥ ಪಶ್ಚಿಮಾಮ್ನಾಯ ಸಮಯವಿದ್ಯೇಶ್ವರೀ ಕುಬ್ಜಿಕಾ ದೇವ್ಯಂಬಾ ಶ್ರೀಪಾ।ಪೂ।ತ।ನಮಃ॥\\
\as{೪ ೧೫॥} ಪಶ್ಚಿಮಾಮ್ನಾಯ ಸಮಷ್ಟಿರೂಪಿಣೀ ಮಹಾತ್ರಿಪುರಸುಂದರೀ ಶ್ರೀಪಾ।ಪೂ।ತ।ನಮಃ॥\\
\as{೪ ಹ್‌ಸ್‌ಖ್‌ಫ್ರೇಂ  ಮಹಾಚಂಡಯೋಗೀಶ್ವರಿ ಕಾಳಿಕೇ ಫಟ್ }॥ ಉತ್ತರಾಮ್ನಾಯ ಸಮಯವಿದ್ಯೇಶ್ವರೀ ಕಾಳಿಕಾ ದೇವ್ಯಂಬಾ ಶ್ರೀಪಾ।ಪೂ।ತ।ನಮಃ॥\\
\as{೪ ೧೫॥} ಉತ್ತರಾಮ್ನಾಯ ಸಮಷ್ಟಿರೂಪಿಣೀ ಮಹಾತ್ರಿಪುರಸುಂದರೀ ಶ್ರೀಪಾ।ಪೂ।ತ।ನಮಃ॥

 [\,ಷೋಡಶೀ ಉಪಾಸಕರು :\\
 \as{೪ ಮಖಪರಯಘಚ್ ಮಹಿಚನಡಯಙ್ ಗಂಶಫರ್ }॥ ಊರ್ಧ್ವಾಮ್ನಾಯ ಸಮಯವಿದ್ಯೇಶ್ವರೀ ಚೈತನ್ಯಭೈರವ್ಯಂಬಾ ಶ್ರೀಪಾ।ಪೂ।ತ।ನಮಃ॥\\
\as{೪ ೧೬॥} ಊರ್ಧ್ವಾಮ್ನಾಯ ಸಮಷ್ಟಿರೂಪಿಣೀ ಮಹಾತ್ರಿಪುರಸುಂದರೀ ಶ್ರೀಪಾ।ಪೂ।ತ।ನಮಃ॥\\
\as{೪ ಭಗವತಿ ವಿಚ್ಚೇ ಮಹಾಮಾಯೇ ಮಾತಂಗಿನಿ ಬ್ಲೂಂ ಅನುತ್ತರವಾಗ್ವಾದಿನಿ ಹ್‌ಸ್‌ಖ್‌ಫ್ರೇಂ ಹ್‌ಸ್‌ಖ್‌ಫ್ರೇಂ ಹ್‌ಸ್ರೌಃ }॥ ಅನುತ್ತರಶಾಂಕರ್ಯಂಬಾ ಶ್ರೀಪಾ।ಪೂ।ತ।ನಮಃ॥\\
\as{೪ ೧೬॥} ಅನುತ್ತರಾಮ್ನಾಯ ಸಮಷ್ಟಿರೂಪಿಣೀ ಮಹಾತ್ರಿಪುರಸುಂದರೀ ಶ್ರೀಪಾ।ಪೂ।ತ।ನಮಃ॥]\,
\section{ಷಡ್ದರ್ಶನಪೂಜಾ}
\as{೪ ತಾರೇ ತುತ್ತಾರೇ ತುರೇ ಸ್ವಾಹಾ }। ತಾರಾ ದೇವತಾಧಿಷ್ಠಿತ ಬೌದ್ಧ ದರ್ಶನ ಶ್ರೀಪಾ।ಪೂ।ತ।ನಮಃ॥(ಸ್ಥಿತಿಚಕ್ರೇ)\\
\as{೪ (ಗಾಯತ್ರೀ) }। ಬ್ರಹ್ಮ ದೇವತಾಧಿಷ್ಠಿತ ವೈದಿಕ ದರ್ಶನ ಶ್ರೀಪಾ।ಪೂ।ತ।ನಮಃ॥(ಪ್ರಥಮಭೂಪುರೇ)\\
\as{೪ (ಪಂಚಾಕ್ಷರೀ)} । ರುದ್ರ ದೇವತಾಧಿಷ್ಠಿತ ಶೈವ ದರ್ಶನ ಶ್ರೀಪಾ।ಪೂ।ತ।ನಮಃ॥(ಬಿಂದೌ)\\
\as{೪ (ಆದಿತ್ಯಮಂತ್ರಃ)}। ಸೂರ್ಯ ದೇವತಾಧಿಷ್ಠಿತ ಸೌರ ದರ್ಶನ ಶ್ರೀಪಾ।ಪೂ।ತ।ನಮಃ॥(ಸೃಷ್ಟಿಚಕ್ರೇ)\\
\as{೪ (ಅಷ್ಟಾಕ್ಷರೀ)}। ವಿಷ್ಣು ದೇವತಾಧಿಷ್ಠಿತ ವೈಷ್ಣವ ದರ್ಶನ ಶ್ರೀಪಾ।ಪೂ।ತ।ನಮಃ॥(ಶಿವವಾಮಭಾಗೇ)\\
\as{೪ ಶ್ರೀಂಹ್ರೀಂಶ್ರೀಂ }। ಭುವನೇಶ್ವರೀ ದೇವತಾಧಿಷ್ಠಿತ ಶಾಕ್ತ ದರ್ಶನ ಶ್ರೀಪಾ।ಪೂ।ತ।ನಮಃ॥(ಬಿಂದುಂ ಪರಿತಃ)
\newpage
\section{ಷಡಾಧಾರಪೂಜಾ}
\as{೪ ಸಾಂ ಹಂಸಃ } ಸ್ವಚ್ಛಾನಂದವಿಭೂತ್ಯೈ ಸ್ವಾಹಾ ।\\ ಮೂಲಾಧಾರಾಧಿಷ್ಠಾನದೇವತಾಯೈ ಸಾಕಿನೀಸಹಿತ ಗಣನಾಥಸ್ವರೂಪಿಣ್ಯೈ ನಮಃ ।  ಮೂಲಾಧಾರದೇವೀ ಶ್ರೀಪಾದುಕಾಂ ಪೂ।ತ।ನಮಃ॥\\
\as{೪ ಕಾಂ ಸೋಹಂ } ಪರಮಹಂಸವಿಭೂತ್ಯೈ ಸ್ವಾಹಾ ।\\ ಸ್ವಾಧಿಷ್ಠಾನಾಧಿಷ್ಠಾನದೇವತಾಯೈ  ಕಾಕಿನೀಸಹಿತ  ಬ್ರಹ್ಮಸ್ವರೂಪಿಣ್ಯೈ ನಮಃ । ಸ್ವಾಧಿಷ್ಠಾನದೇವೀ ಶ್ರೀಪಾದುಕಾಂ ಪೂ।ತ।ನಮಃ॥\\
\as{೪ ಲಾಂ ಹಂಸಸ್ಸೋಹಂ }ಸ್ವಚ್ಛಾನಂದ ಪರಮಹಂಸ ಪರಮಾತ್ಮನೇ ಸ್ವಾಹಾ ।\\ ಮಣಿಪೂರಾಧಿಷ್ಠಾನದೇವತಾಯೈ ಲಾಕಿನೀಸಹಿತ ವಿಷ್ಣುಸ್ವರೂಪಿಣ್ಯೈ ನಮಃ । ಮಣಿಪೂರದೇವೀ ಶ್ರೀಪಾದುಕಾಂ ಪೂ।ತ।ನಮಃ॥\\
\as{೪ ರಾಂ ಹಂಸಶ್ಶಿವಸ್ಸೋಹಂ } ಸ್ವಾತ್ಮಾನಂ ಬೋಧಯ ಬೋಧಯ ಸ್ವಾಹಾ ।\\ ಅನಾಹತಾಧಿಷ್ಠಾನದೇವತಾಯೈ ರಾಕಿಣೀಸಹಿತ ಸದಾಶಿವಸ್ವರೂಪಿಣ್ಯೈ\\ ನಮಃ । ಅನಾಹತದೇವೀ ಶ್ರೀಪಾದುಕಾಂ ಪೂ।ತ।ನಮಃ॥\\
\as{೪ ಡಾಂ ಸೋಹಂಹಂಸಶ್ಶಿವಃ }ಪರಮಾತ್ಮಾನಂ ಬೋಧಯಬೋಧಯ ಸ್ವಾಹಾ ।\\ ವಿಶುಶುದ್ಧ್ಯಧಿಷ್ಠಾನದೇವತಾಯೈ ಡಾಕಿನೀಸಹಿತ ಜೀವೇಶ್ವರಸ್ವರೂಪಿಣ್ಯೈ ನಮಃ । ವಿಶುದ್ಧಿದೇವೀ ಶ್ರೀಪಾದುಕಾಂ ಪೂ।ತ।ನಮಃ॥\\
\as{೪ ಹಾಂ ಹಂಸಶ್ಶಿವಸ್ಸೋಹಂ ಸೋಹಂಹಂಸಶ್ಶಿವಃ }\\ ಸ್ವಚ್ಛಾನಂದ ಚಿತ್ಪ್ರಕಾಶಾಮೃತಹೇತವೇ ಸ್ವಾಹಾ ।\\ ಆಜ್ಞಾಧಿಷ್ಠಾನದೇವತಾಯೈ ಹಾಕಿನೀಸಹಿತ ಪರಮಾತ್ಮಸ್ವರೂಪಿಣ್ಯೈ ನಮಃ । ಆಜ್ಞಾದೇವೀ ಶ್ರೀಪಾದುಕಾಂ ಪೂ।ತ।ನಮಃ॥
\newpage
\section{ಲಲಿತಾ ದ್ವಾದಶ ನಾಮ ಸ್ತೋತ್ರಮ್~॥}
ಶೃಣು ದ್ವಾದಶ ನಾಮಾನಿ ತಸ್ಯಾ ದೇವ್ಯಾ ಘಟೋದ್ಭವ ~।\\
ಯೇಷಾಮಾಕರ್ಣನಾಮಾತ್ರಾತ್ ಪ್ರಸನ್ನಾ ಸಾ ಭವಿಷ್ಯತಿ~॥

ಪಂಚಮೀ ದಂಡನಾಥಾ ಚ ಸಂಕೇತಾ ಸಮಯೇಶ್ವರೀ~।\\
ತಥಾ ಸಮಯಸಂಕೇತಾ ವಾರಾಹೀ ಪೋತ್ರಿಣೀ ಶಿವಾ~॥

ವಾರ್ತಾಲೀ ಚ ಮಹಾಸೇನಾ ಆಜ್ಞಾಚಕ್ರೇಶ್ವರೀ ತಥಾ~।\\
ಅರಿಘ್ನೀ ಚೇತಿ ಸಂಪ್ರೋಕ್ತಮ್ ನಾಮ ದ್ವಾದಶಕಂ ಮುನೇ~॥

ನಾಮದ್ವಾದಶಕಾಭಿಖ್ಯ ವಜ್ರಪಂಜರ ಮಧ್ಯಗಃ~।\\
ಸಂಕಟೇ ದುಃಖಮಾಪ್ನೋತಿ ನ ಕದಾಚನ ಮಾನವಃ~॥
\authorline{ಇತಿ ಶ್ರೀ ಲಲಿತಾ ದ್ವಾದಶ ನಾಮ ಸ್ತೋತ್ರಂ}
\section{ಲಲಿತಾ ಷೋಡಶನಾಮ ಸ್ತೋತ್ರಮ್~॥}
ಸಂಗೀತಯೋಗಿನೀ ಶ್ಯಾಮಾ ಶ್ಯಾಮಲಾ ಮಂತ್ರನಾಯಿಕಾ ।\\
ಮಂತ್ರಿಣೀ ಸಚಿವೇಶಾನೀ ಪ್ರಧಾನೇಶೀ ಶುಕಪ್ರಿಯಾ~॥

ವೀಣಾವತೀ ವೈಣಿಕೀ ಚ ಮುದ್ರಿಣೀ ಪ್ರಿಯಕಪ್ರಿಯಾ~।\\
ನೀಪಪ್ರಿಯಾ ಕದಂಬೇಶೀ ಕದಂಬವನವಾಸಿನೀ~॥

ಸದಾಮದಾ ಚ ನಾಮಾನಿ ಷೋಡಶೈತಾನಿ ಕುಂಭಜ~।\\
ಏತೈರ್ಯಃ ಸಚಿವೇಶಾನೀಂ ಸಕೃತ್ ಸ್ತೌತಿ ಶರೀರವಾನ್ ।\\
ತಸ್ಯ ತ್ರೈಲೋಕ್ಯಮಖಿಲಂ ವಶೇ ತಿಷ್ಠತ್ಯಸಂಶಯಃ~॥
\authorline{ಇತಿ ಶ್ರೀ ಲಲಿತಾ ಷೋಡಶನಾಮ ಸ್ತೋತ್ರಮ್}
\newpage
\section{ಲಲಿತಾ ಪಂಚವಿಂಶತಿನಾಮ ಸ್ತೋತ್ರಮ್~॥}
ಸಿಂಹಾಸನೇಶೀ ಲಲಿತಾ ಮಹಾರಾಜ್ಞೀ ವರಾಂಕುಶಾ~।\\
ಚಾಪಿನೀ ತ್ರಿಪುರಾ ಚೈವಮಹಾತ್ರಿಪುರಸುಂದರೀ~॥

ಸುಂದರೀ ಚಕ್ರನಾಥಾ ಚ ಸಮ್ರಾಜ್ಞೀ ಚಕ್ರಿಣೀ ತಥಾ~।\\
ಚಕ್ರೇಶ್ವರೀ ಮಹಾದೇವೀ ಕಾಮೇಶೀ ಪರಮೇಶ್ವರೀ ~॥

ಕಾಮರಾಜಪ್ರಿಯಾ ಕಾಮಕೋಟಿಕಾ ಚಕ್ರವರ್ತಿನೀ~।\\
ಮಹಾವಿದ್ಯಾ ಶಿವಾನಂಗವಲ್ಲಭಾ ಸರ್ವಪಾಟಲಾ~॥

ಕುಲನಾಥಾಮ್ನಾಯನಾಥಾ ಸರ್ವಾಮ್ನಾಯ ನಿವಾಸಿನೀ~।\\
ಶೃಂಗಾರ ನಾಯಿಕಾ ಚೇತಿ ಪಂಚವಿಂಶತಿ ನಾಮಭಿಃ~॥

ಸ್ತುವಂತಿ ಯೇ ಮಹಾಭಾಗಾಂ ಲಲಿತಾಂ ಪರಮೇಶ್ವರೀಮ್~।\\
ತೇ ಪ್ರಾಪ್ನುವಂತಿ ಸೌಭಾಗ್ಯಮಷ್ಟೌ ಸಿದ್ಧೀರ್ಮಹದ್ಯಶಃ~॥
\authorline{ಇತಿ ಶ್ರೀ ಲಲಿತಾ ಪಂಚವಿಂಶತಿನಾಮ ಸ್ತೋತ್ರಮ್~॥}

\section{ಶ್ರೀಲಲಿತಾತ್ರಿಶತೀ}
{\bfseries ಅತಿಮಧುರಚಾಪಹಸ್ತಾಮಪರಿಮಿತಾಮೋದಬಾಣಸೌಭಾಗ್ಯಾಂ~।\\
ಅರುಣಾಮತಿಶಯಕರುಣಾಮಭಿನವಕುಲಸುಂದರೀಂ ವಂದೇ ॥\\
ಹಯಗ್ರೀವ ಉವಾಚ\\
ಕಕಾರರೂಪಾ ಕಲ್ಯಾಣೀ ಕಲ್ಯಾಣಗುಣಶಾಲಿನೀ~।\\
ಕಲ್ಯಾಣಶೈಲನಿಲಯಾ ಕಮನೀಯಾ ಕಲಾವತೀ ॥೧॥

ಕಮಲಾಕ್ಷೀ ಕಲ್ಮಷಘ್ನೀ ಕರುಣಾಮೃತ ಸಾಗರಾ~।\\
ಕದಂಬಕಾನನಾವಾಸಾ ಕದಂಬ ಕುಸುಮಪ್ರಿಯಾ ॥೨॥

ಕಂದರ್ಪವಿದ್ಯಾ ಕಂದರ್ಪ ಜನಕಾಪಾಂಗ ವೀಕ್ಷಣಾ~।\\
ಕರ್ಪೂರವೀಟೀ ಸೌರಭ್ಯ ಕಲ್ಲೋಲಿತ ಕಕುಪ್ತಟಾ ॥೩॥

ಕಲಿದೋಷಹರಾ ಕಂಜಲೋಚನಾ ಕಮ್ರವಿಗ್ರಹಾ~।\\
ಕರ್ಮಾದಿ ಸಾಕ್ಷಿಣೀ ಕಾರಯಿತ್ರೀ ಕರ್ಮಫಲಪ್ರದಾ ॥೪॥

ಏಕಾರರೂಪಾ ಚೈಕಾಕ್ಷರ್ಯೇಕಾನೇಕಾಕ್ಷರಾಕೃತಿಃ~।\\
ಏತತ್ತದಿತ್ಯನಿರ್ದೇಶ್ಯಾ ಚೈಕಾನಂದ ಚಿದಾಕೃತಿಃ ॥೫॥

ಏವಮಿತ್ಯಾಗಮಾಬೋಧ್ಯಾ ಚೈಕಭಕ್ತಿ ಮದರ್ಚಿತಾ~।\\
ಏಕಾಗ್ರಚಿತ್ತ ನಿರ್ಧ್ಯಾತಾ ಚೈಷಣಾ ರಹಿತಾದೃತಾ ॥೬॥

ಏಲಾಸುಗಂಧಿಚಿಕುರಾ ಚೈನಃ ಕೂಟ ವಿನಾಶಿನೀ~।\\
ಏಕಭೋಗಾ ಚೈಕರಸಾ ಚೈಕೈಶ್ವರ್ಯ ಪ್ರದಾಯಿನೀ ॥೭॥

ಏಕಾತಪತ್ರ ಸಾಮ್ರಾಜ್ಯ ಪ್ರದಾ ಚೈಕಾಂತಪೂಜಿತಾ~।\\
ಏಧಮಾನಪ್ರಭಾ ಚೈಜದನೇಕಜಗದೀಶ್ವರೀ ॥೮॥

ಏಕವೀರಾದಿ ಸಂಸೇವ್ಯಾ ಚೈಕಪ್ರಾಭವ ಶಾಲಿನೀ~।\\
ಈಕಾರರೂಪಾ ಚೇಶಿತ್ರೀ ಚೇಪ್ಸಿತಾರ್ಥ ಪ್ರದಾಯಿನೀ ॥೯॥

ಈದೃಗಿತ್ಯ ವಿನಿರ್ದೇಶ್ಯಾ ಚೇಶ್ವರತ್ವ ವಿಧಾಯಿನೀ~।\\
ಈಶಾನಾದಿ ಬ್ರಹ್ಮಮಯೀ ಚೇಶಿತ್ವಾದ್ಯಷ್ಟ ಸಿದ್ಧಿದಾ ॥೧೦॥

ಈಕ್ಷಿತ್ರೀಕ್ಷಣ ಸೃಷ್ಟಾಂಡ ಕೋಟಿರೀಶ್ವರ ವಲ್ಲಭಾ~।\\
ಈಡಿತಾ ಚೇಶ್ವರಾರ್ಧಾಂಗ ಶರೀರೇಶಾಧಿ ದೇವತಾ ॥೧೧॥

ಈಶ್ವರ ಪ್ರೇರಣಕರೀ ಚೇಶತಾಂಡವ ಸಾಕ್ಷಿಣೀ~।\\
ಈಶ್ವರೋತ್ಸಂಗ ನಿಲಯಾ ಚೇತಿಬಾಧಾ ವಿನಾಶಿನೀ ॥೧೨॥

ಈಹಾವಿರಹಿತಾ ಚೇಶ ಶಕ್ತಿ ರೀಷತ್ ಸ್ಮಿತಾನನಾ~।\\
ಲಕಾರರೂಪಾ ಲಲಿತಾ ಲಕ್ಷ್ಮೀ ವಾಣೀ ನಿಷೇವಿತಾ ॥೧೩॥

ಲಾಕಿನೀ ಲಲನಾರೂಪಾ ಲಸದ್ದಾಡಿಮ ಪಾಟಲಾ~।\\
ಲಲಂತಿಕಾಲಸತ್ಫಾಲಾ ಲಲಾಟ ನಯನಾರ್ಚಿತಾ ॥೧೪॥

ಲಕ್ಷಣೋಜ್ಜ್ವಲ ದಿವ್ಯಾಂಗೀ ಲಕ್ಷಕೋಟ್ಯಂಡ ನಾಯಿಕಾ~।\\
ಲಕ್ಷ್ಯಾರ್ಥಾ ಲಕ್ಷಣಾಗಮ್ಯಾ ಲಬ್ಧಕಾಮಾ ಲತಾತನುಃ ॥೧೫॥

ಲಲಾಮರಾಜದಲಿಕಾ ಲಂಬಿಮುಕ್ತಾಲತಾಂಚಿತಾ~।\\
ಲಂಬೋದರ ಪ್ರಸೂರ್ಲಭ್ಯಾ ಲಜ್ಜಾಢ್ಯಾ ಲಯವರ್ಜಿತಾ ॥೧೬॥

ಹ್ರೀಂಕಾರ ರೂಪಾ ಹ್ರೀಂಕಾರ ನಿಲಯಾ ಹ್ರೀಂಪದಪ್ರಿಯಾ~।\\
ಹ್ರೀಂಕಾರ ಬೀಜಾ ಹ್ರೀಂಕಾರಮಂತ್ರಾ ಹ್ರೀಂಕಾರಲಕ್ಷಣಾ ॥೧೭॥

ಹ್ರೀಂಕಾರಜಪ ಸುಪ್ರೀತಾ ಹ್ರೀಂಮತೀ ಹ್ರೀಂವಿಭೂಷಣಾ~।\\
ಹ್ರೀಂಶೀಲಾಹ್ರೀಂಪದಾರಾಧ್ಯಾ ಹ್ರೀಂಗರ್ಭಾಹ್ರೀಂಪದಾಭಿಧಾ॥೧೮॥

ಹ್ರೀಂಕಾರವಾಚ್ಯಾ ಹ್ರೀಂಕಾರ ಪೂಜ್ಯಾ ಹ್ರೀಂಕಾರ ಪೀಠಿಕಾ~।\\
ಹ್ರೀಂಕಾರವೇದ್ಯಾ ಹ್ರೀಂಕಾರಚಿಂತ್ಯಾ ಹ್ರೀಂ ಹ್ರೀಂಶರೀರಿಣೀ ॥೧೯॥

ಹಕಾರರೂಪಾ ಹಲಧೃಕ್ಪೂಜಿತಾ ಹರಿಣೇಕ್ಷಣಾ~।\\
ಹರಪ್ರಿಯಾ ಹರಾರಾಧ್ಯಾ ಹರಿಬ್ರಹ್ಮೇಂದ್ರ ವಂದಿತಾ ॥೨೦॥

ಹಯಾರೂಢಾ ಸೇವಿತಾಂಘ್ರಿರ್ಹಯಮೇಧ ಸಮರ್ಚಿತಾ।\\
ಹರ್ಯಕ್ಷವಾಹನಾ ಹಂಸವಾಹನಾ ಹತದಾನವಾ ॥೨೧॥

ಹತ್ಯಾದಿಪಾಪಶಮನೀ ಹರಿದಶ್ವಾದಿ ಸೇವಿತಾ~।\\
ಹಸ್ತಿಕುಂಭೋತ್ತುಂಗ ಕುಚಾ ಹಸ್ತಿಕೃತ್ತಿ ಪ್ರಿಯಾಂಗನಾ ॥೨೨॥

ಹರಿದ್ರಾಕುಂಕುಮಾ ದಿಗ್ಧಾ ಹರ್ಯಶ್ವಾದ್ಯಮರಾರ್ಚಿತಾ~।\\
ಹರಿಕೇಶಸಖೀ ಹಾದಿವಿದ್ಯಾ ಹಾಲಾಮದಾಲಸಾ ॥೨೩॥

ಸಕಾರರೂಪಾ ಸರ್ವಜ್ಞಾ ಸರ್ವೇಶೀ ಸರ್ವಮಂಗಲಾ~।\\
ಸರ್ವಕರ್ತ್ರೀ ಸರ್ವಭರ್ತ್ರೀ ಸರ್ವಹಂತ್ರೀ ಸನಾತನಾ ॥೨೪॥

ಸರ್ವಾನವದ್ಯಾ ಸರ್ವಾಂಗ ಸುಂದರೀ ಸರ್ವಸಾಕ್ಷಿಣೀ~।\\
ಸರ್ವಾತ್ಮಿಕಾ ಸರ್ವಸೌಖ್ಯ ದಾತ್ರೀ ಸರ್ವವಿಮೋಹಿನೀ ॥೨೫॥

ಸರ್ವಾಧಾರಾ ಸರ್ವಗತಾ ಸರ್ವಾವಗುಣವರ್ಜಿತಾ~।\\
ಸರ್ವಾರುಣಾ ಸರ್ವಮಾತಾ ಸರ್ವಭೂಷಣ ಭೂಷಿತಾ ॥೨೬॥

ಕಕಾರಾರ್ಥಾ ಕಾಲಹಂತ್ರೀ ಕಾಮೇಶೀ ಕಾಮಿತಾರ್ಥದಾ~।\\
ಕಾಮಸಂಜೀವನೀ ಕಲ್ಯಾ ಕಠಿನಸ್ತನ ಮಂಡಲಾ ॥೨೭॥

ಕರಭೋರುಃ ಕಲಾನಾಥ ಮುಖೀ ಕಚಜಿತಾಂಬುದಾ~।\\
ಕಟಾಕ್ಷಸ್ಯಂದಿ ಕರುಣಾ ಕಪಾಲಿ ಪ್ರಾಣ ನಾಯಿಕಾ ॥೨೮॥

ಕಾರುಣ್ಯ ವಿಗ್ರಹಾ ಕಾಂತಾ ಕಾಂತಿಧೂತ ಜಪಾವಲಿಃ~।\\
ಕಲಾಲಾಪಾ ಕಂಬುಕಂಠೀ ಕರನಿರ್ಜಿತ ಪಲ್ಲವಾ ॥೨೯॥

ಕಲ್ಪವಲ್ಲೀ ಸಮಭುಜಾ ಕಸ್ತೂರೀ ತಿಲಕಾಂಚಿತಾ~।\\
ಹಕಾರಾರ್ಥಾ ಹಂಸಗತಿರ್ಹಾಟಕಾಭರಣೋಜ್ಜ್ವಲಾ ॥೩೦॥

ಹಾರಹಾರಿ ಕುಚಾಭೋಗಾ ಹಾಕಿನೀ ಹಲ್ಯವರ್ಜಿತಾ~।\\
ಹರಿತ್ಪತಿ ಸಮಾರಾಧ್ಯಾ ಹಠಾತ್ಕಾರ ಹತಾಸುರಾ ॥೩೧॥

ಹರ್ಷಪ್ರದಾ ಹವಿರ್ಭೋಕ್ತ್ರೀ ಹಾರ್ದ ಸಂತಮಸಾಪಹಾ~।\\
ಹಲ್ಲೀಸಲಾಸ್ಯ ಸಂತುಷ್ಟಾ ಹಂಸಮಂತ್ರಾರ್ಥ ರೂಪಿಣೀ ॥೩೨॥

ಹಾನೋಪಾದಾನ ನಿರ್ಮುಕ್ತಾ ಹರ್ಷಿಣೀ ಹರಿಸೋದರೀ~।\\
ಹಾಹಾಹೂಹೂ ಮುಖ ಸ್ತುತ್ಯಾ ಹಾನಿ ವೃದ್ಧಿ ವಿವರ್ಜಿತಾ ॥೩೩॥

ಹಯ್ಯಂಗವೀನ ಹೃದಯಾ ಹರಿಗೋಪಾರುಣಾಂಶುಕಾ~।\\
ಲಕಾರಾಖ್ಯಾ ಲತಾಪೂಜ್ಯಾ ಲಯಸ್ಥಿತ್ಯುದ್ಭವೇಶ್ವರೀ ॥೩೪॥

ಲಾಸ್ಯ ದರ್ಶನ ಸಂತುಷ್ಟಾ ಲಾಭಾಲಾಭ ವಿವರ್ಜಿತಾ~।\\
ಲಂಘ್ಯೇತರಾಜ್ಞಾ ಲಾವಣ್ಯ ಶಾಲಿನೀ ಲಘು ಸಿದ್ಧಿದಾ ॥೩೫॥

ಲಾಕ್ಷಾರಸ ಸವರ್ಣಾಭಾ ಲಕ್ಷ್ಮಣಾಗ್ರಜ ಪೂಜಿತಾ~।\\
ಲಭ್ಯೇತರಾ ಲಬ್ಧ ಭಕ್ತಿ ಸುಲಭಾ ಲಾಂಗಲಾಯುಧಾ ॥೩೬॥

ಲಗ್ನಚಾಮರ ಹಸ್ತ ಶ್ರೀಶಾರದಾ ಪರಿವೀಜಿತಾ~।\\
ಲಜ್ಜಾಪದ ಸಮಾರಾಧ್ಯಾ ಲಂಪಟಾ ಲಕುಲೇಶ್ವರೀ ॥೩೭॥

ಲಬ್ಧಮಾನಾ ಲಬ್ಧರಸಾ ಲಬ್ಧ ಸಂಪತ್ಸಮುನ್ನತಿಃ~।\\
ಹ್ರೀಂಕಾರಿಣೀ ಹ್ರೀಂಕಾರಾದ್ಯಾ ಹ್ರೀಂಮಧ್ಯಾ ಹ್ರೀಂಶಿಖಾಮಣಿಃ ॥೩೮॥

ಹ್ರೀಂಕಾರಕುಂಡಾಗ್ನಿ ಶಿಖಾ ಹ್ರೀಂಕಾರ ಶಶಿಚಂದ್ರಿಕಾ~।\\
ಹ್ರೀಂಕಾರ ಭಾಸ್ಕರರುಚಿರ್ಹ್ರೀಂಕಾರಾಂಭೋದ ಚಂಚಲಾ ॥೩೯॥

ಹ್ರೀಂಕಾರ ಕಂದಾಂಕುರಿಕಾ ಹ್ರೀಂಕಾರೈಕ ಪರಾಯಣಾ~।\\
ಹ್ರೀಂಕಾರ ದೀರ್ಘಿಕಾಹಂಸೀ ಹ್ರೀಂಕಾರೋದ್ಯಾನ ಕೇಕಿನೀ ॥೪೦॥

ಹ್ರೀಂಕಾರಾರಣ್ಯ ಹರಿಣೀ ಹ್ರೀಂಕಾರಾವಾಲ ವಲ್ಲರೀ~।\\
ಹ್ರೀಂಕಾರ ಪಂಜರಶುಕೀ ಹ್ರೀಂಕಾರಾಂಗಣ ದೀಪಿಕಾ ॥೪೧॥

ಹ್ರೀಂಕಾರ ಕಂದರಾ ಸಿಂಹೀ ಹ್ರೀಂಕಾರಾಂಭೋಜ ಭೃಂಗಿಕಾ~।\\
ಹ್ರೀಂಕಾರ ಸುಮನೋ ಮಾಧ್ವೀ ಹ್ರೀಂಕಾರ ತರುಮಂಜರೀ ॥೪೨॥

ಸಕಾರಾಖ್ಯಾ ಸಮರಸಾ ಸಕಲಾಗಮ ಸಂಸ್ತುತಾ~।\\
ಸರ್ವವೇದಾಂತ ತಾತ್ಪರ್ಯಭೂಮಿಃ ಸದಸದಾಶ್ರಯಾ ॥೪೩॥

ಸಕಲಾ ಸಚ್ಚಿದಾನಂದಾ ಸಾಧ್ಯಾ ಸದ್ಗತಿದಾಯಿನೀ~।\\
ಸನಕಾದಿಮುನಿಧ್ಯೇಯಾ ಸದಾಶಿವ ಕುಟುಂಬಿನೀ ॥೪೪॥

ಸಕಾಲಾಧಿಷ್ಠಾನ ರೂಪಾ ಸತ್ಯರೂಪಾ ಸಮಾಕೃತಿಃ~।\\
ಸರ್ವಪ್ರಪಂಚ ನಿರ್ಮಾತ್ರೀ ಸಮಾನಾಧಿಕ ವರ್ಜಿತಾ ॥೪೫॥

ಸರ್ವೋತ್ತುಂಗಾ ಸಂಗಹೀನಾ ಸಗುಣಾ ಸಕಲೇಶ್ವರೀ~।\\
ಕಕಾರಿಣೀ ಕಾವ್ಯಲೋಲಾ ಕಾಮೇಶ್ವರ ಮನೋಹರಾ ॥೪೬॥

ಕಾಮೇಶ್ವರಪ್ರಾಣನಾಡೀ ಕಾಮೇಶೋತ್ಸಂಗ ವಾಸಿನೀ~।\\
ಕಾಮೇಶ್ವರಾಲಿಂಗಿತಾಂಗೀ ಕಾಮೇಶ್ವರ ಸುಖಪ್ರದಾ ॥೪೭॥

ಕಾಮೇಶ್ವರ ಪ್ರಣಯಿನೀ ಕಾಮೇಶ್ವರ ವಿಲಾಸಿನೀ~।\\
ಕಾಮೇಶ್ವರ ತಪಃ ಸಿದ್ಧಿಃ ಕಾಮೇಶ್ವರ ಮನಃ ಪ್ರಿಯಾ ॥೪೮॥

ಕಾಮೇಶ್ವರ ಪ್ರಾಣನಾಥಾ ಕಾಮೇಶ್ವರ ವಿಮೋಹಿನೀ~।\\
ಕಾಮೇಶ್ವರ ಬ್ರಹ್ಮವಿದ್ಯಾ ಕಾಮೇಶ್ವರ ಗೃಹೇಶ್ವರೀ ॥೪೯॥

ಕಾಮೇಶ್ವರಾಹ್ಲಾದಕರೀ ಕಾಮೇಶ್ವರ ಮಹೇಶ್ವರೀ~।\\
ಕಾಮೇಶ್ವರೀ ಕಾಮಕೋಟಿ ನಿಲಯಾ ಕಾಂಕ್ಷಿತಾರ್ಥದಾ ॥೫೦॥

ಲಕಾರಿಣೀ ಲಬ್ಧರೂಪಾ ಲಬ್ಧಧೀರ್ಲಬ್ಧ ವಾಂಛಿತಾ~।\\
ಲಬ್ಧಪಾಪ ಮನೋದೂರಾ ಲಬ್ಧಾಹಂಕಾರ ದುರ್ಗಮಾ ॥೫೧॥

ಲಬ್ಧಶಕ್ತಿರ್ಲಬ್ಧ ದೇಹಾ ಲಬ್ಧೈಶ್ವರ್ಯ ಸಮುನ್ನತಿಃ~।\\
ಲಬ್ಧವೃದ್ಧಿರ್ಲಬ್ಧಲೀಲಾ ಲಬ್ಧಯೌವನ ಶಾಲಿನೀ ॥೫೨॥

ಲಬ್ಧಾತಿಶಯ ಸರ್ವಾಂಗ ಸೌಂದರ್ಯಾ ಲಬ್ಧ ವಿಭ್ರಮಾ~।\\
ಲಬ್ಧರಾಗಾ ಲಬ್ಧಪತಿರ್ಲಬ್ಧ ನಾನಾಗಮಸ್ಥಿತಿಃ ॥೫೩॥

ಲಬ್ಧ ಭೋಗಾ ಲಬ್ಧ ಸುಖಾ ಲಬ್ಧ ಹರ್ಷಾಭಿಪೂರಿತಾ~।\\
ಹ್ರೀಂಕಾರ ಮೂರ್ತಿರ್ಹ್ರೀಂಕಾರ ಸೌಧಶೃಂಗ ಕಪೋತಿಕಾ ॥೫೪॥

ಹ್ರೀಂಕಾರ ದುಗ್ಧಾಬ್ಧಿ ಸುಧಾ ಹ್ರೀಂಕಾರ ಕಮಲೇಂದಿರಾ~।\\
ಹ್ರೀಂಕಾರಮಣಿ ದೀಪಾರ್ಚಿರ್ಹ್ರೀಂಕಾರ ತರುಶಾರಿಕಾ ॥೫೫॥

ಹ್ರೀಂಕಾರ ಪೇಟಕ ಮಣಿರ್ಹ್ರೀಂಕಾರಾದರ್ಶ ಬಿಂಬಿತಾ~।\\
ಹ್ರೀಂಕಾರ ಕೋಶಾಸಿಲತಾ ಹ್ರೀಂಕಾರಾಸ್ಥಾನ ನರ್ತಕೀ ॥೫೬॥

ಹ್ರೀಂಕಾರ ಶುಕ್ತಿಕಾ ಮುಕ್ತಾಮಣಿರ್ಹ್ರೀಂಕಾರ ಬೋಧಿತಾ~।\\
ಹ್ರೀಂಕಾರಮಯ ಸೌವರ್ಣಸ್ತಂಭ ವಿದ್ರುಮ ಪುತ್ರಿಕಾ ॥೫೭॥

ಹ್ರೀಂಕಾರ ವೇದೋಪನಿಷದ್ ಹ್ರೀಂಕಾರಾಧ್ವರ ದಕ್ಷಿಣಾ~।\\
ಹ್ರೀಂಕಾರ ನಂದನಾರಾಮ ನವಕಲ್ಪಕ ವಲ್ಲರೀ ॥೫೮॥

ಹ್ರೀಂಕಾರ ಹಿಮವದ್ಗಂಗಾ ಹ್ರೀಂಕಾರಾರ್ಣವ ಕೌಸ್ತುಭಾ~।\\
ಹ್ರೀಂಕಾರ ಮಂತ್ರ ಸರ್ವಸ್ವಾ ಹ್ರೀಂಕಾರಪರ ಸೌಖ್ಯದಾ ॥೫೯॥}
\authorline {॥ಇತಿ ಶ್ರೀಲಲಿತಾತ್ರಿಶತೀಸ್ತೋತ್ರಂ ಸಂಪೂರ್ಣಂ ॥}
\section{ಸೌಭಾಗ್ಯವಿದ್ಯಾ ಕವಚಂ}
ಅಸ್ಯ ಶ್ರೀ ಮಹಾತ್ರಿಪುರಸುಂದರೀ ಮಂತ್ರವರ್ಣಾತ್ಮಕ ಕವಚ ಮಹಾಮಂತ್ರಸ್ಯ ದಕ್ಷಿಣಾಮೂರ್ತಿರ್ಋಷಿಃ। ಅನುಷ್ಟುಪ್ ಛಂದಃ। ಶ್ರೀಮಹಾತ್ರಿಪುರಸುಂದರೀ ದೇವತಾ~। ಐಂ ಬೀಜಂ~। ಸೌಃ ಶಕ್ತಿಃ~। ಕ್ಲೀಂ ಕೀಲಕಂ~। ಮಮ ಶರೀರರಕ್ಷಣಾರ್ಥೇ ಜಪೇ ವಿನಿಯೋಗಃ~॥\\
\dhyana{ಬಾಲಾರ್ಕಮಂಡಲಾಭಾಸಾಂ ಚತುರ್ಬಾಹುಂ ತ್ರಿಲೋಚನಾಂ~।\\
ಪಾಶಾಂಕುಶ ಧನುರ್ಬಾಣಾನ್ ಧಾರಯಂತೀಂ ಶಿವಾಂ ಭಜೇ ॥}\\
\as{(ಓಂಐಂಹ್ರೀಂಶ್ರೀಂ)}\\
ಕಕಾರಃ ಪಾತು ಮೇ ಶೀರ್ಷಂ ಏಕಾರಃ ಪಾತು ಫಾಲಕಂ।\\
ಈಕಾರಃ ಪಾತು ಮೇ ವಕ್ತ್ರಂ ಲಕಾರಃ ಪಾತು ಕರ್ಣಕಂ॥೧॥

ಹ್ರೀಂಕಾರಃ ಪಾತು ಹೃದಯಂ ವಾಗ್ಭವಶ್ಚ ಸದಾವತು।\\
ಹಕಾರಃ ಪಾತು ಜಠರಂ ಸಕಾರೋ ನಾಭಿದೇಶಕಂ॥೨॥

ಕಕಾರೋವ್ಯಾದ್ವಸ್ತಿಭಾಗಂ ಹಕಾರಃ ಪಾತು ಲಿಂಗಕಂ।\\
ಲಕಾರೋ ಜಾನುನೀ ಪಾತು ಹ್ರೀಂಕಾರೋ ಜಂಘಯುಗ್ಮಕಂ॥೩॥

ಕಾಮರಾಜಃ ಸದಾ ಪಾತು ಜಠರಾದಿ ಪ್ರದೇಶಕಂ।\\
ಸಕಾರಃ ಪಾತು ಮೇ ಜಂಘೇ ಕಕಾರಃ ಪಾತು ಪೃಷ್ಠಕಂ॥೪॥

ಲಕಾರೋವ್ಯಾನ್ನಿತಂಬಂ ಮೇ ಹ್ರೀಂಕಾರಃ ಪಾತು ಮೂಲಕಂ~।\\
ಶಕ್ತಿಬೀಜಃ ಸದಾ ಪಾತು ಮೂಲಾಧಾರಾದಿ ದೇಶಕಂ॥೫॥

ತ್ರಿಪುರಾ ದೇವತಾ ಪಾತು ತ್ರಿಪುರೇಶೀ ಚ ಸರ್ವದಾ।\\
ತ್ರಿಪುರಾ ಸುಂದರೀ ಪಾತು ತ್ರಿಪುರಾಶ್ರೀ ಸ್ತಥಾವತು॥೬॥

ತ್ರಿಪುರಾ ಮಾಲಿನೀ ಪಾತು ತ್ರಿಪುರಾ ಸಿದ್ಧಿದಾ ವತು।\\
ತ್ರಿಪುರಾಂಬಾ ತಥಾ ಪಾತು ಪಾತು ತ್ರಿಪುರಭೈರವೀ॥೭॥

ಅಣಿಮಾದ್ಯಾ ಸ್ತಥಾ ಪಾಂತು ಬ್ರಾಹ್ಮ್ಯಾದ್ಯಾಃ ಪಾಂತು ಮಾಂ ಸದಾ।\\
ದಶಮುದ್ರಾಸ್ತಥಾ ಪಾಂತು ಕಾಮಾಕರ್ಷಣ ಪೂರ್ವಕಾಃ॥೮॥

ಪಾಂತು ಮಾಂ ಷೋಡಶದಲೇ ಯಂತ್ರೇನಂಗ ಕುಮಾರಿಕಾಃ।\\
ಪಾಂತು ಮಾಂ ಪೃಷ್ಠಪತ್ರೇ ತು ಸರ್ವಸಂಕ್ಷೋಭಣಾದಿಕಾಃ॥೯॥

ಪಾಂತು ಮಾಂ ದಶಕೋಣೇ ತು ಸರ್ವಸಿದ್ಧಿ ಪ್ರದಾಯಿಕಾಃ।\\
ಪಾಂತು ಮಾಂ ಬಾಹ್ಯ ದಿಕ್ಕೋಣೇ ಮಧ್ಯ ದಿಕ್ಕೋಣಕೇ ತಥಾ॥೧೦॥

ಸರ್ವಜ್ಞಾ ದ್ಯಾಸ್ತಥಾ ಪಾಂತು ಸರ್ವಾಭೀಷ್ಟ ಪ್ರದಾಯಿಕಾಃ।\\
ವಶಿನ್ಯಾದ್ಯಾಸ್ತಥಾ ಪಾಂತು ವಸು ಪತ್ರಸ್ಯ ದೇವತಾಃ॥೧೧॥

ತ್ರಿಕೋಣ ಸ್ಯಾಂತ ರಾಲೇ ತು ಪಾಂತು ಮಾಮಾಯುಧಾನಿ ಚ।\\
ಕಾಮೇಶ್ವರ್ಯಾದಿಕಾಃ ಪಾಂತು ತ್ರಿಕೋಣೇ ಕೋಣಸಂಸ್ಥಿತಾಃ॥೧೨॥

ಬಿಂದುಚಕ್ರೇ ತಥಾ ಪಾತು ಮಹಾತ್ರಿಪುರಸುಂದರೀ।\as{(ಶ್ರೀಂಹ್ರೀಂಐಂ)}\\
ಇತೀದಂ ಕವಚಂ ದೇವಿ ಕವಚಂ ಮಂತ್ರಸೂಚಕಂ॥೧೩॥

ಯಸ್ಮೈ ಕಸ್ಮೈ ನ ದಾತವ್ಯಂ ನ ಪ್ರಕಾಶ್ಯಂ ಕಥಂಚನ।\\
ಯಸ್ತ್ರಿಸಂಧ್ಯಂ ಪಠೇದ್ದೇವಿ ಲಕ್ಷ್ಮೀಸ್ತಸ್ಯ ಪ್ರಜಾಯತೇ॥೧೪॥

ಅಷ್ಟಮ್ಯಾಂ ಚ ಚತುರ್ದಶ್ಯಾಂ ಯಃ ಪಠೇತ್ ಪ್ರಯತಃ ಸದಾ।\\
ಪ್ರಸನ್ನಾ ಸುಂದರೀ ತಸ್ಯ ಸರ್ವಸಿದ್ಧಿಪ್ರದಾಯಿನೀ॥೧೫॥
\authorline{॥ಇತಿ ಶ್ರೀ ರುದ್ರಯಾಮಲೇ ತಂತ್ರೇ ತ್ರಿಪುರಾ ಹೃದಯೇ ಕವಚರಹಸ್ಯಂ ಸಂಪೂರ್ಣಂ ॥}

ವನಸ್ಪತಿರಸೋತ್ಪನ್ನೋ ಗಂಧಾಢ್ಯೋ ಧೂಪ ಉತ್ತಮಃ ।\\
ಆಘ್ರೇಯಃ ಸರ್ವದೇವಾನಾಂ ಧೂಪೋಽಯಂ ಪ್ರತಿಗೃಹ್ಯತಾಮ್ ॥ಧೂಪಃ॥

ಸುಪ್ರಕಾಶೋ ಮಹಾದೀಪಃ ಸರ್ವತಃ ತಿಮಿರಾಪಹಃ ।\\
ಸ ಬಾಹ್ಯಾಭ್ಯಂತರಜ್ಯೋತಿರ್ದೀಪೋಽಯಂ ಪ್ರತಿಗೃಹ್ಯತಾಂ ॥ದೀಪಃ॥

ಚಿತ್ಪಾತ್ರೇ ಸದ್ಧವಿಃ ಸೌಖ್ಯಂ ವಿವಿಧಾನೇಕ ಭಕ್ಷಣಂ ।\\
ನಿವೇದಯಾಮಿ ದೇವೇಶ ಸಾನುಗಾಯ ಗೃಹಾಣ ತತ್ ॥ನೈವೇದ್ಯಂ॥
\newpage
%(ನಿವೇದಯಾಮಿ ಭಗವನ್ ಜುಷಾಣೇದಂ ಹವಿರ್ಹರ ॥)
%ಶಾಲೀಭಕ್ತಂ ಸುಪಕ್ವಂ ಶಿಶಿರಕರಸಿತಂ ಪಾಯಸಾಪೂಪಸೂಪಂ\\ ಲೇಹ್ಯಂ ಪೇಯಂ ಚ ಚೋಷ್ಯಂ ಸಿತಮಮೃತಫಲಂ ಪಾರಿಕಾದ್ಯಂ ಸುಖಾದ್ಯಂ ।\\ ಜ್ಯಂ ಪ್ರಾಜ್ಯಂ ಸಭೋಜ್ಯಂ ನಯನರುಚಿಕರಂ ರಾಜಿಕೈಲಾಮರೀಚ\\- ಸ್ವಾದೀಯಃ ಶಾಕರಾಜೀಪರಿಕರಮಮೃತಾಹಾರಜೋಷಂ ಜುಷಸ್ವ ॥
ನಮಸ್ತೇ ದೇವ ದೇವೇಶ ಸರ್ವತೃಪ್ತಿಕರಂ ಪರಂ ।\\
ಅಖಂಡಾನಂದ ಸಂಪೂರ್ಣ ಗೃಹಾಣ ಜಲಮುತ್ತಮಮ್ ॥ಸ್ವಾದೂದಕಂ॥

ವಿಚಿತ್ರವಸ್ತ್ರ ಮಧ್ವಾಢ್ಯಂ ಸುಪಾತ್ರೇ ಲಘುದೀಪಿತಂ ।\\
ನೀರಾಜನಂ ಭಜಸ್ವ ತ್ವಂ ಅರ್ಪಯಾಮೀಶ ತೇ ನಮಃ ॥ನೀರಾಜನಂ॥

ಸರ್ವಾಮ್ನಾಯಾಂತ ವೇದ್ಯ ತ್ವಂ ಅಖಂಡಾನಂದ ದಾಯಕ ।\\
ಅರ್ಪಯಾಮಿ ತವಪ್ರೀತ್ಯೈ ಗೃಹಾಣ ಕುಸುಮಾಂಜಲಿಂ ॥ಮಂತ್ರಪುಷ್ಪಂ॥

ಸರ್ವಜ್ಞಂ ಸರ್ವಸಂಸ್ಥಂ ಸಕಲ(ವಿಲಲಿತಂ) ನಿಷ್ಕಲಾಕಾರ ಶಾಂತಂ \\
ಪಂಚಾಶದ್ವೃತ್ತಿಭಿನ್ನಂ ಪ್ರಕೃತಿಪರ ಮಹಾ ಲೋಕಯಾತ್ರಾ ಸನಾಥಂ ॥\\
ಷಟ್‌ತ್ರಿಂಶತ್ತ್ವಗರ್ಭಂ ತ್ರಿಗುಣಪರಿಣತಂ ವಿಶ್ವರೂಪಾಷ್ಟಮೂರ್ತಿಂ\\
ತಂ ವಂದೇ ಯೋಗಿಸಿದ್ಧಂ ಹರಮಭಯಪದಂ ಭಾವನಾಭಾಸ್ಯ ಮೂರ್ತಿಂ ॥ಪ್ರದಕ್ಷಿಣಂ॥

ಗಂಗಾಧರಂ ಚೈವ ಮಹೀಧರಂ ಚ ತ್ರಿಶೂಲಪಾಣಿಂ ಚ ಚತುರ್ಭುಜಂ ಚ ।\\
ತ್ರೈಲೋಕ್ಯನಾಥಂ ಚ ಹರಿಂ ಹರಂ ಚ ರುದ್ರಂ ಚ ವಿಷ್ಣುಂ ಚ ಸದಾ ನಮಸ್ಯೇ ॥ನಮಸ್ಕಾರಾಃ॥
\newpage
\section{ಮಾತೃಕಾಸ್ತೋತ್ರಂ}
ಗಣೇಶ ಗ್ರಹ ನಕ್ಷತ್ರಯೋಗಿನೀ ರಾಶಿ ರೂಪಿಣೀಂ~।\\
ದೇವೀಂ ಮಂತ್ರಮಯೀಂ ನೌಮಿ ಮಾತೃಕಾಂ ಪೀಠ ರೂಪಿಣೀಂ॥೧॥

ಪ್ರಣಮಾಮಿ ಮಹಾದೇವೀಂ ಮಾತೃಕಾಂ ಪರಮೇಶ್ವರೀಂ।\\
ಕಾಲಹಲ್ಲೋಹಲೋಲ್ಲೋಲ ಕಲನಾಶಮಕಾರಿಣೀಂ॥೨॥

ಯದಕ್ಷರೈಕಮಾತ್ರೇಪಿ ಸಂಸಿದ್ಧೇ ಸ್ಪರ್ಧತೇ ನರಃ।\\
ರವಿತಾರ್ಕ್ಷ್ಯೇಂದುಕಂದರ್ಪಶಂಕರಾನಲವಿಷ್ಣುಭಿಃ॥೩॥

ಯದಕ್ಷರ ಶಶಿ ಜ್ಯೋತ್ಸ್ನಾಮಂಡಿತಂ ಭುವನತ್ರಯಂ।\\
ವಂದೇ ಸರ್ವೇಶ್ವರೀಂ ದೇವೀಂ ಮಹಾ ಶ್ರೀ ಸಿದ್ಧಮಾತೃಕಾಂ॥೪॥

ಯದಕ್ಷರಮಹಾಸೂತ್ರಪ್ರೋತಮೇತಜ್ಜಗತ್ತ್ರಯಂ।\\
ಬ್ರಹ್ಮಾಂಡಾದಿಕಟಾಹಾಂತಂ ತಾಂ ವಂದೇ ಸಿದ್ಧಮಾತೃಕಾಂ॥೫॥

ಯದೇಕಾದಶಮಾಧಾರಂ ಬೀಜಂ ಕೋಣತ್ರಯೋದ್ಭವಂ।\\
ಬ್ರಹ್ಮಾಂಡಾದಿ ಕಟಾಹಾಂತಂ ಜಗದದ್ಯಾಪಿ ದೃಶ್ಯತೇ॥೬॥

ಅಕಚಾದಿಟತೋನ್ನದ್ಧಪಯಶಾಕ್ಷರವರ್ಗಿಣೀಂ।\\
ಜ್ಯೇಷ್ಠಾಂಗಬಾಹುಹೃತ್ಪೃಷ್ಠಕಟಿಪಾದನಿವಾಸಿನೀಂ॥೭॥

ತಾಮೀಕರಾಕ್ಷರೋದ್ಧಾರಾಂ ಸಾರಾತ್ಸಾರಾಂ ಪರಾತ್ಪರಾಂ।\\
ಪ್ರಣಮಾಮಿ ಮಹಾದೇವೀಂ ಪರಮಾನಂದರೂಪಿಣೀಂ॥೮॥

ಅದ್ಯಾಪಿ ಯಸ್ಯಾ ಜಾನಂತಿ ನ ಮನಾಗಪಿ ದೇವತಾಃ।\\
ಕೇಯಂ ಕಸ್ಮಾತ್ ಕ್ವ ಕೇನೇತಿ ಸರೂಪಾರೂಪಭಾವನಾಂ॥೯॥
\eject
ವಂದೇ ತಾಮಹ ಮಕ್ಷಯ್ಯಾಂ ಕ್ಷಕಾರಾಕ್ಷರರೂಪಿಣೀಂ~।\\
ದೇವೀಂಕುಲಕಲೋಲ್ಲಾಸ ಪ್ರೋಲ್ಲಸಂತೀಂ ಪರಾಂಶಿವಾಂ॥೧೦॥

ವರ್ಗಾನು ಕ್ರಮ ಯೋಗೇನ ಯಸ್ಯಾಂ ಮಾತ್ರಷ್ಟಕಂ ಸ್ಥಿತಂ।\\
ವಂದೇ ತಾಮಷ್ಟವರ್ಗೋತ್ಥಮಹಾಸಿದ್ಧ್ಯಷ್ಟಕೇಶ್ವರೀಂ ॥೧೧॥

ಕಾಮಪೂರ್ಣಜಕಾರಾಖ್ಯ ಶ್ರೀಪೀಠಾಂತರ್ನಿವಾಸಿನೀಂ।\\
ಚತುರಾಜ್ಞಾಕೋಶಮೂಲಾಂ ನೌಮಿ ಶ್ರೀ ತ್ರಿಪುರಾಮಹಂ॥೧೨॥

ಇತಿ ದ್ವಾದಶಭಿಃ ಶ್ಲೋಕೈಃ ಸ್ತವನಂ ಸರ್ವ ಸಿದ್ಧಿ ಕೃತ್।\\
ದೇವ್ಯಾ ಸ್ತ್ವಖಂಡ ರೂಪಾಯಾಃ ಸ್ತವನಂ ತವ ತಥ್ಯತಃ॥೧೩॥
\begin{center}{\Large ಇತಿ ಸರ್ವಸಿದ್ಧಿಕೃತ್ಸ್ತೋತ್ರಂ ಸಂಪೂರ್ಣಂ॥}\end{center}

\section{ಸೌಭಾಗ್ಯನವರತ್ನಮಾಲಿಕಾ}
ಕಲಯೇ ಕರುಣಾಪಾಂಗೀಂ ಕಾರಣಮೂರ್ತ್ಯೈಕಕಾರಣೀಭೂತಾಮ್ ।\\
ಕಾಮೇಶ್ವರೀಂ ನ ಹೀತರದೇವಂ ಕಂಚಿತ್ಕದಾಚಿದಭಿಯಾಚೇ ॥೧॥

ಅಭಿಯಾಚೇ ಪರಮಾಂ ತಾಮಂಕುಶಪಾಶಪ್ರಸೂನಚಾಪಕರಾಮ್ ।\\
ಅಲಮಲಮನುತ್ತರಾಂಬಾಚರಣಾದನ್ಯೇನ ದೇವತಾಖ್ಯೇನ ॥೨॥

ಗೃಹ್ಯಂತೇ ತ್ರಿಪುರಾಯಾಃ ಕರುಣಕಲ್ಲೋಲವಾಸಿತಕಟಾಕ್ಷಾಃ ।\\
ಸರ್ವೋಪರಿ ಲೋಕೇಽಸ್ಮಿನ್ ಪಶ್ಯಾಮ್ಯಂಬಾಮಿಹೈಕರೂಪಾಂ ತಾಮ್ ॥೩॥

ಈಪ್ಸಿತಮನೀಪ್ಸಿತಂ ವಾ ಮಮಾಸ್ತು ಸತತಂ ತ್ರಿಮೂರ್ತಿಜನಯಿತ್ರ್ಯಾಃ ।\\
ಈಹಾಶೂನ್ಯಾಯಾ ನೋ ಕರೋಮಿ ಶರಣಂ ತತೋಽನ್ಯದೇವಗಣಮ್ ॥೪॥

ಲಲತು ಹೃದಿ ಸೈವ ದೇವೀ ಮಮ ನಿತ್ಯಂ ಯಾ ಮಹೇಶಸಂಸೇವ್ಯಾ ।\\
ಲಲಿತಾ ರಾಜ್ಞೀ ಮಾನ್ಯಾ ಕದಾಪಿ ಸೇವ್ಯಾಸ್ತು ದೇವತಾ ಮಾಽನ್ಯಾ ॥೫॥ 

ಹರ್ಷಯತು ಮಾಮಜಸ್ರಂ ಹರಿಹರಮುಖ್ಯೈಃ ಸಮಾಶ್ರಿತಾಂಘ್ರಿಯುಗಾ ।\\
ಯಾ ಚ ಹಯಾರೂಢಾಖ್ಯಾ ಸೇನಾನೇತ್ರೀ ನಮಾಮಿ ತಾಂ ಬ್ರೂಯಾಮ್ ॥೬॥

ರಕ್ಷತು ಮಾಮಿಕ್ಷುಧನುಃಪುಷ್ಪಶರಾಢ್ಯಕೃಪೇಕ್ಷಯಾ ಸತತಮ್ ।\\
ಕಾಮೇಶ್ವರಾಽಭಿರಾಮಾ ನಾಽನ್ಯಾ ಮಮೇಶ್ವರೀ ಭವತಿ ॥೭॥

ಅಂಕುರಯತು ಹೃದಿ ಭಕ್ತಿಂ ನಿಜಪದಯುಗಲೇ ಸದಾ ಮಹೇಶಾನೀ ।\\
ಅಂಬಾ ಕಾಮಾಕ್ಷೀ ನೈವಾಽನ್ಯಸ್ಯಾಂ ಮಮಾಽಸ್ತು ತಲ್ಲೇಶಃ ॥೮॥

ಸಮ್ಮನುತೇ ಮಮ ಹೃದಯಂ ವಸ್ತುಂ ತಸ್ಯಾಃ ಪದಾಬ್ಜಯೋಃ ಸತತಮ್ ।\\
ಸರ್ವಜ್ಞಾಯಾ ದೇವ್ಯಾ ಘಟಯತು ತನ್ಮೇ ಮಹೇಶ್ವರೀ ಮನಸಃ ॥೯॥



\section{ಜ್ಞಾನ-ಕಲಿಕಾ-ಸ್ತೋತ್ರಂ}
ಶಿವೇ ದೇವಿ ಸಂವಿತ್ಸುಧಾಸಾಗರಾಽಽತ್ಮ-\\
ಸ್ವರೂಪಾಽಸಿ ಸರ್ವಾಂತರಾಽಽತ್ಮೈಕರೂಪಾ ।\\
ನ ಕಿಂಚಿದ್ ವಿನಾ ತ್ವತ್ಕಲಾಮಸ್ತಿ ಲೋಕೇ\\
ತತಃ ಸತ್ಸ್ವರೂಪಾಽಸಿ ಸತ್ಯೇಽಪ್ಯಸತ್ಯೇ ॥೧॥

ಅಸತ್ಯಂ ಪುನಃ ಸತ್ಯಮನ್ಯೇ ದ್ವಿರೂಪಂ\\
ದ್ವಯಾತೀತಮೇಕೇ ಜಗುಃ ಸರ್ವಮೇತತ್ ।\\
ನ ತೇ ತಾಂ ವಿದುರ್ಮಾಯಯಾ ಮೋಹಿತಾಸ್ತೇ\\
ಚಿದಾನಂದರೂಪಾ ತ್ವಮೇವಾಽಸಿ ಸರ್ವಂ ॥೨॥

ಕ್ಷಣಾನಾಂ ಕ್ರಮೈರ್ಭಿನ್ನರೂಪಾಂ ಧರಾದ್ಯೈ-\\
ರ್ಮಿತಾಮಾಹುರೇಕೇ ತಮೋಮಾತ್ರರೂಪಾಂ ।\\
ತಮೋದೀಪ್ತಿಸಂಭಿನ್ನರೂಪಾಂಚ ಶಾಂತ-\\
ಸ್ವರೂಪಾಂ ಮಹೇಶೀಂ ವಿದುಸ್ತ್ವಾಂ ನ ತೇಽಜ್ಞಾಃ ॥೩॥

ಶಿವಾದಿಕ್ಷಿತಿಪ್ರಾಂತತತ್ತ್ವಾವಲಿರ್ಯಾ\\
ವಿಚಿತ್ರಾ ಯದೀಯೇ ಶರೀರೇ ವಿಭಾತಿ ।\\
ಪಟೇ ಚಿತ್ರಕಲ್ಪಾ ಜಲೇ ಸೇಂದುತಾರಾ\\
ನಭೋವತ್ಪರಾ ಸಾ ತ್ವಮೇವಾಽಸಿ ಸರ್ವಾ ॥೪॥

ಅಭಿನ್ನಂ ವಿಭಿನ್ನಂ ಬಹಿರ್ವಾಽನ್ತರೇ ವಾ\\
ವಿಭಾತಿ ಪ್ರಕಾಶಸ್ತಮೋ ವಾಽಪಿ ಸರ್ವಂ ।\\
ಋತೇ ತ್ವಾಂ ಚಿತಿಂ ಯೇನ ನೋ ಭಾತಿ ಕಿಂಚಿತ್\\
ತತಸ್ತ್ವಂ ಸಮಸ್ತಂ ನ ಕಿಂಚಿತ್ವದನ್ಯತ್ ॥೫॥

ನಿರುಧ್ಯಾಽನ್ತರಂಗಂ ವಿಲಾಪ್ಯಾಽಕ್ಷಸಂಘಂ\\
ಪರಿತ್ಯಜ್ಯ ಸರ್ವತ್ರ ಕಾಮಾದಿಭಾವಂ ।\\
ಸ್ಥಿತಾನಾಂ ಮಹಾಯೋಗಿನಾಂ ಚಿತ್ತಭೂಮೌ\\
ಚಿದಾನಂದರೂಪಾ ತ್ವಮೇಕಾ ವಿಭಾಸಿ ॥೬॥

ತಥಾಽನ್ಯೇ ಮನಃ ಸೇಂದ್ರಿಯಂ ಸಂಚರಚ್ಚಾ-\\
ಽಪ್ಯಸಂಯಮ್ಯತನ್ಮಾರ್ಗಕೇಜಾಗರೂಕಾಃ ।\\
ಸ್ವಸಮ್ವಿತ್ಸುಧಾಽಽದರ್ಶದೇಹೇ ಸ್ಫುರಂತಂ\\
ಮಹಾಯೋಗಿನಾಥಾಃ ಪ್ರಪಶ್ಯಂತಿ ಸರ್ವಂ ॥೭॥

ನಿರುಕ್ತೇ ಮಹಾಸಾರಮಾರ್ಗೇಽತಿಸೂಕ್ಷ್ಮೇ\\
ಗತಿಂ ಯೇ ನ ವಿಂದಂತಿ ಮೂಢಸ್ವಭಾವಾಃ ।\\
ಜನಾನ್ ತಾನ್ ಸಮುದ್ಧರ್ತುಮಕ್ಷಾವಗಮ್ಯಂ\\
ಬಹಿಃ ಸ್ಥೂಲರೂಪಂ ವಿಭಿನ್ನಂ ವಿಭರ್ಷಿ ॥೮॥
\eject
ತದಾರಾಧನೇಽನೇಕಮಾರ್ಗಾನ್ ವಿಚಿತ್ರಾನ್\\
ವಿಧಾಯಾಽಥ ಮಾರ್ಗೇಣ ಕೇನಾಪಿ ಯಾಂತಂ ।\\
ನದೀವಾರಿ ಸಿಂಧುರ್ಯಥಾ ಸ್ವೀಕರೋತಿ\\
ಪ್ರದಾಯ ಸ್ವಭಾವಂ ನು ಸ್ವಾತ್ಮೀಕರೋಷಿ ॥೯॥

ತಥಾ ತಾಸುಮೂರ್ತಿಷ್ವನೇಕಾಸು ಮುಖ್ಯಾ\\
ಧನುರ್ಬಾಣಪಾಶಾಂಕುಶಾಢ್ಯೈವ ಮೂರ್ತಿಃ ।\\
ಶರೀರೇಷು ಮೂರ್ಧೇವ ಯೇ ತಾಂ ಭಜೇಯು-\\
ರ್ಜನಾಸ್ತ್ರೈಪುರೀಂ ಮೂರ್ತಿಮತ್ಯುತ್ತಮಾಸ್ತೇ ॥೧೦॥

ಜನಾನ್ ದುಃಖಸಿಂಧೋಃ ಸಮುದ್ಧರ್ತುಕಾಮಾ\\
ಪಥಸ್ತಾನನೇಕಾನ್ ಪ್ರದಿಶ್ಯ ಪ್ರಕೃಷ್ಟಾನ್ ।\\
ದಯಾರ್ದ್ರಸ್ವಭಾವೇತಿ ವಿಖ್ಯಾತಕೀರ್ತಿ\\
ಸ್ತ್ವಮೇಕೈವ ಪೂಜ್ಯಾ ಪರಾಶಕ್ತಿರೂಪಾ ॥೧೧॥

ಸದಾ ತೇ ಪದಾಬ್ಜೇ ಮನಃಷಟ್ಪದೋ ಮೇ\\
ಪಿಬಂಸ್ತದ್ರಸಂ ನಿರ್ವೃತಃ ಸಂಸ್ಥಿತೋಽಸ್ತು ।\\
ಇತಿ ಪ್ರಾರ್ಥನಾಂ ಮೇ ನಿಶಮ್ಯಾಽಽಶು ಮಾತ-\\
ರ್ವಿಧೇಹಿ ಸ್ವದೃಷ್ಟಿಂ ದಯಾರ್ದ್ರಾಮಪೀಷತ್ ॥೧೨॥

\begin{center}{\Large ಇತಿ ಜ್ಞಾನಕಲಿಕಾಸ್ತೋತ್ರಂ ಸಂಪೂರ್ಣಂ}\end{center}

\authorline{{\LARGE ಸದ್ಗುರುಚರಣಾರವಿಂದಾರ್ಪಣಮಸ್ತು\\ಓಂ ತತ್ಸತ್\\********}}



