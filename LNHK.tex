\thispagestyle{empty}
\chapter*{\center ॥ಲಕ್ಷ್ಮೀಹೃದಯನ್ಯಾಸಾಃ॥}
\section{ದೇವೀಮಾತೃಕಾನ್ಯಾಸಃ ।}
ಅಸ್ಯ ಶ್ರೀಕಲಾಮಾತೃಕಾನ್ಯಾಸಸ್ಯ ಪ್ರಜಾಪತಿರ್ಋಷಿಃ । ಗಾಯತ್ರೀ ಛಂದಃ । ಶಾರದಾ ದೇವತಾ । ಹಲೋ ಬೀಜಾನಿ । ಸ್ವರಾಃ ಶಕ್ತಯಃ । ಬಿಂದವಃ ಕೀಲಕಾನಿ । ಲಕ್ಷ್ಮೀಹೃದಯಪಾರಾಯಣಾಂಗತಯಾ ನ್ಯಾಸೇ ವಿನಿಯೋಗಃ ।
\section{ಋಷ್ಯಾದಿನ್ಯಾಸಃ ।}
ಓಂ ಪ್ರಜಾಪತಿಋಷಯೇ ನಮಃ (ಶಿರಸಿ) । ಓಂ ಗಾಯತ್ರೀಚ್ಛಂದಸೇ ನಮಃ (ಮುಖೇ) । ಓಂ ಶಾರದಾದೇವತಾಯೈ ನಮಃ (ಹೃದಿ) । ಓಂ ಹಲ್ಬೀಜೇಭ್ಯೋ ನಮಃ (ಗುಹ್ಯೇ) । ಓಂ ಸ್ವರಶಕ್ತಿಭ್ಯೋ ನಮಃ (ಪಾದಯೋಃ) ।ಬಿಂದುಕೀಲಕೇಭ್ಯೋ ನಮಃ (ನಾಭೌ)। ಓಂ ವಿನಿಯೋಗಾಯ ನಮಃ (ಸರ್ವಾಂಗೇ) ।\\
ಅಂ ಆಂ ಇಂ ಈಂ ++++++ಳಂ ಕ್ಷಂ ।(ಅಂಜಲಿನಾ ಸರ್ವಾಂಗೇ ವಿನ್ಯಸೇತ್ ।)
\section{ಕರನ್ಯಾಸಃ ಷಡಂಗನ್ಯಾಸಶ್ಚ ।}
ಓಂಅಂಕಂಖಂಗಂಘಂಙಂಆಂ ಅಂಗುಷ್ಠಾಭ್ಯಾಂ ನಮಃ।ಹೃದಯಾಯ ನಮಃ।\\
ಓಂಇಂಚಂಛಂಜಂಝಂಞಂಈಂ ತರ್ಜನೀಭ್ಯಾಂ ನಮಃ।ಶಿರಸೇ ಸ್ವಾಹಾ।\\
ಓಂಉಂಟಂಠಂಡಂಢಂಣಂಊಂ ಮಧ್ಯಮಾಭ್ಯಾಂ ನಮಃ।ಶಿಖಾಯೈ ವಷಟ್।\\
ಓಂಏಂತಂಥಂದಂಧಂನಂಐಂ ಅನಾಮಿಕಾಭ್ಯಾಂ ನಮಃ।ಕವಚಾಯ ಹುಂ।\\
ಓಂಓಂಪಂಫಂಬಂಭಂಮಂಔಂ ಕನಿಷ್ಠಿಕಾಭ್ಯಾಂ ನಮಃ।ನೇತ್ರತ್ರಯಾಯ ವೌಷಟ್।\\
ಓಂಅಂಯಂರಂಲಂವಂಶಂಷಂಸಂಹಂಳಂಕ್ಷಂಅಃಕರತಲಕರಪೃಷ್ಠಾಭ್ಯಾಂನಮಃ।ಅಸ್ತ್ರಾಯ ಫಟ್।
\section{ಧ್ಯಾನಂ}
{\bfseries ಶಂಖಚಕ್ರಾಬ್ಜಪರಶುಕಪಾಲಾನ್ಯಕ್ಷಮಾಲಿಕಾಂ ।\\
ಪುಸ್ತಕಾಮೃತಕುಂಭೌ ಚ ತ್ರಿಶೂಲಂ ದಧತೀಂ ಕರೈಃ ॥\\
ಸಿತಪೀತಾಸಿತಶ್ವೇತರಕ್ತವರ್ಣೈಸ್ತ್ರಿಲೋಚನೈಃ ।\\
ಪಂಚಾಸ್ಯೈಃ ಸಂಯುತಾಂ ಚಂದ್ರಸಕಾಂತಿಂ ಶಾರದಾಂ ಭಜೇ ॥}
\section{ಬಹಿರ್ಮಾತೃಕಾನ್ಯಾಸಃ }
ಓಂ ಅಂ ನಿವೃತ್ಯೈ ನಮಃ (ಶಿರಸಿ )।\\
ಓಂ ಆಂ ಪ್ರತಿಷ್ಠಾಯೈ ನಮಃ (ಮುಖವೃತ್ತೇ )।\\
ಓಂ ಇಂ ವಿದ್ಯಾಯೈ ನಮಃ (ದಕ್ಷನೇತ್ರೇ )।\\
ಓಂ ಈಂ ಶಾಂತ್ಯೈ ನಮಃ (ವಾಮನೇತ್ರೇ )।\\
ಓಂ ಉಂ ಇಂಧಿಕಾಯೈ ನಮಃ (ದಕ್ಷಕರ್ಣೇ )।\\
ಓಂ ಊಂ ದೀಪಿಕಾಯೈ ನಮಃ (ವಾಮಕರ್ಣೇ )।\\
ಓಂ ಋಂ ರೇಚಿಕಾಯೈ ನಮಃ (ದಕ್ಷನಾಸಾಪುಟೇ )।\\
ಓಂ ೠಂ ಮೋಚಿಕಾಯೈ ನಮಃ (ವಾಮನಾಸಾಪುಟೇ )।\\
ಓಂ ಲೃಂ ಪರಾಯೈ ನಮಃ (ದಕ್ಷಗಂಡೇ )।\\
ಓಂ ಲೄಂ ಸೂಕ್ಷ್ಮಾಯೈ ನಮಃ (ವಾಮಗಂಡೇ )।\\
ಓಂ ಏಂ ಸೂಕ್ಷ್ಮಾಮೃತಾಯೈ ನಮಃ (ಊರ್ಧ್ವೋಷ್ಠೇ )।\\
ಓಂ ಐಂ ಜ್ಞಾನಾಮೃತಾಯೈ ನಮಃ (ಅಧರೋಷ್ಠೇ )।\\
ಓಂ ಓಂ ಆಪ್ಯಾಯನ್ಯೈ ನಮಃ (ಊರ್ಧ್ವದಂತಪಂಕ್ತೌ )।\\
ಓಂ ಔಂ ವ್ಯಾಪಿನ್ಯೈ ನಮಃ (ಅಧೋದಂತಪಂಕ್ತೌ )।\\
ಓಂ ಅಂ ವ್ಯೋಮರೂಪಾಯೈ ನಮಃ (ಜಿಹ್ವಾಗ್ರೇ )।\\
ಓಂ ಅಃ ಅನಂತಾಯೈ ನಮಃ (ಕಂಠದೇಶೇ )।\\
ಓಂ ಕಂ ಸೃಷ್ಟ್ಯೈ ನಮಃ (ದಕ್ಷಬಾಹುಮೂಲೇ )।\\
ಓಂ ಖಂ ಋದ್ಧ್ಯೈ ನಮಃ (ದಕ್ಷಕೂರ್ಪರೇ )।\\
ಓಂ ಗಂ ಸ್ಮೃತ್ಯೈ ನಮಃ (ದಕ್ಷಮಣಿಬಂಧೇ )।\\
ಓಂ ಘಂ ಮೇಧಾಯೈ ನಮಃ (ದಕ್ಷಹಸ್ತಾಂಗುಲಿಮೂಲೇ )।\\
ಓಂ ಙಂ ಕಾಂತ್ಯೈ ನಮಃ (ದಕ್ಷಹಸ್ತಾಂಗುಲ್ಯಗ್ರೇ )।\\
ಓಂ ಚಂ ಲಕ್ಷ್ಮ್ಯೈ ನಮಃ (ವಾಮಬಾಹುಮೂಲೇ )।\\
ಓಂ ಛಂ ದ್ಯುತ್ಯೈ ನಮಃ (ವಾಮಕೂರ್ಪರೇ )।\\
ಓಂ ಜಂ ಸ್ಥಿರಾಯೈ ನಮಃ (ವಾಮಮಣಿಬಂಧೇ )।\\
ಓಂ ಝಂ ಸ್ಥಿತ್ಯೈ ನಮಃ (ವಾಮಹಸ್ತಾಂಗುಲಿಮೂಲೇ )।\\
ಓಂ ಞಂ ಸಿದ್ಧ್ಯೈ ನಮಃ (ವಾಮಹಸ್ತಾಂಗುಲ್ಯಗ್ರೇ )।\\
ಓಂ ಟಂ ಜರಾಯೈ ನಮಃ (ದಕ್ಷೋರುಮೂಲೇ )।\\
ಓಂ ಠಂ ಪಾಲಿನ್ಯೈ ನಮಃ (ದಕ್ಷಜಾನೌ )।\\
ಓಂ ಡಂ ಕ್ಷಾಂತ್ಯೈ ನಮಃ (ದಕ್ಷಗುಲ್ಫೇ )।\\
ಓಂ ಢಂ ಈಶ್ವರಿಕಾಯೈ ನಮಃ ( ದಕ್ಷಪಾದಾಂಗುಲಿಮೂಲೇ)।\\
ಓಂ ಣಂ ರತ್ಯೈ ನಮಃ (ದಕ್ಷಪಾದಾಂಗುಲ್ಯಗ್ರೇ )।\\
ಓಂ ತಂ ಕಾಮಿಕಾಯೈ ನಮಃ (ವಾಮೋರುಮೂಲೇ )।\\
ಓಂ ಥಂ ವರದಾಯೈ ನಮಃ (ವಾಮಜಾನುನಿ )।\\
ಓಂ ದಂ ಆಹ್ಲಾದಿನ್ಯೈ ನಮಃ (ವಾಮಗುಲ್ಫೇ )।\\
ಓಂ ಧಂ ಪ್ರೀತ್ಯೈ ನಮಃ (ವಾಮಪಾದಾಂಗುಲಿಮೂಲೇ )।\\
ಓಂ ನಂ ದೀರ್ಘಾಯೈ ನಮಃ (ವಾಮಪಾದಾಂಗುಲ್ಯಗ್ರೇ )।\\
ಓಂ ಪಂ ತೀಕ್ಷ್ಣಾಯೈ ನಮಃ (ದಕ್ಷಪಾರ್ಶ್ವೇ )।\\
ಓಂ ಫಂ ರೌದ್ರ್ಯೈ ನಮಃ (ವಾಮಪಾರ್ಶ್ವೇ )।\\
ಓಂ ಬಂ ಭಯಾಯೈ ನಮಃ (ಪೃಷ್ಠೇ )।\\
ಓಂ ಭಂ ನಿದ್ರಾಯೈ ನಮಃ (ನಾಭೌ )।\\
ಓಂ ಮಂ ತಂದ್ರಿಕಾಯೈ ನಮಃ (ಉದರೇ )।\\
ಓಂ ಯಂ ಕ್ಷುಧಾಯೈ ನಮಃ (ಹೃದಿ )।\\
ಓಂ ರಂ ಕ್ರೋಧಿನ್ಯೈ ನಮಃ (ದಕ್ಷಾಂಸೇ )।\\
ಓಂ ಲಂ ಕ್ರಿಯಾಯೈ ನಮಃ (ಕಕುದಿ )।\\
ಓಂ ವಂ ಉತ್ಕಾರ್ಯೈ ನಮಃ (ವಾಮಾಂಸೇ )।\\
ಓಂ ಶಂ ಸಮೃತ್ಯುಕಾಯೈ ನಮಃ (ಹೃದಯಾದಿ ದಕ್ಷಹಸ್ತಾಂತಂ )।\\
ಓಂ ಷಂ ಪೀತಾಯೈ ನಮಃ (ಹೃದಯಾದಿ ವಾಮಹಸ್ತಾಂತಂ )।\\
ಓಂ ಸಂ ಶ್ವೇತಾಯೈ ನಮಃ (ಹೃದಯಾದಿ ದಕ್ಷಪಾದಾಂತಂ )।\\
ಓಂ ಹಂ ಅರುಣಾಯೈ ನಮಃ (ಹೃದಯಾದಿ ವಾಮಪಾದಾಂತಂ)।\\
ಓಂ ಳಂ ಅಸಿತಾಯೈ ನಮಃ (ಹೃದಯಾದಿ ಪಾದಾಂತಂ )।\\
ಓಂ ಕ್ಷಂ ಅನಂತಾಯೈ ನಮಃ (ಹೃದಯಾದಿ ಮಸ್ತಕಾಂತಂ )।
\section{ಅಂತರ್ಮಾತೃಕಾನ್ಯಾಸಃ }
{\bfseries ಆಧಾರೇ ಲಿಂಗನಾಭಿಪ್ರಕಟಿತಹೃದಯೇ ತಾಲುಮೂಲೇ ಲಲಾಟೇ ।\\
ದ್ವೇ ಪತ್ರೇ ಷೋಡಶಾರೇ ದ್ವಿದಶದಶದಲದ್ವಾದಶಾರ್ಧೇ ಚತುಷ್ಕೇ ।\\
ವಾಸಾಂತೇ ಬಾಲಮಧ್ಯೇ ಡಫಕಠಸಹಿತೇ ಕಂಠದೇಶೇ ಸ್ವರಾಣಾಂ\\
ಹಂ ಕ್ಷಂ ತತ್ತ್ವಾರ್ಥಯುಕ್ತಂ ಸಕಲದಲಗತಂ ವರ್ಣರೂಪಂ ನಮಾಮಿ ॥}\\
(ಕಂಠೇ)ಓಂ ಶ್ರೀಂಹ್ರೀಂಐಂ ಐಂಹ್ರೀಂಶ್ರೀಂ ಅಂ ನಮಃಆಂ ++++ಅಂ ನಮಃಅಃ ನಮಃ\\
(ಹೃದಯೇ)ಓಂ ಶ್ರೀಂಹ್ರೀಂಐಂ ಐಂಹ್ರೀಂಶ್ರೀಂ ಕಂ ನಮಃ ಖಂ+++++ಠಂ ನಮಃ\\
(ನಾಭೌ)ಓಂ ಶ್ರೀಂಹ್ರೀಂಐಂ ಐಂಹ್ರೀಂಶ್ರೀಂ ಡಂ ನಮಃ ಢಂ +++++ಫಂ ನಮಃ\\
(ಗುಹ್ಯೇ)ಓಂ ಶ್ರೀಂಹ್ರೀಂಐಂ ಐಂಹ್ರೀಂಶ್ರೀಂ ಬಂ ನಮಃ ಭಂ+++++ಲಂ ನಮಃ\\
(ಆಧಾರೇ)ಓಂ ಶ್ರೀಂಹ್ರೀಂಐಂ ಐಂಹ್ರೀಂಶ್ರೀಂ ವಂ ನಮಃ+++++ಸಂ ನಮಃ\\
(ಭ್ರೂಮಧ್ಯೇ)ಓಂ ಶ್ರೀಂಹ್ರೀಂಐಂ ಐಂಹ್ರೀಂಶ್ರೀಂ ಹಂ ನಮಃ ಕ್ಷಂ ನಮಃ\\
(ಸಹಸ್ರಾರೇ)ಓಂ ಶ್ರೀಂಹ್ರೀಂಐಂ ಐಂಹ್ರೀಂಶ್ರೀಂ ಅಂ ನಮಃ ಆಂ+++++ಕ್ಷಂ ನಮಃ ।\\
(ಪ್ರಾಗ್ವದುತ್ತರನ್ಯಾಸಃ ।)
 \section{ಕರಶುದ್ಧಿನ್ಯಾಸಃ }
ಓಂ ಅಂ ನಮಃ(ದಕ್ಷಕರತಲೇ) ಓಂ ಆಂ ನಮಃ(ದಕ್ಷಕರಪೃಷ್ಠೇ) ಓಂ ಶ್ರೀಂ ನಮಃ (ದಕ್ಷಕರತಲೇ)\\
ಓಂ ಅಂ ನಮಃ(ವಾಮಕರತಲೇ) ಓಂ ಆಂ ನಮಃ(ವಾಮಕರಪೃಷ್ಠೇ) ಓಂ ಶ್ರೀಂ ನಮಃ (ವಾಮಕರತಲೇ)\\
ಓಂ ಅಂ ನಮಃ (ಮಧ್ಯಮಯೋಃ )। ಓಂ ಆಂ ನಮಃ (ಅನಾಮಿಕಯೋಃ) । ಓಂ ಶ್ರೀಂ ನಮಃ (ಕನಿಷ್ಠಿಕಯೋಃ) ।\\
ಓಂ ಅಂ ನಮಃ (ಅಂಗುಷ್ಠಯೋಃ)  । ಓಂ ಆಂ ನಮಃ (ತರ್ಜನ್ಯೋಃ) । ಓಂ ಶ್ರೀಂ ನಮಃ (ಕರತಲಕರಪೃಷ್ಠಯೋಃ) ।
\section{ಪಾದಾದಿಬೀಜನ್ಯಾಸಃ }
ಓಂಶ್ರೀಂಹ್ರೀಂಐಂ ಮಹಾಲಕ್ಷ್ಮ್ಯೈ ನಮಃ ಆಂ ಈಂ ಯಂ ಪಂ ಕಂ ಲಂ ಹಂ (ಪಾದಯೋಃ )।\\
ಓಂ ಶ್ರೀಂ ಹ್ರೀಂ ಐಂ ಮಹಾಲಕ್ಷ್ಮ್ಯೈ ನಮಃ ಹ್ರಾಂ ಹ್ರೀಂ ಆಂ ವ್ಯಾಂ ಯಂ ಭಾಂ ಸಾಂ (ಮುಖೇ) ।\\
ಓಂ ಶ್ರೀಂ ಹ್ರೀಂ ಐಂ ಮಹಾಲಕ್ಷ್ಮ್ಯೈ ನಮಃ ಘ್ರಾಂ ಘ್ರೀಂ ಘ್ರೂಂ ಘ್ರೈಂ ಘ್ರೌಂ ಘ್ರಃ (ನೇತ್ರಯೋಃ )।\\
ಓಂ ಶ್ರೀಂ ಹ್ರೀಂ ಐಂ ಮಹಾಲಕ್ಷ್ಮ್ಯೈ ನಮಃ ಹ್ರಾಂ ಹ್ರೀಂ ಹ್ರೂಂ ಹ್ರೈಂ ಹ್ರೌಂ ಹ್ರಃ (ಜಿಹ್ವಾಗ್ರೇ) ।\\
ಓಂ ಶ್ರೀಂ ಹ್ರೀಂ ಐಂ ಮಹಾಲಕ್ಷ್ಮ್ಯೈ ನಮಃ ಆಂ ಈಂ ಏಂ ಐಂ (ಕುಕ್ಷೌ) ।\\
ಓಂ ಶ್ರೀಂ ಹ್ರೀಂ ಐಂ ಮಹಾಲಕ್ಷ್ಮ್ಯೈ ನಮಃ ಆಂ ಕ್ರೌಂ ಹುಂ ಫಟ್ ಕುರು ಕುರು ಸ್ವಾಹಾ (ಹೃದಯೇ) ।\\
ಓಂ ಶ್ರೀಂ ಹ್ರೀಂ ಐಂ ಮಹಾಲಕ್ಷ್ಮ್ಯೈ ನಮಃ ಶ್ರಾಂ ಶ್ರೀಂ ಶ್ರೂಂ ಶ್ರೈಂ ಶ್ರೌಂ ಶ್ರಃ (ಕಂಠೇ )।\\
ಓಂ ಶ್ರೀಂ ಹ್ರೀಂ ಐಂ ಮಹಾಲಕ್ಷ್ಮ್ಯೈ ನಮಃ ಯಂ ಹಂ ಕಂ ಲಂ ಪಂ ಶ್ರೀಂ (ಶಿರಸಿ) ।
\section{ಮಂತ್ರನ್ಯಾಸಃ }
ಅಸ್ಯ ಶ್ರೀಮಹಾಲಕ್ಷ್ಮೀಮಹಾಮಂತ್ರಸ್ಯ ಭಾರ್ಗವ ಋಷಿಃ । ಅನುಷ್ಟುಬಾದಿನಾನಾಚ್ಛಂದಾಂಸಿ । \\ಆದ್ಯಾದಿ ಶ್ರೀಮಹಾಲಕ್ಷ್ಮೀಃ ದೇವತಾ । ಶ್ರೀಂ ಬೀಜಂ ।  ಹ್ರೀಂ ಶಕ್ತಿಃ । ಐಂ ಕಿಲಕಂ । ಜಪೇ ವಿನಿಯೋಗಃ ।
\begin{multicols}{2}೧.ಓಂ ಶ್ರೀಂ ನಮಃ (ಶಿರಸಿ)।\\೨.ಓಂ ಹ್ರೀಂ ನಮಃ (ಮೂಲಾಧಾರೇ)।\\೩.ಓಂ ಐಂ ನಮಃ (ಹೃದಯೇ)।\\೪.ಓಂ ಶ್ರೀಂ ನಮಃ (ನೇತ್ರಯೋಃ)।\\೫.ಓಂ ಹ್ರೀಂ ನಮಃ (ನಾಸಿಕೇ)।\\೬.ಓಂ ಐಂ ನಮಃ (ಕರ್ಣಯೋಃ)।\\೭.ಓಂ ಶ್ರೀಂ ನಮಃ (ಓಷ್ಠೌ)।\\೮.ಓಂ ಹ್ರೀಂ ನಮಃ (ದಂತಪಂಕ್ತೌ)।\\೯.ಓಂ ಐಂ ನಮಃ (ಜಿಹ್ವಾಗ್ರೇ)।\\೧೦.ಓಂ ಶ್ರೀಂ ನಮಃ (ಕಂಠೇ)।\\೧೧.ಓಂ ಹ್ರೀಂ ನಮಃ (ಭುಜಯೋಃ)।\\೧೨.ಓಂ ಐಂ ನಮಃ (ಸ್ತನಯೋಃ)।\\೧೩.ಓಂ ಶ್ರೀಂ ನಮಃ (ಬಾಹ್ವೋಃ)।\\೧೪.ಓಂ ಹ್ರೀಂ ನಮಃ (ಊರ್ವೋಃ)।\\೧೫.ಓಂ ಐಂ ನಮಃ (ಕಟೌ)।\\೧೬.ಓಂ ಶ್ರೀಂ ನಮಃ (ಜಾನುನೋಃ)।\\೧೭.ಓಂ ಹ್ರೀಂ ನಮಃ (ಗುಲ್ಫಯೋಃ)।\\೧೮.ಓಂ ಐಂ ನಮಃ (ಪಾದಯೋಃ)।\\೧೯.ಓಂ ಶ್ರೀಂ ನಮಃ (ಪೃಷ್ಠೇ)।\\೨೦.ಓಂ ಹ್ರೀಂ ನಮಃ (ದಕ್ಷಿಣಪಾರ್ಶ್ವೇ)।\\೨೧.ಓಂ ಐಂ ನಮಃ (ವಾಮಪಾರ್ಶ್ವೇ)।\\೨೨.ಓಂ ಶ್ರೀಂ ನಮಃ (ಮಣಿಬಂಧಯೋಃ)।\\೨೩.ಓಂ ಹ್ರೀಂ ನಮಃ (ಹಸ್ತಾಂಗುಲೀಷು)।\\೨೪.ಓಂ ಐಂ ನಮಃ (ಪಾದಾಂಗುಲೀಷು)।\\೨೫.ಓಂ ಶ್ರೀಂ ನಮಃ (ನಾಭೌ)।\\೨೬.ಓಂ ಹ್ರೀಂ ನಮಃ (ಹೃದಯೇ)।\\೨೭.ಓಂ ಐಂ ನಮಃ (ಶಿರಸಿ)।\end{multicols}
\section{ಅಥ ಸೂಕ್ತನ್ಯಾಸಃ}
ಹಿರಣ್ಯವರ್ಣಾಂ ********** ಮ ಆವಹ ॥(ಶಿರಸಿ)\\
ತಾಂ ಮ*********** ಪುರುಷಾನಹಮ್॥(ನೇತ್ರಯೋಃ)\\
ಅಶ್ವಪೂರ್ವಾಂ *********ಜುಷತಾಮ್॥(ಕರ್ಣಯೋಃ)\\
ಕಾಂ ಸೋಸ್ಮಿತಾಂ ********* ಶ್ರಿಯಮ್ ॥(ನಾಸಿಕಯೋಃ)\\
ಚನ್ದ್ರಾಂ ಪ್ರಭಾಸಾಂ ********ತ್ವಾಂ ವೃಣೇ॥(ಮುಖೇ)\\
ಆದಿತ್ಯವರ್ಣೇ *********ಬಾಹ್ಯಾ ಅಲಕ್ಷ್ಮೀಃ ॥(ಗಲೇ)\\
ಉಪೈತು ಮಾಂ ******** ದದಾತು ಮೇ ॥(ಬಾಹ್ವೋಃ)\\
ಕ್ಷುತ್ಪಿಪಾಸಾಮ್*********** ಗೃಹಾತ್॥(ಹೃದಿ)\\
ಗಂಧದ್ವಾರಾಂ********* ಶ್ರಿಯಮ್ ॥(ನಾಭೌ)\\
ಮನಸಃ *********** ಯಶಃ ॥(ಗುಹ್ಯೇ)\\
ಕರ್ದಮೇನ********ಪದ್ಮಮಾಲಿನೀಮ್ ॥(ಗುದೇ)\\
ಆಪಃ ಸೃಜನ್ತು ********ಮೇ ಕುಲೇ॥(ಊರ್ವೋಃ)\\
ಆರ್ದ್ರಾಂ ಪುಷ್ಕರಿಣೀಂ **** ಮ ಆವಹ॥(ಜಾನುನೋಃ)\\
ಆರ್ದ್ರಾಂ ಯಃ ಕರಿಣೀಂ*****ಮ ಆವಹ॥(ಜಂಘಯೋಃ)\\
ತಾಂ ಮ ಆವಹ ******ಪುರುಷಾನಹಮ್ ॥(ಪಾದಯೋಃ)\\
ಯಃ ಶುಚಿಃ ********ಸತತಂ ಜಪೇತ್ ॥(ಸರ್ವಾಂಗೇ)
\chapter*{\center ॥ನಾರಾಯಣಹೃದಯನ್ಯಾಸಾಃ॥}
\section{ಕೇಶವಾದಿಮಾತೃಕಾನ್ಯಾಸಃ }
ಕೇಶವಾದಿಮಾತೃಕಾನ್ಯಾಸಸ್ಯ ಸಾಧ್ಯೋ ನಾರಾಯಣ ಋಷಿಃ । ಅಮೃತಗಾಯತ್ರೀಚ್ಛಂದಃ । ಶ್ರೀಲಕ್ಷ್ಮೀನಾರಾಯಣೌ ದೇವತೇ । ಹಲೋ ಬೀಜಾನಿ । ಸ್ವರಾಃ ಶಕ್ತಯಃ ।ಬಿಂದವಃ ಕೀಲಕಾನಿ । ನಾರಾಯಣಹೃದಯಪಾರಾಯಣಾಂಗತಯಾ ನ್ಯಾಸೇ ವಿನಿಯೋಗಃ ।
 \section{ಋಷ್ಯಾದಿನ್ಯಾಸಃ }
ಸಾಧ್ಯನಾರಾಯಣಋಷಯೇ ನಮಃ (ಶಿರಸಿ)। ಗಾಯತ್ರೀಚ್ಛಂದಸೇ ನಮಃ (ಮುಖೇ) । ಲಕ್ಷ್ಮೀನಾರಾಯಣದೇವತಾಭ್ಯಾಂ ನಮಃ (ಹೃದಯೇ)। ಹಲ್ಬೀಜೇಭ್ಯೋ ನಮಃ (ಗುಹ್ಯೇ)। ಸ್ವರಶಕ್ತಿಭ್ಯೋ ನಮಃ (ಪಾದಯೋಃ)।ಬಿಂದುಕೀಲಕೇಭ್ಯೋ ನಮಃ (ನಾಭೌ )।
ಅಂ ಆಂ ಇಂ ಈಂ++++ ಳಂ ಕ್ಷಂ । (ಅಂಜಲಿನಾ ಸರ್ವಾಂಗೇ ವಿನ್ಯಸೇತ್ ।)
\section{ಕರನ್ಯಾಸಃ ಷಡಂಗನ್ಯಾಸಶ್ಚ }
ಓಂ ಹ್ರೀಂ ಅಂಗುಷ್ಠಾಭ್ಯಾಂ ನಮಃ ಹೃದಯಾಯ ನಮಃ ।\\
ಓಂ ಶ್ರೀಂ ತರ್ಜನೀಭ್ಯಾಂ ನಮಃ ಶಿರಸೇ ಸ್ವಾಹಾ ।\\
ಓಂ ಕ್ಲೀಂ ಮಧ್ಯಮಾಭ್ಯಾಂ ನಮಃ ಶಿಖಾಯೈ ವಷಟ್ ।\\
ಓಂ ಹ್ರೀಂ ಅನಾಮಿಕಾಭ್ಯಾಂ ನಮಃ ಕವಚಾಯ ಹುಂ ।\\
ಓಂ ಶ್ರೀಂ ಕನಿಷ್ಠಿಕಾಭ್ಯಾಂ ನಮಃ ನೇತ್ರತ್ರಯಾಯ ವೌಷಟ್ ।\\
ಓಂ ಕ್ಲೀಂ ಕರತಲಕರಪೃಷ್ಠಾಭ್ಯಾಂ ನಮಃ ಅಸ್ತ್ರಾಯ ಫಟ್ ।
\begin{center}
\dhyana{ಶಂಖಚಕ್ರಗದಾಪದ್ಮಕುಂಭಾದರ್ಶಾಬ್ಜಪುಸ್ತಕಂ ।\\
ಬಿಭ್ರಂತಂ ಮೇಘಚಪಲಾವರ್ಣಂ ಲಕ್ಷ್ಮೀಹರಿಂ ಭಜೇ ॥}

(ಪಂಚೋಪಚಾರಪೂಜಾ ।)\end{center}
\newpage
\section{ಬಹಿರ್ಮಾತೃಕಾನ್ಯಾಸಃ}
ಓಂಹ್ರೀಂಶ್ರೀಂಕ್ಲೀಂ ಅಂ ಕ್ಲೀಂಶ್ರೀಂಹ್ರೀಂ ಕೇಶವ ಕೀರ್ತಿಭ್ಯಾಂ ನಮಃ (ಶಿರಸಿ)।\\
ಓಂಹ್ರೀಂಶ್ರೀಂಕ್ಲೀಂ ಆಂ ಕ್ಲೀಂಶ್ರೀಂಹ್ರೀಂ ನಾರಾಯಣ ಕಾಂತಿಭ್ಯಾಂ ನಮಃ (ಮುಖವೃತ್ತೇ)।\\
ಓಂಹ್ರೀಂಶ್ರೀಂಕ್ಲೀಂ ಇಂ ಕ್ಲೀಂಶ್ರೀಂಹ್ರೀಂ ಮಾಧವ ತುಷ್ಟಿಭ್ಯಾಂ ನಮಃ (ದಕ್ಷನೇತ್ರೇ )।\\
ಓಂಹ್ರೀಂಶ್ರೀಂಕ್ಲೀಂ ಈಂ ಕ್ಲೀಂಶ್ರೀಂಹ್ರೀಂ ಗೋವಿಂದ ಪುಷ್ಟಿಭ್ಯಾಂ ನಮಃ (ವಾಮನೇತ್ರೇ )।\\
ಓಂಹ್ರೀಂಶ್ರೀಂಕ್ಲೀಂ ಉಂ ಕ್ಲೀಂಶ್ರೀಂಹ್ರೀಂ ವಿಷ್ಣು ಧೃತಿಭ್ಯಾಂ ನಮಃ (ದಕ್ಷಕರ್ಣೇ)।\\
ಓಂಹ್ರೀಂಶ್ರೀಂಕ್ಲೀಂ ಊಂ ಕ್ಲೀಂಶ್ರೀಂಹ್ರೀಂ ಮಧುಸೂದನ ಶಾಂತಿಭ್ಯಾಂ ನಮಃ (ವಾಮಕರ್ಣೇ)।\\
ಓಂಹ್ರೀಂಶ್ರೀಂಕ್ಲೀಂ ಋ ಕ್ಲೀಂಶ್ರೀಂಹ್ರೀಂ ತ್ರಿವಿಕ್ರಮ ಕ್ರಿಯಾಭ್ಯಾಂ ನಮಃ (ದಕ್ಷನಾಸಾಪುಟೇ)।\\
ಓಂಹ್ರೀಂಶ್ರೀಂಕ್ಲೀಂ ೠಂ ಕ್ಲೀಂಶ್ರೀಂಹ್ರೀಂ ವಾಮನ ದಯಾಭ್ಯಾಂ ನಮಃ (ವಾಮನಾಸಾಪುಟೇ)।\\
ಓಂಹ್ರೀಂಶ್ರೀಂಕ್ಲೀಂ ಲೃಂ ಕ್ಲೀಂಶ್ರೀಂಹ್ರೀಂ ಶ್ರೀಧರ ಮೇಧಾಭ್ಯಾಂ ನಮಃ (ದಕ್ಷಗಂಡೇ)\\
ಓಂಹ್ರೀಂಶ್ರೀಂಕ್ಲೀಂ ಲೄಂ ಕ್ಲೀಂಶ್ರೀಂಹ್ರೀಂ ಹೃಷೀಕೇಶ ಹರ್ಷಾಭ್ಯಾಂ ನಮಃ (ವಾಮಗಂಡೇ)।\\
ಓಂಹ್ರೀಂಶ್ರೀಂಕ್ಲೀಂ ಏಂ ಕ್ಲೀಂಶ್ರೀಂಹ್ರೀಂ ಪದ್ಮನಾಭ ಶ್ರದ್ಧಾಭ್ಯಾಂ ನಮಃ (ಊರ್ಧ್ವೋಷ್ಠೇ)।\\
ಓಂಹ್ರೀಂಶ್ರೀಂಕ್ಲೀಂ ಐಂ ಕ್ಲೀಂಶ್ರೀಂಹ್ರೀಂ ದಾಮೋದರ ಲಜ್ಜಾಭ್ಯಾಂ ನಮಃ (ಅಧರೋಷ್ಠೇ)।\\
ಓಂಹ್ರೀಂಶ್ರೀಂಕ್ಲೀಂ ಓಂ ಕ್ಲೀಂಶ್ರೀಂಹ್ರೀಂ ವಾಸುದೇವ ಲಕ್ಷ್ಮೀಭ್ಯಾಂ ನಮಃ (ಊರ್ಧ್ವದಂತಪಂಕ್ತೌ)।\\
ಓಂಹ್ರೀಂಶ್ರೀಂಕ್ಲೀಂ ಔಂ ಕ್ಲೀಂಶ್ರೀಂಹ್ರೀಂ ಸಂಕರ್ಷಣ ಸರಸ್ವತೀಭ್ಯಾಂ ನಮಃ (ಅಧೋದಂತಪಂಕ್ತೌ)।\\
ಓಂಹ್ರೀಂಶ್ರೀಂಕ್ಲೀಂ ಅಂ ಕ್ಲೀಂಶ್ರೀಂಹ್ರೀಂ ಪ್ರದ್ಯುಮ್ನ ಪ್ರೀತಿಭ್ಯಾಂ ನಮಃ (ಜಿಹ್ವಾಗ್ರೇ)।\\
ಓಂಹ್ರೀಂಶ್ರೀಂಕ್ಲೀಂ ಅಃ ಕ್ಲೀಂಶ್ರೀಂಹ್ರೀಂ ಅನಿರುದ್ಧ ರತಿಭ್ಯಾಂ ನಮಃ (ಕಂಠೇ)।\\
ಓಂಹ್ರೀಂಶ್ರೀಂಕ್ಲೀಂ ಕಂ ಕ್ಲೀಂಶ್ರೀಂಹ್ರೀಂ ಚಕ್ರಿ ಜಯಾಭ್ಯಾಂ ನಮಃ (ದಕ್ಷಬಾಹುಮೂಲೇ)।\\
ಓಂಹ್ರೀಂಶ್ರೀಂಕ್ಲೀಂ ಖಂ ಕ್ಲೀಂಶ್ರೀಂಹ್ರೀಂ ಗದಿ ದುರ್ಗಾಭ್ಯಾಂ ನಮಃ (ದಕ್ಷಕೂರ್ಪರೇ)।\\
ಓಂಹ್ರೀಂಶ್ರೀಂಕ್ಲೀಂ ಗಂ ಕ್ಲೀ ಶ್ರೀಂ ಹ್ರೀಂ ಶಾರ್ಙ್ಗಿ ಪ್ರಭಾಭ್ಯಾಂ ನಮಃ (ದಕ್ಷಮಣಿಬಂಧೇ)।\\
ಓಂಹ್ರೀಂಶ್ರೀಂಕ್ಲೀಂ ಘಂ ಕ್ಲೀಂಶ್ರೀಂಹ್ರೀಂ ಖಡ್ಗಿ ಸತ್ಯಾಭ್ಯಾಂ ನಮಃ (ದಕ್ಷಾಂಗುಲೀಮೂಲೇ\\
ಓಂಹ್ರೀಂಶ್ರೀಂಕ್ಲೀಂ ಙಂ ಕ್ಲೀಂಶ್ರೀಂಹ್ರೀಂ ಶಂಖಿ ಚಂಡಾಭ್ಯಾಂ ನಮಃ (ದಕ್ಷಾಂಗುಲ್ಯಗ್ರೇ)।\\
ಓಂಹ್ರೀಂಶ್ರೀಂಕ್ಲೀಂ ಚಂ ಕ್ಲೀಂಶ್ರೀಂಹ್ರೀಂ ಹಲಿ ವಾಣೀಭ್ಯಾಂ ನಮಃ (ವಾಮಬಾಹುಮೂಲೇ)।\\
ಓಂಹ್ರೀಂಶ್ರೀಂಕ್ಲೀಂ ಛಂ ಕ್ಲೀಂಶ್ರೀಂಹ್ರೀಂ ಮುಸಲಿ ವಿಲಾಸಿನೀಭ್ಯಾಂ ನಮಃ (ವಾಮಕೂರ್ಪರೇ)।\\
ಓಂಹ್ರೀಂಶ್ರೀಂಕ್ಲೀಂ ಜಂ ಕ್ಲೀಂಶ್ರೀಂಹ್ರೀಂ ಶೂಲಿ ವಿಜಯಾಭ್ಯಾಂ ನಮಃ (ವಾಮಮಣಿಬಂಧೇ)।\\
ಓಂಹ್ರೀಂಶ್ರೀಂಕ್ಲೀಂ ಝಂ ಕ್ಲೀಂಶ್ರೀಂಹ್ರೀಂ ಪಾಶಿ ವಿರಜಾಭ್ಯಾಂ ನಮಃ (ವಾಮಾಂಗುಲೀಮೂಲೇ)।\\
ಓಂಹ್ರೀಂಶ್ರೀಂಕ್ಲೀಂ ಞಂ ಕ್ಲೀಂಶ್ರೀಂಹ್ರೀಂ ಅಂಕುಶಿ ವಿಶ್ವಾಭ್ಯಾಂ ನಮಃ (ವಾಮಾಂಗುಲ್ಯಗ್ರೇ)।\\
ಓಂಹ್ರೀಂಶ್ರೀಂಕ್ಲೀಂ ಟಂ ಕ್ಲೀಂಶ್ರೀಂಹ್ರೀಂ ಮುಕುಂದ ವಿನದಾಭ್ಯಾಂ ನಮಃ (ದಕ್ಷೋರುಮೂಲೇ)।\\
ಓಂಹ್ರೀಂಶ್ರೀಂಕ್ಲೀಂ ಠಂ ಕ್ಲೀಂಶ್ರೀಂಹ್ರೀಂ ನಂದಜ ಸುನದಾಭ್ಯಾಂ ನಮಃ (ದಕ್ಷಜಾನುನಿ)।\\
ಓಂಹ್ರೀಂಶ್ರೀಂಕ್ಲೀಂ ಡಂ ಕ್ಲೀಂಶ್ರೀಂಹ್ರೀಂ ನಂದಿ ಸತ್ಯಾಭ್ಯಾಂ ನಮಃ (ದಕ್ಷಗುಲ್ಫೇ)।\\
ಓಂಹ್ರೀಂಶ್ರೀಂಕ್ಲೀಂ ಢಂ ಕ್ಲೀಂಶ್ರೀಂಹ್ರೀಂ ನರರ್ದ್ಧಿಭ್ಯಾಂ ನಮಃ (ದಕ್ಷಪಾದಾಂಗುಲಿಮೂಲೇ)।\\
ಓಂಹ್ರೀಂಶ್ರೀಂಕ್ಲೀಂ ಣಂ ಕ್ಲೀಂಶ್ರೀಂಹ್ರೀಂ ನರಕಜಿತ್ ಸಮೃದ್ಧಿಭ್ಯಾಂ ನಮಃ (ದಕ್ಷಪಾದಾಂಗುಲ್ಯಗ್ರೇ)।\\
ಓಂಹ್ರೀಂಶ್ರೀಂಕ್ಲೀಂ ತಂ ಕ್ಲೀಂಶ್ರೀಂಹ್ರೀಂ ಹರಿ ಶುದ್ಧಿಭ್ಯಾಂ ನಮಃ (ವಾಮೋರುಮೂಲೇ)।\\
ಓಂಹ್ರೀಂಶ್ರೀಂಕ್ಲೀಂ ಥಂ ಕ್ಲೀಂಶ್ರೀಂಹ್ರೀಂ ಕೃಷ್ಣ ಬುದ್ಧಿಭ್ಯಾಂ ನಮಃ (ವಾಮಜಾನುನಿ)।\\
ಓಂಹ್ರೀಂಶ್ರೀಂಕ್ಲೀಂ ದಂ ಕ್ಲೀಂಶ್ರೀಂಹ್ರೀಂ ಸತ್ಯ ಭುಕ್ತಿಭ್ಯಾಂ ನಮಃ (ವಾಮಗುಲ್ಫೇ)।\\
ಓಂಹ್ರೀಂಶ್ರೀಂಕ್ಲೀಂ ಧಂ ಕ್ಲೀಂಶ್ರೀಂಹ್ರೀಂ ಸಾತ್ವತ ಮತಿಭ್ಯಾಂ ನಮಃ (ವಾಮಪಾದಾಂಗುಲಿಮೂಲೇ)।\\
ಓಂಹ್ರೀಂಶ್ರೀಂಕ್ಲೀಂ ನಂ ಕ್ಲೀಂಶ್ರೀಂಹ್ರೀಂ ಸೌರಿ ಕ್ಷಮಾಭ್ಯಾಂ ನಮಃ (ವಾಮಪಾದಾಂಗುಲ್ಯಗ್ರೇ)।\\
ಓಂಹ್ರೀಂಶ್ರೀಂಕ್ಲೀಂ ಪಂ ಕ್ಲೀಂಶ್ರೀಂಹ್ರೀಂ ಶೂರ ರಮಾಭ್ಯಾಂ ನಮಃ (ದಕ್ಷಪಾರ್ಶ್ವೇ)।\\
ಓಂಹ್ರೀಂಶ್ರೀಂಕ್ಲೀಂ ಫಂ ಕ್ಲೀಂಶ್ರೀಂಹ್ರೀಂ ಜನಾರ್ದನೋಮಾಭ್ಯಾಂ ನಮಃ (ವಾಮಪಾರ್ಶ್ವೇ)।\\
ಓಂಹ್ರೀಂಶ್ರೀಂಕ್ಲೀಂ ಬಂ ಕ್ಲೀಂಶ್ರೀಂಹ್ರೀಂ ಭೂಧರ ಕ್ಲೇದಿನೀಭ್ಯಾಂ ನಮಃ (ಪೃಷ್ಠೇ)।\\
ಓಂಹ್ರೀಂಶ್ರೀಂಕ್ಲೀಂ ಭಂ ಕ್ಲೀಂಶ್ರೀಂಹ್ರೀಂ ವಿಶ್ವಮೂರ್ತಿ ಕ್ಲಿನ್ನಾಭ್ಯಾಂ ನಮಃ (ನಾಭೌ)।\\
ಓಂಹ್ರೀಂಶ್ರೀಂಕ್ಲೀಂ ಮಂ ಕ್ಲೀಂಶ್ರೀಂಹ್ರೀಂ ವೈಕುಂಠ ವಸುಧಾಭ್ಯಾಂ ನಮಃ (ಜಠರೇ)।\\
ಓಂಹ್ರೀಂಶ್ರೀಂಕ್ಲೀಂ ಯಂ ಕ್ಲೀಂಶ್ರೀಂಹ್ರೀಂ ತ್ವಗಾತ್ಮಭ್ಯಾಂ ಪುರುಷೋತ್ತಮ ವಸುದಾಭ್ಯಾಂ ನಮಃ (ಹೃದಿ)।\\
ಓಂಹ್ರೀಂಶ್ರೀಂಕ್ಲೀಂ ರಂ ಕ್ಲೀಂಶ್ರೀಂಹ್ರೀಂ ಅಸೃಗಾತ್ಮಭ್ಯಾಂ ಬಲಿ ಪರಾಭ್ಯಾಂ ನಮಃ (ದಕ್ಷಾಂಸೇ)।\\
ಓಂಹ್ರೀಂಶ್ರೀಂಕ್ಲೀಂ ಲಂ ಕ್ಲೀಂಶ್ರೀಂಹ್ರೀಂ ಮಾಂಸಾತ್ಮಭ್ಯಾಂ ಬಲಾನುಜ ಪರಾಯಣಾಭ್ಯಾಂ ನಮಃ (ಕುಕುದಿ)।\\
ಓಂಹ್ರೀಂಶ್ರೀಂಕ್ಲೀಂ ವಂ ಕ್ಲೀಂಶ್ರೀಂಹ್ರೀಂ ಮೇದಆತ್ಮಭ್ಯಾಂ ಬಾಲ ಸೂಕ್ಷ್ಮಾಭ್ಯಾಂ ನಮಃ (ವಾಮಾಂಸೇ)।\\
ಓಂಹ್ರೀಂಶ್ರೀಂಕ್ಲೀಂ ಶಂ ಕ್ಲೀಂಶ್ರೀಂಹ್ರೀಂ ಅಸ್ಥ್ಯಾತ್ಮಭ್ಯಾಂ ವೃಷಘ್ನ ಸಂಧ್ಯಾಭ್ಯಾಂ ನಮಃ\\ (ಹೃದಯಾದಿ ದಕ್ಷಕರಾಂತಂ)।\\
ಓಂಹ್ರೀಂಶ್ರೀಂಕ್ಲೀಂ ಷಂ ಕ್ಲೀಂಶ್ರೀಂಹ್ರೀಂ ಮಜ್ಜಾತ್ಮಭ್ಯಾಂ ವೃಷ ಪ್ರಜ್ಞಾಭ್ಯಾಂ ನಮಃ \\(ಹೃದಯಾದಿ ವಾಮಕರಾಂತಂ)।\\
ಓಂಹ್ರೀಂಶ್ರೀಂಕ್ಲೀಂ ಸಂ ಕ್ಲೀಂಶ್ರೀಂಹ್ರೀಂ ಶುಕ್ರಾತ್ಮಭ್ಯಾಂ ಹಂಸ ಪ್ರಭಾಭ್ಯಾಂ ನಮಃ \\(ಹೃದಯಾದಿ ದಕ್ಷಪಾದಾಂತಂ)।\\
ಓಂಹ್ರೀಂಶ್ರೀಂಕ್ಲೀಂ ಹಂ ಕ್ಲೀಂಶ್ರೀಂಹ್ರೀಂ ಪ್ರಾಣಾತ್ಮಭ್ಯಾಂ ವರಾಹ ನಿಶಾಭ್ಯಾಂ ನಮಃ \\(ಹೃದಯಾದಿ ವಾಮಪಾದಾಂತಂ)।\\
ಓಂಹ್ರೀಂಶ್ರೀಂಕ್ಲೀಂ ಳಂ ಕ್ಲೀಂಶ್ರೀಂಹ್ರೀಂ ಶಕ್ತ್ಯಾತ್ಮಭ್ಯಾಂ ವಿಮಲಾಮೋಘಾಭ್ಯಾಂ ನಮಃ \\(ಹೃದಯಾದಿ ಉದರಾಂತಂ)।\\
ಓಂಹ್ರೀಂಶ್ರೀಂಕ್ಲೀಂ ಕ್ಷಂ ಕ್ಲೀಂಶ್ರೀಂಹ್ರೀಂಕ್ರೋಧಾತ್ಮಭ್ಯಾಂ ನೃಸಿಂಹ ವಿದ್ಯುತಾಭ್ಯಾಂ ನಮಃ \\(ಹೃದಯಾದಿ ಮೂರ್ಧಪರ್ಯಂತಂ)।
\newpage
 \section{ಅಂತರ್ಮಾತೃಕಾನ್ಯಾಸಃ}
\dhyana{ಆಧಾರೇ ಲಿಂಗನಾಭೌ ಪ್ರಕಟಿತಹೃದಯೇ ತಾಲುಮೂಲೇ ಲಲಾಟೇ\\
ದ್ವೇ ಪತ್ರೇ ಷೋಡಶಾರೇ ದ್ವಿದಶದಶದಲೇ ದ್ವಾದಶಾರ್ಧೇ ಚತುಷ್ಕೇ ।\\
ವಾಸಾಂತೇ ಬಾಲಮಧ್ಯೇ ಡಫಕಠಸಹಿತೇ ಕಂಠದೇಶೇ ಸ್ವರಾಣಾಂ\\
ಹಂ ಕ್ಷಂ ತತ್ತ್ವಾರ್ಥಯುಕ್ತಂ ಸಕಲದಲಗತಂ ವರ್ಣರೂಪಂ ನಮಾಮಿ ॥}\\
(ಕಂಠೇ)ಓಂ ಹ್ರೀಂಶ್ರೀಂಕ್ಲೀಂ ಕ್ಲೀಂಶ್ರೀಂಹ್ರೀಂ ಅಂ ನಮಃಆಂ++++++ಅಃ ನಮಃ।\\
(ಹೃದಯೇ)ಓಂ ಹ್ರೀಂಶ್ರೀಂಕ್ಲೀಂ ಕ್ಲೀಂಶ್ರೀಂಹ್ರೀಂ ಕಂ ನಮಃಖಂ ++++ಠಂ ನಮಃ।\\
(ನಾಭೌ)ಓಂ ಹ್ರೀಂಶ್ರೀಂಕ್ಲೀಂ ಕ್ಲೀಂಶ್ರೀಂಹ್ರೀಂ ಡಂ ನಮಃಢಂ +++++ಫಂ ನಮಃ।\\
(ಗುಹ್ಯೇ)ಓಂ ಹ್ರೀಂಶ್ರೀಂಕ್ಲೀಂ ಕ್ಲೀಂಶ್ರೀಂಹ್ರೀಂ ಬಂ ನಮಃಭಂ++++++ಲಂ ನಮಃ।\\
(ಆಧಾರೇ)ಓಂಹ್ರೀಂಶ್ರೀಂಕ್ಲೀಂ ಕ್ಲೀಂಶ್ರೀಂಹ್ರೀಂ ವಂ ನಮಃಶಂ+++++ಸಂ ನಮಃ।\\
(ಭ್ರೂಮಧ್ಯೇ)ಓಂಹ್ರೀಂಶ್ರೀಂಕ್ಲೀಂ ಕ್ಲೀಂಶ್ರೀಂಹ್ರೀಂ ಹಂ ನಮಃ ಕ್ಷಂ ನಮಃ।\\
(ಸಹಸ್ರಾರೇ)ಓಂಹ್ರೀಂಶ್ರೀಂಕ್ಲೀಂ ಕ್ಲೀಂಶ್ರೀಂಹ್ರೀಂ ಅಂ ನಮಃ++++ಕ್ಷಂ ನಮಃ॥\\(ಪ್ರಾಗ್ವದುತ್ತರನ್ಯಾಸಃ।)
\newpage
\section{ಅಷ್ಟಾಕ್ಷರಮಂತ್ರನ್ಯಾಸಃ}
ಅಸ್ಯ ಶ್ರೀ ಅಷ್ಟಾಕ್ಷರೀಮಹಾಮಂತ್ರಸ್ಯ ಸಾಧ್ಯನಾರಾಯಣ ಋಷಿಃ । ಗಾಯತ್ರೀಚ್ಛಂದಃ । ಮಹಾವಿಷ್ಣುರ್ದೇವತಾ । ಓಂ ಇತಿ ಬೀಜಂ । ನಮ ಇತಿ ಶಕ್ತಿಃ। ನಾರಾಯಣಾಯೇತಿ ಕೀಲಕಂ ।
\section{ಋಷ್ಯಾದಿನ್ಯಾಸಃ}
ಸಾಧ್ಯನಾರಾಯಣಋಷಯೇ ನಮಃ(ಶಿರಸಿ)। ಗಾಯತ್ರೀಚ್ಛಂದಸೇ ನಮಃ(ಮುಖೇ)। ಮಹಾವಿಷ್ಣುದೇವತಾಯೈ ನಮಃ(ಹೃದಯೇ)। ಓಂಬೀಜಾಯ ನಮಃ(ಗುಹ್ಯೇ)। ನಮಶ್ಶಕ್ತಯೇ ನಮಃ(ಪಾದಯೋಃ)। ನಾರಾಯಣಾಯ ಕೀಲಕಾಯ ನಮಃ(ನಾಭೌ)॥
\section{ಅಂಗನ್ಯಾಸಕರನ್ಯಾಸೌ}
ಓಂ ಕ್ರುದ್ಧೋಲ್ಕಾಯ ಸ್ವಾಹಾ ಅಂಗುಷ್ಠಾಭ್ಯಾಂ ನಮಃ ಹೃದಯಾಯ ನಮಃ ।\\
ಓಂ ಮಹೋಲ್ಕಾಯ ಸ್ವಾಹಾ ತರ್ಜನೀಭ್ಯಾಂ ನಮಃ ಶಿರಸೇ ಸ್ವಾಹಾ ।\\
ಓಂ ವೀರೋಲ್ಕಾಯ ಸ್ವಾಹಾ ಮಧ್ಯಮಾಭ್ಯಾಂ ನಮಃ ಶಿಖಾಯೈ ವಷಟ್ ।\\
ಓಂ ದ್ವ್ಯುಲ್ಕಾಯ ಸ್ವಾಹಾ ಅನಾಮಿಕಾಭ್ಯಾಂ ನಮಃ ಕವಚಾಯ ಹುಂ ।\\
ಓಂ ಜ್ಞಾನೋಲ್ಕಾಯ ಸ್ವಾಹಾ ಕನಿಷ್ಠಿಕಾಭ್ಯಾಂ ನಮಃ ನೇತ್ರತ್ರಯಾಯ ವೌಷಟ್ ।\\
ಓಂ ಸಹಸ್ರೋಲ್ಕಾಯ ಸ್ವಾಹಾ ಕರತಲಕರಪೃಷ್ಠಾಭ್ಯಾಂ ನಮಃ ಅಸ್ತ್ರಾಯ ಫಟ್ ।
\section{ಮಂತ್ರವರ್ಣೈಃ ಷಡಂಗನ್ಯಾಸಃ}
ಓಂ \as{ಓಂ} ನಮಃ ಹೃದಯಾಯ ನಮಃ । ಓಂ \as{ನಂ} ನಮಃ ಶಿರಸೇ ಸ್ವಾಹಾ । ಓಂ \as{ಮೋಂ} ನಮಃ ಶಿಖಾಯೈ ವಷಟ್ । ಓಂ \as{ನಾಂ} ನಮಃ ಕವಚಾಯ ಹುಂ । ಓಂ \as{ರಾಂ} ನಮಃ ನೇತ್ರತ್ರಯಾಯ ವೌಷಟ್ । ಓಂ \as{ಯಂ} ನಮಃ ಅಸ್ತ್ರಾಯ ಫಟ್ । ಓಂ \as{ಣಾಂ} ನಮಃ ಕುಕ್ಷ್ಯೋಃ । ಓಂ \as{ಯಂ} ನಮಃ ಪೃಷ್ಠೇ ॥
ಓಂ ನಮಃ ಸುದರ್ಶನಾಯಾಸ್ತ್ರಾಯ ಫಟ್ । ಇತಿ ದಿಗ್ಬಂಧಃ ॥
 \section{ಮಂತ್ರವರ್ಣೈಃ ಅಷ್ಟಾಂಗನ್ಯಾಸಃ}
ಓಂ \as{ಓಂ} ನಮಃ ಆಧಾರೇ । ಓಂ \as{ನಂ} ನಮಃ ಹೃದಿ । ಓಂ \as{ಮೋಂ} ನಮಃ ವಕ್ತ್ರೇ । ಓಂ \as{ನಾಂ} ನಮಃ ದಕ್ಷಿಣಭುಜೇ । ಓಂ \as{ರಾಂ} ನಮಃ ವಾಮಭುಜೇ । ಓಂ \as{ಯಂ} ನಮಃ ದಕ್ಷಿಣಪಾದೇ । ಓಂ \as{ಣಾಂ} ನಮಃ ವಾಮಪಾದೇ । ಓಂ \as{ಯಂ} ನಮಃ ನಾಭೌ ॥೧॥

ಓಂ \as{ಓಂ} ನಮಃ ಕಂಠೇ । ಓಂ \as{ನಂ} ನಮಃ ನಾಭೌ । ಓಂ \as{ಮೋಂ} ನಮಃ ಹೃದಿ । ಓಂ \as{ನಾಂ} ನಮಃ ದಕ್ಷಿಣಸ್ತನೇ । ಓಂ \as{ರಾಂ} ನಮಃ ವಾಮಸ್ತನೇ । ಓಂ \as{ಯಂ} ನಮಃ ದಕ್ಷಿಣಪಾರ್ಶ್ವೇ । ಓಂ \as{ಣಾಂ} ನಮಃ ವಾಮಪಾರ್ಶ್ವೇ । ಓಂ \as{ಯಂ} ನಮಃ ಪೃಷ್ಠೇ ॥೨॥

ಓಂ \as{ಓಂ} ನಮಃ ಮೂರ್ಧ್ನಿ । ಓಂ \as{ನಂ} ನಮಃ ಮುಖೇ । ಓಂ \as{ಮೋಂ} ನಮಃ ದಕ್ಷನೇತ್ರೇ । ಓಂ \as{ನಾಂ} ನಮಃ ವಾಮನೇತ್ರೇ । ಓಂ \as{ರಾಂ} ನಮಃ ದಕ್ಷಿಣಕರ್ಣೇ । ಓಂ \as{ಯಂ} ನಮಃ ವಾಮಕರ್ಣೇ । ಓಂ \as{ಣಾಂ} ನಮಃ ದಕ್ಷಿಣನಾಸಾಪುಟೇ । ಓಂ \as{ಯಂ} ನಮಃ ವಾಮನಾಸಾಪುಟೇ ॥೩॥

ಓಂ \as{ಓಂ} ನಮಃ ದಕ್ಷಬಾಹುಮೂಲೇ । ಓಂ \as{ನಂ} ನಮಃ ದಕ್ಷಕೂರ್ಪರೇ । ಓಂ \as{ಮೋಂ} ನಮಃ ದಕ್ಷಮಣಿಬಂಧೇ । ಓಂ \as{ನಾಂ} ನಮಃ ದಕ್ಷಹಸ್ತಾಂಗುಲೀಮೂಲೇ । ಓಂ \as{ರಾಂ} ನಮಃ ದಕ್ಷಹಸ್ತಾಂಗುಲ್ಯಗ್ರೇ । ಓಂ \as{ಯಂ} ನಮಃ ವಾಮಬಾಹುಮೂಲೇ । ಓಂ \as{ಣಾಂ} ನಮಃ ವಾಮಕೂರ್ಪರೇ । ಓಂ \as{ಯಂ} ನಮಃ ವಾಮಮಣಿಬಂಧೇ ॥೪॥

ಓಂ \as{ಓಂ} ನಮಃ ವಾಮಹಸ್ತಾಂಗುಲೀಮೂಲೇ । ಓಂ \as{ನಂ} ನಮಃ ವಾಮಹಸ್ತಾಂಗುಲ್ಯಗ್ರೇ । ಓಂ \as{ಮೋಂ} ನಮಃ ದಕ್ಷೋರುಮೂಲೇ । ಓಂ \as{ನಾಂ} ನಮಃ ದಕ್ಷಿಣಜಾನುನಿ । ಓಂ \as{ರಾಂ} ನಮಃ ದಕ್ಷಿಣಗುಲ್ಫೇ । ಓಂ \as{ಯಂ} ನಮಃ ದಕ್ಷಿಣಪಾದಾಂಗುಲೀಮೂಲೇ । ಓಂ \as{ಣಾಂ} ನಮಃ ದಕ್ಷಿಣಪಾದಾಂಗುಲ್ಯಗ್ರೇ । ಓಂ \as{ಯಂ} ನಮಃ ವಾಮೋರುಮೂಲೇ ॥೫॥

ಓಂ \as{ಓಂ} ನಮಃ ವಾಮಜಾನುನಿ । ಓಂ \as{ನಂ} ನಮಃ ವಾಮಗುಲ್ಫೇ । ಓಂ \as{ಮೋಂ} ನಮಃ ವಾಮಪಾದಾಂಗುಲೀಮೂಲೇ । ಓಂ \as{ನಾಂ} ನಮಃ ವಾಮಪಾದಾಂಗುಲ್ಯಗ್ರೇ ।(ಹೃದಯೇ ಕರಂ ದತ್ವಾ) ಓಂ \as{ರಾಂ} ನಮಃ ತ್ವಚಿ । ಓಂ \as{ಯಂ} ನಮಃ ರಕ್ತೇ । ಓಂ \as{ಣಾಂ} ನಮಃ ಮಾಂಸೇ । ಓಂ \as{ಯಂ} ನಮಃ ಮೇದಸಿ ॥೬॥

ಓಂ \as{ಓಂ} ನಮಃ ಅಸ್ಥ್ನಿ । ಓಂ \as{ನಂ} ನಮಃ ಮಜ್ಜಾಯಾಂ । ಓಂ \as{ಮೋಂ} ನಮಃ ಶುಕ್ರೇ । ಓಂ \as{ನಾಂ} ನಮಃ ಪ್ರಾಣೇ । ಓಂ \as{ರಾಂ} ನಮಃ ಹೃದಿ । ಓಂ \as{ಯಂ} ನಮಃ ದಕ್ಷಿಣಗಲೇ । ಓಂ \as{ಣಾಂ} ನಮಃ ವಾಮಗಲೇ । ಓಂ \as{ಯಂ} ನಮಃ ಹೃದಿ ॥೭॥

ಓಂ \as{ಓಂ} ನಮಃ ಮೂರ್ಧ್ನಿ । ಓಂ \as{ನಂ} ನಮಃ ನೇತ್ರಯೋಃ। ಓಂ \as{ಮೋಂ} ನಮಃ ಮುಖೇ । ಓಂ \as{ನಾಂ} ನಮಃ ಹೃದಿ । ಓಂ \as{ರಾಂ} ನಮಃ ಕುಕ್ಷೌ । ಓಂ \as{ಯಂ} ನಮಃ ಊರ್ವೋಃ । ಓಂ \as{ಣಾಂ} ನಮಃ ಜಂಘಯೋಃ । ಓಂ \as{ಯಂ} ನಮಃ ಪಾದಯೋಃ ॥೮॥(ಇತಿ ಮಂತ್ರವರ್ಣಾಷ್ಟನ್ಯಾಸಾಃ ॥)
 \section{ಆಯುಧನ್ಯಾಸಃ}
ಓಂ ಚಕ್ರಾಯ ನಮಃ ದಕ್ಷಿಣಗಂಡೇ । ಓಂ ಶಂಖಾಯ ನಮಃ ವಾಮಗಂಡೇ । ಓಂ ಗದಾಯೈ ನಮಃ ದಕ್ಷಿಣಾಂಸೇ । ಓಂ ಪದ್ಮಾಯ ನಮಃ ವಾಮಾಂಸೇ ॥
 \section{ಮೂರ್ತಿಪಂಜರನ್ಯಾಸಃ}
ಓಂ \as{ಓಂ ಅಂ} ಧಾತೃಸಹಿತ ಕೇಶವಾಯ ನಮಃ (ಲಲಾಟೇ)\\
ಓಂ \as{ನಂ ಆಂ} ಅರ್ಯಮಸಹಿತ ನಾರಾಯಣಾಯ ನಮಃ (ಕುಕ್ಷೌ)\\
ಓಂ \as{ಮೋಂ ಇಂ} ಮಿತ್ರಸಹಿತ ಮಾಧವಾಯ ನಮಃ (ಹೃದಿ)\\
ಓಂ \as{ಭಂ ಈಂ} ವರುಣಸಹಿತ ಗೋವಿಂದಾಯ ನಮಃ (ಕಂಠೇ)\\
ಓಂ \as{ಗಂ ಉಂ} ಅಂಶುಸಹಿತ ವಿಷ್ಣವೇ ನಮಃ (ದಕ್ಷಪಾರ್ಶ್ವೇ)\\
ಓಂ \as{ವಂ ಊಂ} ಭಗಸಹಿತ ಮಧುಸೂದನಾಯ ನಮಃ (ದಕ್ಷಾಂಸೇ)\\
ಓಂ \as{ತೇಂ ಏಂ} ವಿವಸ್ವತ್ಸಹಿತ ತ್ರಿವಿಕ್ರಮಾಯ ನಮಃ (ಗಲದಕ್ಷಿಣಭಾಗೇ)\\
ಓಂ \as{ವಾಂ ಐಂ} ಇಂದ್ರಸಹಿತ  ವಾಮನಾಯ ನಮಃ (ವಾಮಪಾರ್ಶ್ವೇ)\\
ಓಂ \as{ಸುಂ ಓಂ} ಪೂಷಸಹಿತ ಶ್ರೀಧರಾಯ ನಮಃ (ವಾಮಾಂಸೇ )\\
ಓಂ \as{ದೇಂ ಔಂ} ಪರ್ಜನ್ಯಸಹಿತ  ಹೃಷೀಕೇಶಾಯ ನಮಃ (ಗಲವಾಮಭಾಗೇ)\\
ಓಂ \as{ವಾಂ ಅಂ} ತ್ವಷ್ಟೃಸಹಿತ ಪದ್ಮನಾಭಾಯ ನಮಃ (ಪೃಷ್ಠೇ)\\
ಓಂ \as{ಯಂ ಅಃ} ವಿಷ್ಣುಸಹಿತ  ದಾಮೋದರಾಯ ನಮಃ (ಕಕುದಿ )\\
ಓಂ \as{ಓಂ ನಮೋ ಭಗವತೇ ವಾಸುದೇವಾಯ} (ಮೂರ್ಧ್ನಿ)
 \section{ಕಿರೀಟಮಂತ್ರನ್ಯಾಸಃ}
ಓಂ ಕಿರೀಟ ಕೇಯೂರ ಹಾರ ಮಕರಕುಂಡಲಧರ ಶಂಖಚಕ್ರಗದಾಂಭೋಜ ಪೀತಾಂಬರಧರ ಶ್ರೀವತ್ಸಾಂಕಿತವಕ್ಷಃಸ್ಥಲ ಶ್ರೀಭೂಮೀಸಹಿತಾತ್ಮಜ್ಯೋತಿರ್ದ್ವಯದೀಪ್ತಕರಾಯ ಸಹಸ್ರಾದಿತ್ಯತೇಜಸೇ ನಮಃ ॥(ಇತಿ ಮಂತ್ರೇಣ ಶಿರಆದಿಪಾದಾಂತಂ ವ್ಯಾಪಕಂ ವಿನ್ಯಸೇತ್ ॥)
\begin{center}{\bfseries ಉದ್ಯತ್ಕೋಟಿದಿವಾಕರಾಭಮನಿಶಂ ಶಂಖಂ ಗದಾಂ ಪಂಕಜಂ\\
ಚಕ್ರಂ ಬಿಭ್ರತಮಿಂದಿರಾವಸುಮತೀಸಂಶೋಭಿಪಾರ್ಶ್ವದ್ವಯಂ ।\\
ಕೋಟೀರಾಂಗದಹಾರಕುಂಡಲಧರಂ ಪೀತಾಂಬರಂ ಕೌಸ್ತುಭೋ\\
ದ್ದೀಪ್ತಂ ವಿಶ್ವಧರಂ ಸ್ವವಕ್ಷಸಿ ಲಸಚ್ಛ್ರೀವತ್ಸಚಿಹ್ನಂ ಭಜೇ ॥}\end{center}
\newpage
 \section{ದ್ವಾದಶಾಕ್ಷರವಿಷ್ಣುಮಂತ್ರನ್ಯಾಸಃ}
ಅಸ್ಯ ಶ್ರೀದ್ವಾದಶಾಕ್ಷರವಿಷ್ಣುಮಂತ್ರಸ್ಯ ಪ್ರಜಾಪತಿರ್ಋಷಿಃ । ಗಾಯತ್ರೀ ಛಂದಃ । ವಾಸುದೇವಪರಮಾತ್ಮಾ ದೇವತಾ । ಜಪೇ ವಿನಿಯೋಗಃ ॥
 \section{ಋಷ್ಯಾದಿನ್ಯಾಸಃ}
ಪ್ರಜಾಪತಿಋಷಯೇ ನಮಃ (ಶಿರಸಿ)। ಗಾಯತ್ರೀಚ್ಛಂದಸೇ ನಮಃ (ಮುಖೇ)। ವಾಸುದೇವಪರಮಾತ್ಮದೇವತಾಯೈ ನಮಃ(ಹೃದಯೇ)। ವಿನಿಯೋಗಾಯ ನಮಃ (ಸರ್ವಾಂಗೇ)।
  \section{ಅಂಗುಲಿನ್ಯಾಸಪಂಚಾಂಗನ್ಯಾಸೌ}
ಓಂ \as{ಓಂ} ಅಂಗುಷ್ಠಾಭ್ಯಾಂ ನಮಃ । ಹೃದಯಾಯ ನಮಃ ।\\
ಓಂ \as{ನಮಃ} ತರ್ಜನೀಭ್ಯಾಂ ನಮಃ । ಶಿರಸೇ ಸ್ವಾಹಾ ।\\
ಓಂ \as{ಭಗವತೇ} ಮಧ್ಯಮಾಭ್ಯಾಂ ನಮಃ । ಶಿಖಾಯೈ ವಷಟ್ ।\\
ಓಂ \as{ವಾಸುದೇವಾಯ} ಅನಾಮಿಕಾಭ್ಯಾಂ ನಮಃ । ಕವಚಾಯ ಹುಂ ।\\
ಓಂ \as{ಓಂ ನಮೋ ಭಗವತೇ ವಾಸುದೇವಾಯ} ಕನಿಷ್ಠಿಕಾಭ್ಯಾಂ ನಮಃ । ಅಸ್ತ್ರಾಯ ಫಟ್ ।
\newpage
  \section{ಮಂತ್ರವರ್ಣನ್ಯಾಸಃ}
ಓಂ \as{ಓಂ} ನಮಃ ಮೂರ್ಧ್ನಿ । ಓಂ \as{ನಂ} ನಮಃ ಲಲಾಟೇ । ಓಂ \as{ಮೋಂ} ನಮಃ ನೇತ್ರಯೋಃ । ಓಂ \as{ಭಂ} ನಮಃ ಮುಖೇ । ಓಂ \as{ಗಂ} ನಮಃ ಗಲೇ । ಓಂ \as{ವಂ} ನಮಃ ಬಾಹ್ವೋಃ । ಓಂ \as{ತೇಂ} ನಮಃ ಹೃದಯೇ । ಓಂ \as{ವಾಂ} ನಮಃ ಕುಕ್ಷೌ । ಓಂ \as{ಸುಂ} ನಮಃ ನಾಭೌ । ಓಂ \as{ದೇಂ} ನಮಃ ಲಿಂಗೇ । ಓಂ \as{ವಾಂ} ನಮಃ ಜಾನ್ವೋಃ । ಓಂ \as{ಯಂ} ನಮಃ ಪಾದಯೋಃ ॥
\begin{center}\section{ಧ್ಯಾನಂ}
{\bfseries ಕ್ಷೀರೋದನ್ವತ್ಪ್ರದೇಶೇ ಶುಚಿಮಣಿವಿಲಸತ್ಸೈಕತೇ ಮೌಕ್ತಿಕಾನಾಂ\\
ಮಾಲಾಕೢಪ್ತಾಸನಸ್ಥಃ ಸ್ಫಟಿಕಮಣಿನಿಭೈರ್ಮೌಕ್ತಿಕೈರ್ಮಂಡಿತಾಂಗಃ ।\\
ಶುಭ್ರೈರಭ್ರೈರದಭ್ರೈರುಪರಿವಿರಚಿತೈರ್ಮುಕ್ತಪೀಯೂಷವರ್ಷೈಃ\\
ಆನಂದೀ ನಃ ಪುನೀಯಾದರಿನಳಿನಗದಾಶಂಖಪಾಣಿರ್ಮುಕುಂದಃ ॥

ಭೂಃ ಪಾದೌ ಯಸ್ಯ ನಾಭಿರ್ವಿಯದಸುರನಿಲಶ್ಚಂದ್ರಸೂರ್ಯೌ ಚ ನೇತ್ರೇ\\
ಕರ್ಣಾವಾಶಾಶ್ಶಿರೋ ದ್ಯೌರ್ಮುಖಮಪಿ ದಹನೋ ಯಸ್ಯ ವಾಸ್ತೇಯಮಬ್ಧಿಃ ।\\
ಅಂತಸ್ಥಂ ಯಸ್ಯ ವಿಶ್ವಂ ಸುರನರಖಗಗೋಭೋಗಿಗಂಧರ್ವದೈತ್ಯೈಃ\\
ಚಿತ್ರಂ ರಂರಮ್ಯತೇ ತಂ ತ್ರಿಭುವನವಪುಷಂ ವಿಷ್ಣುಮೀಶಂ ನಮಾಮಿ ॥
\newpage
ಶಾಂತಾಕಾರಂ ಭುಜಗಶಯನಂ ಪದ್ಮನಾಭಂ ಸುರೇಶಂ\\
ವಿಶ್ವಾಧಾರಂ ಗಗನಸದೃಶಂ ಮೇಘವರ್ಣಂ ಶುಭಾಂಗಂ ।\\
ಲಕ್ಷ್ಮೀಕಾಂತಂ ಕಮಲನಯನಂ ಯೋಗಿಹೃದ್ಧ್ಯಾನಗಮ್ಯಂ\\
ವಂದೇ ವಿಷ್ಣುಂ ಭವಭಯಹರಂ ಸರ್ವಲೋಕೈಕನಾಥಂ ॥

ಮೇಘಶ್ಯಾಮಂ ಪೀತಕೌಶೇಯವಾಸಂ ಶ್ರೀವತ್ಸಾಂಕಂ ಕೌಸ್ತುಭೋದ್ಭಾಸಿತಾಂಗಂ ।\\
ಪುಣ್ಯೋಪೇತಂ ಪುಂಡರೀಕಾಯತಾಕ್ಷಂ ವಿಷ್ಣುಂ ವಂದೇ ಸರ್ವಲೋಕೈಕನಾಥಂ ॥

ಸಶಂಖಚಕ್ರಂ ಸಕಿರೀಟಕುಂಡಲಂ ಸಪೀತವಸ್ತ್ರಂ ಸರಸೀರುಹೇಕ್ಷಣಂ ।\\
ಸಹಾರವಕ್ಷಃಸ್ಥಲಶೋಭಿಕೌಸ್ತುಭಂ ನಮಾಮಿ ವಿಷ್ಣುಂ ಶಿರಸಾ ಚತುರ್ಭುಜಂ ॥

ವಿಷ್ಣುಂ ಶಾರದಚಂದ್ರಕೋಟಿಸದೃಶಂ ಶಂಖಂ ರಥಾಂಗಂ ಗದಾ\\
ಮಂಭೋಜಂ ದಧತಂ ಸಿತಾಬ್ಜನಿಲಯಂ ಕಾಂತ್ಯಾ ಜಗನ್ಮೋಹನಂ ।\\
ಆಬದ್ಧಾಂಗದಹಾರಕುಂಡಲಮಹಾಮೌಲಿಂ ಸ್ಫುರತ್ಕಂಕಣಂ\\
ಶ್ರೀವತ್ಸಾಂಕಮುದಾರಕೌಸ್ತುಭಧರಂ ವಂದೇ ಮುನೀಂದ್ರೈಃ ಸ್ತುತಂ ॥}\end{center}
\newpage
 \section{ಆತ್ಮರಕ್ಷಾನ್ಯಾಸಃ ॥}
ಓಂ ಐಂ ಹ್ರೀಂ ಶ್ರೀಂ ಲಕ್ಷ್ಮೀನಾರಾಯಣ ಮಮ ಆತ್ಮಾನಂ ರಕ್ಷ ರಕ್ಷ ॥(ಅಂಜಲಿಂ ಹೃದಿ ಸಮರ್ಪಯೇತ್ ।)
  \section{ಅಥ ಸೂಕ್ತನ್ಯಾಸಃ}
ಸಹಸ್ರಶೀರ್ಷಾ ಪುರುಷಃ*****ದಶಾಂಗುಲಂ॥(ಶಿರಸಿ)\\
ಪುರುಷ ಏವೇದಂ***** ನಾ ತಿರೋಹತಿ॥(ನೇತ್ರಯೋಃ)\\
ಏತಾವಾನಸ್ಯ***** ತ್ರಿಪಾದಸ್ಯಾಮೃತಂ ದಿವಿ॥(ಕರ್ಣಯೋಃ)\\
ತ್ರಿಪಾದೂರ್ದ್ಧ್ವ*****ಶನೇ ಅಭಿ॥(ನಾಸಿಕಯೋಃ)\\
ತಸ್ಮಾತ್ ವಿರಾಳ*****ಪಶ್ಚಾತ್ ಭೂಮಿಮಥೋ ಪುರಃ॥(ಮುಖೇ)\\
ಯತ್ ಪುರುಷೇಣ***** ಇದ್ಧ್ಮಃ ಶರದ್ಧವಿಃ॥(ಗಲೇ)\\
ತಂ ಯಜ್ಞಂ ***** ಋಷಯಶ್ಚ ಯೇ॥(ಬಾಹ್ವೋಃ)\\
ತಸ್ಮಾದ್ಯಜ್ಞಾ*****ಗ್ರಾಮ್ಯಾಶ್ಚ ಯೇ॥(ಹೃದಿ)\\
ತಸ್ಮಾದ್ಯಜ್ಞಾ*****ತಸ್ಮಾದಜಾಯತ॥(ನಾಭೌ)\\
ತಸ್ಮಾದಶ್ವಾ*****ಜಾತಾ ಅಜಾವಯಃ॥(ಗುಹ್ಯೇ)\\
ಯತ್ಪುರುಷಂ ******ಪಾದಾ ಉಚ್ಯೇತೇ॥(ಗುದೇ)\\
ಬ್ರಾಹ್ಮಣೋಽಸ್ಯ *****ಶೂದ್ರೋ ಅಜಾಯತ॥(ಊರ್ವೋಃ)\\
ಚಂದ್ರಮಾ ಮನಸೋ*****  ಪ್ರಾಣಾದ್ವಾಯುರಜಾಯತ॥(ಜಾನುನೋಃ)\\
ನಾಭ್ಯಾ *****ಲೋಕಾನ್ ಅಕಲ್ಪಯನ್॥(ಜಂಘಯೋಃ)\\
ಸಪ್ತಾಸ್ಯಾಸನ್***** ಅಬಧ್ನನ್ಪುರುಷಂ ಪಶುಂ॥(ಪಾದಯೋಃ)\\
ಯಜ್ಞೇನ ಯಜ್ಞಮಯ*****ಪೂರ್ವೇ ಸಾಧ್ಯಾಃ ಸಂತಿ ದೇವಾಃ॥(ಸರ್ವಾಂಗೇ)
\chapter*{\center ॥ ಶ್ರೀಮಹಾಲಕ್ಷ್ಮೀಹೃದಯಂ ॥}
ಅಸ್ಯ ಶ್ರೀಮಹಾಲಕ್ಷ್ಮೀಹೃದಯಸ್ತೋತ್ರಮಹಾಮಂತ್ರಸ್ಯ ಭಾರ್ಗವ ಋಷಿಃ । ಅನುಷ್ಟುಬಾದಿನಾನಾಛಂದಾಂಸಿ । ಆದ್ಯಾದಿಶ್ರೀಮಹಾಲಕ್ಷ್ಮೀರ್ದೇವತಾ । ಶ್ರೀಂ ಬೀಜಂ । ಹ್ರೀಂ ಶಕ್ತಿಃ । ಐಂ ಕೀಲಕಂ । ಶ್ರೀಮಹಾಲಕ್ಷ್ಮೀಪ್ರಸಾದಸಿದ್ಧ್ಯರ್ಥೇ ಜಪೇ ವಿನಿಯೋಗಃ ॥
\section{ಋಷ್ಯಾದಿನ್ಯಾಸಃ}
ಭಾರ್ಗವ ಋಷಯೇ ನಮಃ ।ಅನುಷ್ಟುಬಾದಿನಾನಾಛಂದೋಭ್ಯೋ ನಮಃ । ಆದ್ಯಾದಿಶ್ರೀಮಹಾಲಕ್ಷ್ಮೀದೇವತಾಯೈ ನಮಃ । ಶ್ರೀಂಬೀಜಾಯ ನಮಃ। ಹ್ರೀಂಶಕ್ತಯೇ ನಮಃ । ಐಂಕೀಲಕಾಯ ನಮಃ ।
\newpage
\section{ಕರನ್ಯಾಸಪಂಚಾಂಗನ್ಯಾಸೌ}
ಓಂ ಶ್ರಾಂ ಅಂಗುಷ್ಠಾಭ್ಯಾಂ ನಮಃ । ಹೃದಯಾಯ ನಮಃ ।\\
ಓಂ ಶ್ರೀಂ ತರ್ಜನೀಭ್ಯಾಂ ನಮಃ । ಶಿರಸೇ ಸ್ವಾಹಾ ।\\
ಓಂ ಶ್ರೂಂ ಮಧ್ಯಮಾಭ್ಯಾಂ ನಮಃ । ಶಿಖಾಯೈ ವಷಟ್ ।\\
ಓಂ ಶ್ರೈಂ ಅನಾಮಿಕಾಭ್ಯಾಂ ನಮಃ । ಕವಚಾಯ ಹುಂ ।\\
ಓಂ ಶ್ರೌಂ ಕನಿಷ್ಠಿಕಾಭ್ಯಾಂ ನಮಃ । ನೇತ್ರತ್ರಯಾಯ ವೌಷಟ್ ।\\
ಓಂ ಶ್ರಃ ಕರತಲಕರಪೃಷ್ಠಾಭ್ಯಾಂ ನಮಃ । ಅಸ್ತ್ರಾಯ ಫಟ್ ।

\dhyana{ಪೀತವಸ್ತ್ರಾಂ ಸುವರ್ಣಾಂಗೀಂ ಪದ್ಮಹಸ್ತಾಂ ಗದಾನ್ವಿತಾಂ ।\\
ಲಕ್ಷ್ಮೀಂ ಧ್ಯಾಯೇತ ಮಂತ್ರೇಣ ಸ ಭವೇತ್ ಪೃಥಿವೀಪತಿಃ ॥\\
ಮಾತುಲುಂಗಂ ಗದಾಂ ಖೇಟಂ ಪಾನಪಾತ್ರಂಚ ಬಿಭ್ರತೀಂ ।\\
ನಾಗಂ ಲಿಂಗಂಚ ಯೋನಿಂಚ ಬಿಭ್ರತೀಂ ಚೈವ ಮೂರ್ಧನಿ ॥\\
ವಿಷ್ಣುಸ್ತುತಿಪರಾಂ ಲಕ್ಷ್ಮೀಂ ಸ್ವರ್ಣವರ್ಣಾಂ ಸ್ತುತಿಪ್ರಿಯಾಂ ।\\
ವರಾಭಯಪ್ರದಾಂ ದೇವೀಂ ವಂದೇ ತಾಂ ಕಮಲೇಕ್ಷಣಾಂ ॥
\newpage
ಕೌಶೇಯಪೀತವಸನಾಂ ಅರವಿಂದನೇತ್ರಾಂ\\ ಪದ್ಮದ್ವಯಾಭಯವರೋದ್ಯತಪದ್ಮಹಸ್ತಾಂ ।\\
ಉದ್ಯಚ್ಛತಾರ್ಕಸದೃಶೀಂ ಪರಮಾಂಕಸಂಸ್ಥಾಂ \\ಧ್ಯಾಯೇದ್ವಿಧೀಶನತಪಾದಯುಗಾಂ ಜನಿತ್ರೀಂ ॥

	ಯಾ ಸಾ ಪದ್ಮಾಸನಸ್ಥಾ ವಿಪುಲಕಟಿತಟೀ ಪದ್ಮಪತ್ರಾಯತಾಕ್ಷೀ\\
	ಗಂಭೀರಾವರ್ತನಾಭಿಃ ಸ್ತನಭರನಮಿತಾ ಶುಭ್ರವಸ್ತ್ರೋತ್ತರೀಯಾ ।\\
	ಲಕ್ಷ್ಮೀರ್ದಿವ್ಯೈರ್ಗಜೇಂದ್ರೈರ್ಮಣಿಗಣಖಚಿತೈಃ ಸ್ನಾಪಿತಾ ಹೇಮಕುಂಭೈಃ\\
	ನಿತ್ಯಂ ಸಾ ಪದ್ಮಹಸ್ತಾ ಮಮ ವಸತು ಗೃಹೇ ಸರ್ವಮಾಂಗಲ್ಯಯುಕ್ತಾ ॥

ಹಸ್ತದ್ವಯೇನ ಕಮಲೇ ಧಾರಯಂತೀಂ ಸ್ವಲೀಲಯಾ ।\\
ಹಾರನೂಪುರಸಂಯುಕ್ತಾಂ ಲಕ್ಷ್ಮೀಂ ದೇವೀಂ ವಿಚಿಂತಯೇ ॥}\\(ಲಮಿತ್ಯಾದಿನಾ ಸಂಪೂಜ್ಯ ।)

ಶಂಕಚಕ್ರಗದಾಹಸ್ತೇ ಶುಭ್ರವರ್ಣೇ ಶುಭಾನನೇ ।\\
ಮಮ ದೇಹಿ ವರಂ ಲಕ್ಷ್ಮಿ ಸರ್ವಸಿದ್ಧಿಪ್ರದಾಯಿನೀ ॥
	(ಇತಿ ಸಂಪ್ರಾರ್ಥ್ಯ ॥)\\
{\bfseries ಓಂ ಶ್ರೀಂ ಹ್ರೀಂ ಐಂ ಮಹಾಲಕ್ಷ್ಮ್ಯೈ ಕಮಲಧಾರಿಣ್ಯೈ ಸಿಂಹವಾಹಿನ್ಯೈ ಸ್ವಾಹಾ ॥(೧೦೮)}

ವಂದೇ ಲಕ್ಷ್ಮೀಂ ಪ್ರಹಸಿತಮುಖೀಂ ಶುದ್ಧಜಾಂಬೂನದಾಭಾಂ\\
ತೇಜೋರೂಪಾಂ ಕನಕವಸನಾಂ ಸರ್ವಭೂಷೋಜ್ಜ್ವಲಾಂಗೀಂ ।\\
	ಬೀಜಾಪೂರಂ ಕನಕಕಲಶಂ ಹೇಮಪದ್ಮೇ ದಧಾನಾಂ\\
	ಆದ್ಯಾಂ ಶಕ್ತಿಂ ಸಕಲಜನನೀಂ ವಿಷ್ಣುಮಾಂಕಸಂಸ್ಥಾಂ ॥೧॥

ಶ್ರೀಮತ್ಸೌಭಾಗ್ಯಜನನೀಂ ಸ್ತೌಮಿ ಲಕ್ಷ್ಮೀಂ ಸನಾತನೀಂ ।\\
ಸರ್ವಕಾಮಫಲಾವಾಪ್ತಿಸಾಧನೈಕಸುಖಾವಹಾಂ ॥೨॥

	ಓಂ ಶ್ರೀಂ ಹ್ರೀಂ ಐಂ ಮಹಾಲಕ್ಷ್ಮ್ಯೈ ನಮಃ ॥

	ಸ್ಮರಾಮಿ ನಿತ್ಯಂ ದೇವೇಶಿ ತ್ವಯಾ ಪ್ರೇರಿತಮಾನಸಃ ।\\
	ತ್ವದಾಜ್ಞಾಂ ಶಿರಸಾ ಧೃತ್ವಾ ಭಜಾಮಿ ಪರಮೇಶ್ವರೀಂ ॥೩॥

ಸಮಸ್ತಸಂಪತ್ಸುಖದಾಂ ಮಹಾಶ್ರಿಯಂ ಸಮಸ್ತಸೌಭಾಗ್ಯಕರೀಂ ಮಹಾಶ್ರಿಯಂ ।\\
ಸಮಸ್ತಕಲ್ಯಾಣಕರೀಂ ಮಹಾಶ್ರಿಯಂ ಭಜಾಮ್ಯಹಂ ಜ್ಞಾನಕರೀಂ ಮಹಾಶ್ರಿಯಂ ॥೪॥

	ವಿಜ್ಞಾನಸಂಪತ್ಸುಖದಾಂ ಸನಾತನೀಂ ವಿಚಿತ್ರವಾಗ್ಭೂತಿಕರೀಂ ಮನೋಹರಾಂ ।\\
	ಅನಂತಸಮ್ಮೋದಸುಖಪ್ರದಾಯಿನೀಂ ನಮಾಮ್ಯಹಂ ಭೂತಿಕರೀಂ ಹರಿಪ್ರಿಯಾಂ ॥೫॥

ಸಮಸ್ತಭೂತಾಂತರಸಂಸ್ಥಿತಾ ತ್ವಂ ಸಮಸ್ತಭೋಕ್ತ್ರೀಶ್ಶ್ವರಿ ವಿಶ್ವರೂಪೇ ।\\
ತನ್ನಾಸ್ತಿ ಯತ್ತ್ವದ್ವ್ಯತಿರಿಕ್ತವಸ್ತು ತ್ವತ್ಪಾದಪದ್ಮಂ ಪ್ರಣಮಾಮ್ಯಹಂ ಶ್ರೀಃ ॥೬॥

	ದಾರಿದ್ರ್ಯದುಃಖೌಘತಮೋಪಹಂತ್ರಿ ತ್ವತ್ಪಾದಪದ್ಮಂ ಮಯಿ ಸನ್ನಿಧತ್ಸ್ವ ।\\
	ದೀನಾರ್ತಿವಿಚ್ಛೇದನಹೇತುಭೂತೈಃ ಕೃಪಾಕಟಾಕ್ಷೈರಭಿಷಿಂಚ ಮಾಂ ಶ್ರೀಃ ॥೭॥

ಅಂಬ ಪ್ರಸೀದ ಕರುಣಾಸುಧಯಾರ್ದ್ರದೃಷ್ಟ್ಯಾ \\ಮಾಂ ತ್ವತ್ಕೃಪಾದ್ರವಿಣಗೇಹಮಿಮಂ ಕುರುಷ್ವ ।\\
ಆಲೋಕಯ ಪ್ರಣತಹೃದ್ಗತಶೋಕಹಂತ್ರಿ \\ತ್ವತ್ಪಾದಪದ್ಮಯುಗಲಂ ಪ್ರಣಮಾಮ್ಯಹಂ ಶ್ರೀಃ ॥೮॥

	ಶಾಂತ್ಯೈ ನಮೋಽಸ್ತು ಶರಣಾಗತರಕ್ಷಣಾಯೈ\\ ಕಾಂತ್ಯೈ ನಮೋಽಸ್ತು ಕಮನೀಯಗುಣಾಶ್ರಯಾಯೈ ।\\
	ಕ್ಷಾಂತ್ಯೈ ನಮೋಽಸ್ತು ದುರಿತಕ್ಷಯಕಾರಣಾಯೈ \\ಧಾತ್ರ್ಯೈ ನಮೋಽಸ್ತು ಧನಧಾನ್ಯಸಮೃದ್ಧಿದಾಯೈ ॥೯॥

ಶಕ್ತ್ಯೈ ನಮೋಽಸ್ತು ಶಶಿಶೇಖರಸಂಸ್ತುತಾಯೈ\\ ರತ್ಯೈ ನಮೋಽಸ್ತು ರಜನೀಕರಸೋದರಾಯೈ ।\\
ಭಕ್ತ್ಯೈ ನಮೋಽಸ್ತು ಭವಸಾಗರತಾರಕಾಯೈ \\ಮತ್ಯೈ ನಮೋಽಸ್ತು ಮಧುಸೂದನವಲ್ಲಭಾಯೈ ॥೧೦॥

	ಲಕ್ಷ್ಮ್ಯೈ ನಮೋಽಸ್ತು ಶುಭಲಕ್ಷಣಲಕ್ಷಿತಾಯೈ \\ಸಿದ್ಧ್ಯೈ ನಮೋಽಸ್ತು ಶಿವಸಿದ್ಧಸುಪೂಜಿತಾಯೈ ।\\
	ಧೃತ್ಯೈ ನಮೋಽಸ್ತ್ವಮಿತದುರ್ಗತಿಭಂಜನಾಯೈ \\ಗತ್ಯೈ ನಮೋಽಸ್ತು ವರಸದ್ಗತಿದಾಯಿಕಾಯೈ ॥೧೧॥

ದೇವ್ಯೈ ನಮೋಽಸ್ತು ದಿವಿ ದೇವಗಣಾರ್ಚಿತಾಯೈ \\ಭೂತ್ಯೈ ನಮೋಽಸ್ತು ಭುವನಾರ್ತಿವಿನಾಶನಾಯೈ ।\\
ಧಾತ್ರ್ಯೈ ನಮೋಽಸ್ತು ಧರಣೀಧರವಲ್ಲಭಾಯೈ \\ಪುಷ್ಟ್ಯೈ ನಮೋಽಸ್ತು ಪುರುಷೋತ್ತಮವಲ್ಲಭಾಯೈ ॥೧೨॥

	ಸುತೀವ್ರದಾರಿದ್ರ್ಯವಿದುಃಖಹಂತ್ರ್ಯೈ \\ನಮೋಽಸ್ತು ತೇ ಸರ್ವಭಯಾಪಹಂತ್ರ್ಯೈ ।\\
	ಶ್ರೀವಿಷ್ಣುವಕ್ಷಃಸ್ಥಲಸಂಸ್ಥಿತಾಯೈ \\ನಮೋ ನಮಃ ಸರ್ವವಿಭೂತಿದಾಯೈ ॥೧೩॥

ಜಯತು ಜಯತು ಲಕ್ಷ್ಮೀರ್ಲಕ್ಷಣಾಲಂಕೃತಾಂಗೀ\\ ಜಯತು ಜಯತು ಪದ್ಮಾ ಪದ್ಮಸದ್ಮಾಭಿವಂದ್ಯಾ ।\\
ಜಯತು ಜಯತು ವಿದ್ಯಾ ವಿಷ್ಣುವಾಮಾಂಕಸಂಸ್ಥಾ \\ಜಯತು ಜಯತು ಸಮ್ಯಕ್ಸರ್ವಸಂಪತ್ಕರೀ ಶ್ರೀಃ ॥೧೪॥

ಜಯತು ಜಯತು ದೇವೀ ದೇವಸಂಘಾಭಿಪೂಜ್ಯಾ \\ಜಯತು ಜಯತು ಭದ್ರಾ ಭಾರ್ಗವೀ ಭಾಗ್ಯರೂಪಾ ।\\
ಜಯತು ಜಯತು ನಿತ್ಯಾ ನಿರ್ಮಲಜ್ಞಾನವೇದ್ಯಾ \\ಜಯತು ಜಯತು ಸತ್ಯಾ ಸರ್ವಭೂತಾಂತರಸ್ಥಾ ॥೧೫॥

	ಜಯತು ಜಯತು ರಮ್ಯಾ ರತ್ನಗರ್ಭಾಂತರಸ್ಥಾ \\ಜಯತು ಜಯತು ಶುದ್ಧಾ ಶುದ್ಧಜಾಂಬೂನದಾಭಾ ।\\
	ಜಯತು ಜಯತು ಕಾಂತಾ ಕಾಂತಿಮದ್ಭಾಸಿತಾಂಗೀ \\ಜಯತು ಜಯತು ಶಾಂತಾ ಶೀಘ್ರಮಾಗಚ್ಛ ಸೌಮ್ಯೇ ॥೧೬॥

ಯಸ್ಯಾಃ ಕಲಾಯಾಃ ಕಮಲೋದ್ಭವಾದ್ಯಾ ರುದ್ರಾಶ್ಚ ಶಕ್ರಪ್ರಮುಖಾಶ್ಚ ದೇವಾಃ ।\\
ಜೀವಂತಿ ಸರ್ವೇಽಪಿ ಸಶಕ್ತಯಸ್ತೇ ಪ್ರಭುತ್ವಮಾಪುಃ ಪರಮಾಯುಷಸ್ತೇ ॥೧೭॥

{\bfseries ಪಾದಬೀಜಂ । ಓಂ ಆಂ ಈಂ ಯಂ ಪಂ ಕಂ ಲಂ ಹಂ ॥}

ಲಿಲೇಖ ನಿಟಿಲೇ ವಿಧಿರ್ಮಮ ಲಿಪಿಂ ವಿಸೃಜ್ಯಾಂತರಂ\\
ತ್ವಯಾ ವಿಲಿಖಿತವ್ಯಮೇತದಿತಿ ತತ್ಫಲಪ್ರಾಪ್ತಯೇ ।\\
	ತದಂತರತಲೇ ಸ್ಫುಟಂ ಕಮಲವಾಸಿನಿ ಶ್ರೀರಿಮಾಂ\\
	ಸಮರ್ಪಯ ಸಮುದ್ರಿಕಾಂ ಸಕಲಭಾಗ್ಯಸಂಸೂಚಿಕಾಂ ॥೧೮॥

ತದಿದಂ ತಿಮಿರಂ ಫಾಲೇ ಸ್ಫುಟಂ ಕಮಲವಾಸಿನಿ ।\\
ಶ್ರಿಯಂ ಸಮುದ್ರಿಕಾಂ ದೇಹಿ ಸರ್ವಭಾಗ್ಯಸ್ಯ ಸೂಚಿಕಾಂ ॥೧೯॥

{\bfseries ಮುಖಬೀಜಂ । ಓಂ ಹ್ರಾಂ ಹ್ರೀಂ ಆಂ ವ್ಯಾಂ ಯಂ ಭಾಂ ಸಾಂ ॥}

	ಕಲಯಾ ತೇ ಯಥಾ ದೇವಿ ಜೀವಂತಿ ಸಚರಾಚರಾಃ ।\\
	ತಥಾ ಸಂಪತ್ಕರೇ ಲಕ್ಷ್ಮಿ ಸರ್ವದಾ ಸಂಪ್ರಸೀದ ಮೇ ॥೨೦॥

ಯಥಾ ವಿಷ್ಣುರ್ಧ್ರುವೇ ನಿತ್ಯಂ ಸ್ವಕಲಾಂ ಸನ್ನ್ಯವೇಶಯತ್ ।\\
ತಥೈವ ಸ್ವಕಲಾಂ ಲಕ್ಷ್ಮಿ ಮಯಿ ಸಮ್ಯಕ್ಸಮರ್ಪಯ ॥೨೧॥
\newpage
	ಸರ್ವಸೌಖ್ಯಪ್ರದೇ ದೇವಿ ಭಕ್ತಾನಾಮಭಯಪ್ರದೇ ।\\
	ಅಚಲಾಂ ಕುರು ಯತ್ನೇನ ಕಲಾಂ ಮಯಿ ನಿವೇಶಿತಾಂ ॥೨೨॥

ಮುದಾಸ್ತಾಂ ಮತ್ಫಾಲೇ ಪರಮಪದಲಕ್ಷ್ಮೀಃ ಸ್ಫುಟಕಲಾ\\
ಸದಾ ವೈಕುಂಠಶ್ರೀರ್ನಿವಸತು ಕಲಾ ಮೇ ನಯನಯೋಃ ।\\
	ವಸೇತ್ಸತ್ಯೇ ಲೋಕೇ ಮಮ ವಚಸಿ ಲಕ್ಷ್ಮೀರ್ವರಕಲಾ\\
	ಶ್ರಿಯಃ ಶ್ವೇತದ್ವೀಪೇ ನಿವಸತು ಕಲಾ ಮೇಽಸ್ತು ಕರಯೋಃ ॥೨೩॥

{\bfseries ನೇತ್ರಬೀಜಂ । ಓಂ ಘ್ರಾಂ ಘ್ರೀಂ ಘ್ರೂಂ ಘ್ರೈಂ ಘ್ರೌಂ ಘ್ರಃ ॥}

ತಾವನ್ನಿತ್ಯಂ ಮಮಾಂಗೇಷು ಕ್ಷೀರಾಬ್ಧೌ ಶ್ರೀಕಲಾ ವಸೇತ್ ।\\
ಸೂರ್ಯಾಚಂದ್ರಮಸೌ ಯಾವದ್ಯಾವಲ್ಲಕ್ಷ್ಮೀಪತಿಃ ಶ್ರಿಯಾ ॥೨೪॥

	ಸರ್ವಮಂಗಲಸಂಪೂರ್ಣಾ ಸರ್ವೈಶ್ವರ್ಯಸಮನ್ವಿತಾ ।\\
	ಆದ್ಯಾದಿಶ್ರೀರ್ಮಹಾಲಕ್ಷ್ಮೀಸ್ತ್ವತ್ಕಲಾ ಮಯಿ ತಿಷ್ಠತು ॥೨೫॥

ಅಜ್ಞಾನತಿಮಿರಂ ಹಂತುಂ ಶುದ್ಧಜ್ಞಾನಪ್ರಕಾಶಿಕಾ ।\\
ಸರ್ವೈಶ್ವರ್ಯಪ್ರದಾ ಮೇಽಸ್ತು ತ್ವತ್ಕಲಾ ಮಯಿ ಸಂಸ್ಥಿತಾ ॥೨೬॥

	ಅಲಕ್ಷ್ಮೀಂ ಹರತು ಕ್ಷಿಪ್ರಂ ತಮಃ ಸೂರ್ಯಪ್ರಭಾ ಯಥಾ ।\\
	ವಿತನೋತು ಮಮ ಶ್ರೇಯಸ್ತ್ವತ್ಕಲಾ ಮಯಿ ಸಂಸ್ಥಿತಾ ॥೨೭॥

ಐಶ್ವರ್ಯಮಂಗಲೋತ್ಪತ್ತಿಃ ತ್ವತ್ಕಲಾಯಾಂ ನಿಧೀಯತೇ ।\\
ಮಯಿ ತಸ್ಮಾತ್ಕೃತಾರ್ಥೋಽಸ್ಮಿ ಪಾತ್ರಮಸ್ಮಿ ಸ್ಥಿತೇಸ್ತವ ॥೨೮॥

	ಭವದಾವೇಶಭಾಗ್ಯಾರ್ಹೋ ಭಾಗ್ಯವಾನಸ್ಮಿ ಭಾರ್ಗವಿ ।\\
	ತ್ವತ್ಪ್ರಸಾದಾತ್ಪವಿತ್ರೋಽಹಂ ಲೋಕಮಾತರ್ನಮೋಽಸ್ತು ತೇ ॥೨೯॥

{\bfseries ಜಿಹ್ವಾಬೀಜಂ । ಓಂ ಹ್ರಾಂ ಹ್ರೀಂ ಹ್ರೂಂ ಹ್ರೈಂ ಹ್ರೌಂ ಹ್ರಃ ॥}

ಪುನಾಸಿ ಮಾಂ ತ್ವಂ ಕಲಯೈವ ಯಸ್ಮಾದತಃ ಸಮಾಗಚ್ಛ ಮಮಾಗ್ರತಸ್ತ್ವಂ ।\\
ಪರಂ ಪದಂ ಶ್ರೀರ್ಭವ ಸುಪ್ರಸನ್ನಾ ಮಯ್ಯಚ್ಯುತೇನ ಪ್ರವಿಶಾಽಽದಿಲಕ್ಷ್ಮಿ ॥೩೦॥

	ಶ್ರೀವೈಕುಂಠಸ್ಥಿತೇ ಲಕ್ಷ್ಮಿ ಸಮಾಗಚ್ಛ ಮಮಾಗ್ರತಃ ।\\
	ನಾರಾಯಣೇನ ಸಹ ಮಾಂ ಕೃಪಾದೃಷ್ಟ್ಯಾಽವಲೋಕಯ ॥೩೧॥

	ಸತ್ಯಲೋಕಸ್ಥಿತೇ ಲಕ್ಷ್ಮಿ ತ್ವಂ ಮಮಾಗಚ್ಛ ಸನ್ನಿಧಿಂ ।\\
	ವಾಸುದೇವೇನ ಸಹಿತಾ ಪ್ರಸೀದ ವರದಾ ಭವ ॥೩೨॥

ಶ್ವೇತದ್ವೀಪಸ್ಥಿತೇ ಲಕ್ಷ್ಮಿ ಶೀಘ್ರಮಾಗಚ್ಛ ಸುವ್ರತೇ ।\\
ವಿಷ್ಣುನಾ ಸಹಿತಾ ದೇವಿ ಜಗನ್ಮಾತಃ ಪ್ರಸೀದ ಮೇ ॥೩೩॥

	ಕ್ಷೀರಾಂಬುಧಿಸ್ಥಿತೇ ಲಕ್ಷ್ಮಿ ಸಮಾಗಚ್ಛ ಸಮಾಧವೇ ।\\
	ತ್ವತ್ಕೃಪಾದೃಷ್ಟಿಸುಧಯಾ ಸತತಂ ಮಾಂ ವಿಲೋಕಯ ॥೩೪॥

ರತ್ನಗರ್ಭಸ್ಥಿತೇ ಲಕ್ಷ್ಮಿ ಪರಿಪೂರ್ಣಹಿರಣ್ಮಯಿ ।\\
ಸಮಾಗಚ್ಛ ಸಮಾಗಚ್ಛ ಸ್ಥಿತ್ವಾಶು ಪುರತೋ ಮಮ ॥೩೫॥

	ಸ್ಥಿರಾ ಭವ ಮಹಾಲಕ್ಷ್ಮಿ ನಿಶ್ಚಲಾ ಭವ ನಿರ್ಮಲೇ ।\\
	ಪ್ರಸನ್ನೇ ಕಮಲೇ ದೇವಿ ಪ್ರಸನ್ನಹೃದಯಾ ಭವ ॥೩೬॥

ಶ್ರೀಧರೇ ಶ್ರೀಮಹಾಭೂಮೇ ತ್ವದಂತಃಸ್ಥಂ ಮಹಾನಿಧಿಂ ।\\
ಶೀಘ್ರಮುದ್ಧೃತ್ಯ ಪುರತಃ ಪ್ರದರ್ಶಯ ಸಮರ್ಪಯ ॥೩೭॥

	ವಸುಂಧರೇ ಶ್ರೀವಸುಧೇ ವಸುದೋಗ್ಧ್ರಿ ಕೃಪಾಮಯಿ ।\\
	ತ್ವತ್ಕುಕ್ಷಿಗತಸರ್ವಸ್ವಂ ಶೀಘ್ರಂ ಮೇ ಸಂಪ್ರದರ್ಶಯ ॥೩೮॥
\newpage
ವಿಷ್ಣುಪ್ರಿಯೇ ರತ್ನಗರ್ಭೇ ಸಮಸ್ತಫಲದೇ ಶಿವೇ ।\\
ತ್ವದ್ಗರ್ಭಗತಹೇಮಾದೀನ್ ಸಂಪ್ರದರ್ಶಯ ದರ್ಶಯ ॥೩೯॥

	ರಸಾತಲಗತೇ ಲಕ್ಷ್ಮಿ ಶೀಘ್ರಮಾಗಚ್ಛ ಮೇ ಪುರಃ ।\\
	ನ ಜಾನೇ ಪರಮಂ ರೂಪಂ ಮಾತರ್ಮೇ ಸಂಪ್ರದರ್ಶಯ ॥೪೦॥

ಆವಿರ್ಭವ ಮನೋವೇಗಾತ್ ಶೀಘ್ರಮಾಗಚ್ಛ ಮೇ ಪುರಃ ।\\
ಮಾ ವತ್ಸ ಭೈರಿಹೇತ್ಯುಕ್ತ್ವಾ ಕಾಮಂ ಗೌರಿವ ರಕ್ಷ ಮಾಂ ॥೪೧॥

	ದೇವಿ ಶೀಘ್ರಂ ಮಮಾಗಚ್ಛ ಧರಣೀಗರ್ಭಸಂಸ್ಥಿತೇ ।\\
	ಮಾತಸ್ತ್ವದ್ಭೃತ್ಯಭೃತ್ಯೋಽಹಂ ಮೃಗಯೇ ತ್ವಾಂ ಕುತೂಹಲಾತ್ ॥೪೨॥

ಉತ್ತಿಷ್ಠ ಜಾಗೃಹಿ ತ್ವಂ ಮೇ ಸಮುತ್ತಿಷ್ಠ ಸುಜಾಗೃಹಿ ।\\
ಅಕ್ಷಯಾನ್ ಹೇಮಕಲಶಾನ್ ಸುವರ್ಣೇನ ಸುಪೂರಿತಾನ್ ॥೪೩॥

	ನಿಕ್ಷೇಪಾನ್ಮೇ ಸಮಾಕೃಷ್ಯ ಸಮುದ್ಧೃತ್ಯ ಮಮಾಗ್ರತಃ ।\\
	ಸಮುನ್ನತಾನನಾ ಭೂತ್ವಾ ಸಮಾಧೇಹಿ ಧರಾಂತರಾತ್ ॥೪೪॥
\newpage
ಮತ್ಸನ್ನಿಧಿಂ ಸಮಾಗಚ್ಛ ಮದಾಹಿತಕೃಪಾರಸಾತ್ ।\\
ಪ್ರಸೀದ ಶ್ರೇಯಸಾಂ ದೋಗ್ಧ್ರಿ ಲಕ್ಷ್ಮಿ ಮೇ ನಯನಾಗ್ರತಃ ॥೪೫॥

	ಅತ್ರೋಪವಿಶ್ಯ ಲಕ್ಷ್ಮಿ ತ್ವಂ ಸ್ಥಿರಾ ಭವ ಹಿರಣ್ಮಯಿ ।\\
	ಸುಸ್ಥಿರಾ ಭವ ಸಂಪ್ರೀತ್ಯಾ ಪ್ರಸೀದ ವರದಾ ಭವ ॥೪೬॥

ಆನೀಯ ತ್ವಂ ತಥಾ ದೇವಿ ನಿಧೀನ್ಮೇ ಸಂಪ್ರದರ್ಶಯ ।\\
ಅದ್ಯ ಕ್ಷಣೇನ ಸಹಸಾ ದತ್ತ್ವಾ ಸಂರಕ್ಷ ಮಾಂ ಸದಾ ॥೪೭॥

	ಮಯಿ ತಿಷ್ಠ ತಥಾ ನಿತ್ಯಂ ಯಥೇಂದ್ರಾದಿಷು ತಿಷ್ಠಸಿ ।\\
	ಅಭಯಂ ಕುರು ಮೇ ದೇವಿ ಮಹಾಲಕ್ಷ್ಮಿ ನಮೋಽಸ್ತು ತೇ ॥೪೮॥

ಸಮಾಗಚ್ಛ ಮಹಾಲಕ್ಷ್ಮಿ ಶುದ್ಧಜಾಂಬೂನದಪ್ರಭೇ ।\\
ಪ್ರಸೀದ ಪುರತಃ ಸ್ಥಿತ್ವಾ ಪ್ರಣತಂ ಮಾಂ ವಿಲೋಕಯ ॥೪೯॥

	ಲಕ್ಷ್ಮೀರ್ಭುವಂ ಗತಾ ಭಾಸಿ ಯತ್ರ ಯತ್ರ ಹಿರಣ್ಮಯಿ ।\\
	ತತ್ರ ತತ್ರ ಸ್ಥಿತಾ ತ್ವಂ ಮೇ ತವ ರೂಪಂ ಪ್ರದರ್ಶಯ ॥೫೦॥
\newpage
ಕ್ರೀಡಸೇ ಬಹುಧಾ ಭೂಮೌ ಪರಿಪೂರ್ಣಾ ಹಿರಣ್ಮಯೀ ।\\
ಮಮ ಮೂರ್ಧನಿ ತೇ ಹಸ್ತಮವಿಲಂಬಿತಮರ್ಪಯ ॥೫೧॥

ಫಲದ್ಭಾಗ್ಯೋದಯೇ ಲಕ್ಷ್ಮಿ ಸಮಸ್ತಪುರವಾಸಿನಿ ।\\
ಪ್ರಸೀದ ಮೇ ಮಹಾಲಕ್ಷ್ಮಿ ಪರಿಪೂರ್ಣಮನೋರಥೇ ॥೫೨॥

	ಅಯೋಧ್ಯಾದಿಷು ಸರ್ವೇಷು ನಗರೇಷು ಸಮಾಸ್ಥಿತೇ ।\\
	ವೈಭವೈರ್ವಿವಿಧೈರ್ಯುಕ್ತಾ ಸಮಾಗಚ್ಛ ಬಲಾನ್ವಿತೇ ॥೫೩॥

ಸಮಾಗಚ್ಛ ಸಮಾಗಚ್ಛ ಮಮಾಗ್ರೇ ಭವ ಸುಸ್ಥಿರಾ ।\\
ಕರುಣಾರಸನಿಷ್ಯಂದನೇತ್ರದ್ವಯವಿಲಾಸಿನಿ ॥೫೪॥

	ಸನ್ನಿಧತ್ಸ್ವ ಮಹಾಲಕ್ಷ್ಮಿ ತ್ವತ್ಪಾಣಿಂ ಮಮ ಮಸ್ತಕೇ ।\\
	ಕರುಣಾಸುಧಯಾ ಮಾಂ ತ್ವಮಭಿಷಿಂಚ ಸ್ಥಿರೀಕುರು ॥೫೫॥

ಸರ್ವರಾಜಗೃಹೇ ಲಕ್ಷ್ಮಿ ಸಮಾಗಚ್ಛ ಬಲಾನ್ವಿತೇ ।\\
ಸ್ಥಿತ್ವಾಽಽಶು ಪುರತೋ ಮೇಽದ್ಯ ಪ್ರಸಾದೇನಾಭಯಂ ಕುರು ॥೫೬॥
\newpage
	ಸಾದರಂ ಮಸ್ತಕೇ ಹಸ್ತಂ ಮಮ ತ್ವಂ ಕೃಪಯಾಽರ್ಪಯ ।\\
	ಸರ್ವರಾಜಗೃಹೇ ಲಕ್ಷ್ಮಿ ತ್ವತ್ಕಲಾ ಮಯಿ ತಿಷ್ಠತು ॥೫೭॥

ಆದ್ಯಾದಿಶ್ರೀರ್ಮಹಾಲಕ್ಷ್ಮಿ ವಿಷ್ಣುವಾಮಾಂಕಸಂಸ್ಥಿತೇ ।\\
ಪ್ರತ್ಯಕ್ಷಂ ಕುರು ಮೇ ರೂಪಂ ರಕ್ಷ ಮಾಂ ಶರಣಾಗತಂ ॥೫೮॥

	ಪ್ರಸೀದ ಮೇ ಮಹಾಲಕ್ಷ್ಮಿ ಸುಪ್ರಸೀದ ಮಹಾಶಿವೇ ।\\
	ಅಚಲಾ ಭವ ಸಂಪ್ರೀತ್ಯಾ ಸುಸ್ಥಿರಾ ಭವ ಮದ್ಗೃಹೇ ॥೫೯॥

ಯಾವತ್ತಿಷ್ಠಂತಿ ವೇದಾಶ್ಚ ಯಾವತ್ತ್ವನ್ನಾಮ ತಿಷ್ಠತಿ ।\\
ಯಾವದ್ವಿಷ್ಣುಶ್ಚ ಯಾವತ್ತ್ವಂ ತಾವತ್ಕುರು ಕೃಪಾಂ ಮಯಿ ॥೬೦॥

	ಚಾಂದ್ರೀ ಕಲಾ ಯಥಾ ಶುಕ್ಲೇ ವರ್ಧತೇ ಸಾ ದಿನೇ ದಿನೇ ।\\
	ತಥಾ ದಯಾ ತೇ ಮಯ್ಯೇವ ವರ್ಧತಾಮಭಿವರ್ಧತಾಂ ॥೬೧॥

ಯಥಾ ವೈಕುಂಠನಗರೇ ಯಥಾ ವೈ ಕ್ಷೀರಸಾಗರೇ ।\\
ತಥಾ ಮದ್ಭವನೇ ತಿಷ್ಠ ಸ್ಥಿರಂ ಶ್ರೀವಿಷ್ಣುನಾ ಸಹ ॥೬೨॥
\newpage
	ಯೋಗಿನಾಂ ಹೃದಯೇ ನಿತ್ಯಂ ಯಥಾ ತಿಷ್ಠಸಿ ವಿಷ್ಣುನಾ ।\\
	ತಥಾ ಮದ್ಭವನೇ ತಿಷ್ಠ ಸ್ಥಿರಂ ಶ್ರೀವಿಷ್ಣುನಾ ಸಹ ॥೬೩॥

ನಾರಾಯಣಸ್ಯ ಹೃದಯೇ ಭವತೀ ಯಥಾಽಽಸ್ತೇ\\ ನಾರಾಯಣೋಽಪಿ ತವ ಹೃತ್ಕಮಲೇ ಯಥಾಽಽಸ್ತೇ ।\\
ನಾರಾಯಣಸ್ತ್ವಮಪಿ ನಿತ್ಯಮುಭೌ ತಥೈವ \\ತೌ ತಿಷ್ಠತಾಂ ಹೃದಿ ಮಮಾಪಿ ದಯಾವತಿ ಶ್ರೀಃ ॥೬೪॥

	ವಿಜ್ಞಾನವೃದ್ಧಿಂ ಹೃದಯೇ ಕುರು ಶ್ರೀಃ \\ಸೌಭಾಗ್ಯವೃದ್ಧಿಂ ಕುರು ಮೇ ಗೃಹೇ ಶ್ರೀಃ ।\\
	ದಯಾಸುವೃದ್ಧಿಂ ಕುರುತಾಂ ಮಯಿ ಶ್ರೀಃ \\ಸುವರ್ಣವೃದ್ಧಿಂ ಕುರು ಮೇ ಗೃಹೇ ಶ್ರೀಃ ॥೬೫॥

ನ ಮಾಂ ತ್ಯಜೇಥಾಃ ಶ್ರಿತಕಲ್ಪವಲ್ಲಿ \\ಸದ್ಭಕ್ತಚಿಂತಾಮಣಿಕಾಮಧೇನೋ ।\\
ವಿಶ್ವಸ್ಯ ಮಾತರ್ಭವ ಸುಪ್ರಸನ್ನಾ \\ಗೃಹೇ ಕಲತ್ರೇಷು ಚ ಪುತ್ರವರ್ಗೇ ॥೬೬॥

{\bfseries ಕುಕ್ಷಿಬೀಜಂ । ಓಂ ಆಂ ಈಂ ಏಂ ಐಂ ॥}

	ಆದ್ಯಾದಿಮಾಯೇ ತ್ವಮಜಾಂಡಬೀಜಂ\\ ತ್ವಮೇವ ಸಾಕಾರನಿರಾಕೃತಿಸ್ತ್ವಂ ।\\
	ತ್ವಯಾ ಧೃತಾಶ್ಚಾಬ್ಜಭವಾಂಡಸಂಘಾಃ\\ಚಿತ್ರಂ ಚರಿತ್ರಂ ತವ ದೇವಿ ವಿಷ್ಣೋಃ ॥೬೭॥

ಬ್ರಹ್ಮರುದ್ರಾದಯೋ ದೇವಾ ವೇದಾಶ್ಚಾಪಿ ನ ಶಕ್ನುಯುಃ ।\\
ಮಹಿಮಾನಂ ತವ ಸ್ತೋತುಂ ಮಂದೋಽಹಂ ಶಕ್ನುಯಾಂ ಕಥಂ ॥೬೮॥

	ಅಂಬ ತ್ವದ್ವತ್ಸವಾಕ್ಯಾನಿ ಸೂಕ್ತಾಸೂಕ್ತಾನಿ ಯಾನಿ ಚ ।\\
	ತಾನಿ ಸ್ವೀಕುರು ಸರ್ವಜ್ಞೇ ದಯಾಲುತ್ವೇನ ಸಾದರಂ ॥೬೯॥

ಭವತೀಂ ಶರಣಂ ಗತ್ವಾ ಕೃತಾರ್ಥಾಃ ಸ್ಯುಃ ಪುರಾತನಾಃ ।\\
ಇತಿ ಸಂಚಿಂತ್ಯ ಮನಸಾ ತ್ವಾಮಹಂ ಶರಣಂ ವ್ರಜೇ ॥೭೦॥

	ಅನಂತಾ ನಿತ್ಯಸುಖಿನಃ ತ್ವದ್ಭಕ್ತಾಸ್ತ್ವತ್ಪರಾಯಣಾಃ ।\\
	ಇತಿ ವೇದಪ್ರಮಾಣಾದ್ಧಿ ದೇವಿ ತ್ವಾಂ ಶರಣಂ ವ್ರಜೇ ॥೭೧॥

ತವ ಪ್ರತಿಜ್ಞಾ ಮದ್ಭಕ್ತಾ ನ ನಶ್ಯಂತೀತ್ಯಪಿ ಕ್ವಚಿತ್ ।\\
ಇತಿ ಸಂಚಿಂತ್ಯ ಸಂಚಿಂತ್ಯ ಪ್ರಾಣಾನ್ ಸಂಧಾರಯಾಮ್ಯಹಂ ॥೭೨॥

	ತ್ವದಧೀನಸ್ತ್ವಹಂ ಮಾತಃ ತ್ವತ್ಕೃಪಾ ಮಯಿ ವಿದ್ಯತೇ ।\\
	ಯಾವತ್ಸಂಪೂರ್ಣಕಾಮಃ ಸ್ಯಾಂ ತಾವದ್ದೇಹಿ ದಯಾನಿಧೇ ॥೭೩॥

ಕ್ಷಣಮಾತ್ರಂ ನ ಶಕ್ನೋಮಿ ಜೀವಿತುಂ ತ್ವತ್ಕೃಪಾಂ ವಿನಾ ।\\
ನ ಜೀವಂತೀಹ ಜಲಜಾ ಜಲಂ ತ್ಯಕ್ತ್ವಾ ಜಲಗ್ರಹಾಃ ॥೭೪॥

	ಯಥಾ ಹಿ ಪುತ್ರವಾತ್ಸಲ್ಯಾತ್ ಜನನೀ ಪ್ರಸ್ನುತಸ್ತನೀ ।\\
	ವತ್ಸಂ ತ್ವರಿತಮಾಗತ್ಯ ಸಂಪ್ರೀಣಯತಿ ವತ್ಸಲಾ ॥೭೫॥

ಯದಿ ಸ್ಯಾಂ ತವ ಪುತ್ರೋಽಹಂ ಮಾತಾ ತ್ವಂ ಯದಿ ಮಾಮಕೀ ।\\
ದಯಾಪಯೋಧರಸ್ತನ್ಯಸುಧಾಭಿರಭಿಷಿಂಚ ಮಾಂ ॥೭೬॥

	ಮೃಗ್ಯೋ ನ ಗುಣಲೇಶೋಽಪಿ ಮಯಿ ದೋಷೈಕಮಂದಿರೇ ।\\
	ಪಾಂಸೂನಾಂ ವೃಷ್ಟಿಬಿಂದೂನಾಂ ದೋಷಾಣಾಂ ಚ ನ ಮೇ ಮಿತಿಃ ॥೭೭॥
\newpage
ಪಾಪಿನಾಮಹಮೇವಾಗ್ರ್ಯೋ ದಯಾಲೂನಾಂ ತ್ವಮಗ್ರಣೀಃ ।\\
ದಯನೀಯೋ ಮದನ್ಯೋಽಸ್ತಿ ತವ ಕೋಽತ್ರ ಜಗತ್ತ್ರಯೇ ॥೭೮॥

	ವಿಧಿನಾಹಂ ನ ಸೃಷ್ಟಶ್ಚೇತ್ ನ ಸ್ಯಾತ್ತವ ದಯಾಲುತಾ ।\\
	ಆಮಯೋ ವಾ ನ ಸೃಷ್ಟಶ್ಚೇದೌಷಧಸ್ಯ ವೃಥೋದಯಃ ॥೭೯॥

ಕೃಪಾ ಮದಗ್ರಜಾ ಕಿಂ ತೇ ತ್ವಹಂ ಕಿಂ ವಾ ತದಗ್ರಜಃ ।\\
ವಿಚಾರ್ಯ ದೇಹಿ ಮೇ ವಿತ್ತಂ ತವ ದೇವಿ ದಯಾನಿಧೇ ॥೮೦॥

	ಮಾತಾ ಪಿತಾ ತ್ವಂ ಗುರುಸದ್ಗತೀ ಶ್ರೀಃ\\ತ್ವಮೇವ ಸಂಜೀವನಹೇತುಭೂತಾ ।\\
	ಅನ್ಯಂ ನ ಮನ್ಯೇ ಜಗದೇಕನಾಥೇ\\ ತ್ವಮೇವ ಸರ್ವಂ ಮಮ ದೇವಿ ಸತ್ಯೇ ॥೮೧॥

{\bfseries  ಹೃದಯ ಬೀಜಂ ॥ ಓಂ ಆಂ ಕ್ರೌಂ ಹುಂ ಫಟ್ ಕುರು ಕುರು ಸ್ವಾಹಾ ॥}

ಆದ್ಯಾದಿಲಕ್ಷ್ಮೀರ್ಭವ ಸುಪ್ರಸನ್ನಾ ವಿಶುದ್ಧವಿಜ್ಞಾನಸುಖೈಕದೋಗ್ಧ್ರೀ ।\\
ಅಜ್ಞಾನಹಂತ್ರೀ ತ್ರಿಗುಣಾತಿರಿಕ್ತಾ ಪ್ರಜ್ಞಾನನೇತ್ರೀ ಭವ ಸುಪ್ರಸನ್ನಾ ॥೮೨॥

	ಅಶೇಷವಾಗ್ಜಾಡ್ಯಮಲಾಪಹಂತ್ರೀ ನವಂ ನವಂ ಸ್ಪಷ್ಟಸುವಾಕ್ಪ್ರದಾಯಿನೀ ।\\
	ಮಮೇಹ ಜಿಹ್ವಾಂಗಣರಂಗನರ್ತಕೀ ಭವ ಪ್ರಸನ್ನಾ ವದನೇ ಚ ಮೇ ಶ್ರೀಃ ॥೮೩॥

ಸಮಸ್ತಸಂಪತ್ಸು ವಿರಾಜಮಾನಾ ಸಮಸ್ತತೇಜಶ್ಚಯಭಾಸಮಾನಾ ।\\
ವಿಷ್ಣುಪ್ರಿಯೇ ತ್ವಂ ಭವ ದೀಪ್ಯಮಾನಾ ವಾಗ್ದೇವತಾ ಮೇ ನಯನೇ ಪ್ರಸನ್ನಾ ॥೮೪॥

	ಭಕ್ತ್ಯಾ ನತಾನಾಂ ಸಕಲಾರ್ಥಸಿದ್ಧಿಪ್ರದೇ ಸುಲಾವಣ್ಯದಯಾಪ್ರದೋಗ್ಧ್ರಿ ।\\
	ಸುವರ್ಣದೇ ತ್ವಂ ಸುಮುಖೀ ಭವ ಶ್ರೀರ್ಹಿರಣ್ಮಯೀ ಮೇ ನಯನೇ ಪ್ರಸನ್ನಾ ॥೮೫॥

ಸರ್ವಾರ್ಥದಾ ಸರ್ವಜಗತ್ಪ್ರಸೂತಿಃ ಸರ್ವೇಶ್ವರೀ ಸರ್ವಭಯಾಪಹಂತ್ರೀ ।\\
ಸರ್ವೋನ್ನತಾ ತ್ವಂ ಸುಮುಖೀ ಭವ ಶ್ರೀರ್ಹಿರಣ್ಮಯೀ ಮೇ ನಯನೇ ಪ್ರಸನ್ನಾ ॥೮೬॥

	ಸಮಸ್ತವಿಘ್ನೌಘವಿನಾಶಕಾರಿಣೀ ಸಮಸ್ತಭಕ್ತೋದ್ಧರಣೇ ವಿಚಕ್ಷಣಾ ।\\
	ಅನಂತಸೌಭಾಗ್ಯಸುಖಪ್ರದಾಯಿನೀ ಹಿರಣ್ಮಯೀ ಮೇ ನಯನೇ ಪ್ರಸನ್ನಾ ॥೮೭॥

ದೇವಿ ಪ್ರಸೀದ ದಯನೀಯತಮಾಯ ಮಹ್ಯಂ ದೇವಾಧಿನಾಥಭವದೇವಗಣಾಭಿವಂದ್ಯೇ ।\\
ಮಾತಸ್ತಥೈವ ಭವ ಸನ್ನಿಹಿತಾ ದೃಶೋರ್ಮೇ ಪತ್ಯಾ ಸಮಂ ಮಮ ಮುಖೇ ಭವ ಸುಪ್ರಸನ್ನಾ ॥೮೮॥
\newpage
	ಮಾ ವತ್ಸ ಭೈರಭಯದಾನಕರೋಽರ್ಪಿತಸ್ತೇ\\ ಮೌಲೌ ಮಮೇತಿ ಮಯಿ ದೀನದಯಾನುಕಂಪೇ ।\\
	ಮಾತಃ ಸಮರ್ಪಯ ಮುದಾ ಕರುಣಾಕಟಾಕ್ಷಂ\\ ಮಾಂಗಲ್ಯಬೀಜಮಿಹ ನಃ ಸೃಜ ಜನ್ಮ ಮಾತಃ ॥೮೯॥

{\bfseries  ಕಂಠಬೀಜಂ । ಓಂ ಶ್ರಾಂ ಶ್ರೀಂ ಶ್ರೂಂ ಶ್ರೈಂ ಶ್ರೌಂ ಶ್ರಃ ॥}

ಕಟಾಕ್ಷ ಇಹ ಕಾಮಧುಕ್ ತವ ಮನಸ್ತು ಚಿಂತಾಮಣಿಃ\\ ಕರಃ ಸುರತರುಃ ಸದಾ ನವನಿಧಿಸ್ತ್ವಮೇವೇಂದಿರೇ ।\\
ಭವೇತ್ತವ ದಯಾರಸೋ ಮಮ ರಸಾಯನಂ ಚಾನ್ವಹಂ\\ ಮುಖಂ ತವ ಕಲಾನಿಧಿರ್ವಿವಿಧವಾಂಛಿತಾರ್ಥಪ್ರದಂ ॥೯೦॥

	ಯಥಾ ರಸಸ್ಪರ್ಶನತೋಽಯಸೋಽಪಿ ಸುವರ್ಣತಾ ಸ್ಯಾತ್ಕಮಲೇ ತಥಾ ತೇ ।\\
	ಕಟಾಕ್ಷಸಂಸ್ಪರ್ಶನತೋ ಜನಾನಾಮಮಂಗಲಾನಾಮಪಿ ಮಂಗಲತ್ವಂ ॥೯೧॥
	
ದೇಹೀತಿ ನಾಸ್ತೀತಿ ವಚಃ ಪ್ರವೇಶಾದ್ಭೀತೋ ರಮೇ ತ್ವಾಂ ಶರಣಂ ಪ್ರಪದ್ಯೇ ।\\
ಅತಃ ಸದಾಸ್ಮಿನ್ನಭಯಪ್ರದಾ ತ್ವಂ ಸಹೈವ ಪತ್ಯಾ ಮಯಿ ಸನ್ನಿಧೇಹಿ ॥೯೨॥

	ಕಲ್ಪದ್ರುಮೇಣ ಮಣಿನಾ ಸಹಿತಾ ಸುರಭ್ಯಾ \\ಶ್ರೀಸ್ತೇ ಕಲಾ ಮಯಿ ರಸೇನ ರಸಾಯನೇನ ।\\
	ಆಸ್ತಾಂ ಯತೋ ಮಮ ಚ ದೃಕ್ಶಿರಪಾಣಿಪಾದ\\ಸ್ಪೃಷ್ಟಾಃ ಸುವರ್ಣವಪುಷಃ ಸ್ಥಿರಜಂಗಮಾಃ ಸ್ಯುಃ ॥೯೩॥

ಆದ್ಯಾದಿವಿಷ್ಣೋಃ ಸ್ಥಿರಧರ್ಮಪತ್ನೀ ತ್ವಮೇವ ಪತ್ಯಾ ಮಯಿ ಸನ್ನಿಧೇಹಿ ।\\
ಆದ್ಯಾದಿಲಕ್ಷ್ಮಿ ತ್ವದನುಗ್ರಹೇಣ ಪದೇ ಪದೇ ಮೇ ನಿಧಿದರ್ಶನಂ ಸ್ಯಾತ್ ॥೯೪॥

	ಆದ್ಯಾದಿಲಕ್ಷ್ಮೀಹೃದಯಂ ಪಠೇದ್ಯಃ ಸ ರಾಜ್ಯಲಕ್ಷ್ಮೀಮಚಲಾಂ ತನೋತಿ ।\\
	ಮಹಾದರಿದ್ರೋಽಪಿ ಭವೇದ್ಧನಾಢ್ಯಃ ತದನ್ವಯೇ ಶ್ರೀಃ ಸ್ಥಿರತಾಂ ಪ್ರಯಾತಿ ॥೯೫॥

ಯಸ್ಯ ಸ್ಮರಣಮಾತ್ರೇಣ ತುಷ್ಟಾ ಸ್ಯಾದ್ವಿಷ್ಣುವಲ್ಲಭಾ ।\\
ತಸ್ಯಾಭೀಷ್ಟಂ ದದಾತ್ಯಾಶು ತಂ ಪಾಲಯತಿ ಪುತ್ರವತ್ ॥೯೬॥

ಇದಂ ರಹಸ್ಯಂ ಹೃದಯಂ ಸರ್ವಕಾಮಫಲಪ್ರದಂ ।\\
ಜಪಃ ಪಂಚಸಹಸ್ರಂ ತು ಪುರಶ್ಚರಣಮುಚ್ಯತೇ ॥೯೭॥
\newpage
	ತ್ರಿಕಾಲಂ ಏಕಕಾಲಂ ವಾ ನರೋ ಭಕ್ತಿಸಮನ್ವಿತಃ ।\\
	ಯಃ ಪಠೇತ್ ಶೃಣುಯಾದ್ವಾಪಿ ಸ ಯಾತಿ ಪರಮಾಂ ಶ್ರಿಯಂ ॥೯೮॥

ಮಹಾಲಕ್ಷ್ಮೀಂ ಸಮುದ್ದಿಶ್ಯ ನಿಶಿ ಭಾರ್ಗವವಾಸರೇ ।\\
ಇದಂ ಶ್ರೀಹೃದಯಂ ಜಪ್ತ್ವಾ ಪಂಚವಾರಂ ಧನೀ ಭವೇತ್ ॥೯೯॥

	ಅನೇನ ಹೃದಯೇನಾನ್ನಂ ಗರ್ಭಿಣ್ಯೈ ಚಾಭಿಮಂತ್ರಿತಂ ।\\
	ದದಾತಿ ತತ್ಕುಲೇ ಪುತ್ರೋ ಜಾಯತೇ ಶ್ರೀಪತಿಃ ಸ್ವಯಂ ॥೧೦೦॥

ನರೇಣ ವಾಥವಾ ನಾರ್ಯಾ ಲಕ್ಷ್ಮೀಹೃದಯಮಂತ್ರಿತೇ ।\\
ಜಲೇ ಪೀತೇ ಚ ತದ್ವಂಶೇ ಮಂದಭಾಗ್ಯೋ ನ ಜಾಯತೇ ॥೧೦೧॥

ಯ ಆಶ್ವಿನೇ ಮಾಸಿ ಚ ಶುಕ್ಲಪಕ್ಷೇ ರಮೋತ್ಸವೇ ಸನ್ನಿಹಿತೈಕಭಕ್ತ್ಯಾ ।\\
ಪಠೇತ್ತಥೈಕೋತ್ತರವಾರವೃದ್ಧ್ಯಾ ಲಭೇತ್ಸ ಸೌವರ್ಣಮಯೀಂ ಸುವೃಷ್ಟಿಂ ॥೧೦೨॥

	ಯ ಏಕಭಕ್ತೋಽನ್ವಹಮೇಕವರ್ಷಂ ವಿಶುದ್ಧಧೀಃ ಸಪ್ತತಿವಾರಜಾಪೀ ।\\
	ಸ ಮಂದಭಾಗ್ಯೋಽಪಿ ರಮಾಕಟಾಕ್ಷಾತ್ ಭವೇತ್ಸಹಸ್ರಾಕ್ಷಶತಾಧಿಕಶ್ರೀಃ ॥೧೦೩॥
\newpage
ಶ್ರೀಶಾಂಘ್ರಿಭಕ್ತಿಂ ಹರಿದಾಸದಾಸ್ಯಂ ಪ್ರಸನ್ನಮಂತ್ರಾರ್ಥದೃಢೈಕನಿಷ್ಠಾಂ ।\\
ಗುರೋಃ ಸ್ಮೃತಿಂ ನಿರ್ಮಲಬೋಧಬುದ್ಧಿಂ ಪ್ರದೇಹಿ ಮಾತಃ ಪರಮಂ ಪದಂ ಶ್ರೀಃ ॥೧೦೪॥

ಪೃಥ್ವೀಪತಿತ್ವಂ ಪುರುಷೋತ್ತಮತ್ವಂ ವಿಭೂತಿವಾಸಂ ವಿವಿಧಾರ್ಥಸಿದ್ಧಿಂ ।\\
ಸಂಪೂರ್ಣಕೀರ್ತಿಂ ಬಹುವರ್ಷಭೋಗಂ ಪ್ರದೇಹಿ ಮೇ ದೇವಿ ಪುನಃಪುನಸ್ತ್ವಂ ॥೧೦೫॥

	ವಾದಾರ್ಥಸಿದ್ಧಿಂ ಬಹುಲೋಕವಶ್ಯಂ ವಯಃಸ್ಥಿರತ್ವಂ ಲಲನಾಸು ಭೋಗಂ ।\\
	ಪೌತ್ರಾದಿಲಬ್ಧಿಂ ಸಕಲಾರ್ಥಸಿದ್ಧಿಂ ಪ್ರದೇಹಿ ಮೇ ಭಾರ್ಗವಿ ಜನ್ಮಜನ್ಮನಿ ॥೧೦೬॥

ಸುವರ್ಣವೃದ್ಧಿಂ ಕುರು ಮೇ ಗೃಹೇ ಶ್ರೀಃ ಸುಧಾನ್ಯವೃದ್ಧಿಂ ಕುರು ಮೇ ಗೃಹೇ ಶ್ರೀಃ ।\\
ಕಲ್ಯಾಣವೃದ್ಧಿಂ ಕುರು ಮೇ ಗೃಹೇ ಶ್ರೀಃ ವಿಭೂತಿವೃದ್ಧಿಂ ಕುರು ಮೇ ಗೃಹೇ ಶ್ರೀಃ ॥೧೦೭॥

{\bfseries ಶಿರೋ ಬೀಜಂ । ಓಂ ಯಂ ಹಂ ಕಂ ಲಂ ಪಂ ಶ್ರೀಂ ॥}

	ಧ್ಯಾಯೇಲ್ಲಕ್ಷ್ಮೀಂ ಪ್ರಹಸಿತಮುಖೀಂ ಕೋಟಿಬಾಲಾರ್ಕಭಾಸಾಂ\\
	ವಿದ್ಯುದ್ವರ್ಣಾಂಬರವರಧರಾಂ ಭೂಷಣಾಢ್ಯಾಂ ಸುಶೋಭಾಂ ।\\
ಬೀಜಾಪೂರಂ ಸರಸಿಜಯುಗಂ ಬಿಭ್ರತೀಂ ಸ್ವರ್ಣಪಾತ್ರಂ\\
ಭರ್ತ್ರಾ ಯುಕ್ತಾಂ ಮುಹುರಭಯದಾಂ ಮಹ್ಯಮಪ್ಯಚ್ಯುತಶ್ರೀಃ ॥೧೦೮॥

	ಗುಹ್ಯಾತಿಗುಹ್ಯಗೋಪ್ತ್ರೀ ತ್ವಂ ಗೃಹಾಣಾಸ್ಮತ್ಕೃಪಂ ಜಪಂ ।\\
	ಸಿದ್ಧಿರ್ಭವತು ಮೇ ದೇವಿ ತ್ವತ್ಪ್ರಸಾದಾನ್ಮಯಿ ಸ್ಥಿರಾ ॥೧೦೯॥

	॥ ಇತ್ಯಾಥರ್ವಣರಹಸ್ಯೇ ಶ್ರೀಲಕ್ಷ್ಮೀಹೃದಯಸ್ತೋತ್ರಂ ಸಂಪೂರ್ಣಂ ॥\\
	(ಮೂಲೇನ ದಶವಾರಂ ತರ್ಪಯೇತ್ ।)
\begin{figure}[b] \centering \includegraphics[width=5cm]{page-divider} \end{figure}
\chapter*{\center ॥ಶ್ರೀನಾರಾಯಣಹೃದಯಂ॥}
ಅಸ್ಯ ಶ್ರೀನಾರಾಯಣ ಹೃದಯ ಸ್ತೋತ್ರಮಂತ್ರಸ್ಯ ಭಾರ್ಗವಋಷಿಃ~। ಅನುಷ್ಟುಪ್ಛಂದಃ~। ಶ್ರೀಲಕ್ಷ್ಮೀನಾರಾಯಣೋ ದೇವತಾ।
ಓಂ ಬೀಜಂ~। ನಮಃ ಶಕ್ತಿಃ~। ನಾರಾಯಣಾಯೇತಿ ಕೀಲಕಂ~। ಶ್ರೀಲಕ್ಷ್ಮೀನಾರಾಯಣಪ್ರೀತ್ಯರ್ಥೇ ಜಪೇ ವಿನಿಯೋಗಃ॥
\thispagestyle{empty}
\section{ಋಷ್ಯಾದಿನ್ಯಾಸಃ}
ಭಾರ್ಗವ ಋಷಯೇ ನಮಃ (ಶಿರಸಿ)। ಅನುಷ್ಟುಪ್ಛಂದಸೇ ನಮಃ (ಮುಖೇ)। ಶ್ರೀಲಕ್ಷ್ಮೀನಾರಾಯಣ ದೇವತಾಯೈ ನಮಃ (ಹೃದಯೇ)। ಓಂ ಬೀಜಾಯ ನಮಃ (ಗುಹ್ಯೇ)। ನಮಃ ಶಕ್ತಯೇ ನಮಃ (ಪಾದಯೋಃ)। ನಾರಾಯಣಾಯಕೀಲಕಾಯ ನಮಃ (ನಾಭೌ)।
\newpage
\section{ಕರನ್ಯಾಸಪಂಚಾಂಗನ್ಯಾಸೌ}
ಓಂ ನಾರಾಯಣಃ ಪರಂಜ್ಯೋತಿರಿತ್ಯಂಗುಷ್ಠಾಭ್ಯಾಂ ನಮಃ । ಹೃದಯಾಯ ನಮಃ ।\\
ಓಂ ನಾರಾಯಣಃ ಪರಂಬ್ರಹ್ಮೇತಿ ತರ್ಜನೀಭ್ಯಾಂ ನಮಃ । ಶಿರಸೇ ಸ್ವಾಹಾ ।\\
ಓಂ ನಾರಾಯಣಃ ಪರೋ ದೇವಃ ಇತಿಮಧ್ಯಮಾಭ್ಯಾಂ ನಮಃ । ಶಿಖಾಯೈ ವಷಟ್ ।\\
ಓಂ ನಾರಾಯಣಃ ಪರೋ ಧ್ಯಾತೇತ್ಯನಾಮಿಕಾಭ್ಯಾಂ ನಮಃ । ಕವಚಾಯ ಹುಂ ।\\
ಓಂ ನಾರಾಯಣಃ ಪರಂ ಧಾಮೇತಿ ಕನಿಷ್ಠಿಕಾಭ್ಯಾಂ ನಮಃ । ನೇತ್ರತ್ರಯಾಯ ವೌಷಟ್ ।\\
ಓಂ ನಾರಾಯಣಃ ಪರೋ ಧರ್ಮ ಇತಿ ಕರತಲಕರಪೃಷ್ಠಾಭ್ಯಾಂ ನಮಃ । ಅಸ್ತ್ರಾಯ ಫಟ್ ।\\
ಓಂ ನಮಃ ಸುದರ್ಶನಾಯ ಸಹಸ್ರಾರ ಹುಂ ಫಟ್ ಐಂದ್ರ್ಯಾದಿದಶದಿಶೋ ಬಧ್ನಾಮಿ ನಮಶ್ಚಕ್ರಾಯ ಸ್ವಾಹಾ ।

	{\bfseries ಉದ್ಯದ್ಭಾಸ್ವತ್ಸಮಾಭಾಸಶ್ಚಿದಾನಂದೈಕದೇಹವಾನ್ ।\\
	ಚಕ್ರಶಂಖಗದಾಪದ್ಮಧರೋ ಧ್ಯೇಯೋಽಹಮೀಶ್ವರಃ ॥೧॥

ಲಕ್ಷ್ಮೀಧರಾಭ್ಯಾಮಾಶ್ಲಿಷ್ಟಃ ಸ್ವಮೂರ್ತಿಗಣಮಧ್ಯಗಃ ।\\
ಬ್ರಹ್ಮವಾಯುಶಿವಾಹೀಶವಿಪೈಃ ಶಕ್ರಾದಿಕೈರಪಿ ॥೨॥

	ಸೇವ್ಯಮಾನೋಽಧಿಕಂ ಭಕ್ತ್ಯಾ ನಿತ್ಯನಿಶ್ಶೇಷಭಕ್ತಿಮಾನ್ ।\\
	ಮೂರ್ತಯೋಽಷ್ಟಾವಪಿ ಧ್ಯೇಯಾಶ್ಚಕ್ರಶಂಖವರಾಭಯೈಃ ॥೩॥

ಉದ್ಯದಾದಿತ್ಯಸಂಕಾಶಂ ಪೀತವಾಸಸಮಚ್ಯುತಂ ।\\
ಶಂಖಚಕ್ರಗದಾಪಾಣಿಂ ಧ್ಯಾಯೇಲ್ಲಕ್ಷ್ಮೀಪತಿಂ ಹರಿಂ ॥೪॥\\
 ಓಂ ಐಂ ಹ್ರೀಂ ಶ್ರೀಂ ಶ್ರೀಲಕ್ಷ್ಮೀನಾರಾಯಣಾಯ ಸ್ವಾಹಾ ॥(೧೦೮)}\\
ಅಥ ಮೂಲಾಷ್ಟಕಂ\\
	ನಾರಾಯಣಃ ಪರಂ ಜ್ಯೋತಿರಾತ್ಮಾ ನಾರಾಯಣಃ ಪರಃ ।\\
	ನಾರಾಯಣಃ ಪರಂ ಬ್ರಹ್ಮ ನಾರಾಯಣ ನಮೋಽಸ್ತು ತೇ ॥೧॥

ನಾರಾಯಣಃ ಪರೋ ದೇವೋ ಧಾತಾ ನಾರಾಯಣಃ ಪರಃ ।\\
ನಾರಾಯಣಃ ಪರೋ ಧ್ಯಾತಾ ನಾರಾಯಣ ನಮೋಽಸ್ತು ತೇ ॥೨॥

	ನಾರಾಯಣಃ ಪರಂ ಧಾಮ ಧ್ಯಾನಂ ನಾರಾಯಣಃ ಪರಃ ।\\
	ನಾರಾಯಣಃ ಪರೋ ಧರ್ಮೋ ನಾರಾಯಣ ನಮೋಽಸ್ತು ತೇ ॥೩॥

ನಾರಾಯಣಃ ಪರೋ ದೇವೋ ವಿದ್ಯಾ ನಾರಾಯಣಃ ಪರಃ ।\\
ವಿಶ್ವಂ ನಾರಾಯಣಃ ಸಾಕ್ಷಾನ್ನಾರಾಯಣ ನಮೋಽಸ್ತು ತೇ ॥೪॥

	ನಾರಾಯಣಾದ್ವಿಧಿರ್ಜಾತೋ ಜಾತೋ ನಾರಾಯಣಾದ್ಧರಃ ।\\
	ಜಾತೋ ನಾರಾಯಣಾದಿಂದ್ರೋ ನಾರಾಯಣ ನಮೋಽಸ್ತು ತೇ ॥೫॥

ರವಿರ್ನಾರಾಯಣಸ್ತೇಜಃ ಚಂದ್ರೋ ನಾರಾಯಣಃ ಪರಃ ।\\
ವಹ್ನಿರ್ನಾರಾಯಣಃ ಸಾಕ್ಷಾನ್ನಾರಾಯಣ ನಮೋಽಸ್ತು ತೇ ॥೬॥

	ನಾರಾಯಣ ಉಪಾಸ್ಯಃ ಸ್ಯಾದ್ಗುರುರ್ನಾರಾಯಣಃ ಪರಃ ।\\
	ನಾರಾಯಣಃ ಪರೋ ಬೋಧೋ ನಾರಾಯಣ ನಮೋಽಸ್ತು ತೇ ॥೭॥

ನಾರಾಯಣಃ ಫಲಂ ಮುಖ್ಯಂ ಸಿದ್ಧಿರ್ನಾರಾಯಣಃ ಸುಖಂ ।\\
ಸೇವ್ಯೋ ನಾರಾಯಣಃ ಶುದ್ಧೋ ನಾರಾಯಣ ನಮೋಽಸ್ತು ತೇ ॥೮॥(ಇತಿ ಮೂಲಾಷ್ಟಕಂ )\\
(ಅಷ್ಟವಾರಂ ಮೂಲಂ ಜಪ್ತ್ವಾ ಸಕೃತ್ ತರ್ಪಯೇತ್ ।)
\section{ಅಥ ಪ್ರಾರ್ಥನಾದಶಕಂ}
	ನಾರಾಯಣಸ್ತ್ವಮೇವಾಸಿ ದಹರಾಖ್ಯೇ ಹೃದಿ ಸ್ಥಿತಃ ।\\
	ಪ್ರೇರಕಃ ಪ್ರೇರ್ಯಮಾಣಾನಾಂ ತ್ವಯಾ ಪ್ರೇರಿತ ಮಾನಸಃ ॥೧॥

ತ್ವದಜ್ಞಾಂ ಶಿರಸಾ ಧೃತ್ವಾ ಜಪಾಮಿ ಜನಪಾವನಂ ।\\
ನಾನೋಪಾಸನಮಾರ್ಗಾಣಾಂ ಭಾವಹೃದ್ಭಾವಬೋಧಕಃ ॥೨॥
\newpage
	ಭಾವಾರ್ಥಕೃದ್ ಭಾವಭೂತೋ ಭವಸೌಖ್ಯಪ್ರದೋ ಭವ ।\\
	ತ್ವನ್ಮಾಯಾಮೋಹಿತಂ ವಿಶ್ವಂ ತ್ವಯೈವ ಪರಿಕಲ್ಪಿತಂ ॥೩॥

ತ್ವದಧಿಷ್ಠಾನಮಾತ್ರೇಣ ಸೈವ ಸರ್ವಾರ್ಥಕಾರಿಣೀ ।\\
ತ್ವಮೇವ ತಾಂ ಪುರಸ್ಕೃತ್ಯ ಮಮ ಕಾಮಾನ್ ಸಮರ್ಥಯ ॥೪॥

	ನ ಮೇ ತ್ವದನ್ಯಸ್ತ್ರಾತಾಸ್ತಿ ತ್ವದನ್ಯನ್ನ ಹಿ ದೈವತಂ ।\\
	ತ್ವದನ್ಯಂ ನ ಹಿ ಜಾನಾಮಿ ಪಾಲಕಂ ಪುಣ್ಯರೂಪಕಂ ॥೫॥

ಯಾವತ್ಸಾಂಸಾರಿಕೋ ಭಾವೋ ಮನಃಸ್ಥೋ ಭಾವನಾತ್ಮಕಃ ।\\
ತಾವತ್ಸಿದ್ಧಿರ್ಭವೇತ್ ಸದ್ಯಃ ಸರ್ವಥಾ ಸರ್ವದಾ ವಿಭೋ ॥೬॥

	ಪಾಪಿನಾಮಹಮೇವಾಗ್ರ್ಯೋ ದಯಾಲೂನಾಂ ತ್ವಮಗ್ರಣೀಃ ।\\
	ದಯನೀಯೋ ಮದನ್ಯೋಽಸ್ತಿ ತವ ಕೋಽತ್ರ ಜಗತ್ತ್ರಯೇ ॥೭॥

ತ್ವಯಾಪ್ಯಹಂ ನ ಸೃಷ್ಟಶ್ಚೇತ್ ನ ಸ್ಯಾತ್ತವ ದಯಾಲುತಾ ।\\
ಆಮಯೋ ನೈವ ಸೃಷ್ಟಶ್ಚೇದೌಷಧಸ್ಯ ವೃಥೋದಯಃ ॥೮॥
\newpage
	ಪಾಪಸಂಘಪರಿಕ್ರಾಂತಃ ಪಾಪಾತ್ಮಾ ಪಾಪರೂಪಧೃಕ್ ।\\
	ತ್ವದನ್ಯಃ ಕೋಽತ್ರ ಪಾಪೇಭ್ಯಸ್ತ್ರಾತಾ ಮೇ ಜಗತೀತಲೇ ॥೯॥

ತ್ವಮೇವ ಮಾತಾ ಚ ಪಿತಾ ತ್ವಮೇವ ತ್ವಮೇವ ಬಂಧುಶ್ಚ ಸಖಾ ತ್ವಮೇವ ।\\
ತ್ವಮೇವ ವಿದ್ಯಾ ದ್ರವಿಣಂ ತ್ವಮೇವ ತ್ವಮೇವ ಸರ್ವಂ ಮಮ ದೇವ ದೇವ ॥೧೦॥

	ಪ್ರಾರ್ಥನಾದಶಕಂ ಚೈವ ಮೂಲಷ್ಟಕಮಿತಿದ್ವಯಂ ।\\
	ಯಃ ಪಠೇಚ್ಛೃಣುಯಾನ್ನಿತ್ಯಂ ತಸ್ಯ ಲಕ್ಷ್ಮೀಃ ಸ್ಥಿರಾ ಭವೇತ್ ॥೧೧॥

ನಾರಾಯಣಸ್ಯ ಹೃದಯಂ ಸರ್ವಾಭೀಷ್ಟಫಲಪ್ರದಂ ।\\
ಲಕ್ಷ್ಮೀಹೃದಯಕಂ ಸ್ತೋತ್ರಂ ಯದಿ ಚೈತದ್ವಿನಾಕೃತಂ ॥೧೨॥

	ತತ್ಸರ್ವಂ ನಿಷ್ಫಲಂ ಪ್ರೋಕ್ತಂ ಲಕ್ಷ್ಮೀಃ ಕ್ರುಧ್ಯತಿ ಸರ್ವದಾ ।\\
	ಏತತ್ಸಂಪುಟಿತಂ ಸ್ತೋತ್ರಂ ಸರ್ವಕರ್ಮಫಲಪ್ರದಂ ॥೧೩॥

ಲಕ್ಷ್ಮೀಹೃದಯಕಂ ಚೈವ ತಥಾ ನಾರಾಯಣಾತ್ಮಕಂ ।\\
ಜಪೇದ್ಯಃ ಸಂಕಲೀಕೃತ್ಯ ಸರ್ವಾಭೀಷ್ಟಮವಾಪ್ನುಯಾತ್ ॥೧೪॥
\newpage
	ನಾರಾಯಣಸ್ಯ ಹೃದಯಂ ಆದೌ ಜಪ್ತ್ವಾ ತತಃಪರಂ ।\\
	ಲಕ್ಷ್ಮೀಹೃದಯಕಂ ಸ್ತೋತ್ರಂ ಜಪೇನ್ನಾರಾಯಣಂ ಪುನಃ ॥೧೫॥

ಪುನರ್ನಾರಾಯಣಂ ಜಪ್ತ್ವಾ ಪುನರ್ಲಕ್ಷ್ಮೀಸ್ತವಂ ಜಪೇತ್ ।\\
ಪುನರ್ನಾರಾಯಣಂ ಜಾಪ್ಯಂ ಸಂಕಲೀಕರಣಂ ಭವೇತ್ ॥೧೬॥

	ಏವಂ ಮಧ್ಯೇ ದ್ವಿವಾರೇಣ ಜಪೇತ್ ಸಂಕಲಿತಂ ತು ತತ್ ।\\
	ಲಕ್ಷ್ಮೀಹೃದಯಕಂ ಸ್ತೋತ್ರಂ ಸರ್ವಕಾಮಪ್ರಕಾಶಿತಂ ॥೧೭॥

ತದ್ವಜ್ಜಪಾದಿಕಂ ಕುರ್ಯಾದೇತತ್ಸಂಕಲಿತಂ ಶುಭಂ ।\\
ಸರ್ವಾನ್ಕಾಮಾನವಾಪ್ನೋತಿ ಆಧಿವ್ಯಾಧಿಭಯಂ ಹರೇತ್ ॥೧೮॥

	ಗೋಪ್ಯಮೇತತ್ ಸದಾ ಕುರ್ಯಾತ್ ನ ಸರ್ವತ್ರ ಪ್ರಕಾಶಯೇತ್ ।\\
	ಇತಿ ಗುಹ್ಯತಮಂ ಶಾಸ್ತ್ರಂ ಪ್ರೋಕ್ತಂ ಬ್ರಹ್ಮಾದಿಕೈಃ ಪುರಾ ॥೧೯॥

ತಸ್ಮಾತ್ಸರ್ವಪ್ರಯತ್ನೇನ ಗೋಪಯೇತ್ಸಾಧಯೇದ್ ಸುಧೀಃ ।\\
ಯತ್ರೈತತ್ಪುಸ್ತಕಂ ತಿಷ್ಠೇಲ್ಲಕ್ಷ್ಮೀನಾರಾಯಣಾತ್ಮಕಂ ॥೨೦॥
\newpage
	ಭೂತಪೈಶಾಚವೇತಾಳಭಯಂ ತತ್ರ ನ ಜಾಯತೇ ।\\
	ಲಕ್ಷ್ಮೀಹೃದಯಕಂ ಪ್ರೋಕ್ತಂ ವಿಧಿನಾ ಸಾಧಯೇತ್ ಸುಧೀಃ ॥೨೧॥

ಭೃಗುವಾರೇ ತಥಾ ರಾತ್ರೌ ಪೂಜಯೇತ್ ಪುಸ್ತಕದ್ವಯಂ ।\\
ಸರ್ವಥಾ ಸರ್ವದಾ ಸತ್ಯಂ ಗೋಪಯೇತ್ ಸಾಧಯೇತ್ ಸುಧೀಃ ।\\
	ಗೋಪನಾತ್ ಸಾಧನಾಲ್ಲೋಕೇ ಸರ್ವಾಂ ಸಿದ್ಧಿಂ ಲಭೇನ್ನರಃ ॥೨೨॥\\
		ಇತಿ ನಾರಾಯಣಹೃದಯಸ್ತೋತ್ರಂ ಸಂಪೂರ್ಣಂ ॥\\
		(ಮೂಲೇನ ದಶವಾರಂ ತರ್ಪಯೇತ್ ।)\\
		ಉತ್ತರನ್ಯಾಸಃ
\begin{figure}[b] \centering \includegraphics[width=5cm]{page-divider} \end{figure}
ಗಣಪತಿ
ಗಣಾನಾಂ ತ್ವಾ ಗಣಪತಿಂ ಹವಾಮಹೇ ಕವಿಂ ಕವೀನಾಮುಪಮಶ್ರವಸ್ತಮಂ ।
ಜ್ಯೇಷ್ಠರಾಜಂ ಬ್ರಹ್ಮಣಾಂ ಬ್ರಹ್ಮಣಸ್ಪತ ಆ ನಃ ಶೃಣ್ವನ್ನೂತಿಭಿಃ ಸೀದ ಸಾದನಂ ॥
ಗಂ ಗಣಪತಯೇ ನಮಃ ।
ಏಕದಂತಾಯ ವಿದ್ಮಹೇ ವಕ್ರತುಂಡಾಯ ಧೀಮಹಿ । ತನ್ನೋ ದಂತೀ ಪ್ರಚೋದಯಾತ್ ॥
ವಕ್ರತುಂಡ ಮಹಾಕಾಯ ಸೂರ್ಯಕೋಟಿ ಸಮಪ್ರಭ ।
ನಿರ್ವಿಘ್ನಂ ಕುರು ಮೇ ದೇವ ಸರ್ವಕಾರ್ಯೇಷು ಸರ್ವದಾ ॥

ಉಮಾಮಹೇಶ್ವರ
ಶಾಂತಂ ಪದ್ಮಾಸನಸ್ಥಂ ಶಶಧರಮುಕುಟಂ ಪಂಚವಕ್ತ್ರಂ ತ್ರಿನೇತ್ರಂ
ಶೂಲಂ ವಜ್ರಂ ಚ ಖಡ್ಗಂ ಪರಶುಮಭಯದಂ ದಕ್ಷಭಾಗೇ ವಹಂತಂ~।
ನಾಗಂ ಪಾಶಂ ಚ ಘಂಟಾಂ ಪ್ರಲಯಹುತವಹಂ ಚಾಂಕುಶಂ ವಾಮಭಾಗೇ
ನಾನಾಲಂಕಾರಯುಕ್ತಂ ಸ್ಫಟಿಕಮಣಿನಿಭಂ ಪಾರ್ವತೀಶಂ ನಮಾಮಿ ॥
ಓಂ ಹ್ರೀಂ ನಮಃ ಶಿವಾಯ ।
ಓಂ ತತ್ಪುರುಷಾಯ ವಿದ್ಮಹೇ ಮಹಾದೇವಾಯ ಧೀಮಹಿ । ತನ್ನೋ ರುದ್ರ: ಪ್ರಚೋದಯಾತ್ ॥
ಅಪರಾಧಸಹಸ್ತ್ರಾಣಿ ಕ್ರಿಯಂತೇಽಹರ್ನಿಶಂ ಮಯಾ । ದಾಸೋಽಯಮಿತಿ ಮಾಂ ಮತ್ವಾ ಕ್ಷಮಸ್ವ ಪರಮೇಶ್ವರ ॥

ದುರ್ಗಾ
ಜಾತವೇದಸೇ ಸುನವಾಮ ಸೋಮಮರಾತೀಯತೋ ನಿದಹಾತಿ ವೇದಃ । ಸ ನಃ ಪರ್ಷದತಿ ದುರ್ಗಾಣಿ ವಿಶ್ವಾ ನಾವೇವ ಸಿಂಧುಂ ದುರಿತಾತ್ಯಗ್ನಿಃ ॥
ಓಂ ಹ್ರೀಂ ಶ್ರೀಂ ದುಂ ದುರ್ಗಾಯೈ ನಮಃ ।
ಓಂ ಕಾತ್ಯಾಯನಾಯ ವಿದ್ಮಹೇ ಕನ್ಯಾಕುಮಾರಿ ಧೀಮಹಿ । ತನ್ನೋ ದುರ್ಗಿಃ ಪ್ರಚೋದಯಾತ್ ॥
ಸರ್ವಮಂಗಲಮಾಂಗಲ್ಯೇ ಶಿವೇ ಸರ್ವಾರ್ಥಸಾಧಿಕೇ । ಶರಣ್ಯೇ ತ್ರ್ಯಂಬಕೇ ಗೌರಿ ನಾರಾಯಣಿ ನಮೋಽಸ್ತುತೇ ॥

ಕ್ಷೇತ್ರಪಾಲ
ಕ್ಷೇತ್ರಸ್ಯ ಪತಿನಾ ವಯಂ ಹಿತೇನೇವ ಜಯಾಮಸಿ । ಗಾಮಶ್ವಂ ಪೋಷಯಿತ್ನ್ವಾ ಸ ನೋ ಮೃಳಾತೀದೃಶೇ ॥
ಓಂ ಕ್ಷಾಂ ಕ್ಷಂ ಕ್ಷೇತ್ರಪಾಲಾಯ ನಮಃ ।
ಕ್ಷೇತ್ರಪಾಲಾಯ ವಿದ್ಮಹೇ ತತ್ಪುರುಷಾಯ ಧೀಮಹಿ । ತನ್ನಃ ಕ್ಷೇತ್ರಪಾಲಃ ಪ್ರಚೋದಯಾತ್ ॥
ವಾಸ್ತೋಷ್ಪತಿ
ವಾಸ್ತೋಷ್ಪತೇ ಪ್ರತಿ ಜಾನೀಹ್ಯಸ್ಮಾಂತ್ಸ್ವಾವೇಶೋ ಅನಮೀವೋ ಭವಾನಃ । ಯತ್ತ್ವೇಮಹೇ ಪ್ರತಿ ತನ್ನೋ ಜುಷಸ್ವ ಶಂ ನೋ ಭವ ದ್ವಿಪದೇ ಶಂ ಚತುಷ್ಪದೇ ॥
ವಾಸ್ತುಪುರುಷಾಯ ವಿದ್ಮಹೇ ಕ್ಷೇಮಂಕರಾಯ ಧೀಮಹಿ । ತನ್ನೋ ವಾಸ್ತು ಪ್ರಚೋದಯಾತ್ ॥

(ಅಷ್ಟದಿಕ್ಪಾಲಕಾಃ)
ಇಂದ್ರ
ಇಂದ್ರಂ ವೋ ವಿಶ್ವತಸ್ಪರಿ ಹವಾಮಹೇ ಜನೇಭ್ಯಃ। ಅಸ್ಮಾಕಮಸ್ತು ಕೇವಲಃ॥
 ಸರ್ವಲೋಕಾಧಿಪಂ ಶ್ರೇಷ್ಠಂ ದೇವರ್ಷೀಣಾಂ ಚ ಪಾಲಕಂ । ಪೂರ್ವದಿಕ್ಪತಿಮಿಂದ್ರಂ ವೈ ಸ್ಥಾಪಯಾಮಿ ಸುರೇಶ್ವರಂ ॥
ಓಂ ಸಹಸ್ರನೇತ್ರಾಯ ವಿದ್ಮಹೇ ವಜ್ರಹಸ್ತಾಯ ಧೀಮಹಿ ತನ್ನ ಇಂದ್ರಃ ಪ್ರಚೋದಯಾತ್॥

ಅಗ್ನಿ
ತ್ರಿಪಾದಂ ಮೇಷವಾಹಂ ಚ ಸಪ್ತಜಿಹ್ವಂ ದ್ವಿಶೀರ್ಷಕಂ । ಚತುಃಶ್ರೃಂಗಂ ಪ್ರಸನ್ನಾಸ್ಯಂ ಸ್ಥಾಪಯಾಮ್ಯಗ್ನಿಮುತ್ತಮಂ ।
ಓಂ ಸಪ್ತಜಿಹ್ವಾಯ ವಿದ್ಮಹೇ ಅಗ್ನಿದೇವಾಯ ಧೀಮಹಿ ತನ್ನ ಅಗ್ನಿಃ ಪ್ರಚೋದಯಾತ್॥

ಯಮ
ಅಂತಕಂ ಸರ್ವಲೋಕಾನಾಂ ಧರ್ಮರಾಜ ಇತಿ ಶ್ರುತಃ । ಅತಸ್ತ್ವಾಂ ಸ್ಥಾಪಯಾಮೀಹ ದಕ್ಷಿಣಸ್ಯಾಂ ಸ್ಥಿರೋ ಭವ ॥
ಓಂ ವೈವಸ್ವತಾಯ ವಿದ್ಮಹೇ ದಂಡಹಸ್ತಾಯ ಧೀಮಹಿ । ತನ್ನೋ ಯಮಃ ಪ್ರಚೋದಯಾತ್ ॥

ನಿರ್ಋತಿ
ನೈರ್ಋತ್ಯಾಂ ವಸತಿರ್ಯಸ್ಯ ಕೌಣಪಾನಾಂ ಪತಿಶ್ಚ ಯಃ । ಮಂಡಲೇ ಸ್ಥಾಪಯಾಮ್ಯತ್ರ ನೈರ್ಋತ್ಯಾಂ ತಂ ಹಿ ನಿರ್ಋತಿಂ ॥
ನಿಶಾಚರಾಯ ವಿದ್ಮಹೇ ಖಡ್ಗಹಸ್ತಾಯ ಧೀಮಹಿ । ತನ್ನೋ ನಿರ್ಋತಿಃ ಪ್ರಚೋದಯಾತ್ ॥

ವರುಣ
ಅಪಾಂಪತಿಂ
ಪಾಶಧರಂ ಯಾದಸಾಂ ಪ್ರಭುಮೀಶ್ವರಂ । ವರುಣಂ ಸ್ಥಾಪಯಾಮ್ಯತ್ರ ವಾರುಣ್ಯಾಂ ಮಂಡಲೇ ಶುಭೇ ॥
ಶುದ್ಧಹಸ್ತಾಯ ವಿದ್ಮಹೇ ಪಾಶಹಸ್ತಾಯ ಧೀಮಹಿ । ತನ್ನೋ ವರುಣಃ ಪ್ರಚೋದಯಾತ್ ॥

ವಾಯು
ಆಶುಗಂ ಸ್ಪರ್ಶಬೋಧಂ ಚ ಮೃಗವಾಹಂ ಸಮೀರಣಂ । ಮಂಡಲೇ ಸ್ಥಾಪಯಾಮೀಹ ವಾಯವ್ಯಾಂ ವಾಯುಮುತ್ತಮಂ ॥
ಸರ್ವಪ್ರಾಣಾಯ ವಿದ್ಮಹೇ ಯಷ್ಟಿಹಸ್ತಾಯ ಧೀಮಹಿ । ತನ್ನೋ ವಾಯುಃ ಪ್ರಚೋದಯಾತ್ ॥

ಸೋಮ(ಕುಬೇರ)
ಕ್ಷೀರೋದಾರ್ಣವಸಂಭೂತಂ ಲಕ್ಷ್ಮೀಬಂಧುಂ ನಿಶಾಕರಂ । ಮಂಡಲೇ ಸ್ಥಾಪಯಾಮ್ಯತ್ರ ಸೋಮಂ ಸರ್ವಾಥಸಿದ್ಧಯೇ ॥
ಯಕ್ಷೇಶ್ವರಾಯ ವಿದ್ಮಹೇ ಗದಾಹಸ್ತಾಯ ಧೀಮಹಿ । ತನ್ನೋ ಯಕ್ಷಃ ಪ್ರಚೋದಯಾತ್ ॥
ಓಂ ಕ್ಷೀರಪುತ್ರಾಯ ವಿದ್ಮಹೇ ಅಮೃತತತ್ವಾಯ ಧೀಮಹಿ । ತನ್ನಶ್ಚಂದ್ರ: ಪ್ರಚೋದಯಾತ್ ॥
ಈಶಾನ
ಈಶಾನೀಪಾಲಕಂ ಶ್ರೇಷ್ಠಂ ಸರ್ಶಲೋಕಭಯಂಕರಂ । ಮಂಡಲೇ ಸ್ಥಾಪಯಾಮೀಹ ಈಶಾನಂ ಸರ್ವಸಿದ್ಧಯೇ ॥
ಸರ್ವೇಶ್ವರಾಯ ವಿದ್ಮಹೇ ಶೂಲಹಸ್ತಾಯ ಧೀಮಹಿ । ತನ್ನೋ ರುದ್ರಃ ಪ್ರಚೋದಯಾತ್ ॥

(ನವಗ್ರಹ)
ಸೂರ್ಯ 
ಓಂ ಆ ಕೃಷ್ಣೇನ ರಜಸಾ ವರ್ತಮಾನೋ ನಿವೇಶಯನ್ನಮೃತಂ ಮರ್ತ್ಯಂ ಚ। ಹಿರಣ್ಯಯೇನ ಸವಿತಾ ರಥೇನಾ ದೇವೋ ಯಾತಿ ಭುವನಾನಿ ಪಶ್ಯನ್ ॥
ಜಪಾಕುಸುಮಸಂಕಾಶಂ ಕಾಶ್ಯಪೇಯಂ ಮಹಾದ್ಯುತಿಂ । ತಮೋಽರಿಂ ಸರ್ವಪಾಪಘ್ನಂ ಸೂರ್ಯಮಾವಾಹ್ಯಾಮ್ಯಹಂ ॥
ಆದಿತ್ಯಾಯ ವಿದಮಹೇ ದಿವಾಕರಾಯ ಧೀಮಹಿ । ತನ್ನೋ ಸೂರ್ಯ: ಪ್ರಚೋದಯಾತ್ ॥
ಚಂದ್ರ
ಆಪ್ಯಾಯಸ್ವ ಸಮೇತು ತೇ, ವಿಶ್ವತಃ ಸೋಮ ವೃಷ್ಣ್ಯಂ। ಭವಾ ವಾಜಸ್ಯ ಸಂಗಥೇ ॥
ದಧಿಶಂಖತುಷಾರಾಭಂ ಕ್ಷೀರೋದಾರ್ಣವಸಂಭವಂ । ಜ್ಯೋತ್ಸ್ನಾಪತಿಂ ನಿಶಾನಾರ್ಥ ಸೋಮಮಾವಾಹಯಾಮ್ಯಹಂ ।
ಓಂ ಕ್ಷೀರಪುತ್ರಾಯ ವಿದ್ಮಹೇ ಅಮೃತತತ್ವಾಯ ಧೀಮಹಿ । ತನ್ನೋ ಚಂದ್ರ: ಪ್ರಚೋದಯಾತ್ ॥
ಮಂಗಳ
ಅಗ್ನಿರ್ಮೂರ್ಧಾ ದಿವಃ ಕಕುತ್ಪತಿಃ ಪೃಥಿವ್ಯಾ ಅಯಂ । ಅಪಾಂ ರೇತಾಂಸಿ ಜಿನ್ವತಿ ॥
ಧರಣೀಗರ್ಭಸಂಭೂತ ವಿದ್ಯುತ್ತೇಜಸಮಪ್ರಭಂ । ಕುಮಾರಂ ಶಕ್ತಿಹಸ್ತ ಚ ಭೌಮಮಾವಾಹಯಾಮ್ಯಹಂ ॥
ಕ್ಷಿತಿ ಪುತ್ರಾಯ ವಿದ್ಮಹೇ ಲೋಹಿತಾಂಗಾಯ ಧೀಮಹಿ । ತನ್ನೋ ಭೌಮಃ ಪ್ರಚೋದಯಾತ್ ॥
ಬುಧ
ಉದ್ಬುಧ್ಯಧ್ವಂ ಸಮನಸ: ಸಖಾಯ: ಸಮಗ್ನಿಮಿಂಧವಂ ಬಹವಃ ಸನೀಲಾ:। ದಧಿಕ್ರಾಮಗ್ನಿಮುಷಸಂ ಚ ದವೀಮಿಂದ್ರಾವತೋಽವಸೇ ನಿ ಹ್ವಯೇ ವ:॥
ಪ್ರಿಯಂಗಕಲಿಕಾಭಾಸಂ ರುಪೇಣಾಪ್ರತಿಮಂ ಬುಧಂ । ಸೌಮ್ಯಂ ಸೌಮ್ಯಗುಣೋಪೇತಂ ಬುಧಮಾವಾಹಯಾಮ್ಯಹಂ ॥
ಚಂದ್ರ ಪುತ್ರಾಯ ವಿದ್ಮಹೇ ರೋಹಿಣೀಪ್ರಿಯಾಯ ಧೀಮಹಿ । ತನ್ನೋ ಬುಧಃ ಪ್ರಚೋದಯಾತ್ ॥
ಗುರು
ಬೃಹಸ್ಪತೇ ಅತಿ ಯದರ್ಯೋ ಅರ್ಹಾದ್ ದ್ಯುಮದ್ವಿಭಾತಿ ಕ್ರತುಮಜ್ಜನೇಷು। ಯದ್ದೀದಯಚ್ಛವಸ ಋತಪ್ರಜಾತ ತದಸ್ಮಾಸು ದ್ರವಿಣಂ ಧೇಹಿ ಚಿತ್ರಂ॥
ದೇವಾನಾಂ ಚ ಮುನೀನಾಂ ಚ ಗುರುಂ ಕಾಂಚನಸನ್ನಿಭಂ । ವಂದ್ಯಭೂತಂ ತ್ರಿಲೋಕಾನಾಂ ಗುರೂಮಾವಾಹಯಾಮ್ಯಹಂ ।
ಗುರುದೇವಾಯ ವಿದ್ಮಹೇ ಪರಬ್ರಹ್ಮಾಯ ಧೀಮಹಿ । ತನ್ನೋ ಗುರು: ಪ್ರಚೋದಯಾತ್ ॥
ಶುಕ್ರ
ಶುಕ್ರಂ ತೇ ಅನ್ಯದ್ಯಜತಂ ತೇ ಅನ್ಯದ್ವಿಷುರೂಪೇ ಅಹನೀ ದ್ಯೌರಿವಾಸಿ । ವಿಶ್ವಾ ಹಿ ಮಾಯಾ ಅವಸಿ ಸ್ವಧಾವೋ ಭದ್ರಾ ತೇ ಪೂಷನ್ನಿಹ ರಾತಿರಸ್ತು ॥
ಹಿಮಕುಂದಮೃಣಾಲಾಭಂ ದೈತ್ಯಾನಾಂ ಪರಮಂ ಗುರುಂ । ಸರ್ವಶಾಸ್ತ್ರಪ್ರವಕ್ತಾರಂ ಶುಕ್ರಮಾವಾಹಯಾಮ್ಯಹಂ ॥
ಭೃಗುಪುತ್ರಾಯ ವಿದ್ಮಹೇ ಶ್ವೇತವಾಹನಾಯ ಧೀಮಹಿ । ತನ್ನಃ ಶುಕ್ರಃ ಪ್ರಚೋದಯಾತ್ ॥
ಶನಿ
ಶಮಗ್ನಿರಗ್ನಿಭಿಃ ಕರಚ್ಛಂ ನಸ್ತಪತು ಸೂರ್ಯಃ । ಶಂ ವಾತೋ ವಾತ್ವರಪಾ ಅಪ ಸ್ರಿಧಃ ॥
ನೀಲಾಂಬುಜಸಮಾಭಾಸಂ ರವಿಪುತ್ರಂ ಯಮಾಗ್ರಜಂ । ಛಾಯಾಮಾರ್ತಂಡಸಂಭೂತಂ ಶನಿಮಾವಾಹಯಾಮ್ಯಹಂ ॥
ಕೃಷ್ಣಾಂಗಾಯ ವಿದ್ಮಹೇ ರವಿಪುತ್ರಾಯ ಧೀಮಹಿ । ತನ್ನಃ ಸೌರಿಃ ಪ್ರಚೋದಯಾತ್ ॥
ರಾಹು
ಕಯಾನಶ್ಚಿತ್ರ ಆಭುವದೂತೀಸದಾ ವೃಧ: ಸಖಾ । ಕಯಾಶಶ್ಚಿಷ್ಠಯಾ ವೃತಾ॥
ಅರ್ದ್ಧಕಾಯಂ ಮಹಾವೀರ್ಯಂ ಚಂದ್ರಾದಿತ್ಯವಿಮರ್ದನಂ । ಸಿಂಹಿಕಾಗರ್ಭಸಂಭೂತಂ ರಾಹುಮಾವಾಹಯಾಮ್ಯಹಂ ॥
ನೀಲವರ್ಣಾಯ ವಿದ್ಮಹೇ ಸೈಂಹಿಕೇಯಾಯ ಧೀಮಹಿ । ತನ್ನೋ ರಾಹುಃ ಪ್ರಚೋದಯಾತ್ ॥
ಕೇತು
ಕೇತುಂ ಕೃಣ್ವನ್ನ ಕೇತವೇ ಪೇಶೋ ಮರ್ಯಾ ಅಪೇಶಸೇ । ಸಮುಷದ್ಭಿರಜಾ ಯಥಾಃ ॥
ಪಲಾಶಧೂಮ್ರಸಂಕಾಶಂ ತಾರಕಾಗ್ರಹಮಸ್ತಕಂ । ರೌದ್ರಂ ರೌದ್ರಾತ್ಮಕಂ ಘೋರಂ ಕೇತುಮಾವಾಹಯಾಮ್ಯಹಂ ॥
ಧೂಮ್ರವರ್ಣಾಯ ವಿದ್ಮಹೇ ಕಪೋತವಾಹನಾಯ ಧೀಮಹಿ । ತನ್ನಃ ಕೇತುಃ ಪ್ರಚೋದಯಾತ್ ॥

ಪ್ರಾರ್ಥನಾ
ಬ್ರಹ್ಮಾ ಮುರಾರಿಸ್ತ್ರಿಪುರಾಂತಕಾರೀ ।
ಭಾನುಃ ಶಶೀ ಭೂಮಿಸುತೋ ಬುಧಶ್ಚ ॥
ಗುರುಶ್ಚ ಶುಕ್ರಃ ಶನಿ ರಾಹು ಕೇತವಃ ।
ಸರ್ವಗ್ರಹಾಃ ಶಾಂತಿಕರಾ ಭವಂತು ॥



ರತಿಮನ್ಮಥ
ಅಸ್ಯ ಶ್ರೀ ಕಾಮಗಾಯತ್ರೀ ಮಹಾಮಂತ್ರಸ್ಯ ಬ್ರಹ್ಮಾ ಋಷಿಃ । ಗಾಯತ್ರೀ ಚ್ಛಂದಃ । ಶ್ರೀಂ ಬೀಜಂ । ಹ್ರೀಂ ಶಕ್ತಿಃ । ಕ್ಲೀಂ ಕೀಲಕಂ ॥ ಸರ್ವಜನಮೋಹನಾರ್ಥೇ ಸತಿಪತಿ ಸಾಮರಸ್ಯವೃದ್ಧ್ಯರ್ಥೇ ಶೀಘ್ರವಿವಾಹಪ್ರಾಪ್ತ್ಯರ್ಥೇ  ಜಪೇ ವಿನಿಯೋಗಃ ।
ಕ್ಲಾಂ ಇತ್ಯಾದಿನಾ ನ್ಯಾಸಃ ॥
ಅಶೋಕಾಯ ನಮಸ್ತುಭ್ಯಂ ಕಾಮ ಸ್ತ್ರೀಶೋಕನಾಶನ । ಶೋಕಾರ್ತಿಂ ಹರ ಮೇ ನಿತ್ಯಮಾನಂದಂ ಜನಯಸ್ವ ಮೇ ॥
ದೇವದೇವ ಜಗನ್ನಾಥ ವಾಂಛಿತಾರ್ಥಪ್ರದಾಯಕ । ಕೃತ್ಸ್ನಾನ್ಪೂರಯ ಮೇ ತ್ವರ್ಥಾನ್ ಕಾಮಾನ್ ಕಾಮೇಶ್ವರೀಪ್ರಿಯ ॥
ಜಪಾರುಣಂ ರಕ್ತವಿಭೂಷಣಾಢ್ಯಂ ಮೀನಧ್ವಜಂ ಚಾರುಕೃತಾಂಗರಾಗಂ । ಕರಾಂಬುಜೈರಂಕುಶಮಿಕ್ಷುಚಾಪಪುಷ್ಪಾಸ್ತ್ರಪಾಶೌ ದಧತಂ ಭಜಾಮಿ ॥
ಮನುಃ: 1। ಕ್ಲೀಂ ಕಾಮದೇವಾಯ ನಮಃ । ಹ್ರೀಂ ರತ್ಯೈ ನಮಃ ॥
2। ಕಾಮದೇವಾಯ ಸರ್ವಜನಪ್ರಿಯಾಯ ಸರ್ವಜನಸಮ್ಮೋಹನಾಯ ಜ್ವಲ ಜ್ವಲ ಪ್ರಜ್ವಲ ಸರ್ವಜನಸ್ಯ ಹೃದಯಂ ಮಮ ವಶಂ ಕುರು ಕುರು ಸ್ವಾಹಾ ॥
ಕಾಮದೇವಾಯ ವಿದ್ಮಹೇ ಪುಷ್ಪಬಾಣಾಯ ಧೀಮಹಿ । ತನ್ನೋಽನಂಗಃ ಪ್ರಚೋದಯಾತ್ ॥
ಕುಬೇರಗಾಯತ್ರೀ
ಅಸ್ಯ ಶ್ರೀಕುಬೇರಗಾಯತ್ರೀಮಂತ್ರಸ್ಯ ವಿಶ್ರವಾ ಋಷಿಃ । ಬೃಹತೀ ಚ್ಛಂದಃ । ಶಿವಮಿತ್ರಧನೇಶ್ವರೋ ದೇವತಾ । ಮಮ ದಾರಿದ್ರ್ಯ ದೋಷಪರಿಹಾರಾರ್ಥೇ ಜಪೇ ವಿನಿಯೋಗಃ ॥
ನ್ಯಾಸಃ1।ಯಕ್ಷಾಯ 2।ಕುಬೇರಾಯ 3।ವೈಶ್ರವಣಾಯ 4।ಧನಧಾನ್ಯಾಧಿಪತಯೇ 5।ಧನಧಾನ್ಯಸಮೃದ್ಧಿಂ ಮೇ 6।ದೇಹಿ ದಾಪಯ ಸ್ವಾಹಾ ॥
ಧ್ಯಾನಂ
ಮನುಜವಾಹ್ಯವಿಮಾನವರಸ್ಥಿತಂ ಗುರುಡರತ್ನನಿಭಂ ನಿಧಿನಾಯಕಂ । ಶಿವಸಖಂ ಮುಕುಟಾದಿವಿಭೂಷಿತಂ ವರಗದೇ ದಧತಂ ಭಜ ತುಂದಿಲಂ ॥
ಮನುಃ: ಯಕ್ಷಾಯ ಕುಬೇರಾಯ ವೈಶ್ರವಣಾಯ ಧನಧಾನ್ಯಾಧಿಪತಯೇ ಧನಧಾನ್ಯಸಮೃದ್ಧಿಂ ಮೇ ದೇಹಿ ದಾಪಯ ಸ್ವಾಹಾ ॥
ಯಕ್ಷರಾಜಾಯ ವಿದ್ಮಹೇ ವೈಶ್ರವಣಾಯ ಧೀಮಹಿ । ತನ್ನಃ ಕುಬೇರಃ ಪ್ರಚೋದಯಾತ್ ॥


ಅಷ್ಟದಳಪೂಜಾ
ಪೂರ್ವದಳೇ ಲಕ್ಷ್ಮ್ಯೈ  ನಮಃ, ಶ್ರೀಲಕ್ಷ್ಮ್ಯೈ  ನಮಃ ।
ಆಗ್ನೇಯದಳೇ ಮಹಾಲಕ್ಷ್ಮ್ಯೈ  ನಮಃ, ಜಯಲಕ್ಷ್ಮ್ಯೈ  ನಮಃ ।
ದಕ್ಷಿಣದಳೇ ತ್ರಿಶಕ್ತ್ಯೈ  ನಮಃ, ವಿಜಯಲಕ್ಷ್ಮ್ಯೈ  ನಮಃ ।
ನೈರ್ಋತ್ಯದಳೇ ಸಾಮ್ರಾಜ್ಯದಾಯಿನ್ಯೈ  ನಮಃ, ವರಲಕ್ಷ್ಮ್ಯೈ  ನಮಃ ।
ಪಶ್ಚಿಮದಳೇ ಶ್ರೀವಿದ್ಯಾಯೈ  ನಮಃ, ಸೌಭಾಗ್ಯಲಕ್ಷ್ಮ್ಯೈ  ನಮಃ ।
ವಾಯವ್ಯದಳೇ ಪರಂಜ್ಯೋತಿಷೇ  ನಮಃ, ಧನಲಕ್ಷ್ಮ್ಯೈ  ನಮಃ ।
ಉತ್ತರದಳೇ ಪರನಿಷ್ಕಳಾಯೈ  ನಮಃ, ಮೋಕ್ಷಲಕ್ಷ್ಮ್ಯೈ  ನಮಃ ।
ಈಶಾನ್ಯದಳೇ ತ್ರಿಪುರಸುಂದರ್ಯೈ  ನಮಃ, ಜ್ಞಾನಲಕ್ಷ್ಮ್ಯೈ  ನಮಃ ।

ಷೋಡಶದಳಪೂಜಾ 
ಅಜಪಾಯೈ ನಮಃ ಶ್ರಿಯೈ ನಮಃ ।
ಮಾತೃಕಾಯೈ ನಮಃ ಲಕ್ಷ್ಮ್ಯೈ ನಮಃ ।
ತ್ವರಿತಾಯೈ ನಮಃ ಪದ್ಮಾಯೈ ನಮಃ ।
ಪಾರಿಜಾತೇಶ್ವರ್ಯೈ ನಮಃ ಧಾತ್ರ್ಯೈ ನಮಃ ।
ಪಂಚಬಾಣೇಶ್ಯೈ ನಮಃ ರಮಾಯೈ ನಮಃ ।
ಅಮೃತಪೀಠೇಶ್ವರ್ಯೈ ನಮಃ ವರದಾಯೈ ನಮಃ ।
ಸುಧಾಯೈ ನಮಃ ಲೋಕಮಾತ್ರೇ ನಮಃ ।
ಅಮೃತೇಶ್ವರ್ಯೈ ನಮಃ ಚತುರ್ಭುಜಾಯೈ ನಮಃ ।
ಮಾಹೇಶ್ವರ್ಯೈ ನಮಃ ಋದ್ಧ್ಯೈ ನಮಃ ।
ಅನ್ನಪೂರ್ಣೇಶ್ವರ್ಯೈ ನಮಃ ಸಿದ್ಧ್ಯೈ ।
ಸಿದ್ಧಲಕ್ಷ್ಮ್ಯೈ ನಮಃ ಪುಷ್ಟ್ಯೈ ನಮಃ ।
ಮಾತಂಗ್ಯೈ ನಮಃ ತುಷ್ಟ್ಯೈ ನಮಃ ।
ಭುವನೇಶ್ವರ್ಯೈ ನಮಃ ಇಂದಿರಾಯೈ ನಮಃ ।
ವಾರಾಹ್ಯೈ ನಮಃ ಹರಿಪ್ರಿಯಾಯೈ ನಮಃ ।
ದೇವ್ಯೈ ನಮಃ ಭೂತ್ಯೈ ನಮಃ ।
ಮಹಾತ್ರಿಪುರಸುಂದರ್ಯೈ ನಮಃ ಈಶ್ವರ್ಯೈ ನಮಃ  ।

ವಶಿನ್ಯೈ ನಮಃ । ‌ಕಾಮೇಶ್ವರ್ಯೈ ನಮಃ । ಮೋದಿನ್ಯೈ ನಮಃ । ವಿಮಲಾಯೈ ನಮಃ । ಅರುಣಾಯೈ ನಮಃ । ‌ಜಯಿನ್ಯೈ ನಮಃ । ಸರ್ವೇಶ್ವರ್ಯೈ ನಮಃ । ಕೌಳಿನ್ಯೈ ನಮಃ ।
ವಿದ್ಯಾಯೋಗಿನ್ಯೈ । ದೀಪಿಕಾಯೋಗಿನ್ಯೈ । ಜ್ಞಾನಯೋಗಿನ್ಯೈ । ಆಪ್ಯಾಯಿನೀಯೋಗಿನ್ಯೈ । ವ್ಯಾಪಿನೀಯೋಗಿನ್ಯೈ । ವ್ಯೋಮರೂಪಾಯೋಗಿನ್ಯೈ । ಲಕ್ಷ್ಮೀಯೋಗಿನ್ಯೈ । ಗಂಧಾಕರ್ಷಣೀಯೋಗಿನ್ಯೈ । ರಸಾಕರ್ಷಣೀಯೋಗಿನ್ಯೈ । ಸ್ಪರ್ಶಾಕರ್ಷಣೀಯೋಗಿನ್ಯೈ । ರೂಪಾಕರ್ಷಣೀಯೋಗಿನ್ಯೈ ನಮಃ ।

ಅಮೃತಾಂಭೋನಿಧಯೇ ನಮಃ । ರತ್ನದ್ವೀಪಾಯ ನಮಃ । ನಾನಾವೃಕ್ಷಮಹೋದ್ಯಾನಾಯ ನಮಃ । ಕಲ್ಪವೃಕ್ಷವಾಟಿಕಾಯೈ ನಮಃ । ಸಂತಾನವಾಟಿಕಾಯೈ ನಮಃ । ಹರಿಚಂದನವಾಟಿಕಾಯೈ ನಮಃ । ಮಂದಾರವಾಟಿಕಾಯೈ ನಮಃ । ಪಾರಿಜಾತವಾಟಿಕಾಯೈ ನಮಃ । ಕದಂಬವಾಟಿಕಾಯೈ ನಮಃ । ಪುಷ್ಯರಾಗರತ್ನಪ್ರಾಕಾರಾಯ ನಮಃ । ಪದ್ಮರಾಗರತ್ನಪ್ರಾಕಾರಾಯ ನಮಃ । ಗೋಮೇಧಕರತ್ನಪ್ರಾಕಾರಾಯ ನಮಃ । ವಜ್ರರತ್ನಪ್ರಾಕಾರಾಯ ನಮಃ । ವೈಡೂರ್ಯರತ್ನಪ್ರಾಕಾರಾಯ ನಮಃ । ಇಂದ್ರನೀಲರತ್ನಪ್ರಾಕಾರಾಯ ನಮಃ । ಮುಕ್ತಾರತ್ನಪ್ರಾಕಾರಾಯ ನಮಃ । ಮರಕತರತ್ನಪ್ರಾಕಾರಾಯ ನಮಃ । ವಿದ್ರುಮರತ್ನಪ್ರಾಕಾರಾಯ ನಮಃ । ಮಾಣಿಕ್ಯಮಂಡಪಾಯ ನಮಃ । ಸಹಸ್ರ ಸುವರ್ಣಸ್ತಂಭೋಪೇತ ಚಂತಾಮಣಿ ನವರತ್ನ ಮಂಡಪಾಯ ನಮಃ । ಅಮೃತವಾಪಿಕಾಯೈ ನಮಃ । ಆನಂದವಾಪಿಕಾಯೈ ನಮಃ । ವಿಮರ್ಶವಾಪಿಕಾಯೈ ನಮಃ । ಬಾಲಾತಪೋದ್ಗಾರಕಕ್ಷಾಯ ನಮಃ । ಚಂದ್ರಿಕೋದ್ಗಾರಕಕ್ಷಾಯ ನಮಃ । ಮಹಾಶೃಂಗಾರಪರಿಘಾಯೈ ನಮಃ । ಮಹಾಪದ್ಮಾಟವ್ಯೈ ನಮಃ । ಚಿಂತಾಮಣಿಮಯಗೃಹರಾಜಾಯ ನಮಃ ।

ಸಾಂಗಾಭ್ಯಾಂ ಸಾಯುಧಾಭ್ಯಾಂ ಸವಾಹನಾಭ್ಯಾಂ ಸರ್ವಾಲಂಕಾರಭೂಷಿತಾಭ್ಯಾಂ  ಅಸಮಾನಲಾವಣ್ಯಾಭ್ಯಾಂ ಶ್ರೀಲಕ್ಷ್ಮೀನಾರಾಯಣಾಭ್ಯಾಂ ನಮಃ । ರತ್ನಸಿಂಹಾಸನಂ ಕಲ್ಪಯಾಮಿ ನಮಃ । 

ಶಾಂತಾಕಾರಂ ಭುಜಗಶಯನಂ ಪದ್ಮನಾಭಂ ಸುರೇಶಂ\\
ವಿಶ್ವಾಧಾರಂ ಗಗನ ಸದೃಶಂ ಮೇಘವರ್ಣಂ ಶುಭಾಂಗಂ ।\\
ಲಕ್ಷ್ಮೀಕಾಂತಂ ಕಮಲನಯನಂ ಯೋಗಿಬಿರ್ಧ್ಯಾನಗಮ್ಯಂ\\
ವಂದೇ ವಿಷ್ಣುಂ ಭವಭಯಹರಂ ಸರ್ವಲೋಕೈಕನಾಥಂ ॥

ಉದ್ಯದಾದಿತ್ಯಸಂಕಾಶಂ ಪೀತವಾಸಸಮಚ್ಯುತಂ ।\\
ಶಂಖಚಕ್ರಗದಾಪಾಣಿಂ ಧ್ಯಾಯೇಲ್ಲಕ್ಷ್ಮೀಪತಿಂ ಹರಿಂ ॥

ಉದ್ಯತ್ಪ್ರದ್ಯೋತನಶತರುಚಿಂ ತಪ್ತಹೇಮಾವದಾತಂ\\
ಪಾರ್ಶ್ವದ್ವಂದ್ವೇ ಜಲಧಿಸುತಯಾ ವಿಶ್ವಯೋನ್ಯಾ ಚ ಜುಷ್ಟಂ ।\\
ನಾನಾರತ್ನೋಲ್ಲಸಿತವಿವಿಧಾಕಲ್ಪಮಾಪೀತವಸ್ತ್ರಂ\\
ವಿಷ್ಣುಂ ವಂದೇ ದರಕಮಲಕೌಮೋದಕೀ ಚಕ್ರಹಸ್ತಮ್ ॥

ಹಸ್ತದ್ವಯೇನ ಕಮಲೇ ಧಾರಯಂತೀಂ ಸ್ವಲೀಲಯಾ ।\\
ಹಾರನೂಪುರಸಂಯುಕ್ತಾಂ ಲಕ್ಷ್ಮೀಂ ದೇವೀಂ ವಿಚಿಂತಯೇ ॥

ಪೀತಾಂಬರಧರಂ ದೇವಂ ಗದಾಗರುಡವಾಹನಮ್ । ದುಗ್ಧಾಬ್ಧಿ ಮಧ್ಯನಿಲಯಂ ವಿಷ್ಣಮಾವಾಹಯಾಮ್ಯಹಮ್ ॥
ಮಹಾಲಕ್ಷ್ಮಿ ಸಮಾಗಚ್ಛ ಪದ್ಮನಾಭಪದಾದಿಹ । ಪೂಜಾಮಿಮಾಂ ಗೃಹಾಣ ತ್ವಂ ತ್ವದರ್ಥಂ ದೇವಿ ಸಂಭೃತಾಮ್ ॥

ಈಶಾನಾಯ ನಮಸ್ತುಭ್ಯಂ ಸರ್ವಬೀಜಮಯಂ ಶುಭಂ । ಸ್ವಾತ್ಮಸ್ಥಾಯ ಪರಂ ಶುದ್ಧಮಾಸನಂ ಕಲ್ಪಯಾಮ್ಯಹಮ್ ॥
ಭಕ್ತವಿಘ್ನೋಪಶಮನೀಂ ತಪ್ತಕಾಂಚನ ಸನ್ನಿಭಾಮ್ । ಲೋಕರಕ್ಷಾಕರೀಂ ಮಾಯಾಂ ಆಸನೇ ವಿನಿಯೋಜಯೇ ॥

ಯಸ್ಯ ದರ್ಶನಮಿಚ್ಛಂತಿ ದೇವಾಃ ಸ್ವಾಭೀಷ್ಟಸಿದ್ಧಯೇ । ತಸ್ಮೈ ತೇ ಪದ್ಮನಾಭಾಯ ಸ್ವಾಗತಂ ಕಲ್ಪಯಾಮ್ಯಹಮ್ ॥

ಯದ್ಭಕ್ತಿಲೇಶ ಸಂಪರ್ಕಾತ್ಪರಮಾನಂದ ಸಂಭವಃ । ತಸ್ಮೈ ತೇ ಪದ್ಮನಾಭಾಯ ಪಾದ್ಯಂ ಶುದ್ಧಾಯ ಕಲ್ಪಯೇ ॥
ಗಂಗಾದಿಸಲಿಲಾಧಾರಾಂ ತೀರ್ಥಮಂತ್ರಾಭಿಮಂತ್ರಿತಮ್ । ದೂರಯಾತ್ರಾಶ್ರಮಹರಂ ಪಾದ್ಯಂ ಮೇ ಪ್ರತಿಗೃಹ್ಯತಾಮ್ ॥

ಅರ್ಘ್ಯಂ ಗೃಹಾಣ ದೇವೇಶ ಪತ್ರಪುಷ್ಪ ಸಮನ್ವಿತಮ್ । ಗಂಧಾಕ್ಷತಯುತಂ ದೇವ ಗೃಹಾಣ ಪುರುಷೋತ್ತಮ ॥
ತೀರ್ಥೋದಕೈರ್ಮಹಾಪುಣ್ಯೈಃ ಕಲ್ಪಿತಂ ಪಾಪಹಾರಕೈಃ । ಗೃಹಾಣಾರ್ಘ್ಯಂ ಮಹಾಲಕ್ಷ್ಮಿ ಭಕ್ತಾನಾಮುಪಕಾರಿಣಿ ॥

ಗೃಹಾಣಾಚಮನಂ ದೇವ ಸರ್ವಸಿದ್ಧಿಪ್ರದಾಯಕ । ಸುರವಂದಿತ ಪಾದಾಬ್ಜ ರಮಾಕಾಂತಾಯ ತೇ ನಮಃ ॥
ಕರ್ಪೂರಾಗರುಸಂಯುಕ್ತಂ ಶೀತಲಂ ಜಲಮುತ್ತಮಮ್ । ಲೋಕಮಾತರ್ಗೃಹಾಣೇದಂ ದತ್ತಮಾಚಮನಂ ಮಯಾ ॥

ದಧ್ಯಾಜ್ಯ ಮಧುಸಂಯುಕ್ತಂ ಕ್ಷೀರಶುದ್ಧಜಲೈರ್ಯುತಮ್ । ಮಧುಪರ್ಕಂ ಮಯಾ ದತ್ತಂ ಪ್ರೀತ್ಯಾ ಸ್ವೀಕುರು ಶ್ರೀಪತೇ ॥

ಆಕಲ್ಪಸಂಶೋಭಿತ ದಿವ್ಯಗಾತ್ರ ರಾಕೇಂದು ನೀಕಾಶ ಮುಖಾರವಿಂದ । ದತ್ತಂ ಪುನಶ್ಚಾಚಮನಂ ಗೃಹಾಣ ಬ್ರಹ್ಮಾದಿ ಸಂಸೇವಿತ ಪಾದಪದ್ಮ ॥

ಕೇಶವಾಯ ನಮಸ್ತುಭ್ಯಂ ಪುರಾಣಪುರುಷೋತ್ತಮ । ಸಾಂಗೋಪಾಂಗಮಿದಂ ಸ್ನಾನಂ ಕಲ್ಪಯಾಮಿ ರಮಾಪತೇ ॥
ಸ್ನಾನಾಯ ತೇ ಮಹಾಲಕ್ಷ್ಮಿ ಕರ್ಪೂರಾಗರುವಾಸಿತಮ್ । ಆಹೃತಂ ಸರ್ವತೀರ್ಥೇಭ್ಯೋ ಸಲಿಲಂ ಪ್ರತಿಗೃಹ್ಯತಾಮ್ ॥

ಕಾಮಧೇನು ಸಮುತ್ಪನ್ನಂ ದೇವರ್ಷಿ ಪಿತೃತೃಪ್ತಿದಮ್ । ಪಯೋ ದದಾಮಿ ದೇವೇಶಿ ಸ್ನಾನಾರ್ಥಂ ಪ್ರತಿಗೃಹ್ಯತಾಮ್ ॥

ಪಯಸಾ ತು ಸಮುತ್ಪನ್ನಂ ಮಧುರಾಮ್ಲ ಶಶಿಪ್ರಭಮ್ । ದಧ್ಯಾನೀತಂ ಮಯಾ ಸ್ವಾಮಿನ್ ಸ್ನಾನಾರ್ಥಂ ಪ್ರತಿಗೃಹ್ಯತಾಮ್ ॥

ನವನೀತಸಮುತ್ಪನ್ನಂ ಆಯುರಾರೋಗ್ಯವರ್ಧನಂ । ಘೃತಂ ತುಭ್ಯಂ ಪ್ರದಾಸ್ಯಾಮಿ ಸ್ನಾನಾರ್ಥಂ ಪ್ರತಿಗೃಹ್ಯತಾಂ ॥

ತರುಪುಷ್ಪಸಮಾಕೃಷ್ಟಂ ಸುಸ್ವಾದು ಮಧುರಂ ಮಧು । ತೇಜಃಪುಷ್ಟಿಕರಂ ದಿವ್ಯಂ ಸ್ನಾನಾರ್ಥಂ ಪ್ರತಿಗೃಹ್ಯತಾಂ ॥

ಕ್ಷಸಾರಸಮುದ್ಭೂತಾ ಶರ್ಕರಾ ಪುಷ್ಟಿಕಾರಿಕಾ । ಮಲಾಪಹಾರಿಕಾ ದಿವ್ಯಾ ಸ್ನಾನಾರ್ಥಂ ಪ್ರತಿಗೃಹ್ಯತಾಂ ॥

ಮಲಯಾಚಲಸಂಭೂತಂ ಸುಗಂಧಂ ಶೀತಲಂ ಶುಭಂ । ಸುಕಾಂತಿದಾಯಕಂ ದಿವ್ಯಂ ಸ್ನಾನಾರ್ಥಂ ಪ್ರತಿಗೃಹ್ಯತಾಂ ॥

ಲತಾವೃಕ್ಷಸಮುತ್ಪನ್ನಂ ಸುಗಂಧಂ ಶೋಭನಂ ಪರಂ । ಸಂತೋಷವರ್ಧನಂ ನಿತ್ಯಂ ಸ್ನಾನಾರ್ಥಂ ಪ್ರತಿಗೃಹ್ಯತಾಂ ॥

ಸರ್ವಸಾರಸಮುದ್ಭೂತಂ ಶಕ್ತಿ ಪುಷ್ಟಿಕರಂ ದೃಢಮ್ । ಸುಫಲಂ ಕಾರ್ಯಸಿದ್ಧ್ಯರ್ಥಂ ಸ್ನಾನಾರ್ಥಂ ಪ್ರತಿಗೃಹ್ಯತಾಮ್ ॥

ಪೃಥ್ವೀ ಗರ್ಭಸಂಭೂತಂ ಸುವರ್ಣಂ ಕಾಂತಿದಾಯಕಮ್ । ಆತ್ಮತೇಜಃ ಸ್ವರೂಪಾಯ ತುಭ್ಯಂ ಸ್ನಾನಾರ್ಥಮರ್ಪಯೇ ॥

ಕೇಶವಾಯ ನಮಸ್ತುಭ್ಯಂ ಪುರಾಣಪುರುಷೋತ್ತಮ । ಸಾಂಗೋಪಾಂಗಮಿದಂ ಸ್ನಾನಂ ಕಲ್ಪಯಾಮಿ ರಮಾಪತೇ ॥

ಸ್ನಾನಾಯ ತೇ ಮಹಾಲಕ್ಷ್ಮಿ ಕರ್ಪೂರಾಗರುವಾಸಿತಮ್ । ಆಹೃತಂ ಸರ್ವತೀರ್ಥೇಭ್ಯಃ ಸಲಿಲಂ ಪ್ರತಿಗೃಹ್ಯತಾಮ್ ॥

ನಾರಾಯಣ ನಮಸ್ತೇಽಸ್ತು ನರಕಾರ್ಣವ ತಾರಕ । ತ್ರೈಲೋಕ್ಯವ್ಯಾಪಕೋ ದೇವ ವಸ್ತ್ರಂ ತೇ ಪ್ರತಿಗೃಹ್ಯತಾಮ್ ॥

ಹೇಮವಸ್ತ್ರಂ ಪ್ರಯಚ್ಛಾಮಿ ಹೇಮಕಂಚುಕಯಾ ಯುತಮ್ । ಗೃಹಾಣೇದಂ ಮಹಾಲಕ್ಷ್ಮಿ ನಾರಾಯಣಿ ನಮೋಽಸ್ತು ತೇ ॥

ಸುವರ್ಣತಂತೂದ್ಭವ ಯಜ್ಞಸೂತ್ರಂ ಮುಕ್ತಾಫಲಸ್ಯೂತಮನೇಕರತ್ನಂ । ಗೃಹಾಣ ತದ್ವತ್ ಪ್ರಿಯಮುತ್ತರೀಯಂ ಸ್ವಕರ್ಮಸೂತ್ರಾಂತರಿಣೇ ನಮೋಽಸ್ತು ॥
ರಾಜತಂ ಬ್ರಹ್ಮಸೂತ್ರಂ ಚ ಕಾಂಚನಂ ಚೋತ್ತರೀಯಕಂ । ಗೃಹಾಣೇದಂ ಮಹಾಲಕ್ಷ್ಮಿ ಜಗನ್ಮಾತರ್ನಮೋಽಸ್ತು ತೇ ॥

(ಗ್ರಂಥಿದೋರ ಸ್ಥಾಪನಂ) ಷೋಡಶಗ್ರಂಥಿ ಸಂಯುಕ್ತಂ ಸ್ವರ್ಣಸೂತ್ರಂ ಶುಭಪ್ರದಂ । ಸ್ಥಾಪಯಿತ್ವಾ ಮಹಾಲಕ್ಷ್ಮ್ಯಾಃ ಸಮೀಪೇ ತತ್ಪ್ರಪೂಜಯೇತ್ ॥

ಕೇಯೂರ ಕಟಕೇ ಚೈವ ಹಸ್ತೇ ಚಿತ್ರಾಂಗುಲೀಯಕಂ । ಮಾಣಿಕ್ಯೋಲ್ಲಾಸಿ ಮುಕುಟಂ ಕುಂಡಲೇ ಹಾರಶೋಭಿತಮ್ ॥
ನಾಭೌ ನಾಯಕರತ್ನಂ ಚ ನೂಪುರೇ ಹಾರಪದ್ಮಕಮ್ । ರತ್ನಕಂಕಣ ಕೇಯೂರ ಕಾಂಚೀಕುಂಡಲನೂಪುರಂ ॥
ಮುಕ್ತಾಹಾರಂ ಕಿರೀಟಂ ಚ ವಿವಿಧಾಭರಣಾನಿ ಚ । ಗೃಹಾಣ ಭೂಷಣಂ ದೇವಿ ಸರ್ವಸಂಪತ್ಪ್ರದಾಯಿನಿ ॥

ಶ್ರೀಕೃಷ್ಣಾಯ ಪಾದೌ ನಮಃ । ಪಾದೌ ಪೂಜಯಾಮಿ ॥
ಮಾಧವಾಯ ನಮಃ । ಜಾನುನೀ ಪೂಜಯಾಮಿ ॥
ಪೀತಾಂಬರಧಾರಿಣೇ ನಮಃ । ಊರೂ ಪೂಜಯಾಮಿ ॥
ಮಾಧವಾಯ ನಮಃ । ಕಟಿಂ ಪೂಜಯಾಮಿ ॥
ಗೋವಿಂದಾಯ ನಮಃ । ನಾಭಿಂ ಪೂಜಯಾಮಿ ॥
ಲೋಕಧಾರಿಣೇ ನಮಃ । ಉದರಂ ಪೂಜಯಾಮಿ ॥
ವಿಶ್ವನಾಥಾಯ ನಮಃ । ಹೃದಯಂ ಪೂಜಯಾಮಿ ॥
ಶ್ರೀವತ್ಸಧಾರಿಣೇ ನಮಃ । ಬಾಹೂನ್ ಪೂಜಯಾಮಿ ॥
ಕೌಸ್ತುಭೇಷಾಯ ನಮಃ । ಕಂಠಂ ಪೂಜಯಾಮಿ ॥
ನಾರಾಯಣಾಯ ನಮಃ । ಮುಖಂ ಪೂಜಯಾಮಿ ॥
ಸರ್ವಾತ್ಮನೇ ನಮಃ । ಶಿರಃ ಪೂಜಯಾಮಿ ॥
ಮಹಾವಿಷ್ಣವೇ ನಮಃ । ಸರ್ವಾಂಗಂ ಪೂಜಯಾಮಿ  ॥

ದೇವ್ಯೈ ನಮಃ ।ಪಾದೌ ಪೂಜಯಾಮಿ~॥
ಲಕ್ಷ್ಮ್ಯೈ ನಮಃ ।ಜಂಘೇ ಪೂಜಯಾಮಿ~॥
ಲೋಕರಕ್ಷಾಕಾರಿಣ್ಯೈ ನಮಃ ।ಜಾನುನೀ ಪೂಜಯಾಮಿ~॥
ಚಂದ್ರಸಹೋದರ್ಯೈ ನಮಃ ।ಕಟಿಂ ಪೂಜಯಾಮಿ~॥
ಭಕ್ತವತ್ಸಲಾಯೈ ನಮಃ ।ಉದರಂ ಪೂಜಯಾಮಿ~॥
ಕ್ಷೀರೋದಮಥನೋದ್ಭವಾಯೈ ನಮಃ ।ವಕ್ಷಸ್ಥಲಂ ಪೂಜಯಾಮಿ~॥
ಸರ್ವೈಶ್ವರ್ಯಕರ್ಯೈ ನಮಃ ।ಸ್ತನೌ ಪೂಜಯಾಮಿ~॥
ಮಂಗಲಾಯೈ ನಮಃ ।ಬಾಹೂ ಪೂಜಯಾಮಿ~॥
ಕಮಲಾಯೈ ನಮಃ ।ಹಸ್ತೌ ಪೂಜಯಾಮಿ~॥
ಸಿದ್ಧಲಕ್ಷ್ಮ್ಯೈ ನಮಃ ।ಕಂಠಂ ಪೂಜಯಾಮಿ~॥
ಶ್ರಿಯೈ ನಮಃ ।ಮುಖಂ ಪೂಜಯಾಮಿ~॥
ವರದಾಯೈ ನಮಃ ।ನಾಸಿಕಾಂ ಪೂಜಯಾಮಿ~॥
ಸುರಾರ್ಚಿತಾಯೈ ನಮಃ ।ನೇತ್ರೇ ಪೂಜಯಾಮಿ~॥
ಭಕ್ತಾಭೀಷ್ಟಪ್ರದಾಯೈ ನಮಃ ।ಕರ್ಣೌ ಪೂಜಯಾಮಿ~॥
ರಮಾಯೈ ನಮಃ ।ಲಲಾಟಂ ಪೂಜಯಾಮಿ~॥
ಮಾಯಾಯೈ ನಮಃ ।ಶಿರಃ ಪೂಜಯಾಮಿ~॥
ಶ್ರೀಮಹಾಲಕ್ಷ್ಮ್ಯೈ ನಮಃ । ಸರ್ವಾಂಗಂ ಪೂಜಯಾಮಿ~॥

ಅನಂತಾಯ ಶ್ರೀವಲ್ಲಭಾಯ ಜಗನ್ನಾಥಾಯ ಚತುರ್ಮೂರ್ತಯೇ ಚತುರ್ಭುಜಾಯ ಶ್ರೀಧರಾಯ ಚಕ್ರಪಾಣಯೇ ಶ್ರೀವಿಷ್ಣವೇ ಪದ್ಮನಾಭಾಯ ಶಂಖಪಾಣಯೇ ಗರುಡಧ್ವಜಾಯ ತ್ರಿವಿಕ್ರಮಾಯ ನಾರಸಿಂಹಾಯ ಕಾಮರೂಪಾಯ

ಬಿಲ್ವ ಜಾತೀ ಕರವೀರ ಜಂಬೂ ಚೂತ  ಪುನ್ನಾಗ ಮರುಗ ದೇವದಾರು ಅಶೋಕ ಶತಪತ್ರ ಮಲ್ಲಿಕಾ ಮಾಲತೀ ಶಮೀ  ಅಪಾಮಾರ್ಗ

ಶ್ರೀಕೃಷ್ಣಾಯ ಮಧುಸೂದನಾಯ ಕೇಶವಾಯ ನಾರಾಯಣಾಯ ಗೋವಿಂದಾಯ ವಿಷ್ಣವೇ ಮಧುಸೂದನಾಯ ತ್ರಿವಿಕ್ರ,ಮಾಯ ವಾಮನಾಯ ಶ್ರೀಧರಾಯ ಹೃಷೀಕೇಶಾಯ ಪದ್ಮನಾಭಾಯ ದಾಮೋದರಾಯ ಶ್ರೀಮಹಾವಿಷ್ಣವೇ

ತಕ್ಷಕಾಯ ತ್ರಿವಿಕ್ರಮಾಯ ವಾಸುದೇವಾಯ ನಾರಾಯಣಾಯ ಶ್ರೀಕೃಷ್ಣಾಯ ಹರಯೇ ದಂಷ್ಟ್ರಿಣೇ ವಿಶಿಷ್ಟಾಯ ಶಿಷ್ಟೇಷ್ಟಾಯ ಶಿಖಂಡಿನೇ ನಹುಷಾಯ ವಿಶ್ವಬಾಹವೇ ಮಹೀಧರಾಯ ಅಚ್ಯುತಾಯ ವಿಷ್ಣವೇ

ದ್ರೋಣ ಕರ್ಣಿಕಾ ವಿಷ್ಣುಕ್ರಾಂತಿ ಕೃತಕೀ ಮಂದಾರ ತುಲಸೀ ಸೇವಂತಿಕಾ ಪುನ್ನಾಗ ಕರವೀರ ದತ್ತೂರ ನಂದ್ಯಾವರ್ತ ಮಲ್ಲಿಕಾ ಮಾಲತೀ ಗಿರಿಕರ್ಣಿಕಾ ಸರ್ವ

ಶ್ರೀಮಹಾಲಕ್ಷ್ಮ್ಯೈ ಮಹಾದೇವ್ಯೈ  ಪದ್ಮಧಾರಿಣ್ಯೈ ರಮಾಯೈ ಇಂದಿರಾಯೈ ಚಂಚಲಾಯೈ ಹರಿಪ್ರಿಯಾಯೈ ಲೋಕಮಾತ್ರೇ ಕ್ಷೀರಾಬ್ಧಿತನಯೇ ಮಾಯಾಯೈ ಪದ್ಮಪ್ರಿಯಾಯೈ ವರದಾಯೈ ಲಲಿತಾಯೈ ಶ್ರಿಯೈ ಕಮಲವಾಸಿನ್ಯೈ ಸರ್ವೇಶ್ವರ್ಯೈ ವೈಷ್ಣವ್ಯೈ ಮಂಗಲದೇತಾಯೈ ಜ್ಞಾನದಾಯೈ ಶ್ರೀಲಕ್ಷ್ಮ್ಯೈ 

ಕೇತಕೀ ಮಲ್ಲಿಕಾ ಕರವೀರ ಜಾತೀ ಕುಮುದ ಪಾರಿಜಾತ ಕಮಲ ಸೇವಂತಿಕಾ ನಂದ್ಯಾವರ್ತ ಶತಪತ್ರ ಚಂಪಕ ಕರ್ಣಿಕಾರ ನೀಲೋತ್ಪಲ ಪಾಟಲೀ  ಅಶೋಕ ಮಾಲತೀ ರಕ್ತೋತ್ಪಲ ಸುವರ್ಣ ಮಂದಾರ ಸರ್ವ


\subsection{ಪ್ರಥಮಾವರಣಮ್}
ಓಂ ಕ್ರುದ್ಧೋಲ್ಕಾಯ ಸ್ವಾಹಾ ಹೃದಯಾಂಗ ದೇವತಾಭ್ಯೋ ನಮಃ । ಓಂ ಮಹೋಲ್ಕಾಯ ಸ್ವಾಹಾ ಶಿರೋಽಂಗ ದೇವತಾಭ್ಯೋ ನಮಃ । ಓಂ ವೀರೋಲ್ಕಾಯ ಸ್ವಾಹಾ ಶಿಖಾಂಗ ದೇವತಾಭ್ಯೋ ನಮಃ । ಓಂ ದ್ವ್ಯುಲ್ಕಾಯ ಸ್ವಾಹಾ ಕವಚಾಂಗ ದೇವತಾಭ್ಯೋ ನಮಃ । ಓಂ ಜ್ಞಾನೋಲ್ಕಾಯ ಸ್ವಾಹಾ ನೇತ್ರಾಂಗ ದೇವತಾಭ್ಯೋ ನಮಃ। ಓಂ ಸಹಸ್ರೋಲ್ಕಾಯ ಸ್ವಾಹಾ ಅಸ್ತ್ರಾಂಗ ದೇವತಾಭ್ಯೋ ನಮಃ ॥\\
ಅಭೀಷ್ಟಸಿದ್ಧಿಂ ಮೇ ದೇಹಿ ಶರಣಾಗತ ವತ್ಸಲ~।\\
ಭಕ್ತ್ಯಾ ಸಮರ್ಪಯೇ ತುಭ್ಯಂ ಪ್ರಥಮಾವರಣಾರ್ಚನಂ ॥

\subsection{ದ್ವಿತೀಯಾವರಣಮ್}
ವಾಸುದೇವಾಯ ಸಂಕರ್ಷಣಾಯ ಪ್ರದ್ಯುಮ್ನಾಯ ಅನಿರುದ್ಧಾಯ ಶಾಂತ್ಯೈ ಶ್ರಿಯೈ ಸರಸ್ವತ್ಯೈ ರತ್ಯೈ ಶ್ರೀಮಹಾವಿಷ್ಣವೇ

\subsection{ತೃತೀಯಾವರಣಮ್}
ಚಕ್ರಾಯ ಶಂಖಾಯ ಗದಾಯ ಪದ್ಮಾಯ ಕೌಸ್ತುಭಾಯ ಖಡ್ಗಾಯ ವನಮಾಲಾಯ 

\subsection{ತುರೀಯಾವರಣಮ್}
ಧ್ವಜಾಯ ಗರುಡಾಯ ಶಂಖನಿಧಯೇ ಪದ್ಮನಿಧಯೇ ವಿಘ್ನೇಶ್ವರಾಯ ದುರ್ಗಾಯೈ ವಿಷ್ವಕ್ಸೇನಾಯ ಪರಮಾತ್ಮನೇ 

\subsection{ಪಂಚಮಾವರಣಮ್}
ಓಂ ಬ್ರಾಹ್ಮ್ಯೈ ನಮಃ । ಓಂ ಮಾಹೇಶ್ವರ್ಯೈ ನಮಃ । ಓಂ ಕೌಮಾರ್ಯೈ ನಮಃ । ಓಂ ವೈಷ್ಣವ್ಯೈ ನಮಃ । ಓಂ ವಾರಾಹ್ಯೈ ನಮಃ । ಓಂ ಮಾಹೇಂದ್ರ್ಯೈ ನಮಃ । ಓಂ ಚಾಮುಂಡಾಯೈ ನಮಃ । ಓಂ ಮಹಾಲಕ್ಷ್ಮ್ಯೈ ನಮಃ ॥\\
ಅಭೀಷ್ಟಸಿದ್ಧಿಂ******ಪಂಚಮಾವರಣಾರ್ಚನಂ ॥
\subsection{ಷಷ್ಠಾವರಣಮ್}
ಲಂ ಇಂದ್ರಾಯ ಸುರಾಧಿಪತಯೇ ನಮಃ ।ರಂ ಅಗ್ನಯೇ ತೇಜೋಽಧಿಪತಯೇ ನಮಃ ।ಟಂ ಯಮಾಯ ಪ್ರೇತಾಧಿಪತಯೇ ನಮಃ ।ಓಂ ಕ್ಷಂ ನಿರ್ಋತಯೇ ರಕ್ಷೋಧಿಪತಯೇ ನಮಃ ।ಓಂ ವಂ ವರುಣಾಯ ಜಲಾಧಿಪತಯೇ ನಮಃ ।ಓಂ ಯಂ ವಾಯವೇ ಸರ್ವಪ್ರಾಣಾಧಿಪತಯೇ ನಮಃ ।ಓಂ ಸಂ ಸೋಮಾಯ ನಕ್ಷತ್ರಾಧಿಪತಯೇ ನಮಃ ।ಓಂ ಯಂ ಈಶಾನಾಯ ವಿದ್ಯಾಧಿಪತಯೇ ನಮಃ ।ಓಂ ಐಂ ಬ್ರಹ್ಮಣೇ ಸರ್ವಲೋಕಾಧಿಪತಯೇ ನಮಃ ।ಓಂ ವಂ  ಅನಂತಾಯ ಪಾತಾಲಾ(ನಾಗಾ)ಧಿಪತಯೇ ನಮಃ ।
\subsection{ಸಪ್ತಮಾವರಣಮ್}
ಸೂರ್ಯಾಯ ಚಂದ್ರಾಯ ಅಂಗಾರಕಾಯ ಬುಧಾಯ ಗುರವೇ ಶುಕ್ರಾಯ ಶನೈಶ್ಚರಾಯ ರಾಹವೇ ಕೇತವೇ
\subsection{ಅಷ್ಟಮಾವರಣಮ್}
ಮತ್ಸ್ಯಾಯ ಕೂರ್ಮಾಯ ವರಾಹಾಯ ನಾರಸಿಂಹಾಯ ವಾಮನಾಯ ಪರಶುರಾಮಾಯ ರಾಮಾಯ ಕೃಷ್ಣಾಯ ಬೌದ್ಧಾಯ ಕಲ್ಕಿನೇ ।



ಓಂ ಶ್ರಾಂ ಹೃದಯಾಂಗ ದೇವತಾಭ್ಯೋ ನಮಃ । ಓಂ ಶ್ರೀಂ ಶಿರೋಽಂಗ ದೇವತಾಭ್ಯೋ ನಮಃ । ಓಂ ಶ್ರೂಂ ಶಿಖಾಂಗ ದೇವತಾಭ್ಯೋ ನಮಃ । ಓಂ ಶ್ರೈಂ ಕವಚಾಂಗ ದೇವತಾಭ್ಯೋ ನಮಃ । ಓಂ ಶ್ರೌಂ ನೇತ್ರಾಂಗ ದೇವತಾಭ್ಯೋ ನಮಃ। ಓಂ ಶ್ರಃ ಅಸ್ತ್ರಾಂಗ ದೇವತಾಭ್ಯೋ ನಮಃ ॥\\
ಅಭೀಷ್ಟಸಿದ್ಧಿಂ ಮೇ ದೇಹಿ ಶರಣಾಗತ ವತ್ಸಲೇ~।\\
ಭಕ್ತ್ಯಾ ಸಮರ್ಪಯೇ ತುಭ್ಯಂ ಪ್ರಥಮಾವರಣಾರ್ಚನಂ ॥

ಓಂ ವಂ ವಜ್ರಾಯ ನಮಃ । ಓಂ ಶಂ ಶಕ್ತ್ಯೈ ನಮಃ । ಓಂ ದಂ ದಂಡಾಯ ನಮಃ । ಓಂ ಖಂ ಖಡ್ಗಾಯ ನಮಃ । ಓಂ ಪಾಂ ಪಾಶಾಯ ನಮಃ । ಓಂ ಅಂ ಅಂಕುಶಾಯ ನಮಃ । ಓಂ ಗಂ ಗದಾಯೈ ನಮಃ । ಓಂ  ತ್ರಿಂ  ತ್ರಿಶೂಲಾಯ ನಮಃ । ಓಂ ಪಂ ಪದ್ಮಾಯ ನಮಃ । ಓಂ ಚಂ ಚಕ್ರಾಯ ನಮಃ ॥\\
ಅಭೀಷ್ಟಸಿದ್ಧಿಂ******ದ್ವಿತೀಯಾವರಣಾರ್ಚನಂ ॥

ಓಂ ಅಸಿತಾಂಗಭೈರವಾಯ ನಮಃ । ಓಂ ರುರುಭೈರವಾಯ ನಮಃ । ಓಂ ಚಂಡಭೈರವಾಯ ನಮಃ । ಓಂ ಕ್ರೋಧಭೈರವಾಯ ನಮಃ । ಓಂ ಉನ್ಮತ್ತಭೈರವಾಯ ನಮಃ । ಓಂ ಕಪಾಲಭೈರವಾಯ ನಮಃ । ಓಂ ಭೀಷಣಭೈರವಾಯ ನಮಃ । ಓಂ ಸಂಹಾರಭೈರವಾಯ ನಮಃ । ಓಂ ವರಪ್ರದಭೈರವಾಯ ನಮಃ ॥\\
ಅಭೀಷ್ಟಸಿದ್ಧಿಂ******ತೃತೀಯಾವರಣಾರ್ಚನಂ ॥

ಮೇಷಾಯ ವೃಷಭಾಯ ಮಿಥುನಾಯ ಕಟಕಾಯ ಸಿಂಹಾಯ ಕನ್ಯಾಯೈ ತುಲಾಯೈ ವೃಶ್ಚಿಕಾಯ ಧನುಷೇ ಮಕರಾಯ ಕುಂಭಾಯ ಮೀನಾಯ ।

ಮಾಲಾಯೈ ಕಪಾಲಾಯ ಅಸಯೇ ಕುಲಿಶಾಯ ಗದಾಯೈ ಚಕ್ರಾಯ ತ್ರಿಶೂಲಾಯ ಶಂಖಾಯ ಘಂಟಾಯ ಶಕ್ತ್ಯೈ ದಂಡಾಯ ಚರ್ಮಣೇ ಪಾನಪಾತ್ರಾಯ ಚಾಪಾಯ ಕಮಂಡಲವೇ

ಋಗ್ವೇದಾಯ ಯಜುರ್ವೇದಾಯ ಸಾಮವೇದಾಯ ಉಪನಿಷದ್ವಾಕ್ಯಾಯ ಕಲ್ಪಸೂತ್ರಾಯ ಪಾರಮಾರ್ಥಿಕಾಯ ಭಗವಚ್ಛಾಸ್ತ್ರಾಯ ಪುರಾಣಪುರುಷಾಯ ತಿಥಿವಾರಯೋಗಕರಣನಕ್ಷತ್ರಾಯ ।

ನಿವೃತ್ಯತ್ಮನೇ ಪ್ರತಿಷ್ಠಾತ್ಮನೇ ವಿದ್ಯಾತ್ಮನೇ ಶಾಂತ್ಯಾತ್ಮನೇ ಪೃಥಿವ್ಯಾತ್ಮನೇ ಅಚಲಾತ್ಮನೇ ಆಕಾಶಾತ್ಮನೇ 

ಓಂ ಬ್ರಾಹ್ಮ್ಯೈ ನಮಃ । ಓಂ ಮಾಹೇಶ್ವರ್ಯೈ ನಮಃ । ಓಂ ಕೌಮಾರ್ಯೈ ನಮಃ । ಓಂ ವೈಷ್ಣವ್ಯೈ ನಮಃ । ಓಂ ವಾರಾಹ್ಯೈ ನಮಃ । ಓಂ ಮಾಹೇಂದ್ರ್ಯೈ ನಮಃ । ಓಂ ಚಾಮುಂಡಾಯೈ ನಮಃ । ಓಂ ಮಹಾಲಕ್ಷ್ಮ್ಯೈ ನಮಃ ॥\\
ಅಭೀಷ್ಟಸಿದ್ಧಿಂ******ಷಷ್ಠಾವರಣಾರ್ಚನಂ ॥


ಓಂ ಲಾಂ ಹೃದಯಾಂಗ ದೇವತಾಭ್ಯೋ ನಮಃ । ಓಂ ಲೀಂ ಶಿರೋಽಂಗ ದೇವತಾಭ್ಯೋ ನಮಃ । ಓಂ ಲೂಂ ಶಿಖಾಂಗ ದೇವತಾಭ್ಯೋ ನಮಃ । ಓಂ ಲೈಂ ಕವಚಾಂಗ ದೇವತಾಭ್ಯೋ ನಮಃ । ಓಂ ಲೌಂ ನೇತ್ರಾಂಗ ದೇವತಾಭ್ಯೋ ನಮಃ। ಓಂ ಲಃ ಅಸ್ತ್ರಾಂಗ ದೇವತಾಭ್ಯೋ ನಮಃ ॥\\

ಶ್ರೀದೇವ್ಯೈ ಶರಣಾಗತರಕ್ಷಣಾಯೈ ದುರಿತಕ್ಷಯಕಾರಿಣ್ಯೈ ಧನಧಾನ್ಯಸಮೃದ್ಧಿದಾಯೈ ಮಧುಸೂದನವಲ್ಲಭಾಯೈ ದುರ್ಗತಿಭಂಜನಾಯೈ ವರಸದ್ಗತಿದಾಯಿಕಾಯೈ ದಿವಿದೇವಗಣಾರ್ಚಿತಾಯೈ ಭುವನಾರ್ತಿವಿನಾಶನಾಯೈ ಶಿವಸಿದ್ಧಸುಪೂಜಿತಾಯೈ ಭದ್ರಾಯೈ ಮಂಗಲದೇವತಾಯೈ ರಾಕೇಂದುವದನಾಯೈಕಾತ್ಯಾಯನ್ಯೈ ಲೋಲಲೋಚನಾಯೈ ಸರ್ವಸಂಪತ್ಕರ್ಯೈ

ಕೇಶವಾಯ ನಾರಾಯಣಾಯ ಮಾಧವಾಯ ಗೋವಿಂದಾಯ ವಿಷ್ಣವೇ ಮಧುಸೂದನಾಯ ತ್ರಿವಿಕ್ರಮಾಯ ವಾಮನಾಯ ಶ್ರೀಧರಾಯ ಹೃಷೀಕೇಶಾಯ ಪದ್ಮನಾಭಾಯ ಸಂಕರ್ಷಣಾಯ ವಾಸುದೇವಾಯ ಪ್ರದ್ಯುಮ್ನಾಯ ಅನಿರುದ್ಧಾಯ ಪುರುಷೋತ್ತಮಾಯ ಅಧೋಕ್ಷಜಾಯ  ನಾರಸಿಂಹಾಯ ಅಚ್ಯುತಾಯ ಜನಾರ್ದನಾಯ ಉಪೇಂದ್ರಾಯ ಹರಯೇ ಕೃಷ್ಣಾಯ 


ಶ್ರೀದೇವ್ಯೈ ಅಮೃತೋದ್ಭವಾಯೈ ಕಮಲಪ್ರಿಯಾಯೈ ವಿಷ್ಣುಪತ್ನ್ಯೈ ವಾರಾಹ್ಯೈ ಹರಿವಲ್ಲಭಾಯೈ ಮಹಾಲಕ್ಷ್ಮ್ಯೈ ಅಮೃತಾಂಶುಸಹೋದರ್ಯೈ ಶ್ರೀಮನ್ಮಂದಕಟಾಕ್ಷಲಬ್ಧವಿಭವಾಯೈ ಬ್ರಹ್ಮೇಂದ್ರಗಂಗಾಧರಾಯೈ ಲೋಕೈಕದೀಪಾಂಕುರಾಯೈ ಮುಕುಂದಪ್ರಿಯಾಯೈ ಕಾಶ್ಮೀರಪುರವಾಸಿನ್ಯೈ ಬ್ರಹ್ಮಾಂಡಗರ್ಭಿಣ್ಯೈ ಸರ್ವವೇದಾಂತಸಂಪತ್ತ್ಯೈ ಪುತ್ರಪೌತ್ರಾದಿವೃದ್ಧಿದಾಯೈ ಸದಾನಂದಾಯೈ ಮಹಾಲಕ್ಷ್ಮ್ಯೈ ।



ಕಾಲಾಗರು ಪ್ರಚುರು ಗುಗ್ಗುಲಂ ಗಂಧಧೂಪೈಃ ಸುರಭಿತೈಃ ಖಿಲಧೂಪಮಾನೈಃ । ತ್ವಾಂ ಧೂಪಯಾಮಿ ರಘುಪುಂಗವ ವಾಸುದೇವ ॥
ದಶಾಂಗ ಗುಗ್ಗುಲಂ ದೇವಿ ಸುಗಂಧಂ ಚ ಮನೋಹರಮ್ । ಸ್ವೀಕುರುಷ್ವ ಮಹಾಲಕ್ಷ್ಮಿ ಧೂಪಂ ಭಕ್ತ್ಯಾ ಸಮರ್ಪಯೇ ।

ಸೂರ್ಯೇಂದು ಕೋಟಿಪ್ರಭ ವಾಸುದೇವ ದೀಪಾವಲಿಂ ಗೋಘೃತವರ್ತಿಯುಕ್ತಮ್ ।
ಗೃಹಾಣಲೋಕತ್ರಯ ಪೂಜಿತಾಂಘ್ರಿ ಧರ್ಮಪ್ರದೀಪಾನ್ ಕುರುದೀಪ್ಯಮಾನಾನ್ ॥

ಘೃತವರ್ತಿ ಸಮಾಯುಕ್ತಂ ವಹ್ನಿನಾ ಯೋಜಿತಂ ಮಯಾ । ಗೃಹಾಣ ಮಂಗಲಂ ದೀಪಂ ಮಹಾಲಕ್ಷ್ಮಿ ನಮೋಽಸ್ತು ತೇ ॥

ನೈವೇದ್ಯಂ ಸಫಲಂ ಚೈವ ಭಕ್ಷ್ಯ ಭೋಜ್ಯ ಸಮನ್ವಿತಮ್ । ದಧ್ಯಾಜ್ಯ ಮಧುಸಂಯುಕ್ತಂ  ಗೃಹಾಣ ಪರಮೇಶ್ವರ ॥

ಗುಡಮುದ್ಗಾದಿ ಸಂಯುಕ್ತಂ ಪ್ರದದೇ ಕ್ಷೀರಪಾಯಸಮ್ । ಶರ್ಕರಾದಿ ಚ ನೈವೇದ್ಯಂ ಗೃಹಾಣ ಪರಮೇಶ್ವರಿ ॥

ಶ್ರೀಲಕ್ಷ್ಮೀನಾರಾಯಣಾಭ್ಯಾಂ ನಮಃ । ನೈವೇದ್ಯಂ ಸಮರ್ಪಯಾಮಿ ॥

 ಏಲೋಶೀರಲವಂಗಾದಿ ಕರ್ಪೂರಪರಿವಾಸಿತಮ್ । ಪ್ರಾಶನಾರ್ಥಂ ಕೃತಂ ತೋಯಂ ಗೃಹಾಣ ಪರಮೇಶ್ವರ ॥
 
 ಇದಂ ಶುದ್ಧಜಲಂ ದಿವ್ಯಂ ಸರ್ವತೀರ್ಥಸಮನ್ವಿತಮ್ । ಉತ್ತರಾಪೋಶನಾರ್ಥಂ ತು ಸ್ವೀಕುರುಷ್ವ ಜನಾರ್ದನ ॥
 
ಯಮುನಾಯಾ ಸಮಾನೀತಂ ಸುವರ್ಣಕಲಶೇ ಸ್ಥಿತಮ್ । ಹಸ್ತಪ್ರಕ್ಷಾಳನಾರ್ಥಂ ಚ ಸಲಿಲಂ ಪ್ರತಿಗೃಹ್ಯತಾಮ್ ॥ 
 
ಶೀತಲಂ ಚೋದಕಂ ದಿವ್ಯಂ ಗಂಗಾದಿಸರಿದಾಹೃತಂ । ಗಂಡೂಷಾರ್ಥೇ ಪ್ರದಾಸ್ಯಾಮಿ ಗೃಹಾಣ ಕಮಲಾಪತೇ ॥

ಗಂಧಾದಿ ಪರಿಮಲದ್ರವ್ಯ ವಾಸಿತಂ ಜಲಮುತ್ತಮಮ್ । ವಕ್ತ್ರಶೋಭಾವಿವೃದ್ಧ್ಯರ್ಥಂ ಕ್ಷಾಲನಾಯ ಗೃಹಾಣ ಭೋ ॥

ಶಂಖಚಕ್ರಾಂಕಶೋಭಾಢ್ಯಂ ಮುಕ್ತಿಮೂಲಂ ಪದಂ ತವ । ಕ್ಷಾಲಯಾಮಿ ಜಗನ್ನಾಥ ಗಂಗಾದಿ ಸಲಿಲೈಃ ಶುಭೈಃ ॥

ಮಲಯಾಚಲ ಸಂಭೂತಂ ಕರ್ಪೂರೇಣ ಸಮನ್ವಿತಮ್ । ಗಂಧಂ ಕರೋದ್ವರ್ತನಾಯ ಗೃಹಾಣ ಜಗತೀಪತೇ ॥

ಇದಂ ಫಲಂ ಮಯಾ ದೇವ ಸ್ಥಾಪಿತಂ ಪುರತಸ್ತವ । ತೇನ ಮೇ ಸುಫಲಾವಾಪ್ತಿರ್ಭವೇಜ್ಜನ್ಮನಿ ಜನ್ಮನಿ ॥

ಹಿರಣ್ಯಗರ್ಭಗರ್ಭಸ್ಥಂ ಹೇಮಬೀಜಂ ವಿಭಾವಸೋಃ । ಅನಂತಪುಣ್ಯ ಫಲದಮತಃ ಶಾಂತಿಂ ಪ್ರಯಚ್ಛ ಮೇ ॥

ಪೂಗೀಫಲಂ ಸಕರ್ಪೂರಂ ಲವಂಗಾದಿ ವಿಮಿಶ್ರಿತಮ್ । ನಾಗವಲ್ಲೀ ದಲೈರ್ಯುಕ್ತಂ ತಾಂಬೂಲಂ ಪ್ರತಿಗೃಹ್ಯತಾಮ್ ॥

ಜಗತಾಂ ಮಂಗಲಾರ್ಥಂ ತು ನಿರಾಜನಮಿದಂ ಹರೇ । ದರ್ಶಯಾಮಿ ರಮಾಕಾಂತ ಸಂಗೃಹಾಣ ನಮೋಽಸ್ತು ತೇ ॥

ಸುರಾಸುರಶಿರೋರತ್ನ ವಿರಾಜಿತ ಪದಾಂಬುಜೇ । ನಿರಾಜಯಾಮಿ ತ್ವಾಂ ದೇವಿ ವಾರಿಜಾಸನ ವಂದಿತೇ ॥

ಪುಷ್ಪಾಂಜಲಿಂ ಸ್ವೀಕುರು ಪುಷ್ಕರಾಕ್ಷ ಪ್ರಪನ್ನ ಕಲ್ಪದ್ರುಮಪಾರಿಜಾತ । ಇಂದ್ರಾದಿ ವೃಂದಾರಕವಂದ್ಯಪಾದ ನಮೋಽಸ್ತು ತೇ ದೇವವರ ಪ್ರಸೀದ ॥

ತುಲಸೀ ಬಿಲ್ವ ಮಂದಾರ ಪುಷ್ಪಸಂಚಯ ಕಲ್ಪಿತಂ । ಮಂತ್ರಪುಷ್ಪಂ ಅಂಜಲಿಸ್ಥಂ ಪಾದಯೋಃ ಪರಿಕಲ್ಪಯೇ ॥

ಬ್ರಹ್ಮಾದ್ಯೈರರ್ಚಿತಪದೇ ಜಾತರೂಪ ಸಮಪ್ರಭೇ । ನಾನಾಪುಷ್ಪಯುತಂ ದೇವಿ ಮಂತ್ರಪುಷ್ಪಂ ದದಾಮ್ಯಹಂ ॥





ಲಕ್ಷ್ಮೀನಾರಾಯಣಾಯ ; ಮಹಾಲಕ್ಷ್ಮ್ಯೈ ; ಧ್ಯಾನಂ
ಕೇಶವಾಯ ; ಶ್ರೀದೇವ್ಯೈ ; ಆವಾಹನಂ
ನಾರಾಯಣಾಯ ; ಅಮೃತೋದ್ಭವಾಯೈ ; ಆಸನಂ
ಮಾಧವಾಯ ; ಕಮಲಾಯೈ ; ಪಾದ್ಯಂ

ಗೋವಿಂದಾಯ ವಿಷ್ಣುಪತ್ನ್ಯೈ ಅರ್ಘ್ಯಂ
ಮಧುಸೂದನಾಯ ವಾರಾಹ್ಯೈ ಆಚಮನೀಯಂ
ತ್ರಿವಿಕ್ರಮಾಯ ಹರಿವಲ್ಲಭಾಯ ಮಧುಪರ್ಕಃ
ವಾಮನಾಯ ಮಹಾಲಕ್ಷ್ಮ್ಯೈ ಸ್ನಾನಂ
ಶ್ರೀಧರಾಯ ಅಮೃತಾಂಶುಸಹೋದರ್ಯೈ ವಸ್ತ್ರಂ
ಹೃಷೀಕೇಶಾಯ ಶ್ರೀಮನ್ಮಂದಕಟಾಕ್ಷಲಬ್ಧವಿಭವಾಯೈ ಉಪವೀತಂ
ಪದ್ಮನಾಭಾಯ ಬ್ರಹ್ಮೇಂದ್ರಗಂಗಾಧರಾಯೈ ಆಭರಣಂ
ದಾಮೋದರಾಯ ಲೋಕೈಕದೀಪಾಂಕುರಾಯೈ ಗಂಧಂ
ವಾಸುದೇವಾಯ ಕಾಶ್ಮೀರಪುರವಾಸಿನ್ಯೈ ಅಕ್ಷತಾಃ
ಜಗನ್ನಾಥಾಯ ಬ್ರಹ್ಮಾಂಡಗರ್ಭಿಣ್ಯೈ ಪುಷ್ಪಾಣಿ
ಶ್ರೀಲಕ್ಷ್ಮ್ಯೈ @  ಹರಿದ್ರಾ
ವರಲಕ್ಷ್ಮ್ಯೈ @ ಕುಂಕುಮ
ದಾಮೋದರಾಯ ಸರ್ವವೇದಾಂತಸಂಪತ್ತ್ಯೈ ಧೂಪಃ
ಅನಂತಾಯ ಮೋಕ್ಷಲಕ್ಷ್ಮ್ಯೈ ದೀಪಃ
ಅಮೃತಾಯ ಸಾಮ್ರಾಜ್ಯದಾಯಿನ್ಯೈ ನೈವೇದ್ಯಂ
ವೇದವೇದಾಂಗಾಯ ಜ್ಞಾನಲಕ್ಷ್ಮ್ಯೈ ನೀರಾಜನಂ
ಲೋಕಾಧ್ಯಕ್ಷಾಯ ಲೋಕಜನನ್ಯೈ ಮಂತ್ರಪುಷ್ಪಂ
ಹಿರಣ್ಯಗರ್ಭಾಯ ಶ್ರಿಯೈ ನಮಸ್ಕಾರಾಃ
ಪ್ರಸನ್ನಾಯ ಪದ್ಮಾಕ್ಷ್ಯೈ ಪ್ರಸನ್ನಾರ್ಘ್ಯಾಂ
ವಿಶ್ವವಂದ್ಯಾಯ ವಿಶ್ವವಂದ್ಯಾಯೈ ಪ್ರಾರ್ಥನಾ

ಛತ್ರಂ ಚಾಮರಯುಗಲಂ ಗೀತಂ ವಾದ್ಯಂ ನಾಟ್ಯಂ ಅಶ್ವಾರೋಹಣಂ ಗಜಾರೋಹಣಂ ರಥಾರೋಹಣಂ


 
 
