आख्यातं शब्दसंघातो जातिः संघातवर्तिनी ।\\एको ऽनवयवः शब्दः क्रमो बुद्ध्यनुसंहृतिः ॥१॥

पदम् आद्यं पृथक् सर्वं पदं सापेक्षम् इत्य् अपि ।\\वाक्यं प्रति मतिर् भिन्ना बहुधा न्यायदर्शिनाम् ॥२॥

निघातादिव्यवस्थार्थं शास्त्रे यत् परिभाषितम् ।\\साकाङ्क्षावयवं तेन न सर्वं तुल्यलक्षणं ॥३॥

साकाङ्क्षावयवं भेदे परानाकाङ्क्षशब्दकम् ।\\कर्मप्रधानं गुणवद् एकार्थं वाक्यम् उच्यते ॥४॥

संबोधनपदं यच् च तत् क्रियाया विशेषकम् ।\\तथा तिङन्तं तत्राहुस् तिङन्तस्य विशेषकम् ॥५॥

यथानेकम् अपि क्त्वान्तं तिङन्तस्य विशेषकम् ।\\तथा तिङन्तं तत्राहुस् तिङन्तस्य विशेषकम् ॥६॥

यथैक एव सर्वार्थ प्रकाशः प्रविभज्यते ।\\दृश्यभेदानुकारेण वाक्यार्थावगमस् तथा ॥७॥

चित्रस्यैकस्य रूपस्य यथा भेदनिदर्शनैः ।\\नीलादिभिः समाख्यानं क्रियते भिन्नलक्षणैः ॥८॥

तथैवैकस्य वाक्यस्य निराकाङ्क्षस्य सर्वतः ।\\शब्दान्तरैः समाख्यानं साकाङ्क्षैर् अनुगम्यते ॥९॥

यथा पदे विभज्यन्ते प्रकृतिप्रत्ययादयः ।\\अपोद्धारस् तथा वाक्ये पदानाम् उपपद्यते ॥१०॥

वर्णान्तरसरूपत्वं वर्णभागेषु दृष्यते ।\\पदान्तरसरूपाश् च पदभागा इव स्थिताः ॥११॥

भागैर् अनर्थकैर् युक्ता वृषभोदकयावकाः ।\\अन्वयव्यतिरेकौ तु व्यवहारनिबन्धनम् ॥१२॥

शब्दस्य न विभागो ऽस्ति कुतो ऽर्थस्य भविष्यति ।\\विभागैः प्रक्रियाभेदम् अविद्वान् प्रतिपद्यते ॥१३॥

ब्राह्मणार्थो यथा नास्ति कश् चिद् ब्राह्मणकम्बले ।\\देवदत्तादयो वाक्ये थतैव स्युर् अनर्थकाः ॥१४॥

सामान्यार्थस् तिरोभूतो न विशेषे ऽवतिष्ठते ।\\उपात्तस्य कुतस् त्यागो निवृत्तः क्वावतिष्ठताम् ॥१५॥

अशाब्दो यदि वाक्यार्थः पदार्थो ऽपि तथा भवेत् ।\\एवं सति च संबन्धः शब्दस्यार्थेन हीयते ॥१६॥

विशेशशब्दाः केषां चित् सामान्यप्रतिरूपकाः ।\\शब्दान्तराभिसंबन्धाद् व्यज्यन्ते प्रतिपत्तृषु ॥१७॥

तेषां तु कृत्स्नो वाक्यार्थः प्रतिभेदं समाप्यते ।\\व्यक्तोपव्यञ्जना सिद्धिर् अर्थस्य प्रतिपतृषु ॥१८॥

स व्यक्तः क्रमवाञ् छब्द उपांशु यम् अधीयते ।\\अक्रमस् तु वितत्येव बुद्धिर् यत्रावतिष्ठते ॥१९॥

यथोत्क्षेपविशेषे ऽपि कर्मभेदो न गृह्यते ।\\आवृत्तौ व्यज्यते जातिः कर्मभिर् भ्रमणादिभिः ॥२०॥

वर्णवाक्यपदेष्व् एवं तुल्योपव्यञ्जना श्रुतिः ।\\अत्यन्तभेदे तत्त्वस्य सरूपेव प्रतीयते ॥२१॥

नित्येषु च कुतः पूर्वं परं वा परमार्थतः ।\\एकस्यैव तु सा शक्तिर् यद् एवम् अवभासते ॥२२॥

चिरं क्षिप्रम् इति ज्ञाने कालभेदाद् ऋते यथा ।\\भिन्नकाले प्रकाशेते स धर्मो ह्रस्वदीर्घयोः ॥२३॥

न नित्यः क्रममात्राभिः कालो भेदम् इहार्हति ।\\व्यावर्तिनीनां मात्राणाम् अभावे कीदृशः क्रमः ॥२४॥

ताभ्यो या जायते बुद्धिर् एका सा भागवर्जिता ।\\सा हि स्वशक्त्या भिन्नेव क्रमप्रत्यवमर्शिनी ॥२५॥

क्रमोल्लेखानुषङ्गेण तस्यां यद् बीजम् आहितम् ।\\तत्त्वनानात्वयोस् तस्य निरुक्तिर् नावतिष्ठते ॥२६॥

भावनासमये त्व् एतत् क्रमसामर्थ्यम् अक्रमम् ।\\व्यावृत्तभेदो येनार्थो भेदवान् उपलभ्यते ॥२७॥

पदानि वाक्ये तान्य् एव वर्णास् ते च पदे यदि ।\\वर्णेषु वर्णभागानां भेदः स्यात् परमाणुवत् ॥२८॥

भागानाम् अनुपश्लेषान् न वर्णो न पदं भवेत् ।\\तेषाम् अव्यपदेश्यत्वात् किम् अन्यद् व्यपदिश्यताम् ॥२९॥

यद् अन्तःशब्दतत्त्वं तु भागैर् एकं प्रकाशितम् ।\\तम् आहुर् अपरे शब्दं तस्य वाक्ये तथैकताम् ॥३०॥

अर्थभागैस् तथा तेषाम् अन्तरो ऽर्थः प्रकाश्यते ।\\एकस्यैवात्मनो भेदौ शब्दार्थाव् अपृथक्स्थितौ ॥३१॥

प्रकाशकप्रकाश्यत्वं कार्यकारणरूपता ।\\अन्तर्मात्रात्मनस् तस्य शब्दतत्त्वस्य सर्वदा ॥३२॥

तस्यैवास्तित्वनास्तित्वे सामर्थ्ये समवस्थिते ।\\अक्रमे क्रमनिर्भासे व्यवहारनिबन्धने ॥३३॥

संप्रत्ययप्रमाणत्वात् पदार्थास्तित्वकल्पने ।\\पदार्थाभ्युच्चये त्यागाद् आनर्थक्यं प्रसज्यते ॥३४॥

राजशब्देन राजार्थो भिन्नरूपेण गम्यते ।\\वृत्ताव् आख्यातसदृशं पदम् अन्यत् प्रयुज्यते ॥३५॥

यथाश्वकर्ण इत्य् उक्ते विनैवाश्वेन गम्यते ।\\कश् चिद् एव विशिष्टो ऽर्थः सर्वेषु प्रत्ययस् तथा ॥३६॥

वाक्येषु अर्थान्तरगतः सादृश्यपरिकल्पने ।\\केषां चित् रूढिशब्दत्वं शास्त्र एवानुगम्यते ॥३७॥

उपादायापि ये हेयास् तान् उपायान् प्रचक्षते ।\\उपायानां च नियमो नावश्यम् अवतिष्ठते ॥३८॥

अर्थं कथं चित् पुरुषः कश् चित् संप्रतिपद्यते ।\\संसृष्टा वा विभक्ता चा भेदा वाक्यनिबन्धनाः ॥३९॥

सो ऽयम् इत्य् अभिसंबन्धो बुद्ध्या प्रक्रम्यते यदा ।\\वाक्यार्थस्य तदैको ऽपि वर्णः प्रत्यायकः क्व चित् ॥४०॥

केवलेन पदेनार्थो यावान् एवाभिधीयते ।\\वाक्यस्थं तावतो ऽर्थस्य तद् आहुर् अभिधायकम् ॥४१॥

संबन्धे सति यत् त्व् अन्यद् आधिक्यम् उपजायते ।\\वाक्यार्तम् एव तं प्राहुर् अनेकपदसंश्रयम् ॥४२॥

स त्व् अनेकपदस्थो ऽपि प्रतिभेदं समाप्यते ।\\जातिवत् समुदाये ऽपि संख्यावत् कल्प्यते ऽपरैः ॥४३॥

सर्वभेदानुगुण्यं तु सामान्यम् अपरे विदुः ।\\तद् अर्थान्तरसंसर्गाद् भजते भेदरूपताम् ॥४४॥

भेदान् आकाङ्क्षतस् तस्य या परिप्लवमामता ।\\अवच्छिनत्ति संबन्धस् तां विशेषे निवेशयन् ॥४५॥

कार्यानुमेयः संबन्धो रूपं तस्य न विद्यते ।\\असत्त्वभूतम् अत्यन्तम् अतस् तं प्रतिजानते ॥४६॥

नियतं साधने साध्यं क्रिया नियतसाधना ।\\स संनिधानमात्रेण नियमः संप्रकाशते ॥४७॥

गुणभावेन साकाङ्क्षं तत्र नाम प्रवर्तते ।\\साध्यत्वेन निमित्तानि क्रियापदम् अपेक्षते ॥४८॥

सन्त एव विशेषा ये पदार्थेष्व् अविभाविताः ।\\ते क्रमाद् अनुगम्यन्ते न वाक्यम् अभिधायकम् ॥४९॥

शब्दानां क्रममात्रे च नान्यः शब्दो ऽस्ति वाचकः ।\\क्रमो हि धर्मः कालस्य तेन वाक्यं न विद्यते ॥५०॥

ये च संभविनो भेदाः पदार्थेष्व् अविभाविताः ।\\ते संनिधाने व्यज्यन्ते न तु वर्णेष्व् अयं क्रमः ॥५१॥

वर्णानां च पदानां च क्रममात्रनिवेशिनी ।\\पदाख्या वाक्यसंज्ञा च शब्दत्वं नेष्यते तयोः ॥५२॥

समाने ऽपि तु शब्दत्वे दृष्टः संप्रत्ययः पदात् ।\\प्रतिवर्णं त्व् असौ नास्ति पदस्यार्थम् अतो विदुः ॥५३॥

यथा सावयवा वर्णा विना वाच्येन केन चित् ।\\अर्थवन्तः समुदिता वाक्यम् अप्य् एवम् इष्यते ॥५४॥

अनर्थकान्य् अपायत्वात् पदार्थेनार्थवन्ति वा ।\\क्रमेणोच्चरितान्य् आहुर् वाक्यार्थं भिन्नलक्षणम् ॥५५॥

नित्यत्वे समुदायानां जातेर् वा परिकल्पने ।\\एकस्यैकार्थताम् आहुर् वाक्यस्याव्यभिचारिणीम् ॥५६॥

अभेदपूर्वकाभेदाः कल्पिता वाक्यवादिभिः ।\\भेदपूर्वान् अभेदांस् तु मन्यन्ते पददर्शिनः ॥५७॥

पदप्रकृतिभावश् च वृत्तिभेदेन वर्ण्यते ।\\पदानां संहिता योनिः संहिता वा पदाश्रया ॥५८॥

पदाम्नायश् च यद्य् अन्यः संहिताया निदर्शकः ।\\नित्यस् तत्र कथं कार्यं पदं लक्षणदर्शनात् ॥५९॥

प्रतिवर्णम् असंवेद्यः पदार्थप्रत्ययो यथा ।\\पदेश्व् एवम् असंवेद्यं वाक्यार्थस्य निरूपणम् ॥६०॥

वाक्यार्थः संनिविशते पदेषु सहवृत्तिषु ।\\यथा तथैव वर्णेषु पदार्थः सहवृत्तिषु ॥६१॥

सूक्ष्मं ग्राह्यं यथान्येन संसृष्टं सह गृह्यते ।\\वर्णो ऽप्य् अन्येन वर्णेन संबद्धो वाचकस् तथा ॥६२॥

पदस्योच्चारणाद् अर्थो यथा कश् चिन् निरूप्यते ।\\वर्णानाम् अपि सांनिध्यात् तथा सो ऽर्थः प्रतीयते ॥६३॥

प्राप्तस्य यस्य सामर्थ्यान् नियमार्था पुनः श्रुतिः ।\\तेनात्यन्तं विशेषेण सामान्यं यदि बाध्यते ॥६४॥

यजेतेति ततो द्रव्यं प्राप्तं सामर्थ्यलक्षणम् ।\\व्रीहिश्रुत्या निवर्तेत न स्यात् प्रतिनिधिस् तथा ॥६५॥

तस्माद् व्रीहित्वम् अधिकं व्रीहिशब्दः प्रकल्पयेत् ।\\द्रव्यत्वम् अविरुद्धत्वात् प्राप्त्यर्थः सन् न बाधते ॥६६॥

तेन चापि व्यवच्छिन्ने द्रव्यत्वे सहचारिणि ।\\असंभवाद् विशेषाणां तत्रान्येषाम् अदर्शनम् ॥६७॥

न च सामान्यवत् सर्वे क्रियाशब्देन लक्षिताः ।\\विशेषा न हि सर्वेषां सतां शब्दो ऽभिधायकः ॥६८॥

शुक्लादयो गुणाः सन्तो यथा तत्राविवक्षिताः ।\\तथाविवक्षा भेदानां द्रव्यत्वसहचारिणाम् ॥६९॥

असंनिधौ प्रतिनिधिर् मा भून् नित्यस्य कर्मणः ।\\काम्यस्य वा प्रवृत्तस्य लोप इत्य् उपपद्यते ॥७०॥

विशिष्टैव क्रिया येन वाक्यार्थः परिकल्प्यते ।\\द्रव्याभावे प्रतिनिधौ तस्य तत् स्यात् क्रियान्तरम् ॥७१॥

निर्ज्ञातार्थं पदं यच् च तदर्थे प्रतिपादिते ।\\पिकादि यद् अविज्ञातं तत् किम् इत्य् अनुयुज्यते ॥७२॥

सामर्थ्यप्रापितं यच् च व्यक्त्यर्थम् अनुषज्यते ।\\श्रुतिर् एवानुषङ्गेण बाधिका लिङ्गवाक्ययोः ॥७३॥

अप्राप्तो यस् तु शुक्लादिः संनिधानेन गम्यते ।\\स यत्नप्रापितो वाक्ये श्रुतिधर्मविलक्षणः ॥७४॥

अभिन्नम् एव वाक्यं तु यद्य् अभिन्नार्थम् इष्यते ।\\तत् सर्वं श्रुतिभूतत्वान् न श्रुत्यैव विरोत्स्यते ॥७५॥

वाक्यानां समुदायश् च य एकार्थप्रसिद्धये ।\\साकाङ्क्षावयवस् तत्र वाक्यार्थो ऽपि न विद्यते ॥७६॥

प्रासङ्गिकम् इदं कार्यम् इदं तन्त्रेण लभ्यते ।\\इदम् आवृत्तिभेदाभ्याम् अत्र बाधसमुच्चयौ ॥७७॥

ऊहो ऽस्मिन् विषये न्याय्यः संबन्धो ऽस्य न बाध्यते ।\\सामान्यस्यातिदेशो ऽयं विशेषो ऽत्रातिदिश्यते ॥७८॥

अर्थित्वम् अत्र सामर्थ्यम् अस्मिन्न् अर्थो न भिद्यते ।\\शास्त्रात् प्राप्ताधिकारो ऽयं व्युदासो ऽस्य क्रियान्तरे ॥७९॥

इयं श्रुत्या क्रमप्राप्तिर् इयम् उच्चारणाद् इति ।\\क्रमो ऽयम् अत्र बलवान् अस्मिंस् तु न विवक्षितः ॥८०॥

इदं पराङ्गैः संबद्धम् अङ्गानाम् अप्रयोजकम् ।\\प्रयोजकम् इदं तेषाम् अत्रेदं नान्तरीयकम् ॥८१॥

इदं प्रधानं शेषो ऽयं विनियोगक्रमस् त्व् अयम् ।\\साक्षाद् अस्योपकारीदम् इदम् आराद् विशेषकम् ॥८२॥

शक्तिव्यापारभेदो ऽस्मिन् फलम् अत्र तु भिद्यते ।\\संबन्धाज् जातभेदो ऽयं भेदस् तत्राविवक्षितः ॥८३॥

प्रसज्यप्रतिषेधो ऽयं पर्युदासो ऽयम् अत्र तु ।\\इदं गौणम् इदं मुख्यं व्यापीदं गुरु लघ्व् इदम् ॥८४॥

भेदेनाङ्गाङ्गिभावो ऽस्य बहुभेदं विकल्प्यते ।\\इदं नियम्यते ऽस्यात्र योग्यत्वम् उपजायते ॥८५॥

अस्य वाक्यान्तरे दृष्टाल् लिङ्गाद् भेदो ऽनुमीयते ।\\अयं शब्दैर् अपोद्धृत्य पदार्थः प्रविभज्यते ॥८६॥

इति वाक्येषु ये धर्माः पदार्थोपनिबन्धनाः ।\\सर्वे तेन प्रकल्पेरन् पदं चेत् स्यद् अवाचकम् ॥८७॥

अविभक्ते ऽपि वाक्यार्थे शक्तिभेदाद् अपोद्धृते ।\\वाक्यान्तरविभागेन यथोक्तं न विरुध्यते ॥८८॥

यथैवैकस्य गन्धस्य भेदेन परिकल्पना ।\\पुष्पादिषु तथा वाक्ये ऽप्य् अर्थभेदो ऽभिधीयते ॥८९॥

गवये नरसिंहे वाप्य् एकज्ञानावृते यथा ।\\भागं जात्यन्तरस्यैव सदृशं प्रतिपद्यते ॥९०॥

अप्रसिद्धं तु यं भागम् अदृष्टम् अनुपश्यति ।\\तावत्य् असंविदं मूढः सर्वत्र प्रतिपद्यते ॥९१॥

तथा पिकादियोगेन वाक्ये ऽत्यन्तविलक्षणे ।\\सदृशस्येव संज्ञानम् असतो ऽर्थस्य मन्यते ॥९२॥

एकस्य भागे सादृश्यं भागे भेदश् च लक्ष्यते ।\\निर्भागस्य प्रकाशस्य निर्भागेणैव चेतसा ॥९३॥

तथैव भागे सादृश्यं भागे भेदो ऽवसीयते ।\\भागाभावे ऽपि वाक्यानाम् अत्यन्तं भिन्नधर्मणाम् ॥९४॥

रूपनाशे पदानां स्यात् कथं चावधिकल्पना ।\\अगृहीतावधौ शब्दे कथं चार्थो विविच्यते ॥९५॥

संसर्ग इव रूपाणां शब्दे ऽन्यत्र व्यवस्थितः ।\\नानारूपेषु तद्रूपं तन्त्रेणापरम् इष्यते ॥९६॥

तस्मिन्न् अभेदे भेदानां संसर्ग इव वर्तते ।\\रूपं रूपान्तरात् तस्माद् अनन्यत् प्रविभज्यते ॥९७॥

शास्त्रे प्रत्यायकस्यापि क्वचिद् एकत्वम् आश्रितम् ।\\प्रत्याय्येन क्वचिद् भेदो ग्रहणग्राह्ययोः स्थितः ॥९८॥

ऊ इत्य् अभेदम् आश्रित्य यथासंख्यं प्रकल्पितम् ।\\लृलुटोर् ग्रहणे भेदो ग्राह्याभ्यां परिकल्पितः ॥९९॥

यस्येत्य् एतद् अणो रूपं संज्ञिनाम् अभिधायकम् ।\\न हि प्रतीयमानेन ग्रहणस्यास्ति संभवः ॥१००॥

ऊ इत्य् एतद् अभिन्नं च भिन्नवाक्यनिबन्धनम् ।\\भेदेन ग्रहणं यस्य पररूपम् इव द्वयोः ॥१०१॥

प्लुतस्याङ्गविवृद्धिं च समाहारम् अचोस् तथा ।\\व्युदस्यता पुनर् भेदः शब्देष्व् अत्यन्तम् आश्रितः ॥१०२॥

अर्धर्चादिषु शब्देषु रूपभेदः क्रमाद् यथा ।\\तन्त्रात् तथैकशब्दत्वे भिन्नानां श्रुतिर् अन्यथा ॥१०३॥

संहिताविषये वर्णाः स्वरूपेणाविकारिणः ।\\शब्दान्तरत्वं यान्तीव शक्त्यन्तरपरिग्रहात् ॥१०४॥

इन्द्रियादिविकारेण दृष्टं ग्राह्येषु वस्तुषु ।\\आत्मत्यागाद् ऋते भिन्नं ग्रहणं स क्रमः श्रुतौ ॥१०५॥

अभिधानक्रियाभेदाच् छब्देष्व् अविकृतेष्व् अपि ।\\रूपम् अत्यन्तभेदेन तद् एवैकं प्रकाशते ॥१०६॥

ऋचो वा गीतिमात्रं वा साम द्रव्यान्तरं न तु ।\\गीतिभेदात् तु गृह्यन्ते ता एव विकृता ऋचः ॥१०७॥

उपायाच् छ्रुतिसंहारे भिन्नानाम् एकशेषिणाम् ।\\तन्त्रेणोच्चारणे तेषां शास्त्रे साधुत्वम् उच्यते ॥१०८॥

परिगृह्य श्रुतिं चैकां रूपभेदवताम् अपि ।\\तन्त्रेणोच्चारणं कार्यम् अन्यथा ते न साधवः ॥१०९॥

सरूपाणां च वाक्यानां शास्त्रेणाप्रतिपादितम् ।\\तन्त्रेणोच्चारणाद् एकं रूपं साधूपलभ्यते ॥११०॥

एकस्यानेकरूपत्वं नालिकादिपरिग्रहात् ।\\यथा तथैव तन्त्रात् स्याद् बहूनाम् एकरूपता ॥१११॥

यथा पदसरूपाणां वाक्यानां संभवः पृथक् ।\\तथा वाक्यान्तराभावे स्याद् एषां पृथगर्थता ॥११२॥

अभिधेयः पदस्यार्थो वाक्यस्यार्थः प्रयोजनम् ।\\यस्य तस्य न संबन्धो वाक्यानाम् उपपद्यते ॥११३॥

तत्र क्रियापदान्य् एव व्यपेक्षन्ते परस्परम् ।\\क्रियापदानुषक्तस् तु संबन्धो ऽथ प्रतीयते ॥११४॥

आवृत्तिर् अनुवादो वा पदार्थव्यक्तिकल्पने ।\\प्रत्येकं तु समाप्तो ऽर्थः सहभूतेषु वर्तते ॥११५॥

अविकल्पितवाक्यार्थे विकल्पा भावनाश्रयाः ।\\अत्राधिकरणे वादाः पूर्वेषां बहुधा मताः ॥११६॥

अभ्यासात् प्रतिभाहेतुः सर्वः शब्दो ऽपरैः स्मृतः ।\\बालानां च तिरश्चां च यथार्थप्रतिपादने ॥११७॥

अनागमश् च सो ऽभ्यासः समयः कैश् चिद् इष्यते ।\\अनन्तरम् इदं कार्यम् अस्माद् इत्य् उपदर्शकः ॥११८॥

अस्त्य् अर्थः सर्वशब्दानां इति प्रत्याय्यलक्षणम् ।\\अपूर्वदेवतास्वर्गैः समम् आहुर् गवादिषु ॥११९॥

प्रयोगदर्शनाभ्यासाद् आकारावग्रहस् तु यः ।\\न स शब्दस्य विषयः स हि यत्नान्तराश्रयः ॥१२०॥

के चिद् भेदाः प्रकाश्यन्ते शब्दैस् तदभिधायिभिः ।\\अनुनिष्पादिनः कांश् चिच् छब्दार्थान् इति मन्यते ॥१२१॥

जातेः प्रत्यायके शब्दे या व्यक्तिर् अनुषङ्गिणी ।\\न तद्व्यक्तिगतान् भेदाञ् जातिशब्दो ऽवलम्बते ॥१२२॥

घटादीनां न चाकारान् प्रत्याययति वाचकः ।\\वस्तुमात्रनिवेशित्वात् तद्गतिर् नान्तरीयका ॥१२३॥

क्रिया विना प्रयोगेण न दृष्टा शब्दचोदिता ।\\प्रयोगस् त्व् अनुनिष्पादी शब्दार्थ इति गम्यते ॥१२४॥

नियतास् तु प्रयोगा ये नियतं यच् च साधनम् ।\\तेषां शब्दाभिधेयत्वम् अपरैर् अनुगम्यते ॥१२५॥

समुदायो ऽभिधेयो वाप्य् अविकल्पसमुच्चयः ।\\असत्यो वापि संसर्गः शब्दार्थः कैश् चिद् इष्यते ॥१२६॥

असत्योपाधि यत् सत्यं तद् वा शब्दनिबन्धनाम् ।\\शब्दो वाप्य् अभिजल्पत्वम् आगतो याति वाच्यताम् ॥१२७॥

सो ऽयम् इत्य् अभिसंबन्धाद् रूपम् एकीकृतं यता ।\\शब्दस्यार्थेन तं शब्दम् अभिजल्पं प्रचक्षते ॥१२८॥

तयोर् अपृथगात्मत्वे रूढिर् अव्यभिचारिणी ।\\किं चिद् एव क्व चिद् रूपं प्राधान्येनावतिष्ठते ॥१२९॥

लोके ऽर्थरूपतां शब्दः प्रतिपन्नः प्रवर्तते ।\\शास्त्रे तूभयरूपत्वं प्रविभक्तं विवक्षया ॥१३०॥

अशक्तेः सर्वशक्तेर् वा शब्दैर् एव प्रकल्पिता ।\\एकस्यार्थस्य नियता क्रियादिपरिकल्पना ॥१३१॥

यो वार्थो बुद्धिविषयो बाह्यवस्तुनिबन्धनः ।\\स बाह्यं वस्त्व् इति ज्ञातः शब्दार्थ इति गम्यते ॥१३२॥

आकारवन्तः संवेद्या व्यक्तिस्मृतिनिबन्धनाः ।\\एते प्रत्यवभासन्ते संविन्ंआत्रं त्व् अतो ऽन्यथा ॥१३३॥

यथेन्द्रियं संनिपतद् वैचित्रेणोपदर्शकं ।\\तथैव शब्दाद् अर्थस्य प्रतिपत्तिर् अनेकधा ॥१३४॥

वक्त्रान्यथैव प्रक्रान्तो भिन्नेषु प्रतिपत्तृषु ।\\स्वप्रत्ययानुकारेण शब्दार्थः प्रविभज्यते ॥१३५॥

एकस्मिन्न् अपि दृश्ये ऽर्थे दर्शनं भिद्यते पृथक् ।\\कालान्तरेण चैको ऽपि तं पश्यत्य् अन्यथा पुनः ॥१३६॥

एकस्यापि च शब्दस्य निमित्तैर् अव्यवस्थितैः ।\\एकेन बहुभिश् चार्थो बहुधा परिकल्प्यते ॥१३७॥

तस्माद् अदृष्टतत्त्वानां सापराधं बहुच्छलं ।\\दर्शनं वचनं वापि नित्यम् एवानवस्थितम् ॥१३८॥

ऋषीणां दर्शनं यच् च तत्त्वे किं चिद् अवस्थितम् ।\\न तेन व्यवहारो ऽस्ति न तच् छब्दनिबन्धनं ॥१३९॥

तलवद् दृश्यते व्योम खद्योतो हव्यवाड् इव ।\\नैव चास्ति तलं व्योम्नि न खद्योते हुताशनः ॥१४०॥

तस्मात् प्रत्यक्षम् अप्य् अर्थं विद्वान् ईक्षेत युक्तितः ।\\न दर्शनस्य प्रामाण्याद् दृश्यम् अर्थं प्रकल्पयेत् ॥१४१॥

असमाख्येयतत्त्वानाम् अर्थानां लौकिकैर् यथा ।\\व्यवहारे समाख्यानं तत् प्रज्ञो न विकल्पयेत् ॥१४२॥

विच्छेदग्रहणे ऽर्थानां प्रतिभान्यैव जायते ।\\वाक्यार्थ इति ताम् आहुः पदार्थैर् उपपादिताम् ॥१४३॥

इदं तद् इति सान्येषाम् अनाक्येया कथं चन ।\\प्रत्यात्मवृत्ति सिद्धा सा कर्त्रापि न निरूप्यते ॥१४४॥

उपश्लेषम् इवार्थानां सा करोत्य् अविचारिता ।\\सार्वरूप्यम् इवापन्ना विषयत्वेन वर्तते ॥१४५॥

साक्शाच् छब्देन जनितां भावनानुगमेन वा ।\\इतिकर्तव्यतायां तां न कश् चिद् अतिवर्तते ॥१४६॥

प्रमाणत्वेन तां लोकः सर्वः समनुगच्छति ।\\समारम्भाः प्रतायन्ते तिरश्चाम् अपि तद्वशात् ॥१४७॥

यथा द्रव्यविशेषाणां परिपाकैर् अयत्नजाः ।\\मदादिशक्तयो दृष्टाः प्रतिभास् तद्वतां तथा ॥१४८॥

स्वरवृत्तिं विकुरुते मधौ पुंस्कोकिलस्य कः ।\\जन्त्वादयः कुलायादि करणे शिक्षिताः कथम् ॥१४९॥

आहारप्रीत्यपद्वेष प्लवनादिक्रियासु कः ।\\जात्यन्वयप्रसिद्धासु प्रयोक्ता मृगपक्षिणाम् ॥१५०॥

भावनानुगताद् एतद् आगमाद् एव जायते ।\\आसत्तिविप्रकर्षाभ्याम् आगमस् तु विशिष्यते ॥१५१॥

स्वभाववरणाभास योगादृष्टोपपादिताम् ।\\विशिष्टोपहितां चेति प्रतिभां षड्विधां विदुः ॥१५२॥

यथा संयोगिभिर् द्रव्यैर् लक्षिते ऽर्थे प्रयुज्यते ।\\गोशब्दो न त्व् असौ तेषां विशेशाणां प्रकाशकः ॥१५३॥

आकारवर्णावयवैः संसृष्टेषु गवादिषु ।\\शब्दः प्रवर्तमानो ऽपि न तान् अङ्गीकरोत्य् असौ ॥१५४॥

संस्थानवर्णावयवैर् विशिष्टे ऽर्थे प्रयुज्यते ।\\शब्दो न तस्यावयवे प्रवृत्तिर् उपलभ्यते ॥१५५॥

दुर्लभं कस्य चिल् लोके सर्वावयवदर्शनं ।\\कैश् चित् त्व् अवयवैर् दृष्टैर् अर्थः कृत्सो ऽनुमीयते ॥१५६॥

तथा जात्युत्पलादीनां गन्धेन सहचारिणाम् ।\\नित्यसंबन्धिनां दृष्टं गुणानाम् अवधारणम् ॥१५७॥

संख्याप्रमाणसंस्थान निरपेक्षः प्रवर्तते ।\\बिन्दौ च समुदाये च वाचकः सलिलादिषु ॥१५८॥

संस्कारादिपरिच्छिन्ने तैलादौ यो व्यवस्थितः ।\\आहैकदेशं तत्त्वेन तस्यावयववर्तिना ॥१५९॥

येनार्थेनाभिसंबद्धम् अभिधानं प्रयुज्यते ।\\तदर्थापगमे तस्य प्रयोगो विनिवर्तते ॥१६०॥

यांस् तु संभविनो धर्मान् अन्तर्णीय प्रयुज्यते ।\\शब्दस् तेषां न सांनिध्यं नियमेन व्यपेक्षते ॥१६१॥

यथा रोमशफादीनां व्यभिचारे ऽपि दृश्यते ।\\गोशब्दो न तथा जातेर् विप्रयोगे प्रवर्तते ॥१६२॥

तस्मात् संभविनो ऽर्थस्य शब्दात् संप्रत्यये सति ।\\अदृष्टविप्रयोगार्थः संबन्धित्वेन गम्यते ॥१६३॥

वाचिका द्योतिका व स्युर् द्वित्वादीनां विभक्तयः ।\\स्याद् वा संख्यावतो ऽर्थस्य समुदायो ऽभिधायकः ॥१६४॥

विना संख्याभिधानाद् वा संख्याभेदसमन्वितान् ।\\अर्थान् स्वरूपभेदेन काम्श् चिद् आहुर् गवादयः ॥१६५॥

ये शब्दा नित्यसंबन्धा विवेके ज्ञातशक्तयः ।\\अन्वयव्यतिरेकाभ्यां तेषाम् अर्थो विभज्यते ॥१६६॥

यावच् चाव्यभिचारेण तयोः शक्यं प्रकल्पनम् ।\\नियमस् तत्र न त्व् एवं नियमो नुट्शबादिषु ॥१६७॥

संभवे नाभिधानस्य लक्षणत्वं प्रकल्पते ।\\आपेक्षिक्यो हि संसर्गे नियताः शब्दशक्तयः ॥१६८॥

न कूपसूपयूपानाम् अन्वयो ऽर्थस्य दृश्यते ।\\अतो ऽर्थान्तरवाचित्वं संघातस्यैव गम्यते ॥१६९॥

अन्वाख्यानानि भिद्यन्ते शब्दव्युत्पत्तिकर्मसु ।\\बहूनां संभवे ऽर्थानां निमित्तं किं चिद् इष्यते ॥१७०॥

वैरवासिष्ठगिरिशास् तथैकागारिकादयः ।\\कैश् चित् कथं चिद् आख्याता निमित्तावधिसंकरैः ॥१७१॥

यथा पथः समाख्यानं वृक्षवल्मीकपर्वतैः ।\\अविरुद्धं गवादीनां भिन्नैश् च सहचारिभिः ॥१७२॥

अन्यथा च समाख्यानम् अवस्थाभेददर्शिभिः ।\\क्रियते किंशुकादीनाम् एकदेशावधारणं ॥१७३॥

कैश् चिन् निर्वचनं भिन्नं गिरतेर् गर्जतेर् गमेः ।\\गवतेर् गदतेर् वापि गौर् इत्य् अत्रानुदर्शितम् ॥१७४॥

गौर् इत्य् एव स्वरूपाद् वा गोशब्दो गोषु वर्तते ।\\व्युत्पाद्यते न वा सर्वं कैश् चिच् चोभयथेष्यते ॥१७५॥

सामान्येनोपदेशश् च शास्त्रे लघ्वर्थम् आश्रितः ।\\जात्यन्तरवद् अन्यस्य विशेषाः प्रतिपादकाः ॥१७६॥

अर्थान्तरे च यद् वृत्तं तत् प्रकृत्यन्तरं विदुः ।\\तुल्यरूपं न तद् रूढाव् अन्यस्मिन्न् अनुषज्यते ॥१७७॥

भिन्नाव् इजियजी धातू नियतौ विषयान्तरे ।\\कैश् चित् कथं चिद् उद्दिष्टौ चित्रं हि प्रतिपादनम् ॥१७८॥

एवं च वालवायादि जित्वरीवद् उपाचरेत् ।\\भेदाभेदाभ्युपगमे न विरोधो ऽस्ति कश् चन ॥१७९॥

अडादीनां व्यवस्थार्थं पृथक्त्वेन प्रकल्पनम् ।\\धातूपसर्गयोः शास्त्रे धातुर् एव तु तादृशः ॥१८०॥

तथा हि संग्रामयतेः सोपसर्गाद् विधिः स्मृतः ।\\क्रियाविशेषाः सम्घाते प्रक्रम्यन्ते तथाविधाः ॥१८१॥

कार्याणाम् अन्तरङ्गत्वम् एवं धातूपसर्गयोः ।\\साधनैर् याति संबन्धं तथाभूतैव सा क्रिया ॥१८२॥

प्रयोगार्थेषु सिद्धः सन् भेत्तव्यो ऽर्थो विशिष्यते ।\\प्राक् च साधनसंबन्धात् क्रिया नैवोपजायते ॥१८३॥

धातोः साधनयोगस्य भाविनः प्रक्रमाद् यथा ।\\धातुत्वं कर्मभावश् च तथान्यद् अपि दृश्यताम् ॥१८४॥

बीजकालेषु संबन्धाद् यथा लाक्षारसादयः ।\\वर्णादिपरिणामेन फलानाम् उपकुर्वते ॥१८५॥

बुद्धिस्थाद् अभिसंबन्धात् तथा धातूपसर्गयोः ।\\अभ्यन्तरीकृताद् भेदः पदकाले प्रकाशते ॥१८६॥

क्व चित् संभविनो भेदाः केवलैर् अनिदर्शिताः ।\\उपसर्गेण संबन्धे व्यज्यन्ते प्रनिरादिना ॥१८७॥

स वाचको विशेषाणां संभवाद् द्योतको ऽपि वा ।\\शक्त्याधानाय वा धातोः सहकारी प्रयुज्यते ॥१८८॥

स्थादिभिः केवलैर् यच् च गमनादि न गम्यते ।\\तत्रानुमानाद् द्विविधात् तद्धर्मा प्रादिर् उच्यते ॥१८९॥

अप्रयोगे ऽधिपर्योश् च यावद् दृष्टं क्रियान्तरम् ।\\तस्याभिधायको धातुः सह ताभ्याम् अनर्थकः ॥१९०॥

तथैव स्वार्थिकाः के चित् संघातान्तरवृत्तयः ।\\अनर्थकेन संसृष्टाः प्रकृत्यर्थानुवादिनः ॥१९१॥

निपाता द्योतकाः के चित् पृथगर्थप्रकल्पने ।\\आगमा इव के चित् तु संभूयार्थस्य साधकाः ॥१९२॥

उपरिष्टात् पुरस्ताद् वा द्योतकत्वं न भिद्यते ।\\तेषु प्रयुज्यमानेषु भिन्नार्थेष्व् अपि सर्वथा ॥१९३॥

चादयो न प्रयुज्यन्ते पदत्वे सति केवलाः ।\\प्रत्ययो वाचकत्वे ऽपि केवलो न प्रयुज्यते ॥१९४॥

समुच्चिताभिधाने तु व्यतिरेको न विद्यते ।\\असत्त्वभूतो भावश् च तिङ्पदैर् अभिधीयते ॥१९५॥

समुच्चिताभिधाने ऽपि विशिष्टार्थाभिधायिनाम् ।\\गुणैर् पदानां संबन्धः परतन्त्रास् तु चादयः ॥१९६॥

जनयित्वा क्रिया का चित् संबन्धं विनिवर्तते ।\\श्रूयमाणे क्रियाशब्दे संबन्धो जायते क्व चित् ॥१९७॥

तत्र षष्ठी प्रतिपदं समासस्य निवृत्तये ।\\विहिता दर्शनार्थं तु कारकं प्रत्युदाहृतम् ॥१९८॥

स चोपजातः संबन्धो विनिवृत्ते क्रियापदे ।\\कर्मप्रवचनीयेन तत्र तत्र नियम्यते ॥१९९॥

येन क्रियापदाक्षेपः स कारकविभक्तिभिः ।\\युज्यते विर् यथा तस्य लिखाव् अनुपसर्गता ॥२००॥

तिष्ठतेर् अप्रयोगश् च दृष्टो ऽप्रत्य् अजयन्न् इति ।\\सुन्व् अभीत्य् आभिमुख्ये च केवलो ऽपि प्रयुज्यते ॥२०१॥

कर्मप्रवचनीयत्वं क्रियायोगे विधीयते ।\\षत्वादिविनिवृत्त्यर्थं स्वत्यादीनां विधर्मणाम् ॥२०२॥

हेतुहेतुमतोर् योग परिच्छेदे ऽनुना कृते ।\\आरम्भाद् बाध्यते प्राप्ता तृतीया हेतुलक्षणा ॥२०३॥

क्रियाया द्योतको नायं न संबन्धस्य वाचकः ।\\नापि क्रियापदाक्षेपि संबन्धस्य तु भेदकः ॥२०४॥

अनर्थकानां संघातः सार्थको ऽनर्थकस् तथा ।\\वर्णानां पदम् अर्थेन युक्तं नावयवाः पदे ॥२०५॥

पदानाम् अर्थयुक्तानां संघातो भिद्यते पुनः ।\\अर्थान्तरावबोधेन संबन्धविगमेन च ॥२०६॥

सार्थकानर्थकौ भेदे संबन्धं नाधिगच्छतः ।\\अधिगच्छत इत्य् एके कुटीरादिनिदर्शनात् ॥२०७॥

अर्थवद्भ्यो विशिष्टार्थः संघात उपजायते ।\\नोपजायत इत्य् एके समासस्वार्थिकादिषु ॥२०८॥

के चिद् धि युतसिद्धार्था भेदे निर्ज्ञातशक्तयः ।\\अन्वयव्यतिरेकाभ्यां के चित् कल्पितशक्तयः ॥२०९॥

शास्त्रार्थ एव वर्णानाम् अर्थवत्त्वे प्रदर्शितः ।\\धात्वादीनां हि शुद्धानां लौकिको ऽर्थो न विद्यते ॥२१०॥

कृत्तद्धितानाम् अर्थश् च केवलानाम् अलौकिकः ।\\प्राग् विभक्तेस् तदन्तस्य तथैवार्थो न विद्यते ॥२११॥

अभिव्यक्ततरो यो ऽर्थः प्रत्ययान्तेषु लक्ष्यते ।\\अर्थवत्ताप्रकरणाद् आश्रितः स तथाविधः ॥२१२॥

आत्मभेदो न चेत् कश् चिद् वर्णेभ्यः पदवाक्ययोः ।\\अन्योन्यापेक्षया शक्त्या वर्णः स्याद् अभिधायकः ॥२१३॥

वर्णेन केन चिन् न्यूनः संघातो यो ऽभिधायकः ।\\न चेच् छब्दान्तरम् असाव् अन्यूनस् तेन गम्यते ॥२१४॥

स तस्मिन् वाचके शब्दे निमित्तात् स्मृतिम् आदधत् ।\\साक्षाद् इव व्यवहितं शब्देनार्थम् उपोहते ॥२१५॥

पदवाच्यो यथा नार्थः कश् चिद् गौरखरादिषु ।\\सत्य् अपि प्रत्यये ऽत्यन्तं समुदाये न गम्यते ॥२१६॥

समन्वित इवार्थात्मा पदार्थैर् यः प्रतीयते ।\\पदार्थदर्शनं तत्र तथैवानुपकारकम् ॥२१७॥

समुदायावयवयोर् भिन्नार्थत्वे च वृत्तिषु ।\\युगपद् भेदसंसर्गौ विरुद्धाव् अनुषङ्गिणौ ॥२१८॥

कश् च साधनमात्रार्थान् अध्यादीन् परिकल्पयेत् ।\\अप्रयुक्तपदश् चार्थो बहुव्रीहौ कथं भवेत् ॥२१९॥

प्रज्ञुसंज्ञ्वाद्यवयवैर् न चास्त्य् अर्थावधारणम् ।\\तस्मात् संघात एवैको विशिष्टार्थनिबन्धनम् ॥२२०॥

गर्गा इत्य् एक एवायं बहुष्व् अर्थेषु वर्तते ।\\द्वन्द्वसंज्ञो ऽपि संघातो बहूनाम् अभिधायकः ॥२२१॥

यथैकशेषे भुज्यादिः प्रत्येकम् अवतिष्ठते ।\\क्रियैवं द्वन्द्ववाच्ये ऽर्थे प्रत्येकं प्रविभज्यते ॥२२२॥

यच् च द्वन्द्वपदार्थस्य तच्छब्देन व्यपेक्षणम् ।\\सापि व्यावृत्तरूपे ऽर्थे सर्वनामसरूपता ॥२२३॥

यथा च खदिरच्छेदे भागेषु क्रमवांस् छिदिः ।\\तथा द्वन्द्वपदार्थस्य भागेषु क्रमदर्शनम् ॥२२४॥

सङ्घैकदेशे प्रक्रान्तान् यथा सङ्घानुपातिनः ।\\क्रियाविशेषान् मन्यन्ते स द्वन्द्वावयवे क्रमः ॥२२५॥

प्रतिपादयता वृत्तिम् अबुद्धान् वाक्यपूर्विकाम् ।\\वृत्तौ पदार्थभेदेन प्राधान्यम् उपदर्शितम् ॥२२६॥

अभेदाद् अभिधेयस्य नञ्समासे विकल्पितम् ।\\प्राधान्यं बहुधा भाष्ये दोषास् तु प्रक्रियागताः ॥२२७॥

जहत्स्वार्थविकल्पे च सर्वार्थत्यागम् इच्छता ।\\बहुव्रीहिपदार्थस्य त्यागः सर्वस्य दर्शितः ॥२२८॥

शास्त्रे क्व चित् प्रकृत्यर्थः प्रत्ययेनाभिधीयते ।\\प्रकृतौ विनिवृत्तायां प्रत्ययार्थश् च धातुभिः ॥२२९॥

यम् अर्थम् आहतुर् भिन्नौ प्रत्ययाव् एक एव तम् ।\\क्व चिद् आह पचन्तीति धातुस् ताभ्यां विना क्व चित् ॥२३०॥

अन्वाख्यानस्मृतेर् ये च प्रत्ययार्था निबन्धनम् ।\\निर्दिष्टास् ते प्रकृत्यर्थाः स्मृत्यन्तर उदाहृताः ॥२३१॥

प्रसिद्धेर् उद्वमिकरीत्य् एवं शास्त्रे ऽभिधीयते ।\\व्यवहाराय मन्यन्ते शास्त्रार्थप्रक्रिया यतः ॥२३२॥

शास्त्रेषु प्रक्रियाभेदैर् अविद्यैवोपवर्ण्यते ।\\अनागमविकल्पा तु स्वयं विद्योपवर्तते ॥२३३॥

अनिबद्धं निमित्तेषु निरुपाख्यं फलं यथा ।\\तथा विद्याप्य् अनाख्येया शास्त्रोपायेव लक्ष्यते ॥२३४॥

यथाभ्यासं हि वाग् अर्थे प्रतिपत्तिं समीहते ।\\स्वभाव इव चानादिर् मिथ्याभ्यासो व्यवस्थितः ॥२३५॥

उत्प्रेक्षते सावयवं परमाणुम् अपण्डितः ।\\तथावयविनं युक्तम् अन्यैर् अवयवैः पुनः ॥२३६॥

घटादिदर्शनाल् लोकः परिच्छिन्नो ऽवसीयते ।\\समारम्भाच् च भावानाम् आदिमद् ब्रह्म शाश्वतम् ॥२३७॥

उपायाः शिक्षमाणानां बालानाम् उपलापनाः ।\\असत्ये वर्त्मनि स्थित्वा ततः सत्यं समीहते ॥२३८॥

अन्यथा प्रतिपद्यार्थं पदग्रहणपूर्वकम् ।\\पुनर् वाक्ये तम् एवार्थम् अन्यथा प्रतिपद्यते ॥२३९॥

उपात्ता बहवो ऽप्य् अर्था येष्व् अन्ते प्रतिषेधनम् ।\\क्रियते ते निवर्तन्ते तस्मात् तांस् तत्र नाश्रयेत् ॥२४०॥

वृक्षो नास्तीति वाक्यं च विशिष्टाभावलक्षणम् ।\\नार्थे न बुद्धौ संबन्धो निवृत्तेर् अवतिष्ठते ॥२४१॥

विच्छेदप्रतिपत्तौ च यद्य् अस्तीत्य् अवधार्यते ।\\अशब्दवाच्या सा बुद्धिर् निवर्त्येत स्थिता कथम् ॥२४२॥

अथ यज् ज्ञानम् उत्पन्नं तन् मिथ्येति नञा कृतम् ।\\नञो व्यापारभेदे ऽस्मिन्न् अभावावगतिः कथम् ॥२४३॥

निराधारप्रवृत्तौ च प्राक्प्रवृत्तिर् नञो भवेत् ।\\अथाधारः स एवास्य नियमार्था श्रुतिर् भवेत् ॥२४४॥

नियमद्योतनार्था वाप्य् अनुवादो यथा भवेत् ।\\कश् चिद् एवार्थवांस् तत्र शब्दः शेषास् त्व् अनर्थकाः ॥२४५॥

विरुद्धं चाभिसंबन्धम् उदाहार्यादिभिः कृतम् ।\\वाक्ये समाप्ते वाक्यार्थम् अन्यथा प्रतिपद्यते ॥२४६॥

स्तुतिनिन्दाप्रधानेषु वाक्येष्व् अर्थो न तादृशः ।\\पदानां प्रविभागेन यादृशः परिकल्प्यते ॥२४७॥

अथासंसृष्ट एवार्थः पदेषु समवस्थितः ।\\वाक्यार्थस्याभ्युपायो ऽसाव् एकस्य प्रतिपादने ॥२४८॥

पूर्वं पदेष्व् असंसृष्टो यः क्रमाद् उपचीयते ।\\छिन्नग्रथितकल्पत्वात् तद् विशिष्टतरं विदुः ॥२४९॥

एकम् आहुर् अनेकार्थं शब्दम् अन्ये परीक्षकाः ।\\निमित्तभेदाद् एकस्य सार्वार्थ्यं तस्य भिद्यते ॥२५०॥

यौगपद्यम् अतिक्रम्य पर्याये व्यवतिष्ठते ।\\अर्थप्रकरणाभ्यां वा योगाच् छब्दान्तरेण वा ॥२५१॥

यथा सास्नादिमान् पिण्डो गोशब्देनाभिधीयते ।\\तथा स एव गोशब्दो वाहीके ऽपि व्यवस्थितः ॥२५२॥

सर्वशक्तेस् तु तस्यैव शब्दस्यानेकधर्मणः ।\\प्रसिद्धिभेदाद् गौणत्वं मुख्यत्वं चोपजायते ॥२५३॥

एको मन्त्रस् तथाध्यात्मम् अधिदैवम् अधिक्रतु ।\\असंकरेण सर्वार्थो भिन्नशक्तिर् अवस्थितः ॥२५४॥

गोत्वानुषङ्गो वाहीके निमित्तात् कैश् चिद् इष्यते ।\\अर्थमात्रं विपर्यस्तं शब्दः स्वार्थे व्यवस्थितः ॥२५५॥

तथा स्वरूपं शब्दानां सर्वार्थेष्व् अनुषज्यते ।\\अर्थमात्रं विपर्यस्तं स्वरूपे तु श्रुतिः स्थिता ॥२५६॥

एकत्वं तु सरूपत्वाच् छब्दयोर् गौणमुख्ययोः ।\\प्राहुर् अत्यन्तभेदे ऽपि भेदमार्गानुदर्शिनः ॥२५७॥

सामिधेन्यन्तरं चैवम् आवृत्ताव् अनुषज्यते ।\\मन्त्रास् च विनियोगेन लभन्ते भेदम् ऊहवत् ॥२५८॥

तान्य् आम्नायान्तराण्य् एव पठ्यते किं चिद् एव तु ।\\अनर्थकानां पाठो वा शेषस् त्व् अन्यः प्रतीयते ॥२५९॥

शब्दस्वरूपम् अर्थस् तु पाठे ऽन्यैर् उपवर्ण्यते ।\\अत्यन्तभेदः सर्वेषां तत्संबन्धात् तु तद्वताम् ॥२६०॥

अन्या संस्कारसावित्री कर्मण्य् अन्या प्रयुज्यते ।\\अन्या जपप्रबन्धेषु सा त्व् एकैव प्रतीयते ॥२६१॥

अर्थस्वरूपे शब्दानां स्वरूपाद् वृत्तिम् इच्छतः ।\\वाक्यरूपस्य वाक्यार्थे वृत्तिर् अन्यानपेक्षया ॥२६२॥

अनेकार्थत्वम् एकस्य यैः शब्दस्यानुगम्यते ।\\सिद्ध्यसिद्धिकृता तेषां गौणमुख्यप्रकल्पना ॥२६३॥

अर्थप्रकरणापेक्षो यो वा शब्दान्तरैः सह ।\\युक्तः प्रत्याययत्य् अर्थं तं गौणम् अपरे विदुः ॥२६४॥

शुद्धस्योच्चारणे स्वार्थः प्रसिद्धो यस्य गम्यते ।\\स मुख्य इति विज्ञेयो रूपमात्रनिबन्धनः ॥२६५॥

यस् त्व् अन्यस्य प्रयोगेण यत्नाद् इव नियुज्यते ।\\तम् अप्रसिद्धं मन्यन्ते गौणार्थाभिनिवेशिनम् ॥२६६॥

स्वार्थे प्रवर्तमानो ऽपि यस्यार्थं यो ऽवलम्बते ।\\निमित्तं तत्र मुख्यं स्यान् निमित्ती गौण इष्यते ॥२६७॥

पुराराद् इति भिन्ने ऽर्थे यौ वर्तेते विरोधिनि ।\\अर्थप्रकरणापेक्षं तयोर् अप्य् अवधारणम् ॥२६८॥

वाक्यस्यार्थात् पदार्थानाम् अपोद्धारे प्रकल्पिते ।\\शब्दान्तरेण संबन्धः कस्यैकस्योपपद्यते ॥२६९॥

यच् चाप्य् एकं पदं दृष्टं चरितास्तिक्रियं क्व चित् ।\\तद् वाक्यान्तरम् एवाहुर् न तद् अन्येन युज्यते ॥२७०॥

यच् च को ऽयम् इति प्रश्ने गौर् अश्व इति चोच्यते ।\\प्रश्न एव क्रिया तत्र प्रक्रान्ता दर्शनादिका ॥२७१॥

नैवाधिकत्वं धर्माणां न्यूनता वा प्रयोजिका ।\\आधिक्यम् अपि मन्यन्ते प्रसिद्धेर् न्यूनतां क्व चित् ॥२७२॥

जातिशब्दो ऽन्तरेणापि जातिं यत्र प्रयुज्यते ।\\संबन्धिसदृशाद् धर्मात् तं गौणम् अपरे विदुः ॥२७३॥

विपर्यासाद् इवार्थस्य यत्रार्थान्तरताम् इव ।\\मन्यन्ते स गवादिस् तु गौण इत्य् उच्यते क्व चित् ॥२७४॥

नियताः साधनत्वेन रूपशक्तिसमन्विताः ।\\यथा कर्मसु गम्यन्ते सीरासिमुसलादयः ॥२७५॥

क्रियान्तरे न चैतेषां विभवन्ति न शक्तयः ।\\रूपाद् एव तु तादर्थ्यं नियमेन प्रतीयते ॥२७६॥

तथैव रूपशक्तिभ्याम् उत्पत्त्या समवस्थितः ।\\शब्दो नियततादर्थ्यः शक्त्यान्यत्र प्रयुज्यते ॥२७७॥

श्रुतिमात्रेण यत्रास्य सामर्थ्यम् अवसीयते ।\\तं मुख्यम् अर्थं मन्यन्ते गौणं यत्नोपपादितम् ॥२७८॥

गोयुष्मन्महतां च्व्यर्थे स्वार्थाद् अर्थान्तरे स्थितौ ।\\अर्थान्तरस्य तद्भावस् तत्र मुख्यो ऽपि दृश्यते ॥२७९॥

महत्त्वं शुक्लभावं च प्रकृतिः प्रतिपद्यते ।\\भेदेनापेक्षिता सा तु गौणत्वस्य प्रसाधिका ॥२८०॥

अग्निसोमादयः शब्दा ये स्वरूपपदार्थकाः ।\\संज्ञिभिः संप्रयुज्यन्ते ऽप्रसिद्धेस् तेषु गौणता ॥२८१॥

अग्निदत्तस् तु यो ऽग्निः स्यात् तत्र स्वार्थोपसर्जनः ।\\शब्दो दत्तार्थवृत्तित्वाद् गौणत्वं प्रतिपद्यते ॥२८२॥

निमित्तभेदात् प्रक्रान्ते शब्दव्युत्पत्तिकर्मणि ।\\हरिश्चन्द्रादिषु सुटो भावाभावौ व्यवस्थितौ ॥२८३॥

ऋष्यादौ प्राप्तसंस्कारो यः शब्दो ऽन्येन युज्यते ।\\तत्रान्तरङ्गसंस्कारो बाह्ये ऽर्थे न निवर्तते ॥२८४॥

अत्यन्तविपरीतो ऽपि यथा यो ऽर्थो ऽवधार्यते ।\\यथासंप्रत्ययं शब्दस् तत्र मुख्यः प्रयुज्यते ॥२८५॥

यद्य् अपि प्रत्ययाधीनम् अर्थतत्त्वावधारणम् ।\\न सर्वः प्रत्ययस् तस्मिन् प्रसिद्ध इव जायते ॥२८६॥

दर्शनं सलिले तुल्यं मृगतृष्णादिदर्शनैः ।\\भेदात् तु स्पर्शनादीनां न जलं मृगतृष्णिका ॥२८७॥

यद् असाधारणं कार्यं प्रसिद्धं रज्जुसर्पयोः ।\\तेन भेदपरिच्छेदस् तयोस् तुल्ये ऽपि दर्शने ॥२८८॥

प्रसिद्धार्थविपर्यास निमित्तं यच् च दृश्यते ।\\यस् तस्माल् लक्ष्यते भेदस् तम् असत्यं प्रचक्षते ॥२८९॥

यच् च निम्नोन्नतं चित्रे सरूपं पर्वतादिभिः ।\\न तत्र प्रतिघातादि कार्यं तद्वत् प्रवर्तते ॥२९०॥

स्पर्शप्रबन्धो हस्तेन यथा चक्रस्य संततः ।\\न तथालातचक्रस्य विच्छिन्नं स्पृश्यते हि तत् ॥२९१॥

वप्रप्राकारकल्पैश् च स्पर्शनावरणे यथा ।\\नगरेषु न ते तद्वद् गन्धर्वनगरेष्व् अपि ॥२९२॥

मृगपश्वादिभिर् यावान् मुख्यैर् अर्थः प्रसाध्यते ।\\तावान् न मृन्मयेष्व् अस्ति तस्मात् ते विषयः कनः ॥२९३॥

महान् आव्रियते देशः प्रसिद्धैः पर्वतादिभिः ।\\अल्पदेशान्तरावस्थं प्रतिबिम्बं तु दृश्यते ॥२९४॥

मरणादिनिमित्तं च यथा मुख्या विषादयः ।\\न ते स्वप्नादिषु स्वस्य तद्वद् अर्थस्य साधकाः ॥२९५॥

देशकालेन्द्रियगतैर् भेदैर् यद् दृश्यते ऽन्यथा ।\\यथा प्रसिद्धिर् लोकस्य तथा तद् अवसीयते ॥२९६॥

यच् चोपघातजं ज्ञानं यच् च ज्ञानम् अलौकिकम् ।\\न ताभ्यां व्यवहारो ऽस्ति शब्दा लोकनिबन्धनाः ॥२९७॥

घटादिषु यथा दीपो येनार्थेन प्रयुज्यते ।\\ततो ऽन्यस्यापि सांनिध्यात् स करोति प्रकाशनम् ॥२९८॥

संसर्गिषु तथार्थेषु शब्दो येन प्रयुज्यते ।\\तस्मात् प्रयोजकाद् अन्यान् अपि प्रत्याययत्य् असौ ॥२९९॥

निर्मन्थनं यथारण्योर् अग्न्यर्थम् उपपादितम् ।\\धूमम् अप्य् अनभिप्रेतं जनयत्य् एकसाधनम् ॥३००॥

तथा शब्दो ऽपि कस्मिंश् चित् प्रत्याय्ये ऽर्थे विवक्षिते ।\\अविवक्षितम् अप्य् अर्थं प्रकाशयते संनिधेः ॥३०१॥

यथैवात्यन्तसंसृष्टस् त्यक्तुम् अर्थो न शक्यते ।\\तथा शब्दो ऽपि संबन्धी प्रविवक्तुं न शक्यते ॥३०२॥

अर्थानां संनिधाने ऽपि सति चैषां प्रकाशने ।\\प्रयोजको ऽर्थः शब्दस्य रूपाभेदे ऽपि गम्यते ॥३०३॥

क्व चिद् गुणप्रधानत्वम् अर्थानाम् अविवक्षितम् ।\\क्व चित् सांनिध्यम् अप्य् एषां प्रतिपत्ताव् अकारणम् ॥३०४॥

यच् चानुपात्तं शब्देन तत् कस्मिंश् चित् प्रतीयते ।\\३०५॥

क्व चित् प्रधानम् एवार्थो भवत्य् अयस्य लक्षणम् ॥३०६॥

आख्यातं तद्धितार्थस्य यत् किं चिद् उपदर्शकम् ।\\३०७॥

गुणप्रधानभावस्य तत्र दृष्टो विपर्ययः ॥३०८॥

निर्देशे लिङ्गसंख्यानां संनिधानम् अकारणम् ।\\३०९॥

प्रमाणम् अर्धह्रसादाव् अनुपात्तं प्रतीयते ॥३१०॥

ह्रस्वस्यार्धं च यद् दृष्टं तत् तस्यासंनिधाव् अपि ।\\ह्रस्वस्य लक्षणार्थत्वात् तद्वद् एवाभिधीयते ॥३११॥

दीर्घप्लुताभ्यां तस्य स्यान् मात्रया वा विशेषणम् ।\\जातेर् वा लक्षणाय स्यात् सर्वथा सप्तपर्णवत् ॥३१२॥

गन्तव्यं दृश्यतां सूर्य इति कालस्य लक्षणे ।\\ज्ञायतां काल इत्य् एतत् सोपायम् अभिधीयते ॥३१३॥

विध्यत्य् अधनुषेत्य् अत्र विशेषेण निदर्श्यते ।\\सामान्यम् आश्रयः शक्तेर् यः कश् चित् प्रतिपादकः ॥३१४॥

काकेभ्यो रक्ष्यतां सर्पिर् इति बालो ऽपि चोदितः ।\\उपघातपरे वाक्ये न श्वादिभ्यो न रक्षति ॥३१५॥

प्रक्षालने शरावाणां स्थाननिर्मार्जनं तथा ।\\अनुक्तम् अपि रूपेण भुज्यङ्गत्वात् प्रतीयते ॥३१६॥

वाक्यात् प्रकरणाद् अर्थाद् औचित्याद् देशकालतः ।\\शब्दार्थाः प्रविभज्यन्ते न रूपाद् एव केवलात् ॥३१७॥

संसर्गो विप्रयोगश् च साहचर्यं विरोधिता ।\\अर्थः प्रकरणं लिङ्गं शब्दस्यान्यस्य संनिधिः ॥३१८॥

सामर्थ्यम् औचिती देशः कालो व्यक्तिः स्वरादयः ।\\शब्दार्थस्यानवच्छेदे विशेषस्मृतिहेतवः ॥३१९॥

भेदपक्षे ऽपि सारूप्याद् भिन्नार्थाः प्रतिपत्तृषु ।\\नियता यान्त्य् अभिव्यक्तिं शब्दाः प्रकरणादिभिः ॥३२०॥

नामाख्यातसरूपा ये कार्यान्तरनिबन्धनाः ।\\शब्दा वाक्यस्य तेष्व् अर्थो न रूपाद् अधिगम्यते ॥३२१॥

या प्रवृत्तिनिवृत्त्यर्था स्तुतिनिन्दाप्रकल्पना ।\\कुशलः प्रतिपत्ता ताम् अयथार्थां समीहते ॥३२२॥

विधीयमानं यत् क्रर्म दृष्टादृष्टप्रयोजनम् ।\\स्तूयते सा स्तुतिस् तस्य कर्तुर् एव प्रयोजिका ॥३२३॥

व्याघ्रादिव्यपदेशेन यथा बालो निवर्त्यते ।\\असत्यो ऽपि तथा कश् चित् प्रत्यवायो ऽभिधीयते ॥३२४॥

न संविधानां कृत्वापि प्रत्यवाये तथाविधे ।\\शास्त्रेण प्रतिषिद्धे ऽर्थे विद्वान् कश् चित् प्रवर्तते ॥३२५॥

सर्पेषु संविधायापि सिद्धैर् मन्त्रौषधादिभिः ।\\नान्यथा प्रतिपत्तव्यं न दतो गमयेद् इति ॥३२६॥

क्व चित् तत्त्वसमाख्यानं क्रियते स्तुतिनिन्दयोः ।\\तत्रापि च प्रवृत्तिश् च निवृत्तिश् चोपदिश्यते ॥३२७॥

रूपं सर्वपदार्थानां वाक्यार्थोपनिबन्धनम् ।\\सापेक्षा ये तु वाक्यार्थाः पदार्थैर् एव ते समाः ॥३२८॥

वाक्यं तद् अपि मन्यन्ते यत् पदं चरितक्रियम् ।\\अन्तरेण क्रियाशब्दं वाक्यादेर् द्वित्वदर्शनात् ॥३२९॥

आख्यातशब्दे नियतं साधनं यत्र गम्यते ।\\तद् अप्य् एकं समाप्तार्थं वाक्यम् इत्य् अभिधीयते ॥३३०॥

शब्दव्यवहिता बुद्धिर् अप्रयुक्तपदाश्रया ।\\अनुमानं तदर्थस्य प्रत्यये हेतुर् उच्यते ॥३३१॥

[थिस् वेर्से इस् ओन्ल्य् इन् ऋऔ] अपरे तु पदस्यैव तम् अर्थं प्रतिजानते ।\\शब्दान्तराभिसंबन्धम् अन्तरेण व्यवस्थितम् ॥३३२॥

यस्मिन्न् उच्चरिते शब्दे यदा यो ऽर्थः प्रतीयते ।\\तम् आहुर् अर्थं तस्यैव नान्यद् अर्थस्य लक्षणम् ॥३३३॥

क्रियार्थोपपदेश्व् एवं स्थानिनां गम्यते क्रिया ।\\वृत्तौ निरादिभिश् चैवं क्रान्ताद्यर्थः प्रतीयते ॥३३४॥

तानि शब्दान्तराण्य् एव पर्याया इव लौकिकाः ।\\अर्थप्रकरणाभ्यां तु तेषां स्वार्थो नियम्यते ॥३३५॥

प्रतिबोधाभ्युपायास् तु ये तं तं पुरुषं प्रति ।\\नावश्यं ते ऽभिसंबद्धाः शब्दा ज्ञेयेन वस्तुना ॥३३६॥

असत्यां प्रतिपत्तौ वा मिथ्या वा प्रतिपादने ।\\स्वैर् अर्थैर् नित्यसंबन्धास् ते ते शब्दा व्यवस्थिताः ॥३३७॥

यथाप्रकरणं द्वारम् इत्य् अस्यां कर्मणः श्रुतौ ।\\बधान देहि वेत्य् एतद् उपायाद् अवगम्यते ॥३३८॥

तत्र साधनवृत्तिर् यः शब्दः सत्त्वनिबन्धनः ।\\न स प्रधानभूतस्य साध्यस्यार्थस्य वाचकः ॥३३९॥

स्वार्थमात्रं प्रकाश्यासौ सापेक्षो विनिवर्तते ।\\अर्थस् तु तस्य संबन्धी प्रकल्पयति संनिधिम् ॥३४०॥

पारार्थ्यस्याविशिष्टत्वान् न शब्दाच् छब्दसंनिधिः ।\\नार्थाच् छब्दस्य सांनिध्यं न शब्दाद् अर्थसंनिधिः ॥३४१॥

नष्टरूपम् इवाख्यातम् आक्षिप्तं कर्मवाचिना ।\\यदि प्राप्तं प्रधानत्वं युगपद् भावसत्त्वयोः ॥३४२॥

तैस् तु नामसरूपत्वम् आख्यातस्यास्य वर्ण्यते ।\\अन्वयव्यतिरेकाभ्यां व्यवहारो विभज्यते ॥३४३॥

न चापि रूपात् संदेहे वाचकत्वं निवर्तते ।\\अर्धं पशोर् इति यथा सामर्थ्यात् तद् धि कल्पते ॥३४४॥

सर्वं सत्त्वपदं शुद्धं यदि भावनिबन्धनम् ।\\संसर्गे च विभक्तो ऽस्य तस्यार्थो न पृथग् यदि ॥३४५॥

क्रियाप्रधानम् आख्यातं नाम्नां सत्त्वप्रधानता ।\\चत्वारि पदजातानि सर्वम् एतद् विरुध्यते ॥३४६॥

वाक्यस्य बुद्धौ नित्यत्वम् अर्थयोगं च लौकिकम् ।\\दृष्ट्वा चतुष्ट्वं नास्तीति वदत्य् औदुम्बरायणः ॥३४७॥

व्याप्तिमांश् च लघुश् चैव व्यवहारः पदाश्रयः ।\\लोके शास्त्रे च कार्यार्थं विभागेनैव कल्पितः ॥३४८॥

न लोके प्रतिपत्तॄणाम् अर्थयोगात् प्रसिद्धयः ।\\तस्माद् अलौकिको वाक्याद् अन्यः कश् चिन् न विद्यते ॥३४९॥

अन्यत्र श्रूयमाणैश् च लिङ्गैर् वाक्यैश् च सूचिताः ।\\स्वार्था एव प्रतीयन्ते रूपाभेदाद् अलक्षिताः ॥३५०॥

उत्सर्गवाक्ये यत् त्यक्तम् अशब्दम् इव शब्दवत् ।\\तद् बाधकेषु वाक्येषु श्रुतम् अन्यत्र गम्यते ॥३५१॥

ब्राह्मणानां श्रुतिर् दध्नि प्रक्रान्ता माठराद् विना ।\\माठरस् तक्रसंबन्धात् तत्राचष्टे यथार्थताम् ॥३५२॥

अनेकाख्यातयोगे ऽपि वाक्यं न्यायापवादयोः ।\\एकम् एवेष्यते कैश् चिद् भिन्नरूपम् इव स्थितम् ॥३५३॥

नियमः प्रतिषेधश् च विधिशेषस् तथा सति ।\\द्वितीये यो लुग् आख्यातस् तच्छेषम् अलुकं विदुः ॥३५४॥

निराकाङ्क्षाणि निर्वृत्तौ प्रधानानि परस्परम् ।\\तेषाम् अनुपकारित्वात् कथं स्याद् एकवाक्यता ॥३५५॥

विशेषविधिनार्थित्वाद् वाक्यशेषो ऽनुमीयते ।\\विधेयवन् निवर्त्ये ऽर्थे तस्मात् तुल्यं व्यपेक्षणम् ॥३५६॥

संज्ञाशब्दैकदेशो यस् तस्य लोपो न विद्यते ।\\विशिष्टरूपा सा संज्ञा कृता च न निवर्तते ॥३५७॥

संज्ञान्तराच् च दत्तादेर् नान्या संज्ञा प्रतीयते ।\\संज्ञिनं देवदत्ताख्यं दत्तशब्दः कथं वदेत् ॥३५८॥

सर्वैर् अवयवैस् तुल्यं संबन्धं समुदायवत् ।\\के चिच् छब्दस्वरूपाणां मन्यन्ते सर्वसंज्ञिभिः ॥३५९॥

वर्णानाम् अर्थवत्त्वं तु संज्ञानां संज्ञिभिर् भवेत् ।\\संबद्धो ऽवयवः संज्ञा प्रविवेके न कल्पते ॥३६०॥

सर्वस्वरूपैर् युगपत् संबन्धे सति संज्ञिनः ।\\नैकदेशसरूपेभ्यस् तत्प्रत्यायनसंभवः ॥३६१॥

एकदेशात् तु संघाते केषां चिज् जायते स्मृतिः ।\\स्मृतेस् तु विषयाच् छब्दात् संघातार्थः प्रतीयते ॥३६२॥

एकदेशात् स्मृतिर् भिन्ने संघाते नियता कथम् ।\\कथं प्रतीयमानः स्याच् छब्दो ऽर्थस्याभिधायकः ॥३६३॥

एकदेशसरूपास् तु तैस् तैर् भेदैः समन्विताः ।\\अनुनिष्पादिनः शब्दाः संज्ञासु समवस्थिताः ॥३६४॥

साधारणत्वात् संधिग्धाः सामर्थ्यान् नियताश्रयाः ।\\तेषां ये साधवस् तेषु शास्त्रे लोपादि शिष्यते ॥३६५॥

तुल्यायाम् अनुनिष्पत्तौ ज्येद्राघा इत्य् असाधवः ।\\न ह्य् अन्वाख्यायके शास्त्रे तेषु दत्तादिवत् स्मृतिः ॥३६६॥

कृतणत्वाश् च ये शब्दा नित्याः खरणसादयः ।\\एकद्रव्योपदेशित्वात् तान् साधून् संप्रचक्षते ॥३६७॥

गोत्राण्य् एव तु तान्य् आहुः संज्ञाशक्तिसमन्वयात् ।\\निमित्तापेक्षणं तेषु स्वार्थे नावश्यम् इष्यते ॥३६८॥

व्यवहाराय नियमः संज्ञानां संज्ञिनि क्व चित् ।\\नित्य एव तु संबन्धो डित्थादिषु गवादिवत् ॥३६९॥

कृतकत्वाद् अनित्यत्वं संबन्धस्योपपद्यते ।\\संज्ञायां सा हि पुरुषैर् यथाकामं नियुज्यते ॥३७०॥

यथा हि पांसुलेखानां बालकैर् मधुक्रादयः ।\\संज्ञाः क्रियन्ते सर्वासु संज्ञास्व् एषैव कल्पना ॥३७१॥

वृद्ध्यादीनां च शास्त्रे ऽस्मिञ् छक्त्यवच्छेदलक्षणः ।\\अकृत्रिमो हि संबन्धो विशेषणविशेष्यवत् ॥३७२॥

संज्ञा स्वरूपम् आश्रित्य निमित्ते सति लौकिकी ।\\का चित् प्रवर्तते का चिन् निमित्तासंनिधाव् अपि ॥३७३॥

शास्त्रे ऽपि महती संज्ञा स्वरूपोपनिबन्धना ।\\अनुमानं निमित्तस्य संनिधाने प्रतीयते ॥३७४॥

आवृत्तेर् अनुमानं वा सारूप्यात् तत्र गम्यते ।\\शब्दभेदानुमानं वा शक्तिभेदस्य वा गतिः ॥३७५॥

क्व चिद् विषयभेदेन कृत्रिमा व्यवतिष्ठते ।\\संख्यायाम् एकविषयं व्यवस्थानं द्वयोर् अपि ॥३७६॥

विषयं कृत्रिमस्यापि लौकिकः क्व चिद् उच्चरन् ।\\व्याप्नोति दूरात् संबुद्धौ तथा हि ग्रहणं द्वयोः ॥३७७॥

सङ्घैकशेषद्वन्द्वेषु के चित् सामर्थ्यलक्षणम् ।\\प्रत्याश्रयम् अवस्थानं क्रियाणां प्रतिजानते ॥३७८॥

भोजनं फलरूपाभ्याम् एकैकस्मिन् समाप्यते ।\\अन्यथा हि व्यवस्थाने न तदर्थः प्रकल्प्यते ॥३७९॥

अन्नादानादि रूपां च सर्वे तृप्तिफलां भुजिम् ।\\प्रत्येकं प्रतिपद्यन्ते न तु नाट्यक्रियाम् इव ॥३८०॥

पाद्यवत् सा विभागेन सामर्थ्याद् अवतिष्ठते ।\\भुजिः करोति भुज्यर्थं न तन्त्रेण प्रदीपवत् ॥३८१॥

दृश्यादिस् तु क्रियैकापि तथाभूतेषु कर्मसु ।\\आवृत्तिम् अन्तरेणापि समुदायाश्रया भवेत् ॥३८२॥

भिन्नव्यापाररूपाणां व्यवहारादिदर्शने ।\\कर्तॄणां दर्शनं भिन्नं संभूयार्थस्य साधकम् ॥३८३॥

लक्ष्यस्य लोकसिद्धत्वाच् छास्त्रे लिङ्गस्य दर्शनात् ।\\अर्थिष्व् आदैक्षु भेदेन वृद्धिसंज्ञा समाप्यते ॥३८४॥

शतादानप्रधानत्वाद् दण्डने शतकर्मके ।\\अर्थिनां गुणभेदे ऽपि संख्येयो ऽर्थो न भिद्यते ॥३८५॥

सङ्घस्यैव विधेयत्वात् कार्यवत् प्रतिपादने ।\\तत्र तन्त्रेण संबन्धः समासाभ्यस्तसंज्ञयोः ॥३८६॥

लक्षणार्था श्रुतिर् येषां कां चिद् एव क्रियां प्रति ।\\तैर् व्यस्तैश् च समस्तैश् च स धर्म उपलक्ष्यते ॥३८७॥

वृषलैर् न प्रवेष्टव्यम् इत्य् एतस्मिन् गृहे यथा ।\\प्रत्येकं संहतानां च प्रवेशः प्रतिषिध्यते ॥३८८॥

संभूय त्व् अर्थलिप्सादि प्रतिषेधोपदेशने ।\\पृथग् अप्रतिषिद्धत्वात् प्रवृत्तिर् न विरुध्यते ॥३८९॥

व्यवायलक्षणार्थात्वाद् अट्कुप्वाङादिभिस् तथा ।\\प्रत्येकं वा समस्तैर् वा णत्वं न प्रतिषिध्यते ॥३९०॥

अनुग्रहार्था भोक्तॄणां भुजिर् आरभ्यते यदा ।\\देशकालाद्यभेदेन नानुगृह्णाति तान् असौ ॥३९१॥

पात्रादिभेदान् नानात्वं यस्यैकस्योपदिश्यते ।\\विपर्यये वा भिन्नस्य तस्यैकत्वं प्रकल्प्यते ॥३९२॥

संहत्यापि च कुर्वाणा भेदेन प्रतिपादिताः ।\\स्वं स्वं भोज्यं विभागेन प्राप्तं संभूय भुञ्जते ॥३९३॥

वीप्साया विषयाभावाद् विरोधाद् अन्यसंख्यया ।\\द्विधा समाप्त्ययोगाच् च शतम् सङ्घे ऽवतिष्ठते ॥३९४॥

भुजिर् द्वन्द्वैकशेषाभ्यां यत्रान्यैः सह शिष्यते ।\\तत्रापि लक्षणार्थत्वाद् द्विधा वाक्यं समाप्यते ॥३९५॥

वाक्यान्तराणां प्रत्येकं समाप्तिः कैश् चिद् इष्यते ।\\रूपान्तरेण युक्तानां वाक्यनां तेन संग्रहः ॥३९६॥

न वाक्यस्याभिधेयानि भेदवाक्यानि कानि चित् ।\\तस्मिंस् तूच्चरिते भेदांस् तथान्यान् प्रतिपद्यते ॥३९७॥

येषां समस्तो वाक्यार्थः प्रतिभेदं समाप्यते ।\\तेषां तदानीं भिन्नस्य किं पदार्थस्य सत्तया ॥३९८॥

अथ तैर् एव जनितः सो ऽर्थो भिन्नेषु वर्तते ।\\पूर्वस्यार्थस्य तेन स्याद् विरोधः सह वा स्थितिः ॥३९९॥

सहस्थितौ विरोधित्वं स्याद् विशिष्टाविशिष्टयोः ।\\व्यभिचारी तु संबन्धस् त्यागे ऽर्थस्य प्रसज्यते ॥४००॥

एकः साधारणो वाच्यः प्रतिशब्दम् अवस्थितः ।\\सङ्घे सङ्घिषु चार्थात्मा सम्निधाननिदेशकः ॥४०१॥

यथा साधारणे स्वत्वं त्यागस्य च फलं धने ।\\प्रीतिश् चाविकला तद्वत् संबन्धो ऽर्थेन तद्वताम् ॥४०२॥

वर्णानाम् अर्थवत्तायां तेनैवार्थेन तद्वति ।\\समुदाये न चैकत्वं भेदेन व्यवतिष्ठते ॥४०३॥

एकेनैव प्रदीपेन सर्वे साधारणं धनम् ।\\पश्यन्ति तद्वद् एकेन सुपा संख्याभिधीयते ॥४०४॥

नार्थवत्ता पदे वर्णे वाक्ये चैवं विशिष्यते ।\\अभ्यासात् प्रक्रमो ऽन्यस् तु विरुद्ध इव दृश्यते ॥४०५॥

विनियोगाद् ऋते शब्दो न स्वार्थस्य प्रकाशकः ।\\अर्थाभिधानसंबन्धम् उक्तिद्वारं प्रचक्षते ॥४०६॥

यथा प्रणिहितं चक्षुर् दर्शनायोपकल्पते ।\\तथाभिसंहितः शब्दो भवत्य् अर्थस्य वाचकः ॥४०७॥

क्रियाव्यवेतः संबन्धो दृष्टः करणकर्मभिः ।\\अभिधानियमस् तस्माद् अभिधानाभिधेययोः ॥४०८॥

बहुष्व् एकाभिधानेषु सर्वेष्व् एकार्थकारिषु ।\\यत् प्रयोक्ताभिसंधत्ते शब्दस् तत्रावतिष्ठते ॥४०९॥

आम्नायशब्दान् अभ्यासे के चिद् आहुर् अनर्थकान् ।\\स्वरूपमात्रवृत्तींश् च परेषां प्रतिपादने ॥४१०॥

अभिधानक्रियायोगाद् अर्थस्य प्रतिपादकान् ।\\नियोगभेदान् मन्यन्ते तान् एवैकत्वदर्शिनः ॥४११॥

तेषाम् अत्यन्तनानात्वं नानात्वव्यवहारिणः ।\\अक्षादीनाम् इव प्राहुर् एकजातिसमन्वयात् ॥४१२॥

प्रयोगाद् अभिसंधानम् अन्यद् एषु न विद्यते ।\\विषये यतशक्तित्वात् स तु तत्र व्यवस्थितः ॥४१३॥

नानात्वस्यैव संज्ञानम् अर्थप्रकरणादिभिः ।\\न जात्व् अर्थान्तरे वृत्तिर् अन्यार्थानां कथं चन ॥४१४॥

पदरूपम् च यद् वाक्यम् अस्तित्वोपनिबन्धनम् ।\\कामं विमर्शस् तत्रायं न वाक्यावयवे पदे ॥४१५॥

यथैवानर्थकैर् वर्णैर् विशिष्टो ऽर्थो ऽभिधीयते ।\\पदैर् अनर्थकैर् एवं विशिष्टो ऽर्थो ऽभिधीयते ॥४१६॥

यद् अन्तराले ज्ञानं तु पदार्थेषूपजायते ।\\प्रतिपत्तेर् उपायो ऽसौ प्रक्रमानवधारणात् ॥४१७॥

पूर्वैर् अर्थैर् अनुगतो यथार्थात्मा परः परः ।\\संसर्ग एव प्रक्रान्तस् तथान्येष्व् अर्थवस्तुषु ॥४१८॥

अङ्गीकृते तु केषां चित् साध्येनार्थेन साधने ।\\आराधनियमार्थैव साधनानां पुनः श्रुतिः ॥४१९॥

आधारे नियमाभावात् तदाक्षेपो न विद्यते ।\\सामर्थ्यात् संभवस् तस्य श्रुतिस् त्व् अन्यनिवृत्तये ॥४२०॥

क्रिया क्रियान्तराद् भिन्ना नियताधारसाधना ।\\प्रक्रान्ता प्रतिपत्तॄणां भेदाः संबोधहेतवः ॥४२१॥

अविभागं तु शब्देभ्यः क्रमवद्भ्यो ऽपदक्रमम् ।\\प्रकाशते तदन्येषां वाक्यं वाक्यार्थ एव च ॥४२२॥

स्वरूपं विद्यते यस्य तस्यात्मा न निरूप्यते ।\\नास्ति यस्य स्वरूपं तु तस्यैवात्मा निरूप्यते ॥४२३॥

अशब्दम् अपरे ऽर्थस्य रूपनिर्धारणं विदुः ।\\अर्थावभासरूपा च शब्देभ्यो जायते स्मृतिः ॥४२४॥

अन्यथैवाग्निसंबन्धाद् दाहं दग्धो ऽभिमन्यते ।\\अन्यथा दाहशब्देन दाहार्थः संप्रतीयते ॥४२५॥

पृथङ्निविष्टतत्त्वानां पृथगर्थानुपातिनाम् ।\\इन्द्रियाणां यथा कार्यम् ऋते देहान् न कल्पते ॥४२६॥

तथा पदानां सर्वेषां पृथगर्थनिवेशिनाम् ।\\वाक्येभ्यः प्रविभक्तानाम् अर्थवत्ता न विद्यते ॥४२७॥

संसर्गरूपं संसृष्टेष्व् अर्थवस्तुषु गृह्यते ।\\नात्रोपाख्यायते तत्त्वम् अपदार्थस्य दर्शनात् ॥४२८॥

दर्शनस्यापि यत् सत्यं न तथा दर्शनं स्थितम् ।\\वस्तु संसर्गरूपेण तद् अरूपं निरूप्यते ॥४२९॥

अस्तित्वेनानुषक्तो वा निवृत्त्यात्मनि वा स्थितः ।\\अर्थो ऽभिधीयते यस्माद् अतो वाक्यं प्रयुज्यते ॥४३०॥

क्रियानुषङ्गेण विना न पदार्थः प्रतीयते ।\\सत्यो वा विपरीतो वा व्यवहारे न सो ऽस्त्य् अतः ॥४३१॥

सद् इत्य् एतत् तु यद् वाक्यं तद् अभूद् अस्ति नेति वा ।\\क्रियाभिधानसंबन्धम् अन्तरेण न गम्यते ॥४३२॥

आख्यातपदवाच्ये ऽर्थे साधनोपनिबन्धने ।\\विना सत्त्वाभिधानेन नाकाङ्क्षा विनिवर्तते ॥४३३॥

प्राधान्यात् तु क्रिया पूर्वम् अर्थस्य प्रविभज्यते ।\\साध्यप्रयुक्तान्य् अङ्गानि फलं तस्य प्रयोजकम् ॥४३४॥

प्रयोक्तैवाभिसंधत्ते साध्यसाधनरूपताम् ।\\अर्थस्य चाभिसंबन्ध कल्पनां प्रसमीहते ॥४३५॥

पचिक्रियां करोतीति कर्मत्वेनाभिधीयते ।\\पक्तिः करणरूपं तु साध्यत्वेन प्रतीयते ॥४३६॥

यो ऽंशो येनोपकारेण प्रयोक्तॄणां विवक्षितः ।\\अर्थस्य सर्वशक्तित्वात् स तथैव व्यवस्थितः ॥४३७॥

आराद्वृत्तिषु संबन्धः कदा चिद् अभिधीयते ।\\आश्लिष्टो यो ऽनुपश्लिष्टः स कदा चित् प्रतीयते ॥४३८॥

संसृष्टानां विभक्तत्वं संसर्गश् च विवेकिनाम् ।\\नानात्मकानाम् एकत्वं नानात्वं च विपर्यये ॥४३९॥

सर्वात्मकत्वाद् अर्थस्य नैरात्म्याद् वा व्यवस्थितम् ।\\अत्यन्तयतशक्तित्वाच् छब्द एव निबन्धनम् ॥४४०॥

वस्तूपलक्षणः शब्दो नोपकारस्य वाचकः ।\\न स्वशक्तिः पदार्थानां संस्प्रष्टुं तेन शक्यते ॥४४१॥

संबन्धिधर्मा संयोगः स्वशब्देनाभिधीयते ।\\संबन्धः समवायस् तु संबन्धित्वेन गम्यते ॥४४२॥

लक्षणाद् व्यवतिष्ठन्ते पदार्था न तु वस्तुतः ।\\उपकारात् स एवार्थः कथं चिद् अनुगम्यते ॥४४३॥

वाक्यार्थो यो ऽभिसंबन्धो न तस्यात्मा क्व चित् स्थितः ।\\व्यवहारे पदार्थानां तम् आत्मानं प्रचक्षते ॥४४४॥

पदार्थे समुदाये वा समाप्तो नैव वा क्व चित् ।\\पदार्थरूपभेदेन तस्यात्मा प्रविभज्यते ॥४४५॥

अन्वाख्यानाय यो भेदः प्रतिपत्तिनिबन्धनम् ।\\साकाङ्क्षावयवं भेदे तेनान्यद् उपवर्ण्यते ॥४४६॥

अनेकशक्तेर् एकस्य प्रविभागो ऽनुगम्यते ।\\एकार्थत्वं हि वाक्यस्य मात्रयापि प्रतीयते ॥४४७॥

संप्रत्ययार्थाद् बाह्यो ऽर्थः सन्न् असन् वा विभज्यते ।\\बाह्यीकृत्य विभागस् तु शक्त्यपोद्धारलक्षणः ॥४४८॥

प्रत्ययार्थात्मनियताः शक्तयो न व्यवस्थिताः ।\\अन्यत्र च ततो रूपं न तासाम् उपलभ्यते ॥४४९॥

बहुश्व् अपि तिङन्तेषु साकाङ्क्षेष्व् एकवाक्यता ।\\तिङा तिङ्भ्यो निघातस्य पर्युदासस् तथार्थवान् ॥४५०॥

एकतिङ् यस्य वाक्यं तु शास्त्रे नियतलक्षणम् ।\\तस्यातिङ्ग्रहणेनार्थो वाक्यभेदान् न विद्यते ॥४५१॥

तिङन्तान्तरयुक्तेषु युक्तयुक्तेषु वा पुनः ।\\मृगः पश्यत यातीति भेदाभेदौ न तिष्ठतः ॥४५२॥

इतिकर्तव्यतार्थस्य सामर्थ्याद् यत्र काङ्क्ष्यते ।\\अशब्दलक्षणाकाङ्क्षं समाप्तार्थं तद् उच्यते ॥४५३॥

तत्त्वान्वाख्यानमात्रे तु यावान् अर्थो ऽनुषज्यते ।\\विनापि तत्प्रयोगेण श्रुतेर् वाक्यं समाप्यते ॥४५४॥

चिङ्क्रम्यमाणो ऽधीष्वात्र जपंश् चङ्क्रमणं कुरु ।\\तादर्थ्यस्याविशेषे ऽपि शब्दाद् भेदः प्रतीयते ॥४५५॥

फलवन्तः क्रियाभेदाः क्रियान्तरनिबन्धनाः ।\\असंख्याताः क्रमोद्देशैर् एकाख्यातनिदर्शिताः ॥४५६॥

निवृतभेदा सर्वैव क्रियाख्याते ऽभिधीयते ।\\श्रुतेर् अशक्या भेदानां प्रविभागप्रकल्पना ॥४५७॥

अश्वमेधेन यक्ष्यन्ते राजानः सत्त्रम् आसते ।\\ब्राह्मणा इति नाख्यात रूपाद् भेदः प्रतीयते ॥४५८॥

सकृच् छ्रुता सप्तदशस्व् अनावृत्तापि या क्रिया ।\\प्रजापत्येषु सामर्थ्यात् सा भेदं प्रतिपद्यते ॥४५९॥

देवदत्तादिषु भुजिः प्रत्येकम् अवतिष्ठते ।\\प्रतिस्वतन्त्रं वाक्यं वा भेदेन प्रतिपद्यते ॥४६०॥

उच्चारणे तु वाक्यानाम् अन्यद् रूपं न गृह्यते ।\\प्रतिपत्तौ तु भिन्नानाम् अन्यद् रूपं प्रतीयते ॥४६१॥

एकं ग्रहणवाक्यं च सामान्येनाभिधीयते ।\\कर्तरीति यथा तच् च पश्वादिषु विभज्यते ॥४६२॥

यदि आकाङ्क्षा निवर्तेत तद्भूतस्य सकृच् छ्रुतौ ।\\नैवान्येनाभिसंबन्धं तद् उपेयात् कथं चन ॥४६३॥

एकरूपम् अनेकार्थं तस्माद् उपनिबन्धनम् ।\\योनिर् विभागवाक्यानां तेभ्यो ऽनन्यद् इव स्थितम् ॥४६४॥

क्व चित् क्रिया व्यक्तिभागैर् उपकारे प्रवर्तते ।\\सामान्यभाग एवास्याः क्व चिद् अर्थस्य साधकः ॥४६५॥

कालभिन्नाश् च ये भेदा ये चाप्य् उष्ट्रासिकादिषु ।\\प्रक्रमे जातिभागस्य शब्दात्मा तैर् न भिद्यते ॥४६६॥

एकसंख्येषु भेदेषु भिन्ना जात्यादिभिः क्रियाः ।\\भेदेन विनियुज्यन्ते तच्छब्दस्य सकृच् छ्रुतौ ॥४६७॥

अक्षादेषु यथा भिन्ना भक्षिभञ्जिदिविक्रियाः ।\\प्रयोगकालाभेदे ऽपि प्रतिभेदं पृथक् स्थिताः ॥४६८॥

अक्षिणां तन्त्रिणां तन्त्रम् उपायस् तुल्यरूपता ।\\एषां क्रमो विभक्तानां तन्निबद्धा सकृच् छ्रुतिः ॥४६९॥

द्वाव् अप्य् उपायौ शब्दानां प्रयोगे समवस्थितौ ।\\क्रमो वा यौगपद्यं वा यौ लोको नातिवर्तते ॥४७०॥

क्रमे विभज्यते रूपं यौगपद्ये न भिद्यते ।\\क्रिया तु यौगपद्ये ऽपि क्रमरूपानुपातिनी ॥४७१॥

भेदसंसर्गशक्ती द्वे शब्दाद् भिन्ने इव स्थिते ।\\यौगपद्ये ऽप्य् अनेकेन प्रयोगे भिद्यते श्रुतिः ॥४७२॥

अभिन्नो रूपभेदेन य एको ऽर्थो विवक्षितः ।\\तस्यावयवधर्मेण समुदायो ऽनुगम्यते ॥४७३॥

भेदनिर्वचने त्व् अस्य प्रत्येदं वा समाप्यते ।\\श्रुतिर् वचनभिन्ना वा वाक्यभेदे ऽवतिष्ठते ॥४७४॥

तत्रैकवचनान्तो वा सो ऽक्षशब्दः प्रयुज्यते ।\\प्रत्येकं वा बहुत्वेन प्रविभागो यथाश्रुति ॥४७५॥

द्विष्ठानि यानि वाक्यानि तेष्व् अप्य् एकत्वदर्शिनाम् ।\\अनेकशक्तेर् एकस्य स्वशक्तिः प्रविभज्यते ॥४७६॥

अत्यन्तभिन्नयोर् वा स्यात् प्रयोगे तन्त्रलक्षणः ।\\उपायस् तत्र संसर्गः प्रतिपत्तृषु भिद्यते ॥४७७॥

भेदेनाधिगतौ पूर्वं शब्दौ तुल्यश्रुती पुनः ।\\तन्त्रेण प्रतिपत्तारः प्रयोक्त्रा प्रतिपादिताः ॥४७८॥

एकस्यापि विवक्षायाम् अनुनिष्पद्यते परः ।\\विनाभिसंधिना शब्दः शक्तिरूपः प्रकाशते ॥४७९॥

अनेका शक्तिर् एकस्य युगपच् छ्रूयते क्व चित् ।\\अग्निः प्रकाशदाहाभ्याम् एकत्रापि नियुज्यते ॥४८०॥

आवृत्तिशक्तिभिन्नार्थे वाक्ये सकृद् अपि श्रुते ।\\लिङ्गाद् वा तन्त्रधर्माद् वा विभागो व्यवतिष्ठते ॥४८१॥

संप्रसारणसंज्ञायां लिङ्गाभ्यां वर्णवाक्ययोः ।\\प्रविभागस् तथा सूत्र एकस्मिन्न् एव जायते ॥४८२॥

तथा द्विर्वचने ऽचीति तन्त्रोपायाद् अलक्षणः ।\\एकशेषेण निर्देशो भाष्य एव प्रदर्शितः ॥४८३॥

प्रायेण संक्षेपरुचीन् अल्पविद्यापरिग्रहान् ।\\संप्राप्य वैयाकरणान् संग्रहे ऽस्तम् उपागते ॥४८४॥

कृते ऽथ पातञ्जलिना गुरुणा तीर्थदर्शिना ।\\सर्वेसं न्यायबीजानां महाभाष्ये निबन्धने ॥४८५॥

अलब्धगाधे गाम्भीर्याद् उत्तान इव सौष्ठवात् ।\\तस्मिन्न् अकृतबुद्धीनाम् नैवावास्थित निश्चयः ॥४८६॥

वैजिसौभवहर्यक्षैः शुष्कतर्कानुसारिभिः ।\\आर्षे विप्लाविते ग्रन्थे संग्रहप्रतिकञ्चुके ॥४८७॥

यः पातञ्जलिशिष्येभ्यो भ्रष्टो व्याकरणागमः ।\\कालेन दाक्षिणात्येषु ग्रन्थमात्रो व्यवस्थितः ॥४८८॥

पर्वताद् आगमं लब्ध्वा भाष्यबीजानुसारिभिः ।\\स नीतो बहुशाखत्वं चान्द्राचार्यादिभिः पुनः ॥४८९॥

न्यायप्रस्थानमार्गांस् तान् अभ्यस्य स्वम् च दर्शनम् ।\\प्रणीतो गुरुणास्माकम् अयम् आगमसंग्रहः ॥४९०॥

वर्त्मनाम् अत्र केषाम् चिद् वस्तुमात्रम् उदाहृतम् ।\\काण्डे तृतीये न्यक्षेन भविष्यति विचारणा ॥४९१॥

प्रज्ञा विवेकं लभते भिन्नैर् आगमदर्शनैः ।\\कियद् वा शक्यम् उन्नेतुं स्वतर्कम् अनुधावता ॥४९२॥

तत् तद् उत्प्रेक्षमाणानां पुराणैर् आगमैर् विना ।\\अनुपासितवृद्धानां विद्या नातिप्रसीदति ॥४९३॥
