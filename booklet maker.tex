\documentclass[12pt]{article}
\usepackage[a4paper]{geometry}
\usepackage{pdfpages}
    \includepdfset{pages=1-32}


\begin{document}

    \includepdf[pages=1-95,nup=1x2,landscape,signature=32]{generater.pdf}

\end{document}

<div class="bhashya" id="BR_C03_S05_V01_B02">किम् उषस्तकहोलाभ्याम् एक आत्मा पृष्टः, किं वा भिन्नावात्मानौ तुल्यलक्षणाविति । भिन्नाविति युक्तम् , प्रश्नयोरपुनरुक्तत्वोपपत्तेः ; यदि हि एक आत्मा उषस्तकहोलप्रश्नयोर्विवक्षितः, तत्र एकेनैव प्रश्नेन अधिगतत्वात् तद्विषयो द्वितीयः प्रश्नोऽनर्थकः स्यात् ; न च अर्थवादरूपत्वं वाक्यस्य ; तस्मात् भिन्नावेतावात्मानौ क्षेत्रज्ञपरमात्माख्याविति केचिद्व्याचक्षते । तन्न, ‘ते’ इति प्रतिज्ञानात् ; ‘एष त आत्मा’ इति हि प्रतिवचने प्रतिज्ञातम् ; न च एकस्य कार्यकरणसङ्घातस्य द्वावात्मानौ उपपद्येते ; एको हि कार्यकरणसङ्घातः एकेन आत्मना आत्मवान् ; न च उषस्तस्यान्यः कहोलस्यान्यः जातितो भिन्न आत्मा भवति, द्वयोः अगौणत्वात्मत्वसर्वान्तरत्वानुपपत्तेः ; यदि एकमगौणं ब्रह्म द्वयोः इतरेण अवश्यं गौणेन भवितव्यम् ; तथा आत्मत्वं सर्वान्तरत्वं च — विरुद्धत्वात्पदार्थानाम् ; यदि एकं सर्वान्तरं ब्रह्म आत्मा मुख्यः, इतरेण असर्वान्तरेण अनात्मना अमुख्येन अवश्यं भवितव्यम् ; तस्मात् एकस्यैव द्विः श्रवणं विशेषविवक्षया । यत्तु पूर्वोक्तेन समानं द्वितीये प्रश्नान्तर उक्तम् , तावन्मात्रं पूर्वस्यैवानुवादः — तस्यैव अनुक्तः कश्चिद्विशेषः वक्तव्य इति । कः पुनरसौ विशेष इत्युच्यते — पूर्वस्मिन्प्रश्ने — अस्ति व्यतिरिक्त आत्मा यस्यायं सप्रयोजको बन्ध उक्त इति द्वितीये तु — तस्यैव आत्मनः अशनायादिसंसारधर्मातीतत्वं विशेष उच्यते — यद्विशेषपरिज्ञानात् सन्न्याससहितात् पूर्वोक्ताद्बन्धनात् विमुच्यते । तस्मात् प्रश्नप्रतिवचनयोः ‘एष त आत्मा’ इत्येवमन्तयोः तुल्यार्थतैव । ननु कथम् एकस्यैव आत्मनः अशनायाद्यतीतत्वं तद्वत्त्वं चेति विरुद्धधर्मसमवायित्वमिति — न, परिहितत्वात् ; नामरूपविकारकार्यकरणलक्षणसङ्घातोपाधिभेदसम्पर्कजनितभ्रान्तिमात्रं हि संसारित्वमित्यसकृदवोचाम, विरुद्धश्रुतिव्याख्यानप्रसङ्गेन च ; यथा रज्जुशुक्तिकागगनादयः सर्परजतमलिना भवन्ति पराध्यारोपितधर्मविशिष्टाः, स्वतः केवला एव रज्जुशुक्तिकागगनादयः — न च एवं विरुद्धधर्मसमवायित्वे पदार्थानां कश्चन विरोधः । नामरूपोपाध्यस्तित्वे <span class="qt"><a href="Chandogya_id.html#Ch_C06_S02_V01">‘एकमेवाद्वितीयम्’ (छा. उ. ६ । २ । १)</a></span> <span class="qt"><a href="Brha_id.html#BR_C04_S04_V19">‘नेह नानास्ति किञ्चन’ (बृ. उ. ४ । ४ । १९)</a></span> इति श्रुतयो विरुध्येरन्निति चेत् — न, सलिलफेनदृष्टान्तेन परिहृतत्वात् मदादिदृष्टान्तैश्च ; यदा तु परमार्थदृष्ट्या परमात्मतत्त्वात् श्रुत्यनुसारिभिः अन्यत्वेन निरूप्यमाणे नामरूपे मृदादिविकारवत् वस्त्वन्तरे तत्त्वतो न स्तः — सलिलफेनघटादिविकारवदेव, तदा तत् अपेक्ष्य <span class="qt"><a href="Chandogya_id.html#Ch_C06_S02_V01">‘एकमेवाद्वितीयम्’ (छा. उ. ६ । २ । १)</a></span> <span class="qt"><a href="Brha_id.html#BR_C04_S04_V19">‘नेह नानास्ति किञ्चन’ (बृ. उ. ४ । ४ । १९)</a></span> इत्यादिपरमार्थदर्शनगोचरत्वं प्रतिपद्यते ; रूपवदेव स्वेन रूपेण वर्तमानं केनचिदस्पृष्टस्वभावमपि सत् नामरूपकृतकार्यकरणोपाधिभ्यो विवेकेन नावधार्यते, नामरूपोपाधिदृष्टिरेव च भवति स्वाभाविकी, तदा सर्वोऽयं वस्त्वन्तरास्तित्वव्यवहारः । अस्ति चायं भेदकृतो मिथ्याव्यवहारः, येषां ब्रह्मतत्त्वादन्यत्वेन वस्तु विद्यते, येषां च नास्ति ; परमार्थवादिभिस्तु श्रुत्यनुसारेण निरूप्यमाणे वस्तुनि — किं तत्त्वतोऽस्ति वस्तु किं वा नास्तीति, ब्रह्मैकमेवाद्वितीयं सर्वसंव्यवहारशून्यमिति निर्धार्यते ; तेन न कश्चिद्विरोधः । न हि परमार्थावधारणनिष्ठायां वस्त्वन्तरास्तित्वं प्रतिपद्यामहे — <span class="qt"><a href="Chandogya_id.html#Ch_C06_S02_V01">‘एकमेवाद्वितीयम्’ (छा. उ. ६ । २ । १)</a></span> <span class="qt"><a href="Brha_id.html#BR_C02_S05_V19">‘अनन्तरमबाह्यम्’ (बृ. उ. २ । ५ । १९)</a></span>, <span class="qt"><a href="Brha_id.html#BR_C03_S07_V08">(बृ. उ. ३ । ८ । ८)</a></span> इति श्रुतेः ; न च नामरूपव्यवहारकाले तु अविवेकिनां क्रियाकारकफलादिसंव्यवहारो नास्तीति प्रतिषिध्यते । तस्मात् ज्ञानाज्ञाने अपेक्ष्य सर्वः संव्यवहारः शास्त्रीयो लौकिकश्च ; अतो न काचन विरोधशङ्का । सर्ववादिनामप्यपरिहार्यः परमार्थसंव्यवहारकृतो व्यवहारः ॥ </div>
<div class="bhashya" id="BR_C03_S05_V01_B03">तत्र परमार्थात्मस्वरूपमपेक्ष्य प्रश्नः पुनः — कतमो याज्ञवल्क्य सर्वान्तर इति । प्रत्याह इतरः — योऽशनायापिपासे, अशितुमिच्छा अशनाया, पातुमिच्छा पिपासा ; ते अशनायापिपासे योऽत्येतीति वक्ष्यमाणेन सम्बन्धः । अविवेकिभिः तलमलवदिव गगनं गम्यमानमेव तलमले अत्येति — परमार्थतः — ताभ्यामसंसृष्टस्वभावत्वात् — तथा मूढैः अशनायापिपासादिमद्ब्रह्म गम्यमानमपि — क्षुधितोऽहं पिपासितोऽहमिति, ते अत्येत्येव — परमार्थतः — ताभ्यामसंसृष्टस्वभावत्वात् ; <span class="qt"><a href="Kathaka_id.html#Ka_C02_S02_V11">‘न लिप्यते लोकदुःखेन बाह्यः’ (क. उ. २ । २ । ११)</a></span> इति श्रुतेः — अविद्वल्लोकाध्यारोपितदुःखेनेत्यर्थः । प्राणैकधर्मत्वात् समासकरणमशनायापिपासयोः । शोकं मोहम् — शोक इति कामः ; इष्टं वस्तु उद्दिश्य चिन्तयतो यत् अरमणम् , तत् तृष्णाभिभूतस्य कामबीजम् ; तेन हि कामो दीप्यते ; मोहस्तु विपरीतप्रत्ययप्रभवोऽविवेकः भ्रमः ; स च अविद्या सर्वस्यानर्थस्य प्रसवबीजम् ; भिन्नकार्यत्वात्तयोः शोकमोहयोः असमासकरणम् । तौ मनोऽधिकरणौ ; तथा शरीराधिकरणौ जरां मृत्युं च अत्येति ; जरेति कार्यकरणसङ्घातविपरिणामः वलीपलितादिलिङ्गः ; मृत्युरिति तद्विच्छेदः विपरिणामावसानः ; तौ जरामृत्यू शरीराधिकरणौ अत्येति । ये ते अशनायादयः प्राणमनःशरीराधिकरणाः प्राणिषु अनवरतं वर्तमानाः अहोरात्रादिवत् समुद्रोर्मिवच्च प्राणिषु संसार इत्युच्यन्ते ; योऽसौ दृष्टेर्दृष्टेत्यादिलक्षणः साक्षादव्यवहितः अपरोक्षादगौणः सर्वान्तर आत्मा ब्रह्मादिस्तम्बपर्यन्तानां भूतानाम् अशनायापिपासादिभिः संसारधर्मैः सदा न स्पृश्यते — आकाश इव घनादिमलैः — तम् एतं वै आत्मानं स्वं तत्त्वम् , विदित्वा ज्ञात्वा — अयमहमस्मि परं ब्रह्म सदा सर्वसंसारविनिर्मुक्तं नित्यतृप्तमिति, ब्राह्मणाः — ब्राह्मणानामेवाधिकारो व्युत्थाने, अतो ब्राह्मणग्रहणम् — व्युत्थाय वैपरीत्येनोत्थानं कृत्वा ; कुत इत्याह — पुत्रैषणायाः पुत्रार्थैषणा पुत्रैषणा — पुत्रेणेमं लोकं जयेयमिति लोकजयसाधनं पुत्रं प्रति इच्छा एषणा दारसङ्ग्रहः ; दारसङ्ग्रहमकृत्वेत्यर्थः ; वित्तैषणायाश्च — कर्मसाधनस्य गवादेरुपदानम् — अनेन कर्मकृत्वा पितृलोकं जेष्यामीति, विद्यासंयुक्तेन वा देवलोकम् , केवलया वा हिरण्यगर्भविद्यया दैवेन वित्तेन देवलोकम् । दैवाद्वित्तात् व्युत्थानमेव नास्तीति केचित् , यस्मात् तद्बलेन हि किल व्युत्थानमिति — तदसत् , <span class="qt"><a href="Brha_id.html#BR_C01_S04_V17">‘एतावान्वै कामः’ (बृ. उ. १ । ४ । १७)</a></span> इति पठितत्वात् एषणामध्ये दैवस्य वित्तस्य ; हिरण्यगर्भादिदेवताविषयैव विद्या वित्तमित्युच्यते, देवलोकहेतुत्वात् ; नहि निरुपाधिकप्रज्ञानघनविषया ब्रह्मविद्या देवलोकप्राप्तिहेतुः, <span class="qt"><a href="Brha_id.html#BR_C01_S04_V10">‘तस्मात्तत्सर्वमभवत्’ (बृ. उ. १ । ४ । १०)</a></span> <span class="qt"><a href="Brha_id.html#BR_C01_S04_V01">‘आत्मा ह्येषां स भवति’ (बृ. उ. १ । ४ । १)</a></span> इति श्रुतेः ; तद्बलेन हि व्युत्थानम् , <span class="qt"><a href="Brha_id.html#BR_C03_S05_V01">‘एतं वै तमात्मानं विदित्वा’ (बृ. उ. ३ । ५ । १)</a></span> इति विशेषवचनात् । तस्मात् त्रिभ्योऽप्येतेभ्यः अनात्मलोकप्राप्तिसाधनेभ्यः एषणाविषयेभ्यो व्युत्थाय — एषणा कामः, <span class="qt"><a href="Brha_id.html#BR_C01_S04_V17">‘एतावान्वै कामः’ (बृ. उ. १ । ४ । १७)</a></span> इति श्रुतेः — एतस्मिन् विविधे अनात्मलोकप्राप्तिसाधने तृष्णामकृत्वेत्यर्थः । सर्वा हि साधनेच्छा फलेच्छैव, अतो व्याचष्टे श्रुतिः — एकैव एषणेति ; कथम् ? या ह्येव पुत्रैषणा सा वित्तैषणा, दृष्टफलसाधनत्वतुल्यत्वात् ; या वित्तैषणा सा लोकैषणा ; फलार्थैव सा ; सर्वः फलार्थप्रयुक्त एव हि सर्वं साधनमुपादत्ते ; अत एकैव एषणा या लोकैषणा सा साधनमन्तरेण सम्पादयितुं न शक्यत इति, साध्यसाधनभेदेन उभे हि यस्मात् एते एषणे एव भवतः । तस्मात् ब्रह्मविदो नास्ति कर्म कर्मसाधनं वा — अतो येऽतिक्रान्ता ब्राह्मणाः, सर्वं कर्म कर्मसाधनं च सर्वं देवपितृमानुषनिमित्तं यज्ञोपवीतादि — तेन हि दैवं पित्र्यं मानुषं च कर्म क्रियते, <span class="qt_o">‘निवीतं मनुष्याणाम्’ (तै. सं. २ । ५ । ११ । १)</span> इत्यादिश्रुतेः । तस्मात् पूर्वे ब्राह्मणाः ब्रह्मविदः व्युत्थाय कर्मभ्यः कर्मसाधनेभ्यश्च यज्ञोपवीतादिभ्यः, परमहंसपारिव्राज्यं प्रतिपद्य, भिक्षाचर्यं चरन्ति — भिक्षार्थं चरणं भिक्षाचर्यम् , चरन्ति — त्यक्त्वा स्मार्तं लिङ्गं केवलमाश्रममात्रशरणानां जीवनसाधनं पारिव्राज्यव्यञ्जकम् ; विद्वान् लिङ्गवर्जितः — <span class="qt_o">‘तस्मादलिङ्गो धर्मज्ञोऽव्यक्तलिङ्गोऽव्यक्ताचारः’ (अश्व. ४६ । ५१) (व. १० । १२)</span> इत्यादिस्मृतिभ्यः, <span class="qt"><a href="jbl_id.html#JB_V05">‘अथ परिव्राड्विवर्णवासा मुण्डोऽपरिग्रहः’ (जा. उ. ५)</a></span> इत्यादिश्रुतेः, <span class="qt_o">‘सशिखान्केशान्निकृत्य विसृज्य यज्ञोपवीतम्’ (क. रु. १)</span> इति च ॥ </div>
