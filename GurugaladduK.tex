\chapter*{\center ಕವಚಾನಿ }
\fancyhead[RL]{}
\section{ಶ್ರೀಗುರುಪಾದುಕಾಸ್ತೋತ್ರಂ }
ನಾಲೀಕನೀಕಾಶಪದಾದೃತಾಭ್ಯಾಂ \\ನಾರೀವಿಮೋಹಾದಿನಿವಾರಕಾಭ್ಯಾಂ ।\\
ನಮಜ್ಜನಾಭೀಷ್ಟತತಿಪ್ರದಾಭ್ಯಾಂ \\ನಮೋ ನಮಃ ಶ್ರೀಗುರುಪಾದುಕಾಭ್ಯಾಂ ॥ ೧॥

ಶಮಾದಿಷಟ್ಕಪ್ರದವೈಭವಾಭ್ಯಾಂ\\ ಸಮಾಧಿದಾನವ್ರತದೀಕ್ಷಿತಾಭ್ಯಾಂ ।\\
ರಮಾಧವಾಂಘ್ರಿಸ್ಥಿರಭಕ್ತಿದಾಭ್ಯಾಂ \\ನಮೋ ನಮಃ ಶ್ರೀಗುರುಪಾದುಕಾಭ್ಯಾಂ ॥ ೨॥

ನೃಪಾಲಿಮೌಲಿವ್ರಜರತ್ನಕಾಂತಿ\\ಸರಿದ್ವಿರಾಜಜ್ಝಷಕನ್ಯಕಾಭ್ಯಾಂ ।\\
ನೃಪತ್ವದಾಭ್ಯಾಂ ನತಲೋಕಪಂಕ್ತೇರ್\\ನಮೋ ನಮಃ ಶ್ರೀಗುರುಪಾದುಕಾಭ್ಯಾಂ ॥೩॥
\newpage
ಅನಂತಸಂಸಾರಸಮುದ್ರತಾರ\\ನೌಕಾಯಿತಾಭ್ಯಾಂ ಗುರುಭಕ್ತಿದಾಭ್ಯಾಂ ।\\
ವೈರಾಗ್ಯಸಾಮ್ರಾಜ್ಯದಪೂಜನಾಭ್ಯಾಂ \\ನಮೋ ನಮಃ ಶ್ರೀಗುರುಪಾದುಕಾಭ್ಯಾಂ ॥ ೪॥

ಪಾಪಾಂಧಕಾರಾರ್ಕಪರಂಪರಾಭ್ಯಾಂ\\ ತಾಪತ್ರಯಾಹೀಂದ್ರಖಗೇಶ್ವರಾಭ್ಯಾಂ ।\\
ಜಾಡ್ಯಾಬ್ಧಿಸಂಶೋಷಣವಾಡವಾಭ್ಯಾಂ \\ನಮೋ ನಮಃ ಶ್ರೀಗುರುಪಾದುಕಾಭ್ಯಾಂ ॥ ೫॥

ಕವಿತ್ವವಾರಾಶಿನಿಶಾಕರಾಭ್ಯಾಂ \\ದಾರಿದ್ರ್ಯದಾವಾಂಬುದಮಾಲಿಕಾಭ್ಯಾಂ ।\\
ದೂರೀಕೃತಾನಮ್ರವಿಪತ್ತತಿಭ್ಯಾಂ \\ನಮೋ ನಮಃ ಶ್ರೀಗುರುಪಾದುಕಾಭ್ಯಾಂ ॥ ೬॥

ನತಾ ಯಯೋಃ ಶ್ರೀಪತಿತಾಂ ಸಮೀಯುಃ\\ ಕದಾಚಿದಪ್ಯಾಶು ದರಿದ್ರವರ್ಯಾಃ ।\\
ಮೂಕಶ್ಚ ವಾಚಸ್ಪತಿತಾಂ ಹಿ ತಾಭ್ಯಾಂ \\ನಮೋ ನಮಃ ಶ್ರೀಗುರುಪಾದುಕಾಭ್ಯಾಂ ॥ ೭॥

ಕಾಮಾದಿಸರ್ಪವ್ರಜಭಂಜಕಾಭ್ಯಾಂ\\ ವಿವೇಕವೈರಾಗ್ಯನಿಧಿಪ್ರದಾಭ್ಯಾಂ ।\\
ಬೋಧಪ್ರದಾಭ್ಯಾಂ ದ್ರುತಮೋಕ್ಷದಾಭ್ಯಾಂ \\ನಮೋ ನಮಃ ಶ್ರೀಗುರುಪಾದುಕಾಭ್ಯಾಂ ॥ ೮॥
\newpage
ಸ್ವಾರ್ಚಾಪರಾಣಾಮಖಿಲೇಷ್ಟದಾಭ್ಯಾಂ\\ ಸ್ವಾಹಾಸಹಾಯಾಕ್ಷಧುರಂಧರಾಭ್ಯಾಂ ।\\
ಸ್ವಾಂತಾಚ್ಛಭಾವಪ್ರದಪೂಜನಾಭ್ಯಾಂ \\ನಮೋ ನಮಃ ಶ್ರೀಗುರುಪಾದುಕಾಭ್ಯಾಂ ॥ ೯॥

%=================================================================
\newpage
\section{ಗಾಯತ್ರೀಹೃದಯಂ }
\addcontentsline{toc}{section}{ಗಾಯತ್ರೀಹೃದಯಂ }
\thispagestyle{empty}
ಅಸ್ಯ ಶ್ರೀ ಗಾಯತ್ರೀಹೃದಯಸ್ಯ ನಾರಾಯಣ ಋಷಿಃ~। ಗಾಯತ್ರೀ ಛಂದಃ~। ಪರಮೇಶ್ವರೀ ದೇವತಾ~। ಜಪೇ ವಿನಿಯೋಗಃ ॥
\begin{center}ನ್ಯಾಸಃ\end{center}
ದ್ಯೌರ್ಮೂರ್ಧ್ನಿ ದೈವತಂ~। ದಂತಪಂಕ್ತಾವಶ್ವಿನೌ~। ಉಭೇ ಸಂಧ್ಯೇ ಚೋಷ್ಠೌ~। ಮುಖಮಗ್ನಿಃ~। ಜಿಹ್ವಾ ಸರಸ್ವತೀ~। ಗ್ರೀವಾಯಾಂ ತು ಬೃಹಸ್ಪತಿಃ~। ಸ್ತನಯೋರ್ವಸವೋಽಷ್ಟೌ~। ಬಾಹ್ವೋರ್ಮರುತಃ~। ಹೃದಯೇ ಪರ್ಜನ್ಯಃ~। ಆಕಾಶಮುದರಂ~। ನಾಭಾವಂತರಿಕ್ಷಂ~। ಕಟ್ಯೋ ರಿಂದ್ರಾಗ್ನೀ~। ಜಘನೇ ವಿಜ್ಞಾನಘನಃ ಪ್ರಜಾಪತಿಃ~। ಕೈಲಾಸ ಮಲಯೇ ಊರೂ~। ವಿಶ್ವೇದೇವಾ ಜಾನ್ವೋಃ~। ಜಂಘಾಯಾಂ ಕೌಶಿಕಃ~। ಗುಹ್ಯಮಯನೇ~। ಊರೂ ಪಿತರಃ~। ಪಾದೌ ಪೃಥಿವೀ~। ವನಸ್ಪತ ಯೋಽಙ್ಗುಲೀಷು~। ಋಷಯೋ ರೋಮಾಣಿ~। ನಖಾನಿ ಮುಹೂರ್ತಾನಿ~। ಅಸ್ಥಿಷು ಗ್ರಹಾಃ~। ಅಸೃಙ್ಮಾಂಸಮೃತವಃ~। ಸಂವತ್ಸರಾ ವೈ ನಿಮಿಷಂ~। ಅಹೋರಾತ್ರಾವಾದಿತ್ಯಶ್ಚಂದ್ರಮಾಃ~। ಪ್ರವರಾಂ ದಿವ್ಯಾಂ ಗಾಯತ್ರೀಂ ಸಹಸ್ರನೇತ್ರಾಂ ಶರಣಮಹಂ ಪ್ರಪದ್ಯೇ~। ಓಂ ತತ್ಸವಿತುರ್ವರೇಣ್ಯಾಯ ನಮಃ~। ಓಂ ತತ್ಪೂರ್ವಾಜಯಾಯ ನಮಃ~। ತತ್ಪ್ರಾತರಾದಿತ್ಯಾಯ ನಮಃ~। ತತ್ಪ್ರಾತರಾದಿತ್ಯಪ್ರತಿಷ್ಠಾಯೈ ನಮಃ~। ಪ್ರಾತರಧೀಯಾನೋ ರಾತ್ರಿಕೃತಂ ಪಾಪಂ ನಾಶಯತಿ~। ಸಾಯಮಧೀಯಾನೋ ದಿವಸಕೃತಂ ಪಾಪಂ ನಾಶಯತಿ~। ಸಾಯಂಪ್ರಾತರಧೀಯಾನೋ ಅಪಾಪೋ ಭವತಿ~। ಸರ್ವತೀರ್ಥೇಷು ಸ್ನಾತೋ ಭವತಿ~। ಸರ್ವೈರ್ದೇವೈರ್ಜ್ಞಾತೋ ಭವತಿ~। ಅವಾಚ್ಯವಚನಾತ್ಪೂತೋ ಭವತಿ~। ಅಭಕ್ಷ್ಯಭಕ್ಷಣಾತ್ಪೂತೋ ಭವತಿ~। ಅಭೋಜ್ಯಭೋಜನಾತ್ಪೂತೋ ಭವತಿ~। ಅಚೋಷ್ಯ ಚೋಷಣಾತ್ಪೂತೋ ಭವತಿ~। ಅಸಾಧ್ಯಸಾಧನಾತ್ಪೂತೋ ಭವತಿ~। ದುಷ್ಪ್ರತಿಗ್ರಹಶತಸಹಸ್ರಾತ್ಪೂತೋ ಭವತಿ~। ಸರ್ವಪ್ರತಿಗ್ರಹಾತ್ಪೂತೋ ಭವತಿ~। ಪಂಕ್ತಿದೂಷಣಾತ್ಪೂತೋ ಭವತಿ~। ಅನೃತವಚನಾತ್ಪೂತೋ ಭವತಿ~। ಅಥಾಬ್ರಹ್ಮಚಾರೀ ಬ್ರಹ್ಮಚಾರೀ ಭವತಿ~। ಅನೇನ ಹೃದಯೇನಾಧೀತೇನ ಕ್ರತುಸಹಸ್ರೇಣೇಷ್ಟಂ ಭವತಿ~। ಷಷ್ಟಿಶತ ಸಹಸ್ರಗಾಯತ್ರ್ಯಾ ಜಪ್ಯಾನಿ ಫಲಾನಿ ಭವಂತಿ~। ಅಷ್ಟೌ ಬ್ರಾಹ್ಮಣಾನ್ಸಮ್ಯಗ್ಗ್ರಾಹಯೇತ್~। ತಸ್ಯ ಸಿದ್ಧಿರ್ಭವತಿ~। ಯ ಇದಂ ನಿತ್ಯಮಧೀಯಾನೋ ಬ್ರಾಹ್ಮಣಃ ಪ್ರಾತಃ ಶುಚಿಃ ಸರ್ವಪಾಪೈಃ ಪ್ರಮುಚ್ಯತ ಇತಿ~। ಬ್ರಹ್ಮಲೋಕೇ ಮಹೀಯತೇ~। ಇತ್ಯಾಹ ಭಗವಾನ್ ಶ್ರೀನಾರಾಯಣಃ ॥
\authorline{ಇತಿ ಶ್ರೀಮದ್ದೇವೀಭಾಗವತೇ ಮಹಾಪುರಾಣೇ ಗಾಯತ್ರೀಹೃದಯಂ ॥}
\section{ಗಾಯತ್ರೀ ಕವಚಂ}
\addcontentsline{toc}{section}{ಗಾಯತ್ರೀ ಕವಚಂ}
ಅಸ್ಯ ಶ್ರೀ ಗಾಯತ್ತ್ರೀ ಕವಚಸ್ಯ ಬ್ರಹ್ಮವಿಷ್ಣುಮಹೇಶ್ವರಾಃ ಋಷಯಃ~। ಋಗ್ಯಜುಸ್ಸಾಮಾಥರ್ವಶ್ಛಂದಾಂಸಿ~। ಪರಬ್ರಹ್ಮರೂಪಾ ಗಾಯತ್ರೀ ದೇವತಾ~। ತದ್ಬೀಜಂ~। ಭರ್ಗಃ ಶಕ್ತಿಃ~। ಧಿಯಃ ಕೀಲಕಂ~। ಜಪೇ ವಿನಿಯೋಗಃ ॥

ಗಾಯತ್ರೀ ಪೂರ್ವತಃ ಪಾತು ಸಾವಿತ್ರೀ ಪಾತು ದಕ್ಷಿಣೇ।\\
ಬ್ರಹ್ಮ ಸಂಧ್ಯಾ ತು ಮೇ ಪಶ್ಚಾದುತ್ತರಾಯಾಂ ಸರಸ್ವತೀ॥೧॥

ಪಾರ್ವತೀ ಮೇ ದಿಶಂ ರಾಕ್ಷೇ ತ್ಪಾವಕೀಂ ಜಲಶಾಯಿನೀ।\\
ಯಾತುಧಾನೀಂ ದಿಶಂ ರಕ್ಷೇ ದ್ಯಾತುಧಾನಭಯಂಕರೀ॥೨॥

ಪಾವಮಾನೀಂ ದಿಶಂ ರಕ್ಷೇತ್ಪವಮಾನ ವಿಲಾಸಿನೀ।\\
ದಿಶಂ ರೌದ್ರೀಂಚ ಮೇ ಪಾತು ರುದ್ರಾಣೀ ರುದ್ರ ರೂಪಿಣೀ॥೩॥

ಊರ್ಧ್ವಂ ಬ್ರಹ್ಮಾಣೀ ಮೇ ರಕ್ಷೇ ದಧಸ್ತಾ ದ್ವೈಷ್ಣವೀ ತಥಾ।\\
ಏವಂ ದಶ ದಿಶೋ ರಕ್ಷೇ ತ್ಸರ್ವಾಂಗಂ ಭುವನೇಶ್ವರೀ॥೪॥

ತತ್ಪದಂ ಪಾತು ಮೇ ಪಾದೌ ಜಂಘೇ ಮೇ ಸವಿತುಃ ಪದಂ।\\
ವರೇಣ್ಯಂ ಕಟಿ ದೇಶೇ ತು ನಾಭಿಂ ಭರ್ಗ ಸ್ತಥೈವ ಚ॥೫॥

ದೇವಸ್ಯ ಮೇ ತದ್ಧೃದಯಂ ಧೀಮಹೀತಿ ಚ ಗಲ್ಲಯೋಃ।\\
ಧಿಯಃ ಪದಂ ಚ ಮೇನೇತ್ರೇ ಯಃ ಪದಂ ಮೇ ಲಲಾಟಕಂ॥೬॥

ನಃ ಪಾತು ಮೇ ಪದಂ ಮೂರ್ಧ್ನಿ ಶಿಖಾಯಾಂ ಮೇ ಪ್ರಚೋದಯಾತ್।\\
ತತ್ಪದಂ ಪಾತು ಮೂರ್ಧಾನಂ ಸಕಾರಃ ಪಾತು ಫಾಲಕಂ॥೭॥

ಚಕ್ಷುಷೀ ತು ವಿಕಾರಾರ್ಣಸ್ತುಕಾರಸ್ತು ಕಪೋಲಯೋಃ।\\
ನಾಸಾಪುಟಂ ವಕಾರಾರ್ಣೋ ರೇಕಾರಸ್ತು ಮುಖೇ ತಥಾ ॥೮॥

ಣಿಕಾರ ಊರ್ಧ್ವ ಮೋಷ್ಠಂತು ಯಕಾರಸ್ತ್ವಧರೋಷ್ಠಕಂ।\\
ಆಸ್ಯಮಧ್ಯೇ ಭಕಾರಾರ್ಣೋ ರ್ಗೋಕಾರ ಶ್ಚುಬುಕೇ ತಥಾ ॥೯॥

ದೇಕಾರಃ ಕಂಠ ದೇಶೇತು ವಕಾರಃ ಸ್ಕಂಧ ದೇಶಕಂ।\\
ಸ್ಯಕಾರೋ ದಕ್ಷಿಣಂ ಹಸ್ತಂ ಧೀಕಾರೋ ವಾಮ ಹಸ್ತಕಂ॥೧೦॥

ಮಕಾರೋ ಹೃದಯಂ ರಕ್ಷೇದ್ಧಿಕಾರ ಉದರೇ ತಥಾ।\\
ಧಿಕಾರೋ ನಾಭಿ ದೇಶೇ ತು ಯೋಕಾರಸ್ತು ಕಟಿಂ ತಥಾ॥೧೧॥

ಗುಹ್ಯಂ ರಕ್ಷತು ಯೋಕಾರ ಊರೂ ದ್ವೌ ನಃ ಪದಾಕ್ಷರಂ।\\
ಪ್ರಕಾರೋ ಜಾನುನೀ ರಕ್ಷೇಚ್ಚೋಕಾರೋ ಜಂಘ ದೇಶಕಂ॥೧೨॥

ದಕಾರಂ ಗುಲ್ಫ ದೇಶೇ ತು ಯಾಕಾರಃ ಪದಯುಗ್ಮಕಂ।\\
ತಕಾರೋ ವ್ಯಂಜನಂ ಚೈವ ಸರ್ವಾಂಗಂ ಮೇ ಸದಾವತು ॥೧೩॥

ಇದಂ ತು ಕವಚಂ ದಿವ್ಯಂ ಬಾಧಾ ಶತ ವಿನಾಶನಂ।\\
ಚತುಃಷಷ್ಟಿ ಕಲಾ ವಿದ್ಯಾದಾಯಕಂ ಮೋಕ್ಷಕಾರಕಂ॥೧೪॥

ಮುಚ್ಯತೇ ಸರ್ವ ಪಾಪೇಭ್ಯಃ ಪರಂ ಬ್ರಹ್ಮಾಧಿಗಚ್ಛತಿ~।\\
ಪಠನಾ ಚ್ಛ್ರವಣಾ ದ್ವಾಪಿ ಗೋ ಸಹಸ್ರ ಫಲಂ ಲಭೇತ್ ॥೧೫॥

\authorline{॥ಶ್ರೀ ದೇವೀಭಾಗವತಾಂತರ್ಗತಂ ಗಾಯತ್ರೀ ಕವಚಂ ಸಂಪೂರ್ಣಂ ॥}
%===============================================================================================
\section{ಸೌಭಾಗ್ಯವಿದ್ಯಾ ಕವಚಂ}
\addcontentsline{toc}{section}{ಸೌಭಾಗ್ಯವಿದ್ಯಾ ಕವಚಂ}
ಅಸ್ಯ ಶ್ರೀ ಮಹಾತ್ರಿಪುರಸುಂದರೀ ಮಂತ್ರವರ್ಣಾತ್ಮಕ ಕವಚ ಮಹಾಮಂತ್ರಸ್ಯ ದಕ್ಷಿಣಾಮೂರ್ತಿರ್ಋಷಿಃ। ಅನುಷ್ಟುಪ್ ಛಂದಃ। ಶ್ರೀಮಹಾತ್ರಿಪುರಸುಂದರೀ ದೇವತಾ~। ಐಂ ಬೀಜಂ~। ಸೌಃ ಶಕ್ತಿಃ~। ಕ್ಲೀಂ ಕೀಲಕಂ~। ಮಮ ಶರೀರರಕ್ಷಣಾರ್ಥೇ ಜಪೇ ವಿನಿಯೋಗಃ~॥\\
\dhyana{ಬಾಲಾರ್ಕಮಂಡಲಾಭಾಸಾಂ ಚತುರ್ಬಾಹುಂ ತ್ರಿಲೋಚನಾಂ~।\\
ಪಾಶಾಂಕುಶ ಧನುರ್ಬಾಣಾನ್ ಧಾರಯಂತೀಂ ಶಿವಾಂ ಭಜೇ ॥}\\
\as{(ಓಂಐಂಹ್ರೀಂಶ್ರೀಂ)}\\
ಕಕಾರಃ ಪಾತು ಮೇ ಶೀರ್ಷಂ ಏಕಾರಃ ಪಾತು ಫಾಲಕಂ।\\
ಈಕಾರಃ ಪಾತು ಮೇ ವಕ್ತ್ರಂ ಲಕಾರಃ ಪಾತು ಕರ್ಣಕಂ॥೧॥

ಹ್ರೀಂಕಾರಃ ಪಾತು ಹೃದಯಂ ವಾಗ್ಭವಶ್ಚ ಸದಾವತು।\\
ಹಕಾರಃ ಪಾತು ಜಠರಂ ಸಕಾರೋ ನಾಭಿದೇಶಕಂ॥೨॥

ಕಕಾರೋವ್ಯಾದ್ವಸ್ತಿಭಾಗಂ ಹಕಾರಃ ಪಾತು ಲಿಂಗಕಂ।\\
ಲಕಾರೋ ಜಾನುನೀ ಪಾತು ಹ್ರೀಂಕಾರೋ ಜಂಘಯುಗ್ಮಕಂ॥೩॥

ಕಾಮರಾಜಃ ಸದಾ ಪಾತು ಜಠರಾದಿ ಪ್ರದೇಶಕಂ।\\
ಸಕಾರಃ ಪಾತು ಮೇ ಜಂಘೇ ಕಕಾರಃ ಪಾತು ಪೃಷ್ಠಕಂ॥೪॥

ಲಕಾರೋವ್ಯಾನ್ನಿತಂಬಂ ಮೇ ಹ್ರೀಂಕಾರಃ ಪಾತು ಮೂಲಕಂ~।\\
ಶಕ್ತಿಬೀಜಃ ಸದಾ ಪಾತು ಮೂಲಾಧಾರಾದಿ ದೇಶಕಂ॥೫॥

ತ್ರಿಪುರಾ ದೇವತಾ ಪಾತು ತ್ರಿಪುರೇಶೀ ಚ ಸರ್ವದಾ।\\
ತ್ರಿಪುರಾ ಸುಂದರೀ ಪಾತು ತ್ರಿಪುರಾಶ್ರೀ ಸ್ತಥಾವತು॥೬॥

ತ್ರಿಪುರಾ ಮಾಲಿನೀ ಪಾತು ತ್ರಿಪುರಾ ಸಿದ್ಧಿದಾ ವತು।\\
ತ್ರಿಪುರಾಂಬಾ ತಥಾ ಪಾತು ಪಾತು ತ್ರಿಪುರಭೈರವೀ॥೭॥

ಅಣಿಮಾದ್ಯಾ ಸ್ತಥಾ ಪಾಂತು ಬ್ರಾಹ್ಮ್ಯಾದ್ಯಾಃ ಪಾಂತು ಮಾಂ ಸದಾ।\\
ದಶಮುದ್ರಾಸ್ತಥಾ ಪಾಂತು ಕಾಮಾಕರ್ಷಣ ಪೂರ್ವಕಾಃ॥೮॥

ಪಾಂತು ಮಾಂ ಷೋಡಶದಲೇ ಯಂತ್ರೇನಂಗ ಕುಮಾರಿಕಾಃ।\\
ಪಾಂತು ಮಾಂ ಪೃಷ್ಠಪತ್ರೇ ತು ಸರ್ವಸಂಕ್ಷೋಭಣಾದಿಕಾಃ॥೯॥

ಪಾಂತು ಮಾಂ ದಶಕೋಣೇ ತು ಸರ್ವಸಿದ್ಧಿ ಪ್ರದಾಯಿಕಾಃ।\\
ಪಾಂತು ಮಾಂ ಬಾಹ್ಯ ದಿಕ್ಕೋಣೇ ಮಧ್ಯ ದಿಕ್ಕೋಣಕೇ ತಥಾ॥೧೦॥

ಸರ್ವಜ್ಞಾ ದ್ಯಾಸ್ತಥಾ ಪಾಂತು ಸರ್ವಾಭೀಷ್ಟ ಪ್ರದಾಯಿಕಾಃ।\\
ವಶಿನ್ಯಾದ್ಯಾಸ್ತಥಾ ಪಾಂತು ವಸು ಪತ್ರಸ್ಯ ದೇವತಾಃ॥೧೧॥

ತ್ರಿಕೋಣ ಸ್ಯಾಂತ ರಾಲೇ ತು ಪಾಂತು ಮಾಮಾಯುಧಾನಿ ಚ।\\
ಕಾಮೇಶ್ವರ್ಯಾದಿಕಾಃ ಪಾಂತು ತ್ರಿಕೋಣೇ ಕೋಣಸಂಸ್ಥಿತಾಃ॥೧೨॥

ಬಿಂದುಚಕ್ರೇ ತಥಾ ಪಾತು ಮಹಾತ್ರಿಪುರಸುಂದರೀ।\as{(ಶ್ರೀಂಹ್ರೀಂಐಂ)}\\
ಇತೀದಂ ಕವಚಂ ದೇವಿ ಕವಚಂ ಮಂತ್ರಸೂಚಕಂ॥೧೩॥

ಯಸ್ಮೈ ಕಸ್ಮೈ ನ ದಾತವ್ಯಂ ನ ಪ್ರಕಾಶ್ಯಂ ಕಥಂಚನ।\\
ಯಸ್ತ್ರಿಸಂಧ್ಯಂ ಪಠೇದ್ದೇವಿ ಲಕ್ಷ್ಮೀಸ್ತಸ್ಯ ಪ್ರಜಾಯತೇ॥೧೪॥

ಅಷ್ಟಮ್ಯಾಂ ಚ ಚತುರ್ದಶ್ಯಾಂ ಯಃ ಪಠೇತ್ ಪ್ರಯತಃ ಸದಾ।\\
ಪ್ರಸನ್ನಾ ಸುಂದರೀ ತಸ್ಯ ಸರ್ವಸಿದ್ಧಿಪ್ರದಾಯಿನೀ॥೧೫॥
\authorline{॥ಇತಿ ಶ್ರೀ ರುದ್ರಯಾಮಲೇ ತಂತ್ರೇ ತ್ರಿಪುರಾ ಹೃದಯೇ ಕವಚರಹಸ್ಯಂ ಸಂಪೂರ್ಣಂ ॥}
%=======================================================================================================================
\section{ವಿಷ್ಣುಕವಚಸ್ತೋತ್ರಮ್}
\addcontentsline{toc}{section}{ವಿಷ್ಣುಕವಚಸ್ತೋತ್ರಮ್}
ಅಸ್ಯ ಶ್ರೀವಿಷ್ಣುಕವಚಸ್ತೋತ್ರಮಹಾಮಂತ್ರಸ್ಯ ಬ್ರಹ್ಮಾ ಋಷಿಃ ಅನುಷ್ಟುಪ್ ಛಂದಃ । ಶ್ರೀಮನ್ನಾರಾಯಣೋ ದೇವತಾ । ಶ್ರೀಮನ್ನಾರಾಯಣಪ್ರಸಾದಸಿದ್ಧ್ಯರ್ಥೇ ಜಪೇ ವಿನಿಯೋಗಃ ।\\
ಓಂ ಕೇಶವಾಯ ಅಂಗುಷ್ಟಾಭ್ಯಾಂ ನಮಃ ।\\
ಓಂ ನಾರಾಯಣಾಯ ತರ್ಜನೀಭ್ಯಾಂ ನಮಃ ।\\
ಓಂ ಮಾಧವಾಯ ಮಧ್ಯಮಾಭ್ಯಾಂ ನಮಃ ।\\
ಓಂ ಗೋವಿಂದಾಯ ಅನಾಮಿಕಾಭ್ಯಾಂ ನಮಃ ।\\
ಓಂ ವಿಷ್ಣವೇ ಕನಿಷ್ಠಿಕಾಭ್ಯಾಂ ನಮಃ ।\\
ಓಂ ಮಧುಸೂದನಾಯ ಕರತಲಕರಪೃಷ್ಠಾಭ್ಯಾಂ ನಮಃ ॥

ಓಂ ತ್ರಿವಿಕ್ರಮಾಯ ಹೃದಯಾಯ ನಮಃ ।\\
ಓಂ ವಾಮನಾಯ ಶಿರಸೇ ಸ್ವಾಹಾ ।\\
ಓಂ ಶ್ರೀಧರಾಯ ಶಿಖಾಯೈ ವಷಟ್ ।\\
ಓಂ ಹೃಷೀಕೇಶಾಯ ಕವಚಾಯ ಹುಂ ।\\
ಓಂ ಪದ್ಮನಾಭಾಯ ನೇತ್ರಾಭ್ಯಾಂ ವೌಷಟ್ ।\\
ಓಂ ದಾಮೋದರಾಯ ಅಸ್ತ್ರಾಯ ಫಟ್ ।\\
ಭೂರ್ಭುವಸ್ಯುವರೋಮಿತಿ ದಿಗ್ಬಂಧಃ ॥

\dhyana{ಶಾಂತಾಕಾರಂ ಭುಜಗಶಯನಂ ಪದ್ಮನಾಭಂ ಸುರೇಶಂ\\
ವಿಶ್ವಾಕಾರಂ ಗಗನಸದೃಶಂ ಮೇಘವರ್ಣಂ ಶುಭಾಂಗಂ ।\\
ಲಕ್ಷ್ಮೀಕಾಂತಂ ಕಮಲನಯನಂ ಯೋಗಿಹೃದ್ಧ್ಯಾನಗಮ್ಯಂ\\
ವಂದೇ ವಿಷ್ಣುಂ ಭವಭಯಹರಂ ಸರ್ವಲೋಕೈಕನಾಥಂ ॥

ಮೇಘಶ್ಯಾಮಂ ಪೀತಕೌಶೇಯವಾಸಂ\\ ಶ್ರೀವತ್ಸಾಂಗಂ ಕೌಸ್ತುಭೋದ್ಭಾಸಿತಾಂಗಂ ।\\
ಪುಣ್ಯೋಪೇತಂ ಪುಂಡರೀಕಾಯತಾಕ್ಷಂ\\ ವಿಷ್ಣುಂ ವಂದೇ ಸರ್ವಲೋಕೈಕನಾಥಂ ॥

ಸಶಂಖಚಕ್ರಂ ಸಕಿರೀಟಕುಂಡಲಂ ಸಪೀತವಸ್ತ್ರಂ ಸರಸೀರುಹೇಕ್ಷಣಂ ।\\
ಸಹಾರವಕ್ಷಸ್ಥಲಶೋಭಿಕೌಸ್ತುಭಂ, ನಮಾಮಿವಿಷ್ಣುಂ ಶಿರಸಾಚತುರ್ಭುಜಂ ॥}

ಓಂ ಪೂರ್ವತೋ ಮಾಂ ಹರಿಃ ಪಾತು ಪಶ್ಚಾತ್ ಶ್ರೀಃ ಸದಕ್ಷಿಣೇ ।\\
ಶ್ರೀಕೃಷ್ಣ ಉತ್ತರೇ ಪಾತು ಶ್ರೀ ಗೋ ವಿಷ್ಣುಶ್ಚ ಸರ್ವಶಃ ॥

ಊರ್ಧ್ವಂ ಮೇ ನಂದನೀ ಪಾತು ಅಧಸ್ತಾತ್ ಶಾರ್ಙ್ಗಭೃತ್ ಸದಾ ।\\
ಪಾದೌ ಪಾತು ಸರೋಜಾಂಗೀ ಅಂಗೇ ಪಾತು ಜನಾರ್ದನಃ ॥

ಜಾನುನೀ ಮೇ ಜಗನ್ನಾಥಃ ಊರೂ ಪಾತು ತ್ರಿವಿಕ್ರಮಃ ।\\
ಗುಹ್ಯಂ ಪಾತು ಹೃಷೀಕೇಶಃ ಪೃಷ್ಠಂ ಪಾತು ಮಮಾವ್ಯಯಃ ॥

ಪಾತು ನಾಭಿಂ ಮಮಾನಂತಃ ಕುಕ್ಷಿಂ ರಾಕ್ಷಸಮರ್ದನಃ ।\\
ದಾಮೋದರೋ ಮೇ ಹೃದಯಂ ವಕ್ಷಃ ಪಾತು ನೃಕೇಸರೀ ॥

ಕರೌ ಮೇ ಕಾಲಿಯಾರಾತಿಃ ಭುಜೌ ಭಕ್ತಾರ್ತಿಭಂಜನಃ ।\\
ಕಂಠಂ ಕಾಲಾಂಬುದಶ್ಯಾಮಃ ಸ್ಕಂಧೌ ಮೇ ಕಂಸಮರ್ದನಃ ॥

ನಾರಾಯಣೋ ಮೇಽವ್ಯಾನ್ನಾಸಾಂ ಕರ್ಣೌ ಮೇ ಚ ಪ್ರಭಂಜನಃ ।\\
ಕಪಾಲಂ ಪಾತು ವೈಕುಂಠಃ ಜಿಹ್ವಾಂ ಪಾತು ದಯಾನಿಧಿಃ ॥

ಆಸ್ಯಂ ದಶಾಸ್ಯಹಂತಾವ್ಯಾತ್ ನೇತ್ರೇ ಮೇ ಪದ್ಮಲೋಚನಃ ।\\
ಭ್ರುವೌ ಮೇ ಪಾತು ಭೂಮಾ ಚ ಲಲಾಟಂ ಮೇ ಸದಾಚ್ಯುತಃ ॥

ಮುಖಂ ಮೇ ಪಾತು ಗೋವಿಂದಃ ಶಿಖಾಂ ಗರುಡವಾಹನಃ ।\\
ಮಾಂ ಶೇಷಶಾಯೀ ಸರ್ವೇಭ್ಯೋ ವ್ಯಾಧಿಭ್ಯೋ ಭಕ್ತವತ್ಸಲಃ ॥

ಪಿಶಾಚಾಗ್ನಿಜ್ವರೇಭ್ಯೋ ಮಾಂ ಆಪದ್ಭ್ಯೋಽವತು ಮಾಧವಃ ।\\
ಸರ್ವೇಭ್ಯೋ ದುರಿತೇಭ್ಯಶ್ಚ ಪಾತು ಮಾಂ ಪುರುಷೋತ್ತಮಃ ॥

ಇದಂ ಶ್ರೀವಿಷ್ಣುಕವಚಂ ಸರ್ವಮಂಗಲದಾಯಕಂ ।\\
ಸರ್ವರೋಗಪ್ರಶಮನಂ, ಸರ್ವಶತ್ರುವಿನಾಶನಂ ॥

ಏವಂ ಜಜಾಪ ತತ್ಕಾಲೇ ಸ್ಯಾತ್ಪರಶ್ಚಾಕ್ಷರಂ ಪರಂ ।\\
ತ್ರಿಸ್ಸಂಧ್ಯಂ ಯಃ ಪಠೇಚ್ಛುದ್ಧಃ ಸರ್ವತ್ರ ವಿಜಯೀ ಭವೇತ್ ॥
\authorline{ಇತಿ ಶ್ರೀವಿಷ್ಣುಕವಚಸ್ತೋತ್ರಂ ಸಂಪೂರ್ಣಂ ।}
%========================================================================================================================================================
\section{ಶಿವಕವಚಂ }
\addcontentsline{toc}{section}{ಶಿವಕವಚಂ }
ಅಸ್ಯ ಶ್ರೀ ಶಿವಕವಚ ಸ್ತೋತ್ರಮಹಾಮಂತ್ರಸ್ಯ ಋಷಭಯೋಗೀಶ್ವರ ಋಷಿಃ~। ಅನುಷ್ಟುಪ್ ಛಂದಃ~। ಶ್ರೀಸದಾಶಿವೋ ದೇವತಾ~। ಓಂ ಬೀಜಂ~। ನಮಃ ಶಕ್ತಿಃ~। ಶಿವಾಯೇತಿ ಕೀಲಕಂ~। ಸದಾಶಿವಪ್ರೀತ್ಯರ್ಥೇ ಜಪೇ ವಿನಿಯೋಗಃ~।\\
ಓಂ ನಮೋ ಭಗವತೇ ಜ್ವಲ ಜ್ವಲ ಮಹಾರುದ್ರಾಯ ಶ್ರೀಂ ಹ್ರೀಂ ಕ್ಲೀಂ(*)\\
\as{೧.} * ಹ್ರಾಂ ಸರ್ವಶಕ್ತಿ ಧಾಮ್ನೇ
\as{೨.} * ನಂ ತೃಪ್ತಿಶಕ್ತಿಧಾಮ್ನೇ\\
\as{೩.} * ಮಂ ಅನಾದಿಬೋಧಶಕ್ತಿಧಾಮ್ನೇ
\as{೪.} * ಶಿಂ ಸ್ವತಂತ್ರಶಕ್ತಿಧಾಮ್ನೇ\\
\as{೫.} * ವಾಂ ಅಲುಪ್ತಶಕ್ತಿಧಾಮ್ನೇ
\as{೬.} * ಯಂ ಅನಂತಶಕ್ತಿಧಾಮ್ನೇ ಇತಿ ನ್ಯಾಸಃ\\
\dhyana{ವಜ್ರದಂಷ್ಟ್ರಂ ತ್ರಿನಯನಂ ಕಾಲಕಂಠಮರಿಂದಮಂ~।\\
ಸಹಸ್ರಕರಮತ್ಯುಗ್ರಂ ವಂದೇ ಶಂಭುಂ ಉಮಾಪತಿಂ ॥}

ಋಷಭ ಉವಾಚ ॥\\
ನಮಸ್ಕೃತ್ಯ ಮಹಾದೇವಂ ವಿಶ್ವವ್ಯಾಪಿನಮೀಶ್ವರಂ~।\\
ವಕ್ಷ್ಯೇ ಶಿವಮಯಂ ವರ್ಮ ಸರ್ವರಕ್ಷಾಕರಂ ನೃಣಾಂ ॥೧॥\\
ಶುಚೌ ದೇಶೇ ಸಮಾಸೀನೋ ಯಥಾವತ್ಕಲ್ಪಿತಾಸನಃ~।\\
ಜಿತೇಂದ್ರಿಯೋ ಜಿತಪ್ರಾಣಶ್ಚಿಂತಯೇಚ್ಛಿವಮವ್ಯಯಂ ॥೨॥

ಹೃತ್ಪುಂಡರೀಕಾಂತರಸನ್ನಿವಿಷ್ಟಂ\\ಸ್ವತೇಜಸಾ ವ್ಯಾಪ್ತನಭೋಽವಕಾಶಂ~।\\
ಅತೀಂದ್ರಿಯಂ ಸೂಕ್ಷ್ಮಮನಂತಮಾದ್ಯಂ\\ಧ್ಯಾಯೇತ್ ಪರಾನಂದಮಯಂ ಮಹೇಶಂ ॥೩॥

ಧ್ಯಾನಾವಧೂತಾಖಿಲಕರ್ಮಬಂಧ\\ಶ್ಚಿರಂ ಚಿದಾನಂದನಿಮಗ್ನಚೇತಾಃ~।\\
ಷಡಕ್ಷರನ್ಯಾಸ ಸಮಾಹಿತಾತ್ಮಾ\\ಶೈವೇನ ಕುರ್ಯಾತ್ಕವಚೇನ ರಕ್ಷಾಂ ॥೪॥

ಮಾಂ ಪಾತು ದೇವೋಽಖಿಲದೇವತಾತ್ಮಾ\\ಸಂಸಾರಕೂಪೇ ಪತಿತಂ ಗಭೀರೇ~।\\
ತನ್ನಾಮ ದಿವ್ಯಂ ಪರಮಂತ್ರಮೂಲಂ\\ಧುನೋತು ಮೇ ಸರ್ವಮಘಂ ಹೃದಿಸ್ಥಂ ॥೫॥

ಸರ್ವತ್ರ ಮಾಂ ರಕ್ಷತು ವಿಶ್ವಮೂರ್ತಿ\\ರ್ಜ್ಯೋತಿರ್ಮಯಾನಂದಘನಶ್ಚಿದಾತ್ಮಾ~।\\
ಅಣೋರಣೀಯಾನುರುಶಕ್ತಿರೇಕಃ\\ಸ ಈಶ್ವರಃ ಪಾತು ಭಯಾದಶೇಷಾತ್ ॥೬॥

ಯೋ ಭೂಸ್ವರೂಪೇಣ ಬಿಭರ್ತಿ ವಿಶ್ವಂ\\ಪಾಯಾತ್ಸ ಭೂಮೇರ್ಗಿರಿಶೋಽಷ್ಟಮೂರ್ತಿಃ~।\\
ಯೋಽಪಾಂ ಸ್ವರೂಪೇಣ ನೃಣಾಂ ಕರೋತಿ\\ಸಂಜೀವನಂ ಸೋಽವತು ಮಾಂ ಜಲೇಭ್ಯಃ ॥೭॥

ಕಲ್ಪಾವಸಾನೇ ಭುವನಾನಿ ದಗ್ಧ್ವಾ\\ಸರ್ವಾಣಿ ಯೋ ನೃತ್ಯತಿ ಭೂರಿಲೀಲಃ~।\\
ಸ ಕಾಲರುದ್ರೋಽವತು ಮಾಂ ದವಾಗ್ನೇ\\ರ್ವಾತ್ಯಾದಿಭೀತೇರಖಿಲಾಚ್ಚ ತಾಪಾತ್ ॥೮॥

ಪ್ರದೀಪ್ತವಿದ್ಯುತ್ಕನಕಾವಭಾಸೋ\\ವಿದ್ಯಾವರಾಭೀತಿಕುಠಾರಪಾಣಿಃ~।\\
ಚತುರ್ಮುಖಸ್ತತ್ಪುರುಷಸ್ತ್ರಿನೇತ್ರಃ\\ಪ್ರಾಚ್ಯಾಂ ಸ್ಥಿತೋ ರಕ್ಷತು ಮಾಮಜಸ್ರಂ ॥೯॥

ಕುಠಾರಖೇಟಾಂಕುಶಶೂಲಢಕ್ಕಾ\\ಕಪಾಲಪಾಶಾಕ್ಷಗುಣಾನ್ ದಧಾನಃ~।\\
ಚತುರ್ಮುಖೋ ನೀಲರುಚಿಸ್ತ್ರಿನೇತ್ರಃ\\ಪಾಯಾದಘೋರೋ ದಿಶಿ ದಕ್ಷಿಣಸ್ಯಾಂ ॥೧೦॥

ಕುಂದೇಂದುಶಂಖಸ್ಫಟಿಕಾವಭಾಸೋ\\ವೇದಾಕ್ಷಮಾಲಾವರದಾಭಯಾಂಕಃ~।\\
ತ್ರ್ಯಕ್ಷಶ್ಚತುರ್ವಕ್ತ್ರ ಉರುಪ್ರಭಾವಃ\\ಸದ್ಯೋಽಧಿಜಾತೋಽವತು ಮಾಂ ಪ್ರತೀಚ್ಯಾಂ ॥೧೧॥

 ವರಾಕ್ಷಮಾಲಾಭಯಟಂಕಹಸ್ತಃ\\ಸರೋಜಕಿಂಜಲ್ಕಸಮಾನವರ್ಣಃ~।\\
ತ್ರಿಲೋಚನಶ್ಚಾರುಚತುರ್ಮುಖೋ ಮಾಂ\\ಪಾಯಾದುದೀಚ್ಯಾಂ ದಿಶಿ ವಾಮದೇವಃ ॥೧೨॥

ವೇದಾಭಯೇಷ್ಟಾಂಕುಶಟಂಕಪಾಶ\\ಕಪಾಲಢಕ್ಕಾಕ್ಷರಶೂಲಪಾಣಿಃ~।\\
ಸಿತದ್ಯುತಿಃ ಪಂಚಮುಖೋಽವತಾನ್ಮಾಂ\\ಈಶಾನ ಊರ್ಧ್ವಂ ಪರಮಪ್ರಕಾಶಃ ॥೧೩॥

ಮೂರ್ಧಾನಮವ್ಯಾನ್ಮಮ ಚಂದ್ರಮೌಲಿ\\ರ್ಭಾಲಂ ಮಮಾವ್ಯಾದಥ ಭಾಲನೇತ್ರಃ~।\\
ನೇತ್ರೇ ಮಮಾವ್ಯಾದ್ಭಗನೇತ್ರಹಾರೀ\\ನಾಸಾಂ ಸದಾ ರಕ್ಷತು ವಿಶ್ವನಾಥಃ ॥೧೪॥

ಪಾಯಾಚ್ಛ್ರುತೀ ಮೇ ಶ್ರುತಿಗೀತಕೀರ್ತಿಃ\\ಕಪೋಲಮವ್ಯಾತ್ಸತತಂ ಕಪಾಲೀ~।\\
ವಕ್ತ್ರಂ ಸದಾ ರಕ್ಷತು ಪಂಚವಕ್ತ್ರೋ\\ಜಿಹ್ವಾಂ ಸದಾ ರಕ್ಷತು ವೇದಜಿಹ್ವಃ ॥೧೫॥

ಕಂಠಂ ಗಿರೀಶೋಽವತು ನೀಲಕಂಠಃ\\ಪಾಣಿದ್ವಯಂ ಪಾತು ಪಿನಾಕಪಾಣಿಃ~।\\
ದೋರ್ಮೂಲಮವ್ಯಾನ್ಮಮ ಧರ್ಮಬಾಹು\\ರ್ವಕ್ಷಃಸ್ಥಲಂ ದಕ್ಷಮಖಾಂತಕೋಽವ್ಯಾತ್ ॥೧೬॥

ಮಮೋದರಂ ಪಾತು ಗಿರೀಂದ್ರಧನ್ವಾ\\ಮಧ್ಯಂ ಮಮಾವ್ಯಾನ್ಮದನಾಂತಕಾರೀ~।\\
ಹೇರಂಬತಾತೋ ಮಮ ಪಾತು ನಾಭಿಂ\\ಪಾಯಾತ್ಕಟಿಂ ಧೂರ್ಜಟಿರೀಶ್ವರೋ ಮೇ ॥೧೭॥

ಊರುದ್ವಯಂ ಪಾತು ಕುಬೇರಮಿತ್ರೋ\\ಜಾನುದ್ವಯಂ ಮೇ ಜಗದೀಶ್ವರೋಽವ್ಯಾತ್~।\\
ಜಂಘಾಯುಗಂ ಪುಂಗವಕೇತುರವ್ಯಾತ್\\ಪಾದೌ ಮಮಾವ್ಯಾತ್ಸುರವಂದ್ಯಪಾದಃ ॥೧೮॥

ಮಹೇಶ್ವರಃ ಪಾತು ದಿನಾದಿಯಾಮೇ\\ಮಾಂ ಮಧ್ಯಯಾಮೇಽವತು ವಾಮದೇವಃ~।\\
ತ್ರಿಲೋಚನಃ ಪಾತು ತೃತೀಯಯಾಮೇ\\ವೃಷಧ್ವಜಃ ಪಾತು ದಿನಾಂತ್ಯಯಾಮೇ ॥೨೦॥

ಪಾಯಾನ್ನಿಶಾದೌ ಶಶಿಶೇಖರೋ ಮಾಂ\\ಗಂಗಾಧರೋ ರಕ್ಷತು ಮಾಂ ನಿಶೀಥೇ~।\\
ಗೌರೀಪತಿಃ ಪಾತು ನಿಶಾವಸಾನೇ\\ಮೃತ್ಯುಂಜಯೋ ರಕ್ಷತು ಸರ್ವಕಾಲಂ ॥೨೧॥

ಅಂತಃಸ್ಥಿತಂ ರಕ್ಷತು ಶಂಕರೋ ಮಾಂ\\ಸ್ಥಾಣುಃ ಸದಾ ಪಾತು ಬಹಿಃಸ್ಥಿತಂ ಮಾಂ~।\\
ತದಂತರೇ ಪಾತು ಪತಿಃ ಪಶೂನಾಂ\\ಸದಾಶಿವೋ ರಕ್ಷತು ಮಾಂ ಸಮಂತಾತ್ ॥೨೨॥

ತಿಷ್ಠಂತಮವ್ಯಾದ್ ಭುವನೈಕನಾಥಃ\\ಪಾಯಾದ್ವ್ರಜಂತಂ ಪ್ರಮಥಾಧಿನಾಥಃ~।\\
ವೇದಾಂತವೇದ್ಯೋಽವತು ಮಾಂ ನಿಷಣ್ಣಂ\\ ಮಾಮವ್ಯಯಃ ಪಾತು ಶಿವಃ ಶಯಾನಂ ॥೨೩॥

ಮಾರ್ಗೇಷು ಮಾಂ ರಕ್ಷತು ನೀಲಕಂಠಃ\\ಶೈಲಾದಿದುರ್ಗೇಷು ಪುರತ್ರಯಾರಿಃ~।\\
ಅರಣ್ಯವಾಸಾದಿ ಮಹಾಪ್ರವಾಸೇ\\ಪಾಯಾನ್ಮೃಗವ್ಯಾಧ ಉದಾರಶಕ್ತಿಃ ॥೨೪॥

ಕಲ್ಪಾಂತಕಾಲೋಗ್ರಪಟುಪ್ರಕೋಪ\\ಸ್ಫುಟಾಟ್ಟಹಾಸೋಚ್ಚಲಿತಾಂಡಕೋಶಃ~।\\
ಘೋರಾರಿಸೇನಾರ್ಣವದುರ್ನಿವಾರ\\ಮಹಾಭಯಾದ್ರಕ್ಷತು ವೀರಭದ್ರಃ ॥೨೫॥

ಪತ್ತ್ಯಶ್ವಮಾತಂಗರಥಾವರೂಥಿನೀ\\ಸಹಸ್ರಲಕ್ಷಾಯುತ ಕೋಟಿಭೀಷಣಂ~।\\
ಅಕ್ಷೌಹಿಣೀನಾಂ ಶತಮಾತತಾಯಿನಾಂ\\ಛಿಂದ್ಯಾನ್ಮೃಡೋ ಘೋರಕುಠಾರ ಧಾರಯಾ ॥೨೬॥

ನಿಹಂತು ದಸ್ಯೂನ್ಪ್ರಲಯಾನಲಾರ್ಚಿಃ\\ಜ್ವಲತ್ತ್ರಿಶೂಲಂ ತ್ರಿಪುರಾಂತಕಸ್ಯ~।\\
ಶಾರ್ದೂಲಸಿಂಹರ್ಕ್ಷವೃಕಾದಿಹಿಂಸ್ರಾನ್\\ಸಂತ್ರಾಸಯತ್ವೀಶಧನುಃ ಪಿನಾಕಃ ॥೨೭॥

ದುಃಸ್ವಪ್ನದುಃಶಕುನದುರ್ಗತಿದೌರ್ಮನಸ್ಯ\\ದುರ್ಭಿಕ್ಷದುರ್ವ್ಯಸನದುಃಸಹದುರ್ಯಶಾಂಸಿ~।\\
ಉತ್ಪಾತತಾಪವಿಷಭೀತಿಮಸದ್ಗ್ರಹಾರ್ತಿಂ\\ವ್ಯಾಧೀಂಶ್ಚ ನಾಶಯತು ಮೇ ಜಗತಾಮಧೀಶಃ ॥೨೮॥

ಓಂ ನಮೋ ಭಗವತೇ ಸದಾಶಿವಾಯ ಸಕಲ ತತ್ವಾತ್ಮಕಾಯ ಸರ್ವಮಂತ್ರ ಸ್ವರೂಪಾಯ ಸರ್ವಯಂತ್ರಾಧಿಷ್ಠಿತಾಯ ಸರ್ವತಂತ್ರ ಸ್ವರೂಪಾಯ ಸರ್ವತತ್ವ ವಿದೂರಾಯ ಬ್ರಹ್ಮ ರುದ್ರಾವತಾರಿಣೇ ನೀಲಕಂಠಾಯ ಪಾರ್ವತೀ ಮನೋಹರ ಪ್ರಿಯಾಯ ಸೋಮ ಸೂರ್ಯಾಗ್ನಿ ಲೋಚನಾಯ ಭಸ್ಮೋದ್ಧೂಲಿತ ವಿಗ್ರಹಾಯ ಮಹಾಮಣಿ ಮುಕುಟ ಧಾರಣಾಯ ಮಾಣಿಕ್ಯ ಭೂಷಣಾಯ ಸೃಷ್ಟಿ ಸ್ಥಿತಿ ಪ್ರಲಯ ಕಾಲ ರೌದ್ರಾವತಾರಾಯ ದಕ್ಷಾಧ್ವರ ಧ್ವಂಸಕಾಯ ಮಹಾಕಾಲ ಭೇದನಾಯ ಮೂಲಾಧಾರೈಕ ನಿಲಯಾಯ ತತ್ವಾತೀತಾಯ ಗಂಗಾಧರಾಯ ಸರ್ವದೇವಾಧಿ ದೇವಾಯ ಷಡಾಶ್ರಯಾಯ ವೇದಾಂತ ಸಾರಾಯ ತ್ರಿವರ್ಗ ಸಾಧನಾಯ ಅನಂತ ಕೋಟಿ ಬ್ರಹ್ಮಾಂಡ ನಾಯಕಾಯ ಅನಂತ ವಾಸುಕಿ ತಕ್ಷಕ ಕರ್ಕೋಟಕ ಶಂಖ ಕುಲಿಕ ಪದ್ಮ ಮಹಾಪದ್ಮೇತಿ ಅಷ್ಟ ಮಹಾನಾಗಕುಲ ಭೂಷಣಾಯ ಪ್ರಣವ ಸ್ವರೂಪಾಯ ಚಿದಾಕಾಶಾಯ ಆಕಾಶ ದಿಕ್ ಸ್ವರೂಪಾಯ ಗ್ರಹ ನಕ್ಷತ್ರ ಮಾಲಿನೇ ಸಕಲಾಯ ಕಲಂಕ ರಹಿತಾಯ ಸಕಲ ಲೋಕೈಕ ಕರ್ತ್ರೇ ಸಕಲ ಲೋಕೈಕ ಭರ್ತ್ರೇ ಸಕಲ ಲೋಕೈಕ ಸಂಹರ್ತ್ರೇ ಸಕಲ ಲೋಕೈಕ ಗುರವೇ ಸಕಲ ಲೋಕೈಕ ಸಾಕ್ಷಿಣೇ ಸಕಲ ನಿಗಮಗುಹ್ಯಾಯ ಸಕಲ ವೇದಾಂತಪಾರಗಾಯ ಸಕಲ ಲೋಕೈಕ ವರಪ್ರದಾಯ ಸಕಲ ಲೋಕೈಕ ಶಂಕರಾಯ ಸಕಲ ದುರಿತಾರ್ತಿ ಭಂಜನಾಯ ಸಕಲ ಜಗದಭಯಂಕರಾಯ ಶಶಾಂಕ ಶೇಖರಾಯ ಶಾಶ್ವತ ನಿಜಾವಾಸಾಯ ನಿರಾಕಾರಾಯ ನಿರಾಭಾಸಾಯ ನಿರಾಮಯಾಯ ನಿರ್ಮಲಾಯ ನಿರ್ಮದಾಯ ನಿಶ್ಚಿಂತಾಯ ನಿರಹಂಕಾರಾಯ ನಿರಂಕುಶಾಯ ನಿಷ್ಕಲಂಕಾಯ ನಿರ್ಗುಣಾಯ ನಿಷ್ಕಾಮಾಯ ನಿರುಪಪ್ಲವಾಯ ನಿರುಪದ್ರವಾಯ ನಿರವದ್ಯಾಯ ನಿರಂತರಾಯ ನಿಷ್ಕಾರಣಾಯ ನಿರಾತಂಕಾಯ ನಿಷ್ಪ್ರಪಂಚಾಯ ನಿಸ್ಸಂಗಾಯ ನಿರ್ದ್ವಂದ್ವಾಯ ನಿರಾಧಾರಾಯ ನೀರಾಗಾಯ ನಿಷ್ಕ್ರೋಧಾಯ ನಿರ್ಲೋಪಾಯ ನಿಷ್ಪಾಪಾಯ ನಿರ್ಭಯಾಯ ನಿರ್ವಿಕಲ್ಪಾಯ ನಿರ್ಭೇದಾಯ ನಿಷ್ಕ್ರಿಯಾಯ ನಿಸ್ತುಲಾಯ ನಿಃಸಂಶಯಾಯ ನಿರಂಜನಾಯ ನಿರುಪಮ ವಿಭವಾಯ ನಿತ್ಯ ಶುದ್ಧ ಬುದ್ಧ ಮುಕ್ತ ಪರಿಪೂರ್ಣ ಸಚ್ಚಿದಾನಂದಾದ್ವಯಾಯ ಪರಮ ಶಾಂತ ಸ್ವರೂಪಾಯ ಪರಮ ಶಾಂತ ಪ್ರಕಾಶಾಯ ತೇಜೋರೂಪಾಯ ತೇಜೋಮಯಾಯ ತೇಜೋಽಧಿಪತಯೇ ಜಯ ಜಯ ರುದ್ರ ಮಹಾರುದ್ರ ಮಹಾರೌದ್ರ ಭದ್ರಾವತಾರ ಮಹಾಭೈರವ ಕಾಲಭೈರವ ಕಲ್ಪಾಂತಭೈರವ ಕಪಾಲ ಮಾಲಾಧರ ಖಟ್ವಾಂಗ ಚರ್ಮಖಡ್ಗಧರ ಪಾಶಾಂಕುಶ ಡಮರುಶೂಲ ಚಾಪ ಬಾಣ ಗದಾ ಶಕ್ತಿ ಭಿಂದಿಪಾಲ ತೋಮರ ಮುಸಲ ಮುದ್ಗರ ಪಾಶ ಪರಿಘ ಭುಶುಂಡೀ ಶತಘ್ನೀ ಚಕ್ರಾದ್ಯಾಯುಧ ಭೀಷಣಾಕಾರ ಸಹಸ್ರಮುಖ ದಂಷ್ಟ್ರಾಕರಾಲವದನ ವಿಕಟಾಟ್ಟಹಾಸ ವಿಸ್ಫಾರಿತ ಬ್ರಹ್ಮಾಂಡಮಂಡಲ ನಾಗೇಂದ್ರಕುಂಡಲ ನಾಗೇಂದ್ರಹಾರ ನಾಗೇಂದ್ರವಲಯ ನಾಗೇಂದ್ರಚರ್ಮಧರ ನಗೇಂದ್ರನಿಕೇತನ ಮೃತ್ಯುಂಜಯ ತ್ರ್ಯಂಬಕ ತ್ರಿಪುರಾಂತಕ ವಿಶ್ವರೂಪ ವಿರೂಪಾಕ್ಷ ವಿಶ್ವೇಶ್ವರ ವೃಷಭವಾಹನ ವಿಷವಿಭೂಷಣ ವಿಶ್ವತೋಮುಖ ಸರ್ವತೋಮುಖ ಮಾಂ ರಕ್ಷ ರಕ್ಷ ಜ್ವಲ ಜ್ವಲ ಪ್ರಜ್ವಲ ಪ್ರಜ್ವಲ ಮಹಾಮೃತ್ಯುಭಯಂ ಶಮಯ ಶಮಯ ಅಪಮೃತ್ಯುಭಯಂ ನಾಶಯ ನಾಶಯ ರೋಗಭಯಂ ಉತ್ಸಾದಯೋತ್ಸಾದಯ ವಿಷಸರ್ಪಭಯಂ ಶಮಯ ಶಮಯ ಚೋರಾನ್ ಮಾರಯ ಮಾರಯ ಮಮ ಶತ್ರೂನ್ ಉಚ್ಚಾಟಯೋಚ್ಚಾಟಯ ತ್ರಿಶೂಲೇನ ವಿದಾರಯ ವಿದಾರಯ ಕುಠಾರೇಣ ಭಿಂಧಿ ಭಿಂಧಿ ಖಡ್ಗೇನ ಛಿಂಧಿ ಛಿಂಧಿ ಖಟ್ವಾಂಗೇನ ವಿಪೋಥಯ ವಿಪೋಥಯ ಮುಸಲೇನ ನಿಷ್ಪೇಷಯ ನಿಷ್ಪೇಷಯ ಬಾಣೈಃ ಸಂತಾಡಯ ಸಂತಾಡಯ ಯಕ್ಷ ರಕ್ಷಾಂಸಿ ಭೀಷಯ ಭೀಷಯ ಅಶೇಷ ಭೂತಾನ್ ವಿದ್ರಾವಯ ವಿದ್ರಾವಯ ಕೂಷ್ಮಾಂಡ ಭೂತ ವೇತಾಲ ಮಾರೀಗಣ ಬ್ರಹ್ಮರಾಕ್ಷಸಗಣಾನ್ ಸಂತ್ರಾಸಯ ಸಂತ್ರಾಸಯ ಮಮ ಅಭಯಂ ಕುರು ಕುರು ಮಮ ಪಾಪಂ ಶೋಧಯ ಶೋಧಯ ವಿತ್ರಸ್ತಂ ಮಾಂ ಆಶ್ವಾಸಯ ಆಶ್ವಾಸಯ ನರಕ ಮಹಾಭಯಾನ್ ಮಾಂ ಉದ್ಧರ ಉದ್ಧರ ಅಮೃತ ಕಟಾಕ್ಷ ವೀಕ್ಷಣೇನ ಮಾಂ ಆಲೋಕಯ ಆಲೋಕಯ ಸಂಜೀವಯ ಸಂಜೀವಯ ಕ್ಷುತ್ತೃಷ್ಣಾರ್ತಂ ಮಾಂ ಆಪ್ಯಾಯಯ ಆಪ್ಯಾಯಯ ದುಃಖಾತುರಂ ಮಾಂ ಆನಂದಯ ಆನಂದಯ ಶಿವಕವಚೇನ ಮಾಂ ಆಚ್ಛಾದಯ ಆಚ್ಛಾದಯ ಹರ ಹರ ಮೃತ್ಯುಂಜಯ ತ್ರ್ಯಂಬಕ ಸದಾಶಿವ ಪರಮಶಿವ ನಮಸ್ತೇ ನಮಸ್ತೇ ನಮಃ॥
%==============================================================================================================
\section{ಶ್ರೀ ದಕ್ಷಿಣಾಮೂರ್ತಿ ಕವಚಂ}
\addcontentsline{toc}{section}{ಶ್ರೀ ದಕ್ಷಿಣಾಮೂರ್ತಿ ಕವಚಂ}
ಪಾರ್ವತ್ಯುವಾಚ ॥\\
ನಮಸ್ತೇಸ್ತು ತ್ರಯೀನಾಥ ಪರಮಾನಂದ ಕಾರಕ~।\\
ಕವಚಂ ದಕ್ಷಿಣಾಮೂರ್ತೇಃ ಕೃಪಯಾ ವದ ಮೇ ಪ್ರಭೋ ॥

ಈಶ್ವರ ಉವಾಚ ॥\\
ವಕ್ಷ್ಯೇಹಂ ದೇವ ದೇವೇಶಿ ದಕ್ಷಿಣಾಮೂರ್ತಿರವ್ಯಯಂ।\\
ಕವಚಂ ಸರ್ವಪಾಪಘ್ನಂ ವೇದಾನಾಂ ಜ್ಞಾನಗೋಚರಂ॥

ಅಣಿಮಾದಿ ಮಹಾಸಿದ್ಧಿವಿಧಾನಚತುರಂ ಶುಭಂ।\\
ವೇದಶಾಸ್ತ್ರಪುರಾಣಾನಿ ಕವಿತಾತರ್ಕ ಏವ ಚ ॥

ಬಹುಧಾ ದೇವಿ ಜಾಯಂತೇ ಕವಚಸ್ಯ ಪ್ರಭಾವತಃ~।\\
ಋಷಿರ್ಬ್ರಹ್ಮಾ ಸಮುದ್ದಿಷ್ಟಶ್ಛಂದೋಽನುಷ್ಟುಬುದಾಹೃತಂ॥

ದೇವತಾ ದಕ್ಷಿಣಾಮೂರ್ತಿಃ ಪರಮಾತ್ಮಾ ಸದಾಶಿವಃ।\\
ಬೀಜಂ ವೇದಾದಿಕಂ ಚೈವ ಸ್ವಾಹಾ ಶಕ್ತಿರುದಾಹೃತಾ।\\
ಸರ್ವಜ್ಞತ್ವೇಪಿ ದೇವೇಶಿ ವಿನಿಯೋಗಃ ಪ್ರಚಕ್ಷ್ಯತೇ॥

\dhyana{ಅದ್ವಂದ್ವನೇತ್ರಮಮಲೇಂದು ಕಲಾವತಂಸಂ\\
ಹಂಸಾವಲಂಬಿತಸಮಾನಜಟಾಕಲಾಪಂ~।\\
ಆನೀಲಕಂಠಮುಪಕಂಠಮುನಿಪ್ರವೀರಾನ್\\
ಅಧ್ಯಾಪಯಂತಮವಲೋಕಯ ಲೋಕನಾಥಂ॥}

ಶಿರೋ ಮೇ ದಕ್ಷಿಣಾಮೂರ್ತಿರವ್ಯಾತ್ ಫಾಲಂ ಮಹೇಶ್ವರಃ।\\
ದೃಶೌ ಪಾತು ಮಹಾದೇವಃ ಶ್ರವಣೇ ಚಂದ್ರಶೇಖರಃ॥೧॥

ಕಪೋಲೌ ಪಾತು ಮೇ ರುದ್ರೋ ನಾಸಾಂ ಪಾತು ಜಗದ್ಗುರುಃ।\\
ಮುಖಂ ಗೌರೀಪತಿಃ ಪಾತು ರಸನಾಂ ವೇದರೂಪಧೃತ್।\\
ದಶನಾನ್ ತ್ರಿಪುರಧ್ವಂಸೀ ಚೌಷ್ಠಂ ಪನ್ನಗಭೂಷಣಃ ॥೨॥

ಅಧರಂ ಪಾತು ವಿಶ್ವಾತ್ಮಾ ಹನೂ ಪಾತು ಜಗನ್ಮಯಃ।\\
ಚುಬುಕಂ ದೇವದೇವಸ್ತು ಪಾತು ಕಂಠಂ ಜಟಾಧರಃ॥೩॥

ಸ್ಕಂಧೌ ಮೇ ಪಾತು ಶುದ್ಧಾತ್ಮಾ ಕರೌ ಪಾತು ಯಮಾಂತಕಃ।\\
ಕುಚಾಗ್ರಂ ಕರಮಧ್ಯಂ ಚ ನಖರಾನ್ ಶಂಕರಃ ಸ್ವಯಂ ॥೪॥

ಹೃನ್ಮೇ ಪಶುಪತಿಃ ಪಾತು ಪಾರ್ಶ್ವೇ ಪರಮಪೂರುಷಃ~।\\
ಮಧ್ಯಮಂ ಪಾತು ಶರ್ವೋ ಮೇ ನಾಭಿಂ ನಾರಾಯಣಪ್ರಿಯಃ ॥೫॥

ಕಟಿಂ ಪಾತು ಜಗದ್ಭರ್ತಾ ಸಕ್ಥಿನೀ ಚ ಮೃಡಃ ಸ್ವಯಂ।\\
ಕೃತ್ತಿವಾಸಾಃ ಸ್ವಯಂ ಗುಹ್ಯಮೂರೂ ಪಾತು ಪಿನಾಕಧೃತ್॥೬॥

ಜಾನುನೀ ತ್ರ್ಯಂಬಕಃ ಪಾತು ಜಂಘೇ ಪಾತು ಸದಾಶಿವಃ।\\
ಸ್ಮರಾರಿಃ ಪಾತು ಮೇ ಪಾದೌ ಪಾತು ಸರ್ವಾಂಗಮೀಶ್ವರಃ ॥೭॥

ಇತೀದಂ ಕವಚಂ ದೇವಿ ಪರಮಾನಂದ ದಾಯಕಂ।\\
ಪ್ರಾತಃ ಕಾಲೇ ಶುಚಿರ್ಭೂತ್ವಾ ತ್ರಿವಾರಂ ಸರ್ವದಾ ಪಠೇತ್ ॥೮॥

ನಿತ್ಯಂ ಪೂಜಾ ಸಮಾಯುಕ್ತಃ ಸಂವತ್ಸರಮತಂದ್ರಿತಃ~।\\
ಜಪೇತ್ ತ್ರಿಸಂಧ್ಯಂ ಯೋ ವಿದ್ವಾನ್ ವೇದಶಾಸ್ತ್ರಾರ್ಥಪಾರಗಃ॥೯॥

ಗದ್ಯಪದ್ಯೈಸ್ತಥಾ ಚಾಪಿ ನಾಟಕಾಃ ಸ್ವಯಮೇವ ಹಿ~।\\
ನಿರ್ಗಚ್ಛಂತಿ ಮುಖಾಂಭೋಜಾತ್ ಸತ್ಯಮೇತನ್ನ ಸಂಶಯಃ ॥೧೦॥
\authorline{ಇತಿ ಶ್ರೀ ಬ್ರಹ್ಮವೈವರ್ತಮಹಾಪುರಾಣೇ  ಶ್ರೀ ದಕ್ಷಿಣಾಮೂರ್ತಿಕವಚಂ ಸಂಪೂರ್ಣಂ}
%================================================================================
\section{ಶ್ರೀಗಣೇಶಕವಚಂ }
\addcontentsline{toc}{section}{ಶ್ರೀಗಣೇಶಕವಚಂ }
ಗೌರ್ಯುವಾಚ~॥\\
ಏಷೋಽತಿಚಪಲೋ ದೈತ್ಯಾನ್ಬಾಲ್ಯೇಽಪಿ ನಾಶಯತ್ಯಹೋ~।\\
ಅಗ್ರೇ ಕಿಂ ಕರ್ಮ ಕರ್ತೇತಿ ನ ಜಾನೇ ಮುನಿಸತ್ತಮ ॥೧॥

ದೈತ್ಯಾ ನಾನಾವಿಧಾ ದುಷ್ಟಾಃ ಸಾಧುದೇವದ್ರುಹಃ ಖಲಾಃ~।\\
ಅತೋಽಸ್ಯ ಕಂಠೇ ಕಿಂಚಿತ್ತ್ವಂ ರಕ್ಷಾರ್ಥಂ ಬದ್ಧುಮರ್ಹಸಿ ॥೨॥

ಮುನಿರುವಾಚ~॥\\
\dhyana{ಧ್ಯಾಯೇತ್ಸಿಂಹಗತಂ ವಿನಾಯಕಮಮುಂ ದಿಗ್ಬಾಹುಮಾದ್ಯೇ ಯುಗೇ\\
ತ್ರೇತಾಯಾಂ ತು ಮಯೂರವಾಹನಮಮುಂ ಷಡ್ಬಾಹುಕಂ ಸಿದ್ಧಿದಂ~।\\
ದ್ವಾಪಾರೇ ತು ಗಜಾನನಂ ಯುಗಭುಜಂ ರಕ್ತಾಂಗರಾಗಂ ವಿಭುಂ\\
ತುರ್ಯೇ ತು ದ್ವಿಭುಜಂ ಸಿತಾಂಗರುಚಿರಂ ಸರ್ವಾರ್ಥದಂ ಸರ್ವದಾ ॥೩॥}

ವಿನಾಯಕಃ ಶಿಖಾಂ ಪಾತು ಪರಮಾತ್ಮಾ ಪರಾತ್ಪರಃ~।\\
ಅತಿಸುಂದರಕಾಯಸ್ತು ಮಸ್ತಕಂ ಸುಮಹೋತ್ಕಟಃ ॥೪॥

ಲಲಾಟಂ ಕಶ್ಯಪಃ ಪಾತು ಭ್ರೂಯುಗಂ ತು ಮಹೋದರಃ~।\\
ನಯನೇ ಭಾಲಚಂದ್ರಸ್ತು ಗಜಾಸ್ಯಸ್ತ್ವೋಷ್ಠಪಲ್ಲವೌ ॥೫॥

ಜಿಹ್ವಾಂ ಪಾತು ಗಣಕ್ರೀಡಶ್ಚಿಬುಕಂ ಗಿರಿಜಾಸುತಃ~।\\
ವಾಚಂ ವಿನಾಯಕಃ ಪಾತು ದಂತಾನ್ ರಕ್ಷತು ವಿಘ್ನಹಾ ॥೬॥

ಶ್ರವಣೌ ಪಾಶಪಾಣಿಸ್ತು ನಾಸಿಕಾಂ ಚಿಂತಿತಾರ್ಥದಃ~।\\
ಗಣೇಶಸ್ತು ಮುಖಂ ಕಂಠಂ ಪಾತು ದೇವೋ ಗಣಂಜಯಃ ॥೭॥

ಸ್ಕಂಧೌ ಪಾತು ಗಜಸ್ಕಂಧಃ ಸ್ತನೌ ವಿಘ್ನವಿನಾಶನಃ~।\\
ಹೃದಯಂ ಗಣನಾಥಸ್ತು ಹೇರಂಬೋ ಜಠರಂ ಮಹಾನ್ ॥೮॥

ಧರಾಧರಃ ಪಾತು ಪಾರ್ಶ್ವೌ ಪೃಷ್ಠಂ ವಿಘ್ನಹರಃ ಶುಭಃ~।\\
ಲಿಂಗಂ ಗುಹ್ಯಂ ಸದಾ ಪಾತು ವಕ್ರತುಂಡೋ ಮಹಾಬಲಃ ॥೯॥

ಗಣಕ್ರೀಡೋ ಜಾನುಜಂಘೇ ಊರೂ ಮಂಗಲಮೂರ್ತಿಮಾನ್~।\\
ಏಕದಂತೋ ಮಹಾಬುದ್ಧಿಃ ಪಾದೌ ಗುಲ್ಫೌ ಸದಾಽವತು ॥೧೦॥

ಕ್ಷಿಪ್ರಪ್ರಸಾದನೋ ಬಾಹೂ ಪಾಣೀ ಆಶಾಪ್ರಪೂರಕಃ~।\\
ಅಂಗುಲೀಶ್ಚ ನಖಾನ್ಪಾತು ಪದ್ಮಹಸ್ತೋಽರಿನಾಶನಃ ॥೧೧॥

ಸರ್ವಾಂಗಾನಿ ಮಯೂರೇಶೋ ವಿಶ್ವವ್ಯಾಪೀ ಸದಾಽವತು~।\\
ಅನುಕ್ತಮಪಿ ಯತ್ಸ್ಥಾನಂ ಧೂಮ್ರಕೇತುಃ ಸದಾಽವತು ॥೧೨॥

ಆಮೋದಸ್ತ್ವಗ್ರತಃ ಪಾತು ಪ್ರಮೋದಃ ಪೃಷ್ಠತೋಽವತು~।\\
ಪ್ರಾಚ್ಯಾಂ ರಕ್ಷತು ಬುದ್ಧೀಶ ಆಗ್ನೇಯ್ಯಾಂ ಸಿದ್ಧಿದಾಯಕಃ ॥೧೩॥

ದಕ್ಷಿಣಾಸ್ಯಾಮುಮಾಪುತ್ರೋ ನೈರೃತ್ಯಾಂ ತು ಗಣೇಶ್ವರಃ~।\\
ಪ್ರತೀಚ್ಯಾಂ ವಿಘ್ನಹರ್ತಾಽವ್ಯಾದ್ವಾಯವ್ಯಾಂ ಗಜಕರ್ಣಕಃ ॥೧೪॥

ಕೌಬೇರ್ಯಾಂ ನಿಧಿಪಃ ಪಾಯಾದೀಶಾನ್ಯಾಮೀಶನಂದನಃ~।\\
ದಿವಾಽವ್ಯಾದೇಕದಂತಸ್ತು ರಾತ್ರೌ ಸಂಧ್ಯಾಸು ವಿಘ್ನಹೃತ್ ॥೧೫॥

ರಾಕ್ಷಸಾಸುರವೇತಾಲಗ್ರಹಭೂತಪಿಶಾಚತಃ~।\\
ಪಾಶಾಂಕುಶಧರಃ ಪಾತು ರಜಃಸತ್ತ್ವತಮಃಸ್ಮೃತೀಃ ॥೧೬॥

ಜ್ಞಾನಂ ಧರ್ಮಂ ಚ ಲಕ್ಷ್ಮೀಂ ಚ ಲಜ್ಜಾಂ ಕೀರ್ತಿಂ ತಥಾ ಕುಲಂ~।\\
ವಪುರ್ಧನಂ ಚ ಧಾನ್ಯಂ ಚ ಗೃಹಾನ್ ದಾರಾನ್ಸುತಾನ್ಸಖೀನ್ ॥೧೭॥

ಸರ್ವಾಯುಧಧರಃ ಪೌತ್ರಾನ್ ಮಯೂರೇಶೋಽವತಾತ್ಸದಾ~।\\
ಕಪಿಲೋಽಜಾದಿಕಂ ಪಾತು ಗಜಾಶ್ವಾನ್ವಿಕಟೋಽವತು ॥೧೮॥

ಭೂರ್ಜಪತ್ರೇ ಲಿಖಿತ್ವೇದಂ ಯಃ ಕಂಠೇ ಧಾರಯೇತ್ಸುಧೀಃ~।\\
ನ ಭಯಂ ಜಾಯತೇ ತಸ್ಯ ಯಕ್ಷರಕ್ಷಃಪಿಶಾಚತಃ ॥೧೯॥

ತ್ರಿಸಂಧ್ಯಂ ಜಪತೇ ಯಸ್ತು ವಜ್ರಸಾರತನುರ್ಭವೇತ್~।\\
ಯಾತ್ರಾಕಾಲೇ ಪಠೇದ್ಯಸ್ತು ನಿರ್ವಿಘ್ನೇನ ಫಲಂ ಲಭೇತ್ ॥೨೦॥

ಯುದ್ಧಕಾಲೇ ಪಠೇದ್ಯಸ್ತು ವಿಜಯಂ ಚಾಪ್ನುಯಾದ್ದ್ರುತಂ~।\\
ಮಾರಣೋಚ್ಚಾಟನಾಕರ್ಷಸ್ತಂಭಮೋಹನಕರ್ಮಣಿ ॥೨೧॥

ಸಪ್ತವಾರಂ ಜಪೇದೇತದ್ದಿನಾನಾಮೇಕವಿಂಶತಿಂ~।\\
ತತ್ತತ್ಫಲಮವಾಪ್ನೋತಿ ಸಾಧಕೋ ನಾತ್ರಸಂಶಯಃ ॥೨೨॥

ಏಕವಿಂಶತಿವಾರಂ ಚ ಪಠೇತ್ತಾವದ್ದಿನಾನಿ ಯಃ~।\\
ಕಾರಾಗೃಹಗತಂ ಸದ್ಯೋ ರಾಜ್ಞಾ ವಧ್ಯಂ ಚ ಮೋಚಯೇತ್ ॥೨೩॥

ರಾಜದರ್ಶನವೇಲಾಯಾಂ ಪಠೇದೇತತ್ತ್ರಿವಾರತಃ~।\\
ಸ ರಾಜಾನಂ ವಶಂ ನೀತ್ವಾ ಪ್ರಕೃತೀಶ್ಚ ಸಭಾಂ ಜಯೇತ್ ॥೨೪॥

ಇದಂ ಗಣೇಶಕವಚಂ ಕಶ್ಯಪೇನ ಸಮೀರಿತಂ~।\\
ಮುದ್ಗಲಾಯ ಚ ತೇ ನಾಥ ಮಾಂಡವ್ಯಾಯ ಮಹರ್ಷಯೇ ॥೨೫॥

ಮಹ್ಯಂ ಸ ಪ್ರಾಹ ಕೃಪಯಾ ಕವಚಂ ಸರ್ವಸಿದ್ಧಿದಂ~।\\
ನ ದೇಯಂ ಭಕ್ತಿಹೀನಾಯ ದೇಯಂ ಶ್ರದ್ಧಾವತೇ ಶುಭಂ ॥೨೬॥

ಯಸ್ಯಾನೇನ ಕೃತಾ ರಕ್ಷಾ ನ ಬಾಧಾಸ್ಯ ಭವೇತ್ಕ್ವಚಿತ್~।\\
ರಾಕ್ಷಸಾಸುರವೇತಾಲದೈತ್ಯದಾನವಸಂಭವಾ ॥೨೭॥

\authorline {ಇತಿ ಶ್ರೀಗಣೇಶಪುರಾಣೇ ಗಣೇಶಕವಚಂ ಸಂಪೂರ್ಣಂ ॥}
%=====================================================================================================================
\section{ಶ್ರೀಬಾಲಾತ್ರಿಪುರಸುಂದರೀ ಕವಚಂ}
\addcontentsline{toc}{section}{ಶ್ರೀಬಾಲಾತ್ರಿಪುರಸುಂದರೀ ಕವಚಂ}
ಶ್ರೀ ಪಾರ್ವತ್ಯುವಾಚ ॥\\
ದೇವ ದೇವ ಮಹಾದೇವ ಶಂಕರ ಪ್ರಾಣ ವಲ್ಲಭ~।\\
ಕವಚಂ ಶ್ರೋತುಮಿಚ್ಛಾಮಿ ಬಾಲಾಯಾ ವದ ಮೇ ಪ್ರಭೋ ॥೧॥

ಶ್ರೀ ಮಹೇಶ್ವರ ಉವಾಚ ॥\\
ಶ್ರೀಬಾಲಾಕವಚಂ ದೇವಿ ಮಹಾಪ್ರಾಣಾಧಿಕಂ ಪರಂ~।\\
ವಕ್ಷ್ಯಾಮಿ ಸಾವಧಾನಾ ತ್ವಂ ಶೃಣುಷ್ವಾವಹಿತಾ ಪ್ರಿಯೇ ॥೨॥

ಅಸ್ಯ ಶ್ರೀಬಾಲಾಕವಚಸ್ತೋತ್ರ ಮಹಾಮಂತ್ರಸ್ಯ ಶ್ರೀ ದಕ್ಷಿಣಾಮುರ್ತಿರ್ಋಷಿಃ~। ಪಂಕ್ತಿಶ್ಛಂದಃ। ಬಾಲಾತ್ರಿಪುರಸುಂದರೀ ದೇವತಾ। ಐಂ ಬೀಜಂ। ಸೌಃ ಶಕ್ತಿಃ। ಕ್ಲೀಂ ಕೀಲಕಂ। ಶ್ರೀಬಾಲಾತ್ರಿಪುರಸುಂದರೀದೇವತಾ ಪ್ರಸಾದಸಿದ್ಧ್ಯರ್ಥೇ ಜಪೇ ವಿನಿಯೋಗಃ॥

\dhyana{ಅರುಣ ಕಿರಣ ಜಾಲೈರಂಚಿತಾಶಾವಕಾಶಾ\\
ವಿಧೃತಜಪವಟೀಕಾ ಪುಸ್ತಕಾಭೀತಿಹಸ್ತಾ~।\\
ಇತರಕರವರಾಢ್ಯಾ ಫುಲ್ಲಕಹ್ಲಾರಸಂಸ್ಥಾ\\
ನಿವಸತು ಹೃದಿ ಬಾಲಾ ನಿತ್ಯಕಲ್ಯಾಣಶೀಲಾ }॥೩॥

(ಐಂ)ವಾಗ್ಭವಃ ಪಾತು ಶೀರ್ಷೇ (ಕ್ಲೀಂ) ಕಾಮರಾಜಸ್ತಥಾ ಹೃದಿ~।\\
ಸೌಃ ಶಕ್ತಿಬೀಜಂ ಮಾಂ ಪಾತು ನಾಭೌ ಗುಹ್ಯೇ ಚ ಪಾದಯೋಃ ॥೪॥

ಐಂ ಕ್ಲೀಂ ಸೌಃ ವದನೇ ಪಾತು ಬಾಲಾ ಮಾಂ ಸರ್ವಸಿದ್ಧಯೇ~।\\
ಹ್‌ಸ್‌ರೈಂ ಹ್‌ಸ್‌ಕ್ಲ್ರೀಂ ಹ್‌ಸ್‌ರ್‌ಸೌಃ ಪಾತು ಸ್ಕಂಧೇ ಭೈರವೀ ಕಂಠದೇಶತಃ ॥೫॥

ಸುಂದರೀ ನಾಸದೇಶೇವ್ಯಾಚ್ಚರ್ಚೇ ಕಾಮಕಲಾ ಸದಾ~।\\
ಭ್ರೂನಾಸಯೋರಂತರಾಲೇ ಮಹಾತ್ರಿಪುರಸುಂದರೀ ॥೬॥

ಲಲಾಟೇ ಸುಭಗಾ ಪಾತು ಭಗಾ ಮಾಂ ಕಂಠದೇಶತಃ~।\\
ಭಗೋದಯಾ ತು ಹೃದಯೇ ಉದರೇ ಭಗಸರ್ಪಿಣೀ ॥೭॥

ಭಗಮಾಲಾ ನಾಭಿದೇಶೇ ಲಿಂಗೇ ಪಾತು ಮನೋಭವಾ~।\\
ಗುಹ್ಯೇ ಪಾತು ಮಹಾವೀರಾ ರಾಜರಾಜೇಶ್ವರೀ ಶಿವಾ ॥೮॥

ಚೈತನ್ಯರೂಪಿಣೀ ಪಾತು ಪಾದಯೋರ್ಜಗದಂಬಿಕಾ~।\\
ನಾರಾಯಣೀ ಸರ್ವಗಾತ್ರೇ ಸರ್ವಕಾರ್ಯಶುಭಂಕರೀ ॥೯॥

ಬ್ರಹ್ಮಾಣೀ ಪಾತು ಮಾಂ ಪೂರ್ವೇ ದಕ್ಷಿಣೇ ವೈಷ್ಣವೀ ತಥಾ~।\\
ಪಶ್ಚಿಮೇ ಪಾತು ವಾರಾಹೀ ಹ್ಯುತ್ತರೇ ತು ಮಹೇಶ್ವರೀ ॥೧೦॥

ಆಗ್ನೇಯ್ಯಾಂ ಪಾತು ಕೌಮಾರೀ ಮಹಾಲಕ್ಷ್ಮೀಶ್ಚ ನೈರ್ಋತೇ~।\\
ವಾಯವ್ಯೇ ಪಾತು ಚಾಮುಂಡಾ ಚೇಂದ್ರಾಣೀ ಪಾತು ಚೇಶಕೇ ॥೧೧॥

ಜಲೇ ಪಾತು ಮಹಾಮಾಯಾ ಪೃಥಿವ್ಯಾಂ ಸರ್ವಮಂಗಲಾ~।\\
ಸ್ಕ್ಲೀಂ ಮಾಂ ಸರ್ವತಃ ಪಾತು ಸಕಲಹ್ರೀಂ ಪಾತು ಸಂಧಿಷು ॥೧೨॥

ಜಲೇ ಸ್ಥಲೇ ತಥಾಕಾಶೇ ದಿಕ್ಷು ರಾಜಗೃಹೇ ತಥಾ~।\\
ಕ್ಷೂಂಕ್ಷೇಂಮಾಂ ತ್ವರಿತಾಪಾತು ಸಹ್ರೀಂಸಕ್ಲೀಂ ಮನೋಭವಾ ॥೧೩॥

ಹಂಸಃ ಪಾಯಾನ್ಮಹಾದೇವೀ ಪರಂ ನಿಷ್ಕಲದೇವತಾ~।\\
ವಿಜಯಾ ಮಂಗಲಾ ದೂತೀ ಕಲ್ಯಾಣೀ ಭಗಮಾಲಿನೀ ॥೧೪॥

ಜ್ವಾಲಾಮಾಲಿನಿನಿತ್ಯಾ ಚ ಸರ್ವದಾ ಪಾತು ಮಾಂ ಶಿವಾ~।\\
ಇತ್ಯೇತತ್ಕವಚಂ ದೇವಿ ಬಾಲಾದೇವ್ಯಾಃ ಪ್ರಕೀರ್ತಿತಂ~।\\
ಸರ್ವಸ್ವಂ ಮೇ ತವ ಪ್ರೀತ್ಯಾ ಪ್ರಾಣವದ್ರಕ್ಷಿತಂ ಕುರು॥೧೫॥

\authorline{ಇತಿ ಶ್ರೀರುದ್ರಯಾಮಲೇ ಶ್ರೀಶಿವಪಾರ್ವತಿಸಂವಾದೇ\\ ಶ್ರೀಬಾಲಾತ್ರಿಪುರಸುಂದರೀ ಮಂತ್ರಕವಚಂ ಸಂಪೂರ್ಣಂ ॥}
%===========================================================================================
\section{ಶ್ರೀ ಲಕ್ಷ್ಮೀನಾರಾಯಣ ಕವಚಂ }
\addcontentsline{toc}{section}{ಶ್ರೀ ಲಕ್ಷ್ಮೀನಾರಾಯಣ ಕವಚಂ }
ಕವಚಸ್ಯಾಸ್ಯ ಸುಭಗೇ ಕಥಿತೋಽಯಂ ಮುನಿಃ ಶಿವಃ~।\\
ತ್ರಿಷ್ಟುಪ್ ಛಂದೋ ದೇವತಾ ಚ ಲಕ್ಷ್ಮೀನಾರಾಯಣೋ ಮತಃ ॥೭॥

ರಮಾ(ಶ್ರೀಂ) ಬೀಜಂ ಪರಾ(ಹ್ರೀಂ) ಶಕ್ತಿಸ್ತಾರಂ(ಓಂ) ಕೀಲಕಮೀಶ್ವರಿ~।\\
ಭೋಗಾಪವರ್ಗಸಿದ್ಧ್ಯರ್ಥಂ ವಿನಿಯೋಗ ಇತಿ ಸ್ಮೃತಃ ॥೮॥

\dhyana{ಪೂರ್ಣೇಂದುವದನಂ ಪೀತವಸನಂ ಕಮಲಾಸನಂ~।\\
ಲಕ್ಷ್ಮ್ಯಾ ಶ್ರಿತಂ ಚತುರ್ಬಾಹುಂ ಲಕ್ಷ್ಮೀನಾರಾಯಣಂ ಭಜೇ }॥೯॥

ಓಂ ವಾಸುದೇವೋಽವತು ಮೇ ಮಸ್ತಕಂ ಸಶಿರೋರುಹಂ~।\\
ಹ್ರೀಂ ಲಲಾಟಂ ಸದಾ ಪಾತು ಲಕ್ಷ್ಮೀವಿಷ್ಣುಃ ಸಮಂತತಃ ॥೧೦॥

ಹ್ಸೌಃ ನೇತ್ರೇಽವತಾಲ್ಲಕ್ಷ್ಮೀಗೋವಿಂದೋ ಜಗತಾಂ ಪತಿಃ~।\\
ಹ್ರೀಂ ನಾಸಾಂ ಸರ್ವದಾ ಪಾತು ಲಕ್ಷ್ಮೀದಾಮೋದರಃ ಪ್ರಭುಃ ॥೧೧॥

ಶ್ರೀಂ ಮುಖಂ ಸತತಂ ಪಾತು ದೇವೋ ಲಕ್ಷ್ಮೀತ್ರಿವಿಕ್ರಮಃ~।\\
ಲಕ್ಷ್ಮೀ ಕಂಠಂ ಸದಾ ಪಾತು ದೇವೋ ಲಕ್ಷ್ಮೀಜನಾರ್ದನಃ ॥೧೨॥

ನಾರಾಯಣಾಯ ಬಾಹೂ ಮೇ ಪಾತು ಲಕ್ಷ್ಮೀಗದಾಗ್ರಜಃ~।\\
ನಮಃ ಪಾರ್ಶ್ವೌ ಸದಾ ಪಾತು ಲಕ್ಷ್ಮೀನಂದೈಕನಂದನಃ ॥೧೩॥

ಅಂ ಆಂ ಇಂ ಈಂ ಪಾತು ವಕ್ಷೋ ಓಂ ಲಕ್ಷ್ಮೀತ್ರಿಪುರೇಶ್ವರಃ~।\\
ಉಂ ಊಂ ಋಂ ೠಂ ಪಾತು ಕುಕ್ಷಿಂ ಹ್ರೀಂ ಲಕ್ಷ್ಮೀಗರುಡಧ್ವಜಃ ॥೧೪॥

ಲೃಂ ಲೄಂ ಏಂ ಐಂ ಪಾತು ಪೃಷ್ಠಂ ಹ್ಸೌಃ ಲಕ್ಷ್ಮೀನೃಸಿಂಹಕಃ~।\\
ಓಂ ಔಂ ಅಂ ಅಃ ಪಾತು ನಾಭಿಂ ಹ್ರೀಂ ಲಕ್ಷ್ಮೀವಿಷ್ಟರಶ್ರವಾಃ ॥೧೫॥

ಕಂ ಖಂ ಗಂ ಘಂ ಗುದಂ ಪಾತು ಶ್ರೀಂ ಲಕ್ಷ್ಮೀಕೈಟಭಾಂತಕಃ~।\\
ಚಂ ಛಂ ಜಂ ಝಂ ಪಾತು ಶಿಶ್ನಂ ಲಕ್ಷ್ಮೀ ಲಕ್ಷ್ಮೀಶ್ವರಃ ಪ್ರಭುಃ ॥೧೬॥

ಟಂ ಠಂ ಡಂ ಢಂ ಕಟಿಂ ಪಾತು ನಾರಾಯಣಾಯ ನಾಯಕಃ~।\\
ತಂ ಥಂ ದಂ ಧಂ ಪಾತು ಚೋರೂ ನಮೋ ಲಕ್ಷ್ಮೀಜಗತ್ಪತಿಃ ॥೧೭॥

ಪಂ ಫಂ ಬಂ ಭಂ ಪಾತು ಜಾನೂ ಓಂ ಹ್ರೀಂ ಲಕ್ಷ್ಮೀಚತುರ್ಭುಜಃ~।\\
ಯಂ ರಂ ಲಂ ವಂ ಪಾತು ಜಂಘೇ ಹ್ಸೌಃ ಲಕ್ಷ್ಮೀಗದಾಧರಃ ॥೧೮॥

ಶಂ ಷಂ ಸಂ ಹಂ ಪಾತು ಗುಲ್ಫೌ ಹ್ರೀಂ ಶ್ರೀಂ ಲಕ್ಷ್ಮೀರಥಾಂಗಭೃತ್~।\\
ಳಂ ಕ್ಷಃ ಪಾದೌ ಸದಾ ಪಾತು ಮೂಲಂ ಲಕ್ಷ್ಮೀಸಹಸ್ರಪಾತ್ ॥೧೯॥

ಙಂ ಞಂ ಣಂ ನಂ ಮಂ ಮೇ ಪಾತು ಲಕ್ಷ್ಮೀಶಃ ಸಕಲಂ ವಪುಃ~।\\
ಇಂದ್ರೋ ಮಾಂ ಪೂರ್ವತಃ ಪಾತು ವಹ್ನಿರ್ವಹ್ನೌ ಸದಾವತು ॥೨೦॥

ಯಮೋ ಮಾಂ ದಕ್ಷಿಣೇ ಪಾತು ನೈರೃತ್ಯಾಂ ನಿರೃತಿಶ್ಚ ಮಾಂ~।\\
ವರುಣಃ ಪಶ್ಚಿಮೇಽವ್ಯಾನ್ಮಾಂ ವಾಯವ್ಯೇಽವತು ಮಾಂ ಮರುತ್ ॥೨೧॥

ಉತ್ತರೇ ಧನದಃ ಪಾಯಾದೈಶಾನ್ಯಾಮೀಶ್ವರೋಽವತು~।\\
ವಜ್ರ ಶಕ್ತಿ ದಂಡ ಖಡ್ಗ ಪಾಶ ಯಷ್ಟಿ ಧ್ವಜಾಂಕಿತಾಃ ॥೨೨॥

ಸಶೂಲಾಃ ಸರ್ವದಾ ಪಾಂತು ದಿಗೀಶಾಃ ಪರಮಾರ್ಥದಾಃ~।\\
ಅನಂತಃ ಪಾತ್ವಧೋ ನಿತ್ಯಮೂರ್ಧ್ವೇ ಬ್ರಹ್ಮಾವತಾಚ್ಚ ಮಾಂ ॥೨೩॥

ದಶದಿಕ್ಷು ಸದಾ ಪಾತು ಲಕ್ಷ್ಮೀನಾರಾಯಣಃ ಪ್ರಭುಃ~।\\
ಪ್ರಭಾತೇ ಪಾತು ಮಾಂ ವಿಷ್ಣುರ್ಮಧ್ಯಾಹ್ನೇ ವಾಸುದೇವಕಃ ॥೨೪॥

ದಾಮೋದರೋಽವತಾತ್ ಸಾಯಂ ನಿಶಾದೌ ನರಸಿಂಹಕಃ~।\\
ಸಂಕರ್ಷಣೋಽರ್ಧರಾತ್ರೇಽವ್ಯಾತ್ ಪ್ರಭಾತೇಽವ್ಯಾತ್ ತ್ರಿವಿಕ್ರಮಃ ॥೨೫॥

ಅನಿರುದ್ಧಃ ಸರ್ವಕಾಲಂ ವಿಷ್ವಕ್ಸೇನಶ್ಚ ಸರ್ವತಃ~।\\
ರಣೇ ರಾಜಕುಲೇ ದ್ಯೂತೇ ವಿವಾದೇ ಶತ್ರುಸಂಕಟೇ~।\\
ಓಂ ಹ್ರೀಂ ಹ್ಸೌಃ ಹ್ರೀಂ ಶ್ರೀಂ ಮೂಲಂ ಲಕ್ಷ್ಮೀನಾರಾಯಣೋಽವತು ॥೨೬॥

ಓಂಓಂಓಂರಣರಾಜಚೌರರಿಪುತಃ ಪಾಯಾಚ್ಚ ಮಾಂ ಕೇಶವಃ\\
ಹ್ರೀಂಹ್ರೀಂಹ್ರೀಂಹಹಹಾ ಹಸೌಃ ಹಸಹಸೌಃ ವಹ್ನೇರ್ವತಾನ್ಮಾಧವಃ~।\\
ಹ್ರೀಂಹ್ರೀಂಹ್ರೀಂಜಲಪರ್ವತಾಗ್ನಿಭಯತಃ ಪಾಯಾದನಂತೋ ವಿಭುಃ\\
ಶ್ರೀಂಶ್ರೀಂಶ್ರೀಂಶಶಶಾಲಲಂ ಪ್ರತಿದಿನಂ ಲಕ್ಷ್ಮೀಧವಃ ಪಾತು ಮಾಂ ॥೨೭॥

ಇತೀದಂ ಕವಚಂ ದಿವ್ಯಂ ವಜ್ರಪಂಜರಕಾಭಿಧಂ~।\\
ಲಕ್ಷ್ಮೀನಾರಾಯಣಸ್ಯೇಷ್ಟಂ ಚತುರ್ವರ್ಗಫಲಪ್ರದಂ ॥೨೮॥

ಸರ್ವಸೌಭಾಗ್ಯನಿಲಯಂ ಸರ್ವಸಾರಸ್ವತಪ್ರದಂ~।\\
ಲಕ್ಷ್ಮೀಸಂವನನಂ ತತ್ವಂ ಪರಮಾರ್ಥರಸಾಯನಂ ॥೨೯॥

ಮಂತ್ರಗರ್ಭಂ ಜಗತ್ಸಾರಂ ರಹಸ್ಯಂ ತ್ರಿದಿವೌಕಸಾಂ~।\\
ದಶವಾರಂ ಪಠೇದ್ರಾತ್ರೌ ರತಾಂತೇ ವೈಷ್ಣವೋತ್ತಮಃ ॥೩೦॥

ಸ್ವಪ್ನೇ ವರಪ್ರದಂ ಪಶ್ಯೇಲ್ಲಕ್ಷ್ಮೀನಾರಾಯಣಂ ಸುಧೀಃ~।\\
ತ್ರಿಸಂಧ್ಯಂ ಯಃ ಪಠೇನ್ನಿತ್ಯಂ ಕವಚಂ ಮನ್ಮುಖೋದಿತಂ ॥೩೧॥

ಸ ಯಾತಿ ಪರಮಂ ಧಾಮ ವೈಷ್ಣವಂ ವೈಷ್ಣವೇಶ್ವರಃ~।\\
ಮಹಾಚೀನಪದಸ್ಥೋಽಪಿ ಯಃ ಪಠೇದಾತ್ಮಚಿಂತಕಃ ॥೩೨॥

ಆನಂದಪೂರಿತಸ್ತೂರ್ಣಂ ಲಭೇದ್ ಮೋಕ್ಷಂ ಸ ಸಾಧಕಃ~।\\
ಗಂಧಾಷ್ಟಕೇನ ವಿಲಿಖೇದ್ರವೌ ಭೂರ್ಜೇ ಜಪನ್ಮನುಂ ॥೩೩॥

ಪೀತಸೂತ್ರೇಣ ಸಂವೇಷ್ಟ್ಯ ಸೌವರ್ಣೇನಾಥ ವೇಷ್ಟಯೇತ್~।\\
ಧಾರಯೇದ್ಗುಟಿಕಾಂ ಮೂರ್ಧ್ನಿ ಲಕ್ಷ್ಮೀನಾರಾಯಣಂ ಸ್ಮರನ್ ॥೩೪॥

ರಣೇ ರಿಪೂನ್ ವಿಜಿತ್ಯಾಶು ಕಲ್ಯಾಣೀ ಗೃಹಮಾವಿಶೇತ್~।\\
ವಂಧ್ಯಾ ವಾ ಕಾಕವಂಧ್ಯಾ ವಾ ಮೃತವತ್ಸಾ ಚ ಯಾಂಗನಾ ॥೩೫॥

ಸಾ ಬಧ್ನೀಯಾತ್ ಕಂಠದೇಶೇ ಲಭೇತ್ ಪುತ್ರಾಂಶ್ಚಿರಾಯುಷಃ~।\\
ಗುರೂಪದೇಶತೋ ಧೃತ್ವಾ ಗುರುಂ ಧ್ಯಾತ್ವಾ ಮನುಂ ಜಪನ್ ॥೩೬॥

ವರ್ಣಲಕ್ಷಪುರಶ್ಚರ್ಯಾ ಫಲಮಾಪ್ನೋತಿ ಸಾಧಕಃ~।\\
ಬಹುನೋಕ್ತೇನ ಕಿಂ ದೇವಿ ಕವಚಸ್ಯಾಸ್ಯ ಪಾರ್ವತಿ ॥೩೭॥

ವಿನಾನೇನ ನ ಸಿದ್ಧಿಃ ಸ್ಯಾನ್ಮಂತ್ರಸ್ಯಾಸ್ಯ ಮಹೇಶ್ವರಿ~।\\
ಸರ್ವಾಗಮರಹಸ್ಯಾಢ್ಯಂ ತತ್ವಾತ್ ತತ್ವಂ ಪರಾತ್ ಪರಂ ॥೩೮॥

ಅಭಕ್ತಾಯ ನ ದಾತವ್ಯಂ ಕುಚೈಲಾಯ ದುರಾತ್ಮನೇ~।\\
ದೀಕ್ಷಿತಾಯ ಕುಲೀನಾಯ ಸ್ವಶಿಷ್ಯಾಯ ಮಹಾತ್ಮನೇ ॥೩೯॥

ಮಹಾಚೀನಪದಸ್ಥಾಯ ದಾತವ್ಯಂ ಕವಚೋತ್ತಮಂ~।\\
ಗುಹ್ಯಂ ಗೋಪ್ಯಂ ಮಹಾದೇವಿ ಲಕ್ಷ್ಮೀನಾರಾಯಣಪ್ರಿಯಂ~।\\
ವಜ್ರಪಂಜರಕಂ ವರ್ಮ ಗೋಪನೀಯಂ ಸ್ವಯೋನಿವತ್ ॥೪೦॥
\authorline{॥ಇತಿ ಶ್ರೀ ಲಕ್ಷ್ಮೀನಾರಾಯಣ ಕವಚಂ ಸಂಪೂರ್ಣಂ ॥}
%===========================================================================================
\section{ಶೀತಲಾ ಕವಚಮ್}
\addcontentsline{toc}{section}{ಶೀತಲಾ ಕವಚಮ್}
ಅಸ್ಯ ಶ್ರೀ ಶೀತಲಾಕವಚಸ್ಯ ಮಹೇಶ್ವರಃ ಋಷಿಃ । ಅನುಷ್ಟುಪ್ ಛನ್ದಃ ।\\
ಶೀತಲಾ ದೇವತಾ। ಲಕ್ಷ್ಮೀಬೀಜಂ । ರಮಾ ಶಕ್ತಿಃ ।ತಾರಂ ಕೀಲಕಮ್।\\
ಲೂತಾವಿಸ್ಫೋಟಕಾದೀನಾಂ ಶಾಂತ್ಯರ್ಥೇ ಜಪೇ ವಿನಿಯೋಗಃ ।\\

\dhyana{ಉದ್ಯತ್ಸೂರ್ಯನಿಭಾಂ ನವೇಂದುಮುಕುಟಾಂ ಸೂರ್ಯಾಗ್ನಿನೇತ್ರೋಜ್ಜ್ವಲಾಂ\\
ನಾನಾಗಂಧ ವಿಲೇಪನಾಂ ಮೃದುತನುಂ ದಿವ್ಯಾಂಬರಾಲಂಕೃತಾಮ್ ।\\
ದೋರ್ಭ್ಯಾಂ ಸಂದಧತೀಂ ವರಾಭಯಯುಗಂ ವಾಹೇ ಸ್ಥಿತಾಂ ರಾಸಭೇ\\
ಭಕ್ತಾಭೀಷ್ಟ-ಫಲ-ಪ್ರದಾಂ ಭಗವತೀಂ ಶ್ರೀಶೀತಲಾಂ ತ್ವಾಂ ಭಜೇ ॥}

ಶೀತಲಾ ಪಾತು ಮೇ ಪ್ರಾಣೇ ರುನುಕೀ ಪಾತು ಚಾಪಾನೇ ।\\
ಸಮಾನೇ ಝುನುಕೀ ಪಾತು ಉದಾನೇ ಪಾತು ಮನ್ದಲಾ ॥೧॥

ವ್ಯಾನೇ ಚ ಸೇಢಲಾ ಪಾತು ಮನೋ ಮೇ ಶಾಂಕರೀ ತಥಾ ।\\
ಪಾತು ಮಾಮಿಂದ್ರಿಯಾನ್ ಸರ್ವಾನ್ ಶ್ರೀದುರ್ಗಾ ವಿನ್ಧ್ಯವಾಸಿನೀ ॥೨॥

ಮಮ ಪಾತು ಶಿರೋ ದುರ್ಗಾ ಕಮಲಾ ಪಾತು ಮಸ್ತಕಮ್ ।\\
ಹ್ರೀಂ ಮೇ ಪಾತು ಭ್ರುವೋರ್ಮಧ್ಯೇ ಭವಾನೀ ಭುವನೇಶ್ವರೀ ॥\\
ಪಾತು ಮೇ ಮಧುಮತೀ ದೇವೀ ಓಂಕಾರಂ ಭೃಕುಟಿದ್ವಯಮ್ ॥೩॥

ನಾಸಿಕಾಂ ಶಾರದಾ ಪಾತು ತಮಸಾ ವರ್ತ್ಮಸಂಯುತಮ್ ।\\
ನೇತ್ರೇ ಜ್ವಾಲಾಮುಖೀ ಪಾತು ಭೀಷಣಾ ಪಾತು ಶ್ರುತೀ ಮೇ ॥೪॥

ಕಪೋಲೌ ಕಾಲಿಕಾ ಪಾತು ಸುಮುಖೀ ಪಾತು ಚೋಷ್ಠಯೋಃ ।\\
ಸನ್ಧ್ಯಯೋಃ ತ್ರಿಪುರಾ ಪಾತು ದನ್ತೇ ಚ ರಕ್ತದನ್ತಿಕಾ ॥೫॥

ಜಿಹ್ವಾಂ ಸರಸ್ವತೀ ಪಾತು ತಾಲುಕೇ ಚ ವಾಗ್ವಾದಿನೀ ।\\
ಕಣ್ಠೇ ಪಾತು ತು ಮಾತಂಗೀ ಗ್ರೀವಾಯಾಂ ಭದ್ರಕಾಲಿಕಾ ॥೬॥

ಸ್ಕನ್ಧೌ ಚ ಪಾತು ಮೇ ಛಿನ್ನಾ ಕಕುದಂ ಸ್ಕನ್ದ-ಮಾತರಃ ।\\
ಬಾಹುಯುಗ್ಮೌ ಚ ಮೇ ಪಾತು ಶ್ರೀದೇವೀ ಬಗಲಾಮುಖೀ ॥೭॥

ಕರೌ ಮೇ ಭೈರವೀ ಪಾತು ಪೃಷ್ಠೇ ಪಾತು ಧನುರ್ಧರೀ।\\
ವಕ್ಷಃಸ್ಥಲೇ ಚ ಮೇ ಪಾತು ದುರ್ಗಾ ಮಹಿಷಮರ್ದಿನೀ ॥೮॥

ಹೃದಯೇ ಲಲಿತಾ ಪಾತು ಕುಕ್ಷೌ ಪಾತು ಮಹೇಶ್ವರೀ ।\\
ಪಾರ್ಶ್ವೌ ಚ ಗಿರಿಜಾ ಪಾತು ಚಾನ್ನಪೂರ್ಣಾ ತು ಚೋದರಮ್ ॥೯॥

ನಾಭಿಂ ನಾರಾಯಣೀ ಪಾತು ಕಟಿಂ ಮೇ ಸರ್ವಮಂಗಲಾ ।\\
ಜಂಘಯೋ ರ್ಮೇ ಸದಾ ಪಾತು ದೇವೀ ಕಾತ್ಯಾಯನೀ ಪರಾ ॥೧೦॥

ಬ್ರಹ್ಮಾಣೀ ಶಿಶ್ನಂ ಪಾತು ವೃಷಣಂ ಪಾತು ಕಪಾಲಿನೀ ।\\
ಗುಹ್ಯಂ ಗುಹ್ಯೇಶ್ವರೀ ಪಾತು ಜಾನುನೋರ್ಜಗದೀಶ್ವರೀ ॥೧೧॥

ಪಾತು ಗುಲ್ಫೌ ತು ಕೌಮಾರೀ ಪಾದಪೃಷ್ಠಂ ತು ವೈಷ್ಣವೀ।\\
ವಾರಾಹೀ ಪಾತು ಪಾದಾಗ್ರೇ ಇಂದ್ರಾಣೀ ಸರ್ವಮರ್ಮಸು ॥೧೨॥

ಮಾರ್ಗೇ ರಕ್ಷತು ಚಾಮುಂಡಾ ವನೇ ತು ವನವಾಸಿನೀ ।\\
ಜಲೇ ಚ ವಿಜಯಾ ರಕ್ಷೇತ್ ವಹ್ನೌ ಮೇ ಚಾಪರಾಜಿತಾ ॥೧೩॥

ರಣೇ ಕ್ಷೇಮಂಕರೀ ರಕ್ಷೇತ್ ಸರ್ವತ್ರ ಸರ್ವಮಂಗಲಾ ।\\
ಭವಾನೀ ಪಾತು ಬಂಧೂನ್ ಮೇ ಭಾರ್ಯಾಂ ರಕ್ಷತು ಚಾಂಬಿಕಾ ॥೧೪॥

ಪುತ್ರಾನ್ ರಕ್ಷತು ಮಾಹೇಂದ್ರೀ ಕನ್ಯಕಾಂ ಪಾತು ಶಾಂಭವೀ ।\\
ಗೃಹೇಷು ಸರ್ವಕಲ್ಯಾಣೀ ಪಾತು ನಿತ್ಯಂ ಮಹೇಶ್ವರೀ ॥೧೫॥

ಪೂರ್ವೇ ಕಾದಂಬರೀ ಪಾತು ವಹ್ನೌ ಶುಕ್ಲೇಶ್ವರೀ ತಥಾ ।\\
ದಕ್ಷಿಣೇ ಕರಾಲಿನೀ ಪಾತು, ಪ್ರೇತಾರೂಢಾ ತು ನೈರ್ಋತೇ ॥೧೬॥

ಪಾಶಹಸ್ತಾ ಪಶ್ಚಿಮೇ ಪಾಯಾತ್ ವಾಯವ್ಯೇ ಮೃಗವಾಹಿನೀ ।\\
ಪಾತು ಮೇ ಚೋತ್ತರೇ ದೇವೀ ಯಕ್ಷಿಣೀ ಸಿಂಹವಾಹಿನೀ ।\\
ಈಶಾನೇ ಶೂಲಿನೀ ಪಾತು ಊರ್ಧ್ವೇ ಚ ಖಗಗಾಮಿನೀ ॥೧೭॥

ಅಧಸ್ತಾತ್ ವೈಷ್ಣವೀ ಪಾತು, ಸರ್ವತ್ರ ನಾರಸಿಂಹಿಕಾ ।\\
ಪ್ರಭಾತೇ ಸುಂದರೀ ಪಾತು ಮಧ್ಯಾಹ್ನೇ ಜಗದಮ್ಬಿಕಾ ॥೧೮॥

ಸಾಯಾಹ್ನೇ ಚಂಡಿಕಾ ಪಾತು ನಿಶೀಥೇಽತ್ರ ನಿಶಾಚರೀ ।\\
ನಿಶಾಂತೇ ಖೇಚರೀ ಪಾತು ಸರ್ವದಾ ದಿವ್ಯಯೋಗಿನೀ ॥೧೯॥

ವಾಯೌ ಮಾಂ ಪಾತು ವೇತಾಲೀ ವಾಹನೇ ವಜ್ರಧಾರಿಣೀ ।\\
ಸಿಂಹಾ ಸಿಂಹಾಸನೇ ಪಾತು ಶಯ್ಯಾಂ ಚ ಭಗಮಾಲಿನೀ ॥೨೦॥

ಸರ್ವರೋಗೇಷು ಮಾಂ ಪಾತು ಕಾಲರಾತ್ರಿಸ್ವರೂಪಿಣೀ ।\\
ಯಕ್ಷೇಭ್ಯೋ ಯಾಕಿನೀ ಪಾತು ರಾಕ್ಷಸೇ ಡಾಕಿನೀ ತಥಾ ॥೨೧॥

ಭೂತಪ್ರೇತಪಿಶಾಚೇಭ್ಯೋ ಹಾಕಿನೀ ಪಾತು ಮಾಂ ಸದಾ ।\\
ಮಂತ್ರಂ ಮಂತ್ರಾಭಿಚಾರೇಷು ಶಾಕಿನೀ ಪಾತು ಮಾಂ ಸದಾ ॥೨೨॥

ಸರ್ವತ್ರ ಸರ್ವದಾ ಪಾತು ಶ್ರೀದೇವೀ ಗಿರಿಜಾತ್ಮಜಾ ।\\
ಇತ್ಯೇತತ್ ಕಥಿತಂ ಗುಹ್ಯಂ ಶೀತಲಾಕವಚಮುತ್ತಮಮ್ ॥೨೩॥
%===========================================================================================
\section{ಅನ್ನಪೂರ್ಣಾ ಕವಚಂ}
\addcontentsline{toc}{section}{ಅನ್ನಪೂರ್ಣಾ ಕವಚಂ}
ದೇವ್ಯುವಾಚ॥\\
ಭವತಾ ತ್ವನ್ನಪೂರ್ಣಾಯಾ ಯಾ ಯಾ ವಿದ್ಯಾಃ ಸುದುರ್ಲಭಾಃ~।\\
ಕೃಪಯಾ ಕಥಿತಾಃ ಸರ್ವಾಃ ಶ್ರುತಾಶ್ಚಾಧಿಗತಾ ಮಯಾ ॥೧ ॥

ಸಾಂಪ್ರತಂ ಶ್ರೋತುಮಿಚ್ಛಾಮಿ ಕವಚಂ ಮಂತ್ರ ವಿಗ್ರಹಂ~।\\
ಈಶ್ವರ ಉವಾಚ॥\\
ಶೃಣು ಪಾರ್ವತಿ ವಕ್ಷ್ಯಾಮಿ ಸಾವಧಾನಾವಧಾರಯ॥೨ ॥

ಬ್ರಹ್ಮವಿದ್ಯಾ ಸ್ವರೂಪಂ ಚ ಮಹದೈಶ್ವರ್ಯದಾಯಕಂ।\\
ಪಠನಾದ್ಧಾರಣಾನ್ಮರ್ತ್ಯ ಸ್ತ್ರೈಲೋಕ್ಯೈಶ್ವರ್ಯ ಭಾಗ್ಭವೇತ್॥೩ ॥

ತ್ರೈಲೋಕ್ಯ ರಕ್ಷಣಸ್ಯಾಸ್ಯ ಕವಚಸ್ಯ ಋಷಿಃ ಶಿವಃ।\\
ಛಂದೋ ವಿರಾಡ್ ದೇವತಾ ಸ್ಯಾದನ್ನಪೂರ್ಣಾ ಸಮೃದ್ಧಿದಾ॥೪ ॥

ಧರ್ಮಾರ್ಥಕಾಮಮೋಕ್ಷೇಷು ವಿನಿಯೋಗಃ ಪ್ರಕೀರ್ತಿತಃ।\\
ಹ್ರೀಂ ನಮೋ ಭಗವತ್ಯಂತೇ ಮಾಹೇಶ್ವರಿ ಪದಂ ತತಃ॥೫ ॥

ಅನ್ನಪೂರ್ಣೇ ತತಃ ಸ್ವಾಹಾ ಚೈಷಾ ಸಪ್ತದಶಾಕ್ಷರೀ।\\
ಪಾತು ಮಾಮನ್ನಪೂರ್ಣಾ ಸಾ ಯಾ ಖ್ಯಾತಾ ಭುವನತ್ರಯೇ ॥೬ ॥

ವಿಮಾಯಾ ಪ್ರಣವಾದ್ಯೈಷಾ ತಥಾ ಸಪ್ತದಶಾಕ್ಷರೀ~।\\
ಪಾತ್ವನ್ನಪೂರ್ಣಾ ಸರ್ವಾಂಗೇ ರತ್ನಕುಂಭಾನ್ನಪಾತ್ರದಾ॥೭ ॥

ಶ್ರೀಬೀಜಾದ್ಯಾ ತಥೈವೈಷಾ ದ್ವಿರಂಧ್ರಾರ್ಣಾ ತಥಾ ಮುಖಂ।\\
ಪ್ರಣವಾದ್ಯಾ ಭ್ರುವೌ ಪಾತು ಕಂಠಂ ವಾಗ್ಬೀಜಪೂರ್ವಿಕಾ॥೮ ॥

ಕಾಮಬೀಜಾದಿಕಾ ಚೈಷಾ ಹೃದಯಂ ತು ಮಹೇಶ್ವರೀ।\\
ಐಂಶ್ರೀಂಹ್ರೀಂ ಚ ನಮೋಽಂತೇ ತು ಭಗವತೀ ಪದಂ ತತಃ॥೯ ॥

ಮಾಹೇಶ್ವರಿ ಪದಂ ಚಾನ್ನಪೂರ್ಣೇ ಸ್ವಾಹೇತಿ ಪಾತು ಮೇ।\\
ನಾಭಿಮೇಕೋನವಿಂಶಾರ್ಣಾ ಪಾಯಾನ್ಮಾಹೇಶ್ವರೀ ಸದಾ॥೧೦ ॥

ತಾರಂ ಮಾಯಾ ರಮಾ ಕಾಮಃ ಷೋಡಶಾರ್ಣಾ ತತಃ ಪರಂ।\\
ಶಿರಃಸ್ಥಾ ಸರ್ವದಾ ಪಾತು ವಿಂಶತ್ಯರ್ಣಾತ್ಮಿಕಾ ಪರಾ॥೧೧ ॥

ಅನ್ನಪೂರ್ಣಾ ಮಹಾವಿದ್ಯಾ ಹ್ರೀಂ ಪಾತು ಭುವನೇಶ್ವರೀ।\\
ಶಿರಃ ಶ್ರೀಂ ಹ್ರೀಂ ತಥಾ ಕ್ಲೀಂ ಚ ತ್ರಿಪುಟಾ ಪಾತು ಮೇಗುದಂ॥೧೨॥

ಷಡ್‌ದೀರ್ಘ ಭಾಜಾ ಬೀಜೇನ ಷಡಂಗಾನಿ ಪುನಂತು ಮಾಂ।\\
ಇಂದ್ರೋ ಮಾಂ ಪಾತು ಪೂರ್ವಂ ಚ ವಹ್ನಿಕೋಣೇನಲೋವತು॥೧೩ ॥

ಯಮೋಮಾಂ ದಕ್ಷಿಣೇ ಪಾತು ನೈರ್ಋತ್ಯಾಂ ನಿರ್ಋತಿಶ್ಚ ಮಾಂ।\\
ಪಶ್ಚಿಮೇ ವರುಣಃ ಪಾತು ವಾಯವ್ಯಾಂ ಪವನೋವತು॥೧೪ ॥

ಕುಬೇರಶ್ಚೋತ್ತರೇ ಪಾತು ಚೈಶಾನ್ಯಾಂ ಶಂಕರೋಽವತು।\\
ಊರ್ಧ್ವಾಧಃ ಪಾತು ಸತತಂ ಬ್ರಹ್ಮಾನಂತೋ ಯಥಾಕ್ರಮಾತ್॥೧೫ ॥

ಪಾಂತು ವಜ್ರಾದ್ಯಾಯುಧಾನಿ ದಶದಿಕ್ಷು ಯಥಾಕ್ರಮಂ।\\
ಇತಿ ತೇ ಕಥಿತಂ ಪುಣ್ಯಂ ತ್ರೈಲೋಕ್ಯ ರಕ್ಷಣಂ ಪರಂ॥೧೬ ॥

ಯದ್ಧೃತ್ವಾ ಪಠನಾದ್ದೇವಾಃ ಸರ್ವೈಶ್ವರ್ಯಮವಾಪ್ನುಯುಃ।\\
ಬ್ರಹ್ಮಾ ವಿಷ್ಣುಶ್ಚ ರುದ್ರಶ್ಚ ಧಾರಣಾತ್ ಪಠನಾದ್ಯತಃ॥೧೭ ॥

ಸೃಜತ್ಯವತಿ ಹಂತ್ಯೇವ ಕಲ್ಪೇ ಕಲ್ಪೇ ಪೃಥಕ್ ಪೃಥಕ್।\\
ಪುಷ್ಪಾಂಜಲ್ಯಷ್ಟಕಂ ದೇವ್ಯೈ ಮೂಲೇನೈವ ಸಮರ್ಪಯೇತ್॥೧೮ ॥

ಕವಚಸ್ಯಾಸ್ಯ ಪಠನಾತ್ ಪೂಜಾಯಾಃ ಫಲಮಾಪ್ನುಯಾತ್।\\
ವಾಣೀ ವಕ್ತ್ರೇ ವಸೇತ್ತಸ್ಯ ಸತ್ಯಂ ಸತ್ಯಂ ನ ಸಂಶಯಃ॥೧೯ ॥

ಅಷ್ಟೋತ್ತರ ಶತಂ ಚಾಸ್ಯ ಪುರಶ್ಚರ್ಯಾ ವಿಧಿಃ ಸ್ಮೃತಃ।\\
ಭೂರ್ಜೇ ವಿಲಿಖ್ಯ ಗುಟಿಕಾಂ ಸ್ವರ್ಣಸ್ಥಾಂ ಧಾರಯೇದ್ಯದಿ॥೨೦ ॥

ಕಂಠೇ ವಾ ದಕ್ಷಿಣೇ ಬಾಹೌ ಸೋಪಿ ಪುಣ್ಯವತಾಂ ವರಃ।\\
ಸರ್ವಾಣ್ಯಸ್ತ್ರಾಣಿ ಶಸ್ತ್ರಾಣಿ ತದ್ಗಾತ್ರಂ ಪ್ರಾಪ್ಯ ಪಾರ್ವತಿ।\\
ಮಾಲ್ಯಾನಿ ಕುಸುಮಾನೀವ ಸುಖದಾನಿ ಭವಂತಿ ಹಿ॥೨೧ ॥
\begin{center}{\Large॥ಇತಿ ಭೈರವ ತಂತ್ರೇ ಅನ್ನಪೂರ್ಣಾ ಕವಚಂ॥}\end{center}

%===========================================================================================
\section{ಶ್ರೀದತ್ತಾತ್ರೇಯಕವಚಂ}
\addcontentsline{toc}{section}{ಶ್ರೀದತ್ತಾತ್ರೇಯಕವಚಂ}
ಶ್ರೀಪಾದಃ ಪಾತು ಮೇ ಪಾದಾವೂರೂ ಸಿದ್ಧಾಸನಸ್ಥಿತಃ ।\\
ಪಾಯಾದ್ದಿಗಂಬರೋ ಗುಹ್ಯಂ ನೃಹರಿಃ ಪಾತು ಮೇ ಕಟಿಂ ॥೧॥

ನಾಭಿಂ ಪಾತು ಜಗತ್ಸ್ರಷ್ಟೋದರಂ ಪಾತು ದಲೋದರಃ ।\\
ಕೃಪಾಲುಃ ಪಾತು ಹೃದಯಂ ಷಡ್ಭುಜಃ ಪಾತು ಮೇ ಭುಜೌ ॥೨॥

ಸ್ರಕ್ಕುಂಡೀಶೂಲಡಮರುಶಂಖಚಕ್ರಧರಃ ಕರಾನ್ ।\\
ಪಾತು ಕಂಠಂ ಕಂಬುಕಂಠಃ ಸುಮುಖಃ ಪಾತು ಮೇ ಮುಖಂ ॥೩॥

ಜಿಹ್ವಾಂ ಮೇ ವೇದವಾಕ್ಪಾತು ನೇತ್ರೇ ಮೇ ಪಾತು ದಿವ್ಯದೃಕ್ ।\\
ನಾಸಿಕಾಂ ಪಾತು ಗಂಧಾತ್ಮಾ ಪಾತು ಪುಣ್ಯಶ್ರವಾಃ ಶ್ರುತೀ ॥೪॥

ಲಲಾಟಂ ಪಾತು ಹಂಸಾತ್ಮಾ ಶಿರಃ ಪಾತು ಜಟಾಧರಃ ।\\
ಕರ್ಮೇಂದ್ರಿಯಾಣಿ ಪಾತ್ವೀಶಃ ಪಾತು ಜ್ಞಾನೇಂದ್ರಿಯಾಣ್ಯಜಃ ॥೫॥

ಸರ್ವಾಂತರೋಽನ್ತಕರಣಂ ಪ್ರಾಣಾನ್ಮೇ ಪಾತು ಯೋಗಿರಾಟ್ ।\\
ಉಪರಿಷ್ಟಾದಧಸ್ತಾಚ್ಚ ಪೃಷ್ಠತಃ ಪಾರ್ಶ್ವತೋಽಗ್ರಜಃ ॥೬॥

ಅಂತರ್ಬಹಿಶ್ಚ ಮಾಂ ನಿತ್ಯಂ ನಾನಾರೂಪಧರೋಽವತು ।\\
ವರ್ಜಿತಂ ಕವಚೇನಾವ್ಯಾತ್ಸ್ಥಾನಂ ಮೇ ದಿವ್ಯದರ್ಶನಃ ॥೭॥

ರಾಜತಃ ಶತ್ರುತೋ ಹಿಂಸ್ರಾದ್ ದುಷ್ಪ್ರಯೋಗಾದಿತೋಽಘತಃ ।\\
ಆಧಿವ್ಯಾಧಿಭಯಾರ್ತಿಭ್ಯೋ ದತ್ತಾತ್ರೇಯಃ ಸದಾವತು ॥೮॥

ಧನಧಾನ್ಯಗೃಹಕ್ಷೇತ್ರಸ್ತ್ರೀಪುತ್ರಪಶುಕಿಂಕರಾನ್ ।\\
ಜ್ಞಾತೀಂಶ್ಚ ಪಾತು ನಿತ್ಯಂ ಮೇಽನಸೂಯಾನಂದವರ್ಧನಃ ॥೯॥

ಬಾಲೋನ್ಮತ್ತಪಿಶಾಚಾಭೋ ದ್ಯುನಿಟ್ಸಂಧಿಷು ಪಾತು ಮಾಂ ।\\
ಭೂತಭೌತಿಕಮೃತ್ಯುಭ್ಯೋ ಹರಿಃ ಪಾತು ದಿಗಂಬರಃ ॥೧೦॥

ಯ ಏತದ್ದತ್ತಕವಚಂ ಸನ್ನಹ್ಯಾದ್ಭಕ್ತಿಭಾವಿತಃ ।\\
ಸರ್ವಾನರ್ಥವಿನಿರ್ಮುಕ್ತೋ ಗ್ರಹಪೀಡಾವಿವರ್ಜಿತಃ ॥೧೧॥

ಭೂತಪ್ರೇತಪಿಶಾಚಾದ್ಯೈರ್ದೇವೈರಪ್ಯಪರಾಜಿತಃ ।\\
ಭುಕ್ತ್ವಾತ್ರ ದಿವ್ಯಭೋಗಾನ್ಸ ದೇಹಾಂತೇ ತತ್ಪದಂ ವ್ರಜೇತ್ ॥೧೨॥

\authorline{ಇತಿ ಶ್ರೀಮದ್ ಪರಮಪೂಜನೀಯ ಶ್ರೀವಾಸುದೇವಾನಂದಸರಸ್ವತೀವಿರಚಿತಂ ಶ್ರೀದತ್ತಾತ್ರೇಯಕವಚಂ ಸಂಪೂರ್ಣಂ ।}

%==========================================================================================
\section{॥ ಚಂಡೀಕವಚಮ್ ॥}
ಅಸ್ಯ ಶ್ರೀಚಂಡೀಕವಚಸ್ಯ~। ಬ್ರಹ್ಮಾ ಋಷಿಃ~। ಅನುಷ್ಟುಪ್ ಛಂದಃ~।\\ ಚಾಮುಂಡಾ ದೇವತಾ~। ಅಂಗನ್ಯಾಸೋಕ್ತಮಾತರೋ ಬೀಜಮ್~।\\ದಿಗ್ಬಂಧದೇವತಾಸ್ತತ್ವಮ್~। ಶ್ರೀಜಗದಂಬಾಪ್ರೀತ್ಯರ್ಥೇ ಜಪೇ ವಿನಿಯೋಗಃ ॥

ಓಂ ನಮಶ್ಚಂಡಿಕಾಯೈ ॥ ಮಾರ್ಕಂಡೇಯ ಉವಾಚ ॥\\
ಯದ್ಗುಹ್ಯಂ ಪರಮಂ ಲೋಕೇ ಸರ್ವರಕ್ಷಾಕರಂ ನೃಣಾಮ್~।\\
ಯನ್ನ ಕಸ್ಯಚಿದಾಖ್ಯಾತಂ ತನ್ಮೇ ಬ್ರೂಹಿ ಪಿತಾಮಹ ॥೧॥

ಬ್ರಹ್ಮೋವಾಚ ॥\\
ಅಸ್ತಿ ಗುಹ್ಯತಮಂ ವಿಪ್ರ ಸರ್ವಭೂತೋಪಕಾರಕಮ್~।\\
ದೇವ್ಯಾಸ್ತು ಕವಚಂ ಪುಣ್ಯಂ ತಚ್ಛೃಣುಷ್ವ ಮಹಾಮುನೇ ॥೨॥

ಪ್ರಥಮಂ ಶೈಲಪುತ್ರೀತಿ ದ್ವಿತೀಯಂ ಬ್ರಹ್ಮಚಾರಿಣೀ~।\\
ತೃತೀಯಂ ಚಂದ್ರಘಂಟೇತಿ ಕೂಷ್ಮಾಂಡೇತಿ ಚತುರ್ಥಕಮ್ ॥೩॥

ಪಂಚಮಂ ಸ್ಕಂದಮಾತೇತಿ ಷಷ್ಠಂ ಕಾತ್ಯಾಯನೀತಿ ಚ~।\\
ಸಪ್ತಮಂ ಕಾಲರಾತ್ರಿಶ್ಚ ಮಹಾಗೌರೀತಿ ಚಾಷ್ಟಮಮ್ ॥೪॥

ನವಮಂ ಸಿದ್ಧಿದಾತ್ರೀ ಚ ನವದುರ್ಗಾಃ ಪ್ರಕೀರ್ತಿತಾಃ~।\\
ಉಕ್ತಾನ್ಯೇತಾನಿ ನಾಮಾನಿ ಬ್ರಹ್ಮಣೈವ ಮಹಾತ್ಮನಾ ॥೫॥

ಅಗ್ನಿನಾ ದಹ್ಯಮಾನಸ್ತು ಶತ್ರುಮಧ್ಯೇ ಗತೋ ರಣೇ~।\\
ವಿಷಮೇ ದುರ್ಗಮೇ ಚೈವ ಭಯಾರ್ತಾಃ ಶರಣಂ ಗತಾಃ ॥೬॥

ನ ತೇಷಾಂ ಜಾಯತೇ ಕಿಂಚಿದಶುಭಂ ರಣಸಂಕಟೇ~।\\
ನಾಪದಂ ತಸ್ಯ ಪಶ್ಯಾಮಿ ಶೋಕದುಃಖಭಯಂ ನಹಿ ॥೭॥

ಯೈಸ್ತು ಭಕ್ತ್ಯಾ ಸ್ಮೃತಾ ನೂನಂ ತೇಷಾಂ ಸಿದ್ಧಿಃ ಪ್ರಜಾಯತೇ~।\\
ಪ್ರೇತಸಂಸ್ಥಾ ತು ಚಾಮುಂಡಾ ವಾರಾಹೀ ಮಹಿಷಾಸನಾ ॥೮॥

ಐಂದ್ರೀ ಗಜಸಮಾರೂಢಾ ವೈಷ್ಣವೀ ಗರುಡಾಸನಾ~।\\
ಮಾಹೇಶ್ವರೀ ವೃಷಾರೂಢಾ ಕೌಮಾರೀ ಶಿಖಿವಾಹನಾ ॥೯॥

ಬ್ರಾಹ್ಮೀ ಹಂಸಸಮಾರೂಢಾ ಸರ್ವಾಭರಣಭೂಷಿತಾ~।\\
ನಾನಾಽಭರಣಶೋಭಾಢ್ಯಾ ನಾನಾರತ್ನೋಪಶೋಭಿತಾಃ ॥೧೦॥

ದೃಶ್ಯಂತೇ ರಥಮಾರೂಢಾ ದೇವ್ಯಃ ಕ್ರೋಧಸಮಾಕುಲಾಃ~।\\
ಶಂಖಂ ಚಕ್ರಂ ಗದಾಂ ಶಕ್ತಿಂ ಹಲಂ ಚ ಮುಸಲಾಯುಧಮ್ ॥೧೧॥

ಖೇಟಕಂ ತೋಮರಂ ಚೈವ ಪರಶುಂ ಪಾಶಮೇವ ಚ~।\\
ಕುಂತಾಯುಧಂ ತ್ರಿಶೂಲಂ ಚ ಶಾರ್ಙ್ಗಮಾಯುಧಮುತ್ತಮಮ್ ॥೧೨॥

ದೈತ್ಯಾನಾಂ ದೇಹನಾಶಾಯ ಭಕ್ತಾನಾಮಭಯಾಯ ಚ~।\\
ಧಾರಯಂತ್ಯಾಯುಧಾನೀತ್ಥಂ ದೇವಾನಾಂ ಚ ಹಿತಾಯ ವೈ ॥೧೩॥

ಮಹಾಬಲೇ ಮಹೋತ್ಸಾಹೇ ಮಹಾಭಯವಿನಾಶಿನಿ~।\\
ತ್ರಾಹಿ ಮಾಂ ದೇವಿ ದುಷ್ಪ್ರೇಕ್ಷ್ಯೇ ಶತ್ರೂಣಾಂ ಭಯವರ್ಧಿನಿ ॥೧೪॥

ಪ್ರಾಚ್ಯಾಂ ರಕ್ಷತು ಮಾಮೈಂದ್ರೀ ಆಗ್ನೇಯ್ಯಾಮಗ್ನಿದೇವತಾ~।\\
ದಕ್ಷಿಣೇಽವತು ವಾರಾಹೀ ನೈರ್‌ಋತ್ಯಾಂ ಖಡ್ಗಧಾರಿಣೀ ॥೧೫॥

ಪ್ರತೀಚ್ಯಾಂ ವಾರುಣೀ ರಕ್ಷೇದ್ವಾಯವ್ಯಾಂ ಮೃಗವಾಹಿನೀ~।\\
ಉದೀಚ್ಯಾಂ ರಕ್ಷ ಕೌಬೇರಿ ಈಶಾನ್ಯಾಂ ಶೂಲಧಾರಿಣಿ ॥೧೬॥

ಊರ್ಧ್ವಂ ಬ್ರಹ್ಮಾಣೀ ಮೇ ರಕ್ಷೇದಧಸ್ತಾದ್ವೈಷ್ಣವೀ ತಥಾ~।\\
ಏವಂ ದಶ ದಿಶೋ ರಕ್ಷೇಚ್ಚಾಮುಂಡಾ ಶವವಾಹನಾ ॥೧೭॥

ಜಯಾ ಮೇ ಚಾಗ್ರತಃ ಸ್ಥಾತು ವಿಜಯಾ ಸ್ಥಾತು ಪೃಷ್ಠತಃ~।\\
ಅಜಿತಾ ವಾಮಪಾರ್ಶ್ವೇ ತು ದಕ್ಷಿಣೇ ಚಾಪರಾಜಿತಾ ॥೧೮॥

ಶಿಖಾಂ ಮೇ ದ್ಯೋತಿನೀ ರಕ್ಷೇದುಮಾ ಮೂರ್ಧ್ನಿ ವ್ಯವಸ್ಥಿತಾ~।\\
ಮಾಲಾಧರೀ ಲಲಾಟೇ ಚ ಭ್ರುವೌ ರಕ್ಷೇದ್ಯಶಸ್ವಿನೀ ॥೧೯॥

ತ್ರಿನೇತ್ರಾ ಚ ಭ್ರುವೋರ್ಮಧ್ಯೇ ಯಮಘಂಟಾ ಚ ನಾಸಿಕೇ~।\\
ಶಂಖಿನೀ ಚಕ್ಷುಷೋರ್ಮಧ್ಯೇ ಶ್ರೋತ್ರಯೋರ್ದ್ವಾರವಾಸಿನೀ ॥೨೦॥

ಕಪೋಲೌ ಕಾಲಿಕಾ ರಕ್ಷೇತ್ಕರ್ಣಮೂಲೇ ತು ಶಾಂಕರೀ~।\\
ನಾಸಿಕಾಯಾಂ ಸುಗಂಧಾ ಚ ಉತ್ತರೋಷ್ಠೇ ಚ ಚರ್ಚಿಕಾ ॥೨೧॥

ಅಧರೇ ಚಾಮೃತಕಲಾ ಜಿಹ್ವಾಯಾಂ ಚ ಸರಸ್ವತೀ~।\\
ದಂತಾನ್ ರಕ್ಷತು ಕೌಮಾರೀ ಕಂಠಮಧ್ಯೇ ತು ಚಂಡಿಕಾ ॥೨೨॥

ಘಂಟಿಕಾಂ ಚಿತ್ರಘಂಟಾ ಚ ಮಹಾಮಾಯಾ ಚ ತಾಲುಕೇ~।\\
ಕಾಮಾಕ್ಷೀ ಚಿಬುಕಂ ರಕ್ಷೇದ್ವಾಚಂ ಮೇ ಸರ್ವಮಂಗಲಾ ॥೨೩॥

ಗ್ರೀವಾಯಾಂ ಭದ್ರಕಾಲೀ ಚ ಪೃಷ್ಠವಂಶೇ ಧನುರ್ಧರೀ~।\\
ನೀಲಗ್ರೀವಾ ಬಹಿಃಕಂಠೇ ನಲಿಕಾಂ ನಲಕೂಬರೀ ॥೨೪॥

ಖಡ್ಗಧಾರಿಣ್ಯುಭೌ ಸ್ಕಂಧೌ ಬಾಹೂ ಮೇ ವಜ್ರಧಾರಿಣೀ~।\\
ಹಸ್ತಯೋರ್ದಂಡಿನೀ ರಕ್ಷೇದಂಬಿಕಾ ಚಾಂಗುಲೀಸ್ತಥಾ ॥೨೫॥

ನಖಾಂಛೂಲೇಶ್ವರೀ ರಕ್ಷೇತ್ ಕುಕ್ಷೌ ರಕ್ಷೇನ್ನಲೇಶ್ವರೀ~।\\
ಸ್ತನೌ ರಕ್ಷೇನ್ಮಹಾಲಕ್ಷ್ಮೀರ್ಮನಃ ಶೋಕವಿನಾಶಿನೀ ॥೨೬॥

ಹೃದಯಂ ಲಲಿತಾದೇವೀ ಉದರಂ ಶೂಲಧಾರಿಣೀ~।\\
ನಾಭೌ ಚ ಕಾಮಿನೀ ರಕ್ಷೇದ್ಗುಹ್ಯಂ ಗುಹ್ಯೇಶ್ವರೀ ತಥಾ ॥೨೭॥

ಕಟ್ಯಾಂ ಭಗವತೀ ರಕ್ಷೇಜ್ಜಾನುನೀ ವಿಂಧ್ಯವಾಸಿನೀ~।\\
ಭೂತನಾಥಾ ಚ ಮೇಢ್ರಂ ಮೇ ಊರೂ ಮಹಿಷವಾಹಿನೀ ॥೨೮॥

ಜಂಘೇ ಮಹಾಬಲಾ ಪ್ರೋಕ್ತಾ ಸರ್ವಕಾಮಪ್ರದಾಯಿನೀ~।\\
ಗುಲ್ಫಯೋರ್ನಾರಸಿಂಹೀ ಚ ಪಾದೌ ಚಾಮಿತತೇಜಸೀ ॥೨೯॥

ಪಾದಾಂಗುಲೀಃ ಶ್ರೀರ್ಮೇ ರಕ್ಷೇತ್ಪಾದಾಧಸ್ತಲವಾಸಿನೀ~।\\
ನಖಾಂದಂಷ್ಟ್ರಾಃ ಕರಾಲೀ ಚ ಕೇಶಾಂಶ್ಚೈವೋರ್ಧ್ವಕೇಶಿನೀ ॥೩೦॥

ರೋಮಕೂಪೇಷು ಕೌಬೇರೀ ತ್ವಚಂ ವಾಗೀಶ್ವರೀ ತಥಾ~।\\
ರಕ್ತಮಜ್ಜಾವಸಾಮಾಂಸಾನ್ಯಸ್ಥಿಮೇದಾಂಸಿ ಪಾರ್ವತೀ ॥೩೧॥

ಅಂತ್ರಾಣಿ ಕಾಲರಾತ್ರಿಶ್ಚ ಪಿತ್ತಂ ಚ ಮುಕುಟೇಶ್ವರೀ~।\\
ಪದ್ಮಾವತೀ ಪದ್ಮಕೋಶೇ ಕಫೇ ಚೂಡಾಮಣಿಸ್ತಥಾ ॥೩೨॥

ಜ್ವಾಲಾಮುಖೀ ನಖಜ್ವಾಲಾ ಅಭೇದ್ಯಾ ಸರ್ವಸಂಧಿಷು~।\\
ಶುಕ್ರಂ ಬ್ರಹ್ಮಾಣೀ ಮೇ ರಕ್ಷೇಚ್ಛಾಯಾಂ ಛತ್ರೇಶ್ವರೀ ತಥಾ ॥೩೩॥

ಅಹಂಕಾರಂ ಮನೋ ಬುದ್ಧಿಂ ರಕ್ಷ ಮೇ ಧರ್ಮಚಾರಿಣಿ~।\\
ಪ್ರಾಣಾಪಾನೌ ತಥಾ ವ್ಯಾನಂ ಸಮಾನೋದಾನಮೇವ ಚ ॥೩೪॥

ಯಶಃ ಕೀರ್ತಿಂ ಚ ಲಕ್ಷ್ಮೀಂ ಚ ಸದಾ ರಕ್ಷತು ವೈಷ್ಣವೀ~।\\
ಗೋತ್ರಮಿಂದ್ರಾಣೀ ಮೇ ರಕ್ಷೇತ್ಪಶೂನ್ಮೇ ರಕ್ಷ ಚಂಡಿಕೇ ॥೩೫॥

ಪುತ್ರಾನ್ ರಕ್ಷೇನ್ಮಹಾಲಕ್ಷ್ಮೀರ್ಭಾರ್ಯಾಂ ರಕ್ಷತು ಭೈರವೀ~।\\
ಮಾರ್ಗಂ ಕ್ಷೇಮಕರೀ ರಕ್ಷೇದ್ವಿಜಯಾ ಸರ್ವತಃ ಸ್ಥಿತಾ ॥೩೬॥

ರಕ್ಷಾಹೀನಂ ತು ಯತ್ಸ್ಥಾನಂ ವರ್ಜಿತಂ ಕವಚೇನ ತು~।\\
ತತ್ಸರ್ವಂ ರಕ್ಷ ಮೇ ದೇವಿ ಜಯಂತೀ ಪಾಪನಾಶಿನೀ ॥೩೭॥

ಪದಮೇಕಂ ನ ಗಚ್ಛೇತ್ತು ಯದೀಚ್ಛೇಚ್ಛುಭಮಾತ್ಮನಃ~।\\
ಕವಚೇನಾವೃತೋ ನಿತ್ಯಂ ಯತ್ರ ಯತ್ರಾಧಿಗಚ್ಛತಿ ॥೩೮॥

ತತ್ರ ತತ್ರಾರ್ಥ ಲಾಭಶ್ಚ ವಿಜಯಃ ಸಾರ್ವಕಾಮಿಕಃ~।\\
ಯಂ ಯಂ ಕಾಮಯತೇ ಕಾಮಂ ತಂ ತಂ ಪ್ರಾಪ್ನೋತಿ ನಿಶ್ಚಿತಮ್ ॥೩೯॥

ಪರಮೈಶ್ವರ್ಯಮತುಲಂ ಪ್ರಾಪ್ಸ್ಯತೇ ಭೂತಲೇ ಪುಮಾನ್~।\\
ನಿರ್ಭಯೋ ಜಾಯತೇ ಮರ್ತ್ಯಃ ಸಂಗ್ರಾಮೇಷ್ವ ಪರಾಜಿತಃ ॥೪೦॥

ತ್ರೈಲೋಕ್ಯೇ ತು ಭವೇತ್ಪೂಜ್ಯಃ ಕವಚೇನಾವೃತಃ ಪುಮಾನ್~।\\
ಇದಂ ತು ದೇವ್ಯಾಃ ಕವಚಂ ದೇವಾನಾಮಪಿ ದುರ್ಲಭಮ್ ॥೪೧॥

ಯಃ ಪಠೇತ್ಪ್ರಯತೋ ನಿತ್ಯಂ ತ್ರಿಸಂಧ್ಯಂ ಶ್ರದ್ಧಯಾನ್ವಿತಃ~।\\
ದೈವೀ ಕಲಾ ಭವೇತ್ತಸ್ಯ ತ್ರೈಲೋಕ್ಯೇಪ್ಯಪರಾಜಿತಃ ॥೪೨॥

ಜೀವೇದ್ವರ್ಷಶತಂ ಸಾಗ್ರಮಪಮೃತ್ಯು ವಿವರ್ಜಿತಃ~।\\
ನಶ್ಯಂತಿ ವ್ಯಾಧಯಃ ಸರ್ವೇ ಲೂತಾವಿಸ್ಫೋಟಕಾದಯಃ ॥೪೩॥

ಸ್ಥಾವರಂ ಜಂಗಮಂ ವಾಪಿ ಕೃತ್ರಿಮಂ ಚಾಪಿ ಯದ್ವಿಷಮ್~।\\
ಆಭಿಚಾರಾಣಿ ಸರ್ವಾಣಿ ಮಂತ್ರಯಂತ್ರಾಣಿ ಭೂತಲೇ ॥೪೪॥

ಭೂಚರಾಃ ಖೇಚರಾಶ್ಚೈವ ಜಲಜಾಶ್ಚೋಪದೇಶಿಕಾಃ~।\\
ಸಹಜಾಃ ಕುಲಜಾ ಮಾಲಾಃ ಶಾಕಿನೀ ಡಾಕಿನೀ ತಥಾ ॥೪೫॥

ಅಂತರಿಕ್ಷಚರಾ ಘೋರಾ ಡಾಕಿನ್ಯಶ್ಚ ಮಹಾಬಲಾಃ~।\\
ಗ್ರಹಭೂತಪಿಶಾಚಾಶ್ಚ ಯಕ್ಷಗಂಧರ್ವರಾಕ್ಷಸಾಃ ॥೪೬॥

ಬ್ರಹ್ಮರಾಕ್ಷಸವೇತಾಲಾಃ ಕೂಷ್ಮಾಂಡಾ ಭೈರವಾದಯಃ~।\\
ನಶ್ಯಂತಿ ದರ್ಶನಾತ್ತಸ್ಯ ಕವಚೇ ಹೃದಿ ಸಂಸ್ಥಿತೇ ॥೪೭॥

ಮಾನೋನ್ನತಿರ್ಭವೇದ್ರಾಜ್ಞಸ್ತೇಜೋವೃದ್ಧಿಕರಂ ಪರಮ್~।\\
ಯಶಸಾ ವರ್ಧತೇ ಸೋಽಪಿ ಕೀರ್ತಿಮಂಡಿತಭೂತಲೇ ॥೪೮॥

ಜಪೇತ್ಸಪ್ತಶತೀಂ ಚಂಡೀಂ ಕೃತ್ವಾ ತು ಕವಚಂ ಪುರಾ~।\\
ಯಾವದ್ಭೂಮಂಡಲಂ ಧತ್ತೇ ಸಶೈಲವನಕಾನನಮ್ ॥೪೯॥

ತಾವತ್ತಿಷ್ಠತಿ ಮೇದಿನ್ಯಾಂ ಸಂತತಿಃ ಪುತ್ರಪೌತ್ರಿಕೀ~।\\
ದೇಹಾಂತೇ ಪರಮಂ ಸ್ಥಾನಂ ಯತ್ಸುರೈರಪಿ ದುರ್ಲಭಮ್~।\\
ಪ್ರಾಪ್ನೋತಿ ಪುರುಷೋ ನಿತ್ಯಂ ಮಹಾಮಾಯಾಪ್ರಸಾದತಃ ॥೫೦॥
\authorline{ಇತಿ ಶ್ರೀವಾರಾಹಪುರಾಣೇ ಹರಿಹರಬ್ರಹ್ಮವಿರಚಿತಂ ದೇವ್ಯಾಃ ಕವಚಂ ಸಂಪೂರ್ಣಮ್ ॥}
\section{ಅರ್ಗಲಾ ಸ್ತೋತ್ರಮ್}
ಅಸ್ಯ ಶ್ರೀಅರ್ಗಲಾ ಸ್ತೋತ್ರಮಂತ್ರಸ್ಯ ವಿಷ್ಣುರ್ಋಷಿಃ । ಅನುಷ್ಟುಪ್ಛಂದಃ ।  ಶ್ರೀಮಹಾಲಕ್ಷ್ಮೀರ್ದೇವತಾ ।  ಶ್ರೀಜಗದಂಬಾ ಪ್ರೀತ್ಯರ್ಥೇ ಜಪೇ ವಿನಿಯೋಗಃ ॥

ಓಂ ನಮಶ್ಚಂಡಿಕಾಯೈ ॥\\
ಜಯಂತೀ ಮಂಗಲಾ ಕಾಲೀ ಭದ್ರಕಾಲೀ ಕಪಾಲಿನೀ ।\\
ದುರ್ಗಾ ಕ್ಷಮಾ ಶಿವಾ ಧಾತ್ರೀ ಸ್ವಾಹಾ ಸ್ವಧಾ ನಮೋಽಸ್ತು ತೇ ॥೧॥

ಮಧುಕೈಟಭವಿದ್ರಾವಿ ವಿಧಾತೃವರದೇ ನಮಃ ।\\
ರೂಪಂ ದೇಹಿ ಜಯಂ ದೇಹಿ ಯಶೋ ದೇಹಿ ದ್ವಿಷೋ ಜಹಿ ॥೨॥

ಮಹಿಷಾಸುರನಿರ್ನಾಶವಿಧಾತ್ರಿ ವರದೇ ನಮಃ ।\\
ರೂಪಂ ದೇಹಿ ಜಯಂ ದೇಹಿ ಯಶೋ ದೇಹಿ ದ್ವಿಷೋ ಜಹಿ ॥೩॥

ವಂದಿತಾಂಘ್ರಿಯುಗೇ ದೇವಿ ಸರ್ವಸೌಭಾಗ್ಯದಾಯಿನಿ ।\\
ರೂಪಂ ದೇಹಿ ಜಯಂ ದೇಹಿ ಯಶೋ ದೇಹಿ ದ್ವಿಷೋ ಜಹಿ ॥೪॥

ರಕ್ತಬೀಜವಧೇ ದೇವಿ ಚಂಡಮುಂಡವಿನಾಶಿನಿ ।\\
ರೂಪಂ ದೇಹಿ ಜಯಂ ದೇಹಿ ಯಶೋ ದೇಹಿ ದ್ವಿಷೋ ಜಹಿ ॥೫॥

ಅಚಿಂತ್ಯರೂಪಚರಿತೇ ಸರ್ವಶತ್ರುವಿನಾಶಿನಿ ।\\
ರೂಪಂ ದೇಹಿ ಜಯಂ ದೇಹಿ ಯಶೋ ದೇಹಿ ದ್ವಿಷೋ ಜಹಿ ॥೬॥

ನತೇಭ್ಯಃ ಸರ್ವದಾ ಭಕ್ತ್ಯಾ ಚಂಡಿಕೇ ದುರಿತಾಪಹೇ ।\\
ರೂಪಂ ದೇಹಿ ಜಯಂ ದೇಹಿ ಯಶೋ ದೇಹಿ ದ್ವಿಷೋ ಜಹಿ ॥೭॥

ಸ್ತುವದ್ಭ್ಯೋ ಭಕ್ತಿಪೂರ್ವಂ ತ್ವಾಂ ಚಂಡಿಕೇ ವ್ಯಾಧಿನಾಶಿನಿ ।\\
ರೂಪಂ ದೇಹಿ ಜಯಂ ದೇಹಿ ಯಶೋ ದೇಹಿ ದ್ವಿಷೋ ಜಹಿ ॥೮॥

ಚಂಡಿಕೇ ಸತತಂ ಯೇ ತ್ವಾಮರ್ಚಯಂತೀಹ ಭಕ್ತಿತಃ ।\\
ರೂಪಂ ದೇಹಿ ಜಯಂ ದೇಹಿ ಯಶೋ ದೇಹಿ ದ್ವಿಷೋ ಜಹಿ ॥೯॥

ದೇಹಿ ಸೌಭಾಗ್ಯಮಾರೋಗ್ಯಂ ದೇಹಿ ದೇವಿ ಪರಂ ಸುಖಂ ।\\
ರೂಪಂ ದೇಹಿ ಜಯಂ ದೇಹಿ ಯಶೋ ದೇಹಿ ದ್ವಿಷೋ ಜಹಿ ॥೧೦॥

ವಿಧೇಹಿ ದ್ವಿಷತಾಂ ನಾಶಂ ವಿಧೇಹಿ ಬಲಮುಚ್ಚಕೈಃ ।\\
ರೂಪಂ ದೇಹಿ ಜಯಂ ದೇಹಿ ಯಶೋ ದೇಹಿ ದ್ವಿಷೋ ಜಹಿ ॥೧೧॥

ವಿಧೇಹಿ ದೇವಿ ಕಲ್ಯಾಣಂ ವಿಧೇಹಿ ಪರಮಾಂ ಶ್ರಿಯಂ ।\\
ರೂಪಂ ದೇಹಿ ಜಯಂ ದೇಹಿ ಯಶೋ ದೇಹಿ ದ್ವಿಷೋ ಜಹಿ ॥೧೨॥

ವಿದ್ಯಾವಂತಂ ಯಶಸ್ವಂತಂ ಲಕ್ಷ್ಮೀವಂತಂ ಜನಂ ಕುರು ।\\
ರೂಪಂ ದೇಹಿ ಜಯಂ ದೇಹಿ ಯಶೋ ದೇಹಿ ದ್ವಿಷೋ ಜಹಿ ॥೧೩॥

ಪ್ರಚಂಡದೈತ್ಯದರ್ಪಘ್ನೇ ಚಂಡಿಕೇ ಪ್ರಣತಾಯ ಮೇ ।\\
ರೂಪಂ ದೇಹಿ ಜಯಂ ದೇಹಿ ಯಶೋ ದೇಹಿ ದ್ವಿಷೋ ಜಹಿ ॥೧೪॥

ಚತುರ್ಭುಜೇ ಚತುರ್ವಕ್ತ್ರಸಂಸ್ತುತೇ ಪರಮೇಶ್ವರಿ ।\\
ರೂಪಂ ದೇಹಿ ಜಯಂ ದೇಹಿ ಯಶೋ ದೇಹಿ ದ್ವಿಷೋ ಜಹಿ ॥೧೫॥

ಕೃಷ್ಣೇನ ಸಂಸ್ತುತೇ ದೇವಿ ಶಶ್ವದ್ಭಕ್ತ್ಯಾ ತ್ವಮಂಬಿಕೇ ।\\
ರೂಪಂ ದೇಹಿ ಜಯಂ ದೇಹಿ ಯಶೋ ದೇಹಿ ದ್ವಿಷೋ ಜಹಿ ॥೧೬॥

ಹಿಮಾಚಲಸುತಾನಾಥಸಂಸ್ತುತೇ ಪರಮೇಶ್ವರಿ ।\\
ರೂಪಂ ದೇಹಿ ಜಯಂ ದೇಹಿ ಯಶೋ ದೇಹಿ ದ್ವಿಷೋ ಜಹಿ ॥೧೭॥

ಸುರಾಸುರಶಿರೋರತ್ನನಿಘೃಷ್ಟಚರಣೇಽಂಬಿಕೇ ।\\
ರೂಪಂ ದೇಹಿ ಜಯಂ ದೇಹಿ ಯಶೋ ದೇಹಿ ದ್ವಿಷೋ ಜಹಿ ॥೧೮॥

ಇಂದ್ರಾಣೀಪತಿಸದ್ಭಾವಪೂಜಿತೇ ಪರಮೇಶ್ವರಿ ।\\
ರೂಪಂ ದೇಹಿ ಜಯಂ ದೇಹಿ ಯಶೋ ದೇಹಿ ದ್ವಿಷೋ ಜಹಿ ॥೧೯॥

ದೇವಿ ಭಕ್ತಜನೋದ್ದಾಮದತ್ತಾನಂದೋದಯೇಂಬಿಕೇ ।\\
ರೂಪಂ ದೇಹಿ ಜಯಂ ದೇಹಿ ಯಶೋ ದೇಹಿ ದ್ವಿಷೋ ಜಹಿ ॥೨೦॥

ಪುತ್ರಾನ್ ದೇಹಿ ಧನಂ ದೇಹಿ ಸರ್ವಕಾಮಾಂಶ್ಚ ದೇಹಿ ಮೇ ।\\
ರೂಪಂ ದೇಹಿ ಜಯಂ ದೇಹಿ ಯಶೋ ದೇಹಿ ದ್ವಿಷೋ ಜಹಿ ॥೨೧॥

ಪತ್ನೀಂ ಮನೋರಮಾಂ ದೇಹಿ ಮನೋವೃತ್ತಾನು ಸಾರಿಣೀಂ।\\
ತಾರಿಣೀಂ ದುರ್ಗಸಂಸಾರಸಾಗರಸ್ಯ ಕುಲೋದ್ಭವಾಂ॥೨೨॥

ಇದಂ ಸ್ತೋತ್ರಂ ಪಠಿತ್ವಾ ತು ಮಹಾಸ್ತೋತ್ರಂ ಪಠೇನ್ನರಃ ।\\
ಸ ತು ಸಪ್ತಶತೀ ಸಂಖ್ಯಾವರಮಾಪ್ನೋತಿ ಸಂಪದಾಂ ॥೨೩॥
\authorline{॥ ಮಾರ್ಕಂಡೇಯಪುರಾಣೇ ಅರ್ಗಲಾ ಸ್ತೋತ್ರಂ ॥}
\section{ಕೀಲಕಮ್}
ಓಂ ಅಸ್ಯ ಶ್ರೀಕೀಲಕಮಂತ್ರಸ್ಯ ಶಿವಋಷಿಃ । ಅನುಷ್ಟುಪ್ ಛಂದಃ । ಶ್ರೀಮಹಾಸರಸ್ವತೀ ದೇವತಾ । ಶ್ರೀಜಗದಂಬಾಪ್ರೀತ್ಯರ್ಥಂ ಜಪೇ ವಿನಿಯೋಗಃ ।

ಓಂ ನಮಶ್ಚಂಡಿಕಾಯೈ ॥ ಮಾರ್ಕಂಡೇಯ ಉವಾಚ ॥\\
ವಿಶುದ್ಧಜ್ಞಾನದೇಹಾಯ ತ್ರಿವೇದೀದಿವ್ಯಚಕ್ಷುಷೇ ।\\
ಶ್ರೇಯಃಪ್ರಾಪ್ತಿನಿಮಿತ್ತಾಯ ನಮಃ ಸೋಮಾರ್ಧಧಾರಿಣೇ ॥೧॥

ಸರ್ವಮೇತದ್ವಿನಾ ಯಸ್ತು ಮಂತ್ರಾಣಾಮಪಿ ಕೀಲಕಂ ।\\
ಸೋಽಪಿ ಕ್ಷೇಮಮವಾಪ್ನೋತಿ ಸತತಂ ಜಾಪ್ಯತತ್ಪರಃ ॥೨॥

ಸಿದ್ಧ್ಯಂತ್ಯುಚ್ಚಾಟನಾದೀನಿ ವಸ್ತೂನಿ ಸಕಲಾನ್ಯಪಿ ।\\
ಏತೇನ ಸ್ತುವತಾಂ ನಿತ್ಯಂ ಸ್ತೋತ್ರಮಾತ್ರೇಣ ಸಿದ್ಧ್ಯತಿ ॥೩॥

ನ ಮಂತ್ರೋ ನೌಷಧಂ ತತ್ರ ನ ಕಿಂಚಿದಪಿ ವಿದ್ಯತೇ ।\\
ವಿನಾ ಜಾಪ್ಯೇನ ಸಿದ್ಧ್ಯೇತ ಸರ್ವಮುಚ್ಚಾಟನಾದಿಕಂ ॥೪॥

ಸಮಗ್ರಾಣ್ಯಪಿ ಸಿದ್ಧ್ಯಂತಿ ಲೋಕಶಂಕಾಮಿಮಾಂ ಹರಃ ।\\
ಕೃತ್ವಾ ನಿಮಂತ್ರಯಾಮಾಸ ಸರ್ವಮೇವಮಿದಂ ಶುಭಂ ॥೫॥

ಸ್ತೋತ್ರಂ ವೈ ಚಂಡಿಕಾಯಾಸ್ತು ತಚ್ಚ ಗುಹ್ಯಂ ಚಕಾರ ಸಃ ।\\
ಸಮಾಪ್ತಿರ್ನ ಚ ಪುಣ್ಯಸ್ಯ ತಾಂ ಯಥಾವನ್ನಿಯಂತ್ರಣಾಂ ॥೬॥

ಸೋಽಪಿ ಕ್ಷೇಮಮವಾಪ್ನೋತಿ ಸರ್ವಮೇವ ನ ಸಂಶಯಃ ।\\
ಕೃಷ್ಣಾಯಾಂ ವಾ ಚತುರ್ದಶ್ಯಾಮಷ್ಟಮ್ಯಾಂ ವಾ ಸಮಾಹಿತಃ ॥೭॥

ದದಾತಿ ಪ್ರತಿಗೃಹ್ಣಾತಿ ನಾನ್ಯಥೈಷಾ ಪ್ರಸೀದತಿ ।\\
ಇತ್ಥಂ ರೂಪೇಣ ಕೀಲೇನ ಮಹಾದೇವೇನ ಕೀಲಿತಂ ॥೮॥

ಯೋ ನಿಷ್ಕೀಲಾಂ ವಿಧಾಯೈನಾಂ ನಿತ್ಯಂ ಜಪತಿ ಸುಸ್ಫುಟಂ ।\\
ಸಸಿದ್ಧಃ ಸಗಣಃ ಸೋಽಪಿ ಗಂಧರ್ವೋ ಜಾಯತೇ ವನೇ ॥೯॥

ನ ಚೈವಾಪ್ಯಟತಸ್ತಸ್ಯ ಭಯಂ ಕ್ವಾಪಿ ಹಿ ಜಾಯತೇ ।\\
ನಾಪಮೃತ್ಯುವಶಂ ಯಾತಿ ಮೃತೋ ಮೋಕ್ಷಮವಾಪ್ನುಯಾತ್ ॥೧೦॥

ಜ್ಞಾತ್ವಾ ಪ್ರಾರಭ್ಯ ಕುರ್ವೀತ ಹ್ಯಕುರ್ವಾಣೋ ವಿನಶ್ಯತಿ ।\\
ತತೋ ಜ್ಞಾತ್ವೈವ ಸಂಪನ್ನಮಿದಂ ಪ್ರಾರಭ್ಯತೇ ಬುಧೈಃ ॥೧೧॥

ಸೌಭಾಗ್ಯಾದಿ ಚ ಯತ್ಕಿಂಚಿದ್ ದೃಶ್ಯತೇ ಲಲನಾಜನೇ ।\\
ತತ್ಸರ್ವಂ ತತ್ಪ್ರಸಾದೇನ ತೇನ ಜಾಪ್ಯಮಿದಂ ಶುಭಂ ॥೧೨॥

ಶನೈಸ್ತು ಜಪ್ಯಮಾನೇಽಸ್ಮಿನ್ ಸ್ತೋತ್ರೇ ಸಂಪತ್ತಿರುಚ್ಚಕೈಃ ।\\
ಭವತ್ಯೇವ ಸಮಗ್ರಾಪಿ ತತಃ ಪ್ರಾರಭ್ಯಮೇವ ತತ್ ॥೧೩॥

ಐಶ್ವರ್ಯಂ ಯತ್ಪ್ರಸಾದೇನ ಸೌಭಾಗ್ಯಾರೋಗ್ಯಸಂಪದಃ ।\\
ಶತ್ರುಹಾನಿಃ ಪರೋ ಮೋಕ್ಷಃ ಸ್ತೂಯತೇ ಸಾ ನ ಕಿಂ ಜನೈಃ ॥೧೪॥
\authorline{॥ಭಗವತ್ಯಾಃ ಕೀಲಕಸ್ತೋತ್ರಂ ॥}
\section{ಬ್ರಹ್ಮಸ್ತುತಿಃ}
ಬ್ರಹ್ಮೋವಾಚ ॥\\
ತ್ವಂ ಸ್ವಾಹಾ ತ್ವಂ ಸ್ವಧಾ ತ್ವಂ ಹಿ ವಷಟ್ಕಾರ ಸ್ವರಾತ್ಮಿಕಾ ।\\
ಸುಧಾ ತ್ವಮಕ್ಷರೇ ನಿತ್ಯೇ ತ್ರಿಧಾ ಮಾತ್ರಾತ್ಮಿಕಾ ಸ್ಥಿತಾ ॥೨॥

ಅರ್ಧಮಾತ್ರಾ ಸ್ಥಿತಾ ನಿತ್ಯಾ ಯಾನುಚ್ಚಾರ್ಯಾ ವಿಶೇಷತಃ ।\\
ತ್ವಮೇವ ಸಂಧ್ಯಾ ಸಾವಿತ್ರೀ ತ್ವಂ ದೇವಿ ಜನನೀ ಪರಾ ॥೩॥

ತ್ವಯೈತದ್ಧಾರ್ಯತೇ ವಿಶ್ವಂ ತ್ವಯೈತತ್ಸೃಜ್ಯತೇ ಜಗತ್ ।\\
ತ್ವಯೈತತ್ಪಾಲ್ಯತೇ ದೇವಿ ತ್ವಮತ್ಸ್ಯಂತೇ ಚ ಸರ್ವದಾ ॥೪॥

ವಿಸೃಷ್ಟೌ ಸೃಷ್ಟಿರೂಪಾ ತ್ವಂ ಸ್ಥಿತಿರೂಪಾ ಚ ಪಾಲನೇ ।\\
ತಥಾ ಸಂಹೃತಿರೂಪಾಂತೇ ಜಗತೋಽಸ್ಯ ಜಗನ್ಮಯೇ ॥೫॥

ಮಹಾವಿದ್ಯಾ ಮಹಾಮಾಯಾ ಮಹಾಮೇಧಾ ಮಹಾಸ್ಮೃತಿಃ ।\\
ಮಹಾಮೋಹಾ ಚ ಭವತೀ ಮಹಾದೇವೀ ಮಹೇಶ್ವರೀ ॥೬॥

ಪ್ರಕೃತಿಸ್ತ್ವಂ ಚ ಸರ್ವಸ್ಯ ಗುಣತ್ರಯ ವಿಭಾವಿನೀ ।\\
ಕಾಲರಾತ್ರಿರ್ಮಹಾರಾತ್ರಿರ್ಮೋಹರಾತ್ರಿಶ್ಚ ದಾರುಣಾ ॥೭॥

ತ್ವಂ ಶ್ರೀಸ್ತ್ವಮೀಶ್ವರೀ ತ್ವಂ ಹ್ರೀಸ್ತ್ವಂ ಬುದ್ಧಿರ್ಬೋಧಲಕ್ಷಣಾ ।\\
ಲಜ್ಜಾ ಪುಷ್ಟಿಸ್ತಥಾ ತುಷ್ಟಿಸ್ತ್ವಂ ಶಾಂತಿಃ ಕ್ಷಾಂತಿರೇವ ಚ ॥೮॥

ಖಡ್ಗಿನೀ ಶೂಲಿನೀ ಘೋರಾ ಗದಿನೀ ಚಕ್ರಿಣೀ ತಥಾ ।\\
ಶಂಖಿನೀ ಚಾಪಿನೀ ಬಾಣಭುಶುಂಡೀಪರಿಘಾಯುಧಾ ॥೯॥

ಸೌಮ್ಯಾ ಸೌಮ್ಯತರಾಶೇಷ ಸೌಮ್ಯೇಭ್ಯಸ್ತ್ವತಿ ಸುಂದರೀ ।\\
ಪರಾಪರಾಣಾಂ ಪರಮಾ ತ್ವಮೇವ ಪರಮೇಶ್ವರೀ ॥೧೦॥

ಯಚ್ಚ ಕಿಂಚಿತ್ ಕ್ವಚಿದ್ವಸ್ತು ಸದಸದ್ವಾಖಿಲಾತ್ಮಿಕೇ ।\\
ತಸ್ಯ ಸರ್ವಸ್ಯ ಯಾ ಶಕ್ತಿಃ ಸಾ ತ್ವಂ ಕಿಂ ಸ್ತೂಯಸೇ ಮಯಾ ॥೧೧॥

ಯಯಾ ತ್ವಯಾ ಜಗತ್‌ಸ್ರಷ್ಟಾ ಜಗತ್ಪಾತ್ಯತ್ತಿ ಯೋ ಜಗತ್ ।\\
ಸೋಽಪಿ ನಿದ್ರಾವಶಂ ನೀತಃ ಕಸ್ತ್ವಾಂ ಸ್ತೋತುಮಿಹೇಶ್ವರಃ ॥೧೨॥

ವಿಷ್ಣುಃ ಶರೀರಗ್ರಹಣಮಹಮೀಶಾನ ಏವ ಚ ।\\
ಕಾರಿತಾಸ್ತೇ ಯತೋಽತಸ್ತ್ವಾಂ ಕಃ ಸ್ತೋತುಂ ಶಕ್ತಿಮಾನ್ಭವೇತ್ ॥೧೩॥

ಸಾ ತ್ವಮಿತ್ಥಂ ಪ್ರಭಾವೈಃ ಸ್ವೈರುದಾರೈರ್ದೇವಿ ಸಂಸ್ತುತಾ ।\\
ಮೋಹಯೈತೌ ದುರಾಧರ್ಷಾವಸುರೌ ಮಧುಕೈಟಭೌ ॥೧೪॥

ಪ್ರಬೋಧಂ ಚ ಜಗತ್ಸ್ವಾಮೀ ನೀಯತಾಮಚ್ಯುತೋ ಲಘು ।\\
ಬೋಧಶ್ಚ ಕ್ರಿಯತಾಮಸ್ಯ ಹಂತುಮೇತೌ ಮಹಾಸುರೌ ॥೧೫॥


\section{ಶಕ್ರಾದಿಸ್ತುತಿಃ}
          ಋಷಿರುವಾಚ ॥೧॥\\
     ಶಕ್ರಾದಯಃ ಸುರಗಣಾ ನಿಹತೇಽತಿವೀರ್ಯೇ\\
ತಸ್ಮಿನ್ ದುರಾತ್ಮನಿ ಸುರಾರಿಬಲೇ ಚ ದೇವ್ಯಾ ।\\
     ತಾಂ ತುಷ್ಟುವುಃ ಪ್ರಣತಿನಮ್ರಶಿರೋಧರಾಂಸಾ\\
ವಾಗ್ಭಿಃ ಪ್ರಹರ್ಷಪುಲಕೋದ್ಗಮಚಾರುದೇಹಾಃ ॥೨॥

     ದೇವ್ಯಾ ಯಯಾ ತತಮಿದಂ ಜಗದಾತ್ಮಶಕ್ತ್ಯಾ\\
ನಿಶ್ಶೇಷದೇವಗಣಶಕ್ತಿಸಮೂಹಮೂರ್ತ್ಯಾ ।\\
     ತಾಮಂಬಿಕಾಮಖಿಲದೇವಮಹರ್ಷಿಪೂಜ್ಯಾಂ\\
ಭಕ್ತ್ಯಾ ನತಾಃ ಸ್ಮ ವಿದಧಾತು ಶುಭಾನಿ ಸಾ ನಃ ॥೩॥

     ಯಸ್ಯಾಃ ಪ್ರಭಾವಮತುಲಂ ಭಗವಾನನಂತೋ\\
ಬ್ರಹ್ಮಾ ಹರಶ್ಚ ನ ಹಿ ವಕ್ತುಮಲಂ ಬಲಂ ಚ ।\\
     ಸಾ ಚಂಡಿಕಾಖಿಲಜಗತ್ಪರಿಪಾಲನಾಯ\\
ನಾಶಾಯ ಚಾಶುಭಭಯಸ್ಯ ಮತಿಂ ಕರೋತು ॥೪॥

     ಯಾ ಶ್ರೀಃ ಸ್ವಯಂ ಸುಕೃತಿನಾಂ ಭವನೇಷ್ವಲಕ್ಷ್ಮೀಃ\\
ಪಾಪಾತ್ಮನಾಂ ಕೃತಧಿಯಾಂ ಹೃದಯೇಷು ಬುದ್ಧಿಃ ।\\
     ಶ್ರದ್ಧಾ ಸತಾಂ ಕುಲಜನಪ್ರಭವಸ್ಯ ಲಜ್ಜಾ\\
ತಾಂ ತ್ವಾಂ ನತಾಃ ಸ್ಮ ಪರಿಪಾಲಯ ದೇವಿ ವಿಶ್ವಂ ॥೫॥

     ಕಿಂ ವರ್ಣಯಾಮ ತವ ರೂಪಮಚಿಂತ್ಯಮೇತತ್\\
ಕಿಂ ಚಾತಿವೀರ್ಯಮಸುರ ಕ್ಷಯಕಾರಿ ಭೂರಿ ।\\
     ಕಿಂ ಚಾಹವೇಷು ಚರಿತಾನಿ ತವಾದ್ಭುತಾನಿ\\
ಸರ್ವೇಷು ದೇವ್ಯಸುರ ದೇವ ಗಣಾದಿಕೇಷು ॥೬॥

     ಹೇತುಃ ಸಮಸ್ತ ಜಗತಾಂ ತ್ರಿಗುಣಾಪಿ ದೋಷೈ-\\
ರ್ನ ಜ್ಞಾಯಸೇ ಹರಿಹರಾದಿಭಿರಪ್ಯಪಾರಾ ।\\
     ಸರ್ವಾಶ್ರಯಾಖಿಲಮಿದಂ ಜಗದಂಶಭೂತ-\\
ಮವ್ಯಾಕೃತಾ ಹಿ ಪರಮಾ ಪ್ರಕೃತಿಸ್ತ್ವಮಾದ್ಯಾ ॥೭॥

     ಯಸ್ಯಾಃ ಸಮಸ್ತ ಸುರತಾ ಸಮುದೀರಣೇನ\\
ತೃಪ್ತಿಂ ಪ್ರಯಾತಿ ಸಕಲೇಷು ಮಖೇಷು ದೇವಿ ।\\
     ಸ್ವಾಹಾಸಿ ವೈ ಪಿತೃಗಣಸ್ಯ ಚ ತೃಪ್ತಿಹೇತು-\\
ರುಚ್ಚಾರ್ಯಸೇ ತ್ವಮತ ಏವ ಜನೈಃ ಸ್ವಧಾ ಚ ॥೮॥

     ಯಾ ಮುಕ್ತಿಹೇತುರವಿಚಿಂತ್ಯಮಹಾವ್ರತಾ ತ್ವಂ\\
ಅಭ್ಯಸ್ಯಸೇ ಸುನಿಯತೇಂದ್ರಿಯ ತತ್ತ್ವಸಾರೈಃ ।\\
     ಮೋಕ್ಷಾರ್ಥಿಭಿರ್ಮುನಿಭಿರಸ್ತಸಮಸ್ತದೋಷೈ-\\
ರ್ವಿದ್ಯಾಸಿ ಸಾ ಭಗವತೀ ಪರಮಾ ಹಿ ದೇವಿ ॥೯॥

     ಶಬ್ದಾತ್ಮಿಕಾ ಸುವಿಮಲರ್ಗ್ಯಜುಷಾಂ ನಿಧಾನ-\\
ಮುದ್ಗೀಥರಮ್ಯಪದಪಾಠವತಾಂ ಚ ಸಾಮ್ನಾಂ ।\\
     ದೇವಿ ತ್ರಯೀ ಭಗವತೀ ಭವಭಾವನಾಯ\\
ವಾರ್ತಾಸಿ ಸರ್ವಜಗತಾಂ ಪರಮಾರ್ತಿ ಹಂತ್ರೀ ॥೧೦॥

     ಮೇಧಾಸಿ ದೇವಿ ವಿದಿತಾಖಿಲಶಾಸ್ತ್ರಸಾರಾ\\
ದುರ್ಗಾಸಿ ದುರ್ಗಭವಸಾಗರ ನೌರಸಂಗಾ ।\\
     ಶ್ರೀಃ ಕೈಟಭಾರಿ ಹೃದಯೈಕ ಕೃತಾಧಿವಾಸಾ\\
ಗೌರೀ ತ್ವಮೇವ ಶಶಿಮೌಲಿಕೃತಪ್ರತಿಷ್ಠಾ ॥೧೧॥

     ಈಷತ್ಸಹಾಸಮಮಲಂ ಪರಿಪೂರ್ಣಚಂದ್ರ-\\
ಬಿಂಬಾನುಕಾರಿ ಕನಕೋತ್ತಮ ಕಾಂತಿಕಾಂತಂ ।\\
     ಅತ್ಯದ್ಭುತಂ ಪ್ರಹೃತಮಾತ್ತರುಷಾ ತಥಾಪಿ\\
ವಕ್ತ್ರಂ ವಿಲೋಕ್ಯ ಸಹಸಾ ಮಹಿಷಾಸುರೇಣ ॥೧೨॥

     ದೃಷ್ಟ್ವಾ ತು ದೇವಿ ಕುಪಿತಂ ಭ್ರುಕುಟೀಕರಾಲ-\\
ಮುದ್ಯಚ್ಛಶಾಂಕ ಸದೃಶಚ್ಛವಿ ಯನ್ನ ಸದ್ಯಃ ।\\
     ಪ್ರಾಣಾನ್ಮುಮೋಚ ಮಹಿಷಸ್ತದತೀವ ಚಿತ್ರಂ\\
ಕೈರ್ಜೀವ್ಯತೇ ಹಿ ಕುಪಿತಾಂತಕ ದರ್ಶನೇನ ॥೧೩॥

     ದೇವಿ ಪ್ರಸೀದ ಪರಮಾ ಭವತೀ ಭವಾಯ\\
ಸದ್ಯೋ ವಿನಾಶಯಸಿ ಕೋಪವತೀ ಕುಲಾನಿ ।\\
     ವಿಜ್ಞಾತಮೇತದಧುನೈವ ಯದಸ್ತಮೇತ-\\
ನ್ನೀತಂ ಬಲಂ ಸುವಿಪುಲಂ ಮಹಿಷಾಸುರಸ್ಯ ॥೧೪॥

     ತೇ ಸಮ್ಮತಾ ಜನಪದೇಷು ಧನಾನಿ ತೇಷಾಂ\\
ತೇಷಾಂ ಯಶಾಂಸಿ ನ ಚ ಸೀದತಿ ಬಂಧುವರ್ಗಃ ।\\
     ಧನ್ಯಾಸ್ತ ಏವ ನಿಭೃತಾತ್ಮಜ ಭೃತ್ಯ ದಾರಾ\\
ಯೇಷಾಂ ಸದಾಭ್ಯುದಯದಾ ಭವತೀ ಪ್ರಸನ್ನಾ ॥೧೫॥

     ಧರ್ಮ್ಯಾಣಿ ದೇವಿ ಸಕಲಾನಿ ಸದೈವ ಕರ್ಮಾ-\\
ಣ್ಯತ್ಯಾದೃತಃ ಪ್ರತಿದಿನಂ ಸುಕೃತೀ ಕರೋತಿ ।\\
     ಸ್ವರ್ಗಂ ಪ್ರಯಾತಿ ಚ ತತೋ ಭವತೀಪ್ರಸಾದಾ-\\
ಲ್ಲೋಕತ್ರಯೇಽಪಿ ಫಲದಾ ನನು ದೇವಿ ತೇನ ॥೧೬॥

     ದುರ್ಗೇ ಸ್ಮೃತಾ ಹರಸಿ ಭೀತಿಮಶೇಷಜಂತೋಃ\\
ಸ್ವಸ್ಥೈಃ ಸ್ಮೃತಾ ಮತಿಮತೀವ ಶುಭಾಂ ದದಾಸಿ ।\\
     ದಾರಿದ್ರ್ಯ ದುಃಖ ಭಯಹಾರಿಣಿ ಕಾ ತ್ವದನ್ಯಾ\\
ಸರ್ವೋಪಕಾರ ಕರಣಾಯ ಸದಾಽಽರ್ದ್ರಚಿತ್ತಾ ॥೧೭॥

     ಏಭಿರ್ಹತೈರ್ಜಗದುಪೈತಿ ಸುಖಂ ತಥೈತೇ\\
ಕುರ್ವಂತು ನಾಮ ನರಕಾಯ ಚಿರಾಯ ಪಾಪಂ ।\\
     ಸಂಗ್ರಾಮಮೃತ್ಯುಮಧಿಗಮ್ಯ ದಿವಂ ಪ್ರಯಾಂತು\\
ಮತ್ವೇತಿ ನೂನಮಹಿತಾನ್ವಿನಿಹಂಸಿ ದೇವಿ ॥೧೮॥

     ದೃಷ್ಟ್ವೈವ ಕಿಂ ನ ಭವತೀ ಪ್ರಕರೋತಿ ಭಸ್ಮ\\
ಸರ್ವಾಸುರಾನರಿಷು ಯತ್ಪ್ರಹಿಣೋಷಿ ಶಸ್ತ್ರಂ ।\\
     ಲೋಕಾನ್ಪ್ರಯಾಂತು ರಿಪವೋಽಪಿ ಹಿ ಶಸ್ತ್ರಪೂತಾ\\
ಇತ್ಥಂ ಮತಿರ್ಭವತಿ ತೇಷ್ವಹಿತೇಷು ಸಾಧ್ವೀ ॥೧೯॥

     ಖಡ್ಗಪ್ರಭಾ ನಿಕರ ವಿಸ್ಫುರಣೈಸ್ತಥೋಗ್ರೈಃ\\
ಶೂಲಾಗ್ರಕಾಂತಿನಿವಹೇನ ದೃಶೋಽಸುರಾಣಾಂ ।\\
     ಯನ್ನಾಗತಾ ವಿಲಯಮಂಶುಮದಿಂದುಖಂಡ-\\
ಯೋಗ್ಯಾನನಂ ತವ ವಿಲೋಕಯತಾಂ ತದೇತತ್ ॥೨೦॥

     ದುರ್ವೃತ್ತವೃತ್ತಶಮನಂ ತವ ದೇವಿ ಶೀಲಂ\\
ರೂಪಂ ತಥೈತದವಿಚಿಂತ್ಯಮತುಲ್ಯಮನ್ಯೈಃ ।\\
     ವೀರ್ಯಂ ಚ ಹಂತೃಹೃತದೇವಪರಾಕ್ರಮಾಣಾಂ\\
ವೈರಿಷ್ವಪಿ ಪ್ರಕಟಿತೈವ ದಯಾ ತ್ವಯೇತ್ಥಂ ॥೨೧॥

     ಕೇನೋಪಮಾ ಭವತು ತೇಽಸ್ಯ ಪರಾಕ್ರಮಸ್ಯ\\
ರೂಪಂ ಚ ಶತ್ರುಭಯಕಾರ್ಯತಿಹಾರಿ ಕುತ್ರ ।\\
     ಚಿತ್ತೇ ಕೃಪಾ ಸಮರನಿಷ್ಠುರತಾ ಚ ದೃಷ್ಟಾ\\
ತ್ವಯ್ಯೇವ ದೇವಿ ವರದೇ ಭುವನತ್ರಯೇಽಪಿ ॥೨೨॥

     ತ್ರೈಲೋಕ್ಯಮೇತದಖಿಲಂ ರಿಪುನಾಶನೇನ\\
ತ್ರಾತಂ ತ್ವಯಾ ಸಮರಮೂರ್ಧನಿ ತೇಽಪಿ ಹತ್ವಾ ।\\
     ನೀತಾ ದಿವಂ ರಿಪುಗಣಾ ಭಯಮಪ್ಯಪಾಸ್ತ-\\
ಮಸ್ಮಾಕಮುನ್ಮದಸುರಾರಿಭವಂ ನಮಸ್ತೇ ॥೨೩॥

ಶೂಲೇನ ಪಾಹಿ ನೋ ದೇವಿ ಪಾಹಿ ಖಡ್ಗೇನ ಚಾಂಬಿಕೇ ।\\
ಘಂಟಾಸ್ವನೇನ ನಃ ಪಾಹಿ ಚಾಪಜ್ಯಾನಿಃಸ್ವನೇನ ಚ ॥೨೪॥

ಪ್ರಾಚ್ಯಾಂ ರಕ್ಷ ಪ್ರತೀಚ್ಯಾಂ ಚ ಚಂಡಿಕೇ ರಕ್ಷ ದಕ್ಷಿಣೇ ।\\
ಭ್ರಾಮಣೇನಾತ್ಮಶೂಲಸ್ಯ ಉತ್ತರಸ್ಯಾಂ ತಥೇಶ್ವರಿ ॥೨೫॥

ಸೌಮ್ಯಾನಿ ಯಾನಿ ರೂಪಾಣಿ ತ್ರೈಲೋಕ್ಯೇ ವಿಚರಂತಿ ತೇ ।\\
ಯಾನಿ ಚಾತ್ಯಂತ ಘೋರಾಣಿ ತೈ ರಕ್ಷಾಸ್ಮಾಂಸ್ತಥಾ ಭುವಂ ॥೨೬॥

ಖಡ್ಗಶೂಲಗದಾದೀನಿ ಯಾನಿ ಚಾಸ್ತ್ರಾಣಿ ತೇಽಮ್ಬಿಕೇ ।\\
ಕರಪಲ್ಲವಸಂಗೀನಿ ತೈರಸ್ಮಾನ್ರಕ್ಷ ಸರ್ವತಃ ॥೨೭॥

 ಋಷಿರುವಾಚ ॥೨೮॥\\
ಏವಂ ಸ್ತುತಾ ಸುರೈರ್ದಿವ್ಯೈಃ ಕುಸುಮೈರ್ನಂದನೋದ್ಭವೈಃ ।\\
ಅರ್ಚಿತಾ ಜಗತಾಂ ಧಾತ್ರೀ ತಥಾ ಗಂಧಾನುಲೇಪನೈಃ ॥೨೯॥

ಭಕ್ತ್ಯಾ ಸಮಸ್ತೈಸ್ತ್ರಿದಶೈರ್ದಿವ್ಯೈರ್ಧೂಪೈಃ ಸುಧೂಪಿತಾ ।\\
ಪ್ರಾಹ ಪ್ರಸಾದಸುಮುಖೀ ಸಮಸ್ತಾನ್ ಪ್ರಣತಾನ್ ಸುರಾನ್ ॥೩೦॥

ದೇವ್ಯುವಾಚ ॥೩೧॥\\
ವ್ರಿಯತಾಂ ತ್ರಿದಶಾಃ ಸರ್ವೇ ಯದಸ್ಮತ್ತೋಽಭಿವಾಂಛಿತಂ ॥೩೨॥

ದೇವಾ ಊಚುಃ ॥೩೩॥\\
ಭಗವತ್ಯಾ ಕೃತಂ ಸರ್ವಂ ನ ಕಿಂಚಿದವಶಿಷ್ಯತೇ ।\\
ಯದಯಂ ನಿಹತಃ ಶತ್ರುರಸ್ಮಾಕಂ ಮಹಿಷಾಸುರಃ ॥೩೪॥

ಯದಿ ಚಾಪಿ ವರೋ ದೇಯಸ್ತ್ವಯಾಽಸ್ಮಾಕಂ ಮಹೇಶ್ವರಿ ।\\
ಸಂಸ್ಮೃತಾಽಸಂಸ್ಮೃತಾ ತ್ವಂ ನೋ ಹಿಂಸೇಥಾಃ ಪರಮಾಪದಃ ॥೩೫॥

ಯಶ್ಚ ಮರ್ತ್ಯಃ ಸ್ತವೈರೇಭಿಸ್ತ್ವಾಂ ಸ್ತೋಷ್ಯತ್ಯಮಲಾನನೇ ॥೩೬॥

ತಸ್ಯ ವಿತ್ತರ್ದ್ಧಿವಿಭವೈರ್ಧನದಾರಾದಿ ಸಂಪದಾಂ ।\\
ವೃದ್ಧಯೇಽಸ್ಮತ್ಪ್ರಸನ್ನಾ ತ್ವಂ ಭವೇಥಾಃ ಸರ್ವದಾಂಬಿಕೇ ॥೩೭॥


\section{ನಾರಾಯಣೀಸ್ತುತಿಃ}
ಓಂ ಋಷಿರುವಾಚ ॥೧॥\\
ದೇವ್ಯಾ ಹತೇ ತತ್ರ ಮಹಾಸುರೇಂದ್ರೇ\\
        ಸೇಂದ್ರಾಃ ಸುರಾ ವಹ್ನಿಪುರೋಗಮಾಸ್ತಾಂ ।\\
ಕಾತ್ಯಾಯನೀಂ ತುಷ್ಟುವುರಿಷ್ಟಲಾಭಾದ್\\
      ವಿಕಾಶಿವಕ್ತ್ರಾಬ್ಜವಿಕಾಶಿತಾಶಾಃ ॥೨॥

ದೇವಿ ಪ್ರಪನ್ನಾರ್ತಿಹರೇ ಪ್ರಸೀದ\\
        ಪ್ರಸೀದ ಮಾತರ್ಜಗತೋಽಖಿಲಸ್ಯ ।\\
ಪ್ರಸೀದ ವಿಶ್ವೇಶ್ವರಿ ಪಾಹಿ ವಿಶ್ವಂ\\
        ತ್ವಮೀಶ್ವರೀ ದೇವಿ ಚರಾಚರಸ್ಯ ॥೩॥

ಆಧಾರಭೂತಾ ಜಗತಸ್ತ್ವಮೇಕಾ\\
        ಮಹೀಸ್ವರೂಪೇಣ ಯತಃ ಸ್ಥಿತಾಸಿ ।\\
ಅಪಾಂ ಸ್ವರೂಪಸ್ಥಿತಯಾ ತ್ವಯೈತ-\\
      ದಾಪ್ಯಾಯತೇ ಕೃತ್ಸ್ನಮಲಂಘ್ಯವೀರ್ಯೇ ॥೪॥

ತ್ವಂ ವೈಷ್ಣವೀಶಕ್ತಿರನಂತವೀರ್ಯಾ\\
      ವಿಶ್ವಸ್ಯ ಬೀಜಂ ಪರಮಾಸಿ ಮಾಯಾ ।\\
ಸಮ್ಮೋಹಿತಂ ದೇವಿ ಸಮಸ್ತಮೇತತ್\\
      ತ್ವಂ ವೈ ಪ್ರಸನ್ನಾ ಭುವಿ ಮುಕ್ತಿಹೇತುಃ ॥೫॥

ವಿದ್ಯಾಃ ಸಮಸ್ತಾಸ್ತವ ದೇವಿ ಭೇದಾಃ\\
        ಸ್ತ್ರಿಯಃ ಸಮಸ್ತಾಃ ಸಕಲಾ ಜಗತ್ಸು ।\\
ತ್ವಯೈಕಯಾ ಪೂರಿತಮಂಬಯೈತತ್\\
        ಕಾ ತೇ ಸ್ತುತಿಃ ಸ್ತವ್ಯಪರಾಪರೋಕ್ತಿಃ ॥೬॥

ಸರ್ವಭೂತಾ ಯದಾ ದೇವೀ ಭುಕ್ತಿಮುಕ್ತಿಪ್ರದಾಯಿನೀ ।\\
ತ್ವಂ ಸ್ತುತಾ ಸ್ತುತಯೇ ಕಾ ವಾ ಭವಂತು ಪರಮೋಕ್ತಯಃ ॥೭॥

ಸರ್ವಸ್ಯ ಬುದ್ಧಿರೂಪೇಣ ಜನಸ್ಯ ಹೃದಿ ಸಂಸ್ಥಿತೇ ।\\
ಸ್ವರ್ಗಾಪವರ್ಗದೇ ದೇವಿ ನಾರಾಯಣಿ ನಮೋಽಸ್ತು ತೇ ॥೮॥

ಕಲಾಕಾಷ್ಠಾದಿರೂಪೇಣ ಪರಿಣಾಮಪ್ರದಾಯಿನಿ ।\\
ವಿಶ್ವಸ್ಯೋಪರತೌ ಶಕ್ತೇ ನಾರಾಯಣಿ ನಮೋಽಸ್ತು ತೇ ॥೯॥

ಸರ್ವಮಂಗಲಮಾಂಗಲ್ಯೇ ಶಿವೇ ಸರ್ವಾರ್ಥಸಾಧಿಕೇ ।\\
ಶರಣ್ಯೇ ತ್ರ್ಯಂಬಕೇ ಗೌರಿ ನಾರಾಯಣಿ ನಮೋಽಸ್ತು ತೇ ॥೧೦॥

ಸೃಷ್ಟಿಸ್ಥಿತಿವಿನಾಶಾನಾಂ ಶಕ್ತಿಭೂತೇ ಸನಾತನಿ ।\\
ಗುಣಾಶ್ರಯೇ ಗುಣಮಯೇ ನಾರಾಯಣಿ ನಮೋಽಸ್ತು ತೇ ॥೧೧॥

ಶರಣಾಗತದೀನಾರ್ತಪರಿತ್ರಾಣಪರಾಯಣೇ ।\\
ಸರ್ವಸ್ಯಾರ್ತಿಹರೇ ದೇವಿ ನಾರಾಯಣಿ ನಮೋಽಸ್ತು ತೇ ॥೧೨॥

ಹಂಸಯುಕ್ತವಿಮಾನಸ್ಥೇ ಬ್ರಹ್ಮಾಣೀರೂಪಧಾರಿಣಿ ।\\
ಕೌಶಾಂಭಃಕ್ಷರಿಕೇ ದೇವಿ ನಾರಾಯಣಿ ನಮೋಽಸ್ತು ತೇ ॥೧೩॥

ತ್ರಿಶೂಲಚಂದ್ರಾಹಿಧರೇ ಮಹಾವೃಷಭವಾಹಿನಿ ।\\
ಮಾಹೇಶ್ವರೀಸ್ವರೂಪೇಣ ನಾರಾಯಣಿ ನಮೋಽಸ್ತುತೇ ॥೧೪॥

ಮಯೂರಕುಕ್ಕುಟವೃತೇ ಮಹಾಶಕ್ತಿಧರೇಽನಘೇ ।\\
ಕೌಮಾರೀರೂಪಸಂಸ್ಥಾನೇ ನಾರಾಯಣಿ ನಮೋಽಸ್ತು ತೇ ॥೧೫॥

ಶಂಖಚಕ್ರಗದಾಶಾರ್ಙ್ಗಗೃಹೀತಪರಮಾಯುಧೇ ।\\
ಪ್ರಸೀದ ವೈಷ್ಣವೀರೂಪೇ ನಾರಾಯಣಿ ನಮೋಽಸ್ತು ತೇ ॥೧೬॥

ಗೃಹೀತೋಗ್ರಮಹಾಚಕ್ರೇ ದಂಷ್ಟ್ರೋದ್ಧೃತವಸುಂಧರೇ ।\\
ವರಾಹರೂಪಿಣಿ ಶಿವೇ ನಾರಾಯಣಿ ನಮೋಽಸ್ತು ತೇ ॥೧೭॥

ನೃಸಿಂಹರೂಪೇಣೋಗ್ರೇಣ ಹಂತುಂ ದೈತ್ಯಾನ್ ಕೃತೋದ್ಯಮೇ ।\\
ತ್ರೈಲೋಕ್ಯತ್ರಾಣಸಹಿತೇ ನಾರಾಯಣಿ ನಮೋಽಸ್ತು ತೇ ॥೧೮॥

ಕಿರೀಟಿನಿ ಮಹಾವಜ್ರೇ ಸಹಸ್ರನಯನೋಜ್ಜ್ವಲೇ ।\\
ವೃತ್ರಪ್ರಾಣಹರೇ ಚೈಂದ್ರಿ ನಾರಾಯಣಿ ನಮೋಽಸ್ತು ತೇ ॥೧೯॥

ಶಿವದೂತೀ ಸ್ವರೂಪೇಣ ಹತದೈತ್ಯ ಮಹಾಬಲೇ ।\\
ಘೋರರೂಪೇ ಮಹಾರಾವೇ ನಾರಾಯಣಿ ನಮೋಽಸ್ತು ತೇ ॥೨೦॥

ದಂಷ್ಟ್ರಾ ಕರಾಲವದನೇ ಶಿರೋಮಾಲಾವಿಭೂಷಣೇ ।\\
ಚಾಮುಂಡೇ ಮುಂಡಮಥನೇ ನಾರಾಯಣಿ ನಮೋಽಸ್ತು ತೇ ॥೨೧॥

ಲಕ್ಷ್ಮಿ ಲಜ್ಜೇ ಮಹಾವಿದ್ಯೇ ಶ್ರದ್ಧೇ ಪುಷ್ಟಿ ಸ್ವಧೇ ಧ್ರುವೇ ।\\
ಮಹಾರಾತ್ರಿ ಮಹಾಮಾಯೇ ನಾರಾಯಣಿ ನಮೋಽಸ್ತು ತೇ ॥೨೨॥

ಮೇಧೇ ಸರಸ್ವತಿ ವರೇ ಭೂತಿ ಬಾಭ್ರವಿ ತಾಮಸಿ ।\\
ನಿಯತೇ ತ್ವಂ ಪ್ರಸೀದೇಶೇ ನಾರಾಯಣಿ ನಮೋಽಸ್ತುತೇ ॥೨೩॥

ಸರ್ವಸ್ವರೂಪೇ ಸರ್ವೇಶೇ ಸರ್ವಶಕ್ತಿಸಮನ್ವಿತೇ ।\\
ಭಯೇಭ್ಯಸ್ತ್ರಾಹಿ ನೋ ದೇವಿ ದುರ್ಗೇ ದೇವಿ ನಮೋಽಸ್ತು ತೇ ॥೨೪॥

ಏತತ್ತೇ ವದನಂ ಸೌಮ್ಯಂ ಲೋಚನತ್ರಯಭೂಷಿತಂ ।\\
ಪಾತು ನಃ ಸರ್ವಭೂತೇಭ್ಯಃ ಕಾತ್ಯಾಯನಿ ನಮೋಽಸ್ತು ತೇ ॥೨೫॥

ಜ್ವಾಲಾ ಕರಾಲಮತ್ಯುಗ್ರಮಶೇಷಾಸುರ ಸೂದನಂ ।\\
ತ್ರಿಶೂಲಂ ಪಾತು ನೋ ಭೀತೇರ್ಭದ್ರಕಾಲಿ ನಮೋಽಸ್ತು ತೇ ॥೨೬॥

ಹಿನಸ್ತಿ ದೈತ್ಯತೇಜಾಂಸಿ ಸ್ವನೇನಾಪೂರ್ಯ ಯಾ ಜಗತ್ ।\\
ಸಾ ಘಂಟಾ ಪಾತು ನೋ ದೇವಿ ಪಾಪೇಭ್ಯೋ ನಃ ಸುತಾನಿವ ॥೨೭॥

ಅಸುರಾಸೃಗ್ವಸಾ ಪಂಕ ಚರ್ಚಿತಸ್ತೇ ಕರೋಜ್ಜ್ವಲಃ ।\\
ಶುಭಾಯ ಖಡ್ಗೋ ಭವತು ಚಂಡಿಕೇ ತ್ವಾಂ ನತಾ ವಯಂ ॥೨೮॥

ರೋಗಾನಶೇಷಾನಪಹಂಸಿ ತುಷ್ಟಾ\\
        ರುಷ್ಟಾ ತು ಕಾಮಾನ್ ಸಕಲಾನಭೀಷ್ಟಾನ್ ।\\
ತ್ವಾಮಾಶ್ರಿತಾನಾಂ ನ ವಿಪನ್ನರಾಣಾಂ\\
        ತ್ವಾಮಾಶ್ರಿತಾ ಹ್ಯಾಶ್ರಯತಾಂ ಪ್ರಯಾಂತಿ ॥೨೯॥

ಏತತ್ಕೃತಂ ಯತ್ಕದನಂ ತ್ವಯಾದ್ಯ\\
        ಧರ್ಮದ್ವಿಷಾಂ ದೇವಿ ಮಹಾಸುರಾಣಾಂ ।\\
ರೂಪೈರನೇಕೈರ್ಬಹುಧಾತ್ಮಮೂರ್ತಿಂ\\
        ಕೃತ್ವಾಂಬಿಕೇ ತತ್ಪ್ರಕರೋತಿ ಕಾನ್ಯಾ ॥೩೦॥

ವಿದ್ಯಾಸು ಶಾಸ್ತ್ರೇಷು ವಿವೇಕದೀಪೇ-\\
      ಷ್ವಾದ್ಯೇಷು ವಾಕ್ಯೇಷು ಚ ಕಾ ತ್ವದನ್ಯಾ ।\\
ಮಮತ್ವಗರ್ತೇಽತಿಮಹಾಂಧಕಾರೇ\\
      ವಿಭ್ರಾಮಯಸ್ಯೇತದತೀವ ವಿಶ್ವಂ ॥೩೧॥

ರಕ್ಷಾಂಸಿ ಯತ್ರೋಗ್ರವಿಷಾಶ್ಚ ನಾಗಾ\\
        ಯತ್ರಾರಯೋ ದಸ್ಯುಬಲಾನಿ ಯತ್ರ ।\\
ದಾವಾನಲೋ ಯತ್ರ ತಥಾಬ್ಧಿಮಧ್ಯೇ\\
        ತತ್ರ ಸ್ಥಿತಾ ತ್ವಂ ಪರಿಪಾಸಿ ವಿಶ್ವಂ ॥೩೨॥

ವಿಶ್ವೇಶ್ವರಿ ತ್ವಂ ಪರಿಪಾಸಿ ವಿಶ್ವಂ\\
        ವಿಶ್ವಾತ್ಮಿಕಾ ಧಾರಯಸೀಹ ವಿಶ್ವಂ ।\\
ವಿಶ್ವೇಶವಂದ್ಯಾ ಭವತೀ ಭವಂತಿ\\
        ವಿಶ್ವಾಶ್ರಯಾ ಯೇ ತ್ವಯಿ ಭಕ್ತಿನಮ್ರಾಃ ॥೩೩॥

ದೇವಿ ಪ್ರಸೀದ ಪರಿಪಾಲಯ ನೋಽರಿಭೀತೇ-\\
      ರ್ನಿತ್ಯಂ ಯಥಾಸುರವಧಾದಧುನೈವ ಸದ್ಯಃ ।\\
ಪಾಪಾನಿ ಸರ್ವಜಗತಾಂ ಪ್ರಶಮಂ ನಯಾಶು\\
        ಉತ್ಪಾತಪಾಕಜನಿತಾಂಶ್ಚ ಮಹೋಪಸರ್ಗಾನ್ ॥೩೪॥

ಪ್ರಣತಾನಾಂ ಪ್ರಸೀದ ತ್ವಂ ದೇವಿ ವಿಶ್ವಾರ್ತಿಹಾರಿಣಿ ।\\
ತ್ರೈಲೋಕ್ಯವಾಸಿನಾಮೀಡ್ಯೇ ಲೋಕಾನಾಂ ವರದಾ ಭವ ॥೩೫॥

ದೇವ್ಯುವಾಚ ॥೩೬॥\\
ವರದಾಹಂ ಸುರಗಣಾ ವರಂ ಯಂ ಮನಸೇಚ್ಛಥ ।\\
ತಂ ವೃಣುಧ್ವಂ ಪ್ರಯಚ್ಛಾಮಿ ಜಗತಾಮುಪಕಾರಕಂ ॥೩೭॥

ದೇವಾ ಊಚುಃ ॥೩೮॥\\
ಸರ್ವಾಬಾಧಾಪ್ರಶಮನಂ ತ್ರೈಲೋಕ್ಯಸ್ಯಾಖಿಲೇಶ್ವರಿ ।\\
ಏವಮೇವ ತ್ವಯಾ ಕಾರ್ಯಮಸ್ಮದ್ವೈರಿವಿನಾಶನಂ ॥೩೯॥
%============================================================================================
\section{ಶ್ರೀಶರಭೇಶ್ವರಕವಚಂ}
\addcontentsline{toc}{section}{ಶ್ರೀಶರಭೇಶ್ವರಕವಚಂ}

ಶಿವೋವಾಚ\\
ವಕ್ಷ್ಯಾಮಿ ಶೃಣು ತೇ ದೇವಿ ಸರ್ವರಕ್ಷಣಮದ್ಭುತಂ ।\\
ಶಾರಭಂ ಕವಚಂ ನಾಮ ಚತುರ್ವರ್ಗಫಲಪ್ರದಂ ॥೧॥

ಶರಭಶ್ರೀಸಾಲುವಪಕ್ಷಿರಾಜಾಖ್ಯಕವಚಸ್ಯ ತು ।\\
ಸದಾಶಿವಃ ಋಷಿಶ್ಛಂದೋ ಬೃಹತೀ ಶರಭೇಶ್ವರಃ ॥೨॥

ದೇವತಾ ಪ್ರಣವಂ ಬೀಜಂ ಪ್ರಕೃತಿಃ ಶಕ್ತಿರುಚ್ಯತೇ ।\\
ಚತುರ್ವರ್ಗಾರ್ಥಸಿಧ್ಯರ್ಥೇ ವಿನಿಯೋಗೋಽಥ ಭಾವನಾ ॥೩॥

ರಕ್ತಾಭಂ ಸುಪ್ರಸನ್ನಂ ತ್ರಿಣಯನಮಮೃತೋನ್ಮತ್ತಭಾಷಾಭಿರಾಮಂ\\
ಕಾರುಣ್ಯಾಂಭೋಧಿಮೀಶಂ ಶರಭಮಭಯದಂ ಚಂದ್ರರೇಖಾವತಂಸಂ ।\\
ಶಂಖಧ್ವಾನಾಖಿಲಾಶಾಪ್ರತಿಹತವಿಧಿನಾ ಭಾಸಮಾನಾತ್ಮಭಾವಂ\\
ಸರ್ವೇಶಂ ಸಾಲುವೇಶಂ ಪ್ರಣತಭಯಹರಂ ಪಕ್ಷಿರಾಜಂ ನಮಾಮಿ ॥೪॥

ಓಂ ಶ್ರೀಶಿವಃ ಪುರತಃ ಪಾತು ಮಾಯಾಧೀಶಸ್ತು ಪೃಷ್ಠತಃ ।\\
ಪಿನಾಕೀ ದಕ್ಷಿಣಂ ಪಾತು ವಾಮಪಾರ್ಶ್ವಂ ಮಹೇಶ್ವರಃ ॥೫॥

ಶಿಖಾಗ್ನಂ ಪಾತು ಮೇ ಶಂಭುರ್ನಿಟಿಲಂ ಪಾತು ಶಂಕರಃ ।\\
ಈಶ್ವರೋ ವದನಂ ಪಾತು ಭ್ರುವೋರ್ಮಧ್ಯಂ ಪುರಾಂತಕಃ ॥೬॥

ಭ್ರುವೌ ಪಾತು ಮಮ ಸ್ಥಾಣುಃ ಕಪರ್ದೀ ಪಾತು ಲೋಚನೇ ।\\
ಶರ್ವೋ ಮೇ ಶ್ರೋತ್ರಕೇ ಪಾತು ವಾಗೀಶಃ ಪಾತು ಲಂಬಿಕಾಂ ॥೭॥

ನಾಸಿಕ ಮೇ ವೃಷಾರೂಢೋ ನಾಸಾಗ್ರೇ ವೃಷಭಧ್ವಜಃ ।\\
ಸ್ಮರಾರಿಃ ಪಾತು ಮೇ ತಾಲೂ ಚೋಷ್ಠಯೋರ್ಭಕ್ತವತ್ಸಲಃ ॥೮॥

ಪಾತು ಮೃತ್ಯುಂಜಯೋ ದಂತಾಂಶ್ಚಬುಕಂ ಪಾತು ಭೂತರಾಟ್ ।\\
ಪರಮೇಶಃ ಕಪೋಲೌ ಮೇ ತ್ರಿಕಂ ಪಾತು ಕಪಾಲಭೃತ್ ॥೯॥

ಕಂಠಂ ಪಶುಪತಿಃ ಪಾತು ಶೂಲೀ ಪಾತು ಹನೂ ಮಮ ।\\
ಸ್ಕಂಧದ್ವಯಂ ಹರಃ ಪಾತು ಧೂರ್ಜಟಿಃ ಪಾತು ಮೇ ಭುಜೌ ॥೧೦॥

ಭುಜಸಂಧಿಂ ಮಹಾದೇವಃ ಈಶಾನೋ ಮೇ ಪ್ರಕೂರ್ಪರಂ ।\\
ಮಧ್ಯಸಂಧಿಂ ಜಗನ್ನಾಥಃ ಪ್ರಕೋಷ್ಠೇ ಚಂದ್ರಶೇಖರಃ ॥೧೧॥

ಮಣಿ ಬಂಧೇ ತ್ರಿಣೇತ್ರೋ ಮೇ ಭೀಮಃ ಪಾತು ಕರಸ್ಥಲೇ ।\\
ಕರಪೃಷ್ಠೇ ಮೃಡಃ ಪಾತು ರುದ್ರೋಽಙ್ಗುಷ್ಠದ್ವಯಂ ಮಮ ॥೧೨॥

ಉಮಾಸಹಾಯಸ್ತರ್ಜನ್ಯೌ ಭರ್ಗೋ ಮೇ ಪಾತು ಮಧ್ಯಮೇ ।\\
ಅನಾಮಿಕೇ ಕರಾಲಾಸ್ಯಃ ಕಾಲಕಂಠಃ ಕನಿಷ್ಠಿಕೇ ॥೧೩॥

ಗಂಗಾಧರೋಽಙಗುಲೀಪರ್ವಾಣ್ಯಪ್ರಮೇಯೋ ನಖಾನಿ ಮೇ ।\\
ವಕ್ಷಸ್ತತ್ಪುರುಷಃ ಪಾತು ಕಕ್ಷೇ ದಕ್ಷಾಧ್ವರಾಂತಕಃ ॥೧೪॥

ಅಘೋರೋ ಹೃದಯಂ ಪಾತು ವಾಮದೇವಃ ಸ್ತನದ್ವಯಂ ।\\
ಫಾಲದೃಕ್ ಜಠರಂ ಪಾತು ನಾಭಿಂ ನಾರಾಯಣೋಽವ್ಯಯಃ ॥೧೫॥

ಗುಹ್ಯಂ ಪ್ರಜಾಕರಃ ಪಾತು ಕುಕ್ಷಿಪಾರ್ಶ್ವೇ ಮಹಾಬಲಃ ।\\
ಸುದ್ಯೋಜಾತಃ ಕಟಿಂ ಪಾತು ಪೃಷ್ಠಭಾಗಂ ತು ಭೈರವಃ ॥೧೬॥

ಮೋಹನೋ ಜಘನಂ ಪಾತು ಗುದಂ ಮಮ ಜಿತೇಂದ್ರಿಯಃ ।\\
ಊರೂಯುಗಂ ಹರಃ ಪಾತು ಜಾನುಯುಗ್ಮಂ ಭವಾಂತಕಃ ॥೧೭॥

ಹುಂಕಾರಃ ಪಾತು ಮೇ ಜಂಘೇ ಫಟ್ಕಾರೋ ಮಮ ಗುಲ್ಫಕೇ ।\\
ಖಟ್ಕಾರಃ ಪಾದಪೃಷ್ಠೇ ಮೇ ವಷಟ್ಕಾರೋಽಙ್ಘ್ರಿಣೋಸ್ತಲೇ ॥೧೮॥

ಸ್ವಹಾಕಾರೋಽಙ್ಗುಲೀಪಾರ್ಶ್ವಂ ಸ್ವಧಾಕಾರೋಽಙಗುಲೀರ್ಮಮ ।\\
ತ್ವರಿತಃ ಸಂಧಿಬಂಧಂ ಮೇ ರೋಮಕೂಪಾಣಿ ಸಿಹ್ಮಜಿತ್ ॥೧೯॥

ತ್ವಚಃ ಪಾತು ಮನೋವೇಗಃ ಕಾಲಜಿತ್ ರುಧಿರಂ ಮಮ ।\\
ಪುಷ್ಟಿದಃ ಪಾತು ಮೇ ಮಾಂಸಂ ಮೇದೋ ಮೇ ಸ್ವಸ್ತಿದೋಽವತು ॥೨೦॥

ಸರ್ವಾತ್ಮಾಽಸ್ಥಿಚಯಂ ಪಾತು ಮಜ್ಜಾಂ ಮಮ ಜಗತ್ಪ್ರಭುಃ ।\\
ಶುಕ್ಲಂ ಬುದ್ಧಿಕರಃ ಪಾತು ಬುದ್ಧಿಂ ವಾಚಾಮಧೀಶ್ವರಃ ॥೨೧॥

ಮೂಲಾಧಾರಾಂಬುಜಂ ಪಾತು ಭಗವಾನ್ ಶರಭೇಶ್ವರಃ ।\\
ಸ್ವಾಧಿಷ್ಠಾನಮಜಃ ಪಾತು ಮಣಿಪೂರಂ ಹರಿಪ್ರಿಯಃ ॥೨೨॥

ಅನಾಹತಂ ಸಾಲುವೇಶೋ ವಿಶುದ್ಧಿಂ ಜೀವನಾಯಕಃ ।\\
ಸರ್ವಜ್ಞಾನಪ್ರದೋ ದೇವೋ ಲಲಾಟಂ ಮೇ ಸದಾಶಿವಃ ॥೨೩॥

ಬ್ರಹ್ಮರಂಧ್ರಂ ಮಹಾದೇವಃ ಪಕ್ಷಿರಾಜೋಽಖಿಲಾತ್ಮವಾನ್ ।\\
ಸರ್ವಲೋಕವಶೀಕಾರಃ ಪಾತು ಮಾಂ ಪರಗರ್ವಜಿತ್ ॥೨೪॥

ವಜ್ರಮುಷ್ಟಠಿವರಾಭೀತಿಹಸ್ತಃ ಕಾಲಾಗಿಸನ್ನಿಭಃ ।\\
ವಿಜಯಾಸಹಿತಃ ಪಾತು ಚೈಂದ್ರೀಂ ಕಕುಭಮಗ್ನಿಜಿತ್ ॥೨೫॥

ಶಕ್ತಿಶೂಲಕಪಾಲಾಸಿಹಸ್ತಃ ಸೌದಾಮಿನೀಪ್ರಭಃ ।\\
ಜಯಾಯುತೋ ಮಹಾಭೀಮಃ ಪಾತು ವೈಶ್ವಾನರೀಂ ದಿಶಂ ॥೨೬॥

ದಂಡಾಸಿಮುಸಲೀ ಸೀರಪಾಶಾಂಕುಶಕರಾಂಬುಜಃ ।\\
ಕಮಾಂತಕೋಽಜಿತಾಯುಕ್ತಃ ದಿಶಂ ಯಾಮ್ಯಾಂ ಸದಾಽವತು ॥೨೭॥

ಖಡ್ಗಖೇಟಾಗ್ನಿಪರಶುಹಸ್ತಃ ಶತ್ರುವಿಮರ್ದನಃ ।\\
ಅಪರಾಜಿತಯಾ ಯುಕ್ತಃ ಸದಾವ್ಯಾನ್ನೈರೃತೀಂ ದಿಶಂ ॥೨೮॥

ಪಾಶಾಂಕುಶಧನುರ್ಬಾಣಪಾಣಿರ್ಘೋಣೀಯುತೋಽಗ್ರಗಃ ।\\
ಹರಿದ್ರಾಭೋಽನಿಶಂ ಪಾಯಾದ್ವಾರುಣೀಂ ದಿಶಮಾತ್ಮಜಿತ್ ॥೨೯॥

ಧ್ವಜಛತ್ರಗದಾಕುಂತಭುಜೋ ದುರ್ಗಾಯುತೋಽವ್ಯಯಃ ।\\
ಚಂಡವೇಗಃ ಶಿವಃ ಪಾತು ಸತತಂ ಮಾರುತೀಂ ದಿಶಂ ॥೩೦॥

ಗದಾಕ್ಷಸ್ರಗ್ವರಾಭೀತಿಕರಾಂಭೋಜಃ ಶ್ರಿಯಾ ಯುತಃ ।\\
ಕನಕಾಭೋ ಮಹಾತೇಜಾಃ ಪಾತು ಕೌಬೇರಕೀಂ ದಿಶಂ ॥೩೧॥

ತ್ರಿಶೂಲಾಹಿಕಪಾಲಾಗ್ನಿದೋಸ್ತಲೋ ವಿದ್ಯಯಾ ಯುತಃ ।\\
ಭಸ್ಮೋದ್ಧೂಲಿತಸರ್ವಾಂಗೋ ದಿಶಂ ಪಾತು ಶಿವಪ್ರಿಯಾಂ ॥೩೨॥

ಜಪಸ್ರಕ್ಪುಸ್ತಕಾಂಭೋಜಕಮಂಡಲುಕರಾಂಬುಜಃ ।\\
ಊರ್ಧ್ವಂ ಪಾತು ಗಿರಾ ಯುಕ್ತಃ ಸರ್ವಭೂತಹಿತೇ ರತಃ ॥೩೩॥

ಶಂಖಚಕ್ರಗದಾಽಭೀತಿಹಸ್ತಃ ಪದ್ಮಾಯುತೋಽವ್ಯಯಃ ।\\
ನೀಲಾಂಜನಸಮೋ ನೀಲಃ ಪಾತಾಲಂ ಪಾತ್ವನಾರತಂ ॥೩೪॥

ಅನುಕ್ತಾ ವಿದಿಶಃ ಪಾತು ಸಾಳವೋ ನಾರಸಿಹ್ಮಜಿತ್ ।\\
ಶರಭಃ ಪಾತು ಸಂಗ್ರಾಮೇ ಯುದ್ಧೇ ವೈರಿಕುಲಾಂತಕಃ ॥೩೫॥

ಸರ್ವಸೌಭಾಗ್ಯದಃ ಪಾತು ಜಾಗ್ರತ್ಸ್ವಪ್ನಸುಷುಪ್ತಿಷು ।\\
ಸರ್ವಸಂಪತ್ಪ್ರದಃ ಪಾತು ಧನಧಾನ್ಯಾದಿಕಂ ಮಮ ॥೩೬॥

ಸಂತಾನದಃ ಸುತಾನ್ ಪಾತು ಪುತ್ರಾನಾಯುಷ್ಕರೋಽನಿಶಂ ।\\
ಬಂಧೂನ್ ವೃದ್ಧಿಕರಃ ಪಾತು ಗೃಹಂ ಸರ್ವವಶಂಕರಃ ॥೩೭॥

ಗ್ರಾಮಂ ಗ್ರಾಮೇಶ್ವರಃ ಪಾತು ರಾಜ್ಯಂ ಪಾತು ದಿಗಂಬರಃ ।\\
ರಾಷ್ಟ್ರಂ ಶಾಂತಿಕರಃ ಪಾತು ರಾಜಾನಂ ಧರ್ಮಶಾಸಕಃ ॥೩೮॥

ಮಾರ್ಗಂ ದುಷ್ಟಹರಃ ಪಾತು ಧರ್ಮಂ ಕರ್ಮಾಣಿ ಭೈರವಃ ।\\
ಬಟುಕಃ ಪಾತು ಮೇ ಸರ್ವಂ ವ್ಯವಸ್ಥಾಸು ಭಯೇಷು ಚ ॥೩೯॥

ಸಾಧಕಃ ಪ್ರಣವಂ ಧ್ಯಾಯನ್ ಜ್ವಲದ್ದ್ವಿಃ ಪ್ರಜ್ಜ್ವಲದ್ವಯಂ ।\\
ಪ್ರತ್ಯಾವೃತ್ತಿ ಚತುರ್ಥ್ಯಂತಂ ರಕ್ಷ ರಕ್ಷೇತಿ ಚೋಚ್ಚರೇತ್ ॥೪೦॥

ಸ್ಪರ್ಶಂ ಸ್ಪರ್ಶಂ ಜಪಂ ಕೃತ್ವಾ ಪ್ರತಿಸ್ಥಾನಂ ಸಮಾಹಿತಃ ।\\
ಪ್ರಾರ್ಥಯೇದಖಿಲಂ ಸ್ವೇಷ್ಟಂ ಹೃದಿಸ್ಥಂ ಸಾಳುವೇಶ್ವರಂ ॥೪೧॥

ಯೇ ಗ್ರಾಮಘಾತಕಾಃ ಕ್ರೂರಾಃ ಕಪಟಾ ದುರ್ಮದಾ ಗ್ರಹಾಃ ।\\
ತಸ್ಕರಾಃ ಶತ್ರವಃ ಕ್ರುದ್ಧಾ ವಧೂಸಕ್ತಾಃ ಪಲಾಶನಾಃ ॥೪೨॥

ಛಯಾಚಾರಾಃ ವಿಟಾ ಭ್ರಷ್ಟಾಃ ದಿವಾಚರನಿಶಾಚರಾಃ ।\\
ತೇ ಸರ್ವೇ ಪಕ್ಷಿರಾಜಸ್ಯ ಪಕ್ಷವಾತವರಾಹತಃ ॥೪೩॥

ಸ್ತ್ರೀಬಾಲಸಹಿತಾಃ ಕ್ಷಿಪ್ರಂ ಪಿತೃಮಾತೃಕುಲಾನ್ವಿತಾಃ ।\\
ಭಗ್ನಚಿತ್ತಾ ಜಗಸ್ಥಾನಾ ಯಾಂತು ದೇಶಾತರಂ ಸ್ವಯಂ ॥೪೪॥

ಯೇ ತು ದುಷ್ಟಾಹಾ ಯಕ್ಷಾಃ ಪಿಶಾಚಾ ದೇವಯೋನಯಃ ।\\
ಚತುಷ್ಷಷ್ಟೀಗಣಾ ದುಷ್ಟಾಃ ಸಪ್ತತ್ಯುನ್ಮತ್ತಕಾ ಗ್ರಹಃ ॥೪೫॥

ಅಷ್ಟಾಶೀತಿಮಹಾಭೂತಾಃ ಸಪ್ತಕೋಟಿಮಹಾಗ್ರಹಾಃ ।\\
ನವತಿಜ್ವರಭೇದಾಶ್ಚ ಶತಭೇದಾಶ್ಚ ಕೃತ್ರಿಮಾಃ ॥೪೬॥

ಪಂಚಾಶದ್ ಗಣನಾಥಾಶ್ಚ ನಿಯುತಾ ಕೃತಿಮಾ ಗ್ರಹಾಃ ।\\
ಪ್ರೇತಾರೂಢಾಸ್ತ್ರಯಸ್ತ್ರಿಶತ್ಪಿಂಡದಾನಪರಾಯಣಾಃ ॥೪೭॥

ಅಯುತಾಃ ಕ್ಷುದ್ರಭೇದಾಶ್ಚ ಚತ್ವಾರಿಂಶಚ್ಛಿವಾವಯಾಃ ।\\
ದ್ವಾತ್ರಿಂಶದ್ವಹ್ನಿವಕ್ತ್ರಾಶ್ಚ ವಿಂಶತ್ಮಾರ್ಜಾರವಕ್ತಕಾಃ ॥೪೮॥

ಚತುಷ್ಷಷ್ಟಯಾನುರೂಪಾಶ್ಚ ಯೇ ಚಾನ್ಯೇ ಕ್ಷುದ್ರಯೋನಯಃ ।\\
ತೇ ಸರ್ವ ಸಾಳವೇಶಸ್ಯ ಶಂಖನಿಸ್ವಾನಮೋಹಿತಾಃ ॥೪೯॥

ವಿಷಾಣಾಃ ಸ್ಖಲಿತಸ್ವಾಂತಾಃ ಪ್ರಾಣತಾಣಪರಾಯಣಾಃ ।\\
ಗಚ್ಛಂತ್ವಸತ್ಪ್ರಯೋಕ್ತಾರೋ ದೇಶಾಂತರಮನಿಚ್ಛಾಯಾ ॥೫೦॥

ಯೇ ಚ ಮೂಷಿಕಮಾರ್ಜಾರಶುನಕಾ ರೋಗವೃಶ್ಚಿಕಾಃ ।\\
ಆಶೀವಿಷಶಿತವ್ಯಾಲಾ ವ್ಯಾಘ್ರಋಕ್ಷಾಹಿಸೂಕರಾಃ ॥೫೧॥

ಗೃಧ್ರಾಃ ಶ್ಯೇನಾಃ ಖಗಾಃ ಕಾಕದಂಶಕಾ ಭೃಶಕಾ ಗ್ರಹಾಃ ।\\
ಏತೇ ಶರಭಹಸ್ತಾಗ್ರನಖಕ್ಷತವಿಮೋಕ್ಷಕಾಃ ।\\
ಸ್ರವದ್ರಕ್ತಚ್ಛಟಾಸಿಕ್ತಾಃ ಶಿಲಾತಲನಿತಾಡಿತಾಃ ॥೫೪॥

ಸಂಭಿನ್ನತನವಃ ಶೀಘ್ರಂ ನಶ್ಯಂತ್ವಖಿಲದುಶ್ಚರಃ ।\\
ನ ದಶಂತುರಗಾಃ ಕ್ವಾಪಿ ನಾತಿವಾತೋಽಪಿ ಧಾತು ಚ ॥೫೩॥

ನ ದಹತ್ವಸಹೋ ವಹ್ನಿರ್ನಾಯಾಂತ್ವಾಪೋಽಪಿ ಚಾಧಿಕಂ ।\\
ನ ವರ್ಷತ್ವತಿವೃಷ್ಟಿಶ್ಚ ನ ಪತತ್ವ ಶನಿಃ ಕ್ವಚಿತ್ ॥೫೪॥

ನಾಕ್ರಾಮತ್ವಪಮೃತ್ಯುಶ್ಚ ನಾಪ್ಯುತ್ಪಾತಭಯಂ ಕ್ವಚಿತ್ ।\\
ನಾಪಮೃತ್ಯುರ್ಭವೇತ್ಕುತ್ರಾಪ್ಯಶುಭಂ ನ ಕ್ವಚಿದ್ಭವೇತ್ ॥೫೫॥

ನ ಬದಂತ್ವಸಹಂ ವಾಕ್ಯಂ ಜಂತವೋ ಮಮ ದೇಶಿಕ ।\\
ನಾಸ್ತು ವೈರಂ ಚ ಜಂತೂನಾಂ ಅನ್ಯೋನ್ಯಂ ರಾಜ್ಯಕೇ ಮಮ ॥೫೬॥

ಭವಂತು ಸುಖಿನಃ ಸರ್ವೇ ಸರ್ವಾಃ ಸಂತು ಪತಿವ್ರತಾಃ ।\\
ಸರ್ವಾಃ ಸೂಯಂತು ಸತ್ಪುತ್ರಾನ್ ಪುತ್ರೀಶ್ಚ ಶುಭಲಕ್ಷಣಾಃ ॥೫೭॥

ಸರ್ವೇ ದೇವಾಶ್ಚ ನಂದಂತು ಸಂತು ಕಲ್ಯಾಣಕಾರಣಾಃ ।\\
ರಾಜನ್ವತೀ ಮಹೀ ಚಾಸ್ತು ರಾಜಾ ಭವತು ಧಾರ್ಭಿಕಃ ॥೫೮॥

ಸಂಸ್ತ್ರವಂತು ಪಯೋ ಗಾವಃ ಫಲಂತ್ವೋಷಧಯೋಽಧಿಕಂ ।\\
ಭವಂತು ಫಲದಾ ವೃಕ್ಷಾಃ ವೃದ್ಧಿರ್ಭವತು ಮೇಽಖಿಲಾ ॥೫೯॥

ಮಮಾಸ್ತು ತರಸಾ ನೂನಮಾತ್ಮಜ್ಞಾನಮಚಂಚಲಂ ।\\
ಕಾಮಕ್ರೋಧಮಹಾಲೋಭಸಮದಾ ಮೋಹಮತ್ಸರಾಃ ॥೬೦॥

ನಾಕ್ರಾಮೇಯುಶ್ಚ ಸರ್ವೇಶ ಭಗವನ್ ಕರುಣಾನಿಧೇ ॥೬೧॥

ಗೌರೀವಲ್ಲಭ ಕಾಮಾರೇ ಕಾಲಕೂಟವಿಷಾಶನ ।\\
ಮಾಮುದ್ಧರಾಪದಂಭೋಧೇಸ್ತ್ರಿಪುರಘ್ನಾಂತಕಾಂತಕ ॥೬೨॥

ಸಾಳವೇಶ ಜಗನ್ನಾಥ ಸರ್ವಭೂತಹಿತೇ ರತ ।\\
ತ್ರಾಹಿ ಮಾಂ ತರಸಾ ಘೋರಾಂದುಷ್ಟಾನ್ನಾಶಯ ನಾಶಯ ॥೬೩॥

ಕಾಲಭೈರವ ವಿಶ್ವೇಶ ವಿಶ್ವರಕ್ಷಾಪರಾಯಣ ।\\
ರಕ್ಷ ಮೂಷಿಕಚೋರೇಭ್ಯೋ ಧಾನ್ಯರಾಶಿಂ ಚ ಮೇ ವಿಭೋ ॥೬೪॥

ಪಕ್ಷಿರಾಜ ಮಹಾದೇವ ಪ್ರಣತಾರ್ತಿವಿನಾಶನ ।\\
ತಸ್ಕರೇಣ ಹೃತಂ ವಸ್ತು ದ್ರುತಂ ದಾಪಯ ದಾಪಯ ॥೬೫॥

ಯೇ ಮರ್ಮಘಾತಿನಃ ಕ್ಷುದ್ರಾಃ ಛಿದ್ರೋಪದ್ರವಕಾರಕಾಃ ।\\
ಸರ್ವಾಚಾರಪರಿತ್ಯಕ್ತಾಃ ಮಾನಹೀನಾಶ್ಚ ರೋಧಕಾಃ ॥೬೬॥

ತೇ ಸರ್ವೇ ಸಾಳುವೇಶಸ್ಯ ಮುಸಲಾಯುಧಚೂರ್ಣಿತಾಃ ।\\
ನಶ್ಯಂತು ನಿಮಿಷಾರ್ಧೇನ ಪಾವಕಾವೃತತೂಲವತ್ ॥೬೭॥

ಯೇ ಜನದ್ರೋಹಿಣೋಽಲ್ಪಾಶ್ಚ ದುರಾಲೋಚಿತಭಾಷಿಣಃ ॥೬೮॥

ಸತ್ಕರ್ಮವಿಘ್ನಕರ್ತಾರಶ್ಚಾಸತ್ಕರ್ಮಪರಾ ನರಾಃ ।\\
ತೇ ಸಾಳುವೇಶಹಸ್ತಾಗ್ರಹಸ್ತನಿರ್ಮಿಭಿನ್ನಏಹಿನಃ ॥೬೯॥

ಪತಂತು ಭೂತಲೇ ಯಾಮ್ಯಾಂ ಪ್ರಾಣಾಸ್ತೇಷಾಂ ಪ್ರಯಾಂತು ವೈ ।\\
ತ್ವದಂಘ್ರಿಧ್ಯಾನನಿರ್ದಗ್ಧಪಾಪಕೋಶಾಯ ಮೇ ಶಿವ ॥೭೦॥

ಮಹ್ಯಂ ದ್ರುಹ್ಯಂತಿ ಯೇ ತೇಷಾಂ ವಿಭವಾನಿ ಕ್ಷಯಂತ್ವರಂ ॥೭೧॥

ತ್ವದಾಚಾರಪರಂ ಭಕ್ತಂ ಸಾಧಕಂ ಮಾಂ ವಿವೇಕಿನಂ ।\\
ಆಕ್ರಾಮಂತಿ ಚ ಸಂಗ್ರಾಮೇ ತೇ ಗಚ್ಛಂತು ಪರಾಹತಾಃ ॥೭೨॥

ತ್ವದೀಯೇನೈವ ಮಾರ್ಗೇಣ ಚರತಾಂ ಜಪತಾಂ ಸದಾ ।\\
ಯೇ ವದಂತಿ ಪರೀವಾದಂ ಭ್ರಾಂತಾಃ ಶೀಘ್ರಂ ಭವಂತು ತೇ ॥೭೩॥

ತ್ವದ್ದಾಸಮುಖ್ಯಂ ಮಾಂ ಧೀರಂ ತರ್ಜಯಂತಿ ಬಲಾಚ್ಚಾ ಯೇ ।\\
ಮನಸಾ ಯೇಽವಮನ್ಯಂತೇ ತತ್ಸ್ವಾಂತಂ ಕ್ಷಮತು ಕ್ಷಣಾತ್ ॥೭೪॥

ಮನಸಾ ಕರ್ಮಣಾ ವಾಚಾ ಯೇ ಕುರ್ವಂತ್ಯತಿದುಃಸಹಂ ।\\
ತೇ ಮಹಾಶೋಕರೋಗಾಬ್ಧೌ ಪತಂತ್ವಾಶು ಶಿವಾಜ್ಞಾಯಾ ॥೭೫॥

ಮದೀಯಾನಿ ಪದಾರ್ಥಾನಿ ಗ್ರಹೀತುಂ ಯೋಽವಲೋಕತೇ ।\\
ತತ್ಕ್ಷಣಾದೇವ ನಷ್ಟಾಕ್ಷೋ ಭವತ್ವಾಶು ಶಿವಾಜ್ಞಯಾ ॥೭೬॥

ಮದೀಯದ್ರವ್ಯಮಾದಾಯ ಯೇ ಗಚ್ಛಂತಿ ಹಿ ತಸ್ಕರಾಃ ।\\
ಸಿಹ್ಮಾರಿಪಾಶಸಂಬಂಧಾತ್ ತೇ ಚರಂತು ಪ್ರದಕ್ಷಿಣಂ ॥೭೭॥

ಸೀಮಾಂತರಗತಾಶ್ಚೋರಾ ಗೃಹೀತದ್ರವ್ಯಸಂಚಯಾಃ ।\\
ಅವಶಾವಯವಾಸ್ತೇ ಚ ಗಚ್ಛಂತು ಶರಭಾಜ್ಞಯಾ ॥೭೮॥

ತಸ್ಕರಾ ಯೇಽಪಗೃಹ್ಣಂತಿ ಮಮ ಧಾನ್ಯಧನಾದಿಕಂ ।\\
ಪಕ್ಷಿರಾಟ್ ಕ್ರೋಡಿಕಾಕೃಷ್ಟಾಃ ಸಮಾಗಚ್ಛಂತು ತೇ ದ್ರುತಂ ॥೭೯॥

ಸಮಾಹೃತಪದಾರ್ಥಾದ್ಯಾ ದೇಶಾತೀತಾಶ್ಚ ತಸ್ಕರಾಃ ।\\
ಶರಭೇಶಹಲಾಕೃಷ್ಟಾಸ್ತ ಆಗಚ್ಛಂತು ಸತ್ವರಂ ॥೮೦॥

ಶಾಂತಂ ವಿವೇಕಿನಂ ಭಕ್ತಂ ತ್ವದಂಘ್ರಿಧ್ಯಾನತತ್ಪರಂ ।\\
ದ್ರುಹ್ಯಂತಿ ಯ ಇಹ ಪ್ರಾಣಾಸ್ತೇಷಾಂ ಯಾಂತು ಯಮಂ ಶಿವ ॥೮೧॥

ಷಟ್ತ್ರಿಂಶತ್ಕೋಷ್ಟಕೇ ಯಂತ್ರೇ ರೇಖಾಶೂಲಾದಿಸಾಧ್ಯಕೇ ।\\
ಸ್ವೇಚ್ಛಾಮಂತ್ರಂ ಲಿಖಿತ್ವಾ ತು ಜಪೇದಾರಾಧ್ಯ ಸಾಧಕಃ ॥೮೨॥

ಉದಙ್ಮುಖಃ ಸಹಸಂ ತು ರಕ್ಷಣಾಯ ಜಪೇನ್ನಿಶಿ ।\\
ನಷ್ಟಾಹರಣಕೇ ಪಂಚರಾತ್ರಂ ಪಶ್ಚಿಮದಿಙ್ಮುಖಃ ॥೮೩॥

ಮರಣೇ ಸಪ್ತರಾತ್ರಂ ತು ದಕ್ಷಿಣಾಭಿಮುಖೋ ಜಪೇತ್ ।\\
ರೋಗನಿಗ್ರಹಣೇ ಚಾಷ್ಟರಾತ್ರಮಾಗ್ನೇಯದಿಙಮುಖಃ ॥೮೪॥

ಇತಿ ಗುಹ್ಯ ಮಹಾಮಂತ್ರಂ ಪರಮಂ ಸರ್ವಸಿದ್ಧಿದಂ ।\\
ಶರಭೇಶಾಖ್ಯಕವಚಂ, ಚತುರ್ವರ್ಗಫಲಪ್ರದಂ ॥೮೫॥

ಪ್ರತ್ಯಹಂ ಪ್ರತಿಪಕ್ಷಂ ವಾ ಪ್ರತಿಮಾಸಮಥಾಪಿ ವಾ ।\\
ಯೋ ಜಪೇತ್ ಪ್ರತಿವರ್ಷ ವಾ ವರೇಣ್ಯಃ ಸ ಶಿವೋ ಭವೇತ್ ॥೮೬॥

ಏವಂ ಹಿ ಜಪತಃ ಪುಂಸಃ ಪಾತಕಂ ಚೋಪಪಾತಕಂ ।\\
ತತ್ಸರ್ವಂ ಲಯಮಾಪ್ನೋತಿ ರವಿಣಾ ತಿಮಿರಂ ಯಥಾ ॥೮೭॥

ದಶಾಬ್ದಂ ಯೋ ಜಪೇನ್ನಿತ್ಯಂ ಪ್ರಾತರುತ್ಥಾಯ ಸಾಧಕಃ ।\\
ಸರ್ವಸಿದ್ಧಿಂ ಸಮಾಶ್ರಿತ್ಯ ದೇಹಾಂತೇ ಸ ಶಿವೋ ಭವೇತ್ ॥೮೮॥

ತ್ರಿಕಾಲಂ ಧ್ಯಾನಪೂರ್ವಂ ತು ಜಪೇದ್ದ್ವಾದಶವಾರ್ಷಿಕಂ ।\\
ಸ ದೇವಿ ಕಾಯೇನಾನೇನ ಜೀವನ್ಮುಕ್ತೋ ಭವೇಚ್ಛಿವೇ ॥೮೯॥

ಶತವಾರಂ ಜಪೇನ್ನಿತ್ಯಂ ಮಂಡಲಂ ಯೋ ವರಾನನೇ ।\\
ಸೋಽಣಿಮಾದಿಗುಣಾನ್ ಪ್ರಾಪ್ಯ ವಿಚರೇತ್ಸ್ವೇಚ್ಛಯಾ ಸದಾ ॥೯೦॥

ಅತಲಾದಿಧರಣ್ಯಂತರ್ಭುವನೇಷು ತಥೋಪರಿ ।\\
ವಿಚರೇತ್ಕಾಮಗಃ ಸರ್ವೈಃ ಪೂಜ್ಯಮಾನೋ ಯಥಾಸುಖಂ ॥೯೧॥

ತ್ರಿಮಾಸಂ ಯೋ ಜಪೇನ್ನಿತ್ಯಮಷ್ಟೋತ್ತರಸಹಸ್ರಕಂ ।\\
ಸಹಸಾ ಶರಭೇಶಸ್ಯ ಸಾರೂಪ್ಯಂ ಲಭತೇಽಮ್ಬಿಕೇ ॥೯೨॥

ಷಣ್ಮಾಸಂ ಯೋ ಜಪೇದೇವಂ ಪ್ರಯತಸ್ತು ದೃಢವ್ರತಃ ।\\
ಮದ್ರೂಪಧಾರಕೈರ್ಮತ್ಯೈಃ ಸಹಸಿದ್ಧಿಪ್ರದಾಯಕೈಃ ॥೯೩॥

ಮಮ ಲೋಕೇ ತು ಸಂಪೂಜ್ಯೋ ವಿಷ್ಣುಲೋಕೇ ತಥೈವ ಚ ।\\
ಬ್ರಹ್ಮಾಲೋಕೇ ಚ ರಮತೇ ಸರ್ವತ್ರ ನ ನಿವಾರ್ಯತೇ ॥೯೪॥

ಇಂದ್ರಾಗ್ನಿಯಮರಕ್ಷೇಶಜಲೇಶಪವನೈಃ ಸದಾ ।\\
ಸೋಮೇಶಾನಕಲಕ್ಷ್ಮೀಶೈರ್ದಿಶಾಂ ಪಾಲೈಃ ಸ ಪೂಜ್ಯತೇ ॥೯೫॥

ಆದಿತ್ಯಸೋಮಪೃಥ್ವೀಶಬುಧಶ್ರೀಗುರುಭಾರ್ಗವೈಃ ।\\
ಪೂಜ್ಯತೇ ಸ ಗ್ರಹೈಃ ಸರ್ವೈಃ ಶನಿನಾ ರಾಹುಕೇತುಭಿಃ ॥೯೬॥

ಭೃಗ್ವಙಗಿರಃಪುಲಸ್ತ್ಯೈಶ್ಚ ಪುಲಹಾದಿಮರೀಚಿಭಿಃ ।\\
ದಕ್ಷಕಾಶ್ಯಪಕ್ರತ್ವಬ್ಜಯೋನಿಭಿಃ ಸ ಚ ಪೂಜ್ಯತೇ ॥೯೭॥

ಭೈರವೈರ್ವಸುಭೀ ರುದ್ರೈರಾದಿತ್ಯೈರ್ವಾಲಖಿಲ್ಯಕೈಃ ।\\
ದಿಗ್ಗಜೈಶ್ಚ ಮಹಾನಾಗೈರ್ತ್ದಿವ್ಯಾಸ್ತ್ರೈರ್ದಿವ್ಯವಾಹನೈಃ ॥೯೮॥

ಮಾಹೇಶ್ವರೈರ್ಮಹಾರತ್ನೈಃ ಕಾಮಧೇನುಸುರದ್ರುಮೈಃ ।\\
ಸರಿದ್ಭಿಃ ಸಾಗರೈಃ ಶೈಲೇರ್ದೇವತಾಭಿಸ್ತಪೋಧನೈಃ ॥೯೯॥

ದಾನವೈ ರಾಕ್ಷಸೈಃ ಕ್ರೂರೈಃ ಸಿದ್ಧಗಂಧರ್ವಕಿನ್ನರೈಃ ।\\
ಯಕ್ಷವಿದ್ಯಾಧರೈರ್ನಾಗೈರಪ್ಸರೋಭಿಃ ಸ ಪೂಜ್ಯತೇ ॥೧೦೦॥

ಅಪಸ್ಮಾರಗ್ರಹೈರ್ಭೀಮೈರುನ್ಮತ್ತೈರ್ಬ್ರಹ್ಮರಾಕ್ಷಸೈಃ ।\\
ವೇತಾಲೈಃ ಖೇಚರೈಃ ಕ್ರೂರೈಃ ಕೂಶ್ಮಾಂಡರಾಕ್ಷಸಗ್ರಹೈಃ ॥೧೦೧॥

ಜ್ವಾಲಾವಕ್ತ್ರೈಸ್ತಮೋರೂಪೈಃ ಸ್ತ್ರೀಗ್ರಹೈಃ ಪಾಪವಿಕ್ರಮೈಃ ।\\
ಭೂತಪ್ರೇತಪಿಶಾಚಾದ್ಯೈರ್ಗ್ರಹೈಃ ಸರ್ವೈಃ ಸ ಪೂಜ್ಯತೇ ॥೧೦೨॥

ಬ್ರಾಹ್ಮಣೈಃ ಕ್ಷತ್ರಿಯೈರ್ವೈಶ್ಯೈಃ ಶುದ್ರೈರನ್ಯೈಶ್ಚ ಮಾನವೈಃ ।\\
ಪಶುಪಕ್ಷಿಮೃಗವ್ಯಾಲೈಃ ಪೂಜ್ಯತೇ ಸರ್ವಜಂತುಭಿಃ ॥೧೦೩॥

ಕಿಮತ್ರ ಬಹುನಾ ದೇವಿ ತವ ವಕ್ಷ್ಯೇ ಯಥಾತಥಂ ।\\
ಮಯಾ ಚ ವಿಷ್ಣುನಾ ಚೈವ ವಿಶ್ವಕರ್ತ್ರಾ ಚ ಪಾಲ್ಯತೇ ॥೧೦೪॥

ಭವತ್ಯಾ ಚ ಗಿರಾ ಲಕ್ಷ್ಮ್ಯಾ ಬ್ರಹ್ಮಾಣ್ಯಾದ್ಯಷ್ಟಮಾತೃಭಿಃ ।\\
ಗಣೇಶ್ವರಾದಿಯೋಗೀಂದ್ರೈರ್ಯೋಗಿನೀಭಿಶ್ಚ ಪಾಲ್ಯತೇ ॥೧೦೫॥

ಯ ಇದಂ ಪ್ರಜಪೇದ್ಭಕ್ತ್ಯಾ ಚಾಸಾಧ್ಯಂ ನೈವ ವಿದ್ಯತೇ ।\\
ಕವಚೇಂದ್ರಂ ಮಹಾಮಂತ್ರಂ ಜಪೇತ್ತಸ್ಮಾದನುತ್ತಮಂ ॥೧೦೬॥

ಉಚ್ಚಾಟನೇ ಮರುದ್ವಕ್ತ್ರೋ ವಿದ್ವೇಷೇ ರಾಕ್ಷಸಾನನಃ ।\\
ಪ್ರಾಗಾನನೋಽಭಿವೃದ್ಧೌ ಚ ಸರ್ವೇಷ್ವೀಶಾನದಿಙ್ಮುಖಃ ॥೧೦೭॥

ಯೋ ಜಪೇತ್ಕವಚಂ ನಿತ್ಯಂ ತ್ರಿಕಾಲಂ ಧ್ಯಾನಪೂರ್ವಕಂ ।\\
ಸರ್ವಸಿದ್ಧಿಮವಾಪ್ನೋತಿ ಸಹಸಾ ಸಾಧಕೋತ್ತಮಃ ॥೧೦೮॥

ದೇವ ದೇವ ಮಹಾದೇವ ಶಿವ ಕಾರುಣ್ಯವಾರಿಧೇ ।\\
ಪಾಹಿ ಮಾಂ ಪ್ರಣತಂ ಸ್ವಾಮಿನ್ ಪ್ರಸೀದ ಸತತಂ ಪ್ರಭೋ ॥೧೦೯॥

ಯತ್ಕೃತ್ಯಂ ತನ್ನಕೃತಂ ಯದಕೃತ್ಯಂ ಕೃತ್ಯವದಾಚರಿತಂ ।\\
ಉಭಯೋಃ ಪ್ರಾಯಶ್ಚಿತ್ತಂ ಶಿವ ತವ ನಾಮಾಕ್ಷರದ್ವಿತಯಂ ॥೧೧೦॥

\authorline{ಇತಿ ಶ್ರೀಶರಭೇಶ್ವರಕವಚಂ ಸಂಪೂರ್ಣಂ ।}

%=========================================================================================================================
\section{ಶ್ರೀದತ್ತಾತ್ರೇಯವಜ್ರಕವಚಂ}
\addcontentsline{toc}{section}{ಶ್ರೀದತ್ತಾತ್ರೇಯವಜ್ರಕವಚಂ}
ಋಷಯ ಊಚುಃ ।\\
ಕಥಂ ಸಂಕಲ್ಪಸಿದ್ಧಿಃ ಸ್ಯಾದ್ವೇದವ್ಯಾಸ ಕಲೌ ಯುಗೇ ।\\
ಧರ್ಮಾರ್ಥಕಾಮಮೋಕ್ಷಾಣಾಂ ಸಾಧನಂ ಕಿಮುದಾಹೃತಂ ॥೧॥

ವ್ಯಾಸ ಉವಾಚ ।\\
ಶೃಣ್ವಂತುಋಷಯಃ ಸರ್ವೇ ಶೀಘ್ರಂ ಸಂಕಲ್ಪಸಾಧನಂ ।\\
ಸಕೃದುಚ್ಚಾರಮಾತ್ರೇಣ ಭೋಗಮೋಕ್ಷಪ್ರದಾಯಕಂ ॥೨॥

ಗೌರೀಶೃಂಗೇ ಹಿಮವತಃ ಕಲ್ಪವೃಕ್ಷೋಪಶೋಭಿತಂ ।\\
ದೀಪ್ತೇ ದಿವ್ಯಮಹಾರತ್ನಹೇಮಮಂಡಪಮಧ್ಯಗಂ ॥೩॥

ರತ್ನಸಿಂಹಾಸನಾಸಿನಂ ಪ್ರಸನ್ನಂ ಪರಮೇಶ್ವರಂ ।\\
ಮಂದಸ್ಮಿತಮುಖಾಂಭೋಜಂ ಶಂಕರಂ ಪ್ರಾಹ ಪಾರ್ವತೀ ॥೪॥

ಶ್ರೀದೇವ್ಯುವಾಚ ।\\
ದೇವದೇವ ಮಹಾದೇವ ಲೋಕಶಂಕರ ಶಂಕರ ।\\
ಮಂತ್ರಜಾಲಾನಿ ಸರ್ವಾಣಿ ಯಂತ್ರಜಾಲಾನಿ ಕೃತ್ಸ್ನಶಃ ॥೫॥

ತಂತ್ರಜಾಲಾನ್ಯನೇಕಾನಿ ಮಯಾ ತ್ವತ್ತಃ ಶ್ರುತಾನಿ ವೈ ।\\
ಇದಾನೀಂ ದ್ರಷ್ಟುಮಿಚ್ಛಾಮಿ ವಿಶೇಷೇಣ ಮಹೀತಲಂ ॥೬॥

ಇತ್ಯುದೀರಿತಮಾಕರ್ಣ್ಯ ಪಾರ್ವತ್ಯಾ ಪರಮೇಶ್ವರಃ ।\\
ಕರೇಣಾಮೃಜ್ಯ ಸಂತೋಷಾತ್ಪಾರ್ವತೀಂ ಪ್ರತ್ಯಭಾಷತ ॥೭॥

ಮಯೇದಾನೀಂ ತ್ವಯಾ ಸಾರ್ಧಂ ವೃಷಮಾರುಹ್ಯ ಗಮ್ಯತೇ ।\\
ಇತ್ಯುಕ್ತ್ವಾ ವೃಷಮಾರುಹ್ಯ ಪಾರ್ವತ್ಯಾ ಸಹ ಶಂಕರಃ ॥೮॥

ಯಯೌ ಭೂಮಂಡಲಂ ದ್ರಷ್ಟುಂ ಗೌರ್ಯಾಶ್ಚಿತ್ರಾಣಿ ದರ್ಶಯನ್ ।\\
ಕ್ವಚಿತ್ ವಿಂಧ್ಯಾಚಲಪ್ರಾಂತೇ ಮಹಾರಣ್ಯೇ ಸುದುರ್ಗಮೇ ॥೯॥

ತತ್ರ ವ್ಯಾಹರ್ತುಮಾಯಾಂತಂ ಭಿಲ್ಲಂ ಪರಶುಧಾರಿಣಂ ।\\
ವರ್ಧ್ಯಮಾನಂ ಮಹಾವ್ಯಾಘ್ರಂ ನಖದಂಷ್ಟ್ರಾಭಿರಾವೃತಂ ॥೧೦॥

ಅತೀವ ಚಿತ್ರಚಾರಿತ್ರ್ಯಂ ವಜ್ರಕಾಯಸಮಾಯುತಂ ।\\
ಅಪ್ರಯತ್ನಮನಾಯಾಸಮಖಿಲಂ ಸುಖಮಾಸ್ಥಿತಂ ॥೧೧॥

ಪಲಾಯಂತಂ ಮೃಗಂ ಪಶ್ಚಾದ್ವ್ಯಾಘ್ರೋ ಭೀತ್ಯಾ ಪಲಾಯಿತಃ ।\\
ಏತದಾಶ್ಚರ್ಯಮಾಲೋಕ್ಯ ಪಾರ್ವತೀ ಪ್ರಾಹ ಶಂಕರಂ ॥೧೨॥

ಶ್ರೀಪಾರ್ವತ್ಯುವಾಚ ।\\
ಕಿಮಾಶ್ಚರ್ಯಂ ಕಿಮಾಶ್ಚರ್ಯಮಗ್ರೇ ಶಂಭೋ ನಿರೀಕ್ಷ್ಯತಾಂ ।\\
ಇತ್ಯುಕ್ತಃ ಸ ತತಃ ಶಂಭುರ್ದೃಷ್ಟ್ವಾ ಪ್ರಾಹ ಪುರಾಣವಿತ್ ॥೧೩॥

ಶ್ರೀಶಂಕರ ಉವಾಚ ।\\
ಗೌರಿ ವಕ್ಷ್ಯಾಮಿ ತೇ ಚಿತ್ರಮವಾಙ್ಮಾನಸಗೋಚರಂ ।\\
ಅದೃಷ್ಟಪೂರ್ವಮಸ್ಮಾಭಿರ್ನಾಸ್ತಿ ಕಿಂಚನ್ನ ಕುತ್ರಚಿತ್ ॥೧೪॥

ಮಯಾ ಸಮ್ಯಕ್ ಸಮಾಸೇನ ವಕ್ಷ್ಯತೇ ಶೃಣು ಪಾರ್ವತಿ ।\\
ಅಯಂ ದೂರಶ್ರವಾ ನಾಮ ಭಿಲ್ಲಃ ಪರಮಧಾರ್ಮಿಕಃ ॥೧೫॥

ಸಮಿತ್ಕುಶಪ್ರಸೂನಾನಿ ಕಂದಮೂಲಫಲಾದಿಕಂ ।\\
ಪ್ರತ್ಯಹಂ ವಿಪಿನಂ ಗತ್ವಾ ಸಮಾದಾಯ ಪ್ರಯಾಸತಃ ॥೧೬॥

ಪ್ರಿಯೇ ಪೂರ್ವಂ ಮುನೀಂದ್ರೇಭ್ಯಃ ಪ್ರಯಚ್ಛತಿ ನ ವಾಂಛತಿ ।\\
ತೇಽಪಿ ತಸ್ಮಿನ್ನಪಿ ದಯಾಂ ಕುರ್ವತೇ ಸರ್ವಮೌನಿನಃ ॥೧೭॥

ದಲಾದನೋ ಮಹಾಯೋಗೀ ವಸನ್ನೇವ ನಿಜಾಶ್ರಮೇ ।\\
ಕದಾಚಿದಸ್ಮರತ್ ಸಿದ್ಧಂ ದತ್ತಾತ್ರೇಯಂ ದಿಗಂಬರಂ ॥೧೮॥

ದತ್ತಾತ್ರೇಯಃ ಸ್ಮರ್ತೃಗಾಮೀ ಚೇತಿಹಾಸಂ ಪರೀಕ್ಷಿತುಂ ।\\
ತತ್ಕ್ಷಣಾತ್ಸೋಽಪಿ ಯೋಗೀಂದ್ರೋ ದತ್ತಾತ್ರೇಯಃ ಸಮುತ್ಥಿತಃ ॥೧೯॥

ತಂ ದೃಷ್ಟ್ವಾಽಽಶ್ಚರ್ಯತೋಷಾಭ್ಯಾಂ ದಲಾದನಮಹಾಮುನಿಃ ।\\
ಸಂಪೂಜ್ಯಾಗ್ರೇ ನಿಷೀದಂತಂ ದತ್ತಾತ್ರೇಯಮುವಾಚ ತಂ ॥೨೦॥

ಮಯೋಪಹೂತಃ ಸಂಪ್ರಾಪ್ತೋ ದತ್ತಾತ್ರೇಯ ಮಹಾಮುನೇ ।\\
ಸ್ಮರ್ತೃಗಾಮೀ ತ್ವಮಿತ್ಯೇತತ್ ಕಿಂವದಂತೀ ಪರೀಕ್ಷಿತುಂ ॥೨೧॥

ಮಯಾದ್ಯ ಸಂಸ್ಮೃತೋಽಸಿ ತ್ವಮಪರಾಧಂ ಕ್ಷಮಸ್ವ ಮೇ ।\\
ದತ್ತಾತ್ರೇಯೋ ಮುನಿಂ ಪ್ರಾಹ ಮಮ ಪ್ರಕೃತಿರೀದೃಶೀ ॥೨೨॥

ಅಭಕ್ತ್ಯಾ ವಾ ಸುಭಕ್ತ್ಯಾ ವಾ ಯಃ ಸ್ಮರೇನ್ಮಾಮನನ್ಯಧೀಃ ।\\
ತದಾನೀಂ ತಮುಪಾಗತ್ಯ ದದಾಮಿ ತದಭೀಪ್ಸಿತಂ ॥೨೩॥

ದತ್ತಾತ್ರೇಯೋ ಮುನಿಂ ಪ್ರಾಹ ದಲಾದನಮುನೀಶ್ವರಂ ।\\
ಯದಿಷ್ಟಂ ತದ್ವೃಣೀಷ್ವ ತ್ವಂ ಯತ್ ಪ್ರಾಪ್ತೋಽಹಂ ತ್ವಯಾ ಸ್ಮೃತಃ ॥೨೪॥

ದತ್ತಾತ್ರೇಯಂ ಮುನಿಃ ಪ್ರಾಹ ಮಯಾ ಕಿಮಪಿ ನೋಚ್ಯತೇ ।\\
ತ್ವಚ್ಚಿತ್ತೇ ಯತ್ಸ್ಥಿತಂ ತನ್ಮೇ ಪ್ರಯಚ್ಛ ಮುನಿಪುಂಗವ ॥೨೫॥

ಶ್ರೀದತ್ತಾತ್ರೇಯ ಉವಾಚ ।\\
ಮಮಾಸ್ತಿ ವಜ್ರಕವಚಂ ಗೃಹಾಣೇತ್ಯವದನ್ಮುನಿಂ ।\\
ತಥ್ಯೇತ್ಯಂಗೀಕೃತವತೇ ದಲಾದನಮುನಯೇ ಮುನಿಃ ॥೨೬॥

ಸ್ವವಜ್ರಕವಚಂ ಪ್ರಾಹ ಋಷಿಚ್ಛಂದಃಪುರಃಸರಂ ।\\
ನ್ಯಾಸಂ ಧ್ಯಾನಂ ಫಲಂ ತತ್ರ ಪ್ರಯೋಜನಮಶೇಷತಃ ॥೨೭॥

ಅಸ್ಯ ಶ್ರೀದತ್ತಾತ್ರೇಯವಜ್ರಕವಚಸ್ತೋತ್ರಮಂತ್ರಸ್ಯ । ಕಿರಾತರೂಪೀ ಮಹಾರುದ್ರ ಋಷಿಃ । ಅನುಷ್ಟುಪ್ ಛಂದಃ । ಶ್ರೀದತ್ತಾತ್ರೇಯೋ ದೇವತಾ । ದ್ರಾಂ ಬೀಜಂ । ಆಂ ಶಕ್ತಿಃ । ಕ್ರೌಂ ಕೀಲಕಂ । ಓಂ ಆತ್ಮನೇ ನಮಃ ಓಂ ದ್ರೀಂ ಮನಸೇ ನಮಃ । ಓಂ ಆಂ ದ್ರೀಂ ಶ್ರೀಂ ಸೌಃ ಓಂ ಕ್ಲಾಂ ಕ್ಲೀಂ ಕ್ಲೂಂ ಕ್ಲೈಂ ಕ್ಲೌಂ ಕ್ಲಃ ।  ಶ್ರೀದತ್ತಾತ್ರೇಯಪ್ರಸಾದಸಿದ್ಧ್ಯರ್ಥೇ ಜಪೇ ವಿನಿಯೋಗಃ ।\\
ಓಂ ದ್ರಾಂ ದ್ರೀಂ ಇತ್ಯಾದಿನಾ ನ್ಯಾಸಃ ।\\
\dhyana{ಜಗದಂಕುರಕಂದಾಯ ಸಚ್ಚಿದಾನಂದಮೂರ್ತಯೇ ।\\
ದತ್ತಾತ್ರೇಯಾಯ ಯೋಗೀಂದ್ರಚಂದ್ರಾಯ ಪರಮಾತ್ಮನೇ ॥

ಕದಾ ಯೋಗೀ ಕದಾ ಭೋಗೀ ಕದಾ ನಗ್ನಃ ಪಿಶಾಚವತ್ ।\\
ದತ್ತಾತ್ರೇಯೋ ಹರಿಃ ಸಾಕ್ಷಾದ್ಭುಕ್ತಿಮುಕ್ತಿಪ್ರದಾಯಕಃ ॥

ವಾರಾಣಸೀಪುರಸ್ನಾಯೀ ಕೋಲ್ಹಾಪುರಜಪಾದರಃ ।\\
ಮಾಹುರೀಪುರಭಿಕ್ಷಾಶೀ ಸಹ್ಯಶಾಯೀ ದಿಗಂಬರ ॥

ಇಂದ್ರನೀಲಸಮಾಕಾರಶ್ಚಂದ್ರಕಾಂತಿಸಮದ್ಯುತಿಃ ।\\
ವೈಡುರ್ಯಸದೃಶಸ್ಫೂರ್ತಿಶ್ಚಲತ್ಕಿಂಚಿಜ್ಜಟಾಧರಃ ॥

ಸ್ನಿಗ್ಧಧಾವಲ್ಯಯುಕ್ತಾಕ್ಷೋಽತ್ಯಂತನೀಲಕನೀನಿಕಃ ।\\
ಭ್ರೂವಕ್ಷಃಶ್ಮಶ್ರುನೀಲಾಂಕಃ ಸಶಾಂಕಸದೃಶಾನನಃ ॥

ಹಾಸನಿರ್ಜಿತನೀಹಾರಃ ಕಂಠನಿರ್ಜಿತಕಂಬುಕಃ ।\\
ಮಾಂಸಲಾಂಸೋ ದೀರ್ಘಬಾಹುಃ ಪಾಣಿನಿರ್ಜಿತಪಲ್ಲವಃ ॥

ವಿಶಾಲಪೀನವಕ್ಷಾಶ್ಚ ತಾಮ್ರಪಾಣಿರ್ದಲೋದರಃ ।\\
ಪೃಥುಲಶ್ರೋಣಿಲಲಿತೋ ವಿಶಾಲಜಘನಸ್ಥಲಃ ॥

ರಂಭಾಸ್ತಂಭೋಪಮಾನೋರುರ್ಜಾನುಪೂರ್ವೈಕಜಂಘಕಃ ।\\
ಗೂಢಗುಲ್ಫಃ ಕೂರ್ಮಪೃಷ್ಠಲಸತ್ಪಾದೋಪರಿಸ್ಥಲಃ ॥

ರಕ್ತಾರವಿಂದಸದೃಶರಮಣೀಯಪದಾಧರಃ ।\\
ಚರ್ಮಾಂಬರಧರೋ ಯೋಗೀ ಸ್ಮರ್ತೃಗಾಮೀ ಕ್ಷಣೇ ಕ್ಷಣೇ ॥

ಜ್ಞಾನೋಪದೇಶನೀರತೋ ವಿಪದ್ಧರಣದೀಕ್ಷಿತಃ ।\\
ಸಿದ್ಧಾಸನಸಮಾಸೀನ ಋಜುಕಾಯೋ ಹಸನ್ಮುಖಃ ॥

ವಾಮಹಸ್ತೇನ ವರದೋ ದಕ್ಷಿಣೇನಾಭಯಂಕರಃ ।\\
ಬಾಲೋನ್ಮತ್ತಪಿಶಾಚೀಭಿಃ ಕ್ವಚಿದ್ಯುಕ್ತಃ ಪರೀಕ್ಷಿತಃ ॥

ತ್ಯಾಗೀ ಭೋಗೀ ಮಹಾಯೋಗೀ ನಿತ್ಯಾನಂದೋ ನಿರಂಜನಃ ।\\
ಸರ್ವರೂಪೀ ಸರ್ವದಾತಾ ಸರ್ವಗಃ ಸರ್ವಕಾಮದಃ ॥

ಭಸ್ಮೋದ್ಧೂಲಿತಸರ್ವಾಂಗೋ ಮಹಾಪಾತಕನಾಶನಃ ।\\
ಭುಕ್ತಿಪ್ರದೋ ಮುಕ್ತಿದಾತಾ ಜೀವನ್ಮುಕ್ತೋ ನ ಸಂಶಯಃ ॥

ಏವಂ ಧ್ಯಾತ್ವಾಽನನ್ಯಚಿತ್ತೋ ಮದ್ವಜ್ರಕವಚಂ ಪಠೇತ್ ।\\
ಮಾಮೇವ ಪಶ್ಯನ್ಸರ್ವತ್ರ ಸ ಮಯಾ ಸಹ ಸಙ್ಚರೇತ್ ॥

ದಿಗಂಬರಂ ಭಸ್ಮಸುಗಂಧಲೇಪನಂ ಚಕ್ರಂ ತ್ರಿಶೂಲಂ ಡಮರುಂ ಗದಾಯುಧಂ ।\\
ಪದ್ಮಾಸನಂ ಯೋಗಿಮುನೀಂದ್ರವಂದಿತಂ ದತ್ತೇತಿ ನಾಮಸ್ಮರಣೇನ ನಿತ್ಯಂ ॥}

ಓಂ ದ್ರಾಂ ।\\
ಓಂ ದತ್ತಾತ್ರೇಯಃ ಶಿರಃ ಪಾತು ಸಹಸ್ರಾಬ್ಜೇಷು ಸಂಸ್ಥಿತಃ ।\\
ಭಾಲಂ ಪಾತ್ವಾನಸೂಯೇಯಶ್ಚಂದ್ರಮಂಡಲಮಧ್ಯಗಃ ॥೨೮॥

ಕೂರ್ಚಂ ಮನೋಮಯಃ ಪಾತು ಹಂ ಕ್ಷಂ ದ್ವಿದಲಪದ್ಮಭೂಃ ।\\
ಜ್ಯೋತಿರೂಪೋಽಕ್ಷಿಣೀ ಪಾತು ಪಾತು ಶಬ್ದಾತ್ಮಕಃ ಶ್ರುತೀ ॥೨೯॥

ನಾಸಿಕಾಂ ಪಾತು ಗಂಧಾತ್ಮಾ ಮುಖಂ ಪಾತು ರಸಾತ್ಮಕಃ ।\\
ಜಿಹ್ವಾಂ ವೇದಾತ್ಮಕಃ ಪಾತು ದಂತೋಷ್ಠೌ ಪಾತು ಧಾರ್ಮಿಕಃ ॥೩೦॥

ಕಪೋಲಾವತ್ರಿಭೂಃ ಪಾತು ಪಾತ್ವಶೇಷಂ ಮಮಾತ್ವವಿತ್ ।\\
ಸ್ವರಾತ್ಮಾ ಷೋಡಶಾರಾಬ್ಜಸ್ಥಿತಃ ಸ್ವಾತ್ಮಾಽವತಾದ್ಗಲಂ ॥೩೧॥

ಸ್ಕಂಧೌ ಚಂದ್ರಾನುಜಃ ಪಾತು ಭುಜೌ ಪಾತು ಕೃತಾದಿಭೂಃ ।\\
ಜತ್ರುಣೀ ಶತ್ರುಜಿತ್ ಪಾತು ಪಾತು ವಕ್ಷಃಸ್ಥಲಂ ಹರಿಃ ॥೩೨॥

ಕಾದಿಠಾಂತದ್ವಾದಶಾರಪದ್ಮಗೋ ಮರುದಾತ್ಮಕಃ ।\\
ಯೋಗೀಶ್ವರೇಶ್ವರಃ ಪಾತು ಹೃದಯಂ ಹೃದಯಸ್ಥಿತಃ ॥೩೩॥

ಪಾರ್ಶ್ವೇ ಹರಿಃ ಪಾರ್ಶ್ವವರ್ತೀ ಪಾತು ಪಾರ್ಶ್ವಸ್ಥಿತಃ ಸ್ಮೃತಃ ।\\
ಹಠಯೋಗಾದಿಯೋಗಜ್ಞಃ ಕುಕ್ಷೀ ಪಾತು ಕೃಪಾನಿಧಿಃ ॥೩೪॥

ಡಕಾರಾದಿಫಕಾರಾಂತದಶಾರಸರಸೀರುಹೇ ।\\
ನಾಭಿಸ್ಥಲೇ ವರ್ತಮಾನೋ ನಾಭಿಂ ವಹ್ನ್ಯಾತ್ಮಕೋಽವತು ॥೩೫॥

ವಹ್ನಿತತ್ವಮಯೋ ಯೋಗೀ ರಕ್ಷತಾನ್ಮಣಿಪೂರಕಂ ।\\
ಕಟಿಂ ಕಟಿಸ್ಥಬ್ರಹ್ಮಾಂಡವಾಸುದೇವಾತ್ಮಕೋಽವತು ॥೩೬॥

ವಕಾರಾದಿಲಕಾರಾಂತಷಟ್ಪತ್ರಾಂಬುಜಬೋಧಕಃ ।\\
ಜಲತತ್ವಮಯೋ ಯೋಗೀ ಸ್ವಾಧಿಷ್ಠಾನಂ ಮಮಾವತು ॥೩೭॥

ಸಿದ್ಧಾಸನಸಮಾಸೀನ ಊರೂ ಸಿದ್ಧೇಶ್ವರೋಽವತು ।\\
ವಾದಿಸಾಂತಚತುಷ್ಪತ್ರಸರೋರುಹನಿಬೋಧಕಃ ॥೩೮॥

ಮೂಲಾಧಾರಂ ಮಹೀರೂಪೋ ರಕ್ಷತಾದ್ವೀರ್ಯನಿಗ್ರಹೀ ।\\
ಪೃಷ್ಠಂ ಚ ಸರ್ವತಃ ಪಾತು ಜಾನ್ಯುನ್ಯಸ್ತಕರಾಂಬುಜಃ ॥೩೯॥

ಜಂಘೇ ಪಾತ್ವವಧೂತೇಂದ್ರಃ ಪಾತ್ವಂಘ್ರೀ ತೀರ್ಥಪಾವನಃ ।\\
ಸರ್ವಾಂಗಂ ಪಾತು ಸರ್ವಾತ್ಮಾ ರೋಮಾಣ್ಯವತು ಕೇಶವಃ ॥೪೦॥

ಚರ್ಮ ಚರ್ಮಾಂಬರಃ ಪಾತು ರಕ್ತಂ ಭಕ್ತಿಪ್ರಿಯೋಽವತು ।\\
ಮಾಂಸಂ ಮಾಂಸಕರಃ ಪಾತು ಮಜ್ಜಾಂ ಮಜ್ಜಾತ್ಮಕೋಽವತು ॥೪೧॥

ಅಸ್ಥೀನಿ ಸ್ಥಿರಧೀಃ ಪಾಯಾನ್ಮೇಧಾಂ ವೇಧಾಃ ಪ್ರಪಾಲಯೇತ್ ।\\
ಶುಕ್ರಂ ಸುಖಕರಃ ಪಾತು ಚಿತ್ತಂ ಪಾತು ದೃಢಾಕೃತಿಃ ॥೪೨॥

ಮನೋಬುದ್ಧಿಮಹಂಕಾರಂ ಹೃಷೀಕೇಶಾತ್ಮಕೋಽವತು ।\\
ಕರ್ಮೇಂದ್ರಿಯಾಣಿ ಪಾತ್ವೀಶಃ ಪಾತು ಜ್ಞಾನೇಂದ್ರಿಯಾಣ್ಯಜಃ ॥೪೩॥

ಬಂಧೂನ್ ಬಂಧೂತ್ತಮಃ ಪಾಯಾಚ್ಛತ್ರುಭ್ಯಃ ಪಾತು ಶತ್ರುಜಿತ್ ।\\
ಗೃಹಾರಾಮಧನಕ್ಷೇತ್ರಪುತ್ರಾದೀಂಛಂಕರೋಽವತು ॥೪೪॥

ಭಾರ್ಯಾಂ ಪ್ರಕೃತಿವಿತ್ ಪಾತು ಪಶ್ವಾದೀನ್ಪಾತು ಶಾರ್ಙ್ಗಭೃತ್ ।\\
ಪ್ರಾಣಾನ್ಪಾತು ಪ್ರಧಾನಜ್ಞೋ ಭಕ್ಷ್ಯಾದೀನ್ಪಾತು ಭಾಸ್ಕರಃ ॥೪೫॥

ಸುಖಂ ಚಂದ್ರಾತ್ಮಕಃ ಪಾತು ದುಃಖಾತ್ ಪಾತು ಪುರಾಂತಕಃ ।\\
ಪಶೂನ್ಪಶೂಪತಿಃ ಪಾತು ಭೂತಿಂ ಭೂತೇಶ್ವರೋ ಮಮ ॥೪೬॥

ಪ್ರಾಚ್ಯಾಂ ವಿಷಹರಃ ಪಾತು ಪಾತ್ವಾಗ್ನೇಯ್ಯಾಂ ಮಖಾತ್ಮಕಃ ।\\
ಯಾಮ್ಯಾಂ ಧರ್ಮಾತ್ಮಕಃ ಪಾತು ನೈಋತ್ಯಾಂ ಸರ್ವವೈರಿಹೃತ್ ॥೪೭॥

ವರಾಹಃ ಪಾತು ವಾರುಣ್ಯಾಂ ವಾಯವ್ಯಾಂ ಪ್ರಾಣದೋಽವತು ।\\
ಕೌಬೇರ್ಯಾಂ ಧನದಃ ಪಾತು ಪಾತ್ವೈಶಾನ್ಯಾಂ ಮಹಾಗುರುಃ ॥೪೮॥

ಉರ್ಧ್ವಂ ಪಾತು ಮಹಾಸಿದ್ಧಃ ಪಾತ್ವಧಸ್ತಾಜ್ಜಟಾಧರಃ ।\\
ರಕ್ಷಾಹೀನಂ ತು ಯತ್ಸ್ಥಾನಂ ರಕ್ಷತ್ವಾದಿಮುನೀಶ್ವರಃ ॥೪೯॥

ಏತನ್ಮೇ ವಜ್ರಕವಚಂ ಯಃ ಪಠೇಚ್ಛೃಣುಯಾದಪಿ ।\\
ವಜ್ರಕಾಯಶ್ಚಿರಂಜೀವೀ ದತ್ತಾತ್ರೇಯೋಽಹಮಬ್ರುವಂ ॥೫೦॥

ತ್ಯಾಗೀ ಭೋಗೀ ಮಹಾಯೋಗೀ ಸುಖದುಃಖವಿವರ್ಜಿತಃ ।\\
ಸರ್ವತ್ರಸಿದ್ಧಸಂಕಲ್ಪೋ ಜೀವನ್ಮುಕ್ತೋಽದ್ಯ ವರ್ತತೇ ॥೫೨॥

ಇತ್ಯುಕ್ತ್ವಾಂತರ್ದಧೇ ಯೋಗೀ ದತ್ತಾತ್ರೇಯೋ ದಿಗಂಬರಃ ।\\
ದಲಾದನೋಽಪಿ ತಜ್ಜಪ್ತ್ವಾ ಜೀವನ್ಮುಕ್ತಃ ಸ ವರ್ತತೇ ॥೫೩॥

ಭಿಲ್ಲೋ ದೂರಶ್ರವಾ ನಾಮ ತದಾನೀಂ ಶ್ರುತವಾದಿನಂ ।\\
ಸಕೃಚ್ಛ್ರವಣಮಾತ್ರೇಣ ವಜ್ರಾಂಗೋಽಭವದಪ್ಯಸೌ ॥೫೪॥

ಇತ್ಯೇತದ್ವಜ್ರಕವಚಂ ದತ್ತಾತ್ರೇಯಸ್ಯ ಯೋಗಿನಃ ।\\
ಶ್ರುತ್ವಾಶೇಷಂ ಶಂಭುಮುಖಾತ್ ಪುನರಪ್ಯಾಹ ಪಾರ್ವತೀ ॥೫೫॥

ಪಾರ್ವತ್ಯುವಾಚ ।\\
ಏತತ್ಕವಚಮಾಹಾತ್ಮ್ಯಂ ವದ ವಿಸ್ತರತೋ ಮಮ ।\\
ಕುತ್ರ ಕೇನ ಕದಾ ಜಾಪ್ಯಂ ಕಿಂ ಯಜಾಪ್ಯಂ ಕಥಂ ಕಥಂ ॥೫೬॥

ಉವಾಚ ಶಂಭುಸ್ತತ್ಸರ್ವಂ ಪಾರ್ವತ್ಯಾ ವಿನಯೋದಿತಂ ।\\
ಶ್ರೀಶಿವ ಉವಾಚ ।\\
ಶ್ರುಣು ಪಾರ್ವತಿ ವಕ್ಷ್ಯಾಮಿ ಸಮಾಹಿತಮನಾವಿಲಂ ॥೫೭॥

ಧರ್ಮಾರ್ಥಕಾಮಮೋಕ್ಷಾಣಾಮಿದಮೇವ ಪರಾಯಣಂ ।\\
ಹಸ್ತ್ಯಶ್ವರಥಪಾದಾತಿಸರ್ವೈಶ್ವರ್ಯಪ್ರದಾಯಕಂ ॥೫೮॥

ಪುತ್ರಮಿತ್ರಕಲತ್ರಾದಿಸರ್ವಸಂತೋಷಸಾಧನಂ ।\\
ವೇದಶಾಸ್ತ್ರಾದಿವಿದ್ಯಾನಾಂ ನಿಧಾನಂ ಪರಮಂ ಹಿ ತತ್ ॥೫೯॥

ಸಂಗೀತಶಾಸ್ತ್ರಸಾಹಿತ್ಯಸತ್ಕವಿತ್ವವಿಧಾಯಕಂ ।\\
ಬುದ್ಧಿವಿದ್ಯಾಸ್ಮೃತಿಪ್ರಜ್ಞಾಮತಿಪ್ರೌಢಿಪ್ರದಾಯಕಂ ॥೬೦॥

ಸರ್ವಸಂತೋಷಕರಣಂ ಸರ್ವದುಃಖನಿವಾರಣಂ ।\\
ಶತ್ರು ಸಂಹಾರಕಂ ಶೀಘ್ರಂ ಯಶಃಕೀರ್ತಿವಿವರ್ಧನಂ ॥೬೧॥

ಅಷ್ಟಸಂಖ್ಯಾ ಮಹಾರೋಗಾಃ ಸನ್ನಿಪಾತಾಸ್ತ್ರಯೋದಶ ।\\
ಷಣ್ಣವತ್ಯಕ್ಷಿರೋಗಾಶ್ಚ ವಿಂಶತಿರ್ಮೇಹರೋಗಕಾಃ ॥೬೨॥

ಅಷ್ಟಾದಶ ತು ಕುಷ್ಠಾನಿ ಗುಲ್ಮಾನ್ಯಷ್ಟವಿಧಾನ್ಯಪಿ ।\\
ಅಶೀತಿರ್ವಾತರೋಗಾಶ್ಚ ಚತ್ವಾರಿಶತ್ತು ಪೈತ್ತಿಕಾಃ ॥೬೩॥

ವಿಂಶತಿಶ್ಲೇಷ್ಮರೋಗಾಶ್ಚ ಕ್ಷಯಚಾತುರ್ಥಿಕಾದಯಃ ।\\
ಮಂತ್ರಯಂತ್ರಕುಯೋಗಾದ್ಯಾಃ ಕಲ್ಪತಂತ್ರಾದಿನಿರ್ಮಿತಾಃ ॥೬೪॥

ಬ್ರಹ್ಮರಾಕ್ಷಸವೇತಾಲಕೂಷ್ಮಾಂಡಾದಿಗ್ರಹೋದ್ಭವಾಃ ।\\
ಸಂಗಜಾಃ ದೇಶಕಾಲಸ್ಥಾಸ್ತಾಪತ್ರಯಸಮುತ್ಥಿತಾಃ ॥೬೫॥

ನವಗ್ರಹಸಮುದ್ಭೂತಾ ಮಹಾಪಾತಕಸಂಭವಾಃ ।\\
ಸರ್ವೇ ರೋಗಾಃ ಪ್ರಣಶ್ಯಂತಿ ಸಹಸ್ರಾವರ್ತನಾದ್ಧ್ರುವಂ ॥೬೬॥

ಅಯುತಾವೃತ್ತಿಮಾತ್ರೇಣ ವಂಧ್ಯಾ ಪುತ್ರವತೀ ಭವೇತ್ ।\\
ಅಯುತದ್ವಿತಯಾವೃತ್ತ್ಯಾ ಹ್ಯಪಮೃತ್ಯುಜಯೋ ಭವೇತ್ ॥೬೭॥

ಅಯುತತ್ರಿತಯಾಚ್ಚೈವ ಖೇಚರತ್ವಂ ಪ್ರಜಾಯತೇ ।\\
ಸಹಸ್ರಾದಯುತಾದರ್ವಾಕ್ ಸರ್ವಕಾರ್ಯಾಣಿ ಸಾಧಯೇತ್ ॥೬೮॥

ಲಕ್ಷಾವೃತ್ಯಾ ಕಾರ್ಯಸಿದ್ಧಿರ್ಭವೇತ್ಯೇವ ನ ಸಂಶಯಃ ।\\
ವಿಷವೃಕ್ಷಸ್ಯ ಮೂಲೇಷು ತಿಷ್ಠನ್ ವೈ ದಕ್ಷಿಣಾಮುಖಃ ॥೬೯॥

ಕುರುತೇ ಮಾಸಮಾತ್ರೇಣ ವೈರಿಣಂ ವಿಕಲೇಂದ್ರಿಯಂ ।\\
ಔದುಂಬರತರೋರ್ಮೂಲೇ ವೃದ್ಧಿಕಾಮೇನ ಜಾಪ್ಯತೇ ॥೭೦॥

ಶ್ರೀವಕ್ಷಮೂಲೇ ಶ್ರೀಕಾಮೀ ತಿಂತಿಣೀ ಶಾಂತಿಕರ್ಮಣಿ ।\\
ಓಜಸ್ಕಾಮೋಽಶ್ವತ್ಥಮೂಲೇ ಸ್ತ್ರೀಕಾಮೈಃ ಸಹಕಾರಕೇ ॥೭೧॥

ಜ್ಞಾನಾರ್ಥೀ ತುಲಸೀಮೂಲೇ ಗರ್ಭಗೇಹೇ ಸುತಾರ್ಥಿಭಿಃ ।\\
ಧನಾರ್ಥಿಭಿಸ್ತು ಸುಕ್ಷೇತ್ರೇ ಪಶುಕಾಮೈಸ್ತು ಗೋಷ್ಠಕೇ ॥೭೨॥

ದೇವಾಲಯೇ ಸರ್ವಕಾಮೈಸ್ತತ್ಕಾಲೇ ಸರ್ವದರ್ಶಿತಂ ।\\
ನಾಭಿಮಾತ್ರಜಲೇ ಸ್ಥಿತ್ವಾ ಭಾನುಮಾಲೋಕ್ಯ ಯೋ ಜಪೇತ್ ॥೭೩॥

ಯುದ್ಧೇ ವಾ ಶಾಸ್ತ್ರವಾದೇ ವಾ ಸಹಸ್ರೇಣ ಜಯೋ ಭವೇತ್ ।\\
ಕಂಠಮಾತ್ರೇ ಜಲೇ ಸ್ಥಿತ್ವಾ ಯೋ ರಾತ್ರೌ ಕವಚಂ ಪಠೇತ್ ॥೭೪॥

ಜ್ವರಾಪಸ್ಮಾರಕುಷ್ಠಾದಿತಾಪಜ್ವರನಿವಾರಣಂ ।\\
ಯತ್ರ ಯತ್ಸ್ಯಾತ್ಸ್ಥಿರಂ ಯದ್ಯತ್ಪ್ರಸನ್ನಂ ತನ್ನಿವರ್ತತೇ ॥೭೫॥

ತೇನ ತತ್ರ ಹಿ ಜಪ್ತವ್ಯಂ ತತಃ ಸಿದ್ಧಿರ್ಭವೇದ್ಧ್ರುವಂ ।\\
ಇತ್ಯುಕ್ತ್ವಾನ್ ಚ ಶಿವೋ ಗೌರ್ಯೇ ರಹಸ್ಯಂ ಪರಮಂ ಶುಭಂ ॥೭೬॥

ಯಃ ಪಠೇತ್ ವಜ್ರಕವಚಂ ದತ್ತಾತ್ರೇಯೋಸಮೋ ಭವೇತ್ ।\\
ಏವಂ ಶಿವೇನ ಕಥಿತಂ ಹಿಮವತ್ಸುತಾಯೈ ಪ್ರೋಕ್ತಂ ॥೭೭॥

ದಲಾದಮುನಯೇಽತ್ರಿಸುತೇನಪೂರ್ವಂ ಯಃ ಕೋಽಪಿ ವಜ್ರಕವಚಂ ।\\
ಪಠತೀಹ ಲೋಕೇ ದತ್ತೋಪಮಶ್ಚರತಿ ಯೋಗಿವರಶ್ಚಿರಾಯುಃ ॥೭೮॥

\authorline{॥ಇತಿ ಶ್ರೀರುದ್ರಯಾಮಲೇ ಹಿಮವತ್ಖಂಡೇ ಉಮಾಮಹೇಶ್ವರಸಂವಾದೇ ಶ್ರೀದತ್ತಾತ್ರೇಯವಜ್ರಕವಚಸ್ತೋತ್ರಂ ಸಂಪೂರ್ಣಂ ॥}
%===============================================================================================
\chapter*{\center ॥ ಶ್ರೀಮಹಾಲಕ್ಷ್ಮೀಹೃದಯಂ ॥}
ಅಸ್ಯ ಶ್ರೀಮಹಾಲಕ್ಷ್ಮೀಹೃದಯಸ್ತೋತ್ರಮಹಾಮಂತ್ರಸ್ಯ ಭಾರ್ಗವ ಋಷಿಃ । ಅನುಷ್ಟುಬಾದಿನಾನಾಛಂದಾಂಸಿ । ಆದ್ಯಾದಿಶ್ರೀಮಹಾಲಕ್ಷ್ಮೀರ್ದೇವತಾ । ಶ್ರೀಂ ಬೀಜಂ । ಹ್ರೀಂ ಶಕ್ತಿಃ । ಐಂ ಕೀಲಕಂ । ಶ್ರೀಮಹಾಲಕ್ಷ್ಮೀಪ್ರಸಾದಸಿದ್ಧ್ಯರ್ಥೇ ಜಪೇ ವಿನಿಯೋಗಃ ॥\\
ಶ್ರಾಂ ಇತ್ಯಾದಿನಾ ನ್ಯಾಸಃ॥

\dhyana{ಪೀತವಸ್ತ್ರಾಂ ಸುವರ್ಣಾಂಗೀಂ ಪದ್ಮಹಸ್ತಾಂ ಗದಾನ್ವಿತಾಂ ।\\
ಲಕ್ಷ್ಮೀಂ ಧ್ಯಾಯೇತ ಮಂತ್ರೇಣ ಸ ಭವೇತ್ ಪೃಥಿವೀಪತಿಃ ॥\\
ಮಾತುಲುಂಗಂ ಗದಾಂ ಖೇಟಂ ಪಾನಪಾತ್ರಂಚ ಬಿಭ್ರತೀಂ ।\\
ನಾಗಂ ಲಿಂಗಂಚ ಯೋನಿಂಚ ಬಿಭ್ರತೀಂ ಚೈವ ಮೂರ್ಧನಿ ॥\\
ವಿಷ್ಣುಸ್ತುತಿಪರಾಂ ಲಕ್ಷ್ಮೀಂ ಸ್ವರ್ಣವರ್ಣಾಂ ಸ್ತುತಿಪ್ರಿಯಾಂ ।\\
ವರಾಭಯಪ್ರದಾಂ ದೇವೀಂ ವಂದೇ ತಾಂ ಕಮಲೇಕ್ಷಣಾಂ ॥

ಕೌಶೇಯಪೀತವಸನಾಂ ಅರವಿಂದನೇತ್ರಾಂ\\ ಪದ್ಮದ್ವಯಾಭಯವರೋದ್ಯತಪದ್ಮಹಸ್ತಾಂ ।\\
ಉದ್ಯಚ್ಛತಾರ್ಕಸದೃಶೀಂ ಪರಮಾಂಕಸಂಸ್ಥಾಂ \\ಧ್ಯಾಯೇದ್ವಿಧೀಶನತಪಾದಯುಗಾಂ ಜನಿತ್ರೀಂ ॥

	ಯಾ ಸಾ ಪದ್ಮಾಸನಸ್ಥಾ ವಿಪುಲಕಟಿತಟೀ ಪದ್ಮಪತ್ರಾಯತಾಕ್ಷೀ\\
	ಗಂಭೀರಾವರ್ತನಾಭಿಃ ಸ್ತನಭರನಮಿತಾ ಶುಭ್ರವಸ್ತ್ರೋತ್ತರೀಯಾ ।\\
	ಲಕ್ಷ್ಮೀರ್ದಿವ್ಯೈರ್ಗಜೇಂದ್ರೈರ್ಮಣಿಗಣಖಚಿತೈಃ ಸ್ನಾಪಿತಾ ಹೇಮಕುಂಭೈಃ\\
	ನಿತ್ಯಂ ಸಾ ಪದ್ಮಹಸ್ತಾ ಮಮ ವಸತು ಗೃಹೇ ಸರ್ವಮಾಂಗಲ್ಯಯುಕ್ತಾ ॥

ಹಸ್ತದ್ವಯೇನ ಕಮಲೇ ಧಾರಯಂತೀಂ ಸ್ವಲೀಲಯಾ ।\\
ಹಾರನೂಪುರಸಂಯುಕ್ತಾಂ ಲಕ್ಷ್ಮೀಂ ದೇವೀಂ ವಿಚಿಂತಯೇ ॥}\\(ಲಮಿತ್ಯಾದಿನಾ ಸಂಪೂಜ್ಯ ।)

ಶಂಕಚಕ್ರಗದಾಹಸ್ತೇ ಶುಭ್ರವರ್ಣೇ ಶುಭಾನನೇ ।\\
ಮಮ ದೇಹಿ ವರಂ ಲಕ್ಷ್ಮಿ ಸರ್ವಸಿದ್ಧಿಪ್ರದಾಯಿನೀ ॥
	(ಇತಿ ಸಂಪ್ರಾರ್ಥ್ಯ ॥)\\
{\bfseries ಓಂ ಶ್ರೀಂ ಹ್ರೀಂ ಐಂ ಮಹಾಲಕ್ಷ್ಮ್ಯೈ ಕಮಲಧಾರಿಣ್ಯೈ ಸಿಂಹವಾಹಿನ್ಯೈ ಸ್ವಾಹಾ ॥(೧೦೮)}

ವಂದೇ ಲಕ್ಷ್ಮೀಂ ಪ್ರಹಸಿತಮುಖೀಂ ಶುದ್ಧಜಾಂಬೂನದಾಭಾಂ\\
ತೇಜೋರೂಪಾಂ ಕನಕವಸನಾಂ ಸರ್ವಭೂಷೋಜ್ಜ್ವಲಾಂಗೀಂ ।\\
	ಬೀಜಾಪೂರಂ ಕನಕಕಲಶಂ ಹೇಮಪದ್ಮೇ ದಧಾನಾಂ\\
	ಆದ್ಯಾಂ ಶಕ್ತಿಂ ಸಕಲಜನನೀಂ ವಿಷ್ಣುಮಾಂಕಸಂಸ್ಥಾಂ ॥೧॥

ಶ್ರೀಮತ್ಸೌಭಾಗ್ಯಜನನೀಂ ಸ್ತೌಮಿ ಲಕ್ಷ್ಮೀಂ ಸನಾತನೀಂ ।\\
ಸರ್ವಕಾಮಫಲಾವಾಪ್ತಿಸಾಧನೈಕಸುಖಾವಹಾಂ ॥೨॥

	ಓಂ ಶ್ರೀಂ ಹ್ರೀಂ ಐಂ ಲಕ್ಷ್ಮ್ಯೈ ನಮಃ ॥

	ಸ್ಮರಾಮಿ ನಿತ್ಯಂ ದೇವೇಶಿ ತ್ವಯಾ ಪ್ರೇರಿತಮಾನಸಃ ।\\
	ತ್ವದಾಜ್ಞಾಂ ಶಿರಸಾ ಧೃತ್ವಾ ಭಜಾಮಿ ಪರಮೇಶ್ವರೀಂ ॥೩॥

ಸಮಸ್ತಸಂಪತ್ಸುಖದಾಂ ಮಹಾಶ್ರಿಯಂ \\ಸಮಸ್ತಸೌಭಾಗ್ಯಕರೀಂ ಮಹಾಶ್ರಿಯಂ ।\\
ಸಮಸ್ತಕಲ್ಯಾಣಕರೀಂ ಮಹಾಶ್ರಿಯಂ\\ ಭಜಾಮ್ಯಹಂ ಜ್ಞಾನಕರೀಂ ಮಹಾಶ್ರಿಯಂ ॥೪॥

	ವಿಜ್ಞಾನಸಂಪತ್ಸುಖದಾಂ ಸನಾತನೀಂ \\ವಿಚಿತ್ರವಾಗ್ಭೂತಿಕರೀಂ ಮನೋಹರಾಂ ।\\
	ಅನಂತಸಮ್ಮೋದಸುಖಪ್ರದಾಯಿನೀಂ\\ ನಮಾಮ್ಯಹಂ ಭೂತಿಕರೀಂ ಹರಿಪ್ರಿಯಾಂ ॥೫॥

ಸಮಸ್ತಭೂತಾಂತರಸಂಸ್ಥಿತಾ ತ್ವಂ\\ ಸಮಸ್ತಭೋಕ್ತ್ರೀಶ್ಶ್ವರಿ ವಿಶ್ವರೂಪೇ ।\\
ತನ್ನಾಸ್ತಿ ಯತ್ತ್ವದ್ವ್ಯತಿರಿಕ್ತವಸ್ತು \\ತ್ವತ್ಪಾದಪದ್ಮಂ ಪ್ರಣಮಾಮ್ಯಹಂ ಶ್ರೀಃ ॥೬॥

	ದಾರಿದ್ರ್ಯದುಃಖೌಘತಮೋಪಹಂತ್ರಿ\\ ತ್ವತ್ಪಾದಪದ್ಮಂ ಮಯಿ ಸನ್ನಿಧತ್ಸ್ವ ।\\
	ದೀನಾರ್ತಿವಿಚ್ಛೇದನಹೇತುಭೂತೈಃ \\ಕೃಪಾಕಟಾಕ್ಷೈರಭಿಷಿಂಚ ಮಾಂ ಶ್ರೀಃ ॥೭॥

ಅಂಬ ಪ್ರಸೀದ ಕರುಣಾಸುಧಯಾರ್ದ್ರದೃಷ್ಟ್ಯಾ \\ಮಾಂ ತ್ವತ್ಕೃಪಾದ್ರವಿಣಗೇಹಮಿಮಂ ಕುರುಷ್ವ ।\\
ಆಲೋಕಯ ಪ್ರಣತಹೃದ್ಗತಶೋಕಹಂತ್ರಿ \\ತ್ವತ್ಪಾದಪದ್ಮಯುಗಲಂ ಪ್ರಣಮಾಮ್ಯಹಂ ಶ್ರೀಃ ॥೮॥

	ಶಾಂತ್ಯೈ ನಮೋಽಸ್ತು ಶರಣಾಗತರಕ್ಷಣಾಯೈ\\ ಕಾಂತ್ಯೈ ನಮೋಽಸ್ತು ಕಮನೀಯಗುಣಾಶ್ರಯಾಯೈ ।\\
	ಕ್ಷಾಂತ್ಯೈ ನಮೋಽಸ್ತು ದುರಿತಕ್ಷಯಕಾರಣಾಯೈ \\ಧಾತ್ರ್ಯೈ ನಮೋಽಸ್ತು ಧನಧಾನ್ಯಸಮೃದ್ಧಿದಾಯೈ ॥೯॥

ಶಕ್ತ್ಯೈ ನಮೋಽಸ್ತು ಶಶಿಶೇಖರಸಂಸ್ತುತಾಯೈ\\ ರತ್ಯೈ ನಮೋಽಸ್ತು ರಜನೀಕರಸೋದರಾಯೈ ।\\
ಭಕ್ತ್ಯೈ ನಮೋಽಸ್ತು ಭವಸಾಗರತಾರಕಾಯೈ \\ಮತ್ಯೈ ನಮೋಽಸ್ತು ಮಧುಸೂದನವಲ್ಲಭಾಯೈ ॥೧೦॥

	ಲಕ್ಷ್ಮ್ಯೈ ನಮೋಽಸ್ತು ಶುಭಲಕ್ಷಣಲಕ್ಷಿತಾಯೈ \\ಸಿದ್ಧ್ಯೈ ನಮೋಽಸ್ತು ಶಿವಸಿದ್ಧಸುಪೂಜಿತಾಯೈ ।\\
	ಧೃತ್ಯೈ ನಮೋಽಸ್ತ್ವಮಿತದುರ್ಗತಿಭಂಜನಾಯೈ \\ಗತ್ಯೈ ನಮೋಽಸ್ತು ವರಸದ್ಗತಿದಾಯಿಕಾಯೈ ॥೧೧॥

ದೇವ್ಯೈ ನಮೋಽಸ್ತು ದಿವಿ ದೇವಗಣಾರ್ಚಿತಾಯೈ \\ಭೂತ್ಯೈ ನಮೋಽಸ್ತು ಭುವನಾರ್ತಿವಿನಾಶನಾಯೈ ।\\
ಧಾತ್ರ್ಯೈ ನಮೋಽಸ್ತು ಧರಣೀಧರವಲ್ಲಭಾಯೈ \\ಪುಷ್ಟ್ಯೈ ನಮೋಽಸ್ತು ಪುರುಷೋತ್ತಮವಲ್ಲಭಾಯೈ ॥೧೨॥

	ಸುತೀವ್ರದಾರಿದ್ರ್ಯವಿದುಃಖಹಂತ್ರ್ಯೈ \\ನಮೋಽಸ್ತು ತೇ ಸರ್ವಭಯಾಪಹಂತ್ರ್ಯೈ ।\\
	ಶ್ರೀವಿಷ್ಣುವಕ್ಷಃಸ್ಥಲಸಂಸ್ಥಿತಾಯೈ \\ನಮೋ ನಮಃ ಸರ್ವವಿಭೂತಿದಾಯೈ ॥೧೩॥

ಜಯತು ಜಯತು ಲಕ್ಷ್ಮೀರ್ಲಕ್ಷಣಾಲಂಕೃತಾಂಗೀ\\ ಜಯತು ಜಯತು ಪದ್ಮಾ ಪದ್ಮಸದ್ಮಾಭಿವಂದ್ಯಾ ।\\
ಜಯತು ಜಯತು ವಿದ್ಯಾ ವಿಷ್ಣುವಾಮಾಂಕಸಂಸ್ಥಾ \\ಜಯತು ಜಯತು ಸಮ್ಯಕ್ಸರ್ವಸಂಪತ್ಕರೀ ಶ್ರೀಃ ॥೧೪॥

ಜಯತು ಜಯತು ದೇವೀ ದೇವಸಂಘಾಭಿಪೂಜ್ಯಾ \\ಜಯತು ಜಯತು ಭದ್ರಾ ಭಾರ್ಗವೀ ಭಾಗ್ಯರೂಪಾ ।\\
ಜಯತು ಜಯತು ನಿತ್ಯಾ ನಿರ್ಮಲಜ್ಞಾನವೇದ್ಯಾ \\ಜಯತು ಜಯತು ಸತ್ಯಾ ಸರ್ವಭೂತಾಂತರಸ್ಥಾ ॥೧೫॥

	ಜಯತು ಜಯತು ರಮ್ಯಾ ರತ್ನಗರ್ಭಾಂತರಸ್ಥಾ \\ಜಯತು ಜಯತು ಶುದ್ಧಾ ಶುದ್ಧಜಾಂಬೂನದಾಭಾ ।\\
	ಜಯತು ಜಯತು ಕಾಂತಾ ಕಾಂತಿಮದ್ಭಾಸಿತಾಂಗೀ \\ಜಯತು ಜಯತು ಶಾಂತಾ ಶೀಘ್ರಮಾಗಚ್ಛ ಸೌಮ್ಯೇ ॥೧೬॥

ಯಸ್ಯಾಃ ಕಲಾಯಾಃ ಕಮಲೋದ್ಭವಾದ್ಯಾ\\ ರುದ್ರಾಶ್ಚ ಶಕ್ರಪ್ರಮುಖಾಶ್ಚ ದೇವಾಃ ।\\
ಜೀವಂತಿ ಸರ್ವೇಽಪಿ ಸಶಕ್ತಯಸ್ತೇ \\ಪ್ರಭುತ್ವಮಾಪುಃ ಪರಮಾಯುಷಸ್ತೇ ॥೧೭॥

{\bfseries ಪಾದಬೀಜಂ । ಓಂ ಆಂ ಈಂ ಯಂ ಪಂ ಕಂ ಲಂ ಹಂ ॥}

ಲಿಲೇಖ ನಿಟಿಲೇ ವಿಧಿರ್ಮಮ ಲಿಪಿಂ ವಿಸೃಜ್ಯಾಂತರಂ\\
ತ್ವಯಾ ವಿಲಿಖಿತವ್ಯಮೇತದಿತಿ ತತ್ಫಲಪ್ರಾಪ್ತಯೇ ।\\
	ತದಂತರತಲೇ ಸ್ಫುಟಂ ಕಮಲವಾಸಿನಿ ಶ್ರೀರಿಮಾಂ\\
	ಸಮರ್ಪಯ ಸಮುದ್ರಿಕಾಂ ಸಕಲಭಾಗ್ಯಸಂಸೂಚಿಕಾಂ ॥೧೮॥

ತದಿದಂ ತಿಮಿರಂ ಫಾಲೇ ಸ್ಫುಟಂ ಕಮಲವಾಸಿನಿ ।\\
ಶ್ರಿಯಂ ಸಮುದ್ರಿಕಾಂ ದೇಹಿ ಸರ್ವಭಾಗ್ಯಸ್ಯ ಸೂಚಿಕಾಂ ॥೧೯॥

{\bfseries ಮುಖಬೀಜಂ । ಓಂ ಹ್ರಾಂ ಹ್ರೀಂ ಆಂ ವ್ಯಾಂ ಯಂ ಭಾಂ ಸಾಂ ॥}

	ಕಲಯಾ ತೇ ಯಥಾ ದೇವಿ ಜೀವಂತಿ ಸಚರಾಚರಾಃ ।\\
	ತಥಾ ಸಂಪತ್ಕರೇ ಲಕ್ಷ್ಮಿ ಸರ್ವದಾ ಸಂಪ್ರಸೀದ ಮೇ ॥೨೦॥

ಯಥಾ ವಿಷ್ಣುರ್ಧ್ರುವೇ ನಿತ್ಯಂ ಸ್ವಕಲಾಂ ಸನ್ನ್ಯವೇಶಯತ್ ।\\
ತಥೈವ ಸ್ವಕಲಾಂ ಲಕ್ಷ್ಮಿ ಮಯಿ ಸಮ್ಯಕ್ಸಮರ್ಪಯ ॥೨೧॥

	ಸರ್ವಸೌಖ್ಯಪ್ರದೇ ದೇವಿ ಭಕ್ತಾನಾಮಭಯಪ್ರದೇ ।\\
	ಅಚಲಾಂ ಕುರು ಯತ್ನೇನ ಕಲಾಂ ಮಯಿ ನಿವೇಶಿತಾಂ ॥೨೨॥

ಮುದಾಸ್ತಾಂ ಮತ್ಫಾಲೇ ಪರಮಪದಲಕ್ಷ್ಮೀಃ ಸ್ಫುಟಕಲಾ\\
ಸದಾ ವೈಕುಂಠಶ್ರೀರ್ನಿವಸತು ಕಲಾ ಮೇ ನಯನಯೋಃ ।\\
	ವಸೇತ್ಸತ್ಯೇ ಲೋಕೇ ಮಮ ವಚಸಿ ಲಕ್ಷ್ಮೀರ್ವರಕಲಾ\\
	ಶ್ರಿಯಃ ಶ್ವೇತದ್ವೀಪೇ ನಿವಸತು ಕಲಾ ಮೇಽಸ್ತು ಕರಯೋಃ ॥೨೩॥

{\bfseries ನೇತ್ರಬೀಜಂ । ಓಂ ಘ್ರಾಂ ಘ್ರೀಂ ಘ್ರೂಂ ಘ್ರೈಂ ಘ್ರೌಂ ಘ್ರಃ ॥}

ತಾವನ್ನಿತ್ಯಂ ಮಮಾಂಗೇಷು ಕ್ಷೀರಾಬ್ಧೌ ಶ್ರೀಕಲಾ ವಸೇತ್ ।\\
ಸೂರ್ಯಾಚಂದ್ರಮಸೌ ಯಾವದ್ಯಾವಲ್ಲಕ್ಷ್ಮೀಪತಿಃ ಶ್ರಿಯಾ ॥೨೪॥

	ಸರ್ವಮಂಗಲಸಂಪೂರ್ಣಾ ಸರ್ವೈಶ್ವರ್ಯಸಮನ್ವಿತಾ ।\\
	ಆದ್ಯಾದಿಶ್ರೀರ್ಮಹಾಲಕ್ಷ್ಮೀಸ್ತ್ವತ್ಕಲಾ ಮಯಿ ತಿಷ್ಠತು ॥೨೫॥

ಅಜ್ಞಾನತಿಮಿರಂ ಹಂತುಂ ಶುದ್ಧಜ್ಞಾನಪ್ರಕಾಶಿಕಾ ।\\
ಸರ್ವೈಶ್ವರ್ಯಪ್ರದಾ ಮೇಽಸ್ತು ತ್ವತ್ಕಲಾ ಮಯಿ ಸಂಸ್ಥಿತಾ ॥೨೬॥

	ಅಲಕ್ಷ್ಮೀಂ ಹರತು ಕ್ಷಿಪ್ರಂ ತಮಃ ಸೂರ್ಯಪ್ರಭಾ ಯಥಾ ।\\
	ವಿತನೋತು ಮಮ ಶ್ರೇಯಸ್ತ್ವತ್ಕಲಾ ಮಯಿ ಸಂಸ್ಥಿತಾ ॥೨೭॥

ಐಶ್ವರ್ಯಮಂಗಲೋತ್ಪತ್ತಿಃ ತ್ವತ್ಕಲಾಯಾಂ ನಿಧೀಯತೇ ।\\
ಮಯಿ ತಸ್ಮಾತ್ಕೃತಾರ್ಥೋಽಸ್ಮಿ ಪಾತ್ರಮಸ್ಮಿ ಸ್ಥಿತೇಸ್ತವ ॥೨೮॥

	ಭವದಾವೇಶಭಾಗ್ಯಾರ್ಹೋ ಭಾಗ್ಯವಾನಸ್ಮಿ ಭಾರ್ಗವಿ ।\\
	ತ್ವತ್ಪ್ರಸಾದಾತ್ಪವಿತ್ರೋಽಹಂ ಲೋಕಮಾತರ್ನಮೋಽಸ್ತು ತೇ ॥೨೯॥

{\bfseries ಜಿಹ್ವಾಬೀಜಂ । ಓಂ ಹ್ರಾಂ ಹ್ರೀಂ ಹ್ರೂಂ ಹ್ರೈಂ ಹ್ರೌಂ ಹ್ರಃ ॥}

ಪುನಾಸಿ ಮಾಂ ತ್ವಂ ಕಲಯೈವ ಯಸ್ಮಾ\\ದತಃ ಸಮಾಗಚ್ಛ ಮಮಾಗ್ರತಸ್ತ್ವಂ ।\\
ಪರಂ ಪದಂ ಶ್ರೀರ್ಭವ ಸುಪ್ರಸನ್ನಾ \\ಮಯ್ಯಚ್ಯುತೇನ ಪ್ರವಿಶಾಽಽದಿಲಕ್ಷ್ಮೀಃ ॥೩೦॥

	ಶ್ರೀವೈಕುಂಠಸ್ಥಿತೇ ಲಕ್ಷ್ಮಿ ಸಮಾಗಚ್ಛ ಮಮಾಗ್ರತಃ ।\\
	ನಾರಾಯಣೇನ ಸಹ ಮಾಂ ಕೃಪಾದೃಷ್ಟ್ಯಾಽವಲೋಕಯ ॥೩೧॥

	ಸತ್ಯಲೋಕಸ್ಥಿತೇ ಲಕ್ಷ್ಮಿ ತ್ವಂ ಮಮಾಗಚ್ಛ ಸನ್ನಿಧಿಂ ।\\
	ವಾಸುದೇವೇನ ಸಹಿತಾ ಪ್ರಸೀದ ವರದಾ ಭವ ॥೩೨॥

ಶ್ವೇತದ್ವೀಪಸ್ಥಿತೇ ಲಕ್ಷ್ಮಿ ಶೀಘ್ರಮಾಗಚ್ಛ ಸುವ್ರತೇ ।\\
ವಿಷ್ಣುನಾ ಸಹಿತಾ ದೇವಿ ಜಗನ್ಮಾತಃ ಪ್ರಸೀದ ಮೇ ॥೩೩॥

	ಕ್ಷೀರಾಂಬುಧಿಸ್ಥಿತೇ ಲಕ್ಷ್ಮಿ ಸಮಾಗಚ್ಛ ಸಮಾಧವೇ ।\\
	ತ್ವತ್ಕೃಪಾದೃಷ್ಟಿಸುಧಯಾ ಸತತಂ ಮಾಂ ವಿಲೋಕಯ ॥೩೪॥

ರತ್ನಗರ್ಭಸ್ಥಿತೇ ಲಕ್ಷ್ಮಿ ಪರಿಪೂರ್ಣಹಿರಣ್ಮಯಿ ।\\
ಸಮಾಗಚ್ಛ ಸಮಾಗಚ್ಛ ಸ್ಥಿತ್ವಾಶು ಪುರತೋ ಮಮ ॥೩೫॥

	ಸ್ಥಿರಾ ಭವ ಮಹಾಲಕ್ಷ್ಮಿ ನಿಶ್ಚಲಾ ಭವ ನಿರ್ಮಲೇ ।\\
	ಪ್ರಸನ್ನೇ ಕಮಲೇ ದೇವಿ ಪ್ರಸನ್ನಹೃದಯಾ ಭವ ॥೩೬॥

ಶ್ರೀಧರೇ ಶ್ರೀಮಹಾಭೂಮೇ ತ್ವದಂತಃಸ್ಥಂ ಮಹಾನಿಧಿಂ ।\\
ಶೀಘ್ರಮುದ್ಧೃತ್ಯ ಪುರತಃ ಪ್ರದರ್ಶಯ ಸಮರ್ಪಯ ॥೩೭॥

	ವಸುಂಧರೇ ಶ್ರೀವಸುಧೇ ವಸುದೋಗ್ಧ್ರಿ ಕೃಪಾಮಯಿ ।\\
	ತ್ವತ್ಕುಕ್ಷಿಗತಸರ್ವಸ್ವಂ ಶೀಘ್ರಂ ಮೇ ಸಂಪ್ರದರ್ಶಯ ॥೩೮॥

ವಿಷ್ಣುಪ್ರಿಯೇ ರತ್ನಗರ್ಭೇ ಸಮಸ್ತಫಲದೇ ಶಿವೇ ।\\
ತ್ವದ್ಗರ್ಭಗತಹೇಮಾದೀನ್ ಸಂಪ್ರದರ್ಶಯ ದರ್ಶಯ ॥೩೯॥

	ರಸಾತಲಗತೇ ಲಕ್ಷ್ಮಿ ಶೀಘ್ರಮಾಗಚ್ಛ ಮೇ ಪುರಃ ।\\
	ನ ಜಾನೇ ಪರಮಂ ರೂಪಂ ಮಾತರ್ಮೇ ಸಂಪ್ರದರ್ಶಯ ॥೪೦॥

ಆವಿರ್ಭವ ಮನೋವೇಗಾತ್ ಶೀಘ್ರಮಾಗಚ್ಛ ಮೇ ಪುರಃ ।\\
ಮಾ ವತ್ಸ ಭೈರಿಹೇತ್ಯುಕ್ತ್ವಾ ಕಾಮಂ ಗೌರಿವ ರಕ್ಷ ಮಾಂ ॥೪೧॥

	ದೇವಿ ಶೀಘ್ರಂ ಮಮಾಗಚ್ಛ ಧರಣೀಗರ್ಭಸಂಸ್ಥಿತೇ ।\\
	ಮಾತಸ್ತ್ವದ್ಭೃತ್ಯಭೃತ್ಯೋಽಹಂ ಮೃಗಯೇ ತ್ವಾಂ ಕುತೂಹಲಾತ್ ॥೪೨॥

ಉತ್ತಿಷ್ಠ ಜಾಗೃಹಿ ತ್ವಂ ಮೇ ಸಮುತ್ತಿಷ್ಠ ಸುಜಾಗೃಹಿ ।\\
ಅಕ್ಷಯಾನ್ ಹೇಮಕಲಶಾನ್ ಸುವರ್ಣೇನ ಸುಪೂರಿತಾನ್ ॥೪೩॥

	ನಿಕ್ಷೇಪಾನ್ಮೇ ಸಮಾಕೃಷ್ಯ ಸಮುದ್ಧೃತ್ಯ ಮಮಾಗ್ರತಃ ।\\
	ಸಮುನ್ನತಾನನಾ ಭೂತ್ವಾ ಸಮಾಧೇಹಿ ಧರಾಂತರಾತ್ ॥೪೪॥

ಮತ್ಸನ್ನಿಧಿಂ ಸಮಾಗಚ್ಛ ಮದಾಹಿತಕೃಪಾರಸಾತ್ ।\\
ಪ್ರಸೀದ ಶ್ರೇಯಸಾಂ ದೋಗ್ಧ್ರಿ ಲಕ್ಷ್ಮಿ ಮೇ ನಯನಾಗ್ರತಃ ॥೪೫॥

	ಅತ್ರೋಪವಿಶ್ಯ ಲಕ್ಷ್ಮಿ ತ್ವಂ ಸ್ಥಿರಾ ಭವ ಹಿರಣ್ಮಯಿ ।\\
	ಸುಸ್ಥಿರಾ ಭವ ಸಂಪ್ರೀತ್ಯಾ ಪ್ರಸೀದ ವರದಾ ಭವ ॥೪೬॥

ಆನೀಯ ತ್ವಂ ತಥಾ ದೇವಿ ನಿಧೀನ್ಮೇ ಸಂಪ್ರದರ್ಶಯ ।\\
ಅದ್ಯ ಕ್ಷಣೇನ ಸಹಸಾ ದತ್ತ್ವಾ ಸಂರಕ್ಷ ಮಾಂ ಸದಾ ॥೪೭॥

	ಮಯಿ ತಿಷ್ಠ ತಥಾ ನಿತ್ಯಂ ಯಥೇಂದ್ರಾದಿಷು ತಿಷ್ಠಸಿ ।\\
	ಅಭಯಂ ಕುರು ಮೇ ದೇವಿ ಮಹಾಲಕ್ಷ್ಮಿ ನಮೋಽಸ್ತು ತೇ ॥೪೮॥

ಸಮಾಗಚ್ಛ ಮಹಾಲಕ್ಷ್ಮಿ ಶುದ್ಧಜಾಂಬೂನದಪ್ರಭೇ ।\\
ಪ್ರಸೀದ ಪುರತಃ ಸ್ಥಿತ್ವಾ ಪ್ರಣತಂ ಮಾಂ ವಿಲೋಕಯ ॥೪೯॥

	ಲಕ್ಷ್ಮೀರ್ಭುವಂ ಗತಾ ಭಾಸಿ ಯತ್ರ ಯತ್ರ ಹಿರಣ್ಮಯಿ ।\\
	ತತ್ರ ತತ್ರ ಸ್ಥಿತಾ ತ್ವಂ ಮೇ ತವ ರೂಪಂ ಪ್ರದರ್ಶಯ ॥೫೦॥

ಕ್ರೀಡಸೇ ಬಹುಧಾ ಭೂಮೌ ಪರಿಪೂರ್ಣಾ ಹಿರಣ್ಮಯೀ ।\\
ಮಮ ಮೂರ್ಧನಿ ತೇ ಹಸ್ತಮವಿಲಂಬಿತಮರ್ಪಯ ॥೫೧॥

ಫಲದ್ಭಾಗ್ಯೋದಯೇ ಲಕ್ಷ್ಮಿ ಸಮಸ್ತಪುರವಾಸಿನಿ ।\\
ಪ್ರಸೀದ ಮೇ ಮಹಾಲಕ್ಷ್ಮಿ ಪರಿಪೂರ್ಣಮನೋರಥೇ ॥೫೨॥

	ಅಯೋಧ್ಯಾದಿಷು ಸರ್ವೇಷು ನಗರೇಷು ಸಮಾಸ್ಥಿತೇ ।\\
	ವೈಭವೈರ್ವಿವಿಧೈರ್ಯುಕ್ತಾ ಸಮಾಗಚ್ಛ ಬಲಾನ್ವಿತೇ ॥೫೩॥

ಸಮಾಗಚ್ಛ ಸಮಾಗಚ್ಛ ಮಮಾಗ್ರೇ ಭವ ಸುಸ್ಥಿರಾ ।\\
ಕರುಣಾರಸನಿಷ್ಯಂದನೇತ್ರದ್ವಯವಿಲಾಸಿನಿ ॥೫೪॥

	ಸನ್ನಿಧತ್ಸ್ವ ಮಹಾಲಕ್ಷ್ಮಿ ತ್ವತ್ಪಾಣಿಂ ಮಮ ಮಸ್ತಕೇ ।\\
	ಕರುಣಾಸುಧಯಾ ಮಾಂ ತ್ವಮಭಿಷಿಂಚ ಸ್ಥಿರೀಕುರು ॥೫೫॥

ಸರ್ವರಾಜಗೃಹೇ ಲಕ್ಷ್ಮಿ ಸಮಾಗಚ್ಛ ಬಲಾನ್ವಿತೇ ।\\
ಸ್ಥಿತ್ವಾಽಽಶು ಪುರತೋ ಮೇಽದ್ಯ ಪ್ರಸಾದೇನಾಭಯಂ ಕುರು ॥೫೬॥

	ಸಾದರಂ ಮಸ್ತಕೇ ಹಸ್ತಂ ಮಮ ತ್ವಂ ಕೃಪಯಾಽರ್ಪಯ ।\\
	ಸರ್ವರಾಜಗೃಹೇ ಲಕ್ಷ್ಮಿ ತ್ವತ್ಕಲಾ ಮಯಿ ತಿಷ್ಠತು ॥೫೭॥

ಆದ್ಯಾದಿಶ್ರೀರ್ಮಹಾಲಕ್ಷ್ಮಿ ವಿಷ್ಣುವಾಮಾಂಕಸಂಸ್ಥಿತೇ ।\\
ಪ್ರತ್ಯಕ್ಷಂ ಕುರು ಮೇ ರೂಪಂ ರಕ್ಷ ಮಾಂ ಶರಣಾಗತಂ ॥೫೮॥

	ಪ್ರಸೀದ ಮೇ ಮಹಾಲಕ್ಷ್ಮಿ ಸುಪ್ರಸೀದ ಮಹಾಶಿವೇ ।\\
	ಅಚಲಾ ಭವ ಸಂಪ್ರೀತ್ಯಾ ಸುಸ್ಥಿರಾ ಭವ ಮದ್ಗೃಹೇ ॥೫೯॥

ಯಾವತ್ತಿಷ್ಠಂತಿ ವೇದಾಶ್ಚ ಯಾವತ್ತ್ವನ್ನಾಮ ತಿಷ್ಠತಿ ।\\
ಯಾವದ್ವಿಷ್ಣುಶ್ಚ ಯಾವತ್ತ್ವಂ ತಾವತ್ಕುರು ಕೃಪಾಂ ಮಯಿ ॥೬೦॥

	ಚಾಂದ್ರೀ ಕಲಾ ಯಥಾ ಶುಕ್ಲೇ ವರ್ಧತೇ ಸಾ ದಿನೇ ದಿನೇ ।\\
	ತಥಾ ದಯಾ ತೇ ಮಯ್ಯೇವ ವರ್ಧತಾಮಭಿವರ್ಧತಾಂ ॥೬೧॥

ಯಥಾ ವೈಕುಂಠನಗರೇ ಯಥಾ ವೈ ಕ್ಷೀರಸಾಗರೇ ।\\
ತಥಾ ಮದ್ಭವನೇ ತಿಷ್ಠ ಸ್ಥಿರಂ ಶ್ರೀವಿಷ್ಣುನಾ ಸಹ ॥೬೨॥

	ಯೋಗಿನಾಂ ಹೃದಯೇ ನಿತ್ಯಂ ಯಥಾ ತಿಷ್ಠಸಿ ವಿಷ್ಣುನಾ ।\\
	ತಥಾ ಮದ್ಭವನೇ ತಿಷ್ಠ ಸ್ಥಿರಂ ಶ್ರೀವಿಷ್ಣುನಾ ಸಹ ॥೬೩॥

ನಾರಾಯಣಸ್ಯ ಹೃದಯೇ ಭವತೀ ಯಥಾಽಽಸ್ತೇ\\ ನಾರಾಯಣೋಽಪಿ ತವ ಹೃತ್ಕಮಲೇ ಯಥಾಽಽಸ್ತೇ ।\\
ನಾರಾಯಣಸ್ತ್ವಮಪಿ ನಿತ್ಯಮುಭೌ ತಥೈವ \\ತೌ ತಿಷ್ಠತಾಂ ಹೃದಿ ಮಮಾಪಿ ದಯಾವತ್ತಿ ಶ್ರೀಃ ॥೬೪॥

	ವಿಜ್ಞಾನವೃದ್ಧಿಂ ಹೃದಯೇ ಕುರು ಶ್ರೀಃ \\ಸೌಭಾಗ್ಯವೃದ್ಧಿಂ ಕುರು ಮೇ ಗೃಹೇ ಶ್ರೀಃ ।\\
	ದಯಾಸುವೃದ್ಧಿಂ ಕುರುತಾಂ ಮಯಿ ಶ್ರೀಃ \\ಸುವರ್ಣವೃದ್ಧಿಂ ಕುರು ಮೇ ಗೃಹೇ ಶ್ರೀಃ ॥೬೫॥

ನ ಮಾಂ ತ್ಯಜೇಥಾಃ ಶ್ರಿತಕಲ್ಪವಲ್ಲಿ \\ಸದ್ಭಕ್ತಚಿಂತಾಮಣಿಕಾಮಧೇನೋ ।\\
ವಿಶ್ವಸ್ಯ ಮಾತರ್ಭವ ಸುಪ್ರಸನ್ನಾ \\ಗೃಹೇ ಕಲತ್ರೇಷು ಚ ಪುತ್ರವರ್ಗೇ ॥೬೬॥

{\bfseries ಕುಕ್ಷಿಬೀಜಂ । ಓಂ ಆಂ ಈಂ ಏಂ ಐಂ ॥}

	ಆದ್ಯಾದಿಮಾಯೇ ತ್ವಮಜಾಂಡಬೀಜಂ\\ ತ್ವಮೇವ ಸಾಕಾರನಿರಾಕೃತಿಸ್ತ್ವಂ ।\\
	ತ್ವಯಾ ಧೃತಾಶ್ಚಾಬ್ಜಭವಾಂಡಸಂಘಾಃ\\ಚಿತ್ರಂ ಚರಿತ್ರಂ ತವ ದೇವಿ ವಿಷ್ಣೋಃ ॥೬೭॥

ಬ್ರಹ್ಮರುದ್ರಾದಯೋ ದೇವಾ ವೇದಾಶ್ಚಾಪಿ ನ ಶಕ್ನುಯುಃ ।\\
ಮಹಿಮಾನಂ ತವ ಸ್ತೋತುಂ ಮಂದೋಽಹಂ ಶಕ್ನುಯಾಂ ಕಥಂ ॥೬೮॥

	ಅಂಬ ತ್ವದ್ವತ್ಸವಾಕ್ಯಾನಿ ಸೂಕ್ತಾಸೂಕ್ತಾನಿ ಯಾನಿ ಚ ।\\
	ತಾನಿ ಸ್ವೀಕುರು ಸರ್ವಜ್ಞೇ ದಯಾಲುತ್ವೇನ ಸಾದರಂ ॥೬೯॥

ಭವತೀಂ ಶರಣಂ ಗತ್ವಾ ಕೃತಾರ್ಥಾಃ ಸ್ಯುಃ ಪುರಾತನಾಃ ।\\
ಇತಿ ಸಂಚಿಂತ್ಯ ಮನಸಾ ತ್ವಾಮಹಂ ಶರಣಂ ವ್ರಜೇ ॥೭೦॥

	ಅನಂತಾ ನಿತ್ಯಸುಖಿನಃ ತ್ವದ್ಭಕ್ತಾಸ್ತ್ವತ್ಪರಾಯಣಾಃ ।\\
	ಇತಿ ವೇದಪ್ರಮಾಣಾದ್ಧಿ ದೇವಿ ತ್ವಾಂ ಶರಣಂ ವ್ರಜೇ ॥೭೧॥

ತವ ಪ್ರತಿಜ್ಞಾ ಮದ್ಭಕ್ತಾ ನ ನಶ್ಯಂತೀತ್ಯಪಿ ಕ್ವಚಿತ್ ।\\
ಇತಿ ಸಂಚಿಂತ್ಯ ಸಂಚಿಂತ್ಯ ಪ್ರಾಣಾನ್ ಸಂಧಾರಯಾಮ್ಯಹಂ ॥೭೨॥

	ತ್ವದಧೀನಸ್ತ್ವಹಂ ಮಾತಃ ತ್ವತ್ಕೃಪಾ ಮಯಿ ವಿದ್ಯತೇ ।\\
	ಯಾವತ್ಸಂಪೂರ್ಣಕಾಮಃ ಸ್ಯಾಂ ತಾವದ್ದೇಹಿ ದಯಾನಿಧೇ ॥೭೩॥

ಕ್ಷಣಮಾತ್ರಂ ನ ಶಕ್ನೋಮಿ ಜೀವಿತುಂ ತ್ವತ್ಕೃಪಾಂ ವಿನಾ ।\\
ನ ಜೀವಂತೀಹ ಜಲಜಾ ಜಲಂ ತ್ಯಕ್ತ್ವಾ ಜಲಗ್ರಹಾಃ ॥೭೪॥

	ಯಥಾ ಹಿ ಪುತ್ರವಾತ್ಸಲ್ಯಾತ್ ಜನನೀ ಪ್ರಸ್ನುತಸ್ತನೀ ।\\
	ವತ್ಸಂ ತ್ವರಿತಮಾಗತ್ಯ ಸಂಪ್ರೀಣಯತಿ ವತ್ಸಲಾ ॥೭೫॥

ಯದಿ ಸ್ಯಾಂ ತವ ಪುತ್ರೋಽಹಂ ಮಾತಾ ತ್ವಂ ಯದಿ ಮಾಮಕೀ ।\\
ದಯಾಪಯೋಧರಸ್ತನ್ಯಸುಧಾಭಿರಭಿಷಿಂಚ ಮಾಂ ॥೭೬॥

	ಮೃಗ್ಯೋ ನ ಗುಣಲೇಶೋಽಪಿ ಮಯಿ ದೋಷೈಕಮಂದಿರೇ ।\\
	ಪಾಂಸೂನಾಂ ವೃಷ್ಟಿಬಿಂದೂನಾಂ ದೋಷಾಣಾಂ ಚ ನ ಮೇ ಮಿತಿಃ ॥೭೭॥

ಪಾಪಿನಾಮಹಮೇವಾಗ್ರ್ಯೋ ದಯಾಲೂನಾಂ ತ್ವಮಗ್ರಣೀಃ ।\\
ದಯನೀಯೋ ಮದನ್ಯೋಽಸ್ತಿ ತವ ಕೋಽತ್ರ ಜಗತ್ತ್ರಯೇ ॥೭೮॥

	ವಿಧಿನಾಹಂ ನ ಸೃಷ್ಟಶ್ಚೇತ್ ನ ಸ್ಯಾತ್ತವ ದಯಾಲುತಾ ।\\
	ಆಮಯೋ ವಾ ನ ಸೃಷ್ಟಶ್ಚೇದೌಷಧಸ್ಯ ವೃಥೋದಯಃ ॥೭೯॥

ಕೃಪಾ ಮದಗ್ರಜಾ ಕಿಂ ತೇ ತ್ವಹಂ ಕಿಂ ವಾ ತದಗ್ರಜಃ ।\\
ವಿಚಾರ್ಯ ದೇಹಿ ಮೇ ವಿತ್ತಂ ತವ ದೇವಿ ದಯಾನಿಧೇ ॥೮೦॥

	ಮಾತಾ ಪಿತಾ ತ್ವಂ ಗುರುಸದ್ಗತೀ ಶ್ರೀಃ\\ತ್ವಮೇವ ಸಂಜೀವನಹೇತುಭೂತಾ ।\\
	ಅನ್ಯಂ ನ ಮನ್ಯೇ ಜಗದೇಕನಾಥೇ\\ ತ್ವಮೇವ ಸರ್ವಂ ಮಮ ದೇವಿ ಸತ್ಯೇ ॥೮೧॥

{\bfseries  ಹೃದಯ ಬೀಜಂ ॥ ಓಂ ಆಂ ಕ್ರೌಂ ಹುಂ ಫಟ್ ಕುರು ಕುರು ಸ್ವಾಹಾ ॥}

ಆದ್ಯಾದಿಲಕ್ಷ್ಮೀರ್ಭವ ಸುಪ್ರಸನ್ನಾ \\ವಿಶುದ್ಧವಿಜ್ಞಾನಸುಖೈಕದೋಗ್ಧ್ರೀ ।\\
ಅಜ್ಞಾನಹಂತ್ರೀ ತ್ರಿಗುಣಾತಿರಿಕ್ತಾ \\ಪ್ರಜ್ಞಾನನೇತ್ರೀ ಭವ ಸುಪ್ರಸನ್ನಾ ॥೮೨॥

	ಅಶೇಷವಾಗ್ಜಾಡ್ಯಮಲಾಪಹಂತ್ರೀ\\ ನವಂ ನವಂ ಸ್ಪಷ್ಟಸುವಾಕ್ಪ್ರದಾಯಿನೀ ।\\
	ಮಮೇಹ ಜಿಹ್ವಾಂಗಣರಂಗನರ್ತಕೀ\\ ಭವ ಪ್ರಸನ್ನಾ ವದನೇ ಚ ಮೇ ಶ್ರೀಃ ॥೮೩॥

ಸಮಸ್ತಸಂಪತ್ಸು ವಿರಾಜಮಾನಾ\\ ಸಮಸ್ತತೇಜಶ್ಚಯಭಾಸಮಾನಾ ।\\
ವಿಷ್ಣುಪ್ರಿಯೇ ತ್ವಂ ಭವ ದೀಪ್ಯಮಾನಾ\\ ವಾಗ್ದೇವತಾ ಮೇ ನಯನೇ ಪ್ರಸನ್ನಾ ॥೮೪॥

	ಭಕ್ತ್ಯಾ ನತಾನಾಂ ಸಕಲಾರ್ಥಸಿದ್ಧಿಪ್ರದೇ\\ ಸುಲಾವಣ್ಯದಯಾಪ್ರದೋಗ್ಧ್ರಿ ।\\
	ಸುವರ್ಣದೇ ತ್ವಂ ಸುಮುಖೀ ಭವ ಶ್ರೀ\\ರ್ಹಿರಣ್ಮಯೀ ಮೇ ನಯನೇ ಪ್ರಸನ್ನಾ ॥೮೫॥

ಸರ್ವಾರ್ಥದಾ ಸರ್ವಜಗತ್ಪ್ರಸೂತಿಃ \\ಸರ್ವೇಶ್ವರೀ ಸರ್ವಭಯಾಪಹಂತ್ರೀ ।\\
ಸರ್ವೋನ್ನತಾ ತ್ವಂ ಸುಮುಖೀ ಭವ ಶ್ರೀ\\ರ್ಹಿರಣ್ಮಯೀ ಮೇ ನಯನೇ ಪ್ರಸನ್ನಾ ॥೮೬॥

	ಸಮಸ್ತವಿಘ್ನೌಘವಿನಾಶಕಾರಿಣೀ\\ ಸಮಸ್ತಭಕ್ತೋದ್ಧರಣೇ ವಿಚಕ್ಷಣಾ ।\\
	ಅನಂತಸೌಭಾಗ್ಯಸುಖಪ್ರದಾಯಿನೀ\\ ಹಿರಣ್ಮಯೀ ಮೇ ನಯನೇ ಪ್ರಸನ್ನಾ ॥೮೭॥

ದೇವಿ ಪ್ರಸೀದ ದಯನೀಯತಮಾಯ ಮಹ್ಯಂ\\ ದೇವಾಧಿನಾಥಭವದೇವಗಣಾಭಿವಂದ್ಯೇ ।\\
ಮಾತಸ್ತಥೈವ ಭವ ಸನ್ನಿಹಿತಾ ದೃಶೋರ್ಮೇ\\ ಪತ್ಯಾ ಸಮಂ ಮಮ ಮುಖೇ ಭವ ಸುಪ್ರಸನ್ನಾ ॥೮೮॥

	ಮಾ ವತ್ಸ ಭೈರಭಯದಾನಕರೋಽರ್ಪಿತಸ್ತೇ\\ ಮೌಲೌ ಮಮೇತಿ ಮಯಿ ದೀನದಯಾನುಕಂಪೇ ।\\
	ಮಾತಃ ಸಮರ್ಪಯ ಮುದಾ ಕರುಣಾಕಟಾಕ್ಷಂ\\ ಮಾಂಗಲ್ಯಬೀಜಮಿಹ ನಃ ಸೃಜ ಜನ್ಮ ಮಾತಃ ॥೮೯॥

{\bfseries  ಕಂಠಬೀಜಂ । ಓಂ ಶ್ರಾಂ ಶ್ರೀಂ ಶ್ರೂಂ ಶ್ರೈಂ ಶ್ರೌಂ ಶ್ರಃ ॥}

ಕಟಾಕ್ಷ ಇಹ ಕಾಮಧುಕ್ ತವ ಮನಸ್ತು ಚಿಂತಾಮಣಿಃ\\ ಕರಃ ಸುರತರುಃ ಸದಾ ನವನಿಧಿಸ್ತ್ವಮೇವೇಂದಿರೇ ।\\
ಭವೇತ್ತವ ದಯಾರಸೋ ಮಮ ರಸಾಯನಂ ಚಾನ್ವಹಂ\\ ಮುಖಂ ತವ ಕಲಾನಿಧಿರ್ವಿವಿಧವಾಂಛಿತಾರ್ಥಪ್ರದಂ ॥೯೦॥

	ಯಥಾ ರಸಸ್ಪರ್ಶನತೋಽಯಸೋಽಪಿ\\ ಸುವರ್ಣತಾ ಸ್ಯಾತ್ಕಮಲೇ ತಥಾ ತೇ ।\\
	ಕಟಾಕ್ಷಸಂಸ್ಪರ್ಶನತೋ ಜನಾನಾಮ್ \\ಅಮಂಗಲಾನಾಮಪಿ ಮಂಗಲತ್ವಂ ॥೯೧॥
	
ದೇಹೀತಿ ನಾಸ್ತೀತಿ ವಚಃ ಪ್ರವೇಶಾದ್\\ಭೀತೋ ರಮೇ ತ್ವಾಂ ಶರಣಂ ಪ್ರಪದ್ಯೇ ।\\
ಅತಃ ಸದಾಸ್ಮಿನ್ನಭಯಪ್ರದಾ ತ್ವಂ\\ ಸಹೈವ ಪತ್ಯಾ ಮಯಿ ಸನ್ನಿಧೇಹಿ ॥೯೨॥

	ಕಲ್ಪದ್ರುಮೇಣ ಮಣಿನಾ ಸಹಿತಾ ಸುರಭ್ಯಾ \\ಶ್ರೀಸ್ತೇ ಕಲಾ ಮಯಿ ರಸೇನ ರಸಾಯನೇನ ।\\
	ಆಸ್ತಾಂ ಯತೋ ಮಮ ಚ ದೃಕ್ಶಿರಪಾಣಿಪಾದ\\ಸ್ಪೃಷ್ಟಾಃ ಸುವರ್ಣವಪುಷಃ ಸ್ಥಿರಜಂಗಮಾಃ ಸ್ಯುಃ ॥೯೩॥

ಆದ್ಯಾದಿವಿಷ್ಣೋಃ ಸ್ಥಿರಧರ್ಮಪತ್ನೀ\\ ತ್ವಮೇವ ಪತ್ಯಾ ಮಯಿ ಸನ್ನಿಧೇಹಿ ।\\
ಆದ್ಯಾದಿಲಕ್ಷ್ಮಿ ತ್ವದನುಗ್ರಹೇಣ\\ ಪದೇ ಪದೇ ಮೇ ನಿಧಿದರ್ಶನಂ ಸ್ಯಾತ್ ॥೯೪॥

	ಆದ್ಯಾದಿಲಕ್ಷ್ಮೀಹೃದಯಂ ಪಠೇದ್ಯಃ\\ ಸ ರಾಜ್ಯಲಕ್ಷ್ಮೀಮಚಲಾಂ ತನೋತಿ ।\\
	ಮಹಾದರಿದ್ರೋಽಪಿ ಭವೇದ್ಧನಾಢ್ಯಃ \\ತದನ್ವಯೇ ಶ್ರೀಃ ಸ್ಥಿರತಾಂ ಪ್ರಯಾತಿ ॥೯೫॥

ಯಸ್ಯ ಸ್ಮರಣಮಾತ್ರೇಣ ತುಷ್ಟಾ ಸ್ಯಾದ್ವಿಷ್ಣುವಲ್ಲಭಾ ।\\
ತಸ್ಯಾಭೀಷ್ಟಂ ದದಾತ್ಯಾಶು ತಂ ಪಾಲಯತಿ ಪುತ್ರವತ್ ॥೯೬॥

ಇದಂ ರಹಸ್ಯಂ ಹೃದಯಂ ಸರ್ವಕಾಮಫಲಪ್ರದಂ ।\\
ಜಪಃ ಪಂಚಸಹಸ್ರಂ ತು ಪುರಶ್ಚರಣಮುಚ್ಯತೇ ॥೯೭॥

	ತ್ರಿಕಾಲಂ ಏಕಕಾಲಂ ವಾ ನರೋ ಭಕ್ತಿಸಮನ್ವಿತಃ ।\\
	ಯಃ ಪಠೇತ್ ಶೃಣುಯಾದ್ವಾಪಿ ಸ ಯಾತಿ ಪರಮಾಂ ಶ್ರಿಯಂ ॥೯೮॥

ಮಹಾಲಕ್ಷ್ಮೀಂ ಸಮುದ್ದಿಶ್ಯ ನಿಶಿ ಭಾರ್ಗವವಾಸರೇ ।\\
ಇದಂ ಶ್ರೀಹೃದಯಂ ಜಪ್ತ್ವಾ ಪಂಚವಾರಂ ಧನೀ ಭವೇತ್ ॥೯೯॥

	ಅನೇನ ಹೃದಯೇನಾನ್ನಂ ಗರ್ಭಿಣ್ಯೈ ಚಾಭಿಮಂತ್ರಿತಂ ।\\
	ದದಾತಿ ತತ್ಕುಲೇ ಪುತ್ರೋ ಜಾಯತೇ ಶ್ರೀಪತಿಃ ಸ್ವಯಂ ॥೧೦೦॥

ನರೇಣ ವಾಥವಾ ನಾರ್ಯಾ ಲಕ್ಷ್ಮೀಹೃದಯಮಂತ್ರಿತೇ ।\\
ಜಲೇ ಪೀತೇ ಚ ತದ್ವಂಶೇ ಮಂದಭಾಗ್ಯೋ ನ ಜಾಯತೇ ॥೧೦೧॥

ಯ ಆಶ್ವಿನೇ ಮಾಸಿ ಚ ಶುಕ್ಲಪಕ್ಷೇ\\ ರಮೋತ್ಸವೇ ಸನ್ನಿಹಿತೈಕಭಕ್ತ್ಯಾ ।\\
ಪಠೇತ್ತಥೈಕೋತ್ತರವಾರವೃದ್ಧ್ಯಾ\\ ಲಭೇತ್ಸ ಸೌವರ್ಣಮಯೀಂ ಸುವೃಷ್ಟಿಂ ॥೧೦೨॥

	ಯ ಏಕಭಕ್ತೋಽನ್ವಹಮೇಕವರ್ಷಂ\\ ವಿಶುದ್ಧಧೀಃ ಸಪ್ತತಿವಾರಜಾಪೀ ।\\
	ಸ ಮಂದಭಾಗ್ಯೋಽಪಿ ರಮಾಕಟಾಕ್ಷಾತ್\\ ಭವೇತ್ಸಹಸ್ರಾಕ್ಷಶತಾಧಿಕಶ್ರೀಃ ॥೧೦೩॥

ಶ್ರೀಶಾಂಘ್ರಿಭಕ್ತಿಂ ಹರಿದಾಸದಾಸ್ಯಂ \\ಪ್ರಸನ್ನಮಂತ್ರಾರ್ಥದೃಢೈಕನಿಷ್ಠಾಂ ।\\
ಗುರೋಃ ಸ್ಮೃತಿಂ ನಿರ್ಮಲಬೋಧಬುದ್ಧಿಂ\\ ಪ್ರದೇಹಿ ಮಾತಃ ಪರಮಂ ಪದಂ ಶ್ರೀಃ ॥೧೦೪॥

ಪೃಥ್ವೀಪತಿತ್ವಂ ಪುರುಷೋತ್ತಮತ್ವಂ \\ವಿಭೂತಿವಾಸಂ ವಿವಿಧಾರ್ಥಸಿದ್ಧಿಂ ।\\
ಸಂಪೂರ್ಣಕೀರ್ತಿಂ ಬಹುವರ್ಷಭೋಗಂ\\ ಪ್ರದೇಹಿ ಮೇ ದೇವಿ ಪುನಃಪುನಸ್ತ್ವಂ ॥೧೦೫॥

	ವಾದಾರ್ಥಸಿದ್ಧಿಂ ಬಹುಲೋಕವಶ್ಯಂ\\ ವಯಃಸ್ಥಿರತ್ವಂ ಲಲನಾಸು ಭೋಗಂ ।\\
	ಪೌತ್ರಾದಿಲಬ್ಧಿಂ ಸಕಲಾರ್ಥಸಿದ್ಧಿಂ \\ಪ್ರದೇಹಿ ಮೇ ಭಾರ್ಗವಿ ಜನ್ಮಜನ್ಮನಿ ॥೧೦೬॥

ಸುವರ್ಣವೃದ್ಧಿಂ ಕುರು ಮೇ ಗೃಹೇ ಶ್ರೀಃ\\ ಸುಧಾನ್ಯವೃದ್ಧಿಂ ಕುರು ಮೇ ಗೃಹೇ ಶ್ರೀಃ ।\\
ಕಲ್ಯಾಣವೃದ್ಧಿಂ ಕುರು ಮೇ ಗೃಹೇ ಶ್ರೀಃ \\ವಿಭೂತಿವೃದ್ಧಿಂ ಕುರು ಮೇ ಗೃಹೇ ಶ್ರೀಃ ॥೧೦೭॥

{\bfseries ಶಿರೋ ಬೀಜಂ । ಓಂ ಯಂ ಹಂ ಕಂ ಲಂ ಪಂ ಶ್ರೀಂ ॥}

	ಧ್ಯಾಯೇಲ್ಲಕ್ಷ್ಮೀಂ ಪ್ರಹಸಿತಮುಖೀಂ ಕೋಟಿಬಾಲಾರ್ಕಭಾಸಾಂ\\
	ವಿದ್ಯುದ್ವರ್ಣಾಂಬರವರಧರಾಂ ಭೂಷಣಾಢ್ಯಾಂ ಸುಶೋಭಾಂ ।\\
ಬೀಜಾಪೂರಂ ಸರಸಿಜಯುಗಂ ಬಿಭ್ರತೀಂ ಸ್ವರ್ಣಪಾತ್ರಂ\\
ಭರ್ತ್ರಾ ಯುಕ್ತಾಂ ಮುಹುರಭಯದಾಂ ಮಹ್ಯಮಪ್ಯಚ್ಯುತಶ್ರೀಃ ॥೧೦೮॥

	ಗುಹ್ಯಾತಿಗುಹ್ಯಗೋಪ್ತ್ರೀ ತ್ವಂ ಗೃಹಾಣಾಸ್ಮತ್ಕೃಪಂ ಜಪಂ ।\\
	ಸಿದ್ಧಿರ್ಭವತು ಮೇ ದೇವಿ ತ್ವತ್ಪ್ರಸಾದಾನ್ಮಯಿ ಸ್ಥಿರಾ ॥೧೦೯॥

	॥ ಇತ್ಯಾಥರ್ವಣರಹಸ್ಯೇ ಶ್ರೀಲಕ್ಷ್ಮೀಹೃದಯಸ್ತೋತ್ರಂ ಸಂಪೂರ್ಣಂ ॥\\
	(ಮೂಲೇನ ದಶವಾರಂ ತರ್ಪಯೇತ್ ।)
%======================================================================================================
\chapter*{\center ॥ಶ್ರೀನಾರಾಯಣಹೃದಯಂ॥}
ಅಸ್ಯ ಶ್ರೀನಾರಾಯಣ ಹೃದಯ ಸ್ತೋತ್ರಮಂತ್ರಸ್ಯ ಭಾರ್ಗವಋಷಿಃ~। ಅನುಷ್ಟುಪ್ಛಂದಃ~। ಶ್ರೀಲಕ್ಷ್ಮೀನಾರಾಯಣೋ ದೇವತಾ।\\
ಓಂ ಬೀಜಂ~। ನಮಃ ಶಕ್ತಿಃ~। ನಾರಾಯಣಾಯೇತಿ ಕೀಲಕಂ~। ಶ್ರೀಲಕ್ಷ್ಮೀನಾರಾಯಣಪ್ರೀತ್ಯರ್ಥೇ ಜಪೇ ವಿನಿಯೋಗಃ॥
\thispagestyle{empty}
\section{ಋಷ್ಯಾದಿನ್ಯಾಸಃ}
\addcontentsline{toc}{section}{ಋಷ್ಯಾದಿನ್ಯಾಸಃ}
ಭಾರ್ಗವ ಋಷಯೇ ನಮಃ (ಶಿರಸಿ)। ಶ್ರೀಲಕ್ಷ್ಮೀನಾರಾಯಣ ದೇವತಾಯೈ ನಮಃ (ಹೃದಯೇ)। ಓಂ ಬೀಜಾಯ ನಮಃ (ಗುಹ್ಯೇ)। ನಮಃ ಶಕ್ತಯೇ ನಮಃ (ಪಾದಯೋಃ)। ನಾರಾಯಣಾಯಕೀಲಕಾಯ ನಮಃ (ನಾಭೌ)।\\

 \section{ಕರನ್ಯಾಸಪಂಚಾಂಗನ್ಯಾಸೌ}
\addcontentsline{toc}{section}{ಕರನ್ಯಾಸಪಂಚಾಂಗನ್ಯಾಸೌ}
ಓಂ ನಾರಾಯಣಃ ಪರಂಜ್ಯೋತಿರಿತ್ಯಂಗುಷ್ಠಾಭ್ಯಾಂ ನಮಃ । ಹೃದಯಾಯ ನಮಃ ।\\
ಓಂ ನಾರಾಯಣಃ ಪರಂಬ್ರಹ್ಮೇತಿ ತರ್ಜನೀಭ್ಯಾಂ ನಮಃ । ಶಿರಸೇ ಸ್ವಾಹಾ ।\\
ಓಂ ನಾರಾಯಣಃ ಪರೋ ದೇವಃ ಇತಿಮಧ್ಯಮಾಭ್ಯಾಂ ನಮಃ । ಶಿಖಾಯೈ ವಷಟ್ ।\\
ಓಂ ನಾರಾಯಣಃ ಪರೋ ಧ್ಯಾತೇತ್ಯನಾಮಿಕಾಭ್ಯಾಂ ನಮಃ । ಕವಚಾಯ ಹುಂ ।\\
ಓಂ ನಾರಾಯಣಃ ಪರಂ ಧಾಮೇತಿ ಕನಿಷ್ಠಿಕಾಭ್ಯಾಂ ನಮಃ । ನೇತ್ರತ್ರಯಾಯ ವೌಷಟ್ ।\\
ಓಂ ನಾರಾಯಣಃ ಪರೋ ಧರ್ಮ ಇತಿ ಕರತಲಕರಪೃಷ್ಠಾಭ್ಯಾಂ ನಮಃ । ಅಸ್ತ್ರಾಯ ಫಟ್ ।\\
ಓಂ ನಮಃ ಸುದರ್ಶನಾಯ ಸಹಸ್ರಾರ ಹುಂ ಫಟ್ ಐಂದ್ರ್ಯಾದಿದಶದಿಶೋ ಬಧ್ನಾಮಿ ನಮಶ್ಚಕ್ರಾಯ ಸ್ವಾಹಾ ।\\

	{\bfseries ಉದ್ಯದ್ಭಾಸ್ವತ್ಸಮಾಭಾಸಶ್ಚಿದಾನಂದೈಕದೇಹವಾನ್ ।\\
	ಚಕ್ರಶಂಖಗದಾಪದ್ಮಧರೋ ಧ್ಯೇಯೋಽಹಮೀಶ್ವರಃ ॥೧॥

ಲಕ್ಷ್ಮೀಧರಾಭ್ಯಾಮಾಶ್ಲಿಷ್ಟಃ ಸ್ವಮೂರ್ತಿಗಣಮಧ್ಯಗಃ ।\\
ಬ್ರಹ್ಮವಾಯುಶಿವಾಹೀಶವಿಪೈಃ ಶಕ್ರಾದಿಕೈರಪಿ ॥೨॥

	ಸೇವ್ಯಮಾನೋಽಧಿಕಂ ಭಕ್ತ್ಯಾ ನಿತ್ಯನಿಶ್ಶೇಷಭಕ್ತಿಮಾನ್ ।\\
	ಮೂರ್ತಯೋಽಷ್ಟಾವಪಿ ಧ್ಯೇಯಾಶ್ಚಕ್ರಶಂಖವರಾಭಯೈಃ ॥೩॥

ಉದ್ಯಾದಾದಿತ್ಯಸಂಕಾಶಂ ಪೀತವಾಸಸಮಚ್ಯುತಂ ।\\
ಶಂಖಚಕ್ರಗದಾಪಾಣಿಂ ಧ್ಯಾಯೇಲ್ಲಕ್ಷ್ಮೀಪತಿಂ ಹರಿಂ ॥೪॥\\
 ಓಂ ಐಂ ಹ್ರೀಂ ಶ್ರೀಂ ಶ್ರೀಲಕ್ಷ್ಮೀನಾರಾಯಣಾಯ ಸ್ವಾಹಾ ॥(೧೦೮)}\\
ಅಥ ಮೂಲಾಷ್ಟಕಂ\\
	ನಾರಾಯಣಃ ಪರಂ ಜ್ಯೋತಿರಾತ್ಮಾ ನಾರಾಯಣಃ ಪರಃ ।\\
	ನಾರಾಯಣಃ ಪರಂ ಬ್ರಹ್ಮ ನಾರಾಯಣ ನಮೋಽಸ್ತು ತೇ ॥೧॥

ನಾರಾಯಣಃ ಪರೋ ದೇವೋ ಧಾತಾ ನಾರಾಯಣಃ ಪರಃ ।\\
ನಾರಾಯಣಃ ಪರೋ ಧ್ಯಾತಾ ನಾರಾಯಣ ನಮೋಽಸ್ತು ತೇ ॥೨॥

	ನಾರಾಯಣಃ ಪರಂ ಧಾಮ ಧ್ಯಾನಂ ನಾರಾಯಣಃ ಪರಃ ।\\
	ನಾರಾಯಣಃ ಪರೋ ಧರ್ಮೋ ನಾರಾಯಣ ನಮೋಽಸ್ತು ತೇ ॥೩॥

ನಾರಾಯಣಃ ಪರೋ ದೇವೋ ವಿದ್ಯಾ ನಾರಾಯಣಃ ಪರಃ ।\\
ವಿಶ್ವಂ ನಾರಾಯಣಃ ಸಾಕ್ಷಾನ್ನಾರಾಯಣ ನಮೋಽಸ್ತು ತೇ ॥೪॥

	ನಾರಾಯಣಾದ್ವಿಧಿರ್ಜಾತೋ ಜಾತೋ ನಾರಾಯಣಾದ್ಧರಃ ।\\
	ಜಾತೋ ನಾರಾಯಣಾದಿಂದ್ರೋ ನಾರಾಯಣ ನಮೋಽಸ್ತು ತೇ ॥೫॥

ರವಿರ್ನಾರಾಯಣಸ್ತೇಜಃ ಚಂದ್ರೋ ನಾರಾಯಣಃ ಪರಃ ।\\
ವಹ್ನಿರ್ನಾರಾಯಣಃ ಸಾಕ್ಷಾನ್ನಾರಾಯಣ ನಮೋಽಸ್ತು ತೇ ॥೬॥

	ನಾರಾಯಣ ಉಪಾಸ್ಯಃ ಸ್ಯಾದ್ಗುರುರ್ನಾರಾಯಣಃ ಪರಃ ।\\
	ನಾರಾಯಣಃ ಪರೋ ಬೋಧೋ ನಾರಾಯಣ ನಮೋಽಸ್ತು ತೇ ॥೭॥

ನಾರಾಯಣಃ ಫಲಂ ಮುಖ್ಯಂ ಸಿದ್ಧಿರ್ನಾರಾಯಣಃ ಸುಖಂ ।\\
ಸೇವ್ಯೋ ನಾರಾಯಣಃ ಶುದ್ಧೋ ನಾರಾಯಣ ನಮೋಽಸ್ತು ತೇ ॥೮॥\\(ಇತಿ ಮೂಲಾಷ್ಟಕಂ )\\
(ಅಷ್ಟವಾರಂ ಮೂಲಂ ಜಪ್ತ್ವಾ ಸಕೃತ್ ತರ್ಪಯೇತ್ ।)
\section{ಅಥ ಪ್ರಾರ್ಥನಾದಶಕಂ}
\addcontentsline{toc}{section}{ಅಥ ಪ್ರಾರ್ಥನಾದಶಕಂ}
	ನಾರಾಯಣಸ್ತ್ವಮೇವಾಸಿ ದಹರಾಖ್ಯೇ ಹೃದಿ ಸ್ಥಿತಃ ।\\
	ಪ್ರೇರಕಃ ಪ್ರೇರ್ಯಮಾಣಾನಾಂ ತ್ವಯಾ ಪ್ರೇರಿತ ಮಾನಸಃ ॥೧॥

ತ್ವದಜ್ಞಾಂ ಶಿರಸಾ ಧೃತ್ವಾ ಜಪಾಮಿ ಜನಪಾವನಂ ।\\
ನಾನೋಪಾಸನಮಾರ್ಗಾಣಾಂ ಭಾವಹೃದ್ಭಾವಬೋಧಕಃ ॥೨॥

	ಭಾವಾರ್ಥಕೃದ್ ಭಾವಭೂತೋ ಭವಸೌಖ್ಯಪ್ರದೋ ಭವ ।\\
	ತ್ವನ್ಮಾಯಾಮೋಹಿತಂ ವಿಶ್ವಂ ತ್ವಯೈವ ಪರಿಕಲ್ಪಿತಂ ॥೩॥

ತ್ವದಧಿಷ್ಠಾನಮಾತ್ರೇಣ ಸೈವ ಸರ್ವಾರ್ಥಕಾರಿಣೀ ।\\
ತ್ವಮೇವ ತಾಂ ಪುರಸ್ಕೃತ್ಯ ಮಮ ಕಾಮಾನ್ ಸಮರ್ಥಯ ॥೪॥

	ನ ಮೇ ತ್ವದನ್ಯಸ್ತ್ರಾತಾಸ್ತಿ ತ್ವದನ್ಯನ್ನ ಹಿ ದೈವತಂ ।\\
	ತ್ವದನ್ಯಂ ನ ಹಿ ಜಾನಾಮಿ ಪಾಲಕಂ ಪುಣ್ಯರೂಪಕಂ ॥೫॥

ಯಾವತ್ಸಾಂಸಾರಿಕೋ ಭಾವೋ ಮನಃಸ್ಥೋ ಭಾವನಾತ್ಮಕಃ ।\\
ತಾವತ್ಸಿದ್ಧಿರ್ಭವೇತ್ ಸದ್ಯಃ ಸರ್ವಥಾ ಸರ್ವದಾ ವಿಭೋ ॥೬॥

	ಪಾಪಿನಾಮಹಮೇವಾಗ್ರ್ಯೋ ದಯಾಲೂನಾಂ ತ್ವಮಗ್ರಣೀಃ ।\\
	ದಯನೀಯೋ ಮದನ್ಯೋಽಸ್ತಿ ತವ ಕೋಽತ್ರ ಜಗತ್ತ್ರಯೇ ॥೭॥

ತ್ವಯಾಪ್ಯಹಂ ನ ಸೃಷ್ಟಶ್ಚೇತ್ ನ ಸ್ಯಾತ್ತವ ದಯಾಲುತಾ ।\\
ಆಮಯೋ ನೈವ ಸೃಷ್ಟಶ್ಚೇದೌಷಧಸ್ಯ ವೃಥೋದಯಃ ॥೮॥

	ಪಾಪಸಂಘಪರಿಕ್ರಾಂತಃ ಪಾಪಾತ್ಮಾ ಪಾಪರೂಪಧೃಕ್ ।\\
	ತ್ವದನ್ಯಃ ಕೋಽತ್ರ ಪಾಪೇಭ್ಯಸ್ತ್ರಾತಾ ಮೇ ಜಗತೀತಲೇ ॥೯॥

ತ್ವಮೇವ ಮಾತಾ ಚ ಪಿತಾ ತ್ವಮೇವ \\ತ್ವಮೇವ ಬಂಧುಶ್ಚ ಸಖಾ ತ್ವಮೇವ ।\\
ತ್ವಮೇವ ವಿದ್ಯಾ ದ್ರವಿಣಂ ತ್ವಮೇವ \\ತ್ವಮೇವ ಸರ್ವಂ ಮಮ ದೇವ ದೇವ ॥೧೦॥

	ಪ್ರಾರ್ಥನಾದಶಕಂ ಚೈವ ಮೂಲಷ್ಟಕಮಿತಿದ್ವಯಂ ।\\
	ಯಃ ಪಠೇಚ್ಛೃಣುಯಾನ್ನಿತ್ಯಂ ತಸ್ಯ ಲಕ್ಷ್ಮೀಃ ಸ್ಥಿರಾ ಭವೇತ್ ॥೧೧॥

ನಾರಾಯಣಸ್ಯ ಹೃದಯಂ ಸರ್ವಾಭೀಷ್ಟಫಲಪ್ರದಂ ।\\
ಲಕ್ಷ್ಮೀಹೃದಯಕಂ ಸ್ತೋತ್ರಂ ಯದಿ ಚೈತದ್ವಿನಾಕೃತಂ ॥೧೨॥

	ತತ್ಸರ್ವಂ ನಿಷ್ಫಲಂ ಪ್ರೋಕ್ತಂ ಲಕ್ಷ್ಮೀಃ ಕ್ರುಧ್ಯತಿ ಸರ್ವದಾ ।\\
	ಏತತ್ಸಂಪುಟಿತಂ ಸ್ತೋತ್ರಂ ಸರ್ವಕರ್ಮಫಲಪ್ರದಂ ॥೧೩॥

ಲಕ್ಷ್ಮೀಹೃದಯಕಂ ಚೈವ ತಥಾ ನಾರಾಯಣಾತ್ಮಕಂ ।\\
ಜಪೇದ್ಯಃ ಸಂಕಲೀಕೃತ್ಯ ಸರ್ವಾಭೀಷ್ಟಮವಾಪ್ನುಯಾತ್ ॥೧೪॥

	ನಾರಾಯಣಸ್ಯ ಹೃದಯಂ ಆದೌ ಜಪ್ತ್ವಾ ತತಃಪರಂ ।\\
	ಲಕ್ಷ್ಮೀಹೃದಯಕಂ ಸ್ತೋತ್ರಂ ಜಪೇನ್ನಾರಾಯಣಂ ಪುನಃ ॥೧೫॥

ಪುನರ್ನಾರಾಯಣಂ ಜಪ್ತ್ವಾ ಪುನರ್ಲಕ್ಷ್ಮೀಸ್ತವಂ ಜಪೇತ್ ।\\
ಪುನರ್ನಾರಾಯಣಂ ಜಾಪ್ಯಂ ಸಂಕಲೀಕರಣಂ ಭವೇತ್ ॥೧೬॥

	ಏವಂ ಮಧ್ಯೇ ದ್ವಿವಾರೇಣ ಜಪೇತ್ ಸಂಕಲಿತಂ ತು ತತ್ ।\\
	ಲಕ್ಷ್ಮೀಹೃದಯಕಂ ಸ್ತೋತ್ರಂ ಸರ್ವಕಾಮಪ್ರಕಾಶಿತಂ ॥೧೭॥

ತದ್ವಜ್ಜಪಾದಿಕಂ ಕುರ್ಯಾದೇತತ್ಸಂಕಲಿತಂ ಶುಭಂ ।\\
ಸರ್ವಾನ್ಕಾಮಾನವಾಪ್ನೋತಿ ಆಧಿವ್ಯಾಧಿಭಯಂ ಹರೇತ್ ॥೧೮॥

	ಗೋಪ್ಯಮೇತತ್ ಸದಾ ಕುರ್ಯಾತ್ ನ ಸರ್ವತ್ರ ಪ್ರಕಾಶಯೇತ್ ।\\
	ಇತಿ ಗುಹ್ಯತಮಂ ಶಾಸ್ತ್ರಂ ಪ್ರೋಕ್ತಂ ಬ್ರಹ್ಮಾದಿಕೈಃ ಪುರಾ ॥೧೯॥

ತಸ್ಮಾತ್ಸರ್ವಪ್ರಯತ್ನೇನ ಗೋಪಯೇತ್ಸಾಧಯೇದ್ ಸುಧೀಃ ।\\
ಯತ್ರೈತತ್ಪುಸ್ತಕಂ ತಿಷ್ಠೇಲ್ಲಕ್ಷ್ಮೀನಾರಾಯಣಾತ್ಮಕಂ ॥೨೦॥

	ಭೂತಪೈಶಾಚವೇತಾಳಭಯಂ ತತ್ರ ನ ಜಾಯತೇ ।\\
	ಲಕ್ಷ್ಮೀಹೃದಯಕಂ ಪ್ರೋಕ್ತಂ ವಿಧಿನಾ ಸಾಧಯೇತ್ ಸುಧೀಃ ॥೨೧॥

ಭೃಗುವಾರೇ ತಥಾ ರಾತ್ರೌ ಪೂಜಯೇತ್ ಪುಸ್ತಕದ್ವಯಂ ।\\
ಸರ್ವಥಾ ಸರ್ವದಾ ಸತ್ಯಂ ಗೋಪಯೇತ್ ಸಾಧಯೇತ್ ಸುಧೀಃ ।\\
	ಗೋಪನಾತ್ ಸಾಧನಾಲ್ಲೋಕೇ ಸರ್ವಾಂ ಸಿದ್ಧಿಂ ಲಭೇನ್ನರಃ ॥೨೨॥

	ಇತಿ ನಾರಾಯಣಹೃದಯಸ್ತೋತ್ರಂ ಸಂಪೂರ್ಣಂ ॥
%=========================================================================================


\section{ ಶ್ರೀ ಮಹಾಲಕ್ಷ್ಮ್ಯಷ್ಟಕಂ}
\addcontentsline{toc}{section}{ ಶ್ರೀ ಮಹಾಲಕ್ಷ್ಮ್ಯಷ್ಟಕಂ}
ನಮಸ್ತೇಸ್ತು ಮಹಾಮಾಯೇ ಶ್ರೀಪೀಠೇ ಸುರಪೂಜಿತೇ।\\
ಶಂಖಚಕ್ರಗದಾಹಸ್ತೇ ಮಹಾಲಕ್ಷ್ಮೀ ನಮೋಸ್ತು ತೇ ॥೧॥

	ನಮಸ್ತೇ ಗರುಡಾರೂಢೇ ಕೋಲಾಸುರಭಯಂಕರಿ।\\
	ಸರ್ವಪಾಪಹರೇ ದೇವಿ ಮಹಾಲಕ್ಷ್ಮಿ ನಮೋಸ್ತು ತೇ ॥೨॥

ಸರ್ವಜ್ಞೇ ಸರ್ವವರದೇ ಸರ್ವದುಷ್ಟಭಯಂಕರಿ।\\
ಸರ್ವದುಃಖಹರೇ ದೇವಿ ಮಹಾಲಕ್ಷ್ಮಿ ನಮೋಸ್ತು ತೇ ॥೩॥

	ಸಿದ್ಧಿಬುದ್ಧಿಪ್ರದೇ ದೇವಿ ಭುಕ್ತಿಮುಕ್ತಿಪ್ರದಾಯಿನಿ।\\
	ಮಂತ್ರಮೂರ್ತೇ ಸದಾ ದೇವಿ ಮಹಾಲಕ್ಷ್ಮಿ ನಮೋಸ್ತು ತೇ ॥೪॥

ಆದ್ಯಂತರಹಿತೇ ದೇವಿ ಆದ್ಯಶಕ್ತಿಮಹೇಶ್ವರಿ।\\
ಯೋಗಜ್ಞೇ ಯೋಗಸಂಭೂತೇ ಮಹಾಲಕ್ಷ್ಮಿ ನಮೋಸ್ತು ತೇ ॥೫॥

	ಸ್ಥೂಲಸೂಕ್ಷ್ಮಮಹಾರೌದ್ರೇ ಮಹಾಶಕ್ತಿಮಹೋದರೇ।\\
	ಮಹಾಪಾಪಹರೇ ದೇವಿ ಮಹಾಲಕ್ಷ್ಮಿ ನಮೋಸ್ತು ತೇ ॥೬॥

ಪದ್ಮಾಸನಸ್ಥಿತೇ ದೇವಿ ಪರಬ್ರಹ್ಮಸ್ವರೂಪಿಣಿ।\\
ಪರಮೇಶಿ ಜಗನ್ಮಾತರ್ಮಹಾಲಕ್ಷ್ಮಿ ನಮೋಸ್ತು ತೇ ॥೭॥

	ಶ್ವೇತಾಂಬರಧರೇ ದೇವಿ ನಾನಾಲಂಕಾರಭೂಷಿತೇ।\\
	ಜಗತ್ಸ್ಥಿತೇ ಜಗನ್ಮಾತರ್ಮಹಾಲಕ್ಷ್ಮಿ ನಮೋಸ್ತು ತೇ ॥೮॥

ಮಹಾಲಕ್ಷ್ಮ್ಯಷ್ಟಕಂ ಸ್ತೋತ್ರಂ ಯಃ ಪಠೇದ್ಭಕ್ತಿಮಾನ್ನರಃ।\\
ಸರ್ವಸಿದ್ಧಿಮವಾಪ್ನೋತಿ ರಾಜ್ಯಂ ಪ್ರಾಪ್ನೋತಿ ಸರ್ವದಾ ॥೯॥

	ಏಕಕಾಲೇ ಪಠೇನ್ನಿತ್ಯಂ ಮಹಾಪಾಪವಿನಾಶನಂ।\\
	ದ್ವಿಕಾಲಂ ಯಃ ಪಠೇನ್ನಿತ್ಯಂ ಧನಧಾನ್ಯಸಮನ್ವಿತಃ॥೧೦॥

ತ್ರಿಕಾಲಂ ಯಃ ಪಠೇನ್ನಿತ್ಯಂ ಮಹಾಶತ್ರುವಿನಾಶನಂ।\\
ಮಹಾಲಕ್ಷ್ಮೀರ್ಭವೇನ್ನಿತ್ಯಂ ಪ್ರಸನ್ನಾ ವರದಾ ಶುಭಾ ॥೧೧॥

%===========================================================================================================================================================================================
\section{ಇಂದ್ರಾಕ್ಷೀಸ್ತೋತ್ರಮ್}
\addcontentsline{toc}{section}{ಇಂದ್ರಾಕ್ಷೀಸ್ತೋತ್ರಮ್}
ಅಸ್ಯ ಶ್ರೀ ಇಂದ್ರಾಕ್ಷೀಸ್ತೋತ್ರಮಹಾಮಂತ್ರಸ್ಯ । ಶಚೀಪುರಂದರ ಋಷಿಃ । ಅನುಷ್ಟುಪ್ ಛಂದಃ । ಇಂದ್ರಾಕ್ಷೀ ದುರ್ಗಾ ದೇವತಾ । ಲಕ್ಷ್ಮೀರ್ಬೀಜಂ । ಭುವನೇಶ್ವರೀತಿ ಶಕ್ತಿಃ । ಭವಾನೀತಿ ಕೀಲಕಂ  । ಇಂದ್ರಾಕ್ಷೀಪ್ರಸಾದಸಿದ್ಧ್ಯರ್ಥೇ ಜಪೇ ವಿನಿಯೋಗಃ ।\\
ಕರನ್ಯಾಸಃ\\
ಓಂ ಇಂದ್ರಾಕ್ಷೀತ್ಯಂಗುಷ್ಠಾಭ್ಯಾಂ ನಮಃ ।\\
ಓಂ ಮಹಾಲಕ್ಷ್ಮೀತಿ ತರ್ಜನೀಭ್ಯಾಂ ನಮಃ ।\\
ಓಂ ಮಾಹೇಶ್ವರೀತಿ ಮಧ್ಯಮಾಭ್ಯಾಂ ನಮಃ ।\\
ಓಂ ಅಂಬುಜಾಕ್ಷೀತ್ಯನಾಮಿಕಾಭ್ಯಾಂ ನಮಃ ।\\
ಓಂ ಕಾತ್ಯಾಯನೀತಿ ಕನಿಷ್ಠಿಕಾಭ್ಯಾಂ ನಮಃ ।\\
ಓಂ ಕೌಮಾರೀತಿ ಕರತಲಕರಪೃಷ್ಠಾಭ್ಯಾಂ ನಮಃ ।\\
ಅಂಗನ್ಯಾಸಃ\\
ಓಂ ಇಂದ್ರಾಕ್ಷೀತಿ ಹೃದಯಾಯ ನಮಃ ।\\
ಓಂ ಮಹಾಲಕ್ಷ್ಮೀತಿ ಶಿರಸೇ ಸ್ವಾಹಾ ।\\
ಓಂ ಮಾಹೇಶ್ವರೀತಿ ಶಿಖಾಯೈ ವಷಟ್ ।\\
ಓಂ ಅಂಬುಜಾಕ್ಷೀತಿ ಕವಚಾಯ ಹುಂ ।\\
ಓಂ ಕಾತ್ಯಾಯನೀತಿ ನೇತ್ರತ್ರಯಾಯ ವೌಷಟ್ ।\\
ಓಂ ಕೌಮಾರೀತಿ ಅಸ್ತ್ರಾಯ ಫಟ್ ।\\
ಓಂ ಭೂರ್ಭುವಃ ಸ್ವರೋಂ ಇತಿ ದಿಗ್ಬಂಧಃ ॥

ಧ್ಯಾನಂ\\
ನೇತ್ರಾಣಾಂ ದಶಭಿಶ್ಶತೈಃ ಪರಿವೃತಾಮತ್ಯುಗ್ರಚರ್ಮಾಂಬರಾಂ\\
ಹೇಮಾಭಾಂ ಮಹತೀಂ ವಿಲಂಬಿತಶಿಖಾಮಾಮುಕ್ತಕೇಶಾನ್ವಿತಾಂ ।\\
ಘಂಟಾಮಂಡಿತಪಾದಪದ್ಮಯುಗಲಾಂ ನಾಗೇಂದ್ರಕುಂಭಸ್ತನೀಂ\\
ಇಂದ್ರಾಕ್ಷೀಂ ಪರಿಚಿಂತಯಾಮಿ ಮನಸಾ ಕಲ್ಪೋಕ್ತಸಿದ್ಧಿಪ್ರದಾಂ ॥

ಇಂದ್ರಾಕ್ಷೀಂ ದ್ವಿಭುಜಾಂ ದೇವೀಂ ಪೀತವಸ್ತ್ರದ್ವಯಾನ್ವಿತಾಂ ।\\
ವಾಮಹಸ್ತೇ ವಜ್ರಧರಾಂ ದಕ್ಷಿಣೇನ ವರಪ್ರದಾಂ ॥

ಇಂದ್ರಾಕ್ಷೀಂ ಸಹಸ್ರಯುವತೀಂ ನಾನಾಲಂಕಾರಭೂಷಿತಾಂ ।\\
ಪ್ರಸನ್ನವದನಾಂಭೋಜಾಮಪ್ಸರೋಗಣಸೇವಿತಾಂ ॥

ದ್ವಿಭುಜಾಂ ಸೌಮ್ಯವದನಾಂ ಪಾಶಾಂಕುಶಧರಾಂ ಪರಾಂ ।\\
ತ್ರೈಲೋಕ್ಯಮೋಹಿನೀಂ ದೇವೀಮಿಂದ್ರಾಕ್ಷೀನಾಮಕೀರ್ತಿತಾಂ ॥

ಪೀತಾಂಬರಾಂ ವಜ್ರಧರೈಕಹಸ್ತಾಂ ನಾನಾವಿಧಾಲಂಕರಣಾಂ ಪ್ರಸನ್ನಾಂ ।\\
ತ್ವಾಮಪ್ಸರಸ್ಸೇವಿತಪಾದಪದ್ಮಾಮಿಂದ್ರಾಕ್ಷಿ ವಂದೇ ಶಿವಧರ್ಮಪತ್ನೀಂ ॥

ಇಂದ್ರಾದಿಭಿಃ ಸುರೈರ್ವಂದ್ಯಾಂ ವಂದೇ ಶಂಕರವಲ್ಲಭಾಂ ।\\
ಏವಂ ಧ್ಯಾತ್ವಾ ಮಹಾದೇವೀಂ ಜಪೇತ್ ಸರ್ವಾರ್ಥಸಿದ್ಧಯೇ ॥

ವಜ್ರಿಣೀ ಪೂರ್ವತಃ ಪಾತು ಚಾಗ್ನೇಯ್ಯಾಂ ಪರಮೇಶ್ವರೀ ।\\
ದಂಡಿನೀ ದಕ್ಷಿಣೇ ಪಾತು ನೈರೄತ್ಯಾಂ ಪಾತು ಖಡ್ಗಿನೀ ॥೧॥

ಪಶ್ಚಿಮೇ ಪಾಶಧಾರೀ ಚ ಧ್ವಜಸ್ಥಾ ವಾಯುದಿಙ್ಮುಖೇ ।\\
ಕೌಮೋದಕೀ ತಥೋದೀಚ್ಯಾಂ ಪಾತ್ವೈಶಾನ್ಯಾಂ ಮಹೇಶ್ವರೀ ॥೨॥

ಉರ್ಧ್ವದೇಶೇ ಪದ್ಮಿನೀ ಮಾಮಧಸ್ತಾತ್ ಪಾತು ವೈಷ್ಣವೀ ।\\
ಏವಂ ದಶದಿಶೋ ರಕ್ಷೇತ್ ಸರ್ವದಾ ಭುವನೇಶ್ವರೀ ॥೩॥

ಇಂದ್ರ ಉವಾಚ ।\\
ಇಂದ್ರಾಕ್ಷೀ ನಾಮ ಸಾ ದೇವೀ ದೈವತೈಃ ಸಮುದಾಹೃತಾ ।\\
ಗೌರೀ ಶಾಕಂಭರೀ ದೇವೀ ದುರ್ಗಾ ನಾಮ್ನೀತಿ ವಿಶ್ರುತಾ ॥೪॥

ನಿತ್ಯಾನಂದಾ ನಿರಾಹಾರಾ ನಿಷ್ಕಲಾಯೈ ನಮೋಽಸ್ತು ತೇ ।\\
ಕಾತ್ಯಾಯನೀ ಮಹಾದೇವೀ ಚಂದ್ರಘಂಟಾ ಮಹಾತಪಾಃ ॥೫॥

ಸಾವಿತ್ರೀ ಸಾ ಚ ಗಾಯತ್ರೀ ಬ್ರಹ್ಮಾಣೀ ಬ್ರಹ್ಮವಾದಿನೀ ।\\
ನಾರಾಯಣೀ ಭದ್ರಕಾಲೀ ರುದ್ರಾಣೀ ಕೃಷ್ಣಪಿಂಗಲಾ ॥೬॥

ಅಗ್ನಿಜ್ವಾಲಾ ರೌದ್ರಮುಖೀ ಕಾಲರಾತ್ರಿಸ್ತಪಸ್ವಿನೀ ।\\
ಮೇಘಸ್ಶ್ಯಾಮಾ ಸಹಸ್ರಾಕ್ಷೀ ವಿಕಟಾಂಗೀ ಜಡೋದರೀ ॥೭॥

ಮಹೋದರೀ ಮುಕ್ತಕೇಶೀ ಘೋರರೂಪಾ ಮಹಾಬಲಾ ।\\
ಅಜಿತಾ ಭದ್ರದಾನಂತಾ ರೋಗಹರ್ತ್ರೀ ಶಿವಪ್ರದಾ ॥೮॥

ಶಿವದೂತೀ ಕರಾಲೀ ಚ ಪ್ರತ್ಯಕ್ಷಪರಮೇಶ್ವರೀ ।\\
ಇಂದ್ರಾಣೀ ಇಂದ್ರರೂಪಾ ಚ ಇಂದ್ರಶಕ್ತಿಃ ಪರಾಯಣಾ ॥೯॥

ಸದಾ ಸಮ್ಮೋಹಿನೀ ದೇವೀ ಸುಂದರೀ ಭುವನೇಶ್ವರೀ ।\\
ಏಕಾಕ್ಷರೀ ಪರಬ್ರಹ್ಮಸ್ಥೂಲಸೂಕ್ಷ್ಮಪ್ರವರ್ಧಿನೀ ॥೧೦॥

ರಕ್ಷಾಕರೀ ರಕ್ತದಂತಾ ರಕ್ತಮಾಲ್ಯಾಂಬರಾ ಪರಾ ।\\
ಮಹಿಷಾಸುರಹಂತ್ರೀ ಚ ಚಾಮುಂಡಾ ಖಡ್ಗಧಾರಿಣೀ ॥೧೧॥

ವಾರಾಹೀ ನಾರಸಿಂಹೀ ಚ ಭೀಮಾ ಭೈರವನಾದಿನೀ ।\\
ಶ್ರುತಿಃ ಸ್ಮೃತಿರ್ಧೃತಿರ್ಮೇಧಾ ವಿದ್ಯಾ ಲಕ್ಷ್ಮೀಃ ಸರಸ್ವತೀ ॥೧೨॥

ಅನಂತಾ ವಿಜಯಾಪರ್ಣಾ ಮಾನಸ್ತೋಕಾಪರಾಜಿತಾ ।\\
ಭವಾನೀ ಪಾರ್ವತೀ ದುರ್ಗಾ ಹೈಮವತ್ಯಂಬಿಕಾ ಶಿವಾ ॥೧೩॥

ಶಿವಾ ಭವಾನೀ ರುದ್ರಾಣೀ ಶಂಕರಾರ್ಧಶರೀರಿಣೀ ।\\
ಐರಾವತಗಜಾರೂಢಾ ವಜ್ರಹಸ್ತಾ ವರಪ್ರದಾ ॥೧೪॥

ನಿತ್ಯಾ ಸಕಲಕಲ್ಯಾಣೀ ಸರ್ವೈಶ್ವರ್ಯಪ್ರದಾಯಿನೀ ।\\
ದಾಕ್ಷಾಯಣೀ ಪದ್ಮಹಸ್ತಾ ಭಾರತೀ ಸರ್ವಮಂಗಲಾ ॥೧೫॥

ಕಲ್ಯಾಣೀ ಜನನೀ ದುರ್ಗಾ ಸರ್ವದುರ್ಗವಿನಾಶಿನೀ ।\\
ಇಂದ್ರಾಕ್ಷೀ ಸರ್ವಭೂತೇಶೀ ಸರ್ವರೂಪಾ ಮನೋನ್ಮನೀ ॥೧೬॥

ಮಹಿಷಮಸ್ತಕನೃತ್ಯವಿನೋದನಸ್ಫುಟರಣನ್ಮಣಿನೂಪುರಪಾದುಕಾ ।\\
ಜನನರಕ್ಷಣಮೋಕ್ಷವಿಧಾಯಿನೀ ಜಯತು ಶುಂಭನಿಶುಂಭನಿಷೂದಿನೀ ॥೧೭॥

ಸರ್ವಮಂಗಲಮಾಂಗಲ್ಯೇ ಶಿವೇ ಸರ್ವಾರ್ಥಸಾಧಿಕೇ ।\\
ಶರಣ್ಯೇ ತ್ರ್ಯಂಬಕೇ ದೇವಿ ನಾರಾಯಣಿ ನಮೋಽಸ್ತುತೇ ॥೧೮॥

ಓಂ ಹ್ರೀಂ ಶ್ರೀಂ ಇಂದ್ರಾಕ್ಷ್ಯೈ ನಮಃ। ಓಂ ನಮೋ ಭಗವತಿ ಇಂದ್ರಾಕ್ಷಿ ಸರ್ವಜನಸಮ್ಮೋಹಿನಿ ಕಾಲರಾತ್ರಿ ನಾರಸಿಂಹಿ ಸರ್ವಶತ್ರುಸಂಹಾರಿಣಿ  ಅನಲೇ ಅಭಯೇ  ಅಜಿತೇ ಅಪರಾಜಿತೇ ಮಹಾಸಿಂಹವಾಹಿನಿ ಮಹಿಷಾಸುರಮರ್ದಿನಿ  ಹನ ಹನ ಮರ್ದಯ ಮರ್ದಯ ಮಾರಯ ಮಾರಯ ಶೋಷಯ ಶೋಷಯ ದಾಹಯ ದಾಹಯ ಮಹಾಗ್ರಹಾನ್ ಸಂಹರ ಸಂಹರ ॥೧೯॥

ಯಕ್ಷಗ್ರಹ ರಾಕ್ಷಸಗ್ರಹ ಸ್ಕಂಧಗ್ರಹ ವಿನಾಯಕಗ್ರಹ ಬಾಲಗ್ರಹ ಕುಮಾರಗ್ರಹ ಭೂತಗ್ರಹ ಪ್ರೇತಗ್ರಹ ಪಿಶಾಚಗ್ರಹಾದೀನ್ ಮರ್ದಯ ಮರ್ದಯ ॥೨೦॥

ಭೂತಜ್ವರಪ್ರೇತಜ್ವರಪಿಶಾಚಜ್ವರಾನ್ ಸಂಹರ ಸಂಹರ । ಧೂಮಭೂತಾನ್ ಸಂದ್ರಾವಯ ಸಂದ್ರಾವಯ । ಶಿರಶ್ಶೂಲಕಟಿಶೂಲಾಂಗಶೂಲಪಾರ್ಶ್ವಶೂಲ ಪಾಂಡುರೋಗಾದೀನ್ ಸಂಹರ ಸಂಹರ ॥೨೧॥

ಯರಲವಶಷಸಹ  ಸರ್ವಗ್ರಹಾನ್ ತಾಪಯ ತಾಪಯ,ಸಂಹರ ಸಂಹರ ಛೇದಯ ಛೇದಯ ಹ್ರಾಂ ಹ್ರೀಂ ಹ್ರೂಂ ಫಟ್ ಸ್ವಾಹಾ ॥೨೨॥

ಗುಹ್ಯಾತ್ಗುಹ್ಯಗೋಪ್ತ್ರೀ ತ್ವಂ ಗೃಹಾಣಾಸ್ಮತ್ಕೃತಂ ಜಪಂ ।\\
ಸಿದ್ಧಿರ್ಭವತು ಮೇ ದೇವಿ ತ್ವತ್ಪ್ರಸಾದಾನ್ಮಯಿ ಸ್ಥಿರಾ ॥೨೩॥

ಫಲಶ್ರುತಿಃ\\
ನಾರಾಯಣ ಉವಾಚ ।\\
ಏವಂ ನಾಮವರೈರ್ದೇವೀ ಸ್ತುತಾ ಶಕ್ರೇಣ ಧೀಮತಾ ।\\
ಆಯುರಾರೋಗ್ಯಮೈಶ್ವರ್ಯಮಪಮೃತ್ಯುಭಯಾಪಹಂ ॥೧॥

ವರಂ ಪ್ರಾದಾನ್ಮಹೇಂದ್ರಾಯ ದೇವರಾಜ್ಯಂ ಚ ಶಾಶ್ವತಂ ।\\
ಇಂದ್ರಸ್ತೋತ್ರಮಿದಂ ಪುಣ್ಯಂ ಮಹದೈಶ್ವರ್ಯಕಾರಣಂ ॥೨ ॥

ಕ್ಷಯಾಪಸ್ಮಾರಕುಷ್ಠಾದಿತಾಪಜ್ವರನಿವಾರಣಂ ।\\
ಚೋರವ್ಯಾಘ್ರಭಯಾರಿಷ್ಠವೈಷ್ಣವಜ್ವರವಾರಣಂ ॥೩॥

ಮಾಹೇಶ್ವರಮಹಾಮಾರೀಸರ್ವಜ್ವರನಿವಾರಣಂ ।\\
ಶೀತಪೈತ್ತಕವಾತಾದಿಸರ್ವರೋಗನಿವಾರಣಂ ॥೪॥

ಶತಮಾವರ್ತಯೇದ್ಯಸ್ತು ಮುಚ್ಯತೇ ವ್ಯಾಧಿಬಂಧನಾತ್ ।\\
ಆವರ್ತನಸಹಸ್ರಾತ್ತು ಲಭತೇ ವಾಂಛಿತಂ ಫಲಂ ॥೫॥

ರಾಜಾನಂ ಚ ಸಮಾಪ್ನೋತಿ ಇಂದ್ರಾಕ್ಷೀಂ ನಾತ್ರ ಸಂಶಯ ।\\
ನಾಭಿಮಾತ್ರೇ ಜಲೇ ಸ್ಥಿತ್ವಾ ಸಹಸ್ರಪರಿಸಂಖ್ಯಯಾ ॥೬॥

ಜಪೇತ್ ಸ್ತೋತ್ರಮಿದಂ ಮಂತ್ರಂ ವಾಚಾಸಿದ್ಧಿರ್ಭವೇದ್ಧ್ರುವಂ ।\\
ಸಾಯಂ ಪ್ರಾತಃ ಪಠೇನ್ನಿತ್ಯಂ ಷಣ್ಮಾಸೈಃ ಸಿದ್ಧಿರುಚ್ಯತೇ ॥೭॥

ಸಂವತ್ಸರಮುಪಾಶ್ರಿತ್ಯ ಸರ್ವಕಾಮಾರ್ಥಸಿದ್ಧಯೇ ।\\
ಅನೇನ ವಿಧಿನಾ ಭಕ್ತ್ಯಾ ಮಂತ್ರಸಿದ್ಧಿಃ ಪ್ರಜಾಯತೇ ॥೮॥

ಸಂತುಷ್ಟಾ ಚ ಭವೇದ್ದೇವೀ ಪ್ರತ್ಯಕ್ಷಾ ಸಂಪ್ರಜಾಯತೇ ।\\
ಅಷ್ಟಮ್ಯಾಂ ಚ ಚತುರ್ದಶ್ಯಾಮಿದಂ ಸ್ತೋತ್ರಂ ಪಠೇನ್ನರಃ ॥೯॥

ಧಾವತಸ್ತಸ್ಯ ನಶ್ಯಂತಿ ವಿಘ್ನಸಂಖ್ಯಾ ನ ಸಂಶಯಃ ।\\
ಕಾರಾಗೃಹೇ ಯದಾ ಬದ್ಧೋ ಮಧ್ಯರಾತ್ರೇ ತದಾ ಜಪೇತ್ ॥೧೦॥

ದಿವಸತ್ರಯಮಾತ್ರೇಣ ಮುಚ್ಯತೇ ನಾತ್ರ ಸಂಶಯಃ ।\\
ಸಕಾಮೋ ಜಪತೇ ಸ್ತೋತ್ರಂ ಮಂತ್ರಪೂಜಾವಿಚಾರತಃ ॥೧೧॥

ಪಂಚಾಧಿಕೈರ್ದಶಾದಿತ್ಯೈರಿಯಂ ಸಿದ್ಧಿಸ್ತು ಜಾಯತೇ ।\\
ರಕ್ತಪುಷ್ಪೈ ರಕ್ತವಸ್ತ್ರೈ ರಕ್ತಚಂದನಚರ್ಚಿತೈಃ ॥೧೨॥

ಧೂಪದೀಪೈಶ್ಚ ನೈವೇದ್ಯೈಃ ಪ್ರಸನ್ನಾ ಭಗವತೀ ಭವೇತ್ ।\\
ಏವಂ ಸಂಪೂಜ್ಯ ಇಂದ್ರಾಕ್ಷೀಮಿಂದ್ರೇಣ ಪರಮಾತ್ಮನಾ ॥೧೩॥

ವರಂ ಲಬ್ಧಂ ದಿತೇಃ ಪುತ್ರಾ ಭಗವತ್ಯಾಃ ಪ್ರಸಾದತಃ ।\\
ಏತತ್ ಸ್ತ್ರೋತ್ರಂ ಮಹಾಪುಣ್ಯಂ ಜಪ್ಯಮಾಯುಷ್ಯವರ್ಧನಂ ॥೧೪॥

ಜ್ವರಾತಿಸಾರರೋಗಾಣಾಮಪಮೃತ್ಯೋರ್ಹರಾಯ ಚ ।\\
ದ್ವಿಜೈರ್ನಿತ್ಯಮಿದಂ ಜಪ್ಯಂ ಭಾಗ್ಯಾರೋಗ್ಯಮಭೀಪ್ಸುಭಿಃ ॥೧೫॥
\authorline{॥ಇತಿ ಇಂದ್ರಾಕ್ಷೀಸ್ತೋತ್ರಂ ಸಂಪೂರ್ಣಂ ॥}

%=========================================================================================================================


\section{ಶ್ರೀರಾಮರಕ್ಷಾಸ್ತೋತ್ರ}
\addcontentsline{toc}{section}{ಶ್ರೀರಾಮರಕ್ಷಾಸ್ತೋತ್ರ}

ಅಸ್ಯ ಶ್ರೀರಾಮರಕ್ಷಾಸ್ತೋತ್ರಮಂತ್ರಸ್ಯ । ಬುಧಕೌಶಿಕ ಋಷಿಃ । ಶ್ರೀಸೀತಾರಾಮಚಂದ್ರೋ ದೇವತಾ । ಅನುಷ್ಟುಪ್ ಛಂದಃ । ಸೀತಾ ಶಕ್ತಿಃ । ಶ್ರೀಮದ್ ಹನುಮಾನ ಕೀಲಕಂ । ಶ್ರೀರಾಮಚಂದ್ರಪ್ರೀತ್ಯರ್ಥೇ ರಾಮರಕ್ಷಾಸ್ತೋತ್ರಜಪೇ ವಿನಿಯೋಗಃ ॥

ಅಥ ಧ್ಯಾನಂ ।\\
ಧ್ಯಾಯೇದಾಜಾನುಬಾಹುಂ ಧೃತಶರಧನುಷಂ ಬದ್ಧಪದ್ಮಾಸನಸ್ಥಂ\\
ಪೀತಂ ವಾಸೋ ವಸಾನಂ ನವಕಮಲದಲಸ್ಪರ್ಧಿನೇತ್ರಂ ಪ್ರಸನ್ನಂ ।\\
ವಾಮಾಂಕಾರೂಢ ಸೀತಾಮುಖಕಮಲಮಿಲಲ್ಲೋಚನಂ ನೀರದಾಭಂ\\
ನಾನಾಲಂಕಾರದೀಪ್ತಂ ದಧತಮುರುಜಟಾಮಂಡನಂ ರಾಮಚಂದ್ರಂ ॥

ಚರಿತಂ ರಘುನಾಥಸ್ಯ ಶತಕೋಟಿ ಪ್ರವಿಸ್ತರಂ ।\\
ಏಕೈಕಮಕ್ಷರಂ ಪುಂಸಾಂ ಮಹಾಪಾತಕನಾಶನಂ ॥೧॥

ಧ್ಯಾತ್ವಾ ನೀಲೋತ್ಪಲಶ್ಯಾಮಂ ರಾಮಂ ರಾಜೀವಲೋಚನಂ ।\\
ಜಾನಕೀಲಕ್ಷ್ಮಣೋಪೇತಂ ಜಟಾಮುಕುಟಮಂಡಿತಂ ॥೨॥

ಸಾಸಿತೂಣಧನುರ್ಬಾಣಪಾಣಿಂ ನಕ್ತಂಚರಾಂತಕಂ ।\\
ಸ್ವಲೀಲಯಾ ಜಗತ್ರಾತುಂ ಆವಿರ್ಭೂತಂ ಅಜಂ ವಿಭುಂ ॥೩॥

ರಾಮರಕ್ಷಾಂ ಪಠೇತ್ಪ್ರಾಜ್ಞಃ ಪಾಪಘ್ನೀಂ ಸರ್ವಕಾಮದಾಂ ।\\
ಶಿರೋಮೇ ರಾಘವಃ ಪಾತು ಭಾಲಂ ದಶರಥಾತ್ಮಜಃ ॥೪॥

ಕೌಸಲ್ಯೇಯೋ ದೃಶೌ ಪಾತು ವಿಶ್ವಾಮಿತ್ರಪ್ರಿಯಶ್ರುತೀ ।\\
ಘ್ರಾಣಂ ಪಾತು ಮಖತ್ರಾತಾ ಮುಖಂ ಸೌಮಿತ್ರಿವತ್ಸಲಃ ॥೫॥

ಜಿಹ್ವಾಂ ವಿದ್ಯಾನಿಧಿಃ ಪಾತು ಕಂಠಂ ಭರತವಂದಿತಃ ।\\
ಸ್ಕಂಧೌ ದಿವ್ಯಾಯುಧಃ ಪಾತು ಭುಜೌ ಭಗ್ನೇಶಕಾರ್ಮುಕಃ ॥೬॥

ಕರೌ ಸೀತಾಪತಿಃ ಪಾತು ಹೃದಯಂ ಜಾಮದಗ್ನ್ಯಜಿತ್ ।\\
ಮಧ್ಯಂ ಪಾತು ಖರಧ್ವಂಸೀ ನಾಭಿಂ ಜಾಂಬವದಾಶ್ರಯಃ ॥೭॥

ಸುಗ್ರೀವೇಶಃ ಕಟೀ ಪಾತು ಸಕ್ಥಿನೀ ಹನುಮತ್ಪ್ರಭುಃ ।\\
ಊರೂ ರಘೂತ್ತಮಃ ಪಾತು ರಕ್ಷಃಕುಲವಿನಾಶಕೃತ್ ॥೮॥

ಜಾನುನೀ ಸೇತುಕೃತ್ಪಾತು ಜಂಘೇ ದಶಮುಖಾಂತಕಃ ।\\
ಪಾದೌ ಬಿಭೀಷಣಶ್ರೀದಃ ಪಾತು ರಾಮೋಽಖಿಲಂ ವಪುಃ ॥೯॥

ಏತಾಂ ರಾಮಬಲೋಪೇತಾಂ ರಕ್ಷಾಂ ಯಃ ಸುಕೃತೀ ಪಠೇತ್ ।\\
ಸ ಚಿರಾಯುಃ ಸುಖೀ ಪುತ್ರೀ ವಿಜಯೀ ವಿನಯೀ ಭವೇತ್ ॥೧೦॥

ಪಾತಾಲಭೂತಲವ್ಯೋಮಚಾರಿಣಶ್ಛದ್ಮಚಾರಿಣಃ ।\\
ನ ದ್ರಷ್ಟುಮಪಿ ಶಕ್ತಾಸ್ತೇ ರಕ್ಷಿತಂ ರಾಮನಾಮಭಿಃ ॥೧೧॥

ರಾಮೇತಿ ರಾಮಭದ್ರೇತಿ ರಾಮಚಂದ್ರೇತಿ ವಾ ಸ್ಮರನ್ ।\\
ನರೋ ನ ಲಿಪ್ಯತೇ ಪಾಪೈಃ ಭುಕ್ತಿಂ ಮುಕ್ತಿಂ ಚ ವಿಂದತಿ ॥೧೨॥

ಜಗಜೈತ್ರೈಕಮಂತ್ರೇಣ ರಾಮನಾಮ್ನಾಭಿರಕ್ಷಿತಂ ।\\
ಯಃ ಕಂಠೇ ಧಾರಯೇತ್ತಸ್ಯ ಕರಸ್ಥಾಃ ಸರ್ವಸಿದ್ಧಯಃ ॥೧೩॥

ವಜ್ರಪಂಜರನಾಮೇದಂ ಯೋ ರಾಮಕವಚಂ ಸ್ಮರೇತ್ ।\\
ಅವ್ಯಾಹತಾಜ್ಞಃ ಸರ್ವತ್ರ ಲಭತೇ ಜಯಮಂಗಲಂ ॥೧೪॥

ಆದಿಷ್ಟವಾನ್ ಯಥಾ ಸ್ವಪ್ನೇ ರಾಮರಕ್ಷಾಂಮಿಮಾಂ ಹರಃ ।\\
ತಥಾ ಲಿಖಿತವಾನ್ ಪ್ರಾತಃ ಪ್ರಬುದ್ಧೋ ಬುಧಕೌಶಿಕಃ ॥೧೫॥

ಆರಾಮಃ ಕಲ್ಪವೃಕ್ಷಾಣಾಂ ವಿರಾಮಃ ಸಕಲಾಪದಾಂ ।\\
ಅಭಿರಾಮಸ್ತ್ರಿಲೋಕಾನಾಂ ರಾಮಃ ಶ್ರೀಮಾನ್ ಸ ನಃ ಪ್ರಭುಃ ॥೧೬॥

ತರುಣೌ ರೂಪಸಂಪನ್ನೌ ಸುಕುಮಾರೌ ಮಹಾಬಲೌ ।\\
ಪುಂಡರೀಕವಿಶಾಲಾಕ್ಷೌ ಚೀರಕೃಷ್ಣಾಜಿನಾಂಬರೌ ॥೧೭॥

ಫಲಮೂಲಾಶಿನೌ ದಾಂತೌ ತಾಪಸೌ ಬ್ರಹ್ಮಚಾರಿಣೌ ।\\
ಪುತ್ರೌ ದಶರಥಸ್ಯೈತೌ ಭ್ರಾತರೌ ರಾಮಲಕ್ಷ್ಮಣೌ ॥೧೮॥

ಶರಣ್ಯೌ ಸರ್ವಸತ್ತ್ವಾನಾಂ ಶ್ರೇಷ್ಠೌ ಸರ್ವಧನುಷ್ಮತಾಂ ।\\
ರಕ್ಷಃ ಕುಲನಿಹಂತಾರೌ ತ್ರಾಯೇತಾಂ ನೋ ರಘೂತ್ತಮೌ ॥೧೯॥

ಆತ್ತಸಜ್ಜಧನುಷಾವಿಷುಸ್ಪೃಶಾವಕ್ಷಯಾಶುಗನಿಷಂಗಸಂಗಿನೌ ।\\
ರಕ್ಷಣಾಯ ಮಮ ರಾಮಲಕ್ಷ್ಮಣಾವಗ್ರತಃ ಪಥಿ ಸದೈವ ಗಚ್ಛತಾಂ ॥೨೦॥

ಸನ್ನದ್ಧಃ ಕವಚೀ ಖಡ್ಗೀ ಚಾಪಬಾಣಧರೋ ಯುವಾ ।\\
ಗಚ್ಛನ್ಮನೋರಥೋಽಸ್ಮಾಕಂ ರಾಮಃ ಪಾತು ಸಲಕ್ಷ್ಮಣಃ ॥೨೧॥

ರಾಮೋ ದಾಶರಥಿಃ ಶೂರೋ ಲಕ್ಷ್ಮಣಾನುಚರೋ ಬಲೀ ।\\
ಕಾಕುತ್ಸ್ಥಃ ಪುರುಷಃ ಪೂರ್ಣಃ ಕೌಸಲ್ಯೇಯೋ ರಘುತ್ತಮಃ ॥೨೨॥

ವೇದಾಂತವೇದ್ಯೋ ಯಜ್ಞೇಶಃ ಪುರಾಣಪುರುಷೋತ್ತಮಃ ।\\
ಜಾನಕೀವಲ್ಲಭಃ ಶ್ರೀಮಾನ್ ಅಪ್ರಮೇಯ ಪರಾಕ್ರಮಃ ॥೨೩॥

ಇತ್ಯೇತಾನಿ ಜಪನ್ನಿತ್ಯಂ ಮದ್ಭಕ್ತಃ ಶ್ರದ್ಧಯಾನ್ವಿತಃ ।\\
ಅಶ್ವಮೇಧಾಧಿಕಂ ಪುಣ್ಯಂ ಸಂಪ್ರಾಪ್ನೋತಿ ನ ಸಂಶಯಃ ॥೨೪॥

ರಾಮಂ ದುರ್ವಾದಲಶ್ಯಾಮಂ ಪದ್ಮಾಕ್ಷಂ ಪೀತವಾಸಸಂ ।\\
ಸ್ತುವಂತಿ ನಾಮಭಿರ್ದಿವ್ಯೈಃ ನ ತೇ ಸಂಸಾರಿಣೋ ನರಃ ॥೨೫॥

ರಾಮಂ ಲಕ್ಷ್ಮಣಪೂರ್ವಜಂ ರಘುವರಂ ಸೀತಾಪತಿಂ ಸುಂದರಂ \\
ಕಾಕುತ್ಸ್ಥಂ ಕರುಣಾರ್ಣವಂ ಗುಣನಿಧಿಂ ವಿಪ್ರಪ್ರಿಯಂ ಧಾರ್ಮಿಕಂ ।\\
ರಾಜೇಂದ್ರಂ ಸತ್ಯಸಂಧಂ ದಶರಥತನಯಂ ಶ್ಯಾಮಲಂ ಶಾಂತಮೂರ್ತಿಂ\\
ವಂದೇ ಲೋಕಾಭಿರಾಮಂ ರಘುಕುಲತಿಲಕಂ ರಾಘವಂ ರಾವಣಾರಿಂ ॥೨೬॥

ರಾಮಾಯ ರಾಮಭದ್ರಾಯ ರಾಮಚಂದ್ರಾಯ ವೇಧಸೇ ।\\
ರಘುನಾಥಾಯ ನಾಥಾಯ ಸೀತಾಯಾಃ ಪತಯೇ ನಮಃ ॥೨೭॥

ಶ್ರೀರಾಮ ರಾಮ ರಘುನಂದನ ರಾಮ ರಾಮ\\
ಶ್ರೀರಾಮ ರಾಮ ಭರತಾಗ್ರಜ ರಾಮ ರಾಮ ।\\
ಶ್ರೀರಾಮ ರಾಮ ರಣಕರ್ಕಶ ರಾಮ ರಾಮ\\
ಶ್ರೀರಾಮ ರಾಮ ಶರಣಂ ಭವ ರಾಮ ರಾಮ ॥೨೮॥

ಶ್ರೀರಾಮಚಂದ್ರಚರಣೌ ಮನಸಾ ಸ್ಮರಾಮಿ\\
ಶ್ರೀರಾಮಚಂದ್ರಚರಣೌ ವಚಸಾ ಗೃಣಾಮಿ ।\\
ಶ್ರೀರಾಮಚಂದ್ರಚರಣೌ ಶಿರಸಾ ನಮಾಮಿ\\
ಶ್ರೀರಾಮಚಂದ್ರಚರಣೌ ಶರಣಂ ಪ್ರಪದ್ಯೇ ॥೨೯॥

ಮಾತಾ ರಾಮೋ ಮತ್ಪಿತಾ ರಾಮಚಂದ್ರಃ\\
ಸ್ವಾಮೀ ರಾಮೋ ಮತ್ಸಖಾ ರಾಮಚಂದ್ರಃ ।\\
ಸರ್ವಸ್ವಂ ಮೇ ರಾಮಚಂದ್ರೋ ದಯಾಲು\\
ರ್ನಾನ್ಯಂ ಜಾನೇ ನೈವ ಜಾನೇ ನ ಜಾನೇ ॥೩೦॥

ದಕ್ಷಿಣೇ ಲಕ್ಷ್ಮಣೋ ಯಸ್ಯ ವಾಮೇ ತು ಜನಕಾತ್ಮಜಾ ।\\
ಪುರತೋ ಮಾರುತಿರ್ಯಸ್ಯ ತಂ ವಂದೇ ರಘುನಂದನಂ ॥೩೧॥

ಲೋಕಾಭಿರಾಮಂ ರಣರಂಗಧೀರಂ\\
ರಾಜೀವನೇತ್ರಂ ರಘುವಂಶನಾಥಂ ।\\
ಕಾರುಣ್ಯರೂಪಂ ಕರುಣಾಕರಂ ತಂ\\
ಶ್ರೀರಾಮಚಂದ್ರಂ ಶರಣಂ ಪ್ರಪದ್ಯೇ ॥೩೨॥

ಮನೋಜವಂ ಮಾರುತತುಲ್ಯವೇಗಂ\\
ಜಿತೇಂದ್ರಿಯಂ ಬುದ್ಧಿಮತಾಂ ವರಿಷ್ಠಂ ।\\
ವಾತಾತ್ಮಜಂ ವಾನರಯೂಥಮುಖ್ಯಂ\\
ಶ್ರೀರಾಮದೂತಂ ಶರಣಂ ಪ್ರಪದ್ಯೇ ॥೩೩॥

ಕೂಜಂತಂ ರಾಮ ರಾಮೇತಿ ಮಧುರಂ ಮಧುರಾಕ್ಷರಂ ।\\
ಆರುಹ್ಯ ಕವಿತಾಶಾಖಾಂ ವಂದೇ ವಾಲ್ಮೀಕಿಕೋಕಿಲಂ ॥೩೪॥

ಆಪದಾಂ ಅಪಹರ್ತಾರಂ ದಾತಾರಂ ಸರ್ವಸಂಪದಾಂ ।\\
ಲೋಕಾಭಿರಾಮಂ ಶ್ರೀರಾಮಂ ಭೂಯೋ ಭೂಯೋ ನಮಾಮ್ಯಹಂ ॥೩೫॥

ಭರ್ಜನಂ ಭವಬೀಜಾನಾಂ ಅರ್ಜನಂ ಸುಖಸಂಪದಾಂ ।\\
ತರ್ಜನಂ ಯಮದೂತಾನಾಂ ರಾಮ ರಾಮೇತಿ ಗರ್ಜನಂ ॥೩೬॥

ರಾಮೋ ರಾಜಮಣಿಃ ಸದಾ ವಿಜಯತೇ ರಾಮಂ ರಮೇಶಂ ಭಜೇ\\
ರಾಮೇಣಾಭಿಹತಾ ನಿಶಾಚರಚಮೂ ರಾಮಾಯ ತಸ್ಮೈ ನಮಃ ।\\
ರಾಮಾನ್ನಾಸ್ತಿ ಪರಾಯಣಂ ಪರತರಂ ರಾಮಸ್ಯ ದಾಸೋಸ್ಮ್ಯಹಂ\\
ರಾಮೇ ಚಿತ್ತಲಯಃ ಸದಾ ಭವತು ಮೇ ಭೋ ರಾಮ ಮಾಮುದ್ಧರ ॥೩೭॥

ರಾಮ ರಾಮೇತಿ ರಾಮೇತಿ ರಮೇ ರಾಮೇ ಮನೋರಮೇ ।\\
ಸಹಸ್ರನಾಮ ತತ್ತುಲ್ಯಂ ರಾಮನಾಮ ವರಾನನೇ ॥೩೮॥
\authorline{ಇತಿ ಶ್ರೀಬುಧಕೌಶಿಕವಿರಚಿತಂ ಶ್ರೀರಾಮರಕ್ಷಾಸ್ತೋತ್ರಂ ಸಂಪೂರ್ಣಂ ॥}
%=========================================================================================================================
\section{ಶ್ರೀಶೀತಲಾಸ್ತೋತ್ರ}
\addcontentsline{toc}{section}{ಶ್ರೀಶೀತಲಾಸ್ತೋತ್ರ}
ಅಸ್ಯ ಶ್ರೀಶೀತಲಾಸ್ತೋತ್ರಸ್ಯ ಮಹಾದೇವ ಋಷಿಃ~। ಅನುಷ್ಟುಪ್ ಛಂದಃ। ಶೀತಲಾ ದೇವತಾ। ಲಕ್ಷ್ಮೀ ಬೀಜಂ। ಭವಾನೀ ಶಕ್ತಿಃ~। ಜಪೇ ವಿನಿಯೋಗಃ ॥

ಈಶ್ವರ ಉವಾಚ ॥\\
ವಂದೇಹಂ ಶೀತಲಾಂ ದೇವೀಂ ರಾಸಭಸ್ಥಾಂ ದಿಗಂಬರಾಂ।\\
ಮಾರ್ಜನೀಕಲಶೋಪೇತಾಂ ಶೂರ್ಪಾಲಂಕೃತಮಸ್ತಕಾಂ॥೧॥

ವಂದೇಹಂ ಶೀತಲಾಂ ದೇವೀಂ ಸರ್ವರೋಗಭಯಾಪಹಾಂ।\\
ಯಾಮಾಸಾದ್ಯ ನಿವರ್ತೇತ ವಿಸ್ಫೋಟಕಭಯಂ ಮಹತ್॥೨॥

ಶೀತಲೇ ಶೀತಲೇ ಚೇತಿ ಯೋ ಬ್ರೂಯಾದ್ದಾಹಪೀಡಿತಃ।\\
ವಿಸ್ಫೋಟಕಭಯಂ ಘೋರಂ ಕ್ಷಿಪ್ರಂ ತಸ್ಯ ಪ್ರಣಶ್ಯತಿ॥೩॥

ಯಸ್ತ್ವಾಮುದಕಮಧ್ಯೇ ತು ಸ್ಥಿತ್ವಾ ಪೂಜಯತೇ ನರಃ।\\
ವಿಸ್ಫೋಟಕಭಯಂ ಘೋರಂ ಗೃಹೇ ತಸ್ಯ ನ ಜಾಯತೇ॥೪॥

ಶೀತಲೇ ಜ್ವರದಗ್ಧಸ್ಯ ಪೂತಿಗಂಧಯುತಸ್ಯ ಚ~।\\
ಪ್ರಣಷ್ಟಚಕ್ಷುಷಃ ಪುಂಸಸ್ತ್ವಾಮಾಹುರ್ಜೀವನೌಷಧಂ॥೫॥

ಶೀತಲೇ ತನುಜಾನ್ ರೋಗಾನ್ನೃಣಾಂ ಹರಸಿ ದುಸ್ತ್ಯಜಾನ್।\\
ವಿಸ್ಫೋಟಕವಿದೀರ್ಣಾನಾಂ ತ್ವಮೇಕಾಮೃತವರ್ಷಿಣೀ॥೬॥

ಗಲಗಂಡಗ್ರಹಾ ರೋಗಾ ಯೇ ಚಾನ್ಯೇ ದಾರುಣಾ ನೃಣಾಂ।\\
ತ್ವದನುಧ್ಯಾನಮಾತ್ರೇಣ ಶೀತಲೇ ಯಾಂತಿ ಸಂಕ್ಷಯಂ॥೭॥

ನ ಮಂತ್ರೋ ನೌಷಧಂ ತಸ್ಯ ಪಾಪರೋಗಸ್ಯ ವಿದ್ಯತೇ।\\
ತ್ವಾಮೇಕಾಂ ಶೀತಲೇ ಧಾತ್ರೀಂ ನಾನ್ಯಾಂ ಪಶ್ಯಾಮಿ ದೇವತಾಂ॥೮॥

ಮೃಣಾಲತಂತುಸದೃಶೀಂ ನಾಭಿಹೃನ್ಮಧ್ಯಸಂಸ್ಥಿತಾಂ।\\
ಯಸ್ತ್ವಾಂ ಸಂಚಿಂತಯೇದ್ದೇವಿ ತಸ್ಯ ಮೃತ್ಯುರ್ನ ಜಾಯತೇ॥೯॥

ಅಷ್ಟಕಂ ಶೀತಲಾದೇವ್ಯಾ ಯೋ ನರಃ ಪ್ರಪಠೇತ್ಸದಾ।\\
ವಿಸ್ಫೋಟಕಭಯಂ ಘೋರಂ ಗೃಹೇ ತಸ್ಯ ನ ಜಾಯತೇ॥೧೦॥

ಶ್ರೋತವ್ಯಂ ಪಠಿತವ್ಯಂ ಚ ಶ್ರದ್ಧಾಭಕ್ತಿಸಮನ್ವಿತೈಃ।\\
ಉಪಸರ್ಗವಿನಾಶಾಯ ಪರಂ ಸ್ವಸ್ತ್ಯಯನಂ ಮಹತ್॥೧೧॥

ಶೀತಲೇ ತ್ವಂ ಜಗನ್ಮಾತಾ ಶೀತಲೇ ತ್ವಂ ಜಗತ್ಪಿತಾ।\\
ಶೀತಲೇ ತ್ವಂ ಜಗದ್ಧಾತ್ರೀ ಶೀತಲಾಯೈ ನಮೋ ನಮಃ॥೧೨॥

ರಾಸಭೋ ಗರ್ದಭಶ್ಚೈವ ಖರೋ ವೈಶಾಖನಂದನಃ।\\
ಶೀತಲಾವಾಹನಶ್ಚೈವ ದೂರ್ವಾಕಂದನಿಕೃಂತನಃ॥೧೩॥

ಏತಾನಿ ಖರನಾಮಾನಿ ಶೀತಲಾಗ್ರೇ ತು ಯಃ ಪಠೇತ್।\\
ತಸ್ಯ ಗೇಹೇ ಶಿಶೂನಾಂ ಚ ಶೀತಲಾರುಡ್ ನ ಜಾಯತೇ॥೧೪॥

ಶೀತಲಾಷ್ಟಕಮೇವೇದಂ ನ ದೇಯಂ ಯಸ್ಯ ಕಸ್ಯಚಿತ್।\\
ದಾತವ್ಯಂ ಚ ಸದಾ ತಸ್ಮೈ ಶ್ರದ್ಧಾಭಕ್ತಿಯುತಾಯ ವೈ॥೧೫॥

\authorline{॥ಇತಿ ಶ್ರೀಸ್ಕಂದಮಹಾಪುರಾಣೇ ಶೀತಲಾಷ್ಟಕಂ ಸಂಪೂರ್ಣಂ॥}
%====================================================================================================
\section{ ಶ್ರೀ ಅನಘಾಕವಚಾಷ್ಟಕಂ }
\addcontentsline{toc}{section}{ ಶ್ರೀ ಅನಘಾಕವಚಾಷ್ಟಕಂ }
ಶಿರೋ ಮೇ ಅನಘಾ ಪಾತು ಭಾಲಂ ಮೇ ದತ್ತಭಾಮಿನೀ।\\
ಭ್ರೂಮಧ್ಯಂ ಯೋಗಿನೀ ಪಾತು ನೇತ್ರೇ ಪಾತು ಸುದರ್ಶಿನೀ॥೧॥

ನಾಸಾರಂಧ್ರದ್ವಯಂ ಪಾತು ಯೋಗೇಶೀ ಭಕ್ತವತ್ಸಲಾ~।\\
ಮುಖಂ ಮೇ ಮಧುವಾಕ್ಪಾತು ದತ್ತಚಿತ್ತವಿಹಾರಿಣೀ॥೨॥

ತ್ರಿಕಂಠೀ ಪಾತು ಮೇ ಕಂಠಂ ವಾಚಂ ವಾಚಸ್ಪತಿಪ್ರಿಯಾ।\\
ಸ್ಕಂಧೌ ಮೇ ತ್ರಿಗುಣಾ ಪಾತು ಭುಜೌ ಕಮಲಧಾರಿಣೀ॥೩॥

ಕರೌ ಸೇವಾರತಾ ಪಾತು ಹೃದಯಂ ಮಂದಹಾಸಿನೀ।\\
ಉದರಂ ಅನ್ನದಾ ಪಾತು ಸ್ವಯಂಜಾ ನಾಭಿಮಂಡಲಂ॥೪॥

ಕಮನೀಯಾ ಕಟಿಂ ಪಾತು ಗುಹ್ಯಂ ಗುಹ್ಯೇಶ್ವರೀ ಸದಾ।\\
ಊರೂ ಮೇ ಪಾತು ಜಂಭಘ್ನೀ ಜಾನುನೀ ರೇಣುಕೇಷ್ಟದಾ ॥೫॥

ಪಾದೌ ಪಾದಸ್ಥಿತಾ ಪಾತು ಪುತ್ರದಾ ವೈ ಖಿಲಂ ವಪುಃ।\\
ವಾಮಗಾ ಪಾತು ವಾಮಾಂಗಂ ದಕ್ಷಾಂಗಂ ಗುರುಗಾಮಿನೀ॥೬॥

ಗೃಹಂ ಮೇ ದತ್ತಗೃಹಿಣೀ ಬಾಹ್ಯೇ ಸರ್ವಾತ್ಮಿಕಾವತು।\\
ತ್ರಿಕಾಲೇ ಸರ್ವದಾ ರಕ್ಷೇತ್ ಪತಿಶುಶ್ರೂಷಣೋತ್ಸುಕಾ ॥೭॥

ಜಾಯಾಂ ಮೇ ದತ್ತವಾಮಾಂಗೀ ಅಷ್ಟಪುತ್ರಾ ಸುತೋವತು।\\
ಗೋತ್ರಮತ್ರಿಸ್ನುಷಾ ರಕ್ಷೇದ್ ಅನಘಾ ಭಕ್ತರಕ್ಷಣೀ॥೮॥

ಯಃ ಪಠೇದನಘಾಕವಚಂ ನಿತ್ಯಂ ಭಕ್ತಿಯುತೋ ನರಃ।\\
ತಸ್ಮೈ ಭವತ್ಯನಘಾಂಬಾ ವರದಾ ಸರ್ವಭಾಗ್ಯದಾ॥೯॥
\authorline{ಇತ್ಯನಘಾಕವಚಾಷ್ಟಕಮ್}
%===========================================================================================
\section{ಶ್ರೀದ್ವಾದಶನಾಮಪಂಜರಸ್ತೋತ್ರಂ}
\addcontentsline{toc}{section}{ಶ್ರೀದ್ವಾದಶನಾಮಪಂಜರಸ್ತೋತ್ರಂ}
ಪುರಸ್ತಾತ್ ಕೇಶವಃ ಪಾತು ಚಕ್ರೀ ಜಾಂಬೂನದಪ್ರಭಃ ।\\
ಪಶ್ಚಾನ್ನಾರಾಯಣಃ ಶಂಖೀ ನೀಲಜೀಮೂತಸನ್ನಿಭಃ ॥೧॥

ಇಂದೀವರದಲಶ್ಯಾಮೋ ಮಾಧವೋರ್ಧ್ವಂ ಗದಾಧರಃ ।\\
ಗೋವಿಂದೋ ದಕ್ಷಿಣೇ ಪಾರ್ಶ್ವೇ ಧನ್ವೀ ಚಂದ್ರಪ್ರಭೋ ಮಹಾನ್ ॥೨॥

ಉತ್ತರೇ ಹಲಭೃದ್ವಿಷ್ಣುಃ ಪದ್ಮಕಿಂಜಲ್ಕಸನ್ನಿಭಃ ।\\
ಆಗ್ನೇಯ್ಯಾಮರವಿಂದಾಭೋ ಮುಸಲೀ ಮಧುಸೂದನಃ ॥೩॥

ತ್ರಿವಿಕ್ರಮಃ ಖಡ್ಗಪಾಣಿರ್ನಿರೃತ್ಯಾಂ ಜ್ವಲನಪ್ರಭಃ ।\\
ವಾಯವ್ಯಾಂ ವಾಮನೋ ವಜ್ರೀ ತರುಣಾದಿತ್ಯದೀಪ್ತಿಮಾನ್ ॥೪॥

ಐಶಾನ್ಯಾಂ ಪುಂಡರೀಕಾಭಃ ಶ್ರೀಧರಃ ಪಟ್ಟಸಾಯುಧಃ ।\\
ವಿದ್ಯುತ್ಪ್ರಭೋ ಹೃಷೀಕೇಶೋ ಹ್ಯವಾಚ್ಯಾಂ ದಿಶಿ ಮುದ್ಗರೀ ॥೫॥

ಹೃತ್ಪದ್ಮೇ ಪದ್ಮನಾಭೋ ಮೇ ಸಹಸ್ರಾರ್ಕಸಮಪ್ರಭಃ ।\\
ಸರ್ವಾಯುಧಃ ಸರ್ವಶಕ್ತಿಃ ಸರ್ವಜ್ಞಃ ಸರ್ವತೋಮುಖಃ ॥೬॥

ಇಂದ್ರಗೋಪಕಸಂಕಾಶಃ ಪಾಶಹಸ್ತೋಽಪರಾಜಿತಃ ।\\
ಸ ಬಾಹ್ಯಾಭ್ಯಂತರಂ ದೇಹಂ ವ್ಯಾಪ್ಯ ದಾಮೋದರಃ ಸ್ಥಿತಃ ॥೭॥

ಏವಂ ಸರ್ವತ್ರಮಚ್ಛಿದ್ರಂ ನಾಮದ್ವಾದಶಪಂಜರಂ ।\\
ಪ್ರವಿಷ್ಟೋಽಹಂ ನ ಮೇ ಕಿಂಚಿದ್ಭಯಮಸ್ತಿ ಕದಾಚನ ॥೮॥

ಭಯಂ ನಾಸ್ತಿ ಕದಾಚನ ಓಂ ನಮ ಇತಿ ॥
\authorline{ಇತಿ ಶ್ರೀದ್ವಾದಶನಾಮಪಂಜರಸ್ತೋತ್ರಂ ಸಂಪೂರ್ಣಂ ।}
%=====================================================================================================================

\section{ಶ್ರೀಶನಿವಜ್ರಪಂಜರಕವಚಂ}
\addcontentsline{toc}{section}{ಶ್ರೀಶನಿವಜ್ರಪಂಜರಕವಚಂ}
ಓಂ ಅಸ್ಯ ಶ್ರೀಶನೈಶ್ಚರವಜ್ರಪಂಜರ ಕವಚಸ್ಯ ಕಶ್ಯಪ ಋಷಿಃ । ಅನುಷ್ಟುಪ್ ಛಂದಃ । ಶ್ರೀ ಶನೈಶ್ಚರೋ ದೇವತಾ । ಶ್ರೀಶನೈಶ್ಚರ ಪ್ರೀತ್ಯರ್ಥೇ ಜಪೇ ವಿನಿಯೋಗಃ ॥

ಧ್ಯಾನಂ ।\\
ನೀಲಾಂಬರೋ ನೀಲವಪುಃ ಕಿರೀಟೀ ಗೃಧ್ರಸ್ಥಿತಸ್ತ್ರಾಸಕರೋ ಧನುಷ್ಮಾನ್ ।\\
ಚತುರ್ಭುಜಃ ಸೂರ್ಯಸುತಃ ಪ್ರಸನ್ನಃ ಸದಾ ಮಮ ಸ್ಯಾದ್ ವರದಃ ಪ್ರಶಾಂತಃ ॥೧॥

ಬ್ರಹ್ಮೋವಾಚ ॥\\
ಶೃಣುಧ್ವಮೃಷಯಃ ಸರ್ವೇ ಶನಿಪೀಡಾಹರಂ ಮಹತ್ ।\\
ಕವಚಂ ಶನಿರಾಜಸ್ಯ ಸೌರೇರಿದಮನುತ್ತಮಂ ॥೨॥

ಕವಚಂ ದೇವತಾವಾಸಂ ವಜ್ರಪಂಜರಸಂಜ್ಞಕಂ ।\\
ಶನೈಶ್ಚರಪ್ರೀತಿಕರಂ ಸರ್ವಸೌಭಾಗ್ಯದಾಯಕಂ ॥೩॥

ಓಂ ಶ್ರೀಶನೈಶ್ಚರಃ ಪಾತು ಭಾಲಂ ಮೇ ಸೂರ್ಯನಂದನಃ ।\\
ನೇತ್ರೇ ಛಾಯಾತ್ಮಜಃ ಪಾತು ಪಾತು ಕರ್ಣೌ ಯಮಾನುಜಃ ॥೪॥

ನಾಸಾಂ ವೈವಸ್ವತಃ ಪಾತು ಮುಖಂ ಮೇ ಭಾಸ್ಕರಃ ಸದಾ ।\\
ಸ್ನಿಗ್ಧಕಂಠಶ್ಚ ಮೇ ಕಂಠಂ ಭುಜೌ ಪಾತು ಮಹಾಭುಜಃ ॥೫॥

ಸ್ಕಂಧೌ ಪಾತು ಶನಿಶ್ಚೈವ ಕರೌ ಪಾತುಶುಭಪ್ರದಃ ।\\
ವಕ್ಷಃ ಪಾತು ಯಮಭ್ರಾತಾ ಕುಕ್ಷಿಂ ಪಾತ್ವಸಿತಸ್ತಥಾ ॥೬॥

ನಾಭಿಂ ಗ್ರಹಪತಿಃ ಪಾತು ಮಂದಃ ಪಾತು ಕಟಿಂ ತಥಾ ।\\
ಊರೂ ಮಮಾಂತಕಃ ಪಾತು ಯಮೋ ಜಾನುಯುಗಂ ತಥಾ ॥೭॥

ಪಾದೌ ಮಂದಗತಿಃ ಪಾತು ಸರ್ವಾಂಗಂ ಪಾತು ಪಿಪ್ಪಲಃ ।\\
ಅಂಗೋಪಾಂಗಾನಿ ಸರ್ವಾಣಿ ರಕ್ಷೇನ್ ಮೇ ಸೂರ್ಯನಂದನಃ ॥೮॥

ಇತ್ಯೇತತ್ ಕವಚಂ ದಿವ್ಯಂ ಪಠೇತ್ ಸೂರ್ಯಸುತಸ್ಯ ಯಃ ।\\
ನ ತಸ್ಯ ಜಾಯತೇ ಪೀಡಾ ಪ್ರೀತೋ ಭವತಿ ಸೂರ್ಯಜಃ ॥೯॥

ವ್ಯಯಜನ್ಮದ್ವಿತೀಯಸ್ಥೋ ಮೃತ್ಯುಸ್ಥಾನಗತೋಽಪಿ ವಾ ।\\
ಕಲತ್ರಸ್ಥೋ ಗತೋ ವಾಪಿ ಸುಪ್ರೀತಸ್ತು ಸದಾ ಶನಿಃ ॥೧೦॥

ಅಷ್ಟಮಸ್ಥೇ ಸೂರ್ಯಸುತೇ ವ್ಯಯೇ ಜನ್ಮದ್ವಿತೀಯಗೇ ।\\
ಕವಚಂ ಪಠತೇ ನಿತ್ಯಂ ನ ಪೀಡಾ ಜಾಯತೇ ಕ್ವಚಿತ್ ॥೧೧॥

ಇತ್ಯೇತತ್ಕವಚಂ ದಿವ್ಯಂ ಸೌರೇರ್ಯನ್ನಿರ್ಮಿತಂ ಪುರಾ ।\\
ದ್ವಾದಶಾಷ್ಟಮಜನ್ಮಸ್ಥದೋಷಾನ್ನಾಶಯತೇ ಸದಾ ।\\
ಜನ್ಮಲಗ್ನಸ್ಥಿತಾನ್ ದೋಷಾನ್ ಸರ್ವಾನ್ನಾಶಯತೇ ಪ್ರಭುಃ ॥೧೨॥
\authorline{॥ಇತಿ ಶ್ರೀಬ್ರಹ್ಮಾಂಡಪುರಾಣೇ ಬ್ರಹ್ಮನಾರದಸಂವಾದೇ ಶನಿವಜ್ರಪಂಜರಕವಚಂ ಸಂಪೂರ್ಣಂ ॥}
%=========================================================================================================================
\section{ರಾಹುಕವಚಂ}
\addcontentsline{toc}{section}{ರಾಹುಕವಚಂ}

ಓಂ ಅಸ್ಯ ಶ್ರೀರಾಹುಕವಚಸ್ತೋತ್ರಮಂತ್ರಸ್ಯ ಚಂದ್ರಮಾ ಋಷಿಃ । ಅನುಷ್ಟುಪ್ಛಂದಃ । ರಾಂ ಬೀಜಂ । ನಮಃ ಶಕ್ತಿಃ । ಸ್ವಾಹಾ ಕೀಲಕಂ । ರಾಹುಕೃತ ಪೀಡಾನಿವಾರಣಾರ್ಥೇ ಜಪೇ ವಿನಿಯೋಗಃ ॥

ಪ್ರಣಮಾಮಿ ಸದಾ ರಾಹುಂ ಶೂರ್ಪಾಕಾರಂ ಕಿರೀಟಿನಂ ।\\
ಸೈಂಹಿಕೇಯಂ ಕರಾಲಾಸ್ಯಂ ಲೋಕಾನಾಮಭಯಪ್ರದಂ ॥೧॥

ನೀಲಾಂಬರಃ ಶಿರಃ ಪಾತು ಲಲಾಟಂ ಲೋಕವಂದಿತಃ ।\\
ಚಕ್ಷುಷೀ ಪಾತು ಮೇ ರಾಹುಃ ಶ್ರೋತ್ರೇ ತ್ವರ್ಧಶರೀರವಾನ್ ॥೨॥

ನಾಸಿಕಾಂ ಮೇ ಧೂಮ್ರವರ್ಣಃ ಶೂಲಪಾಣಿರ್ಮುಖಂ ಮಮ ।\\
ಜಿಹ್ವಾಂ ಮೇ ಸಿಂಹಿಕಾಸೂನುಃ ಕಂಠಂ ಮೇ ಕಠಿನಾಂಘ್ರಿಕಃ ॥೩॥

ಭುಜಂಗೇಶೋ ಭುಜೌ ಪಾತು ನೀಲಮಾಲ್ಯಾಂಬರಃ ಕರೌ ।\\
ಪಾತು ವಕ್ಷಃಸ್ಥಲಂ ಮಂತ್ರೀ ಪಾತು ಕುಕ್ಷಿಂ ವಿಧುಂತುದಃ ॥೪॥

ಕಟಿಂ ಮೇ ವಿಕಟಃ ಪಾತು ಊರೂ ಮೇ ಸುರಪೂಜಿತಃ ।\\
ಸ್ವರ್ಭಾನುರ್ಜಾನುನೀ ಪಾತು ಜಂಘೇ ಮೇ ಪಾತು ಜಾಡ್ಯಹಾ ॥೫॥

ಗುಲ್ಫೌ ಗ್ರಹಪತಿಃ ಪಾತು ಪಾದೌ ಮೇ ಭೀಷಣಾಕೃತಿಃ ।\\
ಸರ್ವಾಣ್ಯಂಗಾನಿ ಮೇ ಪಾತು ನೀಲಚಂದನಭೂಷಣಃ ॥೬॥


ರಾಹೋರಿದಂ ಕವಚಮೃದ್ಧಿದವಸ್ತುದಂ ಯೋ\\
ಭಕ್ತ್ಯಾ ಪಠತ್ಯನುದಿನಂ ನಿಯತಃ ಶುಚಿಃ ಸನ್ ।\\
ಪ್ರಾಪ್ನೋತಿ ಕೀರ್ತಿಮತುಲಾಂ ಶ್ರಿಯಮೃದ್ಧಿಮಾಯು\\
ರಾರೋಗ್ಯಮಾತ್ಮವಿಜಯಂ ಚ ಹಿ ತತ್ಪ್ರಸಾದಾತ್ ॥೭॥

\authorline{॥ಇತಿ ಶ್ರೀಮಹಾಭಾರತೇ ಧೃತರಾಷ್ಟ್ರಸಂಜಯಸಂವಾದೇ ದ್ರೋಣಪರ್ವಣಿ ರಾಹುಕವಚಂ ಸಂಪೂರ್ಣಂ ॥}
%=========================================================================================================================
\section{ಶ್ರೀಗುರುಕವಚಂ}
\addcontentsline{toc}{section}{ಶ್ರೀಗುರುಕವಚಂ}
॥ಶ್ರೀಈಶ್ವರ ಉವಾಚ ॥\\
ಶೃಣು ದೇವಿ ಪ್ರವಕ್ಷ್ಯಾಮಿ ಗುಹ್ಯಾದ್ಗುಹ್ಯತರಂ ಮಹತ್ ।\\
ಲೋಕೋಪಕಾರಕಂ ಪ್ರಶ್ನಂ ನ ಕೇನಾಪಿ ಕೃತಂ ಪುರಾ ॥೧॥

ಅದ್ಯ ಪ್ರಭೃತಿ ಕಸ್ಯಾಪಿ ನ ಖ್ಯಾತಂ ಕವಚಂ ಮಯಾ ।\\
ದೇಶಿಕಾಃ ಬಹವಃ ಸಂತಿ ಮಂತ್ರಸಾಧನತತ್ಪರಾಃ ॥೨॥

ನ ತೇಷಾಂ ಜಾಯತೇ ಸಿದ್ಧಿಃ ಮಂತ್ರೈರ್ವಾ ಚಕ್ರಪೂಜನೈಃ ।\\
ಗುರೋರ್ವಿಧಾನಂ ಕವಚಮಜ್ಞಾತ್ವಾ ಕ್ರಿಯತೇ ಜಪಃ ।\\
ವೃಥಾಶ್ರಮೋ ಭವೇತ್ ತಸ್ಯ ನ ಸಿದ್ಧಿರ್ಮಂತ್ರಪೂಜನೈಃ ॥೩॥

ಗುರುಪಾದಂ ಪುರಸ್ಕೃತ್ಯ ಪ್ರಾಪ್ಯತೇ ಕವಚಂ ಶುಭಂ ।\\
ತದಾ ಮಂತ್ರಸ್ಯ ಯಂತ್ರಸ್ಯ ಸಿದ್ಧಿರ್ಭವತಿ ತತ್ಕ್ಷಣಾತ್ ।\\
ಸುಗೋಪ್ಯಂ ತು ಪ್ರಜಪ್ತವ್ಯಂ ನ ವಕ್ತವ್ಯಂ ವರಾನನೇ ॥೪॥


ಓಂ ನಮೋಽಸ್ಯ ಶ್ರೀಗುರುಕವಚನಾಮಮಂತ್ರಸ್ಯ ಪರಮಬ್ರಹ್ಮ ಋಷಿಃ  । ಸರ್ವವೇದಾನುಜ್ಞೋ ದೇವದೇವೋ ಶ್ರೀ ಆದಿಶಿವಃ ದೇವತಾ । ನಮೋ ಹಸೌಂ ಹಂಸಃ ಹಸಕ್ಷಮಲವರಯೂಂ ಸೋಽಹಂ ಹಂಸಃ ಬೀಜಂ । ಸಹಕ್ಷಮಲವರಯೀಂ ಶಕ್ತಿಃಹಂಸಃ ಸೋಽಹಂ ಕೀಲಕಂ । ಸಮಸ್ತಶ್ರೀಗುರುಮಂಡಲಪ್ರೀತ್ಯರ್ಥೇ ಜಪೇ ವಿನಿಯೋಗಃ ।\\

ಓಂ ಹಸಾಂ , ಹಸೀಂ ಇತ್ಯಾದಿನಾ ನ್ಯಾಸಃ\\


\dhyana{ಶ್ರೀಸಿದ್ಧಮಾನವಮುಖಾ ಗುರವಃ ಸ್ವರೂಪಂ\\
ಸಂಸಾರದಾಹಶಮನಂ ದ್ವಿಭುಜಂ ತ್ರಿನೇತ್ರಂ ।\\
ವಾಮಾಂಗಶಕ್ತಿಸಕಲಾಭರಣೈರ್ವಿಭೂಷಂ\\
ಧ್ಯಾಯೇಜ್ಜಪೇತ್ ಸಕಲಸಿದ್ಧಿಫಲಪ್ರದಂ ಚ ॥}

ಓಂ ನಮಃ ಪ್ರಕಾಶಾನಂದನಾಥಃ ತು ಶಿಖಾಯಾಂ ಪಾತು ಮೇ ಸದಾ ।\\
ಪರಶಿವಾನಂದನಾಥಃ ಶಿರೋ ಮೇ ರಕ್ಷಯೇತ್ ಸದಾ ॥೧॥

ಪರಶಕ್ತಿದಿವ್ಯಾನಂದನಾಥೋ ಭಾಲೇ ಚ ರಕ್ಷತು ।\\
ಕಾಮೇಶ್ವರಾನಂದನಾಥೋ ಮುಖಂ ರಕ್ಷತು ಸರ್ವಧೃಕ್ ॥೨॥

ದಿವ್ಯೌಘೋ ಮಸ್ತಕಂ ದೇವಿ ಪಾತು ಸರ್ವಶಿರಃ ಸದಾ ।\\
ಕಂಠಾದಿನಾಭಿಪರ್ಯಂತಂ ಸಿದ್ಧೌಘಾ ಗುರವಃ ಪ್ರಿಯೇ ॥೩॥

ಭೋಗಾನಂದನಾಥ ಗುರುಃ ಪಾತು ದಕ್ಷಿಣಬಾಹುಕಂ ।\\
ಸಮಯಾನಂದನಾಥಶ್ಚ ಸಂತತಂ ಹೃದಯೇಽವತು ॥೪॥

ಸಹಜಾನಂದನಾಥಶ್ಚ ಕಟಿಂ ನಾಭಿಂ ಚ ರಕ್ಷತು ।\\
ಏಷು ಸ್ಥಾನೇಷು ಸಿದ್ಧೌಘಾಃ ರಕ್ಷಂತು ಗುರವಃ ಸದಾ ॥೫॥

ಅಧರೇ ಮಾನವೌಘಾಶ್ಚ ಗುರವಃ ಕುಲನಾಯಿಕೇ ।\\
ಗಗನಾನಂದನಾಥಶ್ಚ ಗುಲ್ಫಯೋಃ ಪಾತು ಸರ್ವದಾ ॥೬॥

ನೀಲೌಘಾನಂದನಾಥಶ್ಚ ರಕ್ಷಯೇತ್ ಪಾದಪೃಷ್ಠತಃ ।\\
ಸ್ವಾತ್ಮಾನಂದನಾಥಗುರುಃ ಪಾದಾಂಗುಲೀಶ್ಚ ರಕ್ಷತು ॥೭॥

ಕಂದೋಲಾನಂದನಾಥಶ್ಚ ರಕ್ಷೇತ್ ಪಾದತಲೇ ಸದಾ ।\\
ಇತ್ಯೇವಂ ಮಾನವೌಘಾಶ್ಚ ನ್ಯಸೇನ್ನಾಭ್ಯಾದಿಪಾದಯೋಃ ॥೮॥

ಗುರುರ್ಮೇ ರಕ್ಷಯೇದುರ್ವ್ಯಾಂ ಸಲಿಲೇ ಪರಮೋ ಗುರುಃ ।\\
ಪರಾಪರಗುರುರ್ವಹ್ನೌ ರಕ್ಷಯೇತ್ ಶಿವವಲ್ಲಭೇ ॥೯॥

ಪರಮೇಷ್ಠೀಗುರುಶ್ಚೈವ ರಕ್ಷಯೇತ್ ವಾಯುಮಂಡಲೇ ।\\
ಶಿವಾದಿಗುರವಃ ಸಾಕ್ಷಾತ್ ಆಕಾಶೇ ರಕ್ಷಯೇತ್ ಸದಾ ॥೧೦॥

ಇಂದ್ರೋ ಗುರುಃ ಪಾತು ಪೂರ್ವೇ ಆಗ್ನೇಯಾಂ ಗುರುರಗ್ನಯಃ ।\\
ದಕ್ಷೇ ಯಮೋ ಗುರುಃ ಪಾತು ನೈಋತ್ಯಾಂ ನಿಋತಿರ್ಗುರುಃ ॥೧೧॥

ವರುಣೋ ಗುರುಃ ಪಶ್ಚಿಮೇ ವಾಯವ್ಯಾಂ ಮಾರುತೋ ಗುರುಃ ।\\
ಉತ್ತರೇ ಧನದಃ ಪಾತು ಐಶಾನ್ಯಾಮೀಶ್ವರೋ ಗುರುಃ ॥೧೨॥

ಊರ್ಧ್ವಂ ಪಾತು ಗುರುರ್ಬ್ರಹ್ಮಾ ಅನಂತೋ ಗುರುರಪ್ಯಧಃ ।\\
ಏವಂ ದಶದಿಶಃ ಪಾಂತು ಇಂದ್ರಾದಿಗುರವಃ ಕ್ರಮಾತ್ ॥೧೩॥

ಶಿರಸಃ ಪಾದಪರ್ಯಂತಂ ಪಾಂತು ದಿವ್ಯೌಘಸಿದ್ಧಯಃ ।\\
ಮಾನವೌಘಾಶ್ಚ ಗುರವೋ ವ್ಯಾಪಕಂ ಪಾಂತು ಸರ್ವದಾ ॥೧೪॥

ಸರ್ವತ್ರ ಗುರುರೂಪೇಣ ಸಂರಕ್ಷೇತ್ ಸಾಧಕೋತ್ತಮಂ ।\\
ಆತ್ಮಾನಂ ಗುರುರೂಪಂ ಚ ಧ್ಯಾಯೇನ್ ಮಂತ್ರಂ ಸದಾ ಬುಧಃ ॥೧೫॥

॥ಫಲಶ್ರುತಿಃ ॥\\
ಇತ್ಯೇವಂ ಗುರುಕವಚಂ ಬ್ರಹ್ಮಲೋಕೇಽಪಿ ದುರ್ಲಭಂ ।\\
ತವ ಪ್ರೀತ್ಯಾ ಮಯಾ ಖ್ಯಾತಂ ನ ಕಸ್ಯ ಕಥಿತಂ ಪ್ರಿಯೇ ॥೧॥

ಪೂಜಾಕಾಲೇ ಪಠೇದ್ ಯಸ್ತು ಜಪಕಾಲೇ ವಿಶೇಷತಃ ।\\
ತ್ರೈಲೋಕ್ಯದುರ್ಲಭಂ ದೇವಿ ಭುಕ್ತಿಮುಕ್ತಿಫಲಪ್ರದಂ ॥೨॥

ಸರ್ವಮಂತ್ರಫಲಂ ತಸ್ಯ ಸರ್ವಯಂತ್ರಫಲಂ ತಥಾ ।\\
ಸರ್ವತೀರ್ಥಫಲಂ ದೇವಿ ಯಃ ಪಠೇತ್ ಕವಚಂ ಗುರೋಃ ॥೩॥

ಅಷ್ಟಗಂಧೇನ ಭೂರ್ಜೇ ಚ ಲಿಖ್ಯತೇ ಚಕ್ರಸಂಯುತಂ ।\\
ಕವಚಂ ಗುರುಪಂಕ್ತೇಸ್ತು ಭಕ್ತ್ಯಾ ಚ ಶುಬವಾಸರೇ ॥೪॥

ಪೂಜಯೇತ್ ಧೂಪದೀಪಾದ್ಯೈಃ ಸುಧಾಭಿಃ ಸಿತಸಂಯುತೈಃ ।\\
ತರ್ಪಯೇತ್ ಗುರುಮಂತ್ರೇಣ ಸಾಧಕಃ ಶುದ್ಧಚೇತಸಾ ॥೫॥

ಧಾರಯೇತ್ ಕವಚಂ ದೇವಿ ಇಹ ಭೂತಭಯಾಪಹಂ ।\\
ಪಠೇನ್ಮಂತ್ರೀ ತ್ರಿಕಾಲಂ ಹಿ ಸ ಮುಕ್ತೋ ಭವಬಂಧನಾತ್ ।\\
ಏವಂ ಕವಚಂ ಪರಮಂ ದಿವ್ಯಸಿದ್ಧೌಘಕಲಾವಾನ್ ॥೬॥

\authorline{॥ಇತಿ ಪುರಶ್ಚರಣರಸೋಲ್ಲಾಸೇ ದ್ವಿತೀಯಪ್ರಶ್ನೇ ದಶಮಪಟಲೇ ಈಶ್ವರದೇವೀಸಂವಾದೇ ಶ್ರೀಗುರುಕವಚಂ ಸಂಪೂರ್ಣಂ ॥}

%======================================================================================================

\section{ಗುರುಕವಚಂ \\(ಕಂಕಾಲಮಾಲಿನೀತಂತ್ರೇ)}
\addcontentsline{toc}{section}{ಗುರುಕವಚಂ}
ದೇವ್ಯುವಾಚ ॥
ಭೂತನಾಥ  ಮಹಾದೇವ  ಕವಚಂ ತಸ್ಯ ಮೇ ವದ ।\\
ಗುರುದೇವಸ್ಯ ದೇವೇಶ  ಸಾಕ್ಷಾದ್ಬ್ರಹ್ಮಸ್ವರೂಪಿಣಃ ॥೧॥

ಈಶ್ವರ ಉವಾಚ ॥
ಅಥಾ ತೇ ಕಥಯಾಮೀಶೇ  ಕವಚಂ ಮೋಕ್ಷದಾಯಕಂ ।\\
ಯಸ್ಯ ಜ್ಞಾನಂ ವಿನಾ ದೇವಿ  ನ ಸಿದ್ಧಿರ್ನ ಚ ಸದ್ಗತಿಃ ॥೨॥

ಬ್ರಹ್ಮಾದಯೋಽಪಿ ಗಿರಿಜೇ  ಸರ್ವತ್ರ ಜಯಿನಃ ಸ್ಮೃತಾಃ ।\\
ಅಸ್ಯ ಪ್ರಸಾದಾತ್ ಸಕಲಾ ವೇದಾಗಮಪುರಃಸರಾಃ ॥೩॥

ಕವಚಸ್ಯಾಸ್ಯ ದೇವೇಶಿ  ಋಷಿರ್ವಿಷ್ಣುರುದಾಹೃತಃ ।\\
ಛಂದೋ ವಿರಾಡ್ ದೇವತಾ ಚ ಗುರುದೇವಃ ಸ್ವಯಂ ಶಿವಃ ॥೪॥

ಛತುರ್ವರ್ಗೇ ಜ್ಞಾನಮಾರ್ಗೇ ವಿನಿಯೋಗಃ ಪ್ರಕೀರ್ತ್ತಿತಃ ।\\
ಸಹಸ್ರಾರೇ ಮಹಾಪದ್ಮೇ ಕರ್ಪೂರಧವಲೋ ಗುರುಃ ॥೫॥

ವಾಮೋರುಸ್ಥಿತಶಕ್ತಿರ್ಯಃ ಸರ್ವತ್ರ ಪರಿರಕ್ಷತು ।\\
ಪರಮಾಖ್ಯೋ ಗುರುಃ ಪಾತು ಶಿರಸಂ ಮಮ ವಲ್ಲಭೇ  ॥೬॥

ಪರಾಪರಾಖ್ಯೋ ನಾಸಾಂ ಮೇ ಪರಮೇಷ್ಠೀ ಮುಖಂ ಸದಾ ।\\
ಕಂಠಂ ಮಮ ಸದಾ ಪಾತು ಪ್ರಹ್ಲಾದಾನಂದನಾಥಕಃ ॥೭॥

ಬಾಹೂ ದ್ವೌ ಸನಕಾನಂದಃ ಕುಮಾರಾನಂದ ಏವ ಚ ।\\
ವಶಿಷ್ಠಾನಂದನಾಥಶ್ಚ ಹೃದಯಂ ಪಾತು ಸರ್ವದಾ ॥೮॥

ಕ್ರೋಧಾನಂದಃ ಕಟಿಂ ಪಾತು ಸುಖಾನಂದಃ ಪದಂ ಮಮ ।\\
ಧ್ಯಾನಾನಂದಶ್ಚ ಸರ್ವಾಂಗಂ ಬೋಧಾನಂದಶ್ಚ ಕಾನನೇ ॥೯॥

ಸರ್ವತ್ರ ಗುರವಃ ಪಾಂತು ಸರ್ವೇ ಈಶ್ವರರೂಪಿಣಃ ।\\
ಇತಿ ತೇ ಕಥಿತಂ ಭದ್ರೇ  ಕವಚಂ ಪರಮಂ ಶಿವೇ  ॥೧೦॥

ಭಕ್ತಿಹೀನೇ ದುರಾಚಾರೇ ದತ್ತ್ವೈತನ್ಮೃತ್ಯುಮಾಪ್ನುಯಾತ್ ॥\\
ಅಸ್ಯೈವ ಪಠನಾದ್ದೇವಿ  ಧಾರಣಾತ್ಛ್ರವಣಾತ್ ಪ್ರಿಯೇ  ॥೧೧॥

ಜಾಯತೇ ಮಂತ್ರಸಿದ್ಧಿಶ್ಚ ಕಿಮನ್ಯತ್ ಕಥಯಾಮಿ ತೇ ।\\
ಕಂಠೇ ವಾ ದಕ್ಷಿಣೇ ಬಾಹೌ ಶಿಖಾಯಾಂ ವೀರವಂದಿತೇ  ॥೧೨॥

ಧಾರಣಾನ್ನಾಶಯೇತ್ ಪಾಪಂ ಗಂಗಾಯಾಂ ಕಲ್ಮಷಂ ಯಥಾ ।\\
ಇದಂ ಕವಚಮಜ್ಞಾತ್ವಾ ಯದಿ ಮಂತ್ರಂ ಜಪೇತ್ ಪ್ರಿಯೇ  ॥೧೩॥

ತತ್ ಸರ್ವಂ ನಿಷ್ಫಲಂ ಕೃತ್ವಾ ಗುರುರ್ಯಾತಿ ಸುನಿಶ್ಚಿತಂ ।\\
ಶಿವೇ ರುಷ್ಟೇ ಗುರುಸ್ತ್ರಾತಾ ಗುರೌ ರುಷ್ಟೇ ನ ಕಶ್ಚನ ॥೧೪॥

\authorline{॥ಇತಿ ಕಂಕಾಲಮಾಲಿನೀತಂತ್ರೇ ಗುರುಕವಚಂ ಸಮಾಪ್ತಂ ॥}
%===============================================================================================
\newpage
\section{ಶ್ರೀಷೋಡಶೀಕವಚಂ}
\addcontentsline{toc}{section}{ಶ್ರೀಷೋಡಶೀಕವಚಂ}
ಶ್ರೀದೇವ್ಯುವಾಚ ।\\
ಭಗವಂದೇವದೇವೇಶ ಭಕ್ತಾನುಗ್ರಹಕಾರಕ ।\\
ಶ್ರೀಮಹಾಷೋಡಶೀದೇವ್ಯಾಃ ಕವಚಂ ವದ ಮೇ ಪ್ರಭೋ ॥೧॥

ಶ್ರೀಭೈರವ ಉವಾಚ ।\\
ಅಧುನಾ ದೇವಿ ವಕ್ಷ್ಯಾಮಿ ಮಹಾಶ್ರೀಷೋಡಶೀಮಯಂ ।\\
ಪರಮಾರ್ಥಾಭಿಧಂ ವರ್ಮ ಶ್ರೀವಿದ್ಯಾಸಾರಮುತ್ತಮಂ ॥೨॥

ಅಸ್ಯ ಶ್ರೀಮಹಾಷೋಡಶೀಕವಚಸ್ಯ ಶಿವಋಷಿಃ । ಪಂಕ್ತಿಛಂದಃ । ಶ್ರೀಮಹಾಷೋದಶೀದೇವತಾ । ಕಏಈಲ ಹ್ರೀಂ ಬೀಜಂ । ಹಸಕಹಲ ಹ್ರೀಂ ಶಕ್ತಿಃ । ಸಕಲ ಹ್ರೀಂ ಕೀಲಕಂ । ಧರ್ಮಾರ್ಥಕಾಮಮೋಕ್ಷಾರ್ಥೇ ವಿನಿಯೋಗಃ ಪ್ರಕೀರ್ತಿತಃ ।\\

ಅಥ ಧ್ಯಾನಂ ।\\
ಸೂರ್ಯಕೋಟಿಸಹಸ್ರಾಭಾಂ ಸರ್ವಾಭರಣಭೂಷಿತಾಂ ।\\
ಪಂಚವಕ್ತ್ರಾಂ ಚತುರ್ಬಾಹುಂ ಶಿವಶಕ್ತ್ಯಾತ್ಮಕಾಂ ಭಜೇ ॥೩॥

ಶಿರೋಽವ್ಯಾನ್ಮೇ ರಮಾಬೀಜಂ ವಾಗ್ಬೀಜಂ ಲೋಚನೇಽವತಾತ್ ।\\
ಪರಾಬೀಜಂ ಶ್ರುತೀ ಪಾಯಾತ್ ಕಾಮಬೀಜಂ ಚ ನಾಸಿಕಾಂ ॥೪॥

ಕಾಮರಾಜಸ್ತಥೋಷ್ಠೌ ಮೇ ಶಕ್ತಿಃ ಪಾಯಾನ್ಮುಖಂ ಮಮ ।\\
ವಾಸವ್ಯಾರದನಾನ್ಪಾತು ತಾರಂ ಜಿಹ್ವಾಂ ಮಮಾವತು ॥೫॥

ಶಕ್ತಿಃ ಕಂಠಂ ತಥಾ ಪಾತು ರಮಾ ಸ್ಕಂಧೌ ಮಮಾವತು ।\\
ತ್ರ್ಯಕ್ಷಂ ಭ್ರುವೌ ಸದಾ ಪತು ಧೃತಿಃ ಪತು ಕರೌ ಮಮ ॥೬॥

ಪರಾಬೀಜಂ ಪಾತು ವಕ್ಷೋ ಲಕ್ಷ್ಮೀ ಕುಕ್ಷಿಂ ಮಮಾವತು ।\\
ಶಿವಕೂಟೋಽವತಾತ್ಪೃಷ್ಠಂ ಬ್ರಹ್ಮಕೂಟಂ ತು ಪಾರ್ಶ್ವಯೋಃ ॥೭॥

ಶಕ್ತಿಕೂಟೋಽವತಾನ್ನಾಭಿಂ ಕಾಂತಿಕೂಟೋ ಗುದಂ ತಥಾ ।\\
ವಾಯುಕೂಟೋಽವತಾದ್ಧಸ್ತೌ ಶಕ್ತಿಕೂಟಸ್ತು ಜಾನುತಃ ॥೮॥

ತಾರಂ ಜಂಘೇ ಸದಾ ಪಾತು ಪಾದೌ ಶಕ್ತಿರ್ಮಮಾವತು ।\\
ಪರಾ ಪ್ರಭಾತಕಾಲೇಽವ್ಯಾತ್ ವಾಙ್ಮಧ್ಯಾಹ್ನೇಽವತಾಚ್ಚ ಮಾಂ ॥೯॥

ಸಾ ಸಾಯಂ ಪಾತು ಸರ್ವತ್ರ ಮೋಹಾಽವ್ಯಾನ್ಮಾಂ ನಿಶೀಥಕೇ ।\\
ಪೂರ್ವಾದಿದಿಕ್ಷು ತಾರೋಽವ್ಯಾತ್ ವಹ್ನಿವಾಯೋಃ ಪರಾವತು ॥೧೦॥

ಜಲದುರ್ಭಿಕ್ಷದಾರಿದ್ರ್ಯಾತ್ ಶಕ್ತಿಃ ಪಾತು ಮಮಾನಿಶಂ ।\\
ದಾರಾಪುತ್ರಧನಾಢ್ಯೇಭ್ಯೋ ಗಜಾಶ್ವಗೃಹಮಂಡಲಾತ್ ॥೧೧॥

ಪಾತು ಮೇ ಕಮಲಾಬೀಜಂ ಕಾಮಃ ಶ್ರೀಯೋಗಿನೀಗಣಾತ್ ।\\
ವಾಗ್ಭವಂ ಸರ್ವದಾ ಪಾತು ಶ್ರೀವಿದ್ಯಾ ಸರ್ವತೋಽವತು ॥೧೨॥

ಇತೀದಂ ಕವಚಂ ದಿವ್ಯಂ ಮಹಾಶ್ರೀಷೋಡಶೀಮಯಂ ।\\
ಮೂಲಮಂತ್ರಮಯಂ ಗೋಪ್ಯಂ ಸರ್ವಾಶಾಪರಿಪೂರಕಂ ॥೧೩॥

ಸರ್ವಶ್ರೇಯಸ್ಕರಂ ನಿತ್ಯಂ ಸ್ತುತಂ ಸಿದ್ಧಿಪ್ರದಂ ಕಲೌ ।\\
ಯಃ ಪಠೇತ್ಸಾಧಕೋ ನಿತ್ಯಂ ಧತ್ತೇ ಮೌನವ್ರತಂ ಶಿವೇ ॥೧೪॥

ಸ ಏವ ಶೋಡಶೀಪುತ್ರಃ ಕಾಮೇಶ್ವರಸಮಪ್ರಭುಃ ।\\
ಶ್ಯಾಮಾತ್ಮಕೋ ಮಹಾಕಾಲರೂಪೋ ವೈರಿವಿಮರ್ದನಃ ॥೧೫॥

ಬಹುಪುತ್ರೋ ರಮಾನಾಥೋ ಮಾಂತ್ರಿಕೋತ್ತಮಶೇಖರಃ ।\\
ಇತ್ಯೇತತ್ಕಥಿತಂ ವರ್ಮ ಮಹಾಶ್ರೀಶೋಡಶೀಮಯಂ ॥೧೬॥

ವಿನಾನೇನ ನ ಸಿದ್ಧಿಃ ಸ್ಯಾನ್ಮಮಾಪಿ ಪರಮೇಶ್ವರಿ ।\\
ಗೋಪ್ಯಂ ಗುಹ್ಯತಮಂ ವರ್ಮ ಮೂಲಮಂತ್ರಮಯಂ ಪರಂ ।\\
ಪರಮಾರ್ಥಾಭಿಧಂ ವಸ್ತು ಗೋಪಯೇತ್ಸ ಸದಾಶಿವಃ ॥೧೭॥
\authorline{ಇತಿ ಶ್ರೀರುದ್ರಯಮಲೇ ತಂತ್ರೇ ಮಹಾಷೋಡಶೀ ಕವಚಂ ಸಂಪೂರ್ಣಂ ।}
%=========================================================================================================================
\section{ತಾರಾಕವಚಂ}
\addcontentsline{toc}{section}{ತಾರಾಕವಚಂ}
ಈಶ್ವರ ಉವಾಚ ।\\
ಕೋಟಿತಂತ್ರೇಷು ಗೋಪ್ಯಾ ಹಿ ವಿದ್ಯಾತಿಭಯಮೋಚಿನೀ ।\\
ದಿವ್ಯಂ ಹಿ ಕವಚಂ ತಸ್ಯಾಃ ಶೃಣುಷ್ವ ಸರ್ವಕಾಮದಂ ॥೧॥

ಅಸ್ಯ ತಾರಾಕವಚಸ್ಯ ಅಕ್ಷೋಭ್ಯ ಋಷಿಃ । ತ್ರಿಷ್ಟುಪ್ ಛಂದಃ ।\\
ಭಗವತೀ ತಾರಾ ದೇವತಾ । ಸರ್ವಮಂತ್ರಸಿದ್ಧಿಸಮೃದ್ಧಯೇ ಜಪೇ ವಿನಿಯೋಗಃ ।\\
ಪ್ರಣವೋ ಮೇ ಶಿರಃ ಪಾತು ಬ್ರಹ್ಮರೂಪಾ ಮಹೇಶ್ವರೀ ।\\
ಲಲಾಟೇ ಪಾತು ಹ್ರೀಂಕಾರೋ ಬೀಜರೂಪಾ ಮಹೇಶ್ವರೀ ॥೨॥

ಸ್ತ್ರೀಂಕಾರೋ ವದನೇ ನಿತ್ಯಂ ಲಜ್ಜಾರೂಪಾ ಮಹೇಶ್ವರೀ ।\\
ಹೂಂಕಾರಃ ಪಾತು ಹೃದಯೇ ಭವಾನೀರೂಪಶಕ್ತಿಧೃಕ್ ॥೩॥

ಫಟ್ಕಾರಃ ಪಾತು ಸರ್ವಾಂಗೇ ಸರ್ವಸಿದ್ಧಿಫಲಪ್ರದಾ ।\\
ಖರ್ವಾ ಮಾಂ ಪಾತು ದೇವೇಶೀ ಗಂಡಯುಗ್ಮೇ ಭಯಾಪಹಾ ॥೪॥

ನಿಮ್ನೋದರೀ ಸದಾ ಸ್ಕಂಧಯುಗ್ಮೇ ಪಾತು ಮಹೇಶ್ವರೀ ।\\
ವ್ಯಾಘ್ರಚರ್ಮಾವೃತಾ ಕಟ್ಯಾಂ ಪಾತು ದೇವೀ ಶಿವಪ್ರಿಯಾ ॥೫॥

ಪೀನೋನ್ನತಸ್ತನೀ ಪಾತು ಪಾರ್ಶ್ವಯುಗ್ಮೇ ಮಹೇಶ್ವರೀ ।\\
ರಕ್ತವರ್ತುಲನೇತ್ರಾ ಚ ಕಟಿದೇಶೇ ಸದಾಽವತು ॥೬॥

ಲಲಜಿಹ್ವಾ ಸದಾ ಪಾತು ನಾಭೌ ಮಾಂ ಭುವನೇಶ್ವರೀ ।\\
ಕರಾಲಾಸ್ಯಾ ಸದಾ ಪಾತು ಲಿಂಗೇ ದೇವೀ ಹರಪ್ರಿಯಾ ॥೭॥

ಪಿಂಗೋಗ್ರೈಕಜಟಾ ಪಾತು ಜಂಘಾಯಾಂ ವಿಘ್ನನಾಶಿನೀ ।\\
ಪ್ರೇತಖರ್ಪರಭೃದ್ದೇವೀ ಜಾನುಚಕ್ರೇ ಮಹೇಶ್ವರೀ ॥೮॥

ನೀಲವರ್ಣಾ ಸದಾ ಪಾತು ಜಾನುನೀ ಸರ್ವದಾ ಮಮ ।\\
ನಾಗಕುಂಡಲಧರ್ತ್ರೀ ಚ ಪಾತು ಪಾದಯುಗೇ ತತಃ ॥೯॥

ನಾಗಹಾರಧರಾ ದೇವೀ ಸರ್ವಾಂಗಂ ಪಾತು ಸರ್ವದಾ ।\\
ನಾಗಕಂಕಧರಾ ದೇವೀ ಪಾತು ಪ್ರಾಂತರದೇಶತಃ ॥೧೦॥

ಚತುರ್ಭುಜಾ ಸದಾ ಪಾತು ಗಮನೇ ಶತ್ರುನಾಶಿನೀ ।\\
ಖಡ್ಗಹಸ್ತಾ ಮಹಾದೇವೀ ಶ್ರವಣೇ ಪಾತು ಸರ್ವದಾ ॥೧೧॥

ನೀಲಾಂಬರಧರಾ ದೇವೀ ಪಾತು ಮಾಂ ವಿಘ್ನನಾಶಿನೀ ।\\
ಕರ್ತ್ರಿಹಸ್ತಾ ಸದಾ ಪಾತು ವಿವಾದೇ ಶತ್ರುಮಧ್ಯತಃ ॥೧೨॥

ಬ್ರಹ್ಮರೂಪಧರಾ ದೇವೀ ಸಂಗ್ರಾಮೇ ಪಾತು ಸರ್ವದಾ ।\\
ನಾಗಕಂಕಣಧರ್ತ್ರೀ ಚ ಭೋಜನೇ ಪಾತು ಸರ್ವದಾ ॥೧೩॥

ಶವಕರ್ಣಾ ಮಹಾದೇವೀ ಶಯನೇ ಪಾತು ಸರ್ವದಾ ।\\
ವೀರಾಸನಧರಾ ದೇವೀ ನಿದ್ರಾಯಾಂ ಪಾತು ಸರ್ವದಾ ॥೧೪॥

ಧನುರ್ಬಾಣಧರಾ ದೇವೀ ಪಾತು ಮಾಂ ವಿಘ್ನಸಂಕುಲೇ ।\\
ನಾಗಾಂಚಿತಕಟೀ ಪಾತು ದೇವೀ ಮಾಂ ಸರ್ವಕರ್ಮಸು ॥೧೫॥

ಛಿನ್ನಮುಂಡಧರಾ ದೇವೀ ಕಾನನೇ ಪಾತು ಸರ್ವದಾ ।\\
ಚಿತಾಮಧ್ಯಸ್ಥಿತಾ ದೇವೀ ಮಾರಣೇ ಪಾತು ಸರ್ವದಾ ॥೧೬॥

ದ್ವೀಪಿಚರ್ಮಧರಾ ದೇವೀ ಪುತ್ರದಾರಧನಾದಿಷು ।\\
ಅಲಂಕಾರಾನ್ವಿತಾ ದೇವೀ ಪಾತು ಮಾಂ ಹರವಲ್ಲಭಾ ॥೧೭॥

ರಕ್ಷ ರಕ್ಷ ನದೀಕುಂಜೇ ಹೂಂ ಹೂಂ ಫಟ್ ಸುಸಮನ್ವಿತೇ ।\\
ಬೀಜರೂಪಾ ಮಹಾದೇವೀ ಪರ್ವತೇ ಪಾತು ಸರ್ವದಾ ॥೧೮॥

ಮಣಿಭೃದ್ವಜ್ರಿಣೀ ದೇವೀ ಮಹಾಪ್ರತಿಸರೇ ತಥಾ ।\\
ರಕ್ಷ ರಕ್ಷ ಸದಾ ಹೂಂ ಹೂಂ ಓಂ ಹ್ರೀಂ ಸ್ವಾಹಾ ಮಹೇಶ್ವರೀ ॥೧೯॥

ಪುಷ್ಪಕೇತುರಜಾರ್ಹೇತಿ ಕಾನನೇ ಪಾತು ಸರ್ವದಾ ।\\
ಓಂ ಹ್ರೀಂ ವಜ್ರಪುಷ್ಪಂ ಹುಂ ಫಟ್ ಪ್ರಾಂತರೇ ಸರ್ವಕಾಮದಾ ॥೨೦॥

ಓಂ ಪುಷ್ಪೇ ಪುಷ್ಪೇ ಮಹಾಪುಷ್ಪೇ ಪಾತು ಪುತ್ರಾನ್ಮಹೇಶ್ವರೀ ।\\
ಹೂಂ ಸ್ವಾಹಾ ಶಕ್ತಿಸಂಯುಕ್ತಾ ದಾರಾನ್ ರಕ್ಷತು ಸರ್ವದಾ ॥೨೧॥

ಓಂ ಆಂ ಹೂಂ ಸ್ವಾಹಾ ಮಹೇಶಾನೀ ಪಾತು ದ್ಯೂತೇ ಹರಪ್ರಿಯಾ ।\\
ಓಂ ಹ್ರೀಂ ಸರ್ವವಿಘ್ನೋತ್ಸಾರಿಣೀ ದೇವೀ ವಿಘ್ನಾನ್ಮಾಂ ಸದಾಽವತು ॥೨೨॥

ಓಂ ಪವಿತ್ರವಜ್ರಭೂಮೇ ಹುಂಫಟ್ಸ್ವಾಹಾ ಸಮನ್ವಿತಾ ।\\
ಪೂರಿಕಾ ಪಾತು ಮಾಂ ದೇವೀ ಸರ್ವವಿಘ್ನವಿನಾಶಿನೀ ॥೨೩॥

ಓಂ ಆಃ ಸುರೇಖೇ ವಜ್ರರೇಖೇ ಹುಂಫಟ್ಸ್ವಾಹಾಸಮನ್ವಿತಾ ।\\
ಪಾತಾಲೇ ಪಾತು ಸಾ ದೇವೀ ಲಾಕಿನೀ ನಾಮಸಂಜ್ಞಿಕಾ ॥೨೪॥

ಹ್ರೀಂಕಾರೀ ಪಾತು ಮಾಂ ಪೂರ್ವೇ ಶಕ್ತಿರೂಪಾ ಮಹೇಶ್ವರೀ ।\\
ಸ್ತ್ರೀಂಕಾರೀ ಪಾತು ದೇವೇಶೀ ವಧೂರೂಪಾ ಮಹೇಶ್ವರೀ ॥೨೫॥

ಹೂಂಸ್ವರೂಪಾ ಮಹಾದೇವೀ ಪಾತು ಮಾಂ ಕ್ರೋಧರೂಪಿಣೀ ।\\
ಫಟ್ಸ್ವರೂಪಾ ಮಹಾಮಾಯಾ ಉತ್ತರೇ ಪಾತು ಸರ್ವದಾ ॥೨೬॥

ಪಶ್ಚಿಮೇ ಪಾತು ಮಾಂ ದೇವೀ ಫಟ್ಸ್ವರೂಪಾ ಹರಪ್ರಿಯಾ ।\\
ಮಧ್ಯೇ ಮಾಂ ಪಾತು ದೇವೇಶೀ ಹೂಂಸ್ವರೂಪಾ ನಗಾತ್ಮಜಾ ॥೨೭॥

ನೀಲವರ್ಣಾ ಸದಾ ಪಾತು ಸರ್ವತೋ ವಾಗ್ಭವಾ ಸದಾ ।\\
ಭವಾನೀ ಪಾತು ಭವನೇ ಸರ್ವೈಶ್ವರ್ಯಪ್ರದಾಯಿನೀ ॥೨೮॥

ವಿದ್ಯಾದಾನರತಾ ದೇವೀ ವಕ್ತ್ರೇ ನೀಲಸರಸ್ವತೀ ।\\
ಶಾಸ್ತ್ರೇ ವಾದೇ ಚ ಸಂಗ್ರಾಮೇ ಜಲೇ ಚ ವಿಷಮೇ ಗಿರೌ ॥೨೯॥

ಭೀಮರೂಪಾ ಸದಾ ಪಾತು ಶ್ಮಶಾನೇ ಭಯನಾಶಿನೀ ।\\
ಭೂತಪ್ರೇತಾಲಯೇ ಘೋರೇ ದುರ್ಗಮಾ ಶ್ರೀಘನಾಽವತು ॥೩೦॥

ಪಾತು ನಿತ್ಯಂ ಮಹೇಶಾನೀ ಸರ್ವತ್ರ ಶಿವದೂತಿಕಾ ।\\
ಕವಚಸ್ಯ ಮಾಹಾತ್ಮ್ಯಂ ನಾಹಂ ವರ್ಷಶತೈರಪಿ ॥೩೧॥

ಶಕ್ನೋಮಿ ಗದಿತುಂ ದೇವಿ ಭವೇತ್ತಸ್ಯ ಫಲಂ ಚ ಯತ್ ।\\
ಪುತ್ರದಾರೇಷು ಬಂಧೂನಾಂ ಸರ್ವದೇಶೇ ಚ ಸರ್ವದಾ ॥೩೨॥

ನ ವಿದ್ಯತೇ ಭಯಂ ತಸ್ಯ ನೃಪಪೂಜ್ಯೋ ಭವೇಚ್ಚ ಸಃ ।\\
ಶುಚಿರ್ಭೂತ್ವಾಽಶುಚಿರ್ವಾಪಿ ಕವಚಂ ಸರ್ವಕಾಮದಂ ॥೩೩॥

ಪ್ರಪಠನ್ ವಾ ಸ್ಮರನ್ಮರ್ತ್ಯೋ ದುಃಖಶೋಕವಿವರ್ಜಿತಃ ।\\
ಸರ್ವಶಾಸ್ತ್ರೇ ಮಹೇಶಾನಿ ಕವಿರಾಡ್ ಭವತಿ ಧ್ರುವಂ ॥೩೪॥

ಸರ್ವವಾಗೀಶ್ವರೋ ಮರ್ತ್ಯೋ ಲೋಕವಶ್ಯೋ ಧನೇಶ್ವರಃ ।\\
ರಣೇ ದ್ಯೂತೇ ವಿವಾದೇ ಚ ಜಯಸ್ತತ್ರ ಭವೇದ್ ಧ್ರುವಂ ॥೩೫॥

ಪುತ್ರಪೌತಾನ್ವಿತೋ ಮರ್ತ್ಯೋ ವಿಲಾಸೀ ಸರ್ವಯೋಷಿತಾಂ ।\\
ಶತ್ರವೋ ದಾಸತಾಂ ಯಾಂತಿ ಸರ್ವೇಷಾಂ ವಲ್ಲಭಃ ಸದಾ ॥೩೬॥

ಗರ್ವೀ ಖರ್ವೀ ಭವತ್ಯೇವ ವಾದೀ ಸ್ಖಲತಿ ದರ್ಶನಾತ್ ।\\
ಮೃತ್ಯುಶ್ಚ ವಶ್ಯತಾಂ ಯಾತಿ ದಾಸಾಸ್ತಸ್ಯಾವನೀಭುಜಃ ॥೩೭॥

ಪ್ರಸಂಗಾತ್ಕಥಿತಂ ಸರ್ವಂ ಕವಚಂ ಸರ್ವಕಾಮದಂ ।\\
ಪ್ರಪಠನ್ವಾ ಸ್ಮರನ್ಮರ್ತ್ಯಃ ಶಾಪಾನುಗ್ರಹಣೇ ಕ್ಷಮಃ ॥೩೮॥

ಆನಂದವೃಂದಸಿಂಧೂನಾಮಧಿಪಃ ಕವಿರಾಡ್ ಭವೇತ್ ।\\
ಸರ್ವವಾಗಿಶ್ವರೋ ಮರ್ತ್ಯೋ ಲೋಕವಶ್ಯಃ ಸದಾ ಸುಖೀ ॥೩೯॥

ಗುರೋಃ ಪ್ರಸಾದಮಾಸಾದ್ಯ ವಿದ್ಯಾಂ ಪ್ರಾಪ್ಯ ಸುಗೋಪಿತಾಂ ।\\
ತತ್ರಾಪಿ ಕವಚಂ ದೇವಿ ದುರ್ಲಭಂ ಭುವನತ್ರಯೇ ॥೪೦॥

ಗುರುರ್ದೇವೋ ಹರಃ ಸಾಕ್ಷಾತ್ತತ್ಪತ್ನೀ ತು ಹರಪ್ರಿಯಾ ।\\
ಅಭೇದೇನ ಭಜೇದ್ಯಸ್ತು ತಸ್ಯ ಸಿದ್ಧಿದೂರತಃ ॥೪೧॥

ಮಂತ್ರಾಚಾರಾ ಮಹೇಶಾನಿ ಕಥಿತಾಃ ಪೂರ್ವವತ್ಪ್ರಿಯೇ ।\\
ನಾಭೌ ಜ್ಯೋತಿಸ್ತಥಾ ರಕ್ತಂ ಹೃದಯೋಪರಿ ಚಿಂತಯೇತ್ ॥೪೨॥

ಐಶ್ವರ್ಯಂ ಸುಕವಿತ್ವಂ ಚ ಮಹಾವಾಗಿಶ್ವರೋ ನೃಪಃ ।\\
ನಿತ್ಯಂ ತಸ್ಯ ಮಹೇಶಾನಿ ಮಹಿಲಾಸಂಗಮಂ ಚರೇತ್ ॥೪೩॥

ಪಂಚಾಚಾರರತೋ ಮರ್ತ್ಯಃ ಸಿದ್ಧೋ ಭವತಿ ನಾನ್ಯಥಾ ।\\
ಶಕ್ತಿಯುಕ್ತೋ ಭವೇನ್ಮರ್ತ್ಯಃ ಸಿದ್ಧೋ ಭವತಿ ನಾನ್ಯಥಾ ॥೪೪॥

ಬ್ರಹ್ಮಾ ವಿಷ್ಣುಶ್ಚ ರುದ್ರಶ್ಚ ಯೇ ದೇವಾಸುರಮಾನುಷಾಃ ।\\
ತಂ ದೃಷ್ಟ್ವಾ ಸಾಧಕಂ ದೇವಿ ಲಜ್ಜಾಯುಕ್ತಾ ಭವಂತಿ ತೇ ॥೪೫॥

ಸ್ವರ್ಗೇ ಮರ್ತ್ಯೇ ಚ ಪಾತಾಲೇ ಯೇ ದೇವಾಃ ಸಿದ್ಧಿದಾಯಕಾಃ ।\\
ಪ್ರಶಂಸಂತಿ ಸದಾ ದೇವಿ ತಂ ದೃಷ್ಟ್ವಾ ಸಾಧಕೋತ್ತಮಂ ॥೪೬॥

ವಿಘ್ನಾತ್ಮಕಾಶ್ಚ ಯೇ ದೇವಾಃ ಸ್ವರ್ಗೇ ಮರ್ತ್ಯೇ ರಸಾತಲೇ ।\\
ಪ್ರಶಂಸಂತಿ ಸದಾ ಸರ್ವೇ ತಂ ದೃಷ್ಟ್ವಾ ಸಾಧಕೋತ್ತಮಂ ॥೪೭॥

ಇತಿ ತೇ ಕಥಿತಂ ದೇವಿ ಮಯಾ ಸಮ್ಯಕ್ಪ್ರಕೀರ್ತಿತಂ ।\\
ಭುಕ್ತಿಮುಕ್ತಿಕರಂ ಸಾಕ್ಷಾತ್ಕಲ್ಪವೃಕ್ಷಸ್ವರೂಪಕಂ ॥೪೮॥

ಆಸಾದ್ಯಾದ್ಯಗುರುಂ ಪ್ರಸಾದ್ಯ ಯ ಇದಂ ಕಲ್ಪದ್ರುಮಾಲಂಬನಂ\\
ಮೋಹೇನಾಪಿ ಮದೇನ ಚಾಪಿ ರಹಿತೋ ಜಾಡ್ಯೇನ ವಾ ಯುಜ್ಯತೇ ।\\
ಸಿದ್ಧೋಽಸೌ ಭುವಿ ಸರ್ವದುಃಖವಿಪದಾಂ ಪಾರಂ ಪ್ರಯಾತ್ಯಂತಕೇ\\
ಮಿತ್ರಂ ತಸ್ಯ ನೃಪಾಶ್ಚ ದೇವಿ ವಿಪದೋ ನಶ್ಯಂತಿ ತಸ್ಯಾಶು ಚ ॥೪೯॥

ತದ್ಗಾತ್ರಂ ಪ್ರಾಪ್ಯ ಶಸ್ತ್ರಾಣಿ ಬ್ರಹ್ಮಾಸ್ತ್ರಾದೀನಿ ವೈ ಭುವಿ ।\\
ತಸ್ಯ ಗೇಹೇ ಸ್ಥಿರಾ ಲಕ್ಷ್ಮೀರ್ವಾಣೀ ವಕ್ತ್ರೇ ವಸೇದ್ ಧ್ರುವಂ ॥೫೦॥

ಇದಂ ಕವಚಮಜ್ಞಾತ್ವಾ ತಾರಾಂ ಯೋ ಭಜತೇ ನರಃ ।\\
ಅಲ್ಪಾಯುರ್ನಿರ್ದ್ಧನೋ ಮೂರ್ಖೋ ಭವತ್ಯೇವ ನ ಸಂಶಯಃ ॥೫೧॥

ಲಿಖಿತ್ವಾ ಧಾರಯೇದ್ಯಸ್ತು ಕಂಠೇ ವಾ ಮಸ್ತಕೇ ಭುಜೇ ।\\
ತಸ್ಯ ಸರ್ವಾರ್ಥಸಿದ್ಧಿಃ ಸ್ಯಾದ್ಯದ್ಯನ್ಮನಸಿ ವರ್ತತೇ ॥೫೨॥

ಗೋರೋಚನಾಕುಂಕುಮೇನ ರಕ್ತಚಂದನಕೇನ ವಾ ।\\
ಯಾವಕೈರ್ವಾ ಮಹೇಶಾನಿ ಲಿಖೇನ್ಮಂತ್ರಂ ಸಮಾಹಿತಃ ॥೫೩॥

ಅಷ್ಟಮ್ಯಾಂ ಮಂಗಲದಿನೇ ಚತುರ್ದ್ದಶ್ಯಾಮಥಾಪಿ ವಾ ।\\
ಸಂಧ್ಯಾಯಾಂ ದೇವದೇವೇಶಿ ಲಿಖೇದ್ಯಂತ್ರಂ ಸಮಾಹಿತಃ ॥೫೪॥

ಮಘಾಯಾಂ ಶ್ರವಣೇ ವಾಪಿ ರೇವತ್ಯಾಂ ವಾ ವಿಶೇಷತಃ ।\\
ಸಿಂಹರಾಶೌ ಗತೇ ಚಂದ್ರೇ ಕರ್ಕಟಸ್ಥೇ ದಿವಾಕರೇ ॥೫೫॥

ಮೀನರಾಶೌ ಗುರೌ ಯಾತೇ ವೃಶ್ಚಿಕಸ್ಥೇ ಶನೈಶ್ಚರೇ ।\\
ಲಿಖಿತ್ವಾ ಧಾರಯೇದ್ಯಸ್ತು ಉತ್ತರಾಭಿಮುಖೋ ಭವೇತ್ ॥೫೬॥

ಶ್ಮಶಾನೇ ಪ್ರಾಂತರೇ ವಾಪಿ ಶೂನ್ಯಾಗಾರೇ ವಿಶೇಷತಃ ।\\
ನಿಶಾಯಾಂ ವಾ ಲಿಖೇನ್ಮಂತ್ರಂ ತಸ್ಯ ಸಿದ್ಧಿರಚಂಚಲಾ ॥೫೭॥

ಭೂರ್ಜಪತ್ರೇ ಲಿಖೇನ್ಮಂತ್ರಂ ಗುರುಣಾ ಚ ಮಹೇಶ್ವರಿ ।\\
ಧ್ಯಾನಧಾರಣಯೋಗೇನ ಧಾರಯೇದ್ಯಸ್ತು ಭಕ್ತಿತಃ ।\\
ಅಚಿರಾತ್ತಸ್ಯ ಸಿದ್ಧಿಃ ಸ್ಯಾನ್ನಾತ್ರ ಕಾರ್ಯಾ ವಿಚಾರಣಾ ॥೫೮॥
\authorline{॥ಇತಿ ಶ್ರೀರುದ್ರಯಾಮಲೇ ತಂತ್ರೇ ಉಗ್ರತಾರಾಕವಚಂ ಸಂಪೂರ್ಣಂ ॥}
%=========================================================================================================================
\section{ಅಥ ಬಗಲಾಮುಖೀಕವಚಮ್}
\addcontentsline{toc}{section}{ಅಥ ಬಗಲಾಮುಖೀಕವಚಮ್}
ಶ್ರುತ್ವಾ ಚ ಬಗಲಾಪೂಜಾಂ ಸ್ತೋತ್ರಂ ಚಾಪಿ ಮಹೇಶ್ವರ ।\\
ಇದಾನೀಂ ಶ್ರೋತುಮಿಚ್ಛಾಮಿ ಕವಚಂ ವದ ಮೇ ಪ್ರಭೋ ॥

ವೈರಿನಾಶಕರಂ ದಿವ್ಯಂ ಸರ್ವಾಶುಭವಿನಾಶನಂ ।\\
ಶುಭದಂ ಸ್ಮರಣಾತ್ಪುಣ್ಯಂ ತ್ರಾಹಿ ಮಾಂ ದುಃಖನಾಶನಂ ॥

ಶ್ರೀಭೈರವ ಉವಾಚ ।\\
ಕವಚಂ ಶೃಣು ವಕ್ಷ್ಯಾಮಿ ಭೈರವಿ ಪ್ರಾಣವಲ್ಲಭೇ ।\\
ಪಠಿತ್ವಾ ಧಾರಯಿತ್ವಾ ತು ತ್ರೈಲೋಕ್ಯೇ ವಿಜಯೀ ಭವೇತ್ ॥

ಓಂ ಅಸ್ಯ ಶ್ರೀಬಗಲಾಮುಖೀಕವಚಸ್ಯ ನಾರದಋಷಿಃ । ಅನುಷ್ಟುಪ್ಛಂದಃ । ಶ್ರೀಬಗಲಾಮುಖೀ ದೇವತಾ । ಲಂ ಬೀಜಂ । ಐಂ ಕೀಲಕಂ । ಪುರುಷಾರ್ಥಚತುಷ್ಟಯೇ ಜಪೇ ವಿನಿಯೋಗಃ ॥

ಶಿರೋ ಮೇ ಬಗಲಾ ಪಾತು ಹೃದಯೈಕಾಕ್ಷರೀ ಪರಾ ।\\
ಓಂ ಹ್ರೀಂ ಓಂ ಮೇ ಲಲಾಟೇ ಚ ಬಗಲಾ ವೈರಿನಾಶಿನೀ ॥

ಗದಾಹಸ್ತಾ ಸದಾ ಪಾತು ಮುಖಂ ಮೇ ಮೋಕ್ಷದಾಯಿನೀ ।\\
ವೈರಿಜಿಹ್ವಾಂಧರಾ ಪಾತು ಕಂಠಂ ಮೇ ಬಗಲಾಮುಖೀ ॥

ಉದರಂ ನಾಭಿದೇಶಂ ಚ ಪಾತು ನಿತ್ಯಂ ಪರಾತ್ಪರಾ ।\\
ಪರಾತ್ಪರತರಾ ಪಾತು ಮಮ ಗುಹ್ಯಂ ಸುರೇಶ್ವರೀ ॥

ಹಸ್ತೌ ಚೈವ ತಥಾ ಪಾತು ಪಾರ್ವತೀ ಪರಿಪಾತು ಮೇ ।\\
ವಿವಾದೇ ವಿಷಮೇ ಘೋರೇ ಸಂಗ್ರಾಮೇ ರಿಪುಸಂಕಟೇ ॥

ಪೀತಾಂಬರಧರಾ ಪಾತು ಸರ್ವಾಂಗಂ ಶಿವನರ್ತಕೀ ।\\
ಶ್ರೀವಿದ್ಯಾಸಮಯೋ ಪಾತು ಮಾತಂಗೀ ದುರಿತಾ ಶಿವಾ ॥

ಪಾತು ಪುತ್ರಂ ಸುತಾಂ ಚೈವ ಕಲತ್ರಂ ಕಾಲಿಕಾ ಮಮ ।\\
ಪಾತು ನಿತ್ಯಂ ಭ್ರಾತರಂ ಮೇ ಪಿತರಂ ಶೂಲಿನೀ ಸದಾ ॥

ಸಂದೇಹಿ ಬಗಲಾದೇವ್ಯಾಃ ಕವಚಂ ಮನ್ಮುಖೋದಿತಂ ।\\
ನೈವ ದೇಯಮಮುಖ್ಯಾಯ ಸರ್ವಸಿದ್ಧಿಪ್ರದಾಯಕಂ ॥

ಪಠನಾದ್ಧಾರಣಾದಸ್ಯ ಪೂಜನಾದ್ವಾಂಛಿತಂ ಲಭೇತ್ ।\\
ಇದಂ ಕವಚಮಜ್ಞಾತ್ವಾ ಯೋ ಜಪೇದ್ ಬಗಲಾಮುಖೀಂ ॥

ಪಿಬಂತಿ ಶೋಣಿತಂ ತಸ್ಯ ಯೋಗಿನ್ಯಃ ಪ್ರಾಪ್ಯಸಾದರಾಃ ।\\
ವಶ್ಯೇ ಚಾಕರ್ಷಣೇ ಚೈವ ಮಾರಣೇ ಮೋಹನೇ ತಥಾ ॥

ಮಹಾಭಯೇ ವಿಪತ್ತೌ ಚ ಪಠೇದ್ವಾಪಾಠಯೇತ್ತು ಯಃ ।\\
ತಸ್ಯ ಸರ್ವಾರ್ಥಸಿದ್ಧಿಃ ಸ್ಯಾದ್ಭಕ್ತಿಯುಕ್ತಸ್ಯ ಪಾರ್ವತೀ ॥

\authorline{ಇತಿ ಶ್ರೀರುದ್ರಯಾಮಲೇ ಬಗಲಾಮುಖೀಕವಚಂ ಸಂಪೂರ್ಣಂ ।}
%=========================================================================================================================
\section{ಶ್ರೀಬಗಲಾಮುಖೀಕವಚಂ ।}
\addcontentsline{toc}{section}{ಶ್ರೀಬಗಲಾಮುಖೀಕವಚಂ ।}
ಕೈಲಾಸಾಚಲಮಧ್ಯಗಂಪುರವಹಂ ಶಾಂತಂತ್ರಿನೇತ್ರಂ ಶಿವ।\\
ವಾಮಸ್ಥಾ ಕವಚಂ ಪ್ರಣಮ್ಯ ಗಿರಿಜಾ ಭೂತಿಪ್ರದಂ ಪೃಚ್ಛತಿ ॥

ದೇವೀ ಶ್ರೀಬಗಲಾಮುಖೀ ರಿಪುಕುಲಾರಣ್ಯಾಗ್ನಿರೂಪಾ ಚ ಯ।\\
ತಸ್ಯಾಶ್ಚಾಪವಿಮುಕ್ತಮಂತ್ರಸಹಿತಂ ಪ್ರೀತ್ಯಾಽಧುನಾ ಬ್ರೂಹಿ ಮಾಂ ॥೧॥

ಶ್ರೀಶಂಕರ ಉವಾಚ।\\
ದೇವೀ ಶ್ರೀಭವವಲ್ಲಭೇ ಶೃಣು ಮಹಾಮಂತ್ರಂ ವಿಭೂತಿಪ್ರದಂ\\
ದೇವ್ಯಾ ವರ್ಮಯುತಂ ಸಮಸ್ತಸುಖದಂ ಸಾಮ್ರಾಜ್ಯದಮುಕ್ತಿಪ್ರದಂ ।\\
ತಾರಂ ರುದ್ರವಧೂಂ ವಿರಂಚಿಮಹಿಲಾವಿಷ್ಣುಪ್ರಿಯಾಕಾಮಯುಕ್\\
ಕಾಂತೇ ಶ್ರೀಬಗಲಾನನೇ ಮಮ ರಿಪೂನ್ನಾಶಾಯ ಯುಗ್ಮಂತ್ವಿತಿ ॥೨॥

ಐಶ್ವರ್ಯಾಣಿ ಪದಂಚ ದೇಹಿ ಯುಗಲಂ ಶೀಘ್ರಂ ಮನೋವಾಂಛಿತ\\
ಕಾರ್ಯಂ ಸಾಧಯ ಯುಗ್ಮಯುಕ್ಛಿವವಧೂವಹ್ನಿಪ್ರಿಯಾಂತೋ ಮನುಃ ।\\
ಕಂಸಾರೇಸ್ತನಯಂಚ ಬೀಜಮಪರಾ ಶಕ್ತಿಶ್ಚ ವಾಣೀ ತಥಾ\\
ಕೀಲಂ ಶ್ರೀಮತಿ ಭೈರವರ್ಷಿಸಹಿತಂ ಛಂದೋ ವಿರಾಟ್ಸಂಯುತಂ ॥೩॥

ಸ್ವೇಷ್ಟಾರ್ಥಸ್ಯ ಪರಸ್ಯ ವೇತ್ತಿ ನಿತರಾಂಕಾರ್ಯಸ್ಯ ಸಂಪ್ರಾಪ್ತಯ।\\
ನಾನಾಸಾಧ್ಯಮಹಾಗದಸ್ಯ ನಿಯತನ್ನಾಶಾಯ ವೀರ್ಯಾಪ್ತಯೇ ।\\
ಧ್ಯಾತ್ವಾ ಶ್ರೀಬಗಲಾನನಾಮನುವರಂಜಪ್ತ್ವಾ ಸಹಸ್ರಾಖ್ಯಕ।\\
ದೀರ್ಘೈಃಷಟ್ಕಯುತೈಶ್ಚ ರುದ್ರಮಹಿತಾಬೀಜೈರ್ವಿನಶ್ಯಾಂಗಕೇ ॥

ಓಂ ಅಸ್ಯ ಶ್ರೀಬಗಲಾಮುಖೀಬ್ರಹ್ಮಾಸ್ತ್ರಮಂತ್ರಕವಚಸ್ಯ ಭೈರವ ಋಷಿಃ । ವಿರಾಟ್ಛಂದಃ । ಶ್ರೀಬಗಲಾಮುಖೀ ದೇವತಾ । ಕ್ಲೀಂ ಬೀಜಂ । ಐಂ ಶಕ್ತಿಃ । ಶ್ರೀಂ ಕೀಲಕಂ । ಮಮ ಪರಸ್ಯ ಚ ಮನೋಭಿಲಷಿತೇಷ್ಟಕಾರ್ಯಸಿದ್ಧಯೇ ವಿನಿಯೋಗಃ ।\\
ಹ್ರಾಂ ಇತ್ಯಾದಿನಾ ನ್ಯಾಸ।\\
ಧ್ಯಾನಂ ।\\
ಸೌವರ್ಣಾಸನಸಂಸ್ಥಿತಾಂ ತ್ರಿನಯನಾಂ ಪೀತಾಂಶುಕೋಲ್ಲಾಸಿನೀ।\\
ಹೇಮಾಭಾಂಗರುಚಿಂ ಶಶಾಂಕಮುಕುಟಾಂ ಸ್ರಕ್ಚಂಪಕಸ್ರಗ್ಯುತಾಂ ।\\
ಹಸ್ತೈರ್ಮುದ್ಗರಪಾಶಬದ್ಧರಸನಾಂ ಸಂಬಿಭ್ರತೀಂ ಭೂಷಣ।\\
ವ್ಯಾಪ್ತಾಂಗೀಂ ಬಗಲಾಮುಖೀಂ ತ್ರಿಜಗತಾಂ ಸಂಸ್ತಂಭಿನೀಂ ಚಿಂತಯೇ ।\\
ಮಂತ್ರೋದ್ಧಾರಃ।\\
ಓಂ ಹ್ರಾಂ ಐಂ ಶ್ರೀಂ ಕ್ಲೀಂ ಶ್ರೀಬಗಲಾನನೇ ಮಮ ರಿಪೂನ್ನಾಶಯ ನಾಶಯ ಮಮೈಶ್ವರ್ಯಾಣಿ ದೇಹಿ ದೇಹಿ ಶೀಘ್ರಮನೋವಾಂಛಿತಕಾರ್ಯಂ ಸಾಧಯ ಸಾಧಯ ಹ್ರೀಂ ಸ್ವಾಹಾ ॥

ಶಿರೋ ಮೇ ಪಾತು ಓಂ ಹ್ರೀಂ ಐಂ ಶ್ರೀಂ ಕ್ಲೀಂ ಪಾತು ಲಲಾಟಕಂ ।\\
ಸಂಬೋಧನಪದಂ ಪಾತು ನೇತ್ರೇ ಶ್ರೀಬಗಲಾನನೇ ॥೧॥

ಶ್ರುತೌ ಮಮ ರಿಪೂನ್ ಪಾತು ನಾಸಿಕಾನ್ನಾಶಯ ದ್ವಯಂ ।\\
ಪಾತು ಗಂಡೌ ಸದಾ ಮಮೈಶ್ವರ್ಯಾಣ್ಯಂತಂತು ಮಸ್ತಕಂ ॥೨॥

ದೇಹಿ ದ್ವಂದ್ವಂ ಸದಾ ಜಿಹ್ವಾಂ ಪಾತು ಶೀಘ್ರಂ ವಚೋ ಮಮ ।\\
ಕಂಠದೇಶಂ ಸ ನಃ ಪಾತು ವಾಂಛಿತಂ ಬಾಹುಮೂಲಕಂ ॥೩॥

ಕಾರ್ಯಂ ಸಾಧಯ ದ್ವಂದ್ವಂತು ಕರೌ ಪಾತು ಸದಾ ಮಮ ।\\
ಮಾಯಾಯುಕ್ತಾ ತಥಾ ಸ್ವಾಹಾ ಹೃದಯಂಪಾತು ಸರ್ವದಾ ॥೪॥

ಅಷ್ಟಾಧಿಕಚತ್ವಾರಿಂಶದಂಡಾಢ್ಯಾ ಬಗಲಾಮುಖೀ ।\\
ರಕ್ಷಾಂಕರೋತು ಸರ್ವತ್ರ ಗೃಹೇಽರಣ್ಯೇ ಸದಾ ಮಮ ॥೫॥

ಬ್ರಹ್ಮಾಸ್ತ್ರಾಖ್ಯೋ ಮನುಃ ಪಾತು ಸರ್ವಾಂಗೇ ಸರ್ವಸಂಧಿಷು ।\\
ಮಂತ್ರರಾಜಃ ಸದಾ ರಕ್ಷಾಂಕರೋತು ಮಮ ಸರ್ವದಾ ॥೬॥

ಓಂ ಹ್ರೀಂ ಪಾತು ನಾಭಿದೇಶಂ ಕಟಿಂ ಮೇ ಬಗಲಾವತು ।\\
ಮುಖೀ ವರ್ಣದ್ವಯಂಪಾತು ಲಿಂಗಂ ಮೇ ಮುಷ್ಕಯುಗ್ಮಕಂ ॥೭॥

ಜಾನುನೀ ಸರ್ವದುಷ್ಟಾನಾಂ ಪಾತು ಮೇ ವರ್ಣಪಂಚಕಂ ।\\
ವಾಚಂ ಮುಖಂತಥಾ ಪಾದಂ ಷಡ್ವರ್ಣಾ ಪರಮೇಶ್ವರೀ ॥೮॥

ಜಂಘಾಯುಗ್ಮಂ ಸದಾ ಪಾತು ಬಗಲಾ ರಿಪುಮೋಹಿನೀ ।\\
ಸ್ತಂಭಯೇತಿ ಪದಂ ಪೃಷ್ಠಂ ಪಾತು ವರ್ಣತ್ರಯಂ ಮಮ ॥೯॥

ಜಿಹ್ವಾಂ ವರ್ಣದ್ವಯಂ ಪಾತು ಗುಲ್ಫೌ ಮೇ ಕೀಲಯೇತಿ ಚ ।\\
ಪಾದೋರ್ಧ್ವಂ ಸರ್ವದಾ ಪಾತು ಬುದ್ಧಿಂ ಪಾದತಲೇ ಮಮ ॥೧೦॥

ವಿನಾಶಯ ಪದಂಪಾತು ಪಾದಾಂಗುಲ್ಯೋರ್ನ್ನಖಾನಿ ಮೇ ।\\
ಹ್ರೀಂ ಬೀಜಂ ಸರ್ವದಾ ಪಾತು ಬುದ್ಧೀಂದ್ರಿಯವಚಾಂಸಿ ಮೇ ॥೧೧॥

ಸರ್ವಾಂಗಂಪ್ರಣವಃ ಪಾತು ಸ್ವಾಹಾ ರೋಮಾಣಿ ಮೇಽವತು ।\\
ಬ್ರಾಹ್ಮೀ ಪೂರ್ವದಲೇ ಪಾತು ಚಾಗ್ನೇಯ್ಯಾಂ ವಿಷ್ಣುವಲ್ಲಭಾ ॥೧೨॥

ಮಾಹೇಶೀ ದಕ್ಷಿಣೇ ಪಾತು ಚಾಮುಂಡಾ ರಾಕ್ಷಸೇಽವತು ।\\
ಕೌಮಾರೀ ಪಶ್ಚಿಮೇ ಪಾತು ವಾಯವ್ಯೇ ಚಾಪರಾಜಿತಾ ॥೧೩॥

ವಾರಾಹೀ ಚೋತ್ತರೇ ಪಾತು ನಾರಸಿಂಹೀ ಶಿವೇಽವತು ।\\
ಊರ್ಧ್ವಂ ಪಾತು ಮಹಾಲಕ್ಷ್ಮೀಃ ಪಾತಾಲೇ ಶಾರದಾಽವತು ॥೧೪॥

ಇತ್ಯಷ್ಟೌ ಶಕ್ತಯಃ ಪಾಂತು ಸಾಯುಧಾಶ್ಚ ಸವಾಹನಾಃ ।\\
ರಾಜದ್ವಾರೇ ಮಹಾದುರ್ಗೇ ಪಾತು ಮಾಂ ಗಣನಾಯಕಃ ॥೧೫॥

ಶ್ಮಶಾನೇ ಜಲಮಧ್ಯೇ ಚ ಭೈರವಶ್ಚ ಸದಾಽವತು ।\\
ದ್ವಿಭುಜಾ ರಕ್ತವಸನಾಃ ಸರ್ವಾಭರಣಭೂಷಿತಾಃ ॥೧೬॥

ಯೋಗಿನ್ಯಃ ಸರ್ವದಾ ಪಾಂತು ಮಹಾರಣ್ಯೇ ಸದಾ ಮಮ ।\\
ಇತಿ ತೇ ಕಥಿತಂದೇವಿ ಕವಚಂ ಪರಮಾದ್ಭುತಂ ॥೧೭॥

ಶ್ರೀವಿಶ್ವವಿಜಯನ್ನಾಮ ಕೀರ್ತಿಶ್ರೀವಿಜಯಪ್ರದಂ ।\\
ಅಪುತ್ರೋ ಲಭತೇ ಪುತ್ರಂ ಧೀರಂ ಶೂರಂ ಶತಾಯುಷಂ ॥೧೮॥

ನಿರ್ದ್ಧನೋ ಧನಮಾಪ್ನೋತಿ ಕವಚಸ್ಯಾಸ್ಯ ಪಾಠತಃ ।\\
ಜಪಿತ್ವಾ ಮಂತ್ರರಾಜಂ ತು ಧ್ಯಾತ್ವಾ ಶ್ರೀಬಗಲಾಮುಖೀಂ ॥೧೯॥

ಪಠೇದಿದಂ ಹಿ ಕವಚನ್ನಿಶಾಯಾನ್ನಿಯಮಾತ್ತು ಯಃ ।\\
ಯದ್ಯತ್ಕಾಮಯತೇ ಕಾಮಂ ಸಾಧ್ಯಾಸಾಧ್ಯೇ ಮಹೀತಲೇ ॥೨೦॥

ತತ್ತತ್ಕಾಮಮವಾಪ್ನೋತಿ ಸಪ್ತರಾತ್ರೇಣ ಶಂಕರೀ ।\\
ಗುರುಂ ಧ್ಯಾತ್ವಾ ಸುರಾಂ ಪೀತ್ವಾ ರಾತ್ರೌ ಶಕ್ತಿಸಮನ್ವಿತಃ ॥೨೧॥

ಕವಚಂ ಯಃ ಪಠೇದ್ದೇವಿ ತಸ್ಯಾಽಸಾಧ್ಯನ್ನ ಕಿಂಚನ ।\\
ಯಂ ಧ್ಯಾತ್ವಾ ಪ್ರಜಪೇನ್ಮಂತ್ರಸಹಸ್ರಂ ಕವಚಂ ಪಠೇತ್ ॥೨೨॥

ತ್ರಿರಾತ್ರೇಣ ವಶಂ ಯಾತಿ ಮೃತ್ಯುಂತನ್ನಾತ್ರ ಸಂಶಯಃ ।\\
ಲಿಖಿತ್ವಾ ಪ್ರತಿಮಾಂ ಶತ್ರೋಃ ಸತಾಲೇನ ಹರಿದ್ರಯಾ ॥೨೩॥

ಲಿಖಿತ್ವಾ ಹೃದಿ ತನ್ನಾಮ ತಂಧ್ಯಾತ್ವಾ ಪ್ರಜಪೇನ್ಮನುಂ ।\\
ಏಕವಿಂಶದಿನಂ ಯಾವತ್ಪ್ರತ್ಯಹಂಚ ಸಹಸ್ರಕಂ ॥೨೪॥

ಜಪ್ತ್ವಾ ಪಠೇತ್ತು ಕವಚಂ ಚತುರ್ವಿಂಶತಿವಾರಕಂ ।\\
ಸಂಸ್ತಂಭಂ ಜಾಯತೇ ಶತ್ರೋರ್ನಾತ್ರ ಕಾರ್ಯಾ ವಿಚಾರಣಾ ॥೨೫॥

ವಿವಾದೇ ವಿಜಯಂತಸ್ಯ ಸಂಗ್ರಾಮೇ ಜಯಮಾಪ್ನುಯಾತ್ ।\\
ಶ್ಮಶಾನೇ ಚ ಭಯಂ ನಾಸ್ತಿ ಕವಚಸ್ಯ ಪ್ರಭಾವತಃ ॥೨೬॥

ನವನೀತಂಚಾಭಿಮಂತ್ರ್ಯ ಸ್ತ್ರೀಣಾಂದದ್ಯಾನ್ಮಹೇಶ್ವರಿ ।\\
ವಂಧ್ಯಾಯಾಂ ಜಾಯತೇ ಪುತ್ರೋ ವಿದ್ಯಾಬಲಸಮನ್ವಿತಃ ॥೨೭॥

ಶ್ಮಶಾನಾಂಗಾರಮಾದಾಯ ಭೌಮೇ ರಾತ್ರೌ ಶನಾವಥ ।\\
ಪಾದೋದಕೇನ ಸ್ಪೃಷ್ಟ್ವಾ ಚ ಲಿಖೇಲ್ಲೌಹಶಲಾಕಯಾ ॥೨೮॥

ಭೂಮೌ ಶತ್ರೋಃ ಸ್ವರೂಪಂಚ ಹೃದಿ ನಾಮ ಸಮಾಲಿಖೇತ್ ।\\
ಹಸ್ತಂತದ್ಧೃದಯೇ ದತ್ವಾ ಕವಚಂತಿಥಿವಾರಕಂ ॥೨೯॥

ಧ್ಯಾತ್ವಾ ಜಪೇನ್ಮಂತ್ರರಾಜನ್ನವರಾತ್ರಂ ಪ್ರಯತ್ನತಃ ।\\
ಮ್ರಿಯತೇ ಜ್ವರದಾಹೇನ ದಶಮೇಽಹ್ನಿ ನ ಸಂಶಯಃ ॥೩೦॥

ಭೂರ್ಜಪತ್ರೇಷ್ವಿದಂ ಸ್ತೋತ್ರಮಷ್ಟಗಂಧೇನ ಸಂಲಿಖೇತ್ ।\\
ಧಾರಯೇದ್ದಕ್ಷಿಣೇ ಬಾಹೌ ನಾರೀ ವಾಮಭುಜೇ ತಥಾ ॥೩೧॥

ಸಂಗ್ರಾಮೇ ಜಯಮಾಪ್ನೋತಿ ನಾರೀ ಪುತ್ರವತೀ ಭವೇತ್ ।\\
ಬ್ರಹ್ಮಾಸ್ತ್ರಾದೀನಿ ಶಸ್ತ್ರಾಣಿ ನೈವ ಕೃಂತಂತಿ ತಂಜನಂ ॥೩೨॥

ಸಂಪೂಜ್ಯ ಕವಚನ್ನಿತ್ಯಂಪೂಜಾಯಾಃ ಫಲಮಾಲಭೇತ್ ।\\
ಬೃಹಸ್ಪತಿಸಮೋ ವಾಪಿ ವಿಭವೇ ಧನದೋಪಮಃ ॥೩೩॥

ಕಾಮತುಲ್ಯಶ್ಚ ನಾರೀಣಾಂ ಶತ್ರೂಣಾಂಚ ಯಮೋಪಮಃ ।\\
ಕವಿತಾಲಹರೀ ತಸ್ಯ ಭವೇದ್ಗಂಗಾಪ್ರವಾಹವತ್ ॥೩೪॥

ಗದ್ಯಪದ್ಯಮಯೀ ವಾಣೀ ಭವೇದ್ದೇವೀಪ್ರಸಾದತಃ ।\\
ಏಕಾದಶಶತಂ ಯಾವತ್ಪುರಶ್ಚರಣಮುಚ್ಯತೇ ॥೩೫॥

ಪುರಶ್ಚರ್ಯಾವಿಹೀನಂತು ನ ಚೇದಂ ಫಲದಾಯಕಂ ।\\
ನ ದೇಯಂ ಪರಶಿಷ್ಯೇಭ್ಯೋ ದುಷ್ಟೇಭ್ಯಶ್ಚ ವಿಶೇಷತಃ ॥೩೬॥

ದೇಯಂ ಶಿಷ್ಯಾಯ ಭಕ್ತಾಯ ಪಂಚತ್ವಂಚಾಽನ್ಯಥಾಪ್ನುಯಾತ್ ।\\
ಇದಂಕವಚಮಜ್ಞಾತ್ವಾ ಭಜೇದ್ಯೋ ಬಗಲಾಮುಖೀಂ ॥

ಶತಕೋಟಿ ಜಪಿತ್ವಾ ತು ತಸ್ಯ ಸಿದ್ಧಿರ್ನ್ನ ಜಾಯತೇ ॥೩೭॥

ದಾರಾಢ್ಯೋ ಮನುಜೋಸ್ಯ ಲಕ್ಷಜಪತಃ ಪ್ರಾಪ್ನೋತಿ ಸಿದ್ಧಿಂಪರಾ।\\
ವಿದ್ಯಾಂ ಶ್ರೀವಿಜಯಂತಥಾ ಸುನಿಯತಂಧಾರಂಚ ವೀರಂ ವರಂ ॥

ಬ್ರಹ್ಮಾಸ್ತ್ರಾಖ್ಯಮನುಂ ವಿಲಿಖ್ಯ ನಿತರಾಂಭೂರ್ಜೇಽಷ್ಟಗಂಧೇನ ವ।\\
ಧೃತ್ವಾ ರಾಜಪುರಂ ವ್ರಜಂತಿ ಖಲು ಯೇ ದಾಸೋಽಸ್ತಿ ತೇಷಾನ್ನೃಪಃ ॥೩೮॥

\authorline{ಇತಿ ವಿಶ್ವಸಾರೋದ್ಧಾರತಂತ್ರೇ ಪಾರ್ವತೀಶ್ವರಸಂವಾದೇ ಬಗಲಾಮುಖೀಕವಚಂ ಸಂಪೂರ್ಣಂ ॥}
%======================================================================================================
\section{ಧೂಮಾವತೀ ಕವಚಂ ।}
\addcontentsline{toc}{section}{ಧೂಮಾವತೀ ಕವಚಂ ।}
ಶ್ರೀಪಾರ್ವತ್ಯುವಾಚ\\
ಧೂಮಾವತ್ಯರ್ಚನಂ ಶಂಭೋ ಶ್ರುತಂ ವಿಸ್ತರತೋಮಯಾ ।\\
ಕವಚಂ ಶ್ರೋತುಮಿಚ್ಛಾಮಿ ತಸ್ಯಾ ದೇವ ವದಸ್ವ ಮೇ ॥೧॥

ಶ್ರೀಭೈರವ ಉವಾಚ\\
ಶೃಣುದೇವಿ ಪರಂ ಗುಹ್ಯಂ ನ ಪ್ರಕಾಶ್ಯಂ ಕಲೌಯುಗೇ ।\\
ಕವಚಂ ಶ್ರೀಧೂಮಾವತ್ಯಾಶ್ಶತ್ರುನಿಗ್ರಹಕಾರಕಂ ॥೨॥

ಬ್ರಹ್ಮಾದ್ಯಾದೇವಿ ಸತತಂ ಯದ್ವಶಾದರಿಘಾತಿನಃ ।\\
ಯೋಗಿನೋಭವಛತ್ರುಘ್ನಾ ಯಸ್ಯಾಧ್ಯಾನ ಪ್ರಭಾವತಃ ॥೩॥

ಓಂ ಅಸ್ಯ ಶ್ರೀಧೂಮಾವತೀಕವಚಸ್ಯ ಪಿಪ್ಪಲಾದ ಋಷಿಃ । ಅನುಷ್ಟುಪ್ಛಂದಃ । ಶ್ರೀಧೂಮಾವತೀ ದೇವತಾ । ಧೂಂ ಬೀಜಂ । ಸ್ವಾಹಾಶಕ್ತಿಃ । ಧೂಮಾವತೀ ಕೀಲಕಂ । ಶತ್ರುಹನನೇ ಪಾಠೇ ವಿನಿಯೋಗಃ ।\\

ಓಂ ಧೂಂ ಬೀಜಂ ಮೇ ಶಿರಃ ಪಾತು ಧೂಂ ಲಲಾಟಂ ಸದಾವತು ।\\
ಧೂಮಾನೇತ್ರಯುಗಂ ಪಾತು ವತೀ ಕರ್ಣೌಸದಾವತು ॥೪॥

ದೀರ್ಘಾತೂದರಮಧ್ಯೇ ತು ನಾಭಿಂ ಮೇ ಮಲಿನಾಂಬರಾ ।\\
ಶೂರ್ಪಹಸ್ತಾ ಪಾತು ಗುಹ್ಯಂ ರೂಕ್ಷಾರಕ್ಷತು ಜಾನುನೀ ॥೫॥

ಮುಖಂ ಮೇ ಪಾತು ಭೀಮಾಖ್ಯಾ ಸ್ವಾಹಾ ರಕ್ಷತು ನಾಸಿಕಾಂ ।\\
ಸರ್ವಂ ವಿದ್ಯಾವತು ಕಷ್ಟಂ ವಿವರ್ಣಾ ಬಾಹುಯುಗ್ಮಕಂ ॥೬॥

ಚಂಚಲಾ ಹೃದಯಂ ಪಾತು ದುಷ್ಟಾ ಪಾರ್ಶ್ವಂ ಸದಾವತು ।\\
ಧೂತಹಸ್ತಾ ಸದಾ ಪಾತು ಪಾದೌ ಪಾತು ಭಯಾವಹಾ ॥೭॥

ಪ್ರವೃದ್ಧರೋಮಾ ತು ಭೃಶಂ ಕುಟಿಲಾ ಕುಟಿಲೇಕ್ಷಣಾ ।\\
ಕ್ಷೃತ್ಪಿಪಾಸಾರ್ದಿತಾ ದೇವೀ ಭಯದಾ ಕಲಹಪ್ರಿಯಾ ॥೮॥

ಸರ್ವಾಂಗಂ ಪಾತು ಮೇ ದೇವೀ ಸರ್ವಶತ್ರುವಿನಾಶಿನೀ ।\\
ಇತಿ ತೇ ಕವಚಂ ಪುಣ್ಯಂ ಕಥಿತಂ ಭುವಿ ದುರ್ಲಭಂ ॥೯॥

ನ ಪ್ರಕಾಶ್ಯಂ ನ ಪ್ರಕಾಶ್ಯಂ ನ ಪ್ರಕಾಶ್ಯಂ ಕಲೌ ಯುಗೇ ।\\
ಪಠನೀಯಂ ಮಹಾದೇವಿ ತ್ರಿಸಂಧ್ಯಂ ಧ್ಯಾನತತ್ಪರೈಃ ।\\
ದುಷ್ಟಾಭಿಚಾರೋ ದೇವೇಶಿ ತದ್ಗಾತ್ರಂ ನೈವ ಸಂಸ್ಪೃಶೇತ್ ॥೧೦॥

\authorline{ಇತಿ ಭೈರವೀ ಭೈರವ ಸಂವಾದೇ ಧೂಮಾವತೀ ತತ್ತ್ವೇ ಧೂಮಾವತೀ ಕವಚಂ ಸಂಪೂರ್ಣಂ }
%===============================================================


\section{ಆದಿತ್ಯಕವಚಂ}
\addcontentsline{toc}{section}{ಆದಿತ್ಯಕವಚಂ}

ಓಂ ಅಸ್ಯ ಶ್ರೀಮದಾದಿತ್ಯಕವಚಸ್ತೋತ್ರಮಹಾಮಂತ್ರಸ್ಯ ಯಾಜ್ಞವಲ್ಕ್ಯೋ ಮಹರ್ಷಿಃ । ಅನುಷ್ಟುಪ್ಜಗತೀಚ್ಛಂದಸೀ । ಘೃಣಿರಿತಿ ಬೀಜಂ । ಸೂರ್ಯ ಇತಿ ಶಕ್ತಿಃ । ಆದಿತ್ಯ ಇತಿ ಕೀಲಕಂ । ಶ್ರೀಸೂರ್ಯನಾರಾಯಣಪ್ರೀತ್ಯರ್ಥೇ ಜಪೇ ವಿನಿಯೋಗಃ ॥

\as{ಉದಯಾಚಲಮಾಗತ್ಯ ವೇದರೂಪಮನಾಮಯಂ ।\\
ತುಷ್ಟಾವ ಪರಯಾ ಭಕ್ತ್ಯಾ ವಾಲಖಿಲ್ಯಾದಿಭಿರ್ವೃತಂ ॥೧॥

ದೇವಾಸುರೈಸ್ಸದಾ ವಂದ್ಯಂ ಗ್ರಹೈಶ್ಚ ಪರಿವೇಷ್ಟಿತಂ ।\\
ಧ್ಯಾಯನ್ ಸ್ತುವನ್ ಪಠನ್ ನಾಮ ಯಸ್ಸೂರ್ಯಕವಚಂ ಸದಾ }॥೨॥

ಘೃಣಿಃ ಪಾತು ಶಿರೋದೇಶಂ ಸೂರ್ಯಃ ಫಾಲಂ ಚ ಪಾತು ಮೇ ।\\
ಆದಿತ್ಯೋ ಲೋಚನೇ ಪಾತು ಶ್ರುತೀ ಪಾತು ಪ್ರಭಾಕರಃ ॥೩॥

ಘ್ರಾಣಂ ಪಾತು ಸದಾ ಭಾನುಃ ಅರ್ಕಃ ಪಾತು ಮುಖಂ ತಥಾ ।\\
ಜಿಹ್ವಾಂ ಪಾತು ಜಗನ್ನಾಥಃ ಕಂಠಂ ಪಾತು ವಿಭಾವಸುಃ ॥೪॥

ಸ್ಕಂಧೌ ಗ್ರಹಪತಿಃ ಪಾತು ಭುಜೌ ಪಾತು ಪ್ರಭಾಕರಃ ।\\
ಅಹಸ್ಕರಃ ಪಾತು ಹಸ್ತೌ ಹೃದಯಂ ಪಾತು ಭಾನುಮಾನ್ ॥೫॥

ಮಧ್ಯಂ ಚ ಪಾತು ಸಪ್ತಾಶ್ವೋ ನಾಭಿಂ ಪಾತು ನಭೋಮಣಿಃ ।\\
ದ್ವಾದಶಾತ್ಮಾ ಕಟಿಂ ಪಾತು ಸವಿತಾ ಪಾತು ಸೃಕ್ಕಿಣೀ ॥೬॥

ಊರೂ ಪಾತು ಸುರಶ್ರೇಷ್ಠೋ ಜಾನುನೀ ಪಾತು ಭಾಸ್ಕರಃ ।\\
ಜಂಘೇ ಪಾತು ಚ ಮಾರ್ತಾಂಡೋ ಗಲಂ ಪಾತು ತ್ವಿಷಾಂಪತಿಃ ॥೭॥

ಪಾದೌ ಬ್ರಧ್ನಸ್ಸದಾ ಪಾತು ಮಿತ್ರೋಽಪಿ ಸಕಲಂ ವಪುಃ ।\\
ವೇದತ್ರಯಾತ್ಮಕ ಸ್ವಾಮಿನ್ ನಾರಾಯಣ ಜಗತ್ಪತೇ ।\\
ಅಯಾತಯಾಮಂ ತಂ ಕಂಚಿದ್ವೇದರೂಪಃ ಪ್ರಭಾಕರಃ ॥೮॥

ಸ್ತೋತ್ರೇಣಾನೇನ ಸಂತುಷ್ಟೋ ವಾಲಖಿಲ್ಯಾದಿಭಿರ್ವೃತಃ ।\\
ಸಾಕ್ಷಾದ್ವೇದಮಯೋ ದೇವೋ ರಥಾರೂಢಸ್ಸಮಾಗತಃ ॥೯॥

ತಂ ದೃಷ್ಟ್ವಾ ಸಹಸೋತ್ಥಾಯ ದಂಡವತ್ಪ್ರಣಮನ್ ಭುವಿ ।\\
ಕೃತಾಂಜಲಿಪುಟೋ ಭೂತ್ವಾ ಸೂರ್ಯಸ್ಯಾಗ್ರೇ ಸ್ಥಿತಸ್ತದಾ ॥೧೦॥

ವೇದಮೂರ್ತಿರ್ಮಹಾಭಾಗೋ ಜ್ಞಾನದೃಷ್ಟಿರ್ವಿಚಾರ್ಯ ಚ ।\\
ಬ್ರಹ್ಮಣಾ ಸ್ಥಾಪಿತಂ ಪೂರ್ವಂ ಯಾತಯಾಮವಿವರ್ಜಿತಂ ॥೧೧॥

ಸತ್ತ್ವಪ್ರಧಾನಂ ಶುಕ್ಲಾಖ್ಯಂ ವೇದರೂಪಮನಾಮಯಂ ।\\
ಶಬ್ದಬ್ರಹ್ಮಮಯಂ ವೇದಂ ಸತ್ಕರ್ಮಬ್ರಹ್ಮವಾಚಕಂ ॥೧೨॥

ಮುನಿಮಧ್ಯಾಪಯಾಮಾಸ ಪ್ರಥಮಂ ಸವಿತಾ ಸ್ವಯಂ ।\\
ತೇನ ಪ್ರಥಮದತ್ತೇನ ವೇದೇನ ಪರಮೇಶ್ವರಃ ॥೧೩॥

ಯಾಜ್ಞವಲ್ಕ್ಯೋ ಮುನಿಶ್ರೇಷ್ಠಃ ಕೃತಕೃತ್ಯೋಽಭವತ್ತದಾ ।\\
ಋಗಾದಿಸಕಲಾನ್ ವೇದಾನ್ ಜ್ಞಾತವಾನ್ ಸೂರ್ಯಸನ್ನಿಧೌ ॥೧೪॥

ಇದಂ ಪ್ರೋಕ್ತಂ ಮಹಾಪುಣ್ಯಂ ಪವಿತ್ರಂ ಪಾಪನಾಶನಂ ।\\
ಯಃ ಪಠೇಚ್ಛೃಣುಯಾದ್ವಾಪಿ ಸರ್ವಪಾಪೈಃ ಪ್ರಮುಚ್ಯತೇ ।\\
ವೇದಾರ್ಥಜ್ಞಾನಸಂಪನ್ನಸ್ಸೂರ್ಯಲೋಕಮಾವಪ್ನುಯಾತ್ ॥೧೫॥

\authorline{ಇತಿ ಸ್ಕಾಂದಪುರಾಣೇ ಗೌರೀಖಂಡೇ ಆದಿತ್ಯಕವಚಂ ಸಮಾಪ್ತಂ ।}
%=======================================================================

\section{ಶ್ರೀಕಮಲಾಕವಚಂ}
\addcontentsline{toc}{section}{ಶ್ರೀಕಮಲಾಕವಚಂ}

ಓಂ ಅಸ್ಯಾಶ್ಚತುರಕ್ಷರಾವಿಷ್ಣುವನಿತಾಯಾಃ ಕವಚಸ್ಯ ಶ್ರೀಭಗವಾನ್ ಶಿವ ಋಷೀಃ । ಅನುಷ್ಟುಪ್ಛಂದಃ । ವಾಗ್ಭವಾ ದೇವತಾ । ವಾಗ್ಭವಂ ಬೀಜಂ । ಲಜ್ಜಾ ಶಕ್ತಿಃ । ರಮಾ ಕೀಲಕಂ । ಕಾಮಬೀಜಾತ್ಮಕಂ ಕವಚಂ । ಮಮ ಸುಕವಿತ್ವಪಾಂಡಿತ್ಯಸಮೃದ್ಧಿಸಿದ್ಧಯೇ ಪಾಠೇ ವಿನಿಯೋಗಃ ।

ಐಂಕಾರೋ ಮಸ್ತಕೇ ಪಾತು ವಾಗ್ಭವಾಂ ಸರ್ವಸಿದ್ಧಿದಾ ।\\
ಹ್ರೀಂ ಪಾತು ಚಕ್ಷುಷೋರ್ಮಧ್ಯೇ ಚಕ್ಷುರ್ಯುಗ್ಮೇ ಚ ಶಾಂಕರೀ ॥೧॥

ಜಿಹ್ವಾಯಾಂ ಮುಖವೃತ್ತೇ ಚ ಕರ್ಣಯೋರ್ದಂತಯೋರ್ನಸಿ ।\\
ಓಷ್ಠಾಧಾರೇ ದಂತಪಂಕ್ತೌ ತಾಲುಮೂಲೇ ಹನೌ ಪುನಃ ॥೨॥

ಪಾತು ಮಾಂ ವಿಷ್ಣುವನಿತಾ ಲಕ್ಷ್ಮೀಃ ಶ್ರೀವರ್ಣರೂಪಿಣೀ ॥

ಕರ್ಣಯುಗ್ಮೇ ಭುಜದ್ವಂದ್ವೇ ಸ್ತನದ್ವಂದ್ವೇ ಚ ಪಾರ್ವತೀ ॥೩॥

ಹೃದಯೇ ಮಣಿಬಂಧೇ ಚ ಗ್ರೀವಾಯಾಂ ಪಾರ್ಶ್ವರ್ಯೋದ್ವಯೋಃ ।\\
ಪೃಷ್ಠದೇಶೇ ತಥಾ ಗುಹ್ಯೇ ವಾಮೇ ಚ ದಕ್ಷಿಣೇ ತಥಾ ॥೪॥

ಉಪಸ್ಥೇ ಚ ನಿತಂಬೇ ಚ ನಾಭೌ ಜಂಘಾದ್ವಯೇ ಪುನಃ ।\\
ಜಾನುಚಕ್ರೇ ಪದದ್ವಂದ್ವೇ ಘುಟಿಕೇಽಙ್ಗುಲಿಮೂಲಕೇ ॥೫॥

ಸ್ವಧಾ ತು ಪ್ರಾಣಶಕ್ತ್ಯಾಂ ವಾ ಸೀಮನ್ಯಾಂ ಮಸ್ತಕೇ ತಥಾ ।\\
ಸರ್ವಾಂಗೇ ಪಾತು ಕಾಮೇಶೀ ಮಹಾದೇವೀ ಸಮುನ್ನತಿಃ ॥೬॥

ಪುಷ್ಟಿಃ ಪಾತು ಮಹಾಮಾಯಾ ಉತ್ಕೃಷ್ಟಿಃ ಸರ್ವದಾಽವತು ।\\
ಋದ್ಧಿಃ ಪಾತು ಸದಾ ದೇವೀ ಸರ್ವತ್ರ ಶಂಭುವಲ್ಲಭಾ ॥೭॥

ವಾಗ್ಭವಾ ಸರ್ವದಾ ಪಾತು ಪಾತು ಮಾಂ ಹರಗೇಹಿನೀ ।\\
ರಮಾ ಪಾತು ಮಹಾದೇವೀ ಪಾತು ಮಾಯಾ ಸ್ವರಾಟ್ ಸ್ವಯಂ ॥೮॥

ಸರ್ವಾಂಗೇ ಪಾತು ಮಾಂ ಲಕ್ಷ್ಮೀರ್ವಿಷ್ಣುಮಾಯಾ ಸುರೇಶ್ವರೀ ।\\
ವಿಜಯಾ ಪಾತು ಭವನೇ ಜಯಾ ಪಾತು ಸದಾ ಮಮ ॥೯॥

ಶಿವದೂತೀ ಸದಾ ಪಾತು ಸುಂದರೀ ಪಾತು ಸರ್ವದಾ ।\\
ಭೈರವೀ ಪಾತು ಸರ್ವತ್ರ ಭೇರುಂಡಾ ಸರ್ವದಾಽವತು ॥೧೦॥

ತ್ವರಿತಾ ಪಾತು ಮಾಂ ನಿತ್ಯಮುಗ್ರತಾರಾ ಸದಾಽವತು ।\\
ಪಾತು ಮಾಂ ಕಾಲಿಕಾ ನಿತ್ಯಂ ಕಾಲರಾತ್ರಿಃ ಸದಾಽವತು ॥೧೧॥

ನವದುರ್ಗಾಃ ಸದಾ ಪಾತು ಕಾಮಾಖ್ಯಾ ಸರ್ವದಾಽವತು ।\\
ಯೋಗಿನ್ಯಃ ಸರ್ವದಾ ಪಾತು ಮುದ್ರಾಃ ಪಾತು ಸದಾ ಸಮ ॥೧೨॥

ಮಾತ್ರಾಃ ಪಾತು ಸದಾ ದೇವ್ಯಶ್ಚಕ್ರಸ್ಥಾ ಯೋಗಿನೀ ಗಣಾಃ ।\\
ಸರ್ವತ್ರ ಸರ್ವಕಾರ್ಯೇಷು ಸರ್ವಕರ್ಮಸು ಸರ್ವದಾ ॥೧೩॥

ಪಾತು ಮಾಂ ದೇವದೇವೀ ಚ ಲಕ್ಷ್ಮೀಃ ಸರ್ವಸಮೃದ್ಧಿದಾ ॥

\authorline{ಇತಿ ವಿಶ್ವಸಾರತಂತ್ರೇ ಶ್ರೀಕಮಲಾಕವಚಂ ಸಂಪೂರ್ಣಂ ॥}
%===================================================================

\section{ಶ್ರೀಕಾಮೇಶ್ವರಕವಚಂ}
\addcontentsline{toc}{section}{ಶ್ರೀಕಾಮೇಶ್ವರಕವಚಂ}

ಶ್ರೀದೇವ್ಯುವಾಚ ।\\
ಭಗವನ್ ಕರುಣಾಂಭೋಧೇ ಶಾಸ್ತ್ರಾಂಭೋನಿಧಿಪಾರಗ ।\\
ದಾಸೀ ಪರಮಭಕ್ತಾಸ್ಮಿ ವರಂ ದಾತುಮಿಹಾರ್ಹಸಿ ॥೧॥

ಶ್ರೀಭೈರವ ಉವಾಚ ।\\
ಕಥಯಸ್ವ ಮಹೇಶಾನಿ ಕಮಿತೋ ವರಮಿಚ್ಛಸಿ ।\\
ಯತ್ಕಿಂಚಿನ್ಮನಸೀಷ್ಟಂ ಸ್ಯಾತ್ತದ್ದಾತುಂ ತೇ ಕ್ಷಮೋಽಸ್ಮ್ಯಹಂ ॥೨॥

ಶ್ರೀದೇವ್ಯುವಾಚ ।\\
ಕಾಮೇಶ್ವರಸ್ಯ ದೇವೇಶ ಕವಚಂ ದೇವದುರ್ಲಭಂ ।\\
ಶೀಘ್ರಂ ಮೇ ದಯಯಾ ಬ್ರೂಹಿ ಯದ್ಯಹಂ ಪ್ರೇಯಸೀ ತವ ॥೩॥

ಶ್ರೀಭೈರವೌವಾಚ ।\\
ಶೃಣುಷ್ವ ಪರಮೇಶಾನಿ ಕವಚಂ ಮನ್ಮುಖೋದಿತಂ ।\\
ಮಹಾಕಾಮೇಶ್ವರಸ್ಯಾಸ್ಯ ನ ದೇಯಂ ಪರಮಾದ್ಭುತಂ ॥೪॥

ಯಂ ಧೃತ್ವಾ ಯಂ ಪಠಿತ್ವಾ ಚ ಯಂ ಶ್ರುತ್ವಾ ಕವಚೋತ್ತಮಂ ।\\
ತ್ರೈಲೋಕ್ಯಾಧಿಪತಿರ್ಭೂತ್ವಾ ಸುಖಿತೋಽಸ್ಮಿ ಮಹೇಶ್ವರಿ ॥೫॥

ತದೇವಂ ವರ್ಣಯಿಷ್ಯಾಮಿ ತವ ಪ್ರೀತ್ಯಾ ವರಾನನೇ ।\\
ರಹಸ್ಯಂ ಪರಮಂ ತತ್ತ್ವಂ ನ ದಾತವ್ಯಂ ದುರಾತ್ಮನೇ ॥೬॥

ಅಸ್ಯಶ್ರೀಕಾಮೇಶ್ವರಕವಚಮಂತ್ರಸ್ಯ ಶ್ರೀದಕ್ಷಿಣಾಮೂರ್ತಿ ಋಷಿಃ । ಪಂಕ್ತಿ ಛಂದಃ । ಶ್ರೀಕಾಮೇಶ್ವರೋ ದೇವತಾ । ಶ್ರೀಂ ಬೀಜಂ, ಹ್ರೀಂ ಶಕ್ತಿಃ, ಕ್ಲೀಂ ಕೀಲಕಂ । ಚತುರ್ವರ್ಗಫಲಪ್ರಾಪ್ತ್ಯರ್ಥಂ ಕವಚಪಾಠೇ ವಿನಿಯೋಗಃ ।\\
\as{ಉದ್ಯಚ್ಚಂದ್ರಸಮಾನದೀಪ್ತಿಮಮೃತಾನಂದೈಕಹೇತುಂ ಪರಂ\\
ಚಿದ್ರೂಪಂ ಭುವನೈಕಸೃಷ್ಟಿಪ್ರಲಯೋದ್ಭೂತೈಕರಕ್ಷಾಪರಂ ।\\
ಕೋಟೀ ಚಂದ್ರಗಲತ್ಸುಧಾದ್ಭುತತನುಂ ಹಾರಾದಿಭೂಷೋಜ್ಜ್ವಲಂ\\
ಶೂಲಂ ಖಡ್ಗವರೌ ತಥಾಭಯಕರಂ ಕಾಮೇಶ್ವರಂ ಭಾವಯೇ ॥}

ಐಂ ಕಂ ಕಾಮೇಶ್ವರಃ ಪಾತು ಕ್ಲೀಂ ಶಿವೋ ಮೇ ಲಲಾಟಕಂ ।\\
ಸೌಂ ಮೇ ಕರ್ಣೌ ಚ ಕಾಮೇಶಃ ತಾರಂ ಮೇ ಲೋಚನದ್ವಯಂ ॥೧॥

ಶ್ರೀಂ ಗಂ ಡೌಂ ಮೇ ಮಹಾಕಾಲೋ ಹ್ರೀಂ ನಾಸಾಂ ಚಂದ್ರಚೂಡಕಃ ।\\
ಹಂ ಮೇ ಓಷ್ಟೌ ತ್ರಿಪುರೇಶಃ ಸಂ ಮೇ ದಂತೌ ಚ ಭೈರವಃ ॥೨॥

ಕ್ಷಂ ಮೇ ಜಿಹ್ವಾಂ ಪಾತು ದೇವೋ ಮಹಾಕಾಮೇಶ್ವರಃ ಪರಃ ।\\
ಸಮ್ಮೋಹನಂ ಸದಾ ಪಾತು ಮಹಾವಟುಕಭೈರವಃ ॥೩॥

ಯಂ ಮೇ ಸ್ಕಂಧೌ ಸದಾ ಪಾತು ಮಹಾಮೃತ್ಯುಂಜಯಸ್ತಥಾ ।\\
ಊಂ ಮೇ ಭುಜೌ ಸದಾ ಪಾತು ಶ್ರೀಂ ದೇವಸ್ತ್ರಿಪುರಾಂತಕಃ ॥೪॥

ಸಂ ಮೇ ಹಸ್ತೌ ತ್ರಿಲೋಕೇಶಃ ಹಂ ನಖಾನ್ ಪಾತು ಸುಂದರಃ ।\\
ಕ್ಷಂ ದಕ್ಷಃ ಪಾತು ಶಾಮೇಶೋ ಮಂ ಕುಕ್ಷಿಂ ಕೃತ್ತಿಕಾ ವರಃ ॥೫॥

ಲಂ ಪಾರ್ಶ್ವೌ ಪಾತು ಮೇ ಬಾಲಾ ವಂ ನಾಭಿಂ ಬ್ರಹ್ಮಭೂಷಕಃ ।\\
ರಂ ವಸ್ತಿಂ ಪಾತು ಮೇ ದೇವೋ ಯಂ ಪೃಷ್ಠಂ ವಿಷ್ಣುರೂಪಕಃ ॥೬॥

ಊಂ ಶಿಶ್ನಂ ಪಾತು ವಿಶ್ವೇಶಃ ಅಮೃತೇಶ್ವರಭೈರವಃ ।\\
ಯಂ ಮೇಢ್ರೇ ಪಾತು ಮೇ ರುದ್ರೋ ರಂ ಗುಹ್ಯಂ ಗುಹ್ಯಕೇಶ್ವರಃ ॥೭॥

ಲಂ ಕಟಿಂ ಪಾತ ಮೇ ನಿತ್ಯಂ ಶಿವಃ ಪರಮಕಾರಣಃ ।\\
ವಂ ಊರೂ ತ್ರಿಪುರಾಧ್ಯಕ್ಷಃ ಲಂ ಜಾನೂ ದೇವಪಾಲಕಃ ॥೮॥

ಮಂ ಜಂಘೇ ಪಾತು ಮಹಿಮಾ ಯಂ ಗುಲ್ಫೌ ಸೂರ್ಯರೂಪಕಃ ।\\
ಶ್ರೀಂ ಜಠರಂ ಹರಃ ಪಾತು ಹ್ರೀಂ ಮನೋಹರಸುಂದರಃ ॥೯॥

ಓಂ ಬುದ್ಧಿಂ ಪಾತು ಮೇ ರುದ್ರಃ ಸೌಃ ರಸಂ ದಕ್ಷಜಾಪತಿಃ ।\\
ಕ್ಲೀಂ ರಕ್ತಂ ಪಾತು ಮೇ ದೇವೋ ಊರ್ಧ್ವಂ ಚ ವಹ್ನಿರೂಪಧೃಕ್ ॥೧೦॥

ಐಂ ಸೌಂ ಸಂ ಗುಹ್ಯಕಃ ಪಾತು ಮೇರುಃ ಪಶುಪತಿಸ್ತಥಾ ।\\
ಅಸ್ಥಿಂ ಮೇ ಗಿರಿಜಾನಾಥೋ ಮಜ್ಜಾಂ ಮೇ ನೀಲಕಂಠಕಃ ॥೧೧॥

ಶುಕ್ರಂ ಭೂತೇಶ್ವರಃ ಪಾತು ಮೂರ್ಧಾನಂ ಶಂಕರೋಽವತು ।\\
ಪಾದೌ ಶ್ರೀಂ ಸುಂದರೀಶಾನೋ ದೇವಃ ಕಾಲಾಂತಕಃ ಪ್ರಭುಃ ॥೧೨॥

ಪಾದಾದಿಶಿರಃಪರ್ಯಂತಮಘೋರಃ ಪಾತು ಸರ್ವದಾ ।\\
ಶಿರಸಃ ಪಾದಪರ್ಯಂತಂ ಸದ್ಯೋಜಾತೋ ಮಮಾವತು ॥೧೩॥

ಪ್ರೌಢರೂಪಃ ಸದಾ ಪಾತು ಸ್ವಾಹಾ ಮೇ ಸಕಲಂ ವಪುಃ ।\\
ಕಾಮೇಶ್ವರಃ ಪ್ರಭಾತೇಽವ್ಯಾನ್ಮಧ್ಯಾಹ್ನೇ ಕಾಲಭೈರವಃ ॥೧೪॥

ಸಾಯಂ ಪಾತು ಕುಲೇಶಾನೋ ನಿಶೀಥೇ ನಿಷ್ಕಲೇಶ್ವರಃ ।\\
ಅರ್ಧರಾತ್ರೇ ವಾಮದೇವೋ ನಿಶಾಂತೇ ಸುಮಹೋದಯಃ ॥೧೫॥

ಸರ್ವದಾ ಸರ್ವತಃ ಪಾತು ಐಂ ಕ್ಲೀಂ ಸೌಂ ಬೀಜರೂಪಧೃಕ್ ।\\
ಪೂರ್ವೇ ಮಾಮಸಿತಾಂಗೋಽವ್ಯಾತ್ ದಕ್ಷಿಣೇ ರುರುಭೈರವಃ ॥೧೬॥

ಪಶ್ಚಿಮೇ ಪಾತು ಚಂಡೇಶೋ ಹ್ಯುತ್ತರೇ ಕ್ರೋಧಭೈರವಃ ।\\
ಐಶಾನ್ಯಾಂ ಪಾತು ಚೋನ್ಮತ್ತಃ ಕಪಾಲೇಶಸ್ತಥಾಗ್ನಿಕೇ ॥೧೭॥

ನೈರೃತ್ಯಾಂ ಭೀಷಣಃ ಪಾತು ವಾಯೌ ಸಂಹಾರಕೋಽವತು ।\\
ಊರ್ಧ್ವಂ ಮೇ ಕಾಲರುದ್ರಶ್ಚ ಪಾತಾಲೇ ಪರಮೇಶ್ವರಃ ॥೧೮॥

ಮಧ್ಯೇ ಸದಾಶಿವಃ ಪಾತು ತ್ರಿಕೂಟೇಶಃ ಸದಾವತು ।\\
ದಶದಿಕ್ಷು ಸದಾ ಪಾತು ಮಹಾಕಾಮೇಶ್ವರೋಽವತು ॥೧೯॥

ವಿಸ್ಮಾರಿತಂ ಚ ಯತ್ಸ್ಥಾನಂ ಯತ್ಸ್ಥಾನಂ ನಾಮವರರ್ಜಿತಂ ।\\
ತತ್ಸರ್ವಂ ಪಾತು ಮೇ ದೇವಸ್ತ್ರಿಪುರಃಪತಿರೀಶ್ವರಃ ॥೨೦॥

ಗೃಹೇ ಶರ್ವಃ ಸದಾ ಪಾತು ಬಹಿಃ ಪಾಯಾದ್ವೃಷಧ್ವಜಃ ।\\
ಪಥಿ ತ್ರಿಪುರಸುಂದರ್ಯಃ ಪಾತು ಸರ್ವತ್ರ ಸರ್ವದಾ ॥೨೧॥

ರಣೇ ರಾಜಕುಲೇ ದುರ್ಗೇ ದುರ್ಭಿಕ್ಷೇ ಶತ್ರುಸಂಸದಿ ।\\
ದ್ಯೂತೇ ಮಾರೀಭಯೇ ರಾಷ್ಟ್ರೇ ವಿಪ್ಲವೇ ವಾದಿನಾಂ ಕುಲೇ ॥೨೨॥

ಕಾಮೇಶ್ವರಃ ಸದಾ ಪಾತು ಭಯಸ್ಥಾನೇ ಚ ಸರ್ವದಾ ।\\
ಇತ್ಯೇತತ್ಕವಚಂ ಗುಹ್ಯಂ ತ್ರಿಷು ಲೋಕೇಷು ದುರ್ಲಭಂ ॥೨೩॥

ಮೂಲಮಂತ್ರಮಯಂ ದಿವ್ಯಂ ತ್ರಿಲೋಕೀಸಾರಮುತ್ತಮಂ ।\\
ಅದಾತವ್ಯಮಭಕ್ತಾಯ ಕವಚಂ ಗುಹ್ಯಮೀಶ್ವರಿ ॥೨೪॥

ಅದೃಷ್ಟವ್ಯಮಶ್ರೋತವ್ಯಂ ದೀಕ್ಷಾಹೀನಾಯ ಮಂತ್ರಿಣೇ ।\\
ಪರದೀಕ್ಷಾಯ ಶಿಷ್ಯಾಯ ಪುತ್ರಾಯ ಶರಜನ್ಮನೇ ॥೨೫॥

ನ ದಾತವ್ಯಂ ನ ಶ್ರೋತವ್ಯಮಿತ್ಯಾಜ್ಞಾಂ ಮಾಮಕೀಂ ಶೃಣು ।\\
ಪರಂ ಶ್ರೀಮಹಿಮಾನಂ ಚ ಶೃಣು ಚಾಸ್ಯ ಸುವರ್ಮಣಃ ॥೨೬॥

ಅದೀಕ್ಷಿತೋ ಯದಾ ಮಂತ್ರೀ ವಿದ್ಯಾಂ ಗೃಧ್ನುಃ ಪಠೇದಿದಂ ।\\
ಸುಶಿಕ್ಷಿತ ಇತಿ ಜ್ಞೇಯೋ ಮಾಂತ್ರಿಕಃ ಸಾಧಕೋತ್ತಮಃ ॥೨೭॥

ಗುರುಪೂಜಾಂ ವಿಧಾಯಾಸ್ಯ ವಿಧಿವತ್ಪ್ರಪಠೇತ್ತತಃ ।\\
ಕವಚಂ ತ್ರಿಃಸಕೃದ್ವಾಪಿ ಯಾವಜ್ಜ್ಞಾನಂ ಚ ವಾ ಪುನಃ ॥೨೮॥

ಏತಚ್ಛತಾರ್ಥಮಾವರ್ತ್ಯ ತ್ರೈಲೋಕ್ಯವಿಜಯೀ ಭವೇತ್ ।\\
ಯಃ ಪಠೇನ್ಮನಸಾಂತಸ್ತು ರಾತ್ರೌ ಬ್ರಾಹ್ಮೇ ಮುಹೂರ್ತಕೇ ॥೨೯॥

ಪೂಜಾಕಾಲೇ ನಿಶೀಥೇ ತು ತಸ್ಯ ಹಸ್ತಸ್ಯ ಸಿದ್ಧಯಃ ।\\
ದುಃಸ್ವಪ್ನೇ ಬಂಧನೇ ಘೋರೇ ಕಾಂತಾರೇ ಸಾಗರೇ ಭಯೇ ॥೩೦॥

ರಜಃಸತ್ತ್ವತಮೋರೂಪವರ್ಮ ಕಾಮೇಶ್ವರಸ್ಯ ತು ।\\
ಕಾಂಕ್ಷಂತೇ ಮನಸಾ ಯದ್ಯತ್ತತ್ತತ್ಪ್ರಾಪ್ನೋತಿ ಸಾಧಕಃ ॥೩೧॥

ಮಹಾಕವಿರ್ಭವೇನ್ಮಾಸಾತ್ಸರ್ವಸಿದ್ಧೀಶ್ವರೋ ಭವೇತ್ ।\\
ಕುಂಕುಮೇನ ಲಿಖಿತ್ವಾ ವೈ ಭೂರ್ಜತ್ವಚಿ ರವೌ ದಿನೇ ॥೩೨॥

ಪ್ರಾತಃಕಾಲೇ ಶುಭರ್ಕ್ಷೇ ಚ ಸ್ವರ್ಣಸ್ಥಂ ಧಾರಯೇದ್ಯದಿ ।\\
ಶಿಖಾಯಾಂ ದಕ್ಷಿಣೇ ಬಾಹೌ ಕಂಠೇ ವಾ ಧಾರಯೇದ್ಬುಧಃ ।\\
ಯದ್ಯದಿಷ್ಟಂ ಭವೇತ್ತತ್ತತ್ಸಾಧಕೋ ಲಭತೇಽಚಿರಾತ್ ॥೩೩॥

ಯದ್ಗೃಹೇ ವರ್ತತೇ ವರ್ಮ ಶ್ರೀಕಾಮೇಶ್ವರದೈವತಂ ।\\
ವಿದ್ಯಾಕೀರ್ತಿರ್ಧನಾರೋಗ್ಯಂ ಲಕ್ಷ್ಮೀರ್ವೃದ್ಧಿರ್ನ ಸಂಶಯಃ ॥೩೪॥

ವಿನಾ ಬಲಿಂ ನ ಸಂರಕ್ಷ್ಯಂ ಕವಚಂ ಸಾಧಕತ್ತಮೈಃ ।\\
ಚೇದ್ರಕ್ಷ್ಯಂ ತತ್ಸುತಾನ್ ದಾರಾನ್ನಾಯುರ್ಭಕ್ಷತಿ ಯೋಗಿನೀ ॥೩೫॥

ಬಲಿಂ ದತ್ವಾ ಪಠೇದ್ವರ್ಮ ಧಾರಯೇನ್ಮೂರ್ಧ್ನಿ ಸಾಧಕಃ ।\\
ಪುತ್ರವಾನ್ ಧನವಾನ್ ಶ್ರೀಮಾನ್ ನಾನಾವಿದ್ಯಾನಿಧಿರ್ಭವೇತ್ ।\\
ಇಹ ಭೋಗಾನ್ ಸಮಶ್ನಾತಿ ಪರತ್ರ ಮುಕ್ತಿಭಾಗ್ಭವೇತ್ ॥೩೬॥

ಬ್ರಹ್ಮಾಸ್ತ್ರಾದೀನಿ ಶಸ್ತ್ರಾಣಿ ತದ್ಗಾತ್ರಸ್ಪರ್ಶನಾತ್ತತಃ ।\\
ನಾಶಮಾಯಾತಿ ಸರ್ವತ್ರ ಕವಚಸ್ಯಾಸ್ಯ ಧಾರಣಾತ್ ॥೩೭॥

ಮೃತವತ್ಸಾ ವಾಮಬಾಹೌ ಕವಚಸ್ಯಾಸ್ಯ ಧಾರಣಾತ್ ।\\
ಬಹ್ವಪತ್ಯಾ ಜೀವವತ್ಸಾ ಭವತ್ಯೇವ ನ ಸಂಶಯಃ ॥೩೮॥

ರಣೇಷ್ಟತ್ವಾಚರೇದ್ಯುದ್ಧಂ ಹತ್ವಾ ಶತ್ರೂನ್ ಜಯಂ ಲಭೇತ್ ।\\
ಜಯಂ ಕೃತ್ವಾ ಗೃಹಂ ದೇವಿ ಸ ಪ್ರಾಶ್ಯತಿ ಸುಖೀ ನರಃ ॥೩೯॥

ಮಹಾಭಯೇ ಮಹಾಘೋರೇ ಮಹಾಮಾರೀಭಯೇ ತಥಾ ।\\
ದುರ್ಭಿಕ್ಷೇ ಶತ್ರುಸಂಗ್ರಾಮೇ ಪಠೇತ್ಕವಚಮಾದರಾತ್ ।\\
ಸರ್ವಂ ತತ್ಪ್ರಶಮಂ ಯಾತಿ ಕಾಮೇಶ್ವರಪ್ರಸಾದತಃ ॥೪೦॥

ಕವಚಸ್ಯಾಸ್ಯ ದೇವೇಶಿ ವರ್ಣಿತುಂ ನೈವ ಶಕ್ಯತೇ ।\\
ಮಹಿಮಾನಂ ಮಹಾದೇವಿ ಜಿಹ್ವಾಕೋಟಿಶತೈರಪಿ ॥೪೧॥

ಇದಂ ಕವಚಮಜ್ಞಾತ್ವಾ ಯೋ ಜಪೇತ ಮನುಂ ಶಿವೇ ।\\
ಸಪ್ತಲಕ್ಷಪ್ರಜಪ್ತಾಪಿ ತಸ್ಯ ವಿದ್ಯಾ ನ ಸಿದ್ಧ್ಯತಿ ॥೪೨॥

ಅದಾತವ್ಯಮಿದಂ ವರ್ಮ ಮಂತ್ರಗರ್ಭಂ ರಹಸ್ಯಕಂ ।\\
ಅವಕ್ತವ್ಯಂ ಮಹಾಪುಣ್ಯಂ ಸರ್ವಸಾರಸ್ವತಪ್ರದಂ ॥೪೩॥

ಅದೀಕ್ಷಿತಾಯ ನೋ ದದ್ಯಾತ್ ಕುಚೈಲಾಯ ದುರಾತ್ಮನೇ ।\\
ನ ದೇಯಂ ಪರಶಿಷ್ಯೇಭ್ಯೋಽಭಕ್ತೇಭ್ಯೋಽಪಿ ವಿಶೇಷತಃ ॥೪೪॥

ಶಿಷ್ಯೇಭ್ಯೋ ಭಕ್ತಿಯುಕ್ತೇಭ್ಯಃ ಪ್ರದೇಯಂ ಭಾವನಾವಶಾತ್ ।\\
ಇತಿ ಶ್ರೀದೇವದೇವೇಶಿ ವರ್ಮ ಕಾಮೇಶ್ವರಸ್ಯ ತು ।\\
ಗುಹ್ಯಂ ಗೋಪ್ಯತಮಂ ಗೋಪ್ಯಂ ಕಥಿತಂ ತವ ಭಕ್ತಿತಃ ॥೪೫॥

ಇತ್ಯೇಷ ಪಟಲೋ ದೇವಿ ಪರಮಾದ್ಭುತಕಾರಣಃ ।\\
ಕಥಿತೋ ದೇವದೇವೇಶಿ ತವ ಭಕ್ತಿವಶಾನ್ಮಯಾ ॥೪೬॥

ನ ವಕ್ತವ್ಯೋ ನ ದಾತವ್ಯೋ ದೀಕ್ಷಾಹೀನಾಯ ಪಾರ್ವತಿ ।\\
ಯಥಾವದ್ಗೋಪನೀಯೋಽಸೌ ಸಿದ್ಧ್ಯಷ್ಟಕಫಲೋ ಮತಃ ॥೪೭॥

ಯಾವತ್ಕಾಲಂ ಭವೇದ್ಗುಪ್ತಸ್ತಾವತ್ತು ಫಲದಾಯಕಃ ।\\
ನ ಚೇದ್ಗುಪ್ತೋ ಭವೇದ್ದೇವಿ ಸಿದ್ಧಿಹಾನಿರ್ಭವೇದ್ಧ್ರುವಂ ॥೪೮॥

ರಹಸ್ಯೋದ್ಘಾಟನಾಯಾಸ್ಮಾನ್ಮಹಿಮಾ ಹೀಯತಾನ್ಮನೋಃ ।\\
ತಸ್ಮಾತ್ಸರ್ವತ್ರ ಗೋಪ್ತವ್ಯಃ ಸಾಯಕೈಃ ಪಟಲೋತ್ತಮಃ ।\\
ಯಥೇಷ್ಟಕಾಮದಶ್ಚೇಹ ಪರತ್ರ ಮೋಕ್ಷದಃ ಪ್ರಿಯೇ ॥೪೯॥
\authorline{ಇತಿ ಶ್ರೀವಿಶ್ವಸಾರತಂತ್ರೇ ಪಾರ್ವತೀಪರಮೇಶ್ವರಸಂವಾದೇ ಶ್ರೀಕಾಮೇಶ್ವರಕವಚಂ ಸಂಪೂರ್ಣಂ ॥}
%======================================================================


\section{ಶ್ರೀಕುಲಕುಂಡಲೀಕವಚಸ್ತೋತ್ರಂ}
\addcontentsline{toc}{section}{ಶ್ರೀಕುಲಕುಂಡಲೀಕವಚಸ್ತೋತ್ರಂ}

ಆನಂದಭೈರವೀ ಉವಾಚ ।\\
ಅಥ ವಕ್ಷ್ಯೇ ಮಹಾದೇವ ಕುಂಡಲೀಕವಚಂ ಶುಭಂ ।\\
ಪರಮಾನಂದದಂ ಸಿದ್ಧಂ ಸಿದ್ಧವೃಂದನಿಷೇವಿತಂ ॥೧॥

ಯತ್ಕೃತ್ವಾ ಯೋಗಿನಃ ಸರ್ವೇ ಧರ್ಮಾಧರ್ಮಪ್ರದರ್ಶಕಾಃ ।\\
ಜ್ಞಾನಿನೋ ಮಾನಿನೋ ಧರ್ಮಾನ್ ವಿಚರಂತಿ ಯಥಾಮರಾಃ ॥೨॥

ಸಿದ್ಧಯೋಽಪ್ಯಣಿಮಾದ್ಯಾಶ್ಚ ಕರಸ್ಥಾಃ ಸರ್ವದೇವತಾಃ ।\\
ಏತತ್ಕವಚಪಾಠೇನ ದೇವೇಂದ್ರೋ ಯೋಗಿರಾಡ್ಭವೇತ್ ॥೩॥

ಋಷಯೋ ಯೋಗಿನಃ ಸರ್ವೇ ಜಟಿಲಾಃ ಕುಲಭೈರವಾಃ ।\\
ಪ್ರಾತಃಕಾಲೇ ತ್ರಿವಾರಂ ಚ ಮಧ್ಯಾಹ್ನೇ ವಾರಯುಗ್ಮಕಂ ॥೪॥

ಸಾಯಾಹ್ನೇ ವಾರಮೇಕಂತು ಪಠೇತ್ಕವಚಮೇವ ಚ ।\\
ಪಠೇದೇವಂ ಮಹಾಯೋಗೀ ಕುಂಡಲೀದರ್ಶನಂ ಭವೇತ್ ॥೫॥

ಓಂ ಅಸ್ಯ ಶ್ರೀಕುಲಕುಂಡಲೀಕವಚಸ್ಯ ಬ್ರಹ್ಮೇಂದ್ರ ಋಷಿಃ ।\\
ಗಾಯತ್ರೀ ಛಂದಃ । ಕುಲಕುಂಡಲೀ ದೇವತಾ । ಸರ್ವಾಭೀಷ್ಟಸಿದ್ಧ್ಯರ್ಥೇ ವಿನಿಯೋಗಃ ।

ಓಂ ಈಶ್ವರೀ ಜಗತಾಂ ಧಾತ್ರೀ ಲಲಿತಾ ಸುಂದರೀ ಪರಾ ।\\
ಕುಂಡಲೀ ಕುಲರೂಪಾ ಚ ಪಾತು ಮಾಂ ಕುಲಚಂಡಿಕಾ ॥೬॥

ಶಿರೋ ಮೇ ಲಲಿತಾ ದೇವೀ ಪಾತೂಗ್ರಾಖ್ಯಾ ಕಪೋಲಕಂ ।\\
ಬ್ರಹ್ಮಮಂತ್ರೇಣ ಪುಟಿತಾ ಭ್ರೂಮಧ್ಯಂ ಪಾತು ಮೇ ಸದಾ ॥೭॥

ನೇತ್ರತ್ರಯಂ ಮಹಾಕಾಲೀ ಕಾಲಾಗ್ನಿಭಕ್ಷಿಕಾ ಶಿಖಾಂ ।\\
ದಂತಾವಲೀಂ ವಿಶಾಲಾಕ್ಷೀ ಓಷ್ಠಮಿಷ್ಟಾನುವಾಸಿನೀ ॥೮॥

ಕಾಮಬೀಜಾತ್ಮಿಕಾ ವಿದ್ಯಾ ಅಧರಂ ಪಾತು ಮೇ ಸದಾ ।\\
ಌಯುಗಸ್ಥಾ ಗಂಡಯುಗ್ಮಂ ಮಾಯಾ ವಿಶ್ವಾ ರಸಪ್ರಿಯಾ ॥೯॥

ಭುವನೇಶೀ ಕರ್ಣಯುಗ್ಮಂ ಚಿಬುಕಂ ಕ್ರೋಧಕಾಲಿಕಾ ।\\
ಕಪಿಲಾ ಮೇ ಗಲಂ ಪಾತು ಸರ್ವಬೀಜಸ್ವರೂಪಿಣೀ ॥೧೦॥

ಮಾತೃಕಾವರ್ಣಪುಟಿತಾ ಕುಂಡಲೀ ಕಂಠಮೇವ ಚ ।\\
ಹೃದಯಂ ಕಾಲಪೃಥ್ವೀ ಚ ಕಂಕಾಲೀ ಪಾತು ಮೇ ಮುಖಂ ॥೧೧॥

ಭುಜಯುಗ್ಮಂ ಚತುರ್ವರ್ಗಾ ಚಂಡದೋರ್ದ್ದಂಡಖಂಡಿನೀ ।\\
ಸ್ಕಂಧಯುಗ್ಮಂ ಸ್ಕಂದಮಾತಾ ಹಾಲಾಹಲಗತಾ ಮಮ ॥೧೨॥

ಅಂಗುಲ್ಯಗ್ರಂ ಕುಲಾನಂದಾ ಶ್ರೀವಿದ್ಯಾ ನಖಮಂಡಲಂ ।\\
ಕಾಲಿಕಾ ಭುವನೇಶಾನೀ ಪೃಷ್ಠದೇಶಂ ಸದಾವತು ॥೧೩॥

ಪಾರ್ಶ್ವಯುಗ್ಮಂ ಮಹಾವೀರಾ ವೀರಾಸನಧರಾಭಯಾ ।\\
ಪಾತು ಮಾಂ ಕುಲದರ್ಭಸ್ಥಾ ನಾಭಿಮುದರಮಂಬಿಕಾ ॥೧೪॥

ಕಟಿದೇಶಂ ಪೀಠಸಂಸ್ಥಾ ಮಹಾಮಹಿಷಘಾತಿನೀ ।\\
ಲಿಂಗಸ್ಥಾನಂ ಮಹಾಮುದ್ರಾ ಭಗಂ ಮಾಲಾಮನುಪ್ರಿಯಾ ॥೧೫॥

ಭಗೀರಥಪ್ರಿಯಾ ಧೂಮ್ರಾ ಮೂಲಾಧಾರಂ ಗಣೇಶ್ವರೀ ।\\
ಚತುರ್ದಲಂ ಕಕ್ಷ್ಯಪೂಜ್ಯಾ ದಲಾಗ್ರಂ ಮೇ ವಸುಂಧರಾ ॥೧೬॥

ಶೀರ್ಷಂ ರಾಧಾ ರಣಾಖ್ಯಾ ಚ ಬ್ರಹ್ಮಾಣೀ ಪಾತು ಮೇ ಮುಖಂ ।\\
ಮೇದಿನೀ ಪಾತು ಕಮಲಾ ವಾಗ್ದೇವೀ ಪೂರ್ವಗಂ ದಲಂ ॥೧೭॥

ಛೇದಿನೀ ದಕ್ಷಿಣೇ ಪಾತು ಪಾತು ಚಂಡಾ ಮಹಾತಪಾ ।\\
ಚಂದ್ರಘಂಟಾ ಸದಾ ಪಾತು ಯೋಗಿನೀ ವಾರುಣಂ ದಲಂ ॥೧೮॥

ಉತ್ತರಸ್ಥಂ ದಲಂ ಪಾತು ಪೃಥಿವೀಮಿಂದ್ರಪಾಲಿತಾ ।\\
ಚತುಷ್ಕೋಣಂ ಕಾಮವಿದ್ಯಾ ಬ್ರಹ್ಮವಿದ್ಯಾಬ್ಜಕೋಣಕಂ ॥೧೯॥

ಅಷ್ಟಶೂಲಂ ಸದಾ ಪಾತು ಸರ್ವವಾಹನವಾಹನಾ ।\\
ಚತುರ್ಭುಜಾ ಸದಾ ಪಾತು ಡಾಕಿನೀ ಕುಲಚಂಚಲಾ ॥೨೦॥

ಮೇಢ್ರಸ್ಥಾ ಮದನಾಧಾರಾ ಪಾತು ಮೇ ಚಾರುಪಂಕಜಂ ।\\
ಸ್ವಯಂಭೂಲಿಂಗ ಚಾರ್ವಾಕಾ ಕೋಟರಾಕ್ಷೀ ಮಮಾಸನಂ ॥೨೧॥

ಕದಂಬವನಮಾಪಾತು ಕದಂಬವನವಾಸಿನೀ ।\\
ವೈಷ್ಣವೀ ಪರಮಾ ಮಾಯಾ ಪಾತು ಮೇ ವೈಷ್ಣವಂ ಪದಂ ॥೨೨॥

ಷಡ್ದಲಂ ರಾಕಿಣೀ ಪಾತು ರಾಕಿಣೀ ಕಾಮವಾಸಿನೀ ।\\
ಕಾಮೇಶ್ವರೀ ಕಾಮರೂಪಾ ಶ್ರೀಕೃಷ್ಣಂ ಪೀತವಾಸಸಂ ॥೨೩॥

ವನಮಾಲಾ ವನದುರ್ಗಾ ಶಂಖಂ ಮೇ ಶಂಖಿನೀ ಶಿವಾ ।\\
ಚಕ್ರಂ ಚಕ್ರೇಶ್ವರೀ ಪಾತು ಕಮಲಾಕ್ಷೀ ಗದಾಂ ಮಮ ॥೨೪॥

ಪದ್ಮಂ ಮೇ ಪದ್ಮಗಂಧಾ ಚ ಪದ್ಮಮಾಲಾ ಮನೋಹರಾ ।\\
ರಾದಿಲಾಂತಾಕ್ಷರಂ ಪಾತು ಲಾಕಿನೀ ಲೋಕಪಾಲಿನೀ ॥೨೫॥\\(ಕಾದಿಲಾಂತಾಕ್ಷರಂ)\\
ಷಡ್ದಲೇ ಸ್ಥಿತದೇವಾಂಶ್ಚ ಪಾತು ಕೈಲಾಸವಾಸಿನೀ । (ಷಡ್ವರ್ಗಸ್ಥಿತದೇವೀಶ್ಚ)\\
ಅಗ್ನಿವರ್ಣಾ ಸದಾ ಪಾತು ಗಣಂ ಮೇ ಪರಮೇಶ್ವರೀ ॥೨೬॥

ಮಣಿಪೂರಂ ಸದಾ ಪಾತು ಮಣಿಮಾಲಾವಿಭೂಷಣಾ ।\\
ದಶಾಪತ್ರಂ ದಶವರ್ಣಂ ಡಾದಿಫಾಂತಂ ತ್ರಿವಿಕ್ರಮಾ ॥೨೭॥

ಪಾತು ನೀಲಾ ಮಹಾಕಾಲೀ ಭದ್ರಾ ಭೀಮಾ ಸರಸ್ವತೀ ।\\
ಅಯೋಧ್ಯಾವಾಸಿನೀ ದೇವೀ ಮಹಾಪೀಠನಿವಾಸಿನೀ ॥೨೮॥

ವಾಗ್ಭವಾದ್ಯಾ ಮಹಾವಿದ್ಯಾ ಕುಂಡಲೀ ಕಾಲಕುಂಡಲೀ ।\\
ದಶಚ್ಛದಗತಂ ಪಾತು ರುದ್ರಂ ರುದ್ರಾತ್ಮಕಂ ಮಮ ॥೨೯॥

ಸೂಕ್ಷ್ಮಾತ್ಸೂಕ್ಷ್ಮತರಾ ಪಾತು ಸೂಕ್ಷ್ಮಸ್ಥಾನನಿವಾಸಿನೀ ।\\
ರಾಕಿಣೀ ಲೋಕಜನನೀ ಪಾತು ಕೂಟಾಕ್ಷರಸ್ಥಿತಾ ॥೩೦॥

ತೈಜಸಂ ಪಾತು ನಿಯತಂ ರಜಕೀ ರಾಜಪೂಜಿತಾ ।\\
ವಿಜಯಾ ಕುಲಬೀಜಸ್ಥಾ ತವರ್ಗಂ ತಿಮಿರಾಪಹಾ ॥೩೧॥

ಮಂತ್ರಾತ್ಮಿಕಾ ಮಣಿಗ್ರಂಥಿಭೇದಿನೀ ಪಾತು ಸರ್ವದಾ ।\\
ಗರ್ಭದಾತಾ ಭೃಗುಸುತಾ ಪಾತು ಮಾಂ ನಾಭಿವಾಸಿನೀ ॥೩೨॥

ನಂದಿನೀ ಪಾತು ಸಕಲಂ ಕುಂಡಲೀ ಕಾಲಕಂಪಿತಾ ।\\
ಹೃತ್ಪದ್ಮಂ ಪಾತು ಕಾಲಾಖ್ಯಾ ಧೂಮ್ರವರ್ಣಾ ಮನೋಹರಾ ॥೩೩॥

ದಲದ್ವಾದಶವರ್ಣಂ ಚ ಭಾಸ್ಕರೀ ಭಾವಸಿದ್ಧಿದಾ ।\\
ಪಾತು ಮೇ ಪರಮಾ ವಿದ್ಯಾ ಕವರ್ಗಂ ಕಾಮಚಾರಿಣೀ ॥೩೪॥

ಚವರ್ಗಂ ಚಾರುವಸನಾ ವ್ಯಾಘ್ರಾಸ್ಯಾ ಟಂಕಧಾರಿಣೀ ।\\
ಚಕಾರಂ ಪಾತು ಕೃಷ್ಣಾಖ್ಯಾ ಕಾಕಿನೀಂ ಪಾತು ಕಾಲಿಕಾ ॥೩೫॥

ಟಕುರಾಂಗೀ ಟಕಾರಂ ಮೇ ಜೀವಭಾವಾ ಮಹೋದಯಾ ।\\
ಈಶ್ವರೀ ಪಾತು ವಿಮಲಾ ಮಮ ಹೃತ್ಪದ್ಮವಾಸಿನೀ ॥೩೬॥

ಕರ್ಣಿಕಾಂ ಕಾಲಸಂದರ್ಭಾ ಯೋಗಿನೀ ಯೋಗಮಾತರಂ ।\\
ಇಂದ್ರಾಣೀ ವಾರುಣೀ ಪಾತು ಕುಲಮಾಲಾ ಕುಲಾಂತರಂ ॥೩೭॥

ತಾರಿಣೀ ಶಕ್ತಿಮಾತಾ ಚ ಕಂಠವಾಕ್ಯಂ ಸದಾವತು ।\\
ವಿಪ್ರಚಿತ್ತಾ ಮಹೋಗ್ರೋಗ್ರಾ ಪ್ರಭಾ ದೀಪ್ತಾ ಘನಾಸನಾ ॥೩೮॥

ವಾಕ್ಸ್ತಂಭಿನೀ ವಜ್ರದೇಹಾ ವೈದೇಹೀ ವೃಷವಾಹಿನೀ ।\\
ಉನ್ಮತ್ತಾನಂದಚಿತ್ತಾ ಚ ಕ್ಷಣೋಶೀಶಾ ಭಗಾಂತರಾ ॥೩೯॥

ಮಮ ಷೋಡಶಪತ್ರಾಣಿ ಪಾತು ಮಾತೃತನುಸ್ಥಿತಾ ।\\
ಸುರಾನ್ ರಕ್ಷತು ವೇದಜ್ಞಾ ಸರ್ವಭಾಷಾ ಚ ಕರ್ಣಿಕಾಂ ॥೪೦॥

ಈಶ್ವರಾರ್ಧಾಸನಗತಾ ಪ್ರಪಾಯಾನ್ಮೇ ಸದಾಶಿವಂ ।\\
ಶಾಕಂಭರೀ ಮಹಾಮಾಯಾ ಸಾಕಿನೀ ಪಾತು ಸರ್ವದಾ ॥೪೧॥

ಭವಾನೀ ಭವಮಾತಾ ಚ ಪಾಯಾದ್ಭ್ರೂಮಧ್ಯಪಂಕಜಂ ।\\
ದ್ವಿದಲಂ ವ್ರತಕಾಮಾಖ್ಯಾ ಅಷ್ಟಾಂಗಸಿದ್ಧಿದಾಯಿನೀ ॥೪೨॥

ಪಾತು ನಾಸಾಮಖಿಲಾನಂದಾ ಮನೋರೂಪಾ ಜಗತ್ಪ್ರಿಯಾ ।\\
ಲಕಾರಂ ಲಕ್ಷಣಾಕ್ರಾಂತಾ ಸರ್ವಲಕ್ಷಣಲಕ್ಷಣಾ ॥೪೩॥

ಕೃಷ್ಣಾಜಿನಧರಾ ದೇವೀ ಕ್ಷಕಾರಂ ಪಾತು ಸರ್ವದಾ ।\\
ದ್ವಿದಲಸ್ಥಂ ಸರ್ವದೇವಂ ಸದಾ ಪಾತು ವರಾನನಾ ॥೪೪॥

ಬಹುರೂಪಾ ವಿಶ್ವರೂಪಾ ಹಾಕಿನೀ ಪಾತು ಸಂಸ್ಥಿತಾ ।\\
ಹರಾಪರಶಿವಂ ಪಾತು ಮಾನಸಂ ಪಾತು ಪಂಚಮೀ ॥೪೫॥

ಷಟ್ಚಕ್ರಸ್ಥಾ ಸದಾ ಪಾತು ಷಟ್ಚಕ್ರಕುಲವಾಸಿನೀ ।\\
ಅಕಾರಾದಿಕ್ಷಕಾರಾಂತಾ ಬಿಂದುಸರ್ಗಸಮನ್ವಿತಾ ॥೪೬॥

ಮಾತೃಕಾಣಾಂ ಸದಾ ಪಾತು ಕುಂಡಲೀ ಜ್ಞಾನಕುಂಡಲೀ ।\\
ದೇವಕಾಲೀ ಗತಿಪ್ರೇಮಾ ಪೂರ್ಣಾ ಗಿರಿತಟಂ ಶಿವಾ ॥೪೭॥

ಉಡ್ಡೀಯಾನೇಶ್ವರೀ ದೇವೀ ಸಕಲಂ ಪಾತು ಸರ್ವದಾ ।\\
ಕೈಲಾಸಪರ್ವತಂ ಪಾತು ಕೈಲಾಸಗಿರಿವಾಸಿನೀ ॥೪೮॥

ಪಾತು ಮೇ ಡಾಕಿನೀಶಕ್ತಿರ್ಲಾಕಿನೀ ರಾಕಿಣೀ ಕಲಾ ।\\
ಸಾಕಿನೀ ಹಾಕಿನೀ ದೇವೀ ಷಟ್ಚಕ್ರಾದೀನ್ ಪ್ರಪಾತು ಮೇ ॥೪೯॥

ಕೈಲಾಸಾಖ್ಯಂ ಸದಾ ಪಾತು ಪಂಚಾನನತನೂದ್ಭವಾ ।\\
ಹಿರಣ್ಯವರ್ಣಾ ರಜನೀ ಚಂದ್ರಸೂರ್ಯಾಗ್ನಿಭಕ್ಷಿಣೀ ॥೫೦॥

ಸಹಸ್ರದಲಪದ್ಮಂ ಮೇ ಸದಾ ಪಾತು ಕುಲಾಕುಲಾ ।\\
ಸಹಸ್ರದಲಪದ್ಮಸ್ಥಾ ದೈವತಂ ಪಾತು ಖೇಚರೀ ॥೫೧॥

ಕಾಲೀ ತಾರಾ ಷೋಡಶಾಖ್ಯಾ ಮಾತಂಗೀ ಪದ್ಮವಾಸಿನೀ ।\\
ಶಶಿಕೋಟಿಗಲದ್ರೂಪಾ ಪಾತು ಮೇ ಸಕಲಂ ತಮಃ ॥೫೨॥

ವನೇ ಘೋರೇ ಜಲೇ ದೇಶೇ ಯುದ್ಧೇ ವಾದೇ ಶ್ಮಶಾನಕೇ ।\\
ಸರ್ವತ್ರ ಗಮನೇ ಜ್ಞಾನೇ ಸದಾ ಮಾಂ ಪಾತು ಶೈಲಜಾ ॥೫೩॥

ಪರ್ವತೇ ವಿವಿಧಾಯಾಸೇ ವಿನಾಶೇ ಪಾತು ಕುಂಡಲೀ ।\\
ಪಾದಾದಿಬ್ರಹ್ಮರಂಧ್ರಾಂತಂ ಸರ್ವಾಕಾಶಂ ಸುರೇಶ್ವರೀ ॥೫೪॥

ಸದಾ ಪಾತು ಸರ್ವವಿದ್ಯಾ ಸರ್ವಜ್ಞಾನಂ ಸದಾ ಮಮ ।\\
ನವಲಕ್ಷಮಹಾವಿದ್ಯಾ ದಶದಿಕ್ಷು ಪ್ರಪಾತು ಮಾಂ ॥೫೫॥

ಇತ್ಯೇತತ್ಕವಚಂ ದೇವಿ ಕುಂಡಲಿನ್ಯಾಃ ಪ್ರಸಿದ್ಧಿದಂ ।\\
ಯೇ ಪಠಂತಿ ಧ್ಯಾನಯೋಗೇ ಯೋಗಮಾರ್ಗವ್ಯವಸ್ಥಿತಾಃ ॥೫೬॥

ತೇ ಯಾಂತಿ ಮುಕ್ತಿಪದವೀಮೈಹಿಕೇ ನಾತ್ರ ಸಂಶಯಃ ।\\
ಮೂಲಪದ್ಮೇ ಮನೋಯೋಗಂ ಕೃತ್ತ್ವಾ ಹೃದಾಸನಸ್ಥಿತಃ ॥೫೭॥

ಮಂತ್ರಂ ಧ್ಯಾಯೇತ್ಕುಂಡಲಿನೀಂ ಮೂಲಪದ್ಮಪ್ರಕಾಶಿನೀಂ ।\\
ಧರ್ಯೋದಯಾಂ ದಯಾರುಢಾಮಾಕಾಶಸ್ಥಾನವಾಸಿನೀಂ ॥೫೮॥

ಅಮೃತಾನಂದರಸಿಕಾಂ ವಿಕಲಾಂ ಸುಕಲಾಂ ಶಿತಾಂ ।\\
ಅಜಿತಾಂ ರಕ್ತರಹಿತಾಂ ವಿಶಕ್ತಾಂ ರಕ್ತವಿಗ್ರಹಾಂ ॥೫೯॥

ರಕ್ತನೇತ್ರಾಂ ಕುಲಕ್ಷಿಪ್ತಾಂ ಜ್ಞಾನಾಂಜನಜಯೋಜ್ಜ್ವಲಾಂ ।\\
ವಿಶ್ವಾಕಾರಾಂ ಮನೋರೂಪಾಂ ಮೂಲೇ ಧ್ಯಾತ್ತ್ವಾ ಪ್ರಪೂಜಯೇತ್ ॥೬೦॥

ಯೋ ಯೋಗೀ ಕುರುತೇ ಏವಂ ಸ ಸಿದ್ಧೋ ನಾತ್ರ ಸಂಶಯಃ ।\\
ರೋಗೀ ರೋಗಾತ್ಪ್ರಮುಚ್ಯೇತ ಬದ್ಧೋ ಮುಚ್ಯೇತ ಬಂಧನಾತ್ ॥೬೧॥

ರಾಜ್ಯಂ ಶ್ರಿಯಮವಾಪ್ನೋತಿ ರಾಜ್ಯಹೀನಃ ಪಠೇದ್ಯದಿ ।\\
ಪುತ್ರಹೀನೋ ಲಭೇತ್ಪುತ್ರಂ ಯೋಗಹೀನೋ ಭವೇದ್ವಶೀ ॥೬೨॥

ಕವಚಂ ಧಾರಯೇದ್ಯಸ್ತು ಶಿಖಾಯಾಂ ದಕ್ಷಿಣೇ ಭುಜೇ ।\\
ವಾಮಾ ವಾಮಕರೇ ಧೃತ್ತ್ವಾ ಸರ್ವಾಭೀಷ್ಟಮವಾಪ್ನುಯಾತ್ ॥೬೩॥

ಸ್ವರ್ಣೇ ರೌಪ್ಯೇ ತಥಾ ತಾಮ್ರೇ ಸ್ಥಾಪಯಿತ್ತ್ವಾ ಪ್ರಪೂಜಯೇತ್ ।\\
ಸರ್ವದೇಶೇ ಸರ್ವಕಾಲೇ ಪಠಿತ್ವಾ ಸಿದ್ಧಿಮಾಪ್ನುಯಾತ್ ॥೬೪॥

ಸ ಭೂಯಾತ್ಕುಂಡಲೀಪುತ್ರೋ ನಾತ್ರ ಕಾರ್ಯಾ ವಿಚಾರಣಾ ॥೬೫॥

\authorline{॥ಇತಿ ಶ್ರೀರುದ್ರಯಾಮಲೇ ಉತ್ತರತಂತ್ರೇ ಮಹಾತಂತ್ರೋದ್ದೀಪನೇ ಸಿದ್ಧಿವಿದ್ಯಾಪ್ರಕರಣೇ ಷಟ್ಚಕ್ರಪ್ರಕಾಶೇ ಭೈರವೀಭೈರವಸಂವಾದೇ ಕಂದವಾಸಿನೀಕವಚಂ ಅಥವಾ ಕುಲಕುಂಡಲಿನೀಕವಚಂ ಸಂಪೂರ್ಣಂ ॥}

%===================================================================================
\section{ಶ್ರೀಕಾಲಿಕಾಕವಚಂ}
\addcontentsline{toc}{section}{ಶ್ರೀಕಾಲಿಕಾಕವಚಂ}
ಶ್ರೀಸದಾಶಿವ ಉವಾಚ ।\\
ಕಥಿತಂ ಪರಮಂ ಬ್ರಹ್ಮ ಪ್ರಕೃತೇಃ ಸ್ತವನಂ ಮಹತ್ ।\\
ಆದ್ಯಾಯಾಃ ಶ್ರೀಕಾಲಿಕಾಯಾಃ ಕವಚಂ ಶೃಣು ಸಾಂಪ್ರತಂ ॥೧॥

ತ್ರೈಲೋಕ್ಯವಿಜಯಸ್ಯಾಸ್ಯ ಕವಚಸ್ಯ ಋಷಿಃ ಶಿವಃ ।\\
ಛಂದೋಽನುಷ್ಟುಬ್ದೇವತಾ ಚ ಆದ್ಯಾ ಕಾಲೀ ಪ್ರಕೀರ್ತಿತಾ ॥೨॥

ಮಾಯಾಬೀಜಂ ಬೀಜಮಿತಿ ರಮಾ ಶಕ್ತಿರುದಾಹೃತಾ ।\\
ಕ್ರೀಂ ಕೀಲಕಂ ಕಾಮ್ಯಸಿದ್ಧೌ ವಿನಿಯೋಗಃ ಪ್ರಕೀರ್ತಿತಃ ॥೩॥

ಹ್ರೀಮಾದ್ಯಾ ಮೇ ಶಿರಃ ಪಾತು ಶ್ರೀಂ ಕಾಲೀ ವದನಂ ಮಮ ।\\
ಹೃದಯಂ ಕ್ರೀಂ ಪರಾ ಶಕ್ತಿ ಪಾಯಾತ್ಕಂಠಂ ಪರಾತ್ಪರಾ ॥೪॥

ನೇತ್ರೇ ಪಾತು ಜಗದ್ಧಾತ್ರೀ ಕರ್ಣೌ ರಕ್ಷತು ಶಂಕರೀ ।\\
ಘ್ರಾಣಂ ಪಾತು ಮಹಾಮಾಯಾ ರಸನಾಂ ಸರ್ವಮಂಗಲಾ ॥೫॥

ದಂತಾನ್ ರಕ್ಷತು ಕೌಮಾರೀ ಕಪೋಲೌ ಕಮಲಾಲಯಾ ।\\
ಓಷ್ಠಾಧರೌ ಕ್ಷಮಾ ರಕ್ಷೇಚ್ಚಿಬುಕಂ ಚಾರುಹಾಸಿನೀ ॥೬॥

ಗ್ರೀವಾಂ ಪಾಯಾತ್ಕುಲೇಶಾನೀ ಕಕುತ್ಪಾತು ಕೃಪಾಮಯೀ ।\\
ದ್ವೌ ಬಾಹೂ ಬಾಹುದಾ ರಕ್ಷೇತ್ಕರೌ ಕೈವಲ್ಯದಾಯಿನೀ ॥೭॥

ಸ್ಕಂಧೌ ಕಪರ್ದಿನೀ ಪಾತು ಪೃಷ್ಠಂ ತ್ರೈಲೋಕ್ಯತಾರಿಣೀ ।\\
ಪಾರ್ಶ್ವೇ ಪಾಯಾದಪರ್ಣಾ ಮೇ ಕಟಿಂ ಮೇ ಕಮಠಾಸನಾ ॥೮॥

ನಾಭೌ ಪಾತು ವಿಶಾಲಾಕ್ಷೀ ಪ್ರಜಸ್ಥಾನಂ ಪ್ರಭಾವತೀ ।\\
ಊರೂ ರಕ್ಷತು ಕಲ್ಯಾಣೀ ಪಾದೌ ಮೇ ಪಾತು ಪಾರ್ವತೀ ॥೯॥

ಜಯದುರ್ಗಾವತು ಪ್ರಾಣಾನ್ಸರ್ವಾಂಗಂ ಸರ್ವಸಿದ್ಧಿದಾ ।\\
ರಕ್ಷಾಹೀನಂ ತು ಯತ್ಸ್ಥಾನಂ ವರ್ಜಿತಂ ಕವಚೇನ ಚ ॥೧೦॥

ತತ್ಸರ್ವಂ ಮೇ ಸದಾ ರಕ್ಷೇದಾದ್ಯಾ ಕಾಲೀ ಸನಾತನೀ ।\\
ಇತಿ ತೇ ಕಥಿತಂ ದಿವ್ಯಂ ತ್ರೈಲೋಕ್ಯವಿಜಯಾಭಿಧಂ ॥೧೧॥

ಕವಚಂ ಕಾಲಿಕಾದೇವ್ಯಾ ಆದ್ಯಾಯಾಃ ಪರಮಾದ್ಭುತಂ ।\\
ಪೂಜಾಕಾಲೇ ಪಠೇದ್ಯಸ್ತು ಆದ್ಯಾಧಿಕೃತಮಾನಸಃ ॥೧೨॥
\authorline{ಇತಿ ಮಹಾನಿರ್ವಾಣತಂತ್ರೇ ಶ್ರೀಕಾಲಿಕಾಕವಚಂ ಸಂಪೂರ್ಣಂ ।}
%====================================================================

\section{ಗರುಡಕವಚಂ}
\addcontentsline{toc}{section}{ಗರುಡಕವಚಂ}

ಅಸ್ಯ ಶ್ರೀಗರುಡಕವಚಸ್ತೋತ್ರಮಂತ್ರಸ್ಯ ನಾರದ ಭಗವಾನ್ ಋಷಿಃ ವೈನತೇಯೋ ದೇವತಾ ಅನುಷ್ಟುಪ್ಛಂದಃ ಶ್ರೀವೈನತೇಯಪ್ರೀತ್ಯರ್ಥೇ ಜಪೇ ವಿನಿಯೋಗಃ ।\\

ಓಂ ಶಿರೋ ಮೇ ಗರುಡಃ ಪಾತು ಲಲಾಟೇ ವಿನಿತಾಸುತಃ ।\\
ನೇತ್ರೇ ತು ಸರ್ಪಹಾ ಪಾತು ಕರ್ಣೌ ಪಾತು ಸುರಾಹತಃ ॥೧॥

ನಾಸಿಕಾಂ ಪಾತು ಸರ್ಪಾರಿಃ ವದನಂ ವಿಷ್ಣುವಾಹನಃ ।\\
ಸೂರ್ಯೇತಾಲೂ ಚ ಕಂಠೇ ಚ ಭುಜೌ ಪಾತು ಮಹಾಬಲಃ ॥೨॥

ಹಸ್ತೌ ಖಗೇಶ್ವರಃ ಪಾತು ಕರಾಗ್ರೇ ತರುಣಾಕೃತಿಃ ॥೩॥

ಸ್ತನೌ ಮೇ ವಿಹಗಃ ಪಾತು ಹೃದಯಂ ಪಾತು ಸರ್ಪಹಾ ।\\
ನಾಭಿಂ ಪಾತು ಮಹಾತೇಜಾಃ ಕಟಿಂ ಮೇ ಪಾತು ವಾಯುನಃ ॥೪॥

ಊರೂ ಮೇ ಪಾತು ಉರಗಿರಿಃ ಗುಲ್ಫೌ ವಿಷ್ಣುರಥಃ ಸದಾ ।\\
ಪಾದೌ ಮೇ ತಕ್ಷಕಃ ಸಿದ್ಧಃ ಪಾತು ಪಾದಾಂಗುಲೀಂಸ್ತಥಾ ॥೫॥

ರೋಮಕೂಪಾನಿ ಮೇ ವೀರೋ ತ್ವಚಂ ಪಾತು ಭಯಾಪಹಾ ।\\
ಇತ್ಯೇವಂ ಕವಚಂ ದಿವ್ಯಂ ಪಾಪಘ್ನಂ ಸರ್ವಕಾಮದಂ ॥೬॥

ಯಃ ಪಠೇತ್ಪ್ರಾತರುತ್ಥಾಯ ವಿಷದೋಷಂ ನ ಪಶ್ಯತಿ ।\\
ತ್ರಿಸಂಧ್ಯಂ ಪಠತೇ ನಿತ್ಯಂ ಬಂಧನಾತ್ ಮುಚ್ಯತೇ ನರಃ ।\\
ದ್ವಾದಶಾಹಂ ಪಠೇದ್ಯಸ್ತು ಮುಚ್ಯತೇ ಸರ್ವಕಿಲ್ವಿಷೈಃ ॥೭॥

\authorline{॥ಇತಿ ಶ್ರೀನಾರದಗರುಡಸಂವಾದೇ ಗರುಡಕವಚಂ ಸಂಪೂರ್ಣಂ ॥}

%=======================================================================
\section{ಶ್ರೀಛಿನ್ನಮಸ್ತಾಕವಚಂ}
\addcontentsline{toc}{section}{ಶ್ರೀಛಿನ್ನಮಸ್ತಾಕವಚಂ}
ದೇವ್ಯುವಾಚ ।\\
ಕಥಿತಾಚ್ಛಿನ್ನಮಸ್ತಾಯಾ ಯಾ ಯಾ ವಿದ್ಯಾ ಸುಗೋಪಿತಾಃ ।\\
ತ್ವಯಾ ನಾಥೇನ ಜೀವೇಶ ಶ್ರುತಾಶ್ಚಾಧಿಗತಾ ಮಯಾ ॥೧॥

ಇದಾನೀಂ ಶ್ರೋತುಮಿಚ್ಛಾಮಿ ಕವಚಂ ಸರ್ವಸೂಚಿತಂ ।\\
ತ್ರೈಲೋಕ್ಯವಿಜಯಂ ನಾಮ ಕೃಪಯಾ ಕಥ್ಯತಾಂ ಪ್ರಭೋ ॥೨॥

ಭೈರವ ಉವಾಚ ।\\
ಶ್ರುಣು ವಕ್ಷ್ಯಾಮಿ ದೇವೇಶಿ ಸರ್ವದೇವನಮಸ್ಕೃತೇ ।\\
ತ್ರೈಲೋಕ್ಯವಿಜಯಂ ನಾಮ ಕವಚಂ ಸರ್ವಮೋಹನಂ ॥೩॥

ಸರ್ವವಿದ್ಯಾಮಯಂ ಸಾಕ್ಷಾತ್ಸುರಾತ್ಸುರಜಯಪ್ರದಂ ।\\
ಧಾರಣಾತ್ಪಠನಾದೀಶಸ್ತ್ರೈಲೋಕ್ಯವಿಜಯೀ ವಿಭುಃ ॥೪॥

ಬ್ರಹ್ಮಾ ನಾರಾಯಣೋ ರುದ್ರೋ ಧಾರಣಾತ್ಪಠನಾದ್ಯತಃ ।\\
ಕರ್ತಾ ಪಾತಾ ಚ ಸಂಹರ್ತಾ ಭುವನಾನಾಂ ಸುರೇಶ್ವರಿ ॥೫॥

ನ ದೇಯಂ ಪರಶಿಷ್ಯೇಭ್ಯೋಽಭಕ್ತೇಭ್ಯೋಽಪಿ ವಿಶೇಷತಃ ।\\
ದೇಯಂ ಶಿಷ್ಯಾಯ ಭಕ್ತಾಯ ಪ್ರಾಣೇಭ್ಯೋಽಪ್ಯಧಿಕಾಯ ಚ ॥೬॥

ದೇವ್ಯಾಶ್ಚ ಚ್ಛಿನ್ನಮಸ್ತಾಯಾಃ ಕವಚಸ್ಯ ಚ ಭೈರವಃ ।\\
ಋಷಿಸ್ತು ಸ್ಯಾದ್ವಿರಾಟ್ ಛಂದೋ ದೇವತಾ ಚ್ಛಿನ್ನಮಸ್ತಕಾ ॥೭॥

ತ್ರೈಲೋಕ್ಯವಿಜಯೇ ಮುಕ್ತೌ ವಿನಿಯೋಗಃ ಪ್ರಕೀರ್ತಿತಃ ।\\
ಹುಂಕಾರೋ ಮೇ ಶಿರಃ ಪಾತು ಛಿನ್ನಮಸ್ತಾ ಬಲಪ್ರದಾ ॥೮॥

ಹ್ರಾಂ ಹ್ರೂಂ ಐಂ ತ್ರ್ಯಕ್ಷರೀ ಪಾತು ಭಾಲಂ ವಕ್ತ್ರಂ ದಿಗಂಬರಾ ।\\
ಶ್ರೀಂ ಹ್ರೀಂ ಹ್ರೂಂ ಐಂ ದೃಶೌ ಪಾತು ಮುಂಡಂ ಕರ್ತ್ರಿಧರಾಪಿ ಸಾ ॥೯॥

ಸಾ ವಿದ್ಯಾ ಪ್ರಣವಾದ್ಯಂತಾ ಶ್ರುತಿಯುಗ್ಮಂ ಸದಾಽವತು ।\\
ವಜ್ರವೈರೋಚನೀಯೇ ಹುಂ ಫಟ್ ಸ್ವಾಹಾ ಚ ಧ್ರುವಾದಿಕಾ ॥೧೦॥

ಘ್ರಾಣಂ ಪಾತು ಚ್ಛಿನ್ನಮಸ್ತಾ ಮುಂಡಕರ್ತ್ರಿವಿಧಾರಿಣೀ ।\\
ಶ್ರೀಮಾಯಾಕೂರ್ಚವಾಗ್ಬೀಜೈರ್ವಜ್ರವೈರೋಚನೀಯಹ್ರೂಂ ॥೧೧॥

ಹೂಂ ಫಟ್ ಸ್ವಾಹಾ ಮಹಾವಿದ್ಯಾ ಷೋಡಶೀ ಬ್ರಹ್ಮರೂಪಿಣೀ ।\\
ಸ್ವಪಾರ್ಶ್ರ್ವೇ ವರ್ಣಿನೀ ಚಾಸೃಗ್ಧಾರಾಂ ಪಾಯಯತೀ ಮುದಾ ॥೧೨॥

ವದನಂ ಸರ್ವದಾ ಪಾತು ಚ್ಛಿನ್ನಮಸ್ತಾ ಸ್ವಶಕ್ತಿಕಾ ।\\
ಮುಂಡಕರ್ತ್ರಿಧರಾ ರಕ್ತಾ ಸಾಧಕಾಭೀಷ್ಟದಾಯಿನೀ ॥೧೩॥

ವರ್ಣಿನೀ ಡಾಕಿನೀಯುಕ್ತಾ ಸಾಪಿ ಮಾಮಭಿತೋಽವತು ।\\
ರಾಮಾದ್ಯಾ ಪಾತು ಜಿಹ್ವಾಂ ಚ ಲಜ್ಜಾದ್ಯಾ ಪಾತು ಕಂಠಕಂ ॥೧೪॥

ಕೂರ್ಚಾದ್ಯಾ ಹೃದಯಂ ಪಾತು ವಾಗಾದ್ಯಾ ಸ್ತನಯುಗ್ಮಕಂ ।\\
ರಮಯಾ ಪುಟಿತಾ ವಿದ್ಯಾ ಪಾರ್ಶ್ವೌ ಪಾತು ಸುರೇಶ್ರ್ವರೀ ॥೧೫॥

ಮಾಯಯಾ ಪುಟಿತಾ ಪಾತು ನಾಭಿದೇಶೇ ದಿಗಂಬರಾ ।\\
ಕೂರ್ಚೇಣ ಪುಟಿತಾ ದೇವೀ ಪೃಷ್ಠದೇಶೇ ಸದಾಽವತು ॥೧೬॥

ವಾಗ್ಬೀಜಪುಟಿತಾ ಚೈಷಾ ಮಧ್ಯಂ ಪಾತು ಸಶಕ್ತಿಕಾ ।\\
ಈಶ್ವರೀ ಕೂರ್ಚವಾಗ್ಬೀಜೈರ್ವಜ್ರವೈರೋಚನೀಯಹ್ರೂಂ ॥೧೭॥

ಹೂಂಫಟ್ ಸ್ವಾಹಾ ಮಹಾವಿದ್ಯಾ ಕೋಟಿಸೂರ್ಯಸಮಪ್ರಭಾ ।\\
ಛಿನ್ನಮಸ್ತಾ ಸದಾ ಪಾಯಾದುರುಯುಗ್ಮಂ ಸಶಕ್ತಿಕಾ ॥೧೮॥

ಹ್ರೀಂ ಹ್ರೂಂ ವರ್ಣಿನೀ ಜಾನುಂ ಶ್ರೀಂ ಹ್ರೀಂ ಚ ಡಾಕಿನೀ ಪದಂ ।\\
ಸರ್ವವಿದ್ಯಾಸ್ಥಿತಾ ನಿತ್ಯಾ ಸರ್ವಾಂಗಂ ಮೇ ಸದಾಽವತು ॥೧೯॥

ಪ್ರಾಚ್ಯಾಂ ಪಾಯಾದೇಕಲಿಂಗಾ ಯೋಗಿನೀ ಪಾವಕೇಽವತು ।\\
ಡಾಕಿನೀ ದಕ್ಷಿಣೇ ಪಾತು ಶ್ರೀಮಹಾಭೈರವೀ ಚ ಮಾಂ ॥೨೦॥

ನೈರೃತ್ಯಾಂ ಸತತಂ ಪಾತು ಭೈರವೀ ಪಶ್ಚಿಮೇಽವತು ।\\
ಇಂದ್ರಾಕ್ಷೀ ಪಾತು ವಾಯವ್ಯೇಽಸಿತಾಂಗೀ ಪಾತು ಚೋತ್ತರೇ ॥೨೧॥

ಸಂಹಾರಿಣೀ ಸದಾ ಪಾತು ಶಿವಕೋಣೇ ಸಕರ್ತ್ರಿಕಾ ।\\
ಇತ್ಯಷ್ಟಶಕ್ತಯಃ ಪಾಂತು ದಿಗ್ವಿದಿಕ್ಷು ಸಕರ್ತ್ರಿಕಾಃ ॥೨೨॥

ಕ್ರೀಂ ಕ್ರೀಂ ಕ್ರೀಂ ಪಾತು ಸಾ ಪೂರ್ವಂ ಹ್ರೀಂ ಹ್ರೀಂ ಮಾಂ ಪಾತು ಪಾವಕೇ ।\\
ಹ್ರೂಂ ಹ್ರೂಂ ಮಾಂ ದಕ್ಷಿಣೇ ಪಾತು ದಕ್ಷಿಣೇ ಕಾಲಿಕಾಽವತು ॥೨೩॥

ಕ್ರೀಂ ಕ್ರೀಂ ಕ್ರೀಂ ಚೈವ ನೈರೃತ್ಯಾಂ ಹ್ರೀಂ ಹ್ರೀಂ ಚ ಪಶ್ಚಿಮೇಽವತು ।\\
ಹ್ರೂಂ ಹ್ರೂಂ ಪಾತು ಮರುತ್ಕೋಣೇ ಸ್ವಾಹಾ ಪಾತು ಸದೋತ್ತರೇ ॥೨೪॥

ಮಹಾಕಾಲೀ ಖಡ್ಗಹಸ್ತಾ ರಕ್ಷಃಕೋಣೇ ಸದಾಽವತು ।\\
ತಾರೋ ಮಾಯಾ ವಧೂಃ ಕೂರ್ಚಂ ಫಟ್ಕಾರೋಽಯಂ ಮಹಾಮನುಃ ॥೨೫॥

ಖಡ್ಗಕರ್ತ್ರಿಧರಾ ತಾರಾ ಚೋರ್ಧ್ವದೇಶಂ ಸದಾಽವತು ।\\
ಹ್ರೀಂ ಸ್ತ್ರೀಂ ಹೂಂ ಫಟ್ ಚ ಪಾತಾಲೇ ಮಾಂ ಪಾತು ಚೈಕಜಟಾ ಸತೀ ।\\
ತಾರಾ ತು ಸಹಿತಾ ಖೇಽವ್ಯಾನ್ಮಹಾನೀಲಸರಸ್ವತೀ ॥೨೬॥

ಇತಿ ತೇ ಕಥಿತಂ ದೇವ್ಯಾಃ ಕವಚಂ ಮಂತ್ರವಿಗ್ರಹಂ ।\\
ಯದ್ಧೃತ್ವಾ ಪಠನಾನ್ಭೀಮಃ ಕ್ರೋಧಾಖ್ಯೋ ಭೈರವಃ ಸ್ಮೃತಃ ॥೨೭॥

ಸುರಾಸುರಮುನೀಂದ್ರಾಣಾಂ ಕರ್ತಾ ಹರ್ತಾ ಭವೇತ್ಸ್ವಯಂ ।\\
ಯಸ್ಯಾಜ್ಞಯಾ ಮಧುಮತೀ ಯಾತಿ ಸಾ ಸಾಧಕಾಲಯಂ ॥೨೮॥

ಭೂತಿನ್ಯಾದ್ಯಾಶ್ಚ ಡಾಕಿನ್ಯೋ ಯಕ್ಷಿಣ್ಯಾದ್ಯಾಶ್ಚ ಖೇಚರಾಃ ।\\
ಆಜ್ಞಾಂ ಗೃಹ್ಣಂತಿ ತಾಸ್ತಸ್ಯ ಕವಚಸ್ಯ ಪ್ರಸಾದತಃ ॥೨೯॥

ಏತದೇವಂ ಪರಂ ಬ್ರಹ್ಮಕವಚಂ ಮನ್ಮುಖೋದಿತಂ ।\\
ದೇವೀಮಭ್ಯರ್ಚ ಗಂಧಾದ್ಯೈರ್ಮೂಲೇನೈವ ಪಠೇತ್ಸಕೃತ್ ॥೩೦॥

ಸಂವತ್ಸರಕೃತಾಯಾಸ್ತು ಪೂಜಾಯಾಃ ಫಲಮಾಪ್ನುಯಾತ್ ।\\
ಭೂರ್ಜೇ ವಿಲಿಖಿತಂ ಚೈತದ್ಗುಟಿಕಾಂ ಕಾಂಚನಸ್ಥಿತಾಂ ॥೩೧॥

ಧಾರಯೇದ್ದಕ್ಷಿಣೇ ಬಾಹೌ ಕಂಠೇ ವಾ ಯದಿ ವಾನ್ಯತಃ ।\\
ಸರ್ವೈಶ್ವರ್ಯಯುತೋ ಭೂತ್ವಾ ತ್ರೈಲೋಕ್ಯಂ ವಶಮಾನಯೇತ್ ॥೩೨॥

ತಸ್ಯ ಗೇಹೇ ವಸೇಲ್ಲಕ್ಷ್ಮೀರ್ವಾಣೀ ಚ ವದನಾಂಬುಜೇ ।\\
ಬ್ರಹ್ಮಾಸ್ತ್ರಾದೀನಿ ಶಸ್ತ್ರಾಣಿ ತದ್ಗಾತ್ರೇ ಯಾಂತಿ ಸೌಮ್ಯತಾಂ ॥೩೩॥

ಇದಂ ಕವಚಮಜ್ಞಾತ್ವಾ ಯೋ ಭಜೇಚ್ಛಿನ್ನಮಸ್ತಕಾಂ ।\\
ಸೋಽಪಿ ಶತ್ರಪ್ರಹಾರೇಣ ಮೃತ್ಯುಮಾಪ್ನೋತಿ ಸತ್ವರಂ ॥೩೪॥

\authorline{॥ಇತಿ ಶ್ರೀಭೈರವತಂತ್ರೇ ಭೈರವಭೈರವೀಸಂವಾದೇ ತ್ರೈಲೋಕ್ಯವಿಜಯಂ ನಾಮ ಛಿನ್ನಮಸ್ತಾಕವಚಂ ಸಂಪೂರ್ಣಂ ॥}
%=================================================================================
\section{ನವಗ್ರಹಕವಚಂ}
\addcontentsline{toc}{section}{ನವಗ್ರಹಕವಚಂ}

ಬ್ರಹ್ಮೋವಾಚ ।\\
ಶಿರೋ ಮೇ ಪಾತು ಮಾರ್ತಾಂಡೋ ಕಪಾಲಂ ರೋಹಿಣೀಪತಿಃ ।\\
ಮುಖಮಂಗಾರಕಃ ಪಾತು ಕಂಠಶ್ಚ ಶಶಿನಂದನಃ ॥

ಬುದ್ಧಿಂ ಜೀವಃ ಸದಾ ಪಾತು ಹೃದಯಂ ಭೃಗುನಂದನಃ ।\\
ಜಠರಂಚ ಶನಿಃ ಪಾತು ಜಿಹ್ವಾಂ ಮೇ ದಿತಿನಂದನಃ ॥

ಪಾದೌ ಕೇತುಃ ಸದಾ ಪಾತು ವಾರಾಃ ಸರ್ವಾಂಗಮೇವ ಚ ।\\
ತಿಥಯೋಽಷ್ಟೌ ದಿಶಃ ಪಾಂತು ನಕ್ಷತ್ರಾಣಿ ವಪುಃ ಸದಾ ॥

ಅಂಸೌ ರಾಶಿಃ ಸದಾ ಪಾತು ಯೋಗಾಶ್ಚ ಸ್ಥೈರ್ಯಮೇವ ಚ ।\\
ಗುಹ್ಯಂ ಲಿಂಗಂ ಸದಾ ಪಾಂತು ಸರ್ವೇ ಗ್ರಹಾಃ ಶುಭಪ್ರದಾಃ ।\\
ಅಣಿಮಾದೀನಿ ಸರ್ವಾಣಿ ಲಭತೇ ಯಃ ಪಠೇದ್ ಧ್ರುವಂ ॥

ಏತಾಂ ರಕ್ಷಾಂ ಪಠೇದ್ ಯಸ್ತು ಭಕ್ತ್ಯಾ ಸ ಪ್ರಯತಃ ಸುಧೀಃ ।\\
ಸ ಚಿರಾಯುಃ ಸುಖೀ ಪುತ್ರೀ ರಣೇ ಚ ವಿಜಯೀ ಭವೇತ್ ॥

ಅಪುತ್ರೋ ಲಭತೇ ಪುತ್ರಂ ಧನಾರ್ಥೀ ಧನಮಾಪ್ನುಯಾತ್ ।\\
ದಾರಾರ್ಥೀ ಲಭತೇ ಭಾರ್ಯಾಂ ಸುರೂಪಾಂ ಸುಮನೋಹರಾಂ ॥

ರೋಗೀ ರೋಗಾತ್ಪ್ರಮುಚ್ಯೇತ ಬದ್ಧೋ ಮುಚ್ಯೇತ ಬಂಧನಾತ್ ।\\
ಜಲೇ ಸ್ಥಲೇ ಚಾಂತರಿಕ್ಷೇ ಕಾರಾಗಾರೇ ವಿಶೇಷತಃ ॥

ಯಃ ಕರೇ ಧಾರಯೇನ್ನಿತ್ಯಂ ಭಯಂ ತಸ್ಯ ನ ವಿದ್ಯತೇ ।\\
ಬ್ರಹ್ಮಹತ್ಯಾ ಸುರಾಪಾನಂ ಸ್ತೇಯಂ ಗುರ್ವಂಗನಾಗಮಃ ।\\
ಸರ್ವಪಾಪೈಃ ಪ್ರಮುಚ್ಯೇತ ಕವಚಸ್ಯ ಚ ಧಾರಣಾತ್ ॥

ನಾರೀ ವಾಮಭುಜೇ ಧೃತ್ವಾ ಸುಖೈಶ್ವರ್ಯಸಮನ್ವಿತಾ ।\\
ಕಾಕವಂಧ್ಯಾ ಜನ್ಮವಂಧ್ಯಾ ಮೃತವತ್ಸಾ ಚ ಯಾ ಭವೇತ್ ।\\
ಬಹ್ವಪತ್ಯಾ ಜೀವವತ್ಸಾ ಕವಚಸ್ಯ ಪ್ರಸಾದತಃ ॥

\authorline{ಇತಿ ಗ್ರಹಯಾಮಲೇ ಉತ್ತರಖಂಡೇ ನವಗ್ರಹ ಕವಚಂ ಸಮಾಪ್ತಂ ।}
%=================================================
\section{ನಿತ್ಯಾಕವಚಂ}
\addcontentsline{toc}{section}{ನಿತ್ಯಾಕವಚಂ}
ಸಮಸ್ತಾಪದ್ವಿಮುಕ್ತ್ಯರ್ಥಂ ಸರ್ವಸಂಪದವಾಪ್ತಯೇ ।\\
ಭೂತಪ್ರೇತಪಿಶಾಚಾದಿಪೀಡಾಶಾಂತ್ಯೈ ಸುಖಾಪ್ತಯೇ ॥೧॥

ಸಮಸ್ತರೋಗನಾಶಾಯ ಸಮರೇ ವಿಜಯಾಯ ಚ ।\\
ಚೋರಸಿಂಹದ್ವೀಪಿಗಜ ಗವಯಾದಿಭಯಾನಕೇ ॥೨॥

ಅರಣ್ಯೇ ಶೈಲಗಹನೇ ಮಾರ್ಗೇ ದುರ್ಭಿಕ್ಷಕೇ ತಥಾ ।\\
ಸಲಿಲಾಗ್ನಿ ಮರುತ್ಪೀಡಾಸ್ವಬ್ಧೌ ಪೋತಾದಿಸಂಕಟೇ ॥೩॥

ಪ್ರಜಪ್ಯ ನಿತ್ಯಾಕವಚಂ ಸಕೃತ್ಸರ್ವಂ ತರತ್ಯಸೌ ।\\
ಸುಖೀ ಜೀವತಿ ನಿರ್ದ್ವಂದ್ವೋ ನಿಃಸಪತ್ನೋ ಜಿತೇಂದ್ರಿಯಃ ॥೪॥

ಶೃಣು ತತ್ ಕವಚಂ ದೇವಿ ವಕ್ಷ್ಯೇ ತವ ತವಾತ್ಮಕಂ ।\\
ಯೇನಾಹಮಪಿ ಯುದ್ಧೇಷು ದೇವಾಸುರಜಯೀ ಸದಾ ॥೫॥

ಸರ್ವತಃ ಸರ್ವದಾತ್ಮಾನಂ ಲಲಿತಾ ಪಾತು ಸರ್ವಗಾ ।\\
ಕಾಮೇಶೀ ಪುರತಃ ಪಾತು ಭಗಮಾಲಾ ತ್ವನಂತರಾಂ ॥೬॥

ದಿಶಂ ಪಾತು ತಥಾ ದಕ್ಷಪಾರ್ಶ್ವಂ ಮೇ ಪಾತು ಸರ್ವದಾ ।\\
ನಿತ್ಯಕ್ಲಿನ್ನಾಥ ಭೇರುಂಡಾ ದಿಶಂ ಪಾತು ಸದಾ ಮಮ ॥೭॥

ತಥೈವ ಪಶ್ಚಿಮಂ ಭಾಗಂ ರಕ್ಷೇತ್ ಸಾ ವಹ್ನಿವಾಸಿನೀ ।\\
ಮಹಾವಜ್ರೇಶ್ವರೀ ರಕ್ಷೇದನಂತರದಿಶಂ ಸದಾ ॥೮॥

ವಾಮಪಾರ್ಶ್ವೇ ಸದಾ ಪಾತು ದೂತೀ ಮೇ ತ್ವರಿತಾ ತತಃ ।\\
ಪಾಲಯೇತ್ತು ದಿಶಂ ಚಾನ್ಯಾಂ ರಕ್ಷೇನ್ಮಾಂ ಕುಲಸುಂದರೀ ॥೯॥

ನಿತ್ಯಾ ಮಾಮೂರ್ಧ್ವತಃ ಪಾತು ಸಾಧೋ ಮೇ ಪಾತು ಸರ್ವದಾ ।\\
ನಿತ್ಯಾ ನೀಲಪತಾಕಾಖ್ಯಾ ವಿಜಯಾ ಸರ್ವತಶ್ಚ ಮಾಂ ॥೧೦॥

ಕರೋತು ಮೇ ಮಂಗಲಾನಿ ಸರ್ವದಾ ಸರ್ವಮಂಗಲಾ ।\\
ದೇಹೇಂದ್ರಿಯಮನಃಪ್ರಾಣಾನ್ ಜ್ವಾಲಾಮಾಲಿನಿವಿಗ್ರಹಾ ॥೧೧॥

ಪಾಲಯೇದನಿಶಂ ಚಿತ್ರಾ ಚಿತ್ತಂ ಮೇ ಪಾತು ಸರ್ವದಾ ।\\
ಕಾಮಾತ್ ಕ್ರೋಧಾತ್ತಥಾ ಲೋಭಾನ್ಮೋಹಾತ್ಪಾಯಾನ್ಮದಾದಪಿ ॥೧೨॥

ಪಾಪಾನ್ಮತ್ಸರತಃ ಶೋಕಾತ್ ಸಂಶಯಾತ್ ಸರ್ವತಃ ಸದಾ ।\\
ಸ್ತೈಮಿತ್ಯಾಚ್ಚ ಸಮುದ್ಯೋಗಾದಶುಭೇಷು ತು ಕರ್ಮಸು ॥೧೩॥

ಅಸತ್ಯಾತ್ ಕ್ರೂರಚಿಂತಾತೋ ಹಿಂಸಾತಶ್ಚೌರ್ಯತಸ್ತಥಾ ।\\
ರಕ್ಷಂತು ಮಾಂ ಸರ್ವದಾ ತಾಃ ಕುರ್ವಂತ್ವಿಚ್ಛಾಂ ಶುಭೇಷು ಚ ॥೧೪॥

ನಿತ್ಯಾಃ ಷೋಡಶ ಮಾಂ ಪಾಂತು ಗಜಾರೂಢಾಃ ಸ್ವಶಕ್ತಿಭಿಃ ।\\
ತಥಾ ಹಯಸಮಾರೂಢಾಃ ಪಾಂತು ಮಾಂ ಸರ್ವತಃ ಸದಾ ॥೧೫॥

ಸಿಂಹಾರೂಢಾಸ್ತಥಾ ಪಾಂತು ಮಾಂ ತರಕ್ಷುಗತಾ ಅಪಿ ।\\
ರಥಾರೂಢಾಶ್ಚ ಮಾಂ ಪಾಂತು ಸರ್ವತಃ ಸರ್ವದಾ ರಣೇ ॥೧೬॥

ತಾರ್ಕ್ಷ್ಯಾರೂಢಾಶ್ಚ ಮಾಂ ಪಾಂತು ತಥಾ ವ್ಯೋಮಗತಾಸ್ತಥಾ ।\\
ಭೂಗತಾಃ ಸರ್ವದಾ ಪಾಂತು ಮಾಂಚ ಸರ್ವತ್ರ ಸರ್ವದಾ ॥೧೭॥

ಭೂತಪ್ರೇತಪಿಶಾಚಾಪಸ್ಮಾರಕೃತ್ಯಾದಿಕಾನ್ ಗದಾನ್ ।\\
ದ್ರಾವಯಂತು ಸ್ವಶಕ್ತೀನಾಂ ಭೀಷಣೈರಾಯುಧೈರ್ಮಮ ॥೧೮॥

ಗಜಾಶ್ವದ್ವೀಪಿಪಂಚಾಸ್ಯತಾರ್ಕ್ಷ್ಯಾರೂಢಾಖಿಲಾಯುಧಾಃ ।\\
ಅಸಂಖ್ಯಾಃ ಶಕ್ತಯೋ ದೇವ್ಯಾಃ ಪಾಂತು ಮಾಂ ಸರ್ವತಃ ಸದಾ ॥೧೯॥

ಸಾಯಂ ಪ್ರಾತರ್ಜಪೇನ್ನಿತ್ಯಾಕವಚಂ ಸರ್ವರಕ್ಷಕಂ ।\\
ಕದಾಚಿನ್ನಾಶುಭಂ ಪಶ್ಯೇನ್ನ ಶೃಣೋತಿ ಚ ಮತ್ಸಮಃ ॥೨೦॥

\authorline{ಇತಿ  ನಿತ್ಯಾಕವಚಂ ಸಂಪೂರ್ಣಂ ॥}
%============================================================================
\section{ಶ್ರೀಭುವನೇಶ್ವರೀ ತ್ರೈಲೋಕ್ಯಮೋಹನಕವಚಂ}
\addcontentsline{toc}{section}{ಶ್ರೀಭುವನೇಶ್ವರೀ ತ್ರೈಲೋಕ್ಯಮೋಹನಕವಚಂ}

ಶ್ರೀದೇವ್ಯುವಾಚ।\\
ಭಗವನ್ ಪರಮೇಶಾನ ಸರ್ವಾಗಮವಿಶಾರದ ।\\
ಕವಚಂ ಭುವನೇಶ್ವರ್ಯಾಃ ಕಥಯಸ್ವ ಮಹೇಶ್ವರ ॥

ಶ್ರೀ ಭೈರವ ಉವಾಚ ।\\
ಶೃಣು ದೇವಿ ಮಹೇಶಾನಿ ಕವಚಂ ಸರ್ವಕಾಮದಂ ।\\
ತ್ರೈಲೋಕ್ಯಮೋಹನಂ ನಾಮ ಸರ್ವೇಪ್ಸಿತಫಲಪ್ರದಂ ॥

ಓಂ ಅಸ್ಯ ಶ್ರೀತ್ರೈಲೋಕ್ಯಮೋಹನಕವಚಸ್ಯ ಶ್ರೀಸದಾಶಿವ ಋಷಿಃ । ವಿರಾಟ್ ಛಂದಃ । ಶ್ರೀಭುವನೇಶ್ವರೀ ದೇವತಾ । ಚತುರ್ವರ್ಗಸಿದ್ಧ್ಯರ್ಥಂ ಕವಚಪಾಠೇ ವಿನಿಯೋಗಃ ।\\

ಅಥ ಕವಚಸ್ತೋತ್ರಂ ।\\
ಓಂ ಹ್ರೀಂ ಕ್ಲೀಂ ಮೇ ಶಿರಃ ಪಾತು ಶ್ರೀಂ ಫಟ್ ಪಾತು ಲಲಾಟಕಂ ।\\
ಸಿದ್ಧಪಂಚಾಕ್ಷರೀ ಪಾಯಾನ್ನೇತ್ರೇ ಮೇ ಭುವನೇಶ್ವರೀ ॥೧॥

ಶ್ರೀಂ ಕ್ಲೀಂ ಹ್ರೀಂ ಮೇ ಶ್ರುತೀಃ ಪಾತು ನಮಃ ಪಾತು ಚ ನಾಸಿಕಾಂ ।\\
ದೇವೀ ಷಡಕ್ಷರೀ ಪಾತು ವದನಂ ಮುಂಡಭೂಷಣಾ ॥೨॥

ಓಂ ಹ್ರೀಂ ಶ್ರೀಂ ಐಂ ಗಲಂ ಪಾತು ಜಿಹ್ವಾಂ ಪಾಯಾನ್ಮಹೇಶ್ವರೀ ।\\
ಐಂ ಸ್ಕಂಧೌ ಪಾತು ಮೇ ದೇವೀ ಮಹಾತ್ರಿಭುವನೇಶ್ವರೀ ॥೩॥

ಹ್ರೂಂ ಘಂಟಾಂ ಮೇ ಸದಾ ಪಾತು ದೇವ್ಯೇಕಾಕ್ಷರರೂಪಿಣೀ ।\\
ಐಂ ಹ್ರೀಂ ಶ್ರೀಂ ಹೂಂ ತು ಫಟ್ ಪಾಯಾದೀಶ್ವರೀ ಮೇ ಭುಜದ್ವಯಂ ॥೪॥

ಓಂ ಹ್ರೀಂ ಶ್ರೀಂ ಕ್ಲೀಂ ಐಂ ಫಟ್ ಪಾಯಾದ್ ಭುವನೇಶೀ ಸ್ತನದ್ವಯಂ ।\\
ಹ್ರಾಂ ಹ್ರೀಂ ಐಂ ಫಟ್ ಮಹಾಮಾಯಾ ದೇವೀ ಚ ಹೃದಯಂ ಮಮ ॥೫॥

ಐಂ ಹ್ರೀಂ ಶ್ರೀಂ ಹೂಂ ತು ಫಟ್ ಪಾಯಾತ್ ಪಾರ್ಶ್ವೌ ಕಾಮಸ್ವರೂಪಿಣೀ ।\\
ಓಂ ಹ್ರೀಂ ಕ್ಲೀಂ ಐಂ ನಮಃ ಪಾಯಾತ್ ಕುಕ್ಷಿಂ ಮಹಾಷಡಕ್ಷರೀ ॥೬॥

ಐಂ ಸೌಃ ಐಂ ಐಂ ಕ್ಲೀಂ ಫಟ್ ಸ್ವಾಹಾ ಕಟಿದೇಶೇ ಸದಾಽವತು ।\\
ಅಷ್ಟಾಕ್ಷರೀ ಮಹಾವಿದ್ಯಾ ದೇವೇಶೀ ಭುವನೇಶ್ವರೀ ॥೭॥

ಓಂ ಹ್ರೀಂ ಹ್ರೌಂ ಐಂ ಶ್ರೀಂ ಹ್ರೀಂ ಫಟ್ ಪಾಯಾನ್ಮೇ ಗುಹ್ಯಸ್ಥಲಂ ಸದಾ ।\\
ಷಡಕ್ಷರೀ ಮಹಾವಿದ್ಯಾ ಸಾಕ್ಷಾದ್ ಬ್ರಹ್ಮಸ್ವರೂಪಿಣೀ ॥೮॥

ಐಂ ಹ್ರಾಂ ಹ್ರೌಂ ಹ್ರೂಂ ನಮೋ ದೇವ್ಯೈ ದೇವಿ ಸರ್ವಂ ಪದಂ ತತಃ
ದುಸ್ತರಂ ಪದಂ ತಾರಯ ತಾರಯ ಪ್ರಣವದ್ವಯಂ ।\\
ಸ್ವಾಹಾ ಇತಿ ಮಹಾವಿದ್ಯಾ ಜಾನುನಿ ಮೇ ಸದಾಽವತು ॥೯॥

ಐಂ ಸೌಃ ಓಂ ಐಂ ಕ್ಲೀಂ ಫಟ್ ಸ್ವಾಹಾ ಜಂಘೇಽವ್ಯಾದ್ ಭುವನೇಶ್ವರೀ ।\\
ಓಂ ಹ್ರೀಂ ಶ್ರೀಂ ಕ್ಲೀಂ ಐಂ ಫಟ್ ಪಾಯಾತ್ ಪಾದೌ ಮೇ ಭುವನೇಶ್ವರೀ ॥೧೦॥

ಓಂ ಓಂ ಹ್ರೀಂ ಹ್ರೀಂ ಶ್ರೀಂ ಶ್ರೀಂ ಕ್ಲೀಂ ಕ್ಲೀಂ ಐಂ ಐಂ ಸೌಃ ಸೌಃ ವದ ವದ ।\\
ವಾಗ್ವಾದಿನೀತಿ ಚ ದೇವಿ ವಿದ್ಯಾ ಯಾ ವಿಶ್ವವ್ಯಾಪಿನೀ ॥೧೧॥

ಸೌಃಸೌಃಸೌಃ ಐಂಐಂಐಂ ಕ್ಲೀಂಕ್ಲೀಂಕ್ಲೀಂ ಶ್ರೀಂಶ್ರೀಂಶ್ರೀಂ ಹ್ರೀಂಹ್ರೀಂಹ್ರೀಂ ಓಂ ।\\
ಓಂ ಓಂ ಚತುರ್ದಶಾತ್ಮಿಕಾ ವಿದ್ಯಾ ಪಾಯಾತ್ ಬಾಹೂ ತು ಮೇ ॥೧೨॥

ಸಕಲಂ ಸರ್ವಭೀತಿಭ್ಯಃ ಶರೀರಂ ಭುವನೇಶ್ವರೀ ।\\
ಓಂ ಹ್ರೀಂ ಶ್ರೀಂ ಇಂದ್ರದಿಗ್ಭಾಗೇ ಪಾಯಾನ್ಮೇ ಚಾಪರಾಜಿತಾ ॥೧೩॥

ಸ್ತ್ರೀಂ ಐಂ ಹ್ರೀಂ ವಿಜಯಾ ಪಾಯಾದಿಂದುಮದಗ್ನಿದಿಕ್ಸ್ಥಲೇ ।\\
ಓಂ ಶ್ರೀಂ ಸೌಃ ಕ್ಲೀಂ ಜಯಾ ಪಾತು ಯಾಮ್ಯಾಂ ಮಾಂ ಕವಚಾನ್ವಿತಂ ॥೧೪॥

ಹ್ರೀಂ ಹ್ರೀಂ ಐಂ ಸೌಃ ಹಸೌಃ ಪಾಯಾನ್ನೈಋತಿರ್ಮಾಂ ತು ಪರಾತ್ಮಿಕಾ ।\\
ಓಂ ಶ್ರೀಂ ಶ್ರೀಂ ಹ್ರೀಂ ಸದಾ ಪಾಯಾತ್ ಪಶ್ಚಿಮೇ ಬ್ರಹ್ಮರೂಪಿಣೀ ॥೧೫॥

ಓಂ ಹ್ರಾಂ ಸೌಃ ಮಾಂ ಭಯಾದ್ ರಕ್ಷೇದ್ ವಾಯವ್ಯಾಂ ಮಂತ್ರರೂಪಿಣೀ ।\\
ಐಂ ಕ್ಲೀಂ ಶ್ರೀಂ ಸೌಃ ಸದಾಽವ್ಯಾನ್ಮಾಂ ಕೌವೇರ್ಯಾಂ ನಗನಂದಿನೀ ॥೧೬॥

ಓಂ ಹ್ರೀಂ ಶ್ರೀಂ ಕ್ಲೀಂ ಮಹಾದೇವೀ ಐಶಾನ್ಯಾಂ ಪಾತು ನಿತ್ಯಶಃ ।\\
ಓಂ ಹ್ರೀಂ ಮಂತ್ರಮಯೀ ವಿದ್ಯಾ ಪಾಯಾದೂರ್ಧ್ವಂ ಸುರೇಶ್ವರೀ ॥೧೭॥

ಓಂ ಹ್ರೀಂ ಶ್ರೀಂ ಕ್ಲೀಂ ಐಂ ಮಾಂ ಪಾಯಾದಧಸ್ಥಾ ಭುವನೇಶ್ವರೀ ।\\
ಏವಂ ದಶದಿಶೋ ರಕ್ಷೇತ್ ಸರ್ವಮಂತ್ರಮಯೋ ಶಿವಾ ॥೧೮॥

ಪ್ರಭಾತೇ ಪಾತು ಚಾಮುಂಡಾ ಶ್ರೀಂ ಕ್ಲೀಂ ಐಂ ಸೌಃ ಸ್ವರೂಪಿಣೀ ।\\
ಮಧ್ಯಾಹ್ನೇಽವ್ಯಾನ್ಮಾಮಂಬಾ ಶ್ರೀಂ ಹ್ರೀಂ ಕ್ಲೀಂ ಐಂ ಸೌಃ ಸ್ವರೂಪಿಣೀ ॥೧೯॥

ಸಾಯಂ ಪಾಯಾದುಮಾದೇವೀ ಐಂ ಹ್ರೀಂ ಕ್ಲೀಂ ಸೌಃ ಸ್ವರೂಪಿಣೀ ।\\
ನಿಶಾದೌ ಪಾತು ರುದ್ರಾಣೀ ಓಂ ಕ್ಲೀಂ ಕ್ರೀಂ ಸೌಃ ಸ್ವರೂಪಿಣೀ ॥೨೦॥

ನಿಶೀಥೇ ಪಾತು ಬ್ರಹ್ಮಾಣೀ ಕ್ರೀಂ ಹ್ರೂಂ ಹ್ರೀಂ ಹ್ರೀಂ ಸ್ವರೂಪಿಣೀ ।\\
ನಿಶಾಂತೇ ವೈಷ್ಣವೀ ಪಾಯಾದೋಮೈ ಹ್ರೀಂ ಕ್ಲೀಂ ಸ್ವರೂಪಿಣೀ ॥೨೧॥

ಸರ್ವಕಾಲೇ ಚ ಮಾಂ ಪಾಯಾದೋ ಹ್ರೀಂ ಶ್ರೀಂ ಭುವನೇಶ್ವರೀ ।\\
ಏಷಾ ವಿದ್ಯಾ ಮಯಾ ಗುಪ್ತಾ ತಂತ್ರೇಭ್ಯಶ್ಚಾಪಿ ಸಾಂಪ್ರತಂ ॥೨೨॥

ಫಲಶ್ರುತಿಃ ।\\
ದೇವೇಶಿ ಕಥಿತಾಂ ತುಭ್ಯಂ ಕವಚೇಚ್ಛಾ ತ್ವಯಿ ಪ್ರಿಯೇ ।\\
ಇತಿ ತೇ ಕಥಿತಂ ದೇವಿ ಗುಹ್ಯಾದ್ಗುಹ್ಯಂತರ ಪರಂ ।\\
ತ್ರೈಲೋಕ್ಯಮೋಹನಂ ನಾಮ ಕವಚಂ ಮಂತ್ರವಿಗ್ರಹಂ ।\\
ಬ್ರಹ್ಮವಿದ್ಯಾಮಯಂ ಚೈವ ಕೇವಲಂ ಬ್ರಹ್ಮರೂಪಿಣಂ ॥೧॥

ಮಂತ್ರವಿದ್ಯಾಮಯಂ ಚೈವ ಕವಚಂ ಬನ್ಮುಖೋದಿತಂ ।\\
ಗುರುಮಭ್ಯರ್ಚ್ಯ ವಿಧಿವತ್ ಕವಚಂ ಧಾರಯೇದ್ಯದಿ ।\\
ಸಾಧಕೋ ವೈ ಯಥಾಧ್ಯಾನಂ ತತ್ಕ್ಷಣಾದ್ ಭೈರವೋ ಭವೇತ್ ।\\
ಸರ್ವಪಾಪವಿನಿರ್ಮುಕ್ತಃ ಕುಲಕೋಟಿ ಸಮುದ್ಧರೇತ್ ॥೨॥

ಗುರುಃ ಸ್ಯಾತ್ ಸರ್ವವಿದ್ಯಾಸು ಹ್ಯಧಿಕಾರೋ ಜಪಾದಿಷು ।\\
ಶತಮಷ್ಟೋತ್ತರಂ ಚಾಸ್ಯ ಪುರಶ್ಚರ್ಯಾವಿಧಿಃ ಸ್ಮೃತಾ ।\\
ಶತಮಷ್ಟೋತ್ತರಂ ಜಪ್ತ್ವಾ ತಾವದ್ಧೋಮಾದಿಕಂ ತಥಾ ।\\
ತ್ರೈಲೋಕ್ಯೇ ವಿಚರೇದ್ವೀರೋ ಗಣನಾಥೋ ಯಥಾ ಸ್ವಯಂ ॥೩॥

ಗದ್ಯಪದ್ಯಮಯೀ ವಾಣೀ ಭವೇದ್ ಗಂಗಾಪ್ರವಾಹವತ್ ।\\
ಪುಷ್ಪಾಂಜಲ್ಯಷ್ಟಕಂ ದತ್ವಾ ಮೂಲೇನೈವ ಪಠೇತ್ ಸಕೃತ್ ॥೪॥

asಇತಿ ಶ್ರೀಭುವನೇಶ್ವರೀತ್ರೈಲೋಕ್ಯಮೋಹನಕವಚಂ ಸಂಪೂರ್ಣಂ ।\\
%=========================================================================
\section{ಶ್ರೀತುರೀಯಾಷೋಡಶೀ ತ್ರೈಲೋಕ್ಯವಿಜಯಕವಚಂ}
\addcontentsline{toc}{section}{ಶ್ರೀತುರೀಯಾಷೋಡಶೀ ತ್ರೈಲೋಕ್ಯವಿಜಯಕವಚಂ}
॥ಪೂರ್ವಪೀಠಿಕಾ ॥\\
ಕೈಲಾಸಶಿಖರೇ ರಮ್ಯೇ ದೇವದೇವಂ ಜಗದ್ಗುರುಂ ।\\
ಶಂಕರಂ ಪರಿಪಪ್ರಚ್ಛ ಕುಮಾರಃ ಶಿಖಿವಾಹನಃ ॥೧॥

॥ಶ್ರೀಸ್ಕಂದ ಉವಾಚ ॥\\
ಜನಕ ಶ್ರೀಗುರೋ ದೇವ ಪುತ್ರವಾತ್ಸಲ್ಯವರ್ಧನ! ।\\
ಕವಚಂ ಷೋಡಶಾಕ್ಷರ್ಯಾ ವದ ಮೇ ಪರಮೇಶ್ವರ!॥೨॥

॥ಶ್ರೀಈಶ್ವರ ಉವಾಚ ॥\\
ಶೃಣು ಪುತ್ರ ಮಹಾಭಾಗ! ಕವಚಂ ಮಂತ್ರವಿಗ್ರಹಂ ।\\
ಗೋಪ್ಯಾದ್ಗೋಪ್ಯತರಂ ಗೋಪ್ಯಂ ಗುಹ್ಯಾದ್ಗುಹ್ಯತರಂ ಮಹತ್ ॥೩॥

ತವ ಸ್ನೇಹಾತ್ ಪ್ರವಕ್ಷ್ಯಾಮಿ ನಾಖ್ಯಾತಂ ಯಸ್ಯ ಕಸ್ಯಚಿತ್ ।\\
ಬ್ರಹ್ಮೇಶವಿಷ್ಣುಶಕ್ರೇಭ್ಯೋ ಮಯಾ ನ ಕಥಿತಂ ಪುರಾ ॥೪॥

ದೇವಾಗ್ರೇ ನೈವ ದೈತ್ಯಾಗ್ರೇ ನ ಶೌನಕಗುಣಾಗ್ರಕೇ ।\\
ತವ ಭಕ್ತ್ಯಾ ತು ಕವಚಂ ಗೋಪ್ಯಂ ಕರ್ತುಂ ನ ಶಕ್ಯತೇ ॥೫॥

ಏಕಾದಶಮಹಾವಿದ್ಯಾ ಸವರ್ಣಾ ಸದ್ಗುಣಾನ್ವಿತಾ ।\\
ಸರ್ವಶಕ್ತಿಪ್ರಧಾನಾ ಹಿ ಕವಚಂ ಮನ್ಮುಖೋದಿತಂ ॥೬॥

ಗಣೇಶಶ್ಚ ರವಿರ್ವಿಷ್ಣುಃ ಶಿವಶಕ್ತಿಶ್ಚ ಭೈರವಃ ।\\
ಪಂಚತತ್ತ್ವಾದಿಸರ್ವೇಷಾಂ ವರ್ಮಾವರ್ತೇನ ತುಷ್ಯತಾಂ ॥೭॥

ಊರ್ಧ್ವಾಮ್ನಾಯಂ ಮಹಾಮಂತ್ರಂ ಕವಚಂ ನಿರ್ಮಿತಂ ಮಯಾ ।\\
ತ್ರೈಲೋಕ್ಯವಿಜಯಂ ದಿವ್ಯಂ ತುರೀಯಂ ಕವಚಂ ಶುಭಂ ॥೮॥

॥ಶ್ರೀಸ್ಕಂದ ಉವಾಚ ॥\\
ಕವಚಂ ಪರಮಂ ದೇವ್ಯಾಃ ಶ್ರೋತುಮಿಚ್ಛಾಮ್ಯಽಹಂ ಪ್ರಭೋ ।\\
ಯತ್ಸೂಚಿತಂ ತ್ವಯಾ ಪೂರ್ವ ಧರ್ಮಕಾಮಾರ್ಥಮೋಕ್ಷದಂ ॥೯॥

ತದ್ವಿನಾರಾಧನಂ ನಾಸ್ತಿ ಜಪಂ ಧ್ಯಾನಾದಿ ಕರ್ಮಣಿ ।\\
ತಸ್ಯಾಶ್ಚ ಕವಚಂ ಯಸ್ಮಾತ್ ತಸ್ಮಾತ್ ತದ್ವದ ಮೇ ಪ್ರಭೋ ॥೧೦॥

॥ಶ್ರೀಈಶ್ವರ ಉವಾಚ ॥\\
ಶೃಣು ಪುತ್ರ ಮಹಾಪ್ರಾಜ್ಞ ! ರಹಸ್ಯಾತಿರಹಸ್ಯಕಂ ।\\
ತುರೀಯಂ ಕವಚಂ ದಿವ್ಯಂ ಸರ್ವಮಂತ್ರಮಯಂ ಶುಭಂ ॥೧೧॥

ಪೂಜಾಂತೇ ತು ಜಪಾತ್ಪೂರ್ವಂ ಕವಚಂ ಸಮುದೀರಯೇತ್ ।\\
ಅಸ್ಯ ಶ್ರೀಕವಚಸ್ಯಾಸ್ಯ ಋಷಿರಾನಂದಭೈರವಃ ॥೧೨॥

ಶ್ರೀವಿದ್ಯಾ ದೇವತಾಶ್ಛಂದೋ ಗಾಯತ್ರೀರುಚಿವೃತ್ತಿಕಂ ।\\
ರಮಾಬೀಜಂ ಪರಾಶಕ್ತಿಃ ವಾಗ್ಭವಂ ಕೀಲಕಂ ಸ್ಮೃತಂ ॥೧೩॥

ಮಮ ಸರ್ವಾರ್ಥಸಿದ್ಧಯರ್ಥೇ ಜಪೇ ತು ವಿನಿಯೋಗತಃ ।\\
ಅಂಗುಷ್ಠಾಗ್ರೇ ವಾಗ್ಭವಂ ಚ ಹೃಲ್ಲೇಖಾಂ ತರ್ಜನೀ ನ್ಯಸೇತ್ ॥೧೪॥

ಲಕ್ಷ್ಮೀಬೀಜಂ ಮಧ್ಯಮಾಯಾಮಂಗಿರಾಽನಾಮಿಕಾ ತಥಾ ।\\
ಪರಾಬೀಜಂ ಕನಿಷ್ಠಾಯಾಂ ಲಕ್ಷ್ಮೀಂ ಕರತಲೇ ನ್ಯಸೇತ್ ॥೧೫॥

ಹೃದಯೇ ವಾಗ್ಭವಂ ನ್ಯಸ್ಯಾತ್ ಪರಾಂ ಶಿರಸಿ ಚ ನ್ಯಸೇತ್ ।\\
ಲಕ್ಷ್ಮೀಬೀಜಂ ಶಿಖಾಯಾಂ ಚ ಕವಚೇ ವಾಗ್ಭವಂ ನ್ಯಸೇತ್ ॥೧೬॥

ಹೃಲ್ಲೇಖಾಂ ನೇತ್ರಯೋರ್ನ್ಯಸ್ಯಾದಸ್ತ್ರಂ ತು ಕಮಲಾಂ ನ್ಯಸೇತ್ ।\\
ಭೂರ್ಭುವಃ ಸ್ವರಿತಿ ಮನುನಾ ದಿಗ್ಬಂಧನಮಾಚರೇತ್ ॥೧೭॥

ಧ್ಯಾನಂ ತಸ್ಯ ಪ್ರವಕ್ಷ್ಯಾಮಿ ಧರ್ಮಕಾಮಾರ್ಥಮೋಕ್ಷದಂ ।\\
ನ್ಯಾಸಧ್ಯಾನಾದಿಕಂ ಸರ್ವಂ ಕೃತ್ವಾ ತು ಕವಚಂ ಪಠೇತ್ ॥೧೮॥

ಕ್ಷೀರಸಾಗರಮಧ್ಯಸ್ಥೇ ರತ್ನದ್ವೀಪೇ ಮನೋಹರೇ ।\\
ರತ್ನಸಿಂಹಾಸನೇ ದಿವ್ಯೇ ತತ್ರ ದೇವೀಂ ವಿಚಿಂತಯೇತ್ ॥೧೯॥

ಕೋಟಿಸೂರ್ಯಪ್ರತೀಕಾಶಾಂ ಚಂದ್ರಕೋಟಿನಿಭಾನನಾಂ ।\\
ದಾಡಿಮೀಪುಷ್ಪಸಂಕಾಶಾಂ ಕುಂಕುಮೋದರಸನ್ನಿಭಾಂ ॥೨೦॥

ಜಪಾಕುಸುಮಸಂಕಾಶಾಂ ತ್ರಿನೇತ್ರಾಂ ಚ ಚತುರ್ಭುಜಾಂ ।\\
ಪಾಶಾಂಕುಶಧರಾಂ ರಮ್ಯಾಮಿಕ್ಷುಚಾಪಶರಾನ್ವಿತಾಂ ॥೨೧॥

ಕರ್ಪೂರಶಕಲೋನ್ಮಿಶ್ರತಾಂಬೂಲಪೂರಿತಾನನಾಂ ।\\
ಸರ್ವಶೃಂಗಾರ ವೇಷಾಢ್ಯಾಂ ಸರ್ವಾವಯವಶೋಭಿನೀಂ ॥೨೨॥

ಸರ್ವಾಯುಧಸಮಾಯುಕ್ತಾಂ ಪ್ರಸನ್ನವದನೇ ಕ್ಷಣಾಂ ।\\
ಸಪರಿವಾರಸಾವರಣಾಂ ಸರ್ವೋಪಚಾರಾರ್ಚಿತಾಂ ॥೨೩॥

ಏವಂ ಧ್ಯಾಯೇತ್ ತತೋ ವೀರ ! ಕವಚಂ ಸರ್ವಕಾಮದಂ ।\\
ಆವರ್ತಯೇತ್ ಸ್ವದೇಹೇ ತು ಸರ್ವರಕ್ಷಾಕರಂ ಶುಭಂ ॥೨೪॥

॥ಅಥ ಕವಚಪಾಠಃ ॥

ಶಿಖಾಯಾಂ ಮೇ ಹ್ಸೌಃ ಪಾತು ಶೌ ಮೇ ಪಾತು ಬ್ರಹ್ಮರಂಧ್ರಕೇ ।\\
ಸರ್ವದಾ ಹ್ಲೌಂ ಚ ಮಾಂ ಪಾತು ವಾಮದಕ್ಷಿಣಭಾಗಯೋಃ ॥೧॥

ಐಂ ಹ್ರೀಂ ಶ್ರೀಂ ಸರ್ವದಾ ಪಾತು ಷೋಡಶೀ ಸುಂದರೀ ಪರಾ ।\\
ಶ್ರೀಂ ಹ್ರೀಂ ಕ್ಲೀಂ ಸರ್ವದಾ ಪಾತು ಐಂ ಸೌಂ ಓಂ ಪಾತು ಮೇ ಸದಾ ॥೨॥

ಹ್ರೀಂ ಶ್ರೀಂ ಐಂ ಪಾತು ಸರ್ವತ್ರ ಕ್ನೌಂ ರಂ ಲಂ ಹ್ರೀಂ ಸದಾ ಮಮ ।\\
ಕ್ಲೀಂ ಹಸಕಹಲಹ್ರೀಂ ಮೇ ಪಾತು ಸದಾ ಸೌಃ ಕ್ಲೀಂ ಮಮ ॥೩॥

ಸೌಃ ಐಂ ಕ್ಲೀಂ ಹ್ರೀಂ ಶ್ರಿಯಾ ಪಾತು ಸಬೀಜಾ ಷೋಡಶಾಕ್ಷರೀ ।\\
ಆಪಾದಮಸ್ತಕಂ ಪಾತು ಮಹಾತ್ರಿಪುರಸುಂದರೀ ॥೪॥

ಶ್ರೀಜಯಂತೀ ಮಸ್ತಕೇ ಮಾಂ ಪಾತು ನಿತ್ಯಂ ವಿಭೂತಯೇ ।\\
ಹ್ರೀಂ ಮಂಗಲಾ ಸದಾ ನೇತ್ರೇ ಪಾತು ಸರ್ವಾರ್ಥಸಿದ್ಧಯೇ ॥೫॥

ಕ್ಲೀಂ ಕಾಲಿಕಾ ಕರ್ಣಯುಗ್ಮಂ ಪಾತು ಸರ್ವಶುಭಾವಹಾ ।\\
ಐಂ ಭಾರತೀ ಘ್ರಾಣಯುಗ್ಮಂ ಪಾತು ಸರ್ವಜಯಾಪ್ತಯೇ ॥೬॥

ಸೌಃ ಕರಾಲೀ ಮುಖಂ ಪಾತು ಸರ್ವಲೋಕವಶಾಪ್ತಯೇ ।\\
ಐಂ ಶಾರದಾ ಸದಾ ವಾಚಂ ಪಾತು ಸಾಹಿತ್ಯಸಿದ್ಧಯೇ ॥೭॥

ಓಂ ಕಪಾಲಿನೀ ಮೇ ಕರ್ಣೌ ಪಾತು ಸದ್ಗಾನಸಿದ್ಧಯೇ ।\\
ಹ್ರೀಂ ದುರ್ಗಾ ಸಹಿತಾ ಪಾತು ಸ್ಕಂಧದೇಶೌ ಸದಾ ಮಮ ॥೮॥

ಶ್ರೀಂ ಕ್ಷಮಸಹಿತಾ ಪಾತು ಹೃದಯಂ ಮಮ ಸರ್ವದಾ ।\\
ಕಕಾರಸಹಿತಾ ಧಾತ್ರೀ ಪಾರ್ಶ್ವಯುಗ್ಮಂ ಸದಾಽವತು ॥೯॥

ಏಕಾರಸಹಿತಾ ಸ್ವಾಹಾ ಪಾತು ಮೇ ಜಠರಂ ಸದಾ ।\\
ಈಕಾರಸಹಿತಾ ನಾಭಿಂ ಸ್ವಧಾ ಮಾಂ ಸರ್ವದಾಽವತು ॥೧೦॥

ಲಕಾರಸಹಿತಾ ಬ್ರಾಹ್ಮೀ ಪೃಷ್ಠದೇಶಂ ಸದಾಽವತು ।\\
ಹ್ರೀಂಕಾರಸಹಮಾಹೇಶೀ ಕಟಿಂ ಪಾತು ಸದಾ ಮಮ ॥೧೧॥

ಹಕಾರಸಹಿತಾ ಗುಹ್ಯಂ ಕೌಮಾರೀ ಪಾತು ಸರ್ವದಾ ।\\
ಸಕಾರಸಹಿತಾ ಪಾತು ವೈಷ್ಣವೀ ಗುದದೇಶಕಂ ॥೧೨॥

ಕಕಾರಯುಕ್ತಾವಾರಾಹೀ ಹ್ಯೂರುಯುಗ್ಮಂ ಸದಾಽವತು ।\\
ಹಕಾರಸಹಿತಾ ಜಾನುಯುಗ್ಮಂ ಮಾಹೇಂದ್ರೀ ಮೇಽವತು ॥೧೩॥

ಲಕಾರಯುಕ್ತಾ ಚಾಮುಂಡಾ ಜಂಘಾಯುಗ್ಮಂ ಸದಾಽವತು ।\\
ಹ್ರೀಂಕಾರಸಹಿತಾ ಗುಲ್ಫಯುಗ್ಮಂ ಲಕ್ಷ್ಮೀಃ ಸದಾಽವತು ॥೧೪॥

ಸಕಾರಯುಕ್ತಾ ಮೇ ಪಾದಯುಗ್ಮೇಽವ್ಯಾತ್ ಶಿವದೂತಿಕಾ ।\\
ಕಕಾರಸಹಿತಾ ಪ್ರಾಚ್ಯಾಂ ಚಂಡಾ ರಕ್ಷತು ಸರ್ವದಾ ॥೧೫॥

ಲಕಾರಸಹಿತಾಗ್ನೇಯಾಂ ಪ್ರಚಂಡಾ ಸರ್ವದಾಽವತು ।\\
ಹ್ರೀಂಕಾರಸಹಿತಾ ಪಾತು ದಕ್ಷಿಣೇ ಚಂಡನಾಯಿಕಾ ॥೧೬॥

ಸೌಃಕಾರಸಹಿತಾ ಚಂಡವೇಗಿನೀ ನೈರೃತೇಽವತು ।\\
ಐಂಕಾರಸಂಯುತಾ ಚಂಡಪ್ರಕಾಶಾ ಪಾತು ಪಶ್ಚಿಮೇ ॥೧೭॥

ಕ್ಲೀಂಕಾರಸಹಿತಾ ಪಾತು ಚಂಡಿಕಾ ವಾಯುಗೋಚರೇ ।\\
ಹ್ರೀಂಕಾರಸಹಿತಾ ಪಾತು ಚಾಮುಂಡಾ ಚೋತ್ತರೇ ಮಮ ॥೧೮॥

ಶ್ರೀಂಕಾರಸಹಿತಾ ರೌದ್ರೀ ಪಾಯಾದೈಶಾನ್ಯಕೇ ಮಮ ।\\
ಊರ್ಧ್ವಂ ಬ್ರಹ್ಮಾಣೀ ಮೇ ರಕ್ಷೇತ್ ಷೋಡಶೀಸಹ ಸರ್ವದಾ ॥೧೯॥

ಅಧಸ್ತಾದ್ವೈಷ್ಣವೀ ರಕ್ಷೇತ್ ಪುನಃ ಷೋಡಶೀಸಂಯುತಾ ।\\
ಸರ್ವಾಂಗಂ ಸರ್ವದಾ ಪಾತು ಸಹಿತಾ ಭುವನೇಶ್ವರೀ ॥೨೦॥

ಜಲೇ ದಾವಾನಲೇಽರಣ್ಯೇ ಮಹೋತ್ಪಾತೇ ಚ ಸಾಗರೇ ।\\
ದಿವಾರಾತ್ರೌ ಚ ಮೇ ರಕ್ಷೇದ್ ದೇವೀ ತಾರತ್ರಯೀ ಮಮ ॥೨೧॥

॥ಫಲಶ್ರುತಿ ॥

ಇದಂ ತುರೀಯಾಸಹಿತಂ ಷೋಡಶಾಕ್ಷರಿಕಾತ್ಮಕಂ ।\\
ಅಭೇದ್ಯಂ ಕವಚಂ ತ್ವೇದಂ ಮಂತ್ರಬೀಜಸಮನ್ವಿತಂ ॥೧॥

ಯೋಗಿನೀಚಕ್ರಸಹಿತಂ ತವ ಪ್ರೀತ್ಯಾ ಪ್ರಕಾಶಿತಂ ।\\
ಧಾರಯಸ್ವ ಮಯಾ ದತ್ತಂ ಗೋಪನೀಯಂ ಸುಪುತ್ರಕ ॥೨॥

ನ ಪುತ್ರಾಯ ನ ಶಿಷ್ಯಾಯ ಬಂಧುಭ್ಯೋ ನ ಪ್ರಕಾಶಯೇತ್ ।\\
ಇದಂ ತ್ರಿಪುರಸುಂದರ್ಯಾಸ್ತುರೀಯಂ ಕವಚಂ ಶುಭಂ ॥೩॥

ಗೋಪನೀಯಂ ಪ್ರಯತ್ನೇನ ಮಂತ್ರವರ್ಣಕ್ರಮೋದಿತಂ ।\\
ಪ್ರಾತರಾರಭ್ಯ ಸಾಯಾಂತಂ ಕರ್ಮವೇದಾಂತಮೋಕ್ತಿಕಂ ॥೪॥

ತತ್ಫಲಂ ಸಮವಾಪ್ನೋತಿ ತುರೀಯಕವಚವ್ರತಂ ।\\
ದಶಧಾ ಮಾತೃಕಾನ್ಯಾಸಂ ಲಘುಷೋಢಾ ತತಃ ಪರಂ ॥೫॥

ಶಕ್ತಿನ್ಯಾಸಂ ಮಹಾಷೋಢಾಂ ಕೃತ್ವಾ ಬಾಹ್ಯಾಂತರಂ ನ್ಯಸೇತ್ ।\\
ಶ್ರೀವಿದ್ಯಾಯಾಂ ಮಹಾನ್ಯಾಸಂ ಕ್ರಮಾತ್ ಸಾವರ್ಣತಾಂ ವ್ರಜೇತ್ ॥೬॥

ಪೂಜಾಂತೇ ಯತ್ಫಲಂ ಪ್ರಾಪ್ತಂ ತತ್ಫಲಂ ಕವಚವ್ರತೇ ।\\
ಸವತ್ಸಾಂ ದುಗ್ಧಸಹಿತಾಂ ಸಾಧಕಃ ಕಾಮಧೇನುವತ್ ॥೭॥

ತ್ರೈಲೋಕ್ಯವಿಜಯಾಯೇದಂ ಕವಚಂ ಪರಮಾದ್ಭುತಂ ।\\
ಯಥಾ ಚಿಂತಾಮಣೌ ಪುತ್ರ ! ಮನಸಾ ಪರಿಕಲ್ಪಿತೇ ॥೮॥

ತತ್ಸರ್ವಂ ಲಭತೇ ಶೀಘ್ರಂ ಮಮ ವಾಕ್ಯಾನ್ನ ಸಂಶಯಃ ।\\
ಸಾಯುರಾರೋಗ್ಯಮೈಶ್ವರ್ಯಂ ಸದಾ ಸಂಪತ್ಪ್ರವರ್ಧನಂ ॥೯॥

ಕವಚಸ್ಯ ಪ್ರಭಾವೇಣ ತ್ರೈಲೋಕ್ಯವಿಜಯೀ ಭವೇತ್ ।\\
ಅದೀಕ್ಷಿತಾಯ ನ ದೇಯಂ ಶ್ರದ್ಧಾವಿರಹಿತಾತ್ಮನೇ ॥೧೦॥

ನಾಖ್ಯೇಯಂ ಯಸ್ಯ ಕಸ್ಯಾಪಿ ಕೃತಘ್ನಾಯಾತತಾಯಿನೇ ।\\
ಶಾಂತಾಯ ಗುರುಭಕ್ತಾಯ ದೇಯಂ ಶುದ್ಧಾಯ ಸಾಧನೇ ॥೧೧॥

ಅಜ್ಞಾತ್ವಾ ಕವಚಂ ಚೇದಂ ಯೋ ಜಪೇತ್ ಪರದೇವತಾಂ ।\\
ಸಿದ್ಧಿರ್ನ ಜಾಯತೇ ವತ್ಸ ! ಕಲ್ಪಕೋಟಿಶತೈರಪಿ ॥೧೨॥

ಸ ಏವ ಚ ಗುರು ಸಾಕ್ಷಾತ್ ಕವಚಂ ಯಸ್ತು ಪುತ್ರಕ ।\\
ತ್ರಿಸಂಧ್ಯಂ ಚ ಪಠೇನ್ನಿತ್ಯಮಿದಂ ಕವಚಮುತ್ತಮಂ ॥೧೩॥

ನಿಶಾರ್ಧೇ ಜಪಕಾಲೇ ವಾ ಪ್ರತ್ಯಹಂ ಯಂತ್ರಮಗ್ರತಃ ।\\
ಜಗದ್ವಶ್ಯಂ ಭವೇಚ್ಛೀಘ್ರಂ ನಾತ್ರ ಕಾರ್ಯಾ ವಿಚಾರಣಾ ॥೧೪॥

ಸಪ್ತಕೋಟಿಮಹಾಮಂತ್ರಾಃ ಸವರ್ಣಾಃ ಸಗುಣಾನ್ವಿತಾಃ ।\\
ಸರ್ವೇ ಪ್ರಸನ್ನತಾಂ ಯಾಂತಿ ಸತ್ಯಂ ಸತ್ಯಂ ನ ಸಂಶಯಃ ॥೧೫॥

ಇತಿ ತೇ ಕಥಿತಂ ದಿವ್ಯಂ ಸಗುಣೇ ಭಜನಕ್ರಮಂ ।\\
ನಿರ್ಗುಣಂ ಪರಮಂ ವಕ್ಷ್ಯೇ ತುರೀಯಂ ಕವಚಂ ಶೃಣು ॥೧೬॥

ಕವಚಸ್ಯಾಸ್ಯ ಮಾಹಾತ್ಮ್ಯಂ ವರ್ಣಿತುಂ ನೈವ ಶಕ್ಯತೇ ।\\
ಮೂಲಾದಿಬ್ರಹ್ಮರಂಧ್ರಾಂತಂ ಶ್ರೀಚಕ್ರಂ ಸಮುದೀರಯೇತ್ ॥೧೭॥

ದೇಹಮಧ್ಯೇ ಚ ಸರ್ವಸ್ವಂ ಶ್ರೀಚಕ್ರಂ ಚಿಂತಯೇತ್ ಸುತ ।\\
ಪಂಚವಿಂಶತಿತತ್ತ್ವಂ ಚ ಅತಲಂ ವಿತಲಂ ತಥಾ ॥೧೮॥

ಸುತಲಂ ಚ ತಲಾತಲಂ ಮಹಾತಲಂ ಚ ಪಂಚಮಂ ।\\
ರಸಾತಲಂ ಷಷ್ಠಂ ವಕ್ಷ್ಯೇ ಸಪ್ತಮಂ ಪಾತಾಲಲೋಕಂ ॥೧೯॥

ಭೂರ್ಭುವಃ ಸ್ವರ್ಲೋಕಮತೋ ಮಹಲ್ಲೋಕಜನಸ್ತಥಾ ।\\
ತಪಶ್ಚ ಸತ್ಯಲೋಕಶ್ಚ ಭುವನಾನಿ ಚತುರ್ದಶ ॥೨೦॥

ಸರ್ವಂ ಶ್ರೀಭುವನಂ ಚೈವ ನಿರಾಕಾರಂ ವಿಚಿಂತಯೇತ್ ।\\
ಮಾನಸೇ ಪೂಜಯೇತ್ ಧ್ಯಾಯೇಜ್ಜ್ಯೋತಿರೂಪಂ ಸುಚಿನ್ಮಯಂ ।\\
ಕವಚಂ ಪ್ರಜಪೇದ್ ವತ್ಸ ! ರಾಜರಾಜೇಶ್ವರೋ ಭವೇತ್ ॥೨೧॥

ಇತಿ ಪರಮರಹಸ್ಯಂ ಸರ್ವಮಂತ್ರಾರ್ಥಸಾರಂ\\
ಭಜತಿ ಪರಮಭಕ್ತ್ಯಾ ನಿಶ್ಚಲಂ ನಿರ್ಮಲತ್ವಂ ।\\
ವಿಲಸಿತ ಭುವಿ ಮಧ್ಯೇ ಪುತ್ರಪೌತ್ರಾಭಿವೃಧ್ದಿಂ\\
ಧನಸಕಲಸಮೃಧ್ದಿಂ ಭೋಗಮೋಕ್ಷಪ್ರದಂ ಚ ॥೨೨॥

\authorline{॥ಇತಿ ಶ್ರೀಚೂಡಾಮಣೌ ಶ್ರೀಶಿವಸ್ಕಂದಸಂವಾದೇ ತ್ರೈಲೋಕ್ಯವಿಜಯಂ ನಾಮ ಶ್ರೀತುರೀಯಾಷೋಡಶೀ ಶ್ರೀರಾಜರಾಜೇಶ್ವರೀ ಮಹಾತ್ರಿಪುರಸುಂದರೀಕವಚಂ ಸಂಪೂರ್ಣಂ ॥}
%=========================================================================================================================
\section{ಶ್ರೀಮಹಾಲಕ್ಷ್ಮೀಕವಚಂ}
\addcontentsline{toc}{section}{ಶ್ರೀಮಹಾಲಕ್ಷ್ಮೀಕವಚಂ}


ಶ್ರೀ ಗಣೇಶಾಯ ನಮಃ ।\\
\addcontentsline{toc}{section}{ಶ್ರೀಮಹಾಲಕ್ಷ್ಮೀಕವಚಂ}
ಅಸ್ಯ ಶ್ರೀಮಹಾಲಕ್ಷ್ಮೀಕವಚಮಂತ್ರಸ್ಯ ಬ್ರಹ್ಮಾ ಋಷಿಃ । ಗಾಯತ್ರೀ ಛಂದಃ । ಮಹಾಲಕ್ಷ್ಮೀರ್ದೇವತಾ । ಮಹಾಲಕ್ಷ್ಮೀಪ್ರೀತ್ಯರ್ಥಂ ಜಪೇ ವಿನಿಯೋಗಃ ।

ಇಂದ್ರ ಉವಾಚ ।\\
ಸಮಸ್ತಕವಚಾನಾಂ ತು ತೇಜಸ್ವಿ ಕವಚೋತ್ತಮಂ ।\\
ಆತ್ಮರಕ್ಷಣಮಾರೋಗ್ಯಂ ಸತ್ಯಂ ತ್ವಂ ಬ್ರೂಹಿ ಗೀಷ್ಪತೇ ॥೧॥

ಶ್ರೀಗುರುರುವಾಚ ।\\
ಮಹಾಲಕ್ಷ್ಮ್ಯಾಸ್ತು ಕವಚಂ ಪ್ರವಕ್ಷ್ಯಾಮಿ ಸಮಾಸತಃ ।\\
ಚತುರ್ದಶಸು ಲೋಕೇಷು ರಹಸ್ಯಂ ಬ್ರಹ್ಮಣೋದಿತಂ ॥೨॥

ಬ್ರಹ್ಮೋವಾಚ ।\\
ಶಿರೋ ಮೇ ವಿಷ್ಣುಪತ್ನೀ ಚ ಲಲಾಟಮಮೃತೋದ್ಭವಾ ।\\
ಚಕ್ಷುಷೀ ಸುವಿಶಾಲಾಕ್ಷೀ ಶ್ರವಣೇ ಸಾಗರಾಂಬುಜಾ ॥೩॥

ಘ್ರಾಣಂ ಪಾತು ವರಾರೋಹಾ ಜಿಹ್ವಾಮಾಮ್ನಾಯರೂಪಿಣೀ ।\\
ಮುಖಂ ಪಾತು ಮಹಾಲಕ್ಷ್ಮೀಃ ಕಂಠಂ ವೈಕುಂಠವಾಸಿನೀ ॥೪॥

ಸ್ಕಂಧೌ ಮೇ ಜಾನಕೀ ಪಾತು ಭುಜೌ ಭಾರ್ಗವನಂದಿನೀ ।\\
ಬಾಹೂ ದ್ವೌ ದ್ರವಿಣೀ ಪಾತು ಕರೌ ಹರಿವರಾಂಗನಾ ॥೫॥

ವಕ್ಷಃ ಪಾತು ಚ ಶ್ರೀರ್ದೇವೀ ಹೃದಯಂ ಹರಿಸುಂದರೀ ।\\
ಕುಕ್ಷಿಂ ಚ ವೈಷ್ಣವೀ ಪಾತು ನಾಭಿಂ ಭುವನಮಾತೃಕಾ ॥೬॥

ಕಟಿಂ ಚ ಪಾತು ವಾರಾಹೀ ಸಕ್ಥಿನೀ ದೇವದೇವತಾ ।\\
ಊರೂ ನಾರಾಯಣೀ ಪಾತು ಜಾನುನೀ ಚಂದ್ರಸೋದರೀ ॥೭॥

ಇಂದಿರಾ ಪಾತು ಜಂಘೇ ಮೇ ಪಾದೌ ಭಕ್ತನಮಸ್ಕೃತಾ ।\\
ನಖಾನ್ ತೇಜಸ್ವಿನೀ ಪಾತು ಸರ್ವಾಂಗಂ ಕರುಣಾಮಯೀ ॥೮॥

ಬ್ರಹ್ಮಣಾ ಲೋಕರಕ್ಷಾರ್ಥಂ ನಿರ್ಮಿತಂ ಕವಚಂ ಶ್ರಿಯಃ ।\\
ಯೇ ಪಠಂತಿ ಮಹಾತ್ಮಾನಸ್ತೇ ಚ ಧನ್ಯಾ ಜಗತ್ತ್ರಯೇ ॥೯॥

ಕವಚೇನಾವೃತಾಂಗಾನಾಂ ಜನಾನಾಂ ಜಯದಾ ಸದಾ ।\\
ಮಾತೇವ ಸರ್ವಸುಖದಾ ಭವ ತ್ವಮಮರೇಶ್ವರೀ ॥೧೦॥

ಭೂಯಃ ಸಿದ್ಧಿಮವಾಪ್ನೋತಿ ಪೂರ್ವೋಕ್ತಂ ಬ್ರಹ್ಮಣಾ ಸ್ವಯಂ ।\\
ಲಕ್ಷ್ಮೀರ್ಹರಿಪ್ರಿಯಾ ಪದ್ಮಾ ಏತನ್ನಾಮತ್ರಯಂ ಸ್ಮರನ್ ॥೧೧॥

ನಾಮತ್ರಯಮಿದಂ ಜಪ್ತ್ವಾ ಸ ಯಾತಿ ಪರಮಾಂ ಶ್ರಿಯಂ ।\\
ಯಃ ಪಠೇತ್ಸ ಚ ಧರ್ಮಾತ್ಮಾ ಸರ್ವಾನ್ಕಾಮಾನವಾಪ್ನುಯಾತ್ ॥೧೨॥
\authorline{ಇತಿ ಶ್ರೀಬ್ರಹ್ಮಪುರಾಣೇ ಇಂದ್ರೋಪದಿಷ್ಟಂ ಮಹಾಲಕ್ಷ್ಮೀಕವಚಂ ಸಂಪೂರ್ಣಂ ॥}
%==================================================================
\section{ಶ್ರೀಲಕ್ಷ್ಮೀಕವಚಂ}
\addcontentsline{toc}{section}{ಶ್ರೀಲಕ್ಷ್ಮೀಕವಚಂ}

ಶುಕಂ ಪ್ರತಿ ಬ್ರಹ್ಮೋವಾಚ ।\\
ಮಹಾಲಕ್ಷ್ಮ್ಯಾಃ ಪ್ರವಕ್ಷ್ಯಾಮಿ ಕವಚಂ ಸರ್ವಕಾಮದಂ ।\\
ಸರ್ವಪಾಪಪ್ರಶಮನಂ ದುಷ್ಟವ್ಯಾಧಿವಿನಾಶನಂ ॥೧॥

ಗ್ರಹಪೀಡಾಪ್ರಶಮನಂ ಗ್ರಹಾರಿಷ್ಟಪ್ರಭಂಜನಂ ।\\
ದುಷ್ಟಮೃತ್ಯುಪ್ರಶಮನಂ ದುಷ್ಟದಾರಿದ್ರ್ಯನಾಶನಂ ॥೨॥

ಪುತ್ರಪೌತ್ರಪ್ರಜನನಂ ವಿವಾಹಪ್ರದಮಿಷ್ಟದಂ ।\\
ಚೋರಾರಿಹಂ ಚ ಜಪತಾಂ ಅಖಿಲೇಪ್ಸಿತದಾಯಕಂ ॥೩॥

ಸಾವಧಾನಮನಾ ಭೂತ್ವಾ ಶ್ರುಣು ತ್ವಂ ಶುಕ ಸತ್ತಮ ।\\
ಅನೇಕಜನ್ಮಸಂಸಿದ್ಧಿಲಭ್ಯಂ ಮುಕ್ತಿಫಲಪ್ರದಂ ॥೪॥

ಧನಧಾನ್ಯಮಹಾರಾಜ್ಯಸರ್ವಸೌಭಾಗ್ಯಕಲ್ಪಕಂ ।\\
ಸಕೃತ್ಸ್ಮರಣಮಾತ್ರೇಣ ಮಹಾಲಕ್ಷ್ಮೀಃ ಪ್ರಸೀದತಿ ॥೫॥

ಕ್ಷೀರಾಬ್ಧಿಮಧ್ಯೇ ಪದ್ಮಾನಾಂ ಕಾನನೇ ಮಣಿಮಂಟಪೇ ।\\
ತನ್ಮಧ್ಯೇ ಸುಸ್ಥಿತಾಂ ದೇವೀಂ ಮನೀಷಾಜನಸೇವಿತಾಂ ॥೬॥

ಸುಸ್ನಾತಾಂ ಪುಷ್ಪಸುರಭಿಕುಟಿಲಾಲಕಬಂಧನಾಂ ।\\
ಪೂರ್ಣೇಂದುಬಿಂಬವದನಾಂ ಅರ್ಧಚಂದ್ರಲಲಾಟಿಕಾಂ ॥೭॥

ಇಂದೀವರೇಕ್ಷಣಾಂ ಕಾಮಕೋದಂಡಭ್ರುವಮೀಶ್ವರೀಂ ।\\
ತಿಲಪ್ರಸವಸಂಸ್ಪರ್ಧಿನಾಸಿಕಾಲಂಕೃತಾಂ ಶ್ರಿಯಂ ॥೮॥

ಕುಂದಕುಡ್ಮಲದಂತಾಲಿಂ ಬಂಧೂಕಾಧರಪಲ್ಲವಾಂ ।\\
ದರ್ಪಣಾಕಾರವಿಮಲಕಪೋಲದ್ವಿತಯೋಜ್ಜ್ವಲಾಂ ॥೯॥

ರತ್ನತಾಟಂಕಕಲಿತಕರ್ಣದ್ವಿತಯಸುಂದರಾಂ ।\\
ಮಾಂಗಲ್ಯಾಭರಣೋಪೇತಾಂ ಕಂಬುಕಂಠೀಂ ಜಗತ್ಪ್ರಿಯಾಂ ॥೧೦॥

ತಾರಹಾರಿಮನೋಹಾರಿಕುಚಕುಂಭವಿಭೂಷಿತಾಂ ।\\
ರತ್ನಾಂಗದಾದಿಲಲಿತಕರಪದ್ಮಚತುಷ್ಟಯಾಂ ॥೧೧॥

ಕಮಲೇ ಚ ಸುಪತ್ರಾಢ್ಯೇ ಹ್ಯಭಯಂ ದಧತೀಂ ವರಂ ।\\
ರೋಮರಾಜಿಕಲಾಚಾರುಭುಗ್ನನಾಭಿತಲೋದರೀಂ ॥೧೨॥

ಪತ್ತವಸ್ತ್ರಸಮುದ್ಭಾಸಿಸುನಿತಂಬಾದಿಲಕ್ಷಣಾಂ ।\\
ಕಾಂಚನಸ್ತಂಭವಿಭ್ರಾಜದ್ವರಜಾನೂರುಶೋಭಿತಾಂ ॥೧೩॥

ಸ್ಮರಕಾಹ್ಲಿಕಾಗರ್ವಹಾರಿಜಂಭಾಂ ಹರಿಪ್ರಿಯಾಂ ।\\
ಕಮಠೀಪೃಷ್ಠಸದೃಶಪಾದಾಬ್ಜಾಂ ಚಂದ್ರಸನ್ನಿಭಾಂ ॥೧೪॥

ಪಂಕಜೋದರಲಾವಣ್ಯಸುಂದರಾಂಘ್ರಿತಲಾಂ ಶ್ರಿಯಂ ।\\
ಸರ್ವಾಭರಣಸಂಯುಕ್ತಾಂ ಸರ್ವಲಕ್ಷಣಲಕ್ಷಿತಾಂ ॥೧೫॥

ಪಿತಾಮಹಮಹಾಪ್ರೀತಾಂ ನಿತ್ಯತೃಪ್ತಾಂ ಹರಿಪ್ರಿಯಾಂ ।\\
ನಿತ್ಯಂ ಕಾರುಣ್ಯಲಲಿತಾಂ ಕಸ್ತೂರೀಲೇಪಿತಾಂಗಿಕಾಂ ॥೧೬॥

ಸರ್ವಮಂತ್ರಮಯೀಂ ಲಕ್ಷ್ಮೀಂ ಶ್ರುತಿಶಾಸ್ತ್ರಸ್ವರೂಪಿಣೀಂ ।\\
ಪರಬ್ರಹ್ಮಮಯಾಂ ದೇವೀಂ ಪದ್ಮನಾಭಕುಟುಂಬಿನೀಂ ।\\
ಏವಂ ಧ್ಯಾತ್ವಾ ಮಹಾಲಕ್ಷ್ಮೀಂ ಪಠೇತ್ ತತ್ಕವಚಂ ಪರಂ ॥೧೭॥

\as{ಏಕಂ ನ್ಯಂಚ್ಯನತಿಕ್ಷಮಂ ಮಮಪರಂ ಚಾಕುಂಚ್ಯಪದಾಂಬುಜಂ\\
ಮಧ್ಯೇ ವಿಷ್ಟರಪುಂಡರೀಕಮಭಯಂ ವಿನ್ಯಸ್ತಹಸ್ತಾಂಬುಜಂ ।\\
ತ್ವಾಂ ಪಶ್ಯೇಮ ನಿಷೇದುಷೀಮನುಕಲಂಕಾರುಣ್ಯಕೂಲಂಕಷ\\
ಸ್ಫಾರಾಪಾಂಗತರಂಗಮಂಬ ಮಧುರಂ ಮುಗ್ಧಂ ಮುಖಂ ಬಿಭ್ರತೀಂ} ॥೧೮॥

ಮಹಾಲಕ್ಷ್ಮೀಃ ಶಿರಃ ಪಾತು ಲಲಾಟಂ ಮಮ ಪಂಕಜಾ ।\\
ಕರ್ಣೇ ರಕ್ಷೇದ್ರಮಾ ಪಾತು ನಯನೇ ನಲಿನಾಲಯಾ ॥೧೯॥

ನಾಸಿಕಾಮವತಾದಂಬಾ ವಾಚಂ ವಾಗ್ರೂಪಿಣೀ ಮಮ ।\\
ದಂತಾನವತು ಜಿಹ್ವಾಂ ಶ್ರೀರಧರೋಷ್ಠಂ ಹರಿಪ್ರಿಯಾ ॥೨೦॥

ಚುಬುಕಂ ಪಾತು ವರದಾ ಗಲಂ ಗಂಧರ್ವಸೇವಿತಾ ।\\
ವಕ್ಷಃ ಕುಕ್ಷಿಂ ಕರೌ ಪಾಯೂಂ ಪೃಷ್ಠಮವ್ಯಾದ್ರಮಾ ಸ್ವಯಂ ॥೨೧॥

ಕಟಿಮೂರುದ್ವಯಂ ಜಾನು ಜಘಂ ಪಾತು ರಮಾ ಮಮ ।\\
ಸರ್ವಾಂಗಮಿಂದ್ರಿಯಂ ಪ್ರಾಣಾನ್ ಪಾಯಾದಾಯಾಸಹಾರಿಣೀ ॥೨೨॥

ಸಪ್ತಧಾತೂನ್ ಸ್ವಯಂ ಚಾಪಿ ರಕ್ತಂ ಶುಕ್ರಂ ಮನೋ ಮಮ ।\\
ಜ್ಞಾನಂ ಬುದ್ಧಿಂ ಮಹೋತ್ಸಾಹಂ ಸರ್ವಂ ಮೇ ಪಾತು ಪಂಕಜಾ ॥೨೩॥

ಮಯಾ ಕೃತಂ ಚ ಯತ್ಕಿಂಚಿತ್ತತ್ಸರ್ವಂ ಪಾತು ಸೇಂದಿರಾ ।\\
ಮಮಾಯುರವತಾತ್ ಲಕ್ಷ್ಮೀಃ ಭಾರ್ಯಾಂ ಪುತ್ರಾಂಶ್ಚ ಪುತ್ರಿಕಾ ॥೨೪॥

ಮಿತ್ರಾಣಿ ಪಾತು ಸತತಮಖಿಲಾನಿ ಹರಿಪ್ರಿಯಾ ।\\
ಪಾತಕಂ ನಾಶಯೇತ್ ಲಕ್ಷ್ಮೀಃ ಮಹಾರಿಷ್ಟಂ ಹರೇದ್ರಮಾ ॥೨೫॥

ಮಮಾರಿನಾಶನಾರ್ಥಾಯ ಮಾಯಾಮೃತ್ಯುಂ ಜಯೇದ್ಬಲಂ ।\\
ಸರ್ವಾಭೀಷ್ಟಂ ತು ಮೇ ದದ್ಯಾತ್ ಪಾತು ಮಾಂ ಕಮಲಾಲಯಾ॥೨೬॥

ಫಲಶ್ರುತಿಃ ।\\
ಯ ಇದಂ ಕವಚಂ ದಿವ್ಯಂ ರಮಾತ್ಮಾ ಪ್ರಯತಃ ಪಠೇತ್ ।\\
ಸರ್ವಸಿದ್ಧಿಮವಾಪ್ನೋತಿ ಸರ್ವರಕ್ಷಾಂ ತು ಶಾಶ್ವತೀಂ ॥೨೭॥

ದೀರ್ಘಾಯುಷ್ಮಾನ್ ಭವೇನ್ನಿತ್ಯಂ ಸರ್ವಸೌಭಾಗ್ಯಕಲ್ಪಕಂ ।\\
ಸರ್ವಜ್ಞಃ ಸರ್ವದರ್ಶೀ ಚ ಸುಖದಶ್ಚ ಶುಭೋಜ್ಜ್ವಲಃ ॥೨೮॥

ಸುಪುತ್ರೋ ಗೋಪತಿಃ ಶ್ರೀಮಾನ್ ಭವಿಷ್ಯತಿ ನ ಸಂಶಯಃ ।\\
ತದ್ಗೃಹೇ ನ ಭವೇದ್ಬ್ರಹ್ಮನ್ ದಾರಿದ್ರ್ಯದುರಿತಾದಿಕಂ ॥೨೯॥

ನಾಗ್ನಿನಾ ದಹ್ಯತೇ ಗೇಹಂ ನ ಚೋರಾದ್ಯೈಶ್ಚ ಪೀಡ್ಯತೇ ।\\
ಭೂತಪ್ರೇತಪಿಶಾಚಾದ್ಯಾಃ ಸಂತ್ರಸ್ತಾ ಯಾಂತಿ ದೂರತಃ ॥೩೦॥

ಲಿಖಿತ್ವಾ ಸ್ಥಾಪಯೇದ್ಯತ್ರ ತತ್ರ ಸಿದ್ಧಿರ್ಭವೇತ್ ಧ್ರುವಂ ।\\
ನಾಪಮೃತ್ಯುಮವಾಪ್ನೋತಿ ದೇಹಾಂತೇ ಮುಕ್ತಿಭಾಗ್ಭವೇತ್ ॥೩೧॥

ಆಯುಷ್ಯಂ ಪೌಷ್ಟಿಕಂ ಮೇಧ್ಯಂ ಧಾನ್ಯಂ ದುಃಸ್ವಪ್ನನಾಶನಂ ।\\
ಪ್ರಜಾಕರಂ ಪವಿತ್ರಂ ಚ ದುರ್ಭಿಕ್ಷರ್ತಿವಿನಾಶನಂ ॥೩೨॥

ಚಿತ್ತಪ್ರಸಾದಜನನಂ ಮಹಾಮೃತ್ಯುಪ್ರಶಾಂತಿದಂ ।\\
ಮಹಾರೋಗಜ್ವರಹರಂ ಬ್ರಹ್ಮಹತ್ಯಾದಿಶೋಧನಂ ॥೩೩॥

ಮಹಾಧನಪ್ರದಂ ಚೈವ ಪಠಿತವ್ಯಂ ಸುಖಾರ್ಥಿಭಿಃ ।\\
ಧನಾರ್ಥೀ ಧನಮಾಪ್ನೋತಿ ವಿವಹಾರ್ಥೀ ಲಭೇದ್ವಧೂಂ ॥೩೪॥

ವಿದ್ಯಾರ್ಥೀ ಲಭತೇ ವಿದ್ಯಾಂ ಪುತ್ರಾರ್ಥೀ ಗುಣವತ್ಸುತಂ ।\\
ರಾಜ್ಯಾರ್ಥೀ ರಾಜ್ಯಮಾಪ್ನೋತಿ ಸತ್ಯಮುಕ್ತಂ ಮಯಾ ಶುಕ ॥೩೫॥

ಏತದ್ದೇವ್ಯಾಃಪ್ರಸಾದೇನ ಶುಕಃ ಕವಚಮಾಪ್ತವಾನ್ ।\\
ಕವಚಾನುಗ್ರಹೇಣೈವ ಸರ್ವಾನ್ ಕಾಮಾನವಾಪ ಸಃ ॥೩೬॥

\authorline{ಇತಿ ಲಕ್ಷ್ಮೀಕವಚಂ ಬ್ರಹ್ಮಸ್ತೋತ್ರಂ ಸಮಾಪ್ತಂ ।}
%===============================================================================================
\section{ಸ್ತ್ರೀಗುರು ಧ್ಯಾನಸ್ತುತೀಕವಚಂ}
\addcontentsline{toc}{section}{ಸ್ತ್ರೀಗುರು ಧ್ಯಾನಸ್ತುತೀಕವಚಂ}


ಪಾರ್ವತ್ಯುವಾಚ ॥\\
ಗುರುಧ್ಯಾನಂ ಶ್ರುತಂ ನಾಥ ! ಸರ್ವತಂತ್ರೇಷು ಗೋಪಿತಂ ।\\
ಸ್ತ್ರಿಯಾ ದೀಕ್ಷಾ ಶುಭಾ ಪ್ರೋಕ್ತಾ ಸರ್ವಕಾಮಫಲಪ್ರದಾ ॥೧॥

ಬಹುಜನ್ಮಾರ್ಜಿತಾತ್ ಪುಣ್ಯಾತ್ ಬಹುಭಗ್ಯವಶಾದ್ ಯದಿ ।\\
ಸ್ತ್ರೀಗುರುರ್ಲಭ್ಯತೇ ನಾಥ ! ತಸ್ಯಾ ಧ್ಯಾನಂತು ಕೀದೃಶಂ ॥೨॥
ಕುತ್ರ ವಾ ಸಾ ಗುರುರ್ಧ್ಯೇಯಾ ಶ್ರೋತುಮಿಚ್ಛಾಮಿ ಸಾಂಪ್ರತಂ ।\\
ಕಥಯಸ್ವ ಮಹಾದೇವ ! ಯದ್ಯಹಂ ತವ ವಲ್ಲಭಾ ॥೩॥

ಶಿವ ಉವಾಚ ॥
ಶೃಣು ಪಾರ್ವತಿ ! ವಕ್ಷ್ಯಾಮಿ ತವ ಸ್ನೇಹಪರಿಪ್ಲುತಃ ।\\
ರಹಸ್ಯಂ ಸ್ತ್ರೀಗುರೋರ್ಧ್ಯಾನಂ ಯಥಾ ಧ್ಯೇಯಾ ಚ ಸಾ ಗುರುಃ ॥೪॥

ಸಹಸ್ರಾರೇ ಮಹಾಪದ್ಮೇ ಕಿಂಜಲ್ಕಗಣಶೋಭಿತೇ ।\\
ಪ್ರಫುಲ್ಲಪದ್ಮಪತ್ರಾಕ್ಷೀ ಘನಪೀನಪಯೋಧರಾ ॥೫॥

ಪ್ರಸನ್ನವದನಾ ಕ್ಷೀಣಮಧ್ಯಾ ಧ್ಯಾಯೇಚ್ಛಿವಾಂ ಗುರುಂ ।\\
ಪದ್ಮರಾಗಸಮಾಭಾಸಾಂ ರಕ್ತವಸ್ತ್ರಸುಶೋಭನಾಂ ॥೬॥

ರಕ್ತಕಂಕಣಪಾಣಿಂಚ ರಕ್ತನೂಪುರಶೋಭಿತಾಂ ।\\
ಶರದಿಂದುಪ್ರತೀಕಾಶಾಂ ರಕ್ತೋದ್ಭಾಸಿತಕುಂಡಲಾಂ ॥೭॥

ಸ್ವನಾಥವಾಮಭಾಗಸ್ಥಾಂ ವರಾಭಯಕರಾಂಬುಜಾಂ ।\\
ಇತಿ ತೇ ಕಥಿತಂ ದೇವಿ ! ಸ್ತ್ರೀಗುರೋರ್ಧ್ಯಾನಮುತ್ತಮಂ ।\\
ಗೋಪನೀಯಂ ಪ್ರಯತ್ನೇನ ನ ಪ್ರಕಾಶ್ಯಂ ಕದಾಚನ ॥೮॥

ಅಥ ಸ್ತುತಿಃ ॥ಮಾತೃಕಾಭೇದತಂತ್ರೇ ಸಪ್ತಪಟಲೇ ॥

ದೇವ್ಯುವಾಚ ॥
ಸ್ತುತಿಂ ಚ ಕವಚಂ ನಾಥ ! ಶ್ರೋತುಮಿಚ್ಛಾಮಿ ಸಾಂಪ್ರತಂ ।\\
ಶ್ರೀಗುರೋಃ ಕವಚಂ ಪ್ರೋಕ್ತಂ ತ್ವಯಾ ನಾಥ ! ಪುರಾ ಪ್ರಭೋ ! ॥೧॥

ಇದಾನೀಂ ಸ್ತ್ರೀಗುರೋಃ ಸ್ತ್ರೋತ್ರಂ ಕವಚಂ ಮಯಿ ಕಥ್ಯತಾಂ ।\\
ಯಸ್ಯ ವಿಜ್ಞಾನಮಾತ್ರೇಣ ಪುನರ್ಜನ್ಮ ನ ಜಾಯತೇ ॥೨॥

ಶ್ರೀಶಿವ ಉವಾಚ ॥

ಶೃಣು ದೇವಿ ! ಪ್ರವಕ್ಷ್ಯಾಮಿ ಸ್ತೋತ್ರಂ ಪರಮಗೋಪನಂ ।\\
ಯಸ್ಯ ಶ್ರವಣಮಾತ್ರೇಣ ಸಂಸಾರಾನ್ಮುಚ್ಯತೇ ನರಃ ॥೧॥

ನಮಸ್ತೇ ದೇವದೇವೇಶಿ ! ನಮಸ್ತೇ ಹರಪೂಜಿತೇ ! ।\\
ಬ್ರಹ್ಮವಿದ್ಯಾಸ್ವರೂಪಾಯೈ ತಸ್ಯೈ ನಿತ್ಯಂ ನಮೋ ನಮಃ ॥೨॥

ಅಜ್ಞಾನತಿಮಿರಾಂಧಸ್ಯ ಜ್ಞಾನಾಂಜನಶಲಾಕಯಾ ।\\
ಯಯಾ ಚಕ್ಷುರುನ್ಮೀಲಿತಂ ತಸ್ಯೈ ನಿತ್ಯಂ ನಮೋ ನಮಃ ॥೩॥

ಭವಬಂಧನಪಾರಸ್ಯ ತಾರಿಣೀ ಜನನೀ ಪರಾ ।\\
ಜ್ಞಾನದಾ ಮೋಕ್ಷದಾ ನಿತ್ಯಂ ತಸ್ಯೈ ನಿತ್ಯಂ ನಮೋ ನಮಃ ॥೪॥

ಶ್ರೀನಾಥವಾಮಭಾಗಸ್ಥಾ ಸದಯಾ ಸುರಪೂಜಿತಾ ।\\
ಸದಾ ವಿಜ್ಞಾನದಾತ್ರೀ ಚ ತಸ್ಯೈ ನಿತ್ಯಂ ನಮೋ ನಮಃ ॥೫॥

ಸಹಸ್ರಾರೇ ಮಹಾಪದ್ಮೇ ಸದಾನಂದಸ್ವರೂಪಿಣೀ ।\\
ಮಹಾಮೋಕ್ಷಪ್ರದಾ ದೇವೀ ತಸ್ಯೈ ನಿತ್ಯಂ ನಮೋ ನಮಃ ॥೬॥

ಬ್ರಹ್ಮವಿಷ್ಣುಸ್ವರೂಪಾ ಚ ಮಹಾರುದ್ರಸ್ವರೂಪಿಣೀ ।\\
ತ್ರಿಗುಣಾತ್ಮಸ್ವರೂಪಾ ಚ ತಸ್ಯೈ ನಿತ್ಯಂ ನಮೋ ನಮಃ ॥೭॥

ಚಂದ್ರಸೂರ್ಯಾಗ್ನಿರೂಪಾ ಚ ಮದಾಘೂರ್ಣಿತಲೋಚನಾ ।\\
ಸ್ವನಾಥಂಚ ಸಮಾಲಿಂಗ್ಯ ತಸ್ಯೈ ನಿತ್ಯಂ ನಮೋ ನಮಃ ॥೮॥

ಬ್ರಹ್ಮವಿಷ್ಣುಶಿವತ್ವಾದಿಜೀವನ್ಮುಕ್ತಿಪ್ರದಾಯಿನೀ ।\\
ಜ್ಞಾನವಿಜ್ಞಾನಯೋರ್ದಾತ್ರೀ ತಸ್ಯೈ ಶ್ರೀಗುರವೇ ನಮಃ ॥೯॥

ಇದಂ ಸ್ತೋತ್ರಂ ಮಹೇಶಾನಿ ! ಯಃ ಪಠೇದ್ಭಕ್ತಿಸಂಯುತಃ ।\\
ಸ ಸಿದ್ಧಿಂ ಲಭತೇ ನಿತ್ಯಂ ಸತ್ಯಂ ಸತ್ಯಂ ನ ಸಂಶಯಃ ॥೧೦॥

ಪ್ರಾತಃಕಾಲೇ ಪಠೇದ್ ಯಸ್ತು ಗುರುಪೂಜಾಪುರಃಸರಂ ।\\
ಸ ಏವ ಧನ್ಯೋ ಲೋಕೇಶೋ ದೇವೀಪುತ್ರ ಇವ ಕ್ಷಿತೌ ।\\

\authorline{ಇತಿ ಮಾತೃಕಾಭೇದತಂತ್ರೇ ಸ್ತ್ರೀಗುರೋಃ ಸ್ತೋತ್ರಂ ಸಂಪೂರ್ಣಂ ॥೧೧॥}



\section{ಸ್ತ್ರೀಗುರುಕವಚಂ }
\addcontentsline{toc}{section}{ಸ್ತ್ರೀಗುರುಕವಚಂ }

ಸ್ತೋತ್ರಂ ಸಮಾಪ್ತಂ ದೇವೇಶಿ ! ಕವಚಂ ಶೃಣು ಸಾದರಂ ।\\
ಯಸ್ಯ ಸ್ಮರಣಮಾತ್ರೇಣ ವಾಗೀಶಸಮತಾಂ ವ್ರಜೇತ್ ॥

ಸ್ತ್ರೀಗುರೋಃ ಕವಚಸ್ಯಾಸ್ಯ ಸದಾಶಿವ ಋಷಿಃ ಸ್ಮೃತಃ ।\\
ತಾರಾಖ್ಯಾ ದೇವತಾ ಖ್ಯಾತಾ ಚತುರ್ವರ್ಗಫಲಪ್ರದಾ ॥೧॥

ಕ್ಲೀಂ ಬೀಜಂ ಚಕ್ಷುಷೋರ್ಮಧ್ಯೇ ಸರ್ವಾಂಗಂ ಮೇ ಸದಾವತು ।\\
ಐಂ ಬೀಜಂ ಮೇ ಮುಖಂ ಪಾತು ಹ್ರೀಂ ಜಿಹ್ವಾಂ ಪರಿರಕ್ಷತು ॥೨॥

ಶ್ರೀಂ ಬೀಜಂ ಸ್ಕಂಧದೇಶಂ ಮೇ ಹಸಖಫ್ರೇಂ ಭುಜದ್ವಯಂ ।\\
ಹಕಾರಃ ಕಂಠದೇಶಂ ಮೇ ಸಕಾರಃ ಷೋಡಶಂ ದಲಂ ॥೩॥

ಕ್ಷವರ್ಣಸ್ತದಧಃ ಪಾತು ಲಕಾರೋ ಹೃದಯಂ ಮಮ ।\\
ವಕಾರಃ ಪೃಷ್ಠದೇಶಂಚ ರಕಾರೋ ದಕ್ಷಪಾರ್ಶ್ವಕಂ ॥೪॥

ಯೂಂಕಾರೋ ವಾಮಪಾರ್ಶ್ವಂಚ ಸಕಾರೋ ಮೇರುಮೇವ ಚ ।\\
ಹಕಾರೋ ಮೇ ದಕ್ಷಭುಜಂ ಕ್ಷಕಾರೋ ವಾಮಹಸ್ತಕಂ ॥೫॥

ಮಕಾರಶ್ಚಾಂಗುಲೀಂ ಪಾತು ಲಕಾರೋ ಮೇ ನಖಂ ವತು ।\\
ವಕಾರೋ ಮೇ ನಿತಂಬಂಚ ರಕಾರೋ ಜಠರಂ ವತು ॥೬॥

ಷೀಂಕಾರಃ ಪಾದಯುಗಲಂ ಹೇಸೌಃ ಸರ್ವಾಂಗಂ ಮೇಽವತು ।\\
ಹೇಸೌರ್ಲಿಂಗಂಚ ಲೋಮಾನಿ ಕೇಶಂಚ ಪರಿರಕ್ಷತು ॥೭॥

ಐಂ ಬೀಜಂ ಪಾತು ಪೂರ್ವೇ ಮೇ ಹ್ರೀಂ ಬೀಜಂ ದಕ್ಷಿಣೇಽವತು ।\\
ಶ್ರೀಂ ಬೀಜಂ ಪಶ್ಚಿಮೇ ಪಾತು ಉತ್ತರೇ ಭೂತಸಂಭವಂ ॥೮॥

ಶ್ರೀಂ ಪಾತು ಚಾಗ್ನಿಕೋಣೇ ಚ ವೇದಾಖ್ಯಾ ನೈಋತೇಽವತು ।\\
ದೇವಾಂಬಾ ಪಾತು ವಾಯವ್ಯಾಂ ಶಂಭೌ ಶ್ರೀಪಾದುಕಾಂ ತಥಾ ॥೯॥

ಪೂಜಯಾಮಿ ತಥಾ ಚೋರ್ದ್ಧ್ವಂ ನಮಶ್ಚಾಧಃ ಸದಾಽವತು ।\\
ಇತಿ ತೇ ಕಥಿತಂ ಕಾಂತೇ ! ಕವಚಂ ಪರಮಾದ್ಭುತಂ ॥೧೦॥

ಗುರುಮಂತ್ರಂ ಜಪಿತ್ವಾ ತು ಕವಚಂ ಪ್ರಫಠೇದ್ಯದಿ ।\\
ಸಸಿದ್ಧಃ ಸಗಣಃ ಸೋಽಪಿ ಶಿವ ಏವ ನ ಸಂಶಯಃ ॥೧೧॥

ಪೂಜಾಕಾಲೇ ಪಠೇದ್ ಯಸ್ತು ಕವಚಂ ಮಂತ್ರವಿಗ್ರಹಂ ।\\
ಪೂಜಾಫಲಂ ಭವೇತ್ತಸ್ಯ ಸತ್ಯಂ ಸತ್ಯಂ ಸುರೇಶ್ವರಿ ! ॥೧೨॥

ತ್ರಿಸಂಧ್ಯಂ ಯಃ ಪಠೇದ್ದೇವಿ ! ಸ ಸಿದ್ಧೋ ನಾತ್ರ ಸಂಶಯಃ ।\\
ಭೂರ್ಜೇ ವಿಲಿಖ್ಯ ಗುಲಿಕಾಂ ಸ್ವರ್ಣಸ್ಥಾಂ ಧಾರಯೇದ್ ಯದಿ ॥೧೩॥

ತಸ್ಯ ದರ್ಶನಮಾತ್ರೇಣ ವಾದಿನೋ ನಿಷ್ಪ್ರಭಾಂ ಗತಾಃ ।\\
ವಿವಾದೇ ಜಯಮಾಪ್ನೋತಿ ರಣೇ ಚ ನಿರೃತೇಃ ಸಮಃ ॥೧೪॥

ಸಭಾಯಾಂ ಜಯಮಾಪ್ನೋತಿ ಮಮ ತುಲ್ಯೋ ನ ಸಂಶಯಃ ।\\
ಸಹಸ್ರಾರೇ ಭಾವಯನ್ ಯಸ್ತ್ರಿಸಂಧ್ಯಂ ಪ್ರಪಠೇದ್ ಯದಿ ॥೧೫॥

ಸ ಏವ ಸಿದ್ಧೋ ಲೋಕೇಶೋ ನಿರ್ವಾಣಪದಮೀಯತೇ ।\\
ಸಮಸ್ತಮಂಗಲಂ ನಾಮ ಕವಚಂ ಪರಮಾದ್ಭುತಂ ॥೧೬॥

ಯಸ್ಮೈ ಕಸ್ಮೈ ನ ದಾತವ್ಯಂ ನ ಪ್ರಕಾಶ್ಯಂ ಕದಾಚನ ।\\
ದೇಯಂ ಶಿಷ್ಯಾಯ ಶಾಂತಾಯ ಚಾನ್ಯಥಾ ಪತನಂ ಭವೇತ್ ॥೧೭॥

ಅಭಕ್ತೇಭ್ಯಶ್ಚ ದೇವೇಶಿ ! ಪುತ್ರೇಭ್ಯೋಽಪಿ ನ ದರ್ಶಯೇತ್ ।\\
ಇದಂ ಕವಚಮಜ್ಞಾತ್ವಾ ದಶವಿದ್ಯಾಶ್ಚ ಯೋ ಜಪೇತ್ ॥೧೮॥

ಸ ನಾಪ್ನೋತಿ ಫಲಂ ತಸ್ಯ ಚಾಂತೇ ಚ ನರಕಂ ವ್ರಜೇತ್ ।\\
ಸಮಾಪ್ತಂ ಕವಚಂ ದೇವಿ ! ಕಿಮನ್ಯತ್ ಶ್ರೋತುಮಿಚ್ಛಸಿ ।\\
ತವ ಸ್ನೇಹಾನುಬಂಧೇನ ಕಿಂ ಮಯಾ ನ ಪ್ರಕಾಶಿತಂ ॥೧೯॥

\authorline{ಇತಿ ಮಾತೃಕಾಭೇದತಂತ್ರೇ ಸ್ತ್ರೀಗುರುಕವಚಂ ಸಮಾಪ್ತಂ ॥}

\section{ ಸ್ತ್ರೀಗುರುಗೀತಾ }
\addcontentsline{toc}{section}{ ಸ್ತ್ರೀಗುರುಗೀತಾ }

ಪಾರ್ವತ್ಯುವಾಚ ॥\\
ಲೋಕೇಶ ! ಕಥ್ಯತಾಂ ದೇವ ! ಗುರುಗೀತಾ ಮಯಿ ಪ್ರಭೋ ! ।\\

ಶ್ರೀಭಗವಾನುವಾಚ ॥\\
ಶೃಣು ತಾರಿಣಿ ! ವಕ್ಷ್ಯಾಮಿ ಗೀತಾಂ ಬ್ರಹ್ಮಮಯೀಂ ಪರಾಂ ।\\
ಗುರುಸ್ತ್ವಂ ಸರ್ವಶಾಸ್ತ್ರಾಣಾಮಹಮೇವ ಪ್ರಕಾಶಕಃ ॥೧॥

ತ್ವಮೇವ ಗುರುರೂಪೇಣ ಲೋಕಾನಾಂ ತ್ರಾಣಕಾರಿಣೀ ।\\
ಗಯಾ ಗಂಗಾ ಕಾಶಿಕಾ ಚ ತ್ವಮೇವ ಸಕಲಂ ಜಗತ್ ॥೨॥

ಕಾವೇರೀ ಯಮುನಾ ರೇವಾ ಕರತೋಯಾ ಸರಸ್ವತೀ ।\\
ಗೋಮತೀ ಚಂದ್ರಭಾಗಾ ಚ ತ್ವಮೇವ ಕುಲಪಾಲಿಕೇ ! ॥೩॥

ಬ್ರಹ್ಮಾಂಡಂ ಸಕಲಂ ದೇವಿ ! ಕೋಟಿಬ್ರಹ್ಮಾಂಡಮೇವ ಚ ।\\
ನ ಹಿ ತೇ ವಕ್ತುಮರ್ಹಾಮಿ ಕ್ರಿಯಾಜಾಲಂ ಮಹೇಶ್ವರಿ ! ॥೪॥

ಉಕ್ತ್ವಾ ಚೋಕ್ತ್ವಾ ಭಾವಯಿತ್ವಾ ಭಿಕ್ಷುಕೋಽಹಂ ನಗಾತ್ಮಜೇ ! ।\\
ಕಥಂ ತ್ವಂ ಜನನೀ ಭೂತ್ವಾ ಬಧೂಸ್ತ್ವಂ ಮಮ ದೇಹಿನಾಂ ॥೫॥

ತವ ಚಕ್ರಂ ಮಹೇಶಾನಿ ! ಅತೀತಂ ಪರಮಾತ್ಮನಿ ।\\
ಇತಿ ತೇ ಕಥಿತಾ ಗೀತಾ ಗುರುದೇವಸ್ಯ ಬ್ರಹ್ಮಣಃ ॥೬॥

ಸಂಕ್ಷೇಪೇಣ ಮಹೇಶಾನಿ ! ಪ್ರಭುರೇವ ಗುರುಃ ಸ್ವಯಂ ।\\
ಜಗತ್ ಸಮಸ್ತಮಾಸ್ಥಾಯ ಗುರುಸ್ತ್ವೇಕೋ ಹಿ ಕೇವಲಂ ॥೭॥

ಇತಿ ತಂ ತೋಷಯಿತ್ವಾ ಚ ನುತಿಭಿಃ ಸ್ತುತಿಭಿಸ್ತಥಾ ।\\
ನಾನಾವಿಧದ್ರವ್ಯದಾನೈಃ ಸಿದ್ಧಃ ಸ್ಯಾತ್ ಸಾಧಕೋತ್ತಮಃ ॥೮॥

\authorline{ಇತಿ ಸ್ತ್ರೀಗುರುಗೀತಾ ಸಮಾಪ್ತಾ ॥}
%==============================================================================

\section{ಶ್ರೀಹಯಗ್ರೀವಕವಚಂ}
\addcontentsline{toc}{section}{ಶ್ರೀಹಯಗ್ರೀವಕವಚಂ}


ಅಸ್ಯ ಶ್ರೀಹಯಗ್ರೀವಕವಚಮಹಾಮಂತ್ರಸ್ಯ ಹಯಗ್ರೀವ ಋಷಿಃ । ಅನುಷ್ಟುಪ್ ಛಂದಃ । ಶ್ರೀಹಯಗ್ರೀವಃ ಪರಮಾತ್ಮಾ ದೇವತಾ । ಓಂ ಶ್ರೀಂ ವಾಗೀಶ್ವರಾಯ ನಮ ಇತಿ ಬೀಜಂ । ಓಂ ಕ್ಲೀಂ ವಿದ್ಯಾಧರಾಯ ನಮ ಇತಿ ಶಕ್ತಿಃ । ಓಂ ಸೌಂ ವೇದನಿಧಯೇ ನಮೋ ನಮ ಇತಿ ಕೀಲಕಂ । ಮಮ ಶ್ರೀಹಯಗ್ರೀವಪ್ರಸಾದ ಸಿಧ್ಯರ್ಥೇ ಜಪೇ ವಿನಿಯೋಗಃ ॥

\as{ಕಲಶಾಂಬುಧಿಸಂಕಾಶಂ ಕಮಲಾಯತಲೋಚನಂ ।\\
ಕಲಾನಿಧಿಕೃತಾವಾಸಂ ಕರ್ಣಿಕಾಂತರವಾಸಿನಂ ॥೧॥

ಜ್ಞಾನಮುದ್ರಾಕ್ಷವಲಯಶಂಖಚಕ್ರಲಸತ್ಕರಂ ।\\
ಭೂಷಾಕಿರಣಸಂದೋಹವಿರಾಜಿತದಿಗಂತರಂ ॥೨॥

ವಕ್ತ್ರಾಬ್ಜನಿರ್ಗತೋದ್ದಾಮವಾಣೀಸಂತಾನಶೋಭಿತಂ ।\\
ದೇವತಾಸಾರ್ವಭೌಮಂ ತಂ ಧ್ಯಾಯೇ ದಿಷ್ಟಾರ್ಥಸಿದ್ಧಯೇ ॥೩॥}

ಓಂ ಹಯಗ್ರೀವಶ್ಶಿರಃ ಪಾತು ಲಲಾಟಂ ಚಂದ್ರಮಧ್ಯಗಃ ।\\
ಶಾಸ್ತ್ರದೃಷ್ಟಿರ್ದೃಶೌ ಪಾತು ಶಬ್ದಬ್ರಹ್ಮಾತ್ಮಕಶ್ಶ್ರುತೀ ॥೧॥

ಘ್ರಾಣಂ ಗಂಧಾತ್ಮಕಃ ಪಾತು ವದನಂ ಯಜ್ಞಸಂಭವಃ ।\\
ಜಿಹ್ವಾಂ ವಾಗೀಶ್ವರಃ ಪಾತು ಮುಕುಂದೋ ದಂತಸಂಹತೀಃ ॥೨॥

ಓಷ್ಠಂ ಬ್ರಹ್ಮಾತ್ಮಕಃ ಪಾತು ಪಾತು ನಾರಾಯಣೋಽಧರಂ ।\\
ಶಿವಾತ್ಮಾ ಚಿಬುಕಂ ಪಾತು ಕಪೋಲೌ ಕಮಲಾ ಪ್ರಭುಃ ॥೩॥

ವಿದ್ಯಾತ್ಮಾ ಪೀಠಕಂ ಪಾತು ಕಂಠಂ ನಾದಾತ್ಮಕೋ ಮಮ ।\\
ಭುಜೌ ಚತುರ್ಭುಜಃ ಪಾತು ಕರೌ ದೈತ್ಯೇಂದ್ರಮರ್ದನಃ ॥೪॥

ಜ್ಞಾನಾತ್ಮಾ ಹೃದಯಂ ಪಾತು ವಿಶ್ವಾತ್ಮಾ ತು ಕುಚದ್ವಯಂ ।\\
ಮಧ್ಯಮಂ ಪಾತು ಸರ್ವಾತ್ಮಾ ಪಾತು ಪೀತಾಂಬರಃ ಕಟಿಂ ॥೫॥

ಕುಕ್ಷಿಂ ಕುಕ್ಷಿಸ್ಥವಿಶ್ವೋ ಮೇ ಬಲಿಭಂಗೋ ವಲಿತ್ರಯಂ ।\\
ನಾಭಿಂ ಮೇ ಪದ್ಮನಾಭೋಽವ್ಯಾದ್ಗುಹ್ಯಂ ಗುಹ್ಯಾರ್ಥಬೋಧಕೃತ್ ॥೬॥

ಊರೂ ದಾಮೋದರಃ ಪಾತು ಜಾನುನೀ ಮಧುಸೂದನಃ ।\\
ಪಾತು ಜಂಘೇ ಮಹಾವಿಷ್ಣುರ್ಗುಲ್ಭೌ ಪಾತು ಜನಾರ್ದನಃ ॥೭॥

ಪಾದೌ ತ್ರಿವಿಕ್ರಮಃ ಪಾತು ಪಾತು ಪಾದಾಂಗುಲೀರ್ಹರಿಃ ।\\
ಸರ್ವಾಂಗಂ ಸರ್ವಗಃ ಪಾತು ಪಾತು ರೋಮಾಣಿ ಕೇಶವಃ ॥೮॥

ಧಾತುನ್ನಾಡೀಗತಃ ಪಾತು ಭಾರ್ಯಾಂ ಲಕ್ಷ್ಮೀಪತಿರ್ಮಮ ।\\
ಪುತ್ರಾನ್ವಿಶ್ವಕುಟುಂಬೀ ಮೇ ಪಾತು ಬಂಧೂನ್ಸುರೇಶ್ವರಃ ॥೯॥

ಮಿತ್ರಂ ಮಿತ್ರಾತ್ಮಕಃ ಪಾತು ವಹ್ನ್ಯಾತ್ಮಾ ಶತ್ರುಸಂಹತೀಃ ।\\
ಪ್ರಾಣಾನ್ವಾಯ್ವಾತ್ಮಕಃ ಪಾತು ಕ್ಷೇತ್ರಂ ವಿಶ್ವಂಭರಾತ್ಮಕಃ ॥೧೦॥

ವರುಣಾತ್ಮಾ ರಸಾನ್ಪಾತು ವ್ಯೋಮಾತ್ಮಾ ಹೃದ್ಗುಹಾಂತರಂ ।\\
ದಿವಾರಾತ್ರಂ ಹೃಷೀಕೇಶಃ ಪಾತು ಸರ್ವಂ ಜಗದ್ಗುರುಃ ॥೧೧॥

ವಿಷಮೇ ಸಂಕಟೇ ಚೈವ ಪಾತು ಕ್ಷೇಮಂಕರೋ ಮಮ ।\\
ಸಚ್ಚಿದಾನಂದರೂಪೋ ಮೇ ಜ್ಞಾನಂ ರಕ್ಷತು ಸರ್ವದಾ ॥೧೨॥

ಪ್ರಾಚ್ಯಾಂ ರಕ್ಷತು ಸರ್ವಾತ್ಮಾ ಆಗ್ನೇಯ್ಯಾಂ ಜ್ಞಾನದೀಪಕಃ ।\\
ಯಾಮ್ಯಾಂ ಬೋಧಪ್ರದಃ ಪಾತು ನೈರೃತ್ಯಾಂ ಚಿದ್ಧನಪ್ರಭಃ ॥೧೩॥

ವಿದ್ಯಾನಿಧಿಸ್ತು ವಾರುಣ್ಯಾಂ ವಾಯವ್ಯಾಂ ಚಿನ್ಮಯೋಽವತು ।\\
ಕೌಬೇರ್ಯಾಂ ವಿತ್ತದಃ ಪಾತು ಐಶಾನ್ಯಾಂ ಚ ಜಗದ್ಗುರುಃ ॥೧೪॥

ಉರ್ಧ್ವಂ ಪಾತು ಜಗತ್ಸ್ವಾಮೀ ಪಾತ್ವಧಸ್ತಾತ್ಪರಾತ್ಪರಃ ।\\
ರಕ್ಷಾಹೀನಂತು ಯತ್ಸ್ಥಾನಂ ರಕ್ಷತ್ವಖಿಲನಾಯಕಃ ॥೧೪॥

ಏವಂ ನ್ಯಸ್ತಶರೀರೋಽಸೌ ಸಾಕ್ಷಾದ್ವಾಗೀಶ್ವರೋ ಭವೇತ್ ।\\
ಆಯುರಾರೋಗ್ಯಮೈಶ್ವರ್ಯಂ ಸರ್ವಶಾಸ್ತ್ರಪ್ರವಕ್ತೃತಾಂ ॥೧೬॥

ಲಭತೇ ನಾತ್ರ ಸಂದೇಹೋ ಹಯಗ್ರೀವಪ್ರಸಾದತಃ ।\\
ಇತೀದಂ ಕೀರ್ತಿತಂ ದಿವ್ಯಂ ಕವಚಂ ದೇವಪೂಜಿತಂ ॥೧೭॥

\authorline{ಇತಿ ಹಯಗ್ರೀವತಂತ್ರೇ  ಶ್ರೀಹಯಗ್ರೀವಕವಚಂ ಸಂಪೂರ್ಣಂ ॥}
%=========================================================================================================


\section{ಶ್ರೀ ತ್ರೈಲೋಕ್ಯಮೋಹನಕವಚಂ }
\addcontentsline{toc}{section}{ಶ್ರೀ ತ್ರೈಲೋಕ್ಯಮೋಹನಕವಚಂ }
(ಉತ್ಕೀಲನಮಂತ್ರಂ ಪಠಿತ್ವಾ ಕವಚಂ ಪಠೇತ್)\\
ಓಂ ಹಸಕಲಹ್ರೀಂ ಹ್ರೀಂಹ್ರೀಂಹ್ರೀಂ ಸಕಲಹ್ರೀಂ ಕ್ಲೀಂಕ್ಲೀಂಕ್ಲೀಂ  ಶ್ರೀಂಶ್ರೀಂಶ್ರೀಂ ಕ್ರೀಂಕ್ರೀಂಕ್ರೀಂ ರುದ್ರಸೂಚ್ಯಗ್ರೇಣ ಮೂಲವಿದ್ಯಾಶಾಪಂ ಉತ್ಕೀಲಯೋತ್ಕೀಲಯ ಸ್ವಾಹಾ ॥

ಶ್ರೀ ದೇವ್ಯುವಾಚ ॥\\
ದೇವ ದೇವ ಜಗನ್ನಾಥ ಸಚ್ಚಿದಾನಂದವಿಗ್ರಹ~।\\
ಪಂಚಕೃತ್ಯ ಪರೇಶಾನ ಪರಮಾನಂದದಾಯಕ ॥

ಶ್ರೀಮತ್ತ್ರಿಪುರಸುಂದರ್ಯಾ ಯಾ ಯಾ ವಿದ್ಯಾಸ್ತ್ವಯೋದಿತಾಃ~।\\
ಕೃಪಯಾ ಕಥಿತಾಃ ಸರ್ವಾಃ ಶ್ರುತಾಶ್ಚಾಧಿಗತಾ ಮಯಾ ॥

ಪ್ರಾಣನಾಥಾಧುನಾ ಬ್ರೂಹಿ ಕವಚಂ ಮಂತ್ರವಿಗ್ರಹಂ~।\\
ತ್ರೈಲೋಕ್ಯಮೋಹನಂ ಚೇತಿ ನಾಮತಃ ಕಥಿತಂ ಪುರಾ~।\\
ಇದಾನೀಂ ಶ್ರೋತುಮಿಚ್ಛಾಮಿ ಸರ್ವಾರ್ಥಂ ಕವಚಂ ಸ್ಫುಟಂ ॥

ಈಶ್ವರ ಉವಾಚ ॥\\
ಶೃಣು ದೇವಿ ಪ್ರವಕ್ಷ್ಯಾಮಿ ಸುಂದರಿ ಪ್ರಾಣವಲ್ಲಭೇ~।\\
ತ್ರೈಲೋಕ್ಯಮೋಹನಂ ನಾಮ ಸರ್ವವಿದ್ಯೌಘವಿಗ್ರಹಂ ॥

ಯದ್ಧೃತ್ವಾ ದಾನವಾನ್ ವಿಷ್ಣುಃ ನಿಜಘಾನ ಮುಹುರ್ಮುಹುಃ~।\\
ಸೃಷ್ಟಿಂ ವಿತನುತೇ ಬ್ರಹ್ಮಾ ಯದ್ಧೃತ್ವಾ ಪಠನಾದ್ಯತಃ ॥

ಸಂಹರ್ತಾಹಂ ಯತೋ ದೇವಿ ದೇವೇಶೋ ವಾಸವೋ ಯತಃ~।\\
ಧನಾಧಿಪಃ ಕುಬೇರೋಽಪಿ ಯತಃ ಸರ್ವೇ ದಿಗೀಶ್ವರಾಃ ॥

ನ ದೇಯಂ ಯದಶಿಷ್ಯೇಭ್ಯೋ ದೇಯಂ ಶಿಷ್ಯೇಭ್ಯ ಏವ ಚ~।\\
ಅಭಕ್ತೇಭ್ಯೋಽಪಿ ಪುತ್ರೇಭ್ಯೋ ದತ್ವಾ ಮೃತ್ಯುಮವಾಪ್ನುಯಾತ್ ॥

ಶ್ರೀಮತ್ತ್ರಿಪುರಸುಂದರ್ಯಾಃ ಕವಚಸ್ಯ ಋಷಿಃ ಶಿವಃ~।\\
ಛಂದೋ ವಿರಾಡ್ ದೇವತಾ ಚ ಶ್ರೀಮತ್ತ್ರಿಪುರಸುಂದರೀ~।\\
ಧರ್ಮಾರ್ಥಕಾಮಮೋಕ್ಷೇಷು ವಿನಿಯೋಗಃ ಪ್ರಕೀರ್ತಿತಃ ॥\\

({\bfseries ಓಂಐಂಹ್ರೀಂಶ್ರೀಂ}) ಶಿರೋ ಮೇ ವಾಗ್ಭವಂ ಪಾತು ಕಏಈಲಹ್ರೀಂ ಸ್ವರೂಪಕಂ।\\
ಹಸಕಲಹ್ರೀಂ ಲಲಾಟಂ ಚ ಪಾತು ಕಾಮೇಶ್ವರೀ ಮಮ~।\\
ಹಕಹಲಹ್ರೀಂ ದೃಶೌ ಚ ಪಾತು ಕಾಮೇಶಮಧ್ಯಮಂ~।\\
ಕಹಯಲಹ್ರೀಂ ಶ್ರುತೀ ತು ಪಾತು ಕಾಮಂ ತುರೀಯಕಂ~।\\
ಹಕಲಸಹ್ರೀಂ ಶಕ್ತ್ಯಾಖ್ಯಂ ಪಾತು ಜಿಹ್ವಾಂ ಚ ಪಂಚಮಂ~॥\\
ಕಏಈಲಹ್ರೀಂ ಹಸಕಲಹ್ರೀಂ ಹಕಹಲಹ್ರೀಂ\\
ಕಹಯಲಹ್ರೀಂ ಹಕಲಸಹ್ರೀಂ  ಪರಮಾತ್ಮರೂಪಿಣೀ~।\\
ವದನಂ ಸಕಲಂ ಪಾತು ಪಂಚಕೂಟೈಸ್ತು ಪಂಚಮೀ ॥

ಕಏಈಲಹ್ರೀಂ ಘ್ರಾಣಂ ಮೇ ಪಾತು ವಾಗ್ಭವಸಂಜ್ಞಕಂ~।\\
ಹಸಕಹಲಹ್ರೀಂ ಕಂಠಂ ಪಾತು ಕಾಮೇಶಸಂಜ್ಞಕಂ~।\\
ಸಕಲಹ್ರೀಂ ಶಕ್ತಿಕೂಟಂ ಸ್ಕಂಧೌ ಪಾತು ಸದಾ ಮಮ~॥\\
ಕಏಈಲಹ್ರೀಂ ಹಸಕಹಲಹ್ರೀಂ ಸಕಲಹ್ರೀಂ ಚ।\\
ಕಾಮೇನೋಪಾಸಿತಾ ವಿದ್ಯಾ ಕಕ್ಷದೇಶೇ ಸದಾವತು ॥

ಕ್ಲೀಂಹ್ರೀಂಶ್ರೀಂಐಂಸೌಃ ಓಂಹ್ರೀಂಶ್ರೀಂ ಐಂಕ್ಲೀಂಸೌಃ\\ ಶ್ರೀಂಹ್ರೀಂಕ್ಲೀಂಐಂಸೌಃ॥\\
ಕಾಮಾದಿಷೋಡಶೀ ಪಾತು ಭುಜೌ ತ್ರಿಪುರಸುಂದರೀ ॥

ಓಂಕ್ಲೀಂಹ್ರೀಂಶ್ರೀಂ ಐಂಕ್ಲೀಂಸೌಃ  ಶ್ರೀಂಹ್ರೀಂಕ್ಲೀಂ\\ ಸ್ತ್ರೀಂಐಂಕ್ರೋಂಕ್ರೀಂಶ್ರೀಂಹೂಂ।\\
ತಾರಾದಿಷೋಡಶೀ ಪಾತು ಮಣಿಬಂಧದ್ವಯಂ ತಥಾ ॥

ಶ್ರೀಂಹ್ರೀಂಐಂಕ್ಲೀಂಸೌಃ ಓಂಹ್ರೀಂಶ್ರೀಂ ಐಂಕ್ಲೀಂಸೌಃ\\ ಸೌಃಐಂಕ್ಲೀಂಹ್ರೀಂಶ್ರೀಂ।\\
ರಮಾದಿಷೋಡಶೀ ಪಾತು ಕರೌ ತ್ರಿಪುರಸುಂದರೀ ॥

ಹ್ರೀಂಹ್ಸೌಃ ಓಂಐಂಹ್ರೀಂಶ್ರೀಂ ಐಂಕ್ಲೀಂಸೌಃ \\ಸ್ತ್ರೀಂಕ್ರೋಂಐಂಹ್ರೀಂ ಹೂಂಹೂಂಶ್ರೀಂ~।\\
ಮಾಯಾದಿಕಾ ತು ಹೃತ್ಪಾತು ಶ್ರೀಮತ್ತ್ರಿಪುರಸುಂದರೀ ॥

ಐಂಹ್ರೀಂಶ್ರೀಂಕ್ಲೀಂಸೌಃ ಐಂಕ್ಲೀಂಸೌಃ ಸೌಃಕ್ಲೀಂಐಂ ಸೌಃಕ್ಲೀಂಶ್ರೀಂಹ್ರೀಂಐಂ~।\\
ವಾಗಾದಿಷೋಡಶೀ ಪಾತು ಸ್ತನೌ ಮೇ ಸುಂದರೀ ಪರಾ ॥

ಕಏಶ್ರೀಂಕಏಈಲಹ್ರೀಂ ಕ್ಲೀಂಹಸಕಹಲಹ್ರೀಂ ಸೌಃಸಕಲಹ್ರೀಂ।\\
ನಖವರ್ಣಾಖ್ಯವಿದ್ಯೇಯಂ ಪಾರ್ಶ್ವೌ ಪಾತ್ವಪರಾಜಿತಾ ॥

ಹ್ರೀಂಹ್ಸೌಂಸ್ಹೌಂ ಹ್ರೀಂಸ್ಹೌಂಹ್ಸೌಂ ಕ್ಲೀಂಹಸಕಹಲಹ್ರೀಂಹ್ರೀಂಹ್ರೀಂ\\ ಸೌಃಸೌಃ ಹಹಸಕಹಲಹ್ರೀಂ ಕ್ಲೀಂಹ್ಸೌಂಸ್ಹೌಂ ಹ್ರೀಂಸ್ಹೌಂಹ್ಸೌಂಹ್ರೀಂ~।\\
ಏಕತ್ರಿಂಶದ್ವರ್ಣರೂಪಾ ಮಹಾಪುರುಷಪೂಜಿತಾ ॥\\
ಮಹಾಗುಹ್ಯಸ್ವರೂಪಾ ಚ ಕೇವಲಾನಂದಚಿನ್ಮಯೀ~।\\
ಕಟಿದೇಶಂ ಸದಾ ಪಾತು ಪರಬ್ರಹ್ಮಸ್ವರೂಪಿಣೀ ॥

ಹಸಕಲಹ್ರೀಂ ಪೃಷ್ಠದೇಶೇ ದೇವೀರಕ್ಷತು ವೈ ಸದಾ~।\\
ಹಸಕಹಲಹ್ರೀಂ ಕುಕ್ಷಿದೇಶಂ ಮಹಾವಿದ್ಯಾ ಚ ಪಾತು ಮಾಂ~।\\
ಸಕಲಹ್ರೀಂ ಶಕ್ತಿಕೂಟಂ ಪಾತು ವಕ್ಷಸ್ಥಲಂ ಮಮ~॥\\
ಹಸಕಲಹ್ರೀಂ ಹಸಕಹಲಹ್ರೀಂ ಸಕಲಹ್ರೀಂ ಮೇ~।\\
ಲೋಪಾಮುದ್ರಾಪಂಚದಶೀ ಮಧ್ಯದೇಶಂ ಸದಾವತು ॥

ಕಹಏಈಲಹ್ರೀಂ ನಾಭಿಂ ಪಾತು ಹಕಏಈಲಹ್ರೀಂ ಕಟಿಂ ಪಾತು~।\\
ಸಕ್ಥಿನೀ ಮೇ ಸದಾ ಪಾತು ಸಕಏಈಲಹ್ರೀಂ ಸದಾ॥\\
ಕಹಏಈಲಹ್ರೀಂ ಹಕಏಈಲಹ್ರೀಂ ಸಕಏಈಲಹ್ರೀಂ ಮೇ।\\
ವಸುಚಂದ್ರಾಮಾನವೀ ಮಾಂ ಸಾ ಸದಾ ಸರ್ವತೋಽವತು ॥

ಸಹಕಏಈಲಹ್ರೀಂ ಮೇ ಊರುಯುಗ್ಮಂ ಸದಾವತು~।\\
ಸಹಕಹಏಈಲಹ್ರೀಂ ಗುಹ್ಯಂ ಪಾತು ವರಪ್ರದಾ~।\\
ಹಸಕಏಈಲಹ್ರೀಂ ತು ಜಾನುನೀ ಪಾತು ಮೇ ಸದಾ~॥\\
ಸಹಕಏಈಲಹ್ರೀಂ ಸಹಕಹಏಈಲಹ್ರೀಂ ಹಸಕಏಈಲಹ್ರೀಂ~।\\
ಚಂದ್ರವಿದ್ಯಾ ಚ ಪಾತು ಮಾಂ ಪಕ್ಷಾದ್ಯಕ್ಷರವರ್ಣಕಾ~।\\
ಜಲಜೇ ಭಯಸಂಘಾತೇ ಸದಾ ಮಾಂ ಪರಿರಕ್ಷತು ॥

ಹಸಕಏಈಲಹ್ರೀಂ ಗುಲ್ಫಯುಗ್ಮಂ ಮಮ ವೈ ಸರ್ವದಾವತು~।\\
ಹಸಕಹಏಈಲಹ್ರೀಂ ಪಾದೌ ಪಾಯಾತ್ ಸನಾತನೀ~।\\
ಸಹಕಏಈಲಹ್ರೀಂ ಮೇ ಪ್ರಪದೌ ಪಾತು ಸರ್ವದಾ~॥\\
ಹಸಕಏಈಲಹ್ರೀಂ ಹಸಕಹಏಈಲಹ್ರೀಂ\\
ಸಹಕಏಈಲಹ್ರೀಂ ಸದಾ ಕುಬೇರೇಣ ಪ್ರಪೂಜಿತಾ~।\\
ದ್ವಾವಿಂಶತ್ಯಕ್ಷರೀ ವಿದ್ಯಾ ಸರ್ವಾಂಗಂ ಮೇ ಸದಾವತು ॥

ಕಏಈಲಹ್ರೀಂ ಪ್ರಾಚ್ಯಾಂ ತು ತ್ರಿಪುರಾ ಪರಿರಕ್ಷತು~।\\
ಹಸಕಹಲಹ್ರೀಂ ಪಾತು ವಹ್ನಿಕೋಣೇ ನಿರಂತರಂ~।\\
ಸಹಸಕಲಹ್ರೀಂ ಯಾಮ್ಯಾಂ ತು ಪಾತು ಮೇ ಸರ್ವಸಿದ್ಧಿದಾ~॥\\
ಕಏಈಲಹ್ರೀಂ ಹಸಕಹಲಹ್ರೀಂ ಸಹಸಕಲಹ್ರೀಂ ತು~।\\
ಅಗಸ್ತ್ಯವಿದ್ಯಾ ಸಾ ಸೇವ್ಯಾ ಚಕ್ರಸ್ಥಾ ಮಾಂ ಸದಾವತು ॥

ಸಏಈಲಹ್ರೀಂ ಚ ನಿತ್ಯಂ ನೈರೃತ್ಯಾಂ ಮಾಂ ಸದಾವತು~।\\
ಸಹಕಹಲಹ್ರೀಂ ಚೈವ ಪ್ರತೀಚ್ಯಾಂ ಪಾತು ಪಾರ್ವತೀ ॥\\
ಸಕಲಹ್ರೀಂ ತು ವಾಯವ್ಯೇ ಸದಾ ಮಾಂ ಪರಿರಕ್ಷತು~।\\
ಸಏಈಲಹ್ರೀಂ ಸಹಕಹಲಹ್ರೀಂ ಸಕಲಹ್ರೀಂ ತು~॥\\
ನಂದ್ಯಾರಾಧಿತವಿದ್ಯೇಯಂ ಸರ್ವಾಂಗೇ ಮಾಂ ಸದಾವತು ॥

ಹಸಕಲಹ್ರೀಂ ಉತ್ತರೇ ಚ ಪಾತು ಮಾಂ ಜಗದೀಶ್ವರೀ~।\\
ಸಹಕಲಹ್ರೀಂ ಈಶದಿಶಿ ಶಿವಪತ್ನೀ ಚ ಪಾತು ಮಾಂ~।\\
ಸಕಹಲಹ್ರೀಂ ಸುಂದರೀ ಊರ್ಧ್ವೇ ಮಾಂ ಪಾತು ಸರ್ವದಾ~॥\\
ಹಸಕಲಹ್ರೀಂ ಸಹಕಲಹ್ರೀಂ ಸಕಹಲಹ್ರೀಂ ಮಾಂ~।\\
ಅಧೋ ರಕ್ಷತು ಮೇ ನಿತ್ಯಂ ಸೂರ್ಯಪೂಜ್ಯಾ ಮಹೋದಯಾ ॥

ಕಏಈಲಹ್ರೀಂ ಹಸಕಹಲಹ್ರೀಂ ಸಕಲಹ್ರೀಂ ಮೇ~।\\
ಸರ್ವಾಂಗಂ ಶಕ್ರಸಂಪೂಜ್ಯಾ ಸತತಂ ಪರಿರಕ್ಷತು ॥

ಕಏಈಲಹ್ರೀಂ ಹಕಹಲಹ್ರೀಂ ಹಸಕಲಹ್ರೀಂ ಚ~।\\
ಬ್ರಹ್ಮಾಣೀ ಮಾಂ ಸದಾ ಪಾಯಾತ್ ಶ್ರೀಮತ್ತ್ರಿಪುರಸುಂದರೀ ॥

ಹಸಕಲಹ್ರೀಂ ಹಸಕಹಲಹ್ರೀಂ ಸಕಲಹ್ರೀಂ \\ಹಸಕಲ ಹಸಕಹಲ ಸಕಲಹ್ರೀಂ ಚ ಶಾಂಕರೀ।\\
ಚತುಃಕೂಟಾ ಮಹಾವಿದ್ಯಾ ಪಾತಾಲೇ ಮಾಂ ಸದಾವತು ॥

ಹಸಕಲಹ್ರೀಂ ಆಧಾರಂ ಹಸಕಹಲಹ್ರೀಂ ಚ ಲಿಂಗಕೇ~।\\
ಸಕಲಹ್ರೀಂ ಪಾತು ನಾಭಿಂ ಸಹಕಲಹ್ರೀಂ ಅನಾಹತಂ ।\\
ಸಹಕಹಲಹ್ರೀಂ ಕಂಠಂ ಸಹಸಕಲಹ್ರೀಂ ತಥಾ~॥\\
ಹಸಕಲಹ್ರೀಂ ಹಸಕಹಲಹ್ರೀಂ ಸಕಲಹ್ರೀಂ \\ಸಹಕಲಹ್ರೀಂ ಸಹಕಹಲಹ್ರೀಂ ಸಹಸಕಲಹ್ರೀಂ~।\\
ಮನೋಭವಾ ಸದಾ ಪಾತು ರದಸಂಖ್ಯಾ ಮಹಾಪ್ರಭಾ~।\\
ಷಟ್ಕೂಟಾ ವೈಷ್ಣವೀ ಸಾ ವೈ ಪಾತು ಮಾಂ ಸುಂದರೀ ಪರಾ ॥

ಕಏಈಲಹರೀ ಹಸಕಹಲರೀ ಸಕಲಹರೀ~।\\
ದುರ್ವಾಸಸಾ ಪ್ರಪೂಜ್ಯಾ ಚ ದಿಕ್ಷು ವಿದ್ಯಾ ಸದಾವತು ॥

ಕಹಏಈಲಹ್ರೀಂ ಹಲಏಈಲಹ್ರೀಂ ಸಕಏಈಲಹ್ರೀಂ~।\\
ಕ್ರೋಧೇನ ಪೂಜಿತಾ ನಿತ್ಯಂ ವಿದಿಕ್ಷು ಪರಿರಕ್ಷತು ॥

ಹಸಕಲಹ್ರೀಂ ಹಸಕಹಹಲಹ್ರೀಂ ಸಕಲಹ್ರೀಂ~।\\
ಮಹಾಜ್ಞಾನಮಯೀ ಪಾತು ನಿತ್ಯಂ ಮಾಂ ಷೋಡಶೀ ಪರಾ ॥

ಶ್ರೀಂಹ್ರೀಂಕ್ಲೀಂಐಂಸೌಃ ಓಂಹ್ರೀಂಶ್ರೀಂ ಕಏಈಲಹ್ರೀಂ\\ ಹಸಕಹಲಹ್ರೀಂ ಸಕಲಹ್ರೀಂ ಸೌಃಐಂಕ್ಲೀಂಹ್ರೀಂಶ್ರೀಂ~।\\
ಸರ್ವಾಂಗಂ ಮೇ ಮಹಾವಿದ್ಯಾ ಬೀಜರೂಪಾ ಚ ಷೋಡಶೀ ॥

ಓಂ ಕ್ಲೀಂಹ್ರೀಂಶ್ರೀಂ ಐಂಕ್ಲೀಂಸೌಃ ಕಏಈಲಹ್ರೀಂ\\ಹಸಕಹಲಹ್ರೀಂ ಸಕಲಹ್ರೀಂ ಸ್ತ್ರೀಂಐಂಕ್ರೋಂಕ್ರೀಂ \\ಈಂ ಹೂಂ ಷೋಡಶಸ್ವರರೂಪಿಣಿ  ಶ್ರೀಮತ್ತ್ರಿಪುರಸುಂದರಿ \\ಹ್ರಾಂಹ್ರೀಂಹ್ರೂಂ ಫಟ್ ಸರ್ವಸಿದ್ಧಿಂ ಪ್ರಯಚ್ಛ ಪ್ರಯಚ್ಛ ಸ್ವಾಹಾ~।\\
ಶ್ರೀಮಹಾಷೋಡಶೀ ವಿದ್ಯಾನಾಖ್ಯಾತಾ ಭುವನತ್ರಯೇ~।\\
ಜ್ಞಾನೇನ ಮೃತ್ಯುಹಾ ಸಾ ಮಾಂ ಶಿರಸ್ಥಾ ಸರ್ವತೋಽವತು ॥\\
ಶ್ರೀಮಹಾಷೋಡಶೀ ಪೂರ್ಣಾ ಮಹಾದೇವೇನ ಪೂಜಿತಾ~।\\
ಯಸ್ಯಾ ವಿಜ್ಞಾನಮಾತ್ರೇಣ ಮೃತ್ಯೋರ್ಮೃತ್ಯುರ್ಭವೇತ್ಸ್ವಯಂ ॥

ಕ್ಲೀಂಐಂಶ್ರೀಂ ಕಏಈಲಹ್ರೀಂ ಹಸಕಹಲಹ್ರೀಂ ಸಕಲಹ್ರೀಂ~।\\
ಕಾಮವಾಗೀಶ್ವರೀ ಲಕ್ಷ್ಮೀಸ್ತ್ರಿಕೂಟಾ ಪರಮೇಶ್ವರೀ ॥

ಐಂ ಕಏಈಲಹ್ರೀಂ ಕ್ಲೀಂ ಹಸಕಹಲಹ್ರೀಂ ಸೌಃ ಸಕಲಹ್ರೀಂ\\ಸೋಹಂ ಹೌಂ ಹಂಸಃ ಹ್ರೀಂ ಸಕಲಹ್ರೀಂ ಸೌಃ \\ ಹಸಕಹಲಹ್ರೀಂ ಕ್ಲೀಂ ಕಏಈಲಹ್ರೀಂ ಐಂ ಬ್ರಹ್ಮಸ್ವರೂಪಿಣೀ~।\\
ನೇತ್ರವೇದಯುತೈರ್ವರ್ಣೈರ್ಯುತಾ ಸಾ ಸರ್ವತೋಽವತು~॥\\
ಬ್ರಹ್ಮಸ್ವರೂಪಿಣೀ ಚೇಯಂ ಪರಮಾನಂದಚಿದ್ಘನಾ~।\\
ಅಷ್ಟಾದಶಾಕ್ಷರೀ ವಿದ್ಯಾ ಸದಾ ಮಾಂ ಪರಿರಕ್ಷತು ॥({\bfseries ಶ್ರೀಂಹ್ರೀಂಐಂ})

ಇತಿ ತೇ ಕಥಿತಂ ದೇವಿ ಬ್ರಹ್ಮವಿದ್ಯಾಕಲೇಬರಂ~।\\
ತ್ರೈಲೋಕ್ಯಮೋಹನಂ ನಾಮ ಕವಚಂ ಬ್ರಹ್ಮರೂಪಕಂ ॥

ಸಪ್ತಲಕ್ಷಮಹಾವಿದ್ಯಾಃ ತಂತ್ರಾದೌ ಕಥಿತಾಃ ಪ್ರಿಯೇ~।\\
ತಾಸಾಂ ಸಾರಾತ್ಸಾರತಯಾ ಯಾ ಯಾ ವಿದ್ಯಾಃ ಸುಗೋಪಿತಾಃ ॥

ಬಹುನಾತ್ರಕಿಮುಕ್ತೇನ ಶ್ರೀಮಹಾಷೋಡಶೀ ಪರಾ~।\\
ಪ್ರಕಾಶಿತಾ ಮಯಾ ದೇವಿ ಯಾಂ ಪೃಚ್ಛಸಿ ಪುನಃ ಪುನಃ ॥

ಮಹಾವಿದ್ಯಾಮಯಂ ಬ್ರಹ್ಮಕವಚಂ ಮನ್ಮುಖೋದಿತಂ~।\\
ಗುರುಮಭ್ಯರ್ಚ್ಯ ವಿಧಿವತ್ ಕವಚಂ ಪ್ರಪಠೇತ್ತತಃ ॥

ದೇವೀಮಭ್ಯರ್ಚ್ಯ ವಿಧಿವತ್ ಪುರಶ್ಚರ್ಯಾಂ ಸಮಾಚರೇತ್~।\\
ಅಷ್ಟೋತ್ತರಶತಂ ಜಪ್ತ್ವಾ ದಶಾಂಶಂ ಹವನಾದಿಕಂ ॥

ತತಃ ಸುಸಿದ್ಧಕವಚಃ ಪುಣ್ಯಾತ್ಮಾ ಮದನೋಪಮಃ~।\\
ಮಂತ್ರಸಿದ್ಧಿರ್ಭವೇತ್ತಸ್ಯ ಪುರಶ್ಚರ್ಯಾಂ ವಿನಾ ತತಃ ॥

ಗದ್ಯಪದ್ಯಮಯೀ ವಾಣೀ ತಸ್ಯ ವಕ್ತ್ರಾತ್ ಪ್ರಜಾಯತೇ~।\\
ವಕ್ತ್ರೇ ತಸ್ಯವಸೇದ್ವಾಣೀ ಕಮಲಾ ನಿಶ್ಚಲಾ ಗೃಹೇ ॥

ಪುಷ್ಪಾಂಜಲ್ಯಷ್ಟಕಂ ದತ್ವಾ ಮೂಲೇನೈವ ಪಠೇತ್ ಸಕೃತ್~।\\
ಅಪಿ ವರ್ಷಸಹಸ್ರಾಣಾಂ ಪೂಜಾಫಲಮವಾಪ್ನುಯಾತ್ ॥

ಆತ್ಮಾನಂ ತನ್ಮಯಂ ಕೃತ್ವಾ ಯಃ ಪಠೇತ್ ಕವಚಂ ಪರಂ~।\\
ಯಂ ಯಂ ಪಶ್ಯತಿ ವೈ ಶೀಘ್ರಂ ಸ ಸ ದಾಸೋ ಭವೇದ್ಧೃವಂ ॥

ವಿಲಿಖ್ಯ ಭೂರ್ಜೇ ಘುಟಿಕಾಂ ಸ್ವರ್ಣಸ್ಥಾಂ ಧಾರಯೇದ್ಯದಿ~।\\
ಕಂಠೇ ವಾ ಯದಿ ವಾ ಬಾಹೌ ಸ ಕುರ್ಯಾದ್ದಾಸವಜ್ಜಗತ್ ॥

ತ್ರಿಲೋಕೀಂ ಕ್ಷೋಭಯತ್ಯೇವ ತ್ರೈಲೋಕ್ಯವಿಜಯೀ ಭವೇತ್~।\\
ತದ್ಗಾತ್ರಂ ಪ್ರಾಪ್ಯ ಶಸ್ತ್ರಾಣಿ ಬ್ರಹ್ಮಾಸ್ತ್ರಾದೀನಿ ಪಾರ್ವತಿ ॥

ಮಾಲ್ಯಾನಿ ಕುಸುಮಾನೀವ ಸುಖದಾನಿ ಭವಂತಿ ಹಿ~।\\
ಸ್ಪರ್ಧಾಂ ನಿರಸ್ಯ ಭವನೇ ಲಕ್ಷ್ಮೀರ್ವಾಣೀ ವಸೇತ್ತತಃ ॥

ಇದಂ ಕವಚಮಜ್ಞಾತ್ವಾ ಯೋ ಜಪೇತ್ಸುಂದರೀಂ ಪರಾಂ~।\\
ನವಲಕ್ಷಂ ಪ್ರಜಪ್ತ್ವಾಽಪಿ ತಸ್ಯ ವಿದ್ಯಾ ನ ಸಿದ್ಧ್ಯತಿ ॥

ಸ ಶಸ್ತ್ರಘಾತಮಾಪ್ನೋತಿ ಸೋಽಚಿರಾನ್ಮೃತ್ಯುಮಾಪ್ನುಯಾತ್~।\\
ಇದಮೇವ ಪರಂ ಯಸ್ಮಾದ್ ಭುಕ್ತಿಮುಕ್ತಿಪ್ರದಾಯಕಂ~।\\
ತಸ್ಮಾತ್ಸರ್ವಪ್ರಯತ್ನೇನ ಪಠನೀಯಂ ಮುುಮುಕ್ಷುಭಿಃ ॥

\authorline{ಇತಿ ಶ್ರೀರುದ್ರಯಾಮಲೇ ಗೌರೀಶ್ವರಸಂವಾದೇ ಶ್ರೀರಾಜರಾಜೇಶ್ವರೀ \\ಮಹಾತ್ರಿಪುರಸುಂದರ್ಯಾಃ ತ್ರೈಲೋಕ್ಯಮೋಹನಂ ನಾಮ ಕವಚಂ ಸಂಪೂರ್ಣಂ॥}
%====================================================================================================
\section{ಸರಸ್ವತೀ ಕವಚಮ್\\ (ರುದ್ರಯಾಮಲಾಂತರ್ಗತಮ್ )}
\addcontentsline{toc}{section}{ಸರಸ್ವತೀ ಕವಚಮ್}

ಅಸ್ಯ ಶ್ರೀ ಸರಸ್ವತೀ ಕವಚಮಂತ್ರಸ್ಯ ಆಶ್ವಲಾಯನ ಋಷಿಃ । ಅನುಷ್ಟುಪ್ ಛಂದಃ । ಶ್ರೀ ಸರಸ್ವತೀ ದೇವತಾ । ಐಂ ಬೀಜಂ । ಹ್ರೀಂ ಶಕ್ತಿಃ । ಕ್ಲೀಂ ಕೀಲಕಂ । ಸರಸ್ವತೀ ಪ್ರಸಾದಸಿದ್ಧ್ಯರ್ಥೇ ಜಪೇ ವಿನಿಯೋಗಃ ॥
\dhyana{ದೋರ್ಭಿರ್ಯುಕ್ತಾ ಚತುರ್ಭಿಃ ಸ್ಫಟಿಕಮಣಿಮಯೀಮಕ್ಷಮಾಲಾಂದಧಾನಾ\\
ಹಸ್ತೇನೈಕೇನ ಪದ್ಮಂ ಸಿತಮಪಿ ಚ ಶುಕಂ ಪುಸ್ತಕಂ ಚಾಪರೇಣ ।\\
ಭಾಸಾ ಕುಂದೇಂದುಶಂಖಸ್ಫಟಿಕಮಣಿನಿಭಾ ಭಾಸಮಾನಾಽಸಮಾನಾ\\
ಸಾ ಮೇ ವಾಗ್ದೇವತೇಯಂ ನಿವಸತು ವದನೇ ಸರ್ವದಾ ಸುಪ್ರಸನ್ನಾ ॥}


ಸರಸ್ವತೀ ಶಿರಃ ಪಾತು ಫಾಲಂ ಫಾಲಾಕ್ಷಸೋದರೀ ।\\
ಶ್ರುತೀ ಶ್ರುತಿಮಯೀ ಪಾತು ನೇತ್ರೇರ್ಕೇಂದ್ವಗ್ನಿಲೋಚನಾ ॥೧॥

ಘ್ರಾಣಂ ಪ್ರಾಣನಿಧಿಃ ಪಾತು ಕಪೋಲೌ ಕಾಮಿತಾರ್ಥದಾ ।\\
ವಕ್ತ್ರಂ ವಿದ್ಯಾತ್ಮಿಕಾ ಪಾತು ಸ್ಕಂಧೌ ಸ್ಕಂದಸಮರ್ಚಿತಾ ॥೨॥

ಭುಜೌ ಚತುರ್ಭುಜಾ ಪಾತು ಕರೌ ಕಾಂಕ್ಷಿತದಾಯಿಕಾ ।\\
ಪಾರ್ಶ್ವೌ ಮೇ ಪಾತು ದೇವೇಶೀ ವಕ್ಷೋ ಬ್ರಹ್ಮಮುಖಾಸನಾ ॥೩॥

ಕುಕ್ಷಿಮಕ್ಷರರೂಪಾ ಚ ನಾಭಿಂ ನಾಭಿಜವಲ್ಲಭಾ ।\\
ಮಧ್ಯಂ ಸುಮಧ್ಯಮಾ ಪಾತು ಗುಹ್ಯಂ ಸರ್ವಾಂಗ ಸುಂದರೀ ॥೪॥

ಊರೂ ಮೇ ಪಾತು ವಾಗ್ದೇವೀ ಜಾನುನೀ ಜಗದೀಶ್ವರೀ ।\\
ಜಂಘೇ ಪಾತು ಮಹಾದೇವೀ ಗುಲ್ಫೌ ಮೇ ಗುಣರೂಪಿಣೀ॥೫॥

ಪಾದೌ ವೇದಾತ್ಮಿಕಾ ಪಾತು ಸರ್ವಾಂಗಂ ಮಾತೃಕಾತ್ಮಿಕಾ ।\\
ಇತೀದಂ ಕವಚಂ ದಿವ್ಯಂ ವಾಗ್ದೇವ್ಯಾಃ ಪ್ರೀತಿಕಾರಣಂ ॥೬॥

ಫಲಶ್ರುತಿಃ॥\\
ಜಡಾನಾಂ ಬುದ್ಧಿದಂ ಪುಣ್ಯಂ ಮೂಕಾನಾಂ ವಾಕ್ಪ್ರದಾಯಕಂ ।\\
ಅಂಧಾನಾಂ ದೃಷ್ಟಿದಂ ಚೈವ ಸರ್ವಜ್ಞಾನಪ್ರದಾಯಕಂ ॥೭॥

ವಾಗ್ವಶ್ಯಜನಕಂ ನೄಣಾಂ ತಥಾ ಭೂ-ಪಾಲಮೋಹನಂ ।\\
ವಾಕ್‌ಸ್ತಂಭಕಾರಕಂ ಚೈವ ಸಭಾಯಾಂ ಪ್ರತಿವಾದಿನಾಂ ॥೮॥

ಪುತ್ರಪ್ರದಮಪುತ್ರಾಣಾಂ ಧನದಂ ಧನಕಾಮಿನಾಂ ।\\
ಮೋಕ್ಷದಂ ಮೋಕ್ಷಕಾಮಾನಾಂ ಮಂತ್ರಸಿದ್ಧಿ ಪ್ರದಾಯಕಂ ॥೯॥

ಬಹುನಾ ಕಿಮಿಹೋಕ್ತೇನ ಸತ್ಯಂ ಸತ್ಯಂ ಮುನೀಶ್ವರ ।\\
ಆಶ್ವಲಾಯನಸಂಪ್ರೋಕ್ತಂ ಷಣ್ಮಾಸಂ ಜಪತಾಂ ನೃಣಾಂ ॥೧೦॥

ಕವಿತಾವಾಕ್ಪಟುತ್ವಂ ಚ ಜಾಯತೇ ನಾತ್ರ ಸಂಶಯಃ ।\\
ಶುಕ್ರವಾರೇ ವಿಶೇಷೇಣ ಜಪ್ತವ್ಯಂ ದೇವಿ ಸಾಧಕೈಃ ॥೧೧॥

ಸತ್ಯಂ ಸಾರಸ್ವತಂ ಚೈವ ಸ್ಥಿರತಾಮೇತಿ ತತ್ಕುಲೇ ।\\
ಪೌರ್ಣಮಾಸ್ಯಾಮಮಾವಸ್ಯಾಂಂ ದೇವಿ ಸಾರಸ್ವತೇ ತಥಾ ॥೧೨॥

ಯೋಗೇ ವಿಶೇಷೇ ಜಪ್ತವ್ಯಂ ವಿದ್ಯಾರ್ಥಿಭಿರತಂದ್ರಿತೈಃ ।\\
ಅನೇನ ಕವಚೇನೈವ ನ್ಯಸ್ತಾಂಗೋ ಮೂಲಮಂತ್ರಕಂ ॥೧೩॥

ಅಷ್ಟೋತ್ತರಶತಂ ಜಪ್ತ್ವಾ  ವಚಾಂ ಶ್ವೇತಾಂ ಚ ಭಕ್ಷಯೇತ್ ।\\
ಪ್ರಾತಃ ಕಾಲೇ ತು ಮಾಸೈಕಂ ವಾಕ್ಸಿದ್ಧಿರತುಲಾ ಭವೇತ್ ॥೧೪॥

ಗೋಮಯೇನ ಮೃದಾ ವಾಪಿ ನಿರ್ಮಾಯ ಪ್ರತಿವಾದಿನಮ್ ।\\
ವಾಮಪಾದೇನ ಚಾಕ್ರಮ್ಯ ತಜ್ಜಿಹ್ವಾಂ ಕವಚಂ ಜಪೇತ್ ॥೧೫॥

ಮೂಕೋ ವೈ ಜಾಯತೇ ಶೀಘ್ರಮುನ್ಮತ್ತೋ ವಾ ಭವೇದ್ಧ್ರುವಮ್ ।\\
ಯಂ ಯಂ ಕಾಮಯತೇ ಕಾಮಂ ತಂತಮುದ್ದಿಶ್ಯ ಪಾರ್ವತಿ ॥೧೬॥

ಅಷ್ಟೋತ್ತರಶತಂ ಜಪ್ತ್ವಾ ಫಲಂ ವಿಂದತಿ ಮಾನವಃ ।\\
ಅಶ್ವತ್ಥೇ ರಾಜವಶ್ಯಾರ್ಥೀ ತೇಜಸೇಽಭಿಮುಖೋ ರವೇಃ ॥೧೭॥

ಕನ್ಯಾರ್ಥೀ ಚಂಡಿಕಾಗೇಹೇ ಗೇಹೇ ಶತ್ರುಕೃತೇ ಮಮ ।\\
ಶ್ರೀಕಾಮೋ ಬಿಲ್ವಮೂಲೇ ತು ಉದ್ಯಾನೇ ಸ್ತ್ರೀವಶೀ ಭವೇತ್ ।\\
ಪುತ್ರಾರ್ಥೀ ದಕ್ಷಿಣಾಮೂರ್ತೇಃ ಸನ್ನಿಧೌ ಮಮ ಪಾರ್ವತೀ ॥೧೮॥
\authorline{ಇತಿ ಶ್ರೀರುದ್ರಯಾಮಲೇ ಉಮಾಮಹೇಶ್ವರಸಂವಾದೇ \\ಸರಸ್ವತೀಕವಚಂ ಸಂಪೂರ್ಣಂ ॥}
%=================================================================================================

