‌\setcounter{page}{0}
\thispagestyle{empty}
\fancyhead[RL]{}
\chapter*{\center {\LARGE ಶ್ರೀಮಹಾತ್ರಿಪುರಸುಂದರೀಸಹಿತ}\\ ತ್ರಿಪುರಾಂತಕೇಶ್ವರಾಯ ನಮಃ}
\thispagestyle{empty}
\section{ಗುರುಪ್ರಾರ್ಥನಾ}
\addcontentsline{toc}{section}{ಗುರುಪ್ರಾರ್ಥನಾ}
ಓಂ ಆಬ್ರಹ್ಮಲೋಕಾದಾಶೇಷಾತ್ ಆಲೋಕಾಲೋಕಪರ್ವತಾತ್।\\
ಯೇ ವಸಂತಿ ದ್ವಿಜಾ ದೇವಾಃ ತೇಭ್ಯೋ ನಿತ್ಯಂ ನಮಾಮ್ಯಹಂ॥

ಓಂ ನಮೋ ಬ್ರಹ್ಮಾದಿಭ್ಯೋ ಬ್ರಹ್ಮವಿದ್ಯಾಸಂಪ್ರದಾಯಕರ್ತೃಭ್ಯೋ \\ವಂಶರ್ಷಿಭ್ಯೋ ಮಹದ್ಭ್ಯೋ ನಮೋ ಗುರುಭ್ಯಃ~।\\
ಸರ್ವೋಪಪ್ಲವರಹಿತಃ ಪ್ರಜ್ಞಾನಘನಃ ಪ್ರತ್ಯಗರ್ಥೋ ಬ್ರಹ್ಮೈವಾಹಮಸ್ಮಿ ॥

ಸಚ್ಚಿದಾನಂದ ರೂಪಾಯ ಬಿಂದು ನಾದಾಂತರಾತ್ಮನೇ~।\\
ಆದಿಮಧ್ಯಾಂತ ಶೂನ್ಯಾಯ ಗುರೂಣಾಂ ಗುರವೇ ನಮಃ ॥

ಶುದ್ಧಸ್ಫಟಿಕಸಂಕಾಶಂ ದ್ವಿನೇತ್ರಂ ಕರುಣಾನಿಧಿಂ~।\\
ವರಾಭಯಪ್ರದಂ ವಂದೇ ಶ್ರೀಗುರುಂ ಶಿವರೂಪಿಣಂ ॥

ಗುರುರ್ಬ್ರಹ್ಮಾ ಗ್ರುರುರ್ವಿಷ್ಣುಃ ಗುರುರ್ದೇವೋ ಮಹೇಶ್ವರಃ~।\\
ಗುರುಃ ಸಾಕ್ಷಾತ್ ಪರಂ ಬ್ರಹ್ಮ ತಸ್ಮೈ ಶ್ರೀ ಗುರವೇ ನಮಃ ॥

ಶುಕ್ಲಾಂಬರಧರಂ ವಿಷ್ಣುಂ ಶಶಿವರ್ಣಂ ಚತುರ್ಭುಜಂ~।\\
ಪ್ರಸನ್ನವದನಂ ಧ್ಯಾಯೇತ್ ಸರ್ವವಿಘ್ನೋಪಶಾಂತಯೇ ॥

ಮಂಜುಶಿಂಜಿತ ಮಂಜೀರಂ ವಾಮಮರ್ಧಂ ಮಹೇಶಿತುಃ~।\\
ಆಶ್ರಯಾಮಿ ಜಗನ್ಮೂಲಂ ಯನ್ಮೂಲಂ ವಚಸಾಮಪಿ ॥

ಶ್ರೀ ವಿದ್ಯಾಂ ಜಗತಾಂ ಧಾತ್ರೀಂ ಸರ್ಗಸ್ಥಿತಿ ಲಯೇಶ್ವರೀಂ~।\\
ನಮಾಮಿ ಲಲಿತಾಂ ನಿತ್ಯಂ ಭಕ್ತಾನಾಮಿಷ್ಟದಾಯಿನೀಂ ॥

ಶ್ರೀನಾಥಾದಿ ಗುರುತ್ರಯಂ ಗಣಪತಿಂ ಪೀಠತ್ರಯಂ ಭೈರವಂ\\
ಸಿದ್ಧೌಘಂ ಬಟುಕತ್ರಯಂ ಪದಯುಗಂ ದೂತೀಕ್ರಮಂ ಮಂಡಲಂ~।\\
ವೀರದ್ವ್ಯಷ್ಟ ಚತುಷ್ಕಷಷ್ಟಿನವಕಂ ವೀರಾವಲೀಪಂಚಕಂ\\
ಶ್ರೀಮನ್ಮಾಲಿನಿಮಂತ್ರರಾಜಸಹಿತಂ ವಂದೇ ಗುರೋರ್ಮಂಡಲಂ॥

ಬಿಂದುತ್ರಿಕೋಣಸಂಯುಕ್ತಂ ವಸುಕೋಣಸಮನ್ವಿತಂ~।\\
ದಶಕೋಣದ್ವಯೋಪೇತಂ ಭುವನಾರ ಸಮನ್ವಿತಂ ॥\\
ದಲಾಷ್ಟಕ ಸಮೋಪೇತಂ ದಲಷೋಡಶಕಾನ್ವಿತಂ~।\\
ವೃತ್ತತ್ರಯಾನ್ವಿತಂ ಭೂಮೀಸದನತ್ರಯ ಭೂಷಿತಂ~।\\
ನಮಾಮಿ ಲಲಿತಾಚಕ್ರಂ ಭಕ್ತಾನಾಮಿಷ್ಟದಾಯಕಂ ॥
\section{ದ್ವಾರಪೂಜಾ}
\addcontentsline{toc}{section}{ದ್ವಾರಪೂಜಾ}
ಓಂ ಐಂಹ್ರೀಂಶ್ರೀಂ ವಂ ವಟುಕಾಯ ನಮಃ(ದ್ವಾರಸ್ಯ ಅಧಃ)\\
೪ ಭಂ ಭದ್ರಕಾಲ್ಯೈ ನಮಃ (ದ್ವಾರಸ್ಯ ವಾಮಭಾಗೇ)\\
೪ ಭಂ ಭೈರವಾಯ ನಮಃ(ದ್ವಾರಸ್ಯ ದಕ್ಷಭಾಗೇ)\\
೪ ಲಂ ಲಂಬೋದರಾಯ ನಮಃ(ದ್ವಾರಸ್ಯ ಊರ್ಧ್ವಭಾಗೇ)
\section{ಮಂತ್ರಾಚಮನಮ್}
\addcontentsline{toc}{section}{ಮಂತ್ರಾಚಮನಮ್}
\dhyana{೪ ಐಂ ಕಏಈಲಹ್ರೀಂ} ಆತ್ಮತತ್ವಂ ಶೋಧಯಾಮಿ ನಮಃ ಸ್ವಾಹಾ।\\
\dhyana{೪ ಕ್ಲೀಂ ಹಸಕಹಲಹ್ರೀಂ} ವಿದ್ಯಾತತ್ವಂ ಶೋಧಯಾಮಿ ನಮಃ ಸ್ವಾಹಾ।\\
\dhyana{೪ ಸೌಃ ಸಕಲಹ್ರೀಂ} ಶಿವತತ್ವಂ ಶೋಧಯಾಮಿ ನಮಃ ಸ್ವಾಹಾ।\\
\dhyana{೪ ಐಂ (೫) ಕ್ಲೀಂ (೬) ಸೌಃ (೪)} ಸರ್ವತತ್ವಂ ಶೋಧಯಾಮಿ ನಮಃ ಸ್ವಾಹಾ।
\section{ಗುರುಪಾದುಕಾಮಂತ್ರಃ}
\addcontentsline{toc}{section}{ಗುರುಪಾದುಕಾಮಂತ್ರಃ}
\dhyana{ಓಂಐಂಹ್ರೀಂಶ್ರೀಂಐಂಕ್ಲೀಂಸೌಃ ಹಂಸಃ ಶಿವಃ ಸೋಽಹಂ ಹ್‌ಸ್‌ಖ್‌ಫ್ರೇಂ ಹಸಕ್ಷಮಲವರಯೂಂ ಹ್‌ಸೌಃ ಸಹಕ್ಷಮಲವರಯೀಂ ಸ್‌ಹೌಃ ಹಂಸಃ ಶಿವಃ ಸೋಽಹಂ~॥} ಸ್ವರೂಪ ನಿರೂಪಣ ಹೇತವೇ ಶ್ರೀಗುರವೇ ನಮಃ~। ಶ್ರೀಪಾದುಕಾಂ ಪೂಜಯಾಮಿ ನಮಃ ॥\\
\dhyana{೭ ಸೋಽಹಂ ಹಂಸಃ ಶಿವಃ ಹ್‌ಸ್‌ಖ್‌ಫ್ರೇಂ ಹಸಕ್ಷಮಲವರಯೂಂ ಹ್‌ಸೌಃ ಸಹಕ್ಷಮಲವರಯೀಂ ಸ್‌ಹೌಃ ಸೋಽಹಂ ಹಂಸಃ ಶಿವಃ॥} ಸ್ವಚ್ಛಪ್ರಕಾಶ ವಿಮರ್ಶಹೇತವೇ ಪರಮಗುರವೇ ನಮಃ।ಶ್ರೀಪಾದುಕಾಂ ಪೂಜಯಾಮಿ ನಮಃ॥\\
\dhyana{೭ ಹಂಸಃಶಿವಃ ಸೋಽಹಂಹಂಸಃ ಹ್‌ಸ್‌ಖ್‌ಫ್ರೇಂ ಹಸಕ್ಷಮಲವರಯೂಂ ಹ್‌ಸೌಃ ಸಹಕ್ಷಮಲವರಯೀಂ ಸ್‌ಹೌಃ ಹಂಸಃ ಶಿವಃ ಸೋಽಹಂ ಹಂಸಃ॥} ಸ್ವಾತ್ಮಾರಾಮ ಪರಮಾನಂದ ಪಂಜರ ವಿಲೀನ ತೇಜಸೇ ಪರಮೇಷ್ಠಿಗುರವೇ ನಮಃ~। ಶ್ರೀಪಾದುಕಾಂ ಪೂಜಯಾಮಿ ನಮಃ॥
\section{ಘಂಟಾನಾದಃ}
\addcontentsline{toc}{section}{ಘಂಟಾನಾದಃ}
೪ ಹೇ ಘಂಟೇ ಸುಸ್ವರೇ ರಮ್ಯೇ ಘಂಟಾಧ್ವನಿವಿಭೂಷಿತೇ।\\
ವಾದಯಂತಿ ಪರಾನಂದೇ ಘಂಟಾದೇವಂ ಪ್ರಪೂಜಯೇ॥\\
\as{ಓಂ ಜಗದ್ಧ್ವನಿಮಂತ್ರಮಾತಃ ಸ್ವಾಹಾ ॥}\\
ಆಗಮಾರ್ಥಂ ಚ ದೇವಾನಾಂ ಗಮನಾರ್ಥಂ ಚ ರಕ್ಷಸಾಂ।\\
ಕುರ್ಯಾತ್ ಘಂಟಾರವಂ ತತ್ರ ದೇವತಾಹ್ವಾನ ಲಾಂಛನಂ॥

ವಿಷ್ಣುಶಕ್ತಿಸಮೋಪೇತೇ ಸರ್ವವರ್ಣೇ ಮಹೀತಲೇ~।\\
ಅನೇಕರತ್ನಸಂಭೂತೇ ಭೂಮಿದೇವಿ ನಮೋಽಸ್ತು ತೇ॥

ಪೃಥ್ವೀತಿ ಮಂತ್ರಸ್ಯ ಮೇರುಪೃಷ್ಠ ಋಷಿಃ~। ಸುತಲಂ ಛಂದಃ~।\\ಆದಿಕೂರ್ಮೋ ದೇವತಾ ॥ ಆಸನೇ ವಿನಿಯೋಗಃ\\
ಪೃಥ್ವಿ ತ್ವಯಾ ಧೃತಾ ಲೋಕಾ ದೇವಿ ತ್ವಂ ವಿಷ್ಣುನಾ ಧೃತಾ~।\\
ತ್ವಂ ಚ ಧಾರಯ ಮಾಂ ದೇವಿ ಪವಿತ್ರಂ ಕುರು ಚಾಸನಂ ॥

ಅಪಸರ್ಪಂತು ತೇ ಭೂತಾಃ ಯೇ ಭೂತಾ ಭೂಮಿ ಸಂಸ್ಥಿತಾಃ~।\\
ಯೇ ಭೂತಾಃ ವಿಘ್ನಕರ್ತಾರಸ್ತೇನಶ್ಯಂತು ಶಿವಾಜ್ಞಯಾ ॥

ಅಪಕ್ರಾಮಂತು ಭೂತಾನಿ ಪಿಶಾಚಾಃ ಸರ್ವತೋ ದಿಶಂ।\\
ಸರ್ವೇಷಾಮವಿರೋಧೇನ ಪೂಜಾ ಕರ್ಮಸಮಾರಭೇ ॥

ಸ್ಯೋನಾ ಪೃಥಿವೀತ್ಯಸ್ಯ ಮೇಧಾತಿಥಿಃ ಕಾಣ್ವ ಋಷಿಃ । ಗಾಯತ್ರೀ ಛಂದಃ । ಪೃಥಿವೀ ದೇವತಾ । ಭೂಪ್ರಾರ್ಥನೇ ವಿನಿಯೋಗಃ ॥\\
ಸ್ಯೋನಾ ಪೃಥಿವಿ ಭವಾನೃಕ್ಷರಾ ನಿವೇಶನೀ~।\\ ಯಚ್ಛಾ ನಃ ಶರ್ಮ ಸಪ್ರಥಃ ॥

ಧನುರ್ಧರಾಯೈ ಚ ವಿದ್ಮಹೇ ಸರ್ವಸಿದ್ಧ್ಯೈ ಚ ಧೀಮಹಿ~।\\ ತನ್ನೋ ಧರಾ ಪ್ರಚೋದಯಾತ್ ॥

ಲಂ ಪೃಥಿವ್ಯೈ ನಮಃ~। ರಂ ರಕ್ತಾಸನಾಯ ನಮಃ~। ವಿಂ ವಿಮಲಾಸನಾಯ ನಮಃ~। ಯಂ ಯೋಗಾಸನಾಯ ನಮಃ~। ಕೂರ್ಮಾಸನಾಯ ನಮಃ~। ಅನಂತಾಸನಾಯ ನಮಃ~। ವೀರಾಸನಾಯ ನಮಃ~। ಖಡ್ಗಾಸನಾಯ ನಮಃ~। ಶರಾಸನಾಯ ನಮಃ~। ಪಂ ಪದ್ಮಾಸನಾಯ ನಮಃ~। ಪರಮಸುಖಾಸನಾಯ ನಮಃ॥

೪ ರಕ್ತದ್ವಾದಶಶಕ್ತಿಯುಕ್ತಾಯ ದ್ವೀಪನಾಥಾಯ ನಮಃ ॥

೪ ಶ್ರೀಲಲಿತಾಮಹಾತ್ರಿಪುರಸುಂದರಿ ಆತ್ಮಾನಂ ರಕ್ಷ ರಕ್ಷ ॥

ಓಂ ಗುಂ ಗುರುಭ್ಯೋ ನಮಃ~। ಪರಮಗುರುಭ್ಯೋ ನಮಃ~। ಪರಮೇಷ್ಠಿ\\ಗುರುಭ್ಯೋ ನಮಃ~। ಗಂ ಗಣಪತಯೇ ನಮಃ~। ದುಂ ದುರ್ಗಾಯೈ ನಮಃ~। ಸಂ ಸರಸ್ವತ್ಯೈ ನಮಃ~। ವಂ ವಟುಕಾಯ ನಮಃ~। ಕ್ಷಂ ಕ್ಷೇತ್ರಪಾಲಾಯ ನಮಃ~। ಯಾಂ ಯೋಗಿನೀಭ್ಯೋ ನಮಃ~। ಅಂ ಆತ್ಮನೇ ನಮಃ~। ಪಂ ಪರಮಾತ್ಮನೇ ನಮಃ~। ಸಂ ಸರ್ವಾತ್ಮನೇ ನಮಃ ॥

೪ ಓಂ ನಮೋ ಭಗವತಿ ತಿರಸ್ಕರಿಣಿ ಮಹಾಮಾಯೇ ಮಹಾನಿದ್ರೇ ಸಕಲ \\ಪಶುಜನ ಮನಶ್ಚಕ್ಷುಃಶ್ರೋತ್ರತಿರಸ್ಕರಣಂ ಕುರು ಕುರು ಸ್ವಾಹಾ ॥

೪ ಓಂ ಹಸಂತಿ ಹಸಿತಾಲಾಪೇ ಮಾತಂಗಿ ಪರಿಚಾರಿಕೇ~।\\
ಮಮ ವಿಘ್ನಾಪದಾಂ ನಾಶಂ ಕುರು ಕುರು ಠಃಠಃಠಃ ಹುಂ ಫಟ್ ಸ್ವಾಹಾ ॥

೪ ಓಂ ನಮೋ ಭಗವತಿ ಜ್ವಾಲಾಮಾಲಿನಿ ದೇವದೇವಿ ಸರ್ವಭೂತ  ಸಂಹಾರ ಕಾರಿಕೇ ಜಾತವೇದಸಿ ಜ್ವಲಂತಿ ಜ್ವಲ ಜ್ವಲ ಪ್ರಜ್ವಲ ಪ್ರಜ್ವಲ ಹ್ರಾಂ ಹ್ರೀಂ ಹ್ರೂಂ ರರ ರರ ರರರ ಹುಂ ಫಟ್ ಸ್ವಾಹಾ~। ಸಹಸ್ರಾರ ಹುಂ ಫಟ್~।\\ ಭೂರ್ಭುವಃಸುವರೋಮಿತಿ ದಿಗ್ಬಂಧಃ ॥

೪ ಸಮಸ್ತ ಪ್ರಕಟ ಗುಪ್ತ ಗುಪ್ತತರ ಸಂಪ್ರದಾಯ ಕುಲೋತ್ತೀರ್ಣ ನಿಗರ್ಭ ರಹಸ್ಯಾ\-ತಿರಹಸ್ಯ ಪರಾಪರಾತಿರಹಸ್ಯ ಯೋಗಿನೀ ದೇವತಾಭ್ಯೋ ನಮಃ ॥

೪ ಐಂ ಹ್ರಃ ಅಸ್ತ್ರಾಯ ಫಟ್ ॥

೪ ಶ್ರೀಗುರೋ ದಕ್ಷಿಣಾಮೂರ್ತೇ ಭಕ್ತಾನುಗ್ರಹಕಾರಕ~।\\
ಅನುಜ್ಞಾಂ ದೇಹಿ ಭಗವನ್ ಶ್ರೀಚಕ್ರ ಯಜನಾಯ ಮೇ ॥

೪ ಅತಿಕ್ರೂರ ಮಹಾಕಾಯ ಕಲ್ಪಾಂತದಹನೋಪಮ~।\\
ಭೈರವಾಯ ನಮಸ್ತುಭ್ಯಮನುಜ್ಞಾಂ ದಾತುಮರ್ಹಸಿ ॥

೪ ಮೂಲಶೃಂಗಾಟಕಾತ್ ಸುಷುಮ್ನಾಪಥೇನ ಜೀವಶಿವಂ ಪರಮಶಿವಪದೇ \\ಯೋಜಯಾಮಿ ಸ್ವಾಹಾ~।\\
ಯಂ ೮ ಸಂಕೋಚಶರೀರಂ ಶೋಷಯ ಶೋಷಯ ಸ್ವಾಹಾ~।\\
ರಂ ೮ ಸಂಕೋಚಶರೀರಂ ದಹ ದಹ ಪಚ ಪಚ ಸ್ವಾಹಾ~।\\
ವಂ ೮ ಪರಮಶಿವಾಮೃತಂ ವರ್ಷಯ ವರ್ಷಯ ಸ್ವಾಹಾ~।\\
ಲಂ ೮ ಶಾಂಭವಶರೀರಮುತ್ಪಾದಯೋತ್ಪಾದಯ ಸ್ವಾಹಾ~।\\
ಹಂಸಃ ಸೋಹಮವತರಾವತರ ಶಿವಪದಾತ್ ಜೀವ ಸುಷುಮ್ನಾಪಥೇನ ಪ್ರವಿಶ ಮೂಲಶೃಂಗಾಟಕಮುಲ್ಲಸೋಲ್ಲಸ ಜ್ವಲ ಜ್ವಲ ಪ್ರಜ್ವಲ ಪ್ರಜ್ವಲ ಹಂಸಃ ಸೋಹಂ ಸ್ವಾಹಾ ॥\\
೪ ಆಂ ಸೋಹಂ (ಇತಿ ತ್ರಿಃ ಹೃದಿ) ಇತಿ ಭೂತಶುದ್ಧಿಃ ॥\\
ತತಃ ಪ್ರಾಣಾನಾಯಮ್ಯ, ದೇಶಕಾಲೌ ಸಂಕೀರ್ತ್ಯ ಸಂಕಲ್ಪಯೇತ್ ॥
\section{ಅಥ ಪ್ರಾಣಪ್ರತಿಷ್ಠಾ}
\addcontentsline{toc}{section}{ಅಥ ಪ್ರಾಣಪ್ರತಿಷ್ಠಾ}
ಓಂ ಐಂಹ್ರೀಂಶ್ರೀಂ ಆಂಹ್ರೀಂಕ್ರೋಂ ಯಂರಂಲಂವಂಶಂಷಂಸಂಹಂ ॥ ಓಂ ಹಂಸಃ ಸೋಽಹಂ । ಸೋಽಹಂ ಹಂಸಃ ಶಿವಃ ॥ ಶ್ರೀಚಕ್ರಸ್ಯ  ಪ್ರಾಣಾ ಇಹ ಪ್ರಾಣಾಃ~।\\
೪ ಆಂಹ್ರೀಂಕ್ರೋಂ ಶ್ರೀಚಕ್ರಸ್ಯ ಜೀವ ಇಹ ಸ್ಥಿತಃ~। ಸರ್ವೇಂದ್ರಿಯಾಣಿ~। ವಾಙ್ಮನಸ್ತ್ವಕ್ಚಕ್ಷುಃ ಶ್ರೋತ್ರ ಜಿಹ್ವಾಘ್ರಾಣಪ್ರಾಣಾ ಇಹೈವಾಗತ್ಯ  ಅಸ್ಮಿನ್ ಚಕ್ರೇ ಸುಖಂ ಚಿರಂ ತಿಷ್ಠಂತು ಸ್ವಾಹಾ ॥
\section{ಲಘುನ್ಯಾಸಃ}
\addcontentsline{toc}{section}{ಲಘುನ್ಯಾಸಃ}
ಓಂ ಅಥಾತ್ಮಾನಂ ಶಿವಾತ್ಮಾನಂ ಶ್ರೀರುದ್ರರೂಪಂ ಧ್ಯಾಯೇತ್ ॥
\dhyana{ಶುದ್ಧಸ್ಫಟಿಕಸಂಕಾಶಂ ತ್ರಿಣೇತ್ರಂ ಪಂಚವಕ್ತ್ರಕಂ ।\\
ಗಂಗಾಧರಂ ದಶಭುಜಂ ಸರ್ವಾಭರಣಭೂಷಿತಂ ॥

ನೀಲಗ್ರೀವಂ ಶಶಾಂಕಾಂಕಂ ನಾಗಯಜ್ಞೋಪವೀತಿನಂ ।\\
ವ್ಯಾಘ್ರಚರ್ಮೋತ್ತರೀಯಂ ಚ ವರೇಣ್ಯಮಭಯಪ್ರದಂ ॥

ಕಮಂಡಲ್ವಕ್ಷಸೂತ್ರಾಭ್ಯಾಮನ್ವಿತಂ ಶೂಲಪಾಣಿನಂ ।\\
ಜ್ವಲಂತಂ ಪಿಂಗಳಜಟಾಶಿಖಾಮುದ್ದ್ಯೋತಧಾರಿಣಂ ॥

ವೃಷಸ್ಕಂಧಸಮಾರೂಢಂ ಉಮಾದೇಹಾರ್ಧಧಾರಿಣಂ ।\\
ಅಮೃತೇನಾಪ್ಲುತಂ ಶಾಂತಂ ದಿವ್ಯಭೋಗಸಮನ್ವಿತಂ ॥

ದಿಗ್ದೇವತಾಸಮಾಯುಕ್ತಂ ಸುರಾಸುರನಮಸ್ಕೃತಂ ।\\
ನಿತ್ಯಂ ಚ ಶಾಶ್ವತಂ ಶುದ್ಧಂ ಧ್ರುವಮಕ್ಷರಮವ್ಯಯಂ ॥

ಸರ್ವವ್ಯಾಪಿನಮೀಶಾನಂ ರುದ್ರಂ ವೈ ವಿಶ್ವರೂಪಿಣಂ ।\\
ಏವಂ ಧ್ಯಾತ್ವಾ ದ್ವಿಜಃ ಸಮ್ಯಕ್ ತತೋ ಯಜನಮಾರಭೇತ್ ॥}

ಅಥಾತ್ಮನಿ ದೇವತಾಃ ಸ್ಥಾಪಯೇತ್ ॥

ಪ್ರಜನನೇ ಬ್ರಹ್ಮಾ ತಿಷ್ಠತು । ಪಾದಯೋರ್ವಿಷ್ಣುಸ್ತಿಷ್ಠತು । ಹಸ್ತಯೋರ್ಹರಸ್ತಿಷ್ಠತು । ಬಾಹ್ವೋರಿಂದ್ರಸ್ತಿಷ್ಠತು । ಜಠರೇಽಅಗ್ನಿಸ್ತಿಷ್ಠತು । ಹೃದಯೇ ಶಿವಸ್ತಿಷ್ಠತು । ಕಂಠೇ ವಸವಸ್ತಿಷ್ಠಂತು । ವಕ್ತ್ರೇ ಸರಸ್ವತೀ ತಿಷ್ಠತು । ನಾಸಿಕಯೋರ್ವಾಯುಸ್ತಿಷ್ಠತು । ನಯನಯೋಶ್ಚಂದ್ರಾದಿತ್ಯೌ ತಿಷ್ಠೇತಾಂ । ಕರ್ಣಯೋರಶ್ವಿನೌ ತಿಷ್ಠೇತಾಂ । ಲಲಾಟೇ ರುದ್ರಾಸ್ತಿಷ್ಠಂತು । ಮೂರ್ಧ್ನ್ಯಾದಿತ್ಯಾಸ್ತಿಷ್ಠಂತು । ಶಿರಸಿ ಮಹಾದೇವಸ್ತಿಷ್ಠತು । ಶಿಖಾಯಾಂ ವಾಮದೇವಾಸ್ತಿಷ್ಠತು । ಪೃಷ್ಠೇ ಪಿನಾಕೀ ತಿಷ್ಠತು । ಪುರತಃ ಶೂಲೀ ತಿಷ್ಠತು । ಪಾರ್ಶ್ವಯೋಃ ಶಿವಾಶಂಕರೌ ತಿಷ್ಠೇತಾಂ । ಸರ್ವತೋ ವಾಯುಸ್ತಿಷ್ಠತು । ತತೋ ಬಹಿಃ ಸರ್ವತೋಽಗ್ನಿರ್ಜ್ವಾಲಾಮಾಲಾಪರಿವೃತಸ್ತಿಷ್ಠತು । ಸರ್ವೇಷ್ವಂಗೇಷು ಸರ್ವಾ ದೇವತಾ ಯಥಾಸ್ಥಾನಂ ತಿಷ್ಠಂತು। ಮಾಂ ರಕ್ಷಂತು।

\as{ಅ॒ಗ್ನಿರ್ಮೇ॑} ವಾ॒ಚಿ ಶ್ರಿ॒ತಃ । ವಾಗ್ಘೃದ॑ಯೇ । ಹೃದ॑ಯಂ॒ ಮಯಿ॑ । ಅ॒ಹಮ॒ಮೃತೇ᳚ । ಅ॒ಮೃತಂ॒ ಬ್ರಹ್ಮ॑ಣಿ । \as{ವಾ॒ಯುರ್ಮೇ᳚} ಪ್ರಾ॒ಣೇ ಶ್ರಿ॒ತಃ । ಪ್ರಾ॒ಣೋ ಹೃದ॑ಯೇ । ಹೃದ॑ಯಂ॒ ಮಯಿ॑ । ಅ॒ಹಮ॒ಮೃತೇ᳚ । ಅ॒ಮೃತಂ॒ ಬ್ರಹ್ಮ॑ಣಿ । \as{ಸೂರ್ಯೋ॑} ಮೇ॒ ಚಕ್ಷು॑ಷಿ ಶ್ರಿ॒ತಃ । ಚ॒ಕ್ಷುರ್ಹೃದ॑ಯೇ । ಹೃದ॑ಯಂ॒ ಮಯಿ॑ । ಅ॒ಹಮ॒ಮೃತೇ᳚ । ಅ॒ಮೃತಂ॒ ಬ್ರಹ್ಮ॑ಣಿ । \as{ಚಂ॒ದ್ರ॒ಮಾ} ಮೇ॒ ಮನ॑ಸಿ ಶ್ರಿ॒ತಃ । ಮನೋ॒ ಹೃದ॑ಯೇ । ಹೃದ॑ಯಂ॒ ಮಯಿ॑ । ಅ॒ಹಮ॒ಮೃತೇ᳚ । ಅ॒ಮೃತಂ॒ ಬ್ರಹ್ಮ॑ಣಿ । \as{ದಿಶೋ॑} ಮೇ॒ ಶ್ರೋತ್ರೇ᳚ ಶ್ರಿ॒ತಾಃ । ಶ್ರೋತ್ರ॒ಗ್ಂ॒ ಹೃದ॑ಯೇ । ಹೃದ॑ಯಂ॒ ಮಯಿ॑ । ಅ॒ಹಮ॒ಮೃತೇ᳚ । ಅ॒ಮೃತಂ॒ ಬ್ರಹ್ಮ॑ಣಿ । \as{ಆಪೋ॑} ಮೇ॒ ರೇತ॑ಸಿ ಶ್ರಿ॒ತಾಃ । ರೇ॒ತೋ ಹೃದ॑ಯೇ । ಹೃದ॑ಯಂ॒ ಮಯಿ॑ । ಅ॒ಹಮ॒ಮೃತೇ᳚ । ಅ॒ಮೃತಂ॒ ಬ್ರಹ್ಮ॑ಣಿ । \as{ಪೃ॒ಥಿ॒ವೀ} ಮೇ ಶರೀ॑ರೇ ಶ್ರಿ॒ತಾ । ಶರೀ॑ರ॒ಗ್ಂ ಹೃದ॑ಯೇ । ಹೃದ॑ಯಂ॒ ಮಯಿ॑ । ಅ॒ಹಮ॒ಮೃತೇ᳚ । ಅ॒ಮೃತಂ॒ ಬ್ರಹ್ಮ॑ಣಿ । \as{ಓ॒ಷ॒ಧಿ॒ ವ॒ನ॒ಸ್ಪ॒ತಯೋ॑} ಮೇ॒ ಲೋಮ॑ಸು ಶ್ರಿ॒ತಾಃ । ಲೋಮಾ॑ನಿ॒ ಹೃದ॑ಯೇ । ಹೃದ॑ಯಂ॒ ಮಯಿ॑ । ಅ॒ಹಮ॒ಮೃತೇ᳚ । ಅ॒ಮೃತಂ॒ ಬ್ರಹ್ಮ॑ಣಿ । \as{ಇಂದ್ರೋ॑} ಮೇ॒ ಬಲೇ᳚ ಶ್ರಿ॒ತಃ । ಬಲ॒ಗ್ಂ॒ ಹೃದ॑ಯೇ । ಹೃದ॑ಯಂ॒ ಮಯಿ॑ । ಅ॒ಹಮ॒ಮೃತೇ᳚ । ಅ॒ಮೃತಂ॒ ಬ್ರಹ್ಮ॑ಣಿ । \as{ಪ॒ರ್ಜನ್ಯೋ॑} ಮೇ ಮೂ॒ರ್ಧ್ನಿ ಶ್ರಿ॒ತಃ । ಮೂ॒ರ್ಧಾ ಹೃದ॑ಯೇ । ಹೃದ॑ಯಂ॒ ಮಯಿ॑ । ಅ॒ಹಮ॒ಮೃತೇ᳚ । ಅ॒ಮೃತಂ॒ ಬ್ರಹ್ಮ॑ಣಿ । \as{ಈಶಾ॑ನೋ} ಮೇ ಮ॒ನ್ಯೌ ಶ್ರಿ॒ತಃ । ಮ॒ನ್ಯುರ್-ಹೃದ॑ಯೇ । ಹೃದ॑ಯಂ॒ ಮಯಿ॑ । ಅ॒ಹಮ॒ಮೃತೇ᳚ । ಅ॒ಮೃತಂ॒ ಬ್ರಹ್ಮ॑ಣಿ । \as{ಆ॒ತ್ಮಾ} ಮ॑ ಆ॒ತ್ಮನಿ॑ ಶ್ರಿ॒ತಃ । ಆ॒ತ್ಮಾ ಹೃದ॑ಯೇ । ಹೃದ॑ಯಂ॒ ಮಯಿ॑ । ಅ॒ಹಮ॒ಮೃತೇ᳚ । ಅ॒ಮೃತಂ॒ ಬ್ರಹ್ಮ॑ಣಿ । ಪುನ॑ರ್ಮ ಆ॒ತ್ಮಾ ಪುನ॒ರಾಯು॒ರಾಗಾ᳚ತ್ । ಪುನಃ॑ ಪ್ರಾ॒ಣಃ ಪು॑ನ॒ರಾಕೂ॑ತ॒ಮಾಗಾ᳚ತ್ । ವೈ॒ಶ್ವಾ॒ನ॒ರೋ ರ॒ಶ್ಮಿಭಿ॑ರ್ವಾವೃಧಾ॒ನಃ । ಅಂ॒ತಸ್ತಿ॑ಷ್ಠತ್ವ॒ಮೃತ॑ಸ್ಯ ಗೋ॒ಪಾಃ ॥

ಏವಂ ಯಥಾಲಿಂಗಮಂಗಾನಿ ಸಂಮೃಜ್ಯ, ದೇವಮಾತ್ಮಾನಂ ಚ ಪ್ರತ್ಯಾರಾಧಯೇತ್ ॥\\
\dhyana{ಆರಾಧಿತೋ ಮನುಷ್ಯೈಸ್ತ್ವಂ ಸಿದ್ಧೈರ್ದೇವಾಸುರಾದಿಭಿಃ ।\\
ಆರಾಧಯಾಮಿ ಭಕ್ತ್ಯಾ ತ್ವಾಂ ಮಾಂ ಗೃಹಾಣ ಮಹೇಶ್ವರ ॥}

ಆ ತ್ವಾ ವಹಂತು ಹರಯಃ ಸಚೇತಸಃ ಶ್ವೇತೈರಶ್ವೈಃ  ಸಹಕೇತುಮದ್ಭಿಃ~। ವಾತಾಜಿರೈರ್ಬಲವದ್ಭಿರ್ಮನೋಜವೈರಾಯಾಹಿ ಶೀಘ್ರಂ ಮಮ ಹವ್ಯಾಯ ಶರ್ವೋಮ್ । ಈಶಾನಮಾವಾಹಯಾಮೀತ್ಯಾವಾಹ್ಯ\\
\dhyana{ಶಂಕರಸ್ಯ ಚರಿತಂ ಕಥಾಮೃತಂ ಚಂದ್ರಶೇಖರ ಗುಣಾನುಕೀರ್ತನಮ್ ।\\
ನೀಲಕಂಠ ತವ ಪಾದಸೇವನಂ ಸಂಭವಂತು ಮಮ ಜನ್ಮಜನ್ಮನಿ ॥\\
ಸ್ವಾಮಿನ್ ಸರ್ವಜಗನ್ನಾಥ ಯಾವತ್ಪೂಜಾವಸಾನಕಮ್ ।\\
ತಾವತ್ತ್ವಂ ಪ್ರೀತಿ ಭಾವೇನ ಬಿಂಬೇಽಸ್ಮಿನ್ ಸನ್ನಿಧಿಂ ಕುರು ॥}
\section{ಮಾತೃಕಾಸರಸ್ವತೀನ್ಯಾಸಃ}
\addcontentsline{toc}{section}{ಮಾತೃಕಾಸರಸ್ವತೀನ್ಯಾಸಃ}
ಅಸ್ಯ ಶ್ರೀಮಾತೃಕಾಸರಸ್ವತೀ ನ್ಯಾಸಮಂತ್ರಸ್ಯ ಬ್ರಹ್ಮಣೇ ಋಷಯೇ ನಮಃ (ಶಿರಸಿ) ಗಾಯತ್ರೀ ಛಂದಸೇ ನಮಃ(ಮುಖೇ)~। ಮಾತೃಕಾ ಸರಸ್ವತೀ ದೇವತಾಯೈ ನಮಃ (ಹೃದಯೇ)~। ಹಲ್ಭ್ಯೋ ಬೀಜೇಭ್ಯೋ ನಮಃ (ಗುಹ್ಯೇ)~। ಸ್ವರೇಭ್ಯಃ ಶಕ್ತಿಭ್ಯೋ ನಮಃ (ಪಾದಯೋ)~। ಬಿಂದುಭ್ಯಃ ಕೀಲಕೇಭ್ಯೋ ನಮಃ (ನಾಭೌ)~। ಶ್ರೀವಿದ್ಯಾಂಗತ್ವೇನ ನ್ಯಾಸೇ ವಿನಿಯೋಗಾಯ ನಮಃ(ಸರ್ವಾಂಗೇ)~॥

ಅಂ ಆಂ ಇಂ ಈಂ ಉಂ ಊಂ ಋಂ ೠಂ ಲೃಂ ಲೄಂ ಏಂ ಐಂ ಓಂ ಔಂ ಅಂ ಅಃ ಕಂಖಂಗಂಘಂಙಂ ಚಂಛಂಜಂಝಂಞಂ ಟಂಠಂಡಂಢಂಣಂ ತಂಥಂದಂಧಂನಂ ಪಂಫಂಬಂಭಂಮಂ ಯಂರಂಲಂವಂ ಶಂಷಂಸಂಹಂಳಂಕ್ಷಂ~। (ಇತಿ ಅಂಜಲಿನಾ ತ್ರಿಃ ವ್ಯಾಪಕಂ ನ್ಯಸ್ಯ)

ಓಂ ಐಂಹ್ರೀಂಶ್ರೀಂ ಐಂಕ್ಲೀಂಸೌಃ\\ಅಂ ಕಂ ಖಂ ಗಂ ಘಂ ಙಂ ಆಂ ಅಂಗುಷ್ಠಾಭ್ಯಾಂ ನಮಃ ।\\
೭ ಇಂ ಚಂಛಂಜಂಝಂಞಂ ಈಂ ತರ್ಜನೀಭ್ಯಾಂ ನಮಃ ।\\
೭ ಉಂ ಟಂಠಂಡಂಢಂಣಂ ಊಂ ಮಧ್ಯಮಾಭ್ಯಾಂ ನಮಃ ।\\
೭ ಏಂ ತಂಥಂದಂಧಂನಂ ಐಂ ಅನಾಮಿಕಾಭ್ಯಾಂ ನಮಃ ।\\
೭ ಓಂ ಪಂಫಂಬಂಭಂಮಂ ಔಂ ಕನಿಷ್ಠಿಕಾಭ್ಯಾಂ ನಮಃ ।\\
೭ ಅಂ ಯಂರಂಲಂವಂಶಂಷಂಸಂಹಂಳಂಕ್ಷಂ ಅಃ ಕರತಲಕರಪೃಷ್ಠಾಭ್ಯಾಂ ನಮಃ।\\
ಏವಮೇವಾಂಗನ್ಯಾಸಂ ವಿಧಾಯ ಧ್ಯಾಯೇತ್ \\
\dhyana{ಪಂಚಾಶದ್ವರ್ಣಭೇದೈರ್ವಿಹಿತವದನದೋಃಪಾದಹೃತ್ಕುಕ್ಷಿವಕ್ಷೋ\\
ದೇಶಾಂ ಭಾಸ್ವತ್ಕಪರ್ದಾಕಲಿತಶಶಿಕಲಾಮಿಂದುಕುಂದಾವದಾತಾಂ~।\\
ಅಕ್ಷಸ್ರಕ್ಕುಂಭಚಿಂತಾಲಿಖಿತವರಕರಾಂ ತ್ರೀಕ್ಷಣಾಂ ಪದ್ಮಸಂಸ್ಥಾಂ\\
ಅಚ್ಛಾಕಲ್ಪಾಮತುಚ್ಛಸ್ತನಜಘನಭರಾಂ ಭಾರತೀಂ ತಾಂ ನಮಾಮಿ ॥}\\
ಲಮಿತ್ಯಾದಿನಾ ಪಂಚೋಪಚಾರ ಪೂಜಾ~॥\\
\as{ಓಂ ಐಂಹ್ರೀಂಶ್ರೀಂ ಐಂಕ್ಲೀಂಸೌಃ ಅಂ} ನಮಃ ಹಂಸಃ~।(ಶಿರಸಿ)\\
\as{೭ ಆಂ} ನಮಃ ಹಂಸಃ~।(ಮುಖವೃತ್ತೇ)\\
\as{೭ ಇಂ} ನಮಃ ಹಂಸಃ~।(ದಕ್ಷನೇತ್ರೇ)\\
\as{೭ ಈಂ} ನಮಃ ಹಂಸಃ~।(ವಾಮನೇತ್ರೇ)\\
\as{೭ ಉಂ} ನಮಃ ಹಂಸಃ~।(ದಕ್ಷಕರ್ಣೇ)\\
\as{೭ ಊಂ} ನಮಃ ಹಂಸಃ~।(ವಾಮಕರ್ಣೇ)\\
\as{೭ ಋಂ} ನಮಃ ಹಂಸಃ~।(ದಕ್ಷನಾಸಾಯಾಂ)\\
\as{೭ ೠಂ} ನಮಃ ಹಂಸಃ~।(ವಾಮನಾಸಾಯಾಂ)\\
\as{೭ ಲೃಂ} ನಮಃ ಹಂಸಃ~।(ದಕ್ಷಗಂಡೇ)\\
\as{೭ ಲೄಂ} ನಮಃ ಹಂಸಃ~।(ವಾಮಗಂಡೇ)\\
\as{೭ ಏಂ} ನಮಃ ಹಂಸಃ~।(ಊರ್ಧ್ವೋಷ್ಠೇ)\\
\as{೭ ಐಂ} ನಮಃ ಹಂಸಃ~।(ಅಧರೋಷ್ಠೇ)\\
\as{೭ ಓಂ} ನಮಃ ಹಂಸಃ~।(ಊರ್ಧ್ವದಂತಪಂಕ್ತೌ)\\
\as{೭ ಔಂ} ನಮಃ ಹಂಸಃ~।(ಅಧೋದಂತಪಂಕ್ತೌ)\\
\as{೭ ಅಂ} ನಮಃ ಹಂಸಃ~।(ಜಿಹ್ವಾಯಾಂ)\\
\as{೭ ಅಃ} ನಮಃ ಹಂಸಃ~।(ಕಂಠೇ)\\
\as{೭ ಕಂ} ನಮಃ ಹಂಸಃ~।(ದಕ್ಷ ಬಾಹುಮೂಲೇ)\\
\as{೭ ಖಂ} ನಮಃ ಹಂಸಃ~।(ದಕ್ಷಕೂರ್ಪರೇ)\\
\as{೭ ಗಂ} ನಮಃ ಹಂಸಃ~।(ದಕ್ಷಮಣಿಬಂಧೇ)\\
\as{೭ ಘಂ} ನಮಃ ಹಂಸಃ~।(ದಕ್ಷಕರಾಂಗುಲಿಮೂಲೇ)\\
\as{೭ ಙಂ} ನಮಃ ಹಂಸಃ~।(ದಕ್ಷಕರಾಂಗುಲ್ಯಗ್ರೇ)\\
\as{೭ ಚಂ} ನಮಃ ಹಂಸಃ~।(ವಾಮಬಾಹುಮೂಲೇ)\\
\as{೭ ಛಂ} ನಮಃ ಹಂಸಃ~।(ವಾಮಕೂರ್ಪರೇ)\\
\as{೭ ಜಂ} ನಮಃ ಹಂಸಃ~।(ವಾಮಮಣಿಬಂಧೇ)\\
\as{೭ ಝಂ} ನಮಃ ಹಂಸಃ~।(ವಾಮಕರಾಂಗುಲಿಮೂಲೇ)\\
\as{೭ ಞಂ} ನಮಃ ಹಂಸಃ~।(ವಾಮಕರಾಂಗುಲ್ಯಗ್ರೇ)\\
\as{೭ ಟಂ} ನಮಃ ಹಂಸಃ~।(ದಕ್ಷೋರುಮೂಲೇ)\\
\as{೭ ಠಂ} ನಮಃ ಹಂಸಃ~।(ದಕ್ಷಜಾನುನಿ)\\
\as{೭ ಡಂ} ನಮಃ ಹಂಸಃ~।(ದಕ್ಷಗುಲ್ಫೇ)\\
\as{೭ ಢಂ} ನಮಃ ಹಂಸಃ~।(ದಕ್ಷಪಾದಾಂಗುಲಿಮೂಲೇ)\\
\as{೭ ಣಂ} ನಮಃ ಹಂಸಃ~।(ದಕ್ಷಪಾದಾಂಗುಲ್ಯಗ್ರೇ)\\
\as{೭ ತಂ} ನಮಃ ಹಂಸಃ~।(ವಾಮೋರುಮೂಲೇ)\\
\as{೭ ಥಂ} ನಮಃ ಹಂಸಃ~।(ವಾಮಜಾನುನಿ)\\
\as{೭ ದಂ} ನಮಃ ಹಂಸಃ~।(ವಾಮಗುಲ್ಫೇ)\\
\as{೭ ಧಂ} ನಮಃ ಹಂಸಃ~।(ವಾಮಪಾದಾಂಗುಲಿಮೂಲೇ)\\
\as{೭ ನಂ} ನಮಃ ಹಂಸಃ~।(ವಾಮಪಾದಾಂಗುಲ್ಯಗ್ರೇ)\\
\as{೭ ಪಂ} ನಮಃ ಹಂಸಃ~।(ದಕ್ಷಪಾರ್ಶ್ವೇ)\\
\as{೭ ಫಂ} ನಮಃ ಹಂಸಃ~।(ವಾಮಪಾರ್ಶ್ವೇ)\\
\as{೭ ಬಂ} ನಮಃ ಹಂಸಃ~।(ಪೃಷ್ಠೇ)\\
\as{೭ ಭಂ} ನಮಃ ಹಂಸಃ~।(ನಾಭೌ)\\
\as{೭ ಮಂ} ನಮಃ ಹಂಸಃ~।(ಜಠರೇ)\\
\as{೭ ಯಂ} ನಮಃ ಹಂಸಃ~।(ಹೃದಿ )\\
\as{೭ ರಂ} ನಮಃ ಹಂಸಃ~।(ದಕ್ಷಾಂಸೇ)\\
\as{೭ ಲಂ} ನಮಃ ಹಂಸಃ~।(ಕಕುದಿ)\\
\as{೭ ವಂ} ನಮಃ ಹಂಸಃ~।(ವಾಮಾಂಸೇ)\\
\as{೭ ಶಂ} ನಮಃ ಹಂಸಃ~।(ಹೃದಯಾದಿ ದಕ್ಷ ಕರಾಂಗುಲ್ಯಂತಂ )\\
\as{೭ ಷಂ} ನಮಃ ಹಂಸಃ~।(ಹೃದಯಾದಿ ವಾಮ ಕರಾಂಗುಲ್ಯಂತಂ )\\
\as{೭ ಸಂ} ನಮಃ ಹಂಸಃ~।(ಹೃದಯಾದಿ ದಕ್ಷ ಪಾದಾಂತಂ )\\
\as{೭ ಹಂ} ನಮಃ ಹಂಸಃ~।(ಹೃದಯಾದಿ ವಾಮ ಪಾದಾಂತಂ )\\
\as{೭ ಳಂ} ನಮಃ ಹಂಸಃ~।(ಕಟ್ಯಾದಿ ಪಾದಪರ್ಯಂತಂ)\\
\as{೭ ಕ್ಷಂ} ನಮಃ ಹಂಸಃ~।(ಕಟ್ಯಾದಿ ಶಿರಃಪರ್ಯಂತಂ )\\
\dhyana{ಆಧಾರೇ ಲಿಂಗನಾಭೌ ಹೃದಯಸರಸಿಜೇ ತಾಲುಮೂಲೇ ಲಲಾಟೇ\\
ದ್ವೇ ಪತ್ರೇ ಷೋಡಶಾರೇ ದ್ವಿದಶದಶದಲೇ ದ್ವಾದಶಾರ್ಧೇ ಚತುಷ್ಕೇ~।\\
ವಾಸಾಂತೇ ಬಾಲಮಧ್ಯೇ ಡಫಕಠಸಹಿತೇ ಕಂಠದೇಶೇ ಸ್ವರಾಣಾಂ\\
ಹಂ ಕ್ಷಂ ತತ್ವಾರ್ಥಯುಕ್ತಂ ಸಕಲದಲಗತಂ ವರ್ಣರೂಪಂ ನಮಾಮಿ ॥}

\as{೭ ಅಂ} ನಮಃ ಹಂಸಃ, \as{ಆಂ} ನಮಃ ಹಂಸಃ+ + +\as{ಅಃ} ನಮಃ ಹಂಸಃ॥ಕಂಠೇ\\
\as{೭ ಕಂ} ನಮಃ ಹಂಸಃ + + \as{ಠಂ} ನಮಃ ಹಂಸಃ ॥ಹೃದಯೇ\\
\as{೭ ಡಂ} ನಮಃ ಹಂಸಃ+ + \as{ಫಂ} ನಮಃ ಹಂಸಃ ॥ನಾಭೌ\\
\as{೭ ಬಂ} ನಮಃ ಹಂಸಃ + + \as{ಲಂ} ನಮಃ ಹಂಸಃ ॥ಗುಹ್ಯೇ\\
\as{೭ ವಂ} ನಮಃ ಹಂಸಃ + + \as{ಸಂ} ನಮಃ ಹಂಸಃ ॥ಆಧಾರೇ\\
\as{೭ ಹಂ} ನಮಃ ಹಂಸಃ, \as{ಕ್ಷಂ} ನಮಃ ಹಂಸಃ ॥ಭ್ರೂಮಧ್ಯೇ\\
\as{೭ ಅಂ} ನಮಃ ಹಂಸಃ + +\as{ಕ್ಷಂ} ನಮಃ ಹಂಸಃ॥ಸಹಸ್ರದಲಕಮಲೇ\\
 ಪೂರ್ವವತ್ ಉತ್ತರನ್ಯಾಸಃ ॥
\section{ಕರಶುದ್ಧಿನ್ಯಾಸಃ}
\addcontentsline{toc}{section}{ಕರಶುದ್ಧಿನ್ಯಾಸಃ}
\as{೪ ಅಂ} ನಮಃ (ದಕ್ಷಕರಮಧ್ಯೇ)\\
\as{೪ ಆಂ} ನಮಃ (ದಕ್ಷಕರಪೃಷ್ಠೇ)\\
\as{೪ ಸೌಃ} ನಮಃ (ದಕ್ಷಕರಪಾರ್ಶ್ವಯೋಃ)\\
\as{೪ ಅಂ} ನಮಃ (ವಾಮಕರಮಧ್ಯೇ)\\
\as{೪ ಆಂ} ನಮಃ (ವಾಮಕರಪೃಷ್ಠೇ)\\
\as{೪ ಸೌಃ} ನಮಃ (ವಾಮಕರಪಾರ್ಶ್ವಯೋಃ)\\
\as{೪ ಅಂ} ನಮಃ (ಮಧ್ಯಮಯೋಃ)\\
\as{೪ ಆಂ} ನಮಃ (ಅನಾಮಿಕಯೋಃ)\\
\as{೪ ಸೌಃ} ನಮಃ (ಕನಿಷ್ಠಿಕಯೋಃ)\\
\as{೪ ಅಂ} ನಮಃ (ಅಂಗುಷ್ಠಯೋಃ)\\
\as{೪ ಆಂ} ನಮಃ (ತರ್ಜನ್ಯೋಃ)\\
\as{೪ ಸೌಃ} ನಮಃ (ಉಭಯ ಕರತಲ ಕರಪೃಷ್ಠಯೋಃ)

\section{ಆತ್ಮರಕ್ಷಾನ್ಯಾಸಃ}
\addcontentsline{toc}{section}{ಆತ್ಮರಕ್ಷಾನ್ಯಾಸಃ}
\as{ಓಂ ಐಂಹ್ರೀಂಶ್ರೀಂ ಐಂಕ್ಲೀಂಸೌಃ} ಶ್ರೀಮಹಾತ್ರಿಪುರಸುಂದರಿ ಆತ್ಮಾನಂ ರಕ್ಷರಕ್ಷ॥\\ಇತಿ ಹೃದಿ ಅಂಜಲಿಸಮರ್ಪಣಂ~।
\section{ಬಾಲಾಷಡಂಗನ್ಯಾಸಃ}
\addcontentsline{toc}{section}{ಬಾಲಾಷಡಂಗನ್ಯಾಸಃ}
\as{೪ ಐಂ} ಹೃದಯಾಯ ನಮಃ\\
\as{೪ ಕ್ಲೀಂ} ಶಿರಸೇ ಸ್ವಾಹಾ\\
\as{೪ ಸೌಃ }ಶಿಖಾಯೈ ವಷಟ್\\
\as{೪ ಐಂ} ಕವಚಾಯ ಹುಂ\\
\as{೪ ಕ್ಲೀಂ }ನೇತ್ರತ್ರಯಾಯ ವೌಷಟ್\\
\as{೪ ಸೌಃ} ಅಸ್ತ್ರಾಯ ಫಟ್ ॥
\section{ಚತುರಾಸನನ್ಯಾಸಃ}
\addcontentsline{toc}{section}{ಚತುರಾಸನನ್ಯಾಸಃ}
{\bfseries ೪ ಹ್ರೀಂಕ್ಲೀಂಸೌಃ }ದೇವ್ಯಾತ್ಮಾಸನಾಯ ನಮಃ। ಪಾದಯೋಃ\\
{\bfseries ೪ ಹೈಂ ಹ್‌ಕ್ಲೀಂ ಹ್ಸೌಃ} ಶ್ರೀಚಕ್ರಾಸನಾಯ ನಮಃ। ಜಾನುನೋಃ\\
{\bfseries ೪ ಹ್‌ಸೈಂ ಹ್‌ಸ್‌ಕ್ಲೀಂ ಹ್‌ಸ್ಸೌಃ} ಸರ್ವಮಂತ್ರಾಸನಾಯ ನಮಃ। ಊರುಮೂಲಯೋಃ\\
{\bfseries ೪ ಹ್ರೀಂಕ್ಲೀಂಬ್ಲೇಂ} ಸಾಧ್ಯಸಿದ್ಧಾಸನಾಯ ನಮಃ। ಆಧಾರೇ
\section{ವಶಿನ್ಯಾದಿವಾಗ್ದೇವತಾನ್ಯಾಸಃ}
\addcontentsline{toc}{section}{ವಶಿನ್ಯಾದಿವಾಗ್ದೇವತಾನ್ಯಾಸಃ}
\as{ಓಂ ಐಂಹ್ರೀಂಶ್ರೀಂ ಅಂ ಆಂ ಇಂ ಈಂ+ + ಅಂ ಅಃ~। ರ್ಬ್ಲೂಂ॥}\\ ವಶಿನೀ ವಾಗ್ದೇವತಾಯೈ ನಮಃ। ಶಿರಸಿ\\
\as{೪ ಕಂ ಖಂ ಗಂ ಘಂ ಙಂ। ಕ್‌ಲ್‌ಹ್ರೀಂ॥ ॥}\\ ಕಾಮೇಶ್ವರೀ ವಾಗ್ದೇವತಾಯೈ ನಮಃ। ಲಲಾಟೇ\\
\as{೪ ಚಂ ಛಂ ಜಂ ಝಂ ಞಂ। ನ್‌ವ್ಲೀಂ ॥}\\ ಮೋದಿನೀ ವಾಗ್ದೇವತಾಯೈ ನಮಃ। ಭ್ರೂಮಧ್ಯೇ\\
\as{೪ ಟಂ ಠಂ ಡಂ ಢಂ ಣಂ। ಯ್ಲೂಂ ॥}\\ ವಿಮಲಾ ವಾಗ್ದೇವತಾಯೈ ನಮಃ । ಕಂಠೇ\\
\as{೪ ತಂ ಥಂ ದಂ ಧಂ ನಂ~। ಜ್‌ಮ್ರೀಂ ॥}\\ ಅರುಣಾ ವಾಗ್ದೇವತಾಯೈ ನಮಃ। ಹೃದಯೇ\\
\as{೪ ಪಂ ಫಂ ಬಂ ಭಂ ಮಂ। ಹ್‌ಸ್‌ಲ್‌ವ್ಯೂಂ ॥}\\ ಜಯಿನೀ ವಾಗ್ದೇವತಾಯೈ ನಮಃ। ನಾಭೌ\\
\as{೪ ಯಂ ರಂ ಲಂ ವಂ~। ಝ್‌ಮ್‌ರ್ಯೂಂ ॥}\\ ಸರ್ವೇಶ್ವರೀ ವಾಗ್ದೇವತಾಯೈ ನಮಃ। ಗುಹ್ಯೇ\\
\as{೪ ಶಂ ಷಂ ಸಂ ಹಂ ಳಂ ಕ್ಷಂ~। ಕ್ಷ್‌ಮ್ರೀಂ ॥}\\ ಕೌಲಿನೀ ವಾಗ್ದೇವತಾಯೈ ನಮಃ। ಆಧಾರೇ
\section{ಬಹಿಶ್ಚಕ್ರನ್ಯಾಸಃ}
\addcontentsline{toc}{section}{ಬಹಿಶ್ಚಕ್ರನ್ಯಾಸಃ}
{\bfseries ೪ ಅಂಆಂಸೌಃ ।} ಚತುರಶ್ರ ತ್ರಯಾತ್ಮಕ ತ್ರೈಲೋಕ್ಯಮೋಹನ ಚಕ್ರಾಧಿಷ್ಠಾತ್ರ್ಯೈ ಅಣಿಮಾದ್ಯಷ್ಟಾವಿಂಶತಿ ಶಕ್ತಿಸಹಿತ ಪ್ರಕಟಯೋಗಿನೀ ರೂಪಾಯೈ  ತ್ರಿಪುರಾದೇವ್ಯೈ ನಮಃ॥(ಪಾದಯೋಃ)\\
{\bfseries ೪ ಐಂಕ್ಲೀಂಸೌಃ ।} ಷೋಡಶದಲ ಪದ್ಮಾತ್ಮಕ ಸರ್ವಾಶಾ ಪರಿಪೂರಕ ಚಕ್ರಾಧಿಷ್ಠಾತ್ರ್ಯೈ ಕಾಮಾಕರ್ಷಣ್ಯಾದಿ ಷೋಡಶಶಕ್ತಿ ಸಹಿತ ಗುಪ್ತಯೋಗಿನೀ ರೂಪಾಯೈ ತ್ರಿಪುರೇಶೀ ದೇವ್ಯೈ ನಮಃ॥(ಜಾನುನೋಃ)\\
{\bfseries ೪ ಹ್ರೀಂಕ್ಲೀಂಸೌಃ ।} ಅಷ್ಟದಲ ಪದ್ಮಾತ್ಮಕ ಸರ್ವ ಸಂಕ್ಷೋಭಣ ಚಕ್ರಾಧಿಷ್ಠಾತ್ರ್ಯೈ ಅನಂಗ ಕುಸುಮಾದ್ಯಷ್ಟಶಕ್ತಿ ಸಹಿತ ಗುಪ್ತತರ ಯೋಗಿನೀ ರೂಪಾಯೈ ತ್ರಿಪುರಸುಂದರೀ ದೇವ್ಯೈ ನಮಃ॥(ಊರುಮೂಲಯೋಃ)\\
{\bfseries ೪ ಹೈಂಹ್‌ಕ್ಲೀಂಹ್ಸೌಃ ।} ಚತುರ್ದಶಾರಾತ್ಮಕ ಸರ್ವಸೌಭಾಗ್ಯದಾಯಕ ಚಕ್ರಾಧಿಷ್ಠಾತ್ರ್ಯೈ ಸರ್ವ ಸಂಕ್ಷೋಭಿಣ್ಯಾದಿ ಚತುರ್ದಶ ಶಕ್ತಿಸಹಿತ ಸಂಪ್ರದಾಯ ಯೋಗಿನೀ ರೂಪಾಯೈ  ತ್ರಿಪುರವಾಸಿನೀ ದೇವ್ಯೈ ನಮಃ~॥(ನಾಭೌ)\\
{\bfseries ೪ ಹ್‌ಸೈಂಹ್‌ಸ್‌ಕ್ಲೀಂಹ್‌ಸ್ಸೌಃ ।} ಬಹಿರ್ದಶಾರಾತ್ಮಕ ಸರ್ವಾರ್ಥ ಸಾಧಕ ಚಕ್ರಾಧಿಷ್ಠಾತ್ರ್ಯೈ ಸರ್ವಸಿದ್ಧಿಪ್ರದಾದಿ ದಶಶಕ್ತಿ ಸಹಿತ ಕುಲೋತ್ತೀರ್ಣ ಯೋಗಿನೀ ರೂಪಾಯೈ  ತ್ರಿಪುರಾಶ್ರೀ ದೇವ್ಯೈ ನಮಃ॥(ಹೃದಯೇ)\\
{\bfseries ೪ ಹ್ರೀಂ ಕ್ಲೀಂ ಬ್ಲೇಂ ।} ಅಂತರ್ದಶಾರಾತ್ಮಕ ಸರ್ವರಕ್ಷಾಕರ ಚಕ್ರಾಧಿಷ್ಠಾತ್ರ್ಯೈ ಸರ್ವಜ್ಞಾದಿ ದಶಶಕ್ತಿ ಸಹಿತ ನಿಗರ್ಭ ಯೋಗಿನೀ ರೂಪಾಯೈ  ತ್ರಿಪುರಮಾಲಿನೀ ದೇವ್ಯೈ ನಮಃ॥(ಕಂಠೇ)\\
{\bfseries ೪ ಹ್ರೀಂ ಶ್ರೀಂ ಸೌಃ ।} ಅಷ್ಟಾರಾತ್ಮಕ ಸರ್ವರೋಗಹರ ಚಕ್ರಾಧಿಷ್ಠಾತ್ರ್ಯೈ ವಶಿನ್ಯಾದ್ಯಷ್ಟ ಶಕ್ತಿಸಹಿತ ರಹಸ್ಯ ಯೋಗಿನೀ ರೂಪಾಯೈ  ತ್ರಿಪುರಾಸಿದ್ಧಾ ದೇವ್ಯೈ ನಮಃ॥(ಮುಖೇ)\\
{\bfseries ೪ ಹ್‌ಸ್‌ರೈಂ ಹ್‌ಸ್‌ಕ್ಲ್ರೀಂ ಹ್‌ಸ್‌ರ್ಸೌಃ ।} ತ್ರಿಕೋಣಾತ್ಮಕ ಸರ್ವಸಿದ್ಧಿ ಪ್ರದಚಕ್ರಾಧಿಷ್ಠಾತ್ರ್ಯೈ ಕಾಮೇಶ್ವರ್ಯಾದಿ ತ್ರಿಶಕ್ತಿ ಸಹಿತ ಅತಿರಹಸ್ಯಯೋಗಿನೀ ರೂಪಾಯೈ  ತ್ರಿಪುರಾಂಬಾ ದೇವ್ಯೈ ನಮಃ~॥(ನೇತ್ರಯೋಃ)\\
{\bfseries ೪ ೧೫॥} ಬಿಂದ್ವಾತ್ಮಕ ಸರ್ವಾನಂದಮಯ ಚಕ್ರಾಧಿಷ್ಠಾತ್ರ್ಯೈ ಷಡಂಗಾಯುಧ ದಶಶಕ್ತಿ ಸಹಿತ ಪರಾಪರಾತಿ ರಹಸ್ಯಯೋಗಿನೀ ರೂಪಾಯೈ ಮಹಾ ತ್ರಿಪುರಸುಂದರೀ ದೇವ್ಯೈ ನಮಃ~॥(ಶಿರಸಿ)
\section{ಅಂತಶ್ಚಕ್ರನ್ಯಾಸಃ}
\addcontentsline{toc}{section}{ಅಂತಶ್ಚಕ್ರನ್ಯಾಸಃ}
{\bfseries ೪ ಅಂಆಂಸೌಃ।} ಚತುರಶ್ರ ತ್ರಯಾತ್ಮಕ ತ್ರೈಲೋಕ್ಯಮೋಹನ ಚಕ್ರಾಧಿಷ್ಠಾತ್ರ್ಯೈ ಅಣಿಮಾದ್ಯಷ್ಟಾವಿಂಶತಿ ಶಕ್ತಿ ಸಹಿತ ಪ್ರಕಟಯೋಗಿನೀ ರೂಪಾಯೈ  ತ್ರಿಪುರಾ ದೇವ್ಯೈ ನಮಃ॥(ಅಕುಲಸಹಸ್ರಾರೇ)\\
{\bfseries ೪ ಐಂಕ್ಲೀಂಸೌಃ ।} ಷೋಡಶ ದಲ ಪದ್ಮಾತ್ಮಕ ಸರ್ವಾಶಾಪರಿಪೂರಕ ಚಕ್ರಾಧಿಷ್ಠಾತ್ರ್ಯೈ ಕಾಮಾಕರ್ಷಣ್ಯಾದಿ ಷೋಡಶ ಶಕ್ತಿ ಸಹಿತ ಗುಪ್ತಯೋಗಿನೀ ರೂಪಾಯೈ  ತ್ರಿಪುರೇಶೀ ದೇವ್ಯೈ ನಮಃ॥(ವಿಷುಚಕ್ರೇ)\\
{\bfseries ೪ ಹ್ರೀಂ ಕ್ಲೀಂ ಸೌಃ ।} ಅಷ್ಟದಲ ಪದ್ಮಾತ್ಮಕ ಸರ್ವಸಂಕ್ಷೋಭಣ ಚಕ್ರಾಧಿಷ್ಠಾತ್ರ್ಯೈ ಅನಂಗ ಕುಸುಮಾದ್ಯಷ್ಟ ಶಕ್ತಿಸಹಿತ ಗುಪ್ತತರ ಯೋಗಿನೀ ರೂಪಾಯೈ  ತ್ರಿಪುರಸುಂದರೀ ದೇವ್ಯೈ ನಮಃ॥(ಮೂಲಾಧಾರೇ)\\
{\bfseries ೪ ಹೈಂ ಹ್‌ಕ್ಲೀಂ ಹ್ಸೌಃ ।} ಚತುರ್ದಶಾರಾತ್ಮಕ ಸರ್ವಸೌಭಾಗ್ಯದಾಯಕ ಚಕ್ರಾಧಿಷ್ಠಾತ್ರ್ಯೈ ಸರ್ವಸಂಕ್ಷೋಭಿಣ್ಯಾದಿ ಚತುರ್ದಶ ಶಕ್ತಿಸಹಿತ ಸಂಪ್ರದಾಯ ಯೋಗಿನೀ ರೂಪಾಯೈ  ತ್ರಿಪುರವಾಸಿನೀ ದೇವ್ಯೈ ನಮಃ॥(ಸ್ವಾಧಿಷ್ಠಾನೇ)\\
{\bfseries ೪ ಹ್‌ಸೈಂ ಹ್‌ಸ್‌ಕ್ಲೀಂ ಹ್‌ಸ್ಸೌಃ।} ಬಹಿರ್ದಶಾರಾತ್ಮಕ ಸರ್ವಾರ್ಥ ಸಾಧಕ ಚಕ್ರಾಧಿಷ್ಠಾತ್ರ್ಯೈ ಸರ್ವಸಿದ್ಧಿ ಪ್ರದಾದಿ ದಶ ಶಕ್ತಿ ಸಹಿತ ಕುಲೋತ್ತೀರ್ಣ ಯೋಗಿನೀ ರೂಪಾಯೈ  ತ್ರಿಪುರಾಶ್ರೀ ದೇವ್ಯೈ ನಮಃ॥(ಮಣಿಪೂರೇ)\\
{\bfseries ೪ ಹ್ರೀಂ ಕ್ಲೀಂ ಬ್ಲೇಂ ।} ಅಂತರ್ದಶಾರಾತ್ಮಕ ಸರ್ವರಕ್ಷಾಕರ ಚಕ್ರಾಧಿಷ್ಠಾತ್ರ್ಯೈ ಸರ್ವಜ್ಞಾದಿ ದಶಶಕ್ತಿ ಸಹಿತ ನಿಗರ್ಭಯೋಗಿನೀ ರೂಪಾಯೈ  ತ್ರಿಪುರಮಾಲಿನೀ ದೇವ್ಯೈ ನಮಃ॥(ಅನಾಹತೇ)\\
{\bfseries ೪ ಹ್ರೀಂ ಶ್ರೀಂ ಸೌಃ ।} ಅಷ್ಟಾರಾತ್ಮಕ ಸರ್ವರೋಗಹರ ಚಕ್ರಾಧಿಷ್ಠಾತ್ರ್ಯೈ ವಶಿನ್ಯಾದ್ಯಷ್ಟ ಶಕ್ತಿಸಹಿತ ರಹಸ್ಯ ಯೋಗಿನೀ ರೂಪಾಯೈ  ತ್ರಿಪುರಾಸಿದ್ಧಾ ದೇವ್ಯೈ ನಮಃ॥(ವಿಶುದ್ಧೌ)\\
{\bfseries ೪ ಹ್‌ಸ್‌ರೈಂ ಹ್‌ಸ್‌ಕ್ಲ್ರೀಂ ಹ್‌ಸ್‌ರ್ಸೌಃ ।} ತ್ರಿಕೋಣಾತ್ಮಕ ಸರ್ವಸಿದ್ಧಿಪ್ರದ ಚಕ್ರಾಧಿಷ್ಠಾತ್ರ್ಯೈ ಕಾಮೇಶ್ವರ್ಯಾದಿ ತ್ರಿಶಕ್ತಿಸಹಿತ ಅತಿರಹಸ್ಯಯೋಗಿನೀ ರೂಪಾಯೈ  ತ್ರಿಪುರಾಂಬಾ ದೇವ್ಯೈ ನಮಃ॥(ಲಂಬಿಕಾಗ್ರೇ)\\
{\bfseries ೪ ೧೫~॥}ಬಿಂದ್ವಾತ್ಮಕ ಸರ್ವಾನಂದಮಯ ಚಕ್ರಾಧಿಷ್ಠಾತ್ರ್ಯೈ ಷಡಂಗಾಯುಧ ದಶ ಶಕ್ತಿಸಹಿತ ಪರಾಪರಾತಿರಹಸ್ಯ ಯೋಗಿನೀ ರೂಪಾಯೈ ಮಹಾ ತ್ರಿಪುರಸುಂದರೀ ದೇವ್ಯೈ ನಮಃ॥(ಆಜ್ಞಾಯಾಂ)\\
{\bfseries ೪ ಅಂ ಆಂ ಸೌಃ} ನಮಃ(ಬಿಂದೌ)\\
{\bfseries ೪ ಐಂ ಕ್ಲೀಂ ಸೌಃ} ನಮಃ(ಅರ್ಧಚಂದ್ರೇ)\\
{\bfseries ೪ ಹ್ರೀಂ ಕ್ಲೀಂ ಸೌಃ} ನಮಃ(ರೋಧಿನ್ಯಾಂ)\\
{\bfseries ೪ ಹೈಂ ಹ್‌ಕ್ಲೀಂ ಹ್ಸೌಃ} ನಮಃ(ನಾದೇ)\\
{\bfseries ೪ ಹ್‌ಸೈಂ ಹ್‌ಸ್‌ಕ್ಲೀಂ ಹ್ಸ್ಸೌಃ} ನಮಃ(ನಾದಾಂತೇ)\\
{\bfseries ೪ ಹ್ರೀಂ ಕ್ಲೀಂ ಬ್ಲೇಂ} ನಮಃ (ಶಕ್ತೌ)\\
{\bfseries ೪ ಹ್ರೀಂ ಶ್ರೀಂ ಸೌಃ} ನಮಃ (ವ್ಯಾಪಿಕಾಯಾಂ)\\
{\bfseries ೪ ಹ್‌ಸ್‌ರೈಂ ಹ್‌ಸ್‌ಕ್ಲ್ರೀಂ ಹ್‌ಸ್‌ರ್ಸೌಃ} ನಮಃ(ಸಮನಾಯಾಂ)\\
{\bfseries ೪ (ಪಂಚದಶೀ)}ನಮಃ(ಉನ್ಮನಾಯಾಂ)\\
{\bfseries ೪ (ಷೋಡಶೀ)}ನಮಃ(ಬ್ರಹ್ಮರಂಧ್ರೇ)
\section{ಕಾಮೇಶ್ವರ್ಯಾದಿ ನ್ಯಾಸಃ}
\addcontentsline{toc}{section}{ಕಾಮೇಶ್ವರ್ಯಾದಿ ನ್ಯಾಸಃ}
{\bfseries ೪ ಐಂ ೫॥} ಸೂರ್ಯಚಕ್ರೇ ಕಾಮಗಿರಿಪೀಠೇ ಮಿತ್ರೀಶನಾಥ ನವಯೋನಿ ಚಕ್ರಾತ್ಮಕ ಆತ್ಮತತ್ತ್ವ  ಸಂಹಾರಕೃತ್ಯ ಜಾಗ್ರದ್ದಶಾಧಿಷ್ಠಾಯಕ ಇಚ್ಛಾಶಕ್ತಿ ವಾಗ್ಭವಾತ್ಮಕ ಪರಾಪರಶಕ್ತಿ ಸ್ವರೂಪ ಮಹಾಕಾಮೇಶ್ವರೀ ರುದ್ರಾತ್ಮಶಕ್ತಿ  ಶ್ರೀಪಾದುಕಾಂ ಪೂಜಯಾಮಿ ನಮಃ॥(ಆಧಾರೇ)\\
{\bfseries೪ ಕ್ಲೀಂ ೬॥} ಸೋಮಚಕ್ರೇ ಪೂರ್ಣಗಿರಿಪೀಠೇ ಉಡ್ಡೀಶನಾಥ  ದಶಾರದ್ವಯ ಚತುರ್ದಶಾರ ಚಕ್ರಾತ್ಮಕ ವಿದ್ಯಾತತ್ವ ಸ್ಥಿತಿಕೃತ್ಯ ಸ್ವಪ್ನದಶಾಧಿಷ್ಠಾಯಕ ಜ್ಞಾನಶಕ್ತಿ ಕಾಮರಾಜಾತ್ಮಕ ಕಾಮಕಲಾ ಸ್ವರೂಪ ಮಹಾವಜ್ರೇಶ್ವರೀ ವಿಷ್ಣ್ವಾತ್ಮಶಕ್ತಿ  ಶ್ರೀಪಾದುಕಾಂ ಪೂಜಯಾಮಿ ನಮಃ॥(ಅನಾಹತೇ)\\
{\bfseries೪ ಸೌಃ ೪॥} ಅಗ್ನಿಚಕ್ರೇ ಜಾಲಂಧರಪೀಠೇ  ಷಷ್ಠೀಶನಾಥ ಅಷ್ಟದಳ ಷೋಡಶದಳ ಚತುರಸ್ರ ಚಕ್ರಾತ್ಮಕ ಶಿವತತ್ತ್ವ  ಸೃಷ್ಟಿಕೃತ್ಯ ಸುಷುಪ್ತಿದಶಾಧಿಷ್ಠಾಯಕ ಕ್ರಿಯಾಶಕ್ತಿ ಶಕ್ತಿಬೀಜಾತ್ಮಕ ವಾಗೀಶ್ವರೀ ಸ್ವರೂಪ ಮಹಾಭಗಮಾಲಿನೀ ಬ್ರಹ್ಮಾತ್ಮಶಕ್ತಿ  ಶ್ರೀಪಾದುಕಾಂ ಪೂಜಯಾಮಿ ನಮಃ॥(ಆಜ್ಞಾಯಾಮ್)\\
{\bfseries ೪ ಐಂ೫ ಕ್ಲೀಂ೬ ಸೌಃ೪} ಪರಬ್ರಹ್ಮಚಕ್ರೇ ಮಹೋಡ್ಯಾಣಪೀಠೇ ಚರ್ಯಾನಂದನಾಥ ಸಮಸ್ತಚಕ್ರಾತ್ಮಕ ಸಪರಿವಾರ ಪರಮತತ್ವ ಸೃಷ್ಟಿ ಸ್ಥಿತಿ ಸಂಹಾರಕೃತ್ಯ ತುರೀಯ ದಶಾಧಿಷ್ಠಾಯಕ ಇಚ್ಛಾ ಜ್ಞಾನ ಕ್ರಿಯಾ ಶಾಂತಶಕ್ತಿ ವಾಗ್ಭವ ಕಾಮರಾಜ ಶಕ್ತಿ ಬೀಜಾತ್ಮಕ ಪರಮಶಕ್ತಿ ಸ್ವರೂಪ ಶ್ರೀ ಮಹಾತ್ರಿಪುರಸುಂದರೀ ಪರಬ್ರಹ್ಮಾತ್ಮಶಕ್ತಿ ಶ್ರೀಪಾದುಕಾಂ ಪೂಜಯಾಮಿ ನಮಃ~।(ಬ್ರಹ್ಮರಂಧ್ರೇ)
\section{ಮೂಲವಿದ್ಯಾನ್ಯಾಸ}
\addcontentsline{toc}{section}{ಮೂಲವಿದ್ಯಾನ್ಯಾಸ}
ಅಸ್ಯ ಶ್ರೀ ಮೂಲವಿದ್ಯಾನ್ಯಾಸಮಹಾಮಂತ್ರಸ್ಯ ದಕ್ಷಿಣಾಮೂರ್ತಿಃ ಋಷಿಃ~। ಪಂಕ್ತಿಶ್ಛಂದಃ~। ಶ್ರೀಮಹಾತ್ರಿಪುರಸುಂದರೀ ದೇವತಾ~। ಐಂ ಕಏಈಲಹ್ರೀಂ ಇತಿ ಬೀಜಂ~। ಕ್ಲೀಂ ಹಸಕಹಲಹ್ರೀಂ ಇತಿ ಶಕ್ತಿಃ~। ಸೌಃ ಸಕಲಹ್ರೀಂ ಇತಿ ಕೀಲಕಂ~। ಪೂಜಾಯಾಂ ವಿನಿಯೋಗಃ~।\\
ಕೂಟತ್ರಯೇಣ ನ್ಯಾಸಂ ಧ್ಯಾನಂ ಚ ವಿಧಾಯ~।\\
\as{೪ ಕಂ} ನಮಃ (ಶಿರಸಿ)\\
\as{೪ ಏಂ} ನಮಃ (ಮೂಲಾಧಾರೇ)\\
\as{೪ ಈಂ} ನಮಃ (ಹೃದಯೇ)\\
\as{೪ ಲಂ} ನಮಃ (ದಕ್ಷನೇತ್ರೇ)\\
\as{೪ ಹ್ರೀಂ} ನಮಃ (ವಾಮನೇತ್ರೇ)\\
\as{೪ ಹಂ} ನಮಃ (ಭ್ರೂಮಧ್ಯೇ)\\
\as{೪ ಸಂ} ನಮಃ (ದಕ್ಷಕರ್ಣೇ)\\
\as{೪ ಕಂ} ನಮಃ (ವಾಮಕರ್ಣೇ)\\
\as{೪ ಹಂ} ನಮಃ (ಮುಖೇ)\\
\as{೪ ಲಂ} ನಮಃ (ದಕ್ಷಾಂಸೇ)\\
\as{೪ ಹ್ರೀಂ} ನಮಃ (ವಾಮಾಂಸೇ)\\
\as{೪ ಸಂ} ನಮಃ (ಪೃಷ್ಠೇ)\\
\as{೪ ಕಂ} ನಮಃ (ದಕ್ಷಜಾನುನಿ)\\
\as{೪ ಲಂ} ನಮಃ (ವಾಮಜಾನುನಿ)\\
\as{೪ ಹ್ರೀಂ} ನಮಃ (ನಾಭೌ)\\
\as{೪ ೧೫} ನಮಃ (ಸರ್ವಾಂಗೇ)\\
ಪ್ರಾಗ್ವದುತ್ತರನ್ಯಾಸಃ
\section{ಷೋಡಶಾಕ್ಷರೀ ನ್ಯಾಸಃ}
\addcontentsline{toc}{section}{ಷೋಡಶಾಕ್ಷರೀ ನ್ಯಾಸಃ}
ಅಸ್ಯ ಶ್ರೀ ಷೋಡಶಾಕ್ಷರೀನ್ಯಾಸಮಹಾಮಂತ್ರಸ್ಯ ಆನಂದಭೈರವ ಋಷಿಃ~। ಅನುಷ್ಟುಪ್ಛಂದಃ~। ಶ್ರೀಮಹಾತ್ರಿಪುರಸುಂದರೀ ದೇವತಾ~। ಐಂ ಬೀಜಂ~। ಕ್ಲೀಂ ಶಕ್ತಿಃ~। ಸೌಃ ಕೀಲಕಂ~। ಪೂಜಾಯಾಂ ವಿನಿಯೋಗಃ~। ಮೂಲಾಕ್ಷರೈಃ ನ್ಯಾಸಃ~।\\
\as{೪ (ಮೂಲಮಂತ್ರ)} ನಮಃ ॥ದಕ್ಷಕರಾಂಗುಷ್ಠಾನಾಮಿಕಾಭ್ಯಾಂ ಶಿರಸಿ\\
\as{೪ (ಮೂಲಮಂತ್ರ)} ನಮಃ ॥ಮಹಾಸೌಭಾಗ್ಯಂ ಮೇ ದೇಹಿ॥ಶಿರ ಆದಿ ಪಾದಾಂತಂ ಶರೀರವಾಮಭಾಗೇ\\
\as{೪ (ಮೂಲಮಂತ್ರ)} ನಮಃ ॥ ಮಮ ಶತ್ರೂನ್ನಿಗೃಹ್ಣಾಮಿ ॥ರಿಪುಜಿಹ್ವಾಗ್ರಮುದ್ರಯಾ ವಾಮಪಾದಸ್ಯ ಅಧಃ\\
\as{೪ (ಮೂಲಮಂತ್ರ)} ನಮಃ ॥ ತ್ರೈಲೋಕ್ಯಸ್ಯಾಹಂ ಕರ್ತಾ॥ತ್ರಿಖಂಡಯಾ ಫಾಲೇ\\
\as{೪ (ಮೂಲಮಂತ್ರ)} ನಮಃ ॥ ತ್ರಿಖಂಡಯಾ ಮುಖೇ\\
\as{೪ (ಮೂಲಮಂತ್ರ)} ನಮಃ ॥ ತ್ರಿಖಂಡಯಾ ವಾಮಕರ್ಣಾದಿ ದಕ್ಷಿಣಕರ್ಣಪರ್ಯಂತಂ\\
\as{೪ (ಮೂಲಮಂತ್ರ)} ನಮಃ ॥ ತ್ರಿಖಂಡಯಾ ಗಲಾದಿ ಶಿರಃಪರ್ಯಂತಂ\\
\as{೪ (ಮೂಲಮಂತ್ರ)} ನಮಃ ॥ ತ್ರಿಖಂಡಯಾ ಶಿರ ಆದಿ ಪಾದಪರ್ಯಂತಂ\\
\as{೪ (ಮೂಲಮಂತ್ರ)} ನಮಃ ॥ ಪಾದಾದಿ ಶಿರಃಪರ್ಯಂತಂ\\
\as{೪ (ಮೂಲಮಂತ್ರ)} ನಮಃ ॥ ಯೋನಿಮುದ್ರಯಾ ಮುಖೇ\\
\as{೪ (ಮೂಲಮಂತ್ರ)} ನಮಃ ॥ ಯೋನಿಮುದ್ರಯಾ ಲಲಾಟೇ
\section{ಸಂಮೋಹನನ್ಯಾಸಃ}
\addcontentsline{toc}{section}{ಸಂಮೋಹನನ್ಯಾಸಃ}
\as{೪ (ಮೂಲಮಂತ್ರ)} ನಮಃ ॥ಅನಾಮಿಕಯಾ ತ್ರಿವಾರಂ ಶಿರಸಿ ಪರಿಭ್ರಾಮ್ಯ\\
\as{೪ (ಮೂಲಮಂತ್ರ)} ನಮಃ ॥ಅಂಗುಷ್ಠಅನಾಮಿಕಾಭ್ಯಾಂ ಬ್ರಹ್ಮರಂಧ್ರೇ\\
\as{೪ (ಮೂಲಮಂತ್ರ)} ನಮಃ ॥ಮಣಿಬಂಧಯೋಃ\\
\as{೪ (ಮೂಲಮಂತ್ರ)} ನಮಃ ॥ಲಲಾಟೇ\\
\as{೪ (ಮೂಲಮಂತ್ರ)} ನಮಃ ॥ಇತಿ ಶಾಕ್ತತಿಲಕಂ ಪ್ರಕಲ್ಪಯೇತ್
\section{ಸಂಹಾರನ್ಯಾಸಃ}
\addcontentsline{toc}{section}{ಸಂಹಾರನ್ಯಾಸಃ}
\as{೪ ಶ್ರೀಂ} ನಮಃ (ಪಾದಯೋಃ)\\
\as{೪ ಹ್ರೀಂ} ನಮಃ (ಜಂಘಯೋಃ)\\
\as{೪ ಕ್ಲೀಂ} ನಮಃ (ಜಾನುನೋಃ)\\
\as{೪ ಐಂ} ನಮಃ (ಸ್ಫಿಚೋಃ)\\
\as{೪ ಸೌಃ} ನಮಃ (ಪೃಷ್ಠೇ)\\
\as{೪ ಓಂ} ನಮಃ (ಲಿಂಗೇ)\\
\as{೪ ಹ್ರೀಂ} ನಮಃ (ನಾಭೌ)\\
\as{೪ ಶ್ರೀಂ} ನಮಃ (ಪಾರ್ಶ್ವಯೋಃ)\\
\as{೪ ಐಂ ಕಏಈಲಹ್ರೀಂ} ನಮಃ (ಸ್ತನಯೋಃ)\\
\as{೪ ಕ್ಲೀಂ ಹಸಕಹಲಹ್ರೀಂ} ನಮಃ (ಅಂಸಯೋಃ)\\
\as{೪ ಸೌಃ ಸಕಲಹ್ರೀಂ} ನಮಃ (ಕರ್ಣಯೋಃ)\\
\as{೪ ಸೌಃ} ನಮಃ (ಮೂರ್ಧ್ನಿ)\\
\as{೪ ಐಂ} ನಮಃ (ಮುಖೇ)\\
\as{೪ ಕ್ಲೀಂ} ನಮಃ (ನೇತ್ರಯೋಃ)\\
\as{೪ ಹ್ರೀಂ} ನಮಃ (ಉಪಕರ್ಣಯೋಃ)\\
\as{೪ ಶ್ರೀಂ} ನಮಃ (ಕರ್ಣಯೋಃ)
\section{ಸೃಷ್ಟಿನ್ಯಾಸಃ}
\addcontentsline{toc}{section}{ಸೃಷ್ಟಿನ್ಯಾಸಃ}
\as{೪ ಶ್ರೀಂ} ನಮಃ (ಬ್ರಹ್ಮರಂಧ್ರೇ)\\
\as{೪ ಹ್ರೀಂ} ನಮಃ (ಲಲಾಟೇ)\\
\as{೪ ಕ್ಲೀಂ} ನಮಃ (ನೇತ್ರಯೋಃ)\\
\as{೪ ಐಂ} ನಮಃ (ಕರ್ಣಯೋಃ)\\
\as{೪ ಸೌಃ} ನಮಃ (ನಾಸಿಕಯೋಃ)\\
\as{೪ ಓಂ} ನಮಃ (ಗಂಡಯೋಃ)\\
\as{೪ ಹ್ರೀಂ} ನಮಃ (ದಂತಪಂಕ್ತೌ)\\
\as{೪ ಶ್ರೀಂ} ನಮಃ (ಓಷ್ಠಯೋಃ)\\
\as{೪ ಐಂ ಕಏಈಲಹ್ರೀಂ} ನಮಃ (ಜಿಹ್ವಾಯಾಂ)\\
\as{೪ ಕ್ಲೀಂ ಹಸಕಹಲಹ್ರೀಂ} ನಮಃ (ಕಂಠೇ)\\
\as{೪ ಸೌಃ ಸಕಲಹ್ರೀಂ} ನಮಃ (ಪೃಷ್ಠೇ)\\
\as{೪ ಸೌಃ} ನಮಃ (ಸರ್ವಾಂಗೇ)\\
\as{೪ ಐಂ} ನಮಃ (ಹೃದಯೇ)\\
\as{೪ ಕ್ಲೀಂ} ನಮಃ (ಸ್ತನಯೋಃ)\\
\as{೪ ಹ್ರೀಂ} ನಮಃ (ಉದರೇ)\\
\as{೪ ಶ್ರೀಂ} ನಮಃ (ಲಿಂಗೇ)
\section{ಸ್ಥಿತಿನ್ಯಾಸಃ}
\addcontentsline{toc}{section}{ಸ್ಥಿತಿನ್ಯಾಸಃ}
\as{೪ ಶ್ರೀಂ} ನಮಃ (ಅಂಗುಷ್ಠಯೋಃ)\\
\as{೪ ಹ್ರೀಂ} ನಮಃ (ತರ್ಜನ್ಯೋಃ)\\
\as{೪ ಕ್ಲೀಂ} ನಮಃ (ಮಧ್ಯಮಯೋಃ)\\
\as{೪ ಐಂ} ನಮಃ (ಅನಾಮಿಕಯೋಃ)\\
\as{೪ ಸೌಃ} ನಮಃ (ಕನಿಷ್ಠಿಕಯೋಃ)\\
\as{೪ ಓಂ} ನಮಃ (ಮೂರ್ಧ್ನಿ)\\
\as{೪ ಹ್ರೀಂ} ನಮಃ (ಮುಖೇ)\\
\as{೪ ಶ್ರೀಂ} ನಮಃ (ಹೃದಯೇ)\\
\as{೪ ಐಂ ಕಏಈಲಹ್ರೀಂ} ನಮಃ (ಪಾದಾದಿನಾಭಿಪರ್ಯಂತಂ)\\
\as{೪ ಕ್ಲೀಂ ಹಸಕಹಲಹ್ರೀಂ} ನಮಃ (ನಾಭೇರ್ವಿಶುದ್ಧಿಪರ್ಯಂತಂ)\\
\as{೪ ಸೌಃ ಸಕಲಹ್ರೀಂ} ನಮಃ (ವಿಶುದ್ಧೇರ್ಬ್ರಹ್ಮರಂಧ್ರಾಂತಂ)\\
\as{೪ ಸೌಃ} ನಮಃ (ಪಾದಾಂಗುಷ್ಠಯೋಃ)\\
\as{೪ ಐಂ} ನಮಃ (ಪಾದತರ್ಜನ್ಯೋಃ)\\
\as{೪ ಕ್ಲೀಂ} ನಮಃ (ಪಾದಮಧ್ಯಮಯೋಃ)\\
\as{೪ ಹ್ರೀಂ} ನಮಃ (ಪಾದಾನಾಮಿಕಯೋಃ)\\
\as{೪ ಶ್ರೀಂ} ನಮಃ (ಪಾದಕನಿಷ್ಠಿಕಯೋಃ)\\
\as{೪ ೧೬} ನಮಃ (ಸರ್ವಾಂಗೇ)\\
ಪ್ರಾಗ್ವದುತ್ತರನ್ಯಾಸಃ
\section{ವರ್ಧಿನೀಕಲಶಸ್ಥಾಪನಮ್}
\addcontentsline{toc}{section}{ವರ್ಧಿನೀಕಲಶಸ್ಥಾಪನಮ್}
ಬಿಂದುತ್ರಿಕೋಣವೃತ್ತಚತುರಸ್ರಾತ್ಮಕಂ ಮಂಡಲಂ ವಿಧಾಯ , ಚತುರಸ್ರೇ ಆಗ್ನೇಯಾದಿಷು,

\as{೪ ಐಂ } ಹೃದಯಾಯ ನಮಃ । ಹೃದಯಶಕ್ತಿ ಶ್ರೀಪಾದುಕಾಂ ಪೂ । ನಮಃ ॥\\
\as{೪ ಕ್ಲೀಂ } ಶಿರಸೇ ಸ್ವಾಹಾ । ಶಿರಃಶಕ್ತಿ ಶ್ರೀಪಾದುಕಾಂ ಪೂ । ನಮಃ ॥\\
\as{೪ ಸೌಃ } ಶಿಖಾಯೈ ವಷಟ್ । ಶಿಖಾಶಕ್ತಿ ಶ್ರೀಪಾದುಕಾಂ ಪೂ । ನಮಃ ॥\\
\as{೪ ಐಂ } ಕವಚಾಯ ಹುಂ । ಕವಚಶಕ್ತಿ ಶ್ರೀಪಾದುಕಾಂ ಪೂ । ನಮಃ ॥\\
\as{೪ ಕ್ಲೀಂ} ನೇತ್ರತ್ರಯಾಯ ವೌಷಟ್ । ನೇತ್ರಶಕ್ತಿ ಶ್ರೀಪಾದುಕಾಂ ಪೂ । ನಮಃ ॥\\
\as{೪ ಸೌಃ } ಅಸ್ತ್ರಾಯ ಫಟ್ । ಅಸ್ತ್ರಶಕ್ತಿ ಶ್ರೀಪಾದುಕಾಂ ಪೂ । ನಮಃ ॥ 

ಇತಿ ಷಡಂಗಾನಿ ವಿನ್ಯಸ್ಯ, ತ್ರಿಕೋಣೇ ಸ್ವಾಗ್ರಾದಿಕೋಣೇಷು\\
\as{೪ ಐಂ ಕಏಈಲಹ್ರೀಂ} ನಮಃ ॥\\
\as{೪ ಕ್ಲೀಂ ಹಸಕಹಲಹ್ರೀಂ} ನಮಃ ॥\\
\as{೪ ಸೌಃ ಸಕಲಹ್ರೀಂ} ನಮಃ ॥

ಇತಿ ಸಂಪೂಜ್ಯ, ವಿದ್ಯಯಾ ಬಿಂದುಂ ಸಂಪೂಜ್ಯ,

\as{೪ ಐಂ೫} ಅಂ ಅಗ್ನಿಮಂಡಲಾಯ ಧರ್ಮಪ್ರದದಶಕಲಾತ್ಮನೇ ವರ್ಧಿನೀಕಲಶಾಧಾರಾಯ ನಮಃ । ಇತಿ ಪಾತ್ರಾಧಾರಂ\\
\as{೪ ಕ್ಲೀಂ೬} ಉಂ ಅರ್ಕಮಂಡಲಾಯ ಅರ್ಥಪ್ರದದ್ವಾದಶಕಲಾತ್ಮನೇ ವರ್ಧಿನೀಕಲಶಾಯ ನಮಃ । ಇತಿ ಪಾತ್ರಂ ಚ ನಿಧಾಯ\\
\as{೪ ಸೌಃ೪} ಮಂ ಸೋಮಮಂಡಲಾಯ ಕಾಮಪ್ರದಷೋಡಶಕಲಾತ್ಮನೇ ವರ್ಧಿನೀಕಲಶಾಮೃತಾಯ ನಮಃ । ಇತಿ ಜಲಮಾಪೂರ್ಯ

ಕಲಶಸ್ಯ ಮುಖೇ ವಿಷ್ಣುಃ ಕಂಠೇ ರುದ್ರ ಸಮಾಶ್ರಿತಾಃ  ।\\
ಮೂಲೇ ತತ್ರ ಸ್ಥಿತೋಬ್ರಹ್ಮಾ, ಮಧ್ಯೇ ಮಾತೃ ಗಣಾಃ ಸ್ಮೃತಾಃ ॥

ಕುಕ್ಷೌ ತು ಸಾಗರಾಃ ಸರ್ವೇ ಸಪ್ತ ದ್ವೀಪ ವಸುಂಧರಾ।\\
ಋಗ್ವೇದೋಽಥ ಯಜುರ್ವೇದಃ ಸಾಮವೇದೋ ಹ್ಯಥರ್ವಣಃ।\\
ಅಂಗೈಶ್ಚ ಸಹಿತಾಃ ಸರ್ವೇ ಕಲಶಾಂಬು ಸಮಾಶ್ರಿತಾಃ ॥

ಅತ್ರ ಗಾಯತ್ರಿ ಸಾವಿತ್ರೀ ಶಾಂತಿಃ ಪುಷ್ಟಿಕರೀ ತಥಾ ।\\
ಆಯಾಂತು ದೇವ ಪೂಜಾರ್ಥಂ ದುರಿತಕ್ಷಯ ಕಾರಕಾಃ ॥

ಸರ್ವೇ ಸಮುದ್ರಾಃ ಸರಿತಾ ತೀರ್ಥಾನಿ ಜಲದಾ ನದಾಃ ।\\
ಗಂಗೇ ಚ ಯಮುನೇ ಚೈವ ಗೋದಾವರಿ ಸರಸ್ವತಿ ।\\
ನರ್ಮದೇ ಸಿಂಧು ಕಾವೇರಿ ಜಲೇಽಸ್ಮಿನ್ ಸನ್ನಿಧಿಂ ಕುರು ॥

ಆಪೋ ವಾ ಇದಗ್ಂ ಸರ್ವಂ,ವಿಶ್ವಾ ಭೂತಾನ್ಯಾಪಃ ಪ್ರಾಣಾ ವಾ ಆಪಃ ಪಶವ ಆಪೋಽನ್ನಮಾಪೋಽಮೃತಮಾಪಃ ಸಮ್ರಾಢಾಪೋ ವಿರಾಡಾಪಃ ಸ್ವರೂಡಾಪಶ್ಛಂದಾಗ್ಂಸ್ಯಾಪೋ, ಜೋತಿಗ್ ಷ್ಯಾಪೋ ಯಜೂಗ್ಂಷ್ಯಾಪಃ ಸತ್ಯಮಾಪಃ ಸರ್ವಾ ದೇವತಾ ಆಪೋ ಭೂರ್ಭುವಸ್ಸುವರಾಪ ಓಂ ॥

ಇಮಂ ಮೇ ಗಂಗೇ ಯಮುನೇ ಸರಸ್ವತಿ ಶುತದ್ರಿಸ್ತೋಮಗ್ಂ ಸಚತಾ ಪರುಷ್ಣಿಯಾ ।\\
ಅಸಿಕ್ನಿಯಾ ಮರುಧ್ವೃದೇ ವಿಸತ್ಸಯಾರ್ಜೀಕಿಯೇ  ಶೃಣುಹ್ಯಾ ಸುಶೋಮಯಾ ॥\\
ಸಿತಾಸಿತೇ ಸರಿತೇ ಯತ್ರ ಸಂಗತೇ ತತ್ರಾ ಪ್ಲುತೋಸೋದಿವ ಮುತ್ಪತಂತಿ ।\\
ಏವೈ ತನ್ವಾಂ ಅಂ ವಿಸೃಜಂತಿ ಧೀರಾಸ್ತೇಜನಾಸೋ ಅಮೃತತ್ವಂ ಭಜಂತೇ ॥

ಸಿತಮಕರನಿಷಣ್ಣಾಂ ಶುಭ್ರವರ್ಣಾಂ ತ್ರಿನೇತ್ರಾಂ\\ಕರಧೃತ ಕಲಶೋದ್ಭ್ಯ ತ್ಪಂಕಜಾ ಭೀತ್ಯಭೀಷ್ಟಾಂ ।\\
ವಿಧಿ ಹರಿಹರ ರೂಪಾಂ ಸೇಂದು ಕೋಟೀರಚೂಡಾಂ\\ಭಸಿತ ಸಿತ ದುಕೂಲಾಂ ಜಾಹ್ನವೀಂ ತಾಂ ನಮಾಮಿ ॥\\
ಗಂಗಾದಿ ಸರ್ವ ತೀರ್ಥೇಭ್ಯೋ ನಮಃ । ಇತಿ ಪಠಿತ್ವಾ, ಮೂಲೇನಾಭಿಮಂತ್ರ್ಯ ಧೇನುಮುದ್ರಾಂ ಪ್ರದರ್ಶ್ಯ ಶಂಖಂ ಪೂಜಯೇತ್ ॥

\section{ಸಾಮಾನ್ಯಾರ್ಘ್ಯ ವಿಧಿಃ}
\addcontentsline{toc}{section}{ಸಾಮಾನ್ಯಾರ್ಘ್ಯ ವಿಧಿಃ}
ಬಿಂದು ತ್ರಿಕೋಣ ಷಟ್ಕೋಣ ವೃತ್ತ ಚತುರಸ್ರಾತ್ಮಕಂ ಮಂಡಲಂ ವಿಧಾಯ , ಚತುರಸ್ರೇ ಆಗ್ನೇಯಾದಿಷು,\\
\as{೪ ಐಂ } ಹೃದಯಾಯ ನಮಃ । ಹೃದಯಶಕ್ತಿ ಶ್ರೀಪಾದುಕಾಂ ಪೂ । ನಮಃ ॥\\
\as{೪ ಕ್ಲೀಂ } ಶಿರಸೇ ಸ್ವಾಹಾ । ಶಿರಃಶಕ್ತಿ ಶ್ರೀಪಾದುಕಾಂ ಪೂ । ನಮಃ ॥\\
\as{೪ ಸೌಃ } ಶಿಖಾಯೈ ವಷಟ್ । ಶಿಖಾಶಕ್ತಿ ಶ್ರೀಪಾದುಕಾಂ ಪೂ । ನಮಃ ॥\\
\as{೪ ಐಂ } ಕವಚಾಯ ಹುಂ । ಕವಚಶಕ್ತಿ ಶ್ರೀಪಾದುಕಾಂ ಪೂ । ನಮಃ ॥\\
\as{೪ ಕ್ಲೀಂ} ನೇತ್ರತ್ರಯಾಯ ವೌಷಟ್ । ನೇತ್ರಶಕ್ತಿ ಶ್ರೀಪಾದುಕಾಂ ಪೂ । ನಮಃ ॥\\
\as{೪ ಸೌಃ } ಅಸ್ತ್ರಾಯ ಫಟ್ । ಅಸ್ತ್ರಶಕ್ತಿ ಶ್ರೀಪಾದುಕಾಂ ಪೂ । ನಮಃ ॥

ಷಟ್ಕೋಣೇ ಸ್ವಾಗ್ರಾದಿಕೋಣೇಷು\\
\as{೪ ಐಂ } ಹೃದಯಾಯ ನಮಃ । ಹೃದಯಶಕ್ತಿ ಶ್ರೀಪಾದುಕಾಂ ಪೂ । ನಮಃ ॥\\
\as{೪ ಕ್ಲೀಂ } ಶಿರಸೇ ಸ್ವಾಹಾ । ಶಿರಃಶಕ್ತಿ ಶ್ರೀಪಾದುಕಾಂ ಪೂ । ನಮಃ ॥\\
\as{೪ ಸೌಃ } ಶಿಖಾಯೈ ವಷಟ್ । ಶಿಖಾಶಕ್ತಿ ಶ್ರೀಪಾದುಕಾಂ ಪೂ । ನಮಃ ॥\\
\as{೪ ಐಂ } ಕವಚಾಯ ಹುಂ । ಕವಚಶಕ್ತಿ ಶ್ರೀಪಾದುಕಾಂ ಪೂ । ನಮಃ ॥\\
\as{೪ ಕ್ಲೀಂ} ನೇತ್ರತ್ರಯಾಯ ವೌಷಟ್ । ನೇತ್ರಶಕ್ತಿ ಶ್ರೀಪಾದುಕಾಂ ಪೂ । ನಮಃ ॥\\
\as{೪ ಸೌಃ } ಅಸ್ತ್ರಾಯ ಫಟ್ । ಅಸ್ತ್ರಶಕ್ತಿ ಶ್ರೀಪಾದುಕಾಂ ಪೂ । ನಮಃ ॥

ಇತಿ ಚ ಷಡಂಗಾನಿ ವಿನ್ಯಸ್ಯ, ತ್ರಿಕೋಣೇ ಸ್ವಾಗ್ರಾದಿಕೋಣೇಷು\\
\as{೪ ಐಂ ಕಏಈಲಹ್ರೀಂ} ನಮಃ ॥\\
\as{೪ ಕ್ಲೀಂ ಹಸಕಹಲಹ್ರೀಂ} ನಮಃ ॥\\
\as{೪ ಸೌಃ ಸಕಲಹ್ರೀಂ} ನಮಃ ॥\\

ಇತಿ ಸಂಪೂಜ್ಯ, ವಿದ್ಯಯಾ ಬಿಂದುಂ ಸಂಪೂಜ್ಯ, ಅಸ್ತ್ರಾಯ ಫಟ್ ಇತಿ ಸಾಮಾನ್ಯಾರ್ಘ್ಯಪಾತ್ರಾಧಾರಂ ಪ್ರಕ್ಷಾಲ್ಯ\\
\as{೪ ಅಂ} ಅಗ್ನಿಮಂಡಲಾಯ ಧರ್ಮಪ್ರದದಶಕಲಾತ್ಮನೇ ಸಾಮಾನ್ಯಾರ್ಘ್ಯಪಾತ್ರಾಧಾರಾಯ ನಮಃ । ಇತಿ ಮಂಡಲೋಪರಿ ನಿಧಾಯ

೪ ಅಗ್ನಿಂ ದೂತಂ +++++ಸುಕ್ರತುಮ್ ॥ ರಾಂರೀಂರೂಂರೈಂರೌಂರಃ ರಮಲವರಯೂಂ ಅಗ್ನಿಮಂಡಲಾಯ ನಮಃ । ಇತಿ ಅಗ್ನಿಮಂಡಲಂ ವಿಭಾವ್ಯ , ದಶ ವಹ್ನಿಕಾಲಾಃ ಪೂಜಯೇತ್ ।

ಓಂಐಂಹ್ರೀಂಶ್ರೀಂ ಯಂ ಧೂಮ್ರಾರ್ಚಿಷೇ   ನಮಃ~। ೪  ರಂ ಊಷ್ಮಾಯೈ~। ೪  ಲಂ ಜ್ವಲಿನ್ಯೈ~। ೪  ವಂ ಜ್ವಾಲಿನ್ಯೈ~। ೪  ಶಂ ವಿಸ್ಫುಲಿಂಗಿನ್ಯೈ ~। ೪  ಷಂ ಸುಶ್ರಿಯೈ ~। ೪  ಸಂ ಸುರೂಪಾಯೈ ~। ೪  ಹಂ ಕಪಿಲಾಯೈ ~। ೪  ಳಂ ಹವ್ಯವಹಾಯೈ ~। ೪  ಕ್ಷಂ ಕವ್ಯವಹಾಯೈ ನಮಃ~॥

ಅಸ್ತ್ರಾಯ ಫಟ್ ಇತಿ ಶಂಖಂ ಪ್ರಕ್ಷಾಳ್ಯ\\
\as{೪ ಉಂ} ಸೂರ್ಯಮಂಡಲಾಯ ಅರ್ಥಪ್ರದದ್ವಾದಶಕಲಾತ್ಮನೇ ಸಾಮಾನ್ಯಾರ್ಘ್ಯಪಾತ್ರಾಯ ನಮಃ । ಇತಿ ನಿಧಾಯ

೪ ಆಕೃಷ್ಣೇನ +++++ನಿ ಪಶ್ಯನ್ ॥ ಹಾಂಹೀಂಹೂಂಹೈಂಹೌಂಹಃ ಹಮಲವರಯೂಂ ಸೂರ್ಯಮಂಡಲಾಯ ನಮಃ । ಇತಿ ಸೂರ್ಯಮಂಡಲಂ ವಿಭಾವ್ಯ , ದ್ವಾದಶ ಸೂರ್ಯಕಾಲಾಃ ಪೂಜಯೇತ್ ।

ಓಂಐಂಹ್ರೀಂಶ್ರೀಂ ಕಂ ಭಂ ತಪಿನ್ಯೈ ನಮಃ~। ೪ ಖಂ ಬಂ ತಾಪಿನ್ಯೈ~। ೪ ಗಂ ಫಂ ಧೂಮ್ರಾಯೈ~। ೪ ಘಂ ಪಂ ಮರೀಚ್ಯೈ~। ೪ ಙಂ ನಂ ಜ್ವಾಲಿನ್ಯೈ~। ೪ ಚಂ ಧಂ ರುಚ್ಯೈ~। ೪ ಛಂ ದಂ ಸುಷುಮ್ನಾಯೈ~। ೪ ಜಂ ಥಂ ಭೋಗದಾಯೈ~। ೪ ಝಂ ತಂ ವಿಶ್ವಾಯೈ~। ೪ ಞಂ ಣಂ ಬೋಧಿನ್ಯೈ~। ೪ ಟಂ ಢಂ ಧಾರಿಣ್ಯೈ~। ೪ ಠಂ ಡಂ ಕ್ಷಮಾಯೈ ನಮಃ~॥

\as{೪ ಮಂ} ಸೋಮಮಂಡಲಾಯ ಕಾಮಪ್ರದಷೋಡಶಕಲಾತ್ಮನೇ ಸಾಮಾನ್ಯಾರ್ಘ್ಯಪಾತ್ರಾಮೃತಾಯ ನಮಃ । ಇತಿ ಕಲಶೋದಕೇನ ಶಂಖಂ ಪ್ರಪೂರ್ಯ, ಕ್ಷೀರಬಿಂದುಂ ದತ್ವಾ  ಗಂಧಾದಿಭಿರಭ್ಯರ್ಚ್ಯ,

೪ ಆಪ್ಯಾಯಸ್ವ +++++ಸಂಗಥೇ ॥ ಸಾಂಸೀಂಸೂಂಸೈಂಸೌಂಸಃ ಸಮಲವರಯೂಂ ಸೋಮಮಂಡಲಾಯ ನಮಃ । ಇತಿ ಸೋಮಮಂಡಲಂ ವಿಭಾವ್ಯ ,ತತ್ರ ಷೋಡಶ ಸೋಮಕಾಲಾಃ ಪೂಜಯೇತ್ ।

ಓಂಐಂಹ್ರೀಂಶ್ರೀಂ ಅಂ ಅಮೃತಾಯೈ ನಮಃ~। ೪ ಆಂ ಮಾನದಾಯೈ~। ೪ ಇಂ ಪೂಷಾಯೈ~। ೪ ಈಂ ತುಷ್ಟ್ಯೈ~। ೪ ಉಂ ಪುಷ್ಟ್ಯೈ~। ೪ ಊಂ ರತ್ಯೈ~। ೪ ಋಂ ಧೃತ್ಯೈ~। ೪ ೠಂ ಶಶಿನ್ಯೈ~। ೪ ಲೃಂ ಚಂದ್ರಿಕಾಯೈ~। ೪ ಲೄಂ ಕಾಂತ್ಯೈ~। ೪ ಏಂ ಜ್ಯೋತ್ಸ್ನಾಯೈ~। ೪ ಐಂ ಶ್ರಿಯೈ~। ೪ ಓಂ ಪ್ರೀತ್ಯೈ~। ೪ ಔಂ ಅಂಗದಾಯೈ~। ೪ ಅಂ ಪೂರ್ಣಾಯೈ~। ೪ ಅಃಪೂರ್ಣಾಮೃತಾಯೈ ನಮಃ~॥

ತತಃ ಆಗ್ನೇಯಾದಿ ದಿಕ್ಷು ಮಧ್ಯೇ ಚ \\
\as{೪ ಐಂ } ಹೃದಯಾಯ ನಮಃ । ಹೃದಯಶಕ್ತಿ ಶ್ರೀಪಾದುಕಾಂ ಪೂ । ನಮಃ ॥\\
\as{೪ ಕ್ಲೀಂ } ಶಿರಸೇ ಸ್ವಾಹಾ । ಶಿರಃಶಕ್ತಿ ಶ್ರೀಪಾದುಕಾಂ ಪೂ । ನಮಃ ॥\\
\as{೪ ಸೌಃ } ಶಿಖಾಯೈ ವಷಟ್ । ಶಿಖಾಶಕ್ತಿ ಶ್ರೀಪಾದುಕಾಂ ಪೂ । ನಮಃ ॥\\
\as{೪ ಐಂ } ಕವಚಾಯ ಹುಂ । ಕವಚಶಕ್ತಿ ಶ್ರೀಪಾದುಕಾಂ ಪೂ । ನಮಃ ॥\\
\as{೪ ಕ್ಲೀಂ} ನೇತ್ರತ್ರಯಾಯ ವೌಷಟ್ । ನೇತ್ರಶಕ್ತಿ ಶ್ರೀಪಾದುಕಾಂ ಪೂ । ನಮಃ ॥\\
\as{೪ ಸೌಃ } ಅಸ್ತ್ರಾಯ ಫಟ್ । ಅಸ್ತ್ರಶಕ್ತಿ ಶ್ರೀಪಾದುಕಾಂ ಪೂ । ನಮಃ ॥\\
ಇತಿ ಚ ಷಡಂಗಾನಿ ವಿನ್ಯಸ್ಯ\\

\as{ಅಸ್ತ್ರಾಯ ಫಟ್ } ಇತಿ ಸಂರಕ್ಷ್ಯ । \as{ಕವಚಾಯ ಹುಂ } ಇತಿ ಅವಕುಂಠನ ಮುದ್ರಯಾ ಅವಕುಂಠ್ಯ\\
ಪ್ರಣವೇನ ಅಷ್ಟವಾರಮಭಿಮಂತ್ರ್ಯ ಧೇನುಮುದ್ರಾಂ ಯೋನಿಮುದ್ರಾಂ ಚ ಪ್ರದರ್ಶ್ಯ ।  ಪೂಜಾದ್ರವ್ಯಾಣಿ ಆತ್ಮಾನಂ ಚ ಸಂಪ್ರೋಕ್ಷ್ಯ , ಕಲಶೇ ಕಿಂಚಿನ್ನಿಕ್ಷಿಪೇತ್ ।

\section{ವಿಶೇಷಾರ್ಘ್ಯ ವಿಧಿಃ}
\addcontentsline{toc}{section}{ವಿಶೇಷಾರ್ಘ್ಯ ವಿಧಿಃ}
ಬಿಂದು ತ್ರಿಕೋಣ ಷಟ್ಕೋಣ ವೃತ್ತ ಚತುರಸ್ರಾತ್ಮಕಂ ಮಂಡಲಂ ವಿಧಾಯ , ಚತುರಸ್ರೇ ಆಗ್ನೇಯಾದಿಷು,\\
\as{೪ ಐಂ } ಹೃದಯಾಯ ನಮಃ । ಹೃದಯಶಕ್ತಿ ಶ್ರೀಪಾದುಕಾಂ ಪೂ । ನಮಃ ॥\\
\as{೪ ಕ್ಲೀಂ } ಶಿರಸೇ ಸ್ವಾಹಾ । ಶಿರಃಶಕ್ತಿ ಶ್ರೀಪಾದುಕಾಂ ಪೂ । ನಮಃ ॥\\
\as{೪ ಸೌಃ } ಶಿಖಾಯೈ ವಷಟ್ । ಶಿಖಾಶಕ್ತಿ ಶ್ರೀಪಾದುಕಾಂ ಪೂ । ನಮಃ ॥\\
\as{೪ ಐಂ } ಕವಚಾಯ ಹುಂ । ಕವಚಶಕ್ತಿ ಶ್ರೀಪಾದುಕಾಂ ಪೂ । ನಮಃ ॥\\
\as{೪ ಕ್ಲೀಂ} ನೇತ್ರತ್ರಯಾಯ ವೌಷಟ್ । ನೇತ್ರಶಕ್ತಿ ಶ್ರೀಪಾದುಕಾಂ ಪೂ । ನಮಃ ॥\\
\as{೪ ಸೌಃ } ಅಸ್ತ್ರಾಯ ಫಟ್ । ಅಸ್ತ್ರಶಕ್ತಿ ಶ್ರೀಪಾದುಕಾಂ ಪೂ । ನಮಃ ॥

ಷಟ್ಕೋಣೇ ಸ್ವಾಗ್ರಾದಿಕೋಣೇಷು\\
\as{೪ ಐಂ೫ } ಹೃದಯಾಯ ನಮಃ । ಹೃದಯಶಕ್ತಿ ಶ್ರೀಪಾದುಕಾಂ ಪೂ । ನಮಃ ॥\\
\as{೪ ಕ್ಲೀಂ೬ } ಶಿರಸೇ ಸ್ವಾಹಾ । ಶಿರಃಶಕ್ತಿ ಶ್ರೀಪಾದುಕಾಂ ಪೂ । ನಮಃ ॥\\
\as{೪ ಸೌಃ೪ } ಶಿಖಾಯೈ ವಷಟ್ । ಶಿಖಾಶಕ್ತಿ ಶ್ರೀಪಾದುಕಾಂ ಪೂ । ನಮಃ ॥\\
\as{೪ ಐಂ೫ } ಕವಚಾಯ ಹುಂ । ಕವಚಶಕ್ತಿ ಶ್ರೀಪಾದುಕಾಂ ಪೂ । ನಮಃ ॥\\
\as{೪ ಕ್ಲೀಂ೬} ನೇತ್ರತ್ರಯಾಯ ವೌಷಟ್ । ನೇತ್ರಶಕ್ತಿ ಶ್ರೀಪಾದುಕಾಂ ಪೂ । ನಮಃ॥\\
\as{೪ ಸೌಃ೪ } ಅಸ್ತ್ರಾಯ ಫಟ್ । ಅಸ್ತ್ರಶಕ್ತಿ ಶ್ರೀಪಾದುಕಾಂ ಪೂ । ನಮಃ ॥

ಇತಿ ಚ ಷಡಂಗಾನಿ ವಿನ್ಯಸ್ಯ, ತ್ರಿಕೋಣೇ ಸ್ವಾಗ್ರಾದಿಕೋಣೇಷು\\
\as{೪ ಐಂ ಕಏಈಲಹ್ರೀಂ} ನಮಃ ॥\\
\as{೪ ಕ್ಲೀಂ ಹಸಕಹಲಹ್ರೀಂ} ನಮಃ ॥\\
\as{೪ ಸೌಃ ಸಕಲಹ್ರೀಂ} ನಮಃ ॥\\

ಇತಿ ಸಂಪೂಜ್ಯ, ಮೂಲೇನ ಬಿಂದುಂ ಸಂಪೂಜ್ಯ, 

ಅಸ್ತ್ರ ಮಂತ್ರೇಣ ವಿಶೇಷಾರ್ಘ್ಯಪಾತ್ರಾಧಾರಂ ಪ್ರಕ್ಷಾಲ್ಯ\\
\as{೪ ಐಂ೫ ಅಂ} ಅಗ್ನಿಮಂಡಲಾಯ ಧರ್ಮಪ್ರದದಶಕಲಾತ್ಮನೇ ವಿಶೇಷಾರ್ಘ್ಯಪಾತ್ರಾಧಾರಾಯ ನಮಃ । ಇತಿ ಮಂಡಲೋಪರಿ ನಿಧಾಯ\\
ಅಸ್ತ್ರ ಮಂತ್ರೇಣ ವಿಶೇಷಾರ್ಘ್ಯಪಾತ್ರಂ ಪ್ರಕ್ಷಾಳ್ಯ
\as{೪ ಕ್ಲೀಂ೬ ಉಂ} ಸೂರ್ಯಮಂಡಲಾಯ ಅರ್ಥಪ್ರದ ದ್ವಾದಶಕಲಾತ್ಮನೇ ವಿಶೇಷಾರ್ಘ್ಯ ಪಾತ್ರಾಯ ನಮಃ । ಇತಿ ಪಾತ್ರಂ ನಿಧಾಯ\\
\as{೪ ಸೌಃ೪ ಮಂ} ಸೋಮಮಂಡಲಾಯ ಕಾಮಪ್ರದ ಷೋಡಶಕಲಾತ್ಮನೇ ವಿಶೇಷಾರ್ಘ್ಯ ಪಾತ್ರಾಮೃತಾಯ ನಮಃ ।\\ಅಂ ಆಂ ಇಂ ಈಂ++++ಳಂ ಕ್ಷಂ ॥\\ ಕ್ಷಂ ಳಂ ++++ಈಂ ಇಂ ಆಂ ಅಂ ॥ ನಮಃ ॥\\
ಮೂಲೇನ ಸಪ್ತವಾರಂ ಶಕ್ತಿಪಂಚಾಕ್ಷರ್ಯಾ ಪಂಚವಾರಂ ಚಾಭಿಮಂತ್ರ್ಯ\\
ಆಂ ಸೋಹಂ~। ಆಂ ಹ್ರೀಂ ಕ್ರೋಂ ಯರಲವಶಷಸಹೋಂ ॥\\
ಸುಧಾದೇವ್ಯಾಃ ಪ್ರಾಣಾ ಇಹ ಪ್ರಾಣಾಃ~।\\
ಸುಧಾದೇವ್ಯಾಃ ಜೀವ ಇಹ ಸ್ಥಿತಃ~।\\
ಸುಧಾದೇವ್ಯಾಃ ಸರ್ವೇಂದ್ರಿಯಾಣಿ~।\\
ಸುಧಾದೇವ್ಯಾಃ ವಾಙ್ಮನಸ್ತ್ವಕ್ಚಕ್ಷುಃ ಶ್ರೋತ್ರ ಜಿಹ್ವಾಘ್ರಾಣಪ್ರಾಣಾ\\
ಇಹೈವಾಗತ್ಯ ಸುಖಂ ಚಿರಂ ತಿಷ್ಠಂತು ಸ್ವಾಹಾ ॥\\
೪ ೧೫ ತಾಂ ಚಿನ್ಮಯೀಂ ಆನಂದಲಕ್ಷಣಾಂ ಅಮೃತಕಲಶ ಪಿಶಿತಹಸ್ತದ್ವಯಾಂ ಪ್ರಸನ್ನಾಂ ದೇವೀಂ ಪೂಜಯಾಮಿ ಸ್ವಾಹಾ ।\\
ಶಂನೋ ದೇವೀರಭಿ+++++++ವಂತು ನಃ ॥\\
ಗಂಧಾದಿಭಿಃ ಸಮಂತ್ರಮಭ್ಯರ್ಚ್ಯ,
\as{ವಷಟ್} ಇತಿ ಕಿಂಚಿದುದ್ಧೃತ್ಯ,\\
\as{ಸ್ವಾಹಾ} ಇತಿ ತತ್ರೈವ ವಿಸೃಜ್ಯ,\\
\as{ಹುಂ} ಇತಿ ಅವಗುಂಠ್ಯ,\\
\as{ವೌಷಟ್} ಇತಿ ಧೇನುಮುದ್ರಾಂ ಪ್ರದರ್ಶ್ಯ,\\
\as{ಫಟ್} ಇತಿ ಸಂರಕ್ಷ್ಯ,\\
\as{ನಮಃ} ಇತಿ ಪುಷ್ಪಂ ವಿಕೀರ್ಯ,\\
\as{ಐಂ} ಇತಿ ಯೋನ್ಯಾ ನಮಸ್ಕೃತ್ಯ,\\
ಮೂಲೇನ ಸಕೃದಭಿಮಂತ್ರ್ಯ, ಪಾತ್ರಂ ಸ್ಪೃಷ್ಟ್ವಾ
\subsection{ವಹ್ನಿಕಲಾಃ}ಓಂಐಂಹ್ರೀಂಶ್ರೀಂ ಯಂ ಧೂಮ್ರಾರ್ಚಿಷೇ   ನಮಃ~। ೪  ರಂ ಊಷ್ಮಾಯೈ~। ೪  ಲಂ ಜ್ವಲಿನ್ಯೈ~। ೪  ವಂ ಜ್ವಾಲಿನ್ಯೈ~। ೪  ಶಂ ವಿಸ್ಫುಲಿಂಗಿನ್ಯೈ ~। ೪  ಷಂ ಸುಶ್ರಿಯೈ ~। ೪  ಸಂ ಸುರೂಪಾಯೈ ~। ೪  ಹಂ ಕಪಿಲಾಯೈ ~। ೪  ಳಂ ಹವ್ಯವಾಹಾಯೈ ~। ೪  ಕ್ಷಂ ಕವ್ಯವಾಹಾಯೈ ನಮಃ~॥
\subsection{ಸೂರ್ಯಕಲಾಃ}ಓಂಐಂಹ್ರೀಂಶ್ರೀಂ ಕಂ ಭಂ ತಪಿನ್ಯೈ ನಮಃ~। ೪ ಖಂ ಬಂ ತಾಪಿನ್ಯೈ~। ೪ ಗಂ ಫಂ ಧೂಮ್ರಾಯೈ~। ೪ ಘಂ ಪಂ ಮರೀಚ್ಯೈ~। ೪ ಙಂ ನಂ ಜ್ವಾಲಿನ್ಯೈ~। ೪ ಚಂ ಧಂ ರುಚ್ಯೈ~। ೪ ಛಂ ದಂ ಸುಷುಮ್ನಾಯೈ~। ೪ ಜಂ ಥಂ ಭೋಗದಾಯೈ~। ೪ ಝಂ ತಂ ವಿಶ್ವಾಯೈ~। ೪ ಞಂ ಣಂ ಬೋಧಿನ್ಯೈ~। ೪ ಟಂ ಢಂ ಧಾರಿಣ್ಯೈ~। ೪ ಠಂ ಡಂ ಕ್ಷಮಾಯೈ ನಮಃ~॥
\subsection{ಸೋಮಕಲಾಃ}ಓಂಐಂಹ್ರೀಂಶ್ರೀಂ ಅಂ ಅಮೃತಾಯೈ ನಮಃ~। ೪ ಆಂ ಮಾನದಾಯೈ~। ೪ ಇಂ ಪೂಷಾಯೈ~। ೪ ಈಂ ತುಷ್ಟ್ಯೈ~। ೪ ಉಂ ಪುಷ್ಟ್ಯೈ~। ೪ ಊಂ ರತ್ಯೈ~। ೪ ಋಂ ಧೃತ್ಯೈ~। ೪ ೠಂ ಶಶಿನ್ಯೈ~। ೪ ಲೃಂ ಚಂದ್ರಿಕಾಯೈ~। ೪ ಲೄಂ ಕಾಂತ್ಯೈ~। ೪ ಏಂ ಜ್ಯೋತ್ಸ್ನಾಯೈ~। ೪ ಐಂ ಶ್ರಿಯೈ~। ೪ ಓಂ ಪ್ರೀತ್ಯೈ~। ೪ ಔಂ ಅಂಗದಾಯೈ~। ೪ ಅಂ ಪೂರ್ಣಾಯೈ~। ೪ ಅಃ ಪೂರ್ಣಾಮೃತಾಯೈ ನಮಃ~॥
\subsection{ಬ್ರಹ್ಮಕಲಾಃ} ಓಂಐಂಹ್ರೀಂಶ್ರೀಂ ಕಂ ಸೃಷ್ಟ್ಯೈ ನಮಃ~। ೪ ಖಂ ಋದ್ಧ್ಯೈ~। ೪ ಗಂ ಸ್ಮೃತ್ಯೈ~। ೪ ಘಂ ಮೇಧಾಯೈ~। ೪ ಙಂ ಕಾಂತ್ಯೈ~। ೪ ಚಂ ಲಕ್ಷ್ಮ್ಯೈ~। ೪ ಛಂ ದ್ಯುತ್ಯೈ~। ೪ ಜಂ ಸ್ಥಿರಾಯೈ~। ೪ ಝಂ ಸ್ಥಿತ್ಯೈ~। ೪ ಞಂ ಸಿದ್ಧ್ಯೈ~॥
\subsection{ವಿಷ್ಣುಕಲಾಃ} ಓಂಐಂಹ್ರೀಂಶ್ರೀಂ ಟಂ ಜರಾಯೈ ನಮಃ~। ೪ ಠಂ ಪಾಲಿನ್ಯೈ~। ೪ ಡಂ ಶಾಂತ್ಯೈ~। ೪ ಢಂ ಈಶ್ವರ್ಯೈ~। ೪ ಣಂ ರತ್ಯೈ~। ೪ ತಂ ಕಾಮಿಕಾಯೈ~। ೪ ಥಂ ವರದಾಯೈ~। ೪ ದಂ ಆಹ್ಲಾದಿನ್ಯೈ~। ೪ ಧಂ ಪ್ರೀತ್ಯೈ~। ೪ ನಂ ದೀರ್ಘಾಯೈ~॥
\subsection{ರುದ್ರಕಲಾಃ} ಓಂಐಂಹ್ರೀಂಶ್ರೀಂ ಪಂ ತೀಕ್ಷ್ಣಾಯೈ ನಮಃ~। ೪ ಫಂ ರೌದ್ರ್ಯೈ~। ೪ ಬಂ ಭಯಾಯೈ~। ೪ ಭಂ ನಿದ್ರಾಯೈ~। ೪ ಮಂ ತಂದ್ರ್ಯೈ~। ೪ ಯಂ ಕ್ಷುಧಾಯೈ~। ೪ ರಂ ಕ್ರೋಧಿನ್ಯೈ~। ೪ ಲಂ ಕ್ರಿಯಾಯೈ~। ೪ ವಂ ಉದ್ಗಾರ್ಯೈ~। ೪ ಶಂ ಮೃತ್ಯು~॥
\subsection{ಈಶ್ವರಕಲಾಃ} ಓಂಐಂಹ್ರೀಂಶ್ರೀಂ ಷಂ ಪೀತಾಯೈ~। ೪ ಸಂ ಶ್ವೇತಾಯೈ~। ೪ ಹಂ ಅರುಣಾಯೈ~। ೪ ಕ್ಷಂ ಅಸಿತಾಯೈ~॥
\subsection{ಸದಾಶಿವಕಲಾಃ} ಓಂಐಂಹ್ರೀಂಶ್ರೀಂ ಅಂ ನಿವೃತ್ತ್ಯೈ ನಮಃ~। ೪ ಆಂ ಪ್ರತಿಷ್ಠಾಯೈ~। ೪ ಇಂ ವಿದ್ಯಾಯೈ~। ೪ ಈಂ ಶಾಂತ್ಯೈ~। ೪ ಉಂ ಇಂಧಿಕಾಯೈ~। ೪ ಊಂ ದೀಪಿಕಾಯೈ~। ೪ ಋಂ ರೇಚಿಕಾಯೈ~। ೪ ೠಂ ಮೋಚಿಕಾಯೈ~। ೪ ಲೃಂ ಪರಾಯೈ~। ೪ ಲೄಂ ಸೂಕ್ಷ್ಮಾಯೈ~। ೪ ಏಂ ಸೂಕ್ಷ್ಮಾಮೃತಾಯೈ~। ೪ ಐಂ ಜ್ಞಾನಾಯೈ~। ೪ ಓಂ ಜ್ಞಾನಾಮೃತಾಯೈ~। ೪ ಔಂ ಆಪ್ಯಾಯನ್ಯೈ~। ೪ ಅಂ ವ್ಯಾಪಿನ್ಯೈ~। ೪ ಅಃ ವ್ಯೋಮರೂಪಾಯೈ~॥

ಅಖಂಡೈಕರಸಾನಂದಕರೇ ಪರಸುಧಾತ್ಮನಿ ।\\ ಸ್ವಚ್ಛಂದಸ್ಫುರಣಾಮತ್ರ ನಿಧೇಹಿ ಕುಲನಾಯಿಕೇ ॥

ಅಕುಲಸ್ಥಾಮೃತಾಕಾರೇ ಶುದ್ಧಜ್ಞಾನಕರೇ ಪರೇ ।\\ ಅಮೃತತ್ವಂ ನಿಧೇಹ್ಯಸ್ಮಿನ್ ವಸ್ತುನಿ ಕ್ಲಿನ್ನರೂಪಿಣಿ ॥

ತ್ವದ್ರೂಪಿಣ್ಯೈಕರಸ್ಯತ್ವಂ ಕೃತ್ವಾ ಹ್ಯೇತತ್ ಸ್ವರೂಪಿಣಿ ।\\ ಭೂತ್ವಾ ಪರಾಮೃತಾಕಾರೇ ಮಯಿ ಚಿತ್ಸ್ಫುರಣಂ ಕುರು ॥

\as{೪ ಐಂ ಬ್ಲೂಂ ಝ್ರೌಂ ಜುಂ ಸಃ ಅಮೃತೇ ಅಮೃತೋದ್ಭವೇ ಅಮೃತೇಶ್ವರಿ ಅಮೃತವರ್ಷಿಣಿ ಅಮೃತಂ ಸ್ರಾವಯ ಸ್ರಾವಯ ಸ್ವಾಹಾ ॥ ನಮಃ ॥\\೪ ಐಂ ವದ ವದ ವಾಗ್ವಾದಿನಿ ಐಂ ಕ್ಲೀಂ ಕ್ಲಿನ್ನೇ ಕ್ಲೇದಿನಿ ಕ್ಲೇದಯ ಮಹಾಕ್ಷೋಭಂ ಕುರು ಕುರು ಕ್ಲೀಂ ಸೌಃ ಮೋಕ್ಷಂ ಕುರು ಕುರು ಹ್ಸೌಂ ಸೌಃ ॥ ನಮಃ ॥}

ಏವಮಭಿಮಂತ್ರ್ಯ  ವಿಶೇಷಾರ್ಘ್ಯಬಿಂದುಭಿಃ ಗುರುತ್ರಯಂ ಸ್ವಶಿರಸಿ ತರ್ಪಯೇತ್ ॥\\
\dhyana{೪ ಹಂಸಃಶಿವಃ ಸೋಽಹಂಹಂಸಃ ಹ್‌ಸ್‌ಖ್‌ಫ್ರೇಂ ಹಸಕ್ಷಮಲವರಯೂಂ ಹ್‌ಸೌಃ ಸಹಕ್ಷಮಲವರಯೀಂ ಸ್‌ಹೌಃ ಹಂಸಃ ಶಿವಃ ಸೋಽಹಂ ಹಂಸಃ॥} ಸ್ವಾತ್ಮಾರಾಮ ಪರಮಾನಂದ ಪಂಜರ ವಿಲೀನ ತೇಜಸೇ ಪರಮೇಷ್ಠಿಗುರವೇ ನಮಃ~।ಶ್ರೀಪಾದುಕಾಂ ಪೂ । ತ । ನಮಃ ॥\\
\dhyana{೪ ಸೋಽಹಂ ಹಂಸಃ ಶಿವಃ ಹ್‌ಸ್‌ಖ್‌ಫ್ರೇಂ ಹಸಕ್ಷಮಲವರಯೂಂ ಹ್‌ಸೌಃ ಸಹಕ್ಷಮಲವರಯೀಂ ಸ್‌ಹೌಃ ಸೋಽಹಂ ಹಂಸಃ ಶಿವಃ॥} ಸ್ವಚ್ಛಪ್ರಕಾಶ ವಿಮರ್ಶಹೇತವೇ ಪರಮಗುರವೇ ನಮಃ।ಶ್ರೀಪಾದುಕಾಂ ಪೂ । ತ । ನಮಃ ॥\\
\dhyana{೪ ಹಂಸಃ ಶಿವಃ ಸೋಽಹಂ ಹ್‌ಸ್‌ಖ್‌ಫ್ರೇಂ ಹಸಕ್ಷಮಲವರಯೂಂ ಹ್‌ಸೌಃ ಸಹಕ್ಷಮಲವರಯೀಂ ಸ್‌ಹೌಃ ಹಂಸಃ ಶಿವಃ ಸೋಽಹಂ~॥} ಸ್ವರೂಪ ನಿರೂಪಣ ಹೇತವೇ ಶ್ರೀಗುರವೇ ನಮಃ~।ಶ್ರೀಪಾದುಕಾಂ ಪೂ । ತ । ನಮಃ ॥

ಅಥ ಪೂಜಾ ನಿರ್ವಿಘ್ನತಾ ಸಿದ್ಧ್ಯರ್ಥಂ ಶ್ರೀಗುರು ದಕ್ಷಿಣಾಮೂರ್ತಿಸಹಿತ ಮಹಾಗಣಪತಿಪೂಜಾಂ ಕರಿಷ್ಯೇ ॥

\newpage
\section{ಗುರುಗಣಪತಿಪೂಜಾ}
\addcontentsline{toc}{section}{ಗುರುಗಣಪತಿಪೂಜಾ}
\thispagestyle{empty}
ಭವಸಂಚಿತಪಾಪೌಘವಿಧ್ವಂಸನವಿಚಕ್ಷಣಂ~।\\
ವಿಘ್ನಾಂಧಕಾರಭಾಸ್ವಂತಂ ವಿಘ್ನರಾಜಮಹಂ ಭಜೇ ॥\\
\as{ಓಂ ಲಂ ನಮಸ್ತೇ ಗಣಪತಯೇ ॥ ತ್ವಮೇವ ಪ್ರತ್ಯಕ್ಷಂ ತತ್ವಮಸಿ ॥ \\ತ್ವಮೇವ ಕೇವಲಂ ಕರ್ತಾಸಿ ॥ ತ್ವಮೇವ ಕೇವಲಂ ಧರ್ತಾಸಿ ॥ ತ್ವಮೇವ ಕೇವಲಂ ಹರ್ತಾಸಿ ॥ ತ್ವಮೇವ ಸರ್ವಂ ಖಲ್ವಿದಂ ಬ್ರಹ್ಮಾಸಿ ॥ ತ್ವಂ ಸಾಕ್ಷಾದಾತ್ಮಾಸಿ ನಿತ್ಯಂ ॥}ಧ್ಯಾನಮ್॥

ಅತ್ರಾಗಚ್ಛ ಜಗದ್ವಂದ್ಯ ಸುರರಾಜಾರ್ಚಿತೇಶ್ವರ~।\\
ಅನಾಥನಾಥ ಸರ್ವಜ್ಞ ಗೌರೀಗರ್ಭಸಮುದ್ಭವ ॥\\
\as{ಋತಂ ವಚ್ಮಿ ॥ ಸತ್ಯಂ ವಚ್ಮಿ ॥}ಆವಾಹನಮ್॥

ಮೌಕ್ತಿಕೈಃ ಪುಷ್ಯರಾಗೈಶ್ಚ ನಾನಾರತ್ನವಿರಾಜಿತಂ~।\\
ರತ್ನಸಿಂಹಾಸನಂ ಚಾರು ಪ್ರೀತ್ಯರ್ಥಂ ಪ್ರತಿಗೃಹ್ಯತಾಂ ॥\\
\as{ಅವ ತ್ವಂ ಮಾಂ ॥ ಅವ ವಕ್ತಾರಂ ॥ ಅವ ಶ್ರೋತಾರಂ ॥ ಅವ ದಾತಾರಂ ॥ ಅವ ಧಾತಾರಂ ॥ ಅವಾನೂಚಾನಮವ ಶಿಷ್ಯಂ ॥ ಅವ ಪಶ್ಚಾತ್ತಾತ್ ॥ ಅವ ಪುರಸ್ತಾತ್ ॥ ಅವೋತ್ತರಾತ್ತಾತ್ ॥ ಅವ ದಕ್ಷಿಣಾತ್ತಾತ್ ॥ ಅವ ಚೋರ್ಧ್ವಾತ್ತಾತ್ ॥ ಅವಾಧರಾತ್ತಾತ್ ॥ ಸರ್ವತೋ ಮಾಂ ಪಾಹಿ ಪಾಹಿ ಸಮಂತಾತ್ ॥}ಆಸನಮ್॥

ಗಜವಕ್ತ್ರ ನಮಸ್ತೇಽಸ್ತು ಸರ್ವಾಭೀಷ್ಟಪ್ರದಾಯಕ~।\\
ಭಕ್ತ್ಯಾ ಪಾದ್ಯಂ ಮಯಾ ದತ್ತಂ ಗೃಹಾಣ ದ್ವಿರದಾನನ ॥\\
\as{ತ್ವಂ ವಾಙ್ಮಯಸ್ತ್ವಂ ಚಿನ್ಮಯಃ ॥ ತ್ವಮಾನಂದಮಯಸ್ತ್ವಂ ಬ್ರಹ್ಮಮಯಃ ॥ ತ್ವಂ ಸಚ್ಚಿದಾನಂದಾದ್ವಿತೀಯೋಽಸಿ। ತ್ವಂ ಪ್ರತ್ಯಕ್ಷಂ ಬ್ರಹ್ಮಾಸಿ। ತ್ವಂ ಜ್ಞಾನಮಯೋ ವಿಜ್ಞಾನಮಯೋಽಸಿ ॥}ಪಾದ್ಯಮ್॥

ಗೌರೀಪುತ್ರ ನಮಸ್ತೇಽಸ್ತು ಶಂಕರಪ್ರಿಯನಂದನ~।\\
ಗೃಹಾಣಾರ್ಘ್ಯಂ ಮಯಾ ದತ್ತಂ ಗಂಧಪುಷ್ಪಾಕ್ಷತೈರ್ಯುತಂ ॥\\
\as{ಸರ್ವಂ ಜಗದಿದಂ ತ್ವತ್ತೋ ಜಾಯತೇ ॥ ಸರ್ವಂ ಜಗದಿದಂ ತ್ವತ್ತಸ್ತಿಷ್ಠತಿ ॥ ಸರ್ವಂ ಜಗದಿದಂ ತ್ವಯಿ ಲಯಮೇಷ್ಯತಿ ॥ ಸರ್ವಂ ಜಗದಿದಂ ತ್ವಯಿ ಪ್ರತ್ಯೇತಿ ॥ ತ್ವಂ ಭೂಮಿರಾಪೋಽನಲೋಽನಿಲೋ ನಭಃ ॥ ತ್ವಂ ಚತ್ವಾರಿ ವಾಕ್ಪದಾನಿ ॥}\\ಅರ್ಘ್ಯಮ್॥

ಅನಾಥನಾಥ ಸರ್ವಜ್ಞ ಗೀರ್ವಾಣವರಪೂಜಿತ~।\\
ಗೃಹಾಣಾಚಮನಂ ದೇವ ತುಭ್ಯಂ ದತ್ತಂ ಮಯಾ ಪ್ರಭೋ ॥\\
\as{ತ್ವಂ ಗುಣತ್ರಯಾತೀತಃ।  ತ್ವಂ ದೇಹತ್ರಯಾತೀತಃ। ತ್ವಂ ಕಾಲತ್ರಯಾತೀತಃ। ತ್ವಂ ಮೂಲಾಧಾರಸ್ಥಿತೋಽಸಿ ನಿತ್ಯಂ ॥ ತ್ವಂ ಶಕ್ತಿತ್ರಯಾತ್ಮಕಃ। ತ್ವಾಂ ಯೋಗಿನೋ ಧ್ಯಾಯಂತಿ ನಿತ್ಯಂ ॥ ತ್ವಂ ಬ್ರಹ್ಮಾ  ತ್ವಂ ವಿಷ್ಣುಸ್ತ್ವಂ ರುದ್ರಸ್ತ್ವಂ ಇಂದ್ರಸ್ತ್ವಂ ಅಗ್ನಿಸ್ತ್ವಂ ವಾಯುಸ್ತ್ವಂ ಸೂರ್ಯಸ್ತ್ವಂ ಚಂದ್ರಮಾಸ್ತ್ವಂ ಬ್ರಹ್ಮಭೂರ್ಭುವಃ ಸ್ವರೋಂ ॥}ಆಚಮನಮ್॥

ದಧಿಕ್ಷೀರಸಮಾಯುಕ್ತಂ ಮಧ್ವಾಜ್ಯೇನ ಸಮನ್ವಿತಂ~।\\
ಮಧುಪರ್ಕಂ ಗೃಹಾಣೇಮಂ ಗಜವಕ್ತ್ರ ನಮೋಽಸ್ತು ತೇ ॥ಮಧುಪರ್ಕಃ॥

ಗಂಗಾ ಸರಸ್ವತೀ ರೇವಾ ಪಯೋಷ್ಣೀ ಯಮುನಾಜಲೈಃ~।\\
ಸ್ನಪಯಾಮಿ ಗಣೇಶಾನ ತಥಾ ಶಾಂತಂ ಕುರುಷ್ವ ಮಾಂ॥ಸ್ನಾನಮ್॥

ದಧಿ ಕ್ಷೀರ ಘೃತೈರ್ಯುಕ್ತಂ ಶರ್ಕರಾ ಮಧುಮಿಶ್ರಿತಂ ।\\
ಪಂಚಾಮೃತಂ ಗೃಹಾಣ ತ್ವಂ ಕೃಪಯಾ ಪುರುಷೋತ್ತಮ  ॥\\ಪಂಚಾಮೃತಸ್ನಾನಂ॥

ರಕ್ತವಸ್ತ್ರದ್ವಯಂ ಚಾರು ದೇವಯೋಗ್ಯಂ ಚ ಮಂಗಳಂ~।\\
ಶುಭಪ್ರದ ಗೃಹಾಣ ತ್ವಂ ಲಂಬೋದರ ಹರಾತ್ಮಜ ॥\\
\as{ಗಣಾದದೀನ್ ಪೂರ್ವಮುಚ್ಚಾರ್ಯ ವರ್ಣಾದೀಂಸ್ತದನಂತರಂ। ಅನುಸ್ವಾರಃ ಪರತರಃ। ಅರ್ಧೇಂದುಲಸಿತಂ। ತಾರೇಣ ಋದ್ಧಂ। ಏತತ್ತವ ಮನುಸ್ವರೂಪಂ। ಗಕಾರಃ ಪೂರ್ವರೂಪಂ। ಅಕಾರೋ ಮಧ್ಯಮರೂಪಂ। ಅನುಸ್ವಾರಶ್ಚಾಂತ್ಯರೂಪಂ। ಬಿಂದುರುತ್ತರರೂಪಂ। ನಾದಃ ಸಂಧಾನಂ ॥ ಸಂಹಿತಾ ಸಂಧಿಃ। ಸೈಷಾ ಗಣೇಶವಿದ್ಯಾ। ಗಣಕ ಋಷಿಃ। ನಿಚೃದ್ಗಾಯತ್ರೀಛಂದಃ~। ಗಣಪತಿರ್ದೇವತಾ। ಓಂ ಗಂ ಗಣಪತಯೇ ನಮಃ ॥}ವಸ್ತ್ರಮ್॥

ರಾಜತಂ ಬ್ರಹ್ಮಸೂತ್ರಂ ಚ ಕಾಂಚನಂ ಚೋತ್ತರೀಯಕಂ~।\\
ಗೃಹಾಣ ಚಾರು ಸರ್ವಜ್ಞ ಭಕ್ತಾನಾಂ ವರದೋ ಭವ ॥\\
\as{ಏಕದಂತಾಯ ವಿದ್ಮಹೇ ವಕ್ರತುಂಡಾಯ ಧೀಮಹಿ। \\ತನ್ನೋ ದಂತೀ ಪ್ರಚೋದಯಾತ್ ॥}ಉಪವೀತಮ್॥

ತತಃ ಸಿಂದೂರಕಂ ದೇವ ಗೃಹಾಣ ಗಣನಾಯಕ~।\\
ಅಂಗಲೇಪನಭಾವಾರ್ಥಂ ಸದಾನಂದವಿವರ್ಧನಂ ॥ಸಿಂದೂರಮ್॥

ನಾನಾವಿಧಾನಿ ದಿವ್ಯಾನಿ ನಾನಾರತ್ನೋಜ್ವಲಾನಿ ಚ~।\\
ಭೂಷಣಾನಿ ಗೃಹಾಣೇಶ ಪಾರ್ವತೀಪ್ರಿಯನಂದನ ॥ಆಭರಣಂ॥

ಚಂದನಾಗರುಕಸ್ತೂರೀ ಕರ್ಪೂರಕುಂಕುಮಾನ್ವಿತಂ~।\\
ವಿಲೇಪನಂ ಸುರಶ್ರೇಷ್ಠ ಪ್ರೀತ್ಯರ್ಥಂ ಪ್ರತಿಗೃಹ್ಯತಾಂ ॥\\
\as{ಏಕದಂತಂ ಚತುರ್ಹಸ್ತಂ ಪಾಶಮಂ ಕುಶಧಾರಿಣಂ ॥ \\ರದಂ ಚ ವರದಂ ಹಸ್ತೈರ್ಬಿಭ್ರಾಣಂ ಮೂಷಕಧ್ವಜಂ॥}ಗಂಧಃ॥

ರಕ್ತಾಕ್ಷತಾಂಶ್ಚ ದೇವೇಶ ಗೃಹಾಣ ದ್ವಿರದಾನನ~।\\
ಲಲಾಟಫಲಕೇ ಚಂದ್ರಸ್ತಸ್ಯೋಪರ್ಯವಧಾರಯ ॥\\
\as{ರಕ್ತಂ ಲಂಬೋದರಂ ಶೂರ್ಪಕರ್ಣಕಂ ರಕ್ತವಾಸಸಂ ॥ \\ರಕ್ತಗಂಧಾನುಲಿಪ್ತಾಂಗಂ ರಕ್ತಪುಷ್ಪೈಃ ಸುಪೂಜಿತಂ॥}ಅಕ್ಷತಾಃ॥


ಮಾಲ್ಯಾದೀನಿ ಸುಗಂಧೀನಿ ಮಾಲತ್ಯಾದೀನಿ ಚ ಪ್ರಭೋ~।\\
ಮಯಾಹೃತಾನಿ ಪೂಜಾರ್ಥಂ ಪುಷ್ಪಾಣಿ ಸ್ವೀಕುರು ಪ್ರಭೋ ॥ಪುಷ್ಪಾಣಿ॥

ಗೌರೀಪುತ್ರ ಮಹಾಕಾಯ ಗೀರ್ವಾಣಗಣಪೂಜಿತ~।\\
ದೂರ್ವಾಯುಗ್ಮಂ ಪ್ರದಾಸ್ಯಾಮಿ ಸ್ವೀಕುರುಷ್ವ ಗಣಾಧಿಪ ॥ದೂರ್ವಾ॥

ಭಕ್ತಾನಾಂ ಸುಖದಾತಾ ತ್ವಂ ಸರ್ವಮಂಗಳಕಾರಕ~।\\
ಹರಿದ್ರಾಂ ತೇ ಪ್ರಯಚ್ಛಾಮಿ ಗೃಹಾಣ ಗಣನಾಯಕ ॥ಹರಿದ್ರಾ॥

ಕುಂಕುಮಂ ಕಾಂತಿದಂ ದಿವ್ಯಂ ಸರ್ವಕಾಮಫಲಪ್ರದಂ~।\\
ಸರ್ವದೇವೈಶ್ಚ ಸಂಪೂಜ್ಯಂ ಗೃಹಾಣ ವರದಾಯಕ ॥ಕುಂಕುಮಮ್॥

\section{ದಕ್ಷಿಣಾಮೂರ್ತಿ ಅಂಗಪೂಜಾ ॥}
\addcontentsline{toc}{section}{ದಕ್ಷಿಣಾಮೂರ್ತಿ ಅಂಗಪೂಜಾ ॥}
ಓಂ ಪಾಪನಾಶಾಯ ನಮಃ~। ಪಾದೌ ಪೂಜಯಾಮಿ॥\\
ಓಂ ಗುರವೇ ನಮಃ~। ಗುಲ್ಫೌ ಪೂಜಯಾಮಿ॥\\
ಓಂ ಜ್ಞಾನೇಶ್ವರಾಯ ನಮಃ~। ಜಂಘೇ ಪೂಜಯಾಮಿ॥\\
ಓಂ ಜಾಹ್ನವೀಪತಯೇ ನಮಃ~। ಜಾನುನೀ ಪೂಜಯಾಮಿ॥\\
ಓಂ ಉತ್ತಮೋತ್ತಮಾಯ ನಮಃ~। ಊರೂ ಪೂಜಯಾಮಿ॥\\
ಓಂ ಕಂದರ್ಪನಾಶಾಯ ನಮಃ~। ಕಟಿಂ ಪೂಜಯಾಮಿ॥\\
ಓಂ ಗೌರೀಶಾಯ ನಮಃ~। ಗುಹ್ಯಂ ಪೂಜಯಾಮಿ॥\\
ಓಂ ನಿರಂಜನಾಯ ನಮಃ~। ನಾಭಿಂ ಪೂಜಯಾಮಿ॥\\
ಓಂ ಸ್ಕಂದಗುರವೇ ನಮಃ~। ಸ್ಕಂಧೌ ಪೂಜಯಾಮಿ॥\\
ಓಂ ಹಿರಣ್ಯಬಾಹವೇ ನಮಃ~। ಬಾಹೂ ಪೂಜಯಾಮಿ॥\\
ಓಂ ಹರಾಯ ನಮಃ~। ಹಸ್ತೌ ಪೂಜಯಾಮಿ॥\\
ಓಂ ನೀಲಕಂಠಾಯ ನಮಃ~। ಕಂಠಂ ಪೂಜಯಾಮಿ॥\\
ಓಂ ಕಮಲವದನಾಯ ನಮಃ~। ಮುಖಂ ಪೂಜಯಾಮಿ॥\\
ಓಂ ಉಗ್ರಾಯ ನಮಃ~। ನಾಸಿಕಾಂ ಪೂಜಯಾಮಿ॥\\
ಓಂ ನಳಿನಾಕ್ಷಾಯ ನಮಃ~। ನೇತ್ರೇ ಪೂಜಯಾಮಿ॥\\
ಓಂ ಅಗ್ನಿತಿಲಕಾಯ ನಮಃ~। ತೃತೀಯನೇತ್ರಂ ಪೂಜಯಾಮಿ ।\\
ಓಂ ಭಸಿತಭೂಷಣಾಯ ನಮಃ~। ಲಲಾಟಂ ಪೂಜಯಾಮಿ॥\\
ಓಂ ಚಂದ್ರಮೌಲಯೇ ನಮಃ~। ಮೌಲಿಂ ಪೂಜಯಾಮಿ॥\\
ಓಂ ಗಂಗಾಧರಾಯ ನಮಃ~। ಶಿರಃ ಪೂಜಯಾಮಿ॥\\
ಓಂ ದಕ್ಷಿಣಾಮೂರ್ತಯೇ ನಮಃ~। ಸರ್ವಾಂಗಾನಿ ಪೂಜಯಾಮಿ॥
\section{ಪತ್ರ ಪೂಜಾ ॥}
\addcontentsline{toc}{section}{ಪತ್ರ ಪೂಜಾ ॥}
ಓಂ ಬಿಲ್ವೇಶಾಯ ನಮಃ~। ಬಿಲ್ವಪತ್ರಂ ಸಮರ್ಪಯಾಮಿ॥\\
ಓಂ ತುಷ್ಟಾಯ ನಮಃ~। ತುಲಸೀಪತ್ರಂ ಸಮರ್ಪಯಾಮಿ॥\\
ಓಂ ಆನಂದಾಯ ನಮಃ~। ಅರ್ಕಪತ್ರಂ ಸಮರ್ಪಯಾಮಿ॥\\
ಓಂ ಜಗದ್ಗುರವೇ ನಮಃ~। ಜಂಬೀರಪತ್ರಂ ಸಮರ್ಪಯಾಮಿ॥\\
ಓಂ ನಿರ್ಗುಣಾಯ ನಮಃ~। ನಿರ್ಗುಂಡೀಪತ್ರಂ ಸಮರ್ಪಯಾಮಿ॥\\
ಓಂ ದುರಿತಶಮನಾಯ ನಮಃ~। ದೂರ್ವಾಪತ್ರಂ ಸಮರ್ಪಯಾಮಿ॥\\
ಓಂ ಕುಮುದನೇತ್ರಾಯ ನಮಃ~। ಕುಶಪತ್ರಂ ಸಮರ್ಪಯಾಮಿ॥\\
ಓಂ ಮೃಗಧರಾಯ ನಮಃ~। ಮರುಗಪತ್ರಂ ಸಮರ್ಪಯಾಮಿ॥\\
ಓಂ ಕಾಮಾರಯೇ ನಮಃ~। ಕಾಮಕಸ್ತೂರಿಕಾಪತ್ರಂ ಸಮರ್ಪಯಾಮಿ॥\\
ಓಂ ಗಿರಿಜಾಪತಯೇ ನಮಃ~। ಗಿರಿಕರ್ಣಿಕಾಪತ್ರಂ ಸಮರ್ಪಯಾಮಿ॥\\
ಓಂ ಮಹಾದೇವಾಯ ನಮಃ~। ಮಾಚೀಪತ್ರಂ ಸಮರ್ಪಯಾಮಿ॥\\
ಓಂ ಭಕ್ತಜನಪ್ರಿಯಾಯ ನಮಃ~। ಧಾತ್ರೀಪತ್ರಂ ಸಮರ್ಪಯಾಮಿ॥\\
ಓಂ ವಿಷ್ಣುಪ್ರಿಯಾಯ ನಮಃ~। ವಿಷ್ಣುಕ್ರಾಂತಿಪತ್ರಂ ಸಮರ್ಪಯಾಮಿ॥\\
ಓಂ ದಾಂತಾಯ ನಮಃ~। ದ್ರೋಣಪತ್ರಂ ಸಮರ್ಪಯಾಮಿ॥\\
ಓಂ ಧೂತಕಲ್ಮಶಾಯ ನಮಃ~। ಧತ್ತೂರಪತ್ರಂ ಸಮರ್ಪಯಾಮಿ॥\\
ಓಂ ಶಮಪ್ರಾಪ್ತಾಯ ನಮಃ~। ಶಮೀಪತ್ರಂ ಸಮರ್ಪಯಾಮಿ॥\\
ಓಂ ಸಾಂಬಶಿವಾಯ ನಮಃ~। ಸೇವಂತಿಕಾಪತ್ರಂ ಸಮರ್ಪಯಾಮಿ॥\\
ಓಂ ಚರ್ಮವಾಸಸೇ ನಮಃ~। ಚಂಪಕಪತ್ರಂ ಸಮರ್ಪಯಾಮಿ॥\\
ಓಂ ಕರುಣಾಕರಾಯ ನಮಃ~। ಕರವೀರಪತ್ರಂ ಸಮರ್ಪಯಾಮಿ॥\\ 
ಓಂ ಅಪರಾಜಿತಾಯ ನಮಃ~। ಅಶೋಕಪತ್ರಂ ಸಮರ್ಪಯಾಮಿ॥\\
ಓಂ ಪುಣ್ಯಮೂರ್ತಯೇ ನಮಃ~। ಪುನ್ನಾಗಪತ್ರಂ ಸಮರ್ಪಯಾಮಿ॥\\
ಓಂ ದಕ್ಷಿಣಾಮೂರ್ತಯೇ ನಮಃ~। ಸರ್ವಪತ್ರಂ ಸಮರ್ಪಯಾಮಿ॥
\section{ಪುಷ್ಪಪೂಜಾ॥}
\addcontentsline{toc}{section}{ಪುಷ್ಪಪೂಜಾ॥}
ಓಂ ಆತ್ಮಸಂಭವಾಯ ನಮಃ~। ಅರ್ಕಪುಷ್ಪಂ ಸಮರ್ಪಯಾಮಿ॥\\
ಓಂ ದ್ರವಿಣಪ್ರದಾಯ ನಮಃ~। ದ್ರೋಣಪುಷ್ಪಂ ಸಮರ್ಪಯಾಮಿ॥\\
ಓಂ ಧರ್ಮಸೇತವೇ ನಮಃ~। ಧತ್ತೂರಪುಷ್ಪಂ ಸಮರ್ಪಯಾಮಿ॥\\
ಓಂ ಬೃಹದ್ಗರ್ಭಾಯ ನಮಃ~। ಬೃಹತೀಪುಷ್ಪಂ ಸಮರ್ಪಯಾಮಿ॥\\
ಓಂ ಉಮಾಪತಯೇ ನಮಃ~। ಬಕುಲಪುಷ್ಪಂ ಸಮರ್ಪಯಾಮಿ॥\\
ಓಂ ಜಗತ್ಪತಯೇ ನಮಃ~। ಜಾತೀಪುಷ್ಪಂ ಸಮರ್ಪಯಾಮಿ॥\\
ಓಂ ಕಾಲಾಂತಕಾಯ ನಮಃ~। ಕರವೀರಪುಷ್ಪಂ ಸಮರ್ಪಯಾಮಿ॥\\
ಓಂ ನೀಲಕಂಠಾಯ ನಮಃ~। ನೀಲೋತ್ಪಲಪುಷ್ಪಂ ಸಮರ್ಪಯಾಮಿ॥\\
ಓಂ ಪರಮೇಶ್ವರಾಯ ನಮಃ~। ಪುನ್ನಾಗಪುಷ್ಪಂ ಸಮರ್ಪಯಾಮಿ॥\\
ಓಂ ವೃಷಧ್ವಜಾಯ ನಮಃ~। ವೈಜಯಂತಿಕಾಪುಷ್ಪಂ ಸಮರ್ಪಯಾಮಿ॥\\
ಓಂ ಗಿರಿಧನ್ವನೇ ನಮಃ~। ಗಿರಿಕರ್ಣಿಕಾಪುಷ್ಪಂ ಸಮರ್ಪಯಾಮಿ॥\\
ಓಂ ಚಂದ್ರಚೂಡಾಯ ನಮಃ~। ಚಂಪಕಪುಷ್ಪಂ ಸಮರ್ಪಯಾಮಿ॥\\
ಓಂ ಸೋಮೇಶಾಯ ನಮಃ~। ಸೇವಂತಿಕಾಪುಷ್ಪಂ ಸಮರ್ಪಯಾಮಿ॥\\
ಓಂ ಮಹೇಶ್ವರಾಯ ನಮಃ~। ಮಲ್ಲಿಕಾಪುಷ್ಪಂ ಸಮರ್ಪಯಾಮಿ॥\\
ಓಂ ಜಟಾಧರಾಯ ನಮಃ~। ಜಪಾಪುಷ್ಪಂ ಸಮರ್ಪಯಾಮಿ॥\\
ಓಂ ದಕ್ಷಿಣಾಮೂರ್ತಯೇ ನಮಃ~। ಸರ್ವಾಣಿ ಪುಷ್ಪಾಣಿ ಸಮರ್ಪಯಾಮಿ॥
\section{ಗಣೇಶ ಅಂಗಪೂಜಾ }
\addcontentsline{toc}{section}{ಗಣೇಶ ಅಂಗಪೂಜಾ }
ಓಂ ಗಣೇಶಾಯ ನಮಃ~। ಪಾದೌ ಪೂಜಯಾಮಿ ॥\\
ಓಂ ಗೌರೀಪುತ್ರಾಯ ನಮಃ~। ಗುಲ್ಫೌ ಪೂಜಯಾಮಿ ॥\\
ಓಂ ಅಘನಾಶನಾಯ ನಮಃ~। ಜಾನುನೀ ಪೂಜಯಾಮಿ ॥\\
ಓಂ ವಿಘ್ನರಾಜಾಯ ನಮಃ~। ಜಂಘೇ ಪೂಜಯಾಮಿ ॥\\
ಓಂ ಆಖುವಾಹನಾಯ ನಮಃ~। ಊರೂ ಪೂಜಯಾಮಿ ॥\\
ಓಂ ಹೇರಂಬಾಯ ನಮಃ~। ಕಟಿಂ ಪೂಜಯಾಮಿ ॥\\
ಓಂ ಲಂಬೋದರಾಯ ನಮಃ~। ಉದರಂ ಪೂಜಯಾಮಿ ॥\\
ಓಂ ವಕ್ರತುಂಡಾಯ ನಮಃ~। ನಾಭಿಂ ಪೂಜಯಾಮಿ ॥\\
ಓಂ ಗಣನಾಥಾಯ ನಮಃ~। ಹೃದಯಂ ಪೂಜಯಾಮಿ ॥\\
ಓಂ ಸ್ಥೂಲಕಂಠಾಯ ನಮಃ~। ಕಂಠಂ ಪೂಜಯಾಮಿ ॥\\
ಓಂ ಸ್ಕಂದಾಗ್ರಜಾಯ ನಮಃ~। ಸ್ಕಂಧೌ ಪೂಜಯಾಮಿ ॥\\
ಓಂ ಪರಶುಹಸ್ತಾಯ ನಮಃ~। ಹಸ್ತಾನ್ ಪೂಜಯಾಮಿ ॥\\
ಓಂ ಗಜವಕ್ತ್ರಾಯ ನಮಃ~। ವಕ್ತ್ರಂ ಪೂಜಯಾಮಿ ॥\\
ಓಂ ವಿಘ್ನಹರ್ತ್ರೇ ನಮಃ~। ನೇತ್ರೇ ಪೂಜಯಾಮಿ ॥\\
ಓಂ ಶೂರ್ಪಕರ್ಣಾಯ ನಮಃ~। ಕರ್ಣೌ ಪೂಜಯಾಮಿ ॥\\
ಓಂ ಏಕದಂತಾಯ ನಮಃ~। ದಂತಂ ಪೂಜಯಾಮಿ ॥\\
ಓಂ ಫಾಲಚಂದ್ರಾಯ ನಮಃ~। ಲಲಾಟಂ ಪೂಜಯಾಮಿ ॥\\
ಓಂ ಸರ್ವೇಶ್ವರಾಯ ನಮಃ~। ಶಿರಃ ಪೂಜಯಾಮಿ ॥\\
ಅಂಗಪೂಜಾಂ ಸಮರ್ಪಯಾಮಿ ॥
\section{ಪತ್ರ ಪೂಜಾ }
\addcontentsline{toc}{section}{ಪತ್ರ ಪೂಜಾ }
ಓಂ ಸುಮುಖಾಯ ನಮಃ~। ಮಾಚೀಪತ್ರಂ ಸಮರ್ಪಯಾಮಿ ॥\\
ಓಂ ಗಣಾಧಿಪಾಯ ನಮಃ~। ಭೃಂಗರಾಜಪತ್ರಂ ಸಮರ್ಪಯಾಮಿ ॥\\
ಓಂ ಉಮಾಪುತ್ರಾಯ ನಮಃ~। ಬಿಲ್ವಪತ್ರಂ ಸಮರ್ಪಯಾಮಿ ॥\\
ಓಂ ಗಜಾನನಾಯ ನಮಃ~। ಶ್ವೇತದೂರ್ವಾಯುಗ್ಮಂ ಸಮರ್ಪಯಾಮಿ ॥\\
ಓಂ ಲಂಬೋದರಾಯ ನಮಃ~। ಬದರೀಪತ್ರಂ ಸಮರ್ಪಯಾಮಿ ॥\\
ಓಂ ಹರಸೂನವೇ ನಮಃ~। ಧತ್ತೂರಪತ್ರಂ ಸಮರ್ಪಯಾಮಿ ॥\\
ಓಂ ಗಜಕರ್ಣಕಾಯ ನಮಃ~। ಕೇತಕೀಪತ್ರಂ ಸಮರ್ಪಯಾಮಿ ॥\\
ಓಂ ಗುಹಾಗ್ರಜಾಯ ನಮಃ~। ಅಪಾಮಾರ್ಗಪತ್ರಂ ಸಮರ್ಪಯಾಮಿ ॥\\
ಓಂ ಏಕದಂತಾಯ ನಮಃ~। ಚೂತಪತ್ರಂ ಸಮರ್ಪಯಾಮಿ ॥\\
ಓಂ ಇಭವಕ್ತ್ರಾಯ ನಮಃ~। ಶಮೀಪತ್ರಂ ಸಮರ್ಪಯಾಮಿ ॥\\
ಓಂ ವಿನಾಯಕಾಯ ನಮಃ~। ಕರವೀರಪತ್ರಂ ಸಮರ್ಪಯಾಮಿ ॥\\
ಓಂ ಕಪಿಲಾಯ ನಮಃ~। ಅರ್ಕಪತ್ರಂ ಸಮರ್ಪಯಾಮಿ ॥\\
ಓಂ ವರಪ್ರದಾಯ ನಮಃ~। ಅರ್ಜುನಪತ್ರಂ ಸಮರ್ಪಯಾಮಿ ॥\\
ಓಂ ವಕ್ರತುಂಡಾಯ ನಮಃ~। ಆಮಲಕಪತ್ರಂ ಸಮರ್ಪಯಾಮಿ ॥\\
ಓಂ ವಿಘ್ನರಾಜಾಯ ನಮಃ~। ವಿಷ್ಣುಕ್ರಾಂತಿಪತ್ರಂ ಸಮರ್ಪಯಾಮಿ ॥\\
ಓಂ ವಟವೇ ನಮಃ~। ದಾಡಿಮೀಪತ್ರಂ ಸಮರ್ಪಯಾಮಿ ॥\\
ಓಂ ಸುರಸೇವಿತಾಯ ನಮಃ~। ದೇವದಾರುಪತ್ರಂ ಸಮರ್ಪಯಾಮಿ ॥\\
ಓಂ ಫಾಲಚಂದ್ರಾಯ ನಮಃ~। ಮರುಗಪತ್ರಂ ಸಮರ್ಪಯಾಮಿ ॥\\
ಓಂ ಹೇರಂಬಾಯ ನಮಃ~। ಸಿಂದೂರಪತ್ರಂ ಸಮರ್ಪಯಾಮಿ ॥\\
ಓಂ ಶೂರ್ಪಕರ್ಣಾಯ ನಮಃ~। ಜಾಜೀಪತ್ರಂ ಸಮರ್ಪಯಾಮಿ ॥\\
ಓಂ ಸರ್ವೇಶ್ವರಾಯ ನಮಃ~। ಅಗಸ್ತ್ಯಪತ್ರಂ ಸಮರ್ಪಯಾಮಿ ॥\\
ಪತ್ರಪೂಜಾಂ ಸಮರ್ಪಯಾಮಿ ॥
\section{ಪುಷ್ಪಪೂಜಾ }
\addcontentsline{toc}{section}{ಪುಷ್ಪಪೂಜಾ }
ಓಂ ಸುಮುಖಾಯ ನಮಃ~। ದಾಡಿಮೀ ಪುಷ್ಪಂ ಸಮರ್ಪಯಾಮಿ ॥\\
ಓಂ ಏಕದಂತಾಯ ನಮಃ~। ಯೂಥಿಕಾ ಪುಷ್ಪಂ ಸಮರ್ಪಯಾಮಿ ॥\\
ಓಂ ಕಪಿಲಾಯ ನಮಃ~। ಮಲ್ಲಿಕಾ ಪುಷ್ಪಂ ಸಮರ್ಪಯಾಮಿ ॥\\
ಓಂ ಗಜಕರ್ಣಕಾಯ ನಮಃ~। ಚಂಪಕ ಪುಷ್ಪಂ ಸಮರ್ಪಯಾಮಿ ॥\\
ಓಂ ಲಂಬೋದರಾಯ ನಮಃ~। ಕಲ್ಹಾರ ಪುಷ್ಪಂ ಸಮರ್ಪಯಾಮಿ ॥\\
ಓಂ ವಿಕಟಾಯ ನಮಃ~। ಕೇತಕೀ ಪುಷ್ಪಂ ಸಮರ್ಪಯಾಮಿ ॥\\
ಓಂ ವಿಘ್ನನಾಶಿನೇ ನಮಃ~। ಬಕುಲ ಪುಷ್ಪಂ ಸಮರ್ಪಯಾಮಿ ॥\\
ಓಂ ಧೂಮ್ರಕೇತವೇ ನಮಃ~। ಶತಪತ್ರ ಪುಷ್ಪಂ ಸಮರ್ಪಯಾಮಿ ॥\\
ಓಂ ಗಣಾಧ್ಯಕ್ಷಾಯ ನಮಃ~। ಚೂತ ಪುಷ್ಪಂ ಸಮರ್ಪಯಾಮಿ ॥\\
ಓಂ ಫಾಲಚಂದ್ರಾಯ ನಮಃ~। ನಂದ್ಯಾವರ್ತ ಪುಷ್ಪಂ ಸಮರ್ಪಯಾಮಿ ॥\\
ಓಂ ಗಜಾನನಾಯ ನಮಃ~। ಕುಮುದ ಪುಷ್ಪಂ ಸಮರ್ಪಯಾಮಿ ॥\\
ಓಂ ಗುಹಾಗ್ರಜಾಯ ನಮಃ~। ಅಶೋಕ ಪುಷ್ಪಂ ಸಮರ್ಪಯಾಮಿ ॥\\
ಓಂ ಭಕ್ತಪ್ರಿಯಾಯ ನಮಃ~। ಜಾತೀ ಪುಷ್ಪಂ ಸಮರ್ಪಯಾಮಿ ॥\\
ಓಂ ಗೌರೀಪುತ್ರಾಯ ನಮಃ~। ಆಮಲಕ ಪುಷ್ಪಂ ಸಮರ್ಪಯಾಮಿ ॥\\
ಓಂ ಗಣಂಜಯಾಯ ನಮಃ~। ಅರ್ಕ ಪುಷ್ಪಂ ಸಮರ್ಪಯಾಮಿ ॥\\
ಓಂ ಹೇರಂಬಾಯ ನಮಃ~। ಬದರೀ ಪುಷ್ಪಂ ಸಮರ್ಪಯಾಮಿ ॥\\
ಓಂ ದ್ವೈಮಾತುರಾಯ ನಮಃ~। ಪಾರಿಜಾತ ಪುಷ್ಪಂ ಸಮರ್ಪಯಾಮಿ ॥\\
ಓಂ ಕುಮಾರಗುರವೇ ನಮಃ~। ಜಂಬೀರ ಪುಷ್ಪಂ ಸಮರ್ಪಯಾಮಿ ॥\\
ಓಂ ಅವ್ಯಕ್ತಾಯ ನಮಃ~। ಕಮಲ ಪುಷ್ಪಂ ಸಮರ್ಪಯಾಮಿ ॥\\
ಓಂ ಬೀಜಾಪೂರಾಯ ನಮಃ~। ನಾಗಲಿಂಗ ಪುಷ್ಪಂ ಸಮರ್ಪಯಾಮಿ ॥\\
ಓಂ ಸಿದ್ಧಿವಿನಾಯಕಾಯ ನಮಃ~। ಕರ್ಣಿಕಾರ ಪುಷ್ಪಂ ಸಮರ್ಪಯಾಮಿ ॥\\
ಪುಷ್ಪಪೂಜಾಂ ಸಮರ್ಪಯಾಮಿ ॥
\section{ನಾಮಪೂಜಾ\\ಶ್ರೀದಕ್ಷಿಣಾಮೂರ್ತಿ ಅಷ್ಟೋತ್ತರ ಶತನಾಮಸ್ತೋತ್ರಂ }
\addcontentsline{toc}{section}{ಶ್ರೀದಕ್ಷಿಣಾಮೂರ್ತಿ ಅಷ್ಟೋತ್ತರ ಶತನಾಮಸ್ತೋತ್ರಂ }
\dhyana{ವಟವೃಕ್ಷ ತಟಾಸೀನಂ ಯೋಗಿ ಧ್ಯೇಯಾಂಘ್ರಿ ಪಂಕಜಂ।\\
ಶರಚ್ಚಂದ್ರ ನಿಭಂ ಪೂಜ್ಯಂ ಜಟಾಮುಕುಟ ಮಂಡಿತಂ ॥

ಗಂಗಾಧರಂ ಲಲಾಟಾಕ್ಷಂ ವ್ಯಾಘ್ರ ಚರ್ಮಾಂಬರಾವೃತಂ।\\
ನಾಗಭೂಷಂ ಪರಂಬ್ರಹ್ಮ ದ್ವಿಜರಾಜಾವತಂಸಕಂ ॥

ಅಕ್ಷಮಾಲಾ ಜ್ಞಾನಮುದ್ರಾ ವೀಣಾ ಪುಸ್ತಕ ಶೋಭಿತಂ।\\
ಶುಕಾದಿ ವೃದ್ಧ ಶಿಷ್ಯಾಢ್ಯಂ ವೇದ ವೇದಾಂತಗೋಚರಂ॥\\
ಯುವಾನಂ ಮನ್ಮಥಾರಾತಿಂ ದಕ್ಷಿಣಾಮೂರ್ತಿಮಾಶ್ರಯೇ॥}

ಓಂ ವಿದ್ಯಾರೂಪೀ ಮಹಾಯೋಗೀ ಶುದ್ಧ ಜ್ಞಾನೀ ಪಿನಾಕಧೃತ್~।\\
ರತ್ನಾಲಂಕೃತ ಸರ್ವಾಂಗೀ ರತ್ನಮೌಳಿರ್ಜಟಾಧರಃ ॥೧॥

ಗಂಗಾಧಾರ್ಯಚಲಾವಾಸೀ ಮಹಾಜ್ಞಾನೀ ಸಮಾಧಿಕೃತ್।\\
ಅಪ್ರಮೇಯೋ ಯೋಗನಿಧಿಸ್ತಾರಕೋ ಭಕ್ತವತ್ಸಲಃ॥೨॥

ಬ್ರಹ್ಮರೂಪೀ ಜಗದ್ವ್ಯಾಪೀ ವಿಷ್ಣುಮೂರ್ತಿಃ ಪುರಾತನಃ~।\\
ಉಕ್ಷವಾಹಶ್ಚರ್ಮವಾಸಾಃ ಪೀತಾಂಬರ ವಿಭೂಷಣಃ॥೩॥

ಮೋಕ್ಷದಾಯೀ ಮೋಕ್ಷ ನಿಧಿಶ್ಚಾಂಧಕಾರಿರ್ಜಗತ್ಪತಿಃ।\\
ವಿದ್ಯಾಧಾರೀ ಶುಕ್ಲ ತನುಃ ವಿದ್ಯಾದಾಯೀ ಗಣಾಧಿಪಃ॥೪॥

ಪ್ರೌಢಾಪಸ್ಮೃತಿ ಸಂಹರ್ತಾ ಶಶಿಮೌಳಿರ್ಮಹಾಸ್ವನಃ~।\\
ಸಾಮ ಪ್ರಿಯೋಽವ್ಯಯಃ ಸಾಧುಃ ಸರ್ವ ವೇದೈರಲಂಕೃತಃ ॥೫॥

ಹಸ್ತೇ ವಹ್ನಿಧರಃ ಶ್ರೀಮಾನ್ ಮೃಗಧಾರೀ ವಶಂಕರಃ~।\\
ಯಜ್ಞನಾಥಃ ಕ್ರತುಧ್ವಂಸೀ ಯಜ್ಞಭೋಕ್ತಾ ಯಮಾಂತಕಃ॥೬॥

ಭಕ್ತಾನುಗ್ರಹ ಮೂರ್ತಿಶ್ಚ ಭಕ್ತಸೇವ್ಯೋ ವೃಷಧ್ವಜಃ~।\\
ಭಸ್ಮೋದ್ಧೂಲಿತ ಸರ್ವಾಂಗಃ ಚಾಕ್ಷಮಾಲಾಧರೋ ಮಹಾನ್ ॥೭॥

ತ್ರಯೀಮೂರ್ತಿಃ ಪರಂಬ್ರಹ್ಮ ನಾಗರಾಜೈರಲಂಕೃತಃ~।\\
ಶಾಂತರೂಪೋ ಮಹಾಜ್ಞಾನೀ ಸರ್ವ ಲೋಕ ವಿಭೂಷಣಃ ॥೮॥

ಅರ್ಧನಾರೀಶ್ವರೋ ದೇವೋ ಮುನಿಸೇವ್ಯಸ್ಸುರೋತ್ತಮಃ~।\\
ವ್ಯಾಖ್ಯಾನದೇವೋ ಭಗವಾನ್ ರವಿ ಚಂದ್ರಾಗ್ನಿ ಲೋಚನಃ ॥೯॥

ಜಗದ್ಗುರುರ್ಮಹಾದೇವೋ ಮಹಾನಂದ ಪರಾಯಣಃ~।\\
ಜಟಾಧಾರೀ ಮಹಾಯೋಗೀ ಜ್ಞಾನಮಾಲೈರಲಂಕೃತಃ ॥೧೦॥

ವ್ಯೋಮಗಂಗಾ ಜಲ ಸ್ಥಾನಃ ವಿಶುದ್ಧೋ ಯತಿರೂರ್ಜಿತಃ~।\\
ತತ್ತ್ವಮೂರ್ತಿರ್ಮಹಾಯೋಗೀ ಮಹಾಸಾರಸ್ವತಪ್ರದಃ ॥೧೧॥

ವ್ಯೋಮಮೂರ್ತಿಶ್ಚ ಭಕ್ತಾನಾಂ ಇಷ್ಟಃ ಕಾಮಫಲಪ್ರದಃ~।\\
ಪರಮೂರ್ತಿಃ ಚಿತ್ಸ್ವರೂಪೀ ತೇಜೋಮೂರ್ತಿರನಾಮಯಃ ॥೧೨॥

ವೇದವೇದಾಂಗ ತತ್ತ್ವಜ್ಞಃ ಚತುಃಷಷ್ಟಿ ಕಲಾನಿಧಿಃ~।\\
ಭವರೋಗ ಭಯಧ್ವಂಸೀ ಭಕ್ತಾನಾಮಭಯಪ್ರದಃ ॥೧೩॥

ನೀಲಗ್ರೀವೋ ಲಲಾಟಾಕ್ಷೋ ಗಜ ಚರ್ಮಾಗತಿಪ್ರದಃ~।\\
ಅರಾಗೀ ಕಾಮದಶ್ಚಾಥ ತಪಸ್ವೀ ವಿಷ್ಣುವಲ್ಲಭಃ ॥೧೪॥

ಬ್ರಹ್ಮಚಾರೀ ಚ ಸನ್ಯಾಸೀ ಗೃಹಸ್ಥಾಶ್ರಮ ಕಾರಣಃ~।\\
ದಾಂತಃ ಶಮವತಾಂ ಶ್ರೇಷ್ಠೋ ಸತ್ಯರೂಪೋ ದಯಾಪರಃ ॥೧೫॥

ಯೋಗಪಟ್ಟಾಭಿರಾಮಶ್ಚ ವೀಣಾಧಾರೀ ವಿಚೇತನಃ~।\\
ಮತಿಪ್ರಜ್ಞಾ ಸುಧಾಧಾರೀ ಮುದ್ರಾಪುಸ್ತಕ ಧಾರಣಃ ॥೧೬॥

ವೇತಾಲಾದಿ ಪಿಶಾಚೌಘ ರಾಕ್ಷಸೌಘ ವಿನಾಶನಃ~।\\
ರಾಜ ಯಕ್ಷ್ಮಾದಿ ರೋಗಾಣಾಂ ವಿನಿಹಂತಾ ಸುರೇಶ್ವರಃ ॥೧೭॥
\authorline{॥ಇತಿ ಶ್ರೀ ದಕ್ಷಿಣಾಮೂರ್ತಿ ಅಷ್ಟೋತ್ತರ ಶತನಾಮ ಸ್ತೋತ್ರಂ ಸಂಪೂರ್ಣಂ ॥}
\section{ಶ್ರೀವಿಘ್ನೇಶ್ವರಾಷ್ಟೋತ್ತರ ಶತನಾಮಸ್ತೋತ್ರಂ}
\addcontentsline{toc}{section}{ಶ್ರೀವಿಘ್ನೇಶ್ವರಾಷ್ಟೋತ್ತರ ಶತನಾಮಸ್ತೋತ್ರಂ }
\dhyana{ಗಜವದನಮಚಿಂತ್ಯಂ ತೀಕ್ಷ್ಣದಂಷ್ಟ್ರಂ ತ್ರಿಣೇತ್ರಂ\\
ಬೃಹದುದರಮಶೇಷಂ ಭೂತಿರೂಪಂ ಪುರಾಣಮ್~।\\
ಅಮರವರಸುಪೂಜ್ಯಂ ರಕ್ತವರ್ಣಂ ಸುರೇಶಂ\\
 ಪಶುಪತಿಸುತಮೀಶಂ ವಿಘ್ನರಾಜಂ ನಮಾಮಿ ॥}
  
ವಿನಾಯಕೋ ವಿಘ್ನರಾಜೋ ಗೌರೀಪುತ್ರೋ ಗಣೇಶ್ವರಃ~।\\
ಸ್ಕಂದಾಗ್ರಜೋಽವ್ಯಯಃ ಪೂತೋ ದಕ್ಷೋಽಧ್ಯಕ್ಷೋ ದ್ವಿಜಪ್ರಿಯಃ ॥೧॥

ಅಗ್ನಿಗರ್ಭಚ್ಛಿದಿಂದ್ರಶ್ರೀಪ್ರದೋ ವಾಣೀಬಲಪ್ರದಃ~।\\
ಸರ್ವಸಿದ್ಧಿಪ್ರದಶ್ಶರ್ವತನಯಃ ಶರ್ವರೀಪ್ರಿಯಃ ॥೨॥

ಸರ್ವಾತ್ಮಕಃ ಸೃಷ್ಟಿಕರ್ತಾ ದೇವಾನೀಕಾರ್ಚಿತಶ್ಶಿವಃ~।\\
ಶುದ್ಧೋ ಬುದ್ಧಿಪ್ರಿಯಶ್ಶಾಂತೋ ಬ್ರಹ್ಮಚಾರೀ ಗಜಾನನಃ ॥೩॥

ದ್ವೈಮಾತ್ರೇಯೋ ಮುನಿಸ್ತುತ್ಯೋ ಭಕ್ತವಿಘ್ನವಿನಾಶನಃ~।\\
ಏಕದಂತಶ್ಚತುರ್ಬಾಹುಶ್ಚತುರಶ್ಶಕ್ತಿಸಂಯುತಃ ॥೪॥

ಲಂಬೋದರಶ್ಶೂರ್ಪಕರ್ಣೋ ಹರಿರ್ಬ್ರಹ್ಮ ವಿದುತ್ತಮಃ~।\\
ಕಾಲೋ ಗ್ರಹಪತಿಃ ಕಾಮೀ ಸೋಮಸೂರ್ಯಾಗ್ನಿಲೋಚನಃ ॥೫॥

ಪಾಶಾಂಕುಶಧರಶ್ಚಂಡೋ ಗುಣಾತೀತೋ ನಿರಂಜನಃ~।\\
ಅಕಲ್ಮಷಸ್ಸ್ವಯಂಸಿದ್ಧಸ್ಸಿದ್ಧಾರ್ಚಿತಪದಾಂಬುಜಃ ॥೬॥

ಬೀಜಪೂರಫಲಾಸಕ್ತೋ ವರದಶ್ಶಾಶ್ವತಃ ಕೃತೀ~।\\
ದ್ವಿಜಪ್ರಿಯೋ ವೀತಭಯೋ ಗದೀ ಚಕ್ರೀಕ್ಷುಚಾಪಧೃತ್ ॥೭॥

ಶ್ರೀದೋಽಜ ಉತ್ಪಲಕರಃ ಶ್ರೀಪತಿಃ ಸ್ತುತಿಹರ್ಷಿತಃ~।\\
ಕುಲಾದ್ರಿಭೇತ್ತಾ ಜಟಿಲಃ ಕಲಿಕಲ್ಮಷನಾಶನಃ ॥೮॥

ಚಂದ್ರಚೂಡಾಮಣಿಃ ಕಾಂತಃ ಪಾಪಹಾರೀ ಸಮಾಹಿತಃ~।\\
ಆಶ್ರಿತಶ್ಶ್ರೀಕರಸ್ಸೌಮ್ಯೋ ಭಕ್ತವಾಂಛಿತದಾಯಕಃ ॥೯॥

ಶಾಂತಃ ಕೈವಲ್ಯಸುಖದಸ್ಸಚ್ಚಿದಾನಂದವಿಗ್ರಹಃ~।\\
ಜ್ಞಾನೀ ದಯಾಯುತೋ ದಾಂತೋ ಬ್ರಹ್ಮ ದ್ವೇಷವಿವರ್ಜಿತಃ ॥೧೦॥

ಪ್ರಮತ್ತದೈತ್ಯಭಯದಃ ಶ್ರೀಕಂಠೋ ವಿಬುಧೇಶ್ವರಃ~।\\
ರಮಾರ್ಚಿತೋವಿಧಿರ್ನಾಗರಾಜಯಜ್ಞೋಪವೀತಕಃ ॥೧೧॥

ಸ್ಥೂಲಕಂಠಃ ಸ್ವಯಂಕರ್ತಾ ಸಾಮಘೋಷಪ್ರಿಯಃ ಪರಃ~।\\
ಸ್ಥೂಲತುಂಡೋಽಗ್ರಣೀರ್ಧೀರೋ ವಾಗೀಶಸ್ಸಿದ್ಧಿದಾಯಕಃ ॥೧೨॥

ದೂರ್ವಾಬಿಲ್ವಪ್ರಿಯೋಽವ್ಯಕ್ತಮೂರ್ತಿರದ್ಭುತಮೂರ್ತಿಮಾನ್~।\\
ಶೈಲೇಂದ್ರತನುಜೋತ್ಸಂಗಖೇಲನೋತ್ಸುಕಮಾನಸಃ ॥೧೩॥

ಸ್ವಲಾವಣ್ಯಸುಧಾಸಾರೋ ಜಿತಮನ್ಮಥವಿಗ್ರಹಃ~।\\
ಸಮಸ್ತಜಗದಾಧಾರೋ ಮಾಯೀ ಮೂಷಕವಾಹನಃ ॥೧೪॥

ಹೃಷ್ಟಸ್ತುಷ್ಟಃ ಪ್ರಸನ್ನಾತ್ಮಾ ಸರ್ವಸಿದ್ಧಿಪ್ರದಾಯಕಃ~।\\
ಅಷ್ಟೋತ್ತರಶತೇನೈವಂ ನಾಮ್ನಾಂ ವಿಘ್ನೇಶ್ವರಂ ವಿಭುಂ ॥೧೫॥

ತುಷ್ಟಾವ ಶಂಕರಃ ಪುತ್ರಂ ತ್ರಿಪುರಂ ಹಂತುಮುದ್ಯತಃ~।\\
ಯಃ ಪೂಜಯೇದನೇನೈವ ಭಕ್ತ್ಯಾ ಸಿದ್ಧಿವಿನಾಯಕಂ ॥೧೬॥

ದೂರ್ವಾದಲೈರ್ಬಿಲ್ವಪತ್ರೈಃ ಪುಷ್ಪೈರ್ವಾ ಚಂದನಾಕ್ಷತೈಃ~।\\
ಸರ್ವಾನ್ಕಾಮಾನವಾಪ್ನೋತಿ ಸರ್ವವಿಘ್ನೈಃ ಪ್ರಮುಚ್ಯತೇ ॥
\authorline{ಇತಿ ಶ್ರೀವಿಘ್ನೇಶ್ವರಾಷ್ಟೋತ್ತರ ಶತನಾಮಸ್ತೋತ್ರಂ}

ದಶಾಂಗಂ ಗುಗ್ಗುಲಂ ಧೂಪಂ ಸುಗಂಧಂ ಚ ಮನೋಹರಂ~।\\
ಕಪಿಲಾಘೃತಸಂಯುಕ್ತಂ ಗೃಹಾಣ ಧೂಪಮುತ್ತಮಮ್॥ಧೂಪಃ॥

ಸಾಜ್ಯಂ ತ್ರಿವರ್ತಿಸಂಯುಕ್ತಂ ವಹ್ನಿನಾ ಯೋಜಿತಂ ಮಯಾ~।\\
ಗೃಹಾಣ ಮಂಗಳಂ ದೀಪಂ ತ್ರೈಲೋಕ್ಯತಿಮಿರಾಪಹಂ ॥

ಭಕ್ತ್ಯಾ ದೀಪಂ ಪ್ರಯಚ್ಛಾಮಿ ದೇವಾಯ ಪರಮಾತ್ಮನೇ~।\\
ತ್ರಾಹಿ ಮಾಂ ನರಕಾತ್ ಘೋರಾತ್ ವಿಘ್ನರಾಜ ನಮೋಽಸ್ತು ತೇ ॥ದೀಪಃ॥

ನೈವೇದ್ಯಂ ಷಡ್ರಸೋಪೇತಂ ಪಂಚಭಕ್ಷ್ಯಸಮನ್ವಿತಮ್~।\\
ಚೂತಾದಿ ಫಲಸಂಯುಕ್ತಂ ಗೃಹ್ಯತಾಂ ಗಣನಾಯಕ ॥ನೈವೇದ್ಯಮ್॥

ಪೂಗೀಫಲಸಮಾಯುಕ್ತಂ ನಾಗವಲ್ಲೀ ದಲೈರ್ಯುತಂ~।\\
ಕರ್ಪೂರಚೂರ್ಣಸಂಯುಕ್ತಂ ತಾಂಬೂಲಂ ಪ್ರತಿಗೃಹ್ಯತಾಂ ॥ತಾಂಬೂಲಮ್॥

ಇದಂ ಫಲಂ ಮಯಾದೇವ ಸ್ಥಾಪಿತಂ ಪುರತಸ್ತವ~।\\
ತೇನ ಮೇ ಸುಫಲಾವಾಪ್ತಿಃ ಭವೇಜ್ಜನ್ಮನಿ ಜನ್ಮನಿ ॥ಪೂರ್ಣಫಲಮ್॥

ಹಿರಣ್ಯಗರ್ಭಗರ್ಭಸ್ಥಂ ಹೇಮಬೀಜಂ ವಿಭಾವಸೋಃ~।\\
ಅನಂತಪುಣ್ಯಫಲದಂ ಅತಃ ಶಾಂತಿಂ ಪ್ರಯಚ್ಛ ಮೇ ॥ಸುವರ್ಣಪುಷ್ಪದಕ್ಷಿಣಾ॥

ಘೃತವರ್ತಿ ಸಮಾಯುಕ್ತಂ ಕರ್ಪೂರಶಕಲೈಸ್ತಥಾ~।\\
ನೀರಾಜನಂ ಮಯಾ ದತ್ತಂ ಗೃಹಾಣ ದ್ವಿರದಾನನ ॥\\
\as{ಭಕ್ತಾನುಕಂಪಿನಂ ದೇವಂ ಜಗತ್ಕಾರಣಮಚ್ಯುತಂ। ಆವಿರ್ಭೂತಂ ಚ ಸೃಷ್ಟ್ಯಾದೌ ಪ್ರಕೃತೇಃ ಪುರುಷಾತ್ಪರಂ ॥ ಏವಂ ಧ್ಯಾಯತಿ ಯೋ ನಿತ್ಯಂ ಸ ಯೋಗೀ \\ಯೋಗಿನಾಂ ವರಃ ॥}ನೀರಾಜನಮ್॥

ಗಣಾಧಿಪನಮಸ್ತೇಽಸ್ತು ಉಮಾಪುತ್ರಾಘನಾಶನ~।\\
ವಿನಾಯಕೇಶತನಯ ಸರ್ವಸಿದ್ಧಿಪ್ರದಾಯಕ ॥

ಏಕದಂತೇಭವದನ ತಥಾ ಮೂಷಕವಾಹನ~।\\
ಕುಮಾರಗುರವೇ ತುಭ್ಯಂ ಅರ್ಪಯಾಮಿ ಸುಮಾಂಜಲಿಂ ॥ಮಂತ್ರಪುಷ್ಪಂ॥

ಪ್ರದಕ್ಷಿಣಂ ಕರಿಷ್ಯಾಮಿ ಸತತಂ ಮೋದಕಪ್ರಿಯ~।\\
ನಮಸ್ತೇ ವಿಘ್ನರಾಜಾಯ ನಮಸ್ತೇ ವಿಘ್ನನಾಶನ ॥ಪ್ರದಕ್ಷಿಣಂ॥

ನಮಸ್ತುಭ್ಯಂ ಗಣೇಶಾಯ ನಮಸ್ತೇ ವಿಘ್ನನಾಶನ~।\\
ಈಪ್ಸಿತಂ ಮೇ ವರಂ ದೇಹಿ ಪರತ್ರ ಚ ಪರಾಂ ಗತಿಂ ॥

\as{ನಮೋ ವ್ರಾತಪತಯೇ ನಮೋ ಗಣಪತಯೇ ನಮಃ ಪ್ರಮಥಪತಯೇ ನಮಸ್ತೇಽಸ್ತು ಲಂಬೋದರಾಯೈಕದಂತಾಯ ವಿಘ್ನನಾಶಿನೇ ಶಿವಸುತಾಯ ವರದಮೂರ್ತಯೇ ನಮಃ ॥}ನಮಸ್ಕಾರಾಃ ॥

ಅರ್ಘ್ಯಂ ಗೃಹಾಣ ಹೇರಂಬ ಸರ್ವಭದ್ರಪ್ರದಾಯಕ~।\\
ಗಂಧಪುಷ್ಪಾಕ್ಷತೈರ್ಯುಕ್ತಂ ಪಾತ್ರಸ್ಥಂ ಪಾಪನಾಶನ ॥ಪ್ರಸನ್ನಾರ್ಘ್ಯಮ್॥

ನಾರೋಗ್ಯಮದ್ಯ ಕಲಯೇ ನ ಚ ಭೋಗ್ಯಜಾತಂ\\
ಭಾಗ್ಯಂ ನ ವಾ ವಿಷಯಲಂಪಟತಾನಿದಾನಂ~।\\
ವೈರಾಗ್ಯಮೇವ ಸುತರಾಮುಪಲಂಭಯನ್ ಮಾಂ\\
ಸ್ವಾಮಿನ್ ಸಮುದ್ಧರ ಭವಾಂಬುನಿಧೇರ್ದುರಂತಾತ್ ॥ಪ್ರಾರ್ಥನಾ॥
\newpage
\section{ಮಂದಿರಪೂಜಾ}
\addcontentsline{toc}{section}{ಮಂದಿರಪೂಜಾ}
ಓಂ ಐಂಹ್ರೀಂಶ್ರೀಂ ಅಮೃತಾಂಭೋನಿಧಯೇ ನಮಃ\\
೪ ರತ್ನದ್ವೀಪಾಯ ನಮಃ\\
೪ ನಾನಾವೃಕ್ಷಮಹೋದ್ಯಾನಾಯ ನಮಃ\\
೪ ಕಲ್ಪವೃಕ್ಷವಾಟಿಕಾಯೈ ನಮಃ\\
೪ ಸಂತಾನವಾಟಿಕಾಯೈ ನಮಃ\\
೪ ಹರಿಚಂದನವಾಟಿಕಾಯೈ ನಮಃ\\
೪ ಮಂದಾರವಾಟಿಕಾಯೈ ನಮಃ\\
೪ ಪಾರಿಜಾತವಾಟಿಕಾಯೈ ನಮಃ\\
೪ ಕದಂಬವಾಟಿಕಾಯೈ ನಮಃ\\
೪ ಪುಷ್ಯರಾಗರತ್ನಪ್ರಾಕಾರಾಯ ನಮಃ\\
೪ ಪದ್ಮರಾಗರತ್ನಪ್ರಾಕಾರಾಯ ನಮಃ\\
೪ ಗೋಮೇಧಕರತ್ನಪ್ರಾಕಾರಾಯ ನಮಃ\\
೪ ವಜ್ರರತ್ನಪ್ರಾಕಾರಾಯ ನಮಃ\\
೪ ವೈಡೂರ್ಯರತ್ನಪ್ರಾಕಾರಾಯ ನಮಃ\\
೪ ಇಂದ್ರನೀಲರತ್ನಪ್ರಾಕಾರಾಯ ನಮಃ\\
೪ ಮುಕ್ತಾರತ್ನಪ್ರಾಕಾರಾಯ ನಮಃ\\
೪ ಮರಕತರತ್ನಪ್ರಾಕಾರಾಯ ನಮಃ\\
೪ ವಿದ್ರುಮರತ್ನಪ್ರಾಕಾರಾಯ ನಮಃ\\
೪ ಮಾಣಿಕ್ಯಮಂಡಪಾಯ ನಮಃ\\
೪ ಸಹಸ್ರಸ್ತಂಭಮಂಡಪಾಯ ನಮಃ\\
೪ ಅಮೃತವಾಪಿಕಾಯೈ ನಮಃ\\
೪ ಆನಂದವಾಪಿಕಾಯೈ ನಮಃ\\
೪ ವಿಮರ್ಶವಾಪಿಕಾಯೈ ನಮಃ\\
೪ ಬಾಲಾತಪೋದ್ಗಾರಕಕ್ಷಾಯ ನಮಃ\\
೪ ಚಂದ್ರಿಕೋದ್ಗಾರಕಕ್ಷಾಯ ನಮಃ\\
೪ ಮಹಾಶೃಂಗಾರಪರಿಘಾಯೈ ನಮಃ\\
೪ ಮಹಾಪದ್ಮಾಟವ್ಯೈ ನಮಃ\\
೪ ಚಿಂತಾಮಣಿಮಯಗೃಹರಾಜಾಯ ನಮಃ\\
೪ ಪೂರ್ವಾಮ್ನಾಯಮಯಪೂರ್ವದ್ವಾರಾಯ ನಮಃ\\
೪ ದಕ್ಷಿಣಾಮ್ನಾಯಮಯದಕ್ಷಿಣದ್ವಾರಾಯ ನಮಃ\\
೪ ಪಶ್ಚಿಮಾಮ್ನಾಯಮಯಪಶ್ಚಿಮದ್ವಾರಾಯ ನಮಃ\\
೪ ಉತ್ತರಾಮ್ನಾಯಮಯೋತ್ತರದ್ವಾರಾಯ ನಮಃ\\
೪ ರತ್ನಪ್ರದೀಪವಲಯಾಯ ನಮಃ\\
೪ ಮಣಿಮಯಮಹಾಸಿಂಹಾಸನಾಯ ನಮಃ\\
೪ ಬ್ರಹ್ಮಮಯೈಕಮಂಚಪಾದಾಯ ನಮಃ(ಆಗ್ನೇಯ್ಯಾಂ)\\
೪ ವಿಷ್ಣುಮಯೈಕಮಂಚಪಾದಾಯ ನಮಃ(ನೈರೃತ್ಯಾಂ)\\
೪ ರುದ್ರಮಯೈಕಮಂಚಪಾದಾಯ ನಮಃ(ವಾಯವ್ಯಾಂ)\\
೪ ಈಶ್ವರಮಯೈಕಮಂಚಪಾದಾಯ ನಮಃ(ಐಶಾನ್ಯಾಂ)\\
೪ ಸದಾಶಿವಮಯೈಕಮಂಚಫಲಕಾಯ ನಮಃ\\
೪ ಹಂಸತೂಲಿಕಾತಲ್ಪಾಯ ನಮಃ\\
೪ ಹಂಸತೂಲಿಕಾತಲ್ಪಮಹೋಪಧಾನಾಯ ನಮಃ\\
೪ ಕೌಸುಂಭಾಸ್ತರಣಾಯ ನಮಃ\\
೪ ಮಹಾವಿತಾನಕಾಯ ನಮಃ\\
೪ ಮಹಾಮಾಯಾಜವನಿಕಾಯೈ ನಮಃ

೪ ದೀಪದೇವಿ ಮಹಾದೇವಿ ಶುಭಂ ಭವತು ಮೇ ಸದಾ~।\\
ಯಾವತ್ ಪೂಜಾಸಮಾಪ್ತಿಃ ಸ್ಯಾತ್ ತಾವತ್ ಪ್ರಜ್ವಲ ಸುಸ್ಥಿರಾ ॥

\as{೪ ಐಂ ಕಏಈಲಹ್ರೀಂ ಕ್ಲೀಂ ಹಸಕಹಲಹ್ರೀಂ ಸೌಃ ಸಕಲಹ್ರೀಂ} ನಮಃ ॥(ಬಿಂದೌ)\\
\as{೪ ಐಂ ಕಏಈಲಹ್ರೀಂ} ನಮಃ~।(ಅಗ್ರಕೋಣೇ)\\
\as{೪ ಕ್ಲೀಂ ಹಸಕಹಲಹ್ರೀಂ} ನಮಃ~।(ಆಗ್ನೇಯಕೋಣೇ)\\
\as{೪ ಸೌಃ ಸಕಲಹ್ರೀಂ} ನಮಃ~।(ಐಶಾನ್ಯಕೋಣೇ)

ಉಪಾಸ್ಯ ದೇವತಾಃ ಆವಾಹಯೇತ್ ॥
\section{ಅಥ ನಿತ್ಯಕ್ಲಿನ್ನಾ ಮಂತ್ರಾಃ\\೧। ಶ್ರೀಕಾಮೇಶ್ವರೀನಿತ್ಯಾ}
\addcontentsline{toc}{section}{ಅಥ ನಿತ್ಯಕ್ಲಿನ್ನಾ ಮಂತ್ರಾಃ}
\addcontentsline{toc}{section}{೧। ಶ್ರೀಕಾಮೇಶ್ವರೀನಿತ್ಯಾ}
ಅಸ್ಯ ಶ್ರೀಕಾಮೇಶ್ವರೀನಿತ್ಯಾಮಹಾಮಂತ್ರಸ್ಯ ಸಮ್ಮೋಹನ ಋಷಿಃ~। ಗಾಯತ್ರೀ ಛಂದಃ~। ಶ್ರೀಕಾಮೇಶ್ವರೀದೇವತಾ~। ಕಂ ಬೀಜಂ~। ಇಂ ಶಕ್ತಿಃ। ಲಂ ಕೀಲಕಂ~।\\
\as{ನ್ಯಾಸಃ :}೧.ಓಂ ಐಂ ೨.ಓಂ ಸಕಲಹ್ರೀಂ ೩.ಓಂ ನಿತ್ಯ  ೪.ಓಂ ಕ್ಲಿನ್ನೇ ೫.ಓಂ ಮದದ್ರವೇ ೬.ಓಂ ಸೌಃ \\
{\bfseries ದೇವೀಂ ಧ್ಯಾಯೇಜ್ಜಗದ್ಧಾತ್ರೀಂ ಜಪಾಕುಸುಮಸನ್ನಿಭಾಂ~।\\
ಬಾಲಭಾನುಪ್ರತೀಕಾಶಾಂ ಶಾತಕುಂಭಸಮಪ್ರಭಾಂ ॥\\
ರಕ್ತವಸ್ತ್ರಪರೀಧಾನಾಂ ಸಂಪದ್ವಿದ್ಯಾವಶಂಕರೀಂ~।\\
ನಮಾಮಿ ವರದಾಂ ದೇವೀಂ ಕಾಮೇಶೀಮಭಯಪ್ರದಾಂ ॥\\}
ಮನುಃ :{\bfseries  ಐಂ ಸಕಲಹ್ರೀಂ ನಿತ್ಯಕ್ಲಿನ್ನೇ ಮದದ್ರವೇ ಸೌಃ~॥}
\section{೨। ಭಗಮಾಲಿನೀನಿತ್ಯಾ}
\addcontentsline{toc}{section}{೨। ಭಗಮಾಲಿನೀನಿತ್ಯಾ}
ಅಸ್ಯ ಶ್ರೀಭಗಮಾಲಿನೀನಿತ್ಯಾಮಹಾಮಂತ್ರಸ್ಯ ಸುಭಗಋಷಿಃ~। ಗಾಯತ್ರೀ ಛಂದಃ~। ಶ್ರೀಭಗಮಾಲಿನೀ ದೇವತಾ~। ಹ್‌ರ್‌ಬ್ಲೇಂ ಬೀಜಂ~।  ಶ್ರೀಂ ಶಕ್ತಿಃ~। ಕ್ಲೀಂ ಕೀಲಕಂ~।\\
\as{ನ್ಯಾಸಃ :}೧.ಓಂ ಐಂ  ೨.ಓಂ ಭಗಭುಗೇ ೩.ಓಂ ಭಗಿನಿ  ೪.ಓಂ ಭಗೋದರಿ  ೫.ಓಂ ಭಗಮಾಲೇ ೬.ಓಂ ಭಗಾವಹೇ\\
{\bfseries ಭಗರೂಪಾಂ ಭಗಮಯಾಂ ದುಕೂಲವಸನಾಂ ಶಿವಾಂ~।\\
ಸರ್ವಾಲಂಕಾರಸಂಯುಕ್ತಾಂ ಸರ್ವಲೋಕವಶಂಕರೀಂ ॥\\
ಭಗೋದರೀಂ ಮಹಾದೇವೀಂ ರಕ್ತೋತ್ಪಲಸಮಪ್ರಭಾಂ~।\\
ಕಾಮೇಶ್ವರಾಂಕನಿಲಯಾಂ ವಂದೇ ಶ್ರೀಭಗಮಾಲಿನೀಂ ॥\\}
ಮನುಃ :{\bfseries  ಐಂ ಭಗಭುಗೇ ಭಗಿನಿ ಭಗೋದರಿ ಭಗಮಾಲೇ ಭಗಾವಹೇ ಭಗಗುಹ್ಯೇ ಭಗಯೋನಿ ಭಗನಿಪಾತನಿ ಸರ್ವಭಗವಶಂಕರಿ ಭಗರೂಪೇ ನಿತ್ಯಕ್ಲಿನ್ನೇ ಭಗಸ್ವರೂಪೇ ಸರ್ವಾಣಿ ಭಗಾನಿ ಮೇ ಹ್ಯಾನಯ ವರದೇ ರೇತೇ ಸುರೇತೇ ಭಗಕ್ಲಿನ್ನೇ ಕ್ಲಿನ್ನದ್ರವೇ ಕ್ಲೇದಯ ದ್ರಾವಯ ಅಮೋಘೇ ಭಗವಿಚ್ಚೇ ಕ್ಷುಭ ಕ್ಷೋಭಯ ಸರ್ವಸತ್ವಾನ್ ಭಗೇಶ್ವರಿ ಐಂ ಬ್ಲೂಂ ಜಂ ಬ್ಲೂಂ ಭೇಂ ಬ್ಲೂಂ ಮೋಂ ಬ್ಲೂಂ ಹೇಂ ಬ್ಲೂಂ ಹೇಂ ಕ್ಲಿನ್ನೇ ಸರ್ವಾಣಿ ಭಗಾನಿ ಮೇ ವಶಮಾನಯ ಸ್ತ್ರೀಂ ಹ್‌ರ್‌ಬ್ಲೇಂ ಹ್ರೀಂ ॥}
\section{೩।ನಿತ್ಯಕ್ಲಿನ್ನಾ}
\addcontentsline{toc}{section}{೩।ನಿತ್ಯಕ್ಲಿನ್ನಾ}
ಅಸ್ಯ ಶ್ರೀನಿತ್ಯಕ್ಲಿನ್ನಾಮಹಾಮಂತ್ರಸ್ಯ ಬ್ರಹ್ಮಾ ಋಷಿಃ। ವಿರಾಟ್ಛಂದಃ~।\\ ಶ್ರೀನಿತ್ಯಕ್ಲಿನ್ನಾನಿತ್ಯಾದೇವತಾ। ಹ್ರೀಂ ಬೀಜಂ। ಸ್ವಾಹಾ ಶಕ್ತಿಃ । ನ್ನೇ ಕೀಲಕಂ~।\\
\as{ನ್ಯಾಸಃ :}೧.ಓಂ ಹ್ರೀಂ ೨.ಓಂ ನಿತ್ಯ  ೩.ಓಂ ಕ್ಲಿನ್ನೇ ೪.ಓಂ ಮದ ೫.ಓಂ ದ್ರವೇ ೬.ಓಂ ಸ್ವಾಹಾ\\
{\bfseries ಪದ್ಮರಾಗಮಣಿಪ್ರಖ್ಯಾಂ ಹೇಮತಾಟಂಕಸಂಯುತಾಂ~।\\
ರಕ್ತವಸ್ತ್ರಧರಾಂ ದೇವೀಂ ರಕ್ತಮಾಲ್ಯಾನುಲೇಪನಾಂ~।\\
ಅಂಜನಾಂಚಿತನೇತ್ರಾಂ ತಾಂ ಪದ್ಮಪತ್ರನಿಭೇಕ್ಷಣಾಂ~।\\
ನಿತ್ಯಕ್ಲಿನ್ನಾಂ ನಮಸ್ಯಾಮಿ ಚತುರ್ಭುಜವಿರಾಜಿತಾಂ~॥\\}
ಮನುಃ :{\bfseries  ಹ್ರೀಂ ನಿತ್ಯಕ್ಲಿನ್ನೇ ಮದದ್ರವೇ ಸ್ವಾಹಾ ॥}
\section{೪।ಭೇರುಂಡಾ}
\addcontentsline{toc}{section}{೪।ಭೇರುಂಡಾ}
ಶ್ರೀಭೇರುಂಡಾನಿತ್ಯಾಮಹಾಮಂತ್ರಸ್ಯ ಮಹಾವಿಷ್ಣುಃ ಋಷಿಃ। \\ಗಾಯತ್ರೀಛಂದಃ~। ಭೇರುಂಡಾನಿತ್ಯಾ ದೇವತಾ। ಭ್ರೋಂ ಬೀಜಂ।\\ ಸ್ವಾಹಾ ಶಕ್ತಿಃ। ಕ್ರೋಂ ಕೀಲಕಂ~।\\
\as{ನ್ಯಾಸಃ :}೧.ಓಂ ಕ್ರೋಂ  ೨.ಓಂ ಭ್ರೋಂ ೩.ಓಂ ಕ್ರೋಂ  ೪.ಓಂ ಝ್ರೋಂ ೫.ಓಂ ಛ್ರೋಂ  ೬.ಓಂ ಜ್ರೋಂ \\
{\bfseries ಶುದ್ಧಸ್ಫಟಿಕಸಂಕಾಶಾಂ ಪದ್ಮಪತ್ರಸಮಪ್ರಭಾಂ~।\\
ಮಧ್ಯಾಹ್ನಾದಿತ್ಯಸಂಕಾಶಾಂ ಶುಭ್ರವಸ್ತ್ರಸಮನ್ವಿತಾಂ~॥\\
ಶ್ವೇತಚಂದನಲಿಪ್ತಾಂಗೀಂ ಶುಭ್ರಮಾಲ್ಯವಿಭೂಷಿತಾಂ~।\\	
ಬಿಭ್ರತೀಂ ಚಿನ್ಮಯೀಂ ಮುದ್ರಾಮಕ್ಷಮಾಲಾಂ ಚ ಪುಸ್ತಕಂ~॥\\
ಸಹಸ್ರಪದ್ಮಕಮಲೇ ಸಮಾಸೀನಾಂ ಶುಚಿಸ್ಮಿತಾಂ~।\\
ಸರ್ವವಿದ್ಯಾಪ್ರದಾಂ ದೇವೀಂ ಭೇರುಂಡಾಂ ಪ್ರಣಮಾಮ್ಯಹಂ~॥\\}
ಮನುಃ :{\bfseries  ಕ್ರೋಂ ಭ್ರೋಂ ಕ್ರೋಂ ಝ್ರೋಂ ಛ್ರೋಂ ಜ್ರೋಂ ಸ್ವಾಹಾ ॥}
\section{೫।ವಹ್ನಿವಾಸಿನೀನಿತ್ಯಾ}
\addcontentsline{toc}{section}{೫।ವಹ್ನಿವಾಸಿನೀನಿತ್ಯಾ}
ಶ್ರೀವಹ್ನಿವಾಸಿನೀನಿತ್ಯಾ ಮಂತ್ರಸ್ಯ ವಸಿಷ್ಠ ಋಷಿಃ। ಗಾಯತ್ರೀಛಂದಃ~।\\ ಶ್ರೀವಹ್ನಿವಾಸಿನೀನಿತ್ಯಾದೇವತಾ। ಹ್ರೀಂ ಬೀಜಂ। ನಮಃ ಶಕ್ತಿಃ।\\ ವಹ್ನಿವಾಸಿನ್ಯೈ ಕೀಲಕಂ~। ಹ್ರಾಂ ಇತ್ಯಾದಿನ್ಯಾಸಃ~।\\
{\bfseries ವಹ್ನಿಕೋಟಿಪ್ರತೀಕಾಶಾಂ ಸೂರ್ಯಕೋಟಿಸಮಪ್ರಭಾಂ~।\\
ಅಗ್ನಿಜ್ವಾಲಾಸಮಾಕೀರ್ಣಾಂ ಸರ್ವರೋಗಾಪಹಾರಿಣೀಂ~।\\
ಕಾಲಮೃತ್ಯುಪ್ರಶಮನೀಂ ಭಯಮೃತ್ಯುನಿವಾರಿಣೀಂ~।\\
ಪರಮಾಯುಷ್ಯದಾಂ ವಂದೇ ನಿತ್ಯಾಂ ಶ್ರೀವಹ್ನಿವಾಸಿನೀಂ ॥\\}
ಮನುಃ :{\bfseries ಓಂ ಹ್ರೀಂ ವಹ್ನಿವಾಸಿನ್ಯೈ ನಮಃ~॥}
\section{೬। ಮಹಾವಜ್ರೇಶ್ವರೀ}
\addcontentsline{toc}{section}{೬। ಮಹಾವಜ್ರೇಶ್ವರೀ}
ಶ್ರೀಮಹಾವಜ್ರೇಶ್ವರೀನಿತ್ಯಾ ಮಂತ್ರಸ್ಯ ಬ್ರಹ್ಮಾ ಋಷಿಃ। ಗಾಯತ್ರೀಛಂದಃ~। ಶ್ರೀಮಹಾವಜ್ರೇಶ್ವರೀನಿತ್ಯಾದೇವತಾ। ಹ್ರೀಂ ಬೀಜಂ। ಹ್ರೀಂ ಶಕ್ತಿಃ।\\
ಐಂ ಕೀಲಕಂ~।\\
\as{ನ್ಯಾಸಃ :}೧.ಓಂ ಹ್ರೀಂ ಕ್ಲಿನ್ನೇ ಹ್ರೀಂ  ೨.ಓಂ ಹ್ರೀಂ ಐಂ ಹ್ರೀಂ  ೩.ಓಂ ಹ್ರೀಂ ಕ್ರೋಂ ಹ್ರೀಂ ೪.ಓಂ ಹ್ರೀಂ ನಿತ್ಯ ಹ್ರೀಂ  ೫.ಓಂ ಹ್ರೀಂ ಮದ ಹ್ರೀಂ ೬.ಓಂ ಹ್ರೀಂ ದ್ರವೇ ಹ್ರೀಂ \\
{\bfseries ತಪ್ತಕಾಂಚನಸಂಕಾಶಾಂ ಕನಕಾಭರಣಾನ್ವಿತಾಂ~।\\
ಹೇಮತಾಟಂಕಸಂಯುಕ್ತಾಂ ಕಸ್ತೂರೀತಿಲಕಾನ್ವಿತಾಂ ॥\\
ಹೇಮಚಿಂತಾಕಸಂಯುಕ್ತಾಂ ಪೂರ್ಣಚಂದ್ರಮುಖಾಂಬುಜಾಂ~।\\
ಪೀತಾಂಬರಸಮೋಪೇತಾಂ ಪುಣ್ಯಮಾಲ್ಯವಿಭೂಷಿತಾಂ ॥\\
ಮುಕ್ತಾಹಾರಸಮೋಪೇತಾಂ ಮುಕುಟೇನ ವಿರಾಜಿತಾಂ~।\\
ಮಹಾವಜ್ರೇಶ್ವರೀಂ ವಂದೇ ಸರ್ವೈಶ್ವರ್ಯಫಲಪ್ರದಾಂ ॥\\}
ಮನುಃ :{\bfseries  ಓಂ ಹ್ರೀಂ ಕ್ಲಿನ್ನೇ ಐಂ ಕ್ರೋಂ ನಿತ್ಯಮದದ್ರವೇ ಹ್ರೀಂ~॥}
\section{೭।ಶಿವಾದೂತೀನಿತ್ಯಾ}
\addcontentsline{toc}{section}{೭।ಶಿವಾದೂತೀನಿತ್ಯಾ}
ಶ್ರೀಶಿವಾದೂತೀನಿತ್ಯಾ ಮಂತ್ರಸ್ಯ ರುದ್ರಋಷಿಃ~। ಗಾಯತ್ರೀ ಛಂದಃ~।\\ ಶ್ರೀಶಿವಾದೂತೀನಿತ್ಯಾ ದೇವತಾ~। ಹ್ರೀಂ ಬೀಜಂ~। ನಮಃ ಶಕ್ತಿಃ~।\\ ಶಿವಾದೂತ್ಯೈ ಕೀಲಕಂ~। ಹ್ರಾಂ ಇತ್ಯಾದಿನಾ ನ್ಯಾಸಃ~।
{\bfseries ಬಾಲಸೂರ್ಯಪ್ರತೀಕಾಶಾಂ ಬಂಧೂಕಪ್ರಸವಾರುಣಾಂ~।\\
ವಿಧಿವಿಷ್ಣುಶಿವಸ್ತುತ್ಯಾಂ ದೇವಗಂಧರ್ವಸೇವಿತಾಂ ॥\\
ರಕ್ತಾರವಿಂದಸಂಕಾಶಾಂ ಸರ್ವಾಭರಣಭೂಷಿತಾಂ~।\\
ಶಿವದೂತೀಂ ನಮಸ್ಯಾಮಿ ರತ್ನಸಿಂಹಾಸನಸ್ಥಿತಾಂ ॥\\}
ಮನುಃ :{\bfseries  ಓಂ ಹ್ರೀಂ ಶಿವಾದೂತ್ಯೈ ನಮಃ~॥}
\section{೮। ತ್ವರಿತಾ ನಿತ್ಯಾ}
\addcontentsline{toc}{section}{೮। ತ್ವರಿತಾ ನಿತ್ಯಾ}
ಅಸ್ಯ ಶ್ರೀತ್ವರಿತಾನಿತ್ಯಾ ಮಹಾಮಂತ್ರಸ್ಯ ಈಶ್ವರ ಋಷಿಃ~। ವಿರಾಟ್ ಛಂದಃ~। ತ್ವರಿತಾನಿತ್ಯಾ ದೇವತಾ। ಹೂಂ ಬೀಜಂ~। ಸ್ತ್ರೀಂ ಶಕ್ತಿಃ~। ಹ್ರೀಂ ಕೀಲಕಂ~।\\
ಹ್ರಾಂ ಇತ್ಯಾದಿನ್ಯಾಸಃ ।\\
{\bfseries ರಕ್ತಾರವಿಂದಸಂಕಾಶಾಮುದ್ಯತ್ಸೂರ್ಯಸಮಪ್ರಭಾಂ~।\\
ದಧತೀಮಂಕುಶಂ ಪಾಶಂ ಬಾಣಾನ್ ಚಾಪಂ ಮನೋಹರಂ ॥\\
ಚತುರ್ಭುಜಾಂ ಮಹಾದೇವೀಮಪ್ಸರೋಗಣಸಂಕುಲಾಂ~।\\
ನಮಾಮಿ ತ್ವರಿತಾಂ ನಿತ್ಯಾಂ ಭಕ್ತಾನಾಮಭಯಪ್ರದಾಂ ॥\\}
ಮನುಃ :{\bfseries  ಓಂ ಹ್ರೀಂ ಹೂಂ ಖೇ ಚ ಛೇ ಕ್ಷಃ ಸ್ತ್ರೀಂ ಹೂಂ ಕ್ಷೇ ಹ್ರೀಂ ಫಟ್~॥}
\section{೯। ಕುಲಸುಂದರೀನಿತ್ಯಾ}
\addcontentsline{toc}{section}{೯। ಕುಲಸುಂದರೀನಿತ್ಯಾ}
ಅಸ್ಯ ಶ್ರೀ ಕುಲಸುಂದರೀ ನಿತ್ಯಾ ಮಹಾಮಂತ್ರಸ್ಯ ದಕ್ಷಿಣಾಮೂರ್ತಿಃ ಋಷಿಃ~। ಪಂಕ್ತಿಶ್ಛಂದಃ~। ಶ್ರೀ ಕುಲಸುಂದರೀ ನಿತ್ಯಾ ದೇವತಾ~। ಐಂ ಬೀಜಂ~। ಸೌಃ ಶಕ್ತಿಃ~। ಕ್ಲೀಂ ಕೀಲಕಂ~। ಆಂ , ಈಂ , ಇತ್ಯಾದಿನ್ಯಾಸಃ~।\\
{\bfseries ಅರುಣಕಿರಣಜಾಲೈ ರಂಜಿತಾಶಾವಕಾಶಾ\\
ವಿಧೃತಜಪವಟೀಕಾ ಪುಸ್ತಕಾಭೀತಿಹಸ್ತಾ ॥\\
ಇತರಕರವರಾಢ್ಯಾ ಫುಲ್ಲಕಹ್ಲಾರಸಂಸ್ಥಾ\\
ನಿವಸತು ಹೃದಿ ಬಾಲಾ ನಿತ್ಯಕಲ್ಯಾಣಶೀಲಾ ॥\\}
ಮನುಃ :{\bfseries  ಐಂ ಕ್ಲೀಂ ಸೌಃ~॥}
\section{೧೦।ನಿತ್ಯಾನಿತ್ಯಾ}
\addcontentsline{toc}{section}{೧೦।ನಿತ್ಯಾನಿತ್ಯಾ}
ಅಸ್ಯ ಶ್ರೀನಿತ್ಯಾನಿತ್ಯಾಮಹಾಮಂತ್ರಸ್ಯ ದಕ್ಷಿಣಾಮೂರ್ತಿಃ ಋಷಿಃ।\\ ಪಂಕ್ತಿಃ ಛಂದಃ~। ಶ್ರೀನಿತ್ಯಾನಿತ್ಯಾದೇವತಾ। ಐಂ ಬೀಜಂ। ಔಃ ಶಕ್ತಿಃ~।\\ ಈಂ ಕೀಲಕಂ~। ಹ್ರೀಂ ಹ್ಸಾಂ, ಹ್ರೀಂ ಹ್ಸೀಂ ಇತ್ಯಾದಿನಾ ನ್ಯಾಸಃ~।\\
{\bfseries ಉದ್ಯತ್ಪ್ರದ್ಯೋತನನಿಭಾಂ ಜಪಾಕುಸುಮಸನ್ನಿಭಾಂ~।\\
ಹರಿಚಂದನಲಿಪ್ತಾಂಗೀಂ ರಕ್ತಮಾಲ್ಯ ವಿಭೂಷಿತಾಂ ॥\\
ರತ್ನಾಭರಣಭೂಷಾಂಗೀಂ ರಕ್ತವಸ್ತ್ರಸುಶೋಭಿತಾಂ~।\\
ಜಗದಂಬಾಂ ನಮಸ್ಯಾಮಿ ನಿತ್ಯಾಂ ಶ್ರೀಪರಮೇಶ್ವರೀಂ ॥\\}
ಮನುಃ :{\bfseries  ಓಂ ಹಸಕಲರಡೈಂ ಹಸಕಲರಡೀಂ ಹಸಕಲರಡೌಃ ॥}
\section{೧೧।ಶ್ರೀನೀಲಪತಾಕಾನಿತ್ಯಾ}
\addcontentsline{toc}{section}{೧೧।ಶ್ರೀನೀಲಪತಾಕಾನಿತ್ಯಾ}
ಅಸ್ಯ ಶ್ರೀನೀಲಪತಾಕಾನಿತ್ಯಾ ಮಹಾಮಂತ್ರಸ್ಯ ಸಮ್ಮೋಹನ ಋಷಿಃ~। \\ಗಾಯತ್ರೀ ಛಂದಃ~। ಶ್ರೀನೀಲಪತಾಕಾನಿತ್ಯಾ ದೇವತಾ~। ಹ್ರೀಂ ಬೀಜಂ~।\\ ಹ್ರೀಂ ಶಕ್ತಿಃ~। ಕ್ಲೀಂ ಕೀಲಕಂ~।\\
\as{ನ್ಯಾಸಃ :}೧.ಓಂ ಓಂ ಹ್ರೀಂ ಫ್ರೇಂ ೨.ಓಂ ಸ್ರೂಂ ಓಂ ಆಂ ಕ್ಲೀಂ ೩.ಓಂ ಐಂ ಬ್ಲೂಂ ನಿತ್ಯಮದ ೪.ಓಂ ದ್ರ ೫.ಓಂ ವೇ ೬.ಓಂ ಹುಂ \\
{\bfseries ಪಂಚವಕ್ತ್ರಾಂ ತ್ರಿಣಯನಾಮರುಣಾಂಶುಕಧಾರಿಣೀಂ~।\\
ದಶಹಸ್ತಾಂ ಲಸನ್ಮುಕ್ತಾಪ್ರಾಯಾಭರಣಮಂಡಿತಾಂ ॥\\
ನೀಲಮೇಘಸಮಪ್ರಖ್ಯಾಂ ಧೂಮ್ರಾರ್ಚಿಸದೃಶಪ್ರಭಾಂ~।\\
ನೀಲಪುಷ್ಪಸ್ರಜೋಪೇತಾಂ ಧ್ಯಾಯೇನ್ನೀಲಪತಾಕಿನೀಂ ॥\\}
ಮನುಃ :{\bfseries ಓಂ ಹ್ರೀಂ ಫ್ರೇಂ ಸ್ರೂಂ ಓಂ ಆಂ ಕ್ಲೀಂ ಐಂ ಬ್ಲೂಂ ನಿತ್ಯಮದದ್ರವೇ ಹುಂ ಫ್ರೇಂ ಹ್ರೀಂ~॥}
\section{೧೨। ವಿಜಯಾ ನಿತ್ಯಾ}
\addcontentsline{toc}{section}{೧೨। ವಿಜಯಾ ನಿತ್ಯಾ}
ಅಸ್ಯ ಶ್ರೀ ವಿಜಯಾನಿತ್ಯಾಮಹಾಮಂತ್ರಸ್ಯ ಅಹಿರ್ಋಷಿಃ~। ಗಾಯತ್ರೀಛಂದಃ~। ಶ್ರೀವಿಜಯಾನಿತ್ಯಾ ದೇವತಾ~।\\
\as{ನ್ಯಾಸಃ :}೧.ಓಂ ಭಾಂ ೨.ಓಂ ಮೀಂ ೩.ಓಂ ರೂಂ ೪.ಓಂ ಯೈಂ ೫.ಓಂ ಉಂ ೬.ಓಂ ಔಂ \\
{\bfseries ಉದ್ಯದರ್ಕಸಹಸ್ರಾಭಾಂ ದಾಡಿಮೀಪುಷ್ಪಸನ್ನಿಭಾಂ~।\\
ರಕ್ತಕಂಕಣಕೇಯೂರಕಿರೀಟಾಂಗದಸಂಯುತಾಂ ॥\\
ದೇವಗಂಧರ್ವಯೋಗೀಶಮುನಿಸಿದ್ಧನಿಷೇವಿತಾಂ~।\\
ನಮಾಮಿ ವಿಜಯಾಂ ನಿತ್ಯಾಂ ಸಿಂಹೋಪರಿ ಕೃತಾಸನಾಂ ॥\\}
ಮನುಃ :{\bfseries  ಭ ಮ ರ ಯ ಉ ಔಂ ॥}
\section{೧೩। ಸರ್ವಮಂಗಲಾನಿತ್ಯಾ}
\addcontentsline{toc}{section}{೧೩। ಸರ್ವಮಂಗಲಾನಿತ್ಯಾ}
ಅಸ್ಯ ಶ್ರೀಸರ್ವಮಂಗಲಾನಿತ್ಯಾ ಮಹಾಮಂತ್ರಸ್ಯ ಚಂದ್ರ ಋಷಿಃ~।\\ ಗಾಯತ್ರೀ ಛಂದಃ~। ಸರ್ವಮಂಗಲಾನಿತ್ಯಾ ದೇವತಾ~।\\
ಸ್ವಾಂ , ಸ್ವೀಂ ಇತ್ಯಾದಿನಾ ನ್ಯಾಸಃ~।\\
{\bfseries ರಕ್ತೋತ್ಪಲಸಮಪ್ರಖ್ಯಾಂ ಪದ್ಮಪತ್ರನಿಭೇಕ್ಷಣಾಂ~।\\
ಇಕ್ಷುಕಾರ್ಮುಕಪುಷ್ಪೌಘಪಾಶಾಂಕುಶಸಮನ್ವಿತಾಂ ॥\\
ಸುಪ್ರಸನ್ನಾಂ ಶಶಿಮುಖೀಂ ನಾನಾರತ್ನವಿಭೂಷಿತಾಂ~।\\
ಶುಭ್ರಪದ್ಮಾಸನಸ್ಥಾಂ ತಾಂ ಭಜಾಮಿ ಸರ್ವಮಂಗಲಾಂ ॥\\}
ಮನುಃ :{\bfseries  ಸ್ವೌಂ ॥}
\section{೧೪। ಜ್ವಾಲಾಮಾಲಿನೀ}
\addcontentsline{toc}{section}{೧೪। ಜ್ವಾಲಾಮಾಲಿನೀ}
ಅಸ್ಯ ಶ್ರೀ ಜ್ವಾಲಾಮಾಲಿನೀನಿತ್ಯಾಮಹಾಮಂತ್ರಸ್ಯ ಕಶ್ಯಪ ಋಷಿಃ~। ಗಾಯತ್ರೀ ಛಂದಃ~। ಜ್ವಾಲಾಮಾಲಿನೀನಿತ್ಯಾ ದೇವತಾ~। ರಂ ಬೀಜಂ~। ಫಟ್ ಶಕ್ತಿಃ~। ಹುಂ ಕೀಲಕಂ~।\\
\as{ನ್ಯಾಸಃ :}೧.ಓಂ ಓಂ ೨.ಓಂ ನಮಃ ೩.ಓಂ ಭಗವತಿ ೪.ಓಂ ಜ್ವಾಲಾಮಾಲಿನಿ ೫.ಓಂ ದೇವದೇವಿ ೬.ಓಂ ಸರ್ವಭೂತಸಂಹಾರಕಾರಿಕೇ \\
{\bfseries ಅಗ್ನಿಜ್ವಾಲಾಸಮಾಭಾಕ್ಷೀಂ ನೀಲವಕ್ತ್ರಾಂ ಚತುರ್ಭುಜಾಂ~।\\ನೀಲನೀರದಸಂಕಾಶಾಂ ನೀಲಕೇಶೀಂ ತನೂದರೀಂ ॥\\
ಖಡ್ಗಂ ತ್ರಿಶೂಲಂ ಬಿಭ್ರಾಣಾಂ ವರಂ ಸಾಭಯಮೇವ ಚ~।\\ಸಿಂಹಪೃಷ್ಠಸಮಾರೂಢಾಂ ಧ್ಯಾಯೇಜ್ಜ್ವಾಲಾದ್ಯಮಾಲಿನೀಂ ॥\\}
ಮನುಃ :{\bfseries  ಓಂ ನಮೋ ಭಗವತಿ ಜ್ವಾಲಾಮಾಲಿನಿ ದೇವದೇವಿ ಸರ್ವಭೂತಸಂಹಾರಕಾರಿಕೇ ಜಾತವೇದಸಿ ಜ್ವಲಂತಿ ಜ್ವಲ ಜ್ವಲ ಪ್ರಜ್ವಲ ಪ್ರಜ್ವಲ ಹ್ರಾಂ ಹ್ರೀಂ ಹ್ರೂಂ ರರ ರರ ರರರ ಹುಂ ಫಟ್ ಸ್ವಾಹಾ~॥}
\section{೧೫। ಚಿತ್ರಾನಿತ್ಯಾ}
\addcontentsline{toc}{section}{೧೫। ಚಿತ್ರಾನಿತ್ಯಾ}
ಚಿತ್ರಾನಿತ್ಯಾಮಹಾಮಂತ್ರಸ್ಯ ಬ್ರಹ್ಮಾ ಋಷಿಃ~। ಗಾಯತ್ರೀ ಛಂದಃ~। \\ಚಿತ್ರಾ ನಿತ್ಯಾ ದೇವತಾ~। ಚಾಂ ಚೀಂ ಇತ್ಯಾದಿನಾ ನ್ಯಾಸಃ~।\\
{\bfseries ಶುದ್ಧಸ್ಫಟಿಕಸಂಕಾಶಾಂ ಪಲಾಶಕುಸುಮಪ್ರಭಾಂ~।\\
ನೀಲಮೇಘಪ್ರತೀಕಾಶಾಂ ಚತುರ್ಹಸ್ತಾಂ ತ್ರಿಲೋಚನಾಂ ॥\\
ಸರ್ವಾಲಂಕಾರಸಂಯುಕ್ತಾಂ ಪುಷ್ಪಬಾಣೇಕ್ಷುಚಾಪಿನೀಂ~।\\
ಪಾಶಾಂಕುಶಸಮೋಪೇತಾಂ ಧ್ಯಾಯೇಚ್ಚಿತ್ರಾಂ ಮಹೇಶ್ವರೀಂ ॥\\}
ಮನುಃ :{\bfseries  ಚ್ಕೌಂ~॥}


ಹ್ಸಾಂ ಹ್ಸೀಂ ಇತ್ಯಾದಿನಾ ಕರಾಂಗನ್ಯಾಸಂ ವಿಧಾಯ
\section{ಧ್ಯಾನಂ}
\addcontentsline{toc}{section}{ಧ್ಯಾನಂ}
ಪಂಚವಕ್ತ್ರಂ ಚತುರ್ಬಾಹುಂ ಸರ್ವಾಭರಣ ಭೂಷಿತಮ್ ।\\
ಚಂದ್ರಸೂರ್ಯಸಹಸ್ರಾಭಂ ಶಿವಶಕ್ತ್ಯಾತ್ಮಕಂ ಭಜೇ ॥

ಅಮೃತಾರ್ಣವಮಧ್ಯಸ್ಥ ರತ್ನದ್ವೀಪೇ ಮನೋರಮೇ~।\\
ಕಲ್ಪವೃಕ್ಷವನಾಂತಸ್ಥೇ ನವಮಾಣಿಕ್ಯಮಂಡಪೇ ॥

ನವರತ್ನಮಯ ಶ್ರೀಮತ್ ಸಿಂಹಾಸನಾಗತಾಂಬುಜೇ~।\\
ತ್ರಿಕೋಣಾಂತಃ ಸಮಾಸೀನಂ ಚಂದ್ರಸೂರ್ಯಾಯುತಪ್ರಭಂ ॥

ಅರ್ಧಾಂಬಿಕಾಸಮಾಯುಕ್ತಂ ಪ್ರವಿಭಕ್ತವಿಭೂಷಣಂ~।\\
ಕೋಟಿಕಂದರ್ಪಲಾವಣ್ಯಂ ಸದಾ ಷೋಡಶವಾರ್ಷಿಕಂ ॥

ಮಂದಸ್ಮಿತಮುಖಾಂಭೋಜಂ ತ್ರಿನೇತ್ರಂ ಚಂದ್ರಶೇಖರಂ~।\\
ದಿವ್ಯಾಂಬರಸ್ರಗಾಲೇಪಂ ದಿವ್ಯಾಭರಣಭೂಷಿತಂ ॥

ಪಾನಪಾತ್ರಂ ಚ ಚಿನ್ಮುದ್ರಾಂ ತ್ರಿಶೂಲಂ ಪುಸ್ತಕಂ ಕರೈಃ~।\\
ವಿದ್ಯಾ ಸಂಸದಿ ಬಿಭ್ರಾಣಂ ಸದಾನಂದಮುಖೇಕ್ಷಣಂ ॥

ಮಹಾಷೋಢೋದಿತಾಶೇಷದೇವತಾಗಣಸೇವಿತಂ~।\\
ಏವಂ ಚಿತ್ತಾಂಬುಜೇ ಧ್ಯಾಯೇದರ್ಧನಾರೀಶ್ವರಂ ಶಿವಂ ॥

ಪುರುಷಂ ವಾ ಸ್ಮರೇದ್ ದೇವಿ ಸ್ತ್ರೀರೂಪಂ ವಾ ವಿಚಿಂತಯೇತ್~।\\
ಅಥವಾ ನಿಷ್ಕಲಂ ಧ್ಯಾಯೇತ್ ಸಚ್ಚಿದಾನಂದವಿಗ್ರಹಂ~।\\
ಸರ್ವತೇಜೋಮಯಂ ದಿವ್ಯಂ ಸಚರಾಚರವಿಗ್ರಹಂ ॥

\dhyana{ಸಹಸ್ರದಲಪಂಕಜೇ ಸಕಲಶೀತರಶ್ಮಿಪ್ರಭಂ\\
ವರಾಭಯಕರಾಂಬುಜಂ ವಿಮಲಗಂಧಪುಷ್ಪಾಂಬರಂ~।\\
ಪ್ರಸನ್ನವದನೇಕ್ಷಣಂ ಸಕಲದೇವತಾರೂಪಿಣಂ\\
ಸ್ಮರೇಚ್ಛಿರಸಿ ಹಂಸಗಂ ತದಭಿಧಾನಪೂರ್ವಂ ಗುರುಂ ॥}\\
ಓಂ ಐಂಹ್ರೀಂಶ್ರೀಂ ಐಂಕ್ಲೀಂಸೌಃ ಹ್‌ಸ್‌ಖ್‌ಫ್ರೇಂ ಹಸಕ್ಷಮಲವರಯೂಂ ಹ್ಸೌಃ ಸಹಕ್ಷಮಲವರಯೀಂ ಸ್ಹೌಃ ॥ ಇತಿ ಗುರುಂ ಧ್ಯಾತ್ವಾ

ಶಾಂತಂ ಪದ್ಮಾಸನಸ್ಥಂ ಶಶಧರಮುಕುಟಂ ಪಂಚವಕ್ತ್ರಂ ತ್ರಿನೇತ್ರಂ\\
ಶೂಲಂ ವಜ್ರಂ ಚ ಖಡ್ಗಂ ಪರಶುಮಭಯದಂ ದಕ್ಷಭಾಗೇ ವಹಂತಂ~।\\
ನಾಗಂ ಪಾಶಂ ಚ ಘಂಟಾಂ ಪ್ರಲಯಹುತವಹಂ ಚಾಂಕುಶಂ ವಾಮಭಾಗೇ\\
ನಾನಾಲಂಕಾರಯುಕ್ತಂ ಸ್ಫಟಿಕಮಣಿನಿಭಂ ಪಾರ್ವತೀಶಂ ನಮಾಮಿ ॥

ಅಥಾಹಂ ಬೈಂದವೇ ಚಕ್ರೇ ಸರ್ವಾನಂದಮಯಾತ್ಮಕೇ~।\\
ರತ್ನಸಿಂಹಾಸನೇ ರಮ್ಯೇ ಸಮಾಸೀನಾಂ ಶಿವಪ್ರಿಯಾಂ ॥

ಉದ್ಯದ್ಭಾನುಸಹಸ್ರಾಭಾಂ ಜಪಾಪುಷ್ಪಸಮಪ್ರಭಾಂ~।\\
ನವರತ್ನಪ್ರಭಾಯುಕ್ತಮಕುಟೇನ ವಿರಾಜಿತಾಂ ॥

ಚಂದ್ರರೇಖಾಸಮೋಪೇತಾಂ ಕಸ್ತೂರೀತಿಲಕಾಂಕಿತಾಂ~।\\
ಕಾಮಕೋದಂಡಸೌಂದರ್ಯನಿರ್ಜಿತ ಭ್ರೂಲತಾಯುತಾಂ ॥

ಅಂಜನಾಂಚಿತನೇತ್ರಾಂ ತಾಂ ಪದ್ಮಪತ್ರ ನಿಭೇಕ್ಷಣಾಂ~।\\
ಮಣಿಕುಂಡಲಸಂಯುಕ್ತ ಕರ್ಣದ್ವಯ ವಿರಾಜಿತಾಂ ॥

ಮುಕ್ತಾಮಾಣಿಕ್ಯಖಚಿತ ನಾಸಿಕಾಭರಣಾನ್ವಿತಾಂ~।\\
ಶುದ್ಧಮುಕ್ತಾವಲೀಪ್ರಖ್ಯ ದಂತಪಂಕ್ತಿವಿರಾಜಿತಾಂ ॥

ಪಕ್ವಬಿಂಬ ಫಲಾಭಾಸಾಧರದ್ವಯ ರಾಜಿತಾಂ~।\\
ತಾಂಬೂಲಪೂರಿತಮುಖೀಂ ಸುಸ್ಮಿತಾಸ್ಯವಿರಾಜಿತಾಂ ॥

ಆದ್ಯಭೂಷಣಸಂಯುಕ್ತಾಂ ಹೇಮಚಿಂತಾಕಸಂಯುತಾಂ~।\\
ಪದಕೇನಸಮಾಯುಕ್ತಾಂ ಮಹಾಪದಕಸಂಯುತಾಂ ॥

ಮುಕ್ತಾವಲಿಸಮೋಪೇತಾಮೇಕಾವಲಿಸಮನ್ವಿತಾಂ~।\\
ಕೇಯೂರಾಂಗದ ಸಂಯುಕ್ತಾಂ ಚತುರ್ಬಾಹುವಿರಾಜಿತಾಂ॥

ಅಷ್ಟಗಂಧಸಮಾಯುಕ್ತ ಶ್ರೀಚಂದನವಿಲೇಪನಾಂ~।\\
ಹೇಮಕುಂಭಸಮಪ್ರಖ್ಯ ಸ್ತನದ್ವಯವಿರಾಜಿತಾಂ ॥

ರಕ್ತವಸ್ತ್ರಪರೀಧಾನಾಂ ರಕ್ತಕಂಚುಕಸಂಯುತಾಂ~।\\
ಸೂಕ್ಷ್ಮರೋಮಾವಲೀಯುಕ್ತ ತನುಮಧ್ಯವಿರಾಜಿತಾಂ ॥

ಮುಕ್ತಾಮಾಣಿಕ್ಯಖಚಿತ ಕಾಂಚೀಯುತನಿತಂಬಿನೀಂ~।\\
ಸದಾಶಿವಾಂಕಸ್ಥಪೃಥು ಮಹಾಜಘನಮಂಡಲಾಂ ॥

ಕದಲೀಸ್ತಂಭಸಂಕಾಶಸಕ್ಥಿ ದ್ವಯವಿರಾಜಿತಾಂ~।\\
ಕಾಹಲೀಕಾಂತಿವಿಲಸಜ್ಜಂಘಾ ಯುಗಲಶೋಭಿತಾಂ ॥

ಗೂಢಗುಲ್ಫದ್ವಯೋಪೇತಾಂ ರಕ್ತಪಾದದ್ವಯಾನ್ವಿತಾಂ~।\\
ಬ್ರಹ್ಮವಿಷ್ಣುಮಹಾದೇವಶಿರೋಮುಕುಟಜಾತಯಾ ॥

ಕಾಂತ್ಯಾ ವಿರಾಜಿತಪದಾಂ ಭಕ್ತತ್ರಾಣ ಪರಾಯಣಾಂ~।\\
ಇಕ್ಷುಕಾರ್ಮುಕ ಪುಷ್ಪೇಷು ಪಾಶಾಂಕುಶ ಧರಾಂ ಪರಾಂ~।\\
ಸಂವಿತ್ಸ್ವರೂಪಿಣೀಂ ದೇವೀಂ ಧ್ಯಾಯಾಮಿ ಪರಮೇಶ್ವರೀಂ ॥

೪ ಹ್ರೀಂಶ್ರೀಂಸೌಃ ಲಲಿತಾಯಾ ಅಮೃತಚೈತನ್ಯಮೂರ್ತಿಂ ಕಲ್ಪಯಾಮಿ ನಮಃ\\
೪ ಹ್‌ಸ್‌ರೈಂ ಹ್‌ಸ್‌ಕ್ಲ್ರೀಂ ಹ್‌ಸ್‌ರ್ಸೌಃ\\
\dhyana{ಮಹಾಪದ್ಮವನಾಂತಸ್ಥೇ ಕಾರಣಾನಂದವಿಗ್ರಹೇ~।\\
ಸರ್ವಭೂತಹಿತೇ ಮಾತಃ ಏಹ್ಯೇಹಿ ಪರಮೇಶ್ವರಿ ॥\\
ದೇವೇಶಿ ಭಕ್ತಸುಲಭೇ ಸರ್ವಾವರಣಸಂಯುತೇ~।\\
ಯಾವತ್ತ್ವಾಂ ಪೂಜಯಿಷ್ಯಾಮಿ ತಾವತ್ತ್ವಂ ಸುಸ್ಥಿರಾ ಭವ ॥}

ಸುಭಗೇ ನಮಃ~। ಆಂ ಸೋಹಂ~। ಆಂ ಹ್ರೀಂ ಕ್ರೋಂ ಯರಲವಶಷಸಹೋಂ ಶ್ರೀಲಲಿತಾಮಹಾತ್ರಿಪುರಸುಂದರ್ಯಾಃ ಪ್ರಾಣಾ ಇಹ ಪ್ರಾಣಾಃ~।\\ ಶ್ರೀಲಲಿತಾಮಹಾತ್ರಿಪುರಸುಂದರ್ಯಾಃ ಜೀವ ಇಹ ಸ್ಥಿತಃ~।\\ ಶ್ರೀಲಲಿತಾಮಹಾತ್ರಿಪುರಸುಂದರ್ಯಾಃ ಸರ್ವೇಂದ್ರಿಯಾಣಿ~।\\ಶ್ರೀಲಲಿತಾಮಹಾತ್ರಿಪುರಸುಂದರ್ಯಾಃ ವಾಙ್ಮನಸ್ತ್ವಕ್ಚಕ್ಷುಃ ಶ್ರೋತ್ರ\\ಜಿಹ್ವಾಘ್ರಾಣಪ್ರಾಣಾ ಇಹೈವಾಗತ್ಯ ಸುಖಂ ಚಿರಂ ತಿಷ್ಠಂತು ಸ್ವಾಹಾ ॥\\ಆವಾಹಿತಾ ಭವ~। ಸಂಸ್ಥಾಪಿತಾ ಭವ~। ಸನ್ನಿಹಿತಾ ಭವ~। ಸನ್ನಿರುದ್ಧಾ ಭವ~। ಸಮ್ಮುಖಾ ಭವ~। ಅವಗುಂಠಿತಾ ಭವ~। ವ್ಯಾಪ್ತಾ ಭವ~। ಸುಪ್ರಸನ್ನಾ ಭವ~।\\ ವರದಾ ಭವ ॥

\dhyana{ತತೋ ಧ್ಯಾತ್ವಾ ತ್ರಿಕೋಣಸ್ಥಂ ತ್ರಿಣೇತ್ರಂ ಚಂದ್ರಶೇಖರಂ ।\\
ಉಮಾಶೋಭಿತವಾಮಾಂಗಂ ದಿವ್ಯಗಂಧೇನ ಚರ್ಚಿತಂ ॥\\
ಬಾಹುಭಿಃ ಶೂಲಡಮರುವರದಾಭಯಶೋಭಿತಂ ।\\
ಚತುರ್ಭಿರನ್ವಿತಂ ಧ್ಯಾಯೇತ್ ಶಂಕರಂ ಲೋಕಶಂಕರಂ ॥

ವಿದ್ಯಾಬುದ್ಧಿಧನಾದಿಸರ್ವವಿಭವಪ್ರಾಪ್ತ್ಯೈ ಜಗನ್ಮಂಗಲಾಂ\\
ಸರ್ವಾಭಿಷ್ಟಫಲಪ್ರದಾಂ ಸ್ಮಿತಮುಖೀಂ ದಾರಿದ್ರ್ಯರೋಗಾಪಹಾಂ~।\\
ಕಾರುಣ್ಯಾಮೃತವರ್ಷಿನೇತ್ರಯುಗಲಾಂ ತಾಪತ್ರಯಧ್ವಂಸಿನೀಂ\\
ಭಕ್ತತ್ರಾಣಪರಾಯಣಾಂ ಭಗವತೀಂ ಧ್ಯಾಯಾಮಿ ಸರ್ವೇಶ್ವರೀಂ ॥ ಧ್ಯಾನಮ್ ॥}

ಅವಾಹಯಾಮಿದೇವೇಶ ಭವಂತು ನಿತ್ಯನಿರ್ಮಲಂ ।\\
ಗೃಹಾಣ ಸಹ ಪಾರ್ವತ್ಯಾ ತವ ಪೂಜಾಂ ಮಯಾ ಕೃತಾಂ ॥\\
ಆವಾಹಯಾಮಿ ಪುರತಃ ಸ್ಥಿತದಿವ್ಯಮೂರ್ತೌ \\ಹೃತ್ಸ್ಥಾಂ ಸಹಸ್ರದಳಮಾರ್ಗವಿನಿರ್ಗತಾಂ ತ್ವಾಂ~।\\
ಪೂಜಾವಸಾನಸಮಯಾವಧಿ ತಿಷ್ಠ ಮಾತಃ \\ಸಂಪ್ರಾರ್ಥಯಾಮಿ ಜಗದಂಬ ಹಿರಣ್ಯವರ್ಣೇ ॥\\
\as{ಸಹಸ್ರಶೀರ್ಷಾ ಪುರುಷಃ+++++ದಶಾಂಗುಲಂ॥\\
ಹಿರಣ್ಯವರ್ಣಾಂ ++++++++++ ಮ ಆವಹ ॥} ಆವಾಹನಮ್ ॥

ದುಕೂಲವಸ್ತ್ರಸಂಯುಕ್ತಂ ಹಂಸತೂಲೋಪಶೋಭಿತಂ।\\
ರಚಿತಂ ಮೌಕ್ತಿಕೈಃ ದಿವ್ಯೈಃ ಗೃಹ್ಯತಾಮುತ್ತಮಾಸನಂ ॥\\
ದಿವ್ಯಾಸನಂ ಮಣಿಮಯಂ ಪ್ರದದೇ ಜನನ್ಯೈ\\ ಚಿತ್ರೈಃ ಸುಚಿತ್ರಿತಮನರ್ಘದುಕೂಲರಮ್ಯಂ~।\\
ಸಂವಿಶ್ಯ ತತ್ರ ಪರಿಗೃಹ್ಣ ಮಯಾ ಸಪರ್ಯಾಂ\\ ತ್ವಂ ತಾಂ ಮ ಆವಹ ಸುರೇಶ್ವರಿ ಮೋಕ್ಷಲಕ್ಷ್ಮೀಂ ॥\\
\as{ಪುರುಷ ಏವೇದಂ+++++ ನಾ ತಿರೋಹತಿ॥\\
ತಾಂ ಮ+++++++++++ ಪುರುಷಾನಹಮ್॥}ಆಸನಮ್ ॥

ಜಟಾಜೂಟಲಸದ್ಗಂಗಾಜಾಹ್ನವೀಜಲಶೋಭಿತಂ ।\\
ಗಂಗಪುಷ್ಪಾಕ್ಷತೈಃ ಸಾರ್ಧಂ ಪಾದ್ಯಂ ಸ್ವೀಕುರು ಶಂಕರ ॥\\
ದೂರ್ವಾಕ್ಷತಾದಿಸಹಿತೈಃ ಶುಚಿಗಾಂಗತೋಯೈಃ\\ ಪಾದ್ಯಂ ಗೃಹಾಣ ಚರಣಾಬ್ಜಯುಗೇ ಪ್ರದತ್ತಂ~।\\
ಯತ್ತೋಯಧಾರಣಕುತೂಹಲಿನೋ ಗ್ರಹಾಸ್ತ್ವಾಂ \\ನಿತ್ಯಂ ಭ್ರಮಂತಿ ಪರಿತೋ ಹರಿದಶ್ವಪೂರ್ವಾಃ ॥\\
\as{ಏತಾವಾನಸ್ಯ+++++ ತ್ರಿಪಾದ+++++ಸ್ಯಾಮೃತಂ ದಿವಿ॥\\
ಅಶ್ವಪೂರ್ವಾಂ +++++++++ಜುಷತಾಮ್॥}ಪಾದ್ಯಮ್ ॥

ಅನರ್ಘ್ಯಪುಷ್ಪರತ್ನೌಘೈರರ್ಘ್ಯಮೇತತ್ಪ್ರಕಲ್ಪಿತಂ ।\\
ಗೃಹಾಣ ತ್ವಂ ಮಹಾದೇವ ಭವ ಭಕ್ತಜನಪ್ರಿಯ ॥\\
ಅರ್ಘ್ಯಂ ಪ್ರಸೂನಕಲಿತಂ ಕರಪಲ್ಲವೇಷು \\ಯಚ್ಛಾಮಿ ತೇ ಜನನಿ ಪಾಲಯ ಭಕ್ತಮೇನಂ~।\\
ದುಃಖಾಗ್ನಿತಪ್ತಹೃದಯಃ ಸದಯಾಂ ತ್ವದನ್ಯಾಂ\\ ಕಾಂ ಸೋಽಸ್ಮ್ಯಹಂ ಶರಣಮರ್ಥಯಿತುಂ ಸಮರ್ಥಃ ॥\\
\as{ತ್ರಿಪಾದೂರ್ದ್ಧ್ವ+++++-ಶನೇ ಅಭಿ ॥\\
ಕಾಂ ಸೋಸ್ಮಿತಾಂ +++++++++ ಶ್ರಿಯಮ್ ॥} ಅರ್ಘ್ಯಮ್ ॥

ಆಚಮ್ಯತಾಂ ಮಹಾದೇವ ಮಯಾ ಭಕ್ತ್ಯಾ ಪ್ರಕಲ್ಪಿತಂ ।\\
ಸರ್ವತೀರ್ಥಹ್ರದೈಃ ಸ್ವರ್ಣಪಾತ್ರಸ್ಥಿತೈಃ ಜಲೈಃ ಶುಚೈಃ ॥ \\
ಪಾತ್ರೇ ಸುವರ್ಣರಚಿತೇಂಬ ಸುರಾಪಗಾಯಾ \\ನಿಕ್ಷಿಪ್ಯ ಶುದ್ಧಜಲಮಾಚಮನಾಯ ಭಕ್ತ್ಯಾ~।\\
ಅಭ್ಯರ್ಥಯೇ ಭಗವತೀಂ ನಯನಸ್ಥಸೂರ್ಯಚಂದ್ರಾಂ\\ ಪ್ರಭಾವಲಯಮಧ್ಯಗತಾಮಹಂ ತ್ವಾಂ ॥\\
\as{ತಸ್ಮಾತ್ ವಿರಾಳ+++++ಪಶ್ಚಾತ್ ಭೂಮಿಮಥೋ ಪುರಃ ॥\\
ಚನ್ದ್ರಾಂ ಪ್ರಭಾಸಾಂ ++++++++ತ್ವಾಂ ವೃಣೇ ॥} ಆಚಮನಮ್ ॥

ದಧ್ಯಾಜ್ಯಮಧುಸಂಯುಕ್ತಂ ಕ್ಷೀರಂ ಶುದ್ಧಜಲೈರ್ಯುತಂ ।\\
ಮಧುಪರ್ಕಂ ಮಯಾ ದತ್ತ ಪ್ರೀತ್ಯಾ ಸ್ವೀಕುರು ಶಂಕರ ॥\\
ದಧ್ಯಾಜ್ಯ ಮಧು ಸಂಯುಕ್ತಂ ಶರ್ಕರಾಕ್ಷೀರಸಂಯುತಮ್~।\\ಮಧುಪರ್ಕಂ ಗೃಹಾಣ ತ್ವಂ ಅರ್ಪಯಾಮಿ ಶಿವಪ್ರಿಯೇ ॥
\as{ಮಧುವಾತಾ+++++ತ್ವೋಷಧೀಃ ॥} ಮಧುಪರ್ಕಃ ॥

ಓಂ ಐಂ ಹ್ರೀಂ ಶ್ರೀಂ ಶ್ರೀಮಲ್ಲಲಿತಾಮಹಾತ್ರಿಪುರಸುಂದರ್ಯೈ\\ ಪಾದ್ಯಂ ಕಲ್ಪಯಮಿ ನಮಃ\\
ಆಭರಣಾವರೋಪಣಂ ಕಲ್ಪಯಾಮಿ ನಮಃ~।\\
ಸುಗಂಧಿತೈಲಾಭ್ಯಂಗಂ ಕಲ್ಪಯಾಮಿ ನಮಃ~।\\
ಮಜ್ಜನಶಾಲಾಪ್ರವೇಶನಂ ಕಲ್ಪಯಾಮಿ ನಮಃ~।\\
ಮಜ್ಜನಶಾಲಾಸ್ಥ ಮಣಿಪೀಠೋಪವೇಶನಂ ಕಲ್ಪಯಾಮಿ ನಮಃ~।\\
ದಿವ್ಯಸ್ನಾನೀಯೋದ್ವರ್ತನಂ ಕಲ್ಪಯಾಮಿ ನಮಃ~।\\
ಉಷ್ಣೋದಕಸ್ನಾನಂ ಕಲ್ಪಯಾಮಿ ನಮಃ~।\\
ಕನಕ ಕಲಶಚ್ಯುತ ಸಕಲ ತೀರ್ಥಾಭಿಷೇಕಂ ಕಲ್ಪಯಾಮಿ ನಮಃ~॥

ಪೂರ್ವಂ ಪಂಚಾಮೃತೈಃ ಶುದ್ಧ್ಯೈ ಪಶ್ಚಾತ್ ಸ್ವಚ್ಛೋದಕೈಃ ಶುಚೈಃ ।\\
ಸ್ನಪಯಾಮಿ ಮಹಾದೇವ ಭವಂತು ನಿತ್ಯನಿರ್ಮಲಂ ॥\\
ಆಲೇಪನಾಯ ಸುಮತೈಲಮಿದಂ ಗೃಹೀಷ್ವ\\ ಚೋದ್ವರ್ತನಾಯ ಹರಿಚಂದನಕುಂಕುಮಾದೀನ್~।\\
ಸ್ನಾನಾಯ ಶಂಖಕಲಶಾಂಬುವಿಮಿಶ್ರಿತಾಸ್ತಾ\\ ಆಪೋ ಹಿ ತೇ ತನುಮಲಾನ್ಯಪಕರ್ಷಯಂತು ॥\\
\as{ಆಪೋ ಹಿಷ್ಠಾ++++ಚ ನಃ ॥} ಮಲಾಪಕರ್ಷಣಂ ॥

ದಧಿ ಕ್ಷೀರ ಘೃತೈರ್ಯುಕ್ತಂ ಶರ್ಕರಾ ಮಧುಮಿಶ್ರಿತಂ ।\\
ಪಂಚಾಮೃತಂ ಗೃಹಾಣ ತ್ವಂ ಕೃಪಯಾ ಪರಮೇಶ್ವರ  ॥\\
ಪಾತ್ರೇಷು ಪಂಚಸು ಪೃಥಕ್ಪೃಥಗಸ್ತಿ ಮಾತಃ\\ಗವ್ಯಂ ಪಯೋ ದಧಿ ಘೃತಂ ಮಧು ಶರ್ಕರಾ ಚ~।\\
ಖರ್ಜೂರಚೂತಕದಲೀಪ್ರಮುಖಂ ಫಲೌಘಂ\\ ಸ್ನಾನಾಯ ತೇ ಭವತು ಸರ್ವಮಿದಂ ಪುರಸ್ಥಂ ॥ಪಂಚಾಮೃತಸ್ನಾನಂ॥

ಕ್ಷೀರಸ್ನಾನಮ್ ॥\\
ಕಾಮಧೇನುಸಮುದ್ಭೂತಂ ಸರ್ವೇಷಾಂ ಜೀವನಂ ಪರಂ।\\
ಪಾವನಂ ಯಜ್ಞಹೇತುಶ್ಚ ಸ್ನಾನಾರ್ಥಂ ಪ್ರತಿಗೃಹ್ಯತಾಂ ॥ \\
ಸುರಭೇಸ್ತು ಸಮುತ್ಪನ್ನಂ ದೇವಾನಾಮಪಿ ದುರ್ಲಭಮ್~।\\ಪಯೋ ದದಾಮಿ ದೇವೇಶಿ ಸ್ನಾನಾರ್ಥಂ ಪ್ರತಿಗೃಹ್ಯತಾಮ್ ॥\\
\as{ಆಪ್ಯಾಯಸ್ವ++++ಸಂಗಥೇ ॥\\
ಸದ್ಯೋಜಾತಂ+++ ನಮಃ ॥}

ದಧಿಸ್ನಾನಮ್ ॥\\
ಪಯಸಾತು ಸಮುದ್ಭೂತಂ ಮಧುರಾಮ್ಲಶಶಿಪ್ರಭಂ ।\\
ದಧ್ಯಾನೀತಂ ಮಯಾ ದತ್ತಂ ಪ್ರೀತ್ಯಾ ಸ್ವೀಕುರು ಶಂಕರ ॥\\
ಚಂದ್ರಮಂಡಲ ಸಂಕಾಶಂ ಸರ್ವದೇವ ಪ್ರಿಯಂ ಹಿ ಯತ್~।\\ ದದಾಮಿ ದಧಿ ದೇವೇಶಿ ಸ್ನಾನಾರ್ಥಂ ಪ್ರತಿಗೃಹ್ಯತಾಮ್  ॥\\
\as{ದಧಿಕ್ರಾವ್ಣೋ+++++ ತಾರಿಷತ್ ॥\\
ವಾಮದೇವಾಯ ನಮೋ++++ ನಮಃ ॥}

ಘೃತಸ್ನಾನಮ್ ॥\\
ನವನೀತಸಮುತ್ಪನ್ನಂ ಆಯುರಾರೋಗ್ಯವರ್ಧನಂ ।\\
ಘೃತಂ ತುಭ್ಯಂ ಪ್ರದಾಸ್ಯಾಮಿ ಸ್ನಾನಾರ್ಥಂ ಪ್ರತಿಗೃಹ್ಯತಾಂ ॥\\
ಆಜ್ಯಂ ಸುರಾಣಾಮಾಹಾರ ಮಾಜ್ಯಂ ಯಜ್ಞೇ ಪ್ರತಿಷ್ಠಿತಮ್~। \\ ಆಜ್ಯಂ ಪವಿತ್ರಂ ಪರಮಂ ಸ್ನಾನಾರ್ಥಂ ಪ್ರತಿಗೃಹ್ಯತಾಮ್ ॥\\
\as{ಘೃತಂ ಮಿಮಿಕ್ಷೇ +++ಹವ್ಯಮ್ ॥\\
ಅಘೋರೇಭ್ಯೋಥ+++ರೂಪೇಭ್ಯಃ ॥}

ಮಧುಸ್ನಾನಮ್ ॥\\
ತರುಪುಷ್ಪಸಮಾಕೃಷ್ಟಂ ಸುಸ್ವಾದು ಮಧುರಂ ಮಧು ।\\
ತೇಜಃಪುಷ್ಟಿಕರಂ ದಿವ್ಯಂ ಸ್ನಾನಾರ್ಥಂ ಪ್ರತಿಗೃಹ್ಯತಾಂ ॥ \\
ಸರ್ವೌಷಧಿ ಸಮುತ್ಪನ್ನಂ ಪೀಯೂಷಸದೃಶಂ ಮಧು~।\\ ಸ್ನಾನಾರ್ಥಂ ತೇ ಪ್ರದಾಸ್ಯಾಮಿ ಗೃಹಾಣ ಪರಮೇಶ್ವರಿ ॥\\
\as{ಮಧುವಾತಾ ಋತಾಯತೇ ++++ ಸಂತ್ವೋಷಧೀಃ ॥\\
ತತ್ಪುರುಷಾಯ ವಿದ್ಮಹೇ++++ಪ್ರಚೋದಯಾತ್ ॥}

ಶರ್ಕರಾಸ್ನಾನಮ್ ॥\\
ಇಕ್ಷಸಾರಸಮುದ್ಭೂತಾ ಶರ್ಕರಾ ಪುಷ್ಟಿಕಾರಿಕಾ ।\\
ಮಲಾಪಹಾರಿಕಾ ದಿವ್ಯಾ ಸ್ನಾನಾರ್ಥಂ ಪ್ರತಿಗೃಹ್ಯತಾಂ ॥\\
ಇಕ್ಷುದಂಡಾತ್ ಸಮುತ್ಪನ್ನಾ ರಸ್ಯಾ ಸ್ನಿಗ್ಧತರಾ ಶುಭಾ~।\\ಶರ್ಕರೇಯಂ ಮಯಾ ದತ್ತಾ ಸ್ನಾನಾರ್ಥಂ ಪ್ರತಿಗೃಹ್ಯತಾಮ್ ॥\\
\as{ಸ್ವಾದುಃ ಪವಸ್ವ++++ಅದಾಭ್ಯಃ ॥\\
ಈಶಾನಃ ಸರ್ವವಿದ್ಯಾನಾಂ+++ ಸದಾಶಿವೋಮ್ ॥}

 ಫಲಸ್ನಾನಮ್॥\\
ಸರ್ವಸಾರಸಮುದ್ಭೂತಂ ಶಕ್ತಿಪುಷ್ಟಿಕರಂ ದೃಢಂ ।\\
ಸುಫಲಂ ಕಾರ್ಯಸಿದ್ಧ್ಯರ್ಥಂ ಸ್ನಾನಾರ್ಥಂ ಪ್ರತಿಗೃಹ್ಯತಾಂ ॥\\
ಸುಫಲೈಶ್ಚ ಫಲೋದೈರ್ವಾ ಫಲಾನಾಂ ಚ ರಸೈರ್ಯುತಮ್~।\\ ಸ್ನಾನಂ ಸಮರ್ಪಿತಂ ಭಕ್ತ್ಯಾ ಗೃಹ್ಯತಾಂ ಪರಮೇಶ್ವರಿ ॥\\
\as{ಯಾಃ ಫಲಿನೀ+++++ತ್ವಂ ಹಸಃ ॥\\
ಕದ್ರುದ್ರಾಯ +++++ಹೃದೇ ॥}

ಮಲಯಾಚಲಸಂಭೂತಂ ಸುಗಂಧಂ ಶೀತಲಂ ಶುಭಂ ।\\
ಸುಕಾಂತಿದಾಯಕಂ ದಿವ್ಯಂ ಸ್ನಾನಾರ್ಥಂ ಪ್ರತಿಗೃಹ್ಯತಾಂ ॥\\
ಗಂಗಾದಿ ಸಲಿಲಂ ಶುದ್ಧಂ ಸುವರ್ಣ ಕಲಶೇ ಸ್ಥಿತಮ್~।\\ ಸುವಾಸಿತಂ ಸುಗಂಧೇನ ಸ್ನಾನಾರ್ಥಂ ಪ್ರತಿಗೃಹ್ಯತಾಮ್ ॥\\
\as{ಗಂಧದ್ವಾರಾಂ++++ಶ್ರಿಯಮ್ ॥} ಗಂಧೋದಕಸ್ನಾನಮ್ ॥

ಹೇಲಾಸೃಷ್ಟಾಂಡಕೋಟಿಸ್ತ್ವಂ ಹರಕಾಂತೇ ಹರಿಸ್ತುತೇ~।\\ ಹರಿದ್ರೋದಕ ಧಾರಾಭಿಃ ಸ್ನಪಯಾಮಿ ಮಹಾಶಿವೇ ॥\\
\as{ಹಾ॒ರಿ॒ದ್ರ॒ವೇವ॑ ಪತಥೋ॒ ವನೇದುಪ॒ ಸೋಮಂ᳚ ಸು॒ತಂ ಮ॑ಹಿ॒ಷೇವಾವ॑ ಗಚ್ಛಥಃ ।\\ ಸ॒ಜೋಷ॑ಸಾ ಉ॒ಷಸಾ॒ ಸೂರ್ಯೇ᳚ಣ ಚ॒  ತ್ರಿರ್ವ॒ರ್ತಿರ್ಯಾ᳚ತಮಶ್ವಿನಾ ॥}\\
\as{ಅಂಬಿತಮೇ ನದೀತಮೇ+++++ನಸ್ಕೃಧಿ ॥} ಹರಿದ್ರೋದಕಸ್ನಾನಮ್ ॥

ಕುಂದಾಭ ದಂತರುಚಿರೇ ಕಂಜಾಭ ರುಚಿರೇಕ್ಷಣೇ~।\\ಕುಂಕುಮೋದಕ ಧಾರಾಂ ತೇ ಕಲ್ಪಯಾಮಿ ಶಿವಪ್ರಿಯೇ ॥\\
\as{ಪ್ರಣೋ ದೇವೀ++++++ ಮವಿತ್ರ್ಯವತು ॥}ಕುಂಕುಮೋದಕಸ್ನಾನಮ್ ॥

ಲತಾವೃಕ್ಷಸಮುತ್ಪನ್ನಂ ಸುಗಂಧಂ ಶೋಭನಂ ಪರಂ ।\\
ಸಂತೋಷವರ್ಧನಂ ನಿತ್ಯಂ ಸ್ನಾನಾರ್ಥಂ ಪ್ರತಿಗೃಹ್ಯತಾಂ ॥ \\
ಬೃಹತೀಕರವೀರಾದಿ ಪುಷ್ಪಸಂಚಯ ವಾಸಿತಮ್~।\\ಪುಷ್ಪೋದಕಂ ಪ್ರದಾಸ್ಯಾಮಿ ಗೃಹಾಣ ಪರಮೇಶ್ವರಿ ॥\\
\as{ಆಯನೇತೇ ++++++ ಇಮೇ ॥}ಪುಷ್ಪೋದಕಸ್ನಾನಮ್ ॥

ಸ್ವರ್ಣಕುಂಭಮುಖೋತ್ಸೃಷ್ಟ ಗಾಂಗವಾರಿಭಿರಾಪ್ಲುತೇ।\\ ಸುವರ್ಣೋದಕಧಾರಾಭಿ ರಭಿಷಿಕ್ತಾ ಪ್ರಸೀದ ಮೇ ॥\\
\as{ಹಿರಣ್ಯರೂಪಃ++++ತ್ಯನ್ನಮಸ್ಮೈ ॥}ಸುವರ್ಣೋದಕಸ್ನಾನಮ್॥

ಅಕ್ಷತಂ ತಂಡುಲಂ ಶುಭ್ರಂ ಹರಿದ್ರಯಾ ಸುಮಿಶ್ರಿತಂ ।\\
ಮಂಗಲಂ ಮಂಗಲಾರ್ಥಂ ಚ ಸ್ನಾನಾರ್ಥಂ ಪ್ರತಿಗೃಹ್ಯತಾಂ ॥\\
ಅಕ್ಷತೈಶ್ವರ್ಯವೃದ್ಧ್ಯರ್ಥಂ ಇಕ್ಷುಚಾಪಧರೇಽಂಬಿಕೇ~।\\ಅಕ್ಷತೋದಕಧಾರಾಭಿಃ ಸ್ನಪಯಾಮಿ ಪ್ರಸೀದ ಮೇ ॥\\
\as{ಉಪಾಸ್ಮೈ ಗಾಯತಾ++++ಇಯಕ್ಷತೇ ॥} ಅಕ್ಷತೋದಕಸ್ನಾನಮ್ ॥

ನಮಕಾದಿಸುಮಂತ್ರೌಘೈಃ ತ್ವದೀಯೈಃ ಪಾರ್ವತೀಪತೇ ।\\
ಮುದಾ ತ್ವಾಂ ಸ್ನಪಯಿಷ್ಯಾಮಿ ದುರಿತಂ ಮೇ ವಿನಾಶಯ ॥

\section{ರುದ್ರಪ್ರಶ್ನಃ}
\addcontentsline{toc}{section}{ರುದ್ರಪ್ರಶ್ನಃ}
ಶಂಭವೇ ನಮಃ~। ಓಂ ನಮೋ ಭಗವತೇ ರುದ್ರಾಯ॥\\
ಓಂ ನಮಸ್ತೇ ರುದ್ರ ಮನ್ಯವ ಉತೋತ ಇಷವೇ ನಮಃ॥\\
ನಮಸ್ತೇ ಅಸ್ತು ಧನ್ವನೇ ಬಾಹುಭ್ಯಾಮುತ ತೇ ನಮಃ॥\\
ಯಾ ತ ಇಷುಃ ಶಿವತಮಾ ಶಿವಂ ಬಭೂವ ತೇ ಧನುಃ॥\\
ಶಿವ ಶರವ್ಯಾ ಯಾ ತವ ತಯಾ ನೋ ರುದ್ರ ಮೃಡಯ ॥\\
ಯಾ ತೇ ರುದ್ರ ಶಿವಾ ತನೂರಘೋರಾ ಪಾಪಕಾಶಿನೀ॥\\
ತಯಾನಸ್ತನುವಾ ಶಂತಮಯಾ ಗಿರಿಶಂತಾಭಿಚಾಕಶೀಹಿ॥\\
ಯಾಮಿಷುಂ ಗಿರಿಶಂತ ಹಸ್ತೇಬಿಭರ್ಷ್ಯಸ್ತವೇ॥\\
ಶಿವಾಂ ಗಿರಿತ್ರ ತಾಂ ಕುರು ಮಾ ಹಿಗ್ಂಸೀಃ ಪುರುಷಂ ಜಗತ್॥\\
ಶಿವೇನ ವಚಸಾ ತ್ವಾ ಗಿರಿಶಾಚ್ಛಾವದಾಮಸಿ॥\\
ಯಥಾ ನಃ ಸರ್ವಮಿಜ್ಜಗದಯಕ್ಷ್ಮಗ್ಂ ಸುಮನಾ ಅಸತ್॥\\
ಅಧ್ಯವೋಚದಧಿವಕ್ತಾ  ಪ್ರಥಮೋ ದೈವ್ಯೋ ಭಿಷಕ್॥\\
ಅಹೀಗ್ ಶ್ಚ  ಸರ್ವಾನ್ಜಂಭಯಂಥ್ಸರ್ವಾಶ್ಚ ಯಾತುಧಾನ್ಯಃ॥\\
ಅಸೌ ಯಸ್ತಾಮ್ರೋ ಅರುಣ ಉತ ಬಭ್ರುಃ ಸುಮಂಗಲಃ॥\\
ಯೇ ಚೇಮಾಗ್ಂ ರುದ್ರಾ ಅಭಿತೋ ದಿಕ್ಷು\\ ಶ್ರಿತಾ ಸ್ಸಹಸ್ರಶೋವೈಷಾಗ್ಂ ಹೇಡ ಈಮಹೇ॥\\
ಅಸೌ ಯೋವಸರ್ಪತಿ ನೀಲಗ್ರೀವೋ ವಿಲೋಹಿತಃ॥\\
ಉತೈನಂ ಗೋಪಾ ಅದೃಶನ್ನದೃಶನ್ನುದಹಾರ್ಯಃ॥\\
ಉತೈನಂ ವಿಶ್ವಾ  ಭೂತಾನಿ ಸ ದೃಷ್ಟೋ ಮೃಡಯಾತಿ ನಃ॥\\
ನಮೋ ಅಸ್ತು ನೀಲಗ್ರೀವಾಯ ಸಹಸ್ರಾಕ್ಷಾಯ ಮೀಢುಷೇ॥\\
ಅಥೋಯೇ ಅಸ್ಯ ಸತ್ವಾನೋಹಂ ತೇಭ್ಯೋಕರನ್ನಮಃ॥\\
ಪ್ರಮುಂಚ ಧನ್ವನಸ್ತ್ವಮುಭಯೋರಾರ್ತ್ನಿಯೋರ್ಜ್ಯಾಂ॥\\
ಯಾಶ್ಚ ತೇ ಹಸ್ತ ಇಷವಃ ಪರಾ ತಾ ಭಗವೋ ವಪ॥\\
ಅವತತ್ಯ ಧನುಸ್ತ್ವಗ್ಂ ಸಹಸ್ರಾಕ್ಷ ಶತೇಷುಧೇ॥\\
ನಿಶೀರ್ಯ ಶಲ್ಯಾನಾಂ ಮುಖಾ ಶಿವೋ ನಃ ಸುಮನಾ ಭವ॥\\
ವಿಜ್ಯಂ ಧನುಃ ಕಪರ್ದಿನೋ ವಿಶಲ್ಯೋ ಬಾಣವಾಗ್ಂ ಉತ॥\\
ಅನೇಶನ್ನಸ್ಯೇಷವ ಆಭುರಸ್ಯ ನಿಷಂಗಥಿಃ॥\\
ಯಾ ತೇ ಹೇತಿರ್ಮೀಢುಷ್ಟಮ ಹಸ್ತೇ ಬಭೂವ ತೇ ಧನುಃ॥\\
ತಯಾಸ್ಮಾನ್, ವಿಶ್ವತಸ್ತ್ವಮಯಕ್ಷ್ಮಯಾ ಪರಿಬ್ಭುಜ॥\\
ನಮಸ್ತೇ ಅಸ್ತ್ವಾಯುಧಾಯಾನಾತತಾಯ ಧೃಷ್ಣವೇ॥\\
ಉಭಾಭ್ಯಾಮುತ ತೇ ನಮೋ ಬಾಹುಭ್ಯಾಂ ತವ ಧನ್ವನೇ॥\\
ಪರಿ ತೇ ಧನ್ವನೋ ಹೇತಿರಸ್ಮಾನ್ವೃಣಕ್ತು ವಿಶ್ವತಃ॥\\
ಅಥೋ ಯ ಇಷುಧಿಸ್ತವಾರೇ ಅಸ್ಮನ್ನಿಧೇಹಿ ತಂ॥

ನಮಸ್ತೇ ಅಸ್ತು ಭಗವನ್ವಿಶ್ವೇಶ್ವರಾಯ ಮಹಾದೇವಾಯ ತ್ರ್ಯಂಬಕಾಯ ತ್ರಿಪುರಾಂತಕಾಯ  ತ್ರಿಕಾಗ್ನಿಕಾಲಾಯ ಕಾಲಾಗ್ನಿರುದ್ರಾಯ ನೀಲಕಂಠಾಯ ಮೃತ್ಯುಂಜಯಾಯ ಸರ್ವೇಶ್ವರಾಯ ಸದಾಶಿವಾಯ ಶ್ರೀ ಮನ್ಮಹಾದೇವಾಯ ನಮಃ॥

ನಮೋ ಹಿರಣ್ಯಬಾಹವೇ ಸೇನಾನ್ಯೇ ದಿಶಾಂ ಚ ಪತಯೇ ನಮೋ ನಮೋ ವೃಕ್ಷೇಭ್ಯೋ ಹರಿಕೇಶೇಭ್ಯಃ ಪಶೂನಾಂ ಪತಯೇ ನಮೋ ನಮಃ ಸಸ್ಪಿಂಜರಾಯ ತ್ವಿಷೀಮತೇ ಪಥೀನಾಂ ಪತಯೇ ನಮೋ ನಮೋ ಬಭ್ಲುಶಾಯ ವಿವ್ಯಾಧಿನೇನ್ನಾನಾಂ ಪತಯೇ ನಮೋ ನಮೋ ಹರಿಕೇಶಾಯೋಪವೀತಿನೇ ಪುಷ್ಟಾನಾಂ ನಮೋ ನಮೋ ಭವಸ್ಯ ಹೇತ್ಯೈ ಜಗತಾಂ ಪತಯೇ ನಮೋ ನಮೋ ರುದ್ರಾಯಾತತಾವಿನೇ ಕ್ಷೇತ್ರಾಣಾಂ ಪತಯೇ ನಮೋ ನಮಃ ಸೂತಾಯ ಹಂತ್ಯಾಯ ವನಾನಾಂ ಪತಯೇ ನಮೋ ನಮೋ ರೋಹಿತಾಯ ಸ್ಥಪತಯೇ ವೃಕ್ಷಾಣಾಂ ಪತಯೇ ನಮೋ ನಮೋ ಮಂತ್ರಿಣೇ ವಾಣಿಜಾಯ ಕಕ್ಷಾಣಾಂ ಪತಯೇ ನಮೋ ನಮೋ ಭುವಂತಯೇ ವಾರಿವಸ್ಕೃತಾಯೌಷಧೀನಾಂ ಪತಯೇ ನಮೋ ನಮ ಉಚ್ಚೈರ್ಘೋಷಾಯಾಕ್ರಂದಯತೇ ಪತ್ತೀನಾಂ ಪತಯೇ ನಮೋ ನಮಃ ಕೃತ್ಸ್ನ ವೀತಾಯ ಧಾವತೇ ಸತ್ವನಾಂ ಪತಯೇ ನಮಃ॥

ನಮಃ ಸಹಮಾನಾಯ ನಿವ್ಯಾಧಿನ ಆವ್ಯಾಧಿನೀನಾಂ ಪತಯೇ ನಮೋ ನಮಃ ಕಕುಭಾಯ ನಿಷಂಗಿಣೇ ಸ್ತೇನಾನಾಂ ಪತಯೇ ನಮೋ ನಮೋ ನಿಷಂಗಿಣ ಇಷುಧಿಮತೇ ತಸ್ಕರಾಣಾಂ ಪತಯೇ ನಮೋ ನಮೋ ವಂಚತೇ ಪರಿವಂಚತೇ ಸ್ತಾಯೂನಾಂ ಪತಯೇ ನಮೋ ನಮೋ ನಿಚೇರವೇ ಪರಿಚರಾಯಾರಣ್ಯಾನಾಂ ಪತಯೇ ನಮೋ ನಮಃ ಸೃಕಾವಿಭ್ಯೋ ಜಿಘಾಗ್ಂಸದ್ಭ್ಯೋ ಮುಷ್ಣತಾಂ ಪತಯೇ ನಮೋ ನಮೋಸಿಮದ್ಭ್ಯೋ ನಕ್ತಂಚರದ್ಭ್ಯಃ ಪ್ರಕೃಂತಾನಾಂ ಪತಯೇ ನಮೋ ನಮ ಉಷ್ಣೀಷಿಣೇ ಗಿರಿಚರಾಯ ಕುಲುಂಚಾನಾಂ ಪತಯೇ ನಮೋ ನಮ ಇಷುಮದ್ಭ್ಯೋ ಧನ್ವಾವಿಭ್ಯಶ್ಚ ವೋ ನಮೋ ನಮ ಆತನ್ವಾನೇಭ್ಯಃ  ಪ್ರತಿದಧಾನೇಭ್ಯಶ್ಚ ವೋ ನಮೋ ನಮ ಆಯಚ್ಛದ್ಭ್ಯೋ ವಿಸೃಜದ್ಭ್ಯಶ್ಚ ವೋ ನಮೋ ನಮೋಸ್ಯದ್ಭ್ಯೋ ವಿಧ್ಯದ್ಭ್ಯಶ್ಚ ವೋ ನಮೋ ನಮ ಆಸೀನೇಭ್ಯಃ ಶಯಾನೇಭ್ಯಶ್ಚ ವೋ ನಮೋ ನಮಃ ಸ್ವಪದ್ಭ್ಯೋ ಜಾಗ್ರದ್ಭ್ಯಶ್ಚ ವೋ ನಮೋ ನಮಸ್ತಿಷ್ಠದ್ಭ್ಯೋ ಧಾವದ್ಭ್ಯಶ್ಚ ವೋ ನಮೋ ನಮಃ ಸಭಾಭ್ಯಃ ಸಭಾಪತಿಭ್ಯಶ್ಚ ವೋ ನಮೋ ನಮೋ ಅಶ್ವೇಭ್ಯೋಶ್ವಪತಿಭ್ಯಶ್ಚ ವೋ ನಮಃ॥

ನಮ ಆವ್ಯಾಧಿನೀಭ್ಯೋ ವಿವಿಧ್ಯಂತೀಭ್ಯಶ್ಚ ವೋ ನಮೋ ನಮ ಉಗಣಾಭ್ಯಸ್ತೃಗ್ಂಹತೀಭ್ಯಶ್ಚ ವೋ ನಮೋ ನಮೋ ಗೃಥ್ಸೇಭ್ಯೋ ಗೃಥ್ಸಪತಿಭ್ಯಶ್ಚ ವೋ ನಮೋ ನಮೋ ವ್ರಾತೇಭ್ಯೋ ವ್ರಾತಪತಿಭ್ಯಶ್ಚ ವೋ ನಮೋ ನಮೋ ಗಣೇಭ್ಯೋ ಗಣಪತಿಭ್ಯಶ್ಚ ವೋ ನಮೋ ನಮೋ ವಿರೂಪೇಭ್ಯೋ ವಿಶ್ವರೂಪೇಭ್ಯಶ್ಚ ವೋ ನಮೋ ನಮೋ ಮಹದ್ಭ್ಯಃ, ಕ್ಷುಲ್ಲಕೇಭ್ಯಶ್ಚ ವೋ ನಮೋ ನಮೋ ರಥಿಭ್ಯೋ ರಥೇಭ್ಯಶ್ಚ ವೋ ನಮೋ ನಮೋ ರಥೇಭ್ಯೋ ರಥಪತಿಭ್ಯಶ್ಚ ವೋ ನಮೋ ನಮಃ ಸೇನಾಭ್ಯಃ ಸೇನಾನಿಭ್ಯಶ್ಚ ವೋ ನಮೋ ನಮಃ, ಕ್ಷತ್ತೃಭ್ಯಃ ಸಂಗೃಹೀತೃಭ್ಯಶ್ಚ ವೋ ನಮೋ ನಮಸ್ತಕ್ಷಭ್ಯೋ ರಥಕಾರೇಭ್ಯಶ್ಚ ವೋ ನಮೋ ನಮಃ ಕುಲಾಲೇಭ್ಯಃ ಕರ್ಮಾರೇಭ್ಯಶ್ಚ ವೋ ನಮೋ ನಮಃ ಪುಂಜಿಷ್ಟೇಭ್ಯೋ ನಿಷಾದೇಭ್ಯಶ್ಚ ವೋ ನಮೋ ನಮ ಇಷುಕೃದ್ಭ್ಯೋ ಧನ್ವಕೃದ್ಭ್ಯಶ್ಚ ವೋ ನಮೋ ನಮೋ ಮೃಗಯುಭ್ಯಃ ಶ್ವನಿಭ್ಯಶ್ಚ ವೋ ನಮೋ ನಮಃ ಶ್ವಭ್ಯಃ ಶ್ವಪತಿಭ್ಯಶ್ಚ ವೋ ನಮಃ॥

ನಮೋ ಭವಾಯ ಚ ರುದ್ರಾಯ ಚ ನಮಃ ಶರ್ವಾಯ ಚ ಪಶುಪತಯೇ ಚ ನಮೋ ನೀಲಗ್ರೀವಾಯ ಚ ಶಿತಿಕಂಠಾಯ ಚ ನಮಃ ಕಪರ್ದಿನೇ ಚ ವ್ಯುಪ್ತಕೇಶಾಯ ಚ ನಮಃ ಸಹಸ್ರಾಕ್ಷಾಯ ಚ ಶತಧನ್ವನೇ ಚ ನಮೋ ಗಿರಿಶಾಯ ಚ ಶಿಪಿವಿಷ್ಟಾಯ ಚ ನಮೋ ಮೀಢುಷ್ಟಮಾಯ ಚೇಷುಮತೇ ಚ ನಮೋ ಹ್ರಸ್ವಾಯ ಚ ವಾಮನಾಯ ಚ ನಮೋ ಬೃಹತೇ ಚ ವರ್ಷೀಯಸೇ ಚ ನಮೋ ವೃದ್ಧಾಯ ಚ ಸಂವೃದ್ಧ್ವನೇ ಚ ನಮೋ ಅಗ್ರಿಯಾಯ ಚ ಪ್ರಥಮಾಯ ಚ ನಮ ಆಶವೇ ಚಾಜಿರಾಯ ಚ ನಮಃ ಶೀಘ್ರಿಯಾಯ ಚ ಶೀಭ್ಯಾಯ ಚ ನಮ ಊರ್ಮ್ಯಾಯ ಚಾವಸ್ವನ್ಯಾಯ ಚ ನಮಃ  ಸ್ರೋತಸ್ಯಾಯ ಚ ದ್ವೀಪ್ಯಾಯ ಚ ॥

ನಮೋ ಜ್ಯೇಷ್ಠಾಯ ಚ ಕನಿಷ್ಠಾಯ ಚ ನಮಃ ಪೂರ್ವಜಾಯ ಚಾಪರಜಾಯ ಚ ನಮೋ ಮಧ್ಯಮಾಯ ಚಾಪಗಲ್ಭಾಯ ಚ ನಮೋ ಜಘನ್ಯಾಯ ಚ ಬುಧ್ನಿಯಾಯ ಚ ನಮಃ ಸೋಭ್ಯಾಯ ಚ ಪ್ರತಿಸರ್ಯಾಯ ಚ ನಮೋ ಯಾಮ್ಯಾಯ ಚ ಕ್ಷೇಮ್ಯಾಯ ಚ ನಮ ಉರ್ವರ್ಯಾಯ ಚ ಖಲ್ಯಾಯ ಚ ನಮಃ ಶ್ಲೋಕ್ಯಾಯ ಚಾವಸಾನ್ಯಾಯ ಚ ನಮೋ ವನ್ಯಾಯ ಚ ಕಕ್ಷ್ಯಾಯ ಚ ನಮಃ ಶ್ರವಾಯ ಚ ಪ್ರತಿಶ್ರವಾಯ ಚ ನಮ ಆಶುಷೇಣಾಯ ಚಾಶುರಥಾಯ ಚ ನಮಃ ಶೂರಾಯ ಚಾವಭಿಂದತೇ ಚ ನಮೋ ವರ್ಮಿಣೇ ಚ ವರೂಥಿನೇ ಚ ನಮೋ ಬಿಲ್ಮಿನೇ ಚ ಕವಚಿನೇ ಚ ನಮಃ ಶ್ರುತಾಯ ಚ ಶ್ರುತಸೇನಾಯ ಚ॥

ನಮೋ ದುಂದುಭ್ಯಾಯ ಚಾಹನನ್ಯಾಯ ಚ ನಮೋ ಧೃಷ್ಣವೇ ಚ ಪ್ರಮೃಶಾಯ ಚ ನಮೋ ದೂತಾಯ ಚ ಪ್ರಹಿತಾಯ ಚ ನಮೋ ನಿಷಂಗಿಣೇ ಚೇಷುಧಿಮತೇ ಚ ನಮಸ್ತೀಕ್ಷ್ಣೇಷವೇ ಚಾಯುಧಿನೇ ಚ ನಮಃ ಸ್ವಾಯುಧಾಯ ಚ ಸುಧನ್ವನೇ ಚ ನಮಃ ಸ್ರುತ್ಯಾಯ ಚ ಪಥ್ಯಾಯ ಚ ನಮಃ ಕಾಟ್ಯಾಯ ಚ ನೀಪ್ಯಾಯ ಚ ನಮಃ ಸೂದ್ಯಾಯ ಚ ಸರಸ್ಯಾಯ ಚ ನಮೋ ನಾದ್ಯಾಯ ಚ ವೈಶಂತಾಯ ಚ ನಮಃ ಕೂಪ್ಯಾಯ ಚಾವಟ್ಯಾಯ ಚ ನಮೋ ವರ್ಷ್ಯಾಯ ಚಾವರ್ಷ್ಯಾಯ ಚ ನಮೋ ಮೇಘ್ಯಾಯ ಚ ವಿದ್ಯುತ್ಯಾಯ ಚ ನಮ ಈಧ್ರಿಯಾಯ ಚಾತಪ್ಯಾಯ ಚ ನಮೋ ವಾತ್ಯಾಯ ಚ ರೇಷ್ಮಿಯಾಯ ಚ ನಮೋ ವಾಸ್ತವ್ಯಾಯ ಚ ವಾಸ್ತುಪಾಯ ಚ ॥

ನಮಃ ಸೋಮಾಯ ಚ ರುದ್ರಾಯ ಚ ನಮಸ್ತಾಮ್ರಾಯ ಚಾರುಣಾಯ ಚ ನಮಃ ಶಂಗಾಯ ಚ ಪಶುಪತಯೇ ಚ ನಮ ಉಗ್ರಾಯ ಚ ಭೀಮಾಯ ಚ ನಮೋ ಅಗ್ರೇವಧಾಯ ಚ ದೂರೇವಧಾಯ ಚ ನಮೋ ಹಂತ್ರೇ ಚ ಹನೀಯಸೇ ಚ ನಮೋ ವೃಕ್ಷೇಭ್ಯೋ ಹರಿಕೇಶೇಭ್ಯೋ ನಮಸ್ತಾರಾಯ ನಮಃ ಶಂಭವೇ ಚ ಮಯೋ ಭವೇ ಚ ನಮಃ ಶಂಕರಾಯ ಚ ಮಯಸ್ಕರಾಯ ಚ ನಮಃ ಶಿವಾಯ ಚ ಶಿವತರಾಯ ಚ ನಮಸ್ತೀರ್ಥ್ಯಾಯ ಚ ಕೂಲ್ಯಾಯ ಚ ನಮಃ ಪಾರ್ಯಾಯ ಚಾವಾರ್ಯಾಯ ಚ ನಮಃ ಪ್ರತರಣಾಯ ಚೋತ್ತರಣಾಯ ಚ ನಮ ಆತಾರ್ಯಾಯ ಚಾಲಾದ್ಯಾಯ ಚ ನಮಃ ಶಷ್ಪ್ಯಾಯ ಚ ಫೇನ್ಯಾಯ ಚ ನಮಃ ಸಿಕತ್ಯಾಯ ಚ ಪ್ರವಾಹ್ಯಾಯ ಚ॥

ನಮ ಇರಿಣ್ಯಾಯ ಚ ಪ್ರಪಥ್ಯಾಯ ಚ ನಮಃ ಕಿಗ್ಂಶಿಲಾಯ ಚ ಕ್ಷಯಣಾಯ ಚ ನಮಃ ಕಪರ್ದಿನೇ ಚ ಪುಲಸ್ತಯೇ ಚ ನಮೋ ಗೋಷ್ಠ್ಯಾಯ ಚ ಗೃಹ್ಯಾಯ ಚ ನಮಸ್ತಲ್ಪ್ಯಾಯ ಚ ಗೇಹ್ಯಾಯ ಚ ನಮಃ ಕಾಟ್ಯಾಯ ಚ ಗಹ್ವರೇಷ್ಠಾಯ ಚ ನಮೋ ಹ್ರದಯ್ಯಾಯ ಚ ನಿವೇಷ್ಪ್ಯಾಯ ಚ ನಮಃ ಪಾಗ್ಂಸವ್ಯಾಯ ಚ ರಜಸ್ಯಾಯ ಚ ನಮಃ ಶುಷ್ಕ್ಯಾಯ ಚ ಹರಿತ್ಯಾಯ ಚ ನಮೋ ಲೋಪ್ಯಾಯ ಚೋಲಪ್ಯಾಯ ಚ ನಮ ಊರ್ವ್ಯಾಯ ಚ ಸೂರ್ಮ್ಯಾಯ ಚ ನಮಃ ಪರ್ಣ್ಯಾಯ ಚ ಪರ್ಣಶದ್ಯಾಯ ಚ ನಮೋಪಗುರಮಾಣಾಯ ಚಾಭಿಘ್ನತೇ ಚ ನಮ ಆಖ್ಖಿದತೇ ಚ ಪ್ರಖ್ಖಿದತೇ ಚ ನಮೋ ವಃ ಕಿರಿಕೇಭ್ಯೋ ದೇವಾನಾಗ್ಂ ಹೃದಯೇಭ್ಯೋ ನಮೋ ವಿಕ್ಷೀಣಕೇಭ್ಯೋ ನಮೋ ವಿಚಿನ್ವತ್ಕೇಭ್ಯೋ ನಮ ಆನಿರ್ಹತೇಭ್ಯೋ ನಮ ಆಮೀವತ್ಕೇಭ್ಯಃ ॥

ದ್ರಾಪೇ ಅಂಧಸಸ್ಪತೇ ದರಿದ್ರನ್ನೀಲಲೋಹಿತ~। ಏಷಾಂ ಪುರುಷಾಣಾಮೇಷಾಂ ಪಶೂನಾಂ ಮಾ ಭೇರ್ಮಾರೋ ಮೋ ಏಷಾಂ ಕಿಂಚನಾಮಮತ್~। ಯಾ ತೇ ರುದ್ರ ಶಿವಾ ತನೂಃ ಶಿವಾ ವಿಶ್ವಾಹಭೇಷಜೀ~। ಶಿವಾ ರುದ್ರಸ್ಯ ಭೇಷಜೀ ತಯಾ ನೋ ಮೃಡ ಜೀವಸೇ~। ಇಮಾಗ್ಂ ರುದ್ರಾಯ ತವಸೇ ಕಪರ್ದಿನೇ ಕ್ಷಯದ್ವೀರಾಯ ಪ್ರಭರಾಮಹೇ ಮತಿಂ~। ಯಥಾ ನಃ ಶಮಸದ್ದ್ವಿಪದೇ ಚತುಷ್ಪದೇ ವಿಶ್ವಂ ಪುಷ್ಟಂ ಗ್ರಾಮೇ ಅಸ್ಮಿನ್ನನಾತುರಂ~। ಮೃಡಾ ನೋ ರುದ್ರೋತನೋ ಮಯಸ್ಕೃಧಿ ಕ್ಷಯದ್ವೀರಾಯ ನಮಸಾ ವಿಧೇಮ ತೇ~। ಯಚ್ಛಂ ಚ ಯೋಶ್ಚ ಮನುರಾಯಜೇ ಪಿತಾ ತದಶ್ಯಾಮ ತವ ರುದ್ರ ಪ್ರಣೀತೌ~। ಮಾ ನೋ ಮಹಾಂತಮುತ ಮ ನೋ ಅರ್ಭಕಂ ಮಾ ನ ಉಕ್ಷಂತಮುತ ಮಾನ ಉಕ್ಷಿತಂ~। ಮಾ ನೋ ವಧೀಃ ಪಿತರಂ ಮೋತ ಮಾತರಂ ಪ್ರಿಯಾ ಮಾ ನಸ್ತನುವೋ ರುದ್ರ ರೀರಿಷಃ~। ಮಾ ನ ಸ್ತೋಕೇ ತನಯೇ ಮಾ ನ ಆಯುಷಿ ಮಾನೋ ಗೋಷು ಮಾ ನೋ ಅಶ್ವೇಷು ರೀರಿಷಃ~। ವೀರಾನ್ಮಾನೋ ರುದ್ರ ಭಾಮಿತೋವಧೀರ್ ಹವಿಷ್ಮಂತೋ ನಮಸಾ ವಿಧೇಮ ತೇ~। ಆರಾತ್ತೇ ಗೋಘ್ನ ಉತ ಪೂರುಷಘ್ನೇ ಕ್ಷಯದ್ವೀರಾಯ ಸುಮ್ನಮಸ್ಮೇ ತೇ ಅಸ್ತು~। ರಕ್ಷಾ ಚ ನೋ ಅಧಿ ಚ ದೇವ ಬ್ರೂಹ್ಯಧಾ ಚ ನಃ ಶರ್ಮ ಯಚ್ಛ ದ್ವಿಬರ್ಹಾಃ~। ಸ್ತುಹಿ ಶ್ರುತಂ ಗರ್ತಸದಂ ಯುವಾನಂ ಮೃಗನ್ನ ಭೀಮಮುಪಹತ್ನುಮುಗ್ರಂ~। ಮೃಡಾ ಜರಿತ್ರೇ ರುದ್ರ ಸ್ತವಾನೋ ಅನ್ಯಂತೇ ಅಸ್ಮನ್ನಿವಪಂತು ಸೇನಾಃ~। ಪರಿಣೋ ರುದ್ರಸ್ಯ ಹೇತಿರ್ವೃಣಕ್ತು ಪರಿ ತ್ವೇಷಸ್ಯ ದುರ್ಮತಿರಘಾಯೋಃ~। ಅವಸ್ಥಿರಾ ಮಘವದ್ಭ್ಯಸ್ತನುಷ್ವ ಮೀಢ್ವಸ್ತೋಕಾಯ ತನಯಾಯ ಮೃಡಯ~। ಮೀಢುಷ್ಟಮ ಶಿವತಮ ಶಿವೋ ನಃ ಸುಮನಾ ಭವ~। ಪರಮೇ ವೃಕ್ಷ ಆಯುಧನ್ನಿಧಾಯ ಕೃತ್ತಿಂ ವಸಾನ ಆಚರ ಪಿನಾಕಂ ಬಿಭ್ರದಾಗಹಿ~। ವಿಕಿರಿದ ವಿಲೋಹಿತ ನಮಸ್ತೇ ಅಸ್ತು ಭಗವಃ~। ಯಾಸ್ತೇಸಹಸ್ರಗ್ಂ ಹೇತಯೋನ್ಯಮಸ್ಮನ್ನಿವಪಂತು ತಾಃ~। ಸಹಸ್ರಾಣಿ ಸಹಸ್ರಧಾ ಬಾಹುವೋಸ್ತವ ಹೇತಯಃ~। ತಾಸಾ ಮೀಶಾನೋ ಭಗವಃ ಪರಾಚೀನಾ ಮುಖಾ ಕೃಧಿ~॥

ಸಹಸ್ರಾಣಿ ಸಹಸ್ರಶೋ ಯೇ ರುದ್ರಾ ಅಧಿ ಭೂಮ್ಯಾಂ~॥
ತೇಷಾಗ್ಂ ಸಹಸ್ರಯೋಜನೇವ ಧನ್ವಾನಿ ತನ್ಮಸಿ~॥
ಅಸ್ಮಿನ್ಮಹತ್ಯರ್ಣವೇಂತರಿಕ್ಷೇ ಭವಾ ಅಧಿ~॥
ನೀಲಗ್ರೀವಾ ಶಿತಿಕಂಠಾಃ ಶರ್ವಾ ಅಧಃ ಕ್ಷಮಾಚರಾಃ~॥
ನೀಲಗ್ರೀವಾಃ ಶಿತಿಕಂಠಾ ದಿವಗ್ಂ ರುದ್ರಾ ಉಪಶ್ರಿತಾಃ~॥
ಯೇ ವೃಕ್ಷೇಷು ಸಸ್ಪಿಂಜರಾ ನೀಲಗ್ರೀವಾ ವಿಲೋಹಿತಾಃ~॥
ಯೇ ಭೂತಾನಾಮಧಿಪತಯೋ ವಿಶಿಖಾಸಃ ಕಪರ್ದಿನಃ~॥
ಯೇ ಅನ್ನೇಷು ವಿವಿಧ್ಯಂತಿ ಪಾತ್ರೇಷು ಪಿಬತೋ ಜನಾನ್~॥
ಯೇ ಪಥಾಂ ಪಥಿರಕ್ಷಯ ಐಲಬೃದಾ ಯವ್ಯುಧಃ~॥
ಯೇ ತೀರ್ಥಾನಿ ಪ್ರಚರಂತಿ ಸೃಕಾವಂತೋ ನಿಷಂಗಿಣಃ~॥
ಯ ಏತಾವಂತಶ್ಚ ಭೂಯಾಗ್ಂಸಶ್ಚ ದಿಶೋ ರುದ್ರಾ ವಿತಸ್ತಿರೇ~॥
ತೇಷಾಗ್ಂ ಸಹಸ್ರಯೋಜನೇವ ಧನ್ವಾನಿ ತನ್ಮಸಿ~॥\\
ನಮೋ ರುದ್ರೇಭ್ಯೋ ಯೇ ಪೃಥಿವ್ಯಾಂ ಯೇಂತರಿಕ್ಷೇ ಯೇ ದಿವಿ ಯೇಷಾಮನ್ನಂ ವಾತೋ ವರ್ಷಮಿಷವಸ್ತೇಭ್ಯೋ ದಶ ಪ್ರಾಚೀರ್ದಶ ದಕ್ಷಿಣಾ ದಶ ಪ್ರತೀಚೀರ್ದಶೋದೀಚೀರ್ದಶೋರ್ಧ್ವಾಸ್ತೇಭ್ಯೋ ನಮಸ್ತೇ ನೋ ಮೃಡಯಂತು ತೇ ಯಂ ದ್ವಿಷ್ಮೋ ಯಶ್ಚ ನೋ ದ್ವೇಷ್ಟಿ ತಂ ವೋ ಜಂಭೇ ದಧಾಮಿ~॥

ತ್ರ್ಯಂಬಕಂ ಯಜಾಮಹೇ ಸುಗಂಧಿಂ ಪುಷ್ಟಿವರ್ಧನಂ~॥
ಉರ್ವಾರುಕಮಿವ ಬಂಧನಾನ್ಮೃತ್ಯೋರ್ಮುಕ್ಷೀಯ ಮಾಮೃತಾತ್~॥
ಯೋ ರುದ್ರೌ ಅಗ್ನೌ ಯೋ ಅಪ್ಸು ಯ ಓಷಧೀಷು ಯೋ ರುದ್ರೋ ವಿಶ್ವಾ ಭುವನಾ ವಿವೇಶ ತಸ್ಮೈ ರುದ್ರಾಯ ನಮೋ ಅಸ್ತು~॥
ತಮುಷ್ಟುಹಿ ಯಃ ಸ್ವಿಷುಃ ಸುಧನ್ವಾ ಯೋ ವಿಶ್ವಸ್ಯಕ್ಷಯತಿ ಭೇಷಜಸ್ಯ~॥
ಯಕ್ಷ್ವಾಮಹೇ ಸೌಮನಸಾಯ ರುದ್ರಂ ನಮೋಭಿರ್ದೇವಮಸುರಂದುವಸ್ಯ~॥
ಅಯಂ ಮೇ ಹಸ್ತೋ ಭಗವಾನಯಂ ಮೇ ಭಗವತ್ತರಃ~॥
ಅಯಂ ಮೇ ವಿಶ್ವಭೇಷಜೋಯಂ ಶಿವಾಭಿಮರ್ಶನಃ~॥
ಯೇ ತೇ ಸಹಸ್ರಮಯುತಂ ಪಾಶಾ ಮೃತ್ಯೋ ಮರ್ತ್ಯಾಯ ಹಂತವೇ~॥
ತಾನ್ ಯಜ್ಞಸ್ಯ ಮಾಯಯಾ ಸರ್ವಾನವ ಯಜಾಮಹೇ~॥
ಮೃತ್ಯವೇ ಸ್ವಾಹಾ ಮೃತ್ಯವೇ ಸ್ವಾಹಾ~॥
ಓಂ ನಮೋ ಭಗವತೇ ರುದ್ರಾಯ ವಿಷ್ಣವೇ ಮೃತ್ಯುರ್ಮೇ ಪಾಹಿ~॥ ಸದಾಶಿವೋಂ~॥
\section{ಚಮಕಪ್ರಶ್ನಃ }
\addcontentsline{toc}{section}{ಚಮಕಪ್ರಶ್ನಃ }
ಅಗ್ನಾವಿಷ್ಣೂ ಸಜೋಷಸೇಮಾ ವರ್ಧಂತು ವಾಂ ಗಿರಃ~। ದ್ಯುಮ್ನೈರ್ವಾಜೇಭಿರಾಗತಂ ॥ ವಾಜಶ್ಚ ಮೇ ಪ್ರಸವಶ್ಚ ಮೇ ಪ್ರಯತಿಶ್ಚ ಮೇ ಪ್ರಸಿತಿಶ್ಚ ಮೇ ಧೀತಿಶ್ಚ ಮೇ ಕ್ರತುಶ್ಚ ಮೇ ಸ್ವರಶ್ಚ ಮೇ ಶ್ಲೋಕಶ್ಚ ಮೇ ಶ್ರಾವಶ್ಚ ಮೇ ಶ್ರುತಿಶ್ಚ ಮೇ ಜ್ಯೋತಿಶ್ಚ ಮೇ ಸುವಶ್ಚ ಮೇ ಪ್ರಾಣಶ್ಚ ಮೇಪಾನಶ್ಚ ಮೇ ವ್ಯಾನಶ್ಚ ಮೇಸುಶ್ಚ ಮೇ ಚಿತ್ತಂ ಚ ಮ ಆಧೀತಂ ಚ ಮೇ ವಾಕ್ಚ ಮೇ ಮನಶ್ಚ ಮೇ ಚಕ್ಷುಶ್ಚ ಮೇ ಶ್ರೋತ್ರಂ ಚ ಮೇ ದಕ್ಷಶ್ಚ ಮೇ ಬಲಂ ಚ ಮ ಓಜಶ್ಚ ಮೇ ಸಹಶ್ಚ ಮ ಆಯುಶ್ಚ ಮೇ ಜರಾ ಚ ಮ ಆತ್ಮಾ ಚ ಮೇ ತನೂಶ್ಚ ಮೇ ಶರ್ಮ ಚ ಮೇ ವರ್ಮ ಚ ಮೇಂಗಾನಿ ಚ ಮೇಸ್ಥಾನಿ ಚ ಮೇ ಪರೂಗ್ಂಷಿ ಚ ಮೇ ಶರೀರಾಣಿ ಚ ಮೇ ॥೧॥

ಜ್ಯೈಷ್ಠ್ಯಂ ಚ ಮ ಆಧಿಪತ್ಯಂ ಚ ಮೇ ಮನ್ಯುಶ್ಚ ಮೇ ಭಾಮಶ್ಚ ಮೇಮಶ್ಚ ಮೇಂಭಶ್ಚ ಮೇ ಜೇಮಾ ಚ ಮೇ ಮಹಿಮಾ ಚ ಮೇ ವರಿಮಾ ಚ ಮೇ ಪ್ರಥಿಮಾ ಚ ಮೇ ವರ್ಷ್ಮಾ ಚ ಮೇ ದ್ರಾಘುಯಾ ಚ ಮೇ ವೃದ್ಧಂ ಚ ಮೇ ವೃದ್ಧಿಶ್ಚ ಮೇ ಸತ್ಯಂ ಚ ಮೇ ಶ್ರದ್ಧಾ ಚ ಮೇ ಜಗಚ್ಚ ಮೇ ಧನಂ ಚ ಮೇ ವಶಶ್ಚ ಮೇ ತ್ವಿಷಿಶ್ಚ ಮೇ ಕ್ರೀಡಾ ಚ ಮೇ ಮೋದಶ್ಚ ಮೇ ಜಾತಂ ಚ ಮೇ ಜನಿಷ್ಯಮಾಣಂ ಚ ಮೇ ಸೂಕ್ತಂ ಚ ಮೇ ಸುಕೃತಂ ಚ ಮೇ ವಿತ್ತಂ ಚ ಮೇ ವೇದ್ಯಂ ಚ ಮೇ ಭೂತಂ ಚ ಮೇ ಭವಿಷ್ಯಚ್ಚ ಮೇ ಸುಗಂ ಚ ಮೇ ಸುಪಥಂ ಚ ಮ ಋದ್ಧಂ ಚ ಮ ಋದ್ಧಿಶ್ಚ ಮೇ ಕ್ಲೃಪ್ತಂ ಚ ಮೇ ಕ್ಲೃಪ್ತಿಶ್ಚ ಮೇ ಮತಿಶ್ಚ ಮೇ ಸುಮತಿಶ್ಚ ಮೇ ॥೨॥

ಶಂ ಚ ಮೇ ಮಯಶ್ಚ ಮೇ ಪ್ರಿಯಂ ಚ ಮೇನುಕಾಮಶ್ಚ ಮೇ ಕಾಮಶ್ಚ ಮೇ ಸೌಮನಸಶ್ಚ ಮೇ ಭದ್ರಂ ಚ ಮೇ ಶ್ರೇಯಶ್ಚ ಮೇ ವಸ್ಯಶ್ಚ ಮೇ ಯಶಶ್ಚ ಮೇ ಭಗಶ್ಚ ಮೇ ದ್ರವಿಣಂ ಚ ಮೇ ಯಂತಾ ಚ ಮೇ ಧರ್ತಾ ಚ ಮೇ ಕ್ಷೇಮಶ್ಚ ಮೇ ಧೃತಿಶ್ಚ ಮೇ ವಿಶ್ವಂ ಚ ಮೇ ಮಹಶ್ಚ ಮೇ ಸಂವಿಚ್ಚ ಮೇ ಜ್ಞಾತ್ರಂ ಚ ಮೇ ಸೂಶ್ಚ ಮೇ ಪ್ರಸೂಶ್ಚ ಮೇ ಸೀರಂ ಚ ಮೇ ಲಯಶ್ಚ ಮ ಋತಂ ಚ ಮೇ ಮೃತಂ ಚ ಮೇಯಕ್ಷ್ಮಂ ಚ ಮೇನಾಮಯಚ್ಚ ಮೇ ಜೀವಾತುಶ್ಚ ಮೇ ದೀರ್ಘಾಯುತ್ವಂ ಚ ಮೇನಮಿತ್ರಂ ಚ ಮೇಭಯಂ ಚ ಮೇ ಸುಗಂ ಚ ಮೇ ಶಯನಂ ಚ ಮೇ ಸೂಷಾ ಚ ಮೇ ಸುದಿನಂ ಚ ಮೇ ॥೩॥

ಊರ್ಕ್ಚ ಮೇ ಸೂನೃತಾ ಚ ಮೇ ಪಯಶ್ಚ ಮೇ ರಸಶ್ಚ ಮೇ ಘೃತಂ ಚ ಮೇ ಮಧು ಚ ಮೇ ಸಗ್ಧಿಶ್ಚ ಮೇ ಸಪೀತಿಶ್ಚ ಮೇ ಕೃಷಿಶ್ಚ ಮೇ ವೃಷ್ಟಿಶ್ಚ ಮೇ ಜೈತ್ರಂ ಚ ಮ ಔದ್ಭಿದ್ಯಂ ಚ ಮೇ ರಯಿಶ್ಚ ಮೇ ರಾಯಶ್ಚ ಮೇ ಪುಷ್ಟಂ ಚ ಮೇ ಪುಷ್ಟಿಶ್ಚ ಮೇ ವಿಭು ಚ ಮೇ ಪ್ರಭು ಚ ಮೇ ಬಹು ಚ ಮೇ ಭೂಯಶ್ಚ ಮೇ ಪೂರ್ಣಂ ಚ ಮೇ ಪೂರ್ಣತರಂ ಚ ಮೇಕ್ಷಿತಿಶ್ಚ ಮೇ ಕೂಯವಾಶ್ಚ ಮೇ ನ್ನಂ ಚ ಮೇಕ್ಷುಚ್ಚ ಮೇ ವ್ರೀಹಯಶ್ಚ ಮೇ ಯವಾಶ್ಚ ಮೇ ಮಾಷಾಶ್ಚ ಮೇ ತಿಲಾಶ್ಚ ಮೇ ಮುದ್ಗಾಶ್ಚ ಮೇ ಖಲ್ವಾಶ್ಚ ಮೇ ಗೋಧೂಮಾಶ್ಚ ಮೇ ಮಸುರಾಶ್ಚ ಮೇ ಪ್ರಿಯಂಗವಶ್ಚ ಮೇಣವಶ್ಚ ಮೇ ಶ್ಯಾಮಾಕಾಶ್ಚ ಮೇ ನೀವಾರಾಶ್ಚ ಮೇ ॥೪॥

ಅಶ್ಮಾ ಚ ಮೇ ಮೃತ್ತಿಕಾ ಚ ಮೇ ಗಿರಯಶ್ಚ ಮೇ ಪರ್ವತಾಶ್ಚ ಮೇ ಸಿಕತಾಶ್ಚ ಮೇ ವನಸ್ಪತಯಶ್ಚ ಮೇ ಹಿರಣ್ಯಂ ಚ ಮೇ ಯಶ್ಚ ಮೇ ಸೀಸಂ ಚ ಮೇ ತ್ರಪುಶ್ಚ ಮೇ ಶ್ಯಾಮಂ ಚ ಮೇ ಲೋಹಂ ಚ ಮೇಗ್ನಿಶ್ಚ ಮ ಆಪಶ್ಚ ಮೇ ವೀರುಧಶ್ಚ ಮ ಓಷಧಯಶ್ಚ ಮೇ ಕೃಷ್ಟಪಚ್ಯಂ ಚ ಮೇಕೃಷ್ಟಪಚ್ಯಂ ಚ ಮೇ ಗ್ರಾಮ್ಯಾಶ್ಚ ಮೇ ಪಶವ ಆರಣ್ಯಾಶ್ಚ ಯಜ್ಞೇನ ಕಲ್ಪಂತಾಂ ವಿತ್ತಂ ಚ ಮೇ ವಿತ್ತಿಶ್ಚ ಮೇ ಭೂತಂ ಚ ಮೇ ಭೂತಿಶ್ಚ ಮೇ ವಸು ಚ ಮೇ ವಸತಿಶ್ಚ ಮೇ ಕರ್ಮ ಚ ಮೇ ಶಕ್ತಿಶ್ಚ ಮೇ ರ್ಥಶ್ಚ ಮ ಏಮಶ್ಚ ಮ ಇತಿಶ್ಚ ಮೇ ಗತಿಶ್ಚ ಮೇ ॥೫॥

ಅಗ್ನಿಶ್ಚ ಮ ಇಂದ್ರಶ್ಚ ಮೇ ಸೋಮಶ್ಚ ಮ ಇಂದ್ರಶ್ಚ ಮೇ ಸವಿತಾ ಚ ಮ ಇಂದ್ರಶ್ಚ ಮೇ ಸರಸ್ವತೀ ಚ ಮ ಇಂದ್ರಶ್ಚ ಮೇ ಪೂಷಾ ಚ ಮ ಇಂದ್ರಶ್ಚ ಮೇ ಬೃಹಸ್ಪತಿಶ್ಚ ಮ ಇಂದ್ರಶ್ಚ ಮೇ ಮಿತ್ರಶ್ಚ ಮ ಇಂದ್ರಶ್ಚ ಮೇ ವರುಣಶ್ಚ ಮ ಇಂದ್ರಶ್ಚ ಮೇ ತ್ವಷ್ಟಾ ಚ ಮ ಇಂದ್ರಶ್ಚ ಮೇ ಧಾತಾ ಚ ಮ ಇಂದ್ರಶ್ಚ ಮೇ ವಿಷ್ಣುಶ್ಚ ಮ ಇಂದ್ರಶ್ಚ ಮೇಶ್ವಿನೌ ಚ ಮ ಇಂದ್ರಶ್ಚ ಮೇ ಮರುತಶ್ಚ ಮ ಇಂದ್ರಶ್ಚ ಮೇ ವಿಶ್ವೇ ಚ ಮೇ ದೇವಾ ಇಂದ್ರಶ್ಚ ಮೇ ಪೃಥಿವೀ ಚ ಮ ಇಂದ್ರಶ್ಚ ಮೇಂತರೀಕ್ಷಂ ಚ ಮ ಇಂದ್ರಶ್ಚ ಮೇ ದ್ಯೌಶ್ಚ ಮ ಇಂದ್ರಶ್ಚ ಮೇ ದಿಶಶ್ಚ ಮ ಇಂದ್ರಶ್ಚ ಮೇ ಮೂರ್ಧಾ ಚ ಮ ಇಂದ್ರಶ್ಚ ಮೇ ಪ್ರಜಾಪತಿಶ್ಚ ಮ ಇಂದ್ರಶ್ಚ ಮೇ ॥೬॥

ಅಗ್ಂಶುಶ್ಚ ಮೇ ರಶ್ಮಿಶ್ಚ ಮೇದಾಭ್ಯಶ್ಚ ಮೇಧಿಪತಿಶ್ಚ ಮ ಉಪಾಗ್ಂಶುಶ್ಚ ಮೇಂತರ್ಯಾಮಶ್ಚ ಮ ಐಂದ್ರವಾಯಶ್ಚ ಮೇ ಮೈತ್ರಾವರುಣಶ್ಚ ಮ ಆಶ್ವಿನಶ್ಚ ಮೇ ಪ್ರತಿಪಸ್ಥಾನಶ್ಚ ಮೇ ಶುಕ್ರಶ್ಚ ಮೇ ಮಂಥೀ ಚ ಮ ಆಗ್ರಯಣಶ್ಚ ಮೇ ವೈಶ್ವದೇವಶ್ಚ ಮೇ ಧ್ರುವಶ್ಚ ಮೇ ವೈಶ್ವಾನರಶ್ಚ ಮ ಋತುಗ್ರಾಹಾಶ್ಚ ಮೇ ತಿಗ್ರಾಹ್ಯಾಶ್ಚ ಮ ಐಂದ್ರಾಗ್ನಶ್ಚ ಮೇ ವೈಶ್ವದೇವಶ್ಚ ಮೇ ಮರುತ್ವತೀಯಾಶ್ಚ ಮೇ ಮಾಹೇಂದ್ರಶ್ಚ ಮ ಆದಿತ್ಯಶ್ಚ ಮೇ ಸಾವಿತ್ರಶ್ಚ ಮೇ ಸಾರಸ್ವತಶ್ಚ ಮೇ ಪೌಷ್ಣಶ್ಚ ಮೇ ಪಾತ್ನೀವತಶ್ಚ ಮೇ ಹಾರಿಯೋಜನಶ್ಚ ಮೇ ॥೭॥

ಇಧ್ಮಶ್ಚ ಮೇ ಬರ್ಹಿಶ್ಚ ಮೇ ವೇದಿಶ್ಚ ಮೇ ಧಿಷ್ಣಿಯಾಶ್ಚ ಮೇ ಸ್ರುಚಶ್ಚ ಮೇ ಚಮಸಾಶ್ಚ ಮೇ ಗ್ರಾವಾಣಶ್ಚ ಮೇ ಸ್ವರವಶ್ಚ ಮ ಉಪರವಾಶ್ಚ ಮೇಧಿಷವಣೇ ಚ ಮೇ ದ್ರೋಣಕಲಶಶ್ಚ ಮೇ ವಾಯವ್ಯಾನಿ ಚ ಮೇ ಪೂತಭೃಚ್ಚ ಮೇ ಆಧವನೀಯಶ್ಚ ಮ ಆಗ್ನೀಧ್ರಂ ಚ ಮೇ ಹವಿರ್ಧಾನಂ ಚ ಮೇ ಗೃಹಾಶ್ಚ ಮೇ ಸದಶ್ಚ ಮೇ ಪುರೋಡಾಶಾಶ್ಚ ಮೇ ಪಚತಾಶ್ಚ ಮೇವಭೃಥಶ್ಚ ಮೇ ಸ್ವಗಾಕಾರಶ್ಚ ಮೇ ॥೮॥

ಅಗ್ನಿಶ್ಚ ಮೇ ಘರ್ಮಶ್ಚ ಮೇರ್ಕಶ್ಚ ಮೇ ಸೂರ್ಯಶ್ಚ ಮೇ ಪ್ರಾಣಶ್ಚ ಮೇಶ್ವಮೇಧಶ್ಚ ಮೇ ಪೃಥಿವೀ ಚ ಮೇ ದಿತಿಶ್ಚ ಮೇ ದಿತಿಶ್ಚ ಮೇ ದ್ಯೌಶ್ಚ ಮೇ ಶಕ್ಕ್ವರೀರಂಗುಲಯೋ ದಿಶಶ್ಚ ಮೇ ಯಜ್ಞೇನ ಕಲ್ಪಂತಾಮೃಕ್ಚ ಮೇ ಸಾಮ ಚ ಮೇ ಸ್ತೋಮಶ್ಚ ಮೇ ಯಜುಶ್ಚ ಮೇ ದೀಕ್ಷಾ ಚ ಮೇ ತಪಶ್ಚ ಮ ಋತುಶ್ಚ ಮೇ ವ್ರತಂ ಚ ಮೇ ಹೋರಾತ್ರಯೋರ್ವೃಷ್ಟ್ಯಾ ಬೃಹದ್ರಥಂತರೇ ಚ ಮೇ ಯಜ್ಞೇನ ಕಲ್ಪೇತಾಂ ॥೯॥

ಗರ್ಭಾಶ್ಚ ಮೇ ವತ್ಸಾಶ್ಚ ಮೇ ತ್ರ್ಯವಿಶ್ಚ ಮೇ ತ್ರ್ಯವೀ ಚ ಮೇ ದಿತ್ಯವಾಟ್ ಚ ಮೇ ದಿತ್ಯೌಹೀ ಚ ಮೇ ಪಂಚಾವಿಶ್ಚ ಮೇ ಪಂಚಾವೀ ಚ ಮೇ ತ್ರಿವತ್ಸಶ್ಚ ಮೇ ತ್ರಿವತ್ಸಾ ಚ ಮೇ ತುರ್ಯವಾಟ್ ಚ ಮೇ ತುರ್ಯೌಹೀ ಚ ಮೇ ಪಷ್ಠವಾಟ್ ಚ ಮೇ ಪಷ್ಠೌಹೀ ಚ ಮ ಉಕ್ಷಾ ಚ ಮೇ ವಶಾ ಚ ಮ ಋಷಭಶ್ಚ ಮೇ ವೇಹಶ್ಚ ಮೇ ನಡ್ವಾಂಚ ಮೇ ಧೇನುಶ್ಚ ಮ ಆಯುರ್ಯಜ್ಞೇನ ಕಲ್ಪತಾಂ ಪ್ರಾಣೋ ಯಜ್ಞೇನ ಕಲ್ಪತಾಮಪಾನೋ ಯಜ್ಞೇನ ಕಲ್ಪತಾಂ ವ್ಯಾನೋ ಯಜ್ಞೇನ ಕಲ್ಪತಾಂ ಚಕ್ಷುರ್ಯಜ್ಞೇನ ಕಲ್ಪತಾಗ್ ಶ್ರೋತ್ರಂ ಯಜ್ಞೇನ ಕಲ್ಪತಾಂ ಮನೋ ಯಜ್ಞೇನ ಕಲ್ಪತಾಂ ವಾಗ್ಯಜ್ಞೇನ ಕಲ್ಪತಾಮಾತ್ಮಾ ಯಜ್ಞೇನ ಕಲ್ಪತಾಂ ಯಜ್ಞೋ ಯಜ್ಞೇನ ಕಲ್ಪತಾಂ ॥೧೦॥

ಏಕಾ ಚ ಮೇ ತಿಸ್ರಶ್ಚ ಮೇ ಪಂಚ ಚ ಮೇ ಸಪ್ತ ಚ ಮೇ ನವ ಚ ಮ ಏಕದಶ ಚ ಮೇ ತ್ರಯೋದಶ ಚ ಮೇ ಪಂಚದಶ ಚ ಮೇ ಸಪ್ತದಶ ಚ ಮೇ ನವದಶ ಚ ಮ ಏಕ ವಿಗ್ಂಶತಿಶ್ಚ ಮೇ ತ್ರಯೋವಿಗ್ಂಶತಿಶ್ಚ ಮೇ ಪಂಚವಿಗ್ಂಶತಿಶ್ಚ ಮೇ ಸಪ್ತವಿಗ್ಂಶತಿಶ್ಚ ಮೇ ನವವಿಗ್ಂಶತಿಶ್ಚ ಮ ಏಕತ್ರಿಗ್ಂಶಚ್ಚ ಮೇ ತ್ರಯಸ್ತ್ರಿಗ್ಂಶಚ್ಚ ಮೇ ಚತಸ್ರಶ್ಚ ಮೇಷ್ಟೌ ಚ ಮೇ ದ್ವಾದಶ ಚ ಮೇ ಷೋಡಶ ಚ ಮೇ ವಿಗ್ಂಶತಿಶ್ಚ ಮೇ ಚತುರ್ವಿಗ್ಂಶತಿಶ್ಚ ಮೇಷ್ಟಾವಿಗ್ಂಶತಿಶ್ಚ ಮೇ ದ್ವಾತ್ರಿಗ್ಂಶಚ್ಚ ಮೇ ಷಟ್ತ್ರಿಗ್ಂಶಚ್ಚ ಮೇ ಚತ್ವಾರಿಗ್ಂಶಚ್ಚ ಮೇ ಚತುಶ್ಚತ್ವಾರಿಗ್ಂಶಚ್ಚ ಮೇಷ್ಟಾಚತ್ವಾರಿಗ್ಂಶಚ್ಚ ಮೇ ವಾಜಶ್ಚ ಪ್ರಸವಶ್ಚಾಪಿಜಶ್ಚ ಕ್ರತುಶ್ಚ ಸುವಶ್ಚ ಮೂರ್ಧಾ ಚ ವ್ಯಶ್ನಿಯಶ್ಚಾಂತ್ಯಾಯನಶ್ಚಾಂತ್ಯಶ್ಚ ಭೌವನಶ್ಚ ಭುವನಶ್ಚಾಧಿಪತಿಶ್ಚ ॥೧೧॥

ಇಡಾ ದೇವಹೂರ್ಮನುರ್ಯಜ್ಞನೀರ್ಬೃಹಸ್ಪತಿರುಕ್ಥಾಮದಾನಿ ಶಗ್ಂಸಿಷದ್ವಿಶ್ವೇದೇವಾಃ ಸೂಕ್ತವಾಚಃ ಪೃಥಿವೀಮಾತರ್ಮಾ ಮಾ ಹಿಗ್ಂಸೀರ್ಮಧು ಮನಿಷ್ಯೇ ಮಧು ಜನಿಷ್ಯೇ ಮಧು ವಕ್ಷ್ಯಾಮಿ ಮಧು ವದಿಷ್ಯಾಮಿ ಮಧುಮತೀಂ ದೇವೇಭ್ಯೋ ವಾಚಮುದ್ಯಾಸಗ್ಂ ಶುಶ್ರೂಷೇಣ್ಯಾಂ ಮನುಷ್ಯೇಭ್ಯಸ್ತಂ ಮಾ ದೇವಾ ಅವಂತು ಶೋಭಾಯೈ ಪಿತರೋನುಮದಂತು ॥ಓಂ ಶಾಂತಿಃ ಶಾಂತಿಃ ಶಾಂತಿಃ ॥

ಪರಮಾನಂದಬೋಧಾಬ್ಧಿ ನಿಮಗ್ನ ನಿಜಮೂರ್ತಯೇ ।\\
ಸಾಂಗೋಪಾಂಗಮಿದಂ ಸ್ನಾನಂ ಕಲ್ಪಯಾಮಿ ಜಗತ್ಪತೇ ॥\\
ಆನೀಯ ಗಾಂಗಜಲಮಂಬ ಸುಧಾಸಕಾಶಂ \\ಸಂಮಿಶ್ರ್ಯ ರೋಚನಮುಖಾನ್ ಸ್ನಪಯಾಮಿ ತೇಽಹಂ~।\\
ಮಂತ್ರಾಂಸ್ತವಸ್ತವಪರಾನ್ ಪ್ರಪಠಂಛ್ರುತಿಸ್ಥಾನ್ \\ಆದಿತ್ಯವದ್ವಪುರಿದಂ ಕುರು ಕಾಂತಿಯುಕ್ತಂ ॥\\
\as{ಯತ್ಪುರುಷೇಣ+++++ ಇದ್ಧ್ಮಃ ಶರದ್ಧವಿಃ॥\\
ಆದಿತ್ಯವರ್ಣೇ +++++++++ಬಾಹ್ಯಾ ಅಲಕ್ಷ್ಮೀಃ ॥}ಶುದ್ಧೋದಕಸ್ನಾನಂ॥

ಗಜಾಜಿನಪರೀಧಾನ ವ್ಯಾಘ್ರಚರ್ಮಾಂಬರಪ್ರಿಯ ।\\
ದುಕೂಲವಸ್ತ್ರಯುಗಲಂ ಗೃಹ್ಯತಾಂ ಪಾರ್ವತೀಪತೇ ॥ \\
ಪ್ರಾವಾರಯಾಮಿ ಸುದುಕೂಲಮನನ್ಯಲಭ್ಯ\\ಮರ್ಧೋರುಕಂ ಕಟಿತಟೇ ಹೃದಿಕಂಚುಕಂ ಚ~।\\
ಸುಪ್ರೀತಿರಸ್ತು ಮಯಿ ತೇ ಭವತೀಕಟಾಕ್ಷನುನ್ನಾ \\ವರಾಭಯಕರಾ ಸಮುಪೈತು ಮಾಂ ಮಾ ॥\\
\as{ತಂ ಯಜ್ಞಂ +++++ ಋಷಯಶ್ಚ ಯೇ॥\\
ಉಪೈತು ಮಾಂ ++++++++ ದದಾತು ಮೇ ॥\\
ಯುವಂ ವಸ್ತ್ರಾಣಿ ++++ಸಚೇಥೇ ॥}ವಸ್ತ್ರಯುಗ್ಮಮ್ ॥

ಉಪವೀತಂ ಮಹಾದೇವ ಫಣಾಫಣಿಮನೋಹರಂ ।\\
ಬ್ರಹ್ಮಸೂತ್ರಂ ಗೃಹಾಣೇದಂ ಸುವರ್ಣಪರಿಕಲ್ಪಿತಂ ॥\\
ಯಜ್ಞೋಪವೀತಮಿತಿ ಯತ್ಸ್ಮೃತಿಷು ಪ್ರಸಿದ್ಧಂ\\ ಕರ್ಮಾಧಿಕಾರಪರಿಸೂಚಕಸೂತ್ರಮೇತತ್~।\\
ಭಕ್ತ್ಯಾ ಸಮರ್ಪ್ಯ ಚ ಭವಾಮಿ ಸಮಾಧಿಯುಕ್ತೋ \\ನಿಃಕ್ಷುತ್ಪಿಪಾಸ ಇಹ ತೇಽಮ್ಬ ಕೃಪಾಕಟಾಕ್ಷಾತ್ ॥\\
\as{ತಸ್ಮಾದ್ಯಜ್ಞಾ+++++ಗ್ರಾಮ್ಯಾಶ್ಚ ಯೇ॥\\
ಕ್ಷುತ್ಪಿಪಾಸಾಮ್+++++++++++ ಗೃಹಾತ್॥\\
ಯಜ್ಞೋಪವೀತಮ್ ++++ತೇಜಃ ॥} ಉಪವೀತಮ್ ॥

ವಿಭೂತಿಭೂಷಣ ಸ್ವಾಮಿನ್ ವಿಭೂತಿಭಿರುಪಾಸಿತ ।\\
ವಿಭೂತಿಂ ತೇ ಪ್ರಯಚ್ಛಾಮಿ ಭಗವನ್ ವರದೋ ಭವ ॥

ಗೋಗರ್ಭೇನ ಸಮುದ್ಭೂತಂ ಗೋಮಯಂ ಪರಮೌಷಧಂ ।\\
ಅಗ್ನಿಸ್ಪರ್ಶೇನ ಶುದ್ಧಂ ಚ ಭಸ್ಮ ರುದ್ರಪ್ರಿಯಂ ಸದಾ ॥

ಸರ್ವಾಂಗಲೇಪನಂ ಭಸ್ಮ ಸರ್ವಮೃತ್ಯುನಿವಾರಣಂ ।\\
ಸರ್ವತೇಜೋಮಯಂ ದಿವ್ಯಂ ಸರ್ವವ್ಯಾಧಿನಿವಾರಣಂ ॥

ಆಯುರಾರೋಗ್ಯದಮಿದಂ ಸರ್ವಸೌಭಾಗ್ಯದಾಯಕಂ ।\\
ಭಕ್ತ್ಯಾ ಸಮರ್ಪಿತಂ ತುಭ್ಯಂ ಪ್ರೀತ್ಯಾ ಸ್ವೀಕುರು ಶಂಕರ ॥ ಭಸ್ಮ ॥


ನಾನಾಭರಣನಾಗೇಂದ್ರಕಂಕಣಾನೇಕಭೂಷಿತ ।\\
ನಾನಾವಿಧಾನಲಂಕಾರಾನ್ ಗೃಹಾಣ ಪರಮೇಶ್ವರ ॥

ಮಾಣಿಕ್ಯಮುಕ್ತಾಫಲವಿದ್ರುಮೈಶ್ಚ ಗೋಮೇಧವೈಡೂರ್ಯಕಪುಷ್ಯರಾಗೈಃ ।\\
ಪ್ರವಾಲನೀಲೈಶ್ಚ ಕೃತಂ ಗೃಹಾಣ ದಿವ್ಯಂ ಹಿ ರತ್ನಾಭರಣಂ ಚ ದೇವ ॥

ರುದ್ರಾಕ್ಷಮಾಲಾಪ್ರಿಯ ಬಿಲ್ವಮಾಲಾಂ ಸ್ಫಟಿಕಾದಿಮಾಲಾಂ ನವರತ್ನಮಾಲಾಂ।\\
ಭಕ್ತ್ಯಾ ರ್ಪಿತಂ  ಭೂಷಣಮದ್ಯ  ದೇವ ನಾಗೇಂದ್ರರುದ್ರಾಕ್ಷಭೂಷಾಭಿರಾಮ ॥

ಕೇಯೂರಮಂಗದಮಣಿಕಂಕಣಮೇಖಲಾದ್ಯ\\ಮಂಗೇಷು ತೇ ಜನನಿ ಭೂಷಣಮರ್ಪಯಾಮಿ~।\\
ಸ್ವೀಕೃತ್ಯ ಮಾಂ ಕುರು ವಿಭೂಷಿತಮಂಬ ವಾಗ್ಭಿ\\ರ್ದೇಹೇ ಚ ರೋಗರಹಿತಂ ಸುಹಿರಣ್ಯರೂಪಂ ॥\\
\as{ಹಿರಣ್ಯರೂಪಃ+++++ತ್ಯನ್ನಮಸ್ಮೈ ॥ }ಆಭರಣಮ್॥

ಶ್ರೀಖಂಡಂ ಚಂದನಂ ಚಿವ್ಯಂ ಗಂಧಾಢ್ಯಂ ಸುಮನೋಹರಂ ।\\
ವಿಲೇಪನಂ ಸುರಶ್ರೇಷ್ಠ ಪ್ರೀತ್ಯರ್ಥಂ ಪ್ರತಿಗೃಹ್ಯತಾಂ ॥ \\
ಆಲೇಪನಾಯ ಸುಮನೋಹರಗಂಧಯುಕ್ತಂ\\ ಶ್ರೀಚಂದನಂ ದೃಷದಿಮರ್ದಿತಮತ್ಯಮೋಘಂ~।\\
ಸೌವರ್ಣಪಾತ್ರನಿಹಿತಂ ಪ್ರದದೇ ಚ ಗಂಧ\-\\ದ್ವಾರಾಮಿತಿ ಪ್ರಾಥಿತಮಂತ್ರಜಪೇನ ಸಾಕಂ ॥\\
\as{ತಸ್ಮಾದ್ಯಜ್ಞಾತ್+++++ ಸ್ಮಾದಜಾಯತ॥\\
ಗಂಧದ್ವಾರಾಂ+++++++++ ಶ್ರಿಯಮ್ ॥\\
ಗಂಧದ್ವಾರಾಂ+++++++++ ಶ್ರಿಯಮ್ ॥}ಗಂಧಃ ॥


ಅಕ್ಷತಾನ್ ಧವಲಾಕಾರಾನ್ ಶಾಲೀತಂಡುಲಮಿಶ್ರಿತಾನ್ ।\\
ಅನಂತಾಯ ನಮಸ್ತುಭ್ಯಂ  ಅಕ್ಷತಾನ್ ಪ್ರತಿಗೃಹ್ಯತಾಂ ॥\\
ತ್ವಾಮರ್ಚಯಾಮಿ ಸಕಲಾಭ್ಯುದಯಾಯ ಮಾತಃ\\ಹಾರಿದ್ರರಂಜಿತಯವೈ ರುಚಿರಾಕ್ಷತೈಶ್ಚ~।\\
ತ್ವತ್ಪೂಜನಸ್ತುತಿಪರಾ ಶ್ರುತಿರಾದಿದೇಶ \\ಹೇ ಪುತ್ರಕಾಃ ಸತತಮರ್ಚತ ಚಾರ್ಚತೇತಿ ॥\\
\as{ಅರ್ಚತಪ್ರಾರ್ಚತ+++++ ಧೃಷ್ಣ್ವರದಚತ ॥}ಅಕ್ಷತಾಃ ॥

ಗಂಡೇ ವಿಲೇಪಯಿತುಮಂಬ ದದೇ ಹರಿದ್ರಾಂ\\ ದಾರಿದ್ರ್ಯದುಃಖಭಯತಃ ಪರಿಮೋಚಯಾಶು~।\\
ಕಸ್ತೂರಿಕಾತಿಲಕಭೂಷಿತಫಾಲದೇಶೇ \\ಮಾಂಗಲ್ಯಸೂಚಕಸುಕುಂಕುಮಮರ್ಪಯಾಮಿ ॥

ಹರಿದ್ರಾಂ ಸ್ವರ್ಣವರ್ಣಾಭಾಂ ದೇವಮುನ್ಯಂಗನಾಧೃತಾಂ~।\\ಗಂಡಾಲಂಕರಣಾರ್ಥಾಯ ಸ್ವೀಕುರಷ್ವ ಮಯಾರ್ಪಿತಾಮ್ ॥ಹರಿದ್ರಾ॥

ಕುಂಕುಂಮಂ ಸರ್ವಸೌಭಾಗ್ಯ ಸೂಚಕಂ ಫಾಲಭೂಷಣಮ್~।\\ ಸ್ವೀಕುರುಷ್ವ ಜಗನ್ಮಾತಃ ಜಪಾಕುಸುಮ ಭಾಸ್ವರಮ್ ॥ಕುಂಕುಮಮ್॥

ಮಹಾದೇವಿ ನಮಸ್ತುಭ್ಯಂ ಸಂಧ್ಯಾವದರುಣಪ್ರಭಮ್।\\ಸೌಭಾಗ್ಯಸೂಚಕಂ ದಿವ್ಯಂ ಸಿಂದೂರಂ ಪ್ರತಿಗೃಹ್ಯತಾಮ್ ॥ಸಿಂದೂರಮ್॥

ಕರವೀರಜಾತೀಕುಸುಮೈಶ್ಚಂಪಕೈರ್ಬಕುಲೈಃ ಶುಭೈಃ ।\\
ಶತಪತ್ರೈಶ್ಚ ಕಹ್ಲಾರೈರರ್ಚಯೇ ಪರಮೇಶ್ವರ ॥\\
ಕಲ್ಪದ್ರುಮಾದಪಚಿತಾನಿ ಮನೋಹರಾಣಿ \\ಪುಷ್ಪಾಣಿ ಸಂತಿ ಕಬರೀಂ ಪರಿತಸ್ತ್ವದೀಯಾಂ~।\\
ದತ್ತಾಸ್ತಥಾಪಿ ಮಯಾ ತವ ಸಂಭವಂತು \\ಸಂತೋಷದಾಃ ಸುಮನಸೋ ಮನಸಸ್ತು ಕಾಮಂ ॥\\
\as{ತಸ್ಮಾದಶ್ವಾ+++++ಜಾತಾ ಅಜಾವಯಃ॥\\
ಮನಸಃ +++++++++++ ಯಶಃ ॥\\
ಆಯನೇತೇ++++++ಗೃಹಾ ಇಮೇ ॥}ಪುಷ್ಪಾಣಿ ॥

ತುಲಸೀಬಿಲ್ವಮಂದಾರ ಶಮೀಬಕುಲಪಲ್ಲವಾನ್ ।\\
ಮಯಾದತ್ತಾನ್ ಮಹಾದೇವ ಪ್ರೀತ್ಯಾ ಸ್ವೀಕುರು ಶಂಕರ ॥ ಪಾತ್ರಾಣಿ ॥

ಮಲ್ಲಿಕಾಜಾತಿಕುಸುಮೈಃ ಚಂಪಕಾಗಿರಿಕರ್ಣಿಕೈಃ ।\\
ದಿವ್ಯಗಂಧಾಮಿಮಾಂ ಮಾಲಾಂ ಸ್ವೀಕುರು ಪರಮೇಶ್ವರಿ ॥ ಪುಷ್ಪಮಾಲಾ ॥



ಓಂ ಐಂ ಹ್ರೀಂ ಶ್ರೀಂ ಶ್ರೀ ಶ್ರೀಲಲಿತಾಮಹಾತ್ರಿಪುರಸುಂದರ್ಯೈ \\ಧೌತವಸ್ತ್ರಪರಿಮಾರ್ಜನಂ ಕಲ್ಪಯಾಮಿ ನಮಃ~।\\
ಅರುಣದುಕೂಲಪರಿಧಾನಂ ಕಲ್ಪಯಾಮಿ ನಮಃ~।\\
ಅರುಣಕುಚೋತ್ತರೀಯಂ ಕಲ್ಪಯಾಮಿ ನಮಃ~।\\
ಆಲೇಪಮಂಟಪ ಪ್ರವೇಶನಂ ಕಲ್ಪಯಾಮಿ ನಮಃ~।\\
ಆಲೇಪಮಂಟಪಸ್ಥ ಮಣಿಪೀಠೋಪವೇಶನಂ ಕಲ್ಪಯಾಮಿ ನಮಃ~।\\
ಚಂದನಾಗರು ಕುಂಕುಮ ಮೃಗಮದ ಕರ್ಪೂರ ಕಸ್ತೂರೀ ಗೋರೋಚನಾದಿ\\ ದಿವ್ಯಗಂಧ ಸರ್ವಾಂಗೀಣವಿಲೇಪನಂ ಕಲ್ಪಯಾಮಿ ನಮಃ~।\\
ಕೇಶಭರಸ್ಯ ಕಾಲಾಗರುಧೂಪಂ ಕಲ್ಪಯಾಮಿ ನಮಃ~।\\
ಮಲ್ಲಿಕಾ ಮಾಲತೀ ಜಾತೀ ಚಂಪಕ ಅಶೋಕ ಶತಪತ್ರ ಪೂಗ ಗುಹಳೀ ಪುನ್ನಾಗ \\ಕಹ್ಲಾರ ಮುಖ್ಯ ಸರ್ವರ್ತುಕುಸುಮ ಮಾಲಾಃ ಕಲ್ಪಯಾಮಿ ನಮಃ~।\\
ಭೂಷಣಮಂಟಪ ಪ್ರವೇಶನಂ ಕಲ್ಪಯಾಮಿ ನಮಃ~।\\
ಭೂಷಣಮಂಟಪಸ್ಥ ರತ್ನಪೀಠೋಪವೇಶನಂ ಕಲ್ಪಯಾಮಿ ನಮಃ~।\\
ನವಮಣಿಮಕುಟಂ ಕಲ್ಪಯಾಮಿ ನಮಃ~।\\
ಚಂದ್ರಶಕಲಂ ಕಲ್ಪಯಾಮಿ ನಮಃ~।\\
ಸೀಮಂತಸಿಂದೂರಂ ಕಲ್ಪಯಾಮಿ ನಮಃ~।\\
ತಿಲಕರತ್ನಂ ಕಲ್ಪಯಾಮಿ ನಮಃ~।\\
ಕಾಲಾಂಜನಂ ಕಲ್ಪಯಾಮಿ ನಮಃ~।\\
ಪಾಳೀಯುಗಲಂ ಕಲ್ಪಯಾಮಿ ನಮಃ~।\\
ಮಣಿಕುಂಡಲಯುಗಲಂ ಕಲ್ಪಯಾಮಿ ನಮಃ~।\\
ನಾಸಾಭರಣಂ ಕಲ್ಪಯಾಮಿ ನಮಃ~।\\
ಅಧರಯಾವಕಂ ಕಲ್ಪಯಾಮಿ ನಮಃ~।\\
ಆದ್ಯಭೂಷಣಂ ಕಲ್ಪಯಾಮಿ ನಮಃ~।\\
ಕನಕಚಿಂತಾಕಂ ಕಲ್ಪಯಾಮಿ ನಮಃ~।\\
ಪದಕಂ ಕಲ್ಪಯಾಮಿ ನಮಃ~।\\
ಮಹಾಪದಕಂ ಕಲ್ಪಯಾಮಿ ನಮಃ~।\\
ಮುಕ್ತಾವಲಿಂ ಕಲ್ಪಯಾಮಿ ನಮಃ~।\\
ಏಕಾವಲಿಂ ಕಲ್ಪಯಾಮಿ ನಮಃ~।\\
ಛನ್ನವೀರಂ ಕಲ್ಪಯಾಮಿ ನಮಃ~।\\
ಕೇಯೂರಯುಗಲ ಚತುಷ್ಟಯಂ ಕಲ್ಪಯಾಮಿ ನಮಃ~।\\
ವಲಯಾವಲಿಂ ಕಲ್ಪಯಾಮಿ ನಮಃ~।\\
ಊರ್ಮಿಕಾವಲಿಂ ಕಲ್ಪಯಾಮಿ ನಮಃ~।\\
ಕಾಂಚೀದಾಮ ಕಲ್ಪಯಾಮಿ ನಮಃ~।\\
ಕಟಿಸೂತ್ರಂ ಕಲ್ಪಯಾಮಿ ನಮಃ~।\\
ಸೌಭಾಗ್ಯಾಭರಣಂ ಕಲ್ಪಯಾಮಿ ನಮಃ~।\\
ಪಾದಕಟಕಂ ಕಲ್ಪಯಾಮಿ ನಮಃ~।\\
ರತ್ನನೂಪುರಂ ಕಲ್ಪಯಾಮಿ ನಮಃ~।\\
ಪಾದಾಂಗುಲೀಯಕಂ ಕಲ್ಪಯಾಮಿ ನಮಃ~।\\
ವಾಮೋರ್ಧ್ವಕರೇ ಪಾಶಂ ಕಲ್ಪಯಾಮಿ ನಮಃ~।\\
ದಕ್ಷಿಣೋರ್ಧ್ವಕರೇ ಅಂಕುಶಂ ಕಲ್ಪಯಾಮಿ ನಮಃ~।\\
ವಾಮಾಧಃಕರೇ ಪುಂಡ್ರೇಕ್ಷುಚಾಪಂ ಕಲ್ಪಯಾಮಿ ನಮಃ~।\\
ದಕ್ಷಿಣಾಧಃಕರೇ ಪುಷ್ಪಬಾಣಾನ್ ಕಲ್ಪಯಾಮಿ ನಮಃ~।\\
ಶ್ರೀಮನ್ಮಾಣಿಕ್ಯಪಾದುಕಾ ಯುಗಲಂ ಕಲ್ಪಯಾಮಿ ನಮಃ~।\\
ಸ್ವಸಮಾನ ವೇಷಾಭಿರಾವರಣ ದೇವತಾಭಿಃ ಸಹ\\ ಶ್ರೀ ಚಕ್ರಾಧಿರೋಹಣಂ ಕಲ್ಪಯಾಮಿ ನಮಃ~।\\
ಕಾಮೇಶ್ವರಾಂಕಪರ್ಯಂಕೋಪವೇಶನಂ ಕಲ್ಪಯಾಮಿ ನಮಃ~।\\
ಅಮೃತಾಸವ ಚಷಕಂ ಕಲ್ಪಯಾಮಿ ನಮಃ~।\\
ಆಚಮನೀಯಕಂ ಕಲ್ಪಯಾಮಿ ನಮಃ~।\\
ಕರ್ಪೂರವೀಟಿಕಾಂ ಕಲ್ಪಯಾಮಿ ನಮಃ~।\\
ಆನಂದೋಲ್ಲಾಸ ವಿಲಾಸಹಾಸಂ ಕಲ್ಪಯಾಮಿ ನಮಃ~।\\
ಮಂಗಲಾರ್ತಿಕಂ ಕಲ್ಪಯಾಮಿ ನಮಃ~।\\
ಛತ್ರಂ ಕಲ್ಪಯಾಮಿ ನಮಃ~।\\
ಚಾಮರಯುಗಲಂ ಕಲ್ಪಯಾಮಿ ನಮಃ~।\\
ದರ್ಪಣಂ ಕಲ್ಪಯಾಮಿ ನಮಃ~।\\
ತಾಲವೃಂತಂ ಕಲ್ಪಯಾಮಿ ನಮಃ~।\\
ಗಂಧಂ ಕಲ್ಪಯಾಮಿ ನಮಃ~।\\
ಹರಿದ್ರಾ ಚೂರ್ಣಂ ಕಲ್ಪಯಾಮಿ ನಮಃ~।\\
ಕುಂಕುಮ ಚೂರ್ಣಂ ಕಲ್ಪಯಾಮಿ ನಮಃ~।\\
ಸಿಂದೂರ ಚೂರ್ಣಂ ಕಲ್ಪಯಾಮಿ ನಮಃ~।\\
ಪುಷ್ಪಾಣಿ ಕಲ್ಪಯಾಮಿ ನಮಃ~।
\section{ಶಿವ ಅಂಗಪೂಜಾ }
\addcontentsline{toc}{section}{ಶಿವ ಅಂಗಪೂಜಾ }
ಓಂ ಪಾಪನಾಶನಾಯ ನಮಃ~। ಪಾದೌ ಪೂಜಯಾಮಿ॥\\
ಓಂ ಗುರವೇ ನಮಃ~। ಗುಲ್ಫೌ ಪೂಜಯಾಮಿ॥\\
ಓಂ ಜ್ಞಾನಪ್ರದಾಯ ನಮಃ~। ಜಂಘೇ ಪೂಜಯಾಮಿ॥\\
ಓಂ ಜಾಹ್ನವೀಪತಯೇ ನಮಃ~। ಜಾನುನೀ ಪೂಜಯಾಮಿ॥\\
ಓಂ ಉತ್ತಮೋತ್ತಮಾಯ ನಮಃ~। ಊರೂ ಪೂಜಯಾಮಿ॥\\
ಓಂ ಕಂದರ್ಪನಾಶಾಯ ನಮಃ~। ಕಟಿಂ ಪೂಜಯಾಮಿ॥\\
ಓಂ ಗುಹೇಶ್ವರಾಯ ನಮಃ~। ಗುಹ್ಯಂ ಪೂಜಯಾಮಿ॥\\
ಓಂ ನಂದಿಸೇವ್ಯಾಯ ನಮಃ~। ನಾಭಿಂ ಪೂಜಯಾಮಿ॥\\
ಓಂ ಸ್ಕಂದಗುರವೇ ನಮಃ~। ಸ್ಕಂಧೌ ಪೂಜಯಾಮಿ॥\\
ಓಂ ಹಿರಣ್ಯಬಾಹವೇ ನಮಃ~। ಬಾಹೂನ್ ಪೂಜಯಾಮಿ॥\\
ಓಂ ಹರಾಯ ನಮಃ~। ಹಸ್ತಾನ್ ಪೂಜಯಾಮಿ॥\\
ಓಂ ನೀಲಕಂಠಾಯ ನಮಃ~। ಕಂಠಂ ಪೂಜಯಾಮಿ॥\\
ಓಂ ವೇದಮೂರ್ತಯೇ ನಮಃ~। ಮುಖಂ ಪೂಜಯಾಮಿ॥\\
ಓಂ ನಾಗಹಾರಾಯ ನಮಃ~। ನಾಸಿಕಾಂ ಪೂಜಯಾಮಿ॥\\
ಓಂ ತ್ರಿಣೇತ್ರಾಯ ನಮಃ~। ನೇತ್ರಾಣಿ ಪೂಜಯಾಮಿ॥\\
ಓಂ ಭಸಿತಾಭಾಸಾಯ ನಮಃ~। ಲಲಾಟಂ ಪೂಜಯಾಮಿ॥\\
ಓಂ ಇಂದುಮೌಲಯೇ ನಮಃ~। ಮೌಲಿಂ ಪೂಜಯಾಮಿ॥\\
ಓಂ ಶರ್ವಾಯ ನಮಃ~। ಶಿರಃ ಪೂಜಯಾಮಿ॥\\
ಓಂ ಸರ್ವಾತ್ಮನೇ ನಮಃ~। ಸರ್ವಾಂಗಂ ಪೂಜಯಾಮಿ॥
\section{ಪತ್ರ ಪೂಜಾ }
\addcontentsline{toc}{section}{ಪತ್ರ ಪೂಜಾ }
ಓಂ ಉಮಾಪತಯೇ ನಮಃ~। ಬಿಲ್ವಪತ್ರಂ ಸಮರ್ಪಯಾಮಿ॥\\
ಓಂ ಜಗದ್ಗುರವೇ ನಮಃ~। ತುಲಸೀಪತ್ರಂ ಸಮರ್ಪಯಾಮಿ॥\\
ಓಂ ಆನಂದಾಯ ನಮಃ~। ಅರ್ಕಪತ್ರಂ ಸಮರ್ಪಯಾಮಿ॥\\
ಓಂ ಸರ್ವಬಂಧವಿಮೋಚನಾಯ ನಮಃ~। ಜಂಬೀರಪತ್ರಂ ಸಮರ್ಪಯಾಮಿ॥\\
ಓಂ ಲೋಕನಾಥಾಯ ನಮಃ~। ನಿರ್ಗುಂಡೀಪತ್ರಂ ಸಮರ್ಪಯಾಮಿ॥\\
ಓಂ ಜಗತ್ಕಾರಣಾಯ ನಮಃ~। ದೂರ್ವಾಪತ್ರಂ ಸಮರ್ಪಯಾಮಿ॥\\
ಓಂ ನಾಗಭೂಷಣಾಯ ನಮಃ~। ಕುಶಪತ್ರಂ ಸಮರ್ಪಯಾಮಿ॥\\
ಓಂ ಮೃಗಧರಾಯ ನಮಃ~। ಮರುಗಪತ್ರಂ ಸಮರ್ಪಯಾಮಿ॥\\
ಓಂ ಪಶುಪತಯೇ ನಮಃ~। ಕಾಮಕಸ್ತೂರಿಕಾಪತ್ರಂ ಸಮರ್ಪಯಾಮಿ॥\\
ಓಂ ಮುಕುಂದಪ್ರಿಯಾಯ ನಮಃ~। ಗಿರಿಕರ್ಣಿಕಾಪತ್ರಂ ಸಮರ್ಪಯಾಮಿ॥\\
ಓಂ ತ್ರ್ಯಂಬಕಾಯ ನಮಃ~। ಮಾಚೀಪತ್ರಂ ಸಮರ್ಪಯಾಮಿ॥\\
ಓಂ ಭಕ್ತಜನಪ್ರಿಯಾಯ ನಮಃ~। ಧಾತ್ರೀಪತ್ರಂ ಸಮರ್ಪಯಾಮಿ॥\\
ಓಂ ವರದಾಯ ನಮಃ~। ವಿಷ್ಣುಕ್ರಾಂತಿಪತ್ರಂ ಸಮರ್ಪಯಾಮಿ॥\\
ಓಂ ಶಿವಾಯ ನಮಃ~। ದ್ರೋಣಪತ್ರಂ ಸಮರ್ಪಯಾಮಿ॥\\
ಓಂ ಶಂಕರಾಯ ನಮಃ~। ಧತ್ತೂರಪತ್ರಂ ಸಮರ್ಪಯಾಮಿ॥\\
ಓಂ ಶಮಪ್ರಾಪ್ತಾಯ ನಮಃ~। ಶಮೀಪತ್ರಂ ಸಮರ್ಪಯಾಮಿ॥\\
ಓಂ ಸಾಂಬಶಿವಾಯ ನಮಃ~। ಸೇವಂತಿಕಾಪತ್ರಂ ಸಮರ್ಪಯಾಮಿ॥\\
ಓಂ ಚರ್ಮವಾಸಸೇ ನಮಃ~। ಚಂಪಕಪತ್ರಂ ಸಮರ್ಪಯಾಮಿ॥\\
ಓಂ ಬ್ರಾಹ್ಮಣಪ್ರಿಯಾಯ ನಮಃ~। ಕರವೀರಪತ್ರಂ ಸಮರ್ಪಯಾಮಿ॥\\
ಓಂ ಗಂಗಾಧರಾಯ ನಮಃ~। ಅಶೋಕಪತ್ರಂ ಸಮರ್ಪಯಾಮಿ॥\\
ಓಂ ಪುಣ್ಯಮೂರ್ತಯೇ ನಮಃ~। ಪುನ್ನಾಗಪತ್ರಂ ಸಮರ್ಪಯಾಮಿ॥\\
ಓಂ ಉಮಾಮಹೇಶ್ವರಾಯ ನಮಃ~। ಸರ್ವಾಣಿ ಪತ್ರಾಣಿ ಸಮರ್ಪಯಾಮಿ॥
\section{ಪುಷ್ಪಪೂಜಾ}
\addcontentsline{toc}{section}{ಪುಷ್ಪಪೂಜಾ}
ಓಂ ರುದ್ರಾಯ ನಮಃ~। ದ್ರೋಣಪುಷ್ಪಂ ಸಮರ್ಪಯಾಮಿ॥\\
ಓಂ ಪಶುಪತಯೇ ನಮಃ~। ಧತ್ತೂರಪುಷ್ಪಂ ಸಮರ್ಪಯಾಮಿ॥\\
ಓಂ ಸ್ಥಾಣವೇ ನಮಃ~। ಬೃಹತೀಪುಷ್ಪಂ ಸಮರ್ಪಯಾಮಿ॥\\
ಓಂ ನೀಲಕಂಠಾಯ ನಮಃ~। ಅರ್ಕಪುಷ್ಪಂ ಸಮರ್ಪಯಾಮಿ॥\\
ಓಂ ಉಮಾಪತಯೇ ನಮಃ~। ಬಕುಲಪುಷ್ಪಂ ಸಮರ್ಪಯಾಮಿ॥\\
ಓಂ ಕಾಲಕಾಲಾಯ ನಮಃ~। ಜಾತೀಪುಷ್ಪಂ ಸಮರ್ಪಯಾಮಿ॥\\
ಓಂ ಕಾಲಮೂರ್ತಯೇ ನಮಃ~। ಕರವೀರಪುಷ್ಪಂ ಸಮರ್ಪಯಾಮಿ॥\\
ಓಂ ದೇವದೇವಾಯ ನಮಃ~। ಪಂಕಜಪುಷ್ಪಂ ಸಮರ್ಪಯಾಮಿ॥\\
ಓಂ ವಿಶ್ವಪ್ರಿಯಾಯ ನಮಃ~। ಪುನ್ನಾಗಪುಷ್ಪಂ ಸಮರ್ಪಯಾಮಿ॥\\
ಓಂ ವೃಷಭಧ್ವಜಾಯ ನಮಃ~। ವೈಜಯಂತಿಕಾಪುಷ್ಪಂ ಸಮರ್ಪಯಾಮಿ॥\\
ಓಂ ಸದಾಶಿವಾಯ ನಮಃ~। ಗಿರಿಕರ್ಣಿಕಾಪುಷ್ಪಂ ಸಮರ್ಪಯಾಮಿ॥\\
ಓಂ ಶೂಲಿನೇ ನಮಃ~। ಚಂಪಕಪುಷ್ಪಂ ಸಮರ್ಪಯಾಮಿ॥\\
ಓಂ ಸುರೇಶಾಯ ನಮಃ~। ಸೇವಂತಿಕಾಪುಷ್ಪಂ ಸಮರ್ಪಯಾಮಿ॥\\
ಓಂ ನಿರಹಂಕಾರಾಯ ನಮಃ~। ಮಲ್ಲಿಕಾಪುಷ್ಪಂ ಸಮರ್ಪಯಾಮಿ॥\\
ಓಂ ಸತ್ಯವ್ರತಾಯ ನಮಃ~। ಜಪಾಪುಷ್ಪಂ ಸಮರ್ಪಯಾಮಿ॥\\
ಓಂ ಉಮಾಮಹೇಶ್ವರಾಯ ನಮಃ~। ಸರ್ವಾಣಿ ಪುಷ್ಪಾಣಿ ಸಮರ್ಪಯಾಮಿ॥
\section{ಅಥ ಲಲಿತಾ ಅಂಗಪೂಜಾ}
\addcontentsline{toc}{section}{ಅಥ ಲಲಿತಾ ಅಂಗಪೂಜಾ}
೪ ಶ್ರೀಮಾತ್ರೇ ನಮಃ~। ಪಾದೌ ಪೂಜಯಾಮಿ~।\\
೪ ಭಾವನಾಯೈ ನಮಃ~। ಗುಲ್ಫೌ ಪೂಜಯಾಮಿ~।\\
೪ ಭಾವನಾಗಮ್ಯಾಯೈ ನಮಃ~। ಜಂಘೇ ಪೂಜಯಾಮಿ~।\\
೪ ಭವಾರಣ್ಯಕುಠಾರಿಕಾಯೈ ನಮಃ~। ಜಾನುನೀ ಪೂಜಯಾಮಿ~।\\
೪ ಭದ್ರಪ್ರಿಯಾಯೈ ನಮಃ~। ಊರೂ ಪೂಜಯಾಮಿ~।\\
೪ ಭದ್ರಮೂರ್ತ್ಯೈ ನಮಃ~। ಕಟಿಂ ಪೂಜಯಾಮಿ~।\\
೪ ಭಕ್ತಸೌಭಾಗ್ಯದಾಯಿನ್ಯೈ ನಮಃ~। ನಾಭಿಂ ಪೂಜಯಾಮಿ~।\\
೪ ಭಕ್ತಿಪ್ರಿಯಾಯೈ ನಮಃ~। ಉದರಂ ಪೂಜಯಾಮಿ~।\\
೪ ಭಕ್ತಿಗಮ್ಯಾಯೈ ನಮಃ~। ಸ್ತನೌ ಪೂಜಯಾಮಿ~।\\
೪ ಭಕ್ತಿವಶ್ಯಾಯೈ ನಮಃ~। ವಕ್ಷಸ್ಥಲಂ ಪೂಜಯಾಮಿ~।\\
೪ ಕಲ್ಯಾಣ್ಯೈ ನಮಃ~। ಬಾಹೂನ್ ಪೂಜಯಾಮಿ~।\\
೪ ಚಕ್ರಿಣ್ಯೈ ನಮಃ~। ಹಸ್ತಾನ್ ಪೂಜಯಾಮಿ~।\\
೪ ವಿಶ್ವಾತೀತಾಯೈ ನಮಃ~। ಕಂಠಂ ಪೂಜಯಾಮಿ~।\\
೪ ತ್ರಿಪುರಾಯೈ ನಮಃ~। ಜಿಹ್ವಾಂ ಪೂಜಯಾಮಿ~।\\
೪ ವಿರಾಗಿಣ್ಯೈ ನಮಃ~। ಮುಖಂ ಪೂಜಯಾಮಿ~।\\
೪ ಮಹೀಯಸ್ಯೈ ನಮಃ~। ನೇತ್ರೇ ಪೂಜಯಾಮಿ~।\\
೪ ಮನುವಿದ್ಯಾಯೈ ನಮಃ~। ಕರ್ಣೌ ಪೂಜಯಾಮಿ~।\\
೪ ಶಿವಾಯೈ ನಮಃ~। ಲಲಾಟಂ ಪೂಜಯಾಮಿ~।\\
೪ ಶಿವಶಕ್ತ್ಯೈ ನಮಃ~। ಶಿರಃ ಪೂಜಯಾಮಿ~।\\
೪ ಲಲಿತಾಂಬಿಕಾಯೈ ನಮಃ~। ಸರ್ವಾಂಗಂ ಪೂಜಯಾಮಿ~।
\section{ಅಥ ಪತ್ರಪೂಜಾ}
\addcontentsline{toc}{section}{ಅಥ ಪತ್ರಪೂಜಾ}
೪ ಕಲ್ಯಾಣ್ಯೈ ನಮಃ~। ಮಾಚೀಪತ್ರಂ ಸಮರ್ಪಯಾಮಿ~।\\
೪ ಕಮಲಾಕ್ಷ್ಯೈ ನಮಃ~। ಸೇವಂತಿಕಾಪತ್ರಂ ಸಮರ್ಪಯಾಮಿ~।\\
೪ ಏಕಾರರೂಪಾಯೈ ನಮಃ~। ಬಿಲ್ವಪತ್ರಂ ಸಮರ್ಪಯಾಮಿ~।\\
೪ ಏಕಭೋಗಾಯೈ ನಮಃ~। ತುಲಸೀಪತ್ರಂ ಸಮರ್ಪಯಾಮಿ~।\\
೪ ಏಕರಸಾಯೈ ನಮಃ~। ಕಸ್ತೂರಿಕಾಪತ್ರಂ ಸಮರ್ಪಯಾಮಿ~।\\
೪ ಈಶಿತ್ರ್ಯೈ ನಮಃ~। ಮರುವಕಪತ್ರಂ ಸಮರ್ಪಯಾಮಿ~।\\
೪ ಈಶಶಕ್ತ್ಯೈ ನಮಃ~। ಗಿರಿಕರ್ಣಿಕಾಪತ್ರಂ ಸಮರ್ಪಯಾಮಿ~।\\
೪ ಹ್ರೀಂಮತ್ಯೈ ನಮಃ~। ಕರವೀರಪತ್ರಂ ಸಮರ್ಪಯಾಮಿ~।\\
೪ ಕಾಮದಾಯೈ ನಮಃ~। ದಾಡಿಮೀಪತ್ರಂ ಸಮರ್ಪಯಾಮಿ~।\\
೪ ವಿಶ್ವತೋಮುಖ್ಯೈ ನಮಃ~। ವಿಷ್ಣುಕ್ರಾಂತಿಪತ್ರಂ ಸಮರ್ಪಯಾಮಿ~।\\
೪ ಭುವನೇಶ್ವರ್ಯೈ ನಮಃ~। ಜಂಬೀರಪತ್ರಂ ಸಮರ್ಪಯಾಮಿ~।\\
೪ ಶ್ರಿಯೈ ನಮಃ~। ಪದ್ಮಪತ್ರಂ ಸಮರ್ಪಯಾಮಿ~।\\
೪ ರಮಾಯೈ ನಮಃ~। ಚಂಪಕಪತ್ರಂ ಸಮರ್ಪಯಾಮಿ~।\\
೪ ಪುಷ್ಟ್ಯೈ ನಮಃ~। ಅಪಾಮಾರ್ಗಪತ್ರಂ ಸಮರ್ಪಯಾಮಿ~।\\
೪ ಗೌರ್ಯೈ ನಮಃ~। ಮಾಲತೀಪತ್ರಂ ಸಮರ್ಪಯಾಮಿ~।\\
೪ ಸರಸ್ವತ್ಯೈ ನಮಃ~। ಬದರೀಪತ್ರಂ ಸಮರ್ಪಯಾಮಿ~।\\
೪ ದುರ್ಗಾಯೈ ನಮಃ~। ಪಾರಿಜಾತಪತ್ರಂ ಸಮರ್ಪಯಾಮಿ~।\\
೪ ರುದ್ರಾಣ್ಯೈ ನಮಃ~। ಶಮೀಪತ್ರಂ ಸಮರ್ಪಯಾಮಿ~।\\
೪ ಮಾತೃಕಾಯೈ ನಮಃ~। ಧತ್ತೂರಪತ್ರಂ ಸಮರ್ಪಯಾಮಿ~।\\
೪ ಜಗದಂಬಾಯೈ ನಮಃ~। ನಿರ್ಗುಂಡೀಪತ್ರಂ ಸಮರ್ಪಯಾಮಿ~।\\
೪ ಲಲಿತಾಂಬಿಕಾಯೈ ನಮಃ~। ಕದಂಬಪತ್ರಂ ಸಮರ್ಪಯಾಮಿ~।\\
೪ ಶ್ರೀಚಕ್ರವಾಸಿನ್ಯೈ ನಮಃ~। ಸರ್ವಾಣಿ ಪತ್ರಾಣಿ ಸಮರ್ಪಯಾಮಿ~।
\section{ಪುಷ್ಪಪೂಜಾ}
\addcontentsline{toc}{section}{ಪುಷ್ಪಪೂಜಾ}
೪ ಕಾಲಶಕ್ತ್ಯೈ ನಮಃ~। ಜಾಜೀಪುಷ್ಪಂ ಸಮರ್ಪಯಾಮಿ~।\\
೪ ಬ್ರಹ್ಮರೂಪಿಣ್ಯೈ ನಮಃ~। ಕೇತಕೀಪುಷ್ಪಂ ಸಮರ್ಪಯಾಮಿ~।\\
೪ ವಿಷ್ಣುರೂಪಿಣ್ಯೈ ನಮಃ~। ಪುನ್ನಾಗಪುಷ್ಪಂ ಸಮರ್ಪಯಾಮಿ~।\\
೪ ರುದ್ರರೂಪಿಣ್ಯೈ ನಮಃ~। ಕಮಲಪುಷ್ಪಂ ಸಮರ್ಪಯಾಮಿ~।\\
೪ ಸದಾಶಿವರೂಪಿಣ್ಯೈ ನಮಃ~। ಮಲ್ಲಿಕಾಪುಷ್ಪಂ ಸಮರ್ಪಯಾಮಿ~।\\
೪ ಚಿತ್ಕಲಾಯೈ ನಮಃ~। ಸೇವಂತಿಕಾಪುಷ್ಪಂ ಸಮರ್ಪಯಾಮಿ~।\\
೪ ವೇದಮಾತ್ರೇ ನಮಃ~। ಬಕುಲಪುಷ್ಪಂ ಸಮರ್ಪಯಾಮಿ~।\\
೪ ದುರ್ಗಾಯೈ ನಮಃ~। ಜಪಾಪುಷ್ಪಂ ಸಮರ್ಪಯಾಮಿ~।\\
೪ ಮಹಾಲಕ್ಷ್ಮ್ಯೈ ನಮಃ~। ಪಾರಿಜಾತಪುಷ್ಪಂ ಸಮರ್ಪಯಾಮಿ~।\\
೪ ಮಹಾಸರಸ್ವತ್ಯೈ ನಮಃ~। ಮಾಲತೀಪುಷ್ಪಂ ಸಮರ್ಪಯಾಮಿ~।\\
೪ ಚಂಪಕವಾಹಿನ್ಯೈ ನಮಃ~। ಚಂಪಕಪುಷ್ಪಂ ಸಮರ್ಪಯಾಮಿ~।\\
೪ ಕಾಮೇಶ್ವರ್ಯೈ ನಮಃ~। ಕದಂಬಪುಷ್ಪಂ ಸಮರ್ಪಯಾಮಿ~।\\
೪ ಕಾತ್ಯಾಯನ್ಯೈ ನಮಃ~। ಕರವೀರಪುಷ್ಪಂ ಸಮರ್ಪಯಾಮಿ~।\\
೪ ಸರ್ವೇಶ್ವರ್ಯೈ ನಮಃ~। ಅಶೋಕಪುಷ್ಪಂ ಸಮರ್ಪಯಾಮಿ~।\\
೪ ತ್ರಿಪುರಸುಂದರ್ಯೈ ನಮಃ~। ಗಿರಿಕರ್ಣಿಕಾಪುಷ್ಪಂ ಸಮರ್ಪಯಾಮಿ~।\\
೪ ರಾಜರಾಜೇಶ್ವರ್ಯೈ ನಮಃ~। ಬದರೀಪುಷ್ಪಂ ಸಮರ್ಪಯಾಮಿ~।\\
೪ ಶ್ರೀಚಕ್ರವಾಸಿನ್ಯೈ ನಮಃ~। ಸರ್ವಾಣಿ ಪುಷ್ಪಾಣಿ ಸಮರ್ಪಯಾಮಿ~।

\section{ಅಥ ಚತುಃಷಷ್ಟಿ ಯೋಗಿನೀಪೂಜಾ}
\addcontentsline{toc}{section}{ಅಥ ಚತುಃಷಷ್ಟಿ ಯೋಗಿನೀಪೂಜಾ}
\begin{multicols}{2} ಓಂ ಬ್ರಹ್ಮಾಣ್ಯೈ~।\\ ಚಂಡಿಕಾಯೈ~।\\ ರೌದ್ರ್ಯೈ~।\\ ಗೌರ್ಯೈ~।\\ ಇಂದ್ರಾಣ್ಯೈ~।\\ ಕೌಮಾರ್ಯೈ~।\\ ವೈಷ್ಣವ್ಯೈ~।\\ ದುರ್ಗಾಯೈ~।\\ ನಾರಸಿಂಹ್ಯೈ~।\\ ಕಾಲಿಕಾಯೈ~।\\ ಚಾಮುಂಡಾಯೈ~।\\ ಶಿವದೂತ್ಯೈ~।\\ ವಾರಾಹ್ಯೈ~।\\ ಕೌಶಿಕ್ಯೈ~।\\ ಮಾಹೇಶ್ವರ್ಯೈ~।\\ ಶಾಂಕರ್ಯೈ~।\\ ಜಯಂತ್ಯೈ~।\\ ಸರ್ವಮಂಗಳಾಯೈ~।\\ ಕಾಳ್ಯೈ~।\\ ಕಪಾಲಿನ್ಯೈ~।\\ ಮೇಧಾಯೈ~।\\ ಶಿವಾಯೈ~।\\ ಶಾಕಂಭರ್ಯೈ~।\\ ಭೀಮಾಯೈ~।\\ ಶಾಂತಾಯೈ~।\\ ಭ್ರಾಮರ್ಯೈ~।\\ ರುದ್ರಾಣ್ಯೈ~।\\ ಅಂಬಿಕಾಯೈ~।\\ ಕ್ಷಮಾಯೈ~।\\ ಧಾತ್ರ್ಯೈ~।\\ ಸ್ವಾಹಾಯೈ~।\\ ಸ್ವಧಾಯೈ~।\\ ಅಪರ್ಣಾಯೈ~।\\ ಮಹೋದರ್ಯೈ~।\\ ಘೋರರೂಪಾಯೈ~।\\ ಮಹಾಕಾಳ್ಯೈ~।\\ ಭದ್ರಕಾಳ್ಯೈ~।\\ ಭಯಂಕರ್ಯೈ~।\\ ಕ್ಷೇಮಂಕರ್ಯೈ~।\\ ಉಗ್ರದಂಡಾಯೈ~।\\ ಚಂಡನಾಯಿಕಾಯೈ~।\\ ಚಂಡಾಯೈ~।\\ ಚಂಡವತ್ಯೈ~।\\ ಚಂಡ್ಯೈ~।\\ ಮಹಾಮೋಹಾಯೈ~।\\ ಪ್ರಿಯಂಕರ್ಯೈ~।\\ ಕಲವಿಕರಿಣ್ಯೈ~।\\ ದೇವ್ಯೈ~।\\ ಬಲಪ್ರಮಥಿನ್ಯೈ~।\\ ಮದನೋನ್ಮಥಿನ್ಯೈ~।\\ ಸರ್ವಭೂತದಮನಾಯೈ~।\\ ಉಮಾಯೈ~।\\ ತಾರಾಯೈ~।\\ ಮಹಾನಿದ್ರಾಯೈ~।\\ ವಿಜಯಾಯೈ~।\\ ಜಯಾಯೈ~।\\ ಶೈಲಪುತ್ರ್ಯೈ~।\\ ಚಂಡಘಂಟಾಯೈ~।\\ ಸ್ಕಂದಮಾತ್ರೇ~।\\ ಕಾಲರಾತ್ರ್ಯೈ~।\\ ಚಂಡಿಕಾಯೈ~।\\ ಕೂಷ್ಮಾಂಡಿನ್ಯೈ~।\\ ಕಾತ್ಯಾಯನ್ಯೈ~।\\ ಮಹಾಗೌರ್ಯೈ ನಮಃ~॥
\end{multicols}\authorline{ಇತಿ ಚತುಃಷಷ್ಟಿಯೋಗಿನೀಪೂಜಾ}
\section{ ಯೋಗಿನೀಪೂಜಾ}
\addcontentsline{toc}{section}{ ಯೋಗಿನೀಪೂಜಾ}
\begin{multicols}{2} \as{ಅಂ} ಅಮೃತಾಯೈ ನಮಃ~।\\ \as{ಆಂ} ಆಕರ್ಷಿಣ್ಯೈ ನಮಃ~।\\ \as{ಇಂ} ಇಂದ್ರಾಣ್ಯೈ ನಮಃ~।\\ \as{ಈಂ} ಈಶಾನ್ಯೈ ನಮಃ~।\\ \as{ಉಂ} ಉಮಾಯೈ ನಮಃ~।\\ \as{ಊಂ} ಊರ್ಧ್ವಕೇಶ್ಯೈ ನಮಃ~।\\ \as{ಋಂ} ಋದ್ಧಿದಾಯೈ ನಮಃ~।\\ \as{ೠಂ} ೠಕಾರಾಯೈ ನಮಃ~।\\ \as{ಲೃಂ} ಲೃಕಾರಾಯೈ ನಮಃ~।\\ \as{ಲೄಂ} ಲೄಕಾರಾಯೈ ನಮಃ~।\\ \as{ಏಂ} ಏಕಪದಾಯೈ ನಮಃ~।\\ \as{ಐಂ} ಐಶ್ವರ್ಯಾತ್ಮಿಕಾಯೈ ನಮಃ~।\\ \as{ಓಂ} ಓಂಕಾರಾಯೈ ನಮಃ~।\\ \as{ಔಂ} ಔಷಧ್ಯೈ ನಮಃ~।\\ \as{ಅಂ} ಅಂಬಿಕಾಯೈ ನಮಃ~।\\ \as{ಅಃ} ಅಕ್ಷರಾಯೈ ನಮಃ~।\\ \as{ಕಂ} ಕಾಲರಾತ್ರ್ಯೈ ನಮಃ~।\\ \as{ಖಂ} ಖಂಡಿತಾಯೈ ನಮಃ~।\\ \as{ಗಂ} ಗಾಯತ್ರ್ಯೈ ನಮಃ~।\\ \as{ಘಂ} ಘಂಟಾಕರ್ಷಿಣ್ಯೈ ನಮಃ~।\\ \as{ಙಂ} ಙಾರ್ಣಾಯೈ ನಮಃ~।\\ \as{ಚಂ} ಚಂಡಾಯೈ ನಮಃ~।\\ \as{ಛಂ} ಛಾಯಾಯೈ ನಮಃ~।\\ \as{ಜಂ} ಜಯಾಯೈ ನಮಃ~।\\ \as{ಝಂ} ಝಂಕಾರಿಣ್ಯೈ ನಮಃ~।\\ \as{ಞಂ} ಜ್ಞಾನರೂಪಾಯೈ ನಮಃ~।\\ \as{ಟಂ} ಟಂಕಹಸ್ತಾಯೈ ನಮಃ~।\\ \as{ಠಂ} ಠಂಕಾರಿಣ್ಯೈ ನಮಃ~।\\ \as{ಡಂ} ಡಾಮರ್ಯೈ ನಮಃ~।\\ \as{ಢಂ} ಢಂಕಾರಿಣ್ಯೈ ನಮಃ~।\\ \as{ಣಂ} ಣಾರ್ಣಾಯೈ ನಮಃ~।\\ \as{ತಂ} ತಾಮಸ್ಯೈ ನಮಃ~।\\ \as{ಥಂ} ಸ್ಥಾಣ್ವ್ಯೈ ನಮಃ~।\\ \as{ದಂ} ದಾಕ್ಷಾಯಣ್ಯೈ ನಮಃ~।\\ \as{ಧಂ} ಧಾತ್ರ್ಯೈ ನಮಃ~।\\ \as{ನಂ} ನಾರ್ಯೈ ನಮಃ~।\\ \as{ಪಂ} ಪಾರ್ವತ್ಯೈ ನಮಃ~।\\ \as{ಫಂ} ಫಟ್ಕಾರಿಣ್ಯೈ ನಮಃ~।\\ \as{ಬಂ} ಬಂಧಿನ್ಯೈ ನಮಃ ~।\\ \as{ಭಂ} ಭದ್ರಕಾಲ್ಯೈ ನಮಃ ~।\\ \as{ಮಂ} ಮಹಾಮಾಯಾಯೈ ನಮಃ~।\\ \as{ಯಂ} ಯಶಸ್ವಿನ್ಯೈ ನಮಃ~।\\ \as{ರಂ} ರಕ್ತಾಯೈ ನಮಃ~।\\ \as{ಲಂ} ಲಂಬೋಷ್ಠ್ಯೈ ನಮಃ~।\\ \as{ವಂ} ವರದಾಯೈ ನಮಃ~।\\ \as{ಶಂ} ಶ್ರಿಯೈ ನಮಃ~।\\ \as{ಷಂ} ಷಂಡಾಯೈ ನಮಃ~।\\ \as{ಸಂ} ಸರಸ್ವತ್ಯೈ ನಮಃ~।\\ \as{ಹಂ} ಹಂಸವತ್ಯೈ ನಮಃ~।\\ \as{ಕ್ಷಂ} ಕ್ಷಮಾವತ್ಯೈ ನಮಃ ॥
\end{multicols}
\section{ಶಿವ ಆವರಣಪೂಜಾ}
\addcontentsline{toc}{section}{ಶಿವ ಆವರಣಪೂಜಾ}
\subsection{ಪ್ರಥಮಾವರಣಮ್}
ಓಂ ಓಂ ಹೃದಯಾಂಗ ದೇವತಾಭ್ಯೋ ನಮಃ । ಓಂ ನಂ ಶಿರೋಽಂಗ \\ದೇವತಾಭ್ಯೋ ನಮಃ । ಓಂ ಮಂ ಶಿಖಾಂಗ ದೇವತಾಭ್ಯೋ ನಮಃ । ಓಂ ಶಿಂ ಕವಚಾಂಗ ದೇವತಾಭ್ಯೋ ನಮಃ । ಓಂ ವಾಂ ನೇತ್ರಾಂಗ ದೇವತಾಭ್ಯೋ ನಮಃ। ಓಂ ಯಂ ಅಸ್ತ್ರಾಂಗ ದೇವತಾಭ್ಯೋ ನಮಃ ॥\\
ಅಭೀಷ್ಟಸಿದ್ಧಿಂ ಮೇ ದೇಹಿ ಶರಣಾಗತ ವತ್ಸಲ~।\\
ಭಕ್ತ್ಯಾ ಸಮರ್ಪಯೇ ತುಭ್ಯಂ ಪ್ರಥಮಾವರಣಾರ್ಚನಂ ॥
\subsection{ದ್ವಿತೀಯಾವರಣಮ್}
ಓಂ ಗಗನಾ ತತ್ಪುರುಷಾಭ್ಯಾಂ ನಮಃ । ಓಂ  ಕರಾಲಿಕಾ ವಾಮದೇವಾಭ್ಯಾಂ ನಮಃ । ಓಂ  ರಕ್ತಾಘೋರಾಭ್ಯಾಂ ನಮಃ । ಓಂ  ಮಹೋಚ್ಛುಷ್ಮಾ \\ಸದ್ಯೋಜಾತಾಭ್ಯಾಂ ನಮಃ । ಓಂ  ಹೃಲ್ಲೇಖೇಶಾನಾಭ್ಯಾಂ ನಮಃ ॥\\
ಅಭೀಷ್ಟಸಿದ್ಧಿಂ******ದ್ವಿತೀಯಾವರಣಾರ್ಚನಂ ॥
\subsection{ತೃತೀಯಾವರಣಮ್}
ಓಂ ಗಾಯತ್ರ್ಯೈ ನಮಃ । ಓಂ  ಸಾವಿತ್ರ್ಯೈ ನಮಃ । ಓಂ ಸರಸ್ವತ್ಯೈ ನಮಃ । ಓಂ  ವಸುಧಾ ಶಂಖನಿಧಿಭ್ಯಾಂ ನಮಃ । ಓಂ  ವಸುಮತೀ ಪುಷ್ಪನಿಧಿಭ್ಯಾಂ ನಮಃ ॥\\
ಅಭೀಷ್ಟಸಿದ್ಧಿಂ******ತೃತೀಯಾವರಣಾರ್ಚನಂ ॥
\subsection{ಚತುರ್ಥಾವರಣಮ್}
ಓಂ ಅನಂತಾಯ ನಮಃ । ಓಂ  ಸೂಕ್ಷ್ಮಾಯ ನಮಃ । ಓಂ  ಶಿವೋತ್ತಮಾಯ ನಮಃ। ಓಂ  ಏಕನೇತ್ರಾಯ ನಮಃ । ಓಂ  ಏಕರುದ್ರಾಯ ನಮಃ । \\ಓಂ   ತ್ರಿಮೂರ್ತಯೇ ನಮಃ । ಓಂ  ಶ್ರೀಕಂಠಾಯ ನಮಃ ।\\ ಓಂ  ಶಿಖಂಡಿನೇ ನಮಃ ॥\\
ಅಭೀಷ್ಟಸಿದ್ಧಿಂ******ತುರೀಯಾವರಣಾರ್ಚನಂ ॥
\subsection{ಪಂಚಮಾವರಣಮ್}
ಓಂ ಉಮಾಯೈ ನಮಃ । ಓಂ  ಚಂಡೇಶ್ವರಾಯ ನಮಃ । ಓಂ  ನಂದಿನೇ ನಮಃ । ಓಂ  ಮಹಾಕಾಲಾಯ ನಮಃ । ಓಂ  ಗಣೇಶ್ವರಾಯ ನಮಃ । ಓಂ  ಸ್ಕಂದಾಯ ನಮಃ । ಓಂ  ವೀರಭದ್ರಾಯ ನಮಃ । ಓಂ  ಭೃಂಗಿರಿಟಾಯ ನಮಃ ॥\\
ಅಭೀಷ್ಟಸಿದ್ಧಿಂ******ಪಂಚಮಾವರಣಾರ್ಚನಂ ॥
\subsection{ಷಷ್ಠಾವರಣಮ್}
ಓಂ ಬ್ರಾಹ್ಮ್ಯೈ ನಮಃ । ಓಂ ಮಾಹೇಶ್ವರ್ಯೈ ನಮಃ । ಓಂ ಕೌಮಾರ್ಯೈ ನಮಃ । ಓಂ ವೈಷ್ಣವ್ಯೈ ನಮಃ । ಓಂ ವಾರಾಹ್ಯೈ ನಮಃ । ಓಂ ಮಾಹೇಂದ್ರ್ಯೈ ನಮಃ । ಓಂ ಚಾಮುಂಡಾಯೈ ನಮಃ । ಓಂ ಮಹಾಲಕ್ಷ್ಮ್ಯೈ ನಮಃ ॥\\
ಅಭೀಷ್ಟಸಿದ್ಧಿಂ******ಷಷ್ಠಾಖ್ಯಾವರಣಾರ್ಚನಂ ॥
\subsection{ಸಪ್ತಮಾವರಣಮ್}
ಓಂ ಅಸಿತಾಂಗಭೈರವಾಯ ನಮಃ । ಓಂ ರುರುಭೈರವಾಯ ನಮಃ । ಓಂ ಚಂಡಭೈರವಾಯ ನಮಃ । ಓಂ ಕ್ರೋಧಭೈರವಾಯ ನಮಃ । ಓಂ ಉನ್ಮತ್ತಭೈರವಾಯ ನಮಃ । ಓಂ ಕಪಾಲಭೈರವಾಯ ನಮಃ । ಓಂ ಭೀಷಣಭೈರವಾಯ ನಮಃ । ಓಂ ಸಂಹಾರಭೈರವಾಯ ನಮಃ ॥\\%। ಓಂ ವರಪ್ರದಭೈರವಾಯ ನಮಃ
ಅಭೀಷ್ಟಸಿದ್ಧಿಂ******ದಸಪ್ತಮಾವರಣಾರ್ಚನಂ ॥
\subsection{ಅಷ್ಟಮಾವರಣಮ್}
ಓಂ ಲಂ ಇಂದ್ರಾಯ ನಮಃ । ಓಂ ರಂ ಅಗ್ನಯೇ ನಮಃ । ಓಂ ಮಂ ಯಮಾಯ ನಮಃ । ಓಂ ಕ್ಷಂ ನಿರ್ಋತಯೇ ನಮಃ ।ಓಂ ವಂ ವರುಣಾಯ ನಮಃ ।ಓಂ ಯಂ ವಾಯವೇ ನಮಃ । ಓಂ ಕುಂ ಕುಬೇರಾಯ ನಮಃ । ಓಂ ಹಂ ಈಶಾನಾಯ ನಮಃ । ಓಂ ಆಂ ಬ್ರಹ್ಮಣೇ ನಮಃ । ಓಂ ಹ್ರೀಂ ಅನಂತಾಯ ನಮಃ । ಓಂ ನಿಯತ್ಯೈ ನಮಃ । ಓಂ ಕಾಲಾಯ ॥\\
ಅಭೀಷ್ಟಸಿದ್ಧಿಂ******ಅಷ್ಟಮಾವರಣಾರ್ಚನಂ ॥
\subsection{ನವಮಾವರಣಮ್}
ಓಂ ವಂ ವಜ್ರಾಯ ನಮಃ । ಓಂ ಶಂ ಶಕ್ತ್ಯೈ ನಮಃ । ಓಂ ದಂ ದಂಡಾಯ ನಮಃ । ಓಂ ಖಂ ಖಡ್ಗಾಯ ನಮಃ । ಓಂ ಪಾಂ ಪಾಶಾಯ ನಮಃ । ಓಂ ಅಂ ಅಂಕುಶಾಯ ನಮಃ । ಓಂ ಗಂ ಗದಾಯೈ ನಮಃ । ಓಂ  ತ್ರಿಂ  ತ್ರಿಶೂಲಾಯ ನಮಃ । ಓಂ ಪಂ ಪದ್ಮಾಯ ನಮಃ । ಓಂ ಚಂ ಚಕ್ರಾಯ ನಮಃ ॥\\
ಅಭೀಷ್ಟಸಿದ್ಧಿಂ******ನವಮಾವರಣಾರ್ಚನಂ ॥
\section{ಅಥ ಶಿವಾಷ್ಟೋತ್ತರಶತನಾಮಸ್ತೋತ್ರಮ್ }
\addcontentsline{toc}{section}{ಶಿವಾಷ್ಟೋತ್ತರಶತನಾಮಸ್ತೋತ್ರಮ್ }
ಶಿವೋ ಮಹೇಶ್ವರಃ ಶಂಭುಃ ಪಿನಾಕೀ ಶಶಿಶೇಖರಃ~।\\
ವಾಮದೇವೋ ವಿರೂಪಾಕ್ಷಃ ಕಪರ್ದೀ ನೀಲಲೋಹಿತಃ ॥೧ ॥

ಶಂಕರಃ ಶೂಲಪಾಣಿಶ್ಚ ಖಟ್ವಾಂಗೀ ವಿಷ್ಣುವಲ್ಲಭಃ~।\\
ಶಿಪಿವಿಷ್ಟೋಂಬಿಕಾನಾಥಃ ಶ್ರೀಕಂಠೋ ಭಕ್ತವತ್ಸಲಃ ॥೨ ॥

ಭವಃ ಶರ್ವಸ್ತ್ರಿಲೋಕೇಶಃ ಶಿತಿಕಂಠಃ ಶಿವಾಪ್ರಿಯಃ।\\
ಉಗ್ರಃ ಕಪಾಲೀ ಕಾಮಾರಿರಂಧಕಾಸುರಸೂದನಃ ॥೩ ॥

ಗಂಗಾಧರೋ ಲಲಾಟಾಕ್ಷಃ ಕಾಲಕಾಲಃ ಕೃಪಾನಿಧಿಃ~।\\
ಭೀಮಃ ಪರಶುಹಸ್ತಶ್ಚ ಮೃಗಪಾಣಿರ್ಜಟಾಧರಃ ॥೪ ॥

ಕೈಲಾಸವಾಸೀ ಕವಚೀ ಕಠೋರಸ್ತ್ರಿಪುರಾಂತಕಃ।\\
ವೃಷಾಂಕೋ ವೃಷಭಾರೂಢೋ ಭಸ್ಮೋದ್ಧೂಲಿತವಿಗ್ರಹಃ ॥೫ ॥

ಸಾಮಪ್ರಿಯಃ ಸ್ವರಮಯಸ್ತ್ರಯೀಮೂರ್ತಿರನೀಶ್ವರಃ~।\\
ಸರ್ವಜ್ಞಃ ಪರಮಾತ್ಮಾ ಚ ಸೋಮಸೂರ್ಯಾಗ್ನಿಲೋಚನಃ ॥೬॥

ಹವಿರ್ಯಜ್ಞಮಯಃ ಸೋಮಃ ಪಂಚವಕ್ತ್ರಃ ಸದಾಶಿವಃ।\\
ವಿಶ್ವೇಶ್ವರೋ ವೀರಭದ್ರೋ ಗಣನಾಥಃ ಪ್ರಜಾಪತಿಃ ॥೭॥

ಹಿರಣ್ಯರೇತಾ ದುರ್ಧರ್ಷೋ ಗಿರೀಶೋ ಗಿರಿಶೋನಘಃ।\\
ಭುಜಂಗಭೂಷಣೋ ಭರ್ಗೋ ಗಿರಿಧನ್ವಾ ಗಿರಿಪ್ರಿಯಃ ॥೮॥

ಕೃತ್ತಿವಾಸಾಃ ಪುರಾರಾತಿರ್ಭಗವಾನ್ ಪ್ರಮಥಾಧಿಪಃ।\\
ಮೃತ್ಯುಂಜಯಃ ಸೂಕ್ಷ್ಮತನುರ್ಜಗದ್ವ್ಯಾಪೀ ಜಗದ್ಗುರುಃ ॥೯॥

ವ್ಯೋಮಕೇಶೋ ಮಹಾಸೇನಜನಕಶ್ಚಾರುವಿಕ್ರಮಃ।\\
ರುದ್ರೋ ಭೂತಪತಿಃ ಸ್ಥಾಣುರಹಿರ್ಬುಧ್ನ್ಯೋ ದಿಗಂಬರಃ ॥೧೦॥

ಅಷ್ಟಮೂರ್ತಿರನೇಕಾತ್ಮಾ ಸಾತ್ವಿಕಃ ಶುದ್ಧವಿಗ್ರಹಃ।\\
ಶಾಶ್ವತಃ ಖಂಡಪರಶುರಜಃ ಪಾಶವಿಮೋಚನಃ ॥೧೧॥

ಮೃಡಃ ಪಶುಪತಿರ್ದೇವೋ ಮಹಾದೇವೋಽವ್ಯಯೋ ಹರಿಃ।\\
ಪೂಷದಂತಭಿದವ್ಯಗ್ರೋ ದಕ್ಷಾಧ್ವರಹರೋ ಹರಃ ॥೧೨॥

ಭಗನೇತ್ರಭಿದವ್ಯಕ್ತಃ ಸಹಸ್ರಾಕ್ಷಃ ಸಹಸ್ರಪಾತ್।\\
ಅಪವರ್ಗಪ್ರದೋಽನಂತಸ್ತಾರಕಃ ಪರಮೇಶ್ವರಃ ॥೧೩॥

\authorline{॥ಇತಿ ಶಿವಾಷ್ಟೋತ್ತರ ಶತನಾಮಸ್ತೋತ್ರಂ ಸಂಪೂರ್ಣಂ॥}
\section{ಅಥ ನಿತ್ಯಕ್ಲಿನ್ನಾ ಯಜನವಿಧಿಃ\\೧। ಶ್ರೀಕಾಮೇಶ್ವರೀನಿತ್ಯಾ}
\addcontentsline{toc}{section}{ಅಥ ನಿತ್ಯಕ್ಲಿನ್ನಾ ಯಜನವಿಧಿಃ}
\addcontentsline{toc}{section}{೧। ಶ್ರೀಕಾಮೇಶ್ವರೀನಿತ್ಯಾ}
ಅಸ್ಯ ಶ್ರೀಕಾಮೇಶ್ವರೀನಿತ್ಯಾಮಹಾಮಂತ್ರಸ್ಯ ಸಮ್ಮೋಹನ ಋಷಿಃ~। ಗಾಯತ್ರೀ ಛಂದಃ~। ಶ್ರೀಕಾಮೇಶ್ವರೀದೇವತಾ~। ಕಂ ಬೀಜಂ~। ಇಂ ಶಕ್ತಿಃ। ಲಂ ಕೀಲಕಂ~।\\
\as{ನ್ಯಾಸಃ :}೧.ಓಂ ಐಂ ೨.ಓಂ ಸಕಲಹ್ರೀಂ ೩.ಓಂ ನಿತ್ಯ  ೪.ಓಂ ಕ್ಲಿನ್ನೇ ೫.ಓಂ ಮದದ್ರವೇ ೬.ಓಂ ಸೌಃ \\
{\bfseries ದೇವೀಂ ಧ್ಯಾಯೇಜ್ಜಗದ್ಧಾತ್ರೀಂ ಜಪಾಕುಸುಮಸನ್ನಿಭಾಂ~।\\
ಬಾಲಭಾನುಪ್ರತೀಕಾಶಾಂ ಶಾತಕುಂಭಸಮಪ್ರಭಾಂ ॥\\
ರಕ್ತವಸ್ತ್ರಪರೀಧಾನಾಂ ಸಂಪದ್ವಿದ್ಯಾವಶಂಕರೀಂ~।\\
ನಮಾಮಿ ವರದಾಂ ದೇವೀಂ ಕಾಮೇಶೀಮಭಯಪ್ರದಾಂ ॥\\}
ಮನುಃ :{\bfseries  ಐಂ ಸಕಲಹ್ರೀಂ ನಿತ್ಯಕ್ಲಿನ್ನೇ ಮದದ್ರವೇ ಸೌಃ~॥}

ಓಂ ದ್ರಾಂ ಮದನಬಾಣಾಯ ।  ದ್ರೀಂ ಉನ್ಮಾದನಬಾಣಾಯ ।  ಕ್ಲೀಂ ದೀಪನಬಾಣಾಯ ।  ಬ್ಲೂಂ ಮೋಹನಬಾಣಾಯ ।  ಸಃ ಶೋಷಣಬಾಣಾಯ ನಮಃ~॥

ಓಂ ಅನಂಗಕುಸುಮಾ~।  ಅನಂಗಮೇಖಲಾ~।  ಅನಂಗಮದನಾ~।  ಅನಂಗಮದನಾತುರಾ~।  ಅನಂಗಮದವೇಗಿನೀ~।  ಅನಂಗಭುವನಪಾಲಾ~।  ಅನಂಗಶಶಿರೇಖಾ~।  ಅನಂಗಗಗನರೇಖಾ~॥ 

ಓಂ ಅಂ ಶ್ರದ್ಧಾ~।  ಆಂ ಪ್ರೀತಿ~।  ಇಂ ರತಿ~।  ಈಂ ಧೃತಿ~।  ಉಂ ಕಾಂತಿ~।  ಊಂ ಮನೋರಮಾ~।  ಋಂ ಮನೋಹರಾ~।  ೠಂ ಮನೋರಥಾ~।  ಲೃಂ ಮದನಾ~।  ಲೄಂ ಉನ್ಮಾದಿನೀ~।  ಏಂ ಮೋಹಿನೀ~।  ಐಂ ಶಂಖಿನೀ~। ಓಂ  ಶೋಷಿಣೀ~।  ಔಂ ವಶಂಕರೀ~।  ಅಂ ಶಿಂಜಿನೀ~।  ಅಃ ಸುಭಗಾ~॥ 

ಓಂ ಅಂ ಪೂಷಾ~।  ಆಂ ಇದ್ಧಾ~।  ಇಂ ಸುಮನಸಾ~।  ಈಂ ರತಿ~।  ಉಂ ಪ್ರೀತಿ~।  ಊಂ ಧೃತಿ~।  ಋಂ ಋದ್ಧಿ~।  ೠಂ ಸೌಮ್ಯಾ~।  ಲೃಂ ಮರೀಚಿ~।  ಲೄಂ ಅಂಶುಮಾಲಿನೀ~।  ಏಂ ಶಶಿನೀ~।  ಐಂ ಅಂಗಿರಾ~। ಓಂ ಛಾಯಾ~।  ಔಂ ಸಂಪೂರ್ಣಮಂಡಲಾ~।  ಅಂ ತುಷ್ಟಿ~।  ಅಃ ಅಮೃತಾ~॥ 

ಓಂ ಡಾಕಿನೀ~।  ರಾಕಿಣೀ~।  ಲಾಕಿನೀ~।  ಕಾಕಿನೀ~।  ಸಾಕಿನೀ~।  ಹಾಕಿನೀ~॥ 

ಓಂ ವಂ ವಟುಕ~।  ಗಂ ಗಣಪತಿ~।  ದುಂ ದುರ್ಗಾ~।  ಕ್ಷಂ ಕ್ಷೇತ್ರಪಾಲ~॥ 

ಓಂ ಲಂ ಇಂದ್ರಶಕ್ತಿ~।  ರಂ ಅಗ್ನಿಶಕ್ತಿ~।  ಮಂ ಯಮಶಕ್ತಿ~।  ಕ್ಷಂ ನಿರ್ಋತಿಶಕ್ತಿ~।  ವಂ ವರುಣಶಕ್ತಿ~।  ಯಂ ವಾಯುಶಕ್ತಿ~।  ಕುಂ ಕುಬೇರಶಕ್ತಿ~।  ಹಂ ಈಶಾನಶಕ್ತಿ~।  ಆಂ ಬ್ರಹ್ಮಶಕ್ತಿ~।  ಹ್ರೀಂ ಅನಂತಶಕ್ತಿ~॥ 

ಓಂ ವಂ ವಜ್ರಶಕ್ತಿ~।  ಶಂ ಶಕ್ತಿಶಕ್ತಿ~।  ದಂ ದಂಡಶಕ್ತಿ~।  ಖಂ ಖಡ್ಗಶಕ್ತಿ~।  ಪಾಂ ಪಾಶಶಕ್ತಿ~।  ಅಂ ಅಂಕುಶಶಕ್ತಿ~।  ಗಂ ಗದಾಶಕ್ತಿ~।  ತ್ರಿಂ ತ್ರಿಶೂಲಶಕ್ತಿ~।  ಪಂ ಪದ್ಮಶಕ್ತಿ~।  ಚಂ ಚಕ್ರಶಕ್ತಿ~।\\ ಕಾಮೇಶ್ವರ್ಯೈ ವಿದ್ಮಹೇ ನಿತ್ಯಕ್ಲಿನ್ನಾಯೈ ಧೀಮಹಿ ।\\ತನ್ನೋ ನಿತ್ಯಾ ಪ್ರಚೋದಯಾತ್~॥\\
ಇತಿ ಕಾಮೇಶ್ವರೀಆವರಣಪೂಜಾ~।
\section{೨। ಭಗಮಾಲಿನೀನಿತ್ಯಾ}
\addcontentsline{toc}{section}{೨। ಭಗಮಾಲಿನೀನಿತ್ಯಾ}
ಅಸ್ಯ ಶ್ರೀಭಗಮಾಲಿನೀನಿತ್ಯಾಮಹಾಮಂತ್ರಸ್ಯ ಸುಭಗಋಷಿಃ~। ಗಾಯತ್ರೀ ಛಂದಃ~। ಶ್ರೀಭಗಮಾಲಿನೀ ದೇವತಾ~। ಹ್‌ರ್‌ಬ್ಲೇಂ ಬೀಜಂ~।  ಶ್ರೀಂ ಶಕ್ತಿಃ~। ಕ್ಲೀಂ ಕೀಲಕಂ~।\\
\as{ನ್ಯಾಸಃ :}೧.ಓಂ ಐಂ  ೨.ಓಂ ಭಗಭುಗೇ ೩.ಓಂ ಭಗಿನಿ  ೪.ಓಂ ಭಗೋದರಿ  ೫.ಓಂ ಭಗಮಾಲೇ ೬.ಓಂ ಭಗಾವಹೇ\\
{\bfseries ಭಗರೂಪಾಂ ಭಗಮಯಾಂ ದುಕೂಲವಸನಾಂ ಶಿವಾಂ~।\\
ಸರ್ವಾಲಂಕಾರಸಂಯುಕ್ತಾಂ ಸರ್ವಲೋಕವಶಂಕರೀಂ ॥\\
ಭಗೋದರೀಂ ಮಹಾದೇವೀಂ ರಕ್ತೋತ್ಪಲಸಮಪ್ರಭಾಂ~।\\
ಕಾಮೇಶ್ವರಾಂಕನಿಲಯಾಂ ವಂದೇ ಶ್ರೀಭಗಮಾಲಿನೀಂ ॥\\}
ಮನುಃ :{\bfseries  ಐಂ ಭಗಭುಗೇ ಭಗಿನಿ ಭಗೋದರಿ ಭಗಮಾಲೇ ಭಗಾವಹೇ ಭಗಗುಹ್ಯೇ ಭಗಯೋನಿ ಭಗನಿಪಾತನಿ ಸರ್ವಭಗವಶಂಕರಿ ಭಗರೂಪೇ ನಿತ್ಯಕ್ಲಿನ್ನೇ ಭಗಸ್ವರೂಪೇ ಸರ್ವಾಣಿ ಭಗಾನಿ ಮೇ ಹ್ಯಾನಯ ವರದೇ ರೇತೇ ಸುರೇತೇ ಭಗಕ್ಲಿನ್ನೇ ಕ್ಲಿನ್ನದ್ರವೇ ಕ್ಲೇದಯ ದ್ರಾವಯ ಅಮೋಘೇ ಭಗವಿಚ್ಚೇ ಕ್ಷುಭ ಕ್ಷೋಭಯ ಸರ್ವಸತ್ವಾನ್ ಭಗೇಶ್ವರಿ ಐಂ ಬ್ಲೂಂ ಜಂ ಬ್ಲೂಂ ಭೇಂ ಬ್ಲೂಂ ಮೋಂ ಬ್ಲೂಂ ಹೇಂ ಬ್ಲೂಂ ಹೇಂ ಕ್ಲಿನ್ನೇ ಸರ್ವಾಣಿ ಭಗಾನಿ ಮೇ ವಶಮಾನಯ ಸ್ತ್ರೀಂ ಹ್‌ರ್‌ಬ್ಲೇಂ ಹ್ರೀಂ ॥}

ಓಂ ರಾಗಶಕ್ತಿ~।  ದ್ವೇಷಶಕ್ತಿ~।  ಮದನಾ~।  ಮೋಹಿನೀ~।  ಲೋಲಾ~।  ಜಂಭಿನೀ~।  ಉದ್ಯಮಾ~।  ಶುಭಾ~।  ಹ್ಲಾದಿನೀ~।  ದ್ರಾವಿಣೀ~।  ಪ್ರೀತಿ~।  ರತಿ~।  ರಕ್ತಾ~।  ಮನೋರಮಾ~।  ಸರ್ವೋನ್ಮಾದಾ~।  ಸರ್ವಸುಖಾ~।  ಅನಂಗಾ~।  ಅಭಿತೋದ್ಯಮಾ~। ಅನಲ್ಪಾ~।  ವ್ಯಕ್ತವಿಭವಾ~।  ವಿವಿಧವಿಗ್ರಹಾ~।  ಕ್ಷೋಭವಿಗ್ರಹಾ~॥ 

ಓಂ ಇಕ್ಷುಕೋದಂಡ~।  ಪಾಶ~।  ಕಹ್ಲಾರ~।  ಪದ್ಮ~।  ಅಂಕುಶ~।  ಪುಷ್ಪಸಾಯಕ~॥ 

ಓಂ ಲಂ ಇಂದ್ರಶಕ್ತಿ~।  ರಂ ಅಗ್ನಿಶಕ್ತಿ~।  ಮಂ ಯಮಶಕ್ತಿ~।  ಕ್ಷಂ ನಿರ್ಋತಿಶಕ್ತಿ~।  ವಂ ವರುಣಶಕ್ತಿ~।  ಯಂ ವಾಯುಶಕ್ತಿ~।  ಕುಂ ಕುಬೇರಶಕ್ತಿ~।  ಹಂ ಈಶಾನಶಕ್ತಿ~।  ಆಂ ಬ್ರಹ್ಮಶಕ್ತಿ~।  ಹ್ರೀಂ ಅನಂತಶಕ್ತಿ~।  ನಿಯತಿಶಕ್ತಿ~।  ಕಾಲಶಕ್ತಿ~॥ 

ಓಂ ವಂ ವಜ್ರಶಕ್ತಿ~।  ಶಂ ಶಕ್ತಿಶಕ್ತಿ~।  ದಂ ದಂಡಶಕ್ತಿ~।  ಖಂ ಖಡ್ಗಶಕ್ತಿ~।  ಪಾಂ ಪಾಶಶಕ್ತಿ~।  ಅಂ ಅಂಕುಶಶಕ್ತಿ~।  ಗಂ ಗದಾಶಕ್ತಿ~।  ತ್ರಿಂ ತ್ರಿಶೂಲಶಕ್ತಿ~।  ಪಂ ಪದ್ಮಶಕ್ತಿ~।  ಚಂ ಚಕ್ರಶಕ್ತಿ~॥ 

ಭಗಮಾಲಿನ್ಯೈ ವಿದ್ಮಹೇ ಸರ್ವವಶಂಕರ್ಯೈ ಧೀಮಹಿ ।\\ತನ್ನೋ ನಿತ್ಯಾ ಪ್ರಚೋದಯಾತ್~॥\\
ಇತಿ ಭಗಮಾಲಿನೀ ಆವರಣಪೂಜಾ
\section{೩।ನಿತ್ಯಕ್ಲಿನ್ನಾ}
\addcontentsline{toc}{section}{೩।ನಿತ್ಯಕ್ಲಿನ್ನಾ}
ಅಸ್ಯ ಶ್ರೀನಿತ್ಯಕ್ಲಿನ್ನಾಮಹಾಮಂತ್ರಸ್ಯ ಬ್ರಹ್ಮಾ ಋಷಿಃ। ವಿರಾಟ್ಛಂದಃ~।\\ ಶ್ರೀನಿತ್ಯಕ್ಲಿನ್ನಾನಿತ್ಯಾದೇವತಾ। ಹ್ರೀಂ ಬೀಜಂ। ಸ್ವಾಹಾ ಶಕ್ತಿಃ । ನ್ನೇ ಕೀಲಕಂ~।\\
\as{ನ್ಯಾಸಃ :}೧.ಓಂ ಹ್ರೀಂ ೨.ಓಂ ನಿತ್ಯ  ೩.ಓಂ ಕ್ಲಿನ್ನೇ ೪.ಓಂ ಮದ ೫.ಓಂ ದ್ರವೇ ೬.ಓಂ ಸ್ವಾಹಾ\\
{\bfseries ಪದ್ಮರಾಗಮಣಿಪ್ರಖ್ಯಾಂ ಹೇಮತಾಟಂಕಸಂಯುತಾಂ~।\\
ರಕ್ತವಸ್ತ್ರಧರಾಂ ದೇವೀಂ ರಕ್ತಮಾಲ್ಯಾನುಲೇಪನಾಂ~।\\
ಅಂಜನಾಂಚಿತನೇತ್ರಾಂ ತಾಂ ಪದ್ಮಪತ್ರನಿಭೇಕ್ಷಣಾಂ~।\\
ನಿತ್ಯಕ್ಲಿನ್ನಾಂ ನಮಸ್ಯಾಮಿ ಚತುರ್ಭುಜವಿರಾಜಿತಾಂ~॥\\}
ಮನುಃ :{\bfseries  ಹ್ರೀಂ ನಿತ್ಯಕ್ಲಿನ್ನೇ ಮದದ್ರವೇ ಸ್ವಾಹಾ ॥}

ಓಂ ಮದಾವಿಲಾ~।  ಮಂಗಲಾ~।  ಮನ್ಮಥಾರ್ತಾ~।  ಮನಸ್ವಿನೀ~।  ಮೋಹಾ~।  ಆಮೋದಾ~।  ಮಾನಮಯೀ~।  ಮಾಯಾ~।  ಮಂದಾ~।  ಮನೋವತೀ~॥ 

ಓಂ ನಿತ್ಯಾ~।  ನಿರಂಜನಾ~।  ಕ್ಲಿನ್ನಾ~।  ಕ್ಲೇದಿನೀ~।  ಮದನಾತುರಾ~।  ಮದದ್ರವಾ~।  ದ್ರಾವಿಣೀ~।  ದ್ರವಿಣಾ~॥ 

ಓಂ ಕ್ಷೋಭಿಣೀ~।  ಮೋಹಿನೀ~।  ಲೋಲಾ~॥

ಓಂ ಲಂ ಇಂದ್ರಶಕ್ತಿ~।  ರಂ ಅಗ್ನಿಶಕ್ತಿ~।  ಮಂ ಯಮಶಕ್ತಿ~।  ಕ್ಷಂ ನಿರ್ಋತಿಶಕ್ತಿ~।  ವಂ ವರುಣಶಕ್ತಿ~।  ಯಂ ವಾಯುಶಕ್ತಿ~।  ಕುಂ ಕುಬೇರಶಕ್ತಿ~।  ಹಂ ಈಶಾನಶಕ್ತಿ~।  ಆಂ ಬ್ರಹ್ಮಶಕ್ತಿ~।  ಹ್ರೀಂ ಅನಂತಶಕ್ತಿ~।  ನಿಯತಿಶಕ್ತಿ~।  ಕಾಲಶಕ್ತಿ~॥ 

ಓಂ ವಂ ವಜ್ರಶಕ್ತಿ~।  ಶಂ ಶಕ್ತಿಶಕ್ತಿ~।  ದಂ ದಂಡಶಕ್ತಿ~।  ಖಂ ಖಡ್ಗಶಕ್ತಿ~।  ಪಾಂ ಪಾಶಶಕ್ತಿ~।  ಅಂ ಅಂಕುಶಶಕ್ತಿ~।  ಗಂ ಗದಾಶಕ್ತಿ~।  ತ್ರಿಂ ತ್ರಿಶೂಲಶಕ್ತಿ~।  ಪಂ ಪದ್ಮಶಕ್ತಿ~।  ಚಂ ಚಕ್ರಶಕ್ತಿ~॥ 

ನಿತ್ಯಕ್ಲಿನ್ನಾಯೈ ವಿದ್ಮಹೇ ನಿತ್ಯಮದದ್ರವಾಯೈ ಧೀಮಹಿ ।\\ತನ್ನೋ ನಿತ್ಯಾ ಪ್ರಚೋದಯಾತ್ ॥\\
ಇತಿ ನಿತ್ಯಕ್ಲಿನ್ನಾ ಆವರಣಪೂಜಾ
\section{೪।ಭೇರುಂಡಾ}
\addcontentsline{toc}{section}{೪।ಭೇರುಂಡಾ}
ಶ್ರೀಭೇರುಂಡಾನಿತ್ಯಾಮಹಾಮಂತ್ರಸ್ಯ ಮಹಾವಿಷ್ಣುಃ ಋಷಿಃ। \\ಗಾಯತ್ರೀಛಂದಃ~। ಭೇರುಂಡಾನಿತ್ಯಾ ದೇವತಾ। ಭ್ರೋಂ ಬೀಜಂ।\\ ಸ್ವಾಹಾ ಶಕ್ತಿಃ। ಕ್ರೋಂ ಕೀಲಕಂ~।\\
\as{ನ್ಯಾಸಃ :}೧.ಓಂ ಕ್ರೋಂ  ೨.ಓಂ ಭ್ರೋಂ ೩.ಓಂ ಕ್ರೋಂ  ೪.ಓಂ ಝ್ರೋಂ ೫.ಓಂ ಛ್ರೋಂ  ೬.ಓಂ ಜ್ರೋಂ \\
{\bfseries ಶುದ್ಧಸ್ಫಟಿಕಸಂಕಾಶಾಂ ಪದ್ಮಪತ್ರಸಮಪ್ರಭಾಂ~।\\
ಮಧ್ಯಾಹ್ನಾದಿತ್ಯಸಂಕಾಶಾಂ ಶುಭ್ರವಸ್ತ್ರಸಮನ್ವಿತಾಂ~॥\\
ಶ್ವೇತಚಂದನಲಿಪ್ತಾಂಗೀಂ ಶುಭ್ರಮಾಲ್ಯವಿಭೂಷಿತಾಂ~।\\	
ಬಿಭ್ರತೀಂ ಚಿನ್ಮಯೀಂ ಮುದ್ರಾಮಕ್ಷಮಾಲಾಂ ಚ ಪುಸ್ತಕಂ~॥\\
ಸಹಸ್ರಪದ್ಮಕಮಲೇ ಸಮಾಸೀನಾಂ ಶುಚಿಸ್ಮಿತಾಂ~।\\
ಸರ್ವವಿದ್ಯಾಪ್ರದಾಂ ದೇವೀಂ ಭೇರುಂಡಾಂ ಪ್ರಣಮಾಮ್ಯಹಂ~॥\\}
ಮನುಃ :{\bfseries  ಕ್ರೋಂ ಭ್ರೋಂ ಕ್ರೋಂ ಝ್ರೋಂ ಛ್ರೋಂ ಜ್ರೋಂ ಸ್ವಾಹಾ ॥}

ಓಂ ಬ್ರಾಹ್ಮೀ~।  ಮಾಹೇಶ್ವರೀ~।  ಕೌಮಾರೀ~।  ವೈಷ್ಣವೀ~। ವಾರಾಹೀ~।\\  ಇಂದ್ರಾಣೀ~।  ಚಾಮುಂಡಾ~।  ಮಹಾಲಕ್ಷ್ಮೀ~॥ 

ಓಂ ಕೃತಯುಗಶಕ್ತಿ~।  ತ್ರೇತಾಯುಗಶಕ್ತಿ~।  ದ್ವಾಪರಯುಗಶಕ್ತಿ~।  ಕಲಿಯುಗಶಕ್ತಿ~॥ 

ಓಂ ವಿಜಯಾ~।  ವಿಮಲಾ~।  ಶುಭಾ~।  ವಿಶ್ವಾ~।  ವಿಭೂತಿ~।  ವಿನತಾ~।  ವಿವಿಧಾ~।  ವಿಮಲಾ~॥ 

ಓಂ ಕಮಲಾ~।  ಕಾಮಿನೀ~।  ಕಿರಾತೀ~।  ಕೀರ್ತಿ~।  ಕುರ್ದಿನೀ~।  ಕುಲಸುಂದರೀ~।  ಕಲ್ಯಾಣೀ~।  ಕಾಲಕೋಲಾ~।  

ಡಾಕಿನೀ~।  ರಾಕಿಣೀ~।  ಲಾಕಿನೀ~।  ಕಾಕಿನೀ~।  ಸಾಕಿನೀ~।  ಹಾಕಿನೀ~॥ 

ಓಂ ಇಚ್ಛಾಶಕ್ತಿ~।  ಜ್ಞಾನಶಕ್ತಿ~।  ಕ್ರಿಯಾಶಕ್ತಿ~॥ 

ಓಂ ಶರೇಭ್ಯೋ ~।  ಖಡ್ಗಾಯ ~।  ಅಂಕುಶಾಯ ~। ಪಾಶಾಯ ~।  ಗದಾಯೈ ~। ಚರ್ಮಣೇ ~।  ಧನುಷೇ ~।  ವಜ್ರಾಯ  ನಮಃ~॥

ಓಂ ಲಂ ಇಂದ್ರಶಕ್ತಿ~।  ರಂ ಅಗ್ನಿಶಕ್ತಿ~।  ಮಂ ಯಮಶಕ್ತಿ~।  ಕ್ಷಂ ನಿರ್ಋತಿಶಕ್ತಿ~।  ವಂ ವರುಣಶಕ್ತಿ~।  ಯಂ ವಾಯುಶಕ್ತಿ~।  ಕುಂ ಕುಬೇರಶಕ್ತಿ~।  ಹಂ ಈಶಾನಶಕ್ತಿ~।  ಆಂ ಬ್ರಹ್ಮಶಕ್ತಿ~।  ಹ್ರೀಂ ಅನಂತಶಕ್ತಿ~।  ನಿಯತಿಶಕ್ತಿ~।  ಕಾಲಶಕ್ತಿ~॥ 

ಓಂ ವಂ ವಜ್ರಶಕ್ತಿ~।  ಶಂ ಶಕ್ತಿಶಕ್ತಿ~।  ದಂ ದಂಡಶಕ್ತಿ~।  ಖಂ ಖಡ್ಗಶಕ್ತಿ~।  ಪಾಂ ಪಾಶಶಕ್ತಿ~।  ಅಂ ಅಂಕುಶಶಕ್ತಿ~।  ಗಂ ಗದಾಶಕ್ತಿ~।  ತ್ರಿಂ ತ್ರಿಶೂಲಶಕ್ತಿ~।  ಪಂ ಪದ್ಮಶಕ್ತಿ~।  ಚಂ ಚಕ್ರಶಕ್ತಿ~॥ 

ಭೇರುಂಡಾಯೈ ವಿದ್ಮಹೇ ವಿಷಹರಾಯೈ ಧೀಮಹಿ ।\\ತನ್ನೋ ನಿತ್ಯಾ ಪ್ರಚೋದಯಾತ್~।\\
ಇತಿ ಭೇರುಂಡಾಆವರಣಪೂಜಾ~।
\section{೫।ವಹ್ನಿವಾಸಿನೀನಿತ್ಯಾ}
\addcontentsline{toc}{section}{೫।ವಹ್ನಿವಾಸಿನೀನಿತ್ಯಾ}
ಶ್ರೀವಹ್ನಿವಾಸಿನೀನಿತ್ಯಾ ಮಂತ್ರಸ್ಯ ವಸಿಷ್ಠ ಋಷಿಃ। ಗಾಯತ್ರೀಛಂದಃ~।\\ ಶ್ರೀವಹ್ನಿವಾಸಿನೀನಿತ್ಯಾದೇವತಾ। ಹ್ರೀಂ ಬೀಜಂ। ನಮಃ ಶಕ್ತಿಃ।\\ ವಹ್ನಿವಾಸಿನ್ಯೈ ಕೀಲಕಂ~। ಹ್ರಾಂ ಇತ್ಯಾದಿನ್ಯಾಸಃ~।\\
{\bfseries ವಹ್ನಿಕೋಟಿಪ್ರತೀಕಾಶಾಂ ಸೂರ್ಯಕೋಟಿಸಮಪ್ರಭಾಂ~।\\
ಅಗ್ನಿಜ್ವಾಲಾಸಮಾಕೀರ್ಣಾಂ ಸರ್ವರೋಗಾಪಹಾರಿಣೀಂ~।\\
ಕಾಲಮೃತ್ಯುಪ್ರಶಮನೀಂ ಭಯಮೃತ್ಯುನಿವಾರಿಣೀಂ~।\\
ಪರಮಾಯುಷ್ಯದಾಂ ವಂದೇ ನಿತ್ಯಾಂ ಶ್ರೀವಹ್ನಿವಾಸಿನೀಂ ॥\\}
ಮನುಃ :{\bfseries ಓಂ ಹ್ರೀಂ ವಹ್ನಿವಾಸಿನ್ಯೈ ನಮಃ~॥}

ಜ್ವಾಲಿನೀ~। ವಿಸ್ಫುಲಿಂಗಿನೀ~। ಮಂಗಲಾ~। ಮನೋಹರಾ~। ಕನಕಾ~। ಕಿತವಾ~। ವಿಶ್ವಾ~। ವಿವಿಧಾ~॥

ಮೇಷಾ~। ವೃಷಾ~। ಮಿಥುನಾ~। ಕರ್ಕಟಾ~। ಸಿಂಹಾ~। ಕನ್ಯಾ~। ತುಲಾ~। ಕೀಟಾ~। ಚಾಪಾ~। ಮಕರಾ~। ಕುಂಭಾ~। ಮೀನಾ~॥ 

ಘಸ್ಮರಾ~। ಸರ್ವಭಕ್ಷಾ~। ವಿಶ್ವಾ~। ವಿವಿಧೋದ್ಭವಾ~। ಚಿತ್ರಾ~। ನಿಃಸಪತ್ನಾ~।\\ ಪಾವನೀ~। ರಕ್ತಾ~। ನಿರಾತಂಕಾ~। ಅಚಿಂತ್ಯವೈಭವಾ~॥

ಓಂ ಲಂ ಇಂದ್ರಶಕ್ತಿ~।  ರಂ ಅಗ್ನಿಶಕ್ತಿ~।  ಮಂ ಯಮಶಕ್ತಿ~।  ಕ್ಷಂ ನಿರ್ಋತಿಶಕ್ತಿ~।  ವಂ ವರುಣಶಕ್ತಿ~।  ಯಂ ವಾಯುಶಕ್ತಿ~।  ಕುಂ ಕುಬೇರಶಕ್ತಿ~।  ಹಂ ಈಶಾನಶಕ್ತಿ~।  ಆಂ ಬ್ರಹ್ಮಶಕ್ತಿ~।  ಹ್ರೀಂ ಅನಂತಶಕ್ತಿ~।  ನಿಯತಿಶಕ್ತಿ~।  ಕಾಲಶಕ್ತಿ~॥ 

ಓಂ ವಂ ವಜ್ರಶಕ್ತಿ~।  ಶಂ ಶಕ್ತಿಶಕ್ತಿ~।  ದಂ ದಂಡಶಕ್ತಿ~।  ಖಂ ಖಡ್ಗಶಕ್ತಿ~।  ಪಾಂ ಪಾಶಶಕ್ತಿ~।  ಅಂ ಅಂಕುಶಶಕ್ತಿ~।  ಗಂ ಗದಾಶಕ್ತಿ~।  ತ್ರಿಂ ತ್ರಿಶೂಲಶಕ್ತಿ~।  ಪಂ ಪದ್ಮಶಕ್ತಿ~।  ಚಂ ಚಕ್ರಶಕ್ತಿ~॥ 

ವಹ್ನಿವಾಸಿನ್ಯೈ ವಿದ್ಮಹೇ ಸಿದ್ಧಿಪ್ರದಾಯೈ ಧೀಮಹಿ ।\\ತನ್ನೋ ನಿತ್ಯಾ ಪ್ರಚೋದಯಾತ್~।\\
ಇತಿ ವಹ್ನಿವಾಸಿನೀನಿತ್ಯಾ ಆವರಣಪೂಜಾ
\section{೬। ಮಹಾವಜ್ರೇಶ್ವರೀ}
\addcontentsline{toc}{section}{೬। ಮಹಾವಜ್ರೇಶ್ವರೀ}
ಶ್ರೀಮಹಾವಜ್ರೇಶ್ವರೀನಿತ್ಯಾ ಮಂತ್ರಸ್ಯ ಬ್ರಹ್ಮಾ ಋಷಿಃ। ಗಾಯತ್ರೀಛಂದಃ~। ಶ್ರೀಮಹಾವಜ್ರೇಶ್ವರೀನಿತ್ಯಾದೇವತಾ। ಹ್ರೀಂ ಬೀಜಂ। ಹ್ರೀಂ ಶಕ್ತಿಃ।\\
ಐಂ ಕೀಲಕಂ~।\\
\as{ನ್ಯಾಸಃ :}೧.ಓಂ ಹ್ರೀಂ ಕ್ಲಿನ್ನೇ ಹ್ರೀಂ  ೨.ಓಂ ಹ್ರೀಂ ಐಂ ಹ್ರೀಂ  ೩.ಓಂ ಹ್ರೀಂ ಕ್ರೋಂ ಹ್ರೀಂ ೪.ಓಂ ಹ್ರೀಂ ನಿತ್ಯ ಹ್ರೀಂ  ೫.ಓಂ ಹ್ರೀಂ ಮದ ಹ್ರೀಂ ೬.ಓಂ ಹ್ರೀಂ ದ್ರವೇ ಹ್ರೀಂ \\
{\bfseries ತಪ್ತಕಾಂಚನಸಂಕಾಶಾಂ ಕನಕಾಭರಣಾನ್ವಿತಾಂ~।\\
ಹೇಮತಾಟಂಕಸಂಯುಕ್ತಾಂ ಕಸ್ತೂರೀತಿಲಕಾನ್ವಿತಾಂ ॥\\
ಹೇಮಚಿಂತಾಕಸಂಯುಕ್ತಾಂ ಪೂರ್ಣಚಂದ್ರಮುಖಾಂಬುಜಾಂ~।\\
ಪೀತಾಂಬರಸಮೋಪೇತಾಂ ಪುಣ್ಯಮಾಲ್ಯವಿಭೂಷಿತಾಂ ॥\\
ಮುಕ್ತಾಹಾರಸಮೋಪೇತಾಂ ಮುಕುಟೇನ ವಿರಾಜಿತಾಂ~।\\
ಮಹಾವಜ್ರೇಶ್ವರೀಂ ವಂದೇ ಸರ್ವೈಶ್ವರ್ಯಫಲಪ್ರದಾಂ ॥\\}
ಮನುಃ :{\bfseries  ಓಂ ಹ್ರೀಂ ಕ್ಲಿನ್ನೇ ಐಂ ಕ್ರೋಂ ನಿತ್ಯಮದದ್ರವೇ ಹ್ರೀಂ~॥}

ಶೋಣಸಮುದ್ರಾಯ ನಮಃ~। ಕನಕಪೋತಾಯ ನಮಃ~। ರತ್ನಸಿಂಹಾಸನಾಯ ನಮಃ~॥

ಹೃಲ್ಲೇಖಾ~। ಕ್ಲೇದಿನೀ~। ಕ್ಲಿನ್ನಾ~। ಕ್ಷೋಭಿಣೀ~। ಮದನಾ~। ಮದನಾತುರಾ~। \\ನಿರಂಜನಾ।ರಾಗವತೀ।ಮದನಾವತೀ।ಮೇಖಲಾ। ದ್ರಾವಿಣೀ। ವೇಗವತೀ॥ 

ಕಮಲಾ। ಕಲ್ಪಾ। ಕಲಾ। ಕಲಿತಾ। ಕೌತುಕಾ। ಕಿರಾತಾ। ಕಾಲಾ। ಕದನಾ। ಕೌಶಿಕೀ~। ಕಂಬುವಾಹಿನೀ~। ಕಾತರಾ~। ಕಪಟಾ~। ಕೀರ್ತಿ~। ಕುಮಾರೀ~। ಕುಂಕುಮಾ~। ಭಂಜಿನೀ~। ವೇಗಿನೀ~। ಭೋಗಾ~। ಚಪಲಾ~। ಪೇಶಲಾ~। ಸತೀ~। ರತಿ~। ಶ್ರದ್ಧಾ~। ಭೋಗಲೋಲಾ~। ಮದಾ~। ಉನ್ಮತ್ತಾ~। ಮನಸ್ವಿನೀ~॥ 

ಓಂ ಲಂ ಇಂದ್ರಶಕ್ತಿ~।  ರಂ ಅಗ್ನಿಶಕ್ತಿ~।  ಮಂ ಯಮಶಕ್ತಿ~।  ಕ್ಷಂ ನಿರ್ಋತಿಶಕ್ತಿ~।  ವಂ ವರುಣಶಕ್ತಿ~।  ಯಂ ವಾಯುಶಕ್ತಿ~।  ಕುಂ ಕುಬೇರಶಕ್ತಿ~।  ಹಂ ಈಶಾನಶಕ್ತಿ~।  ಆಂ ಬ್ರಹ್ಮಶಕ್ತಿ~।  ಹ್ರೀಂ ಅನಂತಶಕ್ತಿ~।  ನಿಯತಿಶಕ್ತಿ~।  ಕಾಲಶಕ್ತಿ~॥

ಓಂ ವಂ ವಜ್ರಶಕ್ತಿ~।  ಶಂ ಶಕ್ತಿಶಕ್ತಿ~।  ದಂ ದಂಡಶಕ್ತಿ~।  ಖಂ ಖಡ್ಗಶಕ್ತಿ~।  ಪಾಂ ಪಾಶಶಕ್ತಿ~।  ಅಂ ಅಂಕುಶಶಕ್ತಿ~।  ಗಂ ಗದಾಶಕ್ತಿ~।  ತ್ರಿಂ ತ್ರಿಶೂಲಶಕ್ತಿ~।  ಪಂ ಪದ್ಮಶಕ್ತಿ~।  ಚಂ ಚಕ್ರಶಕ್ತಿ~॥

ಮಹಾವಜ್ರೇಶ್ವರ್ಯೈ ವಿದ್ಮಹೇ ವಜ್ರನಿತ್ಯಾಯೈ ಧೀಮಹಿ ।\\ತನ್ನೋ ನಿತ್ಯಾ ಪ್ರಚೋದಯಾತ್~॥\\
ಇತಿ ಮಹಾವಜ್ರೇಶ್ವರೀನಿತ್ಯಾ
\section{೭।ಶಿವಾದೂತೀನಿತ್ಯಾ}
\addcontentsline{toc}{section}{೭।ಶಿವಾದೂತೀನಿತ್ಯಾ}
ಶ್ರೀಶಿವಾದೂತೀನಿತ್ಯಾ ಮಂತ್ರಸ್ಯ ರುದ್ರಋಷಿಃ~। ಗಾಯತ್ರೀ ಛಂದಃ~।\\ ಶ್ರೀಶಿವಾದೂತೀನಿತ್ಯಾ ದೇವತಾ~। ಹ್ರೀಂ ಬೀಜಂ~। ನಮಃ ಶಕ್ತಿಃ~।\\ ಶಿವಾದೂತ್ಯೈ ಕೀಲಕಂ~। ಹ್ರಾಂ ಇತ್ಯಾದಿನಾ ನ್ಯಾಸಃ~।
{\bfseries ಬಾಲಸೂರ್ಯಪ್ರತೀಕಾಶಾಂ ಬಂಧೂಕಪ್ರಸವಾರುಣಾಂ~।\\
ವಿಧಿವಿಷ್ಣುಶಿವಸ್ತುತ್ಯಾಂ ದೇವಗಂಧರ್ವಸೇವಿತಾಂ ॥\\
ರಕ್ತಾರವಿಂದಸಂಕಾಶಾಂ ಸರ್ವಾಭರಣಭೂಷಿತಾಂ~।\\
ಶಿವದೂತೀಂ ನಮಸ್ಯಾಮಿ ರತ್ನಸಿಂಹಾಸನಸ್ಥಿತಾಂ ॥\\}
ಮನುಃ :{\bfseries  ಓಂ ಹ್ರೀಂ ಶಿವಾದೂತ್ಯೈ ನಮಃ~॥}

ಓಂ ವಿಹ್ವಲಾ~।  ಆಕರ್ಷಿಣೀ~।  ಲೋಲಾ~।  ನಿತ್ಯಾ~।  ಮದನಾ~।  ಮಾಲಿನೀ~। \\ವಿನೋದಾ~।  ಕೌತುಕಾ~।  ಪುಣ್ಯಾ~।  ಪುರಾಣಾ~॥ 

ಓಂ ವಾಗೀಶಾ~।  ವರದಾ~।  ವಿಶ್ವಾ~।  ವಿಭವಾ~।  ವಿಘ್ನಕಾರಿಣೀ~।  ವೀರಾ~। \\ವಿಘ್ನಹರಾ~।  ವಿದ್ಯಾ~॥ 

ಓಂ ಸುಮುಖೀ~।  ಸುಂದರೀ~।  ಸಾರಾ~।  ಸಮರಾ~।  ಸರಸ್ವತೀ~।  ಸಮಯಾ~।  ಸರ್ವಗಾ~।  ಸಿದ್ಧಾ~॥ 

ಓಂ ಡಾಕಿನೀ~।  ರಾಕಿಣೀ~।  ಲಾಕಿನೀ~।  ಕಾಕಿನೀ~।  ಸಾಕಿನೀ~।  ಹಾಕಿನೀ~॥ 

ಓಂ ಶಿವಾ~। ವಾಣೀ।ದೂರಸಿದ್ಧಾ।ತ್ಯೈವಿಗ್ರಹವತೀ~। ನಾದಾ।  ಮನೋನ್ಮನೀ॥ 

ಓಂ ಇಚ್ಛಾಶಕ್ತಿ~।  ಜ್ಞಾನಶಕ್ತಿ~।  ಕ್ರಿಯಾಶಕ್ತಿ~॥ 

ಓಂ ಕಮಲಶಕ್ತಿ~।  ಕುಠಾರಶಕ್ತಿ~।  ಖಡ್ಗ ಶಕ್ತಿ~।  ಅಂಕುಶ ಶಕ್ತಿ~।  ರತ್ನಚಷಕ ಶಕ್ತಿ~।\\ಗದಾಶಕ್ತಿ~।  ಖೇಟಶಕ್ತಿ~।  ಪಾಶಶಕ್ತಿ~॥ 

ಓಂ ಲಂ ಇಂದ್ರಶಕ್ತಿ~।  ರಂ ಅಗ್ನಿಶಕ್ತಿ~।  ಮಂ ಯಮಶಕ್ತಿ~।  ಕ್ಷಂ ನಿರ್ಋತಿಶಕ್ತಿ~।  ವಂ ವರುಣಶಕ್ತಿ~।  ಯಂ ವಾಯುಶಕ್ತಿ~।  ಕುಂ ಕುಬೇರಶಕ್ತಿ~।  ಹಂ ಈಶಾನಶಕ್ತಿ~।  ಆಂ ಬ್ರಹ್ಮಶಕ್ತಿ~।  ಹ್ರೀಂ ಅನಂತಶಕ್ತಿ~।  ನಿಯತಿಶಕ್ತಿ~।  ಕಾಲಶಕ್ತಿ~॥

ಓಂ ವಂ ವಜ್ರಶಕ್ತಿ~।  ಶಂ ಶಕ್ತಿಶಕ್ತಿ~।  ದಂ ದಂಡಶಕ್ತಿ~।  ಖಂ ಖಡ್ಗಶಕ್ತಿ~।  ಪಾಂ ಪಾಶಶಕ್ತಿ~।  ಅಂ ಅಂಕುಶಶಕ್ತಿ~।  ಗಂ ಗದಾಶಕ್ತಿ~।  ತ್ರಿಂ ತ್ರಿಶೂಲಶಕ್ತಿ~।  ಪಂ ಪದ್ಮಶಕ್ತಿ~।  ಚಂ ಚಕ್ರಶಕ್ತಿ~॥ 

ಶಿವಾದೂತ್ಯೈ ವಿದ್ಮಹೇ ಶಿವಂಕರ್ಯೈ ಧೀಮಹಿ~। ತನ್ನೋ ನಿತ್ಯಾ ಪ್ರಚೋದಯಾತ್~॥\\
ಇತಿ ಶಿವಾದೂತೀನಿತ್ಯಾ ಆವರಣಪೂಜಾ
\section{೮। ತ್ವರಿತಾ ನಿತ್ಯಾ}
\addcontentsline{toc}{section}{೮। ತ್ವರಿತಾ ನಿತ್ಯಾ}
ಅಸ್ಯ ಶ್ರೀತ್ವರಿತಾನಿತ್ಯಾ ಮಹಾಮಂತ್ರಸ್ಯ ಈಶ್ವರ ಋಷಿಃ~। ವಿರಾಟ್ ಛಂದಃ~। ತ್ವರಿತಾನಿತ್ಯಾ ದೇವತಾ। ಹೂಂ ಬೀಜಂ~। ಸ್ತ್ರೀಂ ಶಕ್ತಿಃ~। ಹ್ರೀಂ ಕೀಲಕಂ~।\\
ಹ್ರಾಂ ಇತ್ಯಾದಿನ್ಯಾಸಃ ।\\
{\bfseries ರಕ್ತಾರವಿಂದಸಂಕಾಶಾಮುದ್ಯತ್ಸೂರ್ಯಸಮಪ್ರಭಾಂ~।\\
ದಧತೀಮಂಕುಶಂ ಪಾಶಂ ಬಾಣಾನ್ ಚಾಪಂ ಮನೋಹರಂ ॥\\
ಚತುರ್ಭುಜಾಂ ಮಹಾದೇವೀಮಪ್ಸರೋಗಣಸಂಕುಲಾಂ~।\\
ನಮಾಮಿ ತ್ವರಿತಾಂ ನಿತ್ಯಾಂ ಭಕ್ತಾನಾಮಭಯಪ್ರದಾಂ ॥\\}
ಮನುಃ :{\bfseries  ಓಂ ಹ್ರೀಂ ಹೂಂ ಖೇ ಚ ಛೇ ಕ್ಷಃ ಸ್ತ್ರೀಂ ಹೂಂ ಕ್ಷೇ ಹ್ರೀಂ ಫಟ್~॥}

ಓಂ ಫಟ್ಕಾರೀ~। ಕಿಂಕರ~। ಜಯಾ~। ವಿಜಯಾ~। ಹುಂಕಾರೀ~। ಖೇಚರೀ~। ಚಂಡಾ~। ಛೇದಿನೀ~। ಕ್ಷೇದಿನೀ~। ಕ್ಷೇಪಿಣೀ~। ಸ್ತ್ರೀಕಾರೀ~। ಹುಂಕಾರೀ~। ಕ್ಷೇಮಕರೀ~॥

ಓಂ ಲಂ ಇಂದ್ರಶಕ್ತಿ~।  ರಂ ಅಗ್ನಿಶಕ್ತಿ~।  ಮಂ ಯಮಶಕ್ತಿ~।  ಕ್ಷಂ ನಿರ್ಋತಿಶಕ್ತಿ~।  ವಂ ವರುಣಶಕ್ತಿ~।  ಯಂ ವಾಯುಶಕ್ತಿ~।  ಕುಂ ಕುಬೇರಶಕ್ತಿ~।  ಹಂ ಈಶಾನಶಕ್ತಿ~।  ಆಂ ಬ್ರಹ್ಮಶಕ್ತಿ~।  ಹ್ರೀಂ ಅನಂತಶಕ್ತಿ~।  ನಿಯತಿಶಕ್ತಿ~।  ಕಾಲಶಕ್ತಿ~॥ 

ಓಂ ವಂ ವಜ್ರಶಕ್ತಿ~।  ಶಂ ಶಕ್ತಿಶಕ್ತಿ~।  ದಂ ದಂಡಶಕ್ತಿ~।  ಖಂ ಖಡ್ಗಶಕ್ತಿ~।  ಪಾಂ ಪಾಶಶಕ್ತಿ~।  ಅಂ ಅಂಕುಶಶಕ್ತಿ~।  ಗಂ ಗದಾಶಕ್ತಿ~।  ತ್ರಿಂ ತ್ರಿಶೂಲಶಕ್ತಿ~।  ಪಂ ಪದ್ಮಶಕ್ತಿ~।  ಚಂ ಚಕ್ರಶಕ್ತಿ~॥ 

ಓಂ ತ್ವರಿತಾಯೈ ವಿದ್ಮಹೇ ಮಹಾನಿತ್ಯಾಯೈ ಧೀಮಹಿ~। ತನ್ನೋ ನಿತ್ಯಾ ಪ್ರಚೋದಯಾತ್~॥\\
ಇತಿ ತ್ವರಿತಾ ಆವರಣಪೂಜಾ
\section{೯। ಕುಲಸುಂದರೀನಿತ್ಯಾ}
\addcontentsline{toc}{section}{೯। ಕುಲಸುಂದರೀನಿತ್ಯಾ}
ಅಸ್ಯ ಶ್ರೀ ಕುಲಸುಂದರೀ ನಿತ್ಯಾ ಮಹಾಮಂತ್ರಸ್ಯ ದಕ್ಷಿಣಾಮೂರ್ತಿಃ ಋಷಿಃ~। ಪಂಕ್ತಿಶ್ಛಂದಃ~। ಶ್ರೀ ಕುಲಸುಂದರೀ ನಿತ್ಯಾ ದೇವತಾ~। ಐಂ ಬೀಜಂ~। ಸೌಃ ಶಕ್ತಿಃ~। ಕ್ಲೀಂ ಕೀಲಕಂ~। ಆಂ , ಈಂ , ಇತ್ಯಾದಿನ್ಯಾಸಃ~।\\
{\bfseries ಅರುಣಕಿರಣಜಾಲೈ ರಂಜಿತಾಶಾವಕಾಶಾ\\
ವಿಧೃತಜಪವಟೀಕಾ ಪುಸ್ತಕಾಭೀತಿಹಸ್ತಾ ॥\\
ಇತರಕರವರಾಢ್ಯಾ ಫುಲ್ಲಕಹ್ಲಾರಸಂಸ್ಥಾ\\
ನಿವಸತು ಹೃದಿ ಬಾಲಾ ನಿತ್ಯಕಲ್ಯಾಣಶೀಲಾ ॥\\}
ಮನುಃ :{\bfseries  ಐಂ ಕ್ಲೀಂ ಸೌಃ~॥}

ಓಂ ವಾಮಾಯೈ~। ಜ್ಯೇಷ್ಠಾಯೈ~। ರೌದ್ರ್ಯೈ~। ಅಂಬಿಕಾಯೈ~। ಇಚ್ಛಾಯೈ~। \\ಜ್ಞಾನಾಯೈ~। ಕ್ರಿಯಾಯೈ~। ಕುಲಿಕಾಯೈ~। ಚಿತ್ರಾಯೈ~। ವಿಷಘ್ನ್ಯೈ~। ದೂತ್ಯೈ~। ಆನಂದಾಯೈ ನಮಃ~॥

ಓಂ ಭಾಷಾಪಾದುಕಾಂ ಪೂಜಯಾಮಿ~।  ಸರಸ್ವತೀ ಪಾದುಕಾಂ ಪೂಜಯಾಮಿ॥ 

ಓಂ ವಾಣೀ~।  ಸಂಸ್ಕೃತಾ~।  ಪ್ರಾಕೃತಾ~।  ಪರಾ~।  ಬಹುರೂಪಾ~।  ಚಿತ್ರರೂಪಾ~।  ರಮ್ಯಾ~।  ಆನಂದಾ~।  ಕೌತುಕಾ~॥ 

ಓಂ ಬ್ರಾಹ್ಮೀ~।  ಮಾಹೇಶ್ವರೀ~।  ಕೌಮಾರೀ~।  ವೈಷ್ಣವೀ~।  ವಾರಾಹೀ~। \\ ಮಾಹೇಂದ್ರೀ~।  ಚಾಮುಂಡಾ~।  ಮಹಾಲಕ್ಷ್ಮೀ~॥ 

ಓಂ ಲಂ ಇಂದ್ರಶಕ್ತಿ~।  ರಂ ಅಗ್ನಿಶಕ್ತಿ~।  ಮಂ ಯಮಶಕ್ತಿ~।  ಕ್ಷಂ ನಿರ್ಋತಿಶಕ್ತಿ~।  ವಂ ವರುಣಶಕ್ತಿ~।  ಯಂ ವಾಯುಶಕ್ತಿ~।  ಕುಂ ಕುಬೇರಶಕ್ತಿ~।  ಹಂ ಈಶಾನಶಕ್ತಿ~।  ಆಂ ಬ್ರಹ್ಮಶಕ್ತಿ~।  ಹ್ರೀಂ ಅನಂತಶಕ್ತಿ~।  ನಿಯತಿಶಕ್ತಿ~।  ಕಾಲಶಕ್ತಿ~॥ 

ಓಂ ವಂ ವಜ್ರಶಕ್ತಿ~।  ಶಂ ಶಕ್ತಿಶಕ್ತಿ~।  ದಂ ದಂಡಶಕ್ತಿ~।  ಖಂ ಖಡ್ಗಶಕ್ತಿ~।  ಪಾಂ ಪಾಶಶಕ್ತಿ~।  ಅಂ ಅಂಕುಶಶಕ್ತಿ~।  ಗಂ ಗದಾಶಕ್ತಿ~।  ತ್ರಿಂ ತ್ರಿಶೂಲಶಕ್ತಿ~।  ಪಂ ಪದ್ಮಶಕ್ತಿ~।  ಚಂ ಚಕ್ರಶಕ್ತಿ~॥ 

ಕುಲಸುಂದರ್ಯೈ ವಿದ್ಮಹೇ ಕಾಮೇಶ್ವರ್ಯೈ ಧೀಮಹಿ ತನ್ನೋ ನಿತ್ಯಾ ಪ್ರಚೋದಯಾತ್~॥\\
ಇತಿ ಕುಲಸುಂದರೀ ಆವರಣಪೂಜಾ
\section{೧೦।ನಿತ್ಯಾನಿತ್ಯಾ}
\addcontentsline{toc}{section}{೧೦।ನಿತ್ಯಾನಿತ್ಯಾ}
ಅಸ್ಯ ಶ್ರೀನಿತ್ಯಾನಿತ್ಯಾಮಹಾಮಂತ್ರಸ್ಯ ದಕ್ಷಿಣಾಮೂರ್ತಿಃ ಋಷಿಃ।\\ ಪಂಕ್ತಿಃ ಛಂದಃ~। ಶ್ರೀನಿತ್ಯಾನಿತ್ಯಾದೇವತಾ। ಐಂ ಬೀಜಂ। ಔಃ ಶಕ್ತಿಃ~।\\ ಈಂ ಕೀಲಕಂ~। ಹ್ರೀಂ ಹ್ಸಾಂ, ಹ್ರೀಂ ಹ್ಸೀಂ ಇತ್ಯಾದಿನಾ ನ್ಯಾಸಃ~।

{\bfseries ಉದ್ಯತ್ಪ್ರದ್ಯೋತನನಿಭಾಂ ಜಪಾಕುಸುಮಸನ್ನಿಭಾಂ~।\\
ಹರಿಚಂದನಲಿಪ್ತಾಂಗೀಂ ರಕ್ತಮಾಲ್ಯ ವಿಭೂಷಿತಾಂ ॥\\
ರತ್ನಾಭರಣಭೂಷಾಂಗೀಂ ರಕ್ತವಸ್ತ್ರಸುಶೋಭಿತಾಂ~।\\
ಜಗದಂಬಾಂ ನಮಸ್ಯಾಮಿ ನಿತ್ಯಾಂ ಶ್ರೀಪರಮೇಶ್ವರೀಂ ॥\\}
ಮನುಃ :{\bfseries  ಓಂ ಹಸಕಲರಡೈಂ ಹಸಕಲರಡೀಂ ಹಸಕಲರಡೌಃ ॥}

ಓಂ ಡಾಕಿನೀ~।  ರಾಕಿಣೀ~।  ಲಾಕಿನೀ~।  ಕಾಕಿನೀ~।  ಸಾಕಿನೀ~।  ಹಾಕಿನೀ~॥ 

ಓಂ ಅ ಶಕ್ತಿ~।  ಆ ಶಕ್ತಿ~।  ಇ ಶಕ್ತಿ~।  ಈ ಶಕ್ತಿ~।  ಉ ಶಕ್ತಿ~।  ಊ ಶಕ್ತಿ~।  ಋ ಶಕ್ತಿ ಶ್ರೀ*********************ಳ ಶಕ್ತಿ~।  ಕ್ಷ ಶಕ್ತಿ~॥ 

ಓಂ ಲಂ ಇಂದ್ರಶಕ್ತಿ~।  ರಂ ಅಗ್ನಿಶಕ್ತಿ~।  ಮಂ ಯಮಶಕ್ತಿ~।  ಕ್ಷಂ ನಿರ್ಋತಿಶಕ್ತಿ~।  ವಂ ವರುಣಶಕ್ತಿ~।  ಯಂ ವಾಯುಶಕ್ತಿ~।  ಕುಂ ಕುಬೇರಶಕ್ತಿ~।  ಹಂ ಈಶಾನಶಕ್ತಿ~।  ಆಂ ಬ್ರಹ್ಮಶಕ್ತಿ~।  ಹ್ರೀಂ ಅನಂತಶಕ್ತಿ~।  ನಿಯತಿಶಕ್ತಿ~।  ಕಾಲಶಕ್ತಿ~॥ 

ಓಂ ವಂ ವಜ್ರಶಕ್ತಿ~।  ಶಂ ಶಕ್ತಿಶಕ್ತಿ~।  ದಂ ದಂಡಶಕ್ತಿ~।  ಖಂ ಖಡ್ಗಶಕ್ತಿ~।  ಪಾಂ ಪಾಶಶಕ್ತಿ~।  ಅಂ ಅಂಕುಶಶಕ್ತಿ~।  ಗಂ ಗದಾಶಕ್ತಿ~।  ತ್ರಿಂ ತ್ರಿಶೂಲಶಕ್ತಿ~।  ಪಂ ಪದ್ಮಶಕ್ತಿ~।  ಚಂ ಚಕ್ರಶಕ್ತಿ~॥ 

ಓಂ ಅಭಯಶಕ್ತಿ~।  ಖಡ್ಗಶಕ್ತಿ। ಪುಸ್ತಕಶಕ್ತಿ। ಪಾಶಶಕ್ತಿ।ಇಕ್ಷುಚಾಪಶಕ್ತಿ। ತ್ರಿಶೂಲಶಕ್ತಿ~।  ಕಪಾಲಶಕ್ತಿ~।  ಇಷುಶಕ್ತಿ~।  ಅಂಕುಶಶಕ್ತಿ~।  ಅಕ್ಷಗುಣಶಕ್ತಿ~।  ಖೇಟಕಶಕ್ತಿ~।  ವರಶಕ್ತಿ~॥

ನಿತ್ಯಭೈರವ್ಯೈ ವಿದ್ಮಹೇ ನಿತ್ಯಾನಿತ್ಯಾಯೈ ಧೀಮಹಿ ।\\ತನ್ನೋ ನಿತ್ಯಾ ಪ್ರಚೋದಯಾತ್ ॥\\
ಇತಿ ನಿತ್ಯಾನಿತ್ಯಾ ಆವರಣಪೂಜಾ~।
\section{೧೧।ಶ್ರೀನೀಲಪತಾಕಾನಿತ್ಯಾ}
\addcontentsline{toc}{section}{೧೧।ಶ್ರೀನೀಲಪತಾಕಾನಿತ್ಯಾ}
ಅಸ್ಯ ಶ್ರೀನೀಲಪತಾಕಾನಿತ್ಯಾ ಮಹಾಮಂತ್ರಸ್ಯ ಸಮ್ಮೋಹನ ಋಷಿಃ~। \\ಗಾಯತ್ರೀ ಛಂದಃ~। ಶ್ರೀನೀಲಪತಾಕಾನಿತ್ಯಾ ದೇವತಾ~। ಹ್ರೀಂ ಬೀಜಂ~।\\ ಹ್ರೀಂ ಶಕ್ತಿಃ~। ಕ್ಲೀಂ ಕೀಲಕಂ~।\\
\as{ನ್ಯಾಸಃ :}೧.ಓಂ ಓಂ ಹ್ರೀಂ ಫ್ರೇಂ ೨.ಓಂ ಸ್ರೂಂ ಓಂ ಆಂ ಕ್ಲೀಂ ೩.ಓಂ ಐಂ ಬ್ಲೂಂ ನಿತ್ಯಮದ ೪.ಓಂ ದ್ರ ೫.ಓಂ ವೇ ೬.ಓಂ ಹುಂ \\
{\bfseries ಪಂಚವಕ್ತ್ರಾಂ ತ್ರಿಣಯನಾಮರುಣಾಂಶುಕಧಾರಿಣೀಂ~।\\
ದಶಹಸ್ತಾಂ ಲಸನ್ಮುಕ್ತಾಪ್ರಾಯಾಭರಣಮಂಡಿತಾಂ ॥\\
ನೀಲಮೇಘಸಮಪ್ರಖ್ಯಾಂ ಧೂಮ್ರಾರ್ಚಿಸ್ಸದೃಶಪ್ರಭಾಂ~।\\
ನೀಲಪುಷ್ಪಸ್ರಜೋಪೇತಾಂ ಧ್ಯಾಯೇನ್ನೀಲಪತಾಕಿನೀಂ ॥\\}
ಮನುಃ :{\bfseries ಓಂ ಹ್ರೀಂ ಫ್ರೇಂ ಸ್ರೂಂ ಓಂ ಆಂ ಕ್ಲೀಂ ಐಂ ಬ್ಲೂಂ ನಿತ್ಯಮದದ್ರವೇ ಹುಂ ಫ್ರೇಂ ಹ್ರೀಂ~॥}

ಓಂ ಅಭಯಾಯ ನಮಃ~।  ಬಾಣಾಯ ನಮಃ~।  ಖಡ್ಗಾಯ ನಮಃ~।  ಶಕ್ತಯೇ ನಮಃ~।  ಅಂಕುಶಾಯ ನಮಃ~।  ಪಾಶಾಯ ನಮಃ~।  ಪತಾಕಾಯ ನಮಃ~। \\ ಚರ್ಮಣೇ ನಮಃ~।  ಶಾರ್ಙ್ಗಚಾಪಾಯ ನಮಃ~।  ವರಾಯ ನಮಃ~॥

ಓಂ ಇಚ್ಛಾಶಕ್ತಿ~।  ಜ್ಞಾನಶಕ್ತಿ~।  ಕ್ರಿಯಾಶಕ್ತಿ~॥ 

ಓಂ ಡಾಕಿನೀ~।  ರಾಕಿಣೀ~।  ಲಾಕಿನೀ~।  ಕಾಕಿನೀ~।  ಸಾಕಿನೀ~।  ಹಾಕಿನೀ~॥ 

ಓಂ ಬ್ರಾಹ್ಮೀ~।  ಮಾಹೇಶ್ವರೀ~।  ಕೌಮಾರೀ~।  ವೈಷ್ಣವೀ~।  ವಾರಾಹೀ~। \\ ಮಾಹೇಂದ್ರೀ~।  ಚಾಮುಂಡಾ~॥ 

ಓಂ ಸುಮುಖೀ~।  ಸುಂದರೀ~।  ಸಾರಾ~।  ಸುಮನಾ~।  ಸರಸ್ವತೀ~।  ಸಮಯಾ~।  ಸರ್ವಗಾ~।  ಸಿದ್ಧಾ~।  ವಿಹ್ವಲಾ~।  ಲೋಲಾ~।  ಮದನಾ~।  ವಿನೋದಾ~।  ಪುಣ್ಯಾ~।  ಆಕರ್ಷಿಣೀ~।  ನಿತ್ಯಾ~।  ಮಾಲಿನೀ~।  ಕೌತುಕಾ~।  ಪುರಾಣಾ~॥ 

ಓಂ ಲಂ ಇಂದ್ರಶಕ್ತಿ~।  ರಂ ಅಗ್ನಿಶಕ್ತಿ~।  ಮಂ ಯಮಶಕ್ತಿ~।  ಕ್ಷಂ ನಿರ್ಋತಿಶಕ್ತಿ~।  ವಂ ವರುಣಶಕ್ತಿ~।  ಯಂ ವಾಯುಶಕ್ತಿ~।  ಕುಂ ಕುಬೇರಶಕ್ತಿ~।  ಹಂ ಈಶಾನಶಕ್ತಿ~।  ಆಂ ಬ್ರಹ್ಮಶಕ್ತಿ~।  ಹ್ರೀಂ ಅನಂತಶಕ್ತಿ~।  ನಿಯತಿಶಕ್ತಿ~।  ಕಾಲಶಕ್ತಿ~॥ 

ಓಂ ವಂ ವಜ್ರಶಕ್ತಿ~।  ಶಂ ಶಕ್ತಿಶಕ್ತಿ~।  ದಂ ದಂಡಶಕ್ತಿ~।  ಖಂ ಖಡ್ಗಶಕ್ತಿ~।  ಪಾಂ ಪಾಶಶಕ್ತಿ~।  ಅಂ ಅಂಕುಶಶಕ್ತಿ~।  ಗಂ ಗದಾಶಕ್ತಿ~।  ತ್ರಿಂ ತ್ರಿಶೂಲಶಕ್ತಿ~। ಪಂ ಪದ್ಮಶಕ್ತಿ~।  ಚಂ ಚಕ್ರಶಕ್ತಿ~॥

ನೀಲಪತಾಕಾಯೈ ವಿದ್ಮಹೇ ಮಹಾನಿತ್ಯಾಯೈ ಧೀಮಹಿ ।\\ತನ್ನೋ ನಿತ್ಯಾ ಪ್ರಚೋದಯಾತ್~॥\\
ಇತಿ ನೀಲಪತಾಕಾನಿತ್ಯಾ ಆವರಣ
\section{೧೨। ವಿಜಯಾ ನಿತ್ಯಾ}
\addcontentsline{toc}{section}{೧೨। ವಿಜಯಾ ನಿತ್ಯಾ}
ಅಸ್ಯ ಶ್ರೀ ವಿಜಯಾನಿತ್ಯಾಮಹಾಮಂತ್ರಸ್ಯ ಅಹಿರ್ಋಷಿಃ~। ಗಾಯತ್ರೀಛಂದಃ~। ಶ್ರೀವಿಜಯಾನಿತ್ಯಾ ದೇವತಾ~।\\
\as{ನ್ಯಾಸಃ :}೧.ಓಂ ಭಾಂ ೨.ಓಂ ಮೀಂ ೩.ಓಂ ರೂಂ ೪.ಓಂ ಯೈಂ ೫.ಓಂ ಉಂ ೬.ಓಂ ಔಂ \\
{\bfseries ಉದ್ಯದರ್ಕಸಹಸ್ರಾಭಾಂ ದಾಡಿಮೀಪುಷ್ಪಸನ್ನಿಭಾಂ~।\\
ರಕ್ತಕಂಕಣಕೇಯೂರಕಿರೀಟಾಂಗದಸಂಯುತಾಂ ॥\\
ದೇವಗಂಧರ್ವಯೋಗೀಶಮುನಿಸಿದ್ಧನಿಷೇವಿತಾಂ~।\\
ನಮಾಮಿ ವಿಜಯಾಂ ನಿತ್ಯಾಂ ಸಿಂಹೋಪರಿ ಕೃತಾಸನಾಂ ॥\\}
ಮನುಃ :{\bfseries  ಭ ಮ ರ ಯ ಉ ಔಂ ॥}

ಓಂ ಲಂ ಇಂದ್ರಶಕ್ತಿ~।  ರಂ ಅಗ್ನಿಶಕ್ತಿ~।  ಮಂ ಯಮಶಕ್ತಿ~।  ಕ್ಷಂ ನಿರ್ಋತಿಶಕ್ತಿ~।  ವಂ ವರುಣಶಕ್ತಿ~।  ಯಂ ವಾಯುಶಕ್ತಿ~।  ಕುಂ ಕುಬೇರಶಕ್ತಿ~।  ಹಂ ಈಶಾನಶಕ್ತಿ~।  ಆಂ ಬ್ರಹ್ಮಶಕ್ತಿ~।  ಹ್ರೀಂ ಅನಂತಶಕ್ತಿ~।  ನಿಯತಿಶಕ್ತಿ~।  ಕಾಲಶಕ್ತಿ~॥ 

ಓಂ ವಂ ವಜ್ರಶಕ್ತಿ~।  ಶಂ ಶಕ್ತಿಶಕ್ತಿ~।  ದಂ ದಂಡಶಕ್ತಿ~।  ಖಂ ಖಡ್ಗಶಕ್ತಿ~।  ಪಾಂ ಪಾಶಶಕ್ತಿ~।  ಅಂ ಅಂಕುಶಶಕ್ತಿ~।  ಗಂ ಗದಾಶಕ್ತಿ~।  ತ್ರಿಂ ತ್ರಿಶೂಲಶಕ್ತಿ~।  ಪಂ ಪದ್ಮಶಕ್ತಿ~।  ಚಂ ಚಕ್ರಶಕ್ತಿ~॥ 

ಓಂ ಜಯಾ~।  ವಿಜಯಾ~।  ದುರ್ಗಾ~।  ಭದ್ರಾ~।  ಭದ್ರಕರೀ~।  ಕ್ಷೇಮಾ~।  \\ಕ್ಷೇಮಕರೀ~।  ನಿತ್ಯಾ~॥ 

ಓಂ ವಿದಾರಿಕಾ~।  ವಿಶ್ವಮಯೀ~।  ವಿಶ್ವಾ।  ವಿಶ್ವವಿಭಂಜಿಕಾ।  ವೀರಾ। ವಿಕ್ಷೋಭಿಣೀ।  ವಿದ್ಯಾ~।  ವಿನೋದಾ~।  ಅಂಚಿತವಿಗ್ರಹಾ~।  ವೀತಶೋಕಾ~।  ವಿಷಗ್ರೀವಾ~।  ವಿಪುಲಾ~।  ವಿಜಯಪ್ರದಾ~।  ವಿಭವಾ~।  ವಿವಿಧಾ~।  ವಿಪ್ರಾ~॥ 

ಓಂ ಮನೋಹರಾ~।  ಮಂಗಲಾ~।  ಮದೋತ್ಸಿಕ್ತಾ~।  ಮನಸ್ವಿನೀ~।  ಮಾನಿನೀ~।  ಮಧುರಾ~।  ಮಾಯಾ~।  ಮೋಹಿನೀ~॥ 

ಓಂ ಮಾತುಲುಂಗಾಯ~।ಓಂ ಸಾಯಕೇಭ್ಯಃ~।ಓಂ ಖಡ್ಗಾಯ~।ಓಂ ಅಂಕುಶಾಯ~।\\ಓಂ ಚಕ್ರಾಯ~।ಓಂ ಶಂಖಾಯ~।ಓಂ ಪಾಶಾಯ~।ಓಂ ಖೇಟಾಯ~।\\ಓಂ ಚಾಪಾಯ~।  ಕಹ್ಲಾರಾಯ ನಮಃ~॥

ವಿಜಯಾದೇವ್ಯೈ ವಿದ್ಮಹೇ ಮಹಾನಿತ್ಯಾಯೈ ಧೀಮಹಿ ತನ್ನೋ ನಿತ್ಯಾ ಪ್ರಚೋದಯಾತ್~॥\\
ಇತಿ ವಿಜಯಾನಿತ್ಯಾ ಆವರಣಪೂಜಾ~।
\section{೧೩। ಸರ್ವಮಂಗಲಾನಿತ್ಯಾ}
\addcontentsline{toc}{section}{೧೩। ಸರ್ವಮಂಗಲಾನಿತ್ಯಾ}
ಅಸ್ಯ ಶ್ರೀಸರ್ವಮಂಗಲಾನಿತ್ಯಾ ಮಹಾಮಂತ್ರಸ್ಯ ಚಂದ್ರ ಋಷಿಃ~।\\ ಗಾಯತ್ರೀ ಛಂದಃ~। ಸರ್ವಮಂಗಲಾನಿತ್ಯಾ ದೇವತಾ~।\\
ಸ್ವಾಂ , ಸ್ವೀಂ ಇತ್ಯಾದಿನಾ ನ್ಯಾಸಃ~।\\
{\bfseries ರಕ್ತೋತ್ಪಲಸಮಪ್ರಖ್ಯಾಂ ಪದ್ಮಪತ್ರನಿಭೇಕ್ಷಣಾಂ~।\\
ಇಕ್ಷುಕಾರ್ಮುಕಪುಷ್ಪೌಘಪಾಶಾಂಕುಶಸಮನ್ವಿತಾಂ ॥\\
ಸುಪ್ರಸನ್ನಾಂ ಶಶಿಮುಖೀಂ ನಾನಾರತ್ನವಿಭೂಷಿತಾಂ~।\\
ಶುಭ್ರಪದ್ಮಾಸನಸ್ಥಾಂ ತಾಂ ಭಜಾಮಿ ಸರ್ವಮಂಗಲಾಂ ॥\\}
ಮನುಃ :{\bfseries  ಸ್ವೌಂ ॥}

ಓಂ ಭದ್ರಾ~।  ಭವಾನೀ~।  ಭವ್ಯಾ~।  ವಿಶಾಲಾಕ್ಷೀ~।  ಶುಚಿಸ್ಮಿತಾ~।  ಕುಂಕುಮಾ~।  ಕಮಲಾ~।  ಕಲ್ಪಾ~॥ 

ಓಂ ಕಲಾ~।  ಪೂರಣೀ~।  ನಿತ್ಯಾ~।  ಅಮೃತಾ~।  ಜೀವಿತಾ~।  ದಯಾ~।  ಅಶೋಕಾ~।  ಅಮಲಾ~।  ಪೂರ್ಣಾ~।  ಪುಣ್ಯಾ~।  ಭಾಗ್ಯಾ~।  ಉದ್ಯತಾ~।  ವಿವೇಕಾ~।  ವಿಭವಾ~।  ವಿಶ್ವಾ~।  ವಿನತಾ~॥ 

ಓಂ ಕಾಮಿನೀ~।  ಖೇಚರೀ~।  ಆರ್ಯಾ~।  ಪುರಾಣಾ~।  ಪರಮೇಶ್ವರೀ~।  ಗೌರೀ~।  ಶಿವಾ~।  ಅಮೇಯಾ~।  ವಿಮಲಾ~।  ವಿಜಯಾ~।  ಪರಾ~।  ಪವಿತ್ರಾ~।  ಪದ್ಮಿನೀ~।  ವಿದ್ಯಾ~।  ವಿಶ್ವೇಶೀ। ಶಿವವಲ್ಲಭಾ।  ಅಶೇಷರೂಪಾ।  ಆನಂದಾ।  ಅಂಬುಜಾಕ್ಷೀ~।  ಅನಿಂದಿತಾ~।  ವರದಾ~।  ವಾಕ್ಪ್ರದಾ~।  ವಾಣೀ।  ವಿವಿಧಾ।  ವೇದವಿಗ್ರಹಾ~।  ವಂದ್ಯಾ~।  ವಾಗೀಶ್ವರೀ~।  ಸತ್ಯಾ~।  ಸಂಯತಾ~।  ಸರಸ್ವತೀ~।  ನಿರ್ಮಲಾ~।  ನಾದರೂಪಾ~॥ 

ಓಂ ಲಂ ಇಂದ್ರಶಕ್ತಿ~।  ರಂ ಅಗ್ನಿಶಕ್ತಿ~।  ಮಂ ಯಮಶಕ್ತಿ~।  ಕ್ಷಂ ನಿರ್ಋತಿಶಕ್ತಿ~।  ವಂ ವರುಣಶಕ್ತಿ~।  ಯಂ ವಾಯುಶಕ್ತಿ~।  ಕುಂ ಕುಬೇರಶಕ್ತಿ~।  ಹಂ ಈಶಾನಶಕ್ತಿ~।  ಆಂ ಬ್ರಹ್ಮಶಕ್ತಿ~।  ಹ್ರೀಂ ಅನಂತಶಕ್ತಿ~।  ನಿಯತಿಶಕ್ತಿ~।  ಕಾಲಶಕ್ತಿ~॥ 

ಓಂ ವಂ ವಜ್ರಶಕ್ತಿ~।  ಶಂ ಶಕ್ತಿಶಕ್ತಿ~।  ದಂ ದಂಡಶಕ್ತಿ~।  ಖಂ ಖಡ್ಗಶಕ್ತಿ~।  ಪಾಂ ಪಾಶಶಕ್ತಿ~।  ಅಂ ಅಂಕುಶಶಕ್ತಿ~।  ಗಂ ಗದಾಶಕ್ತಿ~।  ತ್ರಿಂ ತ್ರಿಶೂಲಶಕ್ತಿ~।  ಪಂ ಪದ್ಮಶಕ್ತಿ~।  ಚಂ ಚಕ್ರಶಕ್ತಿ~॥

ಸರ್ವಮಂಗಲಾಯೈ ವಿದ್ಮಹೇ ಚಂದ್ರಾತ್ಮಿಕಾಯೈ ಧೀಮಹಿ ।\\ತನ್ನೋ ನಿತ್ಯಾ ಪ್ರಚೋದಯಾತ್~॥\\
ಇತಿ ಸರ್ವಮಂಗಲಾ ಆವರಣಪೂಜಾ~।
\section{೧೪। ಜ್ವಾಲಾಮಾಲಿನೀ}
\addcontentsline{toc}{section}{೧೪। ಜ್ವಾಲಾಮಾಲಿನೀ}
ಅಸ್ಯ ಶ್ರೀ ಜ್ವಾಲಾಮಾಲಿನೀನಿತ್ಯಾಮಹಾಮಂತ್ರಸ್ಯ ಕಶ್ಯಪ ಋಷಿಃ~। ಗಾಯತ್ರೀ ಛಂದಃ~। ಜ್ವಾಲಾಮಾಲಿನೀನಿತ್ಯಾ ದೇವತಾ~। ರಂ ಬೀಜಂ~। ಫಟ್ ಶಕ್ತಿಃ~। ಹುಂ ಕೀಲಕಂ~।\\
\as{ನ್ಯಾಸಃ :}೧.ಓಂ ಓಂ ೨.ಓಂ ನಮಃ ೩.ಓಂ ಭಗವತಿ ೪.ಓಂ ಜ್ವಾಲಾಮಾಲಿನಿ ೫.ಓಂ ದೇವದೇವಿ ೬.ಓಂ ಸರ್ವಭೂತಸಂಹಾರಕಾರಿಕೇ \\
{\bfseries ಅಗ್ನಿಜ್ವಾಲಾಸಮಾಭಾಕ್ಷೀಂ ನೀಲವಕ್ತ್ರಾಂ ಚತುರ್ಭುಜಾಂ~।\\ನೀಲನೀರದಸಂಕಾಶಾಂ ನೀಲಕೇಶೀಂ ತನೂದರೀಂ ॥\\
ಖಡ್ಗಂ ತ್ರಿಶೂಲಂ ಬಿಭ್ರಾಣಾಂ ವರಂ ಸಾಭಯಮೇವ ಚ~।\\ಸಿಂಹಪೃಷ್ಠಸಮಾರೂಢಾಂ ಧ್ಯಾಯೇಜ್ಜ್ವಾಲಾದ್ಯಮಾಲಿನೀಂ ॥\\}
ಮನುಃ :{\bfseries  ಓಂ ನಮೋ ಭಗವತಿ ಜ್ವಾಲಾಮಾಲಿನಿ ದೇವದೇವಿ ಸರ್ವಭೂತಸಂಹಾರಕಾರಿಕೇ ಜಾತವೇದಸಿ ಜ್ವಲಂತಿ ಜ್ವಲ ಜ್ವಲ ಪ್ರಜ್ವಲ ಪ್ರಜ್ವಲ ಹ್ರಾಂ ಹ್ರೀಂ ಹ್ರೂಂ ರರ ರರ ರರರ ಹುಂ ಫಟ್ ಸ್ವಾಹಾ~॥}

ಓಂ ಅಭೀತ್ಯೈ ನಮಃ~।  ವಹ್ನಯೇ ನಮಃ~।  ಶಂಖಾಯ ನಮಃ~।  ಬಾಣೇಭ್ಯೋ ನಮಃ~।  ಖಡ್ಗಾಯ ನಮಃ~।  ಅಂಕುಶಾಯ ನಮಃ~।  ಪಾಶಾಯ ನಮಃ~। \\ಖೇಟಾಯ ನಮಃ~।  ಚಾಪಾಯ ನಮಃ~।  ಗದಾಯೈ ನಮಃ~।  ಶೂಲಾಯ ನಮಃ~।  ವರಾಯ ನಮಃ~॥

ಓಂ ಇಚ್ಛಾಶಕ್ತಿ~।  ಜ್ಞಾನಶಕ್ತಿ~।  ಕ್ರಿಯಾಶಕ್ತಿ~॥ 

ಓಂ ಡಾಕಿನೀ~।  ರಾಕಿಣೀ~।  ಲಾಕಿನೀ~।  ಕಾಕಿನೀ~।  ಸಾಕಿನೀ~।  ಹಾಕಿನೀ~॥ 

ಓಂ ಘಸ್ಮರಾ~।  ವಿಶ್ವಕವಲಾ~।  ಲೋಲಾಕ್ಷೀ~।  ಲೋಲಜಿಹ್ವಿಕಾ~।  ಸರ್ವಭಕ್ಷಾ~। \\ ಸಹಸ್ರಾಕ್ಷೀ~।  ನಿಃಸಂಗಾ~।  ಸಂಹೃತಿಪ್ರಿಯಾ~॥ 

ಓಂ ಅಚಿಂತ್ಯಾ~।  ಅಪ್ರಮೇಯಾ~।  ಪೂರ್ಣರೂಪಾ~।  ದುರಾಸದಾ~।  ಸರ್ವಾ~।  ಸಂಸಿದ್ಧಿರೂಪಾ~।  ಪಾವನಾ~।  ಏಕರೂಪಿಣೀ~॥ 

ಓಂ ಬ್ರಾಹ್ಮೀ~।  ಮಾಹೇಶ್ವರೀ~।  ಕೌಮಾರೀ~।  ವೈಷ್ಣವೀ~।  ವಾರಾಹೀ~।\\ ಇಂದ್ರಾಣೀ~।  ಚಾಮುಂಡಾ~।  ಮಹಾಲಕ್ಷ್ಮೀ~॥ 

ಓಂ ಲಂ ಇಂದ್ರಶಕ್ತಿ~।  ರಂ ಅಗ್ನಿಶಕ್ತಿ~।  ಮಂ ಯಮಶಕ್ತಿ~।  ಕ್ಷಂ ನಿರ್ಋತಿಶಕ್ತಿ~।  ವಂ ವರುಣಶಕ್ತಿ~।  ಯಂ ವಾಯುಶಕ್ತಿ~।  ಕುಂ ಕುಬೇರಶಕ್ತಿ~।  ಹಂ ಈಶಾನಶಕ್ತಿ~।  ಆಂ ಬ್ರಹ್ಮಶಕ್ತಿ~।  ಹ್ರೀಂ ಅನಂತಶಕ್ತಿ~।  ನಿಯತಿಶಕ್ತಿ~।  ಕಾಲಶಕ್ತಿ~॥

ಓಂ ವಂ ವಜ್ರಶಕ್ತಿ~।  ಶಂ ಶಕ್ತಿಶಕ್ತಿ~।  ದಂ ದಂಡಶಕ್ತಿ~।  ಖಂ ಖಡ್ಗಶಕ್ತಿ~।  ಪಾಂ ಪಾಶಶಕ್ತಿ~।  ಅಂ ಅಂಕುಶಶಕ್ತಿ~।  ಗಂ ಗದಾಶಕ್ತಿ~।  ತ್ರಿಂ ತ್ರಿಶೂಲಶಕ್ತಿ~।  ಪಂ ಪದ್ಮಶಕ್ತಿ~।  ಚಂ ಚಕ್ರಶಕ್ತಿ~॥

ಜ್ವಾಲಾಮಾಲಿನ್ಯೈ ವಿದ್ಮಹೇ ಮಹಾಜ್ವಾಲಾಯೈ ಧೀಮಹಿ ।\\ತನ್ನೋ ನಿತ್ಯಾ ಪ್ರಚೋದಯಾತ್~॥\\
ಇತಿ ಜ್ವಾಲಾಮಾಲಿನೀನಿತ್ಯಾ ಆವರಣಪೂಜಾ।
\section{೧೫। ಚಿತ್ರಾನಿತ್ಯಾ}
\addcontentsline{toc}{section}{೧೫। ಚಿತ್ರಾನಿತ್ಯಾ}
ಚಿತ್ರಾನಿತ್ಯಾಮಹಾಮಂತ್ರಸ್ಯ ಬ್ರಹ್ಮಾ ಋಷಿಃ~। ಗಾಯತ್ರೀ ಛಂದಃ~। \\ಚಿತ್ರಾ ನಿತ್ಯಾ ದೇವತಾ~। ಚಾಂ ಚೀಂ ಇತ್ಯಾದಿನಾ ನ್ಯಾಸಃ~।\\
{\bfseries ಶುದ್ಧಸ್ಫಟಿಕಸಂಕಾಶಾಂ ಪಲಾಶಕುಸುಮಪ್ರಭಾಂ~।\\
ನೀಲಮೇಘಪ್ರತೀಕಾಶಾಂ ಚತುರ್ಹಸ್ತಾಂ ತ್ರಿಲೋಚನಾಂ ॥\\
ಸರ್ವಾಲಂಕಾರಸಂಯುಕ್ತಾಂ ಪುಷ್ಪಬಾಣೇಕ್ಷುಚಾಪಿನೀಂ~।\\
ಪಾಶಾಂಕುಶಸಮೋಪೇತಾಂ ಧ್ಯಾಯೇಚ್ಚಿತ್ರಾಂ ಮಹೇಶ್ವರೀಂ ॥\\}
ಮನುಃ :{\bfseries  ಚ್ಕೌಂ~॥}

ಓಂ ಅಭಯಾಯ ನಮಃ~।  ಅಂಕುಶಾಯ ನಮಃ।  ಪಾಶಾಯ ನಮಃ।  ವರಾಯ ನಮಃ।  ಇಚ್ಛಾಶಕ್ತಿ~।  ಜ್ಞಾನಶಕ್ತಿ~।  ಕ್ರಿಯಾಶಕ್ತಿ~॥ 

ಓಂ ಅಂ ಆಂ ಇಂ ಈಂ ಉಂ ಊಂ ಋಂ ೠಂ \\ಲೃಂ ಲೄಂ ಏಂ ಐಂ  ಔಂ ಅಂ ಅಃ ಬ್ರಾಹ್ಮೀ~।\\  ಕಂ ಖಂ ಗಂ ಘಂ ಙಂ ಮಾಹೇಶ್ವರೀ~।\\  ಚಂ ಛಂ ಜಂ ಝಂ ಞಂ ಕೌಮಾರೀ~।\\  ಟಂ ಠಂ ಡಂ ಢಂ ಣಂ ವೈಷ್ಣವೀ~। \\ತಂ ಥಂ ದಂ ಧಂ ನಂ ವಾರಾಹೀ~। \\ಪಂ ಫಂ ಬಂ ಭಂ ಮಂ ಇಂದ್ರಾಣೀ~। \\ ಯಂ ರಂ ಲಂ ವಂ ಶಂ ಚಾಮುಂಡಾ~। \\ ಷಂ ಸಂ ಹಂ ಳಂ ಕ್ಷಂ ಮಹಾಲಕ್ಷ್ಮೀ~॥ 

ಓಂ ಭದ್ರಾ~।  ಭವಾನೀ~।  ಭವ್ಯಾ~।  ವಿಶಾಲಾಕ್ಷೀ~।  ಶುಚಿಸ್ಮಿತಾ~।  ಕುಂಕುಮಾ~।  ಕಮಲಾ~।  ಕಲ್ಪಾ~।  ಕಲಾ~।  ಪೂರಣೀ~।  ನಿತ್ಯಾ~।  ಅಮೃತಾ~।  ಜೀವಿತಾ~।  ದಯಾ~।  ಅಶೋಕಾ~।  ಅಮಲಾ~।  ಪೂರ್ಣಾ~।  ಪುಣ್ಯಾ~।  ಭಾಗ್ಯಾ~।  ಉದ್ಯತಾ~।  ವಿವೇಕಾ~।  ವಿಭವಾ~।  ವಿಶ್ವಾ~।  ವಿನತಾ~॥ 

ಓಂ ಕಾಮಿನೀ~।  ಖೇಚರೀ~।  ಆರ್ಯಾ~।  ಪುರಾಣಾ~।  ಪರಮೇಶ್ವರೀ~।  ಗೌರೀ~।  ಶಿವಾ~।  ಅಮೇಯಾ~।  ವಿಮಲಾ~।  ವಿಜಯಾ~।  ಪರಾ~।  ಪವಿತ್ರಾ~।  ಪದ್ಮಿನೀ~।  ವಿದ್ಯಾ~।  ವಿಶ್ವೇಶೀ। ಶಿವವಲ್ಲಭಾ। ಅಶೇಷರೂಪಾ।ಆನಂದಾ~।  ಅಂಬುಜಾಕ್ಷೀ~।  ಅನಿಂದಿತಾ~।  ವರದಾ। ವಾಕ್ಪ್ರದಾ। ವಾಣೀ।  ವಿವಿಧಾ~।  ವೇದವಿಗ್ರಹಾ~।  ವಂದ್ಯಾ~।  ವಾಗೀಶ್ವರೀ~।  ಸತ್ಯಾ~।  ಸಂಯತಾ~।  ಸರಸ್ವತೀ~।  ನಿರ್ಮಲಾ~।  ನಾದರೂಪಾ~॥

ಓಂ ಲಂ ಇಂದ್ರಶಕ್ತಿ~।  ರಂ ಅಗ್ನಿಶಕ್ತಿ~।  ಮಂ ಯಮಶಕ್ತಿ~।  ಕ್ಷಂ ನಿರ್ಋತಿಶಕ್ತಿ~।  ವಂ ವರುಣಶಕ್ತಿ~।  ಯಂ ವಾಯುಶಕ್ತಿ~।  ಕುಂ ಕುಬೇರಶಕ್ತಿ~।  ಹಂ ಈಶಾನಶಕ್ತಿ~।  ಆಂ ಬ್ರಹ್ಮಶಕ್ತಿ~।  ಹ್ರೀಂ ಅನಂತಶಕ್ತಿ~।  ನಿಯತಿಶಕ್ತಿ~।  ಕಾಲಶಕ್ತಿ~॥

ಓಂ ವಂ ವಜ್ರಶಕ್ತಿ~।  ಶಂ ಶಕ್ತಿಶಕ್ತಿ~।  ದಂ ದಂಡಶಕ್ತಿ~।  ಖಂ ಖಡ್ಗಶಕ್ತಿ~।  ಪಾಂ ಪಾಶಶಕ್ತಿ~।  ಅಂ ಅಂಕುಶಶಕ್ತಿ~।  ಗಂ ಗದಾಶಕ್ತಿ~।  ತ್ರಿಂ ತ್ರಿಶೂಲಶಕ್ತಿ~।  ಪಂ ಪದ್ಮಶಕ್ತಿ~।  ಚಂ ಚಕ್ರಶಕ್ತಿ~॥

ಓಂ ವಿಚಿತ್ರಾಯೈ ವಿದ್ಮಹೇ ಮಹಾನಿತ್ಯಾಯೈ ಧೀಮಹಿ~। ತನ್ನೋ ನಿತ್ಯಾ ಪ್ರಚೋದಯಾತ್ ॥
\authorline{ಇತಿಚಿತ್ರಾನಿತ್ಯಾ\\॥ಇತಿ ನಿತ್ಯಾ ಯಜನವಿಧಿಃ ಸಂಪೂರ್ಣಃ ॥}
\newpage


\section{ಶ್ರೀಚಕ್ರಆವರಣಪೂಜಾ}
\addcontentsline{toc}{section}{ಶ್ರೀಚಕ್ರಆವರಣಪೂಜಾ}
ಕೂಟ ತ್ರಯೇಣ ಕರನ್ಯಾಸಂ ಷಡಂಗನ್ಯಾಸಂ ಚ ವಿಧಾಯ , ಯಥಾಶಕ್ತಿ ಮೂಲಂ ಜಪಿತ್ವಾ\\
\as{೪ ದ್ರಾಂ} ಸರ್ವಸಂಕ್ಷೋಭಿಣ್ಯೈ ನಮಃ । \as{ದ್ರೀಂ} ಸರ್ವವಿದ್ರಾವಿಣ್ಯೈ । \as{ಕ್ಲೀಂ} ಸರ್ವಾಕರ್ಷಿಣ್ಯೈ । \as{ಬ್ಲೂಂ} ಸರ್ವವಶಂಕರ್ಯೈ । \as{ಸಃ} ಸರ್ವೋನ್ಮಾದಿನ್ಯೈ । \as{ಕ್ರೋಂ} ಸರ್ವಮಹಾಂಕುಶಾಯೈ । \as{ಹ್‌ಸ್‌ಖ್‌ಫ್ರೇಂ} ಸರ್ವಖೇಚರ್ಯೈ । \as{ಹ್ಸೌಃ} ಸರ್ವಬೀಜಾಯೈ । \as{ಐಂ} ಸರ್ವಯೋನ್ಯೈ । \as{ಹ್‌ಸ್‌ರೈಂ ಹ್‌ಸ್‌ಕ್ಲ್ರೀಂ ಹ್‌ಸ್‌ರ್ಸೌಃ} ಸರ್ವತ್ರಿಖಂಡಾಯೈ ನಮಃ ॥ ಇತಿ ದಶ ಮುದ್ರಾಃ ಪ್ರದರ್ಶ್ಯ

 ತ್ರಿಕೋಣಪೃಷ್ಠೇ ರೇಖಾತ್ರಯೇ ಗುರುಮಂತ್ರೇಣ ಗುರುತ್ರಯಂ ಸಂತರ್ಪ್ಯ, ಬಿಂದೌ ದಕ್ಷಿಣಾಮೂರ್ತಿಂ ಮೂಲವಿದ್ಯಾಂ ಚ ಸಕೃತ್ ತರ್ಪಯಿತ್ವಾ, ಆವಾಹಿತದೇವತಾಃ ಸಂತರ್ಪ್ಯ, ದಕ್ಷಿಣಾಮೂರ್ತಿಂ ಮೂಲವಿದ್ಯಾಂ ಚ ಬಿಂದೌ  ತ್ರಿಃ ಸಂತರ್ಪ್ಯ, ತ್ರಿಕೋಣೇ ಸ್ವಾಗ್ರಾದಿ ಕೋಣೇಷು ಬಿಂದೌ ಚ\\
\as{೪ ಐಂ ಕಏಈಲಹ್ರೀಂ} ಮಹಾಕಾಮೇಶ್ವರೀಶ್ರೀಪಾದುಕಾಂ ಪೂ । ತ । ನಮಃ ॥\\
\as{೪ ಕ್ಲೀಂ ಹಸಕಹಲಹ್ರೀಂ} ಮಹಾವಜ್ರೇಶ್ವರೀಶ್ರೀಪಾದುಕಾಂ ಪೂ।ತ।ನಮಃ॥\\
\as{೪ ಸೌಃ ಸಕಲಹ್ರೀಂ} ಮಹಾಭಗಮಾಲಿನೀಶ್ರೀಪಾದುಕಾಂ ಪೂ । ತ । ನಮಃ ॥\\
\as{೪ ೧೫} ಶ್ರೀಮಹಾತ್ರಿಪುರಸುಂದರೀಶ್ರೀಪಾದುಕಾಂ  ಪೂ । ತ । ನಮಃ ॥

ಆಗ್ನೇಯಾದಿ ಕೋಣೇಷು ಬಿಂದೌ ಚ \\
\as{೪ ಐಂ ಕಏಈಲಹ್ರೀಂ ।} ಸರ್ವಜ್ಞಾಯೈ ಹೃದಯಾಯ ನಮಃ ।\\ ಹೃದಯಶಕ್ತಿ ಶ್ರೀಪಾದುಕಾಂ ಪೂ । ತ । ನಮಃ ॥\\
\as{೪ ಕ್ಲೀಂ ಹಸಕಹಲಹ್ರೀಂ ।} ನಿತ್ಯತೃಪ್ತಾಯೈ ಶಿರಸೇ ಸ್ವಾಹಾ । \\ಶಿರಃಶಕ್ತಿ ಶ್ರೀಪಾದುಕಾಂ ಪೂ । ತ । ನಮಃ ॥\\
\as{೪ ಸೌಃ ಸಕಲಹ್ರೀಂ ।} ಅನಾದಿಬೋಧಿನ್ಯೈ ಶಿಖಾಯೈ ವಷಟ್ ।\\ ಶಿಖಾಶಕ್ತಿ ಶ್ರೀಪಾದುಕಾಂ ಪೂ । ತ । ನಮಃ ॥\\
\as{೪ ಐಂ ಕಏಈಲಹ್ರೀಂ ।} ಸ್ವತಂತ್ರಾಯೈ ಕವಚಾಯ ಹುಂ । \\ಕವಚಶಕ್ತಿ ಶ್ರೀಪಾದುಕಾಂ ಪೂ । ತ । ನಮಃ ॥\\
\as{೪ ಕ್ಲೀಂ ಹಸಕಹಲಹ್ರೀಂ ।} ನಿತ್ಯಾಲುಪ್ತಾಯೈ ನೇತ್ರತ್ರಯಾಯ ವೌಷಟ್ । \\ನೇತ್ರಶಕ್ತಿ ಶ್ರೀಪಾದುಕಾಂ ಪೂ । ತ । ನಮಃ ॥\\
\as{೪ ಸೌಃ ಸಕಲಹ್ರೀಂ ।} ಅಸ್ತ್ರಾಯ ಫಟ್ ।\\ ಅಸ್ತ್ರಶಕ್ತಿ ಶ್ರೀಪಾದುಕಾಂ ಪೂ । ತ । ನಮಃ ॥\\
ಇತಿ ಷಡಂಗದೇವತಾಃ ತರ್ಪಯಿತ್ವಾ\\

\as{ಓಂ ಐಂಹ್ರೀಂಶ್ರೀಂ} ತ್ರಿಪುರಸುಂದರ್ಯೈ ನಮಃ\\
\as{೪} ಹೃದಯದೇವ್ಯೈ ನಮಃ\\
\as{೪} ಶಿರೋದೇವ್ಯೈ ನಮಃ\\
\as{೪} ಶಿಖಾದೇವ್ಯೈ ನಮಃ\\
\as{೪} ಕವಚದೇವ್ಯೈ ನಮಃ\\
\as{೪} ನೇತ್ರದೇವ್ಯೈ ನಮಃ\\
\as{೪} ಅಸ್ತ್ರದೇವ್ಯೈ ನಮಃ

\as{೪} ಕಾಮೇಶ್ವರ್ಯೈ ನಮಃ\\
\as{೪} ಭಗಮಾಲಿನ್ಯೈ ನಮಃ\\
\as{೪} ನಿತ್ಯಕ್ಲಿನ್ನಾಯೈ ನಮಃ\\
\as{೪} ಭೇರುಂಡಾಯೈ ನಮಃ\\
\as{೪} ವಹ್ನಿವಾಸಿನ್ಯೈ ನಮಃ\\
\as{೪} ಮಹಾವಜ್ರೇಶ್ವರ್ಯೈ ನಮಃ\\
\as{೪} ಶಿವಾದೂತ್ಯೈ ನಮಃ\\
\as{೪} ತ್ವರಿತಾಯೈ ನಮಃ\\
\as{೪} ಕುಲಸುಂದರ್ಯೈ ನಮಃ\\
\as{೪} ನಿತ್ಯಾಯೈ ನಮಃ\\
\as{೪} ನೀಲಪತಾಕಾಯೈ ನಮಃ\\
\as{೪} ವಿಜಯಾಯೈ ನಮಃ\\
\as{೪} ಸರ್ವಮಂಗಳಾಯೈ ನಮಃ\\
\as{೪} ಜ್ವಾಲಾಮಾಲಿನ್ಯೈ ನಮಃ\\
\as{೪} ಚಿತ್ರಾಯೈ ನಮಃ\\
\as{೪} ಮಹಾನಿತ್ಯಾಯೈ ನಮಃ

\as{೪ ದಿವ್ಯೌಘಸಿದ್ಧೌಘಮಾನವೌಘೇಭ್ಯೋ ನಮಃ}

\as{೪} ಪರಪ್ರಕಾಶಾನಂದನಾಥಾಯ ನಮಃ\\
\as{೪} ಪರಶಿವಾನಂದನಾಥಾಯ ನಮಃ\\
\as{೪} ಪರಾಶಕ್ತ್ಯಂಬಾಯೈ ನಮಃ\\
\as{೪} ಕೌಲೇಶ್ವರಾನಂದನಾಥಾಯ ನಮಃ\\
\as{೪} ಶುಕ್ಲದೇವ್ಯಂಬಾಯೈ ನಮಃ\\
\as{೪} ಕುಲೇಶ್ವರಾನಂದನಾಥಾಯ ನಮಃ\\
\as{೪} ಕಾಮೇಶ್ವರ್ಯಂಬಾಯೈ ನಮಃ

\as{೪} ಭೋಗಾನಂದನಾಥಾಯ ನಮಃ\\
\as{೪} ಕ್ಲಿನ್ನಾನಂದನಾಥಾಯ ನಮಃ\\
\as{೪} ಸಮಯಾನಂದನಾಥಾಯ ನಮಃ\\
\as{೪} ಸಹಜಾನಂದನಾಥಾಯ ನಮಃ

\as{೪} ಗಗನಾನಂದನಾಥಾಯ ನಮಃ\\
\as{೪} ವಿಶ್ವಾನಂದನಾಥಾಯ ನಮಃ\\
\as{೪} ವಿಮಲಾನಂದನಾಥಾಯ ನಮಃ\\
\as{೪} ಮದನಾನಂದನಾಥಾಯ ನಮಃ\\
\as{೪} ಭುವನಾನಂದನಾಥಾಯ ನಮಃ\\
\as{೪} ಲೀಲಾನಂದನಾಥಾಯ ನಮಃ\\
\as{೪} ಸ್ವಾತ್ಮಾನಂದನಾಥಾಯ ನಮಃ\\
\as{೪} ಪ್ರಿಯಾನಂದನಾಥಾಯ ನಮಃ
\subsection{ಷೋಡಶ್ಯುಪಾಸಕಾನಾಂ ಕೃತೇ ವಿದ್ಯಾರ್ಣವತಂತ್ರೋಕ್ತ ಗುರುಪರಂಪರಾ}
\as{೪} ವ್ಯೋಮಾತೀತಾಂಬಾಯೈ ನಮಃ\\
\as{೪} ವ್ಯೋಮೇಶ್ವರ್ಯಂಬಾಯೈ ನಮಃ\\
\as{೪} ವ್ಯೋಮಗಾಂಬಾಯೈ ನಮಃ\\
\as{೪} ವ್ಯೋಮಚಾರಿಣ್ಯಂಬಾಯೈ ನಮಃ\\
\as{೪} ವ್ಯೋಮಸ್ಥಾಂಬಾಯೈ ನಮಃ

\as{೪} ಉನ್ಮನಾಕಾಶಾನಂದನಾಥಾಯ ನಮಃ\\
\as{೪} ಸಮಾನಾಕಾಶಾನಂದನಾಥಾಯ ನಮಃ\\
\as{೪} ವ್ಯಾಪಕಾಕಾಶಾನಂದನಾಥಾಯ ನಮಃ\\
\as{೪} ಶಕ್ತ್ಯಾಕಾಶಾನಂದನಾಥಾಯ ನಮಃ\\
\as{೪} ಧ್ವನ್ಯಾಕಾಶಾನಂದನಾಥಾಯ ನಮಃ\\
\as{೪} ಧ್ವನಿಮಾತ್ರಾಕಾಶಾನಂದನಾಥಾಯ ನಮಃ\\
\as{೪} ಅನಾಹತಾಕಾಶನಂದನಾಥಾಯ ನಮಃ\\
\as{೪} ಬಿಂದ್ವಾಕಾಶಾನಂದನಾಥಾಯ ನಮಃ\\
\as{೪} ಇಂದ್ವಾಕಾಶಾನಂದನಾಥಾಯ ನಮಃ

\as{೪} ಪರಮಾತ್ಮಾನಂದನಾಥಾಯ ನಮಃ\\
\as{೪} ಶಾಂಭವಾನಂದನಾಥಾಯ ನಮಃ\\
\as{೪} ಚಿನ್ಮುದ್ರಾನಂದನಾಥಾಯ ನಮಃ\\
\as{೪} ವಾಗ್ಭವಾನಂದನಾಥಾಯ ನಮಃ\\
\as{೪} ಲೀಲಾನಂದನಾಥಾಯ ನಮಃ\\
\as{೪} ಸಂಭ್ರಮಾನಂದನಾಥಾಯ ನಮಃ\\
\as{೪} ಚಿದಾನಂದನಾಥಾಯ ನಮಃ\\
\as{೪} ಪ್ರಸನ್ನಾನಂದನಾಥಾಯ ನಮಃ\\
\as{೪} ವಿಶ್ವಾನಂದನಾಥಾಯ ನಮಃ

\as{೪} ಪರಮೇಶ್ವರ ಪರಮೇಶ್ವರ್ಯೈ ನಮಃ\\
\as{೪} ಮಿತ್ರೀಶಮಯ್ಯೈ ನಮಃ\\
\as{೪} ಷಷ್ಠೀಶಮಯ್ಯೈ ನಮಃ\\
\as{೪} ಉಡ್ಡೀಶಮಯ್ಯೈ ನಮಃ\\
\as{೪} ಚರ್ಯಾನಾಥಮಯ್ಯೈ ನಮಃ\\
\as{೪} ಲೋಪಾಮುದ್ರಾಮಯ್ಯೈ ನಮಃ\\
\as{೪} ಅಗಸ್ತ್ಯಮಯ್ಯೈ ನಮಃ

\as{೪} ಕಾಲತಾಪನಮಯ್ಯೈ ನಮಃ\\
\as{೪} ಧರ್ಮಾಚಾರಮಯ್ಯೈ ನಮಃ\\
\as{೪} ಮುಕ್ತಕೇಶೀಶ್ವರಮಯ್ಯೈ ನಮಃ\\
\as{೪} ದೀಪಕಲಾನಾಥಮಯ್ಯೈ ನಮಃ

\as{೪} ವಿಷ್ಣುದೇವಮಯ್ಯೈ ನಮಃ\\
\as{೪} ಪ್ರಭಾಕರದೇವಮಯ್ಯೈ ನಮಃ\\
\as{೪} ತೇಜೋದೇವಮಯ್ಯೈ ನಮಃ\\
\as{೪} ಮನೋಜದೇವಮಯ್ಯೈ ನಮಃ\\
\as{೪} ಕಲ್ಯಾಣದೇವಮಯ್ಯೈ ನಮಃ\\
\as{೪} ರತ್ನದೇವಮಯ್ಯೈ ನಮಃ\\
\as{೪} ವಾಸುದೇವಮಯ್ಯೈ ನಮಃ\\
\as{೪} ಶ್ರೀರಾಮಾನಂದಮಯ್ಯೈ ನಮಃ

ಸಂವಿನ್ಮಯೇ ಪರೇ ದೇವಿ ಪರಾಮೃತರುಚಿಪ್ರಿಯೇ ।\\
 ಅನುಜ್ಞಾಂ ತ್ರಿಪುರೇ ದೇಹಿ ಪರಿವಾರಾರ್ಚನಾಯ ಮೇ ॥
 

\subsection{ಶ್ರೀಚಕ್ರ ಪ್ರಥಮಾವರಣದೇವತಾಃ}
{\bfseries ೪ ಅಂ ಆಂ ಸೌಃ ತ್ರೈಲೋಕ್ಯಮೋಹನ ಚಕ್ರಾಯ ನಮಃ }(ಪುಷ್ಪಾಂಜಲಿಃ)\\
\as{೪ ಅಂ} ಅಣಿಮಾಸಿದ್ಧ್ಯೈ ನಮಃ\\
\as{೪ ಲಂ} ಲಘಿಮಾಸಿದ್ಧ್ಯೈ ನಮಃ\\
\as{೪ ಮಂ} ಮಹಿಮಾಸಿದ್ಧ್ಯೈ ನಮಃ\\
\as{೪ ಈಂ} ಈಶಿತ್ವಸಿದ್ಧ್ಯೈ ನಮಃ\\
\as{೪ ವಂ} ವಶಿತ್ವಸಿದ್ಧ್ಯೈ ನಮಃ\\
\as{೪ ಪಂ} ಪ್ರಾಕಾಮ್ಯಸಿದ್ಧ್ಯೈ ನಮಃ\\
\as{೪ ಭುಂ} ಭುಕ್ತಿಸಿದ್ಧ್ಯೈ ನಮಃ\\
\as{೪ ಇಂ} ಇಚ್ಛಾಸಿದ್ಧ್ಯೈ ನಮಃ\\
\as{೪ ಪಂ} ಪ್ರಾಪ್ತಿಸಿದ್ಧ್ಯೈ ನಮಃ\\
\as{೪ ಸಂ} ಸರ್ವಕಾಮಸಿದ್ಧ್ಯೈ ನಮಃ

\as{೪ ಆಂ} ಬ್ರಾಹ್ಮ್ಯೈ ನಮಃ\\
\as{೪ ಈಂ} ಮಾಹೇಶ್ವರ್ಯೈ ನಮಃ\\
\as{೪ ಊಂ} ಕೌಮಾರ್ಯೈ ನಮಃ\\
\as{೪ ೠಂ} ವೈಷ್ಣವ್ಯೈ ನಮಃ\\
\as{೪ ಲೄಂ} ವಾರಾಹ್ಯೈ ನಮಃ\\
\as{೪ ಐಂ} ಮಾಹೇಂದ್ರ್ಯೈ ನಮಃ\\
\as{೪ ಔಂ} ಚಾಮುಂಡಾಯೈ ನಮಃ\\
\as{೪ ಅಃ} ಮಹಾಲಕ್ಷ್ಮ್ಯೈ ನಮಃ

\as{೪ ದ್ರಾಂ} ಸರ್ವಸಂಕ್ಷೋಭಿಣ್ಯೈ ನಮಃ\\
\as{೪ ದ್ರೀಂ} ಸರ್ವವಿದ್ರಾವಿಣ್ಯೈ ನಮಃ\\
\as{೪ ಕ್ಲೀಂ} ಸರ್ವಾಕರ್ಷಿಣ್ಯೈ ನಮಃ\\
\as{೪ ಬ್ಲೂಂ} ಸರ್ವವಶಂಕರ್ಯೈ ನಮಃ\\
\as{೪ ಸಃ} ಸರ್ವೋನ್ಮಾದಿನ್ಯೈ ನಮಃ\\
\as{೪ ಕ್ರೋಂ} ಸರ್ವಮಹಾಂಕುಶಾಯೈ ನಮಃ\\
\as{೪ ಹ್‌ಸ್‌ಖ್‌ಫ್ರೇಂ} ಸರ್ವಖೇಚರ್ಯೈ ನಮಃ\\
\as{೪ ಹ್ಸೌಃ} ಸರ್ವಬೀಜಾಯೈ ನಮಃ\\
\as{೪ ಐಂ} ಸರ್ವಯೋನ್ಯೈ ನಮಃ\\
\as{೪ ಹ್‌ಸ್‌ರೈಂ ಹ್‌ಸ್‌ಕ್ಲ್ರೀಂ ಹ್‌ಸ್‌ರ್ಸೌಃ} ಸರ್ವತ್ರಿಖಂಡಾಯೈ ನಮಃ\\
\as{೪ ಅಂ ಆಂ ಸೌಃ॥} ತ್ರೈಲೋಕ್ಯಮೋಹನ ಚಕ್ರಸ್ವಾಮಿನ್ಯೈ ನಮಃ

\as{೪ ಅಂ ಆಂ ಸೌಃ॥} ತ್ರಿಪುರಾಚಕ್ರೇಶ್ವರೀ ಶ್ರೀಪಾದುಕಾಂ ಪೂಜಯಾಮಿ ತರ್ಪಯಾಮಿ ನಮಃ ॥\\
\as{೪ ಅಂ} ಅಣಿಮಾಸಿದ್ಧಿ ಶ್ರೀಪಾದುಕಾಂ ಪೂಜಯಾಮಿ ತರ್ಪಯಾಮಿ ನಮಃ ॥\\
\as{೪ ದ್ರಾಂ} ಸರ್ವಸಂಕ್ಷೋಭಿಣೀಮುದ್ರಾಶಕ್ತಿ ಶ್ರೀಪಾದುಕಾಂ ಪೂಜಯಾಮಿ ತರ್ಪಯಾಮಿ ನಮಃ ॥\\
\as{೪ ದ್ರಾಂ ॥}\\
ಮೂಲೇನ ತ್ರಿಃ ಸಂತರ್ಪ್ಯ\\
\as{೪} ಅಭೀಷ್ಟಸಿದ್ಧಿಂ ಮೇ ದೇಹಿ ಶರಣಾಗತ ವತ್ಸಲೇ~।\\
ಭಕ್ತ್ಯಾ ಸಮರ್ಪಯೇ ತುಭ್ಯಂ ಪ್ರಥಮಾವರಣಾರ್ಚನಂ ॥
\subsection{ಶ್ರೀಚಕ್ರ ದ್ವಿತೀಯಾವರಣದೇವತಾಃ}
{\bfseries ೪ ಐಂ ಕ್ಲೀಂ ಸೌಃ ಸರ್ವಾಶಾಪರಿಪೂರಕ ಚಕ್ರಾಯ ನಮಃ}\\
ಪ್ರಕಟಯೋಗಿನ್ಯೈ ನಮಃ\\
\as{೪ ಅಂ} ಕಾಮಾಕರ್ಷಣ್ಯೈ ನಮಃ\\
\as{೪ ಆಂ} ಬುದ್ಧ್ಯಾಕರ್ಷಣ್ಯೈ ನಮಃ\\
\as{೪ ಇಂ} ಅಹಂಕಾರಾಕರ್ಷಣ್ಯೈ ನಮಃ\\
\as{೪ ಈಂ} ಶಬ್ದಾಕರ್ಷಣ್ಯೈ ನಮಃ\\
\as{೪ ಉಂ} ಸ್ಪರ್ಶಾಕರ್ಷಣ್ಯೈ ನಮಃ\\
\as{೪ ಊಂ} ರೂಪಾಕರ್ಷಣ್ಯೈ ನಮಃ\\
\as{೪ ಋಂ} ರಸಾಕರ್ಷಣ್ಯೈ ನಮಃ\\
\as{೪ ೠಂ} ಗಂಧಾಕರ್ಷಣ್ಯೈ ನಮಃ\\
\as{೪ ಲೃಂ} ಚಿತ್ತಾಕರ್ಷಣ್ಯೈ ನಮಃ\\
\as{೪ ಲೄಂ} ಧೈರ್ಯಾಕರ್ಷಣ್ಯೈ ನಮಃ\\
\as{೪ ಏಂ} ಸ್ಮೃತ್ಯಾಕರ್ಷಣ್ಯೈ ನಮಃ\\
\as{೪ ಐಂ} ನಾಮಾಕರ್ಷಣ್ಯೈ ನಮಃ\\
\as{೪ ಓಂ} ಬೀಜಾಕರ್ಷಣ್ಯೈ ನಮಃ\\
\as{೪ ಔಂ} ಆತ್ಮಾಕರ್ಷಣ್ಯೈ ನಮಃ\\
\as{೪ ಅಂ} ಅಮೃತಾಕರ್ಷಣ್ಯೈ ನಮಃ\\
\as{೪ ಅಃ} ಶರೀರಾಕರ್ಷಣ್ಯೈ ನಮಃ\\
\as{೪ ಐಂ ಕ್ಲೀಂ ಸೌಃ॥} ಸರ್ವಾಶಾಪರಿಪೂರಕ ಚಕ್ರಸ್ವಾಮಿನ್ಯೈ ನಮಃ

\as{೪ ಐಂ ಕ್ಲೀಂ ಸೌಃ॥} ತ್ರಿಪುರೇಶೀಚಕ್ರೇಶ್ವರೀ ಶ್ರೀಪಾದುಕಾಂ ಪೂ।ತ।ನಮಃ॥\\
\as{೪ ಲಂ} ಲಘಿಮಾಸಿದ್ಧಿ ಶ್ರೀಪಾದುಕಾಂ ಪೂ।ತ।ನಮಃ॥\\
\as{೪ ದ್ರೀಂ} ಸರ್ವವಿದ್ರಾವಿಣೀಮುದ್ರಾಶಕ್ತಿ ಶ್ರೀಪಾದುಕಾಂ ಪೂ।ತ।ನಮಃ॥\\
\as{೪ ದ್ರೀಂ ॥}\\
ಮೂಲೇನ ತ್ರಿಃ ಸಂತರ್ಪ್ಯ\\
\as{೪} ಅಭೀಷ್ಟಸಿದ್ಧಿಂ******ದ್ವಿತೀಯಾವರಣಾರ್ಚನಂ ॥
\subsection{ಶ್ರೀಚಕ್ರ ತೃತೀಯಾವರಣದೇವತಾಃ}
{\bfseries ೪ ಹ್ರೀಂ ಕ್ಲೀಂ ಸೌಃ ಸರ್ವಸಂಕ್ಷೋಭಣಚಕ್ರಾಯ ನಮಃ}\\
ಗುಪ್ತಯೋಗಿನ್ಯೈ ನಮಃ\\
\as{೪ ಕಂಖಂಗಂಘಂಙಂ} ಅನಂಗಕುಸುಮಾಯೈ ನಮಃ\\
\as{೪ ಚಂಛಂಜಂಝಂಞಂ} ಅನಂಗಮೇಖಲಾಯೈ ನಮಃ\\
\as{೪ ಟಂಠಂಡಂಢಂಣಂ} ಅನಂಗಮದನಾಯೈ ನಮಃ\\
\as{೪ ತಂಥಂದಂಧಂನಂ} ಅನಂಗಮದನಾತುರಾಯೈ ನಮಃ\\
\as{೪ ಪಂಫಂಬಂಭಂಮಂ} ಅನಂಗರೇಖಾಯೈ ನಮಃ\\
\as{೪ ಯಂರಂಲಂವಂ} ಅನಂಗವೇಗಿನ್ಯೈ ನಮಃ\\
\as{೪ ಶಂಷಂಸಂಹಂ} ಅನಂಗಾಂಕುಶಾಯೈ ನಮಃ\\
\as{೪ ಳಂಕ್ಷಂ} ಅನಂಗಮಾಲಿನ್ಯೈ ನಮಃ\\
\as{೪ ಹ್ರೀಂ ಕ್ಲೀಂ ಸೌಃ॥} ಸರ್ವಸಂಕ್ಷೋಭಣ ಚಕ್ರಸ್ವಾಮಿನ್ಯೈ ನಮಃ

\as{೪ ಹ್ರೀಂ ಕ್ಲೀಂ ಸೌಃ॥}  ತ್ರಿಪುರಸುಂದರೀಚಕ್ರೇಶ್ವರೀ ಶ್ರೀಪಾದುಕಾಂ ಪೂ।ತ।ನಮಃ॥\\
\as{೪ ಮಂ} ಮಹಿಮಾಸಿದ್ಧಿ ಶ್ರೀಪಾದುಕಾಂ ಪೂ।ತ।ನಮಃ॥\\
\as{೪ ಕ್ಲೀಂ} ಸರ್ವಾಕರ್ಷಿಣೀಮುದ್ರಾಶಕ್ತಿ ಶ್ರೀಪಾದುಕಾಂ ಪೂ।ತ।ನಮಃ॥\\
\as{೪ ಕ್ಲೀಂ ॥}\\
ಮೂಲೇನ ತ್ರಿಃ ಸಂತರ್ಪ್ಯ\\
\as{೪} ಅಭೀಷ್ಟಸಿದ್ಧಿಂ******ತೃತೀಯಾವರಣಾರ್ಚನಂ ॥
\subsection{ಶ್ರೀಚಕ್ರ ಚತುರ್ಥಾವರಣದೇವತಾಃ}
{\bfseries ೪ ಹೈಂ ಹ್‌ಕ್ಲೀಂ ಹ್‌ಸೌಃ ಸರ್ವಸೌಭಾಗ್ಯದಾಯಕ ಚಕ್ರಾಯ ನಮಃ}\\
ಗುಪ್ತತರಯೋಗಿನ್ಯೈ ನಮಃ\\
\as{೪ ಕಂ} ಸರ್ವಸಂಕ್ಷೋಭಿಣ್ಯೈ ನಮಃ\\
\as{೪ ಖಂ} ಸರ್ವವಿದ್ರಾವಿಣ್ಯೈ ನಮಃ\\
\as{೪ ಗಂ} ಸರ್ವಾಕರ್ಷಿಣ್ಯೈ ನಮಃ\\
\as{೪ ಘಂ} ಸರ್ವಾಹ್ಲಾದಿನ್ಯೈ ನಮಃ\\
\as{೪ ಙಂ} ಸರ್ವಸಮ್ಮೋಹಿನ್ಯೈ ನಮಃ\\
\as{೪ ಚಂ} ಸರ್ವಸ್ತಂಭಿನ್ಯೈ ನಮಃ\\
\as{೪ ಛಂ} ಸರ್ವಜೃಂಭಿಣ್ಯೈ ನಮಃ\\
\as{೪ ಜಂ} ಸರ್ವವಶಂಕರ್ಯೈ ನಮಃ\\
\as{೪ ಝಂ} ಸರ್ವರಂಜನ್ಯೈ ನಮಃ\\
\as{೪ ಞಂ} ಸರ್ವೋನ್ಮಾದಿನ್ಯೈ ನಮಃ\\
\as{೪ ಟಂ} ಸರ್ವಾರ್ಥಸಾಧಿನ್ಯೈ ನಮಃ\\
\as{೪ ಠಂ} ಸರ್ವಸಂಪತ್ತಿಪೂರಣ್ಯೈ ನಮಃ\\
\as{೪ ಡಂ} ಸರ್ವಮಂತ್ರಮಯ್ಯೈ ನಮಃ\\
\as{೪ ಢಂ} ಸರ್ವದ್ವಂದ್ವಕ್ಷಯಂಕರ್ಯೈ ನಮಃ\\
\as{೪ ಹೈಂ ಹ್‌ಕ್ಲೀಂ ಹ್‌ಸೌಃ॥}ಸರ್ವಸೌಭಾಗ್ಯದಾಯಕ ಚಕ್ರಸ್ವಾಮಿನ್ಯೈ ನಮಃ

\as{೪ ಹೈಂ ಹ್‌ಕ್ಲೀಂ ಹ್‌ಸೌಃ॥} ತ್ರಿಪುರವಾಸಿನೀಚಕ್ರೇಶ್ವರೀ ಶ್ರೀಪಾದುಕಾಂ ಪೂ।ತ।ನಮಃ॥\\
\as{೪ ಈಂ} ಈಶಿತ್ವಸಿದ್ಧಿ ಶ್ರೀಪಾದುಕಾಂ ಪೂ।ತ।ನಮಃ॥\\
\as{೪ ಬ್ಲೂಂ} ಸರ್ವವಶಂಕರೀಮುದ್ರಾಶಕ್ತಿ ಶ್ರೀಪಾದುಕಾಂ ಪೂ।ತ।ನಮಃ॥\\
\as{೪ ಬ್ಲೂಂ ॥}\\
ಮೂಲೇನ ತ್ರಿಃ ಸಂತರ್ಪ್ಯ\\
\as{೪} ಅಭೀಷ್ಟಸಿದ್ಧಿಂ******ತುರೀಯಾವರಣಾರ್ಚನಂ ॥
\subsection{ಶ್ರೀಚಕ್ರ ಪಂಚಮಾವರಣದೇವತಾಃ}
{\bfseries ೪ ಹ್‌ಸೈಂ ಹ್‌ಸ್‌ಕ್ಲೀಂ ಹ್‌ಸ್ಸೌಃ ಸರ್ವಾರ್ಥಸಾಧಕ ಚಕ್ರಾಯ ನಮಃ}\\
ಸಂಪ್ರದಾಯಯೋಗಿನ್ಯೈ ನಮಃ\\
\as{೪ ಣಂ} ಸರ್ವಸಿದ್ಧಿಪ್ರದಾಯೈ ನಮಃ\\
\as{೪ ತಂ} ಸರ್ವಸಂಪತ್ಪ್ರದಾಯೈ ನಮಃ\\
\as{೪ ಥಂ} ಸರ್ವಪ್ರಿಯಂಕರ್ಯೈ ನಮಃ\\
\as{೪ ದಂ} ಸರ್ವಮಂಗಳಕಾರಿಣ್ಯೈ ನಮಃ\\
\as{೪ ಧಂ} ಸರ್ವಕಾಮಪ್ರದಾಯೈ ನಮಃ\\
\as{೪ ನಂ} ಸರ್ವದುಃಖವಿಮೋಚನ್ಯೈ ನಮಃ\\
\as{೪ ಪಂ} ಸರ್ವಮೃತ್ಯುಪ್ರಶಮನ್ಯೈ ನಮಃ\\
\as{೪ ಫಂ} ಸರ್ವವಿಘ್ನನಿವಾರಿಣ್ಯೈ ನಮಃ\\
\as{೪ ಬಂ} ಸರ್ವಾಂಗಸುಂದರ್ಯೈ ನಮಃ\\
\as{೪ ಭಂ} ಸರ್ವಸೌಭಾಗ್ಯದಾಯಿನ್ಯೈ ನಮಃ\\
\as{೪ ಹ್‌ಸೈಂ ಹ್‌ಸ್‌ಕ್ಲೀಂ ಹ್‌ಸ್ಸೌಃ ॥} ಸರ್ವಾರ್ಥಸಾಧಕ ಚಕ್ರಸ್ವಾಮಿನ್ಯೈ ನಮಃ

\as{೪ ಹ್‌ಸೈಂ ಹ್‌ಸ್‌ಕ್ಲೀಂ ಹ್‌ಸ್ಸೌಃ ॥} ತ್ರಿಪುರಾಶ್ರೀಚಕ್ರೇಶ್ವರೀ ಶ್ರೀಪಾದುಕಾಂ ಪೂ।ತ।ನಮಃ॥\\
\as{೪ ವಂ} ವಶಿತ್ವಸಿದ್ಧಿ ಶ್ರೀಪಾದುಕಾಂ ಪೂ।ತ।ನಮಃ॥\\
\as{೪ ಸಃ }ಸರ್ವೋನ್ಮಾದಿನೀಮುದ್ರಾಶಕ್ತಿ ಶ್ರೀಪಾದುಕಾಂ ಪೂ।ತ।ನಮಃ॥\\
\as{೪ ಸಃ॥}\\
ಮೂಲೇನ ತ್ರಿಃ ಸಂತರ್ಪ್ಯ\\
\as{೪} ಅಭೀಷ್ಟಸಿದ್ಧಿಂ******ಪಂಚಮಾವರಣಾರ್ಚನಂ ॥
\subsection{ಶ್ರೀಚಕ್ರ ಷಷ್ಠಾವರಣದೇವತಾಃ}
{\bfseries ೪ ಹ್ರೀಂ ಕ್ಲೀಂ ಬ್ಲೇಂ ಸರ್ವರಕ್ಷಾಕರ ಚಕ್ರಾಯ ನಮಃ}\\
ಕುಲೋತ್ತೀರ್ಣಯೋಗಿನ್ಯೈ ನಮಃ\\
\as{೪ ಮಂ} ಸರ್ವಜ್ಞಾಯೈ ನಮಃ\\
\as{೪ ಯಂ} ಸರ್ವಶಕ್ತ್ಯೈ ನಮಃ\\
\as{೪ ರಂ} ಸರ್ವೈಶ್ವರ್ಯಪ್ರದಾಯೈ ನಮಃ\\
\as{೪ ಲಂ} ಸರ್ವಜ್ಞಾನಮಯ್ಯೈ ನಮಃ\\
\as{೪ ವಂ} ಸರ್ವವ್ಯಾಧಿವಿನಾಶಿನ್ಯೈ ನಮಃ\\
\as{೪ ಶಂ} ಸರ್ವಾಧಾರಸ್ವರೂಪಾಯೈ ನಮಃ\\
\as{೪ ಷಂ} ಸರ್ವಪಾಪಹರಾಯೈ ನಮಃ\\
\as{೪ ಸಂ} ಸರ್ವಾನಂದಮಯ್ಯೈ ನಮಃ\\
\as{೪ ಹಂ} ಸರ್ವರಕ್ಷಾಸ್ವರೂಪಿಣ್ಯೈ ನಮಃ\\
\as{೪ ಕ್ಷಂ} ಸರ್ವೇಪ್ಸಿತಫಲಪ್ರದಾಯೈ ನಮಃ\\
\as{೪ ಹ್ರೀಂ ಕ್ಲೀಂ ಬ್ಲೇಂ॥} ಸರ್ವರಕ್ಷಾಕರಚಕ್ರಸ್ವಾಮಿನ್ಯೈ ನಮಃ

\as{೪ ಹ್ರೀಂ ಕ್ಲೀಂ ಬ್ಲೇಂ॥} ತ್ರಿಪುರಮಾಲಿನೀಚಕ್ರೇಶ್ವರೀ ಶ್ರೀಪಾದುಕಾಂ ಪೂ।ತ।ನಮಃ॥\\
\as{೪ ಪಂ} ಪ್ರಾಕಾಮ್ಯಸಿದ್ಧಿ ಶ್ರೀಪಾದುಕಾಂ ಪೂ।ತ।ನಮಃ॥\\
\as{೪ ಕ್ರೋಂ} ಸರ್ವಮಹಾಂಕುಶಾಮುದ್ರಾಶಕ್ತಿ ಶ್ರೀಪಾದುಕಾಂ ಪೂ।ತ।ನಮಃ॥\\
\as{೪ ಕ್ರೋಂ॥}\\
ಮೂಲೇನ ತ್ರಿಃ ಸಂತರ್ಪ್ಯ\\
\as{೪} ಅಭೀಷ್ಟಸಿದ್ಧಿಂ*******ಷಷ್ಠಾಖ್ಯಾವರಣಾರ್ಚನಂ ॥
\subsection{ಶ್ರೀಚಕ್ರ ಸಪ್ತಮಾವರಣದೇವತಾಃ}
{\bfseries ೪ ಹ್ರೀಂ ಶ್ರೀಂ ಸೌಃ ಸರ್ವರೋಗಹರ ಚಕ್ರಾಯ ನಮಃ}\\
ನಿಗರ್ಭಯೋಗಿನ್ಯೈ ನಮಃ\\
\as{೪ ಅಂಆಂ++ಅಃ । ರ್ಬ್ಲೂಂ} ವಶಿನ್ಯೈ ನಮಃ\\
\as{೪ ಕಂಖಂಗಂಘಂಙಂ । ಕ್‌ಲ್‌ಹ್ರೀಂ} ಕಾಮೇಶ್ವರ್ಯೈ ನಮಃ\\
\as{೪ ಚಂಛಂಜಂಝಂಞಂ । ನ್‌ವ್ಲೀಂ} ಮೋದಿನ್ಯೈ ನಮಃ\\
\as{೪ ಟಂಠಂಡಂಢಂಣಂ । ಯ್ಲೂಂ} ವಿಮಲಾಯೈ ನಮಃ\\
\as{೪ ತಂಥಂದಂಧಂನಂ । ಜ್‌ಮ್ರೀಂ} ಅರುಣಾಯೈ ನಮಃ\\
\as{೪ ಪಂಫಂಬಂಭಂಮಂ । ಹ್‌ಸ್‌ಲ್‌ವ್ಯೂಂ} ಜಯಿನ್ಯೈ ನಮಃ\\
\as{೪ ಯಂರಂಲಂವಂ । ಝ್‌ಮ್‌ರ್ಯೂಂ} ಸರ್ವೇಶ್ವರ್ಯೈ ನಮಃ\\
\as{೪ ಶಂಷಂಸಂಹಂಳಂಕ್ಷಂ । ಕ್ಷ್‌ಮ್ರೀಂ} ಕೌಳಿನ್ಯೈ ನಮಃ\\
\as{೪ ಹ್ರೀಂ ಶ್ರೀಂ ಸೌಃ॥} ಸರ್ವರೋಗಹರಚಕ್ರಸ್ವಾಮಿನ್ಯೈ ನಮಃ

\as{೪ ಹ್ರೀಂ ಶ್ರೀಂ ಸೌಃ॥} ತ್ರಿಪುರಾಸಿದ್ಧಾಚಕ್ರೇಶ್ವರೀ ಶ್ರೀಪಾದುಕಾಂ ಪೂ।ತ।ನಮಃ॥\\
\as{೪ ಭುಂ} ಭುಕ್ತಿಸಿದ್ಧಿ ಶ್ರೀಪಾದುಕಾಂ ಪೂ।ತ।ನಮಃ॥\\
\as{೪ ಹ್‌ಸ್‌ಖ್‌ಫ್ರೇಂ} ಸರ್ವಖೇಚರೀಮುದ್ರಾಶಕ್ತಿ ಶ್ರೀಪಾದುಕಾಂ ಪೂ।ತ।ನಮಃ॥\\
\as{೪ ಹ್‌ಸ್‌ಖ್‌ಫ್ರೇಂ ॥}\\
ಮೂಲೇನ ತ್ರಿಃ ಸಂತರ್ಪ್ಯ\\
\as{೪} ಅಭೀಷ್ಟಸಿದ್ಧಿಂ********ಸಪ್ತಮಾವರಣಾರ್ಚನಂ ॥
\subsection{ಶ್ರೀಚಕ್ರ ಅಷ್ಟಮಾವರಣದೇವತಾಃ}
{\bfseries ೪ ಹ್‌ಸ್‌ರೈಂ ಹ್‌ಸ್‌ಕ್ಲ್ರೀಂ ಹ್‌ಸ್‌ರ್ಸೌಃ ಸರ್ವಸಿದ್ಧಿಪ್ರದ ಚಕ್ರಾಯ ನಮಃ}\\
ರಹಸ್ಯಯೋಗಿನ್ಯೈ ನಮಃ\\
\as{೪ ಯಾಂರಾಂಲಾಂವಾಂಸಾಂ ದ್ರಾಂದ್ರೀಂಕ್ಲೀಂಬ್ಲೂಂಸಃ} ಬಾಣಿನ್ಯೈ ನಮಃ\\
\as{೪ ಥಂಧಂ} ಚಾಪಿನ್ಯೈ ನಮಃ\\
\as{೪ ಹ್ರೀಂಆಂ} ಪಾಶಿನ್ಯೈ ನಮಃ\\
\as{೪ ಕ್ರೋಂಕ್ರೋಂ} ಅಂಕುಶಿನ್ಯೈ ನಮಃ\\
\as{೪ ಐಂ೫} ಮಹಾಕಾಮೇಶ್ವರ್ಯೈ ನಮಃ\\
\as{೪ ಕ್ಲೀಂ೬} ಮಹಾವಜ್ರೇಶ್ವರ್ಯೈ ನಮಃ\\
\as{೪ ಸೌಃ೪} ಮಹಾಭಗಮಾಲಿನ್ಯೈ ನಮಃ\\
\as{೪ ೧೫} ಮಹಾಶ್ರೀಸುಂದರ್ಯೈ ನಮಃ\\
\as{೪ ಹ್‌ಸ್‌ರೈಂ ಹ್‌ಸ್‌ಕ್ಲ್ರೀಂ ಹ್‌ಸ್‌ರ್ಸೌಃ ॥} ಸರ್ವಸಿದ್ಧಿಪ್ರದಚಕ್ರಸ್ವಾಮಿನ್ಯೈ ನಮಃ

{\bfseries ೪ ಐಂ ೫॥} ಸೂರ್ಯಚಕ್ರೇ ಕಾಮಗಿರಿಪೀಠೇ ಮಿತ್ರೀಶನಾಥ ನವಯೋನಿ ಚಕ್ರಾತ್ಮಕ ಆತ್ಮತತ್ವ ಸಂಹಾರಕೃತ್ಯ ಜಾಗ್ರದ್ದಶಾಧಿಷ್ಠಾಯಕ ಇಚ್ಛಾಶಕ್ತಿ ವಾಗ್ಭವಾತ್ಮಕ ಪರಾಪರಶಕ್ತಿ ಸ್ವರೂಪ ರುದ್ರಾತ್ಮಶಕ್ತಿ ಮಹಾಕಾಮೇಶ್ವರೀ ಶ್ರೀಪಾದುಕಾಂ ಪೂ।ತ।ನಮಃ॥೧

{\bfseries೪ ಕ್ಲೀಂ ೬॥} ಸೋಮಚಕ್ರೇ ಪೂರ್ಣಗಿರಿಪೀಠೇ ಉಡ್ಡೀಶನಾಥ  ದಶಾರದ್ವಯ ಚತುರ್ದಶಾರ ಚಕ್ರಾತ್ಮಕ ವಿದ್ಯಾತತ್ವ ಸ್ಥಿತಿಕೃತ್ಯ ಸ್ವಪ್ನದಶಾಧಿಷ್ಠಾಯಕ ಜ್ಞಾನಶಕ್ತಿ ಕಾಮರಾಜಾತ್ಮಕ ಕಾಮಕಲಾ ಸ್ವರೂಪ ವಿಷ್ಣ್ವಾತ್ಮಶಕ್ತಿ ಮಹಾವಜ್ರೇಶ್ವರೀ ಶ್ರೀಪಾದುಕಾಂ ಪೂ।ತ।ನಮಃ॥೨

{\bfseries೪ ಸೌಃ ೪॥} ಅಗ್ನಿಚಕ್ರೇ ಜಾಲಂಧರಪೀಠೇ  ಷಷ್ಠೀಶನಾಥ ಅಷ್ಟದಳ ಷೋಡಶದಳ ಚತುರಸ್ರ ಚಕ್ರಾತ್ಮಕ ಶಿವತತ್ವ ಸೃಷ್ಟಿಕೃತ್ಯ ಸುಷುಪ್ತಿದಶಾಧಿಷ್ಠಾಯಕ ಕ್ರಿಯಾಶಕ್ತಿ ಶಕ್ತಿಬೀಜಾತ್ಮಕ ವಾಗೀಶ್ವರೀ ಸ್ವರೂಪ ಬ್ರಹ್ಮಾತ್ಮಶಕ್ತಿ  ಮಹಾಭಗಮಾಲಿನೀ ಶ್ರೀಪಾದುಕಾಂ ಪೂ।ತ।ನಮಃ॥೩

{\bfseries೪ ೧೫॥} ಪರಬ್ರಹ್ಮಚಕ್ರೇ ಮಹೋಡ್ಯಾಣಪೀಠೇ ಚರ್ಯಾನಂದನಾಥ ಸಮಸ್ತಚಕ್ರಾತ್ಮಕ ಸಪರಿವಾರ ಪರಮತತ್ವ ಸೃಷ್ಟಿಸ್ಥಿತಿಸಂಹಾರಕೃತ್ಯ ತುರೀಯದಶಾಧಿಷ್ಠಾಯಕ ಇಚ್ಛಾಜ್ಞಾನಕ್ರಿಯಾಶಾಂತಶಕ್ತಿ ವಾಗ್ಭವ ಕಾಮರಾಜ ಶಕ್ತಿಬೀಜಾತ್ಮಕ ಪರಮಶಕ್ತಿಸ್ವರೂಪ ಪರಬ್ರಹ್ಮಾತ್ಮಶಕ್ತಿ ಶ್ರೀಮಹಾತ್ರಿಪುರಸುಂದರೀ ಶ್ರೀಪಾದುಕಾಂ ಪೂ।ತ।ನಮಃ॥(ಬಿಂದೌ)

\as{೪ ಹ್‌ಸ್‌ರೈಂ ಹ್‌ಸ್‌ಕ್ಲ್ರೀಂ ಹ್‌ಸ್‌ರ್ಸೌಃ ॥} ತ್ರಿಪುರಾಂಬಾ ಚಕ್ರೇಶ್ವರೀ ಶ್ರೀಪಾದುಕಾಂ ಪೂ।ತ।ನಮಃ॥\\
\as{೪ ಇಂ} ಇಚ್ಛಾಸಿದ್ಧಿ ಶ್ರೀಪಾದುಕಾಂ ಪೂ।ತ।ನಮಃ॥\\
\as{೪ ಹ್ಸೌಃ} ಸರ್ವಬೀಜಮುದ್ರಾಶಕ್ತಿ ಶ್ರೀಪಾದುಕಾಂ ಪೂ।ತ।ನಮಃ॥\\
\as{೪ ಹ್ಸೌಃ॥}\\
ಮೂಲೇನ ತ್ರಿಃ ಸಂತರ್ಪ್ಯ\\
\as{೪} ಅಭೀಷ್ಟಸಿದ್ಧಿಂ*******ಅಷ್ಟಮಾವರಣಾರ್ಚನಂ ॥
\subsection{ಶ್ರೀಚಕ್ರ ನವಮಾವರಣದೇವತಾ}
{\bfseries ೪ (೧೫) ಸರ್ವಾನಂದಮಯ ಚಕ್ರಾಯ ನಮಃ}\\
ಅತಿರಹಸ್ಯಯೋಗಿನ್ಯೈ ನಮಃ\\
\as{೪ ೧೫} ಶ್ರೀ ಶ್ರೀ ಮಹಾಭಟ್ಟಾರಿಕಾಯೈ ನಮಃ\\
\as{೪ ೧೫} ಸರ್ವಾನಂದಮಯಚಕ್ರಸ್ವಾಮಿನ್ಯೈ ನಮಃ\\
ಪರಾಪರರಹಸ್ಯಯೋಗಿನ್ಯೈ ನಮಃ

\as{೪ ಅಂ ಆಂ ಸೌಃ} ತ್ರಿಪುರಾಯೈ ನಮಃ\\
\as{೪ ಐಂ ಕ್ಲೀಂ ಸೌಃ} ತ್ರಿಪುರೇಶ್ಯೈ ನಮಃ\\
\as{೪ ಹ್ರೀಂ ಕ್ಲೀಂ ಸೌಃ} ತ್ರಿಪುರಸುಂದರ್ಯೈ ನಮಃ\\
\as{೪ ಹೈಂ ಹ್‌ಕ್ಲೀಂ ಹ್‌ಸೌಃ} ತ್ರಿಪುರವಾಸಿನ್ಯೈ ನಮಃ\\
\as{೪ ಹ್‌ಸೈಂ ಹ್‌ಸ್‌ಕ್ಲೀಂ ಹ್‌ಸ್ಸೌಃ} ತ್ರಿಪುರಾಶ್ರಿಯೈ ನಮಃ\\
\as{೪ ಹ್ರೀಂ ಕ್ಲೀಂ ಬ್ಲೇಂ} ತ್ರಿಪುರಮಾಲಿನ್ಯೈ ನಮಃ\\
\as{೪ ಹ್ರೀಂ ಶ್ರೀಂ ಸೌಃ} ತ್ರಿಪುರಾಸಿದ್ಧಾಯೈ ನಮಃ\\
\as{೪ ಹ್‌ಸ್‌ರೈಂ ಹ್‌ಸ್‌ಕ್ಲ್ರೀಂ ಹ್‌ಸ್‌ರ್ಸೌಃ} ತ್ರಿಪುರಾಂಬಾಯೈ ನಮಃ\\
\as{೪ ೧೫} ಮಹಾತ್ರಿಪುರಸುಂದರ್ಯೈ ನಮಃ

\as{೪} ಮಹಾಮಹೇಶ್ವರ್ಯೈ ನಮಃ\\
\as{೪} ಮಹಾಮಹಾರಾಜ್ಞ್ಯೈ ನಮಃ\\
\as{೪} ಮಹಾಮಹಾಶಕ್ತ್ಯೈ ನಮಃ\\
\as{೪} ಮಹಾಮಹಾಗುಪ್ತಾಯೈ ನಮಃ\\
\as{೪} ಮಹಾಮಹಾಜ್ಞಪ್ತ್ಯೈ ನಮಃ\\
\as{೪} ಮಹಾಮಹಾನಂದಾಯೈ ನಮಃ\\
\as{೪} ಮಹಾಮಹಾಸ್ಪಂದಾಯೈ ನಮಃ\\
\as{೪} ಮಹಾಮಹಾಶಯಾಯೈ ನಮಃ\\
\as{೪} ಮಹಾಮಹಾ ಶ್ರೀಚಕ್ರನಗರಸಾಮ್ರಾಜ್ಞ್ಯೈ ನಮಃ

{\bfseries ೪ (೧೫)} ಲಲಿತಾ ಮಹಾತ್ರಿಪುರಸುಂದರೀ ಪರಾಭಟ್ಟಾರಿಕಾ ಶ್ರೀಪಾದುಕಾಂ ಪೂ।ತ।ನಮಃ॥ಇತಿ ತ್ರಿಃ ಸಂತರ್ಪ್ಯ\\
\as{೪ (೧೫) }ಮಹಾತ್ರಿಪುರಸುಂದರೀ ಚಕ್ರೇಶ್ವರೀ ಶ್ರೀಪಾದುಕಾಂ ಪೂ।ತ।ನಮಃ॥\\
\as{೪ ಪಂ} ಪ್ರಾಪ್ತಿಸಿದ್ಧಿ ಶ್ರೀಪಾದುಕಾಂ ಪೂ।ತ।ನಮಃ॥\\
\as{೪ ಐಂ} ಸರ್ವಯೋನಿಮುದ್ರಾಶಕ್ತಿ ಶ್ರೀಪಾದುಕಾಂ ಪೂ।ತ।ನಮಃ॥\\
\as{೪ ಐಂ ॥}\\
ಮೂಲೇನ ತ್ರಿಃ ಸಂತರ್ಪ್ಯ

[ಷೋಡಶೀ ಉಪಾಸಕರು :\\
\as{೪ ಹಸಕಲ ಹಸಕಹಲ ಸಕಲಹ್ರೀಂ ॥} ತುರೀಯಾಂಬಾ ಶ್ರೀಪಾದುಕಾಂ ಪೂ।ತ।ನಮಃ॥ (ಇತಿ ತ್ರಿಃ ಸಂತರ್ಪ್ಯ)\\
\as{೪} ಸರ್ವಾನಂದಮಯೇ ಚಕ್ರೇ ಮಹೋಡ್ಯಾಣಪೀಠೇ ಚರ್ಯಾನಂದನಾಥಾತ್ಮಕ ತುರೀಯಾತೀತದಶಾಧಿಷ್ಠಾಯಕ ಶಾಂತ್ಯತೀತಕಲಾತ್ಮಕ ಪ್ರಕಾಶ ವಿಮರ್ಶ ಸಾಮರಸ್ಯಾತ್ಮಕ ಪರಬ್ರಹ್ಮಸ್ವರೂಪಿಣೀ ಪರಾಮೃತಶಕ್ತಿಃ ಸರ್ವ ಮಂತ್ರೇಶ್ವರೀ ಸರ್ವಪೀಠೇಶ್ವರೀ ಸರ್ವವೀರೇಶ್ವರೀ ಸಕಲಜಗದುತ್ಪತ್ತಿ ಮಾತೃಕಾ ಸಚಕ್ರಾ ಸದೇವತಾ ಸಾಸನಾ ಸಾಯುಧಾ ಸಶಕ್ತಿಃ ಸವಾಹನಾ ಸಪರಿವಾರಾ ಸಚಕ್ರೇಶೀಕಾ ಪರಯಾ ಅಪರಯಾ ಪರಾಪರಯಾ ಸಪರ್ಯಯಾ ಸರ್ವೋಪಚಾರೈಃ ಸಂಪೂಜಿತಾ ಸಂತರ್ಪಿತಾ ಸಂತುಷ್ಟಾ ಅಸ್ತು ನಮಃ ॥(ಇತಿ ಸಮಷ್ಟ್ಯಂಜಲಿಃ)

\as{೪ ಸಂ} ಸರ್ವಕಾಮಸಿದ್ಧಿ ಶ್ರೀಪಾದುಕಾಂ ಪೂ।ತ।ನಮಃ॥\\
\as{೪ ಹ್‌ಸ್‌ರೈಂ ಹ್‌ಸ್‌ಕ್ಲ್ರೀಂ ಹ್‌ಸ್‌ರ್ಸೌಃ} ಸರ್ವತ್ರಿಖಂಡಾ ಮುದ್ರಾಶಕ್ತಿ ಶ್ರೀಪಾದುಕಾಂ ಪೂ।ತ।ನಮಃ॥\\
\as{೪ ಹ್‌ಸ್‌ರೈಂ ಹ್‌ಸ್‌ಕ್ಲ್ರೀಂ ಹ್‌ಸ್‌ರ್ಸೌಃ ॥}\\
\as{೪ (೧೬)} ಮಹಾತ್ರಿಪುರಸುಂದರೀ ಶ್ರೀಪಾದುಕಾಂ ಪೂ।ತ।ನಮಃ॥(ಇತಿ ತ್ರಿಃ)\\
\as{೪} ಅಭೀಷ್ಟಸಿದ್ಧಿಂ******ನವಮಾವರಣಾರ್ಚನಂ ॥\\
ಇತ್ಯಾವರಣಾರ್ಚನಂ
\section{ಪಂಚಪಂಚಿಕಾಪೂಜಾ}
\subsection{ಪಂಚಲಕ್ಷ್ಮ್ಯಂಬಾಃ}
{\bfseries ೪ ೧೫ ॥}( ಶ್ರೀಮಹಾಲಕ್ಷ್ಮೀಶ್ವರೀ ಬೃಂದಮಂಡಿತಾಸನ ಸಂಸ್ಥಿತಾ~।\\
ಸರ್ವಸೌಭಾಗ್ಯ ಜನನೀ ಶ್ರೀಮಹಾ ತ್ರಿಪುರಸುಂದರೀ ॥)\\
ಶ್ರೀವಿದ್ಯಾ ಲಕ್ಷ್ಮ್ಯಂಬಾ ಶ್ರೀಪಾದುಕಾಂ ಪೂ।ತ।ನಮಃ॥\\
\as{೪ ಶ್ರೀಂ ॥}ಲಕ್ಷ್ಮೀಲಕ್ಷ್ಮ್ಯಂಬಾ ಶ್ರೀಪಾದುಕಾಂ ಪೂ।ತ।ನಮಃ॥೧\\
\as{೪ ಓಂಶ್ರೀಂ ಹ್ರೀಂ ಶ್ರೀಂ ಕಮಲೇ ಕಮಲಾಲಯೇ ಪ್ರಸೀದ ಪ್ರಸೀದ ಶ್ರೀಂ ಹ್ರೀಂ ಶ್ರೀಂ ಓಂ ಮಹಾಲಕ್ಷ್ಮ್ಯೈ ನಮಃ ॥}\\ಮಹಾಲಕ್ಷ್ಮೀಲಕ್ಷ್ಮ್ಯಂಬಾ ಶ್ರೀಪಾದುಕಾಂ ಪೂ।ತ।ನಮಃ॥೨\\
\as{೪ ಶ್ರೀಂ ಹ್ರೀಂ ಕ್ಲೀಂ ॥}ತ್ರಿಶಕ್ತಿಲಕ್ಷ್ಮ್ಯಂಬಾ ಶ್ರೀಪಾದುಕಾಂ ಪೂ।ತ।ನಮಃ॥೩\\
\as{೪ ಶ್ರೀಂ ಸಹಕಲ ಹ್ರೀಂ ಶ್ರೀಂ ॥} \\ಸರ್ವಸಾಮ್ರಾಜ್ಯಲಕ್ಷ್ಮ್ಯಂಬಾ ಶ್ರೀಪಾದುಕಾಂ ಪೂ।ತ।ನಮಃ॥೪
\subsection{ಪಂಚಕೋಶಾಂಬಾಃ}
{\bfseries ೪ ೧೫ ॥}( ಶ್ರೀಮಹಾಕೋಶೇಶ್ವರೀ ಬೃಂದಮಂಡಿತಾಸನ ಸಂಸ್ಥಿತಾ~।\\
ಸರ್ವಸೌಭಾಗ್ಯ ಜನನೀ ಶ್ರೀಮಹಾ ತ್ರಿಪುರಸುಂದರೀ ॥)\\
ಶ್ರೀವಿದ್ಯಾ ಕೋಶಾಂಬಾ ಶ್ರೀಪಾದುಕಾಂ ಪೂ।ತ।ನಮಃ॥\\
\as{೪ ಓಂ ಹ್ರೀಂ ಹಂಸಃ ಸೋಹಂ ಸ್ವಾಹಾ ॥}\\ ಪರಂಜ್ಯೋತಿಃಕೋಶಾಂಬಾ ಶ್ರೀಪಾದುಕಾಂ ಪೂ।ತ।ನಮಃ॥೧\\
\as{೪ ಓಂ ಹಂಸಃ ॥} ಪರಾನಿಷ್ಕಲಾ ಕೋಶಾಂಬಾ ಶ್ರೀಪಾದುಕಾಂ ಪೂ।ತ।ನಮಃ॥೨\\
\as{೪ ಹಂಸಃ ॥}ಅಜಪಾಕೋಶಾಂಬಾ ಶ್ರೀಪಾದುಕಾಂ ಪೂ।ತ।ನಮಃ॥೩\\
\as{೪ ಅಂ ಆಂ ಇಂ ಈಂ++++ಳಂ ಕ್ಷಂ ॥} ಮಾತೃಕಾಕೋಶಾಂಬಾ ಶ್ರೀಪಾದುಕಾಂ ಪೂ।ತ।ನಮಃ॥೪
\subsection{ಪಂಚಕಲ್ಪಲತಾಂಬಾಃ}
{\bfseries ೪ ೧೫ ॥}( ಶ್ರೀಮಹಾಕಲ್ಪಲತೇಶ್ವರೀ ಬೃಂದಮಂಡಿತಾಸನ ಸಂಸ್ಥಿತಾ~।\\
ಸರ್ವಸೌಭಾಗ್ಯ ಜನನೀ ಶ್ರೀಮಹಾ ತ್ರಿಪುರಸುಂದರೀ ॥)\\
ಶ್ರೀವಿದ್ಯಾ ಕಲ್ಪಲತಾಂಬಾ ಶ್ರೀಪಾದುಕಾಂ ಪೂ।ತ।ನಮಃ॥\\
\as{೪ ಹ್ರೀಂ ಕ್ಲೀಂ ಐಂ ಬ್ಲೂಂ ಸ್ತ್ರೀಂ ॥} ತ್ವರಿತಾ ಕಲ್ಪಲತಾಂಬಾ ಶ್ರೀಪಾದುಕಾಂ ಪೂ।ತ।ನಮಃ॥೧\\
\as{೪ ಓಂ ಹ್ರೀಂ ಹ್ರಾಂ ಹಸಕಲಹ್ರೀಂ ಓಂ ಸರಸ್ವತ್ಯೈ ನಮಃ ಹ್‌ಸ್ರೈಂ ॥}\\ಪಾರಿಜಾತೇಶ್ವರೀ ಕಲ್ಪಲತಾಂಬಾ ಶ್ರೀಪಾದುಕಾಂ ಪೂ।ತ।ನಮಃ॥೨\\
\as{೪ ಶ್ರೀಂ ಹ್ರೀಂ ಕ್ಲೀಂ ಐಂ ಕ್ಲೀಂ ಸೌಃ ॥}\\ ತ್ರಿಪುಟಾ ಕಲ್ಪಲತಾಂಬಾ ಶ್ರೀಪಾದುಕಾಂ ಪೂ।ತ।ನಮಃ॥೩\\
\as{೪ ದ್ರಾಂ ದ್ರೀಂ ಕ್ಲೀಂ ಬ್ಲೂಂ ಸಃ ॥}\\ ಪಂಚಬಾಣೇಶ್ವರೀ ಕಲ್ಪಲತಾಂಬಾ ಶ್ರೀಪಾದುಕಾಂ ಪೂ।ತ।ನಮಃ॥೪
\subsection{ಪಂಚಕಾಮದುಘಾಂಬಾಃ}
{\bfseries ೪ ೧೫ ॥}( ಶ್ರೀಮಹಾಕಾಮದುಘೇಶ್ವರೀ ಬೃಂದಮಂಡಿತಾಸನ ಸಂಸ್ಥಿತಾ~।\\
ಸರ್ವಸೌಭಾಗ್ಯ ಜನನೀ ಶ್ರೀಮಹಾ ತ್ರಿಪುರಸುಂದರೀ ॥)\\
ಶ್ರೀವಿದ್ಯಾ ಕಾಮದುಘಾಂಬಾ ಶ್ರೀಪಾದುಕಾಂ ಪೂ।ತ।ನಮಃ॥\\
\as{೪ ಓಂ ಹ್ರೀಂ ಹಂಸಃ ಜುಂ ಸಂಜೀವನಿ ಜೀವಂ ಪ್ರಾಣಗ್ರಂಥಿಸ್ಥಂ ಕುರು ಕುರು ಸ್ವಾಹಾ ॥} ಅಮೃತಪೀಠೇಶ್ವರೀ ಕಾಮದುಘಾಂಬಾ ಶ್ರೀಪಾದುಕಾಂ ಪೂ।ತ।ನಮಃ॥೧\\
\as{೪ ಐಂ ವದ ವದ ವಾಗ್ವಾದಿನಿ ಹ್‌ಸ್ರೈಂ ಕ್ಲೀಂ ಕ್ಲಿನ್ನೇ ಕ್ಲೇದಿನಿ ಮಹಾಕ್ಷೋಭಂ ಕುರು ಕುರು ಹ್‌ಸ್‌ರ್ಕ್ಲೀಂ ಸೌಃ ಓಂ ಮೋಕ್ಷಂ ಕುರು ಕುರು ಹ್‌ಸ್‌ರ್ಸೌಃ~।}\\ಸುಧಾಸೂಕಾಮದುಘಾಂಬಾ ಶ್ರೀಪಾದುಕಾಂ ಪೂ।ತ।ನಮಃ॥೨\\
\as{೪ ಐಂ ಬ್ಲೂಂ ಝ್ರೌಂ ಜುಂ ಸಃ ಅಮೃತೇ ಅಮೃತೋದ್ಭವೇ ಅಮೃತೇಶ್ವರಿ \\ಅಮೃತವರ್ಷಿಣಿ ಅಮೃತಂ ಸ್ರಾವಯ ಸ್ರಾವಯ ಸ್ವಾಹಾ ॥}\\ ಅಮೃತೇಶ್ವರೀ ಕಾಮದುಘಾಂಬಾ ಶ್ರೀಪಾದುಕಾಂ ಪೂ।ತ।ನಮಃ॥೩\\
\as{೪ ಓಂ ಹ್ರೀಂಶ್ರೀಂಕ್ಲೀಂ ನಮೋ ಭಗವತಿ ಮಾಹೇಶ್ವರಿ ಅನ್ನಪೂರ್ಣೇ ಸ್ವಾಹಾ ॥}\\ ಅನ್ನಪೂರ್ಣಾಕಾಮದುಘಾಂಬಾ ಶ್ರೀಪಾದುಕಾಂ ಪೂ।ತ।ನಮಃ॥೪
\subsection{ಪಂಚರತ್ನಾಂಬಾಃ}
{\bfseries ೪ ೧೫ ॥}( ಶ್ರೀಮಹಾರತ್ನೇಶ್ವರೀ ಬೃಂದಮಂಡಿತಾಸನ ಸಂಸ್ಥಿತಾ~।\\
ಸರ್ವಸೌಭಾಗ್ಯ ಜನನೀ ಶ್ರೀಮಹಾ ತ್ರಿಪುರಸುಂದರೀ ॥)\\
ಶ್ರೀವಿದ್ಯಾ ರತ್ನಾಂಬಾ ಶ್ರೀಪಾದುಕಾಂ ಪೂ।ತ।ನಮಃ॥\\
\as{೪ ಜ್‌ಝ್ರೀಂ ಮಹಾಚಂಡೇ ತೇಜಃಕರ್ಷಿಣಿ ಕಾಲಮಂಥಾನೇ ಹಃ ॥}\\ಸಿದ್ಧಲಕ್ಷ್ಮೀರತ್ನಾಂಬಾ ಶ್ರೀಪಾದುಕಾಂ ಪೂ।ತ।ನಮಃ॥೧\\
\as{೪ ಐಂಹ್ರೀಂಶ್ರೀಂ ಐಂಕ್ಲೀಂಸೌಃ ಓಂ ನಮೋ ಭಗವತಿ ರಾಜಮಾತಂಗೀಶ್ವರಿ ಸರ್ವಜನಮನೋಹರಿ ಸರ್ವಮುಖರಂಜನಿ ಕ್ಲೀಂಹ್ರೀಂಶ್ರೀಂ ಸರ್ವರಾಜವಶಂಕರಿ ಸರ್ವಸ್ತ್ರೀಪುರುಷ ವಶಂಕರಿ ಸರ್ವ ದುಷ್ಟಮೃಗವಶಂಕರಿ ಸರ್ವ ಸತ್ವವಶಂಕರಿ ಸರ್ವ ಲೋಕವಶಂಕರಿ ತ್ರೈಲೋಕ್ಯಂ ಮೇ ವಶಮಾನಯ ಸ್ವಾಹಾ ಸೌಃಕ್ಲೀಂಐಂ ಶ್ರೀಂಹ್ರೀಂಐಂ॥}ರಾಜಮಾತಂಗೀಶ್ವರೀರತ್ನಾಂಬಾ ಶ್ರೀಪಾದುಕಾಂ ಪೂ।ತ।ನಮಃ॥೨\\
\as{೪ ಶ್ರೀಂ ಹ್ರೀಂ ಶ್ರೀಂ ॥}ಭುವನೇಶ್ವರೀರತ್ನಾಂಬಾ ಶ್ರೀಪಾದುಕಾಂ ಪೂ।ತ।ನಮಃ॥೩\\
\as{೪ ಐಂ ಗ್ಲೌಂ ಐಂ ನಮೋ ಭಗವತಿ ವಾರ್ತಾಲಿ ವಾರ್ತಾಲಿ ವಾರಾಹಿ ವಾರಾಹಿ ವರಾಹಮುಖಿ ವರಾಹಮುಖಿ ಅಂಧೇ ಅಂಧಿನಿ ನಮಃ ರುಂಧೇ ರುಂಧಿನಿ ನಮಃ ಜಂಭೇ ಜಂಭಿನಿ ನಮಃ ಮೋಹೇ ಮೋಹಿನಿ ನಮಃ ಸ್ತಂಭೇ ಸ್ತಂಭಿನಿ ನಮಃ ಸರ್ವದುಷ್ಟಪ್ರದುಷ್ಟಾನಾಂ ಸರ್ವೇಷಾಂ ಸರ್ವವಾಕ್ಚಿತ್ತ ಚಕ್ಷುರ್ಮುಖಗತಿ ಜಿಹ್ವಾ  ಸ್ತಂಭನಂ ಕುರು ಕುರು ಶೀಘ್ರಂ ವಶ್ಯಂ ಐಂ ಗ್ಲೌಂ ಐಂ ಠಃಠಃಠಃಠಃ ಹುಂ ಫಟ್ ಸ್ವಾಹಾ ॥}ವಾರಾಹೀರತ್ನಾಂಬಾ ಶ್ರೀಪಾದುಕಾಂ ಪೂ।ತ।ನಮಃ॥೪
\section{ಷಡಾಧಾರಪೂಜಾ}
\as{೪ ತಾರೇ ತುತ್ತಾರೇ ತುರೇ ಸ್ವಾಹಾ }। ತಾರಾ ದೇವತಾಧಿಷ್ಠಿತ ಬೌದ್ಧ ದರ್ಶನ ಶ್ರೀಪಾ।ಪೂ।ತ।ನಮಃ॥\\
\as{೪ (ಗಾಯತ್ರೀ) }। ಬ್ರಹ್ಮ ದೇವತಾಧಿಷ್ಠಿತ ವೈದಿಕ ದರ್ಶನ ಶ್ರೀಪಾ।ಪೂ।ತ।ನಮಃ॥\\
\as{೪ (ಪಂಚಾಕ್ಷರೀ)} । ರುದ್ರ ದೇವತಾಧಿಷ್ಠಿತ ಶೈವ ದರ್ಶನ ಶ್ರೀಪಾ।ಪೂ।ತ।ನಮಃ॥\\
\as{೪ (ಆದಿತ್ಯಮಂತ್ರಃ)}। ಸೂರ್ಯ ದೇವತಾಧಿಷ್ಠಿತ ಸೌರ ದರ್ಶನ ಶ್ರೀಪಾ।ಪೂ।ತ।ನಮಃ॥\\
\as{೪ (ಅಷ್ಟಾಕ್ಷರೀ)}। ವಿಷ್ಣು ದೇವತಾಧಿಷ್ಠಿತ ವೈಷ್ಣವ ದರ್ಶನ ಶ್ರೀಪಾ।ಪೂ।ತ।ನಮಃ॥\\
\as{೪ ಶ್ರೀಂಹ್ರೀಂಶ್ರೀಂ }। ಭುವನೇಶ್ವರೀ ದೇವತಾಧಿಷ್ಠಿತ ಶಾಕ್ತ ದರ್ಶನ ಶ್ರೀಪಾ।ಪೂ।ತ।ನಮಃ॥
\section{ಷಡಾಧಾರಪೂಜಾ}
\as{೪ ಸಾಂ ಹಂಸಃ }। ಮೂಲಾಧಾರಾಧಿಷ್ಠಾನದೇವತಾ ಸಾಕಿನೀಸಹಿತ ಗಣನಾಥಸ್ವರೂಪಿಣೀ  ಮೂಲಾಧಾರದೇವೀ ಶ್ರೀಪಾದುಕಾಂ ಪೂ।ತ।ನಮಃ॥\\
\as{೪ ಕಾಂ ಸೋಹಂ }। ಸ್ವಾಧಿಷ್ಠಾನಾಧಿಷ್ಠಾನದೇವತಾ  ಕಾಕಿನೀಸಹಿತ  ಬ್ರಹ್ಮಸ್ವರೂಪಿಣೀ ಸ್ವಾಧಿಷ್ಠಾನದೇವೀ ಶ್ರೀಪಾದುಕಾಂ ಪೂ।ತ।ನಮಃ॥\\
\as{೪ ಲಾಂ ಹಂಸಸ್ಸೋಹಂ }। ಮಣಿಪೂರಾಧಿಷ್ಠಾನದೇವತಾ ಲಾಕಿನೀಸಹಿತ ವಿಷ್ಣುಸ್ವರೂಪಿಣೀ ಮಣಿಪೂರದೇವೀ ಶ್ರೀಪಾದುಕಾಂ ಪೂ।ತ।ನಮಃ॥\\
\as{೪ ರಾಂ ಹಂಸಶ್ಶಿವಸ್ಸೋಹಂ }। ಅನಾಹತಾಧಿಷ್ಠಾನದೇವತಾ ರಾಕಿಣೀಸಹಿತ ಸದಾಶಿವಸ್ವರೂಪಿಣೀ ಅನಾಹತದೇವೀ ಶ್ರೀಪಾದುಕಾಂ ಪೂ।ತ।ನಮಃ॥\\
\as{೪ ಡಾಂ ಸೋಹಂಹಂಸಶ್ಶಿವಃ }। ವಿಶುಶುದ್ಧ್ಯಧಿಷ್ಠಾನದೇವತಾ ಡಾಕಿನೀಸಹಿತ ಜೀವೇಶ್ವರಸ್ವರೂಪಿಣೀ ವಿಶುದ್ಧಿದೇವೀ ಶ್ರೀಪಾದುಕಾಂ ಪೂ।ತ।ನಮಃ॥\\
\as{೪ ಹಾಂ ಹಂಸಶ್ಶಿವಸ್ಸೋಹಂ ಸೋಹಂಹಂಸಶ್ಶಿವಃ }। ಆಜ್ಞಾಧಿಷ್ಠಾನದೇವತಾ ಹಾಕಿನೀಸಹಿತ ಪರಮಾತ್ಮಸ್ವರೂಪಿಣೀ ಆಜ್ಞಾದೇವೀ ಶ್ರೀಪಾದುಕಾಂ ಪೂ।ತ।ನಮಃ॥
\section{ಆಮ್ನಾಯಸಮಷ್ಟಿಪೂಜಾ}
\as{೪ ಹ್‌ಸ್‌ರೈಂ ಹ್‌ಸ್‌ಕ್ಲ್ರೀಂ ಹ್‌ಸ್‌ರ್ಸೌಃ }॥ ಪೂರ್ವಾಮ್ನಾಯ ಸಮಯವಿದ್ಯೇಶ್ವರೀ ಉನ್ಮೋದಿನೀ ದೇವ್ಯಂಬಾ ಶ್ರೀಪಾ।ಪೂ।ತ।ನಮಃ॥\\
\as{೪ ೧೫॥} ಪೂರ್ವಾಮ್ನಾಯ ಸಮಷ್ಟಿರೂಪಿಣೀ ಮಹಾತ್ರಿಪುರಸುಂದರೀ ಶ್ರೀಪಾ।ಪೂ।ತ।ನಮಃ॥\\
\as{೪ ಓಂ ಹ್ರೀಂ ಐಂ ಕ್ಲಿನ್ನೇ ಕ್ಲಿನ್ನಮದದ್ರವೇ ಕುಲೇ ಹ್ಸೌಃ }॥ ದಕ್ಷಿಣಾಮ್ನಾಯ ಸಮಯವಿದ್ಯೇಶ್ವರೀ ಭೋಗಿನೀ ದೇವ್ಯಂಬಾ ಶ್ರೀಪಾ।ಪೂ।ತ।ನಮಃ॥\\
\as{೪ ೧೫॥} ದಕ್ಷಿಣಾಮ್ನಾಯ ಸಮಷ್ಟಿರೂಪಿಣೀ ಮಹಾತ್ರಿಪುರಸುಂದರೀ ಶ್ರೀಪಾ।ಪೂ।ತ।ನಮಃ॥\\
\as{೪ ಹ್ಸ್ರೈಂ ಹ್ಸ್ರೀಂ ಹ್ಸ್ರೌಃ ಹ್‌ಸ್‌ಖ್‌ಫ್ರೇಂ ಭಗವತ್ಯಂಬೇ ಹಸಕ್ಷಮಲವರಯೂಂ ಹ್‌ಸ್‌ಖ್‌ಫ್ರೇಂ ಅಘೋರಮುಖಿ ಛ್ರಾಂ ಛ್ರೀಂ ಕಿಣಿ ಕಿಣಿ ವಿಚ್ಚೇ ಹ್‌ಸ್ರೌಃ ಹ್‌ಸ್‌ಖ್‌ಫ್ರೇಂ ಹ್‌ಸ್ರೌಃ }॥ ಪಶ್ಚಿಮಾಮ್ನಾಯ ಸಮಯವಿದ್ಯೇಶ್ವರೀ ಕುಬ್ಜಿಕಾ ದೇವ್ಯಂಬಾ ಶ್ರೀಪಾ।ಪೂ।ತ।ನಮಃ॥\\
\as{೪ ೧೫॥} ಪಶ್ಚಿಮಾಮ್ನಾಯ ಸಮಷ್ಟಿರೂಪಿಣೀ ಮಹಾತ್ರಿಪುರಸುಂದರೀ ಶ್ರೀಪಾ।ಪೂ।ತ।ನಮಃ॥\\
\as{೪ ಹ್‌ಸ್‌ಖ್‌ಫ್ರೇಂ  ಮಹಾಚಂಡಯೋಗೀಶ್ವರಿ ಕಾಳಿಕೇ ಫಟ್ }॥ ಉತ್ತರಾಮ್ನಾಯ ಸಮಯವಿದ್ಯೇಶ್ವರೀ ಕಾಳಿಕಾ ದೇವ್ಯಂಬಾ ಶ್ರೀಪಾ।ಪೂ।ತ।ನಮಃ॥\\
\as{೪ ೧೫॥} ಉತ್ತರಾಮ್ನಾಯ ಸಮಷ್ಟಿರೂಪಿಣೀ ಮಹಾತ್ರಿಪುರಸುಂದರೀ ಶ್ರೀಪಾ।ಪೂ।ತ।ನಮಃ॥\\
\as{೪ ಮಖಪರಯಘಚ್ ಮಹಿಚನಡಯಙ್ ಗಂಶಫರ್ }॥ ಊರ್ಧ್ವಾಮ್ನಾಯ ಸಮಯವಿದ್ಯೇಶ್ವರೀ ಚೈತನ್ಯಭೈರವ್ಯಂಬಾ ಶ್ರೀಪಾ।ಪೂ।ತ।ನಮಃ॥\\
\as{೪ ೧೬॥} ಊರ್ಧ್ವಾಮ್ನಾಯ ಸಮಷ್ಟಿರೂಪಿಣೀ ಮಹಾತ್ರಿಪುರಸುಂದರೀ ಶ್ರೀಪಾ।ಪೂ।ತ।ನಮಃ॥\\
\as{೪ ಭಗವತಿ ವಿಚ್ಚೇ ಮಹಾಮಾಯೇ ಮಾತಂಗಿನಿ ಬ್ಲೂಂ ಅನುತ್ತರವಾಗ್ವಾದಿನಿ ಹ್‌ಸ್‌ಖ್‌ಫ್ರೇಂ ಹ್‌ಸ್‌ಖ್‌ಫ್ರೇಂ ಹ್‌ಸ್ರೌಃ }॥ ಅನುತ್ತರಶಾಂಕರ್ಯಂಬಾ ಶ್ರೀಪಾ।ಪೂ।ತ।ನಮಃ॥\\
\as{೪ ೧೬॥} ಅನುತ್ತರಾಮ್ನಾಯ ಸಮಷ್ಟಿರೂಪಿಣೀ ಮಹಾತ್ರಿಪುರಸುಂದರೀ ಶ್ರೀಪಾ।ಪೂ।ತ।ನಮಃ॥

ಪುನಃ ಷಡಂಗದೇವತಾಃ ತರ್ಪಯಿತ್ವಾ\\
\as{೪ ಐಂ ಕಏಈಲಹ್ರೀಂ ।} ಸರ್ವಜ್ಞಾಯೈ ಹೃದಯಾಯ ನಮಃ ।\\ ಹೃದಯಶಕ್ತಿ ಶ್ರೀಪಾದುಕಾಂ ಪೂ । ತ । ನಮಃ ॥\\
\as{೪ ಕ್ಲೀಂ ಹಸಕಹಲಹ್ರೀಂ ।} ನಿತ್ಯತೃಪ್ತಾಯೈ ಶಿರಸೇ ಸ್ವಾಹಾ । \\ಶಿರಃಶಕ್ತಿ ಶ್ರೀಪಾದುಕಾಂ ಪೂ । ತ । ನಮಃ ॥\\
\as{೪ ಸೌಃ ಸಕಲಹ್ರೀಂ ।} ಅನಾದಿಬೋಧಿನ್ಯೈ ಶಿಖಾಯೈ ವಷಟ್ ।\\ ಶಿಖಾಶಕ್ತಿ ಶ್ರೀಪಾದುಕಾಂ ಪೂ । ತ । ನಮಃ ॥\\
\as{೪ ಐಂ ಕಏಈಲಹ್ರೀಂ ।} ಸ್ವತಂತ್ರಾಯೈ ಕವಚಾಯ ಹುಂ । \\ಕವಚಶಕ್ತಿ ಶ್ರೀಪಾದುಕಾಂ ಪೂ । ತ । ನಮಃ ॥\\
\as{೪ ಕ್ಲೀಂ ಹಸಕಹಲಹ್ರೀಂ ।} ನಿತ್ಯಾಲುಪ್ತಾಯೈ ನೇತ್ರತ್ರಯಾಯ ವೌಷಟ್ । \\ನೇತ್ರಶಕ್ತಿ ಶ್ರೀಪಾದುಕಾಂ ಪೂ । ತ । ನಮಃ ॥\\
\as{೪ ಸೌಃ ಸಕಲಹ್ರೀಂ ।} ಅಸ್ತ್ರಾಯ ಫಟ್ ।\\ ಅಸ್ತ್ರಶಕ್ತಿ ಶ್ರೀಪಾದುಕಾಂ ಪೂ । ತ । ನಮಃ ॥\\
ಬಿಂದೌ ಮೂಲೇನ ತ್ರಿಃ ಸಂತರ್ಪ್ಯ, ದಶಮುದ್ರಾಃ ಪ್ರದರ್ಶ್ಯ, ಉತ್ತರನ್ಯಾಸಂ ಕುರ್ಯಾತ್ ॥
\section{ಲಲಿತಾ ದ್ವಾದಶ ನಾಮ ಸ್ತೋತ್ರಮ್~॥}
\addcontentsline{toc}{section}{ಲಲಿತಾ ದ್ವಾದಶ ನಾಮ ಸ್ತೋತ್ರಮ್}
ಶೃಣು ದ್ವಾದಶ ನಾಮಾನಿ ತಸ್ಯಾ ದೇವ್ಯಾ ಘಟೋದ್ಭವ ~।\\
ಯೇಷಾಮಾಕರ್ಣನಾಮಾತ್ರಾತ್ ಪ್ರಸನ್ನಾ ಸಾ ಭವಿಷ್ಯತಿ~॥

ಪಂಚಮೀ ದಂಡನಾಥಾ ಚ ಸಂಕೇತಾ ಸಮಯೇಶ್ವರೀ~।\\
ತಥಾ ಸಮಯಸಂಕೇತಾ ವಾರಾಹೀ ಪೋತ್ರಿಣೀ ಶಿವಾ~॥

ವಾರ್ತಾಲೀ ಚ ಮಹಾಸೇನಾ ಆಜ್ಞಾಚಕ್ರೇಶ್ವರೀ ತಥಾ~।\\
ಅರಿಘ್ನೀ ಚೇತಿ ಸಂಪ್ರೋಕ್ತಮ್ ನಾಮ ದ್ವಾದಶಕಂ ಮುನೇ~॥

ನಾಮದ್ವಾದಶಕಾಭಿಖ್ಯ ವಜ್ರಪಂಜರ ಮಧ್ಯಗಃ~।\\
ಸಂಕಟೇ ದುಃಖಮಾಪ್ನೋತಿ ನ ಕದಾಚನ ಮಾನವಃ~॥
\authorline{ಇತಿ ಶ್ರೀ ಲಲಿತಾ ದ್ವಾದಶ ನಾಮ ಸ್ತೋತ್ರಂ}
\section{ಲಲಿತಾ ಷೋಡಶನಾಮ ಸ್ತೋತ್ರಮ್~॥}
\addcontentsline{toc}{section}{ಲಲಿತಾ ಷೋಡಶನಾಮ ಸ್ತೋತ್ರಮ್}
ಸಂಗೀತಯೋಗಿನೀ ಶ್ಯಾಮಾ ಶ್ಯಾಮಲಾ ಮಂತ್ರನಾಯಿಕಾ ।\\
ಮಂತ್ರಿಣೀ ಸಚಿವೇಶಾನೀ ಪ್ರಧಾನೇಶೀ ಶುಕಪ್ರಿಯಾ~॥

ವೀಣಾವತೀ ವೈಣಿಕೀ ಚ ಮುದ್ರಿಣೀ ಪ್ರಿಯಕಪ್ರಿಯಾ~।\\
ನೀಪಪ್ರಿಯಾ ಕದಂಬೇಶೀ ಕದಂಬವನವಾಸಿನೀ~॥

ಸದಾಮದಾ ಚ ನಾಮಾನಿ ಷೋಡಶೈತಾನಿ ಕುಂಭಜ~।\\
ಏತೈರ್ಯಃ ಸಚಿವೇಶಾನೀಂ ಸಕೃತ್ ಸ್ತೌತಿ ಶರೀರವಾನ್ ।\\
ತಸ್ಯ ತ್ರೈಲೋಕ್ಯಮಖಿಲಂ ವಶೇ ತಿಷ್ಠತ್ಯಸಂಶಯಃ~॥
\authorline{ಇತಿ ಶ್ರೀ ಲಲಿತಾ ಷೋಡಶನಾಮ ಸ್ತೋತ್ರಮ್}
\section{ಲಲಿತಾ ಪಂಚವಿಂಶತಿನಾಮ ಸ್ತೋತ್ರಮ್~॥}
\addcontentsline{toc}{section}{ಲಲಿತಾ ಪಂಚವಿಂಶತಿನಾಮ ಸ್ತೋತ್ರಮ್}
ಸಿಂಹಾಸನೇಶೀ ಲಲಿತಾ ಮಹಾರಾಜ್ಞೀ ವರಾಂಕುಶಾ~।\\
ಚಾಪಿನೀ ತ್ರಿಪುರಾ ಚೈವಮಹಾತ್ರಿಪುರಸುಂದರೀ~॥

ಸುಂದರೀ ಚಕ್ರನಾಥಾ ಚ ಸಮ್ರಾಜ್ಞೀ ಚಕ್ರಿಣೀ ತಥಾ~।\\
ಚಕ್ರೇಶ್ವರೀ ಮಹಾದೇವೀ ಕಾಮೇಶೀ ಪರಮೇಶ್ವರೀ ~॥

ಕಾಮರಾಜಪ್ರಿಯಾ ಕಾಮಕೋಟಿಕಾ ಚಕ್ರವರ್ತಿನೀ~।\\
ಮಹಾವಿದ್ಯಾ ಶಿವಾನಂಗವಲ್ಲಭಾ ಸರ್ವಪಾಟಲಾ~॥

ಕುಲನಾಥಾಮ್ನಾಯನಾಥಾ ಸರ್ವಾಮ್ನಾಯ ನಿವಾಸಿನೀ~।\\
ಶೃಂಗಾರ ನಾಯಿಕಾ ಚೇತಿ ಪಂಚವಿಂಶತಿ ನಾಮಭಿಃ~॥

ಸ್ತುವಂತಿ ಯೇ ಮಹಾಭಾಗಾಂ ಲಲಿತಾಂ ಪರಮೇಶ್ವರೀಮ್~।\\
ತೇ ಪ್ರಾಪ್ನುವಂತಿ ಸೌಭಾಗ್ಯಮಷ್ಟೌ ಸಿದ್ಧೀರ್ಮಹದ್ಯಶಃ~॥
\authorline{ಇತಿ ಶ್ರೀ ಲಲಿತಾ ಪಂಚವಿಂಶತಿನಾಮ ಸ್ತೋತ್ರಮ್~॥}
\section{ಸೌಭಾಗ್ಯಾಷ್ಟೋತ್ತರಶತನಾಮಸ್ತೋತ್ರಮ್}
\addcontentsline{toc}{section}{ಸೌಭಾಗ್ಯಾಷ್ಟೋತ್ತರಶತನಾಮಸ್ತೋತ್ರಮ್ }

\as{ಓಂ ಐಂಹ್ರೀಂ ಶ್ರೀಂ}\\
ಕಾಮೇಶ್ವರೀ ಕಾಮಶಕ್ತಿಃ ಕಾಮಸೌಭಾಗ್ಯದಾಯಿನೀ।\\
ಕಾಮರೂಪಾ ಕಾಮಕಲಾ ಕಾಮಿನೀ ಕಮಲಾಸನಾ ॥೧॥

ಕಮಲಾ ಕಲ್ಪನಾಹೀನಾ ಕಮನೀಯಾ ಕಲಾವತೀ~।\\
ಕಮಲಾ ಭಾರತೀಸೇವ್ಯಾ ಕಲ್ಪಿತಾಶೇಷಸಂಸೃತಿಃ ॥೨॥

ಅನುತ್ತರಾಽನಘಾಽನಂತಾಽದ್ಭುತರೂಪಾಽನಲೋದ್ಭವಾ~।\\
ಅತಿಲೋಕಚರಿತ್ರಾಽತಿಸುಂದರ್ಯತಿಶುಭಪ್ರದಾ ॥೩॥

ಅಘಹಂತ್ರ್ಯತಿವಿಸ್ತಾರಾಽರ್ಚನತುಷ್ಟಾಽಮಿತಪ್ರಭಾ~।\\
ಏಕರೂಪೈಕವೀರೈಕನಾಥೈಕಾಂತಾಽರ್ಚನಪ್ರಿಯಾ ॥೪॥

ಏಕೈಕಭಾವತುಷ್ಟೈಕರಸೈಕಾಂತಜನಪ್ರಿಯಾ~।\\
ಏಧಮಾನಪ್ರಭಾವೈಧದ್ಭಕ್ತಪಾತಕನಾಶಿನೀ ॥೫॥

ಏಲಾಮೋದಮುಖೈನೋಽದ್ರಿಶಕ್ರಾಯುಧಸಮಸ್ಥಿತಿಃ~।\\
ಈಹಾಶೂನ್ಯೇಪ್ಸಿತೇಶಾದಿಸೇವ್ಯೇಶಾನವರಾಂಗನಾ ॥೬॥

ಈಶ್ವರಾಽಽಜ್ಞಾಪಿಕೇಕಾರಭಾವ್ಯೇಪ್ಸಿತಫಲಪ್ರದಾ~।\\
ಈಶಾನೇತಿಹರೇಕ್ಷೇಷದರುಣಾಕ್ಷೀಶ್ವರೇಶ್ವರೀ ॥೭॥

ಲಲಿತಾ ಲಲನಾರೂಪಾ ಲಯಹೀನಾ ಲಸತ್ತನುಃ~।\\
ಲಯಸರ್ವಾ ಲಯಕ್ಷೋಣಿರ್ಲಯಕರ್ಣೀ ಲಯಾತ್ಮಿಕಾ ॥೮॥

ಲಘಿಮಾ ಲಘುಮಧ್ಯಾಽಽಢ್ಯಾ ಲಲಮಾನಾ ಲಘುದ್ರುತಾ~।\\
ಹಯಾಽಽರೂಢಾ ಹತಾಽಮಿತ್ರಾ ಹರಕಾಂತಾ ಹರಿಸ್ತುತಾ ॥೯॥

ಹಯಗ್ರೀವೇಷ್ಟದಾ ಹಾಲಾಪ್ರಿಯಾ ಹರ್ಷಸಮುದ್ಧತಾ~।\\
ಹರ್ಷಣಾ ಹಲ್ಲಕಾಭಾಂಗೀ ಹಸ್ತ್ಯಂತೈಶ್ವರ್ಯದಾಯಿನೀ ॥೧೦॥

ಹಲಹಸ್ತಾಽರ್ಚಿತಪದಾ ಹವಿರ್ದಾನಪ್ರಸಾದಿನೀ~।\\
ರಾಮರಾಮಾಽರ್ಚಿತಾ ರಾಜ್ಞೀ ರಮ್ಯಾ ರವಮಯೀ ರತಿಃ ॥೧೧॥

ರಕ್ಷಿಣೀರಮಣೀರಾಕಾ ರಮಣೀಮಂಡಲಪ್ರಿಯಾ~।\\
ರಕ್ಷಿತಾಽಖಿಲಲೋಕೇಶಾ ರಕ್ಷೋಗಣನಿಷೂದಿನೀ ॥೧೨॥

ಅಂಬಾಂತಕಾರಿಣ್ಯಂಭೋಜಪ್ರಿಯಾಂತಕಭಯಂಕರೀ~।\\
ಅಂಬುರೂಪಾಂಬುಜಕರಾಂಬುಜಜಾತವರಪ್ರದಾ ॥೧೩॥

ಅಂತಃಪೂಜಾಪ್ರಿಯಾಂತಃಸ್ವರೂಪಿಣ್ಯಂತರ್ವಚೋಮಯೀ~।\\
ಅಂತಕಾರಾತಿವಾಮಾಂಕಸ್ಥಿತಾಂತಃಸುಖರೂಪಿಣೀ ॥೧೪॥

ಸರ್ವಜ್ಞಾ ಸರ್ವಗಾ ಸಾರಾ ಸಮಾ ಸಮಸುಖಾ ಸತೀ~।\\
ಸಂತತಿಃ ಸಂತತಾ ಸೋಮಾ ಸರ್ವಾ ಸಾಂಖ್ಯಾ ಸನಾತನೀ ॥೧೫॥\as{ಶ್ರೀಂಹ್ರೀಂಐಂ}
\authorline{॥ಇತಿ ಸೌಭಾಗ್ಯಾಷ್ಟೋತ್ತರಶತನಾಮಸ್ತೋತ್ರಂ ॥}
\section{ಶ್ರೀಲಲಿತಾತ್ರಿಶತೀ}
\addcontentsline{toc}{section}{ಶ್ರೀಲಲಿತಾತ್ರಿಶತೀ}
ಅಸ್ಯ ಶ್ರೀಲಲಿತಾತ್ರಿಶತೀಸ್ತೋತ್ರ ಮಹಾಮಂತ್ರಸ್ಯ ಹಯಗ್ರೀವ ಭಗವಾನ್ ಋಷಿಃ । ಅನುಷ್ಟುಪ್ ಛಂದಃ । ಶ್ರೀ ಲಲಿತಾಮಹಾತ್ರಿಪುರಸುಂದರೀ ದೇವತಾ~। ಐಂ ಕಏಈಲಹ್ರೀಂ ಇತಿ ಬೀಜಂ । ಕ್ಲೀಂ ಹಸಕಹಲಹ್ರೀಂ ಇತಿ ಶಕ್ತಿಃ । ಸೌಃ ಸಕಲಹ್ರೀಂ ಇತಿ ಕೀಲಕಮ್ । ಪೂಜಾಯಾಂ ವಿನಿಯೋಗಃ ॥ಕೂಟತ್ರಯೇಣ ಕರಾಂಗನ್ಯಾಸಂ ವಿಧಾಯ ಯಥಾಶಕ್ತಿ ಜಪಿತ್ವಾ ತದ್ದಶಾಂಶೇನ ತರ್ಪಯಿತ್ವಾ\\
{\bfseries ಅತಿಮಧುರಚಾಪಹಸ್ತಾಮಪರಿಮಿತಾಮೋದಬಾಣಸೌಭಾಗ್ಯಾಂ~।\\
ಅರುಣಾಮತಿಶಯಕರುಣಾಮಭಿನವಕುಲಸುಂದರೀಂ ವಂದೇ ॥\\
ಹಯಗ್ರೀವ ಉವಾಚ\\
ಕಕಾರರೂಪಾ ಕಲ್ಯಾಣೀ ಕಲ್ಯಾಣಗುಣಶಾಲಿನೀ~।\\
ಕಲ್ಯಾಣಶೈಲನಿಲಯಾ ಕಮನೀಯಾ ಕಲಾವತೀ ॥೧॥

ಕಮಲಾಕ್ಷೀ ಕಲ್ಮಷಘ್ನೀ ಕರುಣಾಮೃತ ಸಾಗರಾ~।\\
ಕದಂಬಕಾನನಾವಾಸಾ ಕದಂಬ ಕುಸುಮಪ್ರಿಯಾ ॥೨॥

ಕಂದರ್ಪವಿದ್ಯಾ ಕಂದರ್ಪ ಜನಕಾಪಾಂಗ ವೀಕ್ಷಣಾ~।\\
ಕರ್ಪೂರವೀಟೀ ಸೌರಭ್ಯ ಕಲ್ಲೋಲಿತ ಕಕುಪ್ತಟಾ ॥೩॥

ಕಲಿದೋಷಹರಾ ಕಂಜಲೋಚನಾ ಕಮ್ರವಿಗ್ರಹಾ~।\\
ಕರ್ಮಾದಿ ಸಾಕ್ಷಿಣೀ ಕಾರಯಿತ್ರೀ ಕರ್ಮಫಲಪ್ರದಾ ॥೪॥

ಏಕಾರರೂಪಾ ಚೈಕಾಕ್ಷರ್ಯೇಕಾನೇಕಾಕ್ಷರಾಕೃತಿಃ~।\\
ಏತತ್ತದಿತ್ಯನಿರ್ದೇಶ್ಯಾ ಚೈಕಾನಂದ ಚಿದಾಕೃತಿಃ ॥೫॥

ಏವಮಿತ್ಯಾಗಮಾಬೋಧ್ಯಾ ಚೈಕಭಕ್ತಿ ಮದರ್ಚಿತಾ~।\\
ಏಕಾಗ್ರಚಿತ್ತ ನಿರ್ಧ್ಯಾತಾ ಚೈಷಣಾ ರಹಿತಾದೃತಾ ॥೬॥

ಏಲಾಸುಗಂಧಿಚಿಕುರಾ ಚೈನಃ ಕೂಟ ವಿನಾಶಿನೀ~।\\
ಏಕಭೋಗಾ ಚೈಕರಸಾ ಚೈಕೈಶ್ವರ್ಯ ಪ್ರದಾಯಿನೀ ॥೭॥

ಏಕಾತಪತ್ರ ಸಾಮ್ರಾಜ್ಯ ಪ್ರದಾ ಚೈಕಾಂತಪೂಜಿತಾ~।\\
ಏಧಮಾನಪ್ರಭಾ ಚೈಜದನೇಕಜಗದೀಶ್ವರೀ ॥೮॥

ಏಕವೀರಾದಿ ಸಂಸೇವ್ಯಾ ಚೈಕಪ್ರಾಭವ ಶಾಲಿನೀ~।\\
ಈಕಾರರೂಪಾ ಚೇಶಿತ್ರೀ ಚೇಪ್ಸಿತಾರ್ಥ ಪ್ರದಾಯಿನೀ ॥೯॥

ಈದೃಗಿತ್ಯ ವಿನಿರ್ದೇಶ್ಯಾ ಚೇಶ್ವರತ್ವ ವಿಧಾಯಿನೀ~।\\
ಈಶಾನಾದಿ ಬ್ರಹ್ಮಮಯೀ ಚೇಶಿತ್ವಾದ್ಯಷ್ಟ ಸಿದ್ಧಿದಾ ॥೧೦॥

ಈಕ್ಷಿತ್ರೀಕ್ಷಣ ಸೃಷ್ಟಾಂಡ ಕೋಟಿರೀಶ್ವರ ವಲ್ಲಭಾ~।\\
ಈಡಿತಾ ಚೇಶ್ವರಾರ್ಧಾಂಗ ಶರೀರೇಶಾಧಿ ದೇವತಾ ॥೧೧॥

ಈಶ್ವರ ಪ್ರೇರಣಕರೀ ಚೇಶತಾಂಡವ ಸಾಕ್ಷಿಣೀ~।\\
ಈಶ್ವರೋತ್ಸಂಗ ನಿಲಯಾ ಚೇತಿಬಾಧಾ ವಿನಾಶಿನೀ ॥೧೨॥

ಈಹಾವಿರಹಿತಾ ಚೇಶ ಶಕ್ತಿ ರೀಷತ್ ಸ್ಮಿತಾನನಾ~।\\
ಲಕಾರರೂಪಾ ಲಲಿತಾ ಲಕ್ಷ್ಮೀ ವಾಣೀ ನಿಷೇವಿತಾ ॥೧೩॥

ಲಾಕಿನೀ ಲಲನಾರೂಪಾ ಲಸದ್ದಾಡಿಮ ಪಾಟಲಾ~।\\
ಲಲಂತಿಕಾಲಸತ್ಫಾಲಾ ಲಲಾಟ ನಯನಾರ್ಚಿತಾ ॥೧೪॥

ಲಕ್ಷಣೋಜ್ಜ್ವಲ ದಿವ್ಯಾಂಗೀ ಲಕ್ಷಕೋಟ್ಯಂಡ ನಾಯಿಕಾ~।\\
ಲಕ್ಷ್ಯಾರ್ಥಾ ಲಕ್ಷಣಾಗಮ್ಯಾ ಲಬ್ಧಕಾಮಾ ಲತಾತನುಃ ॥೧೫॥

ಲಲಾಮರಾಜದಲಿಕಾ ಲಂಬಿಮುಕ್ತಾಲತಾಂಚಿತಾ~।\\
ಲಂಬೋದರ ಪ್ರಸೂರ್ಲಭ್ಯಾ ಲಜ್ಜಾಢ್ಯಾ ಲಯವರ್ಜಿತಾ ॥೧೬॥

ಹ್ರೀಂಕಾರ ರೂಪಾ ಹ್ರೀಂಕಾರ ನಿಲಯಾ ಹ್ರೀಂಪದಪ್ರಿಯಾ~।\\
ಹ್ರೀಂಕಾರ ಬೀಜಾ ಹ್ರೀಂಕಾರಮಂತ್ರಾ ಹ್ರೀಂಕಾರಲಕ್ಷಣಾ ॥೧೭॥

ಹ್ರೀಂಕಾರಜಪ ಸುಪ್ರೀತಾ ಹ್ರೀಂಮತೀ ಹ್ರೀಂವಿಭೂಷಣಾ~।\\
ಹ್ರೀಂಶೀಲಾಹ್ರೀಂಪದಾರಾಧ್ಯಾ ಹ್ರೀಂಗರ್ಭಾಹ್ರೀಂಪದಾಭಿಧಾ॥೧೮॥

ಹ್ರೀಂಕಾರವಾಚ್ಯಾ ಹ್ರೀಂಕಾರ ಪೂಜ್ಯಾ ಹ್ರೀಂಕಾರ ಪೀಠಿಕಾ~।\\
ಹ್ರೀಂಕಾರವೇದ್ಯಾ ಹ್ರೀಂಕಾರಚಿಂತ್ಯಾ ಹ್ರೀಂ ಹ್ರೀಂಶರೀರಿಣೀ ॥೧೯॥

ಹಕಾರರೂಪಾ ಹಲಧೃಕ್ಪೂಜಿತಾ ಹರಿಣೇಕ್ಷಣಾ~।\\
ಹರಪ್ರಿಯಾ ಹರಾರಾಧ್ಯಾ ಹರಿಬ್ರಹ್ಮೇಂದ್ರ ವಂದಿತಾ ॥೨೦॥

ಹಯಾರೂಢಾ ಸೇವಿತಾಂಘ್ರಿರ್ಹಯಮೇಧ ಸಮರ್ಚಿತಾ।\\
ಹರ್ಯಕ್ಷವಾಹನಾ ಹಂಸವಾಹನಾ ಹತದಾನವಾ ॥೨೧॥

ಹತ್ಯಾದಿಪಾಪಶಮನೀ ಹರಿದಶ್ವಾದಿ ಸೇವಿತಾ~।\\
ಹಸ್ತಿಕುಂಭೋತ್ತುಂಗ ಕುಚಾ ಹಸ್ತಿಕೃತ್ತಿ ಪ್ರಿಯಾಂಗನಾ ॥೨೨॥

ಹರಿದ್ರಾಕುಂಕುಮಾ ದಿಗ್ಧಾ ಹರ್ಯಶ್ವಾದ್ಯಮರಾರ್ಚಿತಾ~।\\
ಹರಿಕೇಶಸಖೀ ಹಾದಿವಿದ್ಯಾ ಹಾಲಾಮದಾಲಸಾ ॥೨೩॥

ಸಕಾರರೂಪಾ ಸರ್ವಜ್ಞಾ ಸರ್ವೇಶೀ ಸರ್ವಮಂಗಲಾ~।\\
ಸರ್ವಕರ್ತ್ರೀ ಸರ್ವಭರ್ತ್ರೀ ಸರ್ವಹಂತ್ರೀ ಸನಾತನಾ ॥೨೪॥

ಸರ್ವಾನವದ್ಯಾ ಸರ್ವಾಂಗ ಸುಂದರೀ ಸರ್ವಸಾಕ್ಷಿಣೀ~।\\
ಸರ್ವಾತ್ಮಿಕಾ ಸರ್ವಸೌಖ್ಯ ದಾತ್ರೀ ಸರ್ವವಿಮೋಹಿನೀ ॥೨೫॥

ಸರ್ವಾಧಾರಾ ಸರ್ವಗತಾ ಸರ್ವಾವಗುಣವರ್ಜಿತಾ~।\\
ಸರ್ವಾರುಣಾ ಸರ್ವಮಾತಾ ಸರ್ವಭೂಷಣ ಭೂಷಿತಾ ॥೨೬॥

ಕಕಾರಾರ್ಥಾ ಕಾಲಹಂತ್ರೀ ಕಾಮೇಶೀ ಕಾಮಿತಾರ್ಥದಾ~।\\
ಕಾಮಸಂಜೀವನೀ ಕಲ್ಯಾ ಕಠಿನಸ್ತನ ಮಂಡಲಾ ॥೨೭॥

ಕರಭೋರುಃ ಕಲಾನಾಥ ಮುಖೀ ಕಚಜಿತಾಂಬುದಾ~।\\
ಕಟಾಕ್ಷಸ್ಯಂದಿ ಕರುಣಾ ಕಪಾಲಿ ಪ್ರಾಣ ನಾಯಿಕಾ ॥೨೮॥

ಕಾರುಣ್ಯ ವಿಗ್ರಹಾ ಕಾಂತಾ ಕಾಂತಿಧೂತ ಜಪಾವಲಿಃ~।\\
ಕಲಾಲಾಪಾ ಕಂಬುಕಂಠೀ ಕರನಿರ್ಜಿತ ಪಲ್ಲವಾ ॥೨೯॥

ಕಲ್ಪವಲ್ಲೀ ಸಮಭುಜಾ ಕಸ್ತೂರೀ ತಿಲಕಾಂಚಿತಾ~।\\
ಹಕಾರಾರ್ಥಾ ಹಂಸಗತಿರ್ಹಾಟಕಾಭರಣೋಜ್ಜ್ವಲಾ ॥೩೦॥

ಹಾರಹಾರಿ ಕುಚಾಭೋಗಾ ಹಾಕಿನೀ ಹಲ್ಯವರ್ಜಿತಾ~।\\
ಹರಿತ್ಪತಿ ಸಮಾರಾಧ್ಯಾ ಹಠಾತ್ಕಾರ ಹತಾಸುರಾ ॥೩೧॥

ಹರ್ಷಪ್ರದಾ ಹವಿರ್ಭೋಕ್ತ್ರೀ ಹಾರ್ದ ಸಂತಮಸಾಪಹಾ~।\\
ಹಲ್ಲೀಸಲಾಸ್ಯ ಸಂತುಷ್ಟಾ ಹಂಸಮಂತ್ರಾರ್ಥ ರೂಪಿಣೀ ॥೩೨॥

ಹಾನೋಪಾದಾನ ನಿರ್ಮುಕ್ತಾ ಹರ್ಷಿಣೀ ಹರಿಸೋದರೀ~।\\
ಹಾಹಾಹೂಹೂ ಮುಖ ಸ್ತುತ್ಯಾ ಹಾನಿ ವೃದ್ಧಿ ವಿವರ್ಜಿತಾ ॥೩೩॥

ಹಯ್ಯಂಗವೀನ ಹೃದಯಾ ಹರಿಗೋಪಾರುಣಾಂಶುಕಾ~।\\
ಲಕಾರಾಖ್ಯಾ ಲತಾಪೂಜ್ಯಾ ಲಯಸ್ಥಿತ್ಯುದ್ಭವೇಶ್ವರೀ ॥೩೪॥

ಲಾಸ್ಯ ದರ್ಶನ ಸಂತುಷ್ಟಾ ಲಾಭಾಲಾಭ ವಿವರ್ಜಿತಾ~।\\
ಲಂಘ್ಯೇತರಾಜ್ಞಾ ಲಾವಣ್ಯ ಶಾಲಿನೀ ಲಘು ಸಿದ್ಧಿದಾ ॥೩೫॥

ಲಾಕ್ಷಾರಸ ಸವರ್ಣಾಭಾ ಲಕ್ಷ್ಮಣಾಗ್ರಜ ಪೂಜಿತಾ~।\\
ಲಭ್ಯೇತರಾ ಲಬ್ಧ ಭಕ್ತಿ ಸುಲಭಾ ಲಾಂಗಲಾಯುಧಾ ॥೩೬॥

ಲಗ್ನಚಾಮರ ಹಸ್ತ ಶ್ರೀಶಾರದಾ ಪರಿವೀಜಿತಾ~।\\
ಲಜ್ಜಾಪದ ಸಮಾರಾಧ್ಯಾ ಲಂಪಟಾ ಲಕುಲೇಶ್ವರೀ ॥೩೭॥

ಲಬ್ಧಮಾನಾ ಲಬ್ಧರಸಾ ಲಬ್ಧ ಸಂಪತ್ಸಮುನ್ನತಿಃ~।\\
ಹ್ರೀಂಕಾರಿಣೀ ಹ್ರೀಂಕಾರಾದ್ಯಾ ಹ್ರೀಂಮಧ್ಯಾ ಹ್ರೀಂಶಿಖಾಮಣಿಃ ॥೩೮॥

ಹ್ರೀಂಕಾರಕುಂಡಾಗ್ನಿ ಶಿಖಾ ಹ್ರೀಂಕಾರ ಶಶಿಚಂದ್ರಿಕಾ~।\\
ಹ್ರೀಂಕಾರ ಭಾಸ್ಕರರುಚಿರ್ಹ್ರೀಂಕಾರಾಂಭೋದ ಚಂಚಲಾ ॥೩೯॥

ಹ್ರೀಂಕಾರ ಕಂದಾಂಕುರಿಕಾ ಹ್ರೀಂಕಾರೈಕ ಪರಾಯಣಾ~।\\
ಹ್ರೀಂಕಾರ ದೀರ್ಘಿಕಾಹಂಸೀ ಹ್ರೀಂಕಾರೋದ್ಯಾನ ಕೇಕಿನೀ ॥೪೦॥

ಹ್ರೀಂಕಾರಾರಣ್ಯ ಹರಿಣೀ ಹ್ರೀಂಕಾರಾವಾಲ ವಲ್ಲರೀ~।\\
ಹ್ರೀಂಕಾರ ಪಂಜರಶುಕೀ ಹ್ರೀಂಕಾರಾಂಗಣ ದೀಪಿಕಾ ॥೪೧॥

ಹ್ರೀಂಕಾರ ಕಂದರಾ ಸಿಂಹೀ ಹ್ರೀಂಕಾರಾಂಭೋಜ ಭೃಂಗಿಕಾ~।\\
ಹ್ರೀಂಕಾರ ಸುಮನೋ ಮಾಧ್ವೀ ಹ್ರೀಂಕಾರ ತರುಮಂಜರೀ ॥೪೨॥

ಸಕಾರಾಖ್ಯಾ ಸಮರಸಾ ಸಕಲಾಗಮ ಸಂಸ್ತುತಾ~।\\
ಸರ್ವವೇದಾಂತ ತಾತ್ಪರ್ಯಭೂಮಿಃ ಸದಸದಾಶ್ರಯಾ ॥೪೩॥

ಸಕಲಾ ಸಚ್ಚಿದಾನಂದಾ ಸಾಧ್ಯಾ ಸದ್ಗತಿದಾಯಿನೀ~।\\
ಸನಕಾದಿಮುನಿಧ್ಯೇಯಾ ಸದಾಶಿವ ಕುಟುಂಬಿನೀ ॥೪೪॥

ಸಕಾಲಾಧಿಷ್ಠಾನ ರೂಪಾ ಸತ್ಯರೂಪಾ ಸಮಾಕೃತಿಃ~।\\
ಸರ್ವಪ್ರಪಂಚ ನಿರ್ಮಾತ್ರೀ ಸಮಾನಾಧಿಕ ವರ್ಜಿತಾ ॥೪೫॥

ಸರ್ವೋತ್ತುಂಗಾ ಸಂಗಹೀನಾ ಸಗುಣಾ ಸಕಲೇಶ್ವರೀ~।\\
ಕಕಾರಿಣೀ ಕಾವ್ಯಲೋಲಾ ಕಾಮೇಶ್ವರ ಮನೋಹರಾ ॥೪೬॥

ಕಾಮೇಶ್ವರಪ್ರಾಣನಾಡೀ ಕಾಮೇಶೋತ್ಸಂಗ ವಾಸಿನೀ~।\\
ಕಾಮೇಶ್ವರಾಲಿಂಗಿತಾಂಗೀ ಕಾಮೇಶ್ವರ ಸುಖಪ್ರದಾ ॥೪೭॥

ಕಾಮೇಶ್ವರ ಪ್ರಣಯಿನೀ ಕಾಮೇಶ್ವರ ವಿಲಾಸಿನೀ~।\\
ಕಾಮೇಶ್ವರ ತಪಃ ಸಿದ್ಧಿಃ ಕಾಮೇಶ್ವರ ಮನಃ ಪ್ರಿಯಾ ॥೪೮॥

ಕಾಮೇಶ್ವರ ಪ್ರಾಣನಾಥಾ ಕಾಮೇಶ್ವರ ವಿಮೋಹಿನೀ~।\\
ಕಾಮೇಶ್ವರ ಬ್ರಹ್ಮವಿದ್ಯಾ ಕಾಮೇಶ್ವರ ಗೃಹೇಶ್ವರೀ ॥೪೯॥

ಕಾಮೇಶ್ವರಾಹ್ಲಾದಕರೀ ಕಾಮೇಶ್ವರ ಮಹೇಶ್ವರೀ~।\\
ಕಾಮೇಶ್ವರೀ ಕಾಮಕೋಟಿ ನಿಲಯಾ ಕಾಂಕ್ಷಿತಾರ್ಥದಾ ॥೫೦॥

ಲಕಾರಿಣೀ ಲಬ್ಧರೂಪಾ ಲಬ್ಧಧೀರ್ಲಬ್ಧ ವಾಂಛಿತಾ~।\\
ಲಬ್ಧಪಾಪ ಮನೋದೂರಾ ಲಬ್ಧಾಹಂಕಾರ ದುರ್ಗಮಾ ॥೫೧॥

ಲಬ್ಧಶಕ್ತಿರ್ಲಬ್ಧ ದೇಹಾ ಲಬ್ಧೈಶ್ವರ್ಯ ಸಮುನ್ನತಿಃ~।\\
ಲಬ್ಧವೃದ್ಧಿರ್ಲಬ್ಧಲೀಲಾ ಲಬ್ಧಯೌವನ ಶಾಲಿನೀ ॥೫೨॥

ಲಬ್ಧಾತಿಶಯ ಸರ್ವಾಂಗ ಸೌಂದರ್ಯಾ ಲಬ್ಧ ವಿಭ್ರಮಾ~।\\
ಲಬ್ಧರಾಗಾ ಲಬ್ಧಪತಿರ್ಲಬ್ಧ ನಾನಾಗಮಸ್ಥಿತಿಃ ॥೫೩॥

ಲಬ್ಧ ಭೋಗಾ ಲಬ್ಧ ಸುಖಾ ಲಬ್ಧ ಹರ್ಷಾಭಿಪೂರಿತಾ~।\\
ಹ್ರೀಂಕಾರ ಮೂರ್ತಿರ್ಹ್ರೀಂಕಾರ ಸೌಧಶೃಂಗ ಕಪೋತಿಕಾ ॥೫೪॥

ಹ್ರೀಂಕಾರ ದುಗ್ಧಾಬ್ಧಿ ಸುಧಾ ಹ್ರೀಂಕಾರ ಕಮಲೇಂದಿರಾ~।\\
ಹ್ರೀಂಕಾರಮಣಿ ದೀಪಾರ್ಚಿರ್ಹ್ರೀಂಕಾರ ತರುಶಾರಿಕಾ ॥೫೫॥

ಹ್ರೀಂಕಾರ ಪೇಟಕ ಮಣಿರ್ಹ್ರೀಂಕಾರಾದರ್ಶ ಬಿಂಬಿತಾ~।\\
ಹ್ರೀಂಕಾರ ಕೋಶಾಸಿಲತಾ ಹ್ರೀಂಕಾರಾಸ್ಥಾನ ನರ್ತಕೀ ॥೫೬॥

ಹ್ರೀಂಕಾರ ಶುಕ್ತಿಕಾ ಮುಕ್ತಾಮಣಿರ್ಹ್ರೀಂಕಾರ ಬೋಧಿತಾ~।\\
ಹ್ರೀಂಕಾರಮಯ ಸೌವರ್ಣಸ್ತಂಭ ವಿದ್ರುಮ ಪುತ್ರಿಕಾ ॥೫೭॥

ಹ್ರೀಂಕಾರ ವೇದೋಪನಿಷದ್ ಹ್ರೀಂಕಾರಾಧ್ವರ ದಕ್ಷಿಣಾ~।\\
ಹ್ರೀಂಕಾರ ನಂದನಾರಾಮ ನವಕಲ್ಪಕ ವಲ್ಲರೀ ॥೫೮॥

ಹ್ರೀಂಕಾರ ಹಿಮವದ್ಗಂಗಾ ಹ್ರೀಂಕಾರಾರ್ಣವ ಕೌಸ್ತುಭಾ~।\\
ಹ್ರೀಂಕಾರ ಮಂತ್ರ ಸರ್ವಸ್ವಾ ಹ್ರೀಂಕಾರಪರ ಸೌಖ್ಯದಾ ॥೫೯॥}
\authorline {॥ಇತಿ ಶ್ರೀಲಲಿತಾತ್ರಿಶತೀಸ್ತೋತ್ರಂ ಸಂಪೂರ್ಣಂ ॥}
\section{ಸೌಭಾಗ್ಯವಿದ್ಯಾ ಕವಚಂ}
\addcontentsline{toc}{section}{ಸೌಭಾಗ್ಯವಿದ್ಯಾ ಕವಚಂ}
ಅಸ್ಯ ಶ್ರೀ ಮಹಾತ್ರಿಪುರಸುಂದರೀ ಮಂತ್ರವರ್ಣಾತ್ಮಕ ಕವಚ ಮಹಾಮಂತ್ರಸ್ಯ ದಕ್ಷಿಣಾಮೂರ್ತಿರ್ಋಷಿಃ। ಅನುಷ್ಟುಪ್ ಛಂದಃ। ಶ್ರೀಮಹಾತ್ರಿಪುರಸುಂದರೀ ದೇವತಾ~। ಐಂ ಬೀಜಂ~। ಸೌಃ ಶಕ್ತಿಃ~। ಕ್ಲೀಂ ಕೀಲಕಂ~। ಮಮ ಶರೀರರಕ್ಷಣಾರ್ಥೇ ಜಪೇ ವಿನಿಯೋಗಃ~॥\\
\dhyana{ಬಾಲಾರ್ಕಮಂಡಲಾಭಾಸಾಂ ಚತುರ್ಬಾಹುಂ ತ್ರಿಲೋಚನಾಂ~।\\
ಪಾಶಾಂಕುಶ ಧನುರ್ಬಾಣಾನ್ ಧಾರಯಂತೀಂ ಶಿವಾಂ ಭಜೇ ॥}\\
\as{(ಓಂಐಂಹ್ರೀಂಶ್ರೀಂ)}
ಕಕಾರಃ ಪಾತು ಮೇ ಶೀರ್ಷಂ ಏಕಾರಃ ಪಾತು ಫಾಲಕಂ।\\
ಈಕಾರಃ ಪಾತು ಮೇ ವಕ್ತ್ರಂ ಲಕಾರಃ ಪಾತು ಕರ್ಣಕಂ॥೧॥

ಹ್ರೀಂಕಾರಃ ಪಾತು ಹೃದಯಂ ವಾಗ್ಭವಶ್ಚ ಸದಾವತು।\\
ಹಕಾರಃ ಪಾತು ಜಠರಂ ಸಕಾರೋ ನಾಭಿದೇಶಕಂ॥೨॥

ಕಕಾರೋವ್ಯಾದ್ವಸ್ತಿಭಾಗಂ ಹಕಾರಃ ಪಾತು ಲಿಂಗಕಂ।\\
ಲಕಾರೋ ಜಾನುನೀ ಪಾತು ಹ್ರೀಂಕಾರೋ ಜಂಘಯುಗ್ಮಕಂ॥೩॥

ಕಾಮರಾಜಃ ಸದಾ ಪಾತು ಜಠರಾದಿ ಪ್ರದೇಶಕಂ।\\
ಸಕಾರಃ ಪಾತು ಮೇ ಜಂಘೇ ಕಕಾರಃ ಪಾತು ಪೃಷ್ಠಕಂ॥೪॥

ಲಕಾರೋವ್ಯಾನ್ನಿತಂಬಂ ಮೇ ಹ್ರೀಂಕಾರಃ ಪಾತು ಮೂಲಕಂ~।\\
ಶಕ್ತಿಬೀಜಃ ಸದಾ ಪಾತು ಮೂಲಾಧಾರಾದಿ ದೇಶಕಂ॥೫॥

ತ್ರಿಪುರಾ ದೇವತಾ ಪಾತು ತ್ರಿಪುರೇಶೀ ಚ ಸರ್ವದಾ।\\
ತ್ರಿಪುರಾ ಸುಂದರೀ ಪಾತು ತ್ರಿಪುರಾಶ್ರೀ ಸ್ತಥಾವತು॥೬॥

ತ್ರಿಪುರಾ ಮಾಲಿನೀ ಪಾತು ತ್ರಿಪುರಾ ಸಿದ್ಧಿದಾ ವತು।\\
ತ್ರಿಪುರಾಂಬಾ ತಥಾ ಪಾತು ಪಾತು ತ್ರಿಪುರಭೈರವೀ॥೭॥

ಅಣಿಮಾದ್ಯಾ ಸ್ತಥಾ ಪಾಂತು ಬ್ರಾಹ್ಮ್ಯಾದ್ಯಾಃ ಪಾಂತು ಮಾಂ ಸದಾ।\\
ದಶಮುದ್ರಾಸ್ತಥಾ ಪಾಂತು ಕಾಮಾಕರ್ಷಣ ಪೂರ್ವಕಾಃ॥೮॥

ಪಾಂತು ಮಾಂ ಷೋಡಶದಲೇ ಯಂತ್ರೇನಂಗ ಕುಮಾರಿಕಾಃ।\\
ಪಾಂತು ಮಾಂ ಪೃಷ್ಠಪತ್ರೇ ತು ಸರ್ವಸಂಕ್ಷೋಭಣಾದಿಕಾಃ॥೯॥

ಪಾಂತು ಮಾಂ ದಶಕೋಣೇ ತು ಸರ್ವಸಿದ್ಧಿ ಪ್ರದಾಯಿಕಾಃ।\\
ಪಾಂತು ಮಾಂ ಬಾಹ್ಯ ದಿಕ್ಕೋಣೇ ಮಧ್ಯ ದಿಕ್ಕೋಣಕೇ ತಥಾ॥೧೦॥

ಸರ್ವಜ್ಞಾ ದ್ಯಾಸ್ತಥಾ ಪಾಂತು ಸರ್ವಾಭೀಷ್ಟ ಪ್ರದಾಯಿಕಾಃ।\\
ವಶಿನ್ಯಾದ್ಯಾಸ್ತಥಾ ಪಾಂತು ವಸು ಪತ್ರಸ್ಯ ದೇವತಾಃ॥೧೧॥

ತ್ರಿಕೋಣ ಸ್ಯಾಂತ ರಾಲೇ ತು ಪಾಂತು ಮಾಮಾಯುಧಾನಿ ಚ।\\
ಕಾಮೇಶ್ವರ್ಯಾದಿಕಾಃ ಪಾಂತು ತ್ರಿಕೋಣೇ ಕೋಣಸಂಸ್ಥಿತಾಃ॥೧೨॥

ಬಿಂದುಚಕ್ರೇ ತಥಾ ಪಾತು ಮಹಾತ್ರಿಪುರಸುಂದರೀ।\as{(ಶ್ರೀಂಹ್ರೀಂಐಂ)}\\
ಇತೀದಂ ಕವಚಂ ದೇವಿ ಕವಚಂ ಮಂತ್ರಸೂಚಕಂ॥೧೩॥

ಯಸ್ಮೈ ಕಸ್ಮೈ ನ ದಾತವ್ಯಂ ನ ಪ್ರಕಾಶ್ಯಂ ಕಥಂಚನ।\\
ಯಸ್ತ್ರಿಸಂಧ್ಯಂ ಪಠೇದ್ದೇವಿ ಲಕ್ಷ್ಮೀಸ್ತಸ್ಯ ಪ್ರಜಾಯತೇ॥೧೪॥

ಅಷ್ಟಮ್ಯಾಂ ಚ ಚತುರ್ದಶ್ಯಾಂ ಯಃ ಪಠೇತ್ ಪ್ರಯತಃ ಸದಾ।\\
ಪ್ರಸನ್ನಾ ಸುಂದರೀ ತಸ್ಯ ಸರ್ವಸಿದ್ಧಿಪ್ರದಾಯಿನೀ॥೧೫॥
\authorline{॥ಇತಿ ಶ್ರೀ ರುದ್ರಯಾಮಲೇ ತಂತ್ರೇ ತ್ರಿಪುರಾ ಹೃದಯೇ ಕವಚರಹಸ್ಯಂ ಸಂಪೂರ್ಣಂ ॥}

ಸುಗಂಧಂ ಗುಗ್ಗುಲಯುತಂ ನವಸರ್ಪಿಃಸಮರ್ಪಿತಂ ।\\
ಧೂಪಂ ಗೃಹಾಣ ದೇವೇಶ ಸರ್ವಮಂಗಳದೋ ಭವ ॥\\
ಶ್ರೀಖಂಡಗುಗ್ಗುಲಸಿತಾಭ್ರಮುಖೈಃ ಸುಗಂಧೈಃ \\ಸಂಧೂಪಯಾಮಿ ಜಗದೀಶ್ವರಿ ಮೇ ಪ್ರಸೀದ~।\\
ಗಂಧೇನ ತೇನ ಕಲಿತಂ ಪುನರುಕ್ತಮಸ್ತು\\ ಲಿಪ್ತಂ ತ್ವದಂಗಮಿಹ ಕುಂಕುಮಕರ್ದಮೇನ ॥\\
\as{ಯತ್ಪುರುಷಂ ವ್ಯದಧುಃ ಪಾದಾ ಉಚ್ಯೇತೇ॥\\
ಕರ್ದಮೇನ++++++++ಪದ್ಮಮಾಲಿನೀಮ್ ॥} ಧೂಪಃ ॥

ದಿವ್ಯವರ್ತಿಸಮಾಯುಕ್ತಂ ಅಜ್ಞಾನತಿಮಿರಾಪಂ ।\\
ಅತಿಭಕ್ತ್ಯಾ ಮಯಾ ದತ್ತಂ ದೀಪೋಽಯಂ ಪ್ರತಿಗೃಹ್ಯತಾಂ ॥\\
ದೀಪಂ ಸಮುಜ್ವಲಮಹಂ ಪ್ರದದೇ ತಥಾಪಿ\\ ನಾನೇನ ಗಚ್ಛತಿ ಮದೀಯ ಹೃದಂಧಕಾರಃ~।\\
ವಿಶ್ವೇಶಿ ತಾವಕಕಟಾಕ್ಷವಿನಿರ್ಗತಾಸ್ತಾ \\ಆಪಃ ಸೃಜಂತು ಮಮ ಹೃತ್ಕಮಲೇ ಪ್ರಕಾಶಂ ॥\\
\as{ಬ್ರಾಹ್ಮಣೋಽಸ್ಯ +++++ಶೂದ್ರೋ ಅಜಾಯತ॥\\
ಆಪಃ ಸೃಜನ್ತು ++++++++ಮೇ ಕುಲೇ॥}ದೀಪಃ ॥

ಶಾಲ್ಯನ್ನಂ ಸಘೃತಂ ಕ್ಷೀರಂ ಹೇಮಪಾತ್ರೇ ಪ್ರಪೂರಿತಂ ।\\
ಪರಮಾನ್ನಂ ಮಯಾ ದತ್ತಂ ಗೃಹಾಣ ಪರಮೇಶ್ವರ ॥\\
ಸ್ವಾದ್ವನ್ನಭಕ್ಷ್ಯಪರಮಾನ್ನಪಯೋಘೃತಾನಿ \\ಖರ್ಜೂರಚೂತಕದಲೀಂ ಚ ನಿವೇದಯಾಮಿ~।\\
ಭುಕ್ತ್ವಾ ಪ್ರಸನ್ನಹೃದಯಾ ದಯಯಾ ಸುದೃಷ್ಟಿ\-\\ಮಾರ್ದ್ರಾಂ ಪುರಸ್ಥಿತಸುತೇ ಮಯಿ ಪಾತಯಾಶು ॥\\
\as{ಚಂದ್ರಮಾ ಮನಸೋ+++++  ಪ್ರಾಣಾದ್ವಾಯುರಜಾಯತ॥\\
ಆರ್ದ್ರಾಂ ಪುಷ್ಕರಿಣೀಂ ++++ ಮ ಆವಹ॥} ನೈವೇದ್ಯಮ್ ॥

ನಾರಿಕೇಳ ಫಲಂ ದಿವ್ಯಂ ಸುಸ್ವಾದೂದಕಸಂಯುತಂ ।\\
ಕದಲೀಪಕ್ವಸಹತಂ ಈಶವರ ಪ್ರತಿಗೃಹ್ಯತಾಂ ॥ ಫಲಮ್ ॥

ಏಲಾಶೀರಲವಂಗಾದಿ ಮೃಗನಾಭಿವಿಮಿಶ್ರಿತಂ ।\\
ಕರ್ಪೂರಚಂದನಯುತಂ ಮಧ್ಯೇ ಸ್ವೀಕುರು ಶಂಕರ ॥ ತಾಂಬೂಲಮ್ ॥

ನಾನಾಗಂಧಸಂಯುಕ್ತಂ ಕರ್ಪೂರಾದಿಸಮನ್ವಿತಂ ।\\
ಹಸ್ತಪ್ರಕ್ಷಾಳನಾರ್ಥೇ ಗೃಹಾಣ ಪರಮೇಶ್ವರ    ॥ ಹಸ್ತಪ್ರಕ್ಷಾಳನಮ್ ॥

ಸುಗಂಧಜಲಮಿದಂ ದತ್ತಂ ನಾನಾಪುಷ್ಪಸಮನ್ವಿತಂ ।\\
ಪಾದಪ್ರಕ್ಷಾಳನಾರ್ಥೇ ಸ್ವೀಕುರುಷ್ವ ದಯಾನಿಧೇ  ॥ ಪಾದಪ್ರಕ್ಷಾಳನಮ್ ॥

ಇದಂ ಶುದ್ಧಜಲಂ ದಿವ್ಯಂ ಕರ್ಪೂರಚಂದನಮಿಶ್ರಿತಂ ।\\
ಕರೋದ್ವರ್ತನಾರ್ಥೇ ಪ್ರೀತ್ಯಾ ಸ್ವೀಕುರು ಶಂಕರ ॥ ಕರೋದ್ವರ್ತನಮ್ ॥

ಚಂದನಾಗರುಕಸ್ತೂರೀ ಕೇಸರ್ಯಾದಿವಿಮಿಶ್ರಿತಂ ।\\
ಪುನರಾಚಮನೀಯಾರ್ಥೇ ಗೃಹಾಣ ಪರಮೇಶ್ವರ ॥ ಪುನರಾಚನಮ್ ॥

ನಾರಿಕೇಳ ಕದಲ್ಯಾದಿ ದ್ರಾಕ್ಷಾ ದಾಡಿಮಸಂಯುತಂ ।\\
ಖರ್ಜೂರ ನಾರಂಗಯುತಂ ಫಲೌಘಂ ಪುರತಃ ಸ್ಥಿತಮ್ ।\\ 
ಉಪಾಹಾರಂ ಪ್ರದಾಸ್ಯಾಮಿ  ಶಂಕರ ಪ್ರತಿಗೃಹ್ಯತಾಂ ॥ ಉಪಾಹಾರಃ ॥

ಭಕ್ತಜನಪ್ರಿಯೋ ದೇವ ಭಕ್ತೇಷ್ಟಫಲದಾಯಕ ।\\
ನಾನಾಫಲಮಿದಂ ದೇವ ಶಂಕರ ಪ್ರತಿಗೃಹ್ಯತಾಂ ॥ ಪೂರ್ಣಫಲಮ್ ॥

ಪೂಗೀಫಲಸಮಾಯುಕ್ತಂ ನಾಗವಲ್ಲೀ ದಲೈರ್ಯುತಂ~।\\
ಕರ್ಪೂರಚೂರ್ಣಸಂಯುಕ್ತಂ ತಾಂಬೂಲಂ ಪ್ರತಿಗೃಹ್ಯತಾಂ ॥ತಾಂಬೂಲಮ್॥

ಕೋಟಿಸೂರ್ಯಪ್ರತೀಕಾಶ ಚಂದ್ರಸೂರ್ಯಾಗ್ನಿಲೋಚನ ।\\
ನೀರಾಜಯಾಮಿ ದೀಪಾತ್ಮನ್ ಅಂತರ್ಧ್ವಾಂತಂ ನಿಷೂದಯ ॥ \\
ಅಜ್ಞಾನಘೋರತಿಮಿರಾಪಗಮಾಯ ಭಕ್ತ್ಯಾ \\ನೀರಾಜಯಾಮಿ ಜನನೀಂ ಸುಹಿರಣ್ಯಪಾತ್ರೇ~।\\
ನಿಕ್ಷಿಪ್ಯ ತೂಲಮಯವರ್ತಿಮಹಂ ಘೃತೇನ \\ಸಾರ್ದ್ರಾಂ ಯಥೇಪ್ಸಿತಸಿತಾಭ್ರಮಪಿ ಪ್ರದೀಪೈಃ ॥\\
\as{ನಾಭ್ಯಾ +++++ಲೋಕಾನ್ ಅಕಲ್ಪಯನ್॥\\
ಆರ್ದ್ರಾಂ ಯಃ ಕರಿಣೀಂ+++++ಮ ಆವಹ॥}ನೀರಾಜನಮ್ ॥

ಚಂಪಕೈಃ ಬಿಲ್ವಮಂದಾರ ಬಕುಳಾದಿ ಸುಮೈಸ್ತಥಾ ।\\
ಪುಷ್ಪಾಂಜಲಿಂ ಪ್ರಯಚ್ಛಾಮಿ ದೇವದೇವ ಮಹೇಶ್ವರ ॥\\
ಪುನ್ನಾಗಪೂಗಕರವೀರತಮಾಲಪುಷ್ಪೈ\-\\ರಚ್ಛಿದ್ರ ನವ್ಯ ತುಲಸೀದಲ ಬಿಲ್ವಪತ್ರೈಃ~।\\
ಸ್ವರ್ಣಾಕ್ಷತೈಃ ಸಹ ದದಾಮಿ ಚ ಮಂತ್ರಪುಷ್ಪಂ\\ ರಾಜಾಧಿರಾಜವಿನುತೇ ಮಯಿ ಸಂಪ್ರಸೀದ ॥
\section{ಭಾವನೋಪನಿಷತ್}
\addcontentsline{toc}{section}{ಭಾವನೋಪನಿಷತ್}
ಓಂ ಭದ್ರಂ ಕರ್ಣೇಭಿಃ ಇತಿ ಶಾಂತಿಃ ॥
ಓಂ ಶ್ರೀಗುರುಃ ಸರ್ವಕಾರಣಭೂತಾ ಶಕ್ತಿಃ~। ತೇನ ನವರಂಧ್ರರೂಪೋ ದೇಹಃ~। ನವಚಕ್ರರೂಪಂ ಶ್ರೀಚಕ್ರಂ~। ವಾರಾಹೀ ಪಿತೃರೂಪಾ~। ಕುರುಕುಲ್ಲಾ ಬಲಿದೇವತಾ ಮಾತಾ~। ಪುರುಷಾರ್ಥಾಃ ಸಾಗರಾಃ~। ದೇಹೋ ನವರತ್ನದ್ವೀಪಃ~। ತ್ವಗಾದಿಸಪ್ತಧಾತುರೋಮ ಸಂಯುಕ್ತಃ। ಸಂಕಲ್ಪಾಃ ಕಲ್ಪತರವಸ್ತೇಜಃ ಕಲ್ಪಕೋ\-ದ್ಯಾನಂ~। ರಸನಯಾ ಭಾವ್ಯಮಾನಾ ಮಧುರಾಮ್ಲತಿಕ್ತಕಟುಕಷಾಯಲವಣರಸಾಃ ಷಡೃತವಃ~। ಜ್ಞಾನಮರ್ಘ್ಯಂ। ಜ್ಞೇಯಂ ಹವಿಃ~। ಜ್ಞಾತಾ ಹೋತಾ~। ಜ್ಞಾತೃಜ್ಞಾನಜ್ಞೇಯಾನಾಮಭೇದಭಾವನಂ ಶ್ರೀಚಕ್ರಪೂಜನಂ~। ನಿಯತಿಃ ಶೃಂಗಾರಾದಯೋ ರಸಾ ಅಣಿಮಾದಯಃ~। ಕಾಮಕ್ರೋಧ ಲೋಭಮೋಹ ಮದ ಮಾತ್ಸರ್ಯ ಪುಣ್ಯ ಪಾಪಮಯಾ ಬ್ರಾಹ್ಮ್ಯಾದ್ಯಷ್ಟಶಕ್ತಯಃ~। ಆಧಾರನವಕಂ ಮುದ್ರಾಶಕ್ತಯಃ~। ಪೃಥಿವ್ಯಪ್ತೇಜೋವಾಯ್ವಾಕಾಶ ಶ್ರೋತ್ರತ್ವಕ್ಚಕ್ಷುರ್ಜಿಹ್ವಾ ಘ್ರಾಣ ವಾಕ್ಪಾಣಿ ಪಾದ ಪಾಯೂಪಸ್ಥಾನಿ ಮನೋವಿಕಾರಾಃ ಕಾಮಾಕರ್ಷಣ್ಯಾದಿ ಷೋಡಶ ಶಕ್ತಯಃ~। ವಚನಾದಾನಗಮನವಿಸರ್ಗಾನಂದ ಹಾನೋಪಾದಾನೋಪೇಕ್ಷಾಖ್ಯ ಬುದ್ಧಯೋನಂಗಕುಸುಮಾದ್ಯಷ್ಟೌ~। ಅಲಂಬುಸಾ ಕುಹೂರ್ವಿಶ್ವೋದರಾ ವಾರಣಾ ಹಸ್ತಿಜಿಹ್ವಾ ಯಶೋವತೀ ಪಯಸ್ವಿನೀ ಗಾಂಧಾರೀ ಪೂಷಾ ಶಂಖಿನೀ ಸರಸ್ವತೀ ಇಡಾ ಪಿಂಗಲಾ ಸುಷುಮ್ನಾ ಚೇತಿ ಚತುರ್ದಶ ನಾಡ್ಯಃ ಸರ್ವಸಂಕ್ಷೋಭಿಣ್ಯಾದಿ ಚತುರ್ದಶ ಶಕ್ತಯಃ~। ಪ್ರಾಣಾಪಾನ ವ್ಯಾನೋದಾನ ಸಮಾನ ನಾಗ ಕೂರ್ಮ ಕೃಕರ ದೇವದತ್ತ ಧನಂಜಯಾ ಇತಿ ದಶ ವಾಯವಃ ಸರ್ವಸಿದ್ಧಿಪ್ರದಾದಿ ಬಹಿರ್ದಶಾರಗಾ ದೇವತಾಃ~। ಏತದ್ವಾಯುಸಂಸರ್ಗ ಕೋಪಾಧಿ ಭೇಧೇನ ರೇಚಕಃ ಪಾಚಕಃ ಶೋಷಕೋ ದಾಹಕಃ ಪ್ಲಾವಕ ಇತಿ ಪ್ರಾಣಮುಖ್ಯಶ್ರೀತ್ವೇನ ಪಂಚಧಾ ಜಠರಾಗ್ನಿರ್ಭವತಿ~। ಕ್ಷಾರಕ ಉದ್ಗಾರಕಃ ಕ್ಷೋಭಕೋ ಜೃಂಭಕೋ ಮೋಹಕ ಇತಿ ನಾಗಪ್ರಾಧಾನ್ಯೇನ ಪಂಚವಿಧಾಸ್ತೇ ಮನುಷ್ಯಾಣಾಂ ದೇಹಗಾ ಭಕ್ಷ್ಯಭೋಜ್ಯ ಚೋಷ್ಯ ಲೇಹ್ಯ ಪೇಯಾತ್ಮಕ ಪಂಚವಿಧಮನ್ನಂ ಪಾಚಯಂತಿ। ಏತಾ ದಶ ವಹ್ನಿಕಲಾಃ ಸರ್ವಜ್ಞಾದ್ಯಾ ಅಂತರ್ದಶಾರಗಾ ದೇವತಾಃ~। ಶೀತೋಷ್ಣಸುಖದುಃಖೇಚ್ಛಾಃ ಸತ್ತ್ವಂ ರಜಸ್ತಮೋ ವಶಿನ್ಯಾದಿಶಕ್ತಯೋಽಷ್ಟೌ~। ಶಬ್ದಾದಿ ತನ್ಮಾತ್ರಾಃ ಪಂಚಪುಷ್ಪಬಾಣಾಃ। ಮನ ಇಕ್ಷುಧನುಃ~। ರಾಗಃ ಪಾಶಃ~। ದ್ವೇಷೋಂಕುಶಃ~। ಅವ್ಯಕ್ತಮಹದಹಂಕಾರಾಃ ಕಾಮೇಶ್ವರೀವಜ್ರೇಶ್ವರೀ ಭಗಮಾಲಿನ್ಯೋಂತಸ್ತ್ರಿಕೋಣಗಾ ದೇವತಾಃ~। ನಿರುಪಾಧಿಕ ಸಂವಿದೇವ ಕಾಮೇಶ್ವರಃ। ಸದಾನಂದಪೂರ್ಣ ಸ್ವಾತ್ಮೈವ ಪರದೇವತಾ ಲಲಿತಾ~। ಲೌಹಿತ್ಯಮೇತಸ್ಯ ಸರ್ವಸ್ಯ ವಿಮರ್ಶಃ~। ಅನನ್ಯಚಿತ್ತತ್ವೇನ ಚ ಸಿದ್ಧಿಃ~। ಭಾವನಾಯಾಃ ಕ್ರಿಯಾ ಉಪಚಾರಃ~। ಅಹಂತ್ವಮಸ್ತಿನಾಸ್ತಿಕರ್ತವ್ಯಮಕರ್ತವ್ಯ ಮುಪಾಸಿತವ್ಯಮಿತಿ ವಿಕಲ್ಪಾನಾ ಮಾತ್ಮನಿ ವಿಲಾಪನಂ ಹೋಮಃ~। ಭಾವನಾವಿಷಯಾಣಾ ಮಭೇದಭಾವನಾ ತರ್ಪಣಂ~। ಪಂಚದಶತಿಥಿರೂಪೇಣ ಕಾಲಸ್ಯ ಪರಿಣಾಮಾವಲೋಕನಂ ಪಂಚದಶನಿತ್ಯಾಃ~। ಏವಂ ಮುಹೂರ್ತತ್ರಿತಯಂ ಮುಹೂರ್ತದ್ವಿತಯಂ ಮುಹೂರ್ತಮಾತ್ರಂ ವಾ ಭಾವನಾಪರೋ ಜೀವನ್ಮುಕ್ತೋ ಭವತಿ~। ಸ ಏವ ಶಿವಯೋಗೀತಿ ಗದ್ಯತೇ~। ಕಾದಿಮತೇನಾಂತಶ್ಚಕ್ರ ಭಾವನಾಃ ಪ್ರತಿಪಾದಿತಾಃ। ಯ ಏವಂ ವೇದ। ಸೋಽಥರ್ವಶಿರ್ಷೋಽಧೀತೇ ॥ ಓಂ ಭದ್ರಂ ಕರ್ಣೇಭಿಃ ಇತಿ ಶಾಂತಿಃ ॥

\as{ರಾಜಾಧಿರಾಜಾಯ+++++ಮಹೇಶ್ವರಃ ॥}ಪುಷ್ಪಾಂಜಲಿಃ ॥

ಪ್ರದಕ್ಷಿಣಂ ಕರೋಮಿತ್ವಾಂ ಪಾಹಿ ಮಾಂ ಮೃಡ ಶಂಕರ ।\\
ಅನಾಥ ನಾಥ ಸರ್ವಜ್ಞ ಭಕ್ತಾನಾಂ ಭದ್ರದಾಯಕ ॥\\
ಕೃತ್ವಾ ಪ್ರದಕ್ಷಿಣಮಹಂ ತವ ದಿವ್ಯಮೂರ್ತೇಃ\\ವಿಶ್ವಪ್ರದಕ್ಷಿಣಫಲಂ ನಿಖಿಲಾಂಡಮಾತಃ~।\\
ಪ್ರಾಪ್ತೋಽಸ್ಮಿ ಬುದ್ಧಿಮತುಲಾಮಖಿಲಾಗಮಜ್ಞಾಂ \\ಭಕ್ತಾಯ ಮೇ ಕರುಣಯಾ ದಿಶ ತಾಂ ಮ ಆಶು ॥\\
\as{ಸಪ್ತಾಸ್ಯಾಸನ್+++++ ಅಬಧ್ನನ್ಪುರುಷಂ ಪಶುಂ॥\\
ತಾಂ ಮ ಆವಹ ++++++ಪುರುಷಾನಹಮ್ ॥}ಪ್ರದಕ್ಷಿಣಮ್॥

ಸದಾನಂದಮಯಂ ದೇವಂ ಸದಾಸಂತೃಪ್ತಿದಾಯಕಂ ।\\ಸದಾವೈರಾಗ್ಯನಿರತಂ ಸದಾಶಾಂತಸ್ವರೂಪಿಣಂ ॥

ನಮಸ್ಕುರ್ವೇ ಮಹಾದೇವ ಭಕ್ತಾನುಗ್ರಹಕಾರಕ ।\\ ನಮಸ್ತೇ ದೇವದೇವೇಶ ನಮಸ್ತೇ ಕರುಣಾರ್ಣವ ॥

ನಮಸ್ತೇ ವೃಷಭಾರೂಢ ನಮಸ್ತೇ ಗಿರಿಜಾಪತೇ ।\\ನಮಸ್ತೇ ದುಃಖೌಘಹರ ವಿಶುದ್ಧಜ್ಞಾನಮೂರ್ತಯೇ ॥

ನಮಸ್ತೇ ತ್ರಿಪುರಾರೀಶ ನಮಸ್ತೇ ಜಗತಾಂ ಪ್ರಭೋ ।\\ ನಮಸ್ತೇ ಹೇ ಜಗದ್ವಂದ್ಯ ನಮಸ್ತೇ ನಾಗಭೂಷಣ ॥

ಗಂಗಾಧರ ನಮಸ್ತೇಸ್ತು  ನಮಸ್ತೇ ಶೂಲಪಾಣಯೇ ।\\ ನಮಸ್ತೇ ಸರ್ವಲೋಕೇಶ ನಮೋ ಭಕ್ತಜನಪ್ರಿಯ ॥

ನಮೋ ಭಕ್ತಪರಾಧೀನ ನಮೋ ಭಕ್ತವರಪ್ರದ ।\\ ನಮಸ್ತೇ ಸೃಷ್ಟಿಕರ್ತ್ರೇ ಚ ನಮಃ ಸಂಸ್ಥಿತಿಹೇತವೇ ॥

ನಮಸ್ತೇ ಲಯಕರ್ತ್ರೇ ಚ ನಮಸ್ತೇ ಜಗತಾಂ ಪ್ರಭೋ ।\\
ಪ್ರಣಮಾಮಿ ಜಗನ್ನಾಥ ಪ್ರಣತಾರ್ತಿಪ್ರಣಾಶನ ।\\ಪ್ರಣಾಮಗೋಚರೇಶಾನ ಸದಾಶಿವ ಜಗತ್ಪ್ರಭೋ ॥

ತುಷಾರಕಿರಣೋಲ್ಲಾಸಿಮುಕುಟಾಲಂಕೃತ ಮಸ್ತಕಮ್ ।\\ಪ್ರಣಮಾಮಿ ಮಹೇಶಾನಂ ನೀಲಗ್ರೀವಂ ಮಹೇಶ್ವರಮ್ ॥

ಯತ್ಪಾದಪದ್ಮಸರಸೀರುವಂದನೇನ \\ಶಕ್ರಾದಯಃ ಸ್ವಪದವೀಂ ಮಹತೀಮವಾಪುಃ~।\\
ಮನ್ವಾದಯೋಽಪಿ ಭುವಿ ಭೂಪತಯೋ ಬಭೂವುಃ\\ತ್ವಾಂ ಯಃ ಶುಚಿಃ ಪ್ರಣಮತಿ ಪ್ರಯತಃ ಸ ಧನ್ಯಃ ॥

ನಮೋ ದೇವ್ಯೈ ಮಹಾದೇವ್ಯೈ ಶಿವಾಯೈ ಸತತಂ ನಮಃ ।\\
ನಮಃ ಪ್ರಕೃತ್ಯೈ ಭದ್ರಾಯೈ ನಿಯತಾಃ ಪ್ರಣತಾ ಸ್ಮರತಾಮ। ॥

ರೌದ್ರಾಯೈ ನಮೋ ನಿತ್ಯಯೈ ಗೌರ್ಯ ಧಾತ್ರ್ಯೈ ನಮೋ ನಮಃ ।\\
ಜ್ಯೋತ್ಯಸ್ತ್ರಾಯೈ ಚೇಂದುರುಪಿಣ್ಯೈ ಸುಖಾಯೈ ಸತತಂ ನಮಃ ॥

ಕಲ್ಯಾಣ್ಯೈ ಪ್ರಣತಾಮೃದ್ಧ್ಯೈ ಸಿದ್ಧ್ಯೈ ಕುರ್ಮೋ ನಮೋ ನಮಃ ।\\
ನೈರ್ಋತ್ಯೈ ಭೂಭೃತಾಂ ಲಕ್ಷ್ಮ್ಯೈ ಶರ್ವಾಣ್ಯೈ ತೇ ನಮೋ ನಮಃ ॥

ದುರ್ಗಾಯೈ ದುರ್ಗಪಾರಾಯೈ ಸಾರಾಯೈ ಸರ್ವಕಾರಿಣ್ಯೈ ।\\
ಖ್ಯಾತ್ಯೈ ತಥೈವ ಕೃಷ್ಣಾಯೈ ಧೂಮ್ರಾಯೈ ಸತತಂ ನಮಃ ॥

ಅತಿಸೌಮ್ಯಾತಿರೌದ್ರಾಯೈ ನತಾಸ್ತಸ್ಯೈ ನಮೋ ನಮಃ ।\\
ನಮೋ ಜಗತ್ಪ್ರತಿಷ್ಠಾಯೈ ದೇವ್ಯೈ ನಮೋ ನಮಃ ॥

ಯಾ ದೇವೀ ಸರ್ವಭೂತೇಷು ವಿಷ್ಣುಮಾಯೇತಿ ಶಬ್ದಿತಾ ।\\
ನಮಸ್ತಸ್ಯೈ ನಮಸ್ತಸ್ಯೈ ನಮಸ್ತಸ್ಯೈ ನಮೋ ನಮಃ ॥

ಯಾ ದೇವೀ ಸರ್ವಭೂತೇಷು ಚೇತನೇತ್ಯಭಿಧೀಯತೇ ।\\
ನಮಸ್ತಸ್ಯೈ ನಮಸ್ತಸ್ಯೈ ನಮಸ್ತಸ್ಯೈ ನಮೋ ನಮಃ ॥

ಯಾ ದೇವೀ ಸರ್ವಭೂತೇಷು ಬುದ್ಧಿರೂಪೇಣ ಸಂಸ್ಥಿತಾ ।\\
ನಮಸ್ತಸ್ಯೈ ನಮಸ್ತಸ್ಯೈ ನಮಸ್ತಸ್ಯೈ ನಮೋ ನಮಃ ॥

ನಿದ್ರಾರೂಪೇಣ ಸಂಸ್ಥಿತಾ ॥ ಕ್ಷುಧಾರೂಪೇಣ ಸಂಸ್ಥಿತಾ ॥\\
ಛಾಯಾರೂಪೇಣ ಸಂಸ್ಥಿತಾ ॥ಶಕ್ತಿರೂಪೇಣ ಸಂಸ್ಥಿತಾ ॥\\
ತೃಷ್ಣಾರೂಪೇಣ ಸಂಸ್ಥಿತಾ ॥ಕ್ಷಾಂತಿರೂಪೇಣ ಸಂಸ್ಥಿತಾ ॥\\
ಜಾತಿರೂಪೇಣ ಸಂಸ್ಥಿತಾ ॥ಲಜ್ಜಾರೂಪೇಣ ಸಂಸ್ಥಿತಾ ॥\\
ಕಾಂತಿರೂಪೇಣ ಸಂಸ್ಥಿತಾ ॥ಲಕ್ಷ್ಮೀರೂಪೇಣ ಸಂಸ್ಥಿತಾ ॥\\
ವೃತ್ತಿರೂಪೇಣ ಸಂಸ್ಥಿತಾ ॥ಸ್ಮೃತಿರೂಪೇಣ ಸಂಸ್ಥಿತಾ ॥\\
ದಯಾರೂಪೇಣ ಸಂಸ್ಥಿತಾ ॥ತುಷ್ಟಿರೂಪೇಣ ಸಂಸ್ಥಿತಾ ॥\\
ಮಾತೃರೂಪೇಣ ಸಂಸ್ಥಿತಾ ॥ಭ್ರಾಂತಿರೂಪೇಣ ಸಂಸ್ಥಿತಾ ॥

ಇಂದ್ರಿಯಾಣಾಮಧಿಷ್ಠಾತ್ರೀ ಭೂತಾನಾಂ ಚಾಖಿಲೇಷು ಯಾ ।\\
ಭೂತೇಷು ಸತತಂ ತಸ್ಯೈ ವ್ಯಾಪ್ತಿದೈವ್ಯೈ ನಮೋ ನಮಃ ॥

ಚಿತಿರೂಪೇಣ ಯಾ ಕೃತ್ಸ್ನಮೇತದ್ವ್ಯಾಪ್ಯ ಸ್ಥಿತಾ ಜಗತ್ ।\\
ನಮಸ್ತಸ್ಯೈ ನಮಸ್ತಸ್ಯೈ ನಮಸ್ತಸ್ಯೈ ನಮೋ ನಮಃ ॥

ಸ್ತುತಾ ಸುರೈಃ ಪೂರ್ವಮಭೀಷ್ಟಸಂಶ್ರಯಾತ್\\ತಥಾ ಸುರೇಂದ್ರೇಣು ದಿನೇಷು ಸೇವಿತಾ ।\\
ಕರೋತು ಸಾ ನಃ ಶುಭಹೇತುರೀಶ್ವರೀ\\ ಶುಭಾನಿ ಭದ್ರಾಣ್ಯಭಿಹಂತು ಚಾಪದಃ ॥

ಯಾ ಸಾಂಪ್ರತಂ ಚೋದ್ಧತದೈತ್ಯತಾಪಿತೈ\\ರಸ್ಮಾಭಿರೀಶಾ ಚ ಸುರೈರ್ನಮಸ್ಯತೇ ।\\
ಯಾ ಚ ಸ್ಮೃತಾ ತತ್ಕ್ಷಣಮೇವ ಹಂತಿ ನಃ\\ ಸರ್ವಾಪದೋ ಭಕ್ತಿವಿನಮ್ರಮೂರ್ತಿಭಿಃ ॥

ಸರ್ವ ಮಂಗಲ ಮಾಂಗಲ್ಯೇ ಶಿವೇ ಸರ್ವಾರ್ಥ ಸಾಧಿಕೇ ।\\ಶರಣ್ಯೇ ತ್ರ್ಯಂಬಕೇ ಗೌರೀ ನಾರಾಯಣೀ ನಮೋಸ್ತುತೇ ॥

\as{ಯಜ್ಞೇನ ಯಜ್ಞಮಯ+++++-ಪೂರ್ವೇ ಸಾಧ್ಯಾಃ ಸಂತಿ ದೇವಾಃ ॥\\
ಯಃ ಶುಚಿಃ ++++++++ಸತತಂ ಜಪೇತ್ ॥}ನಮಸ್ಕಾರಾಃ ॥

ನಮಸ್ತೇ ದೇವದೇವೇಶ ನಮಸ್ತೇ ಕರುಣಾರ್ಣವ ।\\ ನಮೋ ಭಕ್ತಪರಾಧೀನ ಗೃಹಾಣಾರ್ಘ್ಯಂ ನಮೋಸ್ತು ತೇ ॥

ಕುಂಕುಮೇನ ಸಮಾಯುಕ್ತಂ ಚಂದನೇನ ವಿಮಿಶ್ರಿತಮ್ ।\\ಬಿಲ್ವಪತ್ರೇಣ ಸಹಿತಂ ಗೃಹಾಣಾರ್ಘ್ಯಂ ನಮೋಽಸ್ತುತೇ॥ಪ್ರಸನ್ನಾರ್ಘ್ಯಮ್॥\\

\section{ಪ್ರಾರ್ಥನಾ}
\addcontentsline{toc}{section}{ಪ್ರಾರ್ಥನಾ}
ಜಯ ದೇವ ಜಗನ್ನಾಥ ಜಯ ಶಂಕರ ಶಾಶ್ವತ।\\ ಜಯ ಸರ್ವಸುರಾಧ್ಯಕ್ಷ  ಜಯ ಸರ್ವಸುರಾರ್ಚಿತ ॥

ಜಯ ಸರ್ವಗುಣಾತೀತ ಜಯ ಸರ್ವವರಪ್ರದ ।\\ ಜಯ ನಿತ್ಯನಿರಾಧಾರ ಜಯ ವಿಶ್ವಂಭರಾವ್ಯಯ ॥

ಜಯ ವಿಶ್ವೈಕವೇದ್ಯೇಶ  ಜಯ ನಾಗೇಂದ್ರಭೂಷಣ ।\\ ಜಯ ಗೌರೀಪತೇ ಶಂಭೋ ಜಯ ಚಂದ್ರಾರ್ಧಶೇಖರ ॥

ಜಯ ಕೋಟ್ಯರ್ಕಸಂಕಾಶ ಜಯಾನಂತಗುಣಾಶ್ರ ಯ ।\\ ಜಯ ರುದ್ರವಿರುಪಾಕ್ಷ  ಜಯ ಚಿಂತ್ಯನಿರಂಜನ ॥

ಜಯ ನಾಥ ಕೃಪಾಸಿಂಧೋ ಜಯ ಭಕ್ತಾರ್ತ್ತಿಭಞ್ ಜನ ।\\ ಜಯ ದುಸ್ತರಸಂಸಾರಸಾಗರೋತ್ತಾರಣಪ್ರಭೋ ॥

ಪ್ರಸೀದ ಮೇ ಮಹಾಭಾಗ ಸಂಸಾರಾರ್ತಸ್ಯಖಿದ್ಯತ:।\\ ಸರ್ವಪಾಪಕ್ಷಯಂಕೃತ್ವಾರಕ್ಷ ಮಾಂ ಪರಮೇಶ್ವರ ॥

ಮಹಾದಾರಿದ್ರ್ಯಮಗ್ನಸ್ಯ ಮಹಾಪಾಪಹತಸ್ಯ ಚ ।\\ ಮಹಾಶೋಕನಿವಿಷ್ಟಸ್ಯ ಮಹಾರೋಗಾತುರಸ್ಯ ಚ ॥

ಋಣಭಾರಪರೀತಸ್ಯ ದಹ್ಯಮಾನಸ್ಯ ಕರ್ಮಭಿಃ ।\\ ಗ್ರಹೈಃ ಪ್ರಪೀಡ್ಯಮಾನಸ್ಯ ಪ್ರಸೀದ ಮಮ ಶಂಕರ ॥

ತ್ವಂ ಯಥಾ ಪಾಶಬದ್ಧಾನಾಂ ಪಶೂನಾಂ ಪಾಶಮೋಚಕಃ ।\\ ತಥಾ ವ್ರತೇನ ಸಂತುಷ್ಟಃ ಋಣಮೋಕ್ಷಕರೋ ಭವ ॥

ಯನ್ಮಯಾ ಭಕ್ತಿಸಂಯುಕ್ತಂ ಪತ್ರಂ ಪುಷ್ಪಂ ಫಲಂ ದಲಂ ।\\ ಪ್ರಕಲ್ಪಿತಂ ಚ ನೈವೇದ್ಯಂ ತದ್ಗೃಹಾಣ ಜಗತ್ಪತೇ ॥

ಗತಂ ಪಾಪಂ ಗತಂ ದುಃಖಂ ಗತಂ ದಾರಿದ್ರ್ಯಮೇವ ಚ ।\\ ಆಗತಾ ಸುಖಸಂಪತ್ತಿಃ ಪುಣ್ಯಾಚ್ಚ ತವ ದರ್ಶನಾತ್ ॥

ಸಂಪ್ರಾರ್ಥನಾ ತ್ವಯಿ ಮಮೇಯಮಶೇಷವಿದ್ಯಾ\-\\ಮಾಯುಃ ಶತಾಬ್ದಪರಿಮಾಣಮರೋಗತಾಂ ಚ~।\\
ಲಕ್ಷ್ಮೀಮಪಿ ಸ್ಥಿರತರಾಮಖಿಲೇಷ್ಟಸಿದ್ಧಿಂ \\ವೈರಾಗ್ಯವೃದ್ಧಿಮನಿಶಂ ಮಮ ದೇಹಿ ದೇವಿ ॥

ಶ್ರದ್ಧಾಂ ಗುರೌ ತ್ವಯಿ ಚ ಭಕ್ತಿಮಚಂಚಲಾಂ ಮೇ \\ಧರ್ಮೇ ಮತಿಂ ಶ್ರುತಿಪಥೇ ಗತಿಮನ್ಯವಿತ್ತೇ~।\\
ನೇಚ್ಛಾಂ ಚ ಪಾಪವಿಷಯೇ ವಿರತಿಂ ಪ್ರದಾಯ \\ಕೀರ್ತ್ಯಾ ಯುತಂ ಜನನಿ ಕುರ್ವಿತಿ ಚಾರ್ಥಯಾಮಿ ॥

ಸರ್ವ ಮಂಗಲ ಮಾಂಗಲ್ಯೇ ಶಿವೇ ಸರ್ವಾರ್ಥ ಸಾಧಿಕೇ ।\\ ಶರಣ್ಯೇ ತ್ರ್ಯಂಬಕೇ ಗೌರೀ ನಾರಾಯಣೀ ನಮೋಸ್ತುತೇ ॥

ಓಂ ಜಯ ರುದ್ರೇ ವಿರೂಪಾಕ್ಷಿ ಜಯಾತೀತೇ ನಿರಂಜನಿ ।\\ಜಯ ಕಲ್ಯಾಣಸುಖದೇ ಜಯ ಮಂಗಲದೇ ಶುಭೇ ॥

ಜಯ ಸಿದ್ಧಮುನೀನ್ದ್ರಾದಿ ವನ್ದಿತಾಂಘ್ರಿಸರೋರುಹೇ ।\\ಜಯ ವಿಷ್ಣುಪ್ರಿಯೇ ದೇವಿ ಜಯ ಭೂತವಿಭೂತಿದೇ ॥

ಜಯ ರತ್ನಪ್ರದೀಪಾಭೇ ಜಯ ಹೇಮವಿಭಾಸಿತೇ ।\\ಜಯ ಬಾಲೇನ್ದುತಿಲಕೇ ತ್ರ್ಯಮ್ಬಕೇ ಜಯ ವೃದ್ಧಿದೇ ॥

ಸರ್ವಲಕ್ಷ್ಮೀಪ್ರದೇ ದೇವಿ ಸರ್ವರಕ್ಷಾಪ್ರದಾ ಭವ ।\\ಧರ್ಮಾರ್ಥಕಾಮಮೋಕ್ಷಾಖ್ಯ ಚತುರ್ವರ್ಗಫಲಪ್ರದೇ ॥

ಶೈಲಪುತ್ರಿ ನಮಸ್ತೇಽಸ್ತು ಬ್ರಹ್ಮಚಾರಿಣಿ ತೇ ನಮಃ ।\\ಕಾಲರಾತ್ರಿ ನಮಸ್ತೇಽಸ್ತು ನಾರಾಯಣಿ ನಮೋಽಸ್ತುತೇ ॥

ಮಧುಕೈಟಭಹಾರಿಣ್ಯೈ ನಮೋ ಮಹಿಷಮರ್ದಿನಿ ।\\ಧೂಮ್ರಲೋಚನನಿರ್ನಾಶೇ ಚಂಡಮುಂಡವಿನಾಶಿನಿ ॥

ರಕ್ತಬೀಜವಧೇ ದೇವಿ ನಿಶುಮ್ಭಾಸುರಘಾತಿನಿ ।\\ಶುಮ್ಭಪ್ರಾಣಾಪಹಾರಿಣ್ಯೈ ತ್ರ್ಯೈಲೋಕ್ಯವರದೇ ನಮಃ ॥

ದೇವಿ ದೇಹಿ ಪರಂ ರೂಪಂ ದೇವಿ ದೇಹಿ ಪರಂ ಸುಖಮ್ ।\\ಧರ್ಮಂ ದೇಹಿ ಧನಂ ದೇಹಿ ಸರ್ವಕಾಮಾಂಶ್ಚ ದೇಹಿ ಮೇ ॥

ಸುಪುತ್ರಾಂಶ್ಚ ಪಶೂನ್ ಕೋಶಾನ್ ಸುಕ್ಷೇತ್ರಾಣಿ ಸುಖಾನಿ ಚ ।\\ದೇವಿ ದೇಹಿ ಪರಂ ಜ್ಞಾನಮಿಹ ಮುಕ್ತಿ ಸುಖಂ ಕುರು ॥

ಯದ್ಯದಿಚ್ಛಾಮ್ಯಹಂ ನಿತ್ಯಂ ತತ್ತತ್ಸರ್ವಂ ಪ್ರಯಚ್ಛ ಮೇ ।\\ನಾಮ್ನಾಮಷ್ಟೋತ್ತರಶತೈಃ ಸಹಸ್ರೈರ್ವಾ ಯಜೇತ್ಸುಧೀಃ ॥\\

ತವ ಪ್ರತಿಜ್ಞಾಮಾಲಂಬ್ಯ ಕಷ್ಟಾ ನಶ್ಯಂತಿ ನಃ ಕ್ವಚಿತ್ ।\\ಸೇವಿತುಂ ಭವತೀಂ ತಸ್ಮಾತ್ ಕೃತಾ ಪೂಜಾ ತವಾಜ್ಞಯಾ ॥

ಪೂರ್ವಂ ಚ ಕ್ರಮಮಾಲಂಬ್ಯ ತವ ಪೂಜಾ ಮಯಾ ಕೃತಾ ।\\ಸ್ಥಿರಾ ಮಮ ಗೃಹೇ ನಿತ್ಯಂ ಭವ ಸಂತಾಪಹಾರಿಣೀ ॥
 
ನ್ಯೂನಂ ವಾಪ್ಯಧಿಕಂ ವಾಪಿ ಯನ್ಮಯಾ ಮೋಹತಃ ಕೃತಂ ।\\ಸರ್ವಂ ತದಸ್ತು ಸಂಪೂರ್ಣ ತ್ವತ್ಪ್ರಸಾದಾತ್ ಮಹೇಶ್ವರಿ ॥

ಪೂಜಾಜಪಾಗ್ನಿಕಾರ್ಯಾದ್ಯೈಃ ಸುಕೃತಂ ಯನ್ಮಯಾರ್ಜಿತಂ ।\\ತತ್ಸರ್ವಂ ಸಫಲಂ ಮೇಽಸ್ತು ಭುಕ್ತಿಂ ಮುಕ್ತಿಂ ಚ ಸಾಧಯ ॥

ದುರ್ಗೇ ಸ್ಮೃತಾ ಹರಸಿ ಭೀತಿಮಶೇಷಜಂತೋಃ\\ಸ್ವಸ್ಥೈಃ ಸ್ಮೃತಾ ಮತಿಮತೀವ ಶುಭಾಂ ದದಾಸಿ।\\
ದಾರಿದ್ರಯದುಃಖಭಯಹಾರಿಣಿ ಕಾ ತ್ವದನ್ಯಾ\\ಸರ್ವೋಪಕಾರಕರಣಾಯ ಸದಾರ್ದ್ರಚಿತ್ತಾ ॥

ಪ್ರಣತಾನಾಂ ಪ್ರಸೀದ ತ್ವಂ ದೇವಿ ವಿಶ್ವಾರ್ತಿಹಾರಿಣಿ ।\\ತ್ರೈಲೋಕ್ಯವಾಸಿನಾಮೀಡ್ಯೇ ಲೋಕಾನಾಂ ವರದಾ ಭವ ॥ಪ್ರಾರ್ಥನಾ॥

\section{ಮಾತೃಕಾಸ್ತೋತ್ರಂ}
\addcontentsline{toc}{section}{ಮಾತೃಕಾಸ್ತೋತ್ರಂ}
ಗಣೇಶ ಗ್ರಹ ನಕ್ಷತ್ರಯೋಗಿನೀ ರಾಶಿ ರೂಪಿಣೀಂ~।\\
ದೇವೀಂ ಮಂತ್ರಮಯೀಂ ನೌಮಿ ಮಾತೃಕಾಂ ಪೀಠ ರೂಪಿಣೀಂ॥೧॥

ಪ್ರಣಮಾಮಿ ಮಹಾದೇವೀಂ ಮಾತೃಕಾಂ ಪರಮೇಶ್ವರೀಂ।\\
ಕಾಲಹಲ್ಲೋಹಲೋಲ್ಲೋಲ ಕಲನಾಶಮಕಾರಿಣೀಂ॥೨॥

ಯದಕ್ಷರೈಕಮಾತ್ರೇಪಿ ಸಂಸಿದ್ಧೇ ಸ್ಪರ್ಧತೇ ನರಃ।\\
ರವಿತಾರ್ಕ್ಷ್ಯೇಂದುಕಂದರ್ಪಶಂಕರಾನಲವಿಷ್ಣುಭಿಃ॥೩॥

ಯದಕ್ಷರ ಶಶಿ ಜ್ಯೋತ್ಸ್ನಾಮಂಡಿತಂ ಭುವನತ್ರಯಂ।\\
ವಂದೇ ಸರ್ವೇಶ್ವರೀಂ ದೇವೀಂ ಮಹಾ ಶ್ರೀ ಸಿದ್ಧಮಾತೃಕಾಂ॥೪॥

ಯದಕ್ಷರಮಹಾಸೂತ್ರಪ್ರೋತಮೇತಜ್ಜಗತ್ತ್ರಯಂ।\\
ಬ್ರಹ್ಮಾಂಡಾದಿಕಟಾಹಾಂತಂ ತಾಂ ವಂದೇ ಸಿದ್ಧಮಾತೃಕಾಂ॥೫॥

ಯದೇಕಾದಶಮಾಧಾರಂ ಬೀಜಂ ಕೋಣತ್ರಯೋದ್ಭವಂ।\\
ಬ್ರಹ್ಮಾಂಡಾದಿ ಕಟಾಹಾಂತಂ ಜಗದದ್ಯಾಪಿ ದೃಶ್ಯತೇ॥೬॥

ಅಕಚಾದಿಟತೋನ್ನದ್ಧಪಯಶಾಕ್ಷರವರ್ಗಿಣೀಂ।\\
ಜ್ಯೇಷ್ಠಾಂಗಬಾಹುಹೃತ್ಪೃಷ್ಠಕಟಿಪಾದನಿವಾಸಿನೀಂ॥೭॥

ತಾಮೀಕರಾಕ್ಷರೋದ್ಧಾರಾಂ ಸಾರಾತ್ಸಾರಾಂ ಪರಾತ್ಪರಾಂ।\\
ಪ್ರಣಮಾಮಿ ಮಹಾದೇವೀಂ ಪರಮಾನಂದರೂಪಿಣೀಂ॥೮॥

ಅದ್ಯಾಪಿ ಯಸ್ಯಾ ಜಾನಂತಿ ನ ಮನಾಗಪಿ ದೇವತಾಃ।\\
ಕೇಯಂ ಕಸ್ಮಾತ್ ಕ್ವ ಕೇನೇತಿ ಸರೂಪಾರೂಪಭಾವನಾಂ॥೯॥

ವಂದೇ ತಾಮಹ ಮಕ್ಷಯ್ಯಾಂ ಕ್ಷಕಾರಾಕ್ಷರರೂಪಿಣೀಂ~।\\
ದೇವೀಂಕುಲಕಲೋಲ್ಲಾಸ ಪ್ರೋಲ್ಲಸಂತೀಂ ಪರಾಂಶಿವಾಂ॥೧೦॥

ವರ್ಗಾನು ಕ್ರಮ ಯೋಗೇನ ಯಸ್ಯಾಂ ಮಾತ್ರಷ್ಟಕಂ ಸ್ಥಿತಂ।\\
ವಂದೇ ತಾಮಷ್ಟವರ್ಗೋತ್ಥಮಹಾಸಿದ್ಧ್ಯಷ್ಟಕೇಶ್ವರೀಂ ॥೧೧॥

ಕಾಮಪೂರ್ಣಜಕಾರಾಖ್ಯ ಶ್ರೀಪೀಠಾಂತರ್ನಿವಾಸಿನೀಂ।\\
ಚತುರಾಜ್ಞಾಕೋಶಮೂಲಾಂ ನೌಮಿ ಶ್ರೀ ತ್ರಿಪುರಾಮಹಂ॥೧೨॥

ಇತಿ ದ್ವಾದಶಭಿಃ ಶ್ಲೋಕೈಃ ಸ್ತವನಂ ಸರ್ವ ಸಿದ್ಧಿ ಕೃತ್।\\
ದೇವ್ಯಾ ಸ್ತ್ವಖಂಡ ರೂಪಾಯಾಃ ಸ್ತವನಂ ತವ ತಥ್ಯತಃ॥೧೩॥
\authorline{ಇತಿ ಸರ್ವಸಿದ್ಧಿಕೃತ್ಸ್ತೋತ್ರಂ ಸಂಪೂರ್ಣಂ॥}
\newpage\section{ಪುನಃ ಪೂಜಾ}
\addcontentsline{toc}{section}{ಪುನಃ ಪೂಜಾ}
\begin{tabular}{ c c c }
ಮಹಾದೇವಾಯ ನಮಃ & ದುರ್ಗಾಯೈ  ನಮಃ & ಧ್ಯಾನಂ\\
ಮಹೇಶ್ವರಾಯ & ಕಾತ್ಯಾಯನ್ಯೈ & ಆವಾಹನಮ್\\
ಶಂಕರಾಯ & ಗೌರ್ಯೈ & ಪಾದ್ಯಮ್\\
ವೃಷಭಧ್ವಜಾಯ & ಮಹಿಷಮರ್ದಿನ್ಯೈ & ಅರ್ಘ್ಯಮ್\\
ಶೂಲಪಾಣಯೇ & ಚಂಡಿಕಾಯೈ & ಆಚಮನಮ್\\
ತ್ರಿಪುರಾಂತಕಾಯ & ಮಹಾಲಕ್ಷ್ಮ್ಯೈ & ಸ್ನಾನಮ್\\
ಸರ್ವೇಶ್ವರಾಯ & ಸರ್ವೇಶ್ವರ್ಯೈ & ಪಂಚಾಮೃತಮ್\\
ಶ್ರೀಕಂಠಾಯ & ಕಾಲ್ಯೈ & ಶುದ್ಧೋದಕಮ್\\
ಪರಮೇಶ್ವರಾಯ & ಪರಮೇಶ್ವರ್ಯೈ & ವಸ್ತ್ರಮ್\\
ಪಾರ್ವತೀಪತಯೇ & ಮಹಾಭೂತ್ಯೈ & ಉಪವೀತಮ್\\
ನಾಗಭೂಷಣಾಯ & ಸರ್ವಾಭರಣಭೂಷಿತಾಯೈ & ಆಭರಣಮ್\\
ಸತ್ಪರಾಯಣಾಯ & ಸರ್ವಮಂಗಳಾಯೈ & ಗಂಧಮ್\\
ಶಾಂತಾಯ & ವರದಾಯೈ & ಅಕ್ಷತಾಃ\\
ಶಿವಾಯ & ದುರ್ಗಾಯೈ & ಹರಿದ್ರಾ\\
ಭರ್ಗಾಯ & ಭಗವತ್ಯೈ & ಕುಂಕುಮಮ್\\
ರುದ್ರಾಯ & ಚಾಮುಂಡಾಯೈ & ಪುಷ್ಪಮ್\\
ವಿಶ್ವೇಶ್ವರಾಯ & ವಿಜಯಾಯೈ & ಧೂಪಃ\\
ವಿಶುದ್ಧಜ್ಞಾನಾಯ & ಜ್ಞಾನಮೂರ್ತ್ಯೈ & ದೀಪಃ\\
ತ್ರಿಪುರಾಂತಕಾಯ & ತ್ರಿಪುರಾಂತಕ್ಯೈ & ನೈವೇದ್ಯಮ್\\
ಪದ್ಮಾಯ & ಪದ್ಮಾಯೈ & ತಾಂಬೂಲಮ್\\
ಶಿವಾಯ & ಶಿವಶಕ್ತ್ಯೈ & ದೀಪಮಾಲಾ\\
\end{tabular}
\newpage
\begin{tabular}{ c c c }
ಪಾಪನಾಶನಾಯ & ಪಾಪನಾಶಿನ್ಯೈ & ನೀರಾಜನಮ್\\
ಪುರುಷಾಯ & ಪಾರ್ವತ್ಯೈ & ಮಂತ್ರಪುಷ್ಪಮ್\\
ಗೌರಾಂಗಾಯ & ಗೌರ್ಯೈ & ಪ್ರದಕ್ಷಿಣನಮಸ್ಕಾರಾಃ\\
ಬ್ರಹ್ಮಣ್ಯಾಯ & ಬ್ರಾಹ್ಮ್ಯೈ & ಪ್ರಸನ್ನಾರ್ಘ್ಯಮ್\\
ಶುಭಾಯ & ಶುಭಾವಹಾಯೈ & ಪ್ರಾರ್ಥನಾ\\
\end{tabular}


ಕುಬೇರಾಯ ವೈ ಶ್ರವಣಾಯ, ಮಹಾರಾಜಾಯ ನಮಃ ॥\\ಕಾಮೇಶ್ವರಾಯ ನಮಃ ಕಾಮೇಶ್ವರ್ಯೈ ನಮಃ॥\as{ ಛತ್ರಂ ಧಾರಯಾಮಿ॥}

ತದಪ್ಯೇಷಃ ಶ್ಲೋಕೋಭಿಗೀತೋ ಮರುತಃ ಪರಿವೇಷ್ಟಾರೋ ಮರುತ್ತಸ್ಯಾವಸನ್ ಗೃಹೇ॥\\ಶಿವಾಯ ನಮಃ ಕಾಲ್ಯೈನಮಃ॥\as{ ಚಾಮರೇಣ ವೀಜಯಾಮಿ॥}

 ಶ್ರಿಯ ಏವೈನಂ ತಚ್ಛ್ರಿಯಮಾದಧಾತಿ ಸಂತತಮೃಚಾ ವಷಟ್ಕೃತ್ಯಂ ಸಂತತ್ಯೈ ಸಂಧೀಯತೇ॥\\ ಲೋಕಗುರವೇ ನಮಃ  ಲೋಕಮಾತ್ರೇ ನಮಃ॥\as{ಗೀತಂ ಗಾಯಾಮಿ॥}

ಪರ್ಯಾಪ್ತ್ಯಾ ಅನಂತರಾಯಾಯ ಸರ್ವಸ್ತೋಮೋತಿ ರಾತ್ರ ಉತ್ತಮಮಹರ್ಭವತಿ॥\\ಮೃಡಾಯ ನಮಃ ಮೃಡಾನ್ಯೈ ನಮಃ॥\as{ನಾಟ್ಯಂ ನಟಾಮಿ॥}

ಸಾಮ್ರಾಜ್ಯಂ ಭೌಜ್ಯಂ ಸ್ವಾರಾಜ್ಯಂ ವೈರಾಜ್ಯಂ ಪಾರಮೇಷ್ಟ್ಯಂ ರಾಜ್ಯಂ ಮಹಾರಾಜ್ಯಮಾಧಿಪತ್ಯಮಯಂ॥\\ದೇವದೇವಾಯ ನಮಃ  ಮಹಾಲಕ್ಷ್ಮ್ಯೈ ನಮಃ॥\as{ ವಾದ್ಯಂ ವಾದಯಾಮಮಿ॥}

ಸಹಸ್ರಮಾಖ್ಯಾತ್ರೇ ದದ್ಯಾಚ್ಛತಂ ಪ್ರತಿಗರಿತ್ರ ಏತೇ ಚೈವಾಽಽಸನೇ॥\\ಮಹಾದೇವಾಯ ನಮಃ ಮಹಾಸರಸ್ವತ್ಯೈ ನಮಃ॥\as{ ಆಂದೋಲಿಕಾಮಾರೋಹಯಾಮಿ॥}

ಆವಿಕ್ಷಿತಸ್ಯ ಕಾಮಪ್ರೇರ್ವಿಶ್ವೇದೇವಾಃ ಸಭಾಸದ ಇತಿ॥\\ಸ್ವರೂಪಾಯ ನಮಃ ನಿತ್ಯಸ್ವರೂಪಿಣ್ಯೈ ನಮಃ॥\as{ ವ್ಯಜನೇನ ವೀಜಯಾಮಿ॥}

ಪೃಥಿವ್ಯೈ ಸಮುದ್ರಪರ್ಯಂತಾಯಾ ಏಕರಾಳಿತಿ॥\\ಲೋಕಚಕ್ಷುಷೇ ನಮಃ ಲೋಕಚಕ್ಷುಷ್ಯೈ ನಮಃ॥\as{ ದರ್ಪಣಂ ದರ್ಶಯಾಮಿ॥}

ಶ್ವೇತಶ್ಚಾಶ್ವತರೀರಥೋ ಹೋತುಃ. ಪುತ್ರಕಾಮಾ ಹಾಪ್ಯಾಖ್ಯಾಪಯೇರನ್ ಲಭಂತೇ ಹ ಪುತ್ರಾನ್ ಲಭಂತೇ ಹ ಪುತ್ರಾನ್॥\\ಸವಿತ್ರೇ ನಮಃ ಸಾವಿತ್ರ್ಯೈ ನಮಃ॥\as{ ಅಶ್ವಮಾರೋಹಯಾಮಿ॥}

ಪ್ರಜಯಾ ಪಶುಭಿರ್ಯ ಏವಂ ವೇದ॥\\ಮಹೇಂದ್ರಾಯ ನಮಃ ಮಾಹೇಂದ್ರ್ಯೈ ನಮಃ॥\as{ ಗಜಮಾರೋಹಯಾಮಿ॥}

ಸಮಂತ ಪರ್ಯಾಯೀ ಸ್ಯಾಂ ಸಾರ್ವಭೌಮಃ ಸಾರ್ವಾಯುಷ ಆಂತಾದಾ ಪರಾರ್ಧಾತ್॥\\ಮಹೇಶ್ವರಾಯ ನಮಃ ಮಾಹೇಶ್ವರ್ಯೈ ನಮಃ॥\as{ ವೃಷಭಮಾರೋಹಯಾಮಿ॥}

ಉರುಂ ದುಹಾಂ ಯಜಮಾನಾಯ ಯಜ್ಞಂ॥\\ವನಮಾಲಿನೇ ನಮಃ ಲಕ್ಷ್ಮ್ಯೈ ನಮಃ॥\as{ ಮೃಗಮಾರೋಹಯಾಮಿ॥}

ಅನೂರಾಧಾನ್ ಹವಿಷಾ ವರ್ಧಯಂತಃ । ಶತಂ ಜೀವೇವ ಶರದಃ ಸವೀರಾಃ॥\\ಬ್ರಹ್ಮಣೇ ನಮಃ ಬ್ರಾಹ್ಮ್ಯೈ ನಮಃ॥\as{ ಹಂಸಮಾರೋಹಯಾಮಿ॥}

ಪೃಥೂ ರಥೋ ದಕ್ಷಿಣಾಯಾ ಅಯೋಜಿ॥\\ತ್ರಿಪುರಾಂತಕಾಯ ನಮಃ ಚಂಡಿಕಾಯೈ ನಮಃ॥\as{ ರಥಮಾರೋಹಯಾಮಿ॥}

\begin{tabular}{ c c c }
ಭಕ್ತವತ್ಸಲಾಯ ನಮಃ & ಪ್ರಭಾಯೈ ನಮಃ & ಉರಸಾ ನಮಾಮಿ\\
ಮಾಯಾಯ & ಮಾಯಾಯೈ & ಶಿರಸಾ ನಮಾಮಿ\\
ಜಯಾಯ & ಜಯಾಯೈ & ದೃಷ್ಟ್ಯಾ ನಮಾಮಿ\\
ಸೂಕ್ಷ್ಮಾಯ & ಸೂಕ್ಷ್ಮಾಯೈ & ಮನಸಾ ನಮಾಮಿ\\
ವಿಶುದ್ಧಾಯ & ವಿಶುದ್ಧಾಯೈ & ವಚಸಾ ನಮಾಮಿ\\
ಸುಪ್ರಭಾಯ & ಸುಪ್ರಭಾಯೈ & ಪದ್ಭ್ಯಾಂ ನಮಾಮಿ\\
ಕಾಮೇಶ್ವರಾಯ & ಕಾಮೇಶ್ವರ್ಯೈ & ಕರಾಭ್ಯಾಂ ನಮಾಮಿ\\
ವಿಜಯಾಯ & ವಿಜಯಾಯೈ & ಕರ್ಣಾಭ್ಯಾಂ ನಮಾಮಿ\\
ಸರ್ವಸಿದ್ಧಿದಾಯ & ಸರ್ವಸಿದ್ಧಿದಾಯೈ & ಸಾಷ್ಟಾಂಗಂ ನಮಾಮಿ \\
\end{tabular}

ಅಜ್ಞಾನತೋ ಯದಿ ಮಯಾ ಮನುಜಸ್ವಭಾವಾತ್\\ ಪೂಜಾವಿಧೌ ಸ್ಖಲಿತಮತ್ರ ಕೃತಂ ಕ್ಷಮಸ್ವ~।\\
ಪೂಜಾಮಿಮಾಮಭಿನವಾಂ ಹೃದಯಾಭಿರಾಮಾಮ್ \\ಆನಂದತಸ್ತವ ಪದೇಽಂಬ ಸಮರ್ಪಯಾಮಿ ॥
\authorline{{\LARGE ಸದ್ಗುರುಚರಣಾರವಿಂದಾರ್ಪಣಮಸ್ತು\\ಓಂ ತತ್ಸತ್\\********}}

ಗುರುಪೂಜಾ
ಗುರುಗಣಪತಿ ಪ್ರಾರ್ಥನಾ
ಗಣೇಶ ಪೂಜಾ
ಕಲಶ ಶಂಖ ಮಂಟಪಪೂಜಾ ದ್ವಾರಪೂಜಾಪೀಠಪೂಜಾಂ ಕೃತ್ವಾ
ನವಶಕ್ತಿಪೂಜಾ
ಧ್ಯಾನಂ
ಹಂಸಾಭ್ಯಾಂ
ಬಾಲಾರ್ಕಪ್ರಭ
ದಿಗಂಬರಂ ಭಸ್ಮ

ಜಗದುತ್ಪತ್ತಿಕರ್ತ್ರೇ ಚ ಸ್ಥಿತಿಸಂಹಾರಹೇತವೇ ।
ಭವಪಾಶವಿಮುಕ್ತಾಯ ದತ್ತಾತ್ರೇಯ ನಮೋಽಸ್ತು ತೇ ॥
ಆವಾಹಯಾಮಿ ದೇವೇಶ ಭವಂತು ನಿತ್ಯನಿರ್ಮಲಮ್ ।
ಗೃಹಾಣಾನಘಯಾ ಸಹಿತ ತವ ಪೂಜಾಂ ಮಯಾಕೃತಮ್॥ಆವಾಹನಮ್॥

ಜರಾಜನ್ಮವಿನಾಶಾಯ ದೇಹಶುದ್ಧಿಕರಾಯ ಚ ।
ದಿಗಂಬರ ದಯಾಮೂರ್ತೇ ದತ್ತಾತ್ರೇಯ ನಮೋಽಸ್ತು ತೇ ॥
ರಮ್ಯಂ ಸಿಂಹಾಸನಂ ದಿವ್ಯಂ ಸರ್ವಸೌಖ್ಯಕರಂ ಶುಭಮ್।
ವ್ಯಾಘ್ರಾಜಿನ ಸಮಾಕೀರ್ಣಂ ದತ್ತಂ ತೇ ಪ್ರತಿಗೃಹ್ಯತಾಮ್ ॥

ಚರಣಂ ಪವಿತ್ರಂ--- ತರೇಮ ।
ಪಾದ್ಯಂ॥

ಕರ್ಪೂರಕಾಂತಿದೇಹಾಯ ಬ್ರಹ್ಮಮೂರ್ತಿಧರಾಯ ಚ ।
ವೇದಶಾಸ್ತ್ರಪರಿಜ್ಞಾಯ ದತ್ತಾತ್ರೇಯ ನಮೋಽಸ್ತು ತೇ ॥ 
ಚಂದನಾಕ್ಷತ ಪುಷ್ಪಾದಿ ನಾನಾರತ್ನ ಸಮನ್ವಿತಂ ।
ಶುದ್ಧಂ ಕಲ್ಪಿತಂ ಚಾರ್ಘ್ಯಂ ಸದ್ಗುರೋ ಪ್ರತಿಗೃಹ್ಯತಾಮ್ ॥

ಹ್ರಸ್ವದೀರ್ಘಕೃಶಸ್ಥೂಲನಾಮಗೋತ್ರವಿವರ್ಜಿತ ।
ಪಂಚಭೂತೈಕದೀಪ್ತಾಯ ದತ್ತಾತ್ರೇಯ ನಮೋಽಸ್ತು ತೇ ॥ 
ಶೀತಲಂ ವಿಮಲಂ ತೋಯಂ ಕರ್ಪೂರಾದಿ ಸುವಾಸಿತಮ್ ।
ಆಚಮ್ಯತಾಂ ಗುರುಶ್ರೇಷ್ಠ ಮಯಾ ದತ್ತಂ ಚ ಭಕ್ತಿತಃ ॥

ಆದೌ ಬ್ರಹ್ಮಾ ಹರಿರ್ಮಧ್ಯೇ ಹ್ಯಂತೇ ದೇವಸ್ಸದಾಶಿವಃ ।
ಮೂರ್ತಿತ್ರಯಸ್ವರೂಪಾಯ ದತ್ತಾತ್ರೇಯ ನಮೋಽಸ್ತು ತೇ ॥


ಯಜ್ಞಭೋಕ್ತ್ರೇ ಚ ಯಜ್ಞಾಯ ಯಜ್ಞರೂಪಧರಾಯ ಚ ।
ಯಜ್ಞಪ್ರಿಯಾಯ ಸಿದ್ಧಾಯ ದತ್ತಾತ್ರೇಯ ನಮೋಽಸ್ತು ತೇ ॥ಮಧುಪರ್ಕಃ॥


ಗಂಗಾದಿ ಸರ್ವತೀರ್ಥೇಭ್ಯೋ ಸುಗಂಧ ದ್ರವ್ಯ ಮಿಶ್ರಿತಮ್ ।
ಪುನರಾಚಮನಾರ್ಥೇ ತ್ವಂ ಪ್ರೀತ್ಯಾ ಸ್ವೀಕುರು ಸದ್ಗುರೋ ॥

ಮಲಾಪಕರ್ಷಣ
ಭೋಗಾಲಯಾಯ ಭೋಗಾಯ ಯೋಗಯೋಗ್ಯಾಯ ಧಾರಿಣೇ ।
ಜಿತೇಂದ್ರಿಯ ಜಿತಜ್ಞಾಯ ದತ್ತಾತ್ರೇಯ ನಮೋಽಸ್ತು ತೇ ॥
ಗಂಗಾದಿ ಸರ್ವತೀರ್ಥೇಭ್ಯೋ ಆನೀತಂ ತೋಯಮುತ್ತಮ್ ।
ಸ್ನಾನಾರ್ಥಂತು ಮಯಾ ದತ್ತಂ ಪ್ರೀತ್ಯಾ ಸ್ವೀಕುರು ತ್ರಿಮೂರ್ತಯೇ ॥-------------------

(ಮಲಾಪಕರ್ಷಣಸ್ನಾನ ಪಂಚಾಮೃತ ರುದ್ರಪ್ರಶ್ನಃ ಪುರುಷಸೂಕ್ತ ಇತ್ಯಾದಿ )

ದಿಗಂಬರಾಯ ದಿವ್ಯಾಯ ದಿವ್ಯರೂಪಧರಾಯ ಚ ।
ಸದೋದಿತಪರಬ್ರಹ್ಮ ದತ್ತಾತ್ರೇಯ ನಮೋಽಸ್ತು ತೇ ॥
ಮಲಾಪಕರ್ಷಣಂ ಆಪೋಹಿಷ್ಠಾ

ಕ್ಷೀರೇ ಸೋಮಮಾವಾಹಯಾಮಿ...---

ಜಂಬೂದ್ವೀಪೇ ಮಹಾಕ್ಷೇತ್ರೇ ಮಾತಾಪುರನಿವಾಸಿನೇ ।
ಜಯಮಾನ ಸತಾಂ ದೇವ ದತ್ತಾತ್ರೇಯ ನಮೋಽಸ್ತು ತೇ ॥
ಕಾಮಧೇನುಸಮುದ್ಭೂತಂ ಸರ್ವೇಷಾಂ ಜೀವನಂ ಪರಂ।\\
ಪಾವನಂ ಯಜ್ಞಹೇತುಶ್ಚ ಸ್ನಾನಾರ್ಥಂ ಪ್ರತಿಗೃಹ್ಯತಾಂ ॥ \\
\as{ಆಪ್ಯಾಯಸ್ವ++++ಸಂಗಥೇ ॥\\
ಸದ್ಯೋಜಾತಂ+++ ನಮಃ ॥}
ಯದಂಘ್ರಿಕಮಲದ್ವಂದ್ವಂ ದ್ವಂದ್ವತಾಪನಿವಾರಕಂ~।\\
ತಾರಕಂ ಭವಸಿಂಧೋಶ್ಚ ತಂ ಗುರುಂ ಪ್ರಣಮಾಮ್ಯಹಂ~॥

ಪಯಸಾತು ಸಮುದ್ಭೂತಂ ಮಧುರಾಮ್ಲಶಶಿಪ್ರಭಂ ।\\
ದಧ್ಯಾನೀತಂ ಮಯಾ ದತ್ತಂ ಪ್ರೀತ್ಯಾ ಸ್ವೀಕುರು ಶಂಕರ ॥\\
\as{ದಧಿಕ್ರಾವ್ಣೋ+++++ ತಾರಿಷತ್ ॥\\
ವಾಮದೇವಾಯ ನಮೋ++++ ನಮಃ ॥}
ಶೋಷಣಂ ಪಾಪಪಂಕಸ್ಯ ದೀಪನಂ ಜ್ಞಾನತೇಜಸಃ~।\\
ಗುರೋಃ ಪಾದೋದಕಂ ಸಮ್ಯಕ್ ಸಂಸಾರಾರ್ಣವತಾರಕಂ~॥

ನವನೀತಸಮುತ್ಪನ್ನಂ ಆಯುರಾರೋಗ್ಯವರ್ಧನಂ ।\\
ಘೃತಂ ತುಭ್ಯಂ ಪ್ರದಾಸ್ಯಾಮಿ ಸ್ನಾನಾರ್ಥಂ ಪ್ರತಿಗೃಹ್ಯತಾಂ ॥\\
\as{ಘೃತಂ ಮಿಮಿಕ್ಷೇ +++ಹವ್ಯಮ್ ॥\\
ಅಘೋರೇಭ್ಯೋಥ+++ರೂಪೇಭ್ಯಃ ॥}
ತಾಪತ್ರಯಾಗ್ನಿತಪ್ತನಾಮಶಾಂತಪ್ರಾಣಿನಾಂ ಭುವಿ~।\\
ಯಸ್ಯ ಪಾದೋದಕಂ ಗಂಗಾ ತಸ್ಮೈ ಶ್ರೀಗುರವೇ ನಮಃ~॥

ತರುಪುಷ್ಪಸಮಾಕೃಷ್ಟಂ ಸುಸ್ವಾದು ಮಧುರಂ ಮಧು ।\\
ತೇಜಃಪುಷ್ಟಿಕರಂ ದಿವ್ಯಂ ಸ್ನಾನಾರ್ಥಂ ಪ್ರತಿಗೃಹ್ಯತಾಂ ॥ \\
\as{ಮಧುವಾತಾ ಋತಾಯತೇ ++++ ಸಂತ್ವೋಷಧೀಃ ॥\\
ತತ್ಪುರುಷಾಯ ವಿದ್ಮಹೇ++++ಪ್ರಚೋದಯಾತ್ ॥}
ಯತ್ಪಾದರೇಣುರ್ವೈ ನಿತ್ಯಂ ಕೋಽಪಿ ಸಂಸಾರವಾರಿಧೌ~।\\
ಸೇತುಬಂಧಾಯತೇ ನಾಥಂ ದೇಶಿಕಂ ತಮುಪಾಸ್ಮಹೇ~॥

ಇಕ್ಷಸಾರಸಮುದ್ಭೂತಾ ಶರ್ಕರಾ ಪುಷ್ಟಿಕಾರಿಕಾ ।\\
ಮಲಾಪಹಾರಿಕಾ ದಿವ್ಯಾ ಸ್ನಾನಾರ್ಥಂ ಪ್ರತಿಗೃಹ್ಯತಾಂ ॥\\
\as{ಸ್ವಾದುಃ ಪವಸ್ವ++++ಅದಾಭ್ಯಃ ॥\\
ಈಶಾನಃ ಸರ್ವವಿದ್ಯಾನಾಂ+++ ಸದಾಶಿವೋಮ್ ॥}
ಅಜ್ಞಾನತಿಮಿರಾಂಧಸ್ಯ ಜ್ಞಾನಾಂಜನಶಲಾಕಯಾ~।\\
ಚಕ್ಷುರುನ್ಮೀಲಿತಂ ಯೇನ ತಸ್ಮೈ ಶ್ರೀಗುರವೇ ನಮಃ~॥

ಸರ್ವಸಾರಸಮುದ್ಭೂತಂ ಶಕ್ತಿಪುಷ್ಟಿಕರಂ ದೃಢಂ ।\\
ಸುಫಲಂ ಕಾರ್ಯಸಿದ್ಧ್ಯರ್ಥಂ ಸ್ನಾನಾರ್ಥಂ ಪ್ರತಿಗೃಹ್ಯತಾಂ ॥\\
\as{ಯಾಃ ಫಲಿನೀ+++++ತ್ವಂ ಹಸಃ ॥\\
ಕದ್ರುದ್ರಾಯ +++++ಹೃದೇ ॥}
ಅಖಂಡಮಂಡಲಾಕಾರಂ ವ್ಯಾಪ್ತಂ ಯೇನ ಚರಾಚರಂ~।\\
ತತ್ಪದಂ ದರ್ಶಿತಂ ಯೇನ ತಸ್ಮೈ ಶ್ರೀಗುರವೇ ನಮಃ~॥

ಮಲಯಾಚಲಸಂಭೂತಂ ಸುಗಂಧಂ ಶೀತಲಂ ಶುಭಂ ।\\
ಸುಕಾಂತಿದಾಯಕಂ ದಿವ್ಯಂ ಸ್ನಾನಾರ್ಥಂ ಪ್ರತಿಗೃಹ್ಯತಾಂ ॥\\
\as{ಗಂಧದ್ವಾರಾಂ++++ಶ್ರಿಯಮ್ ॥} ಗಂಧೋದಕಸ್ನಾನಮ್ ॥

ಲತಾವೃಕ್ಷಸಮುತ್ಪನ್ನಂ ಸುಗಂಧಂ ಶೋಭನಂ ಪರಂ ।\\
ಸಂತೋಷವರ್ಧನಂ ನಿತ್ಯಂ ಸ್ನಾನಾರ್ಥಂ ಪ್ರತಿಗೃಹ್ಯತಾಂ ॥ \\
\as{ಆಯನೇತೇ ++++++ ಇಮೇ ॥}ಪುಷ್ಪೋದಕಸ್ನಾನಮ್ ॥

ಪೃಥಿವೀಗರ್ಭ ಸಂಭೂತ ಸುವರ್ಣೇನ ಯುತಂ ಜಲಂ ।\\
ಆತ್ಮತೇಜಃ ಸ್ವರೂಪಾಯ ಗುರವೇ ಪ್ರದದಾಮ್ಯಹಮ್ ॥\\
\as{ಹಿರಣ್ಯಗರ್ಭಃ ಸಮವರ್ತತಾಗ್ರೇ ++++ವಿಧೇಮ ॥}ಸುವರ್ಣೋದಕಸ್ನಾನಮ್॥
ಶುದ್ಧೋದಕಸ್ನಾಮ್ ॥


ಅಕ್ಷತಾನಿ ಚ ಕರ್ಮಾಣಿ ಕ್ಷತಾನಿ ತವ ಸೇವಯಾ ।\\
ಅಕ್ಷಯಂ ದೇಹಿ ಮೇ ಜ್ಞಾನಂ ಅಕ್ಷತಾಯ ನಮೋ ನಮಃ ॥\\
\as{ಅರ್ಚತ ಪ್ರಾರ್ಚತ+++++ರ್ಚತ}

ಗಂಗಾದಿ ಸರ್ವತೀರ್ಥಾನಾಂ ಪರಿಪಮಳದ್ರವ್ಯ ಮಿಶ್ರಿತಮ್ ।\\
ಭಕ್ತ್ಯಾ ಸಮರ್ಪಿತಂ ತುಭ್ಯಂ ಸ್ನಾನಾರ್ಥಂ ಪ್ರತಿಗೃಹ್ಯತಾಮ್ ॥\\
ಸಂಯುಕ್ತಾಮೇತತ್-----------------------------------------------------------------------
ಪರಿಮಳೋದಕ


ವ್ಯಕ್ತಾವ್ಯಕ್ತಂ ಭವತೇ ವಿಶ್ವಮಸ್-----------------
ಪತ್ರೋದಕಮ್

ತುಳಸೀ ಬಿಲ್ವಮಂದಾರ ಮಲ್ಲಿಕಾ ಮಿಶ್ರಿತಂ ಜಲಮ್ ।\\
ಶಮೀಕುಶಾಭ್ಯಾಂ ಸಹಿತಂ ಸ್ವೀಕುರುಷ್ವ ಜಗದ್ಗುರೋ ॥

ವಿಭೂತಿಭೂಷಣ ಸ್ವಾಮಿನ್ ವಿಭೂತಿಭಿರುಪಾಸಿತ ।\\
ವಿಭೂತಿಂ ತೇ ಪ್ರಯಚ್ಛಾಮಿ ಭಗವನ್ ವರದೋ ಭವ ॥ವಿಭೂತಿಸ್ನಾನ॥

ಬ್ರಹ್ಮಜ್ಞಾನಮಯೀ ಮುದ್ರಾ ವಸ್ತ್ರೇ ಚಾಕಾಶಭೂತಲೇ ।
ಪ್ರಜ್ಞಾನಘನಬೋಧಾಯ ದತ್ತಾತ್ರೇಯ ನಮೋಽಸ್ತು ತೇ ॥
ದಿಗಂಬರಾಯ ಗುರವೇ ಚಿದ್ವಲ್ಕಲಸುಶೋಭಿನೇ ।
ಲೋಕಲಜ್ಜಾನಿವೃತ್ಯರ್ಥಂ ವಸನಂ ಪರಿಧೀಯತಾಮ್ ॥ವಸ್ತ್ರಂ॥


ಶೂಲಹಸ್ತಗದಾಪಾಣೇ ವನಮಾಲಾಸುಕಂಧರ ।
ಯಜ್ಞಸೂತ್ರಧರ ಬ್ರಹ್ಮನ್ ದತ್ತಾತ್ರೇಯ ನಮೋಽಸ್ತು ತೇ ॥
ಸೂಚನಾತ್ ಸೂತ್ರಮಿತ್ಯಾಹುಃ ಸೂತ್ರನಾಮ ಪರಂ ಪದಂ ।
ತತ್ಸೂತ್ರಂ ದೇಹಿ ಮೇ ಬ್ರಹ್ಮನ್ ಬಾಹ್ಯಸೂತ್ರ ಪ್ರದಾನತಃ ॥ಉಪವೀತ॥

ದತ್ತ ವಿದ್ಯಾಢ್ಯ ಲಕ್ಷ್ಮೀಶ ದತ್ತ ಸ್ವಾತ್ಮಸ್ವರೂಪಿಣೇ ।
ಗುಣನಿರ್ಗುಣರೂಪಾಯ ದತ್ತಾತ್ರೇಯ ನಮೋಽಸ್ತು ತೇ ॥
ಮಾಣಿಕ್ಯಮುಕ್ತಾಫಲವಿದ್ರುಮೈಶ್ಚ ಗೋಮೇಧವೈಡೂರ್ಯಕಪುಷ್ಯರಾಗೈಃ ।\\
ಪ್ರವಾಲನೀಲೈಶ್ಚ ಕೃತಂ ಗೃಹಾಣ ದಿವ್ಯಂ ಹಿ ರತ್ನಾಭರಣಂ ಚ ದೇವ ॥


ಯಜ್ಞಭೋಕ್ತ್ರೇ ಚ ಯಜ್ಞಾಯ ಯಜ್ಞರೂಪಧರಾಯ ಚ ।
ಯಜ್ಞಪ್ರಿಯಾಯ ಸಿದ್ಧಾಯ ದತ್ತಾತ್ರೇಯ ನಮೋಽಸ್ತು ತೇ ॥
ವಿಭೂತಿಭೂಷಣ ಸ್ವಾಮಿನ್ ವಿಭೂತಿಭಿರುಪಾಸಿತ ।\\
ವಿಭೂತಿಂ ತೇ ಪ್ರಯಚ್ಛಾಮಿ ಭಗವನ್ ವರದೋ ಭವ ॥ವಿಭೂತಿ

ಅವಧೂತ ಸದಾನಂದ ಪರಬ್ರಹ್ಮಸ್ವರೂಪಿಣೇ ।
ವಿದೇಹದೇಹರೂಪಾಯ ದತ್ತಾತ್ರೇಯ ನಮೋಽಸ್ತು ತೇ ॥
ಚಂದನಾಗರು ಕಸ್ತೂರೀ ರೋಚನಂ ಕುಂಕುಮಾನ್ವಿತಮ್ ।
ಗಂಧಂ ದಾಸ್ಯಾಮಿ ತೇ ಯೋಗಿನ್ ಲೇಪನಮ್ ಪ್ರತಿಗೃಹ್ಯತಾಮ್ ॥

ಕ್ಷರಾಕ್ಷರಸ್ವರೂಪಾಯ ಪರಾತ್ಪರತರಾಯ ಚ ।
ದತ್ತಮುಕ್ತಿಪರಸ್ತೋತ್ರ ದತ್ತಾತ್ರೇಯ ನಮೋಽಸ್ತು ತೇ ॥
ಹರಿದ್ರಾಂ ಕುಂಕುಮಂ ದಿವ್ಯಂ ಸರ್ವಸೌಭಾಗ್ಯಸೂಚಕಮ್ ।
ಮಯಾ ನಿವೇದಿತಂ ಭಕ್ತ್ಯಾ ಸ್ವೀಕುರುಷ್ವ ಜಗತ್ಪ್ರಭೋ ॥

ಶತ್ರುನಾಶಕರಂ ಸ್ತೋತ್ರಂ ಜ್ಞಾನವಿಜ್ಞಾನದಾಯಕಮ್ ।
ಸರ್ವಪಾಪಂ ಶಮಂ ಯಾತಿ ದತ್ತಾತ್ರೇಯ ನಮೋಽಸ್ತು ತೇ ॥
ಸಿಂದೂರಂ ನಾಗಸಂಭೂತಂ ಫಾಲಶೋಭಾವಿವರ್ಧನಮ್ ।
ಪೂರಣಂ ಭೂಷಣಾನಾಂಚ ಸಿಂದೂರಂ ಪ್ರತಿಗೃಹ್ಯತಾಮ್ ॥----

ದತ್ತಂ ಸನಾತನಂ ನಿತ್ಯಂ ನಿರ್ವಿಕಲ್ಪಂ ನಿರಾಮಯಮ್ ।
ಹರಿ ಶಿವಂ ಮಹಾದೇವಂ ಸರ್ವಭೂತೋಪಕಾರಕಂ ॥
ತಸ್ಮೈ ದತ್ತಂ ನಮೊಽಸ್ತು ತೇ -------
ಸೇವಂತಿಕಾ ಬಕುಲ ಚಂಪಕ ಪಾಟಲಾಬ್ಜೈಃ ಪುನ್ನಾಗಜಾಜಿ ಕರವೀರಕ ಸಾಲಪುಷ್ಪೈಃ ।
ಬಿಲ್ವಪ್ರವಾಲ ತುಲಸೀ ದಲ ಮಲ್ಲಿಕಾದ್ಯೈಃ ತ್ವಾಂ ಪೂಜಯಾಮಿ ಭಗವನ್ ಗುರು ಪ್ರಸೀದ ॥

ನಮೋ ನಾರಾಯಣಂ ಮಹಾವಿಷ್ಣುಂ ಸರ್ಗಸ್ಥಿತ್ಯಂತಕಾರಣಂ ।
ನಿರಾಕಾರಂಚ ಸರ್ವೇಷಾಂ ಕಾರ್ತವೀರ್ಯ ವರಪ್ರದಂ॥----------------
ತಸ್ಮೈ ದತ್ತಾತ್ರೇಯ ನಮೋಽಸ್ತು ತೇ

ತುಳಸೀ ಬಿಲ್ವಮಂದಾರ ನಾನಾಪುಷ್ಪಯುತೈಃ ಕೃತಂ ।
ದಿವ್ಯಗಂಧ ವಿಮಲಂ ಸ್ವೀಕುರುಷ್ವ ದಯಾನಿಧೇ ॥
ಅತ್ರಿಪುತ್ರಂ ಮಹಾತೇಜಂ----- ಮುನಿವಂದ್ಯಂ ಜನಾರ್ದನಂ ।
ದ್ರಾಂ ಬೀಜಂ ವರದಂ ಶುದ್ಧಂ ಹ್ರೀಂ ಬೀಜೇನ ಸಮನ್ವಿತಮ್ ॥

ದತ್ತಾತ್ರೇಯಾಯ ನಮಃ । ಪಾದೌ ಪೂಜಯಾಮಿ॥\\
ಜಗದ್ಗುರವೇ ನಮಃ । ಗುಲ್ಫೌ ಪೂಜಯಾಮಿ॥\\
ಅತ್ರಿಪುತ್ರಾಯ ನಮಃ । ಜಂಘೇ ಪೂಜಯಾಮಿ॥\\
ಅನಸೂಯಾಸೂನವೇ ನಮಃ । ಜಾನುನೀ ಪೂಜಯಾಮಿ॥\\
ನಿರಾಕಾರಾಯ ನಮಃ । ಊರೂ ಪೂಜಯಾಮಿ॥\\
ಮುನಿವಂದ್ಯಾಯ ನಮಃ । ಕಟಿಂ ಪೂಜಯಾಮಿ॥\\
ಕಲಿದೋಷಹರಾಯ ನಮಃ । ಉದರ ಪೂಜಯಾಮಿ॥\\
ಶಾಶ್ವತಾಯ ನಮಃ । ನಾಭಿಂ ಪೂಜಯಾಮಿ॥\\
ಬ್ರಹ್ಮಸ್ವರೂಪಾಯ ನಮಃ । ವಕ್ಷಸ್ಥಲಂ ಪೂಜಯಾಮಿ॥\\
ವಿಷ್ಣುಸ್ವರೂಪಾಯ ನಮಃ । ಬಾಹೂನ್ ಪೂಜಯಾಮಿ॥\\
ಶಿವಸ್ವರೂಪಾಯ ನಮಃ । ಹಸ್ತಾನ್ ಪೂಜಯಾಮಿ॥\\
ವಿಶ್ವವಂದ್ಯಾಯ ನಮಃ । ಕಂಠಂ ಪೂಜಯಾಮಿ॥\\
ಸರ್ವಾಭೀಷ್ಟಪ್ರದಾಯ ನಮಃ । ಮುಖಂ ಪೂಜಯಾಮಿ॥\\
ಕರುಣಾಕರಾಯ ನಮಃ । ಓಷ್ಠ ಪೂಜಯಾಮಿ॥\\
ನಿರಂಜನಾಯ ನಮಃ । ದಂತಾನ್ ಪೂಜಯಾಮಿ॥\\
ದಿಗಂಬರಾಯ ನಮಃ । ಸ್ಮಿತಂ ಪೂಜಯಾಮಿ॥\\
ಮಹಾಬಾಹವೇ ನಮಃ । ನಾಸಿಕಾಂ ಪೂಜಯಾಮಿ॥\\
ಚಂದ್ರಾದಿತ್ಯಸ್ವರೂಪಾಯ ನಮಃ । ನೇತ್ರಾಣಿ ಪೂಜಯಾಮಿ॥\\
ಭಕ್ತಾಭೀಷ್ಟಪ್ರದಾಯ ನಮಃ । ಕರ್ಣೌ ಪೂಜಯಾಮಿ॥\\
ದುಃಖಹರಾಯ ನಮಃ । ಲಲಾಟಂ ಪೂಜಯಾಮಿ॥\\
ಭಸ್ಮಧಾರಿಣೇ ನಮಃ । ಶಿರಃ ಪೂಜಯಾಮಿ॥\\
ಓಂಕಾರರೂಪಾಯ ನಮಃ । ಸರ್ವಾಂಗಂ ಪೂಜಯಾಮಿ॥




ಭಿಕ್ಷಾಟನಂ ಗೃಹೇ ಗ್ರಾಮೇ ಪಾತ್ರಂ ಹೇಮಮಯಂ ಕರೇ ।
ನಾನಾಸ್ವಾದಮಯೀ ಭಿಕ್ಷಾ ದತ್ತಾತ್ರೇಯ ನಮೋಽಸ್ತು ತೇ ॥ 10 ॥

ಅವಧೂತ ಸದಾನಂದ ಪರಬ್ರಹ್ಮಸ್ವರೂಪಿಣೇ ।
ವಿದೇಹದೇಹರೂಪಾಯ ದತ್ತಾತ್ರೇಯ ನಮೋಽಸ್ತು ತೇ ॥ 12 ॥

ಸತ್ಯರೂಪ ಸದಾಚಾರ ಸತ್ಯಧರ್ಮಪರಾಯಣ ।
ಸತ್ಯಾಶ್ರಯಪರೋಕ್ಷಾಯ ದತ್ತಾತ್ರೇಯ ನಮೋಽಸ್ತು ತೇ ॥ 13 ॥

ಇದಂ ಸ್ತೋತ್ರಂ ಮಹದ್ದಿವ್ಯಂ ದತ್ತಪ್ರತ್ಯಕ್ಷಕಾರಕಮ್ ।
ದತ್ತಾತ್ರೇಯಪ್ರಸಾದಾಚ್ಚ ನಾರದೇನ ಪ್ರಕೀರ್ತಿತಮ್ ॥ 18 ॥ 
 



