\titlespacing{\section}{0pt}{*-5}{*-4}

\titleformat*{\section}{\large\bfseries}
\renewcommand*\contentsname {\Large विषयानुक्रमणिका}

\frontmatter
\title{॥परिभाषेन्दुशेखरः॥}
\author{ महामहोपाध्यायश्रीनागोजीभट्टविरचितः }
\date{}
	\begin{titlepage}
	\vfill
	\vfill
		\centering
		\maketitle
	\end{titlepage}
\thispagestyle{empty}
\fancyhead[LE]{\thepage}
\fancyhead[RO]{\thepage}
\fancyhead[RE]{विषयानुक्रमणिका}
\fancyhead[LO]{परिभाषेन्दुशेखरः}
\tableofcontents
\begin{figure}[b]
\centering
\includegraphics[width=4cm]{page-divider}
\end{figure}
\newpage
\thispagestyle{empty}
\mainmatter
\fancyhead[LE]{\thepage}
\fancyhead[RO]{\thepage}
\fancyhead[RE]{मङ्गलम्}
\fancyhead[LO]{परिभाषेन्दुशेखरः}
\renewcommand{\thepage}{\devanagarinumeral{page}}
\section*{\begin{center}मङ्गलम् \end{center}}
\addcontentsline{toc}{section}{मङ्गलम्}
\begin{figure}[h]
\centering
\includegraphics[width=10cm]{Lord-Shiva-And-Parvathi-HD-Wallpaper}

\end{figure}
\begin{center}
{\large\bfseries  नत्वा साम्बं शिवं ब्रह्म नागेशः कुरुते सुधीः~।\newline बालानां सुखबोधाय परिभाषेन्दुशेखरम्~॥}
\end{center}
 प्राचीनव्याकरणतन्त्रे वाचनिकान्यत्र पाणिनीयतन्त्रे ज्ञापकन्यायसिद्धानि भाष्यवार्तिकयोरुपनिबद्धानि यानि परिभाषारूपाणि तानि व्याख्यास्यन्ते~।\\
\newpage
\thispagestyle{empty}
\part*{॥अथ शास्त्रत्वसम्पादनोद्देशनामकं प्रथमं प्रकरणम्॥} 
\addcontentsline{toc}{part}{शास्त्रत्वसम्पादनोद्देशप्रकरणम् }

\fancyhead[RE]{प्रथमं प्रकरणम्}

ननु लण्-अ‍इउण्-सूत्रयोर्णकारद्वयस्यैवोपादानेनाणिण्ग्रहणेषु सन्देहादनिर्णयोऽत आह-
\section*{\begin{center}व्याख्यानतो विशेषप्रतिपत्तिर्न हि सन्देहादलक्षणम्॥१॥\end{center}}
\addcontentsline{toc}{section}{व्याख्यानतो विशेष}
विशेषस्यान्यतराद्यर्थरूपस्य व्याख्यानाच्छिष्टकृतात्प्रतिपत्तिः निश्चयो, यतः  सन्देहाच्छास्त्रमलक्षणमननुष्ठापकं   लक्षणमलक्षणम् , तथा न,  शास्त्रस्य निर्णयजनकत्वौचित्यादित्यर्थः~।
असन्दिग्धानुष्ठानसिद्ध्यर्थेऽत्र शास्त्रे सन्दिग्धोच्चारणरूपाचार्यव्यवहारेण सन्देहनिवृत्तेर्व्याख्यानातिरिक्तनिमित्तानपेक्षत्वं बोध्यत इति यावत्~।
तेनाणुदित्सवर्णस्येत्येतत्परिहाय पूर्वेणाण्ग्रहणम्~, परेणेण्ग्रहणमिति लण्सूत्रे भाष्ये स्पष्टम्~॥१॥\par
तत्र संज्ञापरिभाषाविषये पक्षद्वयमाह-
\section*{\begin{center} यथोद्देशं संज्ञापरिभाषम्~।
कार्यकालं संज्ञापरिभाषम्॥२॥\end{center}}
\addcontentsline{toc}{section}{यथोद्देशं}
उद्देशमनतिक्रम्य यथोद्देशम्~।
उद्देश उपदेशदेशः~।
अधिकरणसाधनश्चायम्~।
यत्र देशे उपदिश्यते तद्देश एव वाक्यार्थबोधेन गृहीतशक्त्या गृहीतपरिभाषार्थेन च सर्वत्र शास्त्रे व्यवहारः~।
देशश्चोच्चारणकाल एवात्र व्यवह्रियते~।
तत्तद्वाक्यार्थबोधे जाते `भविष्यति किञ्चिदनेन प्रयोजनम्' इति ज्ञानमात्रेण सन्तुष्यद्यथाश्रुतग्राहिप्रतिपत्रपेक्षोऽयं पक्ष इति ईदूदेत्सूत्रे कैयटः~।
केचित्तु परिभाषाविषये तस्मिन्नित्यादिवाक्यार्थबोधे सप्तमीनिर्देशादि क्वेति पर्यालोचनायां सकलतत्तद्विद्ध्युपस्थितौ सकलतत्तसंस्काराय गुणभेदं परिकल्प्यैकवाक्यतयैव नियमः~।
कार्यकालपक्षे तु त्रिपाद्यामप्युपस्थितिरिति विशेषः~।
एतदेवाभिप्रेत्य {\bfseries अधिकारो नाम त्रिप्रकारः, कश्चिदेकदेशस्थः सर्वं शास्रमभिज्वलयति, यथा प्रदीपः सुप्रज्वलितः सर्वं वेश्माभिज्वलयति} इति ``षष्ठी स्थाने-'' इति सूत्रे भाष्य उक्तम्~।
अधिकारशब्देन पारार्थ्यात्परिभाषाप्युच्यते~।
`कश्चित्परिभाषारूप' इति कैयटः~।
दीपो यथा प्रभाद्वारा सर्वगृहप्रकाशक एवमेतत्स्वबुद्धिजननद्वारा सर्वशास्त्रोपकारकमिति तत्तात्पर्यम्~।
एतच्च पक्षद्वयसाधारणं भाष्यम्~, पक्षद्वयेऽपि प्रदेशैकवाक्यतया इतः प्रतीतेः~॥\par
तत्रैतावान्विशेषः यथोद्देशे परिभाषादेशे सर्वविधिसूत्रबुद्धावात्मभेदं परिकल्प्य तैरेकवाक्यता परिभाषाणाम्~।
तदुक्तं `ग्क्ङिति च' इति सूत्रे कैयटे~। `यथोद्देशे प्रधानान्यात्मसंस्काराय सन्निधीयमानानि गुणभेदं प्रयुञ्जत' इति~।
कार्यकाले तु तत्तद्विधिप्रदेशे परिभाषाबुद्ध्यैकवाक्यतेति~।
अत्रैकदेशस्थ इत्यनेन तत्र तत्र तत्तद्बुद्धावपि तत्तद्देशस्थत्वं वारयति, यथा व्यवहर्तॄणां कार्यार्थमनेकदेशगमनेऽपि न तत्तद्देशीयत्वव्यवहारः, किन्त्वभिजनदेशीयत्वव्यवहार एव, तद्वन्निषेधवाक्यानामपि निषेध्यविशेषाकाङ्क्षत्वाद्विध्येकवाक्यतयैवान्वय इति परिभाषासादृश्यात्परिभाषात्वेन व्यवहारः ``क्ङिति च'' इत्यत्र भाष्ये~।
तत्रैकवाक्यता पर्युदासन्यायेन ।
प्रसज्यप्रतिषेधेऽपि तेन सह वाक्यार्थबोधमात्रेणैकवाक्यता व्यवहारः॥संज्ञाशास्त्रस्य तु कार्यकालपक्षे न पृथग्वाक्यार्थबोधः, किन्तु प्रदेशवाक्येन सहैव~।
अत एवाणोऽप्रगृह्यस्येत्येतदेकवाक्यतापन्नम् ``अदसो मात्'' इत्येतत्प्रति न मुत्वाद्यसिद्धम्~, असिद्धत्वस्य कार्यार्थतया कार्यज्ञानोत्तरमेव तत्प्रवृत्तिः, कार्यज्ञानं च प्रदेशदेश एवेति तद्देशस्थासिद्धत्वात्पूर्वग्रहणेनाग्रहणात्~।
एवं तद्बोधोत्तरमेव विरोधप्रतिसन्धानञ्चेति तत्रत्यपरत्वमेव विप्रतिषेधसूत्रप्रवृत्तौ बीजम्~।
अत एव कार्यकालपक्षे ``अयादिभ्यः परैव प्रगृह्यसंज्ञा'' इति ``अदसो मात्'' इति सूत्रे भाष्य उक्तम्~॥ आकडाराधिकारस्थभपदसंज्ञाविषये तु यथोद्देशपक्ष एवेति तत्रत्यपरत्वेनैव बाध्यबाधकभावः~।
पदादिपदानां तत्र जातशक्तिग्रहेणैव त्रिपाद्यामपि व्यवहारः~।
अत एव पूर्वत्रासिद्धमिति सूत्रे परिभाषाणामेव त्रिपाद्यामप्रवृत्तिमाशङ्क्य~, कार्यकालपक्षाश्रयणेन समाहितमित्याहुः~।
यथोद्देशपक्षः प्रगृह्यसंज्ञाप्रकरणे भाष्ये~॥
कार्यकालमित्यस्य च कार्येण काल्यते स्वसन्निधिं प्राप्यत इत्यर्थः~।
कार्येण स्वसंस्काराय स्ववृत्तिलिङ्गचिह्नितपरिभाषाणामाक्षेप इति यावत्~।
अत एव ``पूर्वत्रासिद्धम्'' इति सूत्रे भाष्ये त्रिपाद्या असिद्धत्वात्तत्र सपादसप्ताध्यायीस्थपरिभाषाणामप्रवृत्तिमाशङ्क्य यद्यपीदं तत्रासिद्धं तत्त्विह सिद्धमित्युक्त्वा, तावताप्यसिद्धिरित्यभिप्रायके कथमिति प्रश्ने, `कार्यकालं संज्ञापरिभाषम्' यत्र कार्यं तत्रोपस्थितं द्रष्टव्यमित्युक्तम्~।
न च कार्यकालपक्षे ``ङमो ह्रस्वात्'' इत्यदौ ``तस्मादित्युत्तरस्य'' ``तस्मिन्निति निर्दिष्टे पूर्वस्य'' इति परिभाषाद्वयोपस्थितौ परत्वात् `उभयनिर्देशे पञ्चमीनिर्देशो बलीयान्' इति तस्मिन्निति सूत्रस्थभाष्यासङ्गतिः, उभयोरेकदेशस्थत्वेन परत्वादित्यस्यासङ्गत्यापत्तेः, स्पष्टं चेदम् ``इको गुण'' इत्यत्र कैयट इति वाच्यम् ।
विप्रतिषेधसूत्रेऽष्टाध्यायीपाठकृतपरत्वस्याश्रयणेनादोषात्~।
न हि कार्यकालपक्ष इत्येतावता तदपैति~।
पक्षद्वयेऽपि प्रदेशेषु स्वबुद्धिजननाविशेषात्~।
न हि तत्पक्षेऽप्यचेतनस्य शास्त्रस्य स्वदेशं विहाय तद्देशगमनं सम्भवति~।
नाप्यस्मदादिबुद्धिजननेन स्वदेशत्यागो भवति~।
अत एव भाष्ये एकदेशस्थस्यैव सर्वशास्त्राभिज्वालकत्वमुक्तम् ।
अत एव तस्मिन्निति सूत्रे कैयटः सूत्रपाठापेक्षया परत्वस्य व्यवस्थापकत्वमिति~।
इको गुणेतिसूत्रस्थकैयटस्तु चिन्त्य एव ; अन्यथा सर्वशास्त्राणां प्रयोगार्थत्वेन प्रयोगरूपैकदेशस्थत्वेन क्वापि परत्वं न स्यात्~।
किञ्च क्ङिति चेति सूत्रस्थ कैयटरीत्या विधिसूत्राणां यथोद्देशपक्षे परिभाषादेशे सन्निधानेन तेषां परत्वं व्याहन्येत~।
एवञ्च वृक्षेभ्य इत्यत्र सुपि चेत्यतः परत्वात् ``बहुवचने झल्येत्'' इत्येत्वमित्याद्युच्छिद्येत, इत्यलम्॥२॥\par
इत्संज्ञका अनुबन्धाः, तेष्ववयवानवयवत्वसन्देह आह-
\section*{\begin{center}अनेकान्ताः अनुबन्धाः~॥३॥\\ \end{center}}
\addcontentsline{toc}{section}{अनेकान्ताः}
अनेकान्ता अनवयवाः इत्यर्थः~।
यो ह्यनवयवः स कदाचित्तत्रोपलभ्यत एव, अयं तु न तथा, तदर्थभूते विधेये कदाप्यदर्शनात्~।
शित्किदित्यादौ समीपेऽवयवत्वारोपेण समासो बोध्यः~।
``वुञ्छण्कठ-'' इत्यादौ णित्वप्रयुक्तं कार्यं पूर्वस्यैवेत्यादि तु व्याख्यानतो निर्णेयम्~।
``हलन्त्यम्'' इत्यत्रान्त्यशब्दः परसमीपबोधकः~॥३॥\par
वस्तुतस्तु --
\section*{\begin{center}एकान्ताः॥४॥\end{center}}
\addcontentsline{toc}{section}{एकान्ताः}
इत्येव न्याय्यम्~, शास्त्रे तत्रोपलम्भादन्यत्रानुपलम्भाच्च~।
अनवयवो हि काकादिरेकजातीयसम्बन्धेन गृहवृक्षादिषूपलभ्यते~।
नैवमयम्~।
एवं हि बहुव्रीहिरपि न्यायत एवोपपन्नः, अन्त्यादिशब्दे लक्षणा च न~।
किञ्च अनवयवत्वे णशकप्रत्ययादौ णादेरित्वानापत्तिः, प्रत्ययादित्वाभावात्~, दध्नचश्चकारस्य वैयर्थ्यापत्तेश्च~।
इदञ्च ``तस्य लोपः'' इत्यत्र भाष्ये स्पष्टम्~।
तत्र ह्युक्तम्~,``एकान्ता अनुबन्धाः'' इत्येव न्याय्यमिति दिक्~॥४॥\par
 नन्वेकान्तत्वेऽनेकाल्त्वादेव औशादीनां सर्वादेशत्वसिद्ध्या ``अनेकाल्'' सूत्रे शिद्ग्रहणं व्यर्थमत आह-

\section*{\begin{center}नानुबन्धकृतमनेकाल्त्वम्~॥५॥\end{center}}
\addcontentsline{toc}{section}{नानुबन्धकृतम्}
शिद्ग्रहणमेवैतज्ज्ञापकम्~, तेन ``अर्वणस्तृ'' इत्यादेर्न सर्वादेशत्वम्~।
डादिविषये तु सर्वादेशत्वं विनानुबन्धत्वस्यैवाभावेनऽऽनुपूर्व्यात्सिद्धम्~॥५॥\par
नन्वेवमपि `अवदातं मुखम्' इत्यत्र पलोपोत्तरमात्वे कृतेऽदाबिति घुसंज्ञाप्रतिषेधो न स्यात्~, दैपः पकारसत्वेऽनेजन्तत्वादात्वाप्राप्त्या पलोपोत्तरं पकाराभावेनास्य दाप्त्वाभावादत आह-
\section*{\begin{center}नानुबन्धकृतमनेजन्तत्वम्~॥६॥\end{center}}
\addcontentsline{toc}{section}{नानुबन्धकृतमनेजन्तत्वम्}
``उदीचामाङः'' इति निर्देशोऽस्या ज्ञापकः~।
``आदेच उपजेशे'' इति सूत्रेणोपदिश्यमानस्यैजन्तस्यात्वं क्रियते, ङकारसत्वे त्वेजन्तत्वाभावादात्वाप्राप्तेस्तस्यासङ्गतिः~।
न चास्यामवस्थायां तस्य धातुत्वाभावात्कथमात्वम् ? तत्र धातोरित्यस्य निवृत्तेरित्यन्यत्र विस्तरः~।
स्पष्टञ्चेदमं ``दाधा घ्वदाप्'' इति सूत्रे भाष्ये~॥६॥\par
 नन्वेवमपि ``वासरूप'' इति सूत्रेण कविषयेऽणोऽप्यापत्तिरित्यत आह-
\section*{\begin{center}नानुबन्धकृतमसारूप्यम्~॥७॥\end{center}}
\addcontentsline{toc}{section}{नानुबन्धकृतमसारूप्यम्}
``ददातिदधात्योर्विभाषा'' इति णबाधकशस्य विकल्पविधायकमस्यां ज्ञापकम्~, तेन `गोदः' इत्यादौ नाणिति वासरूपसूत्रे भाष्ये स्पष्टम्~॥७॥\par
 ननु संख्याग्रहणे बह्वादीनामेव ग्रहणं स्यात्~, प्रकरणस्याभिधानियामकत्वसिद्धात् `कृत्रिमाकृत्रिमयोः कृत्रिमे कार्यसम्प्रत्ययः' इति न्यायात्~।
अस्ति च प्रकृते बह्वादीनां सङ्ख्यासंज्ञा कृतेति ज्ञानरूपं प्रकरणम्~, न तु लोकप्रसिद्धैकद्व्यादीनामित्यत आह-
\section*{\begin{center}उभयगतिरिह भवति~॥८॥\end{center}}
\addcontentsline{toc}{section}{उभयगतिरिह}
इह - शास्त्रे~।
``सङ्ख्याया अतिशदन्तायाः'' इति निषेधोऽस्या ज्ञापकः~।
न हि कृत्रिमा सङ्ख्या त्यन्ता शदन्ता वास्ति, तेन ``कर्तरि कर्मव्यतिहारे'', ``कण्वमेघेभ्यः करणे'', ``विप्रतिषिद्धं चानधिकरण'', इत्यादौ लौकिकक्रियाद्रव्याद्यवगतिः~।
तत्र क्वोभयगतिः? क्वाकृत्रिमस्यैव ? क्व कृत्रिमस्यैवेत्यत्र लक्ष्यानुसारि व्याख्यानमेव शरणम्~।
अत एव आम्रेडितशब्देन कृत्रिमस्यैव ग्रहणं न तु द्विस्त्रिर्घुष्टमात्रस्य~।
स्पष्टञ्चेदं सङ्ख्यासंज्ञासूत्रे भाष्ये~।
यत्तु संज्ञाशास्त्राणां मच्छास्त्रेऽनेन शब्देनैत एवेति नियमार्थत्वं कृत्रिमाकृत्रिमन्यायबीजमिति~।
तन्न~।
तेषामगृहीतशक्तिग्राहकत्वेन विधित्वे सम्भवति, नियमत्वायोगात्~।
`सर्वे सर्वार्थवाचकाः' इत्यभ्युपगमोऽपि योगिदृष्ट्या, न त्वस्मदादिदृष्ट्या, विशिष्य सर्वशब्दार्थज्ञानस्याशक्यत्वात्~।
सामान्यज्ञानं तु न बोधोपयोगीत्यन्यत्र निरूपितम्~॥८॥\par
 ननु `अध्येता',`शयिता' इत्यादाविङ्‍-शीङोर्ङित्वाद्गुणनिषेधः स्यादत आह-
\section*{\begin{center}कार्यमनुभवन्हि कार्यी निमित्ततया नाश्रीयते॥९॥\end{center}}
\addcontentsline{toc}{section}{कार्यमनुभवन्हि}
``स्थण्डिलाच्छयितरि'' इति निर्देशश्चास्यां ज्ञापकः~।
`ऊर्णुनविषति' इत्यादिसिद्धये कार्यमनुभवन्निति~।
अत्र हि ``द्विर्वचनेऽचि'' इति नुशब्दस्य द्वित्वम्~, अन्यथा ``सन्यङोः'' इत्यस्य षष्ठ्यन्तत्वात्सन्नन्तस्य कार्यित्वेन इसो द्वित्वनिमित्तत्वाभावात्तत्प्रवृत्तिर्न स्यात्~।
वस्तुतः समवायिकारणनिमित्तकारणयोर्भेदस्य सकललोकतन्त्रप्रसिद्धतया तस्य तत्वेनाऽश्रयणाभावेन नैषा ज्ञापकसाध्या~।
अत एव हि प्रयुक्तः~।
स हि तत्वेनानाश्रयणे हेतोः प्रसिद्धत्वं द्योतयतीति तत्वम्~।
``द्विर्वचनेऽचि'' इत्यत्र भाष्ये ध्वनितैषा~॥९॥\par
ननु `प्रणिदापयति' इत्यादौ दारूपस्य विधीयमाना घुसंज्ञा दापेर्न स्यादत आह-
\section*{\begin{center}यदागमास्तद्गुणीभूतास्तद्ग्रहणेन गृह्यन्ते~॥१०॥\end{center}}
\addcontentsline{toc}{section}{यदागमास्तद्गुणीभूताः}
यमुद्दिश्यागमो विहितः, स तद्गुणीभूतः शास्त्रेण तदवयवत्वेन बोधितोऽतस्तद्ग्रहणेन तद्ग्राहकेण तद्बोधकेन शब्देन गृह्यते बोध्यत इत्यर्थः~।
तत्र तद्गुणीभूता इत्यंशो बीजकथनम्~।
लोकेऽपि देवदत्तस्याङ्गाधिक्ये तद्विशिष्टस्यैव देवदत्तग्रहणेन ग्रहणं दृश्यते~।
यमुद्दिश्य विहित इत्युक्तेः `प्रनिदारयति' इत्यादौ न दारित्यस्य घुत्वम्~।
``आने मुक्'' इति मुग्विधानसामर्थ्यादेषानित्या~।
अन्यथा `पचमानः' इत्यादावकारस्य मुकि अनया परिभाषया विशिष्टस्य सवर्णदीर्घे तद्वैयर्थ्यं स्पष्टमेव~।
तेन `दिदीये' इत्यादौ यणादि न, `जहार' इत्यादौ ``आत औ णलः'' इति च न~।
न चाकारादेर्वर्णस्य वर्णान्तरमवयवः कथमिति वाच्यम्~, वचनेनावयवत्वबोधनात्~।
तस्य चावयवत्वसादृश्ये पर्यवसानं बोध्यम्~।
न चोक्तज्ञापकाद्वर्णग्रहणेऽस्या अप्रवृत्तिरिति वाच्यम् ~। ``आने मुक्'' इति सूत्रभाष्येऽकारस्याङ्गावयवस्य मुगित्यर्थे `पचमानः' इत्यत्र
``तास्यनुदात्तेत्'' इति स्वरो न स्यादित्याशङ्क्यादुपदेशभक्तस्तद्ग्रहणेन ग्राहिष्यत इत्युक्तेरसङ्गत्यापत्तेः~।
किञ्च ङमन्तपदावयवस्य ह्रस्वात्परस्य ङमो ङमुडित्यर्थे `कुर्वन्नास्ते' इत्यादौ ङमो ङमुडागमे णत्वप्राप्तिमाशङ्क्य
`यदागमाः' इति न्यायेनाद्यनस्यापि पदान्तग्रहणेन ग्रहणात् ``पदान्तस्य'' इति निषेध इत्यनया परिभषया
`आगमानामागमिधर्मवैशिष्ट्यमपि बोध्यते' इत्याशयक-ङमुट्‍सूत्रस्थभाष्यासङ्गतेः~।
किञ्च गुणादे रपरत्वे रेफविशिष्टे गुणत्वाद्येष्टव्यम्~।
अन्यथा `ऋकारस्य गुणवृद्धी अरारावेव' इति नियमो न स्यात्~।
तच्च वर्णग्रहणे एतदप्रवृत्तौ न सङ्गच्छते~।
अत एव ``रदाभ्यां'' इति सूत्रे भाष्यं `गुणो भवति, वृद्धिर्भवतीति, रेफशिरा गुणवृद्धिसंज्ञकोऽभिनिर्वर्तते' इति~।
अत एव ``नेटि'' ``णेरनिटि'' इत्यदि चरितार्थम्~।
`अनागमकानां सागमकाः आदेशाः' इत्यस्य त्वयमर्थः आर्धधातुकस्येडागम इत्यर्थे ज्ञाते, नित्येषु शब्देष्वागमविधानानुपपत्त्या अर्थापत्तिमूलकवाक्यान्तरकल्पनेनेड्रहितबुद्धिप्रसङ्गे सेड्बुद्धिः कर्तव्येति~।
एवञ्चादेशेष्विवात्रापि बुद्धिविपरिणाम इति न नित्यत्वहानिः~।
स्थानिवत्सूत्रे च नेदृशादेशग्रहणम्~।
साक्षादष्टाध्यायीबोधितस्थान्यादेशभावे चारितार्थ्यात्~।
किञ्चैवं सति स्थानिबुद्ध्यैव कार्यप्रवृत्या `लावस्थायामट्' इति सिद्धान्तासङ्गतिः~।
स्थानिवद्भावविषये `निर्दिश्यमानस्य' इति परिभाषायाः प्रवृत्तौ `तिसृणाम्' इत्यत्र परत्वात्तिस्रादेशे स्थानिवद्भावेन त्रयादेशमाशङ्ख्य, सकृद्गतिन्यायेन समाधानपरभाष्यासङ्गतिः~।
``एरुः'' इत्यादौ स्थानषष्ठीनिर्देशात्तदन्तपरतया पठितवाक्यस्यैव समुदायादेशपरत्वेनाऽऽदेशग्रहणसामर्थ्यात्तस्य स्थानिवत्सूत्रे ग्रहणेन न दोषः~।
आनुमानिकस्थान्यादेशभावकल्पनेऽपि श्रौतस्थान्यादेशभावस्य न त्याग इति ``अचः परस्मिन्'' इत्यादेर्नासङ्गतिः~।
एतेन `यदागमाः' इति परिभाषा स्थानिवत्सूत्रेण गतार्थेत्यपास्तम्~।
एतत्सर्वं ``दाधाघ्वदाप्'' इति सूत्रे भाष्ये स्पष्टम्॥१०॥\par
नन्वेवम् उदस्थादित्यादौ ``उदःस्थास्तम्भोः पूर्वस्य'' इति पूर्वसवर्णापत्तिरत आह-
\section*{\begin{center}निर्दिश्यमानस्याऽदेशा भवन्ति॥११॥\end{center}}
\addcontentsline{toc}{section}{निर्दिश्यमानस्य}
``षष्ठी स्थानेयोगा'' इति सूत्रमावर्तते~।
तत्र द्वितीयस्यायमर्थः - षष्ठ्यन्तं निर्दिश्यमानमुच्चार्यमाणमुच्चार्यमाणसजातीयमेव, निर्दिश्यमानावयवरूपमेव वा स्थानेन स्थाननिरूपितसम्बन्धेन युज्यते, न तु प्रतीयमानमित्यर्थः~।
तेनेदं सिद्धम्~।
न च ``अस्य च्वौ'' इत्यादौ दीर्घाणामादेशानापत्तिः, तेषां निर्दिश्यमानत्वाभावादिति वाच्यम्~, जातिपक्षे दोषाभावात्~।
किञ्च ``न भूसुधियोः'' इति निषेधेन ग्रहणकशास्त्रगृहीतानां निर्दिश्यमानकार्यबोधान्न दोषः~।
इयङुवङोर्ङित्त्वं तु इवर्णोवर्णान्तश्नुधातुभ्रुवामित्यर्थेन धात्वादीनामपि निर्दिष्टत्वादन्त्यादेशत्वाय~।
रीङ्रिङोर्ङित्त्वं तु स्पष्टार्थमेव~।
एतेनेदं ङित्त्वं वर्णग्रहणे निर्दिश्यमानपरिभाषाया अप्रवृत्तिज्ञापकमित्यपास्तम्~।
``हयवरट्'' सूत्रस्थेन  {\bfseries अयोगवहानामुपदेशेऽलोन्त्यविधिः प्रयोजनम्, वृक्षस्त्तत्र~।
नैतदस्ति प्रयोजनम्~, निर्दिश्यमानस्येत्येव सिद्धम्} इति भाष्येण विरोधात्~।
अनया परिभाषया ``येन विधिः'' इति सूत्रबोधिततदन्तस्य स्थानिवत्वाभाव बोधनम्~, `यदागमा-' इति लब्धस्य च~।
तेन `सुपदः', `उदस्थात्' इत्यादिसिद्धिः~।
अनया च स्वस्वनिमित्तसन्निधापितानाम् ``अलोऽन्त्यस्य'' इत्यादीनां समावेश एव, न बाध्यबाधकभावः, विरोधाभावात्~।
नाप्येतयोरङ्गाङ्गिभावः, उभयोरपि परार्थत्वेन तदयोगात्~।
``अनेकाल्शित्'' इति सूत्रे सर्वश्चैतत्परिभाषाबोधित एव गृह्यते~।
यत्तु ``आदेः परस्य'', ``अलोन्त्यस्य'' इत्येतावेव तद्बाधकाविति तन्न ~; ``उदःस्थात्'' इति सूत्रविषयेऽस्याः ``पादः पत्'' इति सूत्रे भाष्ये सञ्चारितत्वात्~।
नाप्येतयोरियं बाधिका, एतयोर्निर्विषयत्वप्रसङ्गादिति ``ति विंशतेः'' इति सूत्रे कैयटः~।
अकज्विषये तु नायं न्यायः, स्थानिवद्भावेन तन्मध्यपतितन्यायेन तद्बुद्ध्यैव कार्यजननात्~।
इयञ्च अवयवषष्ठीविषयेऽपि~।
अत एव ``तदोः सः सा''विति सत्वम्~, `अतिस्यः' इत्यत्रोपसर्गतकारस्य न~।
निर्दिश्यमानयुष्मदाद्यवयवमपर्यन्तस्यैव यूयादयः, न तु `अतियूयम्' इत्यादौ सोपसर्गावयवमपर्यन्तस्येति बोध्यम्~। ``पादः पत्'' इति सूत्रे ``षष्ठी स्थाने-'' इति सूत्रे च भाष्ये च स्पष्टैषा~॥११॥\par
ननु `चेता' इत्यादौ ह्रस्वस्येकारस्य प्रमाणत आन्तर्यादकारोऽपि स्यादत आह-
\section*{\begin{center}यत्रानेकविधमान्तर्यं तत्र स्थानत आन्तर्यं बलीयः॥१२॥\end{center}}
\addcontentsline{toc}{section}{यत्रानेकविधमान्तर्यम्}
अनेकविधं स्थानार्थगुणप्रमाणकृतम्~।
अत्र मानं - ``षष्टी स्थाने'' इत्यत एकदेशानुवृत्त्या स्थानेग्रहणेऽनुवर्तमाने पुनः``स्थानेन्तरतमः'' इति सूत्रे स्थाने ग्रहणमेव~।
तद्धि तृतीयया विपरिणमय्य वाक्यभेदेन स्थानिनः प्रसङ्गे जायमानः सति सम्भवे स्थानत एवान्तरतम इत्यर्थकम्~। तमब्ग्रहणमेवानेकविधान्तर्यसत्तागमकम्~।
स्थानतः स्थानेनेत्यर्थः~।
तत्र स्थानत आन्तर्यम् ``इको यणचि'' इत्यादौ प्रसिद्धमेव~।
अर्थतः``पद्दन्नो-'' इत्यादौ स्थान्यर्थाभिधानसमर्थस्यैवादेशतेति सिद्धान्ताद्यदर्थाभिधानसमर्थो यः, स तस्यादेश इति तत्समानार्थतत्समानवर्णपदादीनां ते, ``तृज्वत्क्रोष्टुः'' इति च~।
गुणतो `वाग्घरिः' इत्यादौ~।
प्रमाणतः ``अदसोऽसेः'' इत्यादौ~।
``स्थानेऽन्तरतमः'' सूत्रे भाष्ये स्पष्टैषा~॥१२॥\par
ननु `प्रोढवान्' इत्यत्र `प्रादूहोढे'ति वृद्धिः स्यादत आह-
\section*{\begin{center}अर्थवद्ग्रहणे नानर्थकस्य~॥१३॥\end{center}}
\addcontentsline{toc}{section}{अर्थवद्ग्रहणे}
विशिष्टरूपोपादाने उपस्थितार्थस्य शब्दं प्रति विशेषणतयान्वयसम्भवे त्यागे मानाभावोऽस्या मूलम्~।
अत्रार्थः कल्पितान्वयव्यतिरेककल्पितः शास्त्रीयोऽपि गृह्यते इति ``सङ्ख्यायाः'' इति सूत्रे भाष्ये स्पष्टम्~।
इयं वर्णग्रहणेषु नेति ``लस्य'' इत्यत्र भाष्ये स्पष्टम्~।
अत एव एषा विशिष्टरूपोपादानविषयेति वृद्धाः~।
एतन्मूलकमेव ``येन विधिः'' इत्यत्र भाष्ये पठ्यते `अलैवानर्थकेन तदन्तविधिः' इति~।
किञ्च ``स्वं रूपम्'' इति शास्त्रे स्वशब्देनात्मीयवाचिनार्थो गृह्यते, रूपशब्देन स्वरूपम्~, एवञ्च तदुभयं शब्दस्य संज्ञीति तदर्थः~।
तत्रार्थो न विशेष्यस्तत्र शास्त्रीयकार्यासम्भवात्, किन्तु शब्दविशेषणम्~।
एवञ्चार्थविशिष्टः शब्दः संज्ञीति फलितम्~।
तेनैषा परिभाषा सिद्धेति भाष्ये स्पष्टम्~॥१३॥\par
नन्वेवमपि `महद्भूतश्चन्द्रमाः' इत्यत्र ``आन्महतः'' इत्यात्त्वापत्तिरत आह-
\section*{\begin{center}गौणमुख्ययोर्मुख्ये कार्यसम्प्रत्ययः~॥१४॥\end{center}}
\addcontentsline{toc}{section}{गौणमुख्ययोर्मुख्ये}
गुणादागतो गौणः~।
यथा गोशब्दस्य जाड्यादिगुणनिमित्तोऽर्थो वाहीकः~।
अप्रसिद्धश्च संज्ञादिरपि तद्गुणारोपादेव बुद्ध्यते~।
मुखमिव प्रधानत्वान्मुख्यः प्रथम इत्यर्थः~।
गौणे ह्यर्थे शब्दः प्रयुज्यमानो मुख्यार्थारोपेण प्रवर्तते~।
एवं चाप्रसिद्धत्वं गौणलाक्षणिकत्वं चात्र गौणत्वम्~।
तेन `प्रियत्रयाणाम्' इत्यादौ त्रयादेशो भवत्येव, तत्र त्रिशब्दार्थस्येतरविशेषणत्वेऽप्युक्तगौणत्वाभावात्~।
किञ्चायं न्यायो न प्रातिपदिककार्ये, किन्तूपात्तं विशिष्यार्थोपस्थापकं विशिष्टरूपं यत्र तादृशपदकार्य एव~।
परिनिष्ठितस्य पदान्तरसम्बन्धे हि `गौर्वाहीकः' इत्यादौ गौणत्वप्रतीतिर्न तु प्रातिपदिकसंस्कारवेलायामित्यन्तरङ्गत्वाज्जातसंस्कारबाधायोगः प्रातिपदिककार्ये प्रवृत्त्यभावे बीजम्~।
श्वशुरसदृशस्यापत्यमित्यर्थके `श्वाशुरिः' इत्यादावत इञः सिद्धये उपात्तमित्यादि~।
न च प्रातिपतिकपदं तादृशमिति वाच्यम्~, तेन हि प्रातिपदिकपदवत्वेनोपस्थितिरिति तस्य विशिष्यार्थोपस्थापकत्वाभावात्~।
निपातपदं तु चादित्वेनैव चादीनामुपस्थापकमिति तदुद्देश्यककार्यविधायके ``ओत्'' इत्यादावेतत्प्रवृत्त्या `गोभवत्' इत्यादौ दोषो न~।
`अग्नीषोमौ माणवकौ' इत्यत्र प्रसिद्धदेवताद्वन्द्ववाच्यग्नीषोमपदस्य तत्सदृशपरत्वेऽप्यन्तरङ्गत्वादीत्वषत्वे भवत एव~।
सदृशलाक्षणिकाग्निसोमपदयोर्द्वन्द्वे, तन्नामकावित्यर्थके च नेत्वषत्वे, आद्ये गौणलाक्षणिकत्वात्~, अन्त्येऽप्रसिद्धत्वात्~।
अत एव `अग्निसोमौ माणवकौ' इत्यत्र गौणमुख्यन्यायेन षत्ववारणपरम्~, ``अग्नेः स्तुत्स्तोमसोमाः'' इति सूत्रस्थं भाष्यं सङ्गच्छते~।
`गां पाठय' इत्यादौ मुख्यगोपदार्थस्य पाठनकर्मत्वासम्भवेन विभक्त्युत्पत्तिवेलायां प्रयोक्तृभिर्गौणार्थत्वस्य प्रतीतावप्यपदस्याप्रयोगेण बोद्धृभिः सर्वत्र पदस्यैव गौणार्थकत्वस्य ग्रहेण `अत्वं त्वं सम्पद्यते',
`अमहान्महान्भूतः' - `त्वद्भवति' इत्यादिभाष्यप्रयोगे त्वाद्यादेशदीर्घादीनां करणेन चास्य न्यायस्य पदकार्यविषयत्वमेवोचितम्~।
अन्यथा वाक्यसंस्कारपक्षे तेषु तदनापत्तिः~।
किञ्च {\bfseries शुक्लाम् इत्युक्ते कर्म निर्दिष्टं कर्ता क्रिया चानिर्दिष्टे} इत्याद्युक्त्या {\bfseries इहेदानीं गामभ्याज कृष्णां देवदत्तेत्यादौ सर्वं निर्दिष्टम्~, गामेव कर्म, देवदत्त एव कर्त्ता, अभ्याजैव क्रिया}
इत्यर्थकेनार्थवत्सूत्रस्थभाष्येण कारकादिमात्रप्रयोगे योग्यसर्वक्रियाध्यहारे प्रसक्ते नियमार्थः क्रियावाचकादिप्रयोग इत्येतत्तात्पर्यकेण सामान्यतः क्रियाजन्यफलाश्रयत्वमात्रविवक्षायां द्वितीयादीनां
साधुत्वान्वाख्यानमित्यर्थलाभेन पाठनक्रियान्वयकाले पदस्यैव गौणर्थत्वप्रतीतिः प्रयोक्तुरपि~।
एतन्मूलकः `अभिव्यक्तपदार्थाः ये' इति श्लोकोऽपि पदकार्यविषयकः~।
ध्वनितं चेदं ``सर्वादीनि'' इति सूत्रे संज्ञाभूतानां प्रतिषेधमारभता वार्त्तिककृता, ``पूर्वपर-'' इति सूत्रेऽसंज्ञायामिति वदता सूत्रकृता, अन्वर्थसंज्ञया तत्प्रत्याख्यानं कुर्वता भाष्यकृता च~।
{\bfseries अर्थाश्रय एतदेवं भवति, शब्दाश्रये च वृद्ध्यात्वे} इति ``ओत्'' सूत्रस्थभाष्यस्य लौकिकार्थवत्वयोग्यपदाश्रय एष न्यायः, तद्रहितशब्दाश्रये च ते इत्यर्थः~।
``गोतः'' इति यथाश्रुतसूत्रे विशिष्टरूपोपादानसत्वेनोक्तरीत्यैव तस्य भाष्यस्य व्याख्येयत्वादित्यलम्~॥१४॥\par
`अर्थवद्ग्रहणे' इत्यस्यापवादमाह -
\section*{\begin{center}अनिनस्मन्ग्रहणान्यर्थवता चानर्थकेन च तदन्तविधिं प्रयोजयन्ति~॥१५॥\end{center}}
\addcontentsline{toc}{section}{अनिनस्मन्ग्रहणानि}
``येन विधिः''  इत्यत्र भाष्ये वचनरूपेण पठितैषा~।
तेन `राज्ञा', `साम्ना' इत्यादावल्लोपः, `दण्डी', `वाग्मी' इत्यादौ ``इन्हन्'' इति नियमः, `सुपयाः', `सुस्रोताः' इत्यादौ ``अत्वसन्तस्य'' इति दीर्घः,
`सुशर्मा', `सुप्रथिमा' इत्यादौ ``मनः'' इति ङीब्निषेधश्च सिद्धः~।
अन्ये तु `परिवीविषीध्वम्' इत्यत्र ढत्वव्यावृत्तये क्रियमाणात् ``इणः षीध्वम्'' इत्यत्राङ्गग्रहणादर्थवत्परिभाषानित्या, तन्मूलकमिदमित्याहुः~।
``विभाषेटः'' इत्यत्रानर्थकस्यैव षीध्वमः सम्भवादत्रापि तस्यैव ग्रहणमिति भ्रमवारणायाङ्गात् इति परे~॥१५॥\par
ननु ``उश्च'' इत्यत्र ``लिङ्‍सिचौ'' इत्यतः `आत्मनेपदेषु' इत्येव सम्बध्येत, अनन्तरत्वादत आह-

\section*{\begin{center}एकयोगनिर्दिष्टानां सह वा प्रवृत्तिः सह वा निवृत्तिः~॥१६॥\end{center}}
\addcontentsline{toc}{section}{एकयोगनिर्दिष्टानाम्}
वाशब्द एवार्थे~।
परस्परान्वितार्थकपदानां सहैवानुवृत्तिनिवृत्ती इत्यर्थः~।
एककार्यनियुक्तानां बहूनां लोके तथैव दर्शनादिति भावः~।
यत्त्वत्र ज्ञापकं ``नेड्वशि'' इत्यत इडित्यनुवर्तमाने ``आर्धधातुकस्येड्'' इत्यत्र पुनरिड्‍ग्रहणम्~।
तद्धि नेत्यस्यासम्बन्धार्थमिति~।
तन्न ~।
``दीधीवेवीटाम्'' इति सूत्रे भाष्ये तत्रत्येड्‍ग्रहणप्रत्याख्यानायेड्‍ग्रहणेऽनुवर्तमाने पुनरिड्‍ग्रहणस्येटो गुणरूपविकारार्थकत्वस्योक्तत्वेन तद्विरोधात्~।
नञो निवृत्तिस्तु `क्वचिदेकदेशोऽपि अनुवर्तते' इति न्यायेन सिद्धा~।
वस्तुतस्तु ``दीधीवेवीटाम्'' इति सूत्रस्थभाष्यमेकदेश्युक्तिः, ``आर्धधातुकस्य'' इति सूत्रस्थेड्‍ग्रहणस्य ``नेड्वशि'' इति सूत्रे भाष्ये प्रत्याख्यानात्~,
तत्करणेन गुरुतरयत्नमाश्रित्यैतत्प्रत्याख्यानस्यायुक्तत्वात्~॥१६॥\par
ननु अलुगधिकारः प्रागानङः, उत्तरपदाधिकारः प्रागङ्गाधिकारादित्यनुपपन्नम्~, एकयोगनिर्दिष्टत्वात्~।
तथा ``दामहायनान्ताच्च'' इत्यादौ ``सङ्ख्याव्ययादेः'' इत्यतः सङ्ख्यादेरित्यनुवर्ततेऽव्ययादेरिति निवृत्तमिति चानुपपन्नमत आह- 
\section*{\begin{center}क्वचिदेकदेशोऽप्यनुवर्तते~॥१७॥\end{center}}
\addcontentsline{toc}{section}{क्वचिदेकदेशोऽपि}
एकार्थे योगः सम्बन्धस्तेन निर्दिष्टयोः समुदायाभिधायिद्वन्द्वनिर्दिष्टयोरित्यर्थ इति ``पक्षात्तिः'' इति सूत्रे कैयटः~।
तावन्मात्रांशे स्वरितत्वप्रतिज्ञाबलाल्लभ्यमिदम्~।
स्पष्टा चेयम् ``दामहायनान्ताच्च'' इति सूत्रे, ``औतोऽम्शसोः'' इति सूत्रे च भाष्ये पूर्वा च~॥१७॥\par
ननु ``त्यदादीनामः'' इत्यादिना `इमम्' इत्यादावनुनासिकः स्यादत आह-
\section*{\begin{center}भाव्यमानेन सवर्णानां ग्रहणं न॥१८॥\end{center}}
\addcontentsline{toc}{section}{भाव्यमानेन}
अणुदित्सूत्रे `अप्रत्ययः' इत्यनेन सामर्थ्यात्सूत्रप्राप्तं~, जातिपक्षेण प्राप्तं~,गुणाभेदकत्वेन च प्राप्तं~, नेत्यर्थः~।
अत एवाणुदित्सूत्रे प्रत्ययाऽदेशाऽगमेषु सवर्णग्रहणाभावं प्रकारान्तरेणोक्त्वा, एवं तर्हि सिद्धे यदप्रत्यय इति प्रतिषेधं शास्ति, तज्ज्ञापयति भवत्येषा परिभाषा `भाव्यमानेन सवर्णानां ग्रहणं न' इति~।
किञ्च, `ज्याद ईयसः' इत्येवान्तर्यतो दीर्घे सिद्धे ``ज्यादात्'' इति दीर्घोच्चारणमस्या ज्ञापकम्~।
अणुदित्सूत्रे ``ज्यादात्'' इति सूत्रे च भाष्ये स्पष्टैषा~।
``चोः कुः'' इत्यादौ भाव्यमानेनापि सवर्णग्रहणम्~, विधेये उदिदुच्चारणसामर्थ्यात्~।
एतदेवाभिप्रेत्य `भाव्यमानोऽण्सवर्णान्न गृह्णाति' इति नव्याः पठन्ति~॥१८॥\par
नन्वेवम् ``अदसोऽसेः'' इत्यादिना `अमू' इत्यादौ दीर्घविधानं न स्यादत आह-
\section*{\begin{center}भाव्यमानोऽप्युकारः सवर्णान्गृह्णाति॥१९॥\end{center}}
\addcontentsline{toc}{section}{भाव्यमानोऽप्युकारः}
``दिव उत्'', ``ऋत उत्'' इति तपरकरणमस्या ज्ञापकम्~।
``तित्स्वरितम्'' इति सूत्रे भाष्ये स्पष्टैषा॥१९॥\par
ननु `गवे हितं गोहितम्' इत्यादौ प्रत्ययलक्षणेनावाद्यादेशापत्तिरत आह-
\section*{\begin{center}वर्णाश्रये नास्ति प्रत्ययलक्षणम्॥२०॥\end{center}}
\addcontentsline{toc}{section}{वर्णाश्रये नास्ति}
वर्णप्राधान्यविषयमेतत्~।
तत्त्वं च ``प्रत्ययलोपे'' इति सूत्रे स्थानिवदित्यनुवृत्त्यैव सिद्धे प्रत्ययलक्षणग्रहणं प्रत्ययेतराविशेषणत्वरूपं यत्र प्राधान्यं तत्रैव प्रवृत्त्यर्थमेतत्सिद्धम्~।
वर्णप्राधान्यं च वर्णस्येतराविशेषणत्वरूपं प्रत्ययनिरूपितविशेष्यतारूपं च~।
तेन `गोहितम्' इत्यादौ अवादि न, `चित्रायां जाता चित्रा' इत्यादावण्योऽकारस्तदन्तान्ङीबिति ङीप् च न~।
इयमल्विधौ स्थानिवत्त्वाप्राप्तावपि प्राप्तप्रत्ययलक्षणविधेर्निषेधिकेति स्पष्टं भाष्ये॥२०॥\par
ननु ``अतः कृकमि-'' इत्यत्र कमिग्रहणेन सिद्धे~, कंसग्रहणं व्यर्थम्~, अत आह-
\section*{\begin{center}उणादयोऽव्युत्पन्नानि प्रातिपदिकानि॥२१॥\end{center}}
\addcontentsline{toc}{section}{उणादयोऽव्युत्पन्नानि}
इदमेवास्या ज्ञापकमिति कैयटादयः~।
कंसेस्तु न कंसोऽनभिधानात्~।
``प्रत्ययस्य लुक्-'' इत्यादौ भाष्ये स्पष्टा~।
``ण्वुल्तृचौ'' इत्यादौ भाष्ये व्युत्पन्नानीत्यपि~।
इदं शाकटायनरीत्या~।
पाणिनेस्त्वव्युत्पत्तिपक्ष एवेति शब्देन्दुशेखरे निरूपितम्~।
``आयनेयीन्-'' इति सूत्रे भाष्ये स्फुटमेतत्॥२१॥\par
ननु `देवदत्तश्चिकीर्षति' इत्यादौ देवादेः सन्नन्तत्वप्रयुक्तधातुत्वाद्यापत्तिरत आह-
\section*{\begin{center}प्रत्ययग्रहणे यस्मात्स विहितस्तदादेस्तदन्तस्य ग्रहणम्॥२२॥\end{center}}
\addcontentsline{toc}{section}{प्रत्ययग्रहणे यस्मात्स}
``यस्मात्प्रत्ययविधिः'' इति सूत्रे `यस्मात्प्रत्ययविधिस्तदादिप्रत्यय' इति योगो विभज्यते~।
गृह्यमाण उपतिष्ठत इति शेषः~।
तेन तदाद्यन्तांशः सिद्धः~।
तदन्तांशस्तु ``येन विधिः'' इत्यनेन सिद्धः~।
स च शब्दरूपं विशेष्यमादाय विशेष्यान्तरासत्त्वे~।
यत्तु प्रत्ययेन स्वप्रकृत्यवयवकसमुदायाक्षेपात्तद्विशेषणत्वेन तदन्तविधिरिति~।
तन्न~। `इयान्' इत्यादौ तस्य तादृशसमुदायेन व्यभिचारेणाक्षेपासम्भवात्~।
यत्र प्रत्ययो निमित्तत्वेनाश्रीयते, तत्र तदादीत्यन्तांशमात्रोपस्थितिरिति ``अङ्गस्य'' इति सूत्रे भाष्यकैयटयोः॥एवं यत्रापि पञ्चम्यन्तात्परः प्रत्यय आश्रीयते, तत्रापि तदादीत्यन्तांशोपस्थितिः, परन्तु तत्र पञ्चम्यन्तता~।
अत  एव``एङ्ह्रस्वात्'' इति सूत्रे एङन्तादित्यर्थलाभः~।
अस्याः परिभाषायाः प्रयोजनान्तरम्~, ``येन विधिः'' इत्यत्र भाष्य उक्तं~, `परमगार्ग्यायण' इति~।
परमगार्ग्यस्यापत्यमिति विग्रहेऽपि गार्ग्यशब्दादेव प्रत्ययः, न विशिष्टात्~।
निष्कृष्य तावन्मात्रेणैकार्थीभावाभावेऽपि वृत्तिर्भवत्येव~।
अत्र चेदं भाष्यमेव मानमित्यन्यत्र विस्तरः~।
प्रत्ययमात्रग्रहणे एषा, न तु, प्रत्ययाप्रत्ययग्रहण इति ``उगितश्च'' इति सूत्रे भाष्ये~।
इयमङ्गसंज्ञा सूत्रे भाष्ये स्पष्टा॥२२॥\par
येन विधिरिति सूत्रे भाष्य एतद्घटकतदन्तांशस्यापवादः पठ्यते-
\section*{\begin{center}प्रत्ययग्रहणे चापञ्चम्याः॥२३॥\end{center}}
\addcontentsline{toc}{section}{प्रत्ययग्रहणे}
यत्र पञ्चम्यन्तात्परः प्रत्ययः कार्यान्तरविधानाय परिगृह्यते, तत्र तदन्तविधिर्नेत्यर्थः~।
यथा ``रदाभ्यां निष्ठातो नः'' इत्यत्र~।
तेन दृषत्तीर्णेत्यादौ धातुतकारस्य न नत्वम्~।
तदन्तेत्यंशानुपस्थितावपि तदादीत्यंशस्योपस्थितौ रेफदान्तात्परस्य निष्ठातस्येत्यर्थ इति न दोषः, तदंशानुपस्थितौ मानाभावात्~।
तदन्तांशोपस्थितौ तूभयोरेकविषयत्वमेव स्यादिति `दृषत्तीर्णः' इत्यादौ दोषः स्यादेव~।
``स्यतासी लृलुटोः'' इत्यादौ लृलुटोः परयोरित्यर्थे नियमेनावधिसाकाङ्क्षत्वेनोपस्थितधातोरित्यस्यावधित्वेन अन्वयान्न तदन्तविधिः~।
``हल्‌ङ्याभ्यः'' इत्यादौ तु न दोषः, तत्र कस्मादिति नियतावध्याकाङ्क्षाया अभावेन पञ्चम्यन्तस्य प्रत्ययविशेषणत्वाभावात्~।
अङ्गसंज्ञासूत्रे तु तदादेः प्रत्यये पर इत्यर्थे पञ्चम्यन्तस्य विशेषणत्वं स्पष्टमेव~।
अत एव ``उत्तमैकाभ्याम्'' इत्यादिनिर्देशाः सङ्गच्छन्ते॥२३॥\par
नन्वेवं `कुमारी ब्राह्मणीरूपा' इत्यादौ ``घरूप'' इति ह्रस्वापत्तिरत आह-

\section*{\begin{center}उत्तरपदाधिकारे प्रत्ययग्रहणे न तदन्तग्रहणम्॥२४॥\end{center}}
\addcontentsline{toc}{section}{उत्तरपदाधिकारे}
``हृदयस्य हृल्लेखयदण्लासेषु'' इत्यत्र लेखग्रहणात्~।
तत्र लेखेति न घञन्तम्~, अनभिघानात्~।
इयं च ``हृदयस्य'' इति सूत्र एव भाष्ये स्पष्टा~॥२४॥\par
नन्वेवम् `परमकारीषगन्धीपुत्रः' इत्यत्रेव `अतिकारीषगन्ध्यापुत्रः' इत्यत्र ``ष्यङः सम्प्रसारणं पुत्रपत्योः''
इति स्यादत आह-
\section*{\begin{center}स्त्रीप्रत्यये चानुपसर्जने न॥२५॥\end{center}}
\addcontentsline{toc}{section}{स्त्रीप्रत्यये}
विषयसप्तमीयम्~।
यः स्त्रीप्रत्ययः स्त्रियं प्राधान्येनाह तत्र तदादिनियमो न, यस्त्वप्राधान्येनाह तत्र तदादिनियमोऽस्त्येवेत्यर्थः~।
प्रत्यासत्त्या यस्य समुदायस्य स्त्रीप्रत्ययान्तत्वमानेयं तदर्थं प्रत्यनुपसर्जनत्वमेवैतत्परिभाषाप्रवृत्तौ निमित्तम्~।
तेन `अतिराजकुमारिः' इत्यादौ राजकुमारीशब्दार्थस्यातिशब्दार्थं प्रत्युपसर्जनत्वेऽपि तदर्थं प्रत्यनुपसर्जनत्वात्तदादानियमाभावेन ह्रस्वसिद्धिः~।
अत एवात्र परिभाषायां न शास्त्रीयमुपसर्जनत्वम्~, असम्भवात्~।
अस्याः `प्रत्ययग्रहणे' इत्यस्यापवादत्वात्तदेकवाक्यत्वापन्नत्वाच्चात्रापि ग्रहणपदसम्बन्धेन स्त्रीप्रत्ययसामान्यग्रहणे तद्विशेषग्रहणे च प्रवृत्तिः, न तु स्त्रीप्रत्ययास्त्रीप्रत्ययग्रहणे~।
ध्वनितञ्चेदम् ``अर्थवत्'' सूत्रे भाष्ये~।
इयं च वाचनिक्येव~।
``ष्यङः'' इति सूत्रे भाष्ये स्पष्टा॥२५॥\par
नन्वेवं ``तरप्तमपौ घः'' इत्यादिना तरबन्तादेः संज्ञा स्यादत आह-
\section*{\begin{center}संज्ञाविधौ प्रत्ययग्रहणे तदन्तग्रहणं नास्ति॥२६॥\end{center}}
\addcontentsline{toc}{section}{संज्ञाविधौ प्रत्ययग्रहणे}
``सुप्तिङन्तम्'' इत्यन्तग्रहणमस्यां ज्ञापकम्~।
न च प्रत्यययोः पदसंज्ञायामपि प्रत्ययग्रहणपरिभाषया तदन्तग्रहणाभावात् ज्ञापितेऽपि फलाभाव इति वाच्यम्~, पदसंज्ञायाः ``स्वादिषु'' इति विषये प्रकृति निष्ठतया पदग्रहणस्य प्रत्ययमात्रग्रहणत्वाभावात्~।
``सुप्तिङन्तम्'' इति सूत्रे भाष्ये स्पष्टा॥२६॥\par
ननु `अवतप्ते नकुलस्थितम्' इत्यादौ नकुलस्थितशब्दस्य क्तान्तत्वाभावात्समासो न स्यादत आह-
\section*{\begin{center}कृद्ग्रहणे गतिकारकपूर्वस्यापि ग्रहणम्॥२७॥\end{center}}
\addcontentsline{toc}{section}{कृद्ग्रहणे गतिकारक}
अस्याश्च `कर्मणि क्तान्त उत्तरपदेऽनन्तरो गतिः प्रकृतिस्वर' इत्यर्थके ``गतिरनन्तरः'' इति सूत्रे अनन्तरग्रहणं ज्ञापकम्~।
तद्धि `अभ्युद्धृतम्' इत्यादावतिव्याप्तिवारणार्थम्~।
प्रत्ययग्रहणपरिभाषयोद्धृतस्य क्तान्तत्वाभावादेव अप्राप्तौ तद्व्यर्थं सदस्या ज्ञापकम्~।
न च `अभ्युद्धृतम्' इत्यादौ परत्वात् ``गतिर्गतौ'' इत्यनेनाभेर्निघात एवेति वाच्यम्~, पादादिस्थत्वेन, पदात्परत्वाभावेन च तदप्राप्तेः~।
अनन्तरग्रहणे कृते तु तत्सामर्थ्याद्गत्याक्षिप्तधातुनिरूपितमेवानन्तर्यं गृह्यत इति न दोषः~।
न च ``अभ्युद्धृतम्'' इत्यादावभिना समासेऽनन्तरस्योदः पूर्वपदत्वाभावेऽपि स्वरार्थं तदिति वाच्यम्~।
``कारकाद्दत्त'' इति सूत्रे `कारकात्' इति योगं विभज्य तत्र गतिग्रहणमनुवर्त्य कारकादेव परं गतिपूर्वपदं क्तान्तमन्तोदात्तमिति नियमेन थाथादिस्वराप्राप्त्या कृत्स्वरेणोद उदात्तत्वसिद्धेः~।
तस्मादनन्तरग्रहणं व्यवहितनिवृत्त्यर्थमेवेति ज्ञापकमेव~।
यत्र गतिकारकसमभिव्याहृतं कृदन्तं तत्र कृद्ग्रहणे तद्विशिष्टस्यैव ग्रहणम्~।
`अपि' शब्दात्तदसमभिव्याहृतस्य केवलस्यापीति तदर्थः~।
अन्यथानया कृद्ग्रहणविषये परत्वात् प्रत्ययग्रहणपरिभाषाया बाध एव स्यादित्यपि ग्रहणम्~।
अत एव `साङ्कूटिनम्' इति `गतिकारकोपपदानाम्' इति `कृद्ग्रहण' इति च परिभाषाभ्यां कृदन्तेन समासे कृते विशिष्टादेवाणि सिद्ध्यति, न तु `सङ्कौटिनम्' इति ``पुंयोगात्'' इति सूत्रे भाष्योक्तं सङ्गच्छते~।
अन्यथा तत्र केवलकूटिन्नित्येतस्यापीनुणन्तत्वात्ततोऽणि पाक्षिकदोषो दुर्वार एव स्यात्~।
स्पष्टं चेदं सर्वं ``समासेऽनञ्पूर्वे'' इति सूत्रे भाष्यकैयटयोः~।
``गतिरनन्तरः'' इत्यत्र तु गतेः पूर्वपदस्य क्तान्त उत्तरपदे परे कार्यविधानात्तत्समवधानेऽपि केवलस्य क्तान्तत्वेन ग्रहणं बोध्यम्~।
इयं च कृद्विशेषग्रहणे कृत्सामान्यग्रहणे च, न तु कृदकृद्ग्रहण इति~।
``अनुपसर्जनात्'' इति सूत्रे भाष्ये स्पष्टम्॥२७॥\par
\section*{\begin{center}पदाङ्गाधिकारे तस्य च तदन्तस्य च॥२८॥\end{center}}
\addcontentsline{toc}{section}{पदाङ्गाधिकारे}
पदमङ्गं च विशेष्यं विशेषणेन च तदन्तविधिः~।
तेन `इष्टकचितम्', `पक्वेष्टकचितम्' इत्यादौ ``इष्टकेषीकामालानां चित-'' इति ह्रस्वः,`महान्', `परममहान्', `परमातिमहान्' इत्यादौ ``सान्तमहतः'' इति दीर्घश्च सिद्धः~।
अत एव `तदुत्तरपदस्य' इति पाठोऽयुक्त इति भाष्ये स्पष्टम्~।
अत्र पदशब्देनोत्तरपदाधिकारः, केवलपदाधिकारश्च~।
``पादस्य पदाज्याति-'' इत्यत्र न तदन्तग्रहणम्~, लक्ष्यानुरोधादिति सर्वम्~, ``येन विधिः'' इत्यत्र भाष्ये स्पष्टम्॥२८॥\par
नन्वेवम् `अस्यापत्यमिः' इत्यादावदन्तप्रातिपदिकाभावादिञ्न स्यात्~, अत आह-
\section*{\begin{center}व्यपदेशिवदेकस्मिन्॥२९॥\end{center}}
\addcontentsline{toc}{section}{व्यपदेशिवदेकस्मिन्}
निमित्तसद्भावात् विशिष्टोऽपदेशो मुख्यो व्यवहारो यस्यास्ति स व्यपदेशी~।
यस्तु व्यपदेशहेत्वभावादविद्यमानव्यपदेशोऽसहायः, स तेन तुल्यं वर्तते, कार्यं प्रतीत्येकस्मिन्नसहायेऽपि तत्कार्यं कर्तव्यमित्यर्थः~।
तेनाकारस्याप्यदन्तत्वान्न क्षतिः~।
एकस्मिन्नित्युक्तेः `सभासन्नयने' आकारस्य नादित्वम्~, दरिद्राधाताविकारस्य नान्तत्वम्~।
अन्यथा `सभासन्नयने भवः' इत्यर्थे ``वृद्धाच्छः'' दरिद्राधातोरिवर्णान्तलक्षणोऽच्च स्यात्~।
अत एव `हरिषु' इत्यादौ सोः पदत्वं न~।
लोकेऽपि बहुपुत्रसत्वे नैकस्मिन् ज्येष्ठकनिष्ठत्वादिव्यवहारोऽयं मे ज्येष्ठः, कनिष्ठः, मध्यम इति, किन्त्वेकपुत्रसत्व एव~।
अनेन अशास्त्रीयस्याप्यतिदेशः~।
अत एव `इयाय' इत्यादावेकाच्त्वनिबन्धनद्वित्वसिद्धिः~।
अत एव `भवति' इत्यादौ `भू' इत्यस्याङ्गत्वम्~, `इयान्' इत्यादौ कार्यकालपक्षे तद्धितान्तत्वनिबन्धनप्रातिपदिकत्वञ्च सिद्ध्यति~।
अन्यथा यस्माद्विहितस्तदादित्वाभावान्न स्यात्~।
यस्तु योऽर्थवांस्तत्रार्थस्य त्यागोपादानाभ्यामेकाज्व्यपदेशः, यथा `इयाय' इत्यादावर्थवतो धातोरयं वर्णरूप एकोऽजिति कैयटः, तन्न~। तस्य ``एकपदा ऋक्'' इत्यत्र भाष्योक्तरीत्या मुख्यव्यवहारसत्वात्~।
`एकपदा ऋक्'' इत्यत्रार्थेन युक्तो व्यपदेशः इति भाष्ये उक्तम्~।
ऋक्त्वादेरर्थशब्दोभयवृत्तित्वेन तस्याः शब्दमात्ररूपं पदमेकोऽवयव इत्यर्थ इति तदाशयः~।
तस्मादेकस्मिंस्तत्तद्धर्मारोपेण युगपद्यथा ज्येष्ठत्वादिव्यवहारः, यथा च `शिलापुत्रकस्य शरीरम्' इत्यादावेकस्मिन्नारोपितानेकावस्थाभिः समुदायरूपत्वाद्यारोपेणैतस्य शरीरमित्यादिव्यवहारः,
तथात्रैकाच्त्वादि व्यवहारोपपत्तिरिति लोकन्यायसिद्धेयम्~।
न चासहाय एवैतत्प्रवृत्तौ भवतीत्यत्र `भू' इत्यस्याङ्गत्वानापत्तिः, ससहायत्वादिति वाच्यम्~, शपमादायाङ्गत्वे कार्ये यस्माद्विहितस्तदादित्वे तस्य ससहायत्वाभावाल्लोके विजातीयकन्यादिसत्वेऽप्येकपुत्रस्य
तस्मिन्नेवायमेव ज्येष्ठ इत्यादिव्यवहारवत्~।
न चैवं ``निजौ चत्वार एकाचः'' इति भाष्यासङ्गतिः, इकारस्यासहायत्वाभावेन तत्रैकाच्त्वानुपपादनादिति वाच्यम्~, `एकस्मिन्' इत्यस्यापर्यालोचनया तत्प्रवृत्तेः~।
अर्थवता व्यपदेशिवद्भाव इत्यत्रार्थवत्पदेनाप्यसहायत्वमुपलक्ष्यते~।
अर्थबोधकेन शब्देन व्यपदेशिसदृशो भावः कार्यं लभ्यत इति तदर्थः, प्रायोऽसहाय एवार्थवत्वात्~।
`कुरुत' इत्यादौ तशब्दाकारोऽचामन्त्य इति व्यवहारे स आदिर्यस्येति व्यवहारे चासहाय एवेति तत्र व्यपदेशिवद्भावेन टिसंज्ञासिद्धिरित्यन्यत्र विस्तरः॥२९॥\par
ननु गर्गादिभ्यो विहितो यञ् तदन्तविधिना परमगर्गादिभ्योऽपि स्यात्~, अत आह-
\section*{\begin{center}ग्रहणवता प्रातिपदिकेन तदन्तविधिर्नास्ति॥३०॥\end{center}}
\addcontentsline{toc}{section}{ग्रहणवता प्रातिपदिकेन}
इयं च `समासप्रत्ययविधौ प्रतिषेधः', `उगिद्वर्णग्रहणवर्जम्' इति वार्तिकस्थप्रत्ययांशानुवादः~।
अत एव ``येन विधिः'' इति सूत्रभाष्ये प्रत्ययविधिभिन्ने ``अप्तृन्-'' इत्यादौ गृह्यमाणप्रातिपदिकेनापि तदन्तविधिप्रतिपादनं `स्वसा', `परमस्वसा' इत्याद्युदाहरणञ्च सङ्गच्छते~।
अत एव च तदन्तविधिसूत्रे भाष्ये समासेत्यादिनिषेधस्य कथनवदस्य न कथनम्~।
सोऽपि निषेधो विशिष्य तत्तद्रूपेण गृहीतप्रातिपदिकसूत्र एव~।
ध्वनितं चेदम्~,``असमासे निष्कादिभ्यः'' इति सूत्रे भाष्ये~।
अत्र च ज्ञापकं ``सपूर्वाच्च'' इति सूत्रम्~।
अन्यथा ``पूर्वादिनिः'' इत्यत्र तदन्तविधिनैव सिद्धे किं तेन॥३०॥\par
नन्वेवं ``सूत्रान्ताट्ठक्'', ``दशान्ताड्डः'', ``एकगोपूर्वात्'' इत्यादेः केवलसूत्रशब्ददशन्शब्दैकशब्दादिष्वपि प्रवृत्तिर्व्यपदेशिवद्भावात्स्यात्~, अत आह-
\section*{\begin{center}व्यपदेशिवद्भावोऽप्रातिपदिकेन॥३१॥\end{center}}
\addcontentsline{toc}{section}{व्यपदेशिवद्भावः}
``पूर्वात्सपूर्वादिनिः'' इत्येकयोग एव कर्तव्ये पृथग्योगकरणमस्या ज्ञापकम्~।
न च ``इष्टादिभ्यः'' इति सूत्रेऽनुवृत्त्यर्थं तथा पाठः, अत एवानिष्टीत्यादिसिद्धिरिति वाच्यम्~, ज्ञापकपरभाष्यप्रामाण्येनानिष्टीत्यादिप्रयोगाणामनिष्टत्वात्~, एकयोगेऽपि तावत उत्तरत्रानुवृत्तौ बाधकाभावाच्च~।
अत एव ``नान्तादसङ्ख्यादेः'' इति चरितार्थम्~।
अन्यथा `पञ्चमः' इत्यादावपि व्यपदेशिवद्भावेन सङ्ख्यादित्वात्तद्वैयर्थ्यं स्पष्टमेव~।
इयं च प्रातिपदिकग्रहणे एव, न तु प्रातिपदिकाप्रातिपदिकग्रहणे~।
तेन ``उगितश्च'' इत्यत्र न दोष इति तत्रैव भाष्ये स्पष्टम्~।
इयं `ग्रहणवता' इति च परिभाषा प्रत्ययविधिविषयैवेति ``असमासे निष्कादिभ्यः'' इति सूत्रे भाष्यकैयटयोः~।
तेन ``अहन्'' इत्यादेः परमाहन्शब्दे केवलाहन्शब्दे च प्रवृत्तिरित्यन्यत्र विस्तरः॥३१॥\par
ननु ``वान्तो यि'' इत्यादौ यादौ प्रत्यय इत्यर्थः कथम्~? अत आह-
\section*{\begin{center}यस्मिन्विधिस्तदादावल्ग्रहणे॥३२॥\end{center}}
\addcontentsline{toc}{section}{यस्मिन्विधिस्तदादावल्ग्रहणे}
तदन्तविधेरपवाद इयम्~।
वाचनिक्येषा ``येन विधिः'' इत्यत्र भाष्ये पठिता~।
अस्याश्च स्वरूपसती सप्तमी निमित्तम्~।
अत एव ``नेड्वशि कृति'' इत्यादौ वशादेः कृत इत्याद्यर्थलाभः~।
इयं च  ``आर्धधातुकस्येड्'' इति सूत्रे वलादेरित्यादिग्रहणसामर्थ्याद्विशेष्यविशेषणयोरुभयोः सप्तम्यन्तत्व एव प्रवर्तते~।
तेन  ``डः सि धुट्'' इत्यादौ सादेः परस्येति नार्थः~।
``तीषसह", ``सेऽसिची'' इत्यादौ यथा तादेरित्याद्यर्थलाभः, तथा शब्देन्दुशखरे निरूपितम्~॥३२॥\par
‘घटपटम्~, घटपटौ’ इत्यादिसिद्धय आह-
\section*{\begin{center}सर्वो द्वन्द्वो विभाषयैकवद्भवति॥३३॥\end{center}}
\addcontentsline{toc}{section}{सर्वो द्वन्द्वो}
"द्वन्द्वश्च प्राणि'' इत्यादिप्रकरणविषयः सर्वो द्वन्द्व इत्यर्थः~।
``चार्थे द्वन्द्वः'' इति सूत्रेण समाहारेतरयोगयोरविशेषेण द्वन्द्वविधानान्न्यायसिद्धेयम्~।
``तिष्यपुनर्वस्वोः'' इति सूत्रस्थं बहुवचनस्येति ग्रहणमस्या ज्ञापकम्~।
तद्धीदं ``तिष्यपुनर्वसु''इत्यत्र तद्व्यावृत्त्यर्थम्~।
न चैवमप्यत्र ``जातिरप्राणिनाम्'' इति नित्यैकवद्भावेन बहुवचनाभावादिदं सूत्रं व्यर्थमिति वाच्यम्~, तद्वैकल्पिकत्वस्याप्यनेन ज्ञापनात्~।
न चैते प्राणिन इति वाच्यम्~,"अपोमयः प्राणः'' इति श्रुतेरद्भिर्विना ग्लायमानप्राणानामेव प्राणित्वात्~।
स्पष्टं चेदं ``तिष्यपुनर्वस्वोः'' इति सूत्रे भाष्ये~।
अत एव ``द्वन्द्वश्च प्राणि'' इत्यादेः प्राण्यङ्गादीनामेव समाहार इति विपरीतनियमो न~॥३३॥\par
\section*{\begin{center}सर्वे विधयश्छन्दसि विकल्प्यन्ते॥३४॥\end{center}}
\addcontentsline{toc}{section}{सर्वे विधयश्छन्दसि}
``व्यत्ययो बहुलम्'' इति सूत्रे भाष्ये बहुलमिति योगविभागेन ``षष्ठीयुक्तश्छन्दसि'' इति सूत्रे वेति योगविभागेन चैषा साधिता~।
तेन `प्रतीपमन्य ऊर्मिर्युध्यति' इत्यादि सिद्धम्~।
युध्यत इति प्राप्नोति~॥३४॥\par
ननु क्षिय इत्यादावियङ्कथम्~? अत आह-
\section*{\begin{center}प्रकृतिवदनुकरणं भवति॥३५॥\end{center}}
\addcontentsline{toc}{section}{प्रकृतिवदनुकरणम्}
`क्षिय' इतीयङ्निर्देशोऽस्या ज्ञापकः~।
तत्रैव प्रातिपदिकत्वनिबन्धनविभक्तिकरणादनित्या चेयमिति ``क्षियो दीर्घात्'' इति सूत्रे भाष्ये स्पष्टम्~॥३५॥\par
 ननु `रामौ' इत्यादौ वृद्धौ कृतायां कार्यकालपक्षे कथं पदत्वम्~? उभयत आश्रयणेऽन्तादिवत्वाभावादत आह-
\section*{\begin{center}एकदेशविकृतमनन्यवत्॥३६॥\end{center}}
\addcontentsline{toc}{section}{एकदेशविकृतम्}
अनन्यवदित्यस्यान्यवन्नेत्यर्थः~।
तत्रान्यसादृश्यनिषेधेऽन्यत्वाभावः सुतराम्~।
अत एव तादृशादर्थबोधः, अन्यथा शक्ततावच्छेदकानुपूर्व्यज्ञानात्ततो बोधो न स्यात्~।
एवञ्च रामिति मान्तस्य यस्माद्विहितस्तत्वम्~, औ इत्यस्य परादिवत्वेन सुप्त्वमिति तदादितदन्तत्वम् आर्थसमाजग्रस्तम्~।
छिन्नपुच्छे शुनि श्वत्वव्यवहारवन्मान्ते तत्त्वं लोकन्यायसिद्धम्~।
अत एव ``प्राग्दीव्यत'' इति सूत्रे भाष्ये दीव्यतिशब्दैकदेशदीव्यच्छब्दानुकरणमिदमित्युक्त्वा, किमर्थं विकृतनिर्देशः~?
एतदेव ज्ञापयत्याचार्यो भवत्येषा परिभाषा ``एकदेशविकृतमनन्यवत्'' इत्युक्तम्~।
एतेनायं न्यायः शास्त्रीयकार्य एव शास्त्रीयविकार एवेत्यपास्तम्~।
विकृतावयवनिबन्धनकार्ये तु नायं~, छिन्ने पुच्छे शुनि पुच्छवत्वव्यवहारवत्विकृतावयवव्यवहारस्य दुरुपपादत्वात्~।
एवमक्तपरिमाणग्रहणेऽपि नायम्~, उक्तयुक्तेः~।
एतत् ``येन विधिः'' इत्यत्र भाष्यकैयटयोर्ध्वनितम्~।
यत्र त्वर्धं तदधिकं वा विकृतम्~, तत्र जातिव्यञ्जकभूयोऽवयवदर्शनाभावेन तत्त्वाप्रतीतौ कार्यसिध्यर्थं विकृतानल्‌रूपावयवत्वप्रतीत्यर्थं च स्थानिवत्सूत्रम्~।
क्वचित्तु लक्ष्यानुरोधान्न्यायानाश्रयणम्~।
तेन `आभीयात्' इत्यादिसिद्धिः~।
स्पष्टं च क्वचिन्न्यायाप्रवृत्तिः ``प्रथमयोः पूर्वसवर्णः'' इत्यत्र कैयटेन दर्शितेत्यन्यत्र विस्तरः॥
\begin{center}{\bfseries॥इति श्रीनागेशभट्टविरचिते परिभाषेन्दुशेखरे शास्त्रत्वसम्पादनोद्देशनामकं प्रथमं प्रकरणम्॥}\end{center}
\begin{figure}[b]
\centering
\includegraphics[width=4cm]{page-divider}
\end{figure}
\part*{॥अथ बाधबीजनामकं द्वितीयं प्रकरणम्॥}
\addcontentsline{toc}{part}{बाधबीजप्रकरणम् }
\fancyhead[RE]{द्वितीयं प्रकरणम्}
\section*{\begin{center}पूर्वपरनित्यान्तरङ्गापवादानामुत्तरोत्तरं बलीयः॥३७॥\end{center}}
\addcontentsline{toc}{section}{पूर्वपरनित्यान्तरङ्ग}
पूर्वात्परं बलवत्~, विप्रतिषेधशास्त्रात्~, पूर्वस्य परं बाधकमिति यावत्~॥\\
नन्वेवं `भिन्धि' इत्यत्र परत्वात्तातङा बाधितो धिर्न स्यादत आह-
\section*{\begin{center}पुनः प्रसङ्गविज्ञानात्सिद्धम् ॥३८॥\end{center}}
\addcontentsline{toc}{section}{पुनः प्रसङ्गविज्ञानात्}
नन्वेवं `तिसृणाम्' इत्यत्र परत्वात्तिस्रादेशे पुनस्त्रयादेशः स्यादत आह-
\section*{\begin{center}सकृद्गतौ विप्रतिषेधे यद्बाधितं तद्बाधितमेव॥३९॥\end{center}}
\addcontentsline{toc}{section}{सकृद्गतौ विप्रतिषेधे}
तत्र क्वचिच्चरितार्थयोरेकस्मिन्युगपदुभयोः कार्ययोरसम्भवेन बाधकाभावात्पर्यायेण तृजादिवच्छास्त्रद्वयप्रसङ्गे नियमार्थं विप्रतिषेधसूत्रमिति सकृद्गतिन्यायसिद्धिः~।
यथा तुल्यबलयोरेकः प्रेष्यो भवति~।
स तयोः पर्यायेण कार्यं करोति, यदा तमुभौ युगपत्प्रेषयतो नानादिक्षु च कार्ये तदोभयोर्न करोति, यौगपद्यासम्भवात्~।
तथा शास्त्रयोर्लक्ष्यार्थयोः क्वचिल्लक्ष्ये यौगपद्येन प्रवृत्त्यसम्भवादप्रतिपत्तौ प्राप्तायामिदं परविध्यर्थम्~।
तत्र कृते यदि पूर्वप्राप्तिरस्ति, तर्हि तदपि भवत्येवेति पुनः प्रसङ्गविज्ञानसिद्धिरिति विप्रतिषेधसूत्रे भाष्ये स्पष्टम्~।
यत्तु कैयटादयो व्यक्तौ पदार्थे प्रतिलक्ष्यं लक्षणोपप्लवादुभयोरपि शास्त्रयोस्तत्तल्लक्ष्ययोरचारितार्थ्येन पर्यायेण द्वयोरपि प्राप्तौ परमेवेति नियमार्थमिदमिति सकृद्गतिन्यायसिद्धिः~।
अत्र पक्षे एतन्नियमवशादेतल्लक्ष्यविषयकपूर्वशास्त्रानुपप्लव एव~।
जातिपक्षे तूद्देश्यतावच्छेदकाक्रान्ते क्वचिल्लक्ष्ये चरितार्थयोर्द्वयोः शास्त्रयोः सत्प्रतिपक्षन्यायेन युगदुभयासम्भवरूपविरोधस्थल उभयोरप्यप्राप्तौ परविध्यर्थमिदमिति पुनः प्रसङ्गविज्ञानसिद्धिरित्याहुः, तन्न~।
व्यक्तिपक्षे सर्वं लक्ष्यं शास्त्रं व्याप्नोति न जातिपक्ष इत्यत्र मानाभावात्~।
`न ब्राह्मणं हन्यात्' इत्यादौ जात्याश्रयसकलव्यक्तिविषयत्वार्थमेव जातिपक्षाश्रयणस्य भाष्ये दर्शनात्~।
अत एव सरूपसूत्रे भष्ये जातौ पदार्थेऽनवयवेन साकल्येन विधिप्रवृत्तेः `गौरनुबन्ध्यः' इत्यादौ सकलगवानुबन्धनासम्भवात्कर्मणो वैगुण्यमुक्तम्~।
द्रव्यवादे चासर्वद्रव्यावगतेः `गौरनुबन्ध्यः' इत्यादावेकः शास्त्रोक्तोऽपरोऽशास्त्रोक्त इत्युक्तम्~।
किञ्च न हि भाष्योक्ततृजादिदृष्टान्तस्य व्यक्तिवक्ष एव सर्वविषयत्वं न जातिपक्ष इत्यत्र मानमस्ति~।
अपि च व्यक्तिपक्षेऽप्यन्यव्यक्तिरूपविषयलाभेन चरितार्थयोरियं व्यक्तिर्विरोधात्स्वविषयकत्वं न कल्पयतीति वक्तुं शक्यम्~।
जातिपक्षेऽपि तज्जात्याश्रयतद्व्यक्तिविषयकत्वमेव ; नैतद्व्यक्तिविषयकत्वमित्यत्र नियामकाभावः~।
तत्र लक्ष्यानुसारात्क्वचिच्छास्त्रीयदृष्टान्ताश्रयणम्~, क्वचिल्लौकिकदृष्टान्ताश्रयणमिति भाष्यसम्मतमार्ग एव युक्त इति बोध्यम्~।
द्वयोः कार्ययोर्यौगपद्येनासम्भव एव विप्रतिषेधशास्त्रोपयोगी~।
इदम् ``इको गुण'' इति सूत्रे कैयटे स्पष्टम्~।
यथा `शिष्टात्' इत्यादौ तातङ्-शीभावयोर्युगपत्प्रवृत्तौ स्वस्वनिमित्तानन्तर्यासम्भवः~।
यद्यपि तातङादेः स्थानिवद्भवेनास्त्येव तत्~, तथाप्यादेशप्रवृत्त्युत्तरमेव सः, न तु तत्प्रवृत्तिकाले~।
एवं नुम्तृज्वत्त्वयोः `प्रियक्रोष्टूनि' इत्यादौ युगपदसम्भवः, यदागमा इत्यस्य नुम्प्रवृत्त्युत्तरं प्रवृत्तः~।
एवं `भिन्धि' इत्यत्र तातङ्-धिभावयोर्युगपदेकस्थानिसम्बन्धस्याङ्गरूपनिमित्तानन्तर्यस्य चासम्भवो बोध्यः~।
नुम्नुटोरपि नुट्यजादिविभक्त्यानन्तर्यबाधः, नुमि ह्रस्वान्ताङ्गबाध इत्यसम्भवाद्विप्रतिषेधः~।
क्वचिदिष्टानुरोधेन पूर्वशास्त्रे स्वरितत्वप्रतिज्ञाबलात्स्वरितेनाधिकं कार्यमित्यर्थात्पूर्वमेव भवति~।
तेन सर्वे पूर्वविप्रतिषेधाः सङ्गृहीता इति ``स्वरितेन-'' इतिसूत्रे भाष्ये~।
विप्रतिषेधसूत्रस्थपरशब्दस्येष्टवाचित्वात्तत्सङ्ग्रह इति विप्रतिषेधसूत्रे भाष्ये~॥
नन्वेवम् `एधते' इत्यादौ परत्वाद्विकरणे ``अनुदात्तङितः'' इत्यादिनियमानुपपत्तिः, तेन व्यवधानात्~, अत आह-
\section*{\begin{center}विकरणेभ्यो नियमो बलीयान्॥४०॥\end{center}}
\addcontentsline{toc}{section}{विकरणेभ्यो नियमः}
अत्र ``वृद्भ्यः स्यसनोः'' इति सूत्रेण स्ये विभाषातङ्विधानं ज्ञापकम्~।
अन्यथा स्वव्यवधाने नियमाप्रवृत्तौ सामान्यशास्त्रेणोभयसिद्धौ विकल्पविधानं व्यर्थं स्यात्~।
अत्रार्थे ज्ञापिते तु `स्य' इति तत्र विषयसप्तमी बोध्येति ``अनुदात्तङितः'' इत्यत्र भाष्यकैयटयोः स्पष्टम्~।
विकरणव्यवधानेऽपि नियमप्रवृत्तेरिदं ज्ञापकमिति ``शदेः शितः'' इत्यत्र भाष्ये ध्वनितम्~।
वस्तुतोऽस्माज्ज्ञापकात् ``अनुदात्तङितः'' इत्यादिप्रकरणं तिबादिविध्येकवाक्यतया विधायकम्~।
तत्र ``धातोः'' इति विहितपञ्चमी, तत्समानाधिकरणम् ``अनुदात्तङित'' इत्यादि विहितविशेषणमेव~।
एवञ्च लावस्थायां स्येऽपि तद्व्यवधाने तङ्सिद्धिः~।
शबादिभ्यस्तु पूर्वमेव नियमः~।
यद्वा लमात्रापेक्षत्वादन्तरङ्गा आदेशाः, लकारविशेषापेक्षत्वात्स्यादयो बहिरङ्गा इति दिग्योगलक्षणपञ्चम्यामपि न दोषः~।
अत्र पक्षे ``वृद्भ्यः स्यसनोः'' इति सूत्रं स्यविषय इति व्याख्येयम्~।
आत्मनेपद शब्दादौ भाविसंज्ञाऽश्रयणीयेति तत्त्वम्~।
भिन्नवाक्यतया सामान्यशास्त्रविहितानां नियमे तु लुगादिनेव नियमेन जातनिवृत्तिरङ्गीकार्या~।
 ``भुक्तवन्तं प्रति मा भुङ्‍क्था इति ब्रूयात्~, किं तेन कृतं स्यात्'' इति न्यायस्तु नात्र शास्त्र आश्रयितुं युक्तः, नियमादिशास्त्राणां वैयर्थ्यापत्तेः~।
ध्वनितं चेदम् ``स्थनेऽन्तरतमः'' इति सूत्रे भाष्ये~।
शास्त्रानर्थक्यं तु वृद्धिसंज्ञा सूत्रे भाष्येतिरस्कृतम्~।
सामान्यशास्त्रेणोत्पत्तिस्तु सरूपसूत्रस्थकैयटरीत्या प्रधानानुरोधेन गुणभेदकल्पनया, तावत्प्रकृतिकल्पनया वा कार्या, प्रत्ययनिवृत्तौ च तत्कल्पितप्रकृतेरपि निवृत्तिः कल्प्येति गौरवमित्यन्यत्र विस्तरः॥
\section*{\begin{center}परान्नित्यं बलवत्॥४१॥\end{center}}
\addcontentsline{toc}{section}{परान्नित्यं बलवत् }
कृताकृतप्रसङ्गित्वात्~।
तत्राकॢप्ताभावकस्याभावकल्पनापेक्षया कॢप्ताभावकस्यैव तत्कल्पनमुचितमिति, नित्यस्य बलवत्वे बीजम्~॥\\
तदाह-
\section*{\begin{center}कृताकृतप्रसङ्गि नित्यम्~, तद्विपरीतमनित्यम्~॥४२॥\end{center}}
\addcontentsline{toc}{section}{कृताकृतप्रसङ्गि नित्यम्}
अत एव `तुदति' इत्यादौ परादपि गुणान्नित्यत्वात् शप्रत्ययादिर्भवति~॥\\
यद्व्यक्तिसम्भन्धितया पूर्वं प्रवृत्तिस्तद्व्यक्तिसम्भन्धितयैव पुनःप्रवृत्तौ कृताकृतप्रसङ्गित्वमित्याशयेनाह-
\section*{\begin{center}शब्दान्तरस्य प्राप्नुवन्विधिरनित्यो भवति~॥४३॥\end{center}}
\addcontentsline{toc}{section}{शब्दान्तरस्य प्राप्नुवन्}
इदं ``शदेः शितः'' इति सूत्रे भाष्ये स्पष्टम्~।
तत्र हि `न्यविशत' इत्यत्र विकरणे कृते तदन्तस्याट्, अकृते विकरणे धातुमात्रस्येत्यडनित्य इत्युक्तम्~॥\\
एतत्तुल्यन्यायेनाह-
\section*{\begin{center}शब्दान्तरात्प्राप्नुवतः, शब्दान्तरे प्राप्नुवतश्चानित्यत्वम्~॥४४॥\end{center}}
\addcontentsline{toc}{section}{शब्दान्तरात्प्राप्नुवतः}
एतन्मूलकमेवाह-
\section*{\begin{center}लक्षणान्तरेण प्राप्नुवन्विधिरनित्यः~॥४५॥\end{center}}
\addcontentsline{toc}{section}{लक्षणान्तरेण प्राप्नुवन्}
अतिदेशविषये इयमसिद्धवत्सूत्रे कैयटेनोक्ता~॥
यदा तु शास्त्रव्यतिरेकेण तद्विधेयकार्ययोरेव नित्यत्वादिविचारो, यदापि व्यक्तिविशेषाश्रयणाभावः, तदाह-
\section*{\begin{center}क्वचित्कृताकृतप्रसङ्गमात्रेणापि नित्यता~॥४६॥\end{center}}
\addcontentsline{toc}{section}{क्वचित्कृताकृत}
कृते द्वितीये नित्यत्वेनाभिमतस्य पुनःप्रसङ्गमात्रं नित्यत्वव्यवहारे प्रयोजकं~, न तु बाधकाबाधितफलोपहितप्रसङ्गोऽपि तथेति भावः~॥\\
तदाह-
\section*{\begin{center}यस्य च लक्षणान्तरेण निमित्तं विहन्यते न तदनित्यम्~॥४७॥\end{center}}
\addcontentsline{toc}{section}{यस्य च लक्षणान्तरेण}
क्वचित्तु बाधकाबाधितफलोपहितप्रसङ्ग एव गृह्यते तदाह-
\section*{\begin{center}यस्य च लक्षणान्तरेण निमित्तं विहन्यते तदप्यनित्यम्~॥४८॥\end{center}}
\addcontentsline{toc}{section}{यस्य च लक्षणान्तरेण}
सप्तमे कैयटेनैतदुपष्टम्भकं लोकव्यवहारद्वयमुदाहृतम्~।
वालिसुग्रीवयोर्युध्यमानयोर्भगवता वालिनि हतेऽपि सुग्रीवस्य वालितः प्राबल्यं न व्यवहरन्ति, भगवत्सहायैः पाण्डवैर्जये लब्धेऽपि पाण्डवानां प्राबल्यं व्यवहरन्ति चेति सर्वं चेदं लक्ष्यानुरोधाद्व्यवस्थितम्~॥\\
 ``लुटः प्रथमस्य'' इति सूत्रे भाष्ये-
\section*{\begin{center}स्वरभिन्नस्य प्राप्नुवन्विधिरनित्यो भवति~॥४९॥\end{center}}
\addcontentsline{toc}{section}{स्वरभिन्नस्य प्राप्नुवन्}
इति पठ्यते~।
यत्र त्वेकस्यैव कार्यस्य परत्वं नित्यत्वं च, तत्रेच्छयान्यतरत्तदुभयं वा तस्य बलवत्वे नियामकमुल्लेख्यम्~।
अत एव तत्र परत्वान्नित्यत्वाच्चेति भाष्ये उच्यते~।
वस्तुतस्तत्र परत्वादित्युक्तिरेकदेशिनः~।
स्पष्टं चेदं विप्रतिषेधसूत्रे कैयटे~।
"णौ चङि'' इति ह्रस्वापेक्षया नित्यत्वान्तरङ्गत्वप्रयुक्तद्वित्वस्य प्रथमतः प्रवृत्तौ नित्यत्वादित्येव भाष्य उक्तम्~।
एवं नित्यान्तरङ्गयोर्बलवत्वमपि यौगपद्यासम्भव एवेति बोध्यम्~॥
नित्यादप्यन्तरङ्गं बलीयः, अन्तरङ्गे बहिरङ्गस्यासिद्धत्वात्~।
तदाह-
\section*{\begin{center}असिद्धं बहिरङ्गमन्तरङ्गे~॥५०॥\end{center}}
\addcontentsline{toc}{section}{असिद्धं बहिरङ्गं}
अन्तः मध्ये, बहिरङ्गशास्त्रीयनिमित्तसमुदायमध्ये, अन्तर्भूतान्यङ्गानि निमित्तानि यस्य तदन्तरङ्गम्~।
एवं तदीयनिमित्तसमुदायाद्बहिर्भूताङ्गकं बहिरङ्गम्~।
एतच्च ``खरवसानयोः'' इति सूत्रे असिद्धवत्सूत्रे च भाष्यकैयटयोः स्पष्टम्~।
अत्राङ्गशब्देन शब्दरूपं निमित्तमेव गृह्यते, शब्दशास्त्रे तस्यैव प्रधानत्वात्~।
तेन अर्थनिमित्तकस्य न बहिरङ्गत्वम्~।
अत एव ``न तिसृचतसृ'' इति निषेधश्चरितार्थः~।
अन्यथा स्त्रीत्वरूपार्थनिमित्तकतिस्रपेक्षयान्तरङ्गत्वात्त्रयादेशे तदसङ्गतिः स्पष्टैव~।
अत एव `त्रयादेशे स्रन्तस्य प्रतिषेधः' इति स्थानिवत्सूत्रस्थभाष्यवार्तिकादि सङ्गच्छते~।
एतेन `गौधेरः',`पचेरन्' इत्यादावेयादीनामङ्गसंज्ञासापेक्षत्वेन बहिरङ्गतयासिद्धत्वाद्वलिलोपो न स्यादिति परास्तम्~।
एयादेशादेरपरनिमित्तकत्वेनान्तरङ्गत्वाच्च~।
ननु ``येन विधिस्तदन्तस्य'' इति सूत्रे भाष्ये ``इकोयणचि'' इत्यादावपि तदन्तविधौ `स्योनः' इत्यत्र अन्तरङ्गत्वाद्यणो गुणबाधकत्वमिष्यते, तन्न सिध्येत्~, ऊनशब्दमाश्रित्य यणादेशः,
नशब्दमाश्रित्य गुण इत्यन्तरङ्गत्वाद्गुण एव स्यादित्युक्तम्~।
अत्र कैयटः~।
सिवेर्बाहुलकादौणादिके न प्रत्यये गुणवलोपोठां प्रसङ्गे ऊठपवादत्वाद्वलोपं बाधते, गुणं त्वन्तरङ्गत्वाद्बाधते~।
गुणो ह्यङ्गसम्बन्धिनीमिग्लक्षणां लघ्वीमुपधामार्धधातुकं चाश्रयति~।
ऊठ्तु वकारान्तमङ्गमनुनासिकादिञ्च प्रत्ययमित्यल्पापेक्षत्वादन्तरङ्गः~।
तत्र कृते यण्गुणौ प्रप्नुत इति~।
एवञ्च संज्ञापेक्षस्यापि बहिरङ्गत्वं स्पष्टमेवोक्तमिति चेन्न~।
तदन्तविधावपि बहुपदार्थापेक्षत्वरूपबहिरङ्गत्वस्य गुणे सत्वेन तत्र दोषकथनपरभाष्यासङ्गतेः~।
बहिरङ्गान्तरङ्गशब्दाभ्यां बह्वपेक्षत्वाल्पापेक्षत्वयोः शब्दमर्यादयालाभाच्च~।
तथा सत्यसिद्धं बह्वपेक्षमल्पापेक्ष इत्येव वदेत्~।
अत एव विप्रतिषेधसूत्रभाष्ये `गुणाद्यणादेशोऽन्तरङ्गत्वात्' इत्यस्य `स्योनः' इत्युदाहरणम् , न तु गुणादूठन्तरङ्गत्वादित्युक्तम्~।
त्वद्रीत्या तदपि वक्तुमुचितम्~।
प्राथम्यात्तदेव वा वक्तुमुचितम्~।
मम त्वन्तरङ्गपरिभाषया तद्वारणासम्भवात्तन्नोक्तम्~।
किञ्च सिद्धान्ते नित्यत्वाद्गुणात्पूर्वमूठ्, गुणस्तु यणा बाधितत्वादनित्यः~।
ऊनशब्दमाश्रित्येत्यादिभाष्यासङ्गत्या चिन्त्यम्~।
वलि लोपेऽन्तरङ्गपरिभाषा न प्रवर्तत इति तु न युक्तम्~।
तत्सूत्रे भाष्य एव व्रश्चादिषु लोपातिप्रसङ्गमाशङ्क्योपदेशसामर्थ्यान्न~।
न च `वृश्चति' इत्यादौ चारितार्थ्यम् , बहिरङ्गतया सम्प्रसारणस्यासिद्धत्वेन पूर्वमेव तत्प्राप्तेरिति भाष्योक्तेः~।
यत्तु नलोपस्य षट्संज्ञायामसिद्धत्वात् `पञ्च' इत्यत्र ``न षट्-'' इति निषेध इति, तच्चिन्त्यम्~।
न लोपस्य हि पदसंज्ञासापेक्षत्वेन बहिरङ्गत्वं वाच्यम् , तच्च न~, संज्ञाकृतबहिरङ्गत्वस्यानाश्रयणात्~।
`पञ्च' इत्यत्र निषेधस्तु स्त्रियां यत्प्राप्नोति, तन्नेति व्याख्यानसामर्थ्येन भूतपूर्वषट्‍त्वमादायेति बोध्यम्~।
अत एव कृतितुग्ग्रहणं चरितार्थम्~।
`वृत्रहभ्याम्' इत्यादौ पदत्वनिमित्तकत्वेऽपि नलोपस्य बहिरङ्गत्वाभावात्~।
भ्यामः पदसंज्ञानिमित्तत्वेऽपि नलोपस्य तन्निमित्तकत्वाभावात्~।
परम्परया निमित्तत्वमादाय बहिरङ्गत्वाश्रयणं तु न मानम्~।
ध्वनितं चेदं ``नलोपः सुप्'' इतिसूत्रे भाष्य इति, तत्रैव भाष्यप्रदीपोद्योते निरूपितम्~।
अन्तरङ्गे कर्तव्ये जातं तत्कालप्राप्तिकं च बहिरङ्गमसिद्धमित्यर्थः~।
व्रश्चादिषु पदसंस्कारपक्षे समानकालत्वमेव द्वयोरिति बोध्यम्~।
एतेन `अन्तरङ्गं बहिरङ्गाद्बलीयः' इति परिभाषान्तरमित्यपास्तम्~।
एनामाश्रित्य विप्रतिषेधसूत्रे भाष्ये तस्याः प्रत्याख्यानाच्च~।
अन्तरङ्गशास्त्रत्वमस्या लिङ्गम्~।
इयं च त्रिपाद्यां न प्रवर्तते, त्रिपाद्या असिद्धत्वात्~।
अस्याञ्च ``वाह ऊठ्'' सूत्रस्थमूठ्ग्रहणं ज्ञापकमित्येषा सपादसप्ताध्यायीस्था~।
अन्यथा सम्प्रसारणमात्रविधानेन लघूपधगुणे ``वृद्धिरेचि'' इति वृद्धौ `विश्वौहः' इत्यादिसिद्धेस्तद्वैयर्थ्यं स्पष्टमेव~।
सत्यां ह्येतस्यां बहिरङ्गसम्प्रसारणस्यासिद्धत्वाल्लघूपधगुणो न स्यात्~।
न च ``पुगन्त'' इति सूत्रे निमित्तमिको विशेषणम्~, अत एव `भिनत्ति' इत्यादौ न गुणः~।
एवं ``नाजानन्तर्ये'' इति निषेधात्कथं परिभाषाप्रवृत्तिरिति वाच्यम् , प्रत्ययस्याङ्गांश उत्थिताकाङ्क्षत्वेन तत्रैवान्वयात्~, ``पुगन्त-'' इत्यादौ कर्मधारयाश्रयणेन प्रत्ययपराङ्गावयवलघूपधारूपेको गुण इति ``इको गुणवृद्धी'' इतिसूत्रभाष्यसम्मतेऽर्थे `भिनत्ति' इत्यादावदोषाच्च~।
अकारान्तोपसर्गेऽनकारान्ते चोपपदे वहेर्वाहेर्वा ण्विविचावनभिधानान्न स्त एव~।
`वार्यूह' इत्यादि तु ऊहतेः क्विपि बोध्यम्~।
धातूनामनेकार्थत्वान्नार्थासङ्गतिः~।
`प्रौहः' इत्याद्यसाध्वेव, वृद्धेरप्राप्तेः~।
अस्योहस्यानर्थक्यान्न ``प्रादूहोढ'' इत्यस्यापि प्रवृत्तिः~।
न च कार्यकालपक्षे त्रिपाद्यामेतत्प्रवृत्तिर्दुर्वारेति वाच्यम् , पूर्वं प्रति परस्यासिद्धत्वादन्तरङ्गाभावेन पूर्वस्य तन्निरूपितबहिरङ्गत्वाभावात्तया तस्यासिद्धत्वप्रतिपादनासम्भवात्~।
न चानया पूर्वस्यासिद्धत्वादभावेन तं प्रति परासिद्धत्वं पूर्वत्रेत्यनेन वक्तुमशक्यमिति वाच्यम्~।
एवं हि विनिगमनाविरहादुभयोरप्यप्रवृत्त्यापत्तेः~।
किञ्च पूर्वत्रेत्यस्य प्रत्यक्षत्वेन तेनानुमानिक्या अस्या बाध एवोचितः~।
अतः कार्यकालपक्षेऽपि त्रिपाद्यामस्या अनुपस्थितिरेव~।
अत एव कार्यकालपक्षमेवोपक्रम्योक्तयुक्तीरुक्त्वा {\bfseries अतोऽयुक्तोऽयं परिहारो न वा बहिरङ्गलक्षणत्वात्} इत्युक्तं विसर्जनीयसूत्रभाष्ये सिद्धान्तिना~।
त्रिपादीस्थेऽन्तरङ्गे कर्तव्ये परिहारो न युक्त इति तदर्थः~।
किन्तु वचनमेवारब्धव्यमिति तदाशयः~।
अत एव `निगाल्यते' इत्यादौ लत्वार्थं `तस्य दोषः' इति वचनमेवारब्धम्~।
अन्यथान्तरङ्गत्वात् णिलोपात्पूर्वं वैकल्पिकलत्वे तद्वैयर्थ्यं स्पष्टमेव~।
येऽपि लक्ष्यानुरोधादानुमानिक्याप्यन्तरङ्गपरिभाषया प्रत्यक्षसिद्धस्य पूर्वत्रेत्यस्य बाधं वदन्ति, तेऽपि लक्षणैकचक्षूर्भिर्नाऽदर्तव्येति दिक्~।
अत एव ``ओमाङोश्च'' इत्याङ्ग्रहणं चरितार्थम्~।
तद्धि ``खट्वा आ ऊढः'' इत्यत्र परमपि सवर्णदीर्घं बाधित्वान्तरङ्गत्वाद्गुणे कृते वृद्धिप्राप्तौ पररूपार्थम्~।
साधनबोधकप्रत्ययोत्पत्त्यनन्तरं पूर्वं धातोरुपसर्गयोगे पश्चात् खट्वाशब्दस्य समुदायेन योगात्गुणस्यान्तरङ्गत्वमिति ``सम्प्रसारणाच्च'' इति सूत्रे भाष्ये स्पष्टम्~।
`एहि' इत्यनुकरणस्य शिवादिशब्दसम्बन्धे तु नास्य प्रवृत्तिः, ज्ञापकपर ``सम्प्रसारणाच्च'' इतिसूत्रस्थभाष्यप्रामाण्येनान्त्यं `प्रकृतिवदनुकरणम्' इत्यतिदेशमादाय लब्धाङ्‍त्वे एतदप्रवृत्तेः~।
यत्तु ``पूर्वं धातुरुपसर्गेण युज्यते पश्चात्साधनेन~।
उपसर्गेण तत्संज्ञकशब्देन, साधनेन कारकेण तत्प्रयुक्तकार्येण च~।
अत एव `अनुभूयते' इत्यादौ सकर्मकत्वात्कर्मणि लकारसिद्धिः' इति तन्न~।
क्रियायाः साध्यत्वेन बोधात्~, साध्यस्य च साधनाकाङ्क्षतया तत्सम्बन्धोत्तरमेव निश्चितक्रियाबोधेन साधनकार्यप्रवृत्त्युत्तरमेव क्रियायोगनिमित्तोपसर्गसंज्ञकस्य सम्बन्धौचित्यात्~।
अत एव ``सुट्‍कात्पूर्वः'' इति सूत्रे `पूर्वं धातुरुपसर्गेण' इत्युक्त्वा {\bfseries नैतत्सारं~, पूर्वं धातुः साधनेन युज्यते, पश्चादुपसर्गेण} इत्युक्त्वोक्तयुक्त्यास्यैव युक्तत्वमुक्तम्~, {\bfseries साधनं हि क्रियां निर्वर्तयति} इत्यादिना भाष्ये~।
उपसर्गद्योत्यार्थान्तर्भावेण धातुनैवार्थाभिधानादुक्तेषु कर्मणि लकारादिसिद्धिः~।
पश्चाच्छ्रोतृबोधाय द्योतकोपसर्गसम्बन्धः~।
एवञ्चान्तरङ्गतरार्थकोपसर्गनिमित्तः सुट् सम् कृ तीत्यवस्थायां द्वित्वादितः पूर्वं प्रवर्तते, ततो द्वित्वादि~।
अत एव ``प्रणिदापयति'' इत्यादौ णत्वं `यदागमाः' इति न्यायेन समाहितं भाष्ये~।
अत एव `प्रत्येति',`प्रत्ययः' इत्यादिसिद्धिः~।
यदुपसर्गनिमित्तकं कार्यमुसर्गार्थाश्रितं विशिष्टोपसर्गनिमित्तत्वात्तदन्तरङ्गम्~।
यत्तु न तथा, तत्र पूर्वागतसाधननिमित्तकमेवान्तरङ्गम्~।
अत एव ``न धातु'' इति सूत्रे {\bfseries `प्रेद्धः' इत्यत्र गुणो बहिरङ्गः} इति भाष्ये उक्तम्~।
किञ्च {\bfseries पूर्वमुपसर्गयोगे धातूपसर्गयोः समासे ऐकस्वर्याद्यापत्तिः} इति ``उपपदमतिङ्'' इति सूत्रे भाष्ये स्पष्टम्~।
भावार्थप्रत्ययस्यापि पूर्वमेवोत्पत्तिः~।
अत एव ``णेरध्ययने'' इति निर्देशः सङ्गच्छते~।
इदं च सामान्यापेक्षं ज्ञापकम् , भावतिङोऽपि पूर्वमुत्पत्तेः~।
अन्यथा तत्र समासापत्तिः~।
तिङि तु `अतिङ्' इति निषेधान्न तत्र दोषः, यदि भावतिङ्युपसर्गयोगोऽस्तीत्यलम्~।
यत्तु `विशेषापेक्षात्सामान्यापेक्षमन्तरङ्गं' विशेषापेक्षे विशेषधर्मस्याधिकस्य निमित्तत्वात्~।
यथा ``रुदादिभ्यः सार्वधातुके'' इत्यत्र रुदादित्वं सार्वधातुकत्वं च~।
तत्र सार्वधातुकज्ञानाय प्रकृतेर्धातुत्वज्ञानं प्रत्ययस्य प्रत्ययत्वज्ञानं चावश्यकमिति यासुडन्तरङ्गः~।
एतेन यत् ``अनुदात्तङितः'' इति सूत्रे कैयटेनोक्तं `लमात्रापेक्षयान्तरङ्गास्तिबादयो लकारविशेषापेक्षत्वाद्बहिरङ्गाः स्यादयः' इति, तत्परास्तम्~।
विशेषापेक्षत्वेऽपि तस्य सामान्यधर्मनिमित्तकत्वाभावेन तत्त्वस्य दुरुपपादत्वात्~, परनिमित्तकत्वेन स्यादीनां बहिरङ्गत्वाच्चेति, तन्न~।
विशेषस्य व्याप्यत्वेन व्यापकस्यानुमानेनोपस्थितावपि तस्य निमित्तत्वे मानाभावेनाधिकधर्मनिमित्तकत्वानुपपादनात्~,
भाष्ये एवंविधान्तरङ्ग बहिरङ्गभावस्य क्वाप्यनुल्लेखाच्च~।
यत्तु मतुप्सूत्रे भाष्ये ``पञ्च गावो यस्य सन्ति, स `पञ्चगुः' इत्यत्र मतुप्प्राप्तिमाशङ्क्य प्रत्येकमसामर्थ्यात्~, समुदायादप्रातिपदिकत्वात्~, समासात्  समासेनोक्तत्वादिति सिद्धान्तिनोक्ते {\bfseries नैतत्सारम्~, उक्तेऽपि हि प्रत्ययार्थे उत्पद्यते द्विगोस्तद्धितो यथा `पाञ्चनापितिः'} इति पूर्वपक्ष्युक्तिर्भाष्ये~।
 ``द्विगोर्लुगनपत्ये'' इति लुग्विधानात्तद्धितार्थद्विगोस्तद्धितो भवति, पञ्चगुशब्दश्च द्विगुरिति तदाशयं कैयटः~।
ततः `द्वैमातुरः',`पाञ्चनापितिः',`पञ्चसु कपालेषु संस्कृतः' इत्यादौ सावकाशद्विगोर्बहुव्रीहिणा प्रकृते परत्वाद्बाध इत्याशयेन {\bfseries नैष द्विगुः, कस्तर्हि~? बहुव्रीहिः} इति सिद्धान्तिनोक्ते तमवकाशमजानानोऽपवादत्वाद्द्विगुः प्राप्नोतीति पूर्वपक्षी~।
अन्यपदार्थे सुबन्तमात्रस्य विधीयमानबहुव्रीहेः सङ्ख्यायास्तद्धितार्थे विधीयमानो द्विगुर्विशेषविहितत्वाद्बाधकः प्राप्नोतीति कैयटः~।
ततः सिद्धान्त्येकदेश्याह {\bfseries अन्तरङ्गत्वाद्बहुव्रीहिः~।
कान्तरङ्गा? अन्यपदार्थे बहुव्रीहिः, विशिष्टेऽन्यपदार्थे द्विगुः, तस्मिंश्चास्य तद्धितेऽस्तिग्रहणं क्रियते} इति~।
अधिकास्त्यर्थापेक्षमत्वर्थनिमित्तो द्विगुर्बहिरङ्ग इति कैयट इति~।
नैषा सिद्धान्त्युक्तिः, एतावताप्यपवादत्वाहानेः~।
अच्सामान्यापेक्षयणो विशिष्टसवर्णाजपेक्षदीर्घेण बाधदर्शनात्~।
किञ्चोक्तरीत्या परत्वेन बाधसिद्धेः~।
किञ्चात्राधिकापेक्षत्वेनैव बहिरङ्गत्वम्~, न केवलविशेषापेक्षत्वेनेति न तद्भाष्यारूढं विशेषापेक्षस्य बहिरङ्गत्वम्~।
अत एव सुबन्तसामान्यापेक्षो बहुव्रीहिः, तद्विशेषापेक्षो द्विगुरिति नोक्तं भाष्ये~।
न चार्थकृतबहिरङ्गत्वस्यानाश्रयणादिदमयुक्तम्~, एकदेश्युक्तित्वेनादोषात्~।
अत एवास्तिग्रहणं नोपाध्यर्थम्~, किन्त्वस्तिशब्दान्मतुबर्थमिति त्वदभिमतं बहिरङ्गत्वमपि द्विगोर्नास्तीति प्रतिपाद्य सिद्धान्तिना ``मत्वर्थे द्विगोः प्रतिषेधो वक्तव्यः'' इति वचनेनैतत्सिद्धमित्युक्तम्~।
अत एव ``तदोः सः सावनन्त्ययोः'' इति सूत्रेऽनन्त्ययोरिति चरितार्थम्~।
अन्यथा प्रत्ययसामान्यापेक्षत्वेनान्तरङ्गत्वादन्त्यस्यात्वेऽनन्त्यस्यैव सत्वे सिद्धे तद्वैय्यर्थ्यं स्पष्टमेव~।
 ``पादः पत्'' इति सूत्रे भाष्यकैयटयोरप्येतदन्तरङ्गत्वाभाव एव सूचित इति सुधियो विभावयन्तु~।
नन्वेवं `असुस्रुवत्' इत्यत्र लघूपधगुणादुवङोऽल्पनिमित्तत्वाभावादुवङ्न स्यदिति चेन्न~।
तत्रान्तःकार्यत्वरूपान्तरङ्गत्वसत्त्वात्~।
अन्तःकार्यत्वं च पूर्वोपस्थितनिमित्तकत्वम्~, अङ्गशब्दस्य निमित्तपरत्वात्~।
इदमन्तरङ्गत्वं लोकन्यायसिद्धमिति {\bfseries मनुष्योऽयं प्रातरुत्थाय शरीरकार्याणि करोति, ततः सुहृदाम् , ततः सम्बन्धिनाम्~।
अर्थानामपि जातिव्यक्तिलिङ्गसंख्याकारकाणां बोधक्रमः शास्त्रकृत्कल्पितः~।
तत्क्रमेणैव च तद्बोधकशब्दप्रादुर्भावः कल्पित इति तत्क्रमेणैव तत्कार्याणीति, `पट्व्या' इत्यादावन्तरङ्गत्वात्पूर्वं पूर्वयणादेशः, परयणादेशस्य बहिरङ्गतयासिद्धत्वात्} इत्यनेन ``अचः परस्मिन्'' इति सूत्रे भाष्ये स्पष्टम्~।
तदपि युगपत्प्राप्तौ पूर्वप्रवृत्तिनियामकमेव~।
यथा 'पट्व्या' इत्यत्र पदस्य विभज्यान्वाख्याने, न तु जातस्य बहिरङ्गस्य तादृशेऽन्तरङ्गेऽसिद्धतानियामकं~, `प्रागुक्तलोकन्यायेन तथैव लाभात्' इति ``वाह ऊठ्'' सूत्रे कैयटे स्पष्टम्~।
अत एव वाय्वोरित्यादौ वलिलोपो यणः स्थानिवत्वेन वारितः,"अचः परस्मिन्'' इत्यत्र भाष्यकृता~।
क्रमेणान्वाख्याने तूक्तोदाहरणे पूर्वप्रवृत्तिकत्वमन्तरङ्गत्वं बहिरङ्गस्यासिद्धत्वमपि निमित्ताभावादप्राप्तिरूपं बोध्यम्~।
यत्तु एवं रीत्या पूर्वस्थानिकमप्यन्तरङ्गमिति, तच्चिन्त्यम्~।
 `स्रजिष्ठः' इत्यादौ विन्मतोर्लुकि टिलोपस्यापवादविन्मतोर्लुक्प्रवृत्त्या जातिपक्षाश्रयणेन वारणप्रयासस्य ``प्रकृत्यैकाच्'' इति सूत्रप्रयोजनखण्डनावसरे भाष्यकृत्कृतस्य नैष्फल्यापत्तेः~।
त्वदुक्तरीत्या विन्मतोर्लुको बहिरङ्गासिद्धत्वेनानायासतस्तद्वारणात्~।
भाष्य ईदृशरीत्या बहिरङ्गासिद्धत्वस्य क्वाप्यनाश्रयणाच्च, परिभाषायामङ्गशब्दस्य निमित्तपरत्वाच्च~।
इयं चोत्तरपदाधिकारस्थबहिरङ्गस्य नासिद्धत्वबोधिकेति ``इच एकाचोऽम्'' इति सूत्रे भाष्ये पूर्वपक्ष्युक्तिरिति सा नादर्तव्या~।
`परन्तपः' इत्यादावनुस्वारे नासिद्धत्वं मुमस्त्रिपाद्यां तदप्रवृत्तेः~।
नव्यमतेऽपि यथोद्देशपक्षाश्रयणेनान्यथासिद्धोदाहरणदानेन तस्य तदुक्तित्वमावश्यकमित्याहुः~।
आभीयेऽन्तरङ्गे आभीयस्य बहिरङ्गस्य समानाश्रयस्य नानेनासिद्धत्वम्~, अन्तरङ्गस्यासिद्धत्वादित्यसिद्धवत्सूत्रे भाष्ये स्पष्टम्~।
एवं सिचि वृद्धेः येन नाप्रप्तिन्यायेनान्तरङ्गबाधकत्वमूलकम् {\bfseries न सिच्यन्तरङ्गमस्ति} इति ``इको गुण'' इति सूत्रे भाष्ये स्पष्टम्~॥\par
नन्वेवम् `अक्षद्यूः' इत्यादौ बहिरङ्गस्योठोऽसिद्धत्वादन्तरङ्गो यण्न स्यात्~, अत आह-
\section*{\begin{center}नाजानन्तर्ये बहिष्ट्वप्रकॢप्तिः॥५१॥\end{center}}
\addcontentsline{toc}{section}{नाजानन्तर्ये}
अत्र ``षत्वतुकोः'' इति सूत्रस्थं तुग्ग्रहणं ज्ञापकम्~।
अन्यथा `अधीत्य'~, `प्रेत्य' इत्यादौ समासोत्तरं ल्यप्प्रवृत्त्या पूर्वं समासे जाते तत्र संहिताया नित्यत्वाद्ल्यबुत्पत्तिपर्यन्तमप्यसंहितयाऽवस्थानासम्भवेन एकादेशे ल्यपि तुगपेक्षया पदद्वयसम्बन्धिवर्णद्वयापेक्षैकादेशस्य बहिरङ्गतयासिद्धत्वेन तद्वैयर्थ्यं स्पष्टमेव~।
 ``पदद्वयसम्बन्धिवर्णद्वयापेक्षं बहिरङ्गम्'' इति `प्रेद्धः' इत्यादौ गुणो बहिरङ्ग इति ग्रन्थेन ``न धातुलोप'' इति सूत्रे ``संयोगान्तस्य लोपः'' इति सूत्रे च भाष्ये स्पष्टम्~।
यत्तु षत्वग्रहणमपि ज्ञापकम्~, अन्यथा ``कोऽसिचत्'' इत्यादौ पदद्वयसम्बन्धिवर्णद्वयापेक्षत्वेन बहिरङ्गस्यैकादेशस्यासिद्धत्वेन षत्वप्रवृत्तौ किं तेनेति~।
तन्न, इणः पूर्वपदसम्बन्धित्वेन षत्वस्यापि पदद्वय सम्बन्धिवर्णद्वयापेक्षत्वेनोभयोः समत्वात्~।
एकादेशस्य परादित्वेन `ओसिचत्' इत्यस्य पदत्वेन पदादित्वाभावान्न ``सात्पदाद्योः'' इत्यनेन निषेधः~।
त्रैपादिकेऽन्तरङ्गे कार्यकालपक्षेऽपि बहिरङ्गपरिभाषाया अप्रवृत्तेः पूर्वमुपपादितत्वाच्च~।
परिभाषार्थस्तु अचोऽन्यानन्तर्यनिमित्तकेऽन्तरङ्गे कर्तव्ये जातस्य बहिरङ्गस्य बहिष्ट्वप्रकॢप्तिर्न~।
बहिष्पदेन बहिरङ्गम्~, तस्य भावो बहिष्ट्वम् , बहिरङ्गत्वम्~।
तत्प्रयुक्तासिद्धत्वस्य न प्रकॢप्तिः, न प्राप्तिरिति~।
 `असिद्धं बहिरङ्गम्' इत्युक्त्वा {\bfseries नाजानन्तर्य इति वक्ष्यामि} इति भाष्योक्त्या तत्रत्यस्य अन्तरङ्ग इत्यस्यानुवृत्तिसूचनात्~।
तेन `पचावेदम्' इत्यादौ न दोषः~।
अन्तरङ्गस्याच्स्थानिककार्यस्यैत्वस्य अन्यानन्तर्यनिमित्तकत्वाभावात्~।
 `जातस्य बहिरङ्गस्य' इत्युक्त्या `अजये इन्द्रम्'~, `धियति' इत्यादौ बहिरङ्गदीर्घगुणादेरसिद्धत्वं सिद्धम्~।
अत एव ``इण्ङिशीनामाद्गुणः, सवर्णदीर्घत्वाच्छचङन्तस्यान्तरङ्गलक्षणत्वात्'' इत्यादिसङ्गच्छते~।
अत एव ``ओमाङोश्च'' इत्याङ्ग्रहणं चरितार्थम्~।
तद्धि `शिवा आ एहि' इति स्थिते परमपि सवर्णदीर्घं बाधित्वा धातूपसर्गकार्यत्वेनान्तरङ्गत्वाद्गुणे वृद्धिबाधनार्थम्~।
ननु ``अक्षद्यूः'' इत्यत्र यणि कृते ऊठोऽसिद्धत्वाद्वलिलोपापत्तिरिति वाच्यम्~, अचोऽन्यानन्तर्यनिमित्तकेऽन्तरङ्गे कर्तव्ये कृते च तस्मिन्यदन्तरङ्गं प्रप्नोति, तत्र च कर्तव्ये नासिद्धत्वमिति तदर्थात्~।
असिद्धपरिभाषाया अनित्यत्वेन तद्वारणे त्वस्या वैयर्थ्यं तेनैव सिद्धेः~।
अत एव ``न लोपः सुप्'' इति सूत्रे कृति तुग्ग्रहणं चरितार्थम्~।
अन्यथा `वृत्रहभ्याम्' इत्यादौ बहिर्भूतभ्याम्निमित्तकपदत्वाश्रयत्वेन बहिरङ्गतया नलोपस्यासिद्धत्वेन सिद्धेस्तद्वैयर्थ्यं स्पष्टमेव~।
मम तु क्यजानन्तर्यसत्वान्न दोषः~।
न चैवं सति ``ह्रस्वस्य पिति'' इतिसूत्रस्य भाष्यविरोधः~।
तत्र हि `ग्रामणि पुत्रः' इत्यत्र ``इको ह्रस्वोऽङ्यः'' इति ह्रस्वे कृते तुकमाशङ्क्य ह्रस्वस्य बहिरङ्गासिद्धत्वेन समाहितम्~।
 ``नाजानन्तर्य'' इत्यस्य सत्वे तत्र तदप्राप्तेरसङ्गतिः स्पष्टमेवेति वाच्यम्~।
तेन भाष्येणास्या अनवकाशत्वबोधनात्~।
एतज्ज्ञापकेन अन्तरङ्गपरिभाषाया अनित्यत्वबोधनस्यैव न्याय्यत्वात्~।
अत एव ``अचः परस्मिन्'' इति सूत्रे भाष्ये `पटु ई आ' इत्यत्र परयणादेशस्य अनयासिद्धत्वात्पूर्वयणादेशः साधितः~।
अत एवैषा भाष्ये पुनः क्वापि नोल्लिखिता~।
अत एव अन्तरङ्गपरिभाषामुपक्रम्य विप्रतिषेधसूत्रेऽस्या बहूनि प्रयोजनानि सन्ति, तदर्थमेषा परिभाषा कर्तव्या, प्रतिविधेयं च दोषेषु इत्युक्तं ``सम्प्रसारणाच्च" इति सूत्रे भाष्ये~।
प्रतिविधानं च परिभाषाविषयेऽनित्यत्वाश्रयणमेवेति ध्वनितमित्यलम्~॥


नन्वेवं `गोमत्प्रियः' इत्यादौ पदद्वयनिमित्तकसमासाश्रितत्वेन बहिरङ्गं लुकं बाधित्वान्तरङ्गत्वाद्धल्ङ्यादिलोपे नुमादयः स्युः, अत आह-
\section*{\begin{center}अन्तरङ्गानपि विधीन्बहिरङ्गो लुग्बाधते~॥५२॥\end{center}}
\addcontentsline{toc}{section}{अन्तरङ्गानपि विधीन्}
अत्र च ``प्रत्ययोत्तरपदयोश्च'' इति सूत्रं ज्ञापकम्~।
`त्वत्कृतम्' इत्यादौ लुगपेक्षया अन्तरङ्गत्वाद्विभक्तिनिमित्तकेन ``त्वमावेकवचने'' इत्यनेन सिद्ध इदं व्यर्थं सत्तज्ज्ञापकम्~।
ननु `तव पुत्रः, त्वत्पुत्रः' इत्यादौ तवममादिबाधनार्थं तदावश्यकमिति चेत्~, एवं तर्ह्यत्रत्यमपर्यन्तग्रहणानुवृत्तिस्तज्ज्ञापिकेति भाष्यकृतः~।
युष्मदादिभ्य आचारक्विप्तु न, सम्पूर्णसूत्रस्य ज्ञापकतापरभाष्यप्रामाण्यात्~, ``ह्रस्वनद्यापः'' इति नुड्विधायकसूत्रस्थभाष्यप्रामाण्येन हलन्तेभ्यः आचारक्विबभावाच्च~।
एवमेवैकार्थकाभ्यां प्रातिपादिकाभ्यां प्रातिपदिकणिचोऽप्यनभिधानं बोध्यम्~।
एतेन तत्राऽदेशार्थं प्रत्ययग्रहणं चरितार्थमित्यपास्तम्~।
ननु मपर्यन्तानुवृत्तिरपि सर्वादेशत्ववारणाय चरितार्था~।
 न चोत्सर्गसमानादेशा अपवादाः इति न्यायेनासिद्धवत्सूत्रस्थभाष्यसम्मतेन मपर्यन्तस्यैवादेशे सिद्धे तदनुवृत्तिर्व्यर्थेति वाच्यम्~, तस्य श्नमकजादौ व्यभिचारादिति चेन्न~।
श्नमि मित्वेन, बहुचि पुरस्ताद्ग्रहणेन, अकचि प्राक्टेर्ग्रहणेन तस्य बाधेऽप्यत्रोत्सर्गस्य त्यागे मानाभावात्~।
अत एव ``तस्मिन्नणि च'' इत्यनेन युष्माकाद्यादेशविधानं चरितार्थम्~, अन्यथा आकङादेशमेव विदध्यात्~।
आकङि तवकाद्यादेशयोरेतदपवादयोरुक्तन्यायेनान्त्यादेशत्वापत्तिः, अतस्तद्विधानम्~, इदमेव च तज्ज्ञपकम्~।
यद्यपि विरोधे बाधकत्वमिति वार्तिकमतेऽयं न्यायः, भाष्यकारस्तु विनापि विरोधं सत्यपि सम्भवे बाधकत्वमिच्छति, इत्यनभिहितसूत्रस्थकैयटरीत्या नायं नियमः, तथापि युष्मकाद्याऽदेशविधानज्ञापित उत्सर्गः स्वीक्रियत एवेति प्रकृते न दोषः~।
एतद्भाष्यमपि तत्स्वीकारे मानम्~।
एवञ्च मपर्यन्तानुवृत्तिः `त्वत्कृतम्' इत्यादौ मपर्यन्तस्याऽदेशविधानार्था~।
तत्र च अन्तरङ्गत्वात्त्वमावित्येव सिद्धे व्यर्था सैतज्ज्ञापिका~।
ज्ञापिते त्वस्मिन्नेतद्विषये तवादीनामप्राप्त्या तदपवादत्वाभावेन मपर्यन्तस्यैवाऽदेशार्थं सा चरितार्थेति तदाशयः~।
यत्तु हरदत्तेनान्तरङ्गप्रवृत्तौ प्रत्यय उत्तरपदे च मपर्यन्तासम्भवेन तदनुवृत्तिर्व्यर्था सती ज्ञापिकेत्युक्तम्~, तन्न~।
अन्तरङ्गाणामप्यपवादबाध्यत्वेन तद्विषये तदप्रवृत्तेः~।
वस्तुत इदं ज्ञापकं वार्तिकरीत्यैव, भाष्यरीत्या तु वाचनिक एवायमर्थ इत्याहुः~।
इयं ``सुपो धातु" इति लुग्विषयैवेति केचित्~।
 ``एङ्ह्रस्वात्सम्बुद्धेः''~, ``न यासयोः'' इति सूत्रस्थाकरप्रामाण्येन लुङ्मात्रविषया~।
आद्ये `हे त्रपु' इत्यादावनेन न्यायेन लोपं बाधित्वा लुग्भवतीति भाष्ये उक्तम्~।
अन्त्येऽन्तरङ्गांश्च विधीन्सर्वोऽपि लुग्बाधते, न तु सुब्लुगेव~।
अत एव ``सनीस्रंस'' इत्यादौ नलोपो न भवति~।
 `पञ्चभिः खट्वाभिः क्रीतः, पञ्चखट्वः' इत्यादावेकादेशात्प्रागेव टापो लुक्~।
अन्यथा कृतैकादेशस्य लुक्यकारश्रवणं न स्यादिति कैयटे उक्तम्~।
एतद्विरोधाद्यत् ``तद्राजस्य'' इति सूत्रे कैयटेनोक्तं `अङ्गानतिक्रान्तोऽत्यङ्गः' इत्यत्र सुपो लुकि बहुवचनपरत्वाभावात् ``तद्राजस्य'' इति लुग्न स्यादिति शङ्कापरभाष्यव्याख्यावसरे अन्तरङ्गानपीति न्यायेनायं लुक्सुब्लुको बाधकः स्यादित्याशङ्क्य सुब्लुक एवानेन बलवत्वं बोध्यत इति तत्प्रौढ्येति द्रष्टव्यम्~।
लुगपेक्षया लुको बलवत्वस्य वक्तुमशक्यत्वादिति तदा शङ्का समाधानं वक्तुं युक्तम्~।
अनेन न्यायेनान्तरङ्गनिमित्तविनाशकलुकस्तत्प्रयोजकसमासादीनां च प्राबल्यं बोध्यत इत्यन्यत्र विस्तरः~॥
नन्वेवं सौमेन्द्रेऽन्तरङ्गत्वादाद्गुणे पूर्वपदात्परेन्द्रशब्दाभावेन ``नेन्द्रस्य परस्य'' इति वृद्धिनिषेधो व्यर्थः~।
अन्तादिवद्भावस्तूभयत आश्रयणे निषिद्धः~।
किञ्च वृद्धिरप्यत्र न प्राप्नोति, अन्तादिवत्त्वोभयाभावेऽपि पूर्वान्तवत्त्वेनैकादेशविशिष्टे पूर्वपदत्वेनेन्द्रशब्दस्य एकदेशविकृतन्यायेन `उभयत आश्रयणे नान्तादिवत्' इत्यस्याभावेन तदाश्रयणेन वोत्तरपदत्वेऽपि तस्यानच्कत्वादेकस्यैकादेशेन परस्य नित्येन ``यस्येति'' लोपेनापहारात्~।
न च परादिवद्भावेनैकादेशविशिष्टस्योत्तरपदत्वमेवास्त्विति तत्सम्भव इति वाच्यम्~।
उत्तरपदाद्यच्स्थानिकत्वाद्वृद्धेस्तदभावेनाप्राप्तेस्ताद्रूप्यानतिदेशात्~।
अन्यथा `खट्वाभिः' इत्यादावपि पूर्वान्तवत्वेनादन्तत्वे भिस ऐसापत्तिरिति भाष्ये स्पष्टम्~।
अत एव `पूर्वेषुकामशमः' इत्यादावन्तरङ्गत्वाद्गुणे वृद्धिर्न स्यादित्याशङ्कितम्~।
तदेकदेशमात्रस्य विकाराभावाच्च~।
तदुक्तं भाष्ये {\bfseries इन्द्रे द्वावचौ, एकः "यस्येति'' लोपेनापहृतोऽपर एकादेशेन, ततोऽनच्क इन्द्रशब्दः सम्पन्नस्तत्र कः प्रसङ्गो वृद्धे}रिति~।\par
मरुदादिभिरिन्द्रस्य द्वन्द्वे इन्द्रस्यैव पूर्वनिपातः, अत आह-

\section*{\begin{center}पूर्वोत्तरपदनिमित्तकार्यात्पूर्वमन्तरङ्गेऽप्येकादेशो न~॥५३॥\end{center}}
\addcontentsline{toc}{section}{पूर्वोत्तरपदनिमित्तकार्यात्}
अत्र च ``नेन्द्रस्य'' इति निषेध एव ज्ञापक इति ``अन्तादिवच्च''~, ``विप्रतिषेधे परम्'' इतिसूत्रयोर्भाष्ये स्पष्टम्~॥\par
नन्वेवमपि `प्रधाय, प्रस्थाय' इत्यादावन्तरङ्गत्वात् हित्वादिषु कृतेषु ल्यप्स्यादत आह-
\section*{\begin{center}अन्तरङ्गानपि विधीन्बहिरङ्गो ल्यब्बाधते~॥५४॥\end{center}}
\addcontentsline{toc}{section}{अन्तरङ्गानपि विधीन्}
 ``अदो जग्धिः'' इतिसूत्रे ति कितीत्येव सिद्धे ल्यब्ग्रहणमस्या ज्ञापकमिति ``अदो जग्धिः'' इत्यत्र भाष्ये स्पष्टम्~॥
नन्वेवमपि `इयाय' इत्यादौ द्वित्वे कृतेऽन्तरङ्गत्वात्सवर्णदीर्घत्वे तदसिद्धिरत आह-
\section*{\begin{center}वार्णादाङ्गं बलीयो भवति~॥५५॥\end{center}}
\addcontentsline{toc}{section}{वार्णादाङ्गम्}
तेनान्तरङ्गमपि सवर्णदीर्घं बाधित्वा वृद्धिरिति तत्सिद्धिः~।
 ``अभ्यासस्यासवर्णे'' इतीयङ्विधायकसूत्रस्थमसवर्णग्रहणमस्या ज्ञापकम्~।
तद्धि `ईषतुः' इत्यादावियङादिव्यावृत्त्यर्थम्~।
एतत्परिभाषाभावे तु `ईषतुः' इत्यादावन्तरङ्गेण सवर्णदीर्घेण बाधात्तद्व्यर्थम्~।
इयङुवङौ ह्यभ्याससम्बन्धनिमित्तकत्वाद्बहिरङ्गौ~।
न चेयङादिरपवादः, येन नाप्राप्तिन्यायेन `इयति' इत्यादिसकललक्ष्यप्राप्तयणपवादत्वस्यैव निर्णयादिति प्राञ्चः~।
परे तु एतत्परिभाषाभावे ``अभ्यासस्य'' इति सूत्रमेव व्यर्थम्~।
न च `इयाय',`इयेष' इत्यादौ चरितार्थम्~।
तयोरपि पूर्वप्रवृत्तगुणस्य पूर्वप्रवृत्तवृद्धेश्च ``द्विर्वचनेऽचि'' इति रूपातिदेशेनापहारे द्वित्वे कृते पुनः प्राप्ते गुणवृद्धी बाधित्वान्तरङ्गत्वात्सवर्णदीर्घापत्तेः~।
न च `इयति' इत्यादौ तच्चरितार्थम् , तावन्मात्रप्रयोजकत्वे `उः' इत्येव ब्रूयात्~।
य्वोरित्यनुवर्तते, ``इणो यण्'' इति साहचर्याद्व्याख्यानाच्च ऋधातोरेव ग्रहणम्~।
अतः इवर्णस्येयङित्यर्थः~।
 ``अभ्यासस्यायतौ'' इति ``अभ्यासस्यार्तेः'' इति वा गुरुत्वान्न युक्तम्~।
न च ``ए,ऐ,ओ,औ'' शब्देभ्य आचारक्विबन्तेभ्यो लिटीयङाद्यर्थं तत्सूत्रमनावश्यकम् , तथा ``उवोणकीयिषति'' इत्याद्यर्थमावश्यकमिति वाच्यम्~, षाष्ठप्रथमाह्निकान्तस्थभाष्यप्रामाण्येन तेषामनभिधानात्~।
अन्त्ये द्वितीयद्विर्वचनस्यैव सत्त्वेन त्वदुक्तप्रयोगस्यैव दुर्लभत्वात्~।
एवञ्च सम्पूर्णसूत्रस्य ज्ञापकता युक्ता~।
यद्यपि भाष्ये {\bfseries यदयमभ्यासस्यासवर्ण इत्यसवर्णग्रहणं करोति} इति ग्रन्थेनासवर्णग्रहणस्यैव ज्ञापकता लभ्यते, तथापि {\bfseries न ह्यन्तरेण गुणवृद्धी असवर्णपराभ्यासो भवति} इति तदुपपादन ग्रन्थेन सम्पूर्णसूत्रस्यैव ज्ञापकता लभ्यते~।
अग्रेऽपि {\bfseries नैतदस्ति ज्ञापकम्~, अत्यर्थमेतत्स्यात्} इत्यनेन सूत्रसार्थक्यमेव दर्शितम्~।
असवर्णग्रहणस्यैव ज्ञापकत्वे तु तद्व्यावर्त्यप्रदर्शनेन तत्सार्थक्यमेव दर्शितं स्यात्~।
न च अकृतव्यूहपरिभाषया `इयेष' इत्यादौ सवर्णदीर्घाप्राप्तिः, यदि दीर्घो न स्यात्~, तर्हि गुणः स्यादिति सम्भावनायाः सत्वेन परिभाषाप्रवृत्तेः सूपपादत्वादिति कथं सम्पूर्णसूत्रस्य ज्ञापकतेति वाच्यम्~।
 तस्या असत्त्वात्~।
सत्त्वे वैतद्भाष्यप्रामाण्येन यत्रान्तरङ्गकार्यप्रवृत्तियोग्यकालोत्तरमेव तन्निमित्तविनाशकबहिरङ्गविधेः प्राप्तिः, तत्रैव तत्परिभाषाप्रवृत्तिस्वीकाराच्च~।
न चान्तरङ्गत्वाद्दीर्घोऽपि `इयाय' इत्यादौ पूर्वान्तवत्त्वेनाभ्यासत्वादिवर्णत्वाच्च णल्यसवर्ण इयङ्‍-विधानेन सूत्रं चरितार्थम्~।
न च ``अचि श्नु'' इत्यनेन सिद्धिः, वृद्धिबाधनार्थत्वादिति वाच्यम्~।
प्रत्यासत्त्या सवर्णपदेन अभ्यासोत्तरखण्डसम्बन्ध्यसवर्णाच एव ग्रहणात्~, शास्त्रबाधकल्पनापेक्षया परिभाषाज्ञापकत्वस्यैवौचित्याच्चेत्याहुः~।
सा चेयं धर्मिग्राहकमानादाङ्गवार्णयोः समानकार्यित्व एव~।
यत्तु समाननिमित्तकत्वरूपसमानाश्रयत्व एवैषेति, तन्न~।
ज्ञापितेऽपि ``इयाय'' ``इयेष''  इत्याद्यसिद्धेः, सूत्रवैयर्थ्यस्य तदवस्थत्वाच्च~।
 `स्योनः' इत्यत्र तु वक्ष्यमाणरीत्यास्या अनित्यत्वादप्रवृत्तौ गुणादन्तरङ्गत्वाद्यणादेशः~।
न चैवमपि `इयाय' इत्यादावियङ्‍दुर्लभः, तत्र कर्तव्ये वृद्ध्यादेः स्थानिवत्त्वेन `असवर्णे' इति प्रतिषेधादिति वाच्यम्~, सूत्रारम्भसामर्थ्यादेव स्थानिवत्त्वाप्रवृत्तेः~।
तच्च सामान्यापेक्षम्~।
अभ्यासकार्ये तदुत्तरखण्डादेशस्य तत्कार्यप्रतिबन्धकीभूतं स्थानिवत्त्वं नेति~।
अत एव `आरति' इत्यादौ यणादेशस्य स्थानिवत्वादभ्यासस्य ``ढ्रलोपे'' इति दीर्घो दुर्लभ इत्यपास्तम्~।
दीर्घविधौ तन्निषेधाच्च~।
 ``अरियियात्'' इत्यत्र स्थानिवत्वेनेयङ्भवत्येव, तस्य स्थानिवत्वस्याभ्यासकार्यप्रतिबन्धकत्वाभावात्~।
  {\bfseries  इयञ्चाङ्गसम्बन्धिन्याङ्ग एव} इति ``स्वरितो वा'' इति सूत्रे भाष्ये~।
तत्र हि `कुमार्यै' इत्यादौ यणुत्तरमाडुक्तः~।
इयञ्चानित्या  ``छ्वोः'' इति सतुग्निर्देशात्~।
अन्यथाङ्गत्वात्पूर्वं तुकः शादेशे तुकोऽप्राप्त्या तद्वैयर्थ्यं स्पष्टमेवेत्यन्यत्र विस्तरः~॥\par
नन्वेवं `सेदुषः' इत्यादौ क्वसोऽन्तरङ्गत्वादिटि, ततः सम्प्रसारणेऽपीटः श्रवणापत्तिरिति चेत्~, अत्र केचित्-

\section*{\begin{center}अकृतव्यूहाः पाणिनीयाः~॥५६॥\end{center}}
\addcontentsline{toc}{section}{अकृतव्यूहाः}
न कृतो विशिष्ट ऊहः निश्चयः, शास्त्रप्रवृत्तिविषयो यैरित्यर्थः~।
भाविनिमित्तविनाश इत्यद्याहारः~।
बहिरङ्गेणान्तरङ्गस्य निमित्तविनाशे पश्चात्सम्भावितेऽन्तरङ्गं नेति यावत्~।
अत्र च ज्ञापकम्~, ``समर्थानां प्रथमात्'' इति सूत्रे समर्थानामिति~।
तद्धि सूत्थितादिभ्यः कृतदीर्घेभ्यः प्रत्ययोत्पत्त्यर्थम्~।
अन्यथान्तरङ्गत्वाद्दीर्घे कृत एव प्रत्ययप्राप्त्या तद्व्यर्थता स्पष्टैव~।
न चात्रैकादेशप्रवृत्तिसमये वृद्ध्यप्राप्त्यैकादेशे कृत आदेशे वृद्धेः प्राप्तावपि तन्निमित्तविनाशाभाव इति वाच्यम्~, तद्द्वारैव तन्निमित्तविनाशसत्वेनाक्षतेः~।
न च सौत्थितौ बहिरङ्गतया वृद्धेरसिद्धत्वान्न तन्निमित्तविनाश इति वाच्यं~, समर्थग्रहणेनैतद्विषये तस्या अप्रवृत्तेरपि ज्ञापनात्~।
यत्तु समर्थग्रहणेनान्तरङ्गपरिभाषाया अनित्यत्वमेव ज्ञाप्यत इति~।
तन्न, असिद्धपरिभाषाया समकालप्राप्तबहिरङ्गस्य पूर्वं जातबहिरङ्गस्य चान्तरङ्गे कर्तव्येऽसिद्धत्वं बोध्यते, न तु जातेऽन्तरङ्गे तस्य तत्त्वं बोध्यते, मानाभावात्~, फलाभावाच्च~।
एवञ्च सूत्थितावेकादेशस्य परिभाषासाध्यत्वाभावेन तदनित्यत्वज्ञापनासम्भवात्~।
 ``अन्तरङ्गानपि विधीन्'' इत्यादेरप्यस्यामेवान्तर्भावः~।
एतत्प्रवृत्तौ च निमित्तविनाशसम्भावनापि निमित्तम्~।
अत एव `गोमद्दण्डी' इत्यादौ हल्ङ्यादिलोपो न~।
अन्यथा हल्ङ्यादिलोपकाले सामासिकलुकोऽप्राप्त्या तदुत्तरं चापहार्याभावादप्राप्त्या लोपस्यैवापत्तेः~।
अस्ति चात्रापि `यदि लोपो न स्यात्, तर्हि लुक्स्यात्' इति सम्भावना~।
 ``अल्लोपोऽनः'' इतिसूत्रस्थतपरकरणं तु परिभाषानित्यत्वज्ञापनेन चरितार्थम्~।
तद्धि `आन' इत्यादौ लोपवारणाय~।
अन्यथा दीर्घाभावे लोपसम्भावनयैतत्परिभाषाबलाद्दीर्घाप्राप्तौ तद्वैयर्थ्यं स्पष्टमेवेत्याहुः~।
 ``समर्थानाम्'' इति सूत्रे कैयटस्तु समर्थवचनेनेयं परिभाषा ज्ञाप्यते `अकृतव्यूहाः पाणिनीयाः' इति~।
तेन `पपुषः' इत्यादावन्तरङ्गात्पूर्वं कृतोऽपीडागमो निवर्तत इति वदन् `न कृतो व्यूहः, विशिष्टस्तर्को, निमित्तकारणविनाशेऽपि कार्यस्थितिरूपो यैः' इत्यर्थमभिप्रैति~।
`निमित्तापाये नैमित्तिकस्याप्यपायः' इति यावत्~।
सूत्थितादिञि वृद्धौ दीर्घनिवृत्तौ सावुत्त्थितिर्मा भूदिति समर्थानामिति~।
लोकन्यायसिद्धश्चायमर्थः~।
तथा हि लोके निमित्तं द्विविधं दृष्टं कार्यस्थितौ नियामकं तदनियामकं च~।
आद्यं यथा न्यायनयेऽपेक्षा बुद्धिः, तन्नाशे द्वित्वनाशाभ्युपगमात्~।
वेदान्तिनये प्रारब्धस्य विक्षेपस्थितिनियामकत्वं च प्रसिद्धमेव~।
द्वितीयं यथा दण्डादि, तन्नाशेऽपि घटनाशदर्शनात्~।
शास्त्रे लक्ष्यानुरोधाद्व्यवस्था~।
भाविनिमित्तविनाशे पूर्वमनुत्पत्तौ तु न कश्चिन्न्यायः नापि सम्प्रतिपन्नो दृष्टान्तः~।
समर्थानामित्यस्यापि लोकसिद्धार्थज्ञापनेन चारितार्थ्यसम्भवे, लोकासिद्धापूर्वतादृशार्थज्ञापकत्वे मानाभाव इति तदाशय इति बोध्यम्~।
परे तु `सेदुषः' इत्यादौ पदावधिकेऽन्वाख्याने `सेद्वस् अस्' इति स्थिते इट्‍सम्प्रसारणयोः प्रप्तौ प्रतिपदविधित्वात्पूर्वं सम्प्रसारणे वलादित्वाभादिटः प्राप्तिरेव नेति तत्सिद्धिरिति स्पष्टं ``समर्थानाम्'' इति सूत्रे कैयटे, ``असिद्धवत्'' सूत्रे कैयटे च स्पष्टमेतत्~।
यद्यपि प्रतिपदविधित्वमनवकाशत्वे सत्येव बाधकत्वे बीजम् , तथापि पूर्वप्रवृत्तौ सावकाशत्वेऽपि नियामकं भवत्येवेति तदाशयः~।
निरूपितं चैतद्बहुशः शब्देन्दुशेखरादौ~।
 ``समर्थानाम्'' इति सूत्रस्थसमर्थग्रहणं तु `विषुणः' इत्यादावकृतसन्धेः प्रत्ययदर्शनेन सर्वत्र तथा भ्रमवारणाय न्यायसिद्धार्थानुवाद एव~।
ध्वनितं चेदं विप्रतिषेधसूत्रे भाष्ये~।
तत्र हि `वैक्षमणिः' इत्यन्तरङ्गपरिभाषोदाहरणमुक्तम्~।
किञ्च विभज्यान्वाख्याने `सु उत्थित अस्' इति स्थिते `वार्णादाङ्गं बलीयः' इति प्राप्तवृद्धिवारणाय समर्थग्रहणमित्यत्रैव कैयटे स्पष्टम्~।
अत एव ``असिद्धवत्'' सूत्रे {\bfseries वसुसम्प्रसारणमज्विधौ सिद्धं वक्तव्यम्~। `पपुषः' इत्यादौ वसोः सम्प्रसारणे कृते आतो लोपो यथा स्यात्} इति भाष्ये उक्तम्~।
पदस्य विभज्यान्वाख्याने पूर्वोक्तकैयटरीत्या पूर्वं सम्प्रसारणे इटोऽप्राप्तावुस्निमित्तक एवाऽतो लोप इति तदसङ्गतिः~।
अत एव `चौ प्रत्यङ्गस्य प्रतिषेधः' इति वचनं वार्तिककृताऽरब्धम्, भाष्यकृता च न प्रत्याख्यातम्~।
प्रत्यङ्गं अन्तरङ्गम्~।
अस्यां परिभाषायां सत्यां तु तद्वैयर्थ्यं स्पष्टमेव~।
अत एव ``छ्वोः'' इति सूत्रे {\bfseries अवश्यमत्र तुगभावार्थो यत्नः कार्यः~। अन्तरङ्गत्वाद्धि तुक्प्राप्नोति} इति भाष्ये उक्तम्~।
एतत्सत्वे तु तुकोऽप्राप्त्या यत्नावश्यकत्वकथनमसङ्गतमिति स्पष्टमेव~।
न चैतदनित्यत्वज्ञापनार्थमेव तदिति तदाशयः, अवश्यमत्रेत्यक्षरस्वारस्यभङ्गापत्तेः~।
किञ्चानयैव ``प्रत्ययोत्तरपदयोश्च'' ``अदो जग्धिर्ल्यप्ति किति'' इत्यनयोश्चारितार्थ्येन तज्ज्ञापकवशाल्लुग्ल्यपोरन्तरङ्गबाधकता भाष्योक्ता भज्येत~।
किञ्चैषा भाष्ये न दृश्यते~।
तदुक्तम् ``असिद्धवत्'' सूत्रे कैयटेन `निमित्तापाये नैमित्तिकस्याप्यपायः इति परिभाषायाः भष्यकृतानाश्रयणात्' इति~।
पदसंस्कारपक्षे `हरिः' इत्यादौ विसर्गे कृते ततो गच्छतीत्यादिसम्बन्धे `हरिः गच्छति' इत्याद्येव साधु~।
तद्विषये पदसंस्कारपक्षानाश्रयणं वेति दिक्~॥\par
`अन्तरङ्गादप्यपवादो बलीयान्'~।
तत्र अपवादपदार्थमाह
\section*{\begin{center}येन नाप्राप्ते यो विधिरारभ्यते स तस्य बाधको भवति~॥५७॥\end{center}}
\addcontentsline{toc}{section}{येन नाप्राप्ते यो विधिः}
प्राप्त इति भावे क्तः, येन नाप्राप्त इत्यस्य यत्कर्तृकावश्यप्राप्तावित्यर्थः~।
नञ्द्वयस्य प्रकृतार्थदार्ढ्यबोधकत्वात्~।
एवञ्च विशेषशास्त्रोद्देश्यविशेषधर्मावच्छिन्नवृत्तिसामान्यधर्मावच्छिन्नोद्देश्यकशास्त्रस्य विशेषशास्त्रेण बाधः~।
तदप्राप्तियोग्येऽचारितार्थ्यं ह्येतस्य बाधकत्वे बीजम्~।
अत एव ``आयादयः'' इति सूत्रे {\bfseries `गोपायिष्यति' इत्यादावायादीन्बाधित्वा परत्वात्स्यादयः प्राप्नुवन्ती}त्याशङ्क्य {\bfseries ``अनवकाशाः आयादयः'' `गोपायति' इत्यादावपि शप्स्यादिः प्रप्नोति~।
न च सति शप्यसति वा न विशेषः~।
अन्यादिदानीमिदमुच्यते नास्ति विशेष इति~।
यदुक्तम् ``आयादीनां स्यादिभिरव्याप्तोऽवकाश इति स नास्त्यवकाश'' }इति भाष्ये उक्तम्~।
एवमत्र तत्प्रवृत्युत्तरं चारितार्थ्येऽपि तदव्याप्तोऽवकाशो नास्तीति सममेव~।
किञ्च तत्प्रवृत्त्युत्तरमपि चारितार्थ्ये तद्बाधबोधनम्~।
अन्यथानवकाशत्वेनैव बाधे सिद्धे एतत्कथनवैयर्थ्यापत्तेः,"तक्रकौण्डिन्य'' न्यायप्रदर्शनस्यापि वैयर्थ्यापत्तेश्च~।
तथा प्रथमद्विर्वचनस्य तदुत्तरं सावकाशेनापि द्वितीयद्विर्वचनेन बाधः~।
यथा वाऽऽदेरपि प्रवृत्त्या चरितार्थेन ``आदेः परस्य'' इत्यनेन ``अलोऽन्त्यस्य'' इत्यस्य बाधः~।
तदुक्तं ``मिदचोऽन्त्यात्'' इति सूत्रे भाष्ये {\bfseries सत्यपि सम्भवे बाधनं भवति} इति~।
अन्यथा ``ब्राह्मणेभ्यो दधि दीयतां~, तक्रं कौण्डिन्याय'' इत्यत्र तक्रदानेन दधिदानस्य बाधो न स्यात्~।
तद्दानोत्तरं तत्पूर्वं वा तद्दानस्य चारितार्थ्यसम्भवात्~।
अत एव विषयभेदेऽप्यपवादत्वम्~।
अत एवाचिरादेशेन नुटोऽप्यपवादत्वाद्बाधमाशङ्क्य ``न तिसृ-'' इति ज्ञापकेन समाहितं ``तृज्वत्'' सूत्रे भाष्ये~।
तेन विषयभेदेऽपवादत्वाभाव एव बोध्यत इति कश्चित्~, तन्न~।
विन्मतोर्लुका टिलोपमात्रस्य बाधानापत्तेः~।
कतिपयसंस्करणे यद्यपि ``अन्तरङ्गादप्यपवादो बलीयान्'' इत्यपि परिभाषारूपेणाङ्गीकृतम्~, किन्तु `परनित्यान्तरङ्गापवादानामुत्तरोत्तरं बलीयः' इत्यनेन गतार्थत्वात्~, परिभाषापाठे पठितत्वात्~, चन्द्रिकाकारेण धृतत्वात्~, `येन नाप्राप्ते यो विधिरारभ्यते स तस्य बाधको भवति' इत्येवात्र परिभाषारूपेण पठ्यते~।
यत्तु ``दयतेर्दिगि'' इति सूत्रे द्वित्वोत्तरं दिग्यादेशस्य चारितार्थ्यं कैयटेनोक्तं~, तत्प्रौढ्या~।
ध्वनितञ्च तेनापि तस्य तथात्वं तदुत्तरग्रन्थेन~।
"असम्भव एव बाधकत्वम्~, विरोधस्य तद्बीजत्वात्'' इति वार्तिकमतं तु भाष्यकृता दूषितत्वान्न लक्ष्यसिद्ध्युपयोगि~।
तक्रकौण्डिन्यन्यायोऽपि तदप्राप्तियोग्येऽचरितार्थ्यविषयो विधेयविषय एव चेति ``तद्धितेष्वचामादेः'' ``धातोरेकाचः'' इत्यादिसूत्रेषु भाष्ये स्पष्टम्~।
क्वचित्तु सर्वथानवकाशत्वादेव बाधकत्वम्~।
यथा ङेरामो याडादिबाधकत्वम्~।
न हि याडादिषु कृतेषु ङेरां प्राप्नोति, निर्दिश्यमानस्य व्यवधानात्~।
तत्र स्वस्य पूर्वं प्रवृत्तिरित्येव तेषां बाधः~।
तत्र बाधके प्रवृत्ते यद्युत्सर्गप्राप्तिर्भवति तदा भवत्येव~।
यथा तत्रैव याडागमः~।
अप्राप्तौ तु न, यथा `पचेयुः' इत्यादौ दीर्घबाधके निरवकाश इयादेशे दीर्घाभावः॥\par
तदेतत्पठ्यते-
\section*{\begin{center}क्वचिदपवादविषयेऽप्युत्सर्गोऽभिनिविशत इति~॥५८॥\end{center}}
\addcontentsline{toc}{section}{क्वचिदपवादविषयेऽपि}
अपवादशब्दोऽत्र बाधकपरः~।
तदुक्तं ``गुणो यङ्लुकोः'' इत्यत्र भाष्ये {\bfseries अभ्यासविकारेष्वपवादा उत्सर्गान्न बाधन्ते~।
`अजीगणत्'~।
अत्र गणेरीत्वमपवादत्वाद्धलादिःशेषं बाधते~।
न गणेरीत्वमपवादत्वाद्धलादिःशेषं बाधते~।
किं तर्हि~? अनवकाशत्वा}दिति ग्रन्थेन गणरूपाभ्यासान्त्यणस्येत्वमित्यर्थे हलादिःशेषेण तन्निवृत्तौ तदनवकाशम्~, ईत्वे तु कृते तस्य प्राप्तिः, अन्त्यहलोऽभावात्~।
अभ्यासविकारेषु बाध्यबाधकभावाभावेन च साधितम्~।
तस्मिंश्च सति लोपे कृते सामर्थ्याच्छिष्टस्यान्त्यस्येत्त्वमिति न दोषः~।
न च येन नाप्राप्तिन्यायेनापवादत्वमप्यस्य सुवचं~, तस्य चरितार्थविषयताया उक्तत्वात्~।
 ``इको झल्'' इत्यत्र भाष्येऽपि ध्वनितमेतत्~।
तत्र हि `अज्झन' इति दीर्घेण गुणोत्तरं फलाभावेनानवकाशत्वाद्गुणे बाधिते दीर्घोत्तरं गुणः स्यात्~, दीर्घविधानं तु मिनोतेर्दीर्घे कृते ``सनि मीमा'' इत्यत्र मीग्रहणेन ग्रहणेऽर्थवत्तत्र पश्चात्प्राप्तगुणबाधनार्थम्~, ``इको झल्'' इति कित्त्वमित्युक्तम्~।
अन्यथापवादत्वेन बाधे तद्विषये उत्सर्गाप्रवृत्तेर्भाष्यस्य सूत्रस्य चासङ्गतिरिति स्पष्टमेव~।
यत्तु काञ्चनीत्यादावपवादमयड्विषयेऽप्यण्भवति, `क्वचिदपवादविषयेऽपि' इति न्यायादिति तन्न~।
 ``अणः कर्मणि च'' इति सूत्रस्थभाष्यविरोधात्~।
तत्र हि {\bfseries अणः पुनर्वचनमपवादविषयेऽनिवृत्त्यर्थम्~।
गोदायो व्रजति} इत्याद्युक्तम्~।
`काञ्चनी' इत्यादौ काञ्चनेन निर्मितेत्यर्थे शैषिकोऽण्बोध्यः~।
अत्रेदं बोध्यं `येन नाप्राप्ते' इत्यत्र येनेत्यस्य यदि स्वेतरेणेत्यर्थः, तदा स्वविषये स्वेतरद्यद्यत्प्राप्नोति तद्बाध्यम् , विध्यन्तराप्राप्तविषयाभावात्~।
इयमेव बाध्यसामान्यचिन्तेति व्यवह्रियते~।
अनवकाशत्वेन बाधेऽप्येषा वक्तुं शक्या यद्युदाहरणमस्ति, विनिगमनाविरहात्~।
यदि तु येनेत्यस्य लक्षणेनेत्यर्थः, कार्येणेत्यर्थो वा तदा बाध्यविशेषचिन्ता~।
अनवकाशत्वेन बाधेऽप्येतद्बाधेन सार्थक्यम्~, उत तद्बाधेनेत्येवं विशेषचिन्ता सम्भवति, यद्युदाहरणमस्ति~॥\par
तत्र कार्येणेत्यर्थे पररूपत्वावच्छिन्ने कार्ये आरभ्यमाणाया वृद्धेस्तद्बाधकत्वे न निर्णीते किं शास्त्रविहितस्येत्येवं तद्विशेषचिन्तायामाह-

\section*{\begin{center}पुरस्तादपवादाः अनन्तरान्विधीन्बाधन्ते, नोत्तरान्~॥५९॥\end{center}}
\addcontentsline{toc}{section}{पुरस्तादपवादाः}
अवश्यं स्वपरस्मिन्बाधनीये प्रथमोपस्थितानन्तरबाधेन चारितार्थ्ये पश्चादुपस्थितस्य ततः परस्य बाधे मानाभावः~।
आकाङ्क्षाया निवृत्तेर्विप्रतिषेधशास्त्रबाधे मानाभावाच्चेत्येतस्य बीजम्~॥\par
 ``नासिकोदरौष्ठजङ्घादन्त-'' इत्यस्यौष्ठाद्यंशे ङीष्निषेधत्वावच्छिन्नबाधकत्वे निर्णीते किंविहितस्येत्याकाङ्क्षायामाह-
\section*{\begin{center}मध्येऽपवादाः पूर्वान्विधीन्बाधन्ते, नोत्तरान्~॥६०॥\end{center}}
\addcontentsline{toc}{section}{मध्येऽपवादाः}
तेनौष्ठादिषु पञ्चसु ``असंयोगोपधात्'' इति प्रतिषेध एव बाध्यते, न तु सहनञ्विद्यमानलक्षण इति ``नासिकोदर'' इत्यत्र भाष्ये स्पष्टम्~।
पूर्वोपस्थितबाधेन नैराकाङ्क्ष्यमस्या बीजम्~॥\par
ननु ``वा छन्दसि'' इत्यनेन ``सेर्ह्यपिच्च'' इत्यनन्तरस्यापित्वस्येव हेरपि विकल्पः स्यात्~।
तथा ``नेटि'' ति निषेधोऽनन्तरहलन्तलक्षणाया इव सिचिवृद्धिमृजिवृद्ध्योरपि स्यात्~, अत उक्तन्यायमूलकमेवाह-
\section*{\begin{center}अनन्तरस्य विधिर्वा भवति प्रतिषेधो वेति~॥६१॥\end{center}}
\addcontentsline{toc}{section}{अनन्तरस्य विधिर्वा}
अत एव ``संख्याव्ययादेः'' इति ङीब्ग्रहणं चरितार्थम्~।
तद्ध्यनन्तरस्य ङीषो विध्यभावाय~।
 ``न क्तिचि''इति सूत्रे दीर्घग्रहणञ्च चरितार्थम्~।
तद्ध्यनन्तरस्यानुदात्तोपदेशेत्यस्यैव निषेधाभावाय~।
मध्येऽपवादन्ययापेक्षयानन्तरस्येति न्यायः प्रबल इति ``अष्टाभ्यः'' इति सूत्रे कैयटः~।
प्रत्यासत्तिमूलकोऽयम्~।
लक्ष्यानुरोधाच्च व्यवस्थेत्यपि पक्षान्तरम्~।
तत्र तत्र क्वचित्स्वरितत्वप्रतिज्ञासामर्थ्येन बाध्यतेऽयं न्यायः~।
यथा ``टिड्ढाण्-'' इति सूत्रेण डापा व्यवहितस्यापि ङीपो विधिः~।
 ``न षट्-'' इत्यादिना द्वयोरपि टाब्ङीपोः प्रतिषेधः~।
इयञ्च ``शि सर्वनामस्थानम्'' इत्यादौ भाष्ये स्पष्टेत्यन्यत्र विस्तरः~॥\par
ननु दधतीत्यादावन्तरङ्गत्वादन्तादेशेऽल्विधौ, स्थानिवत्वाभावाददादेशो न स्यादिति तद्वैयर्थ्यापत्तिः, अत आह-
\section*{\begin{center}पूर्वं ह्यपवादा अभिनिविशन्ते पश्चादुत्सर्गाः~॥६२॥\end{center}}
\addcontentsline{toc}{section}{पूर्वं ह्यपवादाः}
लक्षणैकचक्षुष्को ह्यपवादविषयं पर्यालोच्य तद्विषयत्वाभावनिश्चय उत्सर्गेण तत्तल्लक्ष्यं संस्करोति~।
अन्यथा विकल्पापत्तिरित्यर्थः~।
अभिनिविशन्त इत्यस्य बुद्ध्यारूढा भवन्तीत्यर्थः~।
 `अपवादो यद्यन्यत्र चरितार्थः' इति न्यायस्य तु नात्र प्राप्तिः, अन्तादेशाप्राप्तिविषये चारितार्थ्याभावात्~॥\par
लक्ष्यैकचक्षुष्कस्तु तच्छास्त्रपर्यालोचनं विनाप्यपवादविषयं परित्यज्योत्सर्गेण लक्ष्यं संस्करोति, तस्यापि शास्त्रप्रक्रियास्मरणपूर्वकप्रयोग एव धर्मोत्पत्तेः, तदाह-
\section*{\begin{center}प्रकल्प्य चापवादविषयं तत उत्सर्गोऽभिनिविशते~॥६३॥\end{center}}
\addcontentsline{toc}{section}{प्रकल्प्य चापवादविषयं}
तत इत्यस्यापवादशास्त्रपर्यालोचनात्प्रागपीत्यर्थः~।
प्रकल्प्येत्यस्य परित्यज्येत्यर्थः~।
अत एव प्रातिपदिकार्थसूत्रे भाष्ये इदं द्वयमप्युक्त्वा `न कदाचित्तावदुत्सर्गो भवति, अपवादं तावत्प्रतीक्षते’ इत्यर्थकमुक्तम्~॥\par
एतन्मूलकमेव नवीनाः पठन्ति-
\section*{\begin{center}उपसञ्जनिष्यमाणनिमित्तोऽप्यपवाद उपसञ्जातनिमित्तमप्युत्सर्गं बाधत इति~॥६४॥\end{center}}
\addcontentsline{toc}{section}{उपसञ्जनिष्यमाणनिमित्तोऽपि}
यत्त्वभ्यस्तसंज्ञासूत्रे कैयटेन प्रकल्प्य चेति प्रतीकमुपादाय यथा `न सम्प्रसारण इति परस्य यणः पूर्वं सम्प्रसारणम् , पूर्वस्य तु तन्निमित्तकः प्रतिषेधः' इत्युक्तम्, तत्तु तत उत्सर्ग इत्याद्यक्षरार्थाननुगुणम्~।
यत्त्वपवादवाक्यार्थं विना नोत्सर्गवाक्यार्थ इति, तन्न, `अभिनिविशतेऽपवादविषयम्' इत्यादिपदस्वारस्यभङ्गापत्तेः~।
पदजन्यपदार्थोपस्थितौ वाक्यार्थबोधाभावे कारणभावाच्च~।
यत्र त्वपवादो निषिद्धः, तत्रापवादविषयेऽप्युत्सर्गः प्रवर्तत एव~।
यथा `वृक्षौ' इत्यत्र ``नादिचि'' इति पूर्वसवर्णदीर्घनिषेधादप्रवर्तमानस्य वृद्धिबाधकत्वाभावाद्वृद्धिः प्रवर्तते~।
अत एव ``तौ सत्'' इत्यादि सङ्गच्छते~।
अत एव निर्देशाद्भ्रष्टावसरन्यायस्यात्र शास्त्रेऽनाश्रयणम्~।
ध्वनितं चेदम् ``इको गुण'' इति सूत्रे भाष्य इति भाष्यप्रदीपोद्योते निरूपितम्~।
अत्र `देवदत्तस्य हन्तरि हते देवदत्तस्योन्मज्जनं न' इति न्यायस्य विषय एव नास्ति~।
हते देवदत्त उन्मज्जनं न देवदत्तहननोद्यतस्य, हनने तु भवत्येवोन्मज्जनम्~।
प्रकृतेऽपि न पूर्वसवर्णदीर्घेण वृद्धेर्हननम् , किन्तु हननोद्यमसजातीयं प्रसक्तिमात्रम् , प्रसक्तस्यैव निषेधात्~।
प्रतिपदोक्तत्वमपि निरवकाशत्वे सत्येव बाधप्रयोजकम्~।
स्पष्टञ्चेदं ``शेषाद्विभाषा'' इति सूत्रे भाष्ये~।
तत्र हि {\bfseries शेषग्रहणमनर्थकम् , ये प्रतिपदं विधीयन्ते ते बाधका भविष्यन्ति} इत्याशङ्क्य, {\bfseries अनवकाशा हि विधयो बाधका भवन्ति समासान्ताश्च कबभावे सावकाशाः} इत्युक्तम्~।
क्वचिदनवकाशत्वाभावेऽपि परनित्यादिसमवधाने शीघ्रोपस्थितिकत्वेन पूर्वप्रवृत्तिप्रयोजकं बलवत्वं प्रतिपदविधित्वेनापि, परनित्यान्तरङ्गप्रतिपदविधयो विरोधिसन्निपाते तेषां मिथः परबलीयस्त्वमिति ``प्रत्ययोत्तरपदयोश्च'' इति सूत्रे कैयटेन पाठात्~।
अत एव `रमे' इत्यादौ प्रतिपदोक्तत्वात्पूर्वमेत्व आकारप्रश्लेषाद्धल्ङ्यादिलोपो न प्राप्नोतीत्याशङ्क्य ``एङ्ह्रस्वात्'' इति लोपेन समाहितम्~॥\par
ननु `अयजे इन्द्रम्' इत्यादावन्तरङ्गस्यापि गुणस्यापवादेन सवर्णदीर्घेण बाधः स्यात्~, अत आह-
\section*{\begin{center}अपवादो यद्यन्यत्र चरितार्थस्तर्ह्यन्तरङ्गेण बाध्यते~॥६५॥\end{center}}
\addcontentsline{toc}{section}{अपवादो यद्यन्यत्र}
निरवकाशत्वरूपस्य बाधकत्वबीजस्याभावात्~।
एवञ्च प्रकृतेऽन्तरङ्गेण गुणेन सवर्णदीर्घः समानाश्रये चरितार्थो यण्गुणयोरपवादोऽपि बाध्यते~।
पूर्वोपस्थितिनिमित्तत्वरूपान्तरङ्गत्वविषय इदम्~।
यत्त्वागमादेशयोर्न बाध्यबाधकभावः, भिन्नफलत्वात्, अत एव `ब्राह्मणेभ्यो दधि दीयताम्~, कम्बलः कौण्डिन्याय' इत्यादौ कम्बलदानेन न दधिदानबाध इति ``छ्वोः'' इतिसूत्रे कैयटः~।
तन्न, {\bfseries अपवादो नुग्दीर्घत्वस्य} इति ``दीर्घोऽकितः'' इति सूत्रभाष्यविरोधात्~॥\par
ननु `अजीगणत्' इत्यादौ गणेरीत्वं निरवकाशत्वाद्धलादिःशेषं बाधेत~। तत्राह-
\section*{\begin{center}अभ्यासविकारेषु बाध्यबाधकभावो नास्ति~॥६६॥\end{center}}
\addcontentsline{toc}{section}{अभ्यासविकारेषु}
 ``दीर्घोऽकितः'' इत्यकिद्ग्रहणमस्या ज्ञापकम्~।
अन्यथा `यंयम्यते' इत्यत्र नुकि कृतेऽनजन्तत्वाद्दीर्घाप्राप्त्या तद्वैयर्थ्यं स्पष्टमेव~।
इयं परान्तरङ्गादिबाधकानामप्यबाधकत्वबोधिका~।
तेन `अचीकरत्',`मीमांसते' इत्यादिसिद्धम्~।
आद्ये सन्वद्भावस्य परत्वाद्दीर्घेण बाधः प्राप्नोति~।
अन्त्ये ``मान्बध-'' इति दीर्घेणान्तरङ्गत्वादित्यस्य बाधः प्राप्तः~।
यत्तु यत्रैकैकप्रवृत्त्युत्तरमपि सर्वेषां प्रवृत्तिस्तत्रैवेदमित्यतः ``एक'' इतिसूत्रे कैयटः, तन्न~।
नुकि कृते इत्वाप्राप्त्या ``गुणो यङ्लुकोः'' इति सूत्रस्थभाष्योक्ततदुदाहरणासङ्गतेश्चेत्यन्यत्र विस्तरः~॥\par
ननु तच्छीलादितृन्विषये ण्वुलपि स्यात्~।
न च तृन्नपवादः, असरूपापवादस्य विकल्पेन बाधकत्वात्~।
अत आह-
\section*{\begin{center}ताच्छीलिकेषु वासरूपविधिर्नास्ति~॥६७॥\end{center}}
\addcontentsline{toc}{section}{ताच्छीलिकेषु}
ण्वुलि सिद्धे ``निन्दहिंस-'' आदिसूत्रेणैकाज्भ्यो वुञ्विधानमत्र ज्ञापकम्~।
तत्र ण्वुल्वुञोः स्वरे विशेषाभावात्~।
ताच्छीलिकेष्विति विषयसप्तमी~।
तेन ताच्छीलिकैरताच्छीलिकैश्च वासरूपविधिर्नेति बोध्यम्~।
नन्वेवं `कम्रा, कमना' इत्याद्यसिद्धिः, ``नमिकम्पि-'' इति रेण ``अनुदात्तेतश्च हलादेः'' इति युचो बाधादिति चेन्न~।
 ``सूददीपदीक्षश्च'' इत्यनेन दीपेर्युज्निषेधेनोक्तार्थस्यानित्यत्वात्~॥\par
नन्वेवं `हसितं छात्रस्य हसनम्' इत्यादौ घञ्~,`इच्छति भोक्तुम्' इत्यत्र लिङ्लोटौ, `ईषत्पानः सोमो भवता' इत्यत्र खल्प्राप्नोतीत्यत आह-
\section*{\begin{center}क्तल्युट्‍तुमुन्खलर्थेषु वासरूपविधिर्नास्ति॥६८॥\end{center}}
\addcontentsline{toc}{section}{क्तल्युट्तुमुन्खलर्थेषु}
इदञ्च वासरूपविधेरनित्यत्वात्सिद्धम्~।
तदनित्यत्वे ज्ञपकं च ``अर्हे कृत्यतृचश्च'' इति~।
तत्र हि चकारसमुच्चितलिङा कृत्यतृचोर्बाधो मा भूदिति कृत्यतृज्ग्रहणं क्रियत इत्यन्यत्र विस्तरः~।
 ``वासरूप'' सूत्रे भाष्ये स्पष्टा~॥\par
ननु `श्वः पक्ता' इत्यत्र वासरूपविधिना लृडपि प्राप्नोति, कृते आदेशे वैरूप्यादत आह-
\section*{\begin{center}लादेशेषु वासरूपविधिर्नास्ति~॥६९॥\end{center}}
\addcontentsline{toc}{section}{लादेशेषु वासरूप}
आदेशकृतवैरूप्यवत्सु लकारेषु स नास्तीत्यर्थः~।
अत्र च ``हशश्वतोर्लङ् च'' इति लङ्विधानं ज्ञापकम्~।
अन्यथा ``परोक्षे लिट्'' इति लिटा लङः समावेशोऽसारूप्यात्सिद्ध इति किं लङ्विधानेन~?
शत्रादिभिस्तिङां समावेशार्थं शतृविधायके विभाषाग्रहणानुवृत्तिः ``लिटः कानज्वा'' इति वाग्रहणं च कृतम्~।
तज्ज्ञापयति ``वासरूप'' सूत्रेऽपवाद आदेशत्वानाक्रान्तः प्रत्यय एव गृह्यत इति कैयटादौ ध्वनितम्~।
तत्फलं तु सदादिभ्यो भूतसामान्ये लिटः क्वसुरेव, न तु पक्षे तिङिति बोध्यम्~॥\par
ननु ``ङमो ह्रस्वात्-'' इत्यादौ ङमः परस्याचोऽचि परतो ङम इति वेति सन्देहः स्यात्~, अत आह-
\section*{\begin{center}उभयनिर्देशे पञ्चमीनिर्देशो बलीयान्~॥७०॥\end{center}}
\addcontentsline{toc}{section}{उभयनिर्देशे पञ्चमी}
अचीति सप्तमीनिर्देशस्य ``मय उञ'' इत्युत्तरत्र चारितार्थ्यात्पञ्चमीनिर्देशोऽनवकाश इति ``तस्मादित्युत्तरस्य'' इत्यस्यैव प्रवृत्तिः~।
यत्र तु ``डः सि धुट्'' इत्यादावुभयोरप्यचारितार्थ्यम् , तत्र ``तस्मिन्'' इति सूत्रापेक्षया ``तस्मादित्युत्तरस्य'' इत्यस्य परत्वात्तेनैव व्यवस्था~।
एवमुभयोश्चारितार्थ्येऽपि~।
यथा ``आमि सर्वनाम्नः सुट्'' इत्यादौ~।
तत्राऽमीति सप्तमी ``त्रेस्त्रयः'' इत्यत्र चरितार्था~।
आदिति पञ्चमी ``आज्जसेरसुक्'' इत्यत्र चरितार्थेति स्पष्टं ``तस्मिन्निति'' इति सूत्रे भाष्ये कैयटे च~॥
\begin{center}{\bfseries॥इति श्री नागेशभट्टविरचिते परिभाषेन्दुशेखरे बाधबीजनामकं द्वितीयं प्रकरणम्॥} \end{center}
\begin{figure}[b]
\centering
\includegraphics[width=4cm]{page-divider}
\end{figure}
\part*{॥अथ शास्त्रशेषनामकं तृतीयं प्रकरणम्॥}
\addcontentsline{toc}{part}{शास्त्रशेषप्रकरणम् }
\fancyhead[RE]{तृतीयं प्रकरणम्}
ननु ``अतः कृकमि'' इति सत्वम् `अयस्कुम्भी' इत्यत्र न स्यात्कुम्भशब्दस्यैवोपादानात्~, अत आह-
\section*{\begin{center}प्रातिपदिकग्रहणे लिङ्गविशिष्टस्यापि ग्रहणम्॥७१॥\end{center}}
\addcontentsline{toc}{section}{प्रातिपदिकग्रहणे}
सामान्यरूपेण विशेषरूपेण वा प्रातिपदिकबोधकशब्दग्रहणे सति लिङ्गबोधकप्रत्ययविशिष्टस्यापि तेन ग्रहणं बोध्यम्~।
अपिना केवलस्यापीत्यर्थः~।
अस्याश्च ज्ञापकं समानाधिकरणाधिकारस्थे ``कुमारः श्रमणादिभिः'' इति सूत्रे स्त्रीलिङ्गश्रमणादिशब्दपाठः~।
स्त्रीप्रत्ययविशिष्टश्रमणादिभिश्च कुमारीशब्दस्यैव सामानाधिकरण्यं~, न तु कुमारशब्दस्येति तदेतज्ज्ञापकम्~।
इयञ्च ``द्विषत्परयोः'' इत्याद्युपपदविधौ, समासान्तविधौ, महदात्वे, ञ्नित्स्वरे, राजस्वरे ब्राह्मणकुमारयोः, ``बहोर्नञ्वदुत्तरपदभूम्नि'' इत्यादौ, समाससङ्घातग्रहणेषु च न प्रवर्तत इति ङ्याप्सूत्रे भाष्ये स्पष्टम्~।
विभक्तिनिमित्तककार्ये च नेत्यपि तत्रैव~।
तत्र समासान्तविधाववयवग्रहण एव न, समाससङ्घातग्रहणे तु प्रवर्तत एव, स्वरविधावेव समाससङ्घातग्रहणे तत्र दोषोक्तेः,"बहुव्रीहेरूधसः'' इति सूत्रस्थभाष्याच्च, {\bfseries एतावत्स्वेवानित्यत्वादप्रवृत्तिः, दोषाः खल्वपि साकल्येन परिगणिताः} इति भाष्योक्तेः~।
नन्वेवं ``बहुव्रीहेरूधसो ङीष्'' इति सूत्रस्थभाष्यासङ्गतिः~।
तत्र हि `कुण्डोध्नी' इत्यत्र ``नद्यृतश्च'' इति कबापादितो नद्यन्तबहुव्रीहेरित्यर्थात्~। नद्यन्तस्य बहुव्रीहित्वाभावात्तदसङ्गतिः~।
नद्यन्तानां यः समास इत्यर्थेन च परिहृतम्~।
नद्यन्तप्रकृतिकसुबन्तोत्तरपदकः समास इति चेत्~, न~। अनया परिभाषया स्त्रीप्रत्ययसमभिव्याहारे तद्रहिते दृष्टानां प्रातिपदिकत्वतद्व्याप्यधर्माणां विशिष्टेऽपि पर्याप्तत्वमतिदिश्यत इत्याशयात्~॥\par
नन्वेवं `यूनः पश्य' इत्यत्रेव `युवतीः पश्य' इत्यत्रापि ``श्वयुवोनाम्'' इति सम्प्रसारणं स्यात्~, अत आह-
\section*{\begin{center}विभक्तौ लिङ्गविशिष्टाग्रहणम्॥७२॥\end{center}}
\addcontentsline{toc}{section}{विभक्तौ लिङ्गविशिष्ट}
स्पष्टा चेयं ``युवोरनाकौ'' इत्यत्र भाष्ये~।
घटघटीग्रहणेन लिङ्गविशिष्टपरिभाषया अनित्यत्वात्तन्मूलैषेत्यन्ये~॥\par
ननु ``तस्यापत्यम्'' इत्येकवचननपुंसकाभ्यां निर्देशात् `गार्ग्यो, गार्ग्यौ' इत्याद्ययुक्तम्~, अत आह-
\section*{\begin{center}सूत्रे लिङ्गवचनमतन्त्रम्॥७३॥\end{center}}
\addcontentsline{toc}{section}{सूत्रे लिङ्गवचनम्}
 ``अर्द्धं नपुंसकम्'' इति नपुंसकग्रहणमस्यां ज्ञापकम्~।
नित्यनपुंसकत्वार्थं तु न तदित्यन्यत्र निरूपितम्~।
धान्यपलालन्यायेन नान्तरीयकतया तयोरुपादानमिति ``तस्यापत्यम्'' इत्यत्र भाष्ये स्पष्टम्~।
अत एव आकडारसूत्रे एकेति चरितार्थमित्यन्यत्र विस्तरः~॥\par
ननु ``भृशादिभ्यो भुव्यच्वेः'' इत्यादौ विधीयमानः क्यङ् `क्व दिवा भृशा भवन्ति' इत्यत्रापि स्यात्~, अत आह-
\section*{\begin{center}नञिव युक्तमन्यसदृशाधिकरणे तथा ह्यर्थगतिः॥७४॥\end{center}}
\addcontentsline{toc}{section}{नञिव युक्तमन्य}
नञ्युक्तमिवयुक्तं च यत्किञ्चिद्दृश्यते तत्र तस्माद्भिन्ने तत्सदृशेऽधिकरणे द्रव्ये कार्यं विज्ञायते~।
हि यतः तथार्थगतिरस्ति~।
न हि `अब्राह्मणमानय' इत्युक्ते लोष्टमानीय कृती भवति~।
अतश्च्‌व्यन्तभिन्ने च्व्यन्तसदृशेऽभूततद्भावविषये क्यङिति नोक्तदोषः~।
 ``ओषधेश्च विभक्तावप्रथमायाम्'' इत्यादौ विभक्तिग्रहणमेतन्न्यायसिद्धार्थानुवाद एव~।
एतेन विभक्तावित्याद्यस्यानित्यत्वे ज्ञापकमिति वदन्तः परास्ताः, अनित्यत्वे भाष्यसम्मतफलाभावात्~।
अत एव ``अकर्तरि च'' इति सूत्रे कारकग्रहणं भाष्ये प्रत्याख्यातमिति बोध्यम्~।
स्पष्टा चेयं ``भृशादिभ्यः'' इतिसूत्रे भाष्ये~।
अत्रान्यसदृशेत्युक्त्वा सादृश्यस्य भेदाघटितत्वं सूचयति~।
निरूपितं चैतन्मञ्जूषायाम्~॥\par
ननु `व्याघ्री', `कच्छपी' इत्यादौ सुबन्तेन समासात्ततोऽप्यन्तरङ्गत्वाट्टाप्यदन्तत्वाभावाज्जातिलक्षणो ङीष्न स्यात्~, अत आह-
\section*{\begin{center}गतिकारकोपपदानां कृद्भिः सह समासवचनं प्राक्सुबुत्पत्तेः॥७५॥\end{center}}
\addcontentsline{toc}{section}{गतिकारकोपपदानाम्}
 ``उपपदम्'' इति सूत्रेऽतिङ्ग्रहणेन ``कुगति'' इत्यत्र तदपकर्षणेनातिङन्तश्च समास इत्यर्थात्तयोः सूत्रयोः सुप्सुपेत्यस्य निवृत्त्यैकदेशानुमत्या कारकांशे च सिद्धेयम्~।
तेन `अश्वक्रीती' इति सिद्धा~।
अन्यथा पूर्वं टाप्यदन्तत्वाभावात् ``क्रीतात्करणपूर्वात्'' इति ङीष्न स्यात्~।
अस्या अनित्यत्वात्क्वचित्सुबुत्पत्त्यनन्तरमपि समासः~।
यथा सा हि तस्य धनक्रीतेति~।
अन्ये त्वनित्यत्वे न मानम्~, तत्राजादित्वाट्टाबित्याहुः~।
अत एव `कुम्भकारः' इत्यादौ षष्ठीसमासोऽपि सुबुत्पत्तेः पूर्वमेव~।
षष्ठीसमासाभावे चोपपदसमासकृतैकार्थीभाव इति न तत्र वाक्यमिति भाष्ये स्पष्टम्~।
तत्र हि `षष्ठीसमासादुपपदसमासो विप्रतिषेधेन' इति वार्त्तिकम्~।
अथवा `विभाषा षष्ठी समासो यदा न षष्ठीसमासस्तदोपपदसमासः' इति तत्प्रत्याख्यानं च~।
यद्यप्युपपदसमासस्यान्तरङ्गत्वाभिप्रायकं `न वा षष्ठीसमासाभावादुपपदसमासः' इति वार्तिककृतोक्तम् ,
तथापि तदुभयप्रत्याख्यानपरम्~, `अथवा' इत्यादिभाष्यं परिभाषायां सामान्यतः कारकोपादानेन कारकविभक्त्यन्तेन कृद्भिः समासमात्रस्य सुबुत्पत्तेः पूर्वमेव लाभात्~।
एतेनैषा कारकतद्विशेषयोरुपादान एवेति परास्तम्~।
अस्या विध्येकवाक्यत्वाभावेन विप्रतिषेधादिशास्त्रवत्कार्यव्यवस्थापकत्वेनोपादान एवेत्यर्थालाभाच्च~॥\par
ननु ``उगिदचाम्'' इत्यत्र धातोश्चेदुगित्कार्यं तर्ह्यञ्चतेरेवेति नियमेनाधातोरेव नुमि सिद्धेऽधातुग्रहणं व्यर्थम्~, अत आह-
\section*{\begin{center}साम्प्रतिकाभावे भूतपूर्वगतिः॥७६॥\end{center}}
\addcontentsline{toc}{section}{साम्प्रतिकाभावे}
तत्तद्वचनसामर्थ्यान्न्यायसिद्धेयम्~।
तत्सामर्थ्यादधातुभूतपूर्वस्यापीत्यर्थेन गोमत्यतेः क्विपि `गोमान्' इत्यादौ नुंसिद्धिः~। ``नामि'' इत्यादिसूत्रेषु भाष्ये स्पष्टा॥
\section*{\begin{center}बहुव्रीहौ तद्गुणसंविज्ञानमपि॥७७॥\end{center}}
\addcontentsline{toc}{section}{बहुव्रीहौ तद्गुण}
अपिना अतद्गुणसंविज्ञानम्~।
तेषां गुणानामवयवपदार्थानां संविज्ञानं विशेषान्वयित्वमिति तदर्थः~।
यत्र समवायसम्बन्धेन सम्बन्ध्यन्यपदार्थः, तत्र प्रायस्तद्गुणसंविज्ञानम्~।
अन्यत्र प्रायोऽन्यत्~।
`लम्बकर्णचित्रगू' उदाहरणे~।
`सर्वादीनि', `जक्षित्यादयः' इति चोदाहरणे~।
सर्वनामसंज्ञा सूत्रे भाष्ये स्पष्टा॥\par
ननु ``वदः सुपि क्यप्च'' इति चेनानुकृष्टस्य यतः ``भुवो भावे'' इत्यत्राप्यनुवृत्तिः स्यात्~, अत आह-
\section*{\begin{center}चानुकृष्टं नोत्तरत्र॥७८॥\end{center}}
\addcontentsline{toc}{section}{चानुकृष्टं नोत्तरत्र}
णमुल्यनुवर्तमाने ``अव्ययेऽयथाभिप्रेत-'' इति सूत्रे पुनर्णमुल्ग्रहणमस्या ज्ञापकम्~।
अन्यथा `क्त्वा च' इति वदेत्~।
तद्धि उत्तरत्रोभयोः सम्बन्धार्थम्~।
उदाहरणानि स्फुटानि~।
इदमनित्यम्~।
अत एव ``तृतीया च होः'' इत्यत्र चानुकृष्टाया अपि द्वितीयाया ``अन्तरान्तरेण'' इत्यत्र सम्बन्धः~।
 ``लुटि च कॢपः'' इति सूत्रस्थेनानुवृत्त्यर्थकसकलचकारप्रत्याख्यानेन विरुद्धेयम्~।
व्याख्यानादेवानुवृत्तिनिवृत्त्योर्निर्वाह इति तदाशयः~।
 ``कुलिजाल्लुक्खौ च'' इति सूत्रस्थभाष्यविरुद्धा च~।
तत्र हि ``द्विगोः ष्ठंश्च'' इति सूत्रात्ष्ठनः, तत्र चेनाप्यनुकृष्टस्य `खोऽन्यतरस्याम्' इत्यस्य चानुवृत्तिं स्वीकृत्य `लुक्खौ च' इति भाष्ये प्रत्याख्यातम्~॥\par
नन्वनुदात्तादेरन्तोदात्ताच्च यदुच्यते तद्व्यञ्जनादेर्व्यञ्जनान्ताच्च न प्राप्नोतीत्यत आह-
\section*{\begin{center}स्वरविधौ व्यञ्जनमविद्यमानवत्॥७९॥\end{center}}
\addcontentsline{toc}{section}{स्वरविधौ व्यञ्जनम्}
स्वरोद्देश्यके विधावित्यर्थः~। ``नोत्तरपदेऽनुदात्तादावपृथिवीरुद्रपूषमन्थिषु'' इतिसूत्रे पृथिव्यादिपर्युदासोऽस्या ज्ञापकः~।
अन्यथा पृथिव्यादीनामनुदात्तादित्वाभावादप्राप्तौ तद्वैयर्थ्यं स्पष्टमेव~।
धर्मिग्राहकमानादेव च स्वरोद्देश्यकविषयमिदम्~।
अत एव ``शतुरनुमो नद्यजादी'', ``अचः कर्तृयकि'' इत्यादौ `अच' इत्यादेश्चारितार्थ्यम्~।
अत एव `राजवती' इत्यादौ नलोपस्यासिद्धत्वादन्वतीशब्दत्वात्~, ``अन्तोऽवत्याः'' इति स्वरो न, 'उदश्वित्वान्' इत्यत्र``ह्रस्वनुड्भ्याम्''इति मतुबुदात्तत्वं च नेत्याकरः~।
स्पष्टं चेदं ``समासस्य'' इति सूत्रे भाष्ये~।
``उच्चैरुदात्तः'' इति सूत्रे कैयटस्तु `इयमनावश्यकी, समभिव्याहृताजुपरागेण हलोऽप्युदात्तादिवदवभासात्तदुपपत्तेः' इत्याह~।
तत्र भाष्येऽपि ध्वनितमेतत्॥\par
नन्वेवमपि `राजदृषत्' इत्यादौ ``समासस्य'' इत्यन्तोदात्तत्वं षकाराकारस्य न स्यात्~, अत आह-
\section*{\begin{center}हल्स्वरप्राप्तौ व्यञ्जनमविद्यमानवत्॥८०॥\end{center}}
\addcontentsline{toc}{section}{हल्स्वरप्राप्तौ व्यञ्जनम्}
अस्याश्च ``यतोऽनावः'' इति सूत्रेऽनौ इति प्रतिषेधो ज्ञापकः~।
`नाव्यम्' इत्यत्रादिर्नकारो न स्वरयोग्यः, यश्चाकारस्तद्योग्यो नासावादिरिति स प्रतिषेधोऽनर्थकः~।
न चादिरेव मकार उदात्तगुणविशिष्टान्तरतमाज्रूपोऽस्त्विति वाच्यम् , तथा सति निमित्तभूतद्व्यच्कत्वस्य विनाशादुपजीव्यविरोधेनाद्युदात्तत्वाप्राप्तेः, इत्यन्यत्र विस्तरः~।
स्पष्टा चेयं ``समासस्य'' इति सूत्रे भाष्ये~॥\par
ननु ``पूरण गुण-'' इति निषेधस्तव्यत्यपि स्यात् ``दिव औत्'' इत्यौत्वं दिवेः क्विप्यपि स्यात्~, तथा ``यतोऽनावः'' इति स्वरो ण्यत्यपि स्यात् ``ऋदृशोऽङि गुणः'' इति चङ्यपि स्यात्~, अत आह-

\section*{\begin{center}निरनुबन्धकग्रहणे न सानुबन्धकस्य॥८१॥\end{center}}
\addcontentsline{toc}{section}{निरनुबन्धकग्रहणे}
\section*{\begin{center}तदनुबन्धकग्रहणे नातदनुबन्धकस्य॥८२॥\end{center}}
\addcontentsline{toc}{section}{तदनुबन्धकग्रहणे}
 ``वामदेवाड्ड्यड्ड्यौ" इति सूत्रे ड्यड्यतोर्डित्त्वमनयोर्ज्ञापकम्~।
तद्धि ``ययतोश्चातदर्थे'' इत्यत्र तयोरग्रहणार्थम्~।
नञः परस्य ययदन्तस्योत्तरपदस्यान्त उदात्त इति तदर्थः~।
एवं चावामदेव्येऽव्यय पूर्वपदप्रकृतिस्वर एव भवति~।
तन्मात्रानुबन्धकग्रहणे स चान्यश्चानुबन्धो यस्य तद्ग्रहणं नेत्यन्त्यार्थः~।
एते च प्रत्ययाप्रत्ययसाधारणे, ``दिव औत्'' इत्यादौ सञ्चारितत्वात्~।
वर्णग्रहणे चानयोरप्रवृत्तिरिति स्पष्टम् ``औङ आपः'' इत्यत्र भाष्ये~।
येनानुबन्धेन सानुबन्धकत्वं द्व्यनुबन्धकत्वादि वा तदनुच्चारणे एवैषा, धर्मिग्राहकमानात्~।
तेन ``जश्शसोः'' इत्यत्र नैषेति निरनुबन्धकत्वात्तद्धितशस एवात्र ग्रहणं स्यादिति न शङ्क्यम्~।
एवमन्त्यान्यतरानुबन्धोच्चारणे एव~।
तेन ``वनो र च'' इत्यादौ ङ्वनिप्क्वनिपोरग्रहणसिद्धिः~।
एकानुबन्धकग्रहणे सम्भवतीति त्वर्थो न भाष्यादिसम्मत इत्यन्यत्र विस्तरः~॥\par
ननु `कुटीरः' इत्यादौ स्वार्थिकत्वात्स्वार्थिकानां प्रकृतितो लिङ्गवचनानुवृत्तेर्न्यायप्राप्तत्वात्पुंस्त्वानुपपत्तिः,`अप्कल्पम्' इत्यत्र नपंसकैकवचनयोरनुपपत्तिश्चेत्यत आह-
\section*{\begin{center}क्वचित्स्वार्थिकाः प्रकृतितो लिङ्गवचनान्यतिवर्तन्ते॥८३॥\end{center}}
\addcontentsline{toc}{section}{क्वचित्स्वार्थिकाः प्रकृतितः}
 ``णचः स्त्रियाम्'' इति सूत्रे स्त्रियामित्युक्तिरस्या ज्ञापिका~।
अन्यथा ``कर्मव्यतिहारे णच्स्त्रियाम्'' इति स्त्रियामेव विधानात्किं तेन~? स्पष्टा चेयं बहुज्विधायके भाष्ये~॥\par
ननु `सुपथी नगरी' इति ``युवोरनाकौ'' इति सूत्रभाष्योदाहृते ``इनः स्त्रियाम्'' इति कप्स्यादत आह-
\section*{\begin{center}समासान्तविधिरनित्यः॥८४॥\end{center}}
\addcontentsline{toc}{section}{समासान्तविधिरनित्यः}
 ``प्रतेरंश्वादयस्तत्पुरुषे'' इत्यन्तोदात्तत्वायांश्वादिषु राजन्-शब्दपाठोऽस्या ज्ञापकः~।
अन्यथा टचैवान्तोदात्तत्वे सिद्धे किं तेन~? ``द्वित्रिभ्यां पाद्दन्मूर्द्धसु'' इति स्वरविधायके भाष्ये स्पष्टम्॥\par
ननु `शतानि' इत्यादौ नुमि कृते षट्संज्ञा प्राप्नोति, ततश्च लुक्स्यात्~, तथा `उपादास्त' इत्यत्रात्वे कृते ``स्थाघ्वोरिच्च'' इतीत्त्वं प्राप्नोतीत्यत आह-

\section*{\begin{center}सन्निपातलक्षणो विधिरनिमित्तं तद्विघातस्य॥८५॥\end{center}}
\addcontentsline{toc}{section}{सन्निपातलक्षणो विधिः}
सन्निपातः उभयोः सम्बन्धः, तन्निमित्तो विधिस्तं सन्निपातं यो विहन्ति, तस्यानिमित्तम्~।
उपजीव्यविरोधस्यायुक्तत्वमिति न्यायमूलैषा~।
अत एवात्र सन्निपातशब्देन न पूर्वपरयोः सम्बन्ध एव, किन्तु विशेष्यविशेषणसन्निपातोऽपि गृह्यते~।
अत एव `ग्रामणि कुलम्’ इत्यादौ नपुंसकह्रस्वत्वेऽपि ``पिति कृति'' इति तुग्न~।
प्रातिपदिकाजन्तत्वसन्निपातेन जातस्य ह्रस्वस्य तदविघातकत्वात्~।
तुक्यजन्तत्वविघातः स्पष्ट एव~।
न चार्थाश्रयत्वेन ह्रस्वस्य बहिरङ्गतयासिद्धत्वम्~, अर्थकृतबहिरङ्गत्वानाश्रयणस्योक्तत्वात्~।
किञ्च ``षत्वतुकोरसिद्धः'' इत्येतद्बलात्कृतितुग्ग्रहणाच्च तुग्विधौ बहिरङ्गपरिभाषाया अप्रवृत्तेः~।
सर्वविधसन्निपातग्रहणादेव `वर्णाश्रयः प्रत्ययो वर्णविचालस्यानिमित्तं स्यात्’ इत्येतत्परिभाषादोषनिरूपणावसरे वार्तिककृतोक्तम् , `न हि प्रत्ययः पूर्वपरसन्निपातनिमित्तकः’ ।
स एव च सन्निपातशब्देन गृह्यत इति मत्वा न प्रत्ययः सन्निपातनिमित्तकः इति शङ्कायां तदभ्युपेत्यैवाङ्गसंज्ञा तर्ह्यनिमित्तं स्यादित्येकदेशिनोक्तम् इति न तद्भाष्यविरोधः~।
किञ्चैवं `शैवः, गार्ग्यः, वैनतेयः’ इत्यादावप्यङ्गसंज्ञाया लोपनिमित्तत्वानापत्त्या `वर्णाश्रय' इत्यस्य वैयर्थ्यम्~।
 `ग्रामणि कुलम्’, `ग्रामणिपुत्रः’ इत्यादावुत्तरपदनिमित्तके ह्रस्वत्वे यथाकथञ्चिद्बहिरङ्गपरिभाषयापि वारणं सम्भवतीति ``कृन्मेजन्तः'' इत्यत्र ``ह्रस्वस्य पिति'' इति सूत्रे चैकदेशिना तया परिभाषया तुग् वारितो भाष्ये~।
अत एव परिभाषाफलत्वेनेदमुक्तं ``कृन्मेजन्तः'' इति सूत्रे वार्तिककृतेति केचित्~।
सन्निपातलक्षणविधित्वमस्या लिङ्गम्~।
स्वप्रवृत्तेः प्राक्स्वनिमित्तभूतो यः सन्निपातः, तद्विघातस्य स्वातिरिक्तशास्त्रस्य स्वयमनिमित्तमिति फलति~।
नन्वेवं `रामाय’ इत्यादौ ``सुपि च'' इति दीर्घानापत्तिः, अदन्ताङ्गसन्निपातेन जातस्य यादेशस्य तद्विघातकत्वात्~।
न च यञादित्वसापेक्षदीर्घस्य बहिरङ्गतयासिद्धत्वान्नात्र सन्निपातविघात इति वाच्यम्~, आरोपितासिद्धत्वेऽपि वस्तुतस्तद्विघातस्य जायमानत्वेनैतत्प्रवृत्तेः~।
किञ्चान्तरङ्गे कर्तव्ये बहिरङ्गस्यासिद्धत्वेऽपि तत्र कृते तस्यासिद्धत्वे मानाभावः~।
किञ्चातिदेशिकसन्निपातविघाताभावमादायैतदप्रवृत्तौ `गौरि’ इत्यादौ सम्बुद्धिलोपेऽपि स्थानिवत्त्वेन ह्रस्वनिमित्तसन्निपातविघाताभावात्~, तत्रैतस्यातिव्याप्तिपर-``कृन्मेजन्तः''-इतिसूत्रस्थभाष्यासङ्गतिः~।
सन्निपातस्याशास्त्रीयत्वान्नात्र स्थानिवत्त्वमिति चेत्~, तर्ह्यत्रासिद्धत्वमपि कथमिति विभावय~।
अशास्त्रीयेऽसिद्धत्वाप्रवृत्तेः ``ईदूदेत्'' इतिसूत्रे कैयटेन स्पष्टमुक्तत्वात्~।
एवञ्च पूर्वत्रासिद्धीयेऽपि कार्य एतत्परिभाषाप्रवृत्तिर्भवत्येवेति चेत्~, न ।
 ``कष्टाय'' इति निर्देशेनैतस्या अनित्यत्वात्~।
ययोः सन्निपातस्य विघातकं शास्त्रम् , तयोः सन्निपातनिमित्तकविधावुपादानमपेक्षितमिति तु नाग्रहः~।
अत एव `दाक्षिः' इत्यत्राकारान्तप्रकृति-इञ्सन्निपातनिमित्ताङ्गसंज्ञानया परिभाषयाल्लोपस्य निमित्तं न स्यादित्याशङ्क्यानित्यत्वेन समाहितम् ,``कृन्मेजन्तः'' इतिसूत्रे भाष्ये ।
`न ह्यङ्गसंज्ञायामदन्तस्याङ्गसंज्ञा' इत्युक्तमस्ति~।
न च `कुम्भकारेभ्यः’, `आधये’ इत्यादावव्ययसंज्ञाया अनया परिभाषया वारणपरभाष्यासङ्गतिः~।
अनया परिभाषया लुङ्‍ मा भूत्~, अव्ययत्वं तु स्यादेव~।
लुका हि तदीयसन्निपातस्य विघातः, नाव्ययसंज्ञया~।
संज्ञाफलं त्वकच्स्यादिति वाच्यम् ।
एतदुदाहरणपरभाष्यप्रामाण्येन साक्षात्परम्परया वा स्वनिमित्तसन्निपातविघातकस्य स्वयमनिमित्तमित्यर्थेनादोषात्~।
एतेनात्राकच्स्यादित्यपास्तम्~।
न च कार्यकालपक्षे लुगेकवाक्यतापन्नसंज्ञाबाधेऽप्यकजेकवाक्यतापन्ना संज्ञा स्यादिति वाच्यम्~, अन्तरङ्गायां तदेकवाक्यतापन्नसंज्ञायां बहिरङ्गगुणादेरसिद्धत्वात्~, लुगेकवाक्यतापन्ना तु न गुणादितोऽन्तरङ्गा,
उभयोरपि शब्दतः सुबाश्रयत्वात्~।
 ``न यासयोः'' इति निर्देशाच्चैषानित्या~।
तेन नातिप्रसङ्गः~।
स्पष्टा चेयं ``कृन्मेजन्तः'' इतिसूत्रे भाष्ये~।
अस्या अनित्यत्वे फलानि भाष्ये परिगणितानि ।
वर्णाश्रयः प्रत्ययः वर्णविचालस्यानिमित्तम् - `दाक्षिः’~।
आत्वं पुग्विधेः- क्रापयति~।
पुग्घ्रस्वत्वस्य- अदीदपत्~।
त्यदाद्यकारष्टाब्विधेः- या सेति~।
इड्विधिराकारलोपस्य - पपिवान्~। ``ह्रस्वनुड्भ्यां मतुप्''~, ``अन्तोदात्तादुत्तरपदात्'' इति मतुब्विभक्त्युदात्तत्वं पूर्वनिघातस्य - अग्निमान्~, परमवाचा~।
नदीह्रस्वत्वं सम्बुद्धिलोपस्य - `नदि’~, `कुमारि’ इत्यादि~।
यादेशो दीर्घत्वस्य - कष्टाय~।
इतोऽन्यत्र प्रवृत्तिरेव, `दोषाः खल्वपि साकल्येन परिगण्ताः’ इति भाष्योक्तेरिति अन्यत्र विस्तरः॥\par
ननु `पञ्चेन्द्राण्यो देवता अस्य पञ्चेन्द्रः’ इत्यादौ ``द्विगोर्लुक्'' इत्यणो लुकि ``लुक्तद्धित'' इति स्त्रीप्रत्ययलुक्यानुकः श्रवणापत्तिः, अत आह-
\section*{\begin{center}सन्नियोगशिष्टानामन्यतरापाय उभयोरप्यपायः॥८६॥\end{center}}
\addcontentsline{toc}{section}{सन्नियोगशिष्टानामन्यतर}
अत्र च ``बिल्वकादिभ्यश्छस्य लुक्'' इतिसूत्रस्थं छग्रहणं ज्ञापकम्~।
तद्धि छमात्रस्य लुग्बोधनद्वारा कुकोऽनिवृत्तिर्यथा स्यादित्यर्थम्~।
कृतकुगागमा नडाद्यन्तर्गता बिल्वादय एव तत्र निर्दिष्टा बिल्वकादिशब्देन~।
न चैवमपि छग्रहणं व्यर्थम्~, कृत कुगागमानुवादसामर्थ्यादेव तदनिवृत्तिसिद्धेः, अन्यथा ``बिल्वादिभ्यः'' इत्येव वदेत्~।
लक्षणप्रतिपदोक्तपरिभाषया बिल्वादिपुरस्कारेण विहितप्रत्ययस्यैव लुग्विधानान्नातिप्रसङ्ग इति वाच्यम्~, ततोऽपि प्रतिपदोक्तत्वेन ``बिल्वादिभ्योऽण्'' इति विकाराद्यर्थस्य लुगापत्तिवारणार्थं कुगनुवादचारितार्थ्यात्~।
समुच्चयार्थकचशब्दयोगे तु विधेययोरेककालिकत्वैकदेशत्वनियमान्न्यायसिद्धापीयम्~।
यत्तु ``णाविष्ठवत्'' इत्यनेन पुंवत्वविधानमेतदनित्यत्वज्ञापनार्थम्~।
अन्यथा `एतयति’ इत्यादौ टिलोपेनैव ङीपि निवृत्ते सन्नियोगशिष्टपरिभाषया तस्यापि निवृत्त्या `एतयति’ इत्यादिसिद्धौ पुंवत्ववैयर्थ्यं स्पष्टमेवेति ``टेः'' इति सूत्रे कैयटः, तन्न ।
 `इडबिडमाचष्ट ऐडबिडयति’ इत्यादौ पुंवत्वस्यावश्यकत्वात्~।
’ऐनेयः’, `श्यैनेयः’ इत्यादि तु स्थानिवत्त्वेन सिद्धमित्यन्यत्र विस्तरः~॥\par
ननु `चुरा शीलमस्याः सा चौरी’ इत्यादौ ``शीलम्''~, ``छत्रादिभ्यो णः'' इति णे ङीप्न प्राप्नोतीत्यत आह-
\section*{\begin{center}ताच्छीलिके णेऽण्कृतानि भवन्ति॥८७॥\end{center}}
\addcontentsline{toc}{section}{ताच्छीलिके णे}
 ``अन्'' इत्यणि विहितप्रकृतिभावबाधनार्थं ``कार्मस्ताच्छील्ये'' इति निपातनमस्या ज्ञापकम्~।
ताच्छीलिकणान्तात् ``अणो द्व्यचः'' इति फिञ्सिद्धिरप्यस्याः प्रयोजनमिति नव्याः~।
ताच्छीलिक इत्युक्तेः "तदस्यां प्रहरणम्'' इति णे `दाण्डा' इत्येव~। ``कार्मः'' इति सूत्रे भाष्ये स्पष्टा~॥\par
ननु ``कंसपरिमृड्भ्याम्'' इत्यादौ ``मृजेर्वृद्धिः'' दुर्वारा, इत्यत आह-
\section*{\begin{center}धातोः कार्यमुच्यमानं तत्प्रत्यये भवति॥८८॥\end{center}}
\addcontentsline{toc}{section}{धातोः कार्यमुच्यमानं}
भ्रौणहत्ये तत्वनिपातनमस्या ज्ञापकम्~।
`धातोः स्वरूपग्रहणे तत्प्रत्यये कार्यविज्ञानम्' इति पाठस्तु `प्रसृड्भिः' इत्यादौ ``अनुदात्तस्य चर्दुपधस्य'' इत्यामापादनेन भाष्ये दूषितः~।
यत्कार्यं प्रत्ययनिमित्तं तत्रेयं व्यवस्थापिका~।
तेन पदान्तत्वनिबन्धनं ``नशेर्वा'' इति कुत्वं `प्रणग्भ्याम्' इत्यादौ भवत्येव~।
इयङादिविधौ तु नैषा, ``न भूसुधियोः'' इति निषेधेनानित्यत्वात्~।
 ``मृजेर्वृद्धिः'' इत्यत्र भाष्ये स्पष्टा॥\par
ननु `सर्वके, उच्चकैः' इत्यादौ सर्वनामाव्ययसंज्ञे न स्याताम्~, अत आह-
\section*{\begin{center}तन्मध्यपतितस्तद्ग्रहणेन गृह्यते॥८९॥\end{center}}
\addcontentsline{toc}{section}{तन्मध्यपतितः}
 ``नेदमदसोरकोः'' इति सूत्रे `अकोः' इति निषेधोऽस्या ज्ञापकः~।
{\bfseries तदेकदेशभूतं तद्ग्रहणेन गृह्यते} इति ``येन विधिः'' इति सूत्रे भाष्ये पाठः॥\par
ननु ``गातिस्थाघुपाभूभ्यः'' इति सिचो लुक् `अपासीत्' इत्यादौ पातेरपि स्यात्~, अत आह-

\section*{\begin{center}लुग्विकरणालुग्विकरणयोरलुग्विकरणस्य॥९०॥\end{center}}
\addcontentsline{toc}{section}{लुग्विकरणालुग्विकरणयोः}
अस्याञ्च ज्ञापकः ``स्वरति सूति'' इति सूत्रे सूङिति वक्तव्ये सूतिसूयत्योः पृथङ्निर्देश इति कैयटः~।
तन्न, साहचर्यादलुग्विकरणस्यैव ग्रहणे प्राप्ते पृथङ्निर्देशस्य तज्ज्ञापकत्वासम्भवात्~।
ध्वनिता चेयं परिभाषा ``यस्य विभाषा'' इत्यत्र भाष्ये~।
तत्र हि `विदितः' इति प्रयोगे निषेधमाशङ्क्य {\bfseries यदुपाधेर्विभाषा, तदुपादेर्निषेधः ``विभाषा गमहनविदविशाम्'' इति सूत्रे शविकरणस्य ग्रहणं लुग्विकरणश्चायम्} इत्युक्तम्~।
तत्र चो हेतौ, यतोऽयं लुग्विकरणः, अतो विशिसाहचर्याच्छविकरणस्य ग्रहणम्~, न तु हनिसाहचर्यादस्यापि, एतत्परिभाषाविरोधादिति तदाशयः~।
अत एव परिभाषायां लुग्विकरणस्यैवेति नोक्तम्~।
कण्ठतस्तु भाष्ये एषा क्वापि न पठिता~।
``गातिस्था-'' इति सूत्रे `पिबतेर्ग्रहणं कर्तव्यम्' इति वार्तिककृता {\bfseries सर्वत्रैव पाग्रहणेऽलुग्विकरणस्य ग्रहणम्} इति भाष्यकृता चोक्तम्~।
 ``स्वरतिसूति-'' इति सूत्रे कैयटेन च स्पष्टमुक्ता~॥\par
ननु `प्रजिघाययिषति' इत्यादौ ``हेरचङि'' इति विधीयमानं कुत्वं न स्यादत आह-
\section*{\begin{center}प्रकृतिग्रहणे ण्यधिकस्यापि ग्रहणम् ॥९१॥\end{center}}
\addcontentsline{toc}{section}{प्रकृतिग्रहणे ण्यधिकस्य}
अचङीति प्रतिषेध एवास्या ज्ञापकः~।
इयं च कुत्वविषयैव~। ``हेरचङि'' इति सूत्रे भाष्ये स्पष्टेयम्~॥\par
ननु `युष्मभ्यम्' इत्यादौ ``भ्यसः'' इत्यत्र भ्यमिति च्छेदे भ्यसो भ्यमि कृतेऽन्त्यलोपे एत्वं स्यात्~, अत आह-
\section*{\begin{center}अङ्गवृत्ते पुनर्वृत्तावविधिः ॥९२॥\end{center}}
\addcontentsline{toc}{section}{अङ्गवृत्ते पुनः}
अङ्गेऽङ्गाधिकारे वृत्तं निष्पन्नं यत्कार्यम्~, तस्मिन्सति पुनरन्यस्याङ्गकार्यस्य वृत्तौ प्रवृत्तावविधानं भवतीत्यर्थः~।
एषा च ``ज्यादादीयसः'' इत्याद्विधानेन ज्ञापिता~।
अन्यथेकारलोपेन ``अकृत्सार्व-'' इति दीर्घेण च सिद्धे तद्वैयर्थ्यं स्पष्टमेव~।
अत एव ``ज्ञाजनोर्जा'' ``ज्यादादीयसः'' इति सूत्रयोरेनां ज्ञापयित्वा किं प्रयोजनमिति प्रश्ने पिबतेर्गुणप्रतिषेध उक्तः, स न वक्तव्यः, इत्येव प्रयोजनमुक्तम्~, न तु लक्ष्यसिद्धिरूपम्~।
तदुक्तं ``भ्यसो भ्यम्'' इत्यत्र `अभ्यम्' इति च्छेदः, ``शेषे लोपः'' चान्त्यलोप एव ``अतो गुणे'' इति पररूपेण सिद्धं `युष्मभ्यम्' इत्यन्यत्र निरूपितम्~।
एवञ्च सूत्रद्वयस्थमेतज्ज्ञापनपरं भाष्यं ``भ्यसो भ्यम्'' इति सूत्रस्थं च भाष्यमेकदेश्युक्तिरित्याहुः॥\par
यत्तु\par
 `ओरोत्' इति वाच्ये "ओर्गुणः'' इति गुणग्रहणात्
\section*{\begin{center}संज्ञापूर्वकविधेरनित्यत्वम् ॥९३॥\end{center}}
\addcontentsline{toc}{section}{संज्ञापूर्वकविधेः}
इयञ्च विधेयकोटौ संज्ञापूर्वकत्व एव~।
तेन `स्वायम्भुवम्' इत्यादि सिद्धम्~॥\par
तथा निलोडित्येव सिद्धे आनिग्रहणात्
\section*{\begin{center}आगमशास्त्रमनित्यम् ॥९४॥\end{center}}
\addcontentsline{toc}{section}{आगमशास्त्रम्}
तेन सागरं तर्त्तुकामस्येत्यादि सिद्धम्~॥\par
तथा तनादिपाठादेव सिद्धे ``तनादिकृञ्भ्यः'' इति सूत्रे कृञ्ग्रहणात्
\section*{\begin{center}गणकार्यमनित्यम् ॥९५॥\end{center}}
\addcontentsline{toc}{section}{गणकार्यमनित्यम्}
तेन `न विश्वसेदविश्वस्तम्' इत्यादि सिद्धम्~॥
तथा चक्षिङो ङित्करणात्-
\section*{\begin{center}अनुदात्तेत्त्वलक्षणमात्मनेपदमनित्यम् ॥९६॥\end{center}}
\addcontentsline{toc}{section}{अनुदात्तेत्त्वलक्षणम्}
तेन `स्फायन्निर्मोकः' इत्यादि सिद्धम् ॥\par
तथा विनार्थनञा समासेन ``अनुदात्तं पदमनेकम्'' इत्येव सिद्धे वर्जग्रहणात्
\section*{\begin{center}नञ्घटितमनित्यम् ॥९७॥\end{center}}
\addcontentsline{toc}{section}{नञ्घटितमनित्यम्}
तेन ``नेयङुवङ्-'' इत्यस्यानित्यत्वात् `हे सुभ्रु' इति सिद्धमिति, तन्न~।
भाष्येऽदर्शनात्~, भाष्यानुक्तज्ञापितार्थस्य साधुतानियामकत्वे मानाभावात्~।
भाष्याविचारितप्रयोजनानां सौत्राक्षराणां पारायणादावदृष्टमात्रार्थकत्वकल्पनाया एवौचित्यात्~।
किञ्च ज्ञापितेऽप्यानीत्यस्य न सार्थक्यम्~, आडागमशून्यप्रयोगस्याप्रसिद्धेः~।
आड्ग्रहणं तु लोड्ग्रहणवदिति बोध्यम्~।
अत एव ``घोर्लोपो लेटि वा'' इति सूत्रे वेति प्रत्याख्यातम्~।
लोपेऽप्याट्‍पक्षे आटः श्रवणं भविष्यति `दधात्' इति, अटि `दधत्' इति~।
आगमशास्त्रस्यानित्यत्वे त्वाट्यसति `दधात्' इत्यसिद्ध्या वाग्रहणस्यावश्यकत्वेन तत्प्रत्याख्यानासङ्गतिः स्पष्टैव~।
एतेन यत्कैयटेन केचिदित्यादिनास्यैव वाग्रहणस्य तदनित्यत्वज्ञापकतोक्ता, सापि चिन्त्या, प्रत्याख्यानपरभाष्यविरोधात्~।
चक्षिङो ङकारस्यान्तेदित्त्वाभावसम्पादनेन चारितार्थ्याच्च~।
तनादिसूत्रे कृञ्ग्रहणस्य भाष्ये प्रत्याख्यानाच्च~॥\par
एवमेव-

\section*{\begin{center}आतिदेशिकमनित्यम् ॥९८॥\end{center}}
\addcontentsline{toc}{section}{आतिदेशिकम्}
\section*{\begin{center}सर्वविधिभ्यो लोपविधिरिड्विधिश्च बलवान् ॥९९॥\end{center}}
\addcontentsline{toc}{section}{सर्वविधिभ्यो लोपविधिः}
इत्यादि भाष्यानुक्तं बोध्यम्~।
`स्वायम्भुवम्' इत्यादि लोकेऽसाध्वेव, इत्यन्यत्र विस्तरः ॥
यदपि ननु हन्तेर्यङ्लुक्याशीर्लिङि वधादेशो न स्यात्~, अत आह-
\section*{\begin{center}प्रकृतिग्रहणे यङ्लुगन्तस्यापि ग्रहणम् ॥१००॥\end{center}}
\addcontentsline{toc}{section}{प्रकृतिग्रहणे यङ्लुगन्तस्य}
षाष्ठद्वित्वस्य द्विःप्रयोगसिद्धान्तेन प्रयोगद्वयरूपे समुदाये प्रकृतिरूपत्वबोधनेनेदं न्यायसिद्धम्~।
अत एव जुहुधीत्यादौ द्वित्वे कृते धित्वसिद्धिरिति~।
तदपि न, भाष्येऽदर्शनात्~।
किञ्च तेन सिद्धान्तेन प्रत्येकं द्वयोस्तत्त्वबोधनेऽपि समुदायस्य तत्त्वबोधने मानाभावः~।
अत एव ``दयदेर्दिगि'' इति सूत्रेऽस्तेः परत्वाद्द्वित्वे कृते परस्यास्तेर्भूभावे पूर्वस्य श्रवणं प्राप्नोतीत्याशङ्क्य विषयसप्म्याश्रयणेन परिहृतं भाष्ये~।
अन्यथा त्वदुक्तरीत्या एकाज्द्विर्वचनन्यायेन समुदायस्यैवादेशापत्तौ तदसङ्गतिः स्पष्टैव~।
तस्मादुत्तरखण्डमादायैव यथायोगं तत्तत्कार्यप्रवृत्तिर्बोध्या~।
 ``भूसुवोः'' इत्यस्य तदन्ताङ्गस्येत्यर्थात्प्राप्तस्य गुणनिषेधस्य बोभूत्विति नियम इति न तद्विरोधः~।
तस्माद्धन्तेर्यङ्लुकि `वध्यात्' इत्यादि माधवाद्युदाहृतं चिन्त्यमेवेत्यन्यत्र विस्तरः ॥\par
यदपि ननु ``वृद्धिर्यस्याचामादिः'' इत्यत्रेक्परिभाषोपस्थितौ शालीयाद्यसिद्धिः, अत आह-
\section*{\begin{center}विधौ परिभाषोपतिष्ठते नानुवादे ॥१०१॥\end{center}}
\addcontentsline{toc}{section}{विधौ परिभाषा}
अनूद्यमानविशेषणेषु तन्नियामिका परिभाषा नोपतिष्टत इति तदर्थः~।
विध्यङ्गभूतानां परिभाषाणां विधेयेनासिद्धतया सम्बन्धासम्भवेऽपि तद्विशेषणे व्यवस्थापकत्वेन चरितार्थानां तद्विशेषणव्यवस्थापकत्वे मानाभाव इति तर्कमूलेयम्~।
किञ्च ``उदीचामातः स्थाने'' इति सूत्रे स्थानेग्रहणमस्या लिङ्गम्~।
अन्यथा ``षष्ठी स्थाने'' इति परिभाषयैव तल्लाभे तद्वैयर्थ्यं स्पष्टमेवेति, तन्न~।
 ``उदात्तस्वरितयोर्यणः'' इत्यादौ ``ष्यङः सम्प्रसारणम्'' इति सूत्रभाष्योक्तरीत्या ``अल्लोपोऽनः'' इत्यादौ चैतस्या व्यभिचरितत्वात्~, भाष्यानुक्तत्वाच्च~।
स्थानसम्बन्धो न परिभाषालभ्य इत्यर्थस्य ``षष्टी स्थाने'' इति सूत्रे भाष्ये स्पष्टमुक्तत्वेन त्वदुक्तज्ञापकासम्भवाच्च~।
तत्र स्थानेग्रहणं तु स्पष्टार्थमेव~।
किञ्च `विधौ परिभाषा’ इति प्रवादः ``इको गुणवृद्धी'' ``अचश्च'' इत्यनयोर्विधीयत इत्यध्याहारमूलकः, अन्यत्र तु नास्या फलमित्यन्यत्र विस्तरः ॥\par
ननु `नमस्करोति देवान्', `नमस्यति देवान्' इत्यादौ ``नमःस्वस्ति-'' इति चतुर्थी दुर्वारा, इत्यत आह-
\section*{\begin{center}उपपदविभक्तेः कारकविभक्तिर्बलीयसी ॥१०२॥\end{center}}
\addcontentsline{toc}{section}{उपपदविभक्तेः}
कारकविभक्तित्वञ्च क्रियाजनकार्थकविभक्तित्वम्~।
तच्च प्रथमाया अप्यस्तीति सापि कारकविभक्तिरिति ``सहयुक्ते'' इत्यादिसूत्रेषु भाष्ये ध्वनितम्~।
इयं च वाचनिक्येव~।
अत एव ``यस्य च भावेन'' इति सप्तम्यपेक्षयाधिकरणसप्तम्या बलवत्त्वमनेन न्यायेन, ``तत्र च दीयते'' इति सूत्रे भाष्ये ध्वनितम् , कैयटेन च स्पष्टमुक्तम्~।
एतेन `क्रियान्वयित्वं कारकत्वम्' इत्यपास्तम् ,``यस्य च भावेन'' इति सप्तम्या अपि क्रियान्वयित्वात्~।
ये तु प्रधानीभूतक्रियासम्बन्धनिमित्तकार्यत्वेन कारकविभक्तीनां बलवत्त्वं वदन्ति, तेषामुभयोरपि क्रियासम्बन्धनिमित्तकत्वेन तदसङ्गतिः स्पष्टैव~।
 ``नमो वरिव'' इतिसूत्रे `नमस्यति देवान्' इत्यादौ चतुर्थीवारणाय भाष्ये उपन्यासस्यासङ्गतेश्च~।
एतेन `क्रियाकारकसम्बन्धोऽन्तरङ्गः' इति तन्निमित्ता विभक्तिरन्तरङ्गा, उपपदार्थेन तु यत्किञ्चित्क्रियाकारकभावमूलकः सम्बन्ध इति तन्निमित्ता विभक्तिर्बहिरङ्गेत्यपास्तम् , `नमस्यति' इत्यत्र नमः पदार्थेऽपि क्रियाकारकाभावे नैवान्वयात्~।
अत्र च नमःपदार्थस्यापि क्रियात्वं मुण्डयतौ मुण्डस्येव~।
 ``सहयुक्ते'' इत्यादौ च प्रधाने प्रथमा साधनार्थमियं भाष्य उपन्यस्तेत्यन्यत्र विस्तरः ॥\par
ननु `अदमुयङ्' इत्यादौ पूर्वस्यापि मुत्वापत्तिः, अत आह-
\section*{\begin{center}अनन्त्यविकारेऽन्त्यसदेशस्य ॥१०३॥\end{center}}
\addcontentsline{toc}{section}{अनन्त्यविकारे}
अन्त्यसदेशानन्त्यसदेशयोरेकप्रयोगे युगपत्प्राप्तवन्त्यसदेशस्यैवेति तदर्थः~।
अन्यथा धात्वादेर्नत्वसत्वे `नेता', `सोता' इत्यादावेव स्याताम्~, न तु `नमति', `सिञ्चति' इत्यादौ~।
अन्त्यविकार इति च लिङ्गम्~।
अन्त्येन समानो देशो यस्य सोऽन्त्यसदेशः~।
तत्त्वं चान्त्यवर्णतद्वर्णयोरितराव्यवधानेन बोध्यम्~।
अत एव `विद्ध' इत्याद्यर्थं ``न सम्प्रसारणे'' इति चरितार्थम्~।
"अल्लोपोऽनः'' इत्यादेः `अनस्तक्ष्णा' इत्यादावाद्याकारादावप्रवृत्तिरप्यस्याः फलम् , यजादिस्वादिपरान्नन्ताङ्गस्याकारस्य लोप इत्यर्थस्यैवाङ्गांशे प्रत्ययस्योत्थिताकाङ्क्षतयौचित्यादङ्गावयवयजादिस्वादिपरस्यान इत्यादिक्रमेणानेकत्रानेकक्लिष्टकल्पनापेक्षयास्या उचितत्वात्~।
न चैषा ``ष्यङः सम्प्रसारणम्'' इति सूत्रे भाष्ये प्रत्याख्यातेति भ्रमितव्यम्~, वार्तिकोक्तफलानामनेकक्लिष्टकल्पनाभिरन्यथासिद्धिं प्रदर्श्यापि यान्येतस्याः परिभाषायाः प्रयोजनानि, तदर्थमेषा कर्तव्या प्रतिविधेयं दोषेषु, प्रतिविधानं चोदात्तनिर्देशात्सिद्धमित्युपसंहारात्~।
`मिमार्जिषति' इत्यर्थं चैषा~।
तत्र वृद्धेः पूर्वमन्तरङ्गत्वाद्द्वित्वे परत्वादभ्यासकार्ये ततोऽभ्यासेकारस्य वृद्धिवारणायावश्यकी~।
न च वृद्धौ पुनरभ्यासह्रस्वत्वेन सिद्धिः, `लक्ष्ये लक्षणस्य' इति न्यायेन पुनरप्रवृत्तेः~।
यत्तु ``न सम्प्रसारणे'' इति सूत्रे भाष्ये {\bfseries `नैतस्याः परिभाषायाः प्रयोजनानि'} इत्युक्तम्~।
तस्यायमर्थः एतत्सूत्रे प्रयोजनान्येतस्याः परिभाषाया न भवन्ति, व्यधादावन्त्यसमानादेशयणोऽभावादिति~।
`नैतानि एतस्याः प्रयोजनानि' इति पाठोऽपि क्वचिद्दृश्यते~।
वाचनिक्येवैषा~।
स्पष्टा च ``ष्यङः'' इति सूत्रे ``अदसोऽसेः'' इति सूत्रे च `केचिदन्त्यसदेशस्य' इत्यनेन भाष्य इत्यन्यत्र विस्तरः॥\par
ननु ``अव्यक्तानुकरणस्यातः'' इति पररूपम् , `पटत् इति, पटिति' इत्यदौ ``अलोऽन्त्यस्य'' इत्यन्त्यस्य प्राप्नोतीत्यत आह-
\section*{\begin{center}नानर्थकेऽलोऽन्त्यविधिरनभ्यासविकारे ॥१०४॥\end{center}}
\addcontentsline{toc}{section}{नानर्थकेऽलोऽन्त्यविधिः}
अनभ्यासेत्युक्तेर्बिभर्ति इत्यादौ ``भृञामित्'' इत्याद्यन्त्यस्यैव~।
अभ्यासोऽनर्थकः, अर्थावृत्त्यभावात्~, किन्तूत्तरखण्ड एवार्थवानित्यन्यत्र निरूपितम्~।
एषा ``अलोऽन्त्यात्-'' सूत्रे भाष्ये स्पष्टा~।
फलानामन्यथासिद्धिकरणेन प्रत्याख्याता चेति तत एवावधार्यम् ॥\par
ननु `ब्राह्मणवत्सा च ब्राह्मणीवत्सश्च' इत्यादौ ``पुमान्स्त्रिया'' इत्येकशेषापत्तिः, स्त्रीत्वपुंस्त्वातिरिक्तकृतविशेषाभावात्~, अत आह-
\section*{\begin{center}प्रधानाप्रधानयोः प्रधाने कार्यसम्प्रत्ययः ॥१०५॥\end{center}}
\addcontentsline{toc}{section}{प्रधानाप्रधानयोः}
तेन प्रधानस्त्रीत्वपुंस्त्वातिरिक्ताप्रधानस्त्रीत्वपुंस्त्वकृतविशेषस्यापि सत्वेन न दोषः~।
स्पष्टा चेयं ``नपुंसकमनपुंसकेन" इत्यनयोर्भाष्ये~।
"अन्तरङ्गोपजीव्यादपि प्रधानं प्रबलम् इति "हेतुमति च" इत्यत्र भाष्यकैयटयोः ॥\par
ननु स्वस्रादित्वप्रयुक्तो मातृशब्दस्य ङीब्निषेधः परिच्छेत्तृवाचकमातृशब्देऽपि स्यात्~, अत आह- 

\section*{\begin{center}अवयवप्रसिद्धेः समुदायप्रसिद्धिर्बलीयसी~॥१०६॥\end{center}}
\addcontentsline{toc}{section}{अवयवप्रसिद्धेः}
तेन शुद्धरूढस्य जननीवाचकस्यैव ग्रहणम् , न परिच्छेत्तृवाचकस्य~।
`योगजबोधे तदनालिङ्गितशुद्धरूढजोपस्थितिः प्रतिबन्धिका' इति व्युत्पत्तिरेव तद्बीजम्~।
रथकाराधिकरणन्यायसिद्धोऽयमर्थः~।
कश्चित्तु ``दीधीवेवीटाम्'' इत्यत्रानया परिभाषया दीधीवेवीङोरेव ग्रहणम् , न दीङ्-धीङ्-वेञ्-वीनामिति, तन्न~। तथा सति `दीवेधीवीटाम्' इत्येव वदेदित्यन्ये~॥\par
ननु वातायनार्थे गवाक्षेऽवङो वैकल्पिकत्वात् `गोऽक्षः' इत्याद्यपि स्यात्~, अत आह-
\section*{\begin{center}व्यवस्थितविभाषयापि कार्याणि क्रियन्ते~॥१०७॥\end{center}}
\addcontentsline{toc}{section}{व्यवस्थितविभाषयापि}
लक्ष्यानुसाराद्व्यवस्था बोध्या~।
 ``शाच्छोः'' इति सूत्रे ``लटः शतृ-'' इत्यादिसूत्रेषु च भाष्ये स्पष्टा~॥
\section*{\begin{center}विधिनियमसम्भवे विधिरेव ज्यायान्~॥१०८॥\end{center}}
\addcontentsline{toc}{section}{विधिनियमसम्भवे}
नियमे ह्यश्रुताया अन्यनिवृत्तेः सामर्थ्यात्परिकल्पनमुक्तानुवाददोषश्चेति लाघवाद्विधिरेवेति बोध्यम्~।
 ``यस्य हलः'' इत्यत्र ``इजादेः सनुमः'' इत्यादौ च भाष्ये स्पष्टेयम्~॥\par
ननु ``आशंसायां भूतवच्च'' इत्यनेन लुङ इव लङ्-लिटोरप्यतिदेशः स्यात्~, अत आह-
\section*{\begin{center}सामान्यातिदेशे विशेषानतिदेशः~॥१०९॥\end{center}}
\addcontentsline{toc}{section}{सामान्यातिदेशे}
सामान्योपस्थितिकाले नियमेन विशेषोपस्थापकसामग्र्यभावोऽस्या बीजम्~।
तेनानद्यतनभूतरूपे विशेषे विहितयोस्तयोर्नातिदेशः~।
इयमनित्या, `न ल्यपि' इति लिङ्गात्~।
तेन स्थानिवत्सूत्रेण विशेषातिदेशोऽपि~।
स्पष्टं चैतत्सर्वं स्थानिवत्सूत्रे भाष्ये~॥\par
 ननु ``तित्स्वरितम्'' इति स्वरितत्वं `चिकीर्षति' इत्यादौ स्यात्~, अत आह-
\section*{\begin{center}प्रत्ययाप्रत्यययोः प्रत्ययस्य ग्रहणम्~॥११०॥\end{center}}
\addcontentsline{toc}{section}{प्रत्ययाप्रत्यययोः}
इयं च ``अङ्गस्य'' इतिसूत्रभाष्ये पठिता~।
वर्णग्रहणे च न प्रवर्तत इति तत्रैव कैयटे स्पष्टम्~।
अत एव ``सनाशंसभिक्ष उः''~, ``वले'' इत्यत्र सन्वलयोः प्रत्यययोर्ग्रहणम्~।
परे तु ``तित्स्वरितम्'' इति सूत्र एषा परिभाषा लक्ष्यसंस्काराय भाष्ये क्वापि नाश्रितेति कैयटेनोक्तम्~।
 ``अङ्गस्य'' इति सूत्रे तत्प्रत्याख्यानायैषा भाष्य एकदेशिनोक्ता~।
अत एव `तिति प्रत्ययग्रहणं कर्तव्यम्' इति वार्तिककृतोक्तम् उक्तसूत्रयोर्व्याख्यानात्प्रत्यययोरेव ग्रहणमित्याहुः~॥\par
ननु ``विपराभ्यां जेः'' इत्यात्मनेपदं `परा सेना जयति' इत्यर्थके `परा जयति सेना' इत्यत्र प्राप्नोतीत्यत आह-
\section*{\begin{center}सहचरितासहचरितयोः सहचरितस्यैव ग्रहणम्~॥१११॥\end{center}}
\addcontentsline{toc}{section}{सहचरितासहचरितयोः}
तेन विशब्दसाहचर्यादुपसर्गस्यैव पराशब्दस्य ग्रहणमिति तत्रैव भाष्ये स्पष्टम्~।
सहचरणं सदृशयोरेवेति सहचरितशब्देन सादृश्यवानुच्यते~।
`रामलक्ष्मणौ' इत्यादावपि सादृश्यमेव नियामकम्~।
सदृशयोरेव सहविवक्षा, तयोरेव सहप्रयोग इत्युत्सर्गाच्च~।
ध्वनितं चेदं ``कर्मप्रवचनीययुक्ते द्वितीया'' इति सूत्रे भाष्ये~।
तत्र हि ``पञ्चम्यपाङ्परिभिः'' इति सूत्रेण लक्षणादिद्योतकपरियोगे पञ्चमीमाशङ्क्य {\bfseries यद्यप्ययं परिर्दृष्टापचारो वर्जने चावर्जने च, अयं खल्वपशब्दोऽदृष्टापचारो वर्जनार्थ एव कर्मप्रवचनीयः, तस्य कोऽन्यः सहायो भवितुमर्हत्यन्यो वर्जनार्थात्, यथास्य गोः सहायेनार्थ इति गौरेवानीयते, नाश्वः, न गर्दभः} इत्युक्तम्~।
तेन हि सदृशानामेव प्रयोगे सहायभावो बाधितः~।
 ``द्विस्त्रिश्चतुः'' इति सूत्रे साहचर्येणैव कृत्वोऽर्थस्य ग्रहणे सिद्धे कृत्वोर्थग्रहणादेषानित्या~।
तेन ``दीधीवेवीटाम्'' इत्यत्र धातुसाहचर्येऽप्यागमस्येटो ग्रहणमित्यन्यत्र विस्तरः~॥\par
ननु ``अस्थि-'' इत्यनङ् `प्रियसक्थना ब्राह्मणेन' इत्यत्र न स्यात्~, अङ्गस्य नपुंसकत्वाभावात्~, अत आह-
\section*{\begin{center}श्रुतानुमितयोः श्रुतसम्बन्धो बलवान्~॥११२॥\end{center}}
\addcontentsline{toc}{section}{श्रुतानुमितयोः}
श्रुतेनैव सम्बन्धः, नानुमितेन प्रकरणादिप्राप्तेनेत्यर्थः~।
प्रकरणादितः श्रुतेर्बलवत्वादिति भावः~।
एवं च तत्र लिङ्गमस्थ्यादीनामेव विशेषणम्~, नाङ्गस्य शिशीलुग्नुम्विधिषु तु गृह्यमाणस्याभावात्प्रकरणप्राप्ताङ्गस्यैव विशेषणम्~।
अत एव ``वा नपुंसकस्य'' इति सूत्रे ``वा शौ'' इति न कृतम्~।
तत्र नपुंसकग्रहणं हि गृह्यमाणस्य शत्रन्तस्यैव नपुंसकत्वे यथा स्यात्~,`बहवो ददतो येषु तानि कुलानि बहुददति' इत्यत्र मा भूत्~।
`बहूनि ददन्ति येषु ते बहुददन्तः' इत्यत्र यथा स्यादित्येवमर्थम्~।
स्पष्टं चेदं ``स्वमोर्नपुंसकात्'' इत्यत्र भाष्ये~।
केचित्तु ``अचो रहाभ्यां द्वे'' इत्यत्र श्रुतेन रेफस्य निमित्तत्वेन यरन्तर्भावादनुमितेन कार्यित्वं बाध्यत इत्येतदुदाहरणमाहुः, तन्न~।
तक्रकौण्डिन्यन्यायेन सिद्धेरित्यन्यत्र विस्तरः~॥\par
ननु ``तत्पुरुषे तुल्यार्थः'' इति स्वरः `परमेण कारकेण, परमकारकेण' इत्यादौ स्यात्~, तथा ``गातिस्थाघुपाभूभ्यः'' इति लुक् `पै शेषणे' इत्यतः कृतात्वात्परस्यापि स्यात्~, अत आह
\section*{\begin{center}लक्षणप्रतिपदोक्तयोः प्रतिपदोक्तस्यैव ग्रहणम्~॥११३॥\end{center}}
\addcontentsline{toc}{section}{लक्षणप्रतिपदोक्तयोः}
तत्तद्विभक्तिविशेषाद्यनुवादेन विहितो हि समासादिः प्रतिपदोक्तः, तस्यैव ग्रहणम्~, शीघ्रोपस्थितिकत्वात्~।
द्वितीयो हि विलम्बोपस्थितिकः~।
`पै' इत्यस्य `पा' इति रूपं लक्षणानुसन्धानपूर्वकं विलम्बोपस्थितिकम्~।
पिबतेस्तु तच्छीघ्रोपस्थितिकम्~।
इदमेव ह्येतत्परिभाषाबीजम्~।
इयं वर्णग्रहणेऽपि ``ओत्'' सूत्रे भाष्ये सञ्चारितत्वात्~।
यत्तु वर्णग्रहणे नैषा ``आदेचः'' इत्यत्रोपदेशग्रहणादिति, तत्तु तस्मिन्नेव सूत्रे शब्देन्दुशेखरे दूषितमिति तत एव द्रष्टव्यम्~।
अनित्या चेयं ``भुवश्च महाव्याहृतेः'' इति महाव्याहृतिग्रहणादित्यन्यत्र विस्तरः~॥\par
नन्वेवं देङो दोधातोश्च कृतात्वस्य घुसंज्ञा न स्यात्~, तथा मेङ आत्वे `प्रणिमाता' इत्यादौ ``नेर्गदनद-'' इति णत्वं न स्यात्~, तथा `गै' इत्यस्यात्वे ``घुमास्था-'' इतीत्वं न स्यात्~, अत आह-
\section*{\begin{center}गामादाग्रहणेष्वविशेषः~॥११४॥\end{center}}
\addcontentsline{toc}{section}{गामादाग्रहणेषु}
अत्र च ज्ञापकं दैपः पित्वम्~।
तद्धि ``अदाप्'' इति सामान्यग्रहणार्थम्~।
अन्यथा लाक्षणिकत्वादेव विधौ तद्ग्रहणे सिद्धे किं निषेधे सामान्यग्रहणार्थेन पित्त्वेन~?
तेन चैकदेशानुमतिद्वारा सम्पूर्णा परिभाषा ज्ञाप्यते~।
इयं च लक्षणप्रतिपदोक्तपरिभाषानिरनुबन्धकपरिभाषालुग्विकरणपरिभाषाणां बाधिका~।
 ``दाधा घु'' इतिसूत्रे भाष्ये स्पष्टा~।
 ``गातिस्था'' इतिसूत्रे इणादेशगाग्रहणमेवेष्यत इति न दोष इत्यन्यत्र विस्तरः~॥
\section*{\begin{center}प्रत्येकं वाक्यपरिसमाप्तिः~॥११५॥\end{center}}
\addcontentsline{toc}{section}{प्रत्येकं वाक्यपरिसमाप्तिः}
देवदत्तादयो भोज्यन्ताम् इत्यत्र भुजिवत्~॥\par
नन्वेवं संयोगसंज्ञासमाससंज्ञाभ्यस्तसंज्ञा अपि प्रत्येकं स्युः, अत आह-
\section*{\begin{center}क्वचित्समुदायेऽपि~॥११६॥\end{center}}
\addcontentsline{toc}{section}{क्वचित्समुदायेऽपि}
{\bfseries गर्गाः शतं दण्ड्यन्ताम्~, अर्थिनश्च राजानो हिरण्येन भवन्ति} इत्यादौ दण्डनवत्~।
लक्ष्यानुरोधेन च व्यवस्था~॥\par
ननु ``यूस्त्र्याख्यौ'' इत्यत्र व्यक्तिपक्षे दीर्घनिर्देशादनण्त्वेन ग्राहकसूत्राप्राप्त्योदात्ताद्यन्यतमोच्चारणेऽन्यस्वरकस्य संज्ञा न स्यात्~, अत आह-
\section*{\begin{center}अभेदकाः गुणाः~॥११७॥\end{center}}
\addcontentsline{toc}{section}{अभेदकाः गुणाः}
असति यत्ने स्वरूपेणोच्चारितो गुणो न भेदकः, न विवक्षित इत्यर्थः~।
अत्र च ``अस्थिदधि-'' इत्यादावनङादेरुदात्तस्यैवोच्चारणेन सिद्धे उदात्तग्रहणं ज्ञापकम्~।
स्वरूपेणोच्चारित इत्युक्तेरनुदात्तादेरन्तोदात्तादित्युदात्तादिशब्दोच्चारणे विवक्षैव~।
 ``उञः''~, ``ऊँ'' इत्यत्राननुनासिक एवोच्चारणीये यत्नाधिक्येनानुनासिकोच्चारणाद्विवक्षा बोध्या~।
 ``पथिमथ्यृभुक्षाम्'' इत्यादौ स्थान्यनुरूपतयानुनासिक एवोच्चारणीये निरनुनासिकोच्चारणात्तद्विवक्षा~।
एतदर्थमेवासति यत्न इत्युक्तम्~।
न चैवमस्थ्यादीनां ``नब्विषयस्य'' इत्याद्युदात्ततयान्त्यादेशस्यानङः स्थान्यनुरूपेऽनुदात्त एवोच्चारणीय उदात्तोच्चारणं विवक्षार्थं भविष्यतीति कथमस्य ज्ञापकत्वमिति वाच्यम्~।
परमास्थिशब्दादाबन्तोदात्त उदात्तगुणकस्यापि स्थानिवत्त्वेन, विवक्षायां मानाभावात्~।
{\bfseries चतसर्याद्युदात्तनिपातनं करिष्यते, वधादेशे आद्युदात्तनिपातनं करिष्यते, पदादयोऽन्तोदात्ता निपात्यन्ते, सहस्य स उदात्तो निपात्यते} इत्यादि भाष्यं त्वेकश्रुत्याष्टाध्यायीपाठे क्वचिदुदात्ताद्युच्चारणं विवक्षार्थमित्याशयेन~।
 `त्रैस्वर्येण पाठः' इति पक्षे तु ज्ञापकपरं भाष्यमिति कैयटादयः~।
परे तु, निपातनं नामान्यादृशे प्रयोगे प्राप्तेऽन्यादृशप्रयोगकरणम्~, तत्तद्रूपाद्यत्नात्तत्र तत्रोदात्तादिविवक्षा~।
 ``तिसृचतसृ'' इत्यत्र द्वन्द्वप्रयुक्तेऽन्तोदात्ते उच्चारणीये आद्युदात्तोच्चारणं अन्यत्र स्थान्यनुरूपे स्वर उच्चारणीये तत्तदुच्चारणं विवक्षार्थम्~।
सम्पूर्णाष्टाध्याय्याचार्येणैकश्रुत्या पठितेत्यत्र न मानम्~।
क्वचित्पदस्यैकश्रुत्यापि पाठः, यथा दाण्डिनायनादिसूत्रे ऐक्ष्वाकेति~।
यद्यप्यध्येतार एकश्रुत्यैवाङ्गानि पठन्ति ब्राह्मणवत्~, तथापि व्याख्यानतोऽनुनासिकत्वादिवदुदात्तनिपातनादिज्ञानमित्याहुः~।
विधेयाण्विषये तु ``अप्रत्ययः'' इति निषेधान्न गुणाभेदकत्वेन सवर्णग्रहणम्~, अत एव `घटवत्' इत्यादौ मतोर्मस्य नानुनासिको वकारः~।
अत एव ``तद्वानासाम्'' इति सूत्रनिर्देशः~।
अन्यथा ``प्रत्यये भाषायाम्'' इति नित्यमनुनासिकः स्यात् ।
जातिपक्षे तु नास्योपयोग इति बोध्यम्~।
 `यू' इत्यादौ दीर्घमात्रवृत्तिजातिनिर्देशान्न क्षतिरित्यन्यत्र विस्तरः~॥\par
ननु `सर्वनामानि' इत्यत्र णत्वाभावनिपातनेऽपि लोके सणत्वप्रयोगस्य साधुत्वं स्यात्~, अत आह-
\section*{\begin{center}बाधकान्येव निपातनानि~॥११८॥\end{center}}
\addcontentsline{toc}{section}{बाधकान्येव निपातनानि}
तत्तत्कार्ये नाप्राप्ते निपातनारम्भात्~।
 ``पुराणप्रोक्तेषु'' इति निपातितपुराणशब्देन पुरातन शब्दस्य बाधः प्राप्तोऽपि पृषोदरादित्वान्नेति बोध्यम्~।
पुराणेति पृषोदरादिः पुरातनेति चेत्यन्ये~।
इयं सर्वादिसूत्रभाष्ये स्पष्टा~।
`अबाधकान्यपि निपातनानि' इति तु भाष्यविरुद्धम्~॥\par
ननूखधातोर्द्वित्वे स्वत एव ह्रस्वत्वात्पूर्वमभ्यासह्रस्वाप्राप्तौ हलादिःशेषे सवर्णदीर्घे ह्रस्वापत्तिः, अत आह-
\section*{\begin{center}पर्जन्यवल्लक्षणप्रवृत्तिः~॥११९॥\end{center}}
\addcontentsline{toc}{section}{पर्जन्यवल्लक्षणप्रवृत्तिः}
एवं च ह्रस्वस्यापि ह्रस्वे कृते `लक्ष्ये लक्षणस्य' इति न्यायेन न पुनर्ह्रस्वः~।
तदुक्तम् ``इको झल्'' इतिसूत्रे भाष्ये {\bfseries कृतकारि खल्वपि शास्त्रं पर्जन्यवत्} इति~।
सिद्धेऽपि ह्रस्वादिकारीत्यर्थः~।
न च `लक्ष्ये लक्षणस्य सकृदेव प्रवृत्तिः' इत्यत्र न मानमिति वाच्यम्~, ``समो वा लोपमेके'' इति लोपेनैकसकारस्य द्वित्वेन द्विसकारस्य पुनर्द्वित्वेन च त्रिसकारस्य सिद्धौ ``समः सुटि'' इति सूत्रस्यैव मानत्वात्~, ``सम्प्रसारणाच्च'' ``सिचि वृद्धिः'' इत्यदौ भाष्ये स्पष्टमुक्तत्वाच्च~।
अत्र विकारकृतो लक्ष्यभेदो नेति ``सिचि वृद्धिः'' इतिभाष्यात्प्रतीयत इत्यन्यत्र विस्तरः~।\par
ननु स्यन्दूधातोः `स्यन्त्स्यति' इत्यादावात्मनेपदनिमित्तत्वाभावनिमित्तत्वात् ``न वृद्भ्यश्चतुर्भ्यः'' इति निषेधस्य बहिरङ्गत्वेनान्तरङ्गत्वादूदिल्लक्षणस्येड्विकल्पस्यापत्तिः, अत आह-

\section*{\begin{center}निषेधाश्च बलीयांसः~॥१२०॥\end{center}}
\addcontentsline{toc}{section}{निषेधाश्च बलीयांसः}
अन्तरङ्गादुपजीव्यादपि बलीयांस इत्यर्थः~।
``चतुर्भ्यः'' इति तु स्पष्टार्थमेव~।
अत एव तत्प्रत्याख्यानं भाष्योक्तं सङ्गच्छते~।
अत एव सवर्णसंज्ञादेर्निषेधविषये न विकल्पः~।
अन्यथा मीमांसकरीत्या विधेरुपजीव्यत्वेन प्राबल्यात्तस्य सर्वथा बाधानुपपत्त्या दुर्वारः स इति मञ्जूषायां विस्तरः~।
अत एव ``द्वन्द्वे च''~, ``विभाषा जसि'' इति चरितार्थम्~।
विध्युन्मूलनाय प्रवृत्तिरस्या बीजम्~।
"न लुमता''~, ``कमेर्णिङ्'' इत्यनयोर्भाष्ये स्पष्टैषा~॥\par
नन्वत्यन्तस्वार्थिकानामर्थप्रत्यायकत्वरूपप्रत्ययत्वानुपपत्तिः, अत आह-
\section*{\begin{center}अनिर्दिष्टार्थाः प्रत्ययाः स्वार्थे~॥१२१॥\end{center}}
\addcontentsline{toc}{section}{अनिर्दिष्टार्थाः प्रत्ययाः}
यस्यार्थः प्रकृत्या प्रत्याय्यते, सोऽपि प्रत्यय इत्यस्याप्यङ्गीकारात्तस्य प्रत्ययत्वमिति न दोषः~।
स्वार्थ इत्यस्य स्वीयप्रकृत्यर्थ इत्यर्थः~।
महासंज्ञाबलादर्थाकाङ्क्षायामन्यानुपस्थितिरस्या बीजम्~।
 ``सुपि स्थः'' इत्यादि सूत्रेषु भाष्ये स्पष्टैषा~॥
\section*{\begin{center}योगविभागादिष्टसिद्धिः॥१२२॥\end{center}}
\addcontentsline{toc}{section}{योगविभागादिष्टसिद्धिः}
इष्टसिद्धिरेव, न त्वनिष्टापादनं कार्यमित्यर्थः~।
तत्तत्समानविधिकद्वितीययोगेन विभक्तस्यानित्यत्वज्ञापनमेतद्बीजम्~॥
\section*{\begin{center}पर्यायशब्दानां लाघवगौरवचर्चा नाद्रियते~॥१२३॥\end{center}}
\addcontentsline{toc}{section}{पर्यायशब्दानां लाघव}
तत्र तत्रान्यतरस्याम् , विभाषा, वा इति सूत्रनिर्देशज्ञापितमिदं~॥
\section*{\begin{center}ज्ञापकसिद्धं न सर्वत्र~॥१२४॥\end{center}}
\addcontentsline{toc}{section}{ज्ञापकसिद्धं न सर्वत्र}
स्पष्टमेव पठितव्येऽनुमानाद्बोधनमसार्वत्रिकत्वार्थमित्यर्थः~।
तेन ज्ञापकसिद्धपरिभाषयानिष्टं नापादनीयमिति तात्पर्यम्~।
भाष्येऽपि ध्वनितमेतत् ङ्याप्सूत्रादौ~।
ज्ञापकेति न्यायस्याप्युपलक्षणम्~।
न्यायज्ञापकसिद्धानामपि केषाञ्चित्कथनमन्येषामनित्यत्वबोधनायेति भावः~।
यथा `तत्स्थानापन्ने तद्धर्मलाभः' इति न्यायसिद्धं स्थानिवत्सूत्रम् , ज्ञापकसिद्धं च तत्र ``अनल्विधौ'' इति~॥\par
ननु `द्रोग्धा द्रोग्धा, द्रोढा द्रोढा' इत्यादौ घत्वादीनामसिद्धत्वात्पूर्वं द्वित्वे एकत्र घत्वम्~, अपरत्र ढत्वमित्यस्याप्यापत्तिरत आह-

\section*{\begin{center}पूर्वत्रासिद्धीयमद्वित्वे~॥१२५॥\end{center}}
\addcontentsline{toc}{section}{पूर्वत्रासिद्धीयमद्वित्वे}
द्वित्वभिन्ने पूर्वत्र कर्तव्ये परमसिद्धमित्यर्थः~।
 ``पूर्वत्रासिद्धम्'' इत्यधिकारभवं शास्त्रमस्या लिङ्गम्~।
यत्र च सिद्धत्वासिद्धत्वयोः फले विशेषः, तत्रैवेयम्~।
`कृष्णर्द्धिः' इत्यादौ जश्त्वात्पूर्वमनन्तरं वा द्वित्वे रूपे विशेषाभावेन नास्याः प्रवृत्तिरित्यन्यत्र विस्तरः~।
 ``सर्वस्य द्वे'' इति सूत्रे भाष्ये स्पष्टेयम्~॥\par
ननु `गोष्वश्वेषु च स्वामी' इत्यादिवत् `गोष्वश्वानां च स्वामी' इत्यपि स्यात्~।
 ``स्वामीश्वर-'' इति सूत्रेण षष्ठीसप्तम्योर्विधानात्~, अत आह-
\section*{\begin{center}एकस्या आकृतेश्चरितः प्रयोगो द्वितीयस्यास्तृतीयस्याश्च न भविष्यति~॥१२६॥\end{center}}
\addcontentsline{toc}{section}{एकस्या आकृतेश्चरितः}
यत्रान्याकृतिकरणे भिन्नार्थत्वसम्भावना तद्विषयोऽयं न्याय इत्यन्यत्र विस्तरः~।
 ``कृञ्चानुप्रयुज्यते'' इति सूत्रे भाष्ये स्पष्टेयम्॥\par
ननु `विव्याध' इत्यादौ परत्वाद्धलादिःशेषे वस्य सम्प्रसारणं स्यात्~, अत आह
\section*{\begin{center}सम्प्रसारणं तदाश्रयं च कार्यं बलवत्~॥१२७॥\end{center}}
\addcontentsline{toc}{section}{सम्प्रसारणं तदाश्रयं}
तदाश्रयं ``सम्प्रसारणाच्च'' इति पूर्वरूपम्~।
वस्तुतः "लिट्यभ्यासस्य'' इति सूत्रे उभयेषां ग्रहणस्योभयेषां सम्प्रसारणमेव यथा स्यादित्यर्थकत्वेनेदं सिद्धमित्येषा व्यर्थेति ``लिट्यभ्यासस्य'' इति सूत्रे भाष्ये स्पष्टम्~।
फलान्तरान्यथासिद्धिरपि तत्रैव भाष्ये स्पष्टा~।
 ``णौ च संश्चङोः'' इत्यादौ संश्चङोः इत्यादि विषयसप्तमीति तत्रापि न दोष इत्यन्यत्र विस्तरः~॥\par
यत्तु-
\section*{\begin{center}क्वचिद्विकृतिः प्रकृतिं गृह्णाति~॥१२८॥\end{center}}
\addcontentsline{toc}{section}{क्वचिद्विकृतिः प्रकृतिं}
तेन ``निसमुपविभ्यो ह्वः'' इत्यत्र ह्वाग्रहणेन ह्वेञो ग्रहणसिद्धिः~॥
तथा-
\section*{\begin{center}औपदेशिकप्रायोगिकयोरौपदेशिकस्य ग्रहणम्~॥१२९॥\end{center}}
\addcontentsline{toc}{section}{औपदेशिकप्रायोगिकयोः}
तेन ``दादेर्धातोः'' इत्यत्रौपदेशिकधातोरेव ग्रहणमिति, तन्न~। तयोर्निर्मूलत्वात्~, भाष्याव्यवहृतत्वाच्च~।
{\bfseries न च विकृतिः प्रकृतिं गृह्णाति} इति ``ग्रहिज्या-'' इति सूत्रस्थभाष्येणाद्यायास्तिरस्काराच्च~।
 ``निसमुपविभ्यो ह्वः'' इत्यादौ ह्वेञोऽनुकरणे सौत्रः प्रयोगः~।
आत्वविषय एवात्मनेपदम्~, प्रयोगस्थानामेवानुकरणस्य घुसंज्ञासूत्रे भाष्ये स्पष्टमुक्तत्वादित्यन्ये~।
अन्त्यापि तत्र तत्रोपदेशग्रहणं कुर्वतः सूत्रकृतो वार्तिककृतश्चासम्मता~।
{\bfseries इह हि व्याकरणे सर्वेष्वेव सानुबन्धकग्रहणेषु रूपमाश्रीयते यत्रास्यैतद्रूपमिति, रूपनिर्ग्रहश्च शब्दस्य नान्तरेण लौकिकं प्रयोगम् , तस्मिंश्च लौकिके प्रयोगे सानुबन्धकानां प्रयोगो नास्तीति कृत्वा द्वितीयः प्रयोग उपास्यते, कः~? उपदेशो नाम} इति घुसंज्ञासूत्रस्थभाष्येण प्रायोगिकासम्भवे तद्ग्रहणमित्यर्थस्य लाभेन भाष्यासम्मता च~।
भाष्ये सानुबन्धकेत्यादि प्रकृताभिप्रयेण~।
 ``दादेः'' इति सूत्रे दादिपदस्यौपदेशिकदादित्ववति लक्षणेति न दोष इत्यन्यत्र विस्तरः~॥\par
यदपि ननु `अजर्घाः', `बेभिदीति' इत्यादौ तत्तद्गुणप्रयुक्ता विकरणा यङ्लुकि स्युः, तथा यङ्लुकि `बेभिदिता' इत्यादौ ``एकाचः'' इतीण्निषेधः स्यात्~, अत आह-
\section*{\begin{center}श्तिपा शपानुबन्धेन निर्दिष्टं यद्गणेन च~।\\
यत्रैकाज्ग्रहणं चैव पञ्चैतानि न यङ्लुकि~॥१३०॥\end{center}}
\addcontentsline{toc}{section}{श्तिपा शपानुबन्धेन}
अनुबन्धनिर्देशो द्विधा स्वरूपेण, `ङितः' इत्यादिपदेन च~।
हन्ति, याति, वाति, ``सनीवन्त-'' इति सूत्रे भरेति, ``दीङो युडचि''~, ``अनुदात्त ङितः''~, ``दिवादिभ्यः श्यन्''~, ``एकाच उपदेशे'' इत्युदाहरणानि~।
द्वित्वं ``सनाद्यन्ताः'' इति ``भूवादयः'' इति धातुत्वं च भवत्येव, ``गुणो यङ्लुकोः'' इत्यादिभिर्नुषेधानित्यत्वकल्पनात्~।
तेन भष्भावोऽपि `अजर्घाः' इत्यादौ भवति~।
अत एव ``श्वीदितः'' इति सूत्रे कैयटे `यत्रैकाज्ग्रहणं किञ्चित्' इति पाठः~।
 ``एकाच उपदेशेऽनुदात्तात्'' इति सूत्र एकाज्ग्रहणेनैकदेशानुमत्यैषा ज्ञाप्यते~।
अन्यथोपदेशेऽनेकाचामुदात्तत्वस्यैव सत्त्वेन तद्वैयर्थ्यं स्पष्टमेवेति~।
तदपि न, भाष्यानुक्तत्वात्~।
एकाज्ग्रहणस्य वधिव्यावृत्त्यर्थमावश्यकत्वाच्च~।
न च वधिः स्थान्युपदेश एकाजेवेति वाच्यम् , साक्षादुपदेशसम्भवेनैतद्विषये स्थान्युपदेशाग्रहणात्~, उपदेशत्वावच्छेदेनैकाजित्यर्थाच्च~।
किञ्च उत्तरार्थमेकाज्ग्रहणम्~।
अत एव `जागरितवान्' इत्यादावुपदेश उगन्तत्वमादाय ``श्र्युकः किति'' इतीण्निषेधो न~।
तत्रोपदेशे इत्यनुवृत्तिश्च `स्तीर्णम्' इत्यादाविण्निषेधायेत्याकरे स्पष्टम्~।
न च भाष्ये यङ्लोपो `बेभिदिता' इत्यादाविट्‍प्रवृत्त्यर्थमुपदेशेऽनुदात्तादेकाचः श्रूयमाणादङ्गादित्यर्थे सनीट्‍प्रतिषेधो वक्तव्यः, `बिभित्सति' इति दोषोपन्यासवद्यङ्लुकि दोषाणामुपन्यासेन तत्रेडिष्टः~।
यङ्लोपेत्यादिभाष्यं तूपक्रमोपसंहारबलेन न यङ्लुग्विषयम्~।
किञ्च, तस्य तद्विषयकत्वे यङ्लोपे स्थानिवत्त्वस्येव यङ्लुक्युपायाप्रदर्शनेन न्यूनतापत्तिरिति वाच्यम्~, इड्विषये यङ्लुको लोकेऽनभिधानेन छन्दसि सर्वविधीनां वैकल्पिकत्वेन च तत्र दोषानुपन्यासेनादोषात्~।
अन्यथा {\bfseries एकाज्ग्रहणं किमर्थम्} इति प्रश्नस्य {\bfseries उत्तरत्र जागर्त्यर्थमिह वध्यर्थम्} इत्युत्तरस्य च भाष्ये निरालम्भनत्वापत्तेः~।
न चार्धधातुकाक्षिप्तधातोरेकाच इति विशेषणम्~।
एवं च `बिभित्सति' इत्यादावुत्तरखण्डस्य धातोरेकाच्त्वमस्त्येव, उत्तरखण्डेऽस्तित्ववत्~।
एतच्च ``दयतेः'' इति सूत्रे भाष्ये स्पष्टम्~।
एवं च प्रकृतभाष्यासङ्गतिरिति वाच्यम्~।
आक्षेपे आक्षिप्तस्यान्वये च मानाभावात्~।
अङ्गत्वं तु विशिष्ट एवेति ``एकाचो द्वे'' इति सूत्रे भाष्ये स्पष्टम्~।
निरूपितं च तनादिशेषे शब्देन्दुशेखरे~।
धातुत्वं तूत्तरखण्ड एव~।
अत एव ``एकाचो बशो भष्'' इति सूत्रे धातोरवयवस्यैकाच इति वैयधिकरण्येनान्वये गर्धप्सिद्धिः प्रयोजनमुक्तं भाष्ये, न तु प्रसिद्धम् `अजर्घाः' इति~।
`अजर्घाः'~, `बेभिदीति' इत्यादौ श्नं श्यनादयस्तु ``चर्करीतं च'' इत्यस्यादादौ पाठेन यङ्लुगन्ते गणान्तरप्रयुक्तविकरणस्याप्राप्त्या न भवन्ति~।
छान्दसत्वादेव कार्यान्तराणामपि छन्दसि दृष्टप्रयोगेष्वदृष्टानामभावो बोध्यः~।
भाषायां तु तादृशानामभाव एव~।
श्तिप्शबादिनिर्देशास्तु ``भवतेरः'' इत्यादि सूत्रस्थतन्निर्देशवन्नार्थसाधका इत्यन्यत्र विस्तरः~॥\par
ननु `जभोऽचि, रधेश्च, नेट्यलिटि' इत्येव सूत्र्यताम्~, किं द्वी रधिग्रहणेनेत्याह-
\section*{\begin{center}पदगौरवाद्योगविभागो गरीयान्~॥१३१॥\end{center}}
\addcontentsline{toc}{section}{पदगौरवाद्योगविभागः}
प्रतिवाक्यं भिन्नवाक्यार्थबोधकल्पनेन गौरवं स्पष्टमेव~।
परन्तु भाष्यासम्मतेयम्~।
 ``टाङसि-'' इति सूत्रस्थभाष्यविरुद्धा च~।
तत्र च इनादेशे इकारप्रत्याख्यानं योगविभागेनैव कृतमिति बहवः~॥
\section*{\begin{center}अर्धमात्रालाघवेन पुत्रोत्सवं मन्यन्ते वैयाकरणाः~॥१३२॥\end{center}}
\addcontentsline{toc}{section}{अर्धमात्रालाघवेन}
 ``एओङ्''~, ``ऐऔच्'' सूत्रयोर्ध्वनितैषा भाष्ये~।
तत्रानेकपदघटितसूत्रे प्रायः पदलाघवविचार एव, न तु मात्रालाघवविचार इति ``ऊकालोऽच्'', ``अपृक्त एकाल्'' इत्यादिसूत्रेषु भाष्ये ध्वनितम्~।
तत्र हि सूत्रे अल्ग्रहणहल्ग्रहणयोर्विशेषविचारे संज्ञायां हल्ग्रहणं ``ण्यक्षत्रिय-'' इति सूत्रे अणिञोरिति वाच्यमिति त्रीणि पदान्यल्ग्रहणे, तदेकं स्वादिलोपे हल्ग्रहणं ``ण्य-'' इति सूत्रे अणिञोरिति न वाच्यम्~, अपृक्तस्येति वाच्यमिति त्रीण्येव पदानीति नास्ति लाघवकृतो विशेष इत्युक्तम्~।
 ``अचि श्नु-'' इति सूत्रे इण इत्येव सिद्धे य्वोरिति संमृद्य ग्रहणान्न पूर्वेणेण्ग्रहणम्~।
तत्र विभक्तिनिर्देशे संमृद्य ग्रहणे च सार्धास्त्रिस्रो मात्रा इण्ग्रहणे तिस्रो मात्रा इति लण्सूत्रे भाष्योक्तेः, तथा ``ओतः श्यनि'' इति सूत्रे शितीति न वक्तव्यम्~।
तत्रायमर्थः ``ष्ठिवुक्लमु-'' इति सूत्रेऽशितीति न कर्त्तव्यं भवति इति भाष्ये न केवलं मात्रालाघवं यावदयमप्यर्थ इति कैयटोक्तेः प्रायेणेति शिवम्~॥
\begin{center}॥इति शास्त्रशेषनामकं तृतीयं प्रकरणम्॥\end{center}
\begin{center}{\bfseries॥इति श्रीमन्महोपाध्यायशिवभट्टसुतसतीगर्भजनागोजीभट्टकृतः परिभाषेन्दुशेखरः समाप्तः॥}\end{center}
\begin{figure}[b]
\centering
\includegraphics[width=4cm]{page-divider}
\end{figure}
\section*{\begin{center}परिभाषापाठः\end{center}}
\addcontentsline{toc}{section}{परिभाषापाठः}
\fancyhead[LE,RO]{परिभाषापाठः}
व्याख्यानतो विशेषप्रतिपत्तिर्न हि सन्देहादलक्षणम्॥१॥
 यथोद्देशं संज्ञापरिभाषम्~। अनेकान्ताः अनुबन्धाः~॥३॥
एकान्ताः॥४॥
नानुबन्धकृतमनेकाल्त्वं~॥५॥
नानुबन्धकृतमनेजन्तत्वम्~॥६॥
नानुबन्धकृतमसारूप्यम्~॥७॥
उभयगतिरिह भवति~॥८॥
कार्यमनुभवन्हि कार्यी निमित्ततया नाश्रीयते॥९॥
यदागमास्तद्गुणीभूतास्तद्ग्रहणेन गृह्यन्ते~॥१०॥
निर्दिश्यमानस्याऽदेशा भवन्ति॥११॥
यत्रानेकविधमान्तर्यं तत्र स्थानत आन्तर्यं बलीयः॥१२॥
अर्थवद्ग्रहणे नानर्थकस्य~॥१३॥
गौणमुख्ययोर्मुख्ये कार्यसम्प्रत्ययः~॥१४॥
अनिनस्मन्ग्रहणान्यर्थवता चानर्थकेन च तदन्तविधिं प्रयोजयन्ति~॥१५॥
एकयोगनिर्दिष्टानां सह वा प्रवृत्तिः सह वा निवृत्तिः~॥१६॥
क्वचिदेकदेशोऽप्यनुवर्तते~॥१७॥
भाव्यमानेन सवर्णानां ग्रहणं न॥१८॥
भाव्यमानोऽप्युकारः सवर्णान्गृह्णाति॥१९॥
वर्णाश्रये नास्ति प्रत्ययलक्षणम्॥२०॥
उणादयोऽव्युत्पन्नानि प्रातिपदिकानि॥२१॥
प्रत्ययग्रहणे यस्मात्स विहितस्तदादेस्तदन्तस्य ग्रहणम्॥२२॥
प्रत्ययग्रहणे चापञ्चम्याः॥२३॥
उत्तरपदाधिकारे प्रत्ययग्रहणे न तदन्तग्रहणम्॥२४॥
स्त्रीप्रत्यये चानुपसर्जने न॥२५॥
संज्ञाविधौ प्रत्ययग्रहणे तदन्तग्रहणं नास्ति॥२६॥
कृद्ग्रहणे गतिकारकपूर्वस्यापि ग्रहणम्॥२७॥
पदाङ्गाधिकारे तस्य च तदन्तस्य च॥२८॥
व्यपदेशिवदेकस्मिन्॥२९॥
ग्रहणवता प्रातिपदिकेन तदन्तविधिर्नास्ति॥३०॥
व्यपदेशिवद्भावोऽप्रातिपदिकेन॥३१॥
यस्मिन्विधिस्तदादावल्ग्रहणे॥३२॥
सर्वो द्वन्द्वो विभाषयैकवद्भवति॥३३॥
सर्वे विधयश्छन्दसि विकल्प्यन्ते॥३४॥
प्रकृतिवदनुकरणं भवति॥३५॥
एकदेशविकृतमनन्यवत्॥३६॥
पूर्वपरनित्यान्तरङ्गापवादानामुत्तरोत्तरं बलीयः॥३७॥
पुनः प्रसङ्गविज्ञानात्सिद्धं॥३८॥
सकृद्गतौ विप्रतिषेधे यद्बाधितं तद्बाधितमेव॥३९॥
विकरणेभ्यो नियमो बलीयान्॥४०॥
परान्नित्यं बलवत्॥४१॥
कृताकृतप्रसङ्गि नित्यम्~, तद्विपरीतमनित्यं~॥४२॥
शब्दान्तरस्य प्राप्नुवन्विधिरनित्यो भवति~॥४३॥
शब्दान्तरात्प्राप्नुवतः, शब्दान्तरे प्राप्नुवतश्चानित्यत्वम्~॥४४॥
लक्षणान्तरेण प्राप्नुवन्विधिरनित्यः~॥४५॥
क्वचित्कृताकृतप्रसङ्गमात्रेणापि नित्यता~॥४६॥
यस्य च लक्षणान्तरेण निमित्तं विहन्यते न तदनित्यम्~॥४७॥
यस्य च लक्षणान्तरेण निमित्तं विहन्यते तदप्यनित्यम्~॥४८॥
स्वरिभन्नस्य प्राप्नुवन्विधिरनित्यो भवति~॥४९॥
असिद्धं बहिरङ्गमन्तरङ्गे~॥५०॥
नाजानन्तर्ये बहिष्ट्वप्रकॢप्तिः॥५१॥
अन्तरङ्गानपि विधीन्बहिरङ्गो लुग्बाधते~॥५२॥
पूर्वोत्तरपदनिमित्तकार्यात्पूर्वमन्तरङ्गेऽप्येकादेशो न~॥५३॥
अन्तरङ्गानपि विधीन्बहिरङ्गो ल्यब्बाधते~॥५४॥
वार्णादाङ्गं बलीयो भवति~॥५५॥
 तस्या असत्वात्~।
  अकृतव्यूहाः पाणिनीयाः~॥५६॥
येन नाप्राप्ते यो विधिरारभ्यते स तस्य बाधको भवति~॥५७॥
क्वचिदपवादविषयेऽप्युत्सर्गोऽभिनिविशत इति~॥५८॥
पुरस्तादपवादाः अनन्तरान्विधीन्बाधन्ते, नोत्तरान्~॥५९॥
मध्येऽपवादाः पूर्वान्विधीन्बाधन्ते, नोत्तरान्~॥६०॥
अनन्तरस्य विधिर्वा भवति प्रतिषेधो वेति~॥६१॥
पूर्वं ह्यपवादा अभिनिविशन्ते पश्चादुत्सर्गाः~॥६२॥
प्रकल्प्य चापवादविषयं तत उत्सर्गोऽभिनिविशते~॥६३॥
उपसञ्जनिष्यमाणनिमित्तोऽप्यपवाद उपसञ्जातनिमित्तमप्युत्सर्गं बाधत इति~॥६४॥
अपवादो यद्यन्यत्र चरितार्थस्तर्ह्यन्तरङ्गेण बाध्यते~॥६५॥
अभ्यासविकारेषु बाध्यबाधकभावो नास्ति~॥६६॥
ताच्छीलिकेषु वासरूपविधिर्नास्ति~॥६७॥
क्तल्युट्‍तुमुन्खलर्थेषु वासरूपविधिर्नास्ति॥६८॥
लादेशेषु वासरूपविधिर्नास्ति~॥६९॥
उभयनिर्देशे पञ्चमीनिर्देशो बलीयान्~॥७०॥
प्रातिपदिकग्रहणे लिङ्गविशिष्टस्यापि ग्रहणम्॥७१॥
विभक्तौ लिङ्गविशिष्टाग्रहणम्॥७२॥
सूत्रे लिङ्गवचनमतन्त्रम्॥७३॥
नञिव युक्तमन्यसदृशाधिकरणे तथा ह्यर्थगतिः॥७४॥
गतिकारकोपपदानां कृद्भिः सह समासवचनं प्राक्सुबुत्पत्तेः॥७५॥
साम्प्रतिकाभावे भूतपूर्वगतिः॥७६॥
बहुव्रीहौ तद्गुणसंविज्ञानमपि॥७७॥
चानुकृष्टं नोत्तरत्र॥७८॥
स्वरविधौ व्यञ्जनमविद्यमानवत्॥७९॥
हल्स्वरप्राप्तौ व्यञ्जनमविद्यमानवत्॥८०॥
निरनुबन्धकग्रहणे न सानुबन्धकस्य॥८१॥
तदनुबन्धकग्रहणे नातदनुबन्धकस्य॥८२॥
क्वचित्स्वार्थिकाः प्रकृतितो लिङ्गवचनान्यतिवर्तन्ते॥८३॥
समासान्तविधिरनित्यः॥८४॥
सन्निपातलक्षणो विधिरनिमित्तं तद्विघातस्य॥८५॥
सन्नियोगशिष्टानामन्यतरापाय उभयोरप्यपायः॥८६॥
ताच्छीलिके णेऽण्कृतानि भवन्ति॥८७॥
धातोः कार्यमुच्यमानं तत्प्रत्यये भवति॥८८॥
तन्मध्यपतितस्तद्ग्रहणेन गृह्यते॥८९॥
लुग्विकरणालुग्विकरणयोरलुग्विकरणस्य॥९०॥
प्रकृतिग्रहणे ण्यधिकस्यापि ग्रहणम् ॥९१॥
अङ्गवृत्ते पुनर्वृत्तावविधिः ॥९२॥
संज्ञापूर्वकविधेरनित्यत्वम् ॥९३॥
आगमशास्त्रमनित्यम् ॥९४॥
गणकार्यमनित्यम् ॥९५॥
अनुदात्तेत्त्वलक्षणमात्मनेपदमनित्यम् ॥९६॥
नञ्घटितमनित्यम् ॥९७॥
आतिदेशिकमनित्यम् ॥९८॥
सर्वविधिभ्यो लोपविधिरिड्विधिश्च बलवान् ॥९९॥
प्रकृतिग्रहणे यङ्लुगन्तस्यापि ग्रहणम् ॥१००॥
विधौ परिभाषोपतिष्ठते नानुवादे ॥१०१॥
उपपदविभक्तेः कारकविभक्तिर्बलीयसी ॥१०२॥
अनन्त्यविकारेऽन्त्यसदेशस्य ॥१०३॥
"अल्लोपोऽनः'' इत्यादेः
नानर्थकेऽलोऽन्त्यविधिरनभ्यासविकारे ॥१०४॥
प्रधानाप्रधानयोः प्रधाने कार्यसम्प्रत्ययः ॥१०५॥
अवयवप्रसिद्धेः समुदायप्रसिद्धिर्बलीयसी~॥१०६॥
व्यवस्थितविभाषयापि कार्याणि क्रियन्ते~॥१०७॥
विधिनियमसम्भवे विधिरेव ज्यायान्~॥१०८॥
सामान्यातिदेशे विशेषानतिदेशः~॥१०९॥
प्रत्ययाप्रत्यययोः प्रत्ययस्य ग्रहणम्~॥११०॥
सहचरितासहचरितयोः सहचरितस्यैव ग्रहणम्~॥१११॥
श्रुतानुमितयोः श्रुतसम्बन्धो बलवान्~॥११२॥
लक्षणप्रतिपदोक्तयोः प्रतिपदोक्तस्यैव ग्रहणम्~॥११३॥
गामादाग्रहणेष्वविशेषः~॥११४॥
प्रत्येकं वाक्यपरिसमाप्तिः~॥११५॥
क्वचित्समुदायेऽपि~॥११६॥
अभेदकाः गुणाः~॥११७॥
बाधकान्येव निपातनानि~॥११८॥
पर्जन्यवल्लक्षणप्रवृत्तिः~॥११९॥
निषेधाश्च बलीयांसः~॥१२०॥
अनिर्दिष्टार्थाः प्रत्ययाः स्वार्थे~॥१२१॥
योगविभागादिष्टसिद्धिः॥१२२॥
पर्यायशब्दानां लाघवगौरवचर्चा नाद्रियते~॥१२३॥
ज्ञापकसिद्धं न सर्वत्र~॥१२४॥
पूर्वत्रासिद्धीयमद्वित्वे~॥१२५॥
एकस्या आकृतेश्चरितः प्रयोगो द्वितीयस्यास्तृतीयस्याश्च न भविष्यति~॥१२६॥
सम्प्रसारणं तदाश्रयं च कार्यं बलवत्~॥१२७॥
क्वचिद्विकृतिः प्रकृतिं गृह्णाति~॥१२८॥
औपदेशिकप्रायोगिकयोरौपदेशिकस्य ग्रहणम्~॥१२९॥
श्तिपा शपानुबन्धेन निर्दिष्टं यद्गणेन च~।
यत्रैकाज्ग्रहणं चैव पञ्चैतानि न यङ्लुकि~॥१३०॥
पदगौरवाद्योगविभागो गरीयान्~॥१३१॥
अर्धमात्रालाघवेन पुत्रोत्सवं मन्यन्ते वैयाकरणाः~॥१३२॥

