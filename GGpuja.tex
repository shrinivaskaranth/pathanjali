\section{ಗುರುಗಣಪತಿಪೂಜಾ}
\subsection{ಧ್ಯಾನಂ}


ಶ್ರೀನಾಥಾದಿ ಗುರುತ್ರಯಂ ಗಣಪತಿಂ ಪೀಠತ್ರಯಂ ಭೈರವಂ\\
ಸಿದ್ಧೌಘಂ ಬಟುಕತ್ರಯಂ ಪದಯುಗಂ ದೂತೀಕ್ರಮಂ ಮಂಡಲಂ~।\\
ವೀರದ್ವ್ಯಷ್ಟ ಚತುಷ್ಕಷಷ್ಟಿನವಕಂ ವೀರಾವಲೀಪಂಚಕಂ\\
ಶ್ರೀಮನ್ಮಾಲಿನಿಮಂತ್ರರಾಜಸಹಿತಂ ವಂದೇ ಗುರೋರ್ಮಂಡಲಂ॥

ಸಚ್ಚಿದಾನಂದ ರೂಪಾಯ ಬಿಂದು ನಾದಾಂತರಾತ್ಮನೇ~।\\
ಆದಿಮಧ್ಯಾಂತ ಶೂನ್ಯಾಯ ಗುರೂಣಾಂ ಗುರವೇ ನಮಃ ॥

ಶುದ್ಧಸ್ಫಟಿಕಸಂಕಾಶಂ ದ್ವಿನೇತ್ರಂ ಕರುಣಾನಿಧಿಂ~।\\
ವರಾಭಯಪ್ರದಂ ವಂದೇ ಶ್ರೀಗುರುಂ ಶಿವರೂಪಿಣಂ ॥

ಗುರುರ್ಬ್ರಹ್ಮಾ ಗ್ರುರುರ್ವಿಷ್ಣುಃ ಗುರುರ್ದೇವೋ ಮಹೇಶ್ವರಃ~।\\
ಗುರುಃ ಸಾಕ್ಷಾತ್ ಪರಂ ಬ್ರಹ್ಮ ತಸ್ಮೈ ಶ್ರೀ ಗುರವೇ ನಮಃ ॥


ಓಂ ಐಂಹ್ರೀಂಶ್ರೀಂ ಐಂಕ್ಲೀಂಸೌಃ ಹ್‌ಸ್‌ಖ್‌ಫ್ರೇಂ ಹಸಕ್ಷಮಲವರಯೂಂ ಹ್ಸೌಃ ಸಹಕ್ಷಮಲವರಯೀಂ ಸ್ಹೌಃ ॥  ಶ್ರೀ ಗುರವೇ ನಮಃ ॥

ಹಂಸ ಗಾಯತ್ರೀ ಮಹಾಮಂತ್ರಸ್ಯ ಸದಾಶಿವ ಋಷಿಃ । ಗಾಯತ್ರೀ ಚ್ಛಂದಃ । ಶ್ರೀಸದ್ಗುರುರ್ದೇವತಾ । ಹಂಸಾಂ ಬೀಜಂ । ಹಂಸೀಂ ಶಕ್ತಿಃ । ಹಂಸೂಂ ಕೀಲಕಮ್ ।

ಹ್‌ಸಾಂ, ಹ್‌ಸೀಂ ಇತ್ಯಾದಿನಾ ಷಡಂಗನ್ಯಾಸಂ ವಿಧಾಯ

ಸಹಸ್ರದಲಪಂಕಜೇ ಸಕಲಶೀತರಶ್ಮಿಪ್ರಭಂ\\
ವರಾಭಯಕರಾಂಬುಜಂ ವಿಮಲಗಂಧಪುಷ್ಪಾಂಬರಂ~।\\
ಪ್ರಸನ್ನವದನೇಕ್ಷಣಂ ಸಕಲದೇವತಾರೂಪಿಣಂ\\
ಸ್ಮರೇಚ್ಛಿರಸಿ ಹಂಸಗಂ ತದಭಿಧಾನಪೂರ್ವಂ ಗುರುಂ ॥


ಮೌನವ್ಯಾಖ್ಯಾ ಪ್ರಕಟಿತ ಪರಬ್ರಹ್ಮತತ್ವಂ ಯುವಾನಂ\\
ವರ್ಷಿಷ್ಠಾಂತೇವಸದೃಷಿಗಣೈರಾವೃತಂ ಬ್ರಹ್ಮನಿಷ್ಠೈಃ ।\\
ಆಚಾರ್ಯೇಂದ್ರಂ ಕರಕಲಿತಚಿನ್ಮುದ್ರಮಾನಂದ ಮೂರ್ತಿಂ\\
ಸ್ವಾತ್ಮಾರಾಮಂ ಮುದಿವದನಂ ದಕ್ಷಿಣಾಮೂರ್ತಿಮೀಡೇ ॥


\as{ಓಂಐಂಹ್ರೀಂಶ್ರೀಂ ಶ್ರೀನಂದನಾಥಾಯ ಮಮ ಗುರವೇ ನಮಃ ।\\
ಓಂಐಂಹ್ರೀಂಶ್ರೀಂ ಶ್ರೀನಂದನಾಥಾಯ ಮಮ ಪರಮಗುರವೇ ನಮಃ ।\\
ಓಂಐಂಹ್ರೀಂಶ್ರೀಂ ಶ್ರೀನಂದನಾಥಾಯ ಮಮ ಪರಮೇಷ್ಠಿ ಗುರವೇ ನಮಃ ।\\}


ಬೀಜಾಪೂರಗದೇಕ್ಷುಕಾರ್ಮುಕಲಸಚ್ಚಕ್ರಾಬ್ಜಪಾಶೋತ್ಪಲ\\ವ್ರೀಹ್ಯಗ್ರಸ್ವವಿಷಾಣರತ್ನಕಲಶಪ್ರೋದ್ಯತ್ಕರಾಂಭೋರುಹಃ ।\\
ಧ್ಯ್ಯೇಯೋ ವಲ್ಲಭಯಾ ಸಪದ್ಮಕರಯಾ ಶ್ಲಿಷ್ಟೋ ಜ್ವಲದ್ಭೂಷಯಾ\\ ವಿಶ್ವೋತ್ಪತ್ತಿವಿನಾಶ ಸಂಸ್ಥಿತಿಕರೋ ವಿಘ್ನೋ ವಿಶಿಷ್ಟಾರ್ಥದಃ ॥
\subsection{ಆವಾಹನಮ್}
ಗುರುರ್ಬ್ರಹ್ಮಾ ಗ್ರುರುರ್ವಿಷ್ಣುಃ ಗುರುರ್ದೇವೋ ಮಹೇಶ್ವರಃ~।\\
ಗುರುಃ ಸಾಕ್ಷಾತ್ ಪರಂ ಬ್ರಹ್ಮ ತಸ್ಮೈ ಶ್ರೀ ಗುರವೇ ನಮಃ ॥
\as{ಸಹಸ್ರಶೀರ್ಷಾ ಪುರುಷಃ+++++ದಶಾಂಗುಲಂ॥\\
ಹಿರಣ್ಯವರ್ಣಾಂ ++++++++++ ಮ ಆವಹ ॥} ಆವಾಹನಮ್ ॥

\as{ಪುರುಷ ಏವೇದಂ+++++ ನಾ ತಿರೋಹತಿ॥\\
ತಾಂ ಮ+++++++++++ ಪುರುಷಾನಹಮ್॥}ಆಸನಮ್ ॥

\as{ಏತಾವಾನಸ್ಯ+++++ ತ್ರಿಪಾದ+++++ಸ್ಯಾಮೃತಂ ದಿವಿ॥\\
ಅಶ್ವಪೂರ್ವಾಂ +++++++++ಜುಷತಾಮ್॥}ಪಾದ್ಯಮ್ ॥

\as{ತ್ರಿಪಾದೂರ್ದ್ಧ್ವ+++++-ಶನೇ ಅಭಿ ॥\\
ಕಾಂ ಸೋಸ್ಮಿತಾಂ +++++++++ ಶ್ರಿಯಮ್ ॥} ಅರ್ಘ್ಯಮ್ ॥

\as{ತಸ್ಮಾತ್ ವಿರಾಳ+++++ಪಶ್ಚಾತ್ ಭೂಮಿಮಥೋ ಪುರಃ ॥\\
ಚನ್ದ್ರಾಂ ಪ್ರಭಾಸಾಂ ++++++++ತ್ವಾಂ ವೃಣೇ ॥} ಆಚಮನಮ್ ॥

\as{ಮಧುವಾತಾ+++++ತ್ವೋಷಧೀಃ ॥} ಮಧುಪರ್ಕಃ ॥

\as{ಆಪೋ ಹಿಷ್ಠಾ++++ಚ ನಃ ॥} ಮಲಾಪಕರ್ಷಣಂ ॥

ಕ್ಷೀರಸ್ನಾನಮ್ ॥\\
\as{ಆಪ್ಯಾಯಸ್ವ++++ಸಂಗಥೇ ॥\\
ಸದ್ಯೋಜಾತಂ+++ ನಮಃ ॥}


ದಧಿಸ್ನಾನಮ್ ॥\\
\as{ದಧಿಕ್ರಾವ್ಣೋ+++++ ತಾರಿಷತ್ ॥\\
ವಾಮದೇವಾಯ ನಮೋ++++ ನಮಃ ॥}

ಘೃತಸ್ನಾನಮ್ ॥\\
\as{ಘೃತಂ ಮಿಮಿಕ್ಷೇ +++ಹವ್ಯಮ್ ॥\\
ಅಘೋರೇಭ್ಯೋಥ+++ರೂಪೇಭ್ಯಃ ॥}

ಮಧುಸ್ನಾನಮ್ ॥\\
\as{ಮಧುವಾತಾ ಋತಾಯತೇ ++++ ಸಂತ್ವೋಷಧೀಃ ॥\\
ತತ್ಪುರುಷಾಯ ವಿದ್ಮಹೇ++++ಪ್ರಚೋದಯಾತ್ ॥}

ಶರ್ಕರಾಸ್ನಾನಮ್ ॥\\
\as{ಸ್ವಾದುಃ ಪವಸ್ವ++++ಅದಾಭ್ಯಃ ॥\\
ಈಶಾನಃ ಸರ್ವವಿದ್ಯಾನಾಂ+++ ಸದಾಶಿವೋಮ್ ॥}

 ಫಲಸ್ನಾನಮ್॥\\
\as{ಯಾಃ ಫಲಿನೀ+++++ತ್ವಂ ಹಸಃ ॥\\
ಕದ್ರುದ್ರಾಯ +++++ಹೃದೇ ॥}

\as{ಗಂಧದ್ವಾರಾಂ++++ಶ್ರಿಯಮ್ ॥} ಗಂಧೋದಕಸ್ನಾನಮ್ ॥

\as{ಅಂಬಿತಮೇ ನದೀತಮೇ+++++ನಸ್ಕೃಧಿ ॥} ಹರಿದ್ರೋದಕಸ್ನಾನಮ್ ॥

\as{ಪ್ರಣೋ ದೇವೀ++++++ ಮವಿತ್ರ್ಯವತು ॥}ಕುಂಕುಮೋದಕಸ್ನಾನಮ್ ॥

\as{ಆಯನೇತೇ ++++++ ಇಮೇ ॥}ಪುಷ್ಪೋದಕಸ್ನಾನಮ್ ॥

\as{ಹಿರಣ್ಯರೂಪಃ++++ತ್ಯನ್ನಮಸ್ಮೈ ॥}ಸುವರ್ಣೋದಕಸ್ನಾನಮ್॥

\as{ಉಪಾಸ್ಮೈ ಗಾಯತಾ++++ಇಯಕ್ಷತೇ ॥} ಅಕ್ಷತೋದಕಸ್ನಾನಮ್ ॥

\section{ಭಾವನೋಪನಿಷತ್}
ಓಂ ಭದ್ರಂ ಕರ್ಣೇಭಿಃ ಇತಿ ಶಾಂತಿಃ ॥
ಓಂ ಶ್ರೀಗುರುಃ ಸರ್ವಕಾರಣಭೂತಾ ಶಕ್ತಿಃ~। ತೇನ ನವರಂಧ್ರರೂಪೋ ದೇಹಃ~। ನವಚಕ್ರರೂಪಂ ಶ್ರೀಚಕ್ರಂ~। ವಾರಾಹೀ ಪಿತೃರೂಪಾ~। ಕುರುಕುಲ್ಲಾ ಬಲಿದೇವತಾ ಮಾತಾ~। ಪುರುಷಾರ್ಥಾಃ ಸಾಗರಾಃ~। ದೇಹೋ ನವರತ್ನದ್ವೀಪಃ~। ತ್ವಗಾದಿಸಪ್ತಧಾತುರೋಮ ಸಂಯುಕ್ತಃ। ಸಂಕಲ್ಪಾಃ ಕಲ್ಪತರವಸ್ತೇಜಃ ಕಲ್ಪಕೋ\-ದ್ಯಾನಂ~। ರಸನಯಾ ಭಾವ್ಯಮಾನಾ ಮಧುರಾಮ್ಲತಿಕ್ತಕಟುಕಷಾಯಲವಣರಸಾಃ ಷಡೃತವಃ~। ಜ್ಞಾನಮರ್ಘ್ಯಂ। ಜ್ಞೇಯಂ ಹವಿಃ~। ಜ್ಞಾತಾ ಹೋತಾ~। ಜ್ಞಾತೃಜ್ಞಾನಜ್ಞೇಯಾನಾಮಭೇದಭಾವನಂ ಶ್ರೀಚಕ್ರಪೂಜನಂ~। ನಿಯತಿಃ ಶೃಂಗಾರಾದಯೋ ರಸಾ ಅಣಿಮಾದಯಃ~। ಕಾಮಕ್ರೋಧ ಲೋಭಮೋಹ ಮದ ಮಾತ್ಸರ್ಯ ಪುಣ್ಯ ಪಾಪಮಯಾ ಬ್ರಾಹ್ಮ್ಯಾದ್ಯಷ್ಟಶಕ್ತಯಃ~। ಆಧಾರನವಕಂ ಮುದ್ರಾಶಕ್ತಯಃ~। ಪೃಥಿವ್ಯಪ್ತೇಜೋವಾಯ್ವಾಕಾಶ ಶ್ರೋತ್ರತ್ವಕ್ಚಕ್ಷುರ್ಜಿಹ್ವಾ ಘ್ರಾಣ ವಾಕ್ಪಾಣಿ ಪಾದ ಪಾಯೂಪಸ್ಥಾನಿ ಮನೋವಿಕಾರಾಃ ಕಾಮಾಕರ್ಷಣ್ಯಾದಿ ಷೋಡಶ ಶಕ್ತಯಃ~। ವಚನಾದಾನಗಮನವಿಸರ್ಗಾನಂದ ಹಾನೋಪಾದಾನೋಪೇಕ್ಷಾಖ್ಯ ಬುದ್ಧಯೋನಂಗಕುಸುಮಾದ್ಯಷ್ಟೌ~। ಅಲಂಬುಸಾ ಕುಹೂರ್ವಿಶ್ವೋದರಾ ವಾರಣಾ ಹಸ್ತಿಜಿಹ್ವಾ ಯಶೋವತೀ ಪಯಸ್ವಿನೀ ಗಾಂಧಾರೀ ಪೂಷಾ ಶಂಖಿನೀ ಸರಸ್ವತೀ ಇಡಾ ಪಿಂಗಲಾ ಸುಷುಮ್ನಾ ಚೇತಿ ಚತುರ್ದಶ ನಾಡ್ಯಃ ಸರ್ವಸಂಕ್ಷೋಭಿಣ್ಯಾದಿ ಚತುರ್ದಶ ಶಕ್ತಯಃ~। ಪ್ರಾಣಾಪಾನ ವ್ಯಾನೋದಾನ ಸಮಾನ ನಾಗ ಕೂರ್ಮ ಕೃಕರ ದೇವದತ್ತ ಧನಂಜಯಾ ಇತಿ ದಶ ವಾಯವಃ ಸರ್ವಸಿದ್ಧಿಪ್ರದಾದಿ ಬಹಿರ್ದಶಾರಗಾ ದೇವತಾಃ~। ಏತದ್ವಾಯುಸಂಸರ್ಗ ಕೋಪಾಧಿ ಭೇಧೇನ ರೇಚಕಃ ಪಾಚಕಃ ಶೋಷಕೋ ದಾಹಕಃ ಪ್ಲಾವಕ ಇತಿ ಪ್ರಾಣಮುಖ್ಯಶ್ರೀತ್ವೇನ ಪಂಚಧಾ ಜಠರಾಗ್ನಿರ್ಭವತಿ~। ಕ್ಷಾರಕ ಉದ್ಗಾರಕಃ ಕ್ಷೋಭಕೋ ಜೃಂಭಕೋ ಮೋಹಕ ಇತಿ ನಾಗಪ್ರಾಧಾನ್ಯೇನ ಪಂಚವಿಧಾಸ್ತೇ ಮನುಷ್ಯಾಣಾಂ ದೇಹಗಾ ಭಕ್ಷ್ಯಭೋಜ್ಯ ಚೋಷ್ಯ ಲೇಹ್ಯ ಪೇಯಾತ್ಮಕ ಪಂಚವಿಧಮನ್ನಂ ಪಾಚಯಂತಿ। ಏತಾ ದಶ ವಹ್ನಿಕಲಾಃ ಸರ್ವಜ್ಞಾದ್ಯಾ ಅಂತರ್ದಶಾರಗಾ ದೇವತಾಃ~। ಶೀತೋಷ್ಣಸುಖದುಃಖೇಚ್ಛಾಃ ಸತ್ತ್ವಂ ರಜಸ್ತಮೋ ವಶಿನ್ಯಾದಿಶಕ್ತಯೋಽಷ್ಟೌ~। ಶಬ್ದಾದಿ ತನ್ಮಾತ್ರಾಃ ಪಂಚಪುಷ್ಪಬಾಣಾಃ। ಮನ ಇಕ್ಷುಧನುಃ~। ರಾಗಃ ಪಾಶಃ~। ದ್ವೇಷೋಂಕುಶಃ~। ಅವ್ಯಕ್ತಮಹದಹಂಕಾರಾಃ ಕಾಮೇಶ್ವರೀವಜ್ರೇಶ್ವರೀ ಭಗಮಾಲಿನ್ಯೋಂತಸ್ತ್ರಿಕೋಣಗಾ ದೇವತಾಃ~। ನಿರುಪಾಧಿಕ ಸಂವಿದೇವ ಕಾಮೇಶ್ವರಃ। ಸದಾನಂದಪೂರ್ಣ ಸ್ವಾತ್ಮೈವ ಪರದೇವತಾ ಲಲಿತಾ~। ಲೌಹಿತ್ಯಮೇತಸ್ಯ ಸರ್ವಸ್ಯ ವಿಮರ್ಶಃ~। ಅನನ್ಯಚಿತ್ತತ್ವೇನ ಚ ಸಿದ್ಧಿಃ~। ಭಾವನಾಯಾಃ ಕ್ರಿಯಾ ಉಪಚಾರಃ~। ಅಹಂತ್ವಮಸ್ತಿನಾಸ್ತಿಕರ್ತವ್ಯಮಕರ್ತವ್ಯ ಮುಪಾಸಿತವ್ಯಮಿತಿ ವಿಕಲ್ಪಾನಾ ಮಾತ್ಮನಿ ವಿಲಾಪನಂ ಹೋಮಃ~। ಭಾವನಾವಿಷಯಾಣಾ ಮಭೇದಭಾವನಾ ತರ್ಪಣಂ~। ಪಂಚದಶತಿಥಿರೂಪೇಣ ಕಾಲಸ್ಯ ಪರಿಣಾಮಾವಲೋಕನಂ ಪಂಚದಶನಿತ್ಯಾಃ~। ಏವಂ ಮುಹೂರ್ತತ್ರಿತಯಂ ಮುಹೂರ್ತದ್ವಿತಯಂ ಮುಹೂರ್ತಮಾತ್ರಂ ವಾ ಭಾವನಾಪರೋ ಜೀವನ್ಮುಕ್ತೋ ಭವತಿ~। ಸ ಏವ ಶಿವಯೋಗೀತಿ ಗದ್ಯತೇ~। ಕಾದಿಮತೇನಾಂತಶ್ಚಕ್ರ ಭಾವನಾಃ ಪ್ರತಿಪಾದಿತಾಃ। ಯ ಏವಂ ವೇದ। ಸೋಽಥರ್ವಶಿರ್ಷೋಽಧೀತೇ ॥ ಓಂ ಭದ್ರಂ ಕರ್ಣೇಭಿಃ ಇತಿ ಶಾಂತಿಃ ॥


\dhyana{ಶ್ರೀಸಿದ್ಧಮಾನವಮುಖಾ ಗುರವಃ ಸ್ವರೂಪಂ\\
ಸಂಸಾರದಾಹಶಮನಂ ದ್ವಿಭುಜಂ ತ್ರಿನೇತ್ರಂ ।\\
ವಾಮಾಂಗಶಕ್ತಿಸಕಲಾಭರಣೈರ್ವಿಭೂಷಂ\\
ಧ್ಯಾಯೇಜ್ಜಪೇತ್ ಸಕಲಸಿದ್ಧಿಫಲಪ್ರದಂ ಚ ॥}

ಓಂ ನಮಃ ಪ್ರಕಾಶಾನಂದನಾಥಃ ತು ಶಿಖಾಯಾಂ ಪಾತು ಮೇ ಸದಾ ।\\
ಪರಶಿವಾನಂದನಾಥಃ ಶಿರೋ ಮೇ ರಕ್ಷಯೇತ್ ಸದಾ ॥೧॥

ಪರಶಕ್ತಿದಿವ್ಯಾನಂದನಾಥೋ ಭಾಲೇ ಚ ರಕ್ಷತು ।\\
ಕಾಮೇಶ್ವರಾನಂದನಾಥೋ ಮುಖಂ ರಕ್ಷತು ಸರ್ವಧೃಕ್ ॥೨॥

ದಿವ್ಯೌಘೋ ಮಸ್ತಕಂ ದೇವಿ ಪಾತು ಸರ್ವಶಿರಃ ಸದಾ ।\\
ಕಂಠಾದಿನಾಭಿಪರ್ಯಂತಂ ಸಿದ್ಧೌಘಾ ಗುರವಃ ಪ್ರಿಯೇ ॥೩॥

ಭೋಗಾನಂದನಾಥ ಗುರುಃ ಪಾತು ದಕ್ಷಿಣಬಾಹುಕಂ ।\\
ಸಮಯಾನಂದನಾಥಶ್ಚ ಸಂತತಂ ಹೃದಯೇಽವತು ॥೪॥

ಸಹಜಾನಂದನಾಥಶ್ಚ ಕಟಿಂ ನಾಭಿಂ ಚ ರಕ್ಷತು ।\\
ಏಷು ಸ್ಥಾನೇಷು ಸಿದ್ಧೌಘಾಃ ರಕ್ಷಂತು ಗುರವಃ ಸದಾ ॥೫॥

ಅಧರೇ ಮಾನವೌಘಾಶ್ಚ ಗುರವಃ ಕುಲನಾಯಿಕೇ ।\\
ಗಗನಾನಂದನಾಥಶ್ಚ ಗುಲ್ಫಯೋಃ ಪಾತು ಸರ್ವದಾ ॥೬॥

ನೀಲೌಘಾನಂದನಾಥಶ್ಚ ರಕ್ಷಯೇತ್ ಪಾದಪೃಷ್ಠತಃ ।\\
ಸ್ವಾತ್ಮಾನಂದನಾಥಗುರುಃ ಪಾದಾಂಗುಲೀಶ್ಚ ರಕ್ಷತು ॥೭॥

ಕಂದೋಲಾನಂದನಾಥಶ್ಚ ರಕ್ಷೇತ್ ಪಾದತಲೇ ಸದಾ ।\\
ಇತ್ಯೇವಂ ಮಾನವೌಘಾಶ್ಚ ನ್ಯಸೇನ್ನಾಭ್ಯಾದಿಪಾದಯೋಃ ॥೮॥

ಗುರುರ್ಮೇ ರಕ್ಷಯೇದುರ್ವ್ಯಾಂ ಸಲಿಲೇ ಪರಮೋ ಗುರುಃ ।\\
ಪರಾಪರಗುರುರ್ವಹ್ನೌ ರಕ್ಷಯೇತ್ ಶಿವವಲ್ಲಭೇ ॥೯॥

ಪರಮೇಷ್ಠೀಗುರುಶ್ಚೈವ ರಕ್ಷಯೇತ್ ವಾಯುಮಂಡಲೇ ।\\
ಶಿವಾದಿಗುರವಃ ಸಾಕ್ಷಾತ್ ಆಕಾಶೇ ರಕ್ಷಯೇತ್ ಸದಾ ॥೧೦॥

ಇಂದ್ರೋ ಗುರುಃ ಪಾತು ಪೂರ್ವೇ ಆಗ್ನೇಯಾಂ ಗುರುರಗ್ನಯಃ ।\\
ದಕ್ಷೇ ಯಮೋ ಗುರುಃ ಪಾತು ನೈಋತ್ಯಾಂ ನಿಋತಿರ್ಗುರುಃ ॥೧೧॥

ವರುಣೋ ಗುರುಃ ಪಶ್ಚಿಮೇ ವಾಯವ್ಯಾಂ ಮಾರುತೋ ಗುರುಃ ।\\
ಉತ್ತರೇ ಧನದಃ ಪಾತು ಐಶಾನ್ಯಾಮೀಶ್ವರೋ ಗುರುಃ ॥೧೨॥

ಊರ್ಧ್ವಂ ಪಾತು ಗುರುರ್ಬ್ರಹ್ಮಾ ಅನಂತೋ ಗುರುರಪ್ಯಧಃ ।\\
ಏವಂ ದಶದಿಶಃ ಪಾಂತು ಇಂದ್ರಾದಿಗುರವಃ ಕ್ರಮಾತ್ ॥೧೩॥

ಶಿರಸಃ ಪಾದಪರ್ಯಂತಂ ಪಾಂತು ದಿವ್ಯೌಘಸಿದ್ಧಯಃ ।\\
ಮಾನವೌಘಾಶ್ಚ ಗುರವೋ ವ್ಯಾಪಕಂ ಪಾಂತು ಸರ್ವದಾ ॥೧೪॥

ಸರ್ವತ್ರ ಗುರುರೂಪೇಣ ಸಂರಕ್ಷೇತ್ ಸಾಧಕೋತ್ತಮಂ ।\\
ಆತ್ಮಾನಂ ಗುರುರೂಪಂ ಚ ಧ್ಯಾಯೇನ್ ಮಂತ್ರಂ ಸದಾ ಬುಧಃ ॥೧೫॥

\as{ಯತ್ಪುರುಷೇಣ+++++ ಇದ್ಧ್ಮಃ ಶರದ್ಧವಿಃ॥\\
ಆದಿತ್ಯವರ್ಣೇ +++++++++ಬಾಹ್ಯಾ ಅಲಕ್ಷ್ಮೀಃ ॥}ಶುದ್ಧೋದಕಸ್ನಾನಂ॥

\as{ತಂ ಯಜ್ಞಂ +++++ ಋಷಯಶ್ಚ ಯೇ॥\\
ಉಪೈತು ಮಾಂ ++++++++ ದದಾತು ಮೇ ॥\\
ಯುವಂ ವಸ್ತ್ರಾಣಿ ++++ಸಚೇಥೇ ॥}ವಸ್ತ್ರಯುಗ್ಮಮ್ ॥

\as{ತಸ್ಮಾದ್ಯಜ್ಞಾ+++++ಗ್ರಾಮ್ಯಾಶ್ಚ ಯೇ॥\\
ಕ್ಷುತ್ಪಿಪಾಸಾಮ್+++++++++++ ಗೃಹಾತ್॥\\
ಯಜ್ಞೋಪವೀತಮ್ ++++ತೇಜಃ ॥} ಉಪವೀತಮ್ ॥

\as{ಹಿರಣ್ಯರೂಪಃ+++++ತ್ಯನ್ನಮಸ್ಮೈ ॥ }ಆಭರಣಮ್॥

\as{ತಸ್ಮಾದ್ಯಜ್ಞಾತ್+++++ ಸ್ಮಾದಜಾಯತ॥\\
ಗಂಧದ್ವಾರಾಂ+++++++++ ಶ್ರಿಯಮ್ ॥\\
ಗಂಧದ್ವಾರಾಂ+++++++++ ಶ್ರಿಯಮ್ ॥}ಗಂಧಃ ॥

\as{ಅರ್ಚತಪ್ರಾರ್ಚತ+++++ ಧೃಷ್ಣ್ವರದಚತ ॥}ಅಕ್ಷತಾಃ ॥

\as{ತಸ್ಮಾದಶ್ವಾ+++++ಜಾತಾ ಅಜಾವಯಃ॥\\
ಮನಸಃ +++++++++++ ಯಶಃ ॥\\
ಆಯನೇತೇ++++++ಗೃಹಾ ಇಮೇ ॥}ಪುಷ್ಪಾಣಿ ॥


============================================================================================================================




ಗುರುರ್ಬ್ರಹ್ಮಾ ಗ್ರುರುರ್ವಿಷ್ಣುಃ ಗುರುರ್ದೇವೋ ಮಹೇಶ್ವರಃ~।\\
ಗುರುಃ ಸಾಕ್ಷಾತ್ ಪರಂ ಬ್ರಹ್ಮ ತಸ್ಮೈ ಶ್ರೀ ಗುರವೇ ನಮಃ ॥

ಸಚ್ಚಿದಾನಂದ ರೂಪಾಯ ಬಿಂದು ನಾದಾಂತರಾತ್ಮನೇ~।\\
ಆದಿಮಧ್ಯಾಂತ ಶೂನ್ಯಾಯ ಗುರೂಣಾಂ ಗುರವೇ ನಮಃ ॥

ಶುದ್ಧಸ್ಫಟಿಕಸಂಕಾಶಂ ದ್ವಿನೇತ್ರಂ ಕರುಣಾನಿಧಿಂ~।\\
ವರಾಭಯಪ್ರದಂ ವಂದೇ ಶ್ರೀಗುರುಂ ಶಿವರೂಪಿಣಂ ॥

ಸಹಸ್ರದಲಪಂಕಜೇ ಸಕಲಶೀತರಶ್ಮಿಪ್ರಭಂ\\
ವರಾಭಯಕರಾಂಬುಜಂ ವಿಮಲಗಂಧಪುಷ್ಪಾಂಬರಂ~।\\
ಪ್ರಸನ್ನವದನೇಕ್ಷಣಂ ಸಕಲದೇವತಾರೂಪಿಣಂ\\
ಸ್ಮರೇಚ್ಛಿರಸಿ ಹಂಸಗಂ ತದಭಿಧಾನಪೂರ್ವಂ ಗುರುಂ ॥


ಹೃದಂಬುಜೇ ಕರ್ಣಿಕಮಧ್ಯಸಂಸ್ಥೇ\\ ಸಿಂಹಾಸನೇ ಸಂಸ್ಥಿತದಿವ್ಯಮೂರ್ತಿಂ~।\\
ಧ್ಯಾಯೇದ್ಗುರುಂ ಚಂದ್ರಕಲಾಪ್ರಕಾಶಂ\\ ಸಚ್ಚಿತ್ಸುಖಾಭೀಷ್ಟವರಂ ದಧಾನಂ~॥

ಆವಾಹಯಾಮೀ ಗುರುದೇವಂ ಸಹಸ್ರದಲ ಪಂಕಜೇ ।
ಆಗಚ್ಛ ಮಹಾದೇವ ಸ್ವಯಂಪ್ರಕಾಶ ಮೂರ್ತಯೇ ॥

ಸಹಸ್ರದಳಮಧ್ಯೇ ನಾನಾರತ್ನ ನಿರ್ಮಿತಮಾಸನಂ ।
ಭಕ್ತ್ಯಾ ತುಭ್ಯಂ ಮಯಾ ದೇವ ಸ್ವೀಕುರು ಗುರು ಶಂಕರ ॥

ಗಂಗಾದಿ ಸರ್ವತೀರ್ಥೇಭ್ಯೋ ಸುಗಂಧಾಕ್ಷತಮಿಶ್ರಿತಮ್ ।
ಅರ್ಘ್ಯಂ ತೇ ಮಯಾ ದತ್ತಂ ಸ್ವೀಕುರು ಗುರು ದಯಾನಿಧೇ ॥

ಶೀತಲಂ ವಿಮಲಂ ತೋಯಂ ಕರ್ಪೂರಾದಿ ಸುವಾಸಿತಮ್ ।
ಆಚಮ್ಯತಾಂ ಗುರುಶ್ರೇಷ್ಠ ಮಯಾ ದತ್ತಂ ಚ ಭಕ್ತಿತಃ ॥

ಗಾಂಗವಾರಿ ಮನೋಹಾರಿ ಸರ್ವಪಾಪಹರಂ ಶುಭಮ್ ।
ನಿರ್ಮಲಾಯ ಪ್ರಶಾಂತಾಯ ಶರ್ೀಗುರೋ ಪ್ರತಿಗೃಹ್ಯತಾಮ್ ॥

ಪಂಚಭೂತವಿಕಾರಾಖ್ಯಂ ಪಂಚಕೋಶೇ ಪ್ರಪೂರಿತಮ್ ।
ಪಂಚಾಮೃತಾರ್ಥೇ ದಾಸ್ಯಾಮಿ ಪ್ರಪಂಚ ಭ್ರಾಂತಿ ಶಾಂತಯೇ ॥

ಕಾಮಧೇನುಸಮುತ್ಪನ್ನಂ ಸರ್ವೇಶಾಂ ಜೀವನಂ ಪರಂ ।
ಪಾವನಂ ಯಜ್ಞಹೇತುಂ ಚ ಪಯಃ ಸ್ನಾನಾರ್ಥಮರ್ಪಿತಮ್ ॥

ಯದಂಘ್ರಿಕಮಲದ್ವಂದ್ವಂ ದ್ವಂದ್ವತಾಪನಿವಾರಕಂ~।\\
ತಾರಕಂ ಭವಸಿಂಧೋಶ್ಚ ತಂ ಗುರುಂ ಪ್ರಣಮಾಮ್ಯಹಂ~॥

ಪಯಸಾ ತು ಸಮುತ್ಪನ್ನಂ ಮಧುರಾಮ್ಲ ಶಶಿಪ್ರಭಂ ।
ದಧ್ಯಾನೀತಂ ಮಯಾ ಸ್ವಾಮಿನ್ ಸ್ನಾನಾರ್ಥಂ ಪ್ರತಿಗೃಹ್ಯತಾಂ ॥

ಶೋಷಣಂ ಪಾಪಪಂಕಸ್ಯ ದೀಪನಂ ಜ್ಞಾನತೇಜಸಃ~।\\
ಗುರೋಃ ಪಾದೋದಕಂ ಸಮ್ಯಕ್ ಸಂಸಾರಾರ್ಣವತಾರಕಂ~॥

ನವನೀತಸಮುತ್ಪನ್ನಂ ಆಯುರಾರೋಗ್ಯವರ್ಧಕಮ್ ।
ಘೃತಂ ತುಭ್ಯಂ ಪ್ರದಾಸ್ಯಾಮಿ ಸ್ನಾನಾರ್ಥಂ ಪ್ರತಿಗೃಹ್ಯತಾಂ ॥

ತಾಪತ್ರಯಾಗ್ನಿತಪ್ತನಾಮಶಾಂತಪ್ರಾಣಿನಾಂ ಭುವಿ~।\\
ಗುರುರೇವ ಪರಾ ಗಂಗಾ ತಸ್ಮೈ ಶ್ರೀಗುರವೇ ನಮಃ~॥

ತರುಪುಷ್ಪಸಮಾಕೃಷ್ಟಂ ಸುಸ್ವಾದು ಮಧುರಂ ಮಧು ।\\
ತೇಜಃಪುಷ್ಟಿಕರಂ ದಿವ್ಯಂ ಸ್ನಾನಾರ್ಥಂ ಪ್ರತಿಗೃಹ್ಯತಾಂ ॥ \\

ಅಜ್ಞಾನತಿಮಿರಾಂಧಸ್ಯ ಜ್ಞಾನಾಂಜನಶಲಾಕಯಾ~।\\
ಚಕ್ಷುರುನ್ಮೀಲಿತಂ ಯೇನ ತಸ್ಮೈ ಶ್ರೀಗುರವೇ ನಮಃ~॥

ಇಕ್ಷಸಾರಸಮುದ್ಭೂತಾ ಶರ್ಕರಾ ಪುಷ್ಟಿಕಾರಿಕಾ ।\\
ಮಲಾಪಹಾರಿಕಾ ದಿವ್ಯಾ ಸ್ನಾನಾರ್ಥಂ ಪ್ರತಿಗೃಹ್ಯತಾಂ ॥

ಯತ್ಪಾದರೇಣುರ್ವೈ ನಿತ್ಯಂ ಕೋಽಪಿ ಸಂಸಾರವಾರಿಧೌ~।\\
ಸೇತುಬಂಧಾಯತೇ ನಾಥಂ ದೇಶಿಕಂ ತಮುಪಾಸ್ಮಹೇ~॥

ಸರ್ವಸಾರಸಮುದ್ಭೂತಾಂ ಶಕ್ತಿಪುಷ್ಟಿಕರಂ ದೃಢಂ ।\\
ಸುಫಲಂ ಕಾರ್ಯಸಿದ್ಧ್ಯರ್ಥಂಸ್ನಾನಾರ್ಥಂ ಪ್ರತಿಗೃಹ್ಯತಾಂ ॥

ಅಖಂಡಮಂಡಲಾಕಾರಂ ವ್ಯಾಪ್ತಂ ಯೇನ ಚರಾಚರಂ~।\\
ತತ್ಪದಂ ದರ್ಶಿತಂ ಯೇನ ತಸ್ಮೈ ಶ್ರೀಗುರವೇ ನಮಃ~॥

ಮಲಯಾಚಲಸಂಭೂತಂ ಸುಗಂಧಂ ಶೀತಲಂ ಶುಭಂ ।\\
ಸುಕಾಂತಿದಾಯಕಂ ದಿವ್ಯಂ ಸ್ನಾನಾರ್ಥಂ ಪ್ರತಿಗೃಹ್ಯತಾಂ ॥


ಪೃಥ್ವೀ ಗರ್ಭಸಂಭೂತಂ ಸುರ್ವಣಂ ಕಾಂತಿದಂ ಮಹತ್ ।
ಆತ್ಮತೇಜಃಸ್ವರೂಪಾಯ  ಸ್ನಾನಾರ್ಥಂ ಪ್ರತಿಗೃಹ್ಯತಾಂ ॥

ಅಕ್ಷತಾನಿ ಚ ಕರ್ಮಾಣಿ ಕ್ಷತಾನಿ ತವ ಸೇವಯಾ ।
ಅಕ್ಷಯಂ ದೇಹಿ ಮೇ ಜ್ಞಾನಂ ಅಕ್ಷತಾಯ ನಮೋ ನಮಃ ॥

ನಿರ್ಗುಣಂ ನಿರ್ಮಲಂ ಶಾಂತಂ ಜಂಗಮಂ ಸ್ಥಿರಮೇವ ಚ ।
ವ್ಯಾಪ್ತಂ ಯೇನ ಜಗತ್ಸರ್ವಂ ತಸ್ಮೈ ಶ್ರೀ ಗುರವೇ ನಮಃ ॥

ಯದ್ದರ್ಶನಸ್ಪರ್ಶನತೋ ಶುದ್ಧ್ಯಂತ್ಯಜ್ಞಾನಚೇತಸಃ ।
ತಸ್ಮೈ ಶುದ್ಧಿಕರಂ ಸ್ನಾನಂ ತೇ ಕಿಂ ಕುರ್ವಂತಿ ಕ್ಷಾಲನಮ್ ॥

ಗಂಗಾ ಸರಸ್ವತೀ ರೇವಾ ಪಯೋಷ್ಣೀ ಯಮುನಾಜಲೈಃ~।\\
ತದಿದಂ ಕಲ್ಪಿತಂ ಸ್ವಾಮಿನ್ ಸ್ನಾನಾರ್ಥಂ ಪ್ರತಿಗೃಹ್ಯತಾಂ ॥

ನಾನಾತೀರ್ಥಾದಾಹೃತಂ ಚ ತೋಯಮುಷ್ಣಜಲಂ ಮಯಾಕೃತಮ್ ।
ಸ್ನಾನಾರ್ಥಂ ತೇ ಪ್ರಯಚ್ಛಾಮಿ ಸ್ವೀಕುರುಷ್ವ ದಯಾನಿಧೇ ॥

ಸಂಸಾರವೃಕ್ಷಮಾರೂಢಾ ಪತಂತಿ ನರಕಾರ್ಣವೇ ।
ಸರ್ವೇ ಯೇನೋದ್ಧೃತಾ ಲೀಕಾ ತಸ್ಮೈ ಶ್ರೀ ಗುರವೇ ನಮಃ ॥

ಶ್ವೇತವಸ್ತ್ರದ್ವಯಂ ದೇವ ನಾನಾಚಿತ್ರ ವಿನಿರ್ಮಿತಂ ।
ಭಕ್ತ್ಯಾ ಸಮರ್ಪಯೇ ತುಭ್ಯಾಂ ಸ್ವೀಕುರುಷ್ವ ದಯಾನಿಧೇ ॥

ಯತ್ಸತ್ತ್ವೇನ ಜಗತ್ಸತ್ವಂ ಯತ್ಪ್ರಕಾಶೇನ ಭಾತಿ ತತ್~।\\
ಯದಾನಂದೇನ ನಂದಂತಿ ತಸ್ಮೈ ಶ್ರೀಗುರವೇ ನಮಃ~॥

ಉಪವೀತಮಿದಂ ದೇವ ಸುವರ್ಣತಂತು ನಿರ್ಮಿತಂ ।
ಭಕ್ತ್ಯಾ ಸಮರ್ಪಯೇ ತುಭ್ಯಂ ಸ್ವೀಕುರುಷ್ವ ದಯಾನಿಧೇ ॥

ಜ್ಞಾನಶಕ್ತಿಸಮಾರೂಢತತ್ತ್ವಮಾಲಾವಿಭೂಷಿಣೇ~।\\
ಭುಕ್ತಿಮುಕ್ತಿಪ್ರದಾತ್ರೇ ಚ ತಸ್ಮೈ ಶ್ರೀಗುರವೇ ನಮಃ~॥

ಮಾಣಿಕ್ಯಮುಕ್ತಾಫಲವಿದ್ರುಮೈಶ್ಚ ಗೋಮೇಧವೈಡೂರ್ಯಕಪುಷ್ಯರಾಗೈಃ\\
ಪ್ರವಾಲನೀಲೈಶ್ಚ ಕೃತಂ ಗೃಹಾಣ ದಿವ್ಯಂ ಹಿ ರತ್ನಾಭರಣಂ ಚ ದೇವ ।\\

ಅನೇಕಜನ್ಮಸಂಪ್ರಾಪ್ತಕರ್ಮಬಂಧವಿದಾಹಿನೇ~।\\
ಜ್ಞಾನಾನಲಪ್ರಭಾವೇನ ತಸ್ಮೈ ಶ್ರೀಗುರವೇ ನಮಃ~॥


ಅಕ್ಷತಾನಿ ಚ ಕರ್ಮಾಣಿ ಕ್ಷತಾನಿ ತವ ಸೇವಯಾ 
ಅಕ್ಷಯಂ ದೇಹಿ ಮೇ ಜ್ಞಾನಂ ಅಕ್ಷತಾಯ ನಮೋ ನಮಃ ॥

ಮನ್ನಾಥಃ ಶ್ರೀಜಗನ್ನಾಥೋ ಮದ್ಗುರುಃ ಶ್ರೀಜಗದ್ಗುರುಃ~।\\
ಮಮಾತ್ಮಾ ಸರ್ವಭೂತಾತ್ಮಾ ತಸ್ಮೈ ಶ್ರೀಗುರವೇ ನಮಃ~॥


ಮಾಲ್ಯಾನಿ ಚ ಸುಗಂಧೀನಿ ಮಾಲತ್ಯಾದೀನಿ ಚ ಪ್ರಭೋ ।
ಮಯಾರ್ಪಿತಾನಿ ಪುಷ್ಪಾಣಿ ಗೃಹಾಣ ಕರುಣಾನಿಧೇ ॥


ಕರವೀರಜಾತೀಕುಸುಮೈಶ್ಚಂಪಕೈರ್ಬಕುಲೈಃ ಶುಭೈಃ ।\\
ಶತಪತ್ರೈಶ್ಚ ಕಹ್ಲಾರೈರರ್ಚಯೇ ಪರಮೇಶ್ವರ ॥\\

ಹೇತವೇ ಜಗತಾಮೇವ ಸಂಸಾರಾರ್ಣವಸೇತವೇ~।\\
ಪ್ರಭವೇ ಸರ್ವವಿದ್ಯಾನಾಂ ಶಂಭವೇ ಗುರವೇ ನಮಃ~॥

ಹರಿದ್ರಾಕುಂಕುಮಂ ದಿವ್ಯಂ ಸರ್ವಸೌಭಾಗ್ಯದಾಯಕಂ ।
ತುಭ್ಯಂ ಮಯಾರ್ಪಿತಂ  ಭಕ್ತ್ಯಾ ಸ್ವೀಕುಷ್ವ ದಯಾನಿಧೇ ॥

ಏಕ ಏವ ಪರೋ ಬಂಧುರ್ವಿಷಮೇ ಸಮುಪಸ್ಥಿತೇ~।\\
ಗುರುಃ ಸಕಲಧರ್ಮಾತ್ಮಾ ತಸ್ಮೈ ಶ್ರೀಗುರವೇ ನಮಃ~॥

ಸಿಂಧೂರಂ ನಾಗಸಂಭೂತಂ ಫಾಲಶೋಭಾವಿವರ್ಧನಂ ।
ಪೂರಣಂ ಭೂಷಣಾನಾಂ ಚ ಸಿಂಧೂರಂ ಪ್ರತಿಗೃಹ್ಯತಾಂ ॥



ಅಂಗಪೂಜಾ
ಗುರವೇ ಪಾದೌ
ಅಜ್ಞಾನತಿಮಿರಭಾಸ್ಕರಾಯ ಜಂಘೇ
ವ್ಯಾಪಕಾಯ ಊರೂ
ಕಲಿದೋಷಹರಾಯ ಕಟಿಂ
ಚಿನ್ಮಯಾಯ ಉದರಂ
ಚೇತನಾಯ ನಾಭಿಂ
ಸದಾನಂದಾಯ ಹೃದಯಂ
ನಿರ್ಗುಣಾಯ ವಕ್ಷಸ್ಥಲ
ವರದಾಯ ಬಾಹೂ
ಭಕ್ತಾಭೀಷ್ಟಪ್ರದಾಯ ಕರೌ
ಕರುಣಾಸಾಂದ್ರಾಯ ಕಂಠಂ
ಸ್ಥಿತಪ್ರಜ್ಞಾಯ ಲಲಾಟಂ
ಜ್ಞಾನತೇಜಸೇ ಮುಖಂ
ನಾನಾರೂಪಾಯ ನಾಸಿಕಾಂ
ಸಹಸ್ರಾಕ್ಷಾಯ ನೇತ್ರಾಣಿ
ಓಂಕಾರರೂಪಾಯ ಕರ್ಣೌ
ದಯಾಸಾಗರಾಯ ದಂತಾನ್
ಸ್ವಪ್ರಕಾಶಾಯ ಶಿರಃ
ಸದ್ಗುರವೇ ಸರ್ವಾಂಗಂ


ಯಸ್ಯ ಸ್ಮರಣಮಾತ್ರೇಣ ಜ್ಞಾನಮುತ್ಪದ್ಯತೇ ಸ್ವಯಂ~।\\
ಸ ಏವ ಸರ್ವಸಂಪತ್ತಿಃ ತಸ್ಮಾತ್ಸಂಪೂಜಯೇದ್ಗುರುಂ~॥

ಸಂಸಾರಾದಿತಾರಕಾಯ ಪಾದೌ
ಗುರವೇ ಗುಲ್ಫೌ
ಅಜ್ಞಾನತಿಮಿರಭಾಸ್ಕರಾಯ ಜಂಘೇ
ವ್ಯಾಪಕಾಯ ಊರೂ
ಕಲಿದೋಷಹರಾಯ ಕಟಿಂ
ವಿಶ್ವಂಭರಾಯ ಉದರಂ
ನಾದಬಿಂದುಕಲಾತೀತಾಯ ನಾಭಿಂ
ಶಾಂತಸ್ವರೂಪಾಯ ವಕ್ಷಸ್ಥಲ
ಸತ್ಯಧರ್ಮಸ್ವರೂಪಾಯ ಬಾಹೂ
ಸದಾನಂದಾಯ ಹೃದಯಂ
ಭಕ್ತಾಭೀಷ್ಟವರಪ್ರದಾಯ ಕರೌ
ಕರುಣಾಸಾಂದ್ರಾಯ ಕಂಠಂ
ಆನಂದಸಾಗರಾಯ ಮುಖಂ
ನಿರ್ಗುಣಾಯ ನಾಸಿಕಾಂ
ದಿವ್ಯಚಕ್ಷುಷೇ ನೇತ್ರಾಣಿ
ಸ್ವಪ್ರಕಾಶಾಯ ಲಲಾಟಂ
ಓಂಕಾರರೂಪಾಯ ಕರ್ಣೌ
ದಯಾಸಾಗರಾಯ ದಂತಾನ್
ಮಾಯಾಯುಕ್ತಾಯ ಶಿರಃ
ಸದ್ಗುರವೇ ಸರ್ವಾಂಗಂ

ಪ್ರಥಮಾವರಣಮ್
ಅಣಿಮಾಸಿದ್ಧ್ಯೈ ನಮಃ
ಲಘಿಮಾಸಿದ್ಧ್ಯೈ ನಮಃ
ಮಹಿಮಾಸಿದ್ಧ್ಯೈ ನಮಃ
ಈಶಿತ್ವಸಿದ್ಧ್ಯೈ ನಮಃ
ವಶಿತ್ವಸಿದ್ಧ್ಯೈ ನಮಃ
ಪ್ರಾಕಾಮ್ಯಸಿದ್ಧ್ಯೈ ನಮಃ
ಭುಕ್ತಿಸಿದ್ಧ್ಯೈ ನಮಃ
ಇಚ್ಛಾಸಿದ್ಧ್ಯೈ ನಮಃ
ಪ್ರಾಪ್ತಿಸಿದ್ಧ್ಯೈ ನಮಃ
ಸರ್ವಕಾಮಸಿದ್ಧ್ಯೈ ನ

ದ್ವಿತೀಯಾವರಣಮ್
ಮೇಷಾಯ ನಮಃ
ವೃಷಭಾಯ ನಮಃ
ಮಿಥುನಾಯ ನಮಃ
ಕಟಕಾಯ ನಮಃ
ಸಿಂಹಾಯ ನಮಃ
ಕನ್ಯಾಯೈ ನಮಃ
ತುಲಾಯೈ ನಮಃ
ವೃಶ್ಚಿಕಾಯ ನಮಃ
ಧನುಷೇ ನಮಃ
ಮಕರಾಯ ನಮಃ
ಕುಂಭಾಯ ನಮಃ
ಮೀನಾಯ ನಮಃ

ತೃತೀಯಾವರಣಮ್
ಸೂರ್ಯಾಯ ನಮಃ
ಚಂದ್ರಾಯ ನಮಃ
ಅಂಗಾರಕಾಯ ನಮಃ
ಬುಧಾಯ ನಮಃ
ಗುರವೇ ನಮಃ
ಶುಕ್ರಾಯ ನಮಃ
ಶನೈಶ್ಚರಾಯ ನಮಃ
ರಾಹವೇ ನಮಃ
ಕೇತವೇ ನಮಃ
ಋಗ್ವೇದಾಯ ನಮಃ ।
ಯಜುರ್ವೇದಾಯ ನಮಃ ।
ಸಾಮವೇದಾಯ ನಮಃ ।
ಅಥರ್ವಣವೇದಾಯ ನಮಃ ।

ಚತುರ್ಥಾವರಣಮ್
ಸನಾತನಾಯ ಭೃಗವೇ ಸನಂದನಾಯ ಭರದ್ವಾಜಾಯ ವಸಿಷ್ಠಾಯ ಆತ್ರೇಯಾಯ ಸನತ್ಕುಮಾರಾಯ ನಾರದಾಯ ಪ್ರಭವೇ

ಪಂಚಮಾವರಣಮ್
ಮತ್ಸ್ಯಾಯ ಕೂರ್ಮಾಯ ವರಾಹಾಯ ನೃಸಿಂಹಾಯ ವಾಮನಾಯ ಭಾರ್ಗವಾಯ ರಾಮಾಯ ಕೃಷ್ಣಾಯ ಬುದ್ಧಾಯ ಕಲ್ಕಿನೇ ॥

ಷಷ್ಠಾವರಣಮ್
ಚಿಂತಾಮಣಯೇ ಕಲ್ಪವೃಕ್ಷಾಯ ಕಾಮಧೇನವೇ ತ್ರಿಶೂಲಾಯ ಡಮರುಗಾಯ ಜಪಮಾಲಾಯೈ ಪುಸ್ತಕಾಯ ಶಂಖಾಯ ಚಕ್ರಾಯ ಗದಾಯ ಖಡ್ಗಾಯ ಕಮಲಾಯ

ಸಪ್ತಮಾವರಣಮ್
ಇಂದ್ರಾಯ ನಮಃ ।
ಅಗ್ನಯೇ ನಮಃ ।
ಯಮಾಯ ನಮಃ ।
ನಿರ್ಋತಯೇ ನಮಃ 
ವರುಣಾಯ ನಮಃ ।
ವಾಯವೇ ನಮಃ ।
ಕುಬೇರಾಯ ನಮಃ ।
ಈಶಾನಾಯ ನಮಃ ।
ಬ್ರಹ್ಮಣೇ ನಮಃ ।
ಅನಂತಾಯ ನಮಃ ।
ನಿಯತ್ಯೈ ನಮಃ ।
ಕಾಲಾಯ ॥

ಅಷ್ಟಮಾವರಣಮ್
ಭವಾಯ ಶರ್ವಾಯ ಈಶಾನಾಯ ಪಶುಪತಯೇ ಉಗ್ರಾಯ ರುದ್ರಾಯ ಭೀಮಾಯ ಮಹತೇ ॥

ನವಮಾವರಣಮ್
ಓಂ ಬ್ರಾಹ್ಮ್ಯೈ ನಮಃ ।
ಮಾಹೇಶ್ವರ್ಯೈ ನಮಃ ।
ಕೌಮಾರ್ಯೈ ನಮಃ ।
ವೈಷ್ಣವ್ಯೈ ನಮಃ ।
ವಾರಾಹ್ಯೈ ನಮಃ ।
ಮಾಹೇಂದ್ರ್ಯೈ ನಮಃ ।
ಚಾಮುಂಡಾಯೈ ನಮಃ ।
ಮಹಾಲಕ್ಷ್ಮ್ಯೈ ನಮಃ ॥


ಪತ್ರ ಪೂಜಾ 

ಗುರವೇ ದೂರ್ವಾ
ಪರಮಗುರವೇ ಬಿಲ್ವ
ಪರಮೇಷ್ಠಿಗುರವೇ ತುಳಸೀ
ಜ್ಞಾನಸಾಗರಾಯ ಮಲ್ಲಿಕಾ
ಸ್ವಪ್ರಕಾಶಾಯ ಸೇವಂತಿಕಾ
ಬ್ರಹ್ಮಸ್ವರೂಪಾಯ ಕಮಲಾ
ಶಿವಸ್ವರೂಪಾಯ ಕುಶ
ವಿಷ್ಣುಸ್ವರೂಪಾಯ ವಿಷ್ಣುಕ್ರಾಂತಿ
ಚಿನ್ಮಯಾಯ ಚಂಪಕ
ಮೋಕ್ಷಪಾಣಯೇ ಮಲ್ಲಿಕಾ
ಕರುಣಾಕರಾಯ ಕರವೀರ
ಶೋಕರಹಿತಾಯ ಅಶೋಕ
ಪ್ರಣವಸ್ವರೂಪಾಯ ಪುನ್ನಾಗ
ಅಜ್ಞಾನಹರಣಾಯ ಅತಸೀ
ನಿರ್ಗುಣಾಯ ನಿರ್ಗುಂಡೀ
ಜಗನ್ನಾಥಾಯ ಜಾಜೀ
ಶರಣಾಗತರಕ್ಷಕಾಯ ಶಮೀ
ತ್ರಿಗುಣಾತ್ಮಕಾಯ ಮರುಗ
ಮಂಗಲಾಯ ಮಂದಾರ
ಜ್ಞಾನನಿಧಯೇ ಆಮಲಕ
ಶ್ರೀಗುರವೇ ನಮಃ ಏಕವಿಂಶತಿ ಪತ್ರಾಣಿ ಸಮರ್ಪಯಾಮಿ



ಅಥ ಪುಷ್ಪಪೂಜಾ
ಗುರವೇ ದ್ರೋಣ
ಪರಮಪುರುಷಾಯ ಪಾಟಲೀ
ಮಂಗಲದಾಯಕಾಯ ಮಲ್ಲಿಕಾ
ಚಿನ್ಮಯಾಯ ಗಿರಿಕರ್ಣಿಕಾ
ಪರಮಾತ್ಮನೇ ಪೂಗ
ಪರಮಹಂಸಾಯ ಪುನ್ನಾಗ
ಪರಂಜ್ಯೋತಿಷೇ ಸೇವಂತಿಕಾ
ಭಕ್ತಪ್ರಿಯಾಯ ಮಾಲತೀ
ಮೋಕ್ಷದಾಯಕಾಯ ಮಂದಾರ
ಬ್ರಹ್ಮಸ್ವರೂಪಾಯ ಬಕುಲ
ಆಕಾರರಹಿತಾಯ ಅತಸೀ
ಜಗತ್ಪ್ರಭವೇ ಜಾಜೀ
ಕರುಣಾಕರಾಯ ಕರವೀರ
ವಿಷ್ಣುರೂಪಿಣೇ ತುಳಸೀ
ಸುಗುಣಾಯ ಸೌಗಂಧಿಕಾ
ಆನಂದರೂಪಾಯ ಜಪಾ
ಅನುಗ್ರಹಪ್ರದಾಯ ಚಂಪಕ
ಶಿವರೂಪಾಯ ಶ್ವೇತ
ದಯಾನಿಧಯೇ ದಾಡಿಮ
ಜಗನ್ನಾಥಾಯ ವಿಷ್ಣುಕ್ರಾಂತಿ
ಶ್ರೀಗುರವೇ ಸರ್ವ


ಅಥ ಪಾದುಕಾಪೂಜಾಂ ಕರಿಷ್ಯೇ
ದಿವ್ಯೌಘಸಿದ್ಧೌಘಮಾನವೌಘೇಭ್ಯೋ ನಮಃ
ಪರಪ್ರಕಾಶಾನಂದನಾಥ ಶ್ರೀಪಾದುಕಾಂ ಪೂಜಯಾಮಿ ।
ಪರಶಿವಾನಂದನಾಥ ಶ್ರೀಪಾದುಕಾಂ ಪೂಜಯಾಮಿ ।
ಪರಾಶಕ್ತ್ಯಂಬಾ ಶ್ರೀಪಾದುಕಾಂ ಪೂಜಯಾಮಿ ।
ಕೌಲೇಶ್ವರಾನಂದನಾಥ ಶ್ರೀಪಾದುಕಾಂ ಪೂಜಯಾಮಿ ।
ಶುಕ್ಲದೇವ್ಯಂಬಾ ಶ್ರೀಪಾದುಕಾಂ ಪೂಜಯಾಮಿ ।
ಕುಲೇಶ್ವರಾನಂದನಾಥ ಶ್ರೀಪಾದುಕಾಂ ಪೂಜಯಾಮಿ ।
ಕಾಮೇಶ್ವರ್ಯಂಬಾ ಶ್ರೀಪಾದುಕಾಂ ಪೂಜಯಾಮಿ ।
ಭೋಗಾನಂದನಾಥ ಶ್ರೀಪಾದುಕಾಂ ಪೂಜಯಾಮಿ ।
ಕ್ಲಿನ್ನಾನಂದನಾಥ ಶ್ರೀಪಾದುಕಾಂ ಪೂಜಯಾಮಿ ।
ಸಮಯಾನಂದನಾಥ ಶ್ರೀಪಾದುಕಾಂ ಪೂಜಯಾಮಿ ।
ಸಹಜಾನಂದನಾಥ  ಶ್ರೀಪಾದುಕಾಂ ಪೂಜಯಾಮಿ ।
ಗಗನಾನಂದನಾಥ ಶ್ರೀಪಾದುಕಾಂ ಪೂಜಯಾಮಿ ।
ವಿಶ್ವಾನಂದನಾಥ ಶ್ರೀಪಾದುಕಾಂ ಪೂಜಯಾಮಿ ।
ವಿಮಲಾನಂದನಾಥ ಶ್ರೀಪಾದುಕಾಂ ಪೂಜಯಾಮಿ ।
ಮದನಾನಂದನಾಥ ಶ್ರೀಪಾದುಕಾಂ ಪೂಜಯಾಮಿ ।
ಭುವನಾನಂದನಾಥ ಶ್ರೀಪಾದುಕಾಂ ಪೂಜಯಾಮಿ ।
ಲೀಲಾನಂದನಾಥ ಶ್ರೀಪಾದುಕಾಂ ಪೂಜಯಾಮಿ ।
ಸ್ವಾತ್ಮಾನಂದನಾಥ ಶ್ರೀಪಾದುಕಾಂ ಪೂಜಯಾಮಿ ।
ಪ್ರಿಯಾನಂದನಾಥ ಶ್ರೀಪಾದುಕಾಂ ಪೂಜಯಾಮಿ ।
ಶ್ರೀಗುರವೇ ನಮಃ ।
ಪರಮಗುರವೇ ನಮಃ ।
ಪರಮೇಷ್ಠಿಗುರವೇ ನಮಃ ।

ಏಹಿ ಸೂರ್ಯ ಸಹಸ್ರಾಂಶೋ ತೇಜೋರಾಶೇ ಜಗತ್ಪತೇ ।
ಅನುಕಂಪಯ ಮಾಂ ಭಕ್ತ್ಯಾ ಗೃಹಾಣಾರ್ಘ್ಯಂ ನಮೋಽಸ್ತು ತೇ ॥

ನಮಃ ಕಮಲನಾಭಾಯ ನಮಸ್ತೇ ಜಲಶಾಯಿನೇ ।
ನಮಸ್ತೇ ಕೇಶವಾನಂತ ಗೃಹಾಣಾರ್ಘ್ಯಂ ನಮೋಽಸ್ತು ತೇ ॥

ವರಾಹದಂಷ್ಟ್ರೋದ್ಭವೇ ದೇವಿ ಗಂಗೇ ತ್ರಿಪಥಗಾಮಿನಿ ।
ತುಂಗಭದ್ರೇತಿ ವಿಖ್ಯಾತೇ ಗೃಹಾಣಾರ್ಘ್ಯಂ ನಮೋಽಸ್ತು ತೇ ॥

ವೇದಪಾದೋದ್ಭವೇ ದೇವಿ ಶ್ರೀಶೈಲೋತ್ಸಂಗಗಾಮಿನಿ ।
ತುಂಗಭದ್ರೇತಿ ವಿಖ್ಯಾತೇ ಗೃಹಾಣಾರ್ಘ್ಯಂ ನಮೋಽಸ್ತು ತೇ ॥

ದಕ್ಷಿಣೇ ಕಲಶೇಶಸ್ಯ ನಿಧಿ ಸನ್ನಿಧಿಕಾರಿಣಿ ।
ಕೋಶತೀರ್ಥೇತಿ ವಿಖ್ಯಾತೇ ಗೃಹಾಣಾರ್ಘ್ಯಂ ನಮೋಽಸ್ತು ತೇ ॥

ಕಾಶಪುಷ್ಪ ಪ್ರತೀಕಾಶ ವಹ್ನಿಮಾರುತಸಂಭವ ।
ಮಿತ್ರಾವರುಣಯೋಃ ಪುತ್ರ ಕುಂಭಯೋನೇ ನಮೋಽಸ್ತು ತೇ ॥

ರಾಜಪುತ್ರಿ ನಮಸ್ತುಭ್ಯಂ ಋಷಿಪತ್ನಿ ವರಾನನೇ ।
ಲೋಪಾಮುದ್ರೇತಿ ವಿಖ್ಯಾತೇ ಸುಭಗೇ ಸರ್ವಮಂಗಳೇ ॥

ಪಶ್ಚಿಮೇ ಕಲಶೇಶಸ್ಯ ರುದ್ರೈಕಾದಶಸೇವಿತೇ ।
ರುದ್ರತೀರ್ಥೇತಿ ವಿಖ್ಯಾತೇ ಗೃಹಾಣಾರ್ಘ್ಯಂ ನಮೋಽಸ್ತು ತೇ ॥

ಪಶ್ವಿಮೇ ಕಲಶೇಶಸ್ಯ ಸೇವಿತೇ ದೇವಕೋಟಿಭಿಃ ।
ಕೋಟಿತೀರ್ಥೇತಿ ವಿಖ್ಯಾತೇ ಗೃಹಾಣಾರ್ಘ್ಯಂ ನಮೋಽಸ್ತು ತೇ ॥

ವಾಯವ್ಯಾಂ ಕಲಶೇಶಸ್ಯ ಪಾರ್ವತೀಪ್ರಿಯವಲ್ಲಭೇ ।
ಅಂಬಾತೀರ್ಥೇತಿ ವಿಖ್ಯಾತೇ ಗೃಹಾಣಾರ್ಘ್ಯಂ ನಮೋಽಸ್ತು ತೇ ॥

ಉತ್ತರೇ ಕಲಶೇಶಸ್ಯ ಸರ್ವತೀರ್ಥಸ್ವರೂಪಿಣಿ ।
ವಸಿಷ್ಠತೀರ್ಥೇತಿ ಖ್ಯಾತೇ ಗೃಹಾಣಾರ್ಘ್ಯಂ ನಮೋಽಸ್ತು ತೇ ॥



