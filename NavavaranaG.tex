
\part*{ತ್ರಿಪುರಸುಂದರೀಸಪರ್ಯಾ}
\chapter*{\center ಗುರುಪ್ರಾರ್ಥನಾ}
ಓಂ ಆಬ್ರಹ್ಮಲೋಕಾದಾಶೇಷಾತ್ ಆಲೋಕಾಲೋಕಪರ್ವತಾತ್।\\
ಯೇ ವಸಂತಿ ದ್ವಿಜಾ ದೇವಾಃ ತೇಭ್ಯೋ ನಿತ್ಯಂ ನಮಾಮ್ಯಹಂ॥

ಓಂ ನಮೋ ಬ್ರಹ್ಮಾದಿಭ್ಯೋ ಬ್ರಹ್ಮವಿದ್ಯಾಸಂಪ್ರದಾಯಕರ್ತೃಭ್ಯೋ\\ ವಂಶರ್ಷಿಭ್ಯೋ ಮಹದ್ಭ್ಯೋ ನಮೋ ಗುರುಭ್ಯಃ~॥

ಸರ್ವೋಪಪ್ಲವರಹಿತಃ ಪ್ರಜ್ಞಾನಘನಃ ಪ್ರತ್ಯಗರ್ಥೋ ಬ್ರಹ್ಮೈವಾಹಮಸ್ಮಿ ॥

ಸಚ್ಚಿದಾನಂದ ರೂಪಾಯ ಬಿಂದು ನಾದಾಂತರಾತ್ಮನೇ~।\\
ಆದಿಮಧ್ಯಾಂತ ಶೂನ್ಯಾಯ ಗುರೂಣಾಂ ಗುರವೇ ನಮಃ ॥

ಶುದ್ಧಸ್ಫಟಿಕಸಂಕಾಶಂ ದ್ವಿನೇತ್ರಂ ಕರುಣಾನಿಧಿಂ~।\\
ವರಾಭಯಪ್ರದಂ ವಂದೇ ಶ್ರೀಗುರುಂ ಶಿವರೂಪಿಣಂ ॥

ಗುರುರ್ಬ್ರಹ್ಮಾ ಗ್ರುರುರ್ವಿಷ್ಣುಃ ಗುರುರ್ದೇವೋ ಮಹೇಶ್ವರಃ~।\\
ಗುರುಃ ಸಾಕ್ಷಾತ್ ಪರಂ ಬ್ರಹ್ಮ ತಸ್ಮೈ ಶ್ರೀ ಗುರವೇ ನಮಃ ॥

ಶುಕ್ಲಾಂಬರಧರಂ ವಿಷ್ಣುಂ ಶಶಿವರ್ಣಂ ಚತುರ್ಭುಜಂ~।\\
ಪ್ರಸನ್ನವದನಂ ಧ್ಯಾಯೇತ್ ಸರ್ವವಿಘ್ನೋಪಶಾಂತಯೇ ॥
\newpage
ಮಂಜುಶಿಂಜಿತ ಮಂಜೀರಂ ವಾಮಮರ್ಧಂ ಮಹೇಶಿತುಃ~।\\
ಆಶ್ರಯಾಮಿ ಜಗನ್ಮೂಲಂ ಯನ್ಮೂಲಂ ವಚಸಾಮಪಿ ॥

ಶ್ರೀ ವಿದ್ಯಾಂ ಜಗತಾಂ ಧಾತ್ರೀಂ ಸರ್ಗಸ್ಥಿತಿ ಲಯೇಶ್ವರೀಂ~।\\
ನಮಾಮಿ ಲಲಿತಾಂ ನಿತ್ಯಂ ಭಕ್ತಾನಾಮಿಷ್ಟದಾಯಿನೀಂ ॥

ಶ್ರೀನಾಥಾದಿ ಗುರುತ್ರಯಂ ಗಣಪತಿಂ ಪೀಠತ್ರಯಂ ಭೈರವಂ\\
ಸಿದ್ಧೌಘಂ ಬಟುಕತ್ರಯಂ ಪದಯುಗಂ ದೂತೀಕ್ರಮಂ ಮಂಡಲಂ~।\\
ವೀರದ್ವ್ಯಷ್ಟ ಚತುಷ್ಕಷಷ್ಟಿನವಕಂ ವೀರಾವಲೀಪಂಚಕಂ\\
ಶ್ರೀಮನ್ಮಾಲಿನಿಮಂತ್ರರಾಜಸಹಿತಂ ವಂದೇ ಗುರೋರ್ಮಂಡಲಂ॥

ಬಿಂದುತ್ರಿಕೋಣಸಂಯುಕ್ತಂ ವಸುಕೋಣಸಮನ್ವಿತಂ~।\\
ದಶಕೋಣದ್ವಯೋಪೇತಂ ಭುವನಾರ ಸಮನ್ವಿತಂ ॥\\
ದಲಾಷ್ಟಕ ಸಮೋಪೇತಂ ದಲಷೋಡಶಕಾನ್ವಿತಂ~।\\
ವೃತ್ತತ್ರಯಾನ್ವಿತಂ ಭೂಮೀಸದನತ್ರಯ ಭೂಷಿತಂ~।\\
ನಮಾಮಿ ಲಲಿತಾಚಕ್ರಂ ಭಕ್ತಾನಾಮಿಷ್ಟದಾಯಕಂ ॥
\section{ದ್ವಾರಪೂಜಾ}
ಓಂ ಐಂಹ್ರೀಂಶ್ರೀಂ ವಂ ವಟುಕಾಯ ನಮಃ(ದ್ವಾರಸ್ಯ ಅಧಃ)\\
೪ ಭಂ ಭದ್ರಕಾಲ್ಯೈ ನಮಃ (ದ್ವಾರಸ್ಯ ವಾಮಭಾಗೇ)\\
೪ ಭಂ ಭೈರವಾಯ ನಮಃ(ದ್ವಾರಸ್ಯ ದಕ್ಷಭಾಗೇ)\\
೪ ಲಂ ಲಂಬೋದರಾಯ ನಮಃ(ದ್ವಾರಸ್ಯ ಊರ್ಧ್ವಭಾಗೇ)
\section{ಮಂತ್ರಾಚಮನಮ್}
\dhyana{೪ ಐಂ ಕಏಈಲಹ್ರೀಂ} ಆತ್ಮತತ್ವಂ ಶೋಧಯಾಮಿ ನಮಃ ಸ್ವಾಹಾ।\\
\dhyana{೪ ಕ್ಲೀಂ ಹಸಕಹಲಹ್ರೀಂ} ವಿದ್ಯಾತತ್ವಂ ಶೋಧಯಾಮಿ ನಮಃ ಸ್ವಾಹಾ।\\
\dhyana{೪ ಸೌಃ ಸಕಲಹ್ರೀಂ} ಶಿವತತ್ವಂ ಶೋಧಯಾಮಿ ನಮಃ ಸ್ವಾಹಾ।\\
\dhyana{೪ ಐಂ (೫) ಕ್ಲೀಂ (೬) ಸೌಃ (೪)} ಸರ್ವತತ್ವಂ ಶೋಧಯಾಮಿ ನಮಃ ಸ್ವಾಹಾ।
\section{ಗುರುಪಾದುಕಾಮಂತ್ರಃ}
\dhyana{ಓಂಐಂಹ್ರೀಂಶ್ರೀಂಐಂಕ್ಲೀಂಸೌಃ ಹಂಸಃ ಶಿವಃ ಸೋಽಹಂ ಹ್‌ಸ್‌ಖ್‌ಫ್ರೇಂ ಹಸಕ್ಷಮಲವರಯೂಂ ಹ್‌ಸೌಃ ಸಹಕ್ಷಮಲವರಯೀಂ ಸ್‌ಹೌಃ ಹಂಸಃ ಶಿವಃ ಸೋಽಹಂ~॥} ಸ್ವರೂಪ ನಿರೂಪಣ ಹೇತವೇ ಶ್ರೀಗುರವೇ ನಮಃ~। ಶ್ರೀಪಾದುಕಾಂ ಪೂಜಯಾಮಿ ನಮಃ ॥\\
\dhyana{೭ ಸೋಽಹಂ ಹಂಸಃ ಶಿವಃ ಹ್‌ಸ್‌ಖ್‌ಫ್ರೇಂ ಹಸಕ್ಷಮಲವರಯೂಂ ಹ್‌ಸೌಃ ಸಹಕ್ಷಮಲವರಯೀಂ ಸ್‌ಹೌಃ ಸೋಽಹಂ ಹಂಸಃ ಶಿವಃ॥} ಸ್ವಚ್ಛಪ್ರಕಾಶ ವಿಮರ್ಶಹೇತವೇ ಪರಮಗುರವೇ ನಮಃ।ಶ್ರೀಪಾದುಕಾಂ ಪೂಜಯಾಮಿ ನಮಃ॥\\
\dhyana{೭ ಹಂಸಃಶಿವಃ ಸೋಽಹಂಹಂಸಃ ಹ್‌ಸ್‌ಖ್‌ಫ್ರೇಂ ಹಸಕ್ಷಮಲವರಯೂಂ ಹ್‌ಸೌಃ ಸಹಕ್ಷಮಲವರಯೀಂ ಸ್‌ಹೌಃ ಹಂಸಃ ಶಿವಃ ಸೋಽಹಂ ಹಂಸಃ॥} ಸ್ವಾತ್ಮಾರಾಮ ಪರಮಾನಂದ ಪಂಜರ ವಿಲೀನ ತೇಜಸೇ ಪರಮೇಷ್ಠಿಗುರವೇ ನಮಃ~। ಶ್ರೀಪಾದುಕಾಂ ಪೂಜಯಾಮಿ ನಮಃ॥
\section{ಘಂಟಾನಾದಃ}
೪ ಹೇ ಘಂಟೇ ಸುಸ್ವರೇ ರಮ್ಯೇ ಘಂಟಾಧ್ವನಿವಿಭೂಷಿತೇ।\\
ವಾದಯಂತಿ ಪರಾನಂದೇ ಘಂಟಾದೇವಂ ಪ್ರಪೂಜಯೇ॥\\
\as{ಓಂ ಜಗದ್ಧ್ವನಿಮಂತ್ರಮಾತಃ ಸ್ವಾಹಾ ॥}\\
ಆಗಮಾರ್ಥಂ ಚ ದೇವಾನಾಂ ಗಮನಾರ್ಥಂ ಚ ರಕ್ಷಸಾಂ।\\
ಕುರ್ಯಾತ್ ಘಂಟಾರವಂ ತತ್ರ ದೇವತಾಹ್ವಾನ ಲಾಂಛನಂ॥
\newpage
ವಿಷ್ಣುಶಕ್ತಿಸಮೋಪೇತೇ ಸರ್ವವರ್ಣೇ ಮಹೀತಲೇ~।\\
ಅನೇಕರತ್ನಸಂಭೂತೇ ಭೂಮಿದೇವಿ ನಮೋಽಸ್ತು ತೇ॥

ಪೃಥ್ವೀತಿ ಮಂತ್ರಸ್ಯ ಮೇರುಪೃಷ್ಠ ಋಷಿಃ~। ಸುತಲಂ ಛಂದಃ~।\\ಆದಿಕೂರ್ಮೋ ದೇವತಾ ॥ ಆಸನೇ ವಿನಿಯೋಗಃ ॥\\
ಪೃಥ್ವಿ ತ್ವಯಾ ಧೃತಾ ಲೋಕಾ ದೇವಿ ತ್ವಂ ವಿಷ್ಣುನಾ ಧೃತಾ~।\\
ತ್ವಂ ಚ ಧಾರಯ ಮಾಂ ದೇವಿ ಪವಿತ್ರಂ ಕುರು ಚಾಸನಂ ॥

ಅಪಸರ್ಪಂತು ತೇ ಭೂತಾಃ ಯೇ ಭೂತಾ ಭೂಮಿ ಸಂಸ್ಥಿತಾಃ~।\\
ಯೇ ಭೂತಾಃ ವಿಘ್ನಕರ್ತಾರಸ್ತೇನಶ್ಯಂತು ಶಿವಾಜ್ಞಯಾ ॥

ಅಪಕ್ರಾಮಂತು ಭೂತಾನಿ ಪಿಶಾಚಾಃ ಸರ್ವತೋ ದಿಶಂ।\\
ಸರ್ವೇಷಾಮವಿರೋಧೇನ ಪೂಜಾ ಕರ್ಮಸಮಾರಭೇ ॥

ಸ್ಯೋನಾ ಪೃಥಿವೀತ್ಯಸ್ಯ ಮೇಧಾತಿಥಿಃ ಕಾಣ್ವ ಋಷಿಃ । ಗಾಯತ್ರೀ ಛಂದಃ । ಪೃಥಿವೀ ದೇವತಾ । ಭೂಪ್ರಾರ್ಥನೇ ವಿನಿಯೋಗಃ ॥\\
ಸ್ಯೋ॒ನಾ ಪೃ॑ಥಿವಿ ಭವಾನೃಕ್ಷ॒ರಾ ನಿ॒ವೇಶ॑ನೀ~।\\ ಯಚ್ಛಾ᳚ನಃ॒ ಶರ್ಮ॑ ಸ॒ಪ್ರಥಃ॑ ॥

ಧನುರ್ಧರಾಯೈ ಚ ವಿದ್ಮಹೇ ಸರ್ವಸಿದ್ಧ್ಯೈ ಚ ಧೀಮಹಿ~।\\ ತನ್ನೋ ಧರಾ ಪ್ರಚೋದಯಾತ್ ॥

ಲಂ ಪೃಥಿವ್ಯೈ ನಮಃ~। ರಂ ರಕ್ತಾಸನಾಯ ನಮಃ~। ವಿಂ ವಿಮಲಾಸನಾಯ ನಮಃ~। ಯಂ ಯೋಗಾಸನಾಯ ನಮಃ~। ಕೂರ್ಮಾಸನಾಯ ನಮಃ~। ಅನಂತಾಸನಾಯ ನಮಃ~। ವೀರಾಸನಾಯ ನಮಃ~। ಖಡ್ಗಾಸನಾಯ ನಮಃ~। ಶರಾಸನಾಯ ನಮಃ~। ಪಂ ಪದ್ಮಾಸನಾಯ ನಮಃ~। ಪರಮಸುಖಾಸನಾಯ ನಮಃ॥

೪ ರಕ್ತದ್ವಾದಶಶಕ್ತಿಯುಕ್ತಾಯ ದ್ವೀಪನಾಥಾಯ ನಮಃ ॥

೪ ಶ್ರೀಲಲಿತಾಮಹಾತ್ರಿಪುರಸುಂದರಿ ಆತ್ಮಾನಂ ರಕ್ಷ ರಕ್ಷ ॥

ಓಂ ಗುಂ ಗುರುಭ್ಯೋ ನಮಃ~। ಪರಮಗುರುಭ್ಯೋ ನಮಃ~। ಪರಮೇಷ್ಠಿ\\ಗುರುಭ್ಯೋ ನಮಃ~। ಗಂ ಗಣಪತಯೇ ನಮಃ~। ದುಂ ದುರ್ಗಾಯೈ ನಮಃ~। ಸಂ ಸರಸ್ವತ್ಯೈ ನಮಃ~। ವಂ ವಟುಕಾಯ ನಮಃ~। ಕ್ಷಂ ಕ್ಷೇತ್ರಪಾಲಾಯ ನಮಃ~। ಯಾಂ ಯೋಗಿನೀಭ್ಯೋ ನಮಃ~। ಅಂ ಆತ್ಮನೇ ನಮಃ~। ಪಂ ಪರಮಾತ್ಮನೇ ನಮಃ~। ಸಂ ಸರ್ವಾತ್ಮನೇ ನಮಃ ॥

೪ ಓಂ ನಮೋ ಭಗವತಿ ತಿರಸ್ಕರಿಣಿ ಮಹಾಮಾಯೇ ಮಹಾನಿದ್ರೇ ಸಕಲ \\ಪಶುಜನ ಮನಶ್ಚಕ್ಷುಃಶ್ರೋತ್ರತಿರಸ್ಕರಣಂ ಕುರು ಕುರು ಸ್ವಾಹಾ ॥

೪ ಓಂ ಹಸಂತಿ ಹಸಿತಾಲಾಪೇ ಮಾತಂಗಿ ಪರಿಚಾರಿಕೇ~।\\
ಮಮ ವಿಘ್ನಾಪದಾಂ ನಾಶಂ ಕುರು ಕುರು ಠಃಠಃಠಃ ಹುಂ ಫಟ್ ಸ್ವಾಹಾ ॥

೪ ಓಂ ನಮೋ ಭಗವತಿ ಜ್ವಾಲಾಮಾಲಿನಿ ದೇವದೇವಿ ಸರ್ವಭೂತ ಸಂಹಾರ ಕಾರಿಕೇ ಜಾತವೇದಸಿ ಜ್ವಲಂತಿ ಜ್ವಲ ಜ್ವಲ ಪ್ರಜ್ವಲ ಪ್ರಜ್ವಲ ಹ್ರಾಂ ಹ್ರೀಂ ಹ್ರೂಂ ರರ ರರ ರರರ ಹುಂ ಫಟ್ ಸ್ವಾಹಾ~। ಸಹಸ್ರಾರ ಹುಂ ಫಟ್~।\\ ಭೂರ್ಭುವಃಸುವರೋಮಿತಿ ದಿಗ್ಬಂಧಃ ॥

೪ ಸಮಸ್ತ ಪ್ರಕಟ ಗುಪ್ತ ಗುಪ್ತತರ ಸಂಪ್ರದಾಯ ಕುಲೋತ್ತೀರ್ಣ ನಿಗರ್ಭ ರಹಸ್ಯಾ\-ತಿರಹಸ್ಯ ಪರಾಪರಾತಿರಹಸ್ಯ ಯೋಗಿನೀ ದೇವತಾಭ್ಯೋ ನಮಃ ॥

೪ ಐಂ ಹ್ರಃ ಅಸ್ತ್ರಾಯ ಫಟ್ ॥

೪ ಶ್ರೀಗುರೋ ದಕ್ಷಿಣಾಮೂರ್ತೇ ಭಕ್ತಾನುಗ್ರಹಕಾರಕ~।\\
ಅನುಜ್ಞಾಂ ದೇಹಿ ಭಗವನ್ ಶ್ರೀಚಕ್ರ ಯಜನಾಯ ಮೇ ॥

೪ ಅತಿಕ್ರೂರ ಮಹಾಕಾಯ ಕಲ್ಪಾಂತದಹನೋಪಮ~।\\
ಭೈರವಾಯ ನಮಸ್ತುಭ್ಯಮನುಜ್ಞಾಂ ದಾತುಮರ್ಹಸಿ ॥

೪ ಮೂಲಶೃಂಗಾಟಕಾತ್ ಸುಷುಮ್ನಾಪಥೇನ ಜೀವಶಿವಂ ಪರಮಶಿವಪದೇ \\ಯೋಜಯಾಮಿ ಸ್ವಾಹಾ~।\\
ಯಂ ೮ ಸಂಕೋಚಶರೀರಂ ಶೋಷಯ ಶೋಷಯ ಸ್ವಾಹಾ~।\\
ರಂ ೮ ಸಂಕೋಚಶರೀರಂ ದಹ ದಹ ಪಚ ಪಚ ಸ್ವಾಹಾ~।\\
ವಂ ೮ ಪರಮಶಿವಾಮೃತಂ ವರ್ಷಯ ವರ್ಷಯ ಸ್ವಾಹಾ~।\\
ಲಂ ೮ ಶಾಂಭವಶರೀರಮುತ್ಪಾದಯೋತ್ಪಾದಯ ಸ್ವಾಹಾ~।\\
ಹಂಸಃ ಸೋಹಂ ಅವತರ ಅವತರ ಶಿವಪದಾತ್ ಜೀವಶಿವ\\ ಸುಷುಮ್ನಾಪಥೇನ ಪ್ರವಿಶ ಮೂಲಶೃಂಗಾಟಕಂ ಉಲ್ಲಸೋಲ್ಲಸ\\ ಜ್ವಲ ಜ್ವಲ ಪ್ರಜ್ವಲ ಪ್ರಜ್ವಲ ಹಂಸಃ ಸೋಹಂ ಸ್ವಾಹಾ ॥\\
೪ ಆಂ ಸೋಹಂ (ಇತಿ ತ್ರಿಃ ಹೃದಿ) ಇತಿ ಭೂತಶುದ್ಧಿಃ ॥\\
ತತಃ ಪ್ರಾಣಾನಾಯಮ್ಯ, ದೇಶಕಾಲೌ ಸಂಕೀರ್ತ್ಯ 
\section{ಲಘುಪ್ರಾಣಪ್ರತಿಷ್ಠಾ}
ಆಂ ಸೋಹಂ~। ಆಂ ಹ್ರೀಂ ಕ್ರೋಂ ಯರಲವಶಷಸಹೋಂ ॥\\
ಶ್ರೀಚಕ್ರಸ್ಯ ಪ್ರಾಣಾ ಇಹ ಪ್ರಾಣಾಃ~।
ಶ್ರೀಚಕ್ರಸ್ಯ ಜೀವ ಇಹ ಸ್ಥಿತಃ~।\\
ಶ್ರೀಚಕ್ರಸ್ಯ ಸರ್ವೇಂದ್ರಿಯಾಣಿ~।
ಶ್ರೀಚಕ್ರಸ್ಯ ವಾಙ್ಮನಸ್ತ್ವಕ್ಚಕ್ಷುಃ ಶ್ರೋತ್ರ ಜಿಹ್ವಾಘ್ರಾಣಪ್ರಾಣಾ
ಇಹೈವಾಗತ್ಯ ಸುಖಂ ಚಿರಂ ತಿಷ್ಠಂತು ಸ್ವಾಹಾ ॥

\as{೪ ಅಸು॑ನೀತೇ॒ ಪುನ॑ರ॒ಸ್ಮಾಸು॒ ಚಕ್ಷು॒: ಪುನ॑: ಪ್ರಾ॒ಣಮಿ॒ಹ ನೋ᳚ ಧೇಹಿ॒ ಭೋಗಂ᳚ । ಜ್ಯೋಕ್ಪ॑ಶ್ಯೇಮ॒ ಸೂರ್ಯ॑ಮು॒ಚ್ಚರಂ᳚ತ॒ಮನು॑ಮತೇ ಮೃ॒ಳಯಾ᳚ ನಃ ಸ್ವ॒ಸ್ತಿ ॥}


\chapter*{\center\Huge ಅಥ ನ್ಯಾಸಾಃ\\
{\LARGE ಲಘುನ್ಯಾಸಃ}}
ಓಂ ಅಥಾತ್ಮಾನಂ ಶಿವಾತ್ಮಾನಂ ಶ್ರೀರುದ್ರರೂಪಂ ಧ್ಯಾಯೇತ್ ॥
\dhyana{ಶುದ್ಧಸ್ಫಟಿಕಸಂಕಾಶಂ ತ್ರಿಣೇತ್ರಂ ಪಂಚವಕ್ತ್ರಕಂ ।\\
ಗಂಗಾಧರಂ ದಶಭುಜಂ ಸರ್ವಾಭರಣಭೂಷಿತಂ ॥

ನೀಲಗ್ರೀವಂ ಶಶಾಂಕಾಂಕಂ ನಾಗಯಜ್ಞೋಪವೀತಿನಂ ।\\
ವ್ಯಾಘ್ರಚರ್ಮೋತ್ತರೀಯಂ ಚ ವರೇಣ್ಯಮಭಯಪ್ರದಂ ॥

ಕಮಂಡಲ್ವಕ್ಷಸೂತ್ರಾಭ್ಯಾಮನ್ವಿತಂ ಶೂಲಪಾಣಿನಂ ।\\
ಜ್ವಲಂತಂ ಪಿಂಗಳಜಟಾಶಿಖಾಮುದ್ದ್ಯೋತಧಾರಿಣಂ ॥

ವೃಷಸ್ಕಂಧಸಮಾರೂಢಂ ಉಮಾದೇಹಾರ್ಧಧಾರಿಣಂ ।\\
ಅಮೃತೇನಾಪ್ಲುತಂ ಶಾಂತಂ ದಿವ್ಯಭೋಗಸಮನ್ವಿತಂ ॥

ದಿಗ್ದೇವತಾಸಮಾಯುಕ್ತಂ ಸುರಾಸುರನಮಸ್ಕೃತಂ ।\\
ನಿತ್ಯಂ ಚ ಶಾಶ್ವತಂ ಶುದ್ಧಂ ಧ್ರುವಮಕ್ಷರಮವ್ಯಯಂ ॥

ಸರ್ವವ್ಯಾಪಿನಮೀಶಾನಂ ರುದ್ರಂ ವೈ ವಿಶ್ವರೂಪಿಣಂ ।\\
ಏವಂ ಧ್ಯಾತ್ವಾ ದ್ವಿಜಃ ಸಮ್ಯಕ್ ತತೋ ಯಜನಮಾರಭೇತ್ ॥}

ಅಥಾತ್ಮನಿ ದೇವತಾಃ ಸ್ಥಾಪಯೇತ್ ॥

ಪ್ರಜನನೇ ಬ್ರಹ್ಮಾ ತಿಷ್ಠತು । ಪಾದಯೋರ್ವಿಷ್ಣುಸ್ತಿಷ್ಠತು । ಹಸ್ತಯೋರ್ಹರಸ್ತಿಷ್ಠತು । ಬಾಹ್ವೋರಿಂದ್ರಸ್ತಿಷ್ಠತು । ಜಠರೇಽಅಗ್ನಿಸ್ತಿಷ್ಠತು । ಹೃದಯೇ ಶಿವಸ್ತಿಷ್ಠತು । ಕಂಠೇ ವಸವಸ್ತಿಷ್ಠಂತು । ವಕ್ತ್ರೇ ಸರಸ್ವತೀ ತಿಷ್ಠತು । ನಾಸಿಕಯೋರ್ವಾಯುಸ್ತಿಷ್ಠತು । ನಯನಯೋಶ್ಚಂದ್ರಾದಿತ್ಯೌ ತಿಷ್ಠೇತಾಂ । ಕರ್ಣಯೋರಶ್ವಿನೌ ತಿಷ್ಠೇತಾಂ । ಲಲಾಟೇ ರುದ್ರಾಸ್ತಿಷ್ಠಂತು । ಮೂರ್ಧ್ನ್ಯಾದಿತ್ಯಾಸ್ತಿಷ್ಠಂತು । ಶಿರಸಿ ಮಹಾದೇವಸ್ತಿಷ್ಠತು । ಶಿಖಾಯಾಂ ವಾಮದೇವಾಸ್ತಿಷ್ಠತು । ಪೃಷ್ಠೇ ಪಿನಾಕೀ ತಿಷ್ಠತು । ಪುರತಃ ಶೂಲೀ ತಿಷ್ಠತು । ಪಾರ್ಶ್ವಯೋಃ ಶಿವಾಶಂಕರೌ ತಿಷ್ಠೇತಾಂ । ಸರ್ವತೋ ವಾಯುಸ್ತಿಷ್ಠತು । ತತೋ ಬಹಿಃ ಸರ್ವತೋಽಗ್ನಿರ್ಜ್ವಾಲಾಮಾಲಾಪರಿವೃತಸ್ತಿಷ್ಠತು । ಸರ್ವೇಷ್ವಂಗೇಷು ಸರ್ವಾ ದೇವತಾ ಯಥಾಸ್ಥಾನಂ ತಿಷ್ಠಂತು। ಮಾಂ ರಕ್ಷಂತು।

\as{ಅ॒ಗ್ನಿರ್ಮೇ॑} ವಾ॒ಚಿ ಶ್ರಿ॒ತಃ । ವಾಗ್ಘೃದ॑ಯೇ । ಹೃದ॑ಯಂ॒ ಮಯಿ॑ । ಅ॒ಹಮ॒ಮೃತೇ᳚ । ಅ॒ಮೃತಂ॒ ಬ್ರಹ್ಮ॑ಣಿ । \as{ವಾ॒ಯುರ್ಮೇ᳚} ಪ್ರಾ॒ಣೇ ಶ್ರಿ॒ತಃ । ಪ್ರಾ॒ಣೋ ಹೃದ॑ಯೇ । ಹೃದ॑ಯಂ॒ ಮಯಿ॑ । ಅ॒ಹಮ॒ಮೃತೇ᳚ । ಅ॒ಮೃತಂ॒ ಬ್ರಹ್ಮ॑ಣಿ । \as{ಸೂರ್ಯೋ॑} ಮೇ॒ ಚಕ್ಷು॑ಷಿ ಶ್ರಿ॒ತಃ । ಚ॒ಕ್ಷುರ್ಹೃದ॑ಯೇ । ಹೃದ॑ಯಂ॒ ಮಯಿ॑ । ಅ॒ಹಮ॒ಮೃತೇ᳚ । ಅ॒ಮೃತಂ॒ ಬ್ರಹ್ಮ॑ಣಿ । \as{ಚಂ॒ದ್ರ॒ಮಾ} ಮೇ॒ ಮನ॑ಸಿ ಶ್ರಿ॒ತಃ । ಮನೋ॒ ಹೃದ॑ಯೇ । ಹೃದ॑ಯಂ॒ ಮಯಿ॑ । ಅ॒ಹಮ॒ಮೃತೇ᳚ । ಅ॒ಮೃತಂ॒ ಬ್ರಹ್ಮ॑ಣಿ । \as{ದಿಶೋ॑} ಮೇ॒ ಶ್ರೋತ್ರೇ᳚ ಶ್ರಿ॒ತಾಃ । ಶ್ರೋತ್ರ॒ಗ್ಂ॒ ಹೃದ॑ಯೇ । ಹೃದ॑ಯಂ॒ ಮಯಿ॑ । ಅ॒ಹಮ॒ಮೃತೇ᳚ । ಅ॒ಮೃತಂ॒ ಬ್ರಹ್ಮ॑ಣಿ । \as{ಆಪೋ॑} ಮೇ॒ ರೇತ॑ಸಿ ಶ್ರಿ॒ತಾಃ । ರೇ॒ತೋ ಹೃದ॑ಯೇ । ಹೃದ॑ಯಂ॒ ಮಯಿ॑ । ಅ॒ಹಮ॒ಮೃತೇ᳚ । ಅ॒ಮೃತಂ॒ ಬ್ರಹ್ಮ॑ಣಿ । \as{ಪೃ॒ಥಿ॒ವೀ} ಮೇ ಶರೀ॑ರೇ ಶ್ರಿ॒ತಾ । ಶರೀ॑ರ॒ಗ್ಂ ಹೃದ॑ಯೇ । ಹೃದ॑ಯಂ॒ ಮಯಿ॑ । ಅ॒ಹಮ॒ಮೃತೇ᳚ । ಅ॒ಮೃತಂ॒ ಬ್ರಹ್ಮ॑ಣಿ । \as{ಓ॒ಷ॒ಧಿ॒ ವ॒ನ॒ಸ್ಪ॒ತಯೋ॑} ಮೇ॒ ಲೋಮ॑ಸು ಶ್ರಿ॒ತಾಃ । ಲೋಮಾ॑ನಿ॒ ಹೃದ॑ಯೇ । ಹೃದ॑ಯಂ॒ ಮಯಿ॑ । ಅ॒ಹಮ॒ಮೃತೇ᳚ । ಅ॒ಮೃತಂ॒ ಬ್ರಹ್ಮ॑ಣಿ । \as{ಇಂದ್ರೋ॑} ಮೇ॒ ಬಲೇ᳚ ಶ್ರಿ॒ತಃ । ಬಲ॒ಗ್ಂ॒ ಹೃದ॑ಯೇ । ಹೃದ॑ಯಂ॒ ಮಯಿ॑ । ಅ॒ಹಮ॒ಮೃತೇ᳚ । ಅ॒ಮೃತಂ॒ ಬ್ರಹ್ಮ॑ಣಿ । \as{ಪ॒ರ್ಜನ್ಯೋ॑} ಮೇ ಮೂ॒ರ್ಧ್ನಿ ಶ್ರಿ॒ತಃ । ಮೂ॒ರ್ಧಾ ಹೃದ॑ಯೇ । ಹೃದ॑ಯಂ॒ ಮಯಿ॑ । ಅ॒ಹಮ॒ಮೃತೇ᳚ । ಅ॒ಮೃತಂ॒ ಬ್ರಹ್ಮ॑ಣಿ । \as{ಈಶಾ॑ನೋ} ಮೇ ಮ॒ನ್ಯೌ ಶ್ರಿ॒ತಃ । ಮ॒ನ್ಯುರ್-ಹೃದ॑ಯೇ । ಹೃದ॑ಯಂ॒ ಮಯಿ॑ । ಅ॒ಹಮ॒ಮೃತೇ᳚ । ಅ॒ಮೃತಂ॒ ಬ್ರಹ್ಮ॑ಣಿ । \as{ಆ॒ತ್ಮಾ} ಮ॑ ಆ॒ತ್ಮನಿ॑ ಶ್ರಿ॒ತಃ । ಆ॒ತ್ಮಾ ಹೃದ॑ಯೇ । ಹೃದ॑ಯಂ॒ ಮಯಿ॑ । ಅ॒ಹಮ॒ಮೃತೇ᳚ । ಅ॒ಮೃತಂ॒ ಬ್ರಹ್ಮ॑ಣಿ । ಪುನ॑ರ್ಮ ಆ॒ತ್ಮಾ ಪುನ॒ರಾಯು॒ರಾಗಾ᳚ತ್ । ಪುನಃ॑ ಪ್ರಾ॒ಣಃ ಪು॑ನ॒ರಾಕೂ॑ತ॒ಮಾಗಾ᳚ತ್ । ವೈ॒ಶ್ವಾ॒ನ॒ರೋ ರ॒ಶ್ಮಿಭಿ॑ರ್ವಾವೃಧಾ॒ನಃ । ಅಂ॒ತಸ್ತಿ॑ಷ್ಠತ್ವ॒ಮೃತ॑ಸ್ಯ ಗೋ॒ಪಾಃ ॥

ಏವಂ ಯಥಾಲಿಂಗಮಂಗಾನಿ ಸಂಮೃಜ್ಯ, ದೇವಮಾತ್ಮಾನಂ ಚ ಪ್ರತ್ಯಾರಾಧಯೇತ್ ॥\\
\dhyana{ಆರಾಧಿತೋ ಮನುಷ್ಯೈಸ್ತ್ವಂ ಸಿದ್ಧೈರ್ದೇವಾಸುರಾದಿಭಿಃ ।\\
ಆರಾಧಯಾಮಿ ಭಕ್ತ್ಯಾ ತ್ವಾಂ ಮಾಂ ಗೃಹಾಣ ಮಹೇಶ್ವರ ॥}\\
ಆ ತ್ವಾ॑ ವಹಂತು॒ ಹರ॑ಯಃ॒ ಸಚೇ॑ತಸಃ ಶ್ವೇ॒ತೈರಶ್ವೈಃ᳚  ಸ॒ಹಕೇ॑ತು॒ಮದ್ಭಿಃ॑~। ವಾತಾ॑ಜಿರೈ॒ರ್ಬಲ॑ವದ್ಭಿ॑ರ್ಮನೋ॑ಜವೈ॒ರಾಯಾ॑ಹಿ ಶೀ॒ಘ್ರಂ ಮಮ॑ ಹ॒ವ್ಯಾಯ ಶ॒ರ್ವೋಮ್ ।\\ ಈಶಾನಮಾವಾಹಯಾಮೀತ್ಯಾವಾಹ್ಯ\\
\dhyana{ಶಂಕರಸ್ಯ ಚರಿತಂ ಕಥಾಮೃತಂ ಚಂದ್ರಶೇಖರ ಗುಣಾನುಕೀರ್ತನಮ್ ।\\
ನೀಲಕಂಠ ತವ ಪಾದಸೇವನಂ ಸಂಭವಂತು ಮಮ ಜನ್ಮಜನ್ಮನಿ ॥\\
ಸ್ವಾಮಿನ್ ಸರ್ವಜಗನ್ನಾಥ ಯಾವತ್ಪೂಜಾವಸಾನಕಮ್ ।\\
ತಾವತ್ತ್ವಂ ಪ್ರೀತಿ ಭಾವೇನ ಬಿಂಬೇಽಸ್ಮಿನ್ ಸನ್ನಿಧಿಂ ಕುರು ॥}
\section{ಮಾತೃಕಾಸರಸ್ವತೀನ್ಯಾಸಃ}
\addcontentsline{toc}{section}{ಮಾತೃಕಾಸರಸ್ವತೀನ್ಯಾಸಃ}
ಅಸ್ಯ ಶ್ರೀಮಾತೃಕಾಸರಸ್ವತೀ ನ್ಯಾಸಮಂತ್ರಸ್ಯ ಬ್ರಹ್ಮಣೇ ಋಷಯೇ ನಮಃ (ಶಿರಸಿ) ಗಾಯತ್ರೀ ಛಂದಸೇ ನಮಃ(ಮುಖೇ)~। ಮಾತೃಕಾ ಸರಸ್ವತೀ ದೇವತಾಯೈ ನಮಃ (ಹೃದಯೇ)~। ಹಲ್ಭ್ಯೋ ಬೀಜೇಭ್ಯೋ ನಮಃ (ಗುಹ್ಯೇ)~। ಸ್ವರೇಭ್ಯಃ ಶಕ್ತಿಭ್ಯೋ ನಮಃ (ಪಾದಯೋ)~। ಬಿಂದುಭ್ಯಃ ಕೀಲಕೇಭ್ಯೋ ನಮಃ (ನಾಭೌ)~। ಶ್ರೀವಿದ್ಯಾಂಗತ್ವೇನ ನ್ಯಾಸೇ ವಿನಿಯೋಗಾಯ ನಮಃ(ಸರ್ವಾಂಗೇ)~॥\\
ಅಂ ಆಂ ಇಂ ಈಂ ಉಂ ಊಂ ಋಂ ೠಂ ಲೃಂ ಲೄಂ ಏಂ ಐಂ ಓಂ ಔಂ ಅಂ ಅಃ ಕಂಖಂಗಂಘಂಙಂ ಚಂಛಂಜಂಝಂಞಂ ಟಂಠಂಡಂಢಂಣಂ ತಂಥಂದಂಧಂನಂ ಪಂಫಂಬಂಭಂಮಂ ಯಂರಂಲಂವಂ ಶಂಷಂಸಂಹಂಳಂಕ್ಷಂ~। (ಇತಿ ಅಂಜಲಿನಾ ತ್ರಿಃ ವ್ಯಾಪಕಂ ನ್ಯಸ್ಯ)\\
\as{೪ ಐಂಕ್ಲೀಂಸೌಃ} ಅಂ ಕಂಖಂಗಂಘಂಙಂ ಆಂ ಅಂಗುಷ್ಠಾಭ್ಯಾಂ ನಮಃ ।\\
\as{೭} ಇಂ ಚಂಛಂಜಂಝಂಞಂ ಈಂ ತರ್ಜನೀಭ್ಯಾಂ ನಮಃ ।\\
\as{೭} ಉಂ ಟಂಠಂಡಂಢಂಣಂ ಊಂ ಮಧ್ಯಮಾಭ್ಯಾಂ ನಮಃ ।\\
\as{೭} ಏಂ ತಂಥಂದಂಧಂನಂ ಐಂ ಅನಾಮಿಕಾಭ್ಯಾಂ ನಮಃ ।\\
\as{೭} ಓಂ ಪಂಫಂಬಂಭಂಮಂ ಔಂ ಕನಿಷ್ಠಿಕಾಭ್ಯಾಂ ನಮಃ ।\\
\as{೭} ಅಂ ಯಂರಂಲಂವಂಶಂಷಂಸಂಹಂಳಂಕ್ಷಂ ಅಃ ಕರತಲಕರಪೃಷ್ಠಾಭ್ಯಾಂ\\ ನಮಃ ।
(ಏವಮೇವಾಂಗನ್ಯಾಸಂ ವಿಧಾಯ ಧ್ಯಾಯೇತ್ )
\newpage
\dhyana{ಪಂಚಾಶದ್ವರ್ಣಭೇದೈರ್ವಿಹಿತವದನದೋಃಪಾದಯುಕ್ಕುಕ್ಷಿವಕ್ಷೋ\\
ದೇಶಾಂ ಭಾಸ್ವತ್ಕಪರ್ದಾಕಲಿತಶಶಿಕಲಾಮಿಂದುಕುಂದಾವದಾತಾಂ~।\\
ಅಕ್ಷಸ್ರಕ್ಕುಂಭಚಿಂತಾಲಿಖಿತವರಕರಾಂ ತ್ರೀಕ್ಷಣಾಂ ಪದ್ಮಸಂಸ್ಥಾಂ\\
ಅಚ್ಛಾಕಲ್ಪಾಮತುಚ್ಛಸ್ತನಜಘನಭರಾಂ ಭಾರತೀಂ ತಾಂ ನಮಾಮಿ ॥}\\
ಲಮಿತ್ಯಾದಿನಾ ಪಂಚೋಪಚಾರ ಪೂಜಾ~॥\\
\as{ಓಂ ಐಂಹ್ರೀಂಶ್ರೀಂ ಐಂಕ್ಲೀಂಸೌಃ ಅಂ} ನಮಃ ಹಂಸಃ~।(ಶಿರಸಿ)\\
\as{೭ ಆಂ} ನಮಃ ಹಂಸಃ~।(ಮುಖವೃತ್ತೇ)\\
\as{೭ ಇಂ} ನಮಃ ಹಂಸಃ~।(ದಕ್ಷನೇತ್ರೇ)\\
\as{೭ ಈಂ} ನಮಃ ಹಂಸಃ~।(ವಾಮನೇತ್ರೇ)\\
\as{೭ ಉಂ} ನಮಃ ಹಂಸಃ~।(ದಕ್ಷಕರ್ಣೇ)\\
\as{೭ ಊಂ} ನಮಃ ಹಂಸಃ~।(ವಾಮಕರ್ಣೇ)\\
\as{೭ ಋಂ} ನಮಃ ಹಂಸಃ~।(ದಕ್ಷನಾಸಾಯಾಂ)\\
\as{೭ ೠಂ} ನಮಃ ಹಂಸಃ~।(ವಾಮನಾಸಾಯಾಂ)\\
\as{೭ ಲೃಂ} ನಮಃ ಹಂಸಃ~।(ದಕ್ಷಗಂಡೇ)\\
\as{೭ ಲೄಂ} ನಮಃ ಹಂಸಃ~।(ವಾಮಗಂಡೇ)\\
\as{೭ ಏಂ} ನಮಃ ಹಂಸಃ~।(ಊರ್ಧ್ವೋಷ್ಠೇ)\\
\as{೭ ಐಂ} ನಮಃ ಹಂಸಃ~।(ಅಧರೋಷ್ಠೇ)\\
\as{೭ ಓಂ} ನಮಃ ಹಂಸಃ~।(ಊರ್ಧ್ವದಂತಪಂಕ್ತೌ)\\
\as{೭ ಔಂ} ನಮಃ ಹಂಸಃ~।(ಅಧೋದಂತಪಂಕ್ತೌ)\\
\as{೭ ಅಂ} ನಮಃ ಹಂಸಃ~।(ಜಿಹ್ವಾಯಾಂ)\\
\as{೭ ಅಃ} ನಮಃ ಹಂಸಃ~।(ಕಂಠೇ)\\
\as{೭ ಕಂ} ನಮಃ ಹಂಸಃ~।(ದಕ್ಷ ಬಾಹುಮೂಲೇ)\\
\as{೭ ಖಂ} ನಮಃ ಹಂಸಃ~।(ದಕ್ಷಕೂರ್ಪರೇ)\\
\as{೭ ಗಂ} ನಮಃ ಹಂಸಃ~।(ದಕ್ಷಮಣಿಬಂಧೇ)\\
\as{೭ ಘಂ} ನಮಃ ಹಂಸಃ~।(ದಕ್ಷಕರಾಂಗುಲಿಮೂಲೇ)\\
\as{೭ ಙಂ} ನಮಃ ಹಂಸಃ~।(ದಕ್ಷಕರಾಂಗುಲ್ಯಗ್ರೇ)\\
\as{೭ ಚಂ} ನಮಃ ಹಂಸಃ~।(ವಾಮಬಾಹುಮೂಲೇ)\\
\as{೭ ಛಂ} ನಮಃ ಹಂಸಃ~।(ವಾಮಕೂರ್ಪರೇ)\\
\as{೭ ಜಂ} ನಮಃ ಹಂಸಃ~।(ವಾಮಮಣಿಬಂಧೇ)\\
\as{೭ ಝಂ} ನಮಃ ಹಂಸಃ~।(ವಾಮಕರಾಂಗುಲಿಮೂಲೇ)\\
\as{೭ ಞಂ} ನಮಃ ಹಂಸಃ~।(ವಾಮಕರಾಂಗುಲ್ಯಗ್ರೇ)\\
\as{೭ ಟಂ} ನಮಃ ಹಂಸಃ~।(ದಕ್ಷೋರುಮೂಲೇ)\\
\as{೭ ಠಂ} ನಮಃ ಹಂಸಃ~।(ದಕ್ಷಜಾನುನಿ)\\
\as{೭ ಡಂ} ನಮಃ ಹಂಸಃ~।(ದಕ್ಷಗುಲ್ಫೇ)\\
\as{೭ ಢಂ} ನಮಃ ಹಂಸಃ~।(ದಕ್ಷಪಾದಾಂಗುಲಿಮೂಲೇ)\\
\as{೭ ಣಂ} ನಮಃ ಹಂಸಃ~।(ದಕ್ಷಪಾದಾಂಗುಲ್ಯಗ್ರೇ)\\
\as{೭ ತಂ} ನಮಃ ಹಂಸಃ~।(ವಾಮೋರುಮೂಲೇ)\\
\as{೭ ಥಂ} ನಮಃ ಹಂಸಃ~।(ವಾಮಜಾನುನಿ)\\
\as{೭ ದಂ} ನಮಃ ಹಂಸಃ~।(ವಾಮಗುಲ್ಫೇ)\\
\as{೭ ಧಂ} ನಮಃ ಹಂಸಃ~।(ವಾಮಪಾದಾಂಗುಲಿಮೂಲೇ)\\
\as{೭ ನಂ} ನಮಃ ಹಂಸಃ~।(ವಾಮಪಾದಾಂಗುಲ್ಯಗ್ರೇ)\\
\as{೭ ಪಂ} ನಮಃ ಹಂಸಃ~।(ದಕ್ಷಪಾರ್ಶ್ವೇ)\\
\as{೭ ಫಂ} ನಮಃ ಹಂಸಃ~।(ವಾಮಪಾರ್ಶ್ವೇ)\\
\as{೭ ಬಂ} ನಮಃ ಹಂಸಃ~।(ಪೃಷ್ಠೇ)\\
\as{೭ ಭಂ} ನಮಃ ಹಂಸಃ~।(ನಾಭೌ)\\
\as{೭ ಮಂ} ನಮಃ ಹಂಸಃ~।(ಜಠರೇ)\\
\as{೭ ಯಂ} ನಮಃ ಹಂಸಃ~।(ಹೃದಿ )\\
\as{೭ ರಂ} ನಮಃ ಹಂಸಃ~।(ದಕ್ಷಾಂಸೇ)\\
\as{೭ ಲಂ} ನಮಃ ಹಂಸಃ~।(ಕಕುದಿ)\\
\as{೭ ವಂ} ನಮಃ ಹಂಸಃ~।(ವಾಮಾಂಸೇ)\\
\as{೭ ಶಂ} ನಮಃ ಹಂಸಃ~।(ಹೃದಯಾದಿ ದಕ್ಷ ಕರಾಂಗುಲ್ಯಂತಂ )\\
\as{೭ ಷಂ} ನಮಃ ಹಂಸಃ~।(ಹೃದಯಾದಿ ವಾಮ ಕರಾಂಗುಲ್ಯಂತಂ )\\
\as{೭ ಸಂ} ನಮಃ ಹಂಸಃ~।(ಹೃದಯಾದಿ ದಕ್ಷ ಪಾದಾಂತಂ )\\
\as{೭ ಹಂ} ನಮಃ ಹಂಸಃ~।(ಹೃದಯಾದಿ ವಾಮ ಪಾದಾಂತಂ )\\
\as{೭ ಳಂ} ನಮಃ ಹಂಸಃ~।(ಕಟ್ಯಾದಿ ಪಾದಪರ್ಯಂತಂ)\\
\as{೭ ಕ್ಷಂ} ನಮಃ ಹಂಸಃ~।(ಕಟ್ಯಾದಿ ಶಿರಃಪರ್ಯಂತಂ )\\
\dhyana{ಆಧಾರೇ ಲಿಂಗನಾಭೌ ಹೃದಯಸರಸಿಜೇ ತಾಲುಮೂಲೇ ಲಲಾಟೇ\\
ದ್ವೇ ಪತ್ರೇ ಷೋಡಶಾರೇ ದ್ವಿದಶದಶದಲೇ ದ್ವಾದಶಾರ್ಧೇ ಚತುಷ್ಕೇ~।\\
ವಾಸಾಂತೇ ಬಾಲಮಧ್ಯೇ ಡಫಕಠಸಹಿತೇ ಕಂಠದೇಶೇ ಸ್ವರಾಣಾಂ\\
ಹಂ ಕ್ಷಂ ತತ್ವಾರ್ಥಯುಕ್ತಂ ಸಕಲದಲಗತಂ ವರ್ಣರೂಪಂ ನಮಾಮಿ ॥}

\as{೭ ಅಂ} ನಮಃ ಹಂಸಃ, \as{ಆಂ} ನಮಃ ಹಂಸಃ+ + +\as{ಅಃ} ನಮಃ ಹಂಸಃ॥ಕಂಠೇ\\
\as{೭ ಕಂ} ನಮಃ ಹಂಸಃ + + \as{ಠಂ} ನಮಃ ಹಂಸಃ ॥ಹೃದಯೇ\\
\as{೭ ಡಂ} ನಮಃ ಹಂಸಃ+ + \as{ಫಂ} ನಮಃ ಹಂಸಃ ॥ನಾಭೌ\\
\as{೭ ಬಂ} ನಮಃ ಹಂಸಃ + + \as{ಲಂ} ನಮಃ ಹಂಸಃ ॥ಗುಹ್ಯೇ\\
\as{೭ ವಂ} ನಮಃ ಹಂಸಃ + + \as{ಸಂ} ನಮಃ ಹಂಸಃ ॥ಆಧಾರೇ\\
\as{೭ ಹಂ} ನಮಃ ಹಂಸಃ, \as{ಕ್ಷಂ} ನಮಃ ಹಂಸಃ ॥ಭ್ರೂಮಧ್ಯೇ\\
\as{೭ ಅಂ} ನಮಃ ಹಂಸಃ + +\as{ಕ್ಷಂ} ನಮಃ ಹಂಸಃ॥ಸಹಸ್ರದಲಕಮಲೇ\\
 ಪೂರ್ವವತ್ ಉತ್ತರನ್ಯಾಸಃ ॥
\section{ಕರಶುದ್ಧಿನ್ಯಾಸಃ}
\addcontentsline{toc}{section}{ಕರಶುದ್ಧಿನ್ಯಾಸಃ}
\as{೪ ಅಂ} ನಮಃ ।(ದಕ್ಷಕರಮಧ್ಯೇ)\\
\as{೪ ಆಂ} ನಮಃ ।(ದಕ್ಷಕರಪೃಷ್ಠೇ)\\
\as{೪ ಸೌಃ} ನಮಃ ।(ದಕ್ಷಕರಪಾರ್ಶ್ವಯೋಃ)\\
\as{೪ ಅಂ} ನಮಃ ।(ವಾಮಕರಮಧ್ಯೇ)\\
\as{೪ ಆಂ} ನಮಃ ।(ವಾಮಕರಪೃಷ್ಠೇ)\\
\as{೪ ಸೌಃ} ನಮಃ ।(ವಾಮಕರಪಾರ್ಶ್ವಯೋಃ)\\
\as{೪ ಅಂ} ನಮಃ ।(ಮಧ್ಯಮಯೋಃ)\\
\as{೪ ಆಂ} ನಮಃ ।(ಅನಾಮಿಕಯೋಃ)\\
\as{೪ ಸೌಃ} ನಮಃ ।(ಕನಿಷ್ಠಿಕಯೋಃ)\\
\as{೪ ಅಂ} ನಮಃ ।(ಅಂಗುಷ್ಠಯೋಃ)\\
\as{೪ ಆಂ} ನಮಃ ।(ತರ್ಜನ್ಯೋಃ)\\
\as{೪ ಸೌಃ} ನಮಃ ।(ಉಭಯ ಕರತಲ ಕರಪೃಷ್ಠಯೋಃ)
\section{ಆತ್ಮರಕ್ಷಾನ್ಯಾಸಃ}
\addcontentsline{toc}{section}{ಆತ್ಮರಕ್ಷಾನ್ಯಾಸಃ}
\as{ಓಂ ಐಂಹ್ರೀಂಶ್ರೀಂ ಐಂಕ್ಲೀಂಸೌಃ ।} ಶ್ರೀಮಹಾತ್ರಿಪುರಸುಂದರಿ ಆತ್ಮಾನಂ ರಕ್ಷರಕ್ಷ॥\\ಇತಿ ಹೃದಿ ಅಂಜಲಿಸಮರ್ಪಣಂ~।
\section{ಬಾಲಾಷಡಂಗನ್ಯಾಸಃ}
\addcontentsline{toc}{section}{ಬಾಲಾಷಡಂಗನ್ಯಾಸಃ}
\as{೪ ಐಂ ।} ಹೃದಯಾಯ ನಮಃ\\
\as{೪ ಕ್ಲೀಂ ।} ಶಿರಸೇ ಸ್ವಾಹಾ\\
\as{೪ ಸೌಃ  ।}ಶಿಖಾಯೈ ವಷಟ್\\
\as{೪ ಐಂ ।} ಕವಚಾಯ ಹುಂ\\
\as{೪ ಕ್ಲೀಂ । }ನೇತ್ರತ್ರಯಾಯ ವೌಷಟ್\\
\as{೪ ಸೌಃ ।} ಅಸ್ತ್ರಾಯ ಫಟ್ ॥
\section{ಚತುರಾಸನನ್ಯಾಸಃ}
\addcontentsline{toc}{section}{ಚತುರಾಸನನ್ಯಾಸಃ}
{\bfseries ೪ ಹ್ರೀಂಕ್ಲೀಂಸೌಃ ।}ದೇವ್ಯಾತ್ಮಾಸನಾಯ ನಮಃ। ಪಾದಯೋಃ\\
{\bfseries ೪ ಹೈಂ ಹ್‌ಕ್ಲೀಂ ಹ್ಸೌಃ ।} ಶ್ರೀಚಕ್ರಾಸನಾಯ ನಮಃ। ಜಾನುನೋಃ\\
{\bfseries ೪ ಹ್‌ಸೈಂ ಹ್‌ಸ್‌ಕ್ಲೀಂ ಹ್‌ಸ್ಸೌಃ ।} ಸರ್ವಮಂತ್ರಾಸನಾಯ ನಮಃ ।\\ ಊರುಮೂಲಯೋಃ\\
{\bfseries ೪ ಹ್ರೀಂಕ್ಲೀಂಬ್ಲೇಂ ।} ಸಾಧ್ಯಸಿದ್ಧಾಸನಾಯ ನಮಃ। ಆಧಾರೇ
\section{ವಶಿನ್ಯಾದಿವಾಗ್ದೇವತಾನ್ಯಾಸಃ}
\addcontentsline{toc}{section}{ವಶಿನ್ಯಾದಿವಾಗ್ದೇವತಾನ್ಯಾಸಃ}
\as{ಓಂ ಐಂಹ್ರೀಂಶ್ರೀಂ ಅಂ ಆಂ ಇಂ ಈಂ+ + ಅಂ ಅಃ~। ರ್ಬ್ಲೂಂ॥}\\ ವಶಿನೀ ವಾಗ್ದೇವತಾಯೈ ನಮಃ। ಶಿರಸಿ\\
\as{೪ ಕಂ ಖಂ ಗಂ ಘಂ ಙಂ। ಕ್‌ಲ್‌ಹ್ರೀಂ ॥}\\ ಕಾಮೇಶ್ವರೀ ವಾಗ್ದೇವತಾಯೈ ನಮಃ। ಲಲಾಟೇ\\
\as{೪ ಚಂ ಛಂ ಜಂ ಝಂ ಞಂ। ನ್‌ವ್ಲೀಂ ॥}\\ ಮೋದಿನೀ ವಾಗ್ದೇವತಾಯೈ ನಮಃ। ಭ್ರೂಮಧ್ಯೇ\\
\as{೪ ಟಂ ಠಂ ಡಂ ಢಂ ಣಂ। ಯ್ಲೂಂ ॥}\\ ವಿಮಲಾ ವಾಗ್ದೇವತಾಯೈ ನಮಃ । ಕಂಠೇ\\
\as{೪ ತಂ ಥಂ ದಂ ಧಂ ನಂ~। ಜ್‌ಮ್ರೀಂ ॥}\\ ಅರುಣಾ ವಾಗ್ದೇವತಾಯೈ ನಮಃ। ಹೃದಯೇ\\
\as{೪ ಪಂ ಫಂ ಬಂ ಭಂ ಮಂ। ಹ್‌ಸ್‌ಲ್‌ವ್ಯೂಂ ॥}\\ ಜಯಿನೀ ವಾಗ್ದೇವತಾಯೈ ನಮಃ। ನಾಭೌ\\
\as{೪ ಯಂ ರಂ ಲಂ ವಂ~। ಝ್‌ಮ್‌ರ್ಯೂಂ ॥}\\ ಸರ್ವೇಶ್ವರೀ ವಾಗ್ದೇವತಾಯೈ ನಮಃ। ಗುಹ್ಯೇ\\
\as{೪ ಶಂ ಷಂ ಸಂ ಹಂ ಳಂ ಕ್ಷಂ~। ಕ್ಷ್‌ಮ್ರೀಂ ॥}\\ ಕೌಲಿನೀ ವಾಗ್ದೇವತಾಯೈ ನಮಃ। ಆಧಾರೇ
\section{ಬಹಿಶ್ಚಕ್ರನ್ಯಾಸಃ}
\addcontentsline{toc}{section}{ಬಹಿಶ್ಚಕ್ರನ್ಯಾಸಃ}
{\bfseries ೪ ಅಂಆಂಸೌಃ ।} ಚತುರಶ್ರ ತ್ರಯಾತ್ಮಕ ತ್ರೈಲೋಕ್ಯಮೋಹನ ಚಕ್ರಾಧಿಷ್ಠಾತ್ರ್ಯೈ ಅಣಿಮಾದ್ಯಷ್ಟಾವಿಂಶತಿ ಶಕ್ತಿಸಹಿತ ಪ್ರಕಟಯೋಗಿನೀ ರೂಪಾಯೈ  ತ್ರಿಪುರಾದೇವ್ಯೈ ನಮಃ॥(ಪಾದಯೋಃ)\\
{\bfseries ೪ ಐಂಕ್ಲೀಂಸೌಃ ।} ಷೋಡಶದಲ ಪದ್ಮಾತ್ಮಕ ಸರ್ವಾಶಾ ಪರಿಪೂರಕ ಚಕ್ರಾಧಿಷ್ಠಾತ್ರ್ಯೈ ಕಾಮಾಕರ್ಷಣ್ಯಾದಿ ಷೋಡಶಶಕ್ತಿ ಸಹಿತ ಗುಪ್ತಯೋಗಿನೀ ರೂಪಾಯೈ ತ್ರಿಪುರೇಶೀ ದೇವ್ಯೈ ನಮಃ॥(ಜಾನುನೋಃ)\\
{\bfseries ೪ ಹ್ರೀಂಕ್ಲೀಂಸೌಃ ।} ಅಷ್ಟದಲ ಪದ್ಮಾತ್ಮಕ ಸರ್ವ ಸಂಕ್ಷೋಭಣ ಚಕ್ರಾಧಿಷ್ಠಾತ್ರ್ಯೈ ಅನಂಗ ಕುಸುಮಾದ್ಯಷ್ಟಶಕ್ತಿ ಸಹಿತ ಗುಪ್ತತರ ಯೋಗಿನೀ ರೂಪಾಯೈ ತ್ರಿಪುರಸುಂದರೀ ದೇವ್ಯೈ ನಮಃ॥(ಊರುಮೂಲಯೋಃ)\\
{\bfseries ೪ ಹೈಂಹ್‌ಕ್ಲೀಂಹ್ಸೌಃ ।} ಚತುರ್ದಶಾರಾತ್ಮಕ ಸರ್ವಸೌಭಾಗ್ಯದಾಯಕ ಚಕ್ರಾಧಿಷ್ಠಾತ್ರ್ಯೈ ಸರ್ವ ಸಂಕ್ಷೋಭಿಣ್ಯಾದಿ ಚತುರ್ದಶ ಶಕ್ತಿಸಹಿತ ಸಂಪ್ರದಾಯ ಯೋಗಿನೀ ರೂಪಾಯೈ  ತ್ರಿಪುರವಾಸಿನೀ ದೇವ್ಯೈ ನಮಃ~॥(ನಾಭೌ)\\
{\bfseries ೪ ಹ್‌ಸೈಂಹ್‌ಸ್‌ಕ್ಲೀಂಹ್‌ಸ್ಸೌಃ ।} ಬಹಿರ್ದಶಾರಾತ್ಮಕ ಸರ್ವಾರ್ಥ ಸಾಧಕ ಚಕ್ರಾಧಿಷ್ಠಾತ್ರ್ಯೈ ಸರ್ವಸಿದ್ಧಿಪ್ರದಾದಿ ದಶಶಕ್ತಿ ಸಹಿತ ಕುಲೋತ್ತೀರ್ಣ ಯೋಗಿನೀ ರೂಪಾಯೈ  ತ್ರಿಪುರಾಶ್ರೀ ದೇವ್ಯೈ ನಮಃ॥(ಹೃದಯೇ)\\
{\bfseries ೪ ಹ್ರೀಂ ಕ್ಲೀಂ ಬ್ಲೇಂ ।} ಅಂತರ್ದಶಾರಾತ್ಮಕ ಸರ್ವರಕ್ಷಾಕರ ಚಕ್ರಾಧಿಷ್ಠಾತ್ರ್ಯೈ ಸರ್ವಜ್ಞಾದಿ ದಶಶಕ್ತಿ ಸಹಿತ ನಿಗರ್ಭ ಯೋಗಿನೀ ರೂಪಾಯೈ  ತ್ರಿಪುರಮಾಲಿನೀ ದೇವ್ಯೈ ನಮಃ॥(ಕಂಠೇ)\\
{\bfseries ೪ ಹ್ರೀಂ ಶ್ರೀಂ ಸೌಃ ।} ಅಷ್ಟಾರಾತ್ಮಕ ಸರ್ವರೋಗಹರ ಚಕ್ರಾಧಿಷ್ಠಾತ್ರ್ಯೈ ವಶಿನ್ಯಾದ್ಯಷ್ಟ ಶಕ್ತಿಸಹಿತ ರಹಸ್ಯ ಯೋಗಿನೀ ರೂಪಾಯೈ  ತ್ರಿಪುರಾಸಿದ್ಧಾ ದೇವ್ಯೈ ನಮಃ॥(ಮುಖೇ)\\
{\bfseries ೪ ಹ್‌ಸ್‌ರೈಂ ಹ್‌ಸ್‌ಕ್ಲ್ರೀಂ ಹ್‌ಸ್‌ರ್ಸೌಃ ।} ತ್ರಿಕೋಣಾತ್ಮಕ ಸರ್ವಸಿದ್ಧಿ ಪ್ರದಚಕ್ರಾಧಿಷ್ಠಾತ್ರ್ಯೈ ಕಾಮೇಶ್ವರ್ಯಾದಿ ತ್ರಿಶಕ್ತಿ ಸಹಿತ ಅತಿರಹಸ್ಯಯೋಗಿನೀ ರೂಪಾಯೈ  ತ್ರಿಪುರಾಂಬಾ ದೇವ್ಯೈ ನಮಃ~॥(ನೇತ್ರಯೋಃ)\\
{\bfseries ೪ ೧೫॥} ಬಿಂದ್ವಾತ್ಮಕ ಸರ್ವಾನಂದಮಯ ಚಕ್ರಾಧಿಷ್ಠಾತ್ರ್ಯೈ ಷಡಂಗಾಯುಧ ದಶಶಕ್ತಿ ಸಹಿತ ಪರಾಪರಾತಿ ರಹಸ್ಯಯೋಗಿನೀ ರೂಪಾಯೈ ಮಹಾ ತ್ರಿಪುರಸುಂದರೀ ದೇವ್ಯೈ ನಮಃ~॥(ಶಿರಸಿ)
\section{ಅಂತಶ್ಚಕ್ರನ್ಯಾಸಃ}
\addcontentsline{toc}{section}{ಅಂತಶ್ಚಕ್ರನ್ಯಾಸಃ}
{\bfseries ೪ ಅಂಆಂಸೌಃ।} ಚತುರಶ್ರ ತ್ರಯಾತ್ಮಕ ತ್ರೈಲೋಕ್ಯಮೋಹನ ಚಕ್ರಾಧಿಷ್ಠಾತ್ರ್ಯೈ ಅಣಿಮಾದ್ಯಷ್ಟಾವಿಂಶತಿ ಶಕ್ತಿ ಸಹಿತ ಪ್ರಕಟಯೋಗಿನೀ ರೂಪಾಯೈ  ತ್ರಿಪುರಾ ದೇವ್ಯೈ ನಮಃ॥(ಅಕುಲಸಹಸ್ರಾರೇ)\\
{\bfseries ೪ ಐಂಕ್ಲೀಂಸೌಃ ।} ಷೋಡಶ ದಲ ಪದ್ಮಾತ್ಮಕ ಸರ್ವಾಶಾಪರಿಪೂರಕ ಚಕ್ರಾಧಿಷ್ಠಾತ್ರ್ಯೈ ಕಾಮಾಕರ್ಷಣ್ಯಾದಿ ಷೋಡಶ ಶಕ್ತಿ ಸಹಿತ ಗುಪ್ತಯೋಗಿನೀ ರೂಪಾಯೈ  ತ್ರಿಪುರೇಶೀ ದೇವ್ಯೈ ನಮಃ॥(ವಿಷುಚಕ್ರೇ)\\
{\bfseries ೪ ಹ್ರೀಂ ಕ್ಲೀಂ ಸೌಃ ।} ಅಷ್ಟದಲ ಪದ್ಮಾತ್ಮಕ ಸರ್ವಸಂಕ್ಷೋಭಣ ಚಕ್ರಾಧಿಷ್ಠಾತ್ರ್ಯೈ ಅನಂಗ ಕುಸುಮಾದ್ಯಷ್ಟ ಶಕ್ತಿಸಹಿತ ಗುಪ್ತತರ ಯೋಗಿನೀ ರೂಪಾಯೈ  ತ್ರಿಪುರಸುಂದರೀ ದೇವ್ಯೈ ನಮಃ॥(ಮೂಲಾಧಾರೇ)\\
{\bfseries ೪ ಹೈಂ ಹ್‌ಕ್ಲೀಂ ಹ್ಸೌಃ ।} ಚತುರ್ದಶಾರಾತ್ಮಕ ಸರ್ವಸೌಭಾಗ್ಯದಾಯಕ ಚಕ್ರಾಧಿಷ್ಠಾತ್ರ್ಯೈ ಸರ್ವಸಂಕ್ಷೋಭಿಣ್ಯಾದಿ ಚತುರ್ದಶ ಶಕ್ತಿಸಹಿತ ಸಂಪ್ರದಾಯ ಯೋಗಿನೀ ರೂಪಾಯೈ  ತ್ರಿಪುರವಾಸಿನೀ ದೇವ್ಯೈ ನಮಃ॥(ಸ್ವಾಧಿಷ್ಠಾನೇ)\\
{\bfseries ೪ ಹ್‌ಸೈಂ ಹ್‌ಸ್‌ಕ್ಲೀಂ ಹ್‌ಸ್ಸೌಃ।} ಬಹಿರ್ದಶಾರಾತ್ಮಕ ಸರ್ವಾರ್ಥ ಸಾಧಕ ಚಕ್ರಾಧಿಷ್ಠಾತ್ರ್ಯೈ ಸರ್ವಸಿದ್ಧಿ ಪ್ರದಾದಿ ದಶ ಶಕ್ತಿ ಸಹಿತ ಕುಲೋತ್ತೀರ್ಣ ಯೋಗಿನೀ ರೂಪಾಯೈ  ತ್ರಿಪುರಾಶ್ರೀ ದೇವ್ಯೈ ನಮಃ॥(ಮಣಿಪೂರೇ)\\
{\bfseries ೪ ಹ್ರೀಂ ಕ್ಲೀಂ ಬ್ಲೇಂ ।} ಅಂತರ್ದಶಾರಾತ್ಮಕ ಸರ್ವರಕ್ಷಾಕರ ಚಕ್ರಾಧಿಷ್ಠಾತ್ರ್ಯೈ ಸರ್ವಜ್ಞಾದಿ ದಶಶಕ್ತಿ ಸಹಿತ ನಿಗರ್ಭಯೋಗಿನೀ ರೂಪಾಯೈ  ತ್ರಿಪುರಮಾಲಿನೀ ದೇವ್ಯೈ ನಮಃ॥(ಅನಾಹತೇ)\\
{\bfseries ೪ ಹ್ರೀಂ ಶ್ರೀಂ ಸೌಃ ।} ಅಷ್ಟಾರಾತ್ಮಕ ಸರ್ವರೋಗಹರ ಚಕ್ರಾಧಿಷ್ಠಾತ್ರ್ಯೈ ವಶಿನ್ಯಾದ್ಯಷ್ಟ ಶಕ್ತಿಸಹಿತ ರಹಸ್ಯ ಯೋಗಿನೀ ರೂಪಾಯೈ  ತ್ರಿಪುರಾಸಿದ್ಧಾ ದೇವ್ಯೈ ನಮಃ॥(ವಿಶುದ್ಧೌ)\\
{\bfseries ೪ ಹ್‌ಸ್‌ರೈಂ ಹ್‌ಸ್‌ಕ್ಲ್ರೀಂ ಹ್‌ಸ್‌ರ್ಸೌಃ ।} ತ್ರಿಕೋಣಾತ್ಮಕ ಸರ್ವಸಿದ್ಧಿಪ್ರದ ಚಕ್ರಾಧಿಷ್ಠಾತ್ರ್ಯೈ ಕಾಮೇಶ್ವರ್ಯಾದಿ ತ್ರಿಶಕ್ತಿಸಹಿತ ಅತಿರಹಸ್ಯಯೋಗಿನೀ ರೂಪಾಯೈ  ತ್ರಿಪುರಾಂಬಾ ದೇವ್ಯೈ ನಮಃ॥(ಲಂಬಿಕಾಗ್ರೇ)\\
{\bfseries ೪ ೧೫~॥}ಬಿಂದ್ವಾತ್ಮಕ ಸರ್ವಾನಂದಮಯ ಚಕ್ರಾಧಿಷ್ಠಾತ್ರ್ಯೈ ಷಡಂಗಾಯುಧ ದಶ ಶಕ್ತಿಸಹಿತ ಪರಾಪರಾತಿರಹಸ್ಯ ಯೋಗಿನೀ ರೂಪಾಯೈ ಮಹಾ ತ್ರಿಪುರಸುಂದರೀ ದೇವ್ಯೈ ನಮಃ॥(ಆಜ್ಞಾಯಾಂ)\\
{\bfseries ೪ ಅಂ ಆಂ ಸೌಃ} ನಮಃ(ಬಿಂದೌ)\\
{\bfseries ೪ ಐಂ ಕ್ಲೀಂ ಸೌಃ} ನಮಃ(ಅರ್ಧಚಂದ್ರೇ)\\
{\bfseries ೪ ಹ್ರೀಂ ಕ್ಲೀಂ ಸೌಃ} ನಮಃ(ರೋಧಿನ್ಯಾಂ)\\
{\bfseries ೪ ಹೈಂ ಹ್‌ಕ್ಲೀಂ ಹ್ಸೌಃ} ನಮಃ(ನಾದೇ)\\
{\bfseries ೪ ಹ್‌ಸೈಂ ಹ್‌ಸ್‌ಕ್ಲೀಂ ಹ್ಸ್ಸೌಃ} ನಮಃ(ನಾದಾಂತೇ)\\
{\bfseries ೪ ಹ್ರೀಂ ಕ್ಲೀಂ ಬ್ಲೇಂ} ನಮಃ (ಶಕ್ತೌ)\\
{\bfseries ೪ ಹ್ರೀಂ ಶ್ರೀಂ ಸೌಃ} ನಮಃ (ವ್ಯಾಪಿಕಾಯಾಂ)\\
{\bfseries ೪ ಹ್‌ಸ್‌ರೈಂ ಹ್‌ಸ್‌ಕ್ಲ್ರೀಂ ಹ್‌ಸ್‌ರ್ಸೌಃ} ನಮಃ(ಸಮನಾಯಾಂ)\\
{\bfseries ೪ (ಪಂಚದಶೀ)}ನಮಃ(ಉನ್ಮನಾಯಾಂ)\\
{\bfseries ೪ (ಷೋಡಶೀ)}ನಮಃ(ಬ್ರಹ್ಮರಂಧ್ರೇ)
\section{ಕಾಮೇಶ್ವರ್ಯಾದಿ ನ್ಯಾಸಃ}
\addcontentsline{toc}{section}{ಕಾಮೇಶ್ವರ್ಯಾದಿ ನ್ಯಾಸಃ}
{\bfseries ೪ ಐಂ ೫॥} ಸೂರ್ಯಚಕ್ರೇ ಕಾಮಗಿರಿಪೀಠೇ ಮಿತ್ರೀಶನಾಥ ನವಯೋನಿ ಚಕ್ರಾತ್ಮಕ ಆತ್ಮತತ್ತ್ವ  ಸಂಹಾರಕೃತ್ಯ ಜಾಗ್ರದ್ದಶಾಧಿಷ್ಠಾಯಕ ಇಚ್ಛಾಶಕ್ತಿ ವಾಗ್ಭವಾತ್ಮಕ ಪರಾಪರಶಕ್ತಿ ಸ್ವರೂಪ ಮಹಾಕಾಮೇಶ್ವರೀ ರುದ್ರಾತ್ಮಶಕ್ತಿ  ಶ್ರೀಪಾದುಕಾಂ ಪೂಜಯಾಮಿ ನಮಃ॥(ಆಧಾರೇ)\\
{\bfseries೪ ಕ್ಲೀಂ ೬॥} ಸೋಮಚಕ್ರೇ ಪೂರ್ಣಗಿರಿಪೀಠೇ ಉಡ್ಡೀಶನಾಥ  ದಶಾರದ್ವಯ ಚತುರ್ದಶಾರ ಚಕ್ರಾತ್ಮಕ ವಿದ್ಯಾತತ್ವ ಸ್ಥಿತಿಕೃತ್ಯ ಸ್ವಪ್ನದಶಾಧಿಷ್ಠಾಯಕ ಜ್ಞಾನಶಕ್ತಿ ಕಾಮರಾಜಾತ್ಮಕ ಕಾಮಕಲಾ ಸ್ವರೂಪ ಮಹಾವಜ್ರೇಶ್ವರೀ ವಿಷ್ಣ್ವಾತ್ಮಶಕ್ತಿ  ಶ್ರೀಪಾದುಕಾಂ ಪೂಜಯಾಮಿ ನಮಃ॥(ಅನಾಹತೇ)\\
{\bfseries೪ ಸೌಃ ೪॥} ಅಗ್ನಿಚಕ್ರೇ ಜಾಲಂಧರಪೀಠೇ  ಷಷ್ಠೀಶನಾಥ ಅಷ್ಟದಳ ಷೋಡಶದಳ ಚತುರಸ್ರ ಚಕ್ರಾತ್ಮಕ ಶಿವತತ್ತ್ವ  ಸೃಷ್ಟಿಕೃತ್ಯ ಸುಷುಪ್ತಿದಶಾಧಿಷ್ಠಾಯಕ ಕ್ರಿಯಾಶಕ್ತಿ ಶಕ್ತಿಬೀಜಾತ್ಮಕ ವಾಗೀಶ್ವರೀ ಸ್ವರೂಪ ಮಹಾಭಗಮಾಲಿನೀ ಬ್ರಹ್ಮಾತ್ಮಶಕ್ತಿ  ಶ್ರೀಪಾದುಕಾಂ ಪೂಜಯಾಮಿ ನಮಃ॥(ಆಜ್ಞಾಯಾಮ್)\\
{\bfseries ೪ ಐಂ೫ ಕ್ಲೀಂ೬ ಸೌಃ೪} ಪರಬ್ರಹ್ಮಚಕ್ರೇ ಮಹೋಡ್ಯಾಣಪೀಠೇ ಚರ್ಯಾನಂದನಾಥ ಸಮಸ್ತಚಕ್ರಾತ್ಮಕ ಸಪರಿವಾರ ಪರಮತತ್ವ ಸೃಷ್ಟಿ ಸ್ಥಿತಿ ಸಂಹಾರಕೃತ್ಯ ತುರೀಯ ದಶಾಧಿಷ್ಠಾಯಕ ಇಚ್ಛಾ ಜ್ಞಾನ ಕ್ರಿಯಾ ಶಾಂತಶಕ್ತಿ ವಾಗ್ಭವ ಕಾಮರಾಜ ಶಕ್ತಿ ಬೀಜಾತ್ಮಕ ಪರಮಶಕ್ತಿ ಸ್ವರೂಪ ಶ್ರೀ ಮಹಾತ್ರಿಪುರಸುಂದರೀ ಪರಬ್ರಹ್ಮಾತ್ಮಶಕ್ತಿ ಶ್ರೀಪಾದುಕಾಂ ಪೂಜಯಾಮಿ ನಮಃ~।(ಬ್ರಹ್ಮರಂಧ್ರೇ)
\newpage
\section{ಮೂಲವಿದ್ಯಾನ್ಯಾಸಃ}
\addcontentsline{toc}{section}{ಮೂಲವಿದ್ಯಾನ್ಯಾಸಃ}
ಅಸ್ಯ ಶ್ರೀ ಮೂಲವಿದ್ಯಾನ್ಯಾಸಮಹಾಮಂತ್ರಸ್ಯ ದಕ್ಷಿಣಾಮೂರ್ತಿಃ ಋಷಿಃ~। ಪಂಕ್ತಿಶ್ಛಂದಃ~। ಶ್ರೀಮಹಾತ್ರಿಪುರಸುಂದರೀ ದೇವತಾ~। ಐಂ ಕಏಈಲಹ್ರೀಂ ಇತಿ ಬೀಜಂ~। ಕ್ಲೀಂ ಹಸಕಹಲಹ್ರೀಂ ಇತಿ ಶಕ್ತಿಃ~। ಸೌಃ ಸಕಲಹ್ರೀಂ ಇತಿ ಕೀಲಕಂ~। ಪೂಜಾಯಾಂ ವಿನಿಯೋಗಃ~।\\
ಕೂಟತ್ರಯೇಣ ನ್ಯಾಸಂ ಧ್ಯಾನಂ ಚ ವಿಧಾಯ~।\\
\as{೪ ಕಂ} ನಮಃ (ಶಿರಸಿ)\\
\as{೪ ಏಂ} ನಮಃ (ಆಧಾರೇ)\\
\as{೪ ಈಂ} ನಮಃ (ಹೃದಯೇ)\\
\as{೪ ಲಂ} ನಮಃ (ದಕ್ಷನೇತ್ರೇ)\\
\as{೪ ಹ್ರೀಂ} ನಮಃ (ವಾಮನೇತ್ರೇ)\\
\as{೪ ಹಂ} ನಮಃ (ಭ್ರೂಮಧ್ಯೇ)\\
\as{೪ ಸಂ} ನಮಃ (ದಕ್ಷಕರ್ಣೇ)\\
\as{೪ ಕಂ} ನಮಃ (ವಾಮಕರ್ಣೇ)\\
\as{೪ ಹಂ} ನಮಃ (ಮುಖೇ)\\
\as{೪ ಲಂ} ನಮಃ (ದಕ್ಷಾಂಸೇ)\\
\as{೪ ಹ್ರೀಂ} ನಮಃ (ವಾಮಾಂಸೇ)\\
\as{೪ ಸಂ} ನಮಃ (ಪೃಷ್ಠೇ)\\
\as{೪ ಕಂ} ನಮಃ (ದಕ್ಷಜಾನುನಿ)\\
\as{೪ ಲಂ} ನಮಃ (ವಾಮಜಾನುನಿ)\\
\as{೪ ಹ್ರೀಂ} ನಮಃ (ನಾಭೌ)\\
\as{೪ ೧೫} ನಮಃ (ಸರ್ವಾಂಗೇ)\\
ಪ್ರಾಗ್ವದುತ್ತರನ್ಯಾಸಃ
\section{ಷೋಡಶಾಕ್ಷರೀ ನ್ಯಾಸಃ}
\addcontentsline{toc}{section}{ಷೋಡಶಾಕ್ಷರೀ ನ್ಯಾಸಃ}
ಅಸ್ಯ ಶ್ರೀ ಷೋಡಶಾಕ್ಷರೀನ್ಯಾಸಮಹಾಮಂತ್ರಸ್ಯ ಆನಂದಭೈರವ ಋಷಿಃ~। ಅನುಷ್ಟುಪ್ಛಂದಃ~। ಶ್ರೀಮಹಾತ್ರಿಪುರಸುಂದರೀ ದೇವತಾ~। ಐಂ ಬೀಜಂ~। ಕ್ಲೀಂ ಶಕ್ತಿಃ~। ಸೌಃ ಕೀಲಕಂ~। ಪೂಜಾಯಾಂ ವಿನಿಯೋಗಃ~। ಮೂಲಾಕ್ಷರೈಃ ನ್ಯಾಸಃ~।\\
\as{೪ (ಮೂಲಮಂತ್ರ)} ನಮಃ ॥ದಕ್ಷಕರಾಂಗುಷ್ಠಾನಾಮಿಕಾಭ್ಯಾಂ ಶಿರಸಿ\\
\as{೪ (ಮೂಲಮಂತ್ರ)} ನಮಃ ॥ಮಹಾಸೌಭಾಗ್ಯಂ ಮೇ ದೇಹಿ॥\\ಶಿರ ಆದಿ ಪಾದಾಂತಂ ಶರೀರವಾಮಭಾಗೇ\\
\as{೪ (ಮೂಲಮಂತ್ರ)} ನಮಃ ॥ ಮಮ ಶತ್ರೂನ್ನಿಗೃಹ್ಣಾಮಿ ॥\\ರಿಪುಜಿಹ್ವಾಗ್ರಮುದ್ರಯಾ ವಾಮಪಾದಸ್ಯ ಅಧಃ\\
\as{೪ (ಮೂಲಮಂತ್ರ)} ನಮಃ ॥ ತ್ರೈಲೋಕ್ಯಸ್ಯಾಹಂ ಕರ್ತಾ॥\\ತ್ರಿಖಂಡಯಾ ಫಾಲೇ\\
\as{೪ (ಮೂಲಮಂತ್ರ)} ನಮಃ ॥ ತ್ರಿಖಂಡಯಾ ಮುಖೇ\\
\as{೪ (ಮೂಲಮಂತ್ರ)} ನಮಃ ॥\\ ತ್ರಿಖಂಡಯಾ ದಕ್ಷಕರ್ಣಾದಿ ವಾಮಕರ್ಣಪರ್ಯಂತಂ\\
\as{೪ (ಮೂಲಮಂತ್ರ)} ನಮಃ ॥ ತ್ರಿಖಂಡಯಾ ಗಲಾದಿ ಶಿರಃಪರ್ಯಂತಂ\\
\as{೪ (ಮೂಲಮಂತ್ರ)} ನಮಃ ॥\\ ತ್ರಿಖಂಡಯಾ ಶಿರ ಆದಿ ಪಾದಪರ್ಯಂತಂ ಪಾದಾದಿ ಶಿರಃಪರ್ಯಂತಂ\\
\as{೪ (ಮೂಲಮಂತ್ರ)} ನಮಃ ॥ ಯೋನಿಮುದ್ರಯಾ ಮುಖೇ\\
\as{೪ (ಮೂಲಮಂತ್ರ)} ನಮಃ ॥ ಯೋನಿಮುದ್ರಯಾ ಲಲಾಟೇ
\section{ಸಂಮೋಹನನ್ಯಾಸಃ}
\addcontentsline{toc}{section}{ಸಂಮೋಹನನ್ಯಾಸಃ}
\as{೪ (ಮೂಲಮಂತ್ರ)} ನಮಃ ॥ಅನಾಮಿಕಯಾ ತ್ರಿವಾರಂ ಶಿರಸಿ ಪರಿಭ್ರಾಮ್ಯ\\
\as{೪ (ಮೂಲಮಂತ್ರ)} ನಮಃ ॥ಅಂಗುಷ್ಠಾನಾಮಿಕಾಭ್ಯಾಂ ಬ್ರಹ್ಮರಂಧ್ರೇ\\
\as{೪ (ಮೂಲಮಂತ್ರ)} ನಮಃ ॥ಮಣಿಬಂಧಯೋಃ\\
\as{೪ (ಮೂಲಮಂತ್ರ)} ನಮಃ ॥ಲಲಾಟೇ\\
\as{೪ (ಮೂಲಮಂತ್ರ)} ನಮಃ ॥ಇತಿ ಶಾಕ್ತತಿಲಕಂ ಪ್ರಕಲ್ಪಯೇತ್
\section{ಸಂಹಾರನ್ಯಾಸಃ}
\addcontentsline{toc}{section}{ಸಂಹಾರನ್ಯಾಸಃ}
\as{೪ ಶ್ರೀಂ} ನಮಃ (ಪಾದಯೋಃ)\\
\as{೪ ಹ್ರೀಂ} ನಮಃ (ಜಂಘಯೋಃ)\\
\as{೪ ಕ್ಲೀಂ} ನಮಃ (ಜಾನುನೋಃ)\\
\as{೪ ಐಂ} ನಮಃ (ಸ್ಫಿಚೋಃ)\\
\as{೪ ಸೌಃ} ನಮಃ (ಪೃಷ್ಠೇ)\\
\as{೪ ಓಂ} ನಮಃ (ಲಿಂಗೇ)\\
\as{೪ ಹ್ರೀಂ} ನಮಃ (ನಾಭೌ)\\
\as{೪ ಶ್ರೀಂ} ನಮಃ (ಪಾರ್ಶ್ವಯೋಃ)\\
\as{೪ ಐಂ ಕಏಈಲಹ್ರೀಂ} ನಮಃ (ಸ್ತನಯೋಃ)\\
\as{೪ ಕ್ಲೀಂ ಹಸಕಹಲಹ್ರೀಂ} ನಮಃ (ಅಂಸಯೋಃ)\\
\as{೪ ಸೌಃ ಸಕಲಹ್ರೀಂ} ನಮಃ (ಕರ್ಣಯೋಃ)\\
\as{೪ ಸೌಃ} ನಮಃ (ಮೂರ್ಧ್ನಿ)\\
\as{೪ ಐಂ} ನಮಃ (ಮುಖೇ)\\
\as{೪ ಕ್ಲೀಂ} ನಮಃ (ನೇತ್ರಯೋಃ)\\
\as{೪ ಹ್ರೀಂ} ನಮಃ (ಉಪಕರ್ಣಯೋಃ)\\
\as{೪ ಶ್ರೀಂ} ನಮಃ (ಕರ್ಣಯೋಃ)
\newpage
\section{ಸೃಷ್ಟಿನ್ಯಾಸಃ}
\addcontentsline{toc}{section}{ಸೃಷ್ಟಿನ್ಯಾಸಃ}
\as{೪ ಶ್ರೀಂ} ನಮಃ (ಬ್ರಹ್ಮರಂಧ್ರೇ)\\
\as{೪ ಹ್ರೀಂ} ನಮಃ (ಲಲಾಟೇ)\\
\as{೪ ಕ್ಲೀಂ} ನಮಃ (ನೇತ್ರಯೋಃ)\\
\as{೪ ಐಂ} ನಮಃ (ಕರ್ಣಯೋಃ)\\
\as{೪ ಸೌಃ} ನಮಃ (ನಾಸಿಕಯೋಃ)\\
\as{೪ ಓಂ} ನಮಃ (ಗಂಡಯೋಃ)\\
\as{೪ ಹ್ರೀಂ} ನಮಃ (ದಂತಪಂಕ್ತೌ)\\
\as{೪ ಶ್ರೀಂ} ನಮಃ (ಓಷ್ಠಯೋಃ)\\
\as{೪ ಐಂ ಕಏಈಲಹ್ರೀಂ} ನಮಃ (ಜಿಹ್ವಾಯಾಂ)\\
\as{೪ ಕ್ಲೀಂ ಹಸಕಹಲಹ್ರೀಂ} ನಮಃ (ಕಂಠೇ)\\
\as{೪ ಸೌಃ ಸಕಲಹ್ರೀಂ} ನಮಃ (ಪೃಷ್ಠೇ)\\
\as{೪ ಸೌಃ} ನಮಃ (ಸರ್ವಾಂಗೇ)\\
\as{೪ ಐಂ} ನಮಃ (ಹೃದಯೇ)\\
\as{೪ ಕ್ಲೀಂ} ನಮಃ (ಸ್ತನಯೋಃ)\\
\as{೪ ಹ್ರೀಂ} ನಮಃ (ಉದರೇ)\\
\as{೪ ಶ್ರೀಂ} ನಮಃ (ಲಿಂಗೇ)
\section{ಸ್ಥಿತಿನ್ಯಾಸಃ}
\addcontentsline{toc}{section}{ಸ್ಥಿತಿನ್ಯಾಸಃ}
\as{೪ ಶ್ರೀಂ} ನಮಃ (ಅಂಗುಷ್ಠಯೋಃ)\\
\as{೪ ಹ್ರೀಂ} ನಮಃ (ತರ್ಜನ್ಯೋಃ)\\
\as{೪ ಕ್ಲೀಂ} ನಮಃ (ಮಧ್ಯಮಯೋಃ)\\
\as{೪ ಐಂ} ನಮಃ (ಅನಾಮಿಕಯೋಃ)\\
\as{೪ ಸೌಃ} ನಮಃ (ಕನಿಷ್ಠಿಕಯೋಃ)\\
\as{೪ ಓಂ} ನಮಃ (ಮೂರ್ಧ್ನಿ)\\
\as{೪ ಹ್ರೀಂ} ನಮಃ (ಮುಖೇ)\\
\as{೪ ಶ್ರೀಂ} ನಮಃ (ಹೃದಯೇ)\\
\as{೪ ಐಂ ಕಏಈಲಹ್ರೀಂ} ನಮಃ (ಪಾದಾದಿನಾಭಿಪರ್ಯಂತಂ)\\
\as{೪ ಕ್ಲೀಂ ಹಸಕಹಲಹ್ರೀಂ} ನಮಃ (ನಾಭೇರ್ವಿಶುದ್ಧಿಪರ್ಯಂತಂ)\\
\as{೪ ಸೌಃ ಸಕಲಹ್ರೀಂ} ನಮಃ (ವಿಶುದ್ಧೇರ್ಬ್ರಹ್ಮರಂಧ್ರಾಂತಂ)\\
\as{೪ ಸೌಃ} ನಮಃ (ಪಾದಾಂಗುಷ್ಠಯೋಃ)\\
\as{೪ ಐಂ} ನಮಃ (ಪಾದತರ್ಜನ್ಯೋಃ)\\
\as{೪ ಕ್ಲೀಂ} ನಮಃ (ಪಾದಮಧ್ಯಮಯೋಃ)\\
\as{೪ ಹ್ರೀಂ} ನಮಃ (ಪಾದಾನಾಮಿಕಯೋಃ)\\
\as{೪ ಶ್ರೀಂ} ನಮಃ (ಪಾದಕನಿಷ್ಠಿಕಯೋಃ)\\
\as{೪ ೧೬} ನಮಃ (ಸರ್ವಾಂಗೇ)\\
ಪ್ರಾಗ್ವದುತ್ತರನ್ಯಾಸಃ
\section{ಲಘುಷೋಢಾನ್ಯಾಸಃ}
ಅಸ್ಯ ಶ್ರೀ ಲಘುಷೋಢಾ ನ್ಯಾಸಸ್ಯ ದಕ್ಷಿಣಾಮೂರ್ತಿ- ಋಷಯೇ ನಮಃ (ಶಿರಸಿ)। ಗಾಯತ್ರೀ ಛಂದಸೇ ನಮಃ (ಮುಖೇ)। ಗಣೇಶ ಗ್ರಹ ನಕ್ಷತ್ರ ಯೋಗಿನೀ ರಾಶಿ ಪೀಠರೂಪಿಣ್ಯೈ ಶ್ರೀಮಹಾತ್ರಿಪುರಸುಂದರೀ ದೇವತಾಯೈ ನಮಃ (ಹೃದಯೇ)।  ಶ್ರಿವಿದ್ಯಾಂಗತ್ವೇನ ನ್ಯಾಸೇ ವಿನಿಯೋಗಾಯ ನಮಃ (ಸರ್ವಾಂಗೇ)।

ಓಂ ಐಂಹ್ರೀಂಶ್ರೀಂ \\ಅಂ ಕಂಖಂಗಂಘಂಙಂ ಆಂ ಐಂ ಅಂಗುಷ್ಠಾಭ್ಯಾಂ ನಮಃ ।\\
೪ ಇಂ ಚಂಛಂಜಂಝಂಞಂ ಈಂ ಕ್ಲೀಂ ತರ್ಜನೀಭ್ಯಾಂ ನಮಃ ।\\
೪ ಉಂ ಟಂಠಂಡಂಢಂಣಂ ಊಂ ಸೌಃ ಮಧ್ಯಮಾಭ್ಯಾಂ ನಮಃ ।\\
೪ ಏಂ ತಂಥಂದಂಧಂನಂ ಐಂ ಐಂ ಅನಾಮಿಕಾಭ್ಯಾಂ ನಮಃ ।\\
೪ ಓಂ ಪಂಫಂಬಂಭಂಮಂ ಔಂ ಕ್ಲೀಂ ಕನಿಷ್ಠಿಕಾಭ್ಯಾಂ ನಮಃ ।\\
೪ ಅಂ ಯಂರಂಲಂವಂಶಂಷಂಸಂಹಂಳಂಕ್ಷಂ ಅಃ ಸೌಃ\\ ಕರತಲಕರಪೃಷ್ಠಾಭ್ಯಾಂ ನಮಃ ।\\
ಏವಮೇವಾಂಗನ್ಯಾಸಂ ವಿಧಾಯ ಧ್ಯಾಯೇತ್ 

\dhyana{ಉದ್ಯತ್ಸೂರ್ಯಸಹಸ್ರಾಭಾಂ ಪೀನೋನ್ನತಪಯೋಧರಾಂ ।\\ರಕ್ತಮಾಲ್ಯಾಂಬರಾಲೇಪಾಂ ರತ್ನಭೂಷಣಭೂಷಿತಾಂ ॥

ಪಾಶಾಂಕುಶಧನುರ್ಬಾಣಭಾಸ್ವತ್ಪಾಣಿಚತುಷ್ಟಯಾಂ ।\\ ಲಸನ್ನೇತ್ರತ್ರಯಾಂ ಸ್ವರ್ಣಮಕುಟೋದ್ಭಾಸಿ ಮಸ್ತಕಾಂ ॥

ಗಣೇಶಗ್ರಹನಕ್ಷತ್ರಯೋಗಿನೀ ರಾಶಿರೂಪಿಣೀಂ ।\\ದೇವೀಂ ಪೀಠಮಯೀಂ ಧ್ಯಾಯೇನ್ಮಾತೃಕಾಂ ಸುಂದರೀಂ ಪರಾಂ ॥}
\subsection{ಗಣೇಶನ್ಯಾಸಃ}
\dhyana{ತರುಣಾದಿತ್ಯಸಂಕಾಶಾನ್ ಗಜವಕ್ತ್ರಾಂಸ್ತ್ರಿಲೋಚನಾನ್ ।\\ ಪಾಶಾಂಕುಶವರಾಭೀತಿಕರಾನ್ ಶಕ್ತಿ ಸಮನ್ವಿತಾನ್ ॥

ತಾಸ್ತುಸಿಂಧೂರವರ್ಣಾಭಾಃ ಸರ್ವಾಲಂಕಾರಭೂಷಿತಾಃ ।\\ ಏಕಹಸ್ತಧೃತಾಂಭೋಜಾ ಇತರಾಲಿಂಗಿತಪ್ರಿಯಾಃ ॥}

\as{೪ ಅಂ} ಶ್ರೀಯುಕ್ತಾಯ ವಿಘ್ನೇಶಾಯ ನಮಃ । (ಶಿರಸಿ)\\
\as{೪ ಆಂ} ಹ್ರೀಯುಕ್ತಾಯ ವಿಘ್ನರಾಜಾಯ ನಮಃ । (ಮುಖವೃತ್ತೇ)\\
\as{೪ ಇಂ} ತುಷ್ಟಿಯುಕ್ತಾಯ ವಿನಾಯಕಾಯ ನಮಃ । (ದಕ್ಷನೇತ್ರೇ)\\
\as{೪ ಈಂ} ಶಾಂತಿಯುಕ್ತಾಯ ಶಿವೋತ್ತಮಾಯ ನಮಃ । (ವಾಮನೇತ್ರೇ)\\
\as{೪ ಉಂ} ಪುಷ್ಟಿಯುಕ್ತಾಯ ವಿಘ್ನಹೃತೇ ನಮಃ । (ದಕ್ಷಕರ್ಣೇ)\\
\as{೪ ಊಂ} ಸರಸ್ವತೀಯುಕ್ತಾಯ ವಿಘ್ನಕರ್ತ್ರೇ ನಮಃ । (ವಾಮಕರ್ಣೇ)\\
\as{೪ ಋಂ} ರತಿಯುಕ್ತಾಯ ವಿಘ್ನರಾಜೇ ನಮಃ । (ದಕ್ಷನಾಸಾಯಾಂ)\\
\as{೪ ೠಂ} ಮೇಧಾಯುಕ್ತಾಯ ಗಣನಾಯಕಾಯ ನಮಃ ।\\(ವಾಮನಾಸಾಯಾಂ)\\
\as{೪ ಲೃಂ} ಕಾಂತಿಯುಕ್ತಾಯ ಏಕದಂತಾಯ ನಮಃ । (ದಕ್ಷಗಂಡೇ)\\
\as{೪ ಲೄಂ} ಕಾಮಿನೀಯುಕ್ತಾಯ ದ್ವಿದಂತಾಯ ನಮಃ । (ವಾಮಗಂಡೇ)\\
\as{೪ ಏಂ} ಮೋಹಿನೀಯುಕ್ತಾಯ ಗಜವಕ್ತ್ರಾಯ ನಮಃ । (ಊರ್ಧ್ವೋಷ್ಠೇ)\\
\as{೪ ಐಂ} ಜಟಾಯುಕ್ತಾಯ ನಿರಂಜನಾಯ ನಮಃ । (ಅಧರೋಷ್ಠೇ)\\
\as{೪ ಓಂ} ತೀವ್ರಾಯುಕ್ತಾಯ ಕಪರ್ದಭೃತೇ ನಮಃ । (ಊರ್ಧ್ವದಂತಪಂಕ್ತೌ)\\
\as{೪ ಔಂ} ಜ್ವಾಲಿನೀಯುಕ್ತಾಯ ದೀರ್ಘಮುಖಾಯ ನಮಃ । \\(ಅಧೋದಂತಪಂಕ್ತೌ)\\
\as{೪ ಅಂ} ನಂದಾಯುಕ್ತಾಯ ಶಂಕುಕರ್ಣಾಯ ನಮಃ । (ಜಿಹ್ವಾಯಾಂ)\\
\as{೪ ಅಃ} ಸುರಸಾಯುಕ್ತಾಯ ವೃಷಧ್ವಜಾಯ ನಮಃ । (ಕಂಠೇ)\\
\as{೪ ಕಂ} ಕಾಮರೂಪಿಣೀಯುಕ್ತಾಯ ಗಣನಾಥಾಯ ನಮಃ । \\(ದಕ್ಷ ಬಾಹುಮೂಲೇ)\\
\as{೪ ಖಂ} ಸುಭ್ರೂಯುಕ್ತಾಯ ಗಜೇಂದ್ರಾಯ ನಮಃ । (ದಕ್ಷಕೂರ್ಪರೇ)\\
\as{೪ ಗಂ} ಜಯಿನೀಯುಕ್ತಾಯ ಶೂರ್ಪಕರ್ಣಾಯ ನಮಃ । (ದಕ್ಷಮಣಿಬಂಧೇ)\\
\as{೪ ಘಂ} ಸತ್ಯಾಯುಕ್ತಾಯ ತ್ರಿಲೋಚನಾಯ ನಮಃ । \\(ದಕ್ಷಕರಾಂಗುಲಿಮೂಲೇ)\\
\as{೪ ಙಂ} ವಿಘ್ನೇಶೀಯುಕ್ತಾಯ ಲಂಬೋದರಾಯ ನಮಃ ।\\ (ದಕ್ಷಕರಾಂಗುಲ್ಯಗ್ರೇ)\\
\as{೪ ಚಂ} ಸುರೂಪಾಯುಕ್ತಾಯ ಮಹಾನಾದಾಯ ನಮಃ ।\\ (ವಾಮಬಾಹುಮೂಲೇ)\\
\as{೪ ಛಂ} ಕಾಮದಾಯುಕ್ತಾಯ ಚತುರ್ಮೂರ್ತಯೇ ನಮಃ ।\\ (ವಾಮಕೂರ್ಪರೇ)\\
\as{೪ ಜಂ} ಮದವಿಹ್ವಲಾಯುಕ್ತಾಯ ಸದಾಶಿವಾಯ ನಮಃ ।\\ (ವಾಮಮಣಿಬಂಧೇ)\\
\as{೪ ಝಂ} ವಿಕಟಾಯುಕ್ತಾಯ ಆಮೋದಾಯ ನಮಃ । \\(ವಾಮಕರಾಂಗುಲಿಮೂಲೇ)\\
\as{೪ ಞಂ} ಪೂರ್ಣಾಯುಕ್ತಾಯ ದುರ್ಮುಖಾಯ ನಮಃ । \\(ವಾಮಕರಾಂಗುಲ್ಯಗ್ರೇ)\\
\as{೪ ಟಂ} ಭೂತಿದಾಯುಕ್ತಾಯ ಸುಮುಖಾಯ ನಮಃ । (ದಕ್ಷೋರುಮೂಲೇ)\\
\as{೪ ಠಂ} ಭೂಮಿಯುಕ್ತಾಯ ಪ್ರಮೋದಾಯ ನಮಃ । (ದಕ್ಷಜಾನುನಿ)\\
\as{೪ ಡಂ} ಶಕ್ತಿಯುಕ್ತಾಯ ಏಕಪಾದಾಯ ನಮಃ । (ದಕ್ಷಗುಲ್ಫೇ)\\
\as{೪ ಢಂ} ರಮಾಯುಕ್ತಾಯ ದ್ವಿಜಿಹ್ವಾಯ ನಮಃ । (ದಕ್ಷಪಾದಾಂಗುಲಿಮೂಲೇ)\\
\as{೪ ಣಂ} ಮಾನುಷೀಯುಕ್ತಾಯ ಶೂರಾಯ ನಮಃ । (ದಕ್ಷಪಾದಾಂಗುಲ್ಯಗ್ರೇ)\\
\as{೪ ತಂ} ಮಕರಧ್ವಜಾಯುಕ್ತಾಯ ವೀರಾಯ ನಮಃ । (ವಾಮೋರುಮೂಲೇ)\\
\as{೪ ಥಂ} ವೀರಿಣೀಯುಕ್ತಾಯ ಷಣ್ಮುಖಾಯ ನಮಃ । (ವಾಮಜಾನುನಿ)\\
\as{೪ ದಂ} ಭ್ರುಕುಟೀಯುಕ್ತಾಯ ವರದಾಯ ನಮಃ । (ವಾಮಗುಲ್ಫೇ)\\
\as{೪ ಧಂ} ಲಜ್ಜಾಯುಕ್ತಾಯ ವಾಮದೇವಾಯ ನಮಃ ।\\(ವಾಮಪಾದಾಂಗುಲಿಮೂಲೇ)\\
\as{೪ ನಂ} ದೀರ್ಘಘೋಣಾಯುಕ್ತಾಯ ವಕ್ರತುಂಡಾಯ ನಮಃ ।\\ (ವಾಮಪಾದಾಂಗುಲ್ಯಗ್ರೇ)\\
\as{೪ ಪಂ} ಧನುರ್ಧರಾಯುಕ್ತಾಯ ದ್ವಿರಂಡಕಾಯ ನಮಃ । (ದಕ್ಷಪಾರ್ಶ್ವೇ)\\
\as{೪ ಫಂ} ಯಾಮಿನೀಯುಕ್ತಾಯ ಸೇನಾನ್ಯೇ ನಮಃ । (ವಾಮಪಾರ್ಶ್ವೇ)\\
\as{೪ ಬಂ} ರಾತ್ರಿಯುಕ್ತಾಯ ಗ್ರಾಮಣ್ಯೇ ನಮಃ । (ಪೃಷ್ಠೇ)\\
\as{೪ ಭಂ} ಚಂದ್ರಿಕಾಯುಕ್ತಾಯ ಮತ್ತಾಯ ನಮಃ । (ನಾಭೌ)\\
\as{೪ ಮಂ} ಶಶಿಪ್ರಭಾಯುಕ್ತಾಯ ವಿಮತ್ತಾಯ ನಮಃ । (ಜಠರೇ)\\
\as{೪ ಯಂ} ಲೋಲಾಯುಕ್ತಾಯ ಮತ್ತವಾಹನಾಯ ನಮಃ । (ಹೃದಿ )\\
\as{೪ ರಂ} ಚಪಲಾಯುಕ್ತಾಯ ಜಟಿನೇ ನಮಃ । (ದಕ್ಷಾಂಸೇ)\\
\as{೪ ಲಂ} ಋದ್ಧಿಯುಕ್ತಾಯ ಮುಂಡಿನೇ ನಮಃ । (ಕಕುದಿ)\\
\as{೪ ವಂ} ದುರ್ಭಗಾಯುಕ್ತಾಯ ಖಡ್ಗಿನೇ ನಮಃ । (ವಾಮಾಂಸೇ)\\
\as{೪ ಶಂ} ಸುಭಗಾಯುಕ್ತಾಯ ವರೇಣ್ಯಾಯ ನಮಃ ।\\ (ಹೃದಯಾದಿ ದಕ್ಷ ಕರಾಂಗುಲ್ಯಂತಂ )\\
\as{೪ ಷಂ} ಶಿವಾಯುಕ್ತಾಯ ವೃಷಕೇತನಾಯ ನಮಃ । \\(ಹೃದಯಾದಿ ವಾಮ ಕರಾಂಗುಲ್ಯಂತಂ )\\
\as{೪ ಸಂ} ದುರ್ಗಾಯುಕ್ತಾಯ ಭಕ್ಷ್ಯಪ್ರಿಯಾಯ ನಮಃ । \\(ಹೃದಯಾದಿ ದಕ್ಷ ಪಾದಾಂತಂ )\\
\as{೪ ಹಂ} ಕಾಲೀಯುಕ್ತಾಯ ಗಣೇಶಾಯ ನಮಃ । \\(ಹೃದಯಾದಿ ವಾಮ ಪಾದಾಂತಂ )\\
\as{೪ ಳಂ} ಕಾಲಕುಬ್ಜಿಕಾಯುಕ್ತಾಯ ಮೇಘನಾದಾಯ ನಮಃ ।\\ (ಕಟ್ಯಾದಿ ಪಾದಪರ್ಯಂತಂ)\\
\as{೪ ಕ್ಷಂ} ವಿಘ್ನಹಾರಿಣೀಯುಕ್ತಾಯ ಗಣೇಶ್ವರಾಯ ನಮಃ । \\(ಕಟ್ಯಾದಿ ಶಿರಃಪರ್ಯಂತಂ )
\subsection{ಗ್ರಹನ್ಯಾಸಃ}
\dhyana{ರಕ್ತಂ ಶ್ವೇತಂ  ತಥಾ ರಕ್ತಂ ಶ್ಯಾಮಂ ಪೀತಂ ಚ ಪಾಂಡರಮ್ ।\\ಕೃಷ್ಣಂ ಧೂಮ್ರಂ ಧೂಮ್ರಧೂಮ್ರಂ ಭಾವಯೇದ್ರವಿಪೂರ್ವಕಾನ್ ॥

ಕಾಮರೂಪಧರಾನ್ ದೇವಾನ್ ದಿವ್ಯಾಭರಣ ಭೂಷಿತಾನ್ ।\\ವಾಮೋರುನ್ಯಸ್ಯ ಹಸ್ತಾಂಶ್ಚ ದಕ್ಷಹಸ್ತವರಪ್ರದಾನ್ ॥

ಶಕ್ತಯೋಽಪಿ ತಥಾ ಜ್ಞೇಯಾ ವರಾಭಯಕರಾಂಬುಜಾಃ ।\\ಸ್ವಸ್ವಪ್ರಿಯಾಂಕನಿಲಯಾಃ ಸರ್ವಾಭರಣ ಭೂಷಿತಾಃ ॥}

\as{ಓಂ ಐಂಹ್ರೀಂಶ್ರೀಂ ಅಂ ಆಂ ಇಂ++++ ಅಂ ಅಃ ।}\\ ರೇಣುಕಾಯುಕ್ತಾಯ ಸೂರ್ಯಾಯ ನಮಃ । (ಹೃಜ್ಜಠರಸಂಧೌ)\\
\as{೪ ಕಂಖಂಗಂಘಂಙಂ ।} ಅಮೃತಾಯುಕ್ತಾಯ ಚಂದ್ರಾಯ ನಮಃ ।\\(ಭ್ರೂಮಧ್ಯೇ)\\
\as{೪ ಚಂಛಂಜಂಝಂಞಂ ।} ಧರ್ಮಯುಕ್ತಾಯ ಭೌಮಾಯ ನಮಃ ।\\(ಚಕ್ಷುಷೋಃ)\\
\as{೪ ಟಂಠಂಡಂಢಂಣಂ ।} ಯಶಸ್ವಿನೀಯುಕ್ತಾಯ ಬುಧಾಯ ನಮಃ ।\\(ಶ್ರೋತ್ರಕೂಪಾಧಃ)\\
\as{೪ ತಂಥಂದಂಧಂನಂ ।} ಶಾಂಕರೀಯುಕ್ತಾಯ ಬೃಹಸ್ಪತಯೇ ನಮಃ ।(ಕಂಠೇ)\\
\as{೪ ಪಂಫಂಬಂಭಂಮಂ ।} ಜ್ಞಾನರೂಪಾಯುಕ್ತಾಯ ಶುಕ್ರಾಯ ನಮಃ ।(ಹೃದಿ)\\
\as{೪ ಯಂರಂಲಂವಂ ।} ಶಕ್ತಿಯುಕ್ತಾಯ ಶನೈಶ್ಚರಾಯ ನಮಃ ।(ನಾಭೌ)\\
\as{೪ ಶಂಷಂಸಂಹಂ ।} ಕೃಷ್ಣಾಯುಕ್ತಾಯ ರಾಹವೇ ನಮಃ ।(ಮುಖೇ)\\
\as{೪ ಳಂಕ್ಷಂ ।} ಧೂಮ್ರಾಯುಕ್ತಾಯ ಕೇತವೇ ನಮಃ ।(ಗುದೇ)
\subsection{ನಕ್ಷತ್ರ ನ್ಯಾಸಃ}
\dhyana{ಜ್ವಲತ್ಕಾಲಾನಲಪ್ರಖ್ಯಾ ವರದಾಭಯಪಾಣಿನಃ ।\\ ನತಿಪಾಣ್ಯೋಽಶ್ವಿನೀಪೂರ್ವಾಃ ಸರ್ವಾಭರಣಭೂಷಿತಾಃ ॥}

\as{ಓಂ ಐಂಹ್ರೀಂಶ್ರೀಂ ಅಂ ಆಂ} ಅಶ್ವಿನ್ಯೈ ನಮಃ ।(ಲಲಾಟೇ)\\
\as{೪ ಇಂ } ಭರಣ್ಯೈ ನಮಃ ।(ದಕ್ಷಚಕ್ಷುಷಿ)\\
\as{೪ ಈಂ ಉಂ ಊಂ }ಕೃತ್ತಿಕಾಯೈ ನಮಃ ।(ವಾಮಚಕ್ಷುಷಿ)\\
\as{೪ ಋಂ ೠಂ ಲೃಂ ಲೄಂ }ರೋಹಿಣ್ಯೈ ನಮಃ ।(ದಕ್ಷಕರ್ಣೇ)\\
\as{೪ ಏಂ }ಮೃಗಶೀರ್ಷಾಯ ನಮಃ ।(ವಾಮಕರ್ಣೇ)\\
\as{೪ ಐಂ }ಆರ್ದ್ರಾಯೈ ನಮಃ ।(ದಕ್ಷನಾಸಾಪುಟೇ)\\
\as{೪ ಓಂ ಔಂ} ಪುನರ್ವಸವೇ ನಮಃ ।(ವಾಮನಾಸಾಪುಟೇ)\\
\as{೪ ಕಂ }ಪುಷ್ಯಾಯ ನಮಃ ।(ಕಂಠೇ)\\
\as{೪ ಖಂಗಂ} ಆಶ್ಲೇಷಾಯೈ ನಮಃ ।(ದಕ್ಷಾಂಸೇ)\\
\as{೪ ಘಂಙಂ} ಮಖಾಯೈ ನಮಃ ।(ವಾಮಾಂಸೇ)\\
\as{೪ ಚಂ }ಪೂರ್ವಫಲ್ಗುನ್ಯೈ ನಮಃ ।(ಪೃಷ್ಠೇ)\\
\as{೪ ಛಂಜಂ} ಉತ್ತರಫಲ್ಗುನ್ಯೈ ನಮಃ ।(ದಕ್ಷಪ್ರಕೋಷ್ಠೇ)\\
\as{೪ ಝಂಞಂ} ಹಸ್ತಾಯ ನಮಃ ।(ವಾಮಪ್ರಕೋಷ್ಠೇ)\\
\as{೪ ಟಂಠಂ }ಚಿತ್ರಾಯೈ ನಮಃ ।(ದಕ್ಷಮಣಿಬಂಧೇ)\\
\as{೪ ಡಂ }ಸ್ವಾತ್ಯೈ ನಮಃ ।(ವಾಮಮಣಿಬಂಧೇ)\\
\as{೪ ಢಂಣಂ} ವಿಶಾಖಾಯೈ ನಮಃ ।(ದಕ್ಷಕರತಲೇ)\\
\as{೪ ತಂಥಂದಂ} ಅನೂರಾಧಾಯೈ ನಮಃ ।(ವಾಮಕರತಲೇ)\\
\as{೪ ಧಂ }ಜ್ಯೇಷ್ಠಾಯೈ ನಮಃ ।(ನಾಭೌ)\\
\as{೪ ನಂ ಪಂಫಂ} ಮೂಲಾಯೈ ನಮಃ ।(ಕಟೌ)\\
\as{೪ ಬಂ} ಪೂರ್ವಾಷಾಢಾಯೈ ನಮಃ ।(ದಕ್ಷೋರೌ)\\
\as{೪ ಭಂ }ಉತ್ತರಾಷಾಢಾಯೈ ನಮಃ ।(ವಾಮೋರೌ)\\
\as{೪ ಮಂ }ಶ್ರವಣಾಯ ನಮಃ ।(ದಕ್ಷಜಾನುನಿ)\\
\as{೪ ಯಂ ರಂ} ಧನಿಷ್ಠಾಯೈ ನಮಃ ।(ವಾಮಜಾನುನಿ)\\
\as{೪ ಲಂ} ಶತತಾರಕಾಯೈ ನಮಃ ।(ದಕ್ಷಜಘನೇ)\\
\as{೪ ವಂ ಶಂ} ಪೂರ್ವಾಭಾದ್ರಪದಾಯೈ ನಮಃ ।(ವಾಮಜಘನೇ)\\
\as{೪ ಷಂ ಸಂ ಹಂ}  ಉತ್ತರಾಭಾದ್ರಪದಾಯೈ ನಮಃ ।(ದಕ್ಷಪಾದೇ)\\
\as{೪ ಳಂ ಕ್ಷಂ ಅಂ ಅಃ }ರೇವತ್ಯೈ ನಮಃ ।(ವಾಮಪಾದೇ)
\subsection{ಯೋಗಿನೀನ್ಯಾಸಃ}
\as{(ಡಾಕಿನೀಧ್ಯಾನಮ್) \\}
\as{ಕಂಠಸ್ಥಾನೇ ವಿಶುದ್ಧೌ ನೃಪದಲಕಮಲೇ ಶ್ವೇತವರ್ಣಾಂ ತ್ರಿನೇತ್ರಾಂ\\ ಹಸ್ತೈಃ ಖಟ್ವಾಂಗಖಡ್ಗೌ ತ್ರಿಶಿಖಮಪಿ ಮಹಾಚರ್ಮ ಸಂಧಾರಯಂತೀಂ ॥\\ವಕ್ತ್ರೇಣೈಕೇನಯುಕ್ತಾಂ ಪಶುಜನಭಯದಾಂ ಪಾಯಸಾನ್ನೈಕಸಕ್ತಾಂ\\ ತ್ವಕ್ಸ್ಥಾಂ ವಂದೇಽಮೃತಾದ್ಯೈಃ ಪರಿವೃತವಪುಷಂ ಡಾಕಿನೀಂ ವೀರವಂದ್ಯಾಂ ॥}

\as{ಓಂ ಐಂಹ್ರೀಂಶ್ರೀಂ ಡಾಂಡೀಂಡೂಂಡೈಂಡೌಂಡಃ ।} ಡಮಲವರಯೂಂ ಡಾಕಿನ್ಯೈ ನಮಃ । \as{೪ ಅಂ ಆಂ ಇಂ ಈಂ +++ ಅಃ} ಮಾಂ ರಕ್ಷ ರಕ್ಷ ತ್ವಗಾತ್ಮಾನಂ ನಮಃ ॥ (ಇತಿ ವಿಶುದ್ಧೌ ಕರ್ಣಿಕಾಯಾಂ ಡಾಕಿನೀಂ ವಿನ್ಯಸ್ಯ ತದ್ದಲೇಷು ಆವರಣ ಶಕ್ತೀರ್ನ್ಯಸೇತ್ ।)

\as{ಓಂ ಐಂಹ್ರೀಂಶ್ರೀಂ ಅಂ} ಅಮೃತಾಯೈ ನಮಃ~। \as{ಆಂ} ಆಕರ್ಷಿಣ್ಯೈ ನಮಃ~। \as{ಇಂ} ಇಂದ್ರಾಣ್ಯೈ ನಮಃ~। \as{ಈಂ} ಈಶಾನ್ಯೈ ನಮಃ~। \as{ಉಂ} ಉಮಾಯೈ ನಮಃ~। \as{ಊಂ} ಊರ್ಧ್ವಕೇಶ್ಯೈ ನಮಃ~। \as{ಋಂ} ಋದ್ಧಿದಾಯೈ ನಮಃ~। \as{ೠಂ} ೠಕಾರಾಯೈ ನಮಃ~। \as{ಲೃಂ} ಲೃಕಾರಾಯೈ ನಮಃ~। \as{ಲೄಂ} ಲೄಕಾರಾಯೈ ನಮಃ~। \as{ಏಂ} ಏಕಪದಾಯೈ ನಮಃ~। \as{ಐಂ} ಐಶ್ವರ್ಯಾತ್ಮಿಕಾಯೈ ನಮಃ~। \as{ಓಂ} ಓಂಕಾರಾಯೈ ನಮಃ~। \as{ಔಂ} ಔಷಧ್ಯೈ ನಮಃ~। \as{ಅಂ} ಅಂಬಿಕಾಯೈ ನಮಃ~। \as{ಅಃ} ಅಕ್ಷರಾಯೈ ನಮಃ~॥
\subsection{(ರಾಕಿಣೀಧ್ಯಾನಮ್)}
\as{ಹೃತ್ಪದ್ಮೇ ಭಾನುಪತ್ರೇ ದ್ವಿವದನ ಲಸಿತಾಂ ದಂಷ್ಟ್ರಿಣೀಂ ಶ್ಯಾಮವರ್ಣಾಂ \\ಅಕ್ಷಂ ಶೂಲಂ ಕಪಾಲಂ ಡಮರುಮಪಿ ಭುಜೈರ್ಧಾರಯಂತೀಂ ತ್ರಿಣೇತ್ರಾಂ ।\\ ರಕ್ತಸ್ಥಾಂ ಕಾಲರಾತ್ರಿ ಪ್ರಭೃತಿ ಪರಿವೃತಾಂ ಸ್ನಿಗ್ಧಭಕ್ತೈಕ ಸಕ್ತಾಂ\\ ಶ್ರೀಮದ್ವೀರೇಂದ್ರವಂದ್ಯಾಮಭಿಮತ ಫಲದಾಂ ರಾಕಿಣೀಂ ಭಾವಯಾಮಃ ॥}

\as{ಓಂ ಐಂಹ್ರೀಂಶ್ರೀಂ ರಾಂರೀಂರೂಂರೈಂರೌಂರಃ ।} ರಮಲವರಯೂಂ ರಾಕಿಣ್ಯೈ ನಮಃ । \as{೪ ಕಂ ಖಂ +++ ಠಂ} ಮಾಂ ರಕ್ಷ ರಕ್ಷ ಅಸೃಗಾತ್ಮಾನಂ ನಮಃ ॥ (ಇತಿ ಅನಾಹತೇ ಕರ್ಣಿಕಾಯಾಂ ರಾಕಿಣೀಂ ವಿನ್ಯಸ್ಯ ತದ್ದಲೇಷು ಆವರಣ ಶಕ್ತೀರ್ನ್ಯಸೇತ್ ।)

\as{ಓಂ ಐಂಹ್ರೀಂಶ್ರೀಂ ಕಂ} ಕಾಲರಾತ್ರ್ಯೈ ನಮಃ~। \as{ಖಂ} ಖಂಡಿತಾಯೈ ನಮಃ~। \as{ಗಂ} ಗಾಯತ್ರ್ಯೈ ನಮಃ~। \as{ಘಂ} ಘಂಟಾಕರ್ಷಿಣ್ಯೈ ನಮಃ~। \as{ಙಂ} ಙಾರ್ಣಾಯೈ ನಮಃ~। \as{ಚಂ} ಚಂಡಾಯೈ ನಮಃ~। \as{ಛಂ} ಛಾಯಾಯೈ ನಮಃ~। \as{ಜಂ} ಜಯಾಯೈ ನಮಃ~। \as{ಝಂ} ಝಂಕಾರಿಣ್ಯೈ ನಮಃ~। \as{ಞಂ} ಜ್ಞಾನರೂಪಾಯೈ ನಮಃ~। \as{ಟಂ} ಟಂಕಹಸ್ತಾಯೈ ನಮಃ~। \as{ಠಂ} ಠಂಕಾರಿಣ್ಯೈ ನಮಃ~॥
\subsection{(ಲಾಕಿನೀಧ್ಯಾನಮ್)}
\as{ದಿಕ್ಪತ್ರೇ ನಾಭಿಪದ್ಮೇ ತ್ರಿವದನಲಸಿತಾಂ ದಂಷ್ಟ್ರಿಣೀಂ ರಕ್ತವರ್ಣಾಂ\\ ಶಕ್ತಿಂ ದಂಭೋಲಿದಂಡಾವಭಯಮಪಿ ಭುಜೈರ್ಧಾರಯಂತೀಂ ಮಹೋಗ್ರಾಂ ।\\ ಡಾಮರ್ಯಾದ್ಯೈಃ ಪರೀತಾಂ ಪಶುಜನಭಯದಾಂ ಮಾಂಸಧಾತ್ವೇಕನಿಷ್ಠಾಂ\\ ಗೌಡಾನ್ನಾಸಕ್ತಚಿತ್ತಾಂ ಸಕಲಸುಖಕರೀಂ ಲಾಕಿನೀಂ ಭಾವಯಾಮಃ ॥}

\as{ಓಂ ಐಂಹ್ರೀಂಶ್ರೀಂ ಲಾಂಲೀಂಲೂಂಲೈಂಲೌಂಲಃ ।} ಲಮಲವರಯೂಂ ಲಾಕಿನ್ಯೈ ನಮಃ । \as{೪ ಡಂ ಢಂ +++ ಫಂ} ಮಾಂ ರಕ್ಷ ರಕ್ಷ ಮಾಂಸಾತ್ಮಾನಂ ನಮಃ ॥ (ಇತಿ ಮಣಿಪೂರೇ ಕರ್ಣಿಕಾಯಾಂ ಲಾಕಿನೀಂ ವಿನ್ಯಸ್ಯ ತದ್ದಲೇಷು ಆವರಣ ಶಕ್ತೀರ್ನ್ಯಸೇತ್ ।)

\as{ಓಂ ಐಂಹ್ರೀಂಶ್ರೀಂ ಡಂ} ಡಾಮರ್ಯೈ ನಮಃ~। \as{ಢಂ} ಢಂಕಾರಿಣ್ಯೈ ನಮಃ~। \as{ಣಂ} ಣಾರ್ಣಾಯೈ ನಮಃ~। \as{ತಂ} ತಾಮಸ್ಯೈ ನಮಃ~। \as{ಥಂ} ಸ್ಥಾಣ್ವ್ಯೈ ನಮಃ~। \as{ದಂ} ದಾಕ್ಷಾಯಣ್ಯೈ ನಮಃ~। \as{ಧಂ} ಧಾತ್ರ್ಯೈ ನಮಃ~। \as{ನಂ} ನಾರ್ಯೈ ನಮಃ~। \as{ಪಂ} ಪಾರ್ವತ್ಯೈ ನಮಃ~। \as{ಫಂ} ಫಟ್ಕಾರಿಣ್ಯೈ ನಮಃ~॥
\subsection{(ಕಾಕಿನೀಧ್ಯಾನಮ್)}
\as{ಸ್ವಾಧಿಷ್ಠಾನಾಖ್ಯಪದ್ಮೇ ರಸದಲಲಸಿತೇ ವೇದವಕ್ತ್ರಾಂ ತ್ರಿಣೇತ್ರಾಂ\\ ಹಸ್ತಾಬ್ಜೈರ್ಧಾರಯಂತೀಂ ತ್ರಿಶಿಖಗುಣಕಪಾಲಾಂಕುಶಾನಾತ್ತಗರ್ವಾಂ ।\\ ಮೇದೋಧಾತು ಪ್ರತಿಷ್ಠಾಮಲಿಮದಮುದಿತಾಂ ಬಂಧಿನೀ ಮುಖ್ಯಯುಕ್ತಾಂ\\ ಪೀತಾಂ ದಧ್ಯೋದನೇಷ್ಟಾಮಭಿಮತಫಲದಾಂ ಕಾಕಿನೀಂ ಭಾವಯಾಮಃ ॥}

\as{ಓಂ ಐಂಹ್ರೀಂಶ್ರೀಂ ಕಾಂಕೀಂಕೂಂಕೈಂಕೌಂಕಃ ।} ಕಮಲವರಯೂಂ ಕಾಕಿನ್ಯೈ ನಮಃ । \as{೪ ಬಂ ಭಂ ++ ಲಂ} ಮಾಂ ರಕ್ಷ ರಕ್ಷ ಮೇದ ಆತ್ಮಾನಂ ನಮಃ ॥(ಇತಿ ಸ್ವಾಧಿಷ್ಠಾನೇ ಕರ್ಣಿಕಾಯಾಂ ಕಾಕಿನೀಂ ವಿನ್ಯಸ್ಯ ತದ್ದಲೇಷು ಆವರಣ ಶಕ್ತೀರ್ನ್ಯಸೇತ್ ।)

\as{ಓಂ ಐಂಹ್ರೀಂಶ್ರೀಂ ಬಂ} ಬಂಧಿನ್ಯೈ ನಮಃ ~। \as{ಭಂ} ಭದ್ರಕಾಲ್ಯೈ ನಮಃ ~। \as{ಮಂ} ಮಹಾಮಾಯಾಯೈ ನಮಃ~। \as{ಯಂ} ಯಶಸ್ವಿನ್ಯೈ ನಮಃ~। \as{ರಂ} ರಕ್ತಾಯೈ ನಮಃ~। \as{ಲಂ} ಲಂಬೋಷ್ಠ್ಯೈ ನಮಃ~॥

\subsection{(ಸಾಕಿನೀಧ್ಯಾನಮ್)}
\as{ಮೂಲಾಧಾರಸ್ಥ ಪದ್ಮೇ ಶ್ರುತಿದಲ ಲಸಿತೇ ಪಂಚವಕ್ತ್ರಾಂ ತ್ರಿಣೇತ್ರಾಂ\\ ಧೂಮ್ರಾಭಾಮಸ್ಥಿ ಸಂಸ್ಥಾಂ ಸೃಣಿಮಪಿ ಕಮಲಂ ಪುಸ್ತಕಂ ಜ್ಞಾನಮುದ್ರಾಂ ।\\ ಬಿಭ್ರಾಣಾಂ ಬಾಹುದಂಡೈಃ ಸುಲಲಿತ ವರದಾಪೂರ್ವಶಕ್ತ್ಯಾವೃತಾಂ ತಾಂ \\ ಮುದ್ಗಾನ್ನಾಸಕ್ತಚಿತ್ತಾಂ ಮಧುಮದಮುದಿತಾಂ ಸಾಕಿನೀಂ ಭಾವಯಾಮಃ ॥}

\as{ಓಂ ಐಂಹ್ರೀಂಶ್ರೀಂ ಸಾಂಸೀಂಸೂಂಸೈಂಸೌಂಸಃ ।} ಸಮಲವರಯೂಂ ಸಾಕಿನ್ಯೈ ನಮಃ । \as{೪ ವಂ ಶಂ ಷಂ ಸಂ} ಮಾಂ ರಕ್ಷ ರಕ್ಷ ಅಸ್ಥ್ಯಾತ್ಮಾನಂ ನಮಃ ॥ (ಇತಿ ಮೂಲಾಧಾರೇ ಕರ್ಣಿಕಾಯಾಂ ಸಾಕಿನೀಂ ವಿನ್ಯಸ್ಯ ತದ್ದಲೇಷು ಆವರಣ ಶಕ್ತೀರ್ನ್ಯಸೇತ್ ।)

\as{ಓಂ ಐಂಹ್ರೀಂಶ್ರೀಂ ವಂ} ವರದಾಯೈ ನಮಃ~। \as{ಶಂ} ಶ್ರಿಯೈ ನಮಃ~। \as{ಷಂ} ಷಂಡಾಯೈ ನಮಃ~। \as{ಸಂ} ಸರಸ್ವತ್ಯೈ ನಮಃ~॥
\subsection{(ಹಾಕಿನೀಧ್ಯಾನಮ್)}
\as{ಭ್ರೂಮಧ್ಯೇ ಬಿಂದುಪದ್ಮೇ ದಲಯುಗಕಲಿತೇ ಶುಕ್ಲವರ್ಣಾಂ ಕರಾಬ್ಜೈಃ \\ಬಿಭ್ರಾಣಾಂ ಜ್ಞಾನಮುದ್ರಾಂ ಡಮರುಕಮಮಲಾಮಕ್ಷಮಾಲಾಂ ಕಪಾಲಂ ।\\ ಷಡ್ವಕ್ತ್ರಾಂ ಮಜ್ಜಸಂಸ್ಥಾಂ ತ್ರಿಣಯನಲಸಿತಾಂ ಹಂಸವತ್ಯಾದಿಯುಕ್ತಾಂ\\ ಹಾರಿದ್ರಾನ್ನೈಕಸಕ್ತಾಂ ಸಕಲಸುಖಕರೀಂ ಹಾಕಿನೀಂ ಭಾವಯಾಮಃ ॥}

\as{ಓಂ ಐಂಹ್ರೀಂಶ್ರೀಂ ಹಾಂಹೀಂಹೂಂಹೈಂಹೌಂಹಃ ।} ಹಮಲವರಯೂಂ ಹಾಕಿನ್ಯೈ ನಮಃ । \as{೪ ಹಂ ಕ್ಷಂ} ಮಾಂ ರಕ್ಷ ರಕ್ಷ ಮಜ್ಜಾತ್ಮಾನಂ ನಮಃ ॥ (ಇತಿ ಆಜ್ಞಾಯಾಂ ಕರ್ಣಿಕಾಯಾಂ ಹಾಕಿನೀಂ ವಿನ್ಯಸ್ಯ ತದ್ದಲೇಷು ಆವರಣ ಶಕ್ತೀರ್ನ್ಯಸೇತ್ ।)

\as{ಓಂ ಐಂಹ್ರೀಂಶ್ರೀಂ ಹಂ} ಹಂಸವತ್ಯೈ ನಮಃ~। \as{ಕ್ಷಂ} ಕ್ಷಮಾವತ್ಯೈ ನಮಃ ॥
\subsection{(ಯಾಕಿನೀಧ್ಯಾನಮ್)}
\as{ಮುಂಡವ್ಯೋಮಸ್ಥಪದ್ಮೇ ದಶಶತದಲಕೇ ಕರ್ಣಿಕಾಚಂದ್ರಸಂಸ್ಥಾಂ\\ ರೇತೋನಿಷ್ಠಾಂ ಸಮಸ್ತಾಯುಧಕಲಿತಕರಾಂ ಸರ್ವತೋವಕ್ತ್ರಪದ್ಮಾಂ ।\\ ಆದಿಕ್ಷಾಂತಾರ್ಣಶಕ್ತಿಪ್ರಕರಪರಿವೃತಾಂ ಸರ್ವವರ್ಣಾಂ ಭವಾನೀಂ\\ ಸರ್ವಾನ್ನಾಸಕ್ತಚಿತ್ತಾಂ ಪರಶಿವರಸಿಕಾಂ ಯಾಕಿನೀಂ ಭಾವಯಾಮಃ ॥}

\as{ಓಂ ಐಂಹ್ರೀಂಶ್ರೀಂ ಯಾಂಯೀಂಯೂಂಯೈಂಯೌಂಯಃ ।} ಯಮಲವರಯೂಂ ಯಾಕಿನ್ಯೈ ನಮಃ । \as{೪ ಅಂ ಆಂ ++++ ಹಂ ಕ್ಷಂ} ಮಾಂ ರಕ್ಷ ರಕ್ಷ ಶುಕ್ರಾತ್ಮಾನಂ ನಮಃ ॥ (ಇತಿ ಸಹಸ್ರಾರೇ ಕರ್ಣಿಕಾಯಾಂ ಯಾಕಿನೀಂ ವಿನ್ಯಸ್ಯ ತದ್ದಲೇಷು ಆವರಣ ಶಕ್ತೀರ್ನ್ಯಸೇತ್ ।)
\begin{multicols}{2}
\as{ಓಂ ಐಂಹ್ರೀಂಶ್ರೀಂ \\ಅಂ} ಅಮೃತಾಯೈ ನಮಃ~।\\ \as{ಆಂ} ಆಕರ್ಷಿಣ್ಯೈ ನಮಃ~।\\ \as{ಇಂ} ಇಂದ್ರಾಣ್ಯೈ ನಮಃ~।\\ \as{ಈಂ} ಈಶಾನ್ಯೈ ನಮಃ~।\\ \as{ಉಂ} ಉಮಾಯೈ ನಮಃ~।\\ \as{ಊಂ} ಊರ್ಧ್ವಕೇಶ್ಯೈ ನಮಃ~।\\ \as{ಋಂ} ಋದ್ಧಿದಾಯೈ ನಮಃ~।\\ \as{ೠಂ} ೠಕಾರಾಯೈ ನಮಃ~।\\ \as{ಲೃಂ} ಲೃಕಾರಾಯೈ ನಮಃ~।\\ \as{ಲೄಂ} ಲೄಕಾರಾಯೈ ನಮಃ~।\\ \as{ಏಂ} ಏಕಪದಾಯೈ ನಮಃ~।\\ \as{ಐಂ} ಐಶ್ವರ್ಯಾತ್ಮಿಕಾಯೈ ನಮಃ~।\\ \as{ಓಂ} ಓಂಕಾರಾಯೈ ನಮಃ~।\\ \as{ಔಂ} ಔಷಧ್ಯೈ ನಮಃ~।\\ \as{ಅಂ} ಅಂಬಿಕಾಯೈ ನಮಃ~।\\ \as{ಅಃ} ಅಕ್ಷರಾಯೈ ನಮಃ~।\\ \as{ಕಂ} ಕಾಲರಾತ್ರ್ಯೈ ನಮಃ~।\\ \as{ಖಂ} ಖಂಡಿತಾಯೈ ನಮಃ~।\\ \as{ಗಂ} ಗಾಯತ್ರ್ಯೈ ನಮಃ~।\\ \as{ಘಂ} ಘಂಟಾಕರ್ಷಿಣ್ಯೈ ನಮಃ~।\\ \as{ಙಂ} ಙಾರ್ಣಾಯೈ ನಮಃ~।\\ \as{ಚಂ} ಚಂಡಾಯೈ ನಮಃ~।\\ \as{ಛಂ} ಛಾಯಾಯೈ ನಮಃ~।\\ \as{ಜಂ} ಜಯಾಯೈ ನಮಃ~।\\ \as{ಝಂ} ಝಂಕಾರಿಣ್ಯೈ ನಮಃ~।\\ \as{ಞಂ} ಜ್ಞಾನರೂಪಾಯೈ ನಮಃ~।\\ \as{ಟಂ} ಟಂಕಹಸ್ತಾಯೈ ನಮಃ~।\\ \as{ಠಂ} ಠಂಕಾರಿಣ್ಯೈ ನಮಃ~।\\ \as{ಡಂ} ಡಾಮರ್ಯೈ ನಮಃ~।\\ \as{ಢಂ} ಢಂಕಾರಿಣ್ಯೈ ನಮಃ~।\\ \as{ಣಂ} ಣಾರ್ಣಾಯೈ ನಮಃ~।\\ \as{ತಂ} ತಾಮಸ್ಯೈ ನಮಃ~।\\ \as{ಥಂ} ಸ್ಥಾಣ್ವ್ಯೈ ನಮಃ~।\\ \as{ದಂ} ದಾಕ್ಷಾಯಣ್ಯೈ ನಮಃ~।\\ \as{ಧಂ} ಧಾತ್ರ್ಯೈ ನಮಃ~।\\ \as{ನಂ} ನಾರ್ಯೈ ನಮಃ~।\\ \as{ಪಂ} ಪಾರ್ವತ್ಯೈ ನಮಃ~।\\ \as{ಫಂ} ಫಟ್ಕಾರಿಣ್ಯೈ ನಮಃ~।\\ \as{ಬಂ} ಬಂಧಿನ್ಯೈ ನಮಃ ~।\\ \as{ಭಂ} ಭದ್ರಕಾಲ್ಯೈ ನಮಃ ~।\\ \as{ಮಂ} ಮಹಾಮಾಯಾಯೈ ನಮಃ~।\\ \as{ಯಂ} ಯಶಸ್ವಿನ್ಯೈ ನಮಃ~।\\ \as{ರಂ} ರಕ್ತಾಯೈ ನಮಃ~।\\ \as{ಲಂ} ಲಂಬೋಷ್ಠ್ಯೈ ನಮಃ~।\\ \as{ವಂ} ವರದಾಯೈ ನಮಃ~।\\ \as{ಶಂ} ಶ್ರಿಯೈ ನಮಃ~।\\ \as{ಷಂ} ಷಂಡಾಯೈ ನಮಃ~।\\ \as{ಸಂ} ಸರಸ್ವತ್ಯೈ ನಮಃ~।\\ \as{ಹಂ} ಹಂಸವತ್ಯೈ ನಮಃ~।\\ \as{ಕ್ಷಂ} ಕ್ಷಮಾವತ್ಯೈ ನಮಃ ॥
\end{multicols}
\subsection{ರಾಶಿನ್ಯಾಸಃ}
\dhyana{ರಕ್ತಶ್ವೇತ ಹರಿತ್ಪಾಂಡುಚಿತ್ರಕೃಷ್ಣ ಪಿಶಂಗಕಾನ್ ।\\
ಕಪಿಶಬಭ್ರುಕಿಮ್ಮೀರಕೃಷ್ಣಧೂಮ್ರಾನ್ ಕ್ರಮಾತ್ ಸ್ಮರೇತ್ ॥}
 
\as{ಓಂ ಐಂಹ್ರೀಂಶ್ರೀಂ ಅಂಆಂಇಂಈಂ} ಮೇಷಾಯ ನಮಃ ।(ದಕ್ಷಪಾದೇ)\\
\as{೪ ಉಂಊಂ} ವೃಷಭಾಯ ನಮಃ ।(ಲಿಂಗದಕ್ಷಭಾಗೇ)\\
\as{೪ ಋಂೠಂಲೃಂಲೄಂ} ಮಿಥುನಾಯ ನಮಃ ।(ದಕ್ಷಕುಕ್ಷೌ)\\
\as{೪ ಏಂಐಂ} ಕರ್ಕಾಯ ನಮಃ ।(ಹೃದಯದಕ್ಷಭಾಗೇ)\\
\as{೪ ಓಂಔಂ} ಸಿಂಹಾಯ ನಮಃ ।(ದಕ್ಷಬಾಹುಮೂಲೇ)\\
\as{೪ ಅಂಅಃ ಶಂಷಂಸಂಹಂಳಂ} ಕನ್ಯಾಯೈ ನಮಃ ।(ಶಿರೋದಕ್ಷಭಾಗೇ)\\
\as{೪ ಕಂಖಂಗಂಘಂಙಂ} ತುಲಾಯೈ ನಮಃ ।(ಶಿರೋವಾಮಭಾಗೇ)\\
\as{೪ ಚಂಛಂಜಂಝಂಞಂ} ವೃಶ್ಚಿಕಾಯ ನಮಃ ।(ವಾಮಬಾಹುಮೂಲೇ)\\
\as{೪ ಟಂಠಂಡಂಢಂಣಂ} ಧನುಷೇ ನಮಃ ।(ಹೃದಯವಾಮಭಾಗೇ)\\
\as{೪ ತಂಥಂದಂಧಂನಂ} ಮಕರಾಯ ನಮಃ ।(ವಾಮಕುಕ್ಷೌ)\\
\as{೪ ಪಂಫಂಬಂಭಂಮಂ} ಕುಂಭಾಯ ನಮಃ ।(ಲಿಂಗವಾಮಭಾಗೇ)\\
\as{೪ ಯಂರಂಲಂವಂಕ್ಷಂ~।} ಮೀನಾಯ ನಮಃ ।(ವಾಮಪಾದೇ)
\subsection{ಪೀಠನ್ಯಾಸಃ}
\dhyana{ಸಿತಾಸಿತಾರುಣಶ್ಯಾಮಹರಿತ್ಪೀತಾನ್ಯನುಕ್ರಮಾತ್\\ ಪುನಃ ಕ್ರಮೇಣ ದೇವೇಶಿ ಪಂಚಾಶತ್ಪೀಠಸಂಚಯಃ ॥}

\as{ಓಂ ಐಂಹ್ರೀಂಶ್ರೀಂ ಅಂ} ಕಾಮರೂಪಾಯ ನಮಃ । (ಶಿರಸಿ)\\
\as{೪ ಆಂ} ವಾರಾಣಸ್ಯೈ ನಮಃ ।(ಮುಖವೃತ್ತೇ)\\
\as{೪ ಇಂ} ನೇಪಾಲಾಯ ನಮಃ ।(ದಕ್ಷನೇತ್ರೇ)\\
\as{೪ ಈಂ} ಪೌಂಡ್ರವರ್ಧನಾಯ ನಮಃ ।(ವಾಮನೇತ್ರೇ)\\
\as{೪ ಉಂ} ಪುರಃಸ್ಥಿತಕಾಶ್ಮೀರಾಯ ನಮಃ ।(ದಕ್ಷಕರ್ಣೇ)\\
\as{೪ ಊಂ} ಕಾನ್ಯಕುಬ್ಜಾಯ ನಮಃ ।(ವಾಮಕರ್ಣೇ)\\
\as{೪ ಋಂ} ಪೂರ್ಣಶೈಲಾಯ ನಮಃ ।(ದಕ್ಷನಾಸಾಯಾಂ)\\
\as{೪ ೠಂ} ಅರ್ಬುದಾಚಲಾಯ ನಮಃ ।(ವಾಮನಾಸಾಯಾಂ)\\
\as{೪ ಲೃಂ} ಆಮ್ರಾತಕೇಶ್ವರಾಯ ನಮಃ ।(ದಕ್ಷಗಂಡೇ)\\
\as{೪ ಲೄಂ} ಏಕಾಮ್ರಾಯ ನಮಃ ।(ವಾಮಗಂಡೇ)\\
\as{೪ ಏಂ} ತ್ರಿಸ್ರೋತಸೇ ನಮಃ ।(ಊರ್ಧ್ವೋಷ್ಠೇ)\\
\as{೪ ಐಂ} ಕಾಮಕೋಟಯೇ ನಮಃ ।(ಅಧರೋಷ್ಠೇ)\\
\as{೪ ಓಂ} ಕೈಲಾಸಾಯ ನಮಃ ।(ಊರ್ಧ್ವದಂತಪಂಕ್ತೌ)\\
\as{೪ ಔಂ} ಭೃಗುನಗರಾಯ ನಮಃ ।(ಅಧೋದಂತಪಂಕ್ತೌ)\\
\as{೪ ಅಂ} ಕೇದಾರಾಯ ನಮಃ ।(ಜಿಹ್ವಾಯಾಂ)\\
\as{೪ ಅಃ} ಚಂದ್ರಪುಷ್ಕರಿಣ್ಯೈ ನಮಃ ।(ಕಂಠೇ)\\
\as{೪ ಕಂ} ಶ್ರೀಪುರಾಯ ನಮಃ ।(ದಕ್ಷ ಬಾಹುಮೂಲೇ)\\
\as{೪ ಖಂ} ಓಂಕಾರಾಯ ನಮಃ ।(ದಕ್ಷಕೂರ್ಪರೇ)\\
\as{೪ ಗಂ} ಜಾಲಂಧರಾಯ ನಮಃ ।(ದಕ್ಷಮಣಿಬಂಧೇ)\\
\as{೪ ಘಂ} ಮಾಲವಾಯ ನಮಃ ।(ದಕ್ಷಕರಾಂಗುಲಿಮೂಲೇ)\\
\as{೪ ಙಂ} ಕುಲಾಂತಕಾಯ ನಮಃ ।(ದಕ್ಷಕರಾಂಗುಲ್ಯಗ್ರೇ)\\
\as{೪ ಚಂ} ದೇವೀಕೂಟಾಯ ನಮಃ ।(ವಾಮಬಾಹುಮೂಲೇ)\\
\as{೪ ಛಂ} ಗೋಕರ್ಣಾಯ ನಮಃ ।(ವಾಮಕೂರ್ಪರೇ)\\
\as{೪ ಜಂ} ಮಾರುತೇಶ್ವರಾಯ ನಮಃ ।(ವಾಮಮಣಿಬಂಧೇ)\\
\as{೪ ಝಂ} ಅಟ್ಟಹಾಸಾಯ ನಮಃ ।(ವಾಮಕರಾಂಗುಲಿಮೂಲೇ)\\
\as{೪ ಞಂ} ವಿರಜಾಯೈ ನಮಃ ।(ವಾಮಕರಾಂಗುಲ್ಯಗ್ರೇ)\\
\as{೪ ಟಂ} ರಾಜಗೇಹಾಯ ನಮಃ ।(ದಕ್ಷೋರುಮೂಲೇ)\\
\as{೪ ಠಂ} ಮಹಾಪಥಾಯ ನಮಃ । (ದಕ್ಷಜಾನುನಿ)\\
\as{೪ ಡಂ} ಕೋಲ್ಹಾಪುರಾಯ ನಮಃ । (ದಕ್ಷಗುಲ್ಫೇ)\\
\as{೪ ಢಂ} ಏಲಾಪುರಾಯ ನಮಃ । (ದಕ್ಷಪಾದಾಂಗುಲಿಮೂಲೇ)\\
\as{೪ ಣಂ} ಕಾಲೇಶ್ವರಾಯ ನಮಃ । (ದಕ್ಷಪಾದಾಂಗುಲ್ಯಗ್ರೇ)\\
\as{೪ ತಂ} ಜಯಂತಿಕಾಯೈ ನಮಃ । (ವಾಮೋರುಮೂಲೇ)\\
\as{೪ ಥಂ} ಉಜ್ಜಯಿನ್ಯೈ ನಮಃ । (ವಾಮಜಾನುನಿ)\\
\as{೪ ದಂ} ಚಿತ್ರಾಯೈ ನಮಃ । (ವಾಮಗುಲ್ಫೇ)\\
\as{೪ ಧಂ} ಕ್ಷೀರಿಕಾಯೈ ನಮಃ । (ವಾಮಪಾದಾಂಗುಲಿಮೂಲೇ)\\
\as{೪ ನಂ} ಹಸ್ತಿನಾಪುರಾಯ ನಮಃ । (ವಾಮಪಾದಾಂಗುಲ್ಯಗ್ರೇ)\\
\as{೪ ಪಂ} ಉಡ್ಡೀಶಾಯ ನಮಃ । (ದಕ್ಷಪಾರ್ಶ್ವೇ)\\
\as{೪ ಫಂ} ಪ್ರಯಾಗಾಯ ನಮಃ । (ವಾಮಪಾರ್ಶ್ವೇ)\\
\as{೪ ಬಂ} ಷಷ್ಠೀಶಾಯ ನಮಃ । (ಪೃಷ್ಠೇ)\\
\as{೪ ಭಂ} ಮಾಯಾಪುರ್ಯೈ ನಮಃ । (ನಾಭೌ)\\
\as{೪ ಮಂ} ಜಲೇಶಾಯ ನಮಃ । (ಜಠರೇ)\\
\as{೪ ಯಂ} ಮಲಯಾಯ ನಮಃ । (ಹೃದಿ )\\
\as{೪ ರಂ} ಶ್ರೀಶೈಲಾಯ ನಮಃ । (ದಕ್ಷಾಂಸೇ)\\
\as{೪ ಲಂ} ಮೇರವೇ ನಮಃ । (ಕಕುದಿ)\\
\as{೪ ವಂ} ಗಿರಿವರಾಯ ನಮಃ । (ವಾಮಾಂಸೇ)\\
\as{೪ ಶಂ} ಮಹೇಂದ್ರಾಯ ನಮಃ । (ಹೃದಯಾದಿ ದಕ್ಷ ಕರಾಂಗುಲ್ಯಂತಂ )\\
\as{೪ ಷಂ} ವಾಮನಾಯ ನಮಃ । (ಹೃದಯಾದಿ ವಾಮ ಕರಾಂಗುಲ್ಯಂತಂ )\\
\as{೪ ಸಂ} ಹಿರಣ್ಯಪುರಾಯ ನಮಃ । (ಹೃದಯಾದಿ ದಕ್ಷ ಪಾದಾಂತಂ )\\
\as{೪ ಹಂ} ಮಹಾಲಕ್ಷ್ಮೀಪುರಾಯ ನಮಃ । (ಹೃದಯಾದಿ ವಾಮ ಪಾದಾಂತಂ )\\
\as{೪ ಳಂ} ಓಡ್ಯಾಣಾಯ ನಮಃ । (ಕಟ್ಯಾದಿ ಪಾದಪರ್ಯಂತಂ)\\
\as{೪ ಕ್ಷಂ} ಛಾಯಾಛತ್ರಾಯ ನಮಃ । ।(ಕಟ್ಯಾದಿ ಶಿರಃಪರ್ಯಂತಂ )\\
ಪುನಃ ಪೂರ್ವೋಕ್ತಪ್ರಕಾರೇಣ ಉತ್ತರನ್ಯಾಸಂ ವಿಧಾಯ ಧ್ಯಾಯೇತ್ ॥

\dhyana{ಉದ್ಯತ್ಸೂರ್ಯಸಹಸ್ರಾಭಾಂ ಪೀನೋನ್ನತಪಯೋಧರಾಂ ।\\ರಕ್ತಮಾಲ್ಯಾಂಬರಾಲೇಪಾಂ ರತ್ನಭೂಷಣಭೂಷಿತಾಂ ॥

ಪಾಶಾಂಕುಶಧನುರ್ಬಾಣಭಾಸ್ವತ್ಪಾಣಿಚತುಷ್ಟಯಾಂ ।\\ ಲಸನ್ನೇತ್ರತ್ರಯಾಂ ಸ್ವರ್ಣಮಕುಟೋದ್ಭಾಸಿ ಮಸ್ತಕಾಂ ॥

ಗಣೇಶಗ್ರಹನಕ್ಷತ್ರಯೋಗಿನೀ ರಾಶಿರೂಪಿಣೀಂ ।\\ದೇವೀಂ ಪೀಠಮಯೀಂ ಧ್ಯಾಯೇನ್ಮಾತೃಕಾಂ ಸುಂದರೀಂ ಪರಾಂ ॥}
\authorline{ಇತಿ ಲಘುಷೋಢಾನ್ಯಾಸಃ}
\newpage
\section{ಮಹಾಷೋಢಾನ್ಯಾಸಃ}
ಅಸ್ಯ ಶ್ರೀಮಹಾಷೋಢಾ ನ್ಯಾಸಸ್ಯ ಬ್ರಹ್ಮಾ ಋಷಿಃ (ಶಿರಸಿ)। ಜಗತೀ ಚ್ಛಂದಃ (ಮುಖೇ)। ಶ್ರೀಮದರ್ಧನಾರೀಶ್ವರೋ ದೇವತಾ (ಹೃದಯೇ) । ಶ್ರೀವಿದ್ಯಾಂಗತ್ವೇನ ನ್ಯಾಸೇ ವಿನಿಯೋಗಃ (ಸರ್ವಾಂಗೇ)।

\as{(ಅಂಗುಲೀನ್ಯಾಸಃ)}\\
(ಅಂಗುಷ್ಠಯೋಃ)\\\as{ಓಂ ಐಂಹ್ರೀಂಶ್ರೀಂ ಹ್ಸೌಂಃ ಹ್ಸೌಃ ಹೋಂ} ಈಶಾನಾಯ ನಮಃ ॥\\
(ತರ್ಜನ್ಯೋಃ)\as{೬ ಹೇಂ} ತತ್ಪುರುಷಾಯ ನಮಃ ॥\\
(ಮಧ್ಯಮಯೋಃ)\as{೬ ಹುಂ} ಅಘೋರಾಯ ನಮಃ ॥\\
(ಅನಾಮಿಕಯೋಃ)\as{೬ ಹಿಂ} ವಾಮದೇವಾಯ ನಮಃ ॥\\
(ಕನಿಷ್ಠಿಕಯೋಃ)\as{೬ ಹಂ} ಸದ್ಯೋಜಾತಾಯ ನಮಃ ॥

\as{(ಅಂಗನ್ಯಾಸಃ)}\\
(ಶಿರಸಿ)\as{ಓಂ ಐಂಹ್ರೀಂಶ್ರೀಂ ಹ್ಸೌಂಃ ಹ್ಸೌಃ ಹೋಂ} ಈಶಾನಾಯ ನಮಃ ॥\\
(ಮುಖೇ)\as{೬ ಹೇಂ} ತತ್ಪುರುಷಾಯ ನಮಃ ॥\\
(ಹೃದಯೇ)\as{೬ ಹುಂ} ಅಘೋರಾಯ ನಮಃ ॥\\
(ಗುಹ್ಯೇ)\as{೬ ಹಿಂ} ವಾಮದೇವಾಯ ನಮಃ ॥\\
(ಪಾದಯೋಃ)\as{೬ ಹಂ} ಸದ್ಯೋಜಾತಾಯ ನಮಃ ॥

\as{(ಪಂಚವಕ್ತ್ರನ್ಯಾಸಃ)}\\
(ಶಿರಸಿ)\\\as{ಓಂ ಐಂಹ್ರೀಂಶ್ರೀಂ ಹ್ಸೌಂಃ ಹ್ಸೌಃ ಹೋಂ} ಈಶಾನಾಯೋರ್ಧ್ವವಕ್ತ್ರಾಯ ನಮಃ ॥\\
(ಮುಖೇ)\as{೬ ಹೇಂ} ತತ್ಪುರುಷಾಯ ಪೂರ್ವವಕ್ತ್ರಾಯ ನಮಃ ॥\\
(ದಕ್ಷಕರ್ಣೇ)\as{೬ ಹುಂ} ಅಘೋರಾಯ ದಕ್ಷಿಣವಕ್ತ್ರಾಯ ನಮಃ ॥\\
(ವಾಮಕರ್ಣೇ)\as{೬ ಹಿಂ} ವಾಮದೇವಾಯೋತ್ತರವಕ್ತ್ರಾಯ ನಮಃ ॥\\
(ಕೇಶೇಷು)\as{೬ ಹಂ} ಸದ್ಯೋಜಾತಾಯ ಪಶ್ಚಿಮವಕ್ತ್ರಾಯ ನಮಃ ॥

\as{(ಅಂಗುಲೀಷು ಪಂಚವಕ್ತ್ರನ್ಯಾಸಃ)}\\
(ಅಂಗುಷ್ಠಯೋಃ)\\\as{ಓಂ ಐಂಹ್ರೀಂಶ್ರೀಂ ಹ್ಸೌಂಃ ಹ್ಸೌಃ ಹೋಂ} ಈಶಾನಾಯೋರ್ಧ್ವವಕ್ತ್ರಾಯ ನಮಃ ॥\\
(ತರ್ಜನ್ಯೋಃ)\as{೬ ಹೇಂ} ತತ್ಪುರುಷಾಯ ಪೂರ್ವವಕ್ತ್ರಾಯ ನಮಃ ॥\\
(ಮಧ್ಯಮಯೋಃ)\as{೬ ಹುಂ} ಅಘೋರಾಯ ದಕ್ಷಿಣವಕ್ತ್ರಾಯ ನಮಃ ॥\\
(ಅನಾಮಿಕಯೋಃ)\as{೬ ಹಿಂ} ವಾಮದೇವಾಯೋತ್ತರವಕ್ತ್ರಾಯ ನಮಃ ॥\\
(ಕನಿಷ್ಠಿಕಯೋಃ)\as{೬ ಹಂ} ಸದ್ಯೋಜಾತಾಯ ಪಶ್ಚಿಮವಕ್ತ್ರಾಯ ನಮಃ ॥

ಇತಿ ಅಂಗೇಷು ವಿನ್ಯಸ್ಯ, \as{ಹ್ಸಾಂ, ಹ್ಸೀಂ} ಇತ್ಯಾದಿನಾ ಕರನ್ಯಾಸಂ ಷಡಂಗನ್ಯಾಸಂ ಚ ವಿಧಾಯ ॥\\
ಭೂರ್ಭುವಃ ಸುವರೋಮಿತಿ ದಿಗ್ಬಂಧಃ ॥
\section{ಧ್ಯಾನಂ}
\dhyana{ಪಂಚವಕ್ತ್ರಂ ಚತುರ್ಬಾಹುಂ ಸರ್ವಾಭರಣ ಭೂಷಿತಮ್ ।\\
ಚಂದ್ರಸೂರ್ಯಸಹಸ್ರಾಭಂ ಶಿವಶಕ್ತ್ಯಾತ್ಮಕಂ ಭಜೇ ॥

ಅಮೃತಾರ್ಣವಮಧ್ಯಸ್ಥ ರತ್ನದ್ವೀಪೇ ಮನೋರಮೇ~।\\
ಕಲ್ಪವೃಕ್ಷವನಾಂತಸ್ಥೇ ನವಮಾಣಿಕ್ಯಮಂಡಪೇ ॥

ನವರತ್ನಮಯ ಶ್ರೀಮತ್ ಸಿಂಹಾಸನಾಗತಾಂಬುಜೇ~।\\
ತ್ರಿಕೋಣಾಂತಃ ಸಮಾಸೀನಂ ಚಂದ್ರಸೂರ್ಯಾಯುತಪ್ರಭಂ ॥

ಅರ್ಧಾಂಬಿಕಾಸಮಾಯುಕ್ತಂ ಪ್ರವಿಭಕ್ತವಿಭೂಷಣಂ~।\\
ಕೋಟಿಕಂದರ್ಪಲಾವಣ್ಯಂ ಸದಾ ಷೋಡಶವಾರ್ಷಿಕಂ ॥

ಮಂದಸ್ಮಿತಮುಖಾಂಭೋಜಂ ತ್ರಿನೇತ್ರಂ ಚಂದ್ರಶೇಖರಂ~।\\
ದಿವ್ಯಾಂಬರಸ್ರಗಾಲೇಪಂ ದಿವ್ಯಾಭರಣಭೂಷಿತಂ ॥

ಪಾನಪಾತ್ರಂ ಚ ಚಿನ್ಮುದ್ರಾಂ ತ್ರಿಶೂಲಂ ಪುಸ್ತಕಂ ಕರೈಃ~।\\
ವಿದ್ಯಾ ಸಂಸದಿ ಬಿಭ್ರಾಣಂ ಸದಾನಂದಮುಖೇಕ್ಷಣಂ ॥

ಮಹಾಷೋಢೋದಿತಾಶೇಷದೇವತಾಗಣಸೇವಿತಂ~।\\
ಏವಂ ಚಿತ್ತಾಂಬುಜೇ ಧ್ಯಾಯೇದರ್ಧನಾರೀಶ್ವರಂ ಶಿವಂ ॥

ಪುರುಷಂ ವಾ ಸ್ಮರೇದ್ ದೇವಿ ಸ್ತ್ರೀರೂಪಂ ವಾ ವಿಚಿಂತಯೇತ್~।\\
ಅಥವಾ ನಿಷ್ಕಲಂ ಧ್ಯಾಯೇತ್ ಸಚ್ಚಿದಾನಂದವಿಗ್ರಹಂ~।\\
ಸರ್ವತೇಜೋಮಯಂ ದಿವ್ಯಂ ಸಚರಾಚರವಿಗ್ರಹಂ ॥}\\
\as{ಲಂ} ಇತ್ಯಾದಿನಾ ಪಂಚೋಪಚಾರೈಃ ಸಂಪೂಜ್ಯ ।
\subsection{ಪ್ರಪಂಚನ್ಯಾಸಃ}
\as{ಓಂ ಐಂಹ್ರೀಂಶ್ರೀಂ ಹ್ಸೌಂಃ ಹ್ಸೌಃ \\ ಅಂ} ಪ್ರಪಂಚರೂಪಾಯೈ ಶ್ರಿಯೈ ನಮಃ । (ಶಿರಸಿ)\\
\as{೬ ಆಂ} ದ್ವೀಪರೂಪಾಯೈ ಮಾಯಾಯೈ ನಮಃ । (ಮುಖವೃತ್ತೇ)\\
\as{೬ ಇಂ} ಜಲಧಿರೂಪಾಯೈ ಕಮಲಾಯೈ ನಮಃ । (ದಕ್ಷನೇತ್ರೇ)\\
\as{೬ ಈಂ} ಗಿರಿರೂಪಾಯೈ ವಿಷ್ಣುವಲ್ಲಭಾಯೈ ನಮಃ । (ವಾಮನೇತ್ರೇ)\\
\as{೬ ಉಂ} ಪತ್ತನರೂಪಾಯೈ ಪದ್ಮಧಾರಿಣ್ಯೈ ನಮಃ । (ದಕ್ಷಕರ್ಣೇ)\\
\as{೬ ಊಂ} ಪೀಠರೂಪಾಯೈ ಸಮುದ್ರತನಯಾಯೈ ನಮಃ । (ವಾಮಕರ್ಣೇ)\\
\as{೬ ಋಂ} ಕ್ಷೇತ್ರರೂಪಾಯೈ ಲೋಕಮಾತ್ರೇ ನಮಃ । (ದಕ್ಷನಾಸಾಯಾಂ)\\
\as{೬ ೠಂ} ವನರೂಪಾಯೈ ಕಮಲವಾಸಿನ್ಯೈ ನಮಃ । (ವಾಮನಾಸಾಯಾಂ)\\
\as{೬ ಲೃಂ} ಆಶ್ರಮರೂಪಾಯೈ ಇಂದಿರಾಯೈ ನಮಃ । (ದಕ್ಷಗಂಡೇ)\\
\as{೬ ಲೄಂ} ಗುಹಾರೂಪಾಯೈ ಮಾಯಾಯೈ ನಮಃ । (ವಾಮಗಂಡೇ)\\
\as{೬ ಏಂ} ನದೀರೂಪಾಯೈ ರಮಾಯೈ ನಮಃ । (ಊರ್ಧ್ವೋಷ್ಠೇ)\\
\as{೬ ಐಂ} ಚತ್ವರರೂಪಾಯೈ ಪದ್ಮಾಯೈ ನಮಃ । (ಅಧರೋಷ್ಠೇ)\\
\as{೬ ಓಂ} ಉದ್ಭಿಜ್ಜರೂಪಾಯೈ ನಾರಾಯಣಪ್ರಿಯಾಯೈ ನಮಃ ।\\ (ಊರ್ಧ್ವದಂತಪಂಕ್ತೌ)\\
\as{೬ ಔಂ} ಸ್ವೇದಜರೂಪಾಯೈ ಸಿದ್ಧಲಕ್ಷ್ಮ್ಯೈ ನಮಃ । (ಅಧೋದಂತಪಂಕ್ತೌ)\\
\as{೬ ಅಂ} ಅಂಡಜರೂಪಾಯೈ ರಾಜಲಕ್ಷ್ಮ್ಯೈ ನಮಃ । (ಜಿಹ್ವಾಯಾಂ)\\
\as{೬ ಅಃ} ಜರಾಯುಜರೂಪಾಯೈ ಮಹಾಲಕ್ಷ್ಮ್ಯೈ ನಮಃ । (ಕಂಠೇ)\\
\as{೬ ಕಂ} ಲವರೂಪಾಯೈ ಆರ್ಯಾಯೈ ನಮಃ । (ದಕ್ಷ ಬಾಹುಮೂಲೇ)\\
\as{೬ ಖಂ} ತೃಪ್ತಿರೂಪಾಯೈ ಉಮಾಯೈ ನಮಃ । (ದಕ್ಷಕೂರ್ಪರೇ)\\
\as{೬ ಗಂ} ಕಲಾರೂಪಾಯೈ ಚಂಡಿಕಾಯೈ ನಮಃ । (ದಕ್ಷಮಣಿಬಂಧೇ)\\
\as{೬ ಘಂ} ಕಷ್ಠಾರೂಪಾಯೈ ದುರ್ಗಾಯೈ ನಮಃ । (ದಕ್ಷಕರಾಂಗುಲಿಮೂಲೇ)\\
\as{೬ ಙಂ} ನಿಮೇಷರೂಪಾಯೈ ಶಿವಾಯೈ ನಮಃ । (ದಕ್ಷಕರಾಂಗುಲ್ಯಗ್ರೇ)\\
\as{೬ ಚಂ} ಶ್ವಾಸರೂಪಾಯೈಅಪರ್ಣಾಯೈ ನಮಃ ।\\ (ವಾಮಬಾಹುಮೂಲೇ)\\
\as{೬ ಛಂ} ಘಟಿಕಾರೂಪಾಯೈ ಅಂಬಿಕಾಯೈ ನಮಃ । (ವಾಮಕೂರ್ಪರೇ)\\
\as{೬ ಜಂ} ಮುಹೂರ್ತರೂಪಾಯೈ ಸತ್ಯೈ ನಮಃ । (ವಾಮಮಣಿಬಂಧೇ)\\
\as{೬ ಝಂ} ಪ್ರಹರರೂಪಾಯೈ ಈಶ್ವರ್ಯೈ ನಮಃ । \\(ವಾಮಕರಾಂಗುಲಿಮೂಲೇ)\\
\as{೬ ಞಂ} ದಿವಸರೂಪಾಯೈ ಶಾಂಭವ್ಯೈ ನಮಃ । (ವಾಮಕರಾಂಗುಲ್ಯಗ್ರೇ)\\
\as{೬ ಟಂ} ಸಂಧ್ಯಾರೂಪಾಯೈ ಈಶಾನ್ಯೈ ನಮಃ । (ದಕ್ಷೋರುಮೂಲೇ)\\
\as{೬ ಠಂ} ರಾತ್ರಿರೂಪಾಯೈ ಪಾರ್ವತ್ಯೈ ನಮಃ । (ದಕ್ಷಜಾನುನಿ)\\
\as{೬ ಡಂ} ತಿಥಿರೂಪಾಯೈ ಸರ್ವಮಂಗಲಾಯೈ ನಮಃ । (ದಕ್ಷಗುಲ್ಫೇ)\\
\as{೬ ಢಂ} ವಾರರೂಪಾಯೈ ದಾಕ್ಷಾಯಣ್ಯೈ ನಮಃ । \\(ದಕ್ಷಪಾದಾಂಗುಲಿಮೂಲೇ)\\
\as{೬ ಣಂ} ನಕ್ಷತ್ರರೂಪಾಯೈ ಹೈಮವತ್ಯೈ ನಮಃ । (ದಕ್ಷಪಾದಾಂಗುಲ್ಯಗ್ರೇ)\\
\as{೬ ತಂ} ಯೋಗರೂಪಾಯೈ ಮಹಾಮಾಯಾಯೈ ನಮಃ । \\(ವಾಮೋರುಮೂಲೇ)\\
\as{೬ ಥಂ} ಕರಣರೂಪಾಯೈ ಮಹೇಶ್ವರ್ಯೈ ನಮಃ । (ವಾಮಜಾನುನಿ)\\
\as{೬ ದಂ} ಪಕ್ಷರೂಪಾಯೈ ಮೃಡಾನ್ಯೈ ನಮಃ । (ವಾಮಗುಲ್ಫೇ)\\
\as{೬ ಧಂ} ಮಾಸರೂಪಾಯೈ ರುದ್ರಾಣ್ಯೈ ನಮಃ । \\(ವಾಮಪಾದಾಂಗುಲಿಮೂಲೇ)\\
\as{೬ ನಂ} ರಾಶಿರೂಪಾಯೈ ಶರ್ವಾಣ್ಯೈ ನಮಃ । (ವಾಮಪಾದಾಂಗುಲ್ಯಗ್ರೇ)\\
\as{೬ ಪಂ} ಋತುರೂಪಾಯೈ ಪರಮೇಶ್ವರ್ಯೈ ನಮಃ । (ದಕ್ಷಪಾರ್ಶ್ವೇ)\\
\as{೬ ಫಂ} ಅಯನರೂಪಾಯೈ ಕಾಲ್ಯೈ ನಮಃ । (ವಾಮಪಾರ್ಶ್ವೇ)\\
\as{೬ ಬಂ} ವತ್ಸರರೂಪಾಯೈ ಕಾತ್ಯಾಯನ್ಯೈ ನಮಃ । (ಪೃಷ್ಠೇ)\\
\as{೬ ಭಂ} ಯುಗರೂಪಾಯೈ ಗೌರ್ಯೈ ನಮಃ । (ನಾಭೌ)\\
\as{೬ ಮಂ} ಪ್ರಲಯರೂಪಾಯೈ ಭವಾನ್ಯೈ ನಮಃ । (ಜಠರೇ)\\
\as{೬ ಯಂ} ಪಂಚಭೂತರೂಪಾಯೈ ಬ್ರಾಹ್ಮ್ಯೈ ನಮಃ । (ಹೃದಿ )\\
\as{೬ ರಂ} ಪಂಚತನ್ಮಾತ್ರರೂಪಾಯೈ ವಾಗೀಶ್ವರ್ಯೈ ನಮಃ । (ದಕ್ಷಾಂಸೇ)\\
\as{೬ ಲಂ} ಪಂಚಕರ್ಮೇಂದ್ರಿಯರೂಪಾಯೈ ವಾಣ್ಯೈ ನಮಃ । (ಕಕುದಿ)\\
\as{೬ ವಂ} ಪಂಚಜ್ಞಾನೇಂದ್ರಿಯರೂಪಾಯೈ ಸಾವಿತ್ರ್ಯೈ ನಮಃ । (ವಾಮಾಂಸೇ)\\
\as{೬ ಶಂ} ಪಂಚಪ್ರಾಣರೂಪಾಯೈ ಸರಸ್ವತ್ಯೈ ನಮಃ । \\(ಹೃದಯಾದಿ ದಕ್ಷ ಕರಾಂಗುಲ್ಯಂತಂ )\\
\as{೬ ಷಂ} ಗುಣತ್ರಯರೂಪಾಯೈ ಗಾಯತ್ರ್ಯೈ ನಮಃ । \\(ಹೃದಯಾದಿ ವಾಮ ಕರಾಂಗುಲ್ಯಂತಂ )\\
\as{೬ ಸಂ} ಅಂತಃಕರಣಚತುಷ್ಟಯರೂಪಾಯೈ ವಾಕ್ಪ್ರದಾಯೈ ನಮಃ । \\(ಹೃದಯಾದಿ ದಕ್ಷ ಪಾದಾಂತಂ )\\
\as{೬ ಹಂ} ಅವಸ್ಥಾಚತುಷ್ಟಯರೂಪಾಯೈ ಶಾರದಾಯೈ ನಮಃ । \\(ಹೃದಯಾದಿ ವಾಮ ಪಾದಾಂತಂ )\\
\as{೬ ಳಂ} ಸಪ್ತಧಾತುರೂಪಾಯೈ ಭಾರತ್ಯೈ ನಮಃ । \\(ಕಟ್ಯಾದಿ ಪಾದಪರ್ಯಂತಂ)\\
\as{೬ ಕ್ಷಂ} ದೋಷತ್ರಯರೂಪಾಯೈ ವಿದ್ಯಾತ್ಮಿಕಾಯೈ ನಮಃ ।\\(ಕಟ್ಯಾದಿ ಶಿರಃಪರ್ಯಂತಂ )
 
 \as{ಓಂ ಐಂಹ್ರೀಂಶ್ರೀಂ ಹ್ಸೌಂಃ ಹ್ಸೌಃ ಅಂ ಆಂ ++++ ಳಂ ಕ್ಷಂ} ಸಕಲಪ್ರಪಂಚಾತ್ಮಿಕಾಯೈ  ಶ್ರೀಪರಾಂಬಾ ದೇವ್ಯೈ ನಮಃ \as{ಹ್ಸೌಃ ಹ್ಸೌಂಃ ಶ್ರೀಂಹ್ರೀಂಐಂ ಓಂ ॥}(ಇತಿ ಸರ್ವಾಂಗೇ)
\subsection{ಭುವನನ್ಯಾಸಃ} 
(ಪಾದಯೋಃ) \as{ಓಂ ಐಂಹ್ರೀಂಶ್ರೀಂ ಹ್ಸೌಂಃ ಹ್ಸೌಃ \\ಅಂ ಆಂ ಇಂ} ಅತಲಲೋಕನಿಲಯ ಶತಕೋಟಿ ಗುಹ್ಯಾದಿಯೋಗಿನೀ ಮೂಲದೇವತಾಯುತಾಧಾರಶಕ್ತ್ಯಂಬಾದೇವ್ಯೈ ನಮಃ ॥\\
(ಗುಲ್ಫಯೋಃ) \as{೬ ಈಂ ಉಂ ಊಂ} ವಿತಲಲೋಕನಿಲಯ ಶತಕೋಟಿ ಗುಹ್ಯತರಾನಂತಯೋಗಿನೀ ಮೂಲದೇವತಾಯುತಾಧಾರಶಕ್ತ್ಯಂಬಾದೇವ್ಯೈ ನಮಃ ॥\\
(ಜಂಘಯೋಃ) \as{೬ ಋಂ ೠಂ ಲೃಂ} ಸುತಲಲೋಕನಿಲಯ ಶತಕೋಟಿ ಗುಹ್ಯಾಚಿಂತ್ಯಯೋಗಿನೀ ಮೂಲದೇವತಾಯುತಾಧಾರಶಕ್ತ್ಯಂಬಾದೇವ್ಯೈ ನಮಃ ॥\\
(ಜಾನುನೋಃ) \as{೬ ಲೄಂ ಏಂ ಐಂ} ಮಹಾತಲಲೋಕನಿಲಯ ಶತಕೋಟಿ ಮಹಾಗುಹ್ಯ ಸ್ವತಂತ್ರಯೋಗಿನೀ ಮೂಲದೇವತಾಯುತಾಧಾರಶಕ್ತ್ಯಂಬಾದೇವ್ಯೈ ನಮಃ ॥\\
(ಊರ್ವೋಃ) \as{೬ ಓಂ ಔಂ} ತಲಾತಲಲೋಕನಿಲಯ ಶತಕೋಟಿ ಪರಮಗುಹ್ಯೇಚ್ಛಾಯೋಗಿನೀ ಮೂಲದೇವತಾಯುತಾಧಾರಶಕ್ತ್ಯಂಬಾದೇವ್ಯೈ ನಮಃ ॥\\
(ಸ್ಫಿಚೋಃ) \as{೬ ಅಂ ಅಃ }ರಸಾತಲಲೋಕನಿಲಯ ಶತಕೋಟಿ ರಹಸ್ಯಜ್ಞಾನ ಯೋಗಿನೀ ಮೂಲದೇವತಾಯುತಾಧಾರಶಕ್ತ್ಯಂಬಾದೇವ್ಯೈ ನಮಃ ॥\\
(ಆಧಾರೇ) \as{೬ ಕಂಖಂಗಂಘಂಙಂ} ಪಾತಾಲಲೋಕನಿಲಯ ಶತಕೋಟಿ ರಹಸ್ಯತರ ಕ್ರಿಯಾಯೋಗಿನೀ ಮೂಲದೇವತಾಯುತಾಧಾರಶಕ್ತ್ಯಂಬಾದೇವ್ಯೈ ನಮಃ ॥\\
(ಸ್ವಾಧಿಷ್ಠಾನೇ) \as{೬ ಚಂಛಂಜಂಝಂಞಂ} ಭೂರ್ಲೋಕನಿಲಯ ಶತಕೋಟಿ ಅತಿರಹಸ್ಯ ಡಾಕಿನೀಯೋಗಿನೀ ಮೂಲದೇವತಾಯುತಾಧಾರಶಕ್ತ್ಯಂಬಾದೇವ್ಯೈ ನಮಃ ॥\\
(ಮಣಿಪೂರೇ) \as{೬ ಟಂಠಂಡಂಢಂಣಂ} ಭುವರ್ಲೋಕನಿಲಯ ಶತಕೋಟಿ ಮಹಾರಹಸ್ಯ ರಾಕಿಣೀಯೋಗಿನೀ ಮೂಲದೇವತಾಯುತಾಧಾರಶಕ್ತ್ಯಂಬಾದೇವ್ಯೈ ನಮಃ ॥\\
(ಅನಾಹತೇ) \as{೬ ತಂಥಂದಂಧಂನಂ} ಸ್ವರ್ಲೋಕನಿಲಯ ಶತಕೋಟಿ ಪರಮರಹಸ್ಯ ಲಾಕಿನೀಯೋಗಿನೀ ಮೂಲದೇವತಾಯುತಾಧಾರಶಕ್ತ್ಯಂಬಾದೇವ್ಯೈ ನಮಃ ॥\\
(ವಿಶುದ್ಧೌ) \as{೬ ಪಂಫಂಬಂಭಂಮಂ} ಮಹರ್ಲೋಕನಿಲಯ ಶತಕೋಟಿ ಗುಪ್ತ ಕಾಕಿನೀಯೋಗಿನೀ ಮೂಲದೇವತಾಯುತಾಧಾರಶಕ್ತ್ಯಂಬಾದೇವ್ಯೈ ನಮಃ ॥\\
(ಆಜ್ಞಾಯಾಂ) \as{೬ ಯಂರಂಲಂವಂ} ಜನೋಲೋಕನಿಲಯ ಶತಕೋಟಿ ಗುಪ್ತತರ ಸಾಕಿನೀಯೋಗಿನೀ ಮೂಲದೇವತಾಯುತಾಧಾರಶಕ್ತ್ಯಂಬಾದೇವ್ಯೈ ನಮಃ ॥\\
(ಲಲಾಟೇ) \as{೬ ಶಂಷಂಸಂಹಂ} ತಪೋಲೋಕನಿಲಯ ಶತಕೋಟಿ ಅತಿಗುಪ್ತಹಾಕಿನೀಯೋಗಿನೀ ಮೂಲದೇವತಾಯುತಾಧಾರಶಕ್ತ್ಯಂಬಾದೇವ್ಯೈ ನಮಃ ॥\\
(ಬ್ರಹ್ಮರಂಧ್ರೇ) \as{೬ ಳಂಕ್ಷಂ} ಸತ್ಯಲೋಕನಿಲಯ ಶತಕೋಟಿ ಮಹಾಗುಪ್ತ\\ ಯಾಕಿನೀಯೋಗಿನೀ ಮೂಲದೇವತಾಯುತಾಧಾರಶಕ್ತ್ಯಂಬಾದೇವ್ಯೈ ನಮಃ ॥

 \as{ಓಂ ಐಂಹ್ರೀಂಶ್ರೀಂ ಹ್ಸೌಂಃ ಹ್ಸೌಃ ಅಂ ಆಂ ++++ ಳಂ ಕ್ಷಂ} ಸಕಲಭುವನಾಧಿಪಾಯೈ  ಶ್ರೀಪರಾಂಬಾ ದೇವ್ಯೈ ನಮಃ \as{ಹ್ಸೌಃ ಹ್ಸೌಂಃ ಶ್ರೀಂಹ್ರೀಂಐಂ ಓಂ ॥} (ಇತಿ ಸರ್ವಾಂಗೇ)
\subsection{ಮೂರ್ತಿನ್ಯಾಸಃ}
\as{ಓಂ ಐಂಹ್ರೀಂಶ್ರೀಂ ಹ್ಸೌಂಃ ಹ್ಸೌಃ \\ಅಂ} ಕೇಶವಾಯ ಅಕ್ಷರಶಕ್ತ್ಯೈ ನಮಃ (ಶಿರಸಿ)\\
\as{೬ ಆಂ} ನಾರಾಯಣಾಯ ಆದ್ಯಾಯೈ ಶಕ್ತ್ಯೈ ನಮಃ (ಮುಖೇ)\\
\as{೬ ಇಂ} ಮಾಧವಾಯೇಷ್ಟದಾಯೈ ಶಕ್ತ್ಯೈ ನಮಃ (ದಕ್ಷಾಂಸೇ)\\
\as{೬ ಈಂ} ಗೋವಿಂದಾಯೇಶಾನ್ಯೈ ಶಕ್ತ್ಯೈ ನಮಃ (ವಾಮಾಂಸೇ)\\
\as{೬ ಉಂ} ವಿಷ್ಣವೇ ಉಗ್ರಾಯೈ ಶಕ್ತ್ಯೈ ನಮಃ (ದಕ್ಷಪಾರ್ಶ್ವೇ)\\
\as{೬ ಊಂ} ಮಧುಸೂದನಾಯ ಊರ್ಧ್ವನಯನಾಯೈ ಶಕ್ತ್ಯೈ ನಮಃ\\ (ವಾಮಪಾರ್ಶ್ವೇ)\\
\as{೬ ಋಂ} ತ್ರಿವಿಕ್ರಮಾಯ ಋದ್ಧ್ಯೈ ಶಕ್ತ್ಯೈ ನಮಃ (ದಕ್ಷಕಟೌ)\\
\as{೬ ೠಂ} ವಾಮನಾಯ ರೂಪಿಣ್ಯೈ ಶಕ್ತ್ಯೈ ನಮಃ (ವಾಮಕಟೌ)\\
\as{೬ ಲೃಂ} ಶ್ರೀಧರಾಯ ಲುಪ್ತಾಯೈ ಶಕ್ತ್ಯೈ ನಮಃ (ದಕ್ಷೋರೌ)\\
\as{೬ ಲೄಂ} ಹೃಷೀಕೇಶಾಯ ಲೂನದೋಷಾಯೈ ಶಕ್ತ್ಯೈ ನಮಃ (ವಾಮೋರೌ)\\
\as{೬ ಏಂ} ಪದ್ಮನಾಭಾಯ ಏಕನಾಯಿಕಾಯೈ ಶಕ್ತ್ಯೈ ನಮಃ (ದಕ್ಷಜಾನುನಿ)\\
\as{೬ ಐಂ} ದಾಮೋದರಾಯ ಐಕಕಾರಿಣ್ಯೈ ಶಕ್ತ್ಯೈ ನಮಃ (ವಾಮಜಾನುನಿ)\\
\as{೬ ಓಂ} ವಾಸುದೇವಾಯ ಓಘವತ್ಯೈ ಶಕ್ತ್ಯೈ ನಮಃ (ದಕ್ಷಜಂಘೇ)\\
\as{೬ ಔಂ} ಸಂಕರ್ಷಣಾಯ ಔರ್ವಕಾಮಾಯೈ ಶಕ್ತ್ಯೈ ನಮಃ (ವಾಮಜಂಘೇ)\\
\as{೬ ಅಂ} ಪ್ರದ್ಯುಮ್ನಾಯ ಅಂಜನಪ್ರಭಾಯೈ ಶಕ್ತ್ಯೈ ನಮಃ (ದಕ್ಷಪಾದ)\\
\as{೬ ಅಃ} ಅನಿರುದ್ಧಾಯ ಅಸ್ಥಿಮಾಲಾಧರಾಯೈ ಶಕ್ತ್ಯೈ ನಮಃ (ವಾಮಪಾದ)\\
\as{೬ ಕಂಭಂ} ಭವಾಯ ಕರಭದ್ರಾಯೈ ಶಕ್ತ್ಯೈ ನಮಃ \\(ದಕ್ಷಪಾದಾದಾರಭ್ಯದಕ್ಷೋರುಮೂಲಪರ್ಯಂತಮ್)\\
\as{೬ ಖಂಬಂ} ಶರ್ವಾಯ ಖಗಬಲಾಯೈ ಶಕ್ತ್ಯೈ ನಮಃ \\(ವಾಮಪಾದಾದಾರಭ್ಯವಾಮೋರುಮೂಲಪರ್ಯಂತಮ್)\\
\as{೬ ಗಂಫಂ} ರುದ್ರಾಯ ಗರಿಮಫಲಪ್ರದಾಯೈ ಶಕ್ತ್ಯೈ ನಮಃ (ದಕ್ಷಪಾರ್ಶ್ವೇ)\\
\as{೬ ಘಂಪಂ} ಪಶುಪತಯೇ ಘೋರಪಾದಾಯೈ ಶಕ್ತ್ಯೈ ನಮಃ (ವಾಮಪಾರ್ಶ್ವೇ)\\
\as{೬ ಙಂನಂ} ಉಗ್ರಾಯ ಪಂಕ್ತಿನಾಸಾಯೈ ಶಕ್ತ್ಯೈ ನಮಃ (ದಕ್ಷಬಾಹುಮೂಲೇ)\\
\as{೬ ಚಂಧಂ} ಮಹಾದೇವಾಯ ಚಂದ್ರಧಾರಿಣ್ಯೈ ಶಕ್ತ್ಯೈ ನಮಃ\\ (ವಾಮಬಾಹುಮೂಲೇ)\\
\as{೬ ಛಂದಂ} ಭೀಮಾಯ ಛಂದೋಮಯ್ಯೈ ಶಕ್ತ್ಯೈ ನಮಃ (ಕಂಠೇ)\\
\as{೬ ಜಂಥಂ} ಈಶಾನಾಯ ಜಗತ್ಸ್ಥಾನಾಯೈ ಶಕ್ತ್ಯೈ ನಮಃ (ಮುಖೇ)\\
\as{೬ ಝಂತಂ} ತತ್ಪುರುಷಾಯ ಝಂಕೃತ್ಯೈ ಶಕ್ತ್ಯೈ ನಮಃ (ದಕ್ಷಕರ್ಣೇ)\\
\as{೬ ಞಂಣಂ} ಅಘೋರಾಯ ಜ್ಞಾನದಾಯೈ ಶಕ್ತ್ಯೈ ನಮಃ (ವಾಮಕರ್ಣೇ)\\
\as{೬ ಟಂಢಂ} ಸದ್ಯೋಜಾತಾಯ ಟಂಕಢಕ್ಕಧರಾಯೈ ಶಕ್ತ್ಯೈ ನಮಃ (ಲಲಾಟೇ)\\
\as{೬ ಠಂಡಂ} ವಾಮದೇವಾಯ ಠಂಕೃತಿಡಾಮರ್ಯೈ ಶಕ್ತ್ಯೈ ನಮಃ (ಶಿರಸಿ)\\
\as{೬ ಯಂ} ಬ್ರಹ್ಮಣೇ ಯಕ್ಷಿಣ್ಯೈ ಶಕ್ತ್ಯೈ ನಮಃ (ಆಧಾರೇ)\\
\as{೬ ರಂ} ಪ್ರಜಾಪತಯೇ ರಂಜಿನ್ಯೈ ಶಕ್ತ್ಯೈ ನಮಃ (ಸ್ವಾಧಿಷ್ಠಾನೇ)\\
\as{೬ ಲಂ} ವೇಧಸೇ ಲಕ್ಷ್ಮ್ಯೈ ಶಕ್ತ್ಯೈ ನಮಃ (ಮಣಿಪೂರೇ)\\
\as{೬ ವಂ} ಪರಮೇಷ್ಠಿನೇ ವಜ್ರಿಣ್ಯೈ ಶಕ್ತ್ಯೈ ನಮಃ (ಅನಾಹತೇ)\\
\as{೬ ಶಂ} ಪಿತಾಮಹಾಯ ಶಶಿಧರಾಯೈ ಶಕ್ತ್ಯೈ ನಮಃ (ವಿಶುದ್ಧೌ)\\
\as{೬ ಷಂ} ವಿಧಾತ್ರೇ ಷಡಾಧಾರಾಲಯಾಯೈ ಶಕ್ತ್ಯೈ ನಮಃ (ಆಜ್ಞಾಯಾಂ)\\
\as{೬ ಸಂ} ವಿರಿಂಚಯೇ ಸರ್ವನಾಯಿಕಾಯೈ ಶಕ್ತ್ಯೈ ನಮಃ (ಅರ್ಧೇಂದೌ)\\
\as{೬ ಹಂ} ಸ್ರಷ್ಟ್ರೇ ಹಸಿತಾನನಾಯೈ ಶಕ್ತ್ಯೈ ನಮಃ (ರೋಧಿನ್ಯಾಂ)\\
\as{೬ ಳಂ} ಚತುರಾನನಾಯ ಲಲಿತಾಯೈ ಶಕ್ತ್ಯೈ ನಮಃ (ನಾದೇ)\\
\as{೬ ಕ್ಷಂ} ಹಿರಣ್ಯಗರ್ಭಾಯ ಕ್ಷಮಾಯೈ ಶಕ್ತ್ಯೈ ನಮಃ (ನಾದಾಂತೇ)

 \as{ಓಂ ಐಂಹ್ರೀಂಶ್ರೀಂ ಹ್ಸೌಂಃ ಹ್ಸೌಃ ಅಂ ಆಂ ++++ ಳಂ ಕ್ಷಂ} ಸಕಲತ್ರಿಮೂರ್ತ್ಯಾತ್ಮಿಕಾಯೈ  ಶ್ರೀಪರಾಂಬಾ ದೇವ್ಯೈ ನಮಃ \as{ಹ್ಸೌಃ ಹ್ಸೌಂಃ ಶ್ರೀಂಹ್ರೀಂಐಂ ಓಂ ॥} (ಇತಿ ಸರ್ವಾಂಗೇ)
 \subsection{ಮಂತ್ರನ್ಯಾಸಃ}
(ಮೂಲಾಧಾರೇ)\as{ಓಂ ಐಂಹ್ರೀಂಶ್ರೀಂ ಹ್ಸೌಂಃ ಹ್ಸೌಃ \\ಅಂಆಂಇಂ} ಏಕಲಕ್ಷ ಕೋಟಿಭೇದ ಪ್ರಣವಾದ್ಯೇಕಾಕ್ಷರಾತ್ಮಕ ಅಖಿಲಮಂತ್ರಾಧಿದೇವತಾಯೈ ಸಕಲಫಲಪ್ರದಾಯೈ ಏಕಕೂಟೇಶ್ವರ್ಯಂಬಾದೇವ್ಯೈ ನಮಃ ॥\\
(ಸ್ವಾಧಿಷ್ಠಾನೇ)\as{೬ ಈಂಉಂಊಂ} ದ್ವಿಲಕ್ಷ ಕೋಟಿಭೇದ ಹಂಸಾದಿದ್ವ್ಯಕ್ಷರಾತ್ಮಕ ಅಖಿಲಮಂತ್ರಾಧಿದೇವತಾಯೈ ಸಕಲಫಲಪ್ರದಾಯೈ ದ್ವಿಕೂಟೇಶ್ವರ್ಯಂಬಾದೇವ್ಯೈ ನಮಃ ॥

(ಮಣಿಪೂರೇ)\as{೬ ಋಂೠಂಲೃಂ} ತ್ರಿಲಕ್ಷ ಕೋಟಿಭೇದ ವಹ್ನ್ಯಾದಿತ್ರ್ಯಕ್ಷರಾತ್ಮಕ ಅಖಿಲಮಂತ್ರಾಧಿದೇವತಾಯೈ ಸಕಲಫಲಪ್ರದಾಯೈ ತ್ರಿಕೂಟೇಶ್ವರ್ಯಂಬಾದೇವ್ಯೈ ನಮಃ ॥

(ಅನಾಹತೇ)\as{೬ ಲೄಂಏಂಐಂ} ಚತುರ್ಲಕ್ಷ ಕೋಟಿಭೇದ ಚಂದ್ರಾದಿಚತುರಕ್ಷರಾತ್ಮಕ ಅಖಿಲಮಂತ್ರಾಧಿದೇವತಾಯೈ ಸಕಲಫಲಪ್ರದಾಯೈ ಚತುಷ್ಕೂಟೇಶ್ವರ್ಯಂಬಾದೇವ್ಯೈ ನಮಃ ॥

(ವಿಶುದ್ಧೌ)\as{೬ ಓಂಔಂಅಂಅಃ} ಪಂಚಲಕ್ಷ ಕೋಟಿಭೇದ ಸೂರ್ಯಾದಿಪಂಚಾಕ್ಷರಾತ್ಮಕ ಅಖಿಲಮಂತ್ರಾಧಿದೇವತಾಯೈ ಸಕಲಫಲಪ್ರದಾಯೈ ಪಂಚಕೂಟೇಶ್ವರ್ಯಂಬಾದೇವ್ಯೈ ನಮಃ ॥

(ಆಜ್ಞಾಯಾಂ)\as{೬ ಕಂಖಂಗಂ} ಷಡ್ಲಕ್ಷ ಕೋಟಿಭೇದ ಸ್ಕಂದಾದಿಷಡಕ್ಷರಾತ್ಮಕ ಅಖಿಲಮಂತ್ರಾಧಿದೇವತಾಯೈ ಸಕಲಫಲಪ್ರದಾಯೈ ಷಟ್ಕೂಟೇಶ್ವರ್ಯಂಬಾದೇವ್ಯೈ ನಮಃ ॥

(ಬಿಂದೌ)\as{೬ ಘಂಙಂಚಂ} ಸಪ್ತಲಕ್ಷ ಕೋಟಿಭೇದ ಗಣಪತ್ಯಾದಿಸಪ್ತಾಕ್ಷರಾತ್ಮಕ ಅಖಿಲಮಂತ್ರಾಧಿದೇವತಾಯೈ ಸಕಲಫಲಪ್ರದಾಯೈ ಸಪ್ತಕೂಟೇಶ್ವರ್ಯಂಬಾದೇವ್ಯೈ ನಮಃ ॥

(ಅರ್ಧೇಂದೌ)\as{೬ ಛಂಜಂಝಂ} ಅಷ್ಟಲಕ್ಷ ಕೋಟಿಭೇದ ವಟುಕಾದ್ಯಷ್ಟಾಕ್ಷರಾತ್ಮಕ ಅಖಿಲಮಂತ್ರಾಧಿದೇವತಾಯೈ ಸಕಲಫಲಪ್ರದಾಯೈ ಅಷ್ಟಕೂಟೇಶ್ವರ್ಯಂಬಾದೇವ್ಯೈ ನಮಃ ॥

(ರೋಧಿನ್ಯಾಂ)\as{೬ ಞಂಟಂಠಂ} ನವಲಕ್ಷ ಕೋಟಿಭೇದ ಬ್ರಹ್ಮಾದಿನವಾಕ್ಷರಾತ್ಮಕ ಅಖಿಲಮಂತ್ರಾಧಿದೇವತಾಯೈ ಸಕಲಫಲಪ್ರದಾಯೈ ನವಕೂಟೇಶ್ವರ್ಯಂಬಾದೇವ್ಯೈ ನಮಃ ॥

(ನಾದೇ)\as{೬ ಡಂಢಂಣಂ} ದಶಲಕ್ಷ ಕೋಟಿಭೇದ ವಿಷ್ಣ್ವಾದಿದಶಾಕ್ಷರಾತ್ಮಕ ಅಖಿಲಮಂತ್ರಾಧಿದೇವತಾಯೈ ಸಕಲಫಲಪ್ರದಾಯೈ ದಶಕೂಟೇಶ್ವರ್ಯಂಬಾದೇವ್ಯೈ ನಮಃ ॥

(ನಾದಾಂತೇ)\as{೬ ತಂಥಂದಂ} ಏಕಾದಶಲಕ್ಷ ಕೋಟಿಭೇದ ರುದ್ರಾದ್ಯೇಕಾದಶಾಕ್ಷರಾತ್ಮಕ ಅಖಿಲಮಂತ್ರಾಧಿದೇವತಾಯೈ ಸಕಲಫಲಪ್ರದಾಯೈ ಏಕಾದಶಕೂಟೇಶ್ವರ್ಯಂಬಾದೇವ್ಯೈ ನಮಃ ॥

(ಶಕ್ತೌ)\as{೬ ಧಂನಂಪಂ} ದ್ವಾದಶಲಕ್ಷ ಕೋಟಿಭೇದ ವಾಣ್ಯಾದಿದ್ವಾದಶಾಕ್ಷರಾತ್ಮಕ ಅಖಿಲಮಂತ್ರಾಧಿದೇವತಾಯೈ ಸಕಲಫಲಪ್ರದಾಯೈ ದ್ವಾದಶಕೂಟೇಶ್ವರ್ಯಂಬಾದೇವ್ಯೈ ನಮಃ ॥

(ವ್ಯಾಪಿಕಾಯಾಂ)\as{೬ ಫಂಬಂಭಂ} ತ್ರಯೋದಶಲಕ್ಷ ಕೋಟಿಭೇದ ಲಕ್ಷ್ಮ್ಯಾದಿತ್ರಯೋದಶಾಕ್ಷರಾತ್ಮಕ ಅಖಿಲಮಂತ್ರಾಧಿದೇವತಾಯೈ ಸಕಲಫಲಪ್ರದಾಯೈ ತ್ರಯೋದಶಕೂಟೇಶ್ವರ್ಯಂಬಾದೇವ್ಯೈ ನಮಃ ॥\\
(ಸಮನಾಯಾಂ)\as{೬ ಮಂಯಂರಂ} ಚತುರ್ದಶಲಕ್ಷ ಕೋಟಿಭೇದ ಗೌರ್ಯಾದಿಚತುರ್ದಶಾಕ್ಷರಾತ್ಮಕ ಅಖಿಲಮಂತ್ರಾಧಿದೇವತಾಯೈ ಸಕಲಫಲಪ್ರದಾಯೈ ಚತುರ್ದಶಕೂಟೇಶ್ವರ್ಯಂಬಾದೇವ್ಯೈ ನಮಃ ॥\\
(ಉನ್ಮನಾಯಾಂ)\as{೬ ಲಂವಂಶಂ} ಪಂಚದಶಲಕ್ಷ ಕೋಟಿಭೇದ ದುರ್ಗಾದಿಪಂಚದಶಾಕ್ಷರಾತ್ಮಕ ಅಖಿಲಮಂತ್ರಾಧಿದೇವತಾಯೈ ಸಕಲಫಲಪ್ರದಾಯೈ ಪಂಚದಶಕೂಟೇಶ್ವರ್ಯಂಬಾದೇವ್ಯೈ ನಮಃ ॥\\
(ಧ್ರುವಮಂಡಲೇ)\as{೬ ಷಂಸಂಹಂಳಂಕ್ಷಂ} ಷೋಡಶಲಕ್ಷ ಕೋಟಿಭೇದ ತ್ರಿಪುರಾದಿಷೋಡಶಾಕ್ಷರಾತ್ಮಕ ಅಖಿಲಮಂತ್ರಾಧಿದೇವತಾಯೈ ಸಕಲಫಲಪ್ರದಾಯೈ ಷೋಡಶಕೂಟೇಶ್ವರ್ಯಂಬಾದೇವ್ಯೈ ನಮಃ ॥

 \as{ಓಂ ಐಂಹ್ರೀಂಶ್ರೀಂ ಹ್ಸೌಂಃ ಹ್ಸೌಃ ಅಂ ಆಂ ++++ ಳಂ ಕ್ಷಂ} ಸಕಲಮಂತ್ರಾಧಿದೇವತಾಯೈ  ಶ್ರೀಪರಾಂಬಾ ದೇವ್ಯೈ ನಮಃ \as{ಹ್ಸೌಃ ಹ್ಸೌಂಃ ಶ್ರೀಂಹ್ರೀಂಐಂ ಓಂ ॥} (ಇತಿ ಸರ್ವಾಂಗೇ)
 \newpage
\as{ದೇವತಾನ್ಯಾಸಃ} \\
(ದಕ್ಷಪಾದೇ)\as{ಓಂ ಐಂಹ್ರೀಂಶ್ರೀಂ ಹ್ಸೌಂಃ ಹ್ಸೌಃ \\ಅಂಆಂ} ಸಹಸ್ರಕೋಟಿ ಋಷಿಕುಲಸೇವಿತಾಯೈ ನಿವೃತ್ಯಂಬಾದೇವ್ಯೈ ನಮಃ ॥

(ವಾಮಪಾದೇ)\as{೬ ಇಂಈಂ} ಸಹಸ್ರಕೋಟಿ ಯೋಗಿನೀಕುಲ ಸೇವಿತಾಯೈ\\ ಪ್ರತಿಷ್ಠಾಂಬಾ ದೇವ್ಯೈ ನಮಃ ॥

(ದಕ್ಷಗುಲ್ಫೇ)\as{೬ ಉಂಊಂ} ಸಹಸ್ರಕೋಟಿ ತಪಸ್ವಿಕುಲ ಸೇವಿತಾಯೈ\\ ವಿದ್ಯಾಂಬಾ ದೇವ್ಯೈ ನಮಃ ॥

(ವಾಮಗುಲ್ಫೇ)\as{೬ ಋಂೠಂ} ಸಹಸ್ರಕೋಟಿ ಶಾಂತಕುಲ ಸೇವಿತಾಯೈ\\ ಶಾಂತಾಂಬಾ ದೇವ್ಯೈ ನಮಃ ॥

(ದಕ್ಷಜಂಘಾಯಾಂ)\as{೬ ಲೃಂಲೄಂ} ಸಹಸ್ರಕೋಟಿ ಮುನಿಕುಲ ಸೇವಿತಾಯೈ\\ ಶಾಂತ್ಯತೀತಾಂಬಾ ದೇವ್ಯೈ ನಮಃ ॥

(ವಾಮಜಂಘಾಯಾಂ)\as{೬ ಏಂಐಂ} ಸಹಸ್ರಕೋಟಿ ದೈವತಕುಲ ಸೇವಿತಾಯೈ\\ ಹೃಲ್ಲೇಖಾಂಬಾ ದೇವ್ಯೈ ನಮಃ ॥

(ದಕ್ಷಜಾನುನಿ)\as{೬ ಓಂಔಂ} ಸಹಸ್ರಕೋಟಿ ರಾಕ್ಷಸಕುಲ ಸೇವಿತಾಯೈ\\ ಗಗನಾಂಬಾ ದೇವ್ಯೈ ನಮಃ ॥

(ವಾಮಜಾನುನಿ)\as{೬ ಅಂಅಃ} ಸಹಸ್ರಕೋಟಿ ವಿದ್ಯಾಧರಕುಲ ಸೇವಿತಾಯೈ\\ ರಕ್ತಾಂಬಾ ದೇವ್ಯೈ ನಮಃ ॥

(ದಕ್ಷೋರೌ)\as{೬ ಕಂಖಂ} ಸಹಸ್ರಕೋಟಿ ಸಿದ್ಧಕುಲ ಸೇವಿತಾಯೈ\\ ಮಹೋಚ್ಛುಷ್ಮಾಂಬಾ ದೇವ್ಯೈ ನಮಃ ॥

(ವಾಮೋರೌ)\as{೬ ಗಂಘಂ} ಸಹಸ್ರಕೋಟಿ ಸಾಧ್ಯಕುಲ ಸೇವಿತಾಯೈ\\ ಕರಾಲಿಕಾಂಬಾ ದೇವ್ಯೈ ನಮಃ ॥

(ದಕ್ಷೋರುಮೂಲೇ)\as{೬ ಙಂಚಂ} ಸಹಸ್ರಕೋಟಿ ಅಪ್ಸರಕುಲ ಸೇವಿತಾಯೈ\\ ಜಯಾಂಬಾ ದೇವ್ಯೈ ನಮಃ ॥

(ವಾಮೋರುಮೂಲೇ)\as{೬ ಛಂಜಂ} ಸಹಸ್ರಕೋಟಿ ಗಂಧರ್ವಕುಲ ಸೇವಿತಾಯೈ\\ ವಿಜಯಾಂಬಾ ದೇವ್ಯೈ ನಮಃ ॥

(ದಕ್ಷಪಾರ್ಶ್ವೇ)\as{೬ ಝಂಞಂ} ಸಹಸ್ರಕೋಟಿ ಗುಹ್ಯಕುಲ ಸೇವಿತಾಯೈ\\ ಅಜಿತಾಂಬಾ ದೇವ್ಯೈ ನಮಃ ॥

(ವಾಮಪಾರ್ಶ್ವೇ)\as{೬ ಟಂಠಂ} ಸಹಸ್ರಕೋಟಿ ಯಕ್ಷಕುಲ ಸೇವಿತಾಯೈ\\ ಅಪರಾಜಿತಾಂಬಾ ದೇವ್ಯೈ ನಮಃ ॥

(ದಕ್ಷಸ್ತನೇ)\as{೬ ಡಂಢಂ} ಸಹಸ್ರಕೋಟಿ ಕಿನ್ನರಕುಲ ಸೇವಿತಾಯೈ\\ ವಾಮಾಂಬಾ ದೇವ್ಯೈ ನಮಃ ॥

(ವಾಮಸ್ತನೇ)\as{೬ ಣಂತಂ} ಸಹಸ್ರಕೋಟಿ ಪನ್ನಗಕುಲ ಸೇವಿತಾಯೈ\\ ಜ್ಯೇಷ್ಠಾಂಬಾ ದೇವ್ಯೈ ನಮಃ ॥

(ದಕ್ಷಬಾಹುಮೂಲೇ)\as{೬ ಥಂದಂ} ಸಹಸ್ರಕೋಟಿ ಪಿತೃಕುಲ ಸೇವಿತಾಯೈ\\ ರೌದ್ರ್ಯಂಬಾ ದೇವ್ಯೈ ನಮಃ ॥

(ವಾಮಬಾಹುಮೂಲೇ)\as{೬ ಧಂನಂ} ಸಹಸ್ರಕೋಟಿ ಗಣೇಶ್ವರಕುಲ ಸೇವಿತಾಯೈ\\ ಮಾಯಾಂಬಾ ದೇವ್ಯೈ ನಮಃ ॥

(ದಕ್ಷಭುಜೇ)\as{೬ ಪಂಫಂ} ಸಹಸ್ರಕೋಟಿ ಭೈರವಕುಲ ಸೇವಿತಾಯೈ\\ ಕುಂಡಲಿನ್ಯಂಬಾ ದೇವ್ಯೈ ನಮಃ ॥

(ವಾಮಭುಜೇ)\as{೬ ಬಂಭಂ} ಸಹಸ್ರಕೋಟಿ ವಟುಕಕುಲ ಸೇವಿತಾಯೈ\\ ಕಾಲ್ಯಂಬಾ ದೇವ್ಯೈ ನಮಃ ॥

(ದಕ್ಷಾಂಸೇ)\as{೬ ಮಂಯಂ} ಸಹಸ್ರಕೋಟಿ ಕ್ಷೇತ್ರೇಶಕುಲ ಸೇವಿತಾಯೈ\\ ಕಾಲರಾತ್ರ್ಯಂಬಾ ದೇವ್ಯೈ ನಮಃ ॥

(ವಾಮಾಂಸೇ)\as{೬ ರಂಲಂ} ಸಹಸ್ರಕೋಟಿ ಪ್ರಮಥಕುಲ ಸೇವಿತಾಯೈ\\ ಭಗವತ್ಯಂಬಾ ದೇವ್ಯೈ ನಮಃ ॥

(ದಕ್ಷಕರ್ಣೇ)\as{೬ ವಂಶಂ} ಸಹಸ್ರಕೋಟಿ ಬ್ರಹ್ಮಕುಲ ಸೇವಿತಾಯೈ\\ ಸರ್ವೇಶ್ವರ್ಯಂಬಾ ದೇವ್ಯೈ ನಮಃ ॥

(ವಾಮಕರ್ಣೇ)\as{೬ ಷಂಸಂ} ಸಹಸ್ರಕೋಟಿ ವಿಷ್ಣುಕುಲ ಸೇವಿತಾಯೈ\\ ಸರ್ವಜ್ಞಾತ್ರ್ಯಂಬಾ ದೇವ್ಯೈ ನಮಃ ॥

(ಲಲಾಟೇ)\as{೬ ಹಂಳಂ} ಸಹಸ್ರಕೋಟಿ ರುದ್ರಕುಲ ಸೇವಿತಾಯೈ\\ ಸರ್ವಕರ್ತ್ರ್ಯಂಬಾ ದೇವ್ಯೈ ನಮಃ ॥

(ಬ್ರಹ್ಮರಂಧ್ರೇ)\as{೬ ಕ್ಷಂ} ಸಹಸ್ರಕೋಟಿ ಚರಾಚರಕುಲ ಸೇವಿತಾಯೈ\\ ಕುಲಶಕ್ತ್ಯಂಬಾ ದೇವ್ಯೈ ನಮಃ ॥

 \as{ಓಂ ಐಂಹ್ರೀಂಶ್ರೀಂ ಹ್ಸೌಂಃ ಹ್ಸೌಃ ಅಂ ಆಂ ++++ ಳಂ ಕ್ಷಂ} ಸಮಸ್ತ ದೇವತಾಧಿಪಾಯೈ  ಶ್ರೀಪರಾಂಬಾ ದೇವ್ಯೈ ನಮಃ \as{ಹ್ಸೌಃ ಹ್ಸೌಂಃ ಶ್ರೀಂಹ್ರೀಂಐಂ ಓಂ ॥} (ಇತಿ ಸರ್ವಾಂಗೇ)
\subsection{ಮಾತೃಭೈರವನ್ಯಾಸಃ}
(ಮೂಲಾಧಾರೇ)\as{ಓಂ ಐಂಹ್ರೀಂಶ್ರೀಂ ಹ್ಸೌಂಃ ಹ್ಸೌಃ \\ಕಂಖಂಗಂಘಂಙಂ} ಅನಂತಕೋಟಿ ಭೂಚರೀಕುಲಸಹಿತಾಯೈ \as{ಆಂಕ್ಷಾಂ} ಮಂಗಲಾಂಬಾದೇವ್ಯೈ \as{ಆಂಕ್ಷಾಂ} ಬ್ರಹ್ಮಾಣ್ಯಂಬಾದೇವ್ಯೈ, ಅನಂತಕೋಟಿ ಭೂತಕುಲ ಸಹಿತಾಯ \as{ಅಂಕ್ಷಂ} ಮಂಗಲನಾಥಾಯ \as{ಅಂಕ್ಷಂ} ಅಸಿತಾಂಗಭೈರವನಾಥಾಯ ನಮಃ ॥

(ಸ್ವಾಧಿಷ್ಠಾನೇ)\as{೬ ಚಂಛಂಜಂಝಂಞಂ} ಅನಂತಕೋಟಿ ಖೇಚರೀಕುಲಸಹಿತಾಯೈ \as{ಈಂಳಾಂ} ಚರ್ಚಿಕಾಂಬಾದೇವ್ಯೈ \as{ಈಂಳಾಂ} ಮಾಹೇಶ್ವರ್ಯಂಬಾದೇವ್ಯೈ, ಅನಂತಕೋಟಿ ವೇತಾಳಕುಲ ಸಹಿತಾಯ \as{ಇಂಳಂ} ಚರ್ಚಿಕನಾಥಾಯ \as{ಇಂಳಂ} ರುರುಭೈರವನಾಥಾಯ ನಮಃ ॥

(ಮಣಿಪೂರೇ)\as{೬ ಟಂಠಂಡಂಢಂಣಂ} ಅನಂತಕೋಟಿ ಪಾತಾಳಚರೀಕುಲಸಹಿತಾಯೈ \as{ಊಂಹಾಂ} ಯೋಗೇಶ್ವರ್ಯಂಬಾದೇವ್ಯೈ \as{ಊಂಹಾಂ} ಕೌಮಾರ್ಯಂಬಾದೇವ್ಯೈ, ಅನಂತಕೋಟಿ ಪಿಶಾಚಕುಲ ಸಹಿತಾಯ \as{ಉಂಹಂ} ಯೋಗೇಶ್ವರನಾಥಾಯ \as{ಉಂಹಂ} ಚಂಡಭೈರವನಾಥಾಯ ನಮಃ ॥

(ಅನಾಹತೇ)\as{೬ ತಂಥಂದಂಧಂನಂ} ಅನಂತಕೋಟಿ ದಿಕ್ಚರೀಕುಲಸಹಿತಾಯೈ \as{ೠಂಸಾಂ} ಹರಸಿದ್ಧಾಂಬಾದೇವ್ಯೈ \as{ೠಂಸಾಂ} ವೈಷ್ಣವ್ಯಂಬಾದೇವ್ಯೈ, ಅನಂತಕೋಟಿ ಅಪಸ್ಮಾರಕುಲ ಸಹಿತಾಯ \as{ಋಂಸಂ} ಹರಸಿದ್ಧನಾಥಾಯ \as{ಋಂಸಂ} ಕ್ರೋಧಭೈರವನಾಥಾಯ ನಮಃ ॥

(ವಿಶುದ್ಧೌ)\as{೬ ಪಂಫಂಬಂಭಂಮಂ} ಅನಂತಕೋಟಿ ಸಹಚರೀಕುಲಸಹಿತಾಯೈ \as{ಲೄಂಷಾಂ} ಭಟ್ಟಿನ್ಯಂಬಾದೇವ್ಯೈ \as{ಲೄಂಷಾಂ} ವಾರಾಹ್ಯಂಬಾದೇವ್ಯೈ, ಅನಂತಕೋಟಿ ಬ್ರಹ್ಮರಾಕ್ಷಸಕುಲ ಸಹಿತಾಯ \as{ಲೃಂಷಂ} ಭಟ್ಟಿನಾಥಾಯ \as{ಲೃಂಷಂ} ಉನ್ಮತ್ತಭೈರವನಾಥಾಯ ನಮಃ ॥

(ಆಜ್ಞಾಯಾಂ)\as{೬ ಯಂರಂಲಂವಂ} ಅನಂತಕೋಟಿ ಗಿರಿಚರೀಕುಲಸಹಿತಾಯೈ \as{ಐಂಶಾಂ} ಕಿಲಕಿಲಾಂಬಾದೇವ್ಯೈ \as{ಐಂಶಾಂ} ಇಂದ್ರಾಣ್ಯಂಬಾದೇವ್ಯೈ, ಅನಂತಕೋಟಿ ವಟುಕಕುಲ ಸಹಿತಾಯ \as{ಏಂಶಂ} ಕಿಲಕಿಲನಾಥಾಯ \as{ಏಂಶಂ} ಕಪಾಲಿಭೈರವನಾಥಾಯ ನಮಃ ॥

(ಲಲಾಟೇ)\as{೬ ಶಂಷಂಸಂಹಂ} ಅನಂತಕೋಟಿ ವನಚರೀಕುಲಸಹಿತಾಯೈ \as{ಔಂವಾಂ} ಕಾಲರಾತ್ರ್ಯಂಬಾದೇವ್ಯೈ \as{ಔಂವಾಂ} ಚಾಮುಂಡಾಂಬಾದೇವ್ಯೈ, ಅನಂತಕೋಟಿ ಪ್ರೇತಕುಲ ಸಹಿತಾಯ \as{ಓಂವಂ} ಕಾಲರಾತ್ರಿನಾಥಾಯ \as{ಓಂವಂ} ಭೀಷಣಭೈರವನಾಥಾಯ ನಮಃ ॥

(ಬ್ರಹ್ಮರಂಧ್ರೇ)\as{೬ ಳಂಕ್ಷಂ} ಅನಂತಕೋಟಿ ಜಲಚರೀಕುಲಸಹಿತಾಯೈ \as{ಅಃಲಾಂ} ಚಂಡಿಕಾಂಬಾದೇವ್ಯೈ \as{ಅಃಲಾಂ} ಮಹಾಲಕ್ಷ್ಮ್ಯಂಬಾದೇವ್ಯೈ, ಅನಂತಕೋಟಿ ಕೂಷ್ಮಾಂಡಕುಲ ಸಹಿತಾಯ \as{ಅಂಲಂ} ಚಂಡಿಕಾನಾಥಾಯ \as{ಅಂಲಂ} ಸಂಹಾರಭೈರವನಾಥಾಯ ನಮಃ ॥

\as{ಓಂ ಐಂಹ್ರೀಂಶ್ರೀಂ ಹ್ಸೌಂಃ ಹ್ಸೌಃ ಅಂ ಆಂ ++++ ಳಂ ಕ್ಷಂ} ಸರ್ವಮಾತೃಕಾಭೈರವಾಧಿ ದೇವತಾಯೈ ಶ್ರೀಪರಾಂಬಾ ದೇವ್ಯೈ ನಮಃ \as{ಹ್ಸೌಃ ಹ್ಸೌಂಃ ಶ್ರೀಂಹ್ರೀಂಐಂ ಓಂ ॥} \\ಇತಿ ಸರ್ವಾಂಗೇಷು ವ್ಯಾಪಕಂ ವಿನ್ಯಸ್ಯ \\
\as{ಹ್ಸಾಂ, ಹ್ಸೀಂ} ಇತ್ಯಾದಿನಾ ಉತ್ತರನ್ಯಾಸಂ ವಿಧಾಯ ।\\
ಭೂರ್ಭುವಃ ಸುವರೋಮಿತಿ ದಿಗ್ವಿಮೋಕಃ ॥
\section{ಧ್ಯಾನಂ}
\as{ಪಂಚವಕ್ತ್ರಂ ಚತುರ್ಬಾಹುಂ ಸರ್ವಾಭರಣ ಭೂಷಿತಮ್ ।\\
ಚಂದ್ರಸೂರ್ಯಸಹಸ್ರಾಭಂ ಶಿವಶಕ್ತ್ಯಾತ್ಮಕಂ ಭಜೇ ॥

ಅಮೃತಾರ್ಣವಮಧ್ಯಸ್ಥ ರತ್ನದ್ವೀಪೇ ಮನೋರಮೇ~।\\
ಕಲ್ಪವೃಕ್ಷವನಾಂತಸ್ಥೇ ನವಮಾಣಿಕ್ಯಮಂಡಪೇ ॥

ನವರತ್ನಮಯ ಶ್ರೀಮತ್ ಸಿಂಹಾಸನಾಗತಾಂಬುಜೇ~।\\
ತ್ರಿಕೋಣಾಂತಃ ಸಮಾಸೀನಂ ಚಂದ್ರಸೂರ್ಯಾಯುತಪ್ರಭಂ ॥

ಅರ್ಧಾಂಬಿಕಾಸಮಾಯುಕ್ತಂ ಪ್ರವಿಭಕ್ತವಿಭೂಷಣಂ~।\\
ಕೋಟಿಕಂದರ್ಪಲಾವಣ್ಯಂ ಸದಾ ಷೋಡಶವಾರ್ಷಿಕಂ ॥

ಮಂದಸ್ಮಿತಮುಖಾಂಭೋಜಂ ತ್ರಿನೇತ್ರಂ ಚಂದ್ರಶೇಖರಂ~।\\
ದಿವ್ಯಾಂಬರಸ್ರಗಾಲೇಪಂ ದಿವ್ಯಾಭರಣಭೂಷಿತಂ ॥

ಪಾನಪಾತ್ರಂ ಚ ಚಿನ್ಮುದ್ರಾಂ ತ್ರಿಶೂಲಂ ಪುಸ್ತಕಂ ಕರೈಃ~।\\
ವಿದ್ಯಾ ಸಂಸದಿ ಬಿಭ್ರಾಣಂ ಸದಾನಂದಮುಖೇಕ್ಷಣಂ ॥

ಮಹಾಷೋಢೋದಿತಾಶೇಷದೇವತಾಗಣಸೇವಿತಂ~।\\
ಏವಂ ಚಿತ್ತಾಂಬುಜೇ ಧ್ಯಾಯೇದರ್ಧನಾರೀಶ್ವರಂ ಶಿವಂ ॥

ಪುರುಷಂ ವಾ ಸ್ಮರೇದ್ ದೇವಿ ಸ್ತ್ರೀರೂಪಂ ವಾ ವಿಚಿಂತಯೇತ್~।\\
ಅಥವಾ ನಿಷ್ಕಲಂ ಧ್ಯಾಯೇತ್ ಸಚ್ಚಿದಾನಂದವಿಗ್ರಹಂ~।\\
ಸರ್ವತೇಜೋಮಯಂ ದಿವ್ಯಂ ಸಚರಾಚರವಿಗ್ರಹಂ ॥} \\
ಇತಿ ಧ್ಯಾತ್ವಾ ಶಿರಸಿ ಸ್ವಗುರುಂ ಧ್ಯಾಯೇತ್ ॥

\dhyana{ಸಹಸ್ರದಲಪಂಕಜೇ ಸಕಲಶೀತರಶ್ಮಿಪ್ರಭಂ\\
ವರಾಭಯಕರಾಂಬುಜಂ ವಿಮಲಗಂಧಪುಷ್ಪಾಂಬರಂ~।\\
ಪ್ರಸನ್ನವದನೇಕ್ಷಣಂ ಸಕಲದೇವತಾರೂಪಿಣಂ\\
ಸ್ಮರೇಚ್ಛಿರಸಿ ಹಂಸಗಂ ತದಭಿಧಾನಪೂರ್ವಂ ಗುರುಂ ॥

ಓಂ ಐಂಹ್ರೀಂಶ್ರೀಂ ಐಂಕ್ಲೀಂಸೌಃ ಹ್‌ಸ್‌ಖ್‌ಫ್ರೇಂ\\ ಹಸಕ್ಷಮಲವರಯೂಂ ಹ್ಸೌಃ ಸಹಕ್ಷಮಲವರಯೀಂ ಸ್ಹೌಃ ॥}\\ಇತಿ ಗುರುಪಾದುಕಾಂ ಶಿರಸಿ ನ್ಯಸ್ಯ, ಗುರುಂ  ಧ್ಯಾಯೇತ್ ॥

ಶಾಂತಂ ಪದ್ಮಾಸನಸ್ಥಂ ಶಶಧರಮುಕುಟಂ ಪಂಚವಕ್ತ್ರಂ ತ್ರಿನೇತ್ರಂ\\
ಶೂಲಂ ವಜ್ರಂ ಚ ಖಡ್ಗಂ ಪರಶುಮಭಯದಂ ದಕ್ಷಭಾಗೇ ವಹಂತಂ~।\\
ನಾಗಂ ಪಾಶಂ ಚ ಘಂಟಾಂ ಪ್ರಲಯಹುತವಹಂ ಚಾಂಕುಶಂ ವಾಮಭಾಗೇ\\
ನಾನಾಲಂಕಾರಯುಕ್ತಂ ಸ್ಫಟಿಕಮಣಿನಿಭಂ ಪಾರ್ವತೀಶಂ ನಮಾಮಿ ॥

%ಅಂಆಂಇಂಈಂಉಂಊಂಋಂೠಂಲೃಂಲೄಂಏಂಐಂಓಂಔಂಅಂಅಃಕಂಖಂಗಂಘಂಙಂಚಂಛಂಜಂಝಂಞಂಟಂಠಂಡಂಢಂಣಂತಂಥಂದಂಧಂನಂಪಂಫಂಬಂಭಂಮಂಯಂರಂಲಂವಂಶಂಷಂಸಂಹಂಳಂಕ್ಷಂ~।
\section{ಶ್ರೀಚಕ್ರನ್ಯಾಸಃ}
ಅಸ್ಯ ಶ್ರೀ ಶ್ರೀಚಕ್ರನ್ಯಾಸಸ್ಯ ದಕ್ಷಿಣಾಮೂರ್ತಿಃ ಋಷಿಃ~।(ಶಿರಸಿ) ಪಂಕ್ತಿಶ್ಛಂದಃ~।(ಮುಖೇ) ಶ್ರೀಮಹಾತ್ರಿಪುರಸುಂದರೀ ದೇವತಾ~।(ಹೃದಿ) ಐಂ ಕಏಈಲಹ್ರೀಂ ಇತಿ ಬೀಜಂ~।(ಗುಹ್ಯೇ) ಕ್ಲೀಂ ಹಸಕಹಲಹ್ರೀಂ ಇತಿ ಶಕ್ತಿಃ~।(ಪಾದಯೋಃ) ಸೌಃ ಸಕಲಹ್ರೀಂ ಇತಿ ಕೀಲಕಂ~।(ನಾಭೌ) ಪೂಜಾಯಾಂ ವಿನಿಯೋಗಃ~।\\
ಕೂಟತ್ರಯೇಣ ನ್ಯಾಸಂ ವಿಧಾಯ~।
\section{ಧ್ಯಾನಂ}
\as{ಧ್ಯಾಯೇತ್ ಕಾಮೇಶ್ವರಾಂಕಸ್ಥಾಂ ಕುರುವಿಂದಮಣಿಪ್ರಭಾಂ ।\\ ಶೋಣಾಂಬರಸ್ರಗಾಲೇಪಾಂ ಸರ್ವಾಂಗೀಣವಿಭೂಷಣಾಂ ॥

ಸೌಂದರ್ಯಶೇವಧಿಂ ಸೇಷುಚಾಪಪಾಶಾಂಕುಶೋಜ್ಜ್ವಲಾಮ್ ।\\ ಸ್ವಭಾಭಿರಣಿಮಾದ್ಯಾಭಿಃ ಸೇವ್ಯಾಂ ಸರ್ವನಿಯಾಮಿಕಾಮ್ ॥

ಸಚ್ಚಿದಾನಂದ ವಪುಷಂ ಸದಯಾಪಾಂಗವಿಭ್ರಮಾಮ್ ।\\ ಸರ್ವಲೋಕೈಕ ಜನನೀಂ ಸ್ಮೇರಾಸ್ಯಾಂ ಲಲಿತಾಂಬಿಕಾಮ್ ॥}\\
 ಇತಿ ಧ್ಯಾತ್ವಾ ಪಂಚೋಪಚಾರೈಃ ಸಂಪೂಜ್ಯ

\as{ ಓಂ ಐಂಹ್ರೀಂಶ್ರೀಂ ಸಮಸ್ತ ಪ್ರಕಟ ಗುಪ್ತ ಗುಪ್ತತರ\\ ಸಂಪ್ರದಾಯ ಕುಲೋತ್ತೀರ್ಣ ನಿಗರ್ಭ ರಹಸ್ಯಾತಿರಹಸ್ಯ \\ಪರಾಪರಾತಿರಹಸ್ಯಯೋಗಿನೀ ಚಕ್ರದೇವತಾಭ್ಯೋ ನಮಃ ॥}\\ ಇತಿ ಸರ್ವಾಂಗೇ ವಿನ್ಯಸ್ಯ

\as{ಓಂ ಐಂಹ್ರೀಂಶ್ರೀಂ ಗಂ} ಗಣಪತಯೇ ನಮಃ ॥(ದಕ್ಷೋರೌ)\\
\as{೪ ಕ್ಷಂ} ಕ್ಷೇತ್ರಪಾಲಾಯ ನಮಃ ॥(ದಕ್ಷಾಂಸೇ)\\
\as{೪ ಯಾಂ} ಯೋಗಿನೀಭ್ಯೋ ನಮಃ ॥(ವಾಮಾಂಸೇ)\\
\as{೪ ವಂ} ವಟುಕಾಯ ನಮಃ ॥(ವಾಮೋರೌ)\\
\as{೪ ಲಂ} ಇಂದ್ರಾಯ ನಮಃ ॥(ಪಾದಾಂಗುಷ್ಠದ್ವಯಾಗ್ರೇ)\\
\as{೪ ರಂ} ಅಗ್ನಯೇ ನಮಃ ॥(ದಕ್ಷಜಾನುನಿ)\\
\as{೪ ಟಂ} ಯಮಾಯ ನಮಃ ॥(ದಕ್ಷಪಾರ್ಶ್ವೇ)\\
\as{೪ ಕ್ಷಂ} ನಿರ್ಋತಯೇ ನಮಃ ॥(ದಕ್ಷಾಂಸೇ)\\
\as{೪ ವಂ} ವರುಣಾಯ ನಮಃ ॥(ಮೂರ್ಧ್ನಿ)\\
\as{೪ ಯಂ} ವಾಯವೇ ನಮಃ ॥(ವಾಮಾಂಸೇ)\\
\as{೪ ಸಂ} ಸೋಮಾಯ ನಮಃ ॥(ವಾಮಪಾರ್ಶ್ವೇ)\\
\as{೪ ಹಂ} ಈಶಾನಾಯ ನಮಃ ॥(ವಾಮಜಾನುನಿ)\\
\as{೪ ಹಂಸಃ} ಬ್ರಹ್ಮಣೇ ನಮಃ ॥(ಮೂರ್ಧ್ನಿ)\\
\as{೪ ಅಂ} ಅನಂತಾಯ ನಮಃ ॥(ಆಧಾರೇ)
\subsection{ ತ್ರೈಲೋಕ್ಯಮೋಹನ ಚಕ್ರನ್ಯಾಸಃ}
{\bfseries ೪ ಅಂ ಆಂ ಸೌಃ ತ್ರೈಲೋಕ್ಯಮೋಹನ ಚಕ್ರಾಯ ನಮಃ }(ಇತಿ ವ್ಯಾಪಕಂ ನ್ಯಸ್ಯ)\\
{\bfseries ೪ ಆದ್ಯಚತುರಶ್ರರೇಖಾಯೈ ನಮಃ }\\(ಇತಿ ದಕ್ಷಾಂಸಪೃಷ್ಠಾದಿ ವಕ್ಷ್ಯಮಾಣೇಷ್ವಂಜಲಿನಾ ವ್ಯಾಪಕಂ ನ್ಯಸ್ಯ)\\
\as{೪ ಅಂ} ಅಣಿಮಾಸಿದ್ಧ್ಯೈ ನಮಃ ।(ದಕ್ಷಾಂಸಪೃಷ್ಠೇ)\\
\as{೪ ಲಂ} ಲಘಿಮಾಸಿದ್ಧ್ಯೈ ನಮಃ ।(ದಕ್ಷಪಾಣ್ಯಂಗುಲ್ಯಗ್ರೇಷು)\\
\as{೪ ಮಂ} ಮಹಿಮಾಸಿದ್ಧ್ಯೈ ನಮಃ ।(ದಕ್ಷೋರುಸಂಧೌ)\\
\as{೪ ಈಂ} ಈಶಿತ್ವಸಿದ್ಧ್ಯೈ ನಮಃ ।(ದಕ್ಷಪಾದಾಂಗುಲ್ಯಗ್ರೇಷು)\\
\as{೪ ವಂ} ವಶಿತ್ವಸಿದ್ಧ್ಯೈ ನಮಃ ।(ವಾಮಪಾದಾಂಗುಲ್ಯಗ್ರೇಷು)\\
\as{೪ ಪಂ} ಪ್ರಾಕಾಮ್ಯಸಿದ್ಧ್ಯೈ ನಮಃ ।(ವಾಮೋರುಸಂಧೌ)\\
\as{೪ ಭುಂ} ಭುಕ್ತಿಸಿದ್ಧ್ಯೈ ನಮಃ ।(ವಾಮಪಾಣ್ಯಂಗುಲ್ಯಗ್ರೇಷು)\\
\as{೪ ಇಂ} ಇಚ್ಛಾಸಿದ್ಧ್ಯೈ ನಮಃ ।(ವಾಮಾಂಸಪೃಷ್ಠೇ)\\
\as{೪ ಪಂ} ಪ್ರಾಪ್ತಿಸಿದ್ಧ್ಯೈ ನಮಃ ।(ಶಿಖಾಮೂಲೇ)\\
\as{೪ ಸಂ} ಸರ್ವಕಾಮಸಿದ್ಧ್ಯೈ ನಮಃ ।(ಶಿರಃಪೃಷ್ಠೇ)

{\bfseries ೪ ಮಧ್ಯಚತುರಶ್ರರೇಖಾಯೈ ನಮಃ }(ಇತಿ ವ್ಯಾಪಕಂ ವಿನ್ಯಸ್ಯ)\\
\as{೪ ಆಂ} ಬ್ರಾಹ್ಮ್ಯೈ ನಮಃ ।(ಪಾದಾಂಗುಷ್ಠದ್ವಯೇ)\\
\as{೪ ಈಂ} ಮಾಹೇಶ್ವರ್ಯೈ ನಮಃ ।(ದಕ್ಷಪಾರ್ಶ್ವೇ)\\
\as{೪ ಊಂ} ಕೌಮಾರ್ಯೈ ನಮಃ ।(ಮೂರ್ಧ್ನಿ)\\
\as{೪ ೠಂ} ವೈಷ್ಣವ್ಯೈ ನಮಃ ।(ವಾಮಪಾರ್ಶ್ವೇ)\\
\as{೪ ಲೄಂ} ವಾರಾಹ್ಯೈ ನಮಃ ।(ವಾಮಜಾನುನಿ)\\
\as{೪ ಐಂ} ಮಾಹೇಂದ್ರ್ಯೈ ನಮಃ ।(ದಕ್ಷಜಾನುನಿ)\\
\as{೪ ಔಂ} ಚಾಮುಂಡಾಯೈ ನಮಃ ।(ದಕ್ಷಾಂಸೇ)\\
\as{೪ ಅಃ} ಮಹಾಲಕ್ಷ್ಮ್ಯೈ ನಮಃ ।(ವಾಮಾಂಸೇ)

{\bfseries ೪ ಅಂತ್ಯಚತುರಶ್ರರೇಖಾಯೈ ನಮಃ }(ಇತಿ ವ್ಯಾಪಕಂ ವಿನ್ಯಸ್ಯ)\\
\as{೪ ದ್ರಾಂ} ಸರ್ವಸಂಕ್ಷೋಭಿಣ್ಯೈ ನಮಃ ।(ಪಾದಾಂಗುಷ್ಠದ್ವಯೇ)\\
\as{೪ ದ್ರೀಂ} ಸರ್ವವಿದ್ರಾವಿಣ್ಯೈ ನಮಃ ।(ದಕ್ಷಪಾರ್ಶ್ವೇ)\\
\as{೪ ಕ್ಲೀಂ} ಸರ್ವಾಕರ್ಷಿಣ್ಯೈ ನಮಃ ।(ಮೂರ್ಧ್ನಿ)\\
\as{೪ ಬ್ಲೂಂ} ಸರ್ವವಶಂಕರ್ಯೈ ನಮಃ ।(ವಾಮಪಾರ್ಶ್ವೇ)\\
\as{೪ ಸಃ} ಸರ್ವೋನ್ಮಾದಿನ್ಯೈ ನಮಃ ।(ವಾಮಜಾನುನಿ)\\
\as{೪ ಕ್ರೋಂ} ಸರ್ವಮಹಾಂಕುಶಾಯೈ ನಮಃ ।(ದಕ್ಷಜಾನುನಿ)\\
\as{೪ ಹ್‌ಸ್‌ಖ್‌ಫ್ರೇಂ} ಸರ್ವಖೇಚರ್ಯೈ ನಮಃ ।(ದಕ್ಷಾಂಸೇ)\\
\as{೪ ಹ್ಸೌಃ} ಸರ್ವಬೀಜಾಯೈ ನಮಃ ।(ವಾಮಾಂಸೇ)\\
\as{೪ ಐಂ} ಸರ್ವಯೋನ್ಯೈ ನಮಃ ।(ದ್ವಾದಶಾಂತೇ)\\
\as{೪ ಹ್‌ಸ್‌ರೈಂ ಹ್‌ಸ್‌ಕ್ಲ್ರೀಂ ಹ್‌ಸ್‌ರ್ಸೌಃ} ಸರ್ವತ್ರಿಖಂಡಾಯೈ ನಮಃ ।\\(ಪಾದಾಂಗುಷ್ಠದ್ವಯೇ)\\
\as{೪ ಅಂ ಆಂ ಸೌಃ॥} ತ್ರೈಲೋಕ್ಯಮೋಹನ ಚಕ್ರೇಶ್ವರ್ಯೈ ತ್ರಿಪುರಾಯೈ ನಮಃ ॥(ಹೃದಿ)

\as{೪} ಏತಾಃ ಪ್ರಕಟಯೋಗಿನ್ಯಃ ತ್ರೈಲೋಕ್ಯಮೋಹನೇ ಚಕ್ರೇ ಸಮುದ್ರಾಃ ಸಸಿದ್ಧಯಃ ಸಾಯುಧಾಃ ಸಶಕ್ತಯಃ ಸವಾಹನಾಃ ಸಪರಿವಾರಾಃ ಸರ್ವಾಃ ನ್ಯಸ್ತಾಃ ಸಂತು ನಮಃ (ಇತಿ ಹೃದಿ ಚಕ್ರಸಮರ್ಪಣಂ ನ್ಯಸ್ಯ)
\subsection{ಸರ್ವಾಶಾಪರಿಪೂರಕ ಚಕ್ರನ್ಯಾಸಃ}
{\bfseries ೪ ಐಂ ಕ್ಲೀಂ ಸೌಃ ಸರ್ವಾಶಾಪರಿಪೂರಕ ಚಕ್ರಾಯ ನಮಃ}\\(ಇತಿ ವ್ಯಾಪಕಂ ವಿನ್ಯಸ್ಯ)\\
\as{೪ ಅಂ} ಕಾಮಾಕರ್ಷಣ್ಯೈ ನಿತ್ಯಾಕಲಾಯೈ ನಮಃ ।(ದಕ್ಷಕರ್ಣಪೃಷ್ಠೇ)\\
\as{೪ ಆಂ} ಬುದ್ಧ್ಯಾಕರ್ಷಣ್ಯೈ ನಿತ್ಯಾಕಲಾಯೈ ನಮಃ ।(ದಕ್ಷಾಂಸೇ)\\
\as{೪ ಇಂ} ಅಹಂಕಾರಾಕರ್ಷಣ್ಯೈ ನಿತ್ಯಾಕಲಾಯೈ ನಮಃ ।(ದಕ್ಷಕೂರ್ಪರೇ)\\
\as{೪ ಈಂ} ಶಬ್ದಾಕರ್ಷಣ್ಯೈ ನಿತ್ಯಾಕಲಾಯೈ ನಮಃ ।\\(ದಕ್ಷಕರ ತಲಪೃಷ್ಠಯೋಃ)\\
\as{೪ ಉಂ} ಸ್ಪರ್ಶಾಕರ್ಷಣ್ಯೈ ನಿತ್ಯಾಕಲಾಯೈ ನಮಃ ।(ದಕ್ಷೋರೌ, ದಕ್ಷಸ್ಫಿಚಿ)\\
\as{೪ ಊಂ} ರೂಪಾಕರ್ಷಣ್ಯೈ ನಿತ್ಯಾಕಲಾಯೈ ನಮಃ ।(ದಕ್ಷಜಾನುನಿ)\\
\as{೪ ಋಂ} ರಸಾಕರ್ಷಣ್ಯೈ ನಿತ್ಯಾಕಲಾಯೈ ನಮಃ ।(ದಕ್ಷಗುಲ್ಫೇ)\\
\as{೪ ೠಂ} ಗಂಧಾಕರ್ಷಣ್ಯೈ ನಿತ್ಯಾಕಲಾಯೈ ನಮಃ ।\\(ದಕ್ಷಪಾದತಲೇ ದಕ್ಷಪ್ರಪದೇ)\\
\as{೪ ಲೃಂ} ಚಿತ್ತಾಕರ್ಷಣ್ಯೈ ನಿತ್ಯಾಕಲಾಯೈ ನಮಃ ।\\(ವಾಮಪಾದತಲೇ ವಾಮಪ್ರಪದೇ)\\
\as{೪ ಲೄಂ} ಧೈರ್ಯಾಕರ್ಷಣ್ಯೈ ನಿತ್ಯಾಕಲಾಯೈ ನಮಃ ।(ವಾಮಗುಲ್ಫೇ)\\
\as{೪ ಏಂ} ಸ್ಮೃತ್ಯಾಕರ್ಷಣ್ಯೈ ನಿತ್ಯಾಕಲಾಯೈ ನಮಃ ।(ವಾಮಜಾನುನಿ)\\
\as{೪ ಐಂ} ನಾಮಾಕರ್ಷಣ್ಯೈ ನಿತ್ಯಾಕಲಾಯೈ ನಮಃ ।(ವಾಮೋರೌ, ವಾಮಸ್ಫಿಚಿ)\\
\as{೪ ಓಂ} ಬೀಜಾಕರ್ಷಣ್ಯೈ ನಿತ್ಯಾಕಲಾಯೈ ನಮಃ ।\\(ವಾಮಕರತಲಪೃಷ್ಠಯೋಃ)\\
\as{೪ ಔಂ} ಆತ್ಮಾಕರ್ಷಣ್ಯೈ ನಿತ್ಯಾಕಲಾಯೈ ನಮಃ ।(ವಾಮಕೂರ್ಪರೇ)\\
\as{೪ ಅಂ} ಅಮೃತಾಕರ್ಷಣ್ಯೈ ನಿತ್ಯಾಕಲಾಯೈ ನಮಃ ।(ವಾಮಾಂಸೇ)\\
\as{೪ ಅಃ} ಶರೀರಾಕರ್ಷಣ್ಯೈ ನಿತ್ಯಾಕಲಾಯೈ ನಮಃ ।(ವಾಮಕರ್ಣಪೃಷ್ಠೇ)\\
\as{೪ ಐಂ ಕ್ಲೀಂ ಸೌಃ॥} ಸರ್ವಾಶಾಪರಿಪೂರಕ ಚಕ್ರೇಶ್ವರ್ಯೈ ತ್ರಿಪುರೇಶ್ಯೈ ನಮಃ ॥(ಹೃದಿ)

\as{೪} ಏತಾಃ ಗುಪ್ತಯೋಗಿನ್ಯಃ ಸರ್ವಾಶಾಪರಿಪೂರಕೇ ಚಕ್ರೇ ಸಮುದ್ರಾಃ ಸಸಿದ್ಧಯಃ ಸಾಯುಧಾಃ ಸಶಕ್ತಯಃ ಸವಾಹನಾಃ ಸಪರಿವಾರಾಃ ಸರ್ವಾಃ ನ್ಯಸ್ತಾಃ ಸಂತು ನಮಃ (ಇತಿ ಹೃದಿ ಚಕ್ರಸಮರ್ಪಣಂ ನ್ಯಸ್ಯ)

\subsection{ಸರ್ವಸಂಕ್ಷೋಭಣಚಕ್ರನ್ಯಾಸಃ}
{\bfseries ೪ ಹ್ರೀಂ ಕ್ಲೀಂ ಸೌಃ ಸರ್ವಸಂಕ್ಷೋಭಣಚಕ್ರಾಯ ನಮಃ}(ಇತಿ ವ್ಯಾಪಕಂ ವಿನ್ಯಸ್ಯ)\\
\as{೪ ಕಂಖಂಗಂಘಂಙಂ} ಅನಂಗಕುಸುಮಾಯೈ ನಮಃ ।(ದಕ್ಷಶಂಖೇ)\\
\as{೪ ಚಂಛಂಜಂಝಂಞಂ} ಅನಂಗಮೇಖಲಾಯೈ ನಮಃ ।(ದಕ್ಷಜತ್ರುಣಿ)\\
\as{೪ ಟಂಠಂಡಂಢಂಣಂ} ಅನಂಗಮದನಾಯೈ ನಮಃ ।(ದಕ್ಷೋರೌ)\\
\as{೪ ತಂಥಂದಂಧಂನಂ} ಅನಂಗಮದನಾತುರಾಯೈ ನಮಃ ।(ದಕ್ಷಗುಲ್ಫೇ)\\
\as{೪ ಪಂಫಂಬಂಭಂಮಂ} ಅನಂಗರೇಖಾಯೈ ನಮಃ ।(ವಾಮಗುಲ್ಫೇ)\\
\as{೪ ಯಂರಂಲಂವಂ} ಅನಂಗವೇಗಿನ್ಯೈ ನಮಃ ।(ವಾಮೋರೌ)\\
\as{೪ ಶಂಷಂಸಂಹಂ} ಅನಂಗಾಂಕುಶಾಯೈ ನಮಃ ।(ವಾಮಜತ್ರುಣಿ)\\
\as{೪ ಳಂಕ್ಷಂ} ಅನಂಗಮಾಲಿನ್ಯೈ ನಮಃ ।(ವಾಮಶಂಖೇ)\\
\as{೪ ಹ್ರೀಂ ಕ್ಲೀಂ ಸೌಃ॥} ಸರ್ವಸಂಕ್ಷೋಭಣ ಚಕ್ರೇಶ್ವರ್ಯೈ ತ್ರಿಪುರಸುಂದರ್ಯೈ ನಮಃ ॥(ಹೃದಿ)

\as{೪} ಏತಾಃ ಗುಪ್ತತರಯೋಗಿನ್ಯಃ ಸರ್ವಸಂಕ್ಷೋಭಣೇ ಚಕ್ರೇ ಸಮುದ್ರಾಃ ಸಸಿದ್ಧಯಃ ಸಾಯುಧಾಃ ಸಶಕ್ತಯಃ ಸವಾಹನಾಃ ಸಪರಿವಾರಾಃ ಸರ್ವಾಃ ನ್ಯಸ್ತಾಃ ಸಂತು ನಮಃ (ಇತಿ ಹೃದಿ ಚಕ್ರಸಮರ್ಪಣಂ ನ್ಯಸ್ಯ)
\subsection{ಸರ್ವಸೌಭಾಗ್ಯದಾಯಕಚಕ್ರನ್ಯಾಸಃ}
{\bfseries ೪ ಹೈಂ ಹ್‌ಕ್ಲೀಂ ಹ್‌ಸೌಃ ಸರ್ವಸೌಭಾಗ್ಯದಾಯಕ ಚಕ್ರಾಯ ನಮಃ}\\(ಇತಿ ವ್ಯಾಪಕಂ ವಿನ್ಯಸ್ಯ)\\
\as{೪ ಕಂ} ಸರ್ವಸಂಕ್ಷೋಭಿಣ್ಯೈ ನಮಃ ।(ಲಲಾಟಮಧ್ಯೇ)\\
\as{೪ ಖಂ} ಸರ್ವವಿದ್ರಾವಿಣ್ಯೈ ನಮಃ ।(ಲಲಾಟದಕ್ಷಭಾಗೇ)\\
\as{೪ ಗಂ} ಸರ್ವಾಕರ್ಷಿಣ್ಯೈ ನಮಃ ।(ದಕ್ಷಗಂಡೇ)\\
\as{೪ ಘಂ} ಸರ್ವಾಹ್ಲಾದಿನ್ಯೈ ನಮಃ ।(ದಕ್ಷಾಂಸೇ)\\
\as{೪ ಙಂ} ಸರ್ವಸಮ್ಮೋಹಿನ್ಯೈ ನಮಃ ।(ದಕ್ಷಪಾರ್ಶ್ವೇ)\\
\as{೪ ಚಂ} ಸರ್ವಸ್ತಂಭಿನ್ಯೈ ನಮಃ ।(ದಕ್ಷೋರೌ)\\
\as{೪ ಛಂ} ಸರ್ವಜೃಂಭಿಣ್ಯೈ ನಮಃ ।(ದಕ್ಷಜಂಘಾಯಾಂ)\\
\as{೪ ಜಂ} ಸರ್ವವಶಂಕರ್ಯೈ ನಮಃ ।(ವಾಮಜಂಘಾಯಾಂ)\\
\as{೪ ಝಂ} ಸರ್ವರಂಜನ್ಯೈ ನಮಃ ।(ವಾಮೋರೌ)\\
\as{೪ ಞಂ} ಸರ್ವೋನ್ಮಾದಿನ್ಯೈ ನಮಃ ।(ವಾಮಪಾರ್ಶ್ವೇ)\\
\as{೪ ಟಂ} ಸರ್ವಾರ್ಥಸಾಧಿನ್ಯೈ ನಮಃ ।(ವಾಮಾಂಸೇ)\\
\as{೪ ಠಂ} ಸರ್ವಸಂಪತ್ತಿಪೂರಣ್ಯೈ ನಮಃ ।(ವಾಮಗಂಡೇ)\\
\as{೪ ಡಂ} ಸರ್ವಮಂತ್ರಮಯ್ಯೈ ನಮಃ ।(ಲಲಾಟವಾಮಭಾಗೇ)\\
\as{೪ ಢಂ} ಸರ್ವದ್ವಂದ್ವಕ್ಷಯಂಕರ್ಯೈ ನಮಃ ।(ಶಿರಃಪೃಷ್ಠೇ)\\
\as{೪ ಹೈಂ ಹ್‌ಕ್ಲೀಂ ಹ್‌ಸೌಃ॥}ಸರ್ವಸೌಭಾಗ್ಯದಾಯಕ ಚಕ್ರೇಶ್ವರ್ಯೈ ತ್ರಿಪುರವಾಸಿನ್ಯೈ ನಮಃ ॥(ಹೃದಿ)

\as{೪} ಏತಾಃ ಸಂಪ್ರದಾಯಯೋಗಿನ್ಯಃ ಸರ್ವಸೌಭಾಗ್ಯದಾಯಕೇ ಚಕ್ರೇ ಸಮುದ್ರಾಃ ಸಸಿದ್ಧಯಃ ಸಾಯುಧಾಃ ಸಶಕ್ತಯಃ ಸವಾಹನಾಃ ಸಪರಿವಾರಾಃ ಸರ್ವಾಃ ನ್ಯಸ್ತಾಃ ಸಂತು ನಮಃ (ಇತಿ ಹೃದಿ ಚಕ್ರಸಮರ್ಪಣಂ ನ್ಯಸ್ಯ)
\subsection{ಸರ್ವಾರ್ಥಸಾಧಕಚಕ್ರನ್ಯಾಸಃ}
{\bfseries ೪ ಹ್‌ಸೈಂ ಹ್‌ಸ್‌ಕ್ಲೀಂ ಹ್‌ಸ್ಸೌಃ ಸರ್ವಾರ್ಥಸಾಧಕ ಚಕ್ರಾಯ ನಮಃ}\\(ಇತಿ ವ್ಯಾಪಕಂ ವಿನ್ಯಸ್ಯ)\\
\as{೪ ಣಂ} ಸರ್ವಸಿದ್ಧಿಪ್ರದಾಯೈ ನಮಃ ।(ದಕ್ಷನೇತ್ರೇ ದಕ್ಷನಾಸಾಪುಟೇ)\\
\as{೪ ತಂ} ಸರ್ವಸಂಪತ್ಪ್ರದಾಯೈ ನಮಃ ।(ನಾಸಾಮೂಲೇ, ದಕ್ಷಸೃಕ್ಕಣಿ)\\
\as{೪ ಥಂ} ಸರ್ವಪ್ರಿಯಂಕರ್ಯೈ ನಮಃ ।(ವಾಮನೇತ್ರೇ, ದಕ್ಷಸ್ತನೇ)\\
\as{೪ ದಂ} ಸರ್ವಮಂಗಳಕಾರಿಣ್ಯೈ ನಮಃ ।\\(ವಾಮಬಾಹುಮೂಲೇ,ದಕ್ಷವೃಷಣೇ)\\
\as{೪ ಧಂ} ಸರ್ವಕಾಮಪ್ರದಾಯೈ ನಮಃ ।\\(ವಾಮೋರುಮೂಲೇ, ಸೀವಿನ್ಯಾ ದಕ್ಷಭಾಗೇ)\\
\as{೪ ನಂ} ಸರ್ವದುಃಖವಿಮೋಚನ್ಯೈ ನಮಃ ।\\(ವಾಮಜಾನುನಿ, ಸೀವಿನ್ಯಾ ವಾಮಭಾಗೇ)\\
\as{೪ ಪಂ} ಸರ್ವಮೃತ್ಯುಪ್ರಶಮನ್ಯೈ ನಮಃ ।(ದಕ್ಷಜಾನುನಿ, ವಾಮಸ್ತನೇ )\\
\as{೪ ಫಂ} ಸರ್ವವಿಘ್ನನಿವಾರಿಣ್ಯೈ ನಮಃ ।(ಗುದೇ, ವಾಮವೃಷಣೇ)\\
\as{೪ ಬಂ} ಸರ್ವಾಂಗಸುಂದರ್ಯೈ ನಮಃ ।(ದಕ್ಷೋರುಮೂಲೇ, ವಾಮಸೃಕ್ಕಣಿ)\\
\as{೪ ಭಂ} ಸರ್ವಸೌಭಾಗ್ಯದಾಯಿನ್ಯೈ ನಮಃ ।\\(ದಕ್ಷಬಾಹುಮೂಲೇ, ವಾಮನಾಸಾಪುಟೇ)\\
\as{೪ ಹ್‌ಸೈಂ ಹ್‌ಸ್‌ಕ್ಲೀಂ ಹ್‌ಸ್ಸೌಃ ॥} \\ಸರ್ವಾರ್ಥಸಾಧಕ ಚಕ್ರೇಶ್ವರ್ಯೈ ತ್ರಿಪುರಾಶ್ರಿಯೈ ನಮಃ ॥(ಹೃದಿ)

\as{೪} ಏತಾಃ ಕುಲೋತ್ತೀರ್ಣಯೋಗಿನ್ಯಃ ಸರ್ವಾರ್ಥಸಾಧಕೇ ಚಕ್ರೇ ಸಮುದ್ರಾಃ ಸಸಿದ್ಧಯಃ ಸಾಯುಧಾಃ ಸಶಕ್ತಯಃ ಸವಾಹನಾಃ ಸಪರಿವಾರಾಃ ಸರ್ವಾಃ ನ್ಯಸ್ತಾಃ ಸಂತು ನಮಃ (ಇತಿ ಹೃದಿ ಚಕ್ರಸಮರ್ಪಣಂ ನ್ಯಸ್ಯ)
\subsection{ಸರ್ವರಕ್ಷಾಕರಚಕ್ರನ್ಯಾಸಃ}
{\bfseries ೪ ಹ್ರೀಂ ಕ್ಲೀಂ ಬ್ಲೇಂ ಸರ್ವರಕ್ಷಾಕರ ಚಕ್ರಾಯ ನಮಃ}(ಇತಿ ವ್ಯಾಪಕಂ ವಿನ್ಯಸ್ಯ)\\
\as{೪ ಮಂ} ಸರ್ವಜ್ಞಾಯೈ ನಮಃ ।(ದಕ್ಷನಾಸಾಪುಟೇ)\\
\as{೪ ಯಂ} ಸರ್ವಶಕ್ತ್ಯೈ ನಮಃ ।(ದಕ್ಷಸೃಕ್ಕಣಿ)\\
\as{೪ ರಂ} ಸರ್ವೈಶ್ವರ್ಯಪ್ರದಾಯೈ ನಮಃ ।(ದಕ್ಷಸ್ತನೇ)\\
\as{೪ ಲಂ} ಸರ್ವಜ್ಞಾನಮಯ್ಯೈ ನಮಃ ।(ದಕ್ಷವೃಷಣೇ)\\
\as{೪ ವಂ} ಸರ್ವವ್ಯಾಧಿವಿನಾಶಿನ್ಯೈ ನಮಃ ।(ಸೀವಿನ್ಯಾ ದಕ್ಷಭಾಗೇ)\\
\as{೪ ಶಂ} ಸರ್ವಾಧಾರಸ್ವರೂಪಾಯೈ ನಮಃ ।\\(ವಾಮವೃಷಣೇ, ಸೀವಿನ್ಯಾ ವಾಮಭಾಗೇ)\\
\as{೪ ಷಂ} ಸರ್ವಪಾಪಹರಾಯೈ ನಮಃ ।(ವಾಮಸ್ತನೇ)\\
\as{೪ ಸಂ} ಸರ್ವಾನಂದಮಯ್ಯೈ ನಮಃ ।(ವಾಮಸೃಕ್ಕಣಿ)\\
\as{೪ ಹಂ} ಸರ್ವರಕ್ಷಾಸ್ವರೂಪಿಣ್ಯೈ ನಮಃ ।(ವಾಮನಾಸಾಪುಟೇ)\\
\as{೪ ಕ್ಷಂ} ಸರ್ವೇಪ್ಸಿತಫಲಪ್ರದಾಯೈ ನಮಃ ।(ನಾಸಾಗ್ರೇ)\\
\as{೪ ಹ್ರೀಂ ಕ್ಲೀಂ ಬ್ಲೇಂ॥} ಸರ್ವರಕ್ಷಾಕರಚಕ್ರೇಶ್ವರ್ಯೈ ತ್ರಿಪುರಮಾಲಿನ್ಯೈ ನಮಃ ॥(ಹೃದಿ)

\as{೪} ಏತಾಃ ನಿಗರ್ಭಯೋಗಿನ್ಯಃ ಸರ್ವರಕ್ಷಾಕರೇ ಚಕ್ರೇ ಸಮುದ್ರಾಃ ಸಸಿದ್ಧಯಃ ಸಾಯುಧಾಃ ಸಶಕ್ತಯಃ ಸವಾಹನಾಃ ಸಪರಿವಾರಾಃ ಸರ್ವಾಃ ನ್ಯಸ್ತಾಃ ಸಂತು ನಮಃ (ಇತಿ ಹೃದಿ ಚಕ್ರಸಮರ್ಪಣಂ ನ್ಯಸ್ಯ)
\subsection{ಸರ್ವರೋಗಹರ ಚಕ್ರನ್ಯಾಸಃ}
{\bfseries ೪ ಹ್ರೀಂ ಶ್ರೀಂ ಸೌಃ ಸರ್ವರೋಗಹರ ಚಕ್ರಾಯ ನಮಃ}\\(ಇತಿ ವ್ಯಾಪಕಂ ವಿನ್ಯಸ್ಯ)\\
\as{೪ ಅಂಆಂ++ಅಃ । ರ್ಬ್ಲೂಂ} ವಶಿನ್ಯೈ ನಮಃ ।(ದಕ್ಷಚಿಬುಕೇ)\\
\as{೪ ಕಂಖಂಗಂಘಂಙಂ । ಕ್‌ಲ್‌ಹ್ರೀಂ} ಕಾಮೇಶ್ವರ್ಯೈ ನಮಃ ।(ದಕ್ಷಕಂಠೇ)\\
\as{೪ ಚಂಛಂಜಂಝಂಞಂ । ನ್‌ವ್ಲೀಂ} ಮೋದಿನ್ಯೈ ನಮಃ ।(ಹೃದಯದಕ್ಷಭಾಗೇ)\\
\as{೪ ಟಂಠಂಡಂಢಂಣಂ । ಯ್ಲೂಂ} ವಿಮಲಾಯೈ ನಮಃ ।(ನಾಭಿದಕ್ಷಭಾಗೇ)\\
\as{೪ ತಂಥಂದಂಧಂನಂ । ಜ್‌ಮ್ರೀಂ} ಅರುಣಾಯೈ ನಮಃ ।(ನಾಭಿವಾಮಭಾಗೇ)\\
\as{೪ ಪಂಫಂಬಂಭಂಮಂ । ಹ್‌ಸ್‌ಲ್‌ವ್ಯೂಂ} ಜಯಿನ್ಯೈ ನಮಃ ।\\(ಹೃದಯವಾಮಭಾಗೇ)\\
\as{೪ ಯಂರಂಲಂವಂ । ಝ್‌ಮ್‌ರ್ಯೂಂ} ಸರ್ವೇಶ್ವರ್ಯೈ ನಮಃ ।\\(ವಾಮಕಂಠೇ)\\
\as{೪ ಶಂಷಂಸಂಹಂಳಂಕ್ಷಂ । ಕ್ಷ್‌ಮ್ರೀಂ} ಕೌಳಿನ್ಯೈ ನಮಃ ।(ವಾಮಚಿಬುಕೇ)\\
\as{೪ ಹ್ರೀಂ ಶ್ರೀಂ ಸೌಃ॥} ಸರ್ವರೋಗಹರಚಕ್ರೇಶ್ವರ್ಯೈ ತ್ರಿಪುರಾಸಿದ್ಧಾಯೈ ನಮಃ ॥(ಹೃದಿ)

\as{೪} ಏತಾಃ ರಹಸ್ಯಯೋಗಿನ್ಯಃ ಸರ್ವರೋಗಹರೇ ಚಕ್ರೇ ಸಮುದ್ರಾಃ ಸಸಿದ್ಧಯಃ ಸಾಯುಧಾಃ ಸಶಕ್ತಯಃ ಸವಾಹನಾಃ ಸಪರಿವಾರಾಃ ಸರ್ವಾಃ ನ್ಯಸ್ತಾಃ ಸಂತು ನಮಃ (ಇತಿ ಹೃದಿ ಚಕ್ರಸಮರ್ಪಣಂ ನ್ಯಸ್ಯ)
\subsection{ಆಯುಧನ್ಯಾಸಃ}
ಅಥ ಹೃದಿ ತ್ರಿಕೋಣಂ ವಿಭಾವ್ಯ ತತ್ರ ಪ್ರಾಗಾದಿದಿಕ್ಷು ಕ್ರಮೇಣಾಯುಧಾನಾಂ ಚತುಷ್ಟಯಂ ನ್ಯಸೇತ್ ।

\as{೪ ಯಾಂರಾಂಲಾಂವಾಂಸಾಂ ದ್ರಾಂದ್ರೀಂಕ್ಲೀಂಬ್ಲೂಂಸಃ} ಸರ್ವಜಂಭನೇಭ್ಯೋ ಕಾಮೇಶ್ವರಕಾಮೇಶ್ವರೀ ಬಾಣೇಭ್ಯೋ ನಮಃ ।(ತ್ರಿಕೋಣಪೃಷ್ಠೇ)\\
\as{೪ ಥಂಧಂ} ಸರ್ವಸಂಮೋಹನಾಭ್ಯಾಂ ಕಾಮೇಶ್ವರಕಾಮೇಶ್ವರೀ ಧನುರ್ಭ್ಯಾಂ ನಮಃ ।(ತ್ರಿಕೋಣದಕ್ಷೇ)\\
\as{೪ ಹ್ರೀಂಆಂ} ಸರ್ವವಶೀಕರಣಾಭ್ಯಾಂ ಕಾಮೇಶ್ವರಕಾಮೇಶ್ವರೀ ಪಾಶಾಭ್ಯಾಂ ನಮಃ ।(ತ್ರಿಕೋಣಾಗ್ರೇ)\\
\as{೪ ಕ್ರೋಂಕ್ರೋಂ} ಸರ್ವಸ್ತಂಭನಾಭ್ಯಾಂ ಕಾಮೇಶ್ವರ \\ಕಾಮೇಶ್ವರ್ಯಂಕುಶಾಭ್ಯಾಂ  ನಮಃ ।(ತ್ರಿಕೋಣವಾಮೇ)
\subsection{ಸರ್ವಸಿದ್ಧಿಪ್ರದಚಕ್ರನ್ಯಾಸಃ}
\as{೪ ಹ್‌ಸ್‌ರೈಂ ಹ್‌ಸ್‌ಕ್ಲ್ರೀಂ ಹ್‌ಸ್‌ರ್ಸೌಃ ಸರ್ವಸಿದ್ಧಿಪ್ರದ ಚಕ್ರಾಯ ನಮಃ}\\(ಇತಿ ವ್ಯಾಪಕಂ ವಿನ್ಯಸ್ಯ)\\
\as{೪ ಐಂ೫} ಕಾಮರೂಪಪೀಠಸ್ಥಾಯೈ ಮಹಾಕಾಮೇಶ್ವರ್ಯೈ ನಮಃ ।\\(ತ್ರಿಕೋಣಾಗ್ರೇ)\\
\as{೪ ಕ್ಲೀಂ೬} ಪೂರ್ಣಗಿರಿಪೀಠಸ್ಥಾಯೈ ಮಹಾವಜ್ರೇಶ್ವರ್ಯೈ ನಮಃ ।\\(ತದ್ದಕ್ಷಕೋಣೇ)\\
\as{೪ ಸೌಃ೪} ಜಾಲಂಧರಪೀಠಸ್ಥಾಯೈ ಮಹಾಭಗಮಾಲಿನ್ಯೈ ನಮಃ ।\\(ತದ್ವಾಮಕೋಣೇ)\\
\as{೪ ೧೫} ಓಡ್ಯಾಣಪೀಠಸ್ಥಾಯೈ ಮಹಾಶ್ರೀಸುಂದರ್ಯೈ ನಮಃ । (ಬಿಂದೌ)

ಅಥ ತ್ರಿಕೋಣಬಾಹುಷು ಬಿಂದೌ ಚ ಸಪರ್ಯೋಕ್ತಪ್ರಕಾರೇಣ ಷೋಡಶನಿತ್ಯಾಃ ನ್ಯಸೇತ್ ।

\as{೪ ಅಂ} ಐಂ ಸಕಲಹ್ರೀಂ ನಿತ್ಯಕ್ಲಿನ್ನೇ ಮದದ್ರವೇ ಸೌಃ \as{ಅಂ~॥} ಕಾಮೇಶ್ವರ್ಯೈ ನಮಃ ॥೧\\
\as{೪ ಆಂ} ಐಂ ಭಗಭುಗೇ ಭಗಿನಿ ಭಗೋದರಿ ಭಗಮಾಲೇ ಭಗಾವಹೇ ಭಗಗುಹ್ಯೇ ಭಗಯೋನಿ ಭಗನಿಪಾತನಿ ಸರ್ವಭಗವಶಂಕರಿ ಭಗರೂಪೇ ನಿತ್ಯಕ್ಲಿನ್ನೇ ಭಗಸ್ವರೂಪೇ ಸರ್ವಾಣಿ ಭಗಾನಿ ಮೇ ಹ್ಯಾನಯ ವರದೇ ರೇತೇ ಸುರೇತೇ ಭಗಕ್ಲಿನ್ನೇ ಕ್ಲಿನ್ನದ್ರವೇ ಕ್ಲೇದಯ ದ್ರಾವಯ ಅಮೋಘೇ ಭಗವಿಚ್ಚೇ ಕ್ಷುಭ ಕ್ಷೋಭಯ ಸರ್ವಸತ್ವಾನ್ ಭಗೇಶ್ವರಿ ಐಂ ಬ್ಲೂಂ ಜಂ ಬ್ಲೂಂ ಭೇಂ ಬ್ಲೂಂ ಮೋಂ ಬ್ಲೂಂ ಹೇಂ ಬ್ಲೂಂ ಹೇಂ ಕ್ಲಿನ್ನೇ ಸರ್ವಾಣಿ ಭಗಾನಿ ಮೇ ವಶಮಾನಯ ಸ್ತ್ರೀಂ ಹ್‌ರ್‌ಬ್ಲೇಂ ಹ್ರೀಂ \as{ಆಂ ॥} ಭಗಮಾಲಿನ್ಯೈ ನಮಃ ॥೨\\
\as{೪ ಇಂ} ಓಂ ಹ್ರೀಂ ನಿತ್ಯಕ್ಲಿನ್ನೇ ಮದದ್ರವೇ ಸ್ವಾಹಾ \as{ಇಂ ॥} ನಿತ್ಯಕ್ಲಿನ್ನಾಯೈ ನಮಃ ॥೩\\
\as{೪ ಈಂ} ಓಂ ಕ್ರೋಂಭ್ರೋಂಕ್ರೋಂಝ್ರೋಂಛ್ರೋಂಜ್ರೋಂ ಸ್ವಾಹಾ \as{ಈಂ ॥} ಭೇರುಂಡಾಯೈ ನಮಃ ॥೪\\
\as{೪ ಉಂ} ಓಂ ಹ್ರೀಂ ವಹ್ನಿವಾಸಿನ್ಯೈ ನಮಃ \as{ಉಂ~॥} ವಹ್ನಿವಾಸಿನ್ಯೈ ನಮಃ ॥೫\\
\as{೪ ಊಂ} ಹ್ರೀಂ ಕ್ಲಿನ್ನೇ ಐಂ ಕ್ರೋಂ ನಿತ್ಯಮದದ್ರವೇ ಹ್ರೀಂ \as{ಊಂ~॥} ಮಹಾವಜ್ರೇಶ್ವರ್ಯೈ ನಮಃ ॥೬\\
\as{೪ ಋಂ} ಹ್ರೀಂ ಶಿವಾದೂತ್ಯೈ ನಮಃ \as{ಋಂ~॥} ಶಿವಾದೂತ್ಯೈ ನಮಃ ॥೭\\
\as{೪ ೠಂ} ಓಂ ಹ್ರೀಂ ಹೂಂಖೇಚಛೇಕ್ಷಃಸ್ತ್ರೀಂಹೂಂಕ್ಷೇ ಹ್ರೀಂ ಫಟ್ \as{ೠಂ~॥} ತ್ವರಿತಾಯೈ ನಮಃ ॥೮\\
\as{೪ ಲೃಂ} ಐಂಕ್ಲೀಂಸೌಃ \as{ಲೃಂ~॥} ಕುಲಸುಂದರ್ಯೈ ನಮಃ ॥೯\\
\as{೪ ಲೄಂ} ಹಸಕಲರಡೈಂ ಹಸಕಲರಡೀಂ ಹಸಕಲರಡೌಃ \as{ಲೄಂ ॥} ನಿತ್ಯಾಯೈ ನಮಃ ॥೧೦\\
\as{೪ ಏಂ} ಹ್ರೀಂ ಫ್ರೇಂಸ್ರೂಂಓಂಆಂಕ್ಲೀಂಐಂಬ್ಲೂಂ ನಿತ್ಯಮದದ್ರವೇ ಹುಂಫ್ರೇಂ ಹ್ರೀಂ \as{ಏಂ~॥} ನೀಲಪತಾಕಾಯೈ ನಮಃ ॥೧೧\\
\as{೪ ಐಂ} ಭಮರಯಉಔಂ \as{ಐಂ ॥} ವಿಜಯಾಯೈ ನಮಃ ॥೧೨\\
\as{೪ ಓಂ} ಸ್ವೌಂ \as{ಓಂ ॥} ಸರ್ವಮಂಗಳಾಯೈ ನಮಃ ॥೧೩\\
\as{೪ ಔಂ} ಓಂ ನಮೋ ಭಗವತಿ ಜ್ವಾಲಾಮಾಲಿನಿ ದೇವದೇವಿ ಸರ್ವಭೂತಸಂಹಾರಕಾರಿಕೇ ಜಾತವೇದಸಿ ಜ್ವಲಂತಿ ಜ್ವಲ ಜ್ವಲ ಪ್ರಜ್ವಲ ಪ್ರಜ್ವಲ ಹ್ರಾಂ ಹ್ರೀಂ ಹ್ರೂಂ ರರ ರರ ರರರ ಹುಂ ಫಟ್ ಸ್ವಾಹಾ \as{ಔಂ~॥} ಜ್ವಾಲಾಮಾಲಿನ್ಯೈ ನಮಃ ॥೧೪\\
\as{೪ ಅಂ} ಚ್ಕೌಂ \as{ಅಂ~॥} ಚಿತ್ರಾಯೈ ನಮಃ ॥೧೫\\
\as{೪ ಅಃ} ೧೫ \as{ಅಃ~॥} ಲಲಿತಾ ಮಹಾನಿತ್ಯಾಯೈ ನಮಃ ॥೧೬ 

\as{೪ ಹ್‌ಸ್‌ರೈಂ ಹ್‌ಸ್‌ಕ್ಲ್ರೀಂ ಹ್‌ಸ್‌ರ್ಸೌಃ ॥} ಸರ್ವಸಿದ್ಧಿಪ್ರದಚಕ್ರೇಶ್ವರ್ಯೈ ತ್ರಿಪುರಾಂಬಾಯೈ ನಮಃ ॥(ಹೃದಿ)

\as{೪} ಏತಾಃ ಅತಿರಹಸ್ಯಯೋಗಿನ್ಯಃ ಸರ್ವಸಿದ್ಧಿಪ್ರದೇ ಚಕ್ರೇ ಸಮುದ್ರಾಃ ಸಸಿದ್ಧಯಃ ಸಾಯುಧಾಃ ಸಶಕ್ತಯಃ ಸವಾಹನಾಃ ಸಪರಿವಾರಾಃ ಸರ್ವಾಃ ನ್ಯಸ್ತಾಃ ಸಂತು ನಮಃ (ಇತಿ ಹೃದಿ ಚಕ್ರಸಮರ್ಪಣಂ ನ್ಯಸ್ಯ)
\subsection{ಸರ್ವಾನಂದಮಯಚಕ್ರನ್ಯಾಸಃ}
{\bfseries ೪ (೧೫) ಸರ್ವಾನಂದಮಯ ಚಕ್ರಾಯ ನಮಃ}(ಇತಿ ವ್ಯಾಪಕಂ ವಿನ್ಯಸ್ಯ)\\
\as{೪ ೧೫} ಶ್ರೀ ಶ್ರೀ ಮಹಾಭಟ್ಟಾರಿಕಾಯೈ ನಮಃ ।(ಹೃನ್ಮಧ್ಯೇ)

\as{೪} ಏಷಾ ಪರಾಪರಾತಿರಹಸ್ಯಯೋಗಿನೀ ಸರ್ವಸಿದ್ಧಿಪ್ರದೇ ಚಕ್ರೇ ಸಮುದ್ರಾ ಸಸಿದ್ಧಿಃ ಸಾಯುಧಾ ಸಶಕ್ತಿಃ ಸವಾಹನಾ ಸಪರಿವಾರಾ ನ್ಯಸ್ತಾ ಅಸ್ತು ನಮಃ ॥(ಹೃದಿ) \\
\as{೪ ೧೫} ಸರ್ವಾನಂದಮಯಚಕ್ರೇಶ್ವರ್ಯೈ\\ ಶ್ರೀಮಲ್ಲಲಿತಾಮಹಾತ್ರಿಪುರಸುಂದರ್ಯೈ ನಮಃ ॥\\(ಇತಿ ಹೃದಿ ನ್ಯಸ್ಯ ಯೋನಿಮುದ್ರಾಂ ಪ್ರದರ್ಶ್ಯ ಮೂಲಂ ಜಪ್ತ್ವಾ ಉತ್ತರನ್ಯಾಸಂ ವಿಧಾಯ ಧ್ಯಾಯೇತ್ ।)\\
\as{ಧ್ಯಾಯೇತ್ ಕಾಮೇಶ್ವರಾಂಕಸ್ಥಾಂ ಕುರುವಿಂದಮಣಿಪ್ರಭಾಂ ।\\ ಶೋಣಾಂಬರಸ್ರಗಾಲೇಪಾಂ ಸರ್ವಾಂಗೀಣವಿಭೂಷಣಾಂ ॥

ಸೌಂದರ್ಯಶೇವಧಿಂ ಸೇಷುಚಾಪಪಾಶಾಂಕುಶೋಜ್ಜ್ವಲಾಮ್ ।\\ ಸ್ವಭಾಭಿರಣಿಮಾದ್ಯಾಭಿಃ ಸೇವ್ಯಾಂ ಸರ್ವನಿಯಾಮಿಕಾಮ್ ॥

ಸಚ್ಚಿದಾನಂದ ವಪುಷಂ ಸದಯಾಪಾಂಗವಿಭ್ರಮಾಮ್ ।\\ ಸರ್ವಲೋಕೈಕ ಜನನೀಂ ಸ್ಮೇರಾಸ್ಯಾಂ ಲಲಿತಾಂಬಿಕಾಮ್ ॥}

\authorline{ಇತಿ ನ್ಯಾಸಾಃ}
\newpage
\section{ವರ್ಧಿನೀಕಲಶಸ್ಥಾಪನಮ್}
\addcontentsline{toc}{section}{ವರ್ಧಿನೀಕಲಶಸ್ಥಾಪನಮ್}
ಬಿಂದುತ್ರಿಕೋಣವೃತ್ತಚತುರಸ್ರಾತ್ಮಕಂ ಮಂಡಲಂ ವಿಧಾಯ , ಚತುರಸ್ರೇ ಆಗ್ನೇಯಾದಿಷು,

\as{೪ ಐಂ೫ } ಹೃದಯಾಯ ನಮಃ । ಹೃದಯಶಕ್ತಿ ಶ್ರೀಪಾದುಕಾಂ ಪೂ । ನಮಃ ॥\\
\as{೪ ಕ್ಲೀಂ೬ } ಶಿರಸೇ ಸ್ವಾಹಾ । ಶಿರಃಶಕ್ತಿ ಶ್ರೀಪಾದುಕಾಂ ಪೂ । ನಮಃ ॥\\
\as{೪ ಸೌಃ೪ } ಶಿಖಾಯೈ ವಷಟ್ । ಶಿಖಾಶಕ್ತಿ ಶ್ರೀಪಾದುಕಾಂ ಪೂ । ನಮಃ ॥\\
\as{೪ ಐಂ೫ } ಕವಚಾಯ ಹುಂ । ಕವಚಶಕ್ತಿ ಶ್ರೀಪಾದುಕಾಂ ಪೂ । ನಮಃ ॥\\
\as{೪ ಕ್ಲೀಂ೬} ನೇತ್ರತ್ರಯಾಯ ವೌಷಟ್ । ನೇತ್ರಶಕ್ತಿ ಶ್ರೀಪಾದುಕಾಂ ಪೂ । ನಮಃ ॥\\
\as{೪ ಸೌಃ೪ } ಅಸ್ತ್ರಾಯ ಫಟ್ । ಅಸ್ತ್ರಶಕ್ತಿ ಶ್ರೀಪಾದುಕಾಂ ಪೂ । ನಮಃ ॥ 

ಇತಿ ಷಡಂಗಾನಿ ವಿನ್ಯಸ್ಯ, ತ್ರಿಕೋಣೇ ಸ್ವಾಗ್ರಾದಿಕೋಣೇಷು\\
\as{೪ ಐಂ ಕಏಈಲಹ್ರೀಂ} ನಮಃ ॥\\
\as{೪ ಕ್ಲೀಂ ಹಸಕಹಲಹ್ರೀಂ} ನಮಃ ॥\\
\as{೪ ಸೌಃ ಸಕಲಹ್ರೀಂ} ನಮಃ ॥

ಇತಿ ಸಂಪೂಜ್ಯ, ವಿದ್ಯಯಾ ಬಿಂದುಂ ಸಂಪೂಜ್ಯ,

\as{೪ ಐಂ೫} ಅಂ ಅಗ್ನಿಮಂಡಲಾಯ ಧರ್ಮಪ್ರದದಶಕಲಾತ್ಮನೇ ವರ್ಧಿನೀಕಲಶಾಧಾರಾಯ ನಮಃ । ಇತಿ ಪಾತ್ರಾಧಾರಂ\\
\as{೪ ಕ್ಲೀಂ೬} ಉಂ ಅರ್ಕಮಂಡಲಾಯ ಅರ್ಥಪ್ರದದ್ವಾದಶಕಲಾತ್ಮನೇ ವರ್ಧಿನೀಕಲಶಾಯ ನಮಃ । ಇತಿ ಪಾತ್ರಂ ಚ ನಿಧಾಯ\\
\as{೪ ಸೌಃ೪} ಮಂ ಸೋಮಮಂಡಲಾಯ ಕಾಮಪ್ರದಷೋಡಶಕಲಾತ್ಮನೇ ವರ್ಧಿನೀಕಲಶಾಮೃತಾಯ ನಮಃ । ಇತಿ ಜಲಮಾಪೂರ್ಯ

ಕಲಶಸ್ಯ ಮುಖೇ ವಿಷ್ಣುಃ ಕಂಠೇ ರುದ್ರ ಸಮಾಶ್ರಿತಾಃ  ।\\
ಮೂಲೇ ತತ್ರ ಸ್ಥಿತೋಬ್ರಹ್ಮಾ, ಮಧ್ಯೇ ಮಾತೃ ಗಣಾಃ ಸ್ಮೃತಾಃ ॥

ಕುಕ್ಷೌ ತು ಸಾಗರಾಃ ಸರ್ವೇ ಸಪ್ತ ದ್ವೀಪ ವಸುಂಧರಾ।\\
ಋಗ್ವೇದೋಽಥ ಯಜುರ್ವೇದಃ ಸಾಮವೇದೋ ಹ್ಯಥರ್ವಣಃ।\\
ಅಂಗೈಶ್ಚ ಸಹಿತಾಃ ಸರ್ವೇ ಕಲಶಾಂಬು ಸಮಾಶ್ರಿತಾಃ ॥

ಅತ್ರ ಗಾಯತ್ರಿ ಸಾವಿತ್ರೀ ಶಾಂತಿಃ ಪುಷ್ಟಿಕರೀ ತಥಾ ।\\
ಆಯಾಂತು ದೇವ ಪೂಜಾರ್ಥಂ ದುರಿತಕ್ಷಯ ಕಾರಕಾಃ ॥

ಸರ್ವೇ ಸಮುದ್ರಾಃ ಸರಿತಾ ತೀರ್ಥಾನಿ ಜಲದಾ ನದಾಃ ।\\
ಗಂಗೇ ಚ ಯಮುನೇ ಚೈವ ಗೋದಾವರಿ ಸರಸ್ವತಿ ।\\
ನರ್ಮದೇ ಸಿಂಧು ಕಾವೇರಿ ಜಲೇಽಸ್ಮಿನ್ ಸನ್ನಿಧಿಂ ಕುರು ॥

\as{ಓಂ ಆಪೋ॒ ವಾ ಇ॒ದँ ಸರ್ವಂ॒ ವಿಶ್ವಾ॑ ಭೂ॒ತಾನ್ಯಾಪಃ॑ ಪ್ರಾ॒ಣಾ ವಾ ಆಪಃ॑
ಪ॒ಶವ॒ ಆಪೋಽನ್ನ॒ಮಾಪೋ॑ಽಮೃ॒ತಮಾಪಃ॑ ಸ॒ಮ್ರಾಡಾಪೋ॑ ವಿ॒ರಾಡಾಪಃ॑
ಸ್ವ॒ರಾಡಾಪ॒ಶ್ಛಂದಾँ॒॒ಸ್ಯಾಪೋ॒ ಜ್ಯೋತೀँ॒॒ಷ್ಯಾಪೋ॒
ಯಜೂँ॒॒ಷ್ಯಾಪಃ॑ ಸ॒ತ್ಯಮಾಪ॒ಸ್ಸರ್ವಾ॑ ದೇ॒ವತಾ॒ ಆಪೋ॒
ಭೂರ್ಭುವ॒ಸ್ಸುವ॒ರಾಪ॒ ಓಂ ॥

ಓಂ ಇ॒ಮಂ ಮೇ᳚ ಗಂಗೇ ಯಮುನೇ ಸರ॑ಸ್ವತಿ॒ ಶುತು॑ದ್ರಿ॒ಸ್ತೋಮಂ᳚ ಸಚತಾ॒ ಪರು॒ಷ್ಣ್ಯಾ ।
ಅ॒ಸಿ॒ಕ್ನಿ॒ಯಾ ಮ॑ರುಧ್ವೃದೇ ವಿ॒ತಸ್ತ॒ಯಾರ್ಜೀ᳚ಕೀಯೇ  ಶೃಣು॒ಹ್ಯಾ ಸು॒ಶೋಮ॑ಯಾ ॥
ಸಿತಾ᳚ಸಿತೇ ಸ॒ರಿತೇ॒ ಯತ್ರ॑ ಸಂಗ॒ತೇ ತತ್ರಾ᳚ ಪ್ಲು॒ತೋಸೋ॒ದಿವ॒ ಮುತ್ಪ॑ತಂತಿ ।
ಏವೈ᳚ ತ॒ನ್ವಾ\nicefrac{೧}{೨}(ಅಂ) ವಿಸೃ॑ಜಂತಿ॒ ಧೀರಾ॒ಸ್ತೇಜನಾ᳚ಸೋ ಅಮೃತ॒ತ್ವಂ ಭ॑ಜಂತೇ ॥}
\newpage
\as{ಸಿತಮಕರನಿಶಣ್ಣಾಂ ಶುಭ್ರವರ್ಣಾಂ ತ್ರಿನೇತ್ರಾಂ\\ಕರಧೃತ ಕಲಶೋದ್ಭ್ಯ ತ್ಪಂಕಜಾ ಭೀತ್ಯಭೀಷ್ಟಾಂ ।\\
ವಿಧಿ ಹರಿಹರ ರೂಪಾಂ ಸೇಂದು ಕೋಟೀರಚೂಡಾಂ\\ಭಸಿತ ಸಿತ ದುಕೂಲಾಂ ಜಾಹ್ನವೀಂ ತಾಂ ನಮಾಮಿ ॥}\\
ಗಂಗಾದಿ ಸರ್ವ ತೀರ್ಥೇಭ್ಯೋ ನಮಃ । ಇತಿ ಪಠಿತ್ವಾ, ಮೂಲೇನಾಭಿಮಂತ್ರ್ಯ ಧೇನುಮುದ್ರಾಂ ಪ್ರದರ್ಶ್ಯ ಗಂಗಾಂ ಷೋಡಶೋಪಚಾರೈಃ ಪೂಜಯೇತ್ ॥

\section{ಗಂಗಾಷ್ಟೋತ್ತರಶತನಾಮಾವಲಿಃ}
\begin{multicols}{2}ಗಂಗಾಯೈ~।\\ ವಿಷ್ಣುಪಾದಸಂಭೂತಾಯೈ~।\\ ಹರವಲ್ಲಭಾಯೈ~।\\ ಹಿಮಾಚಲೇಂದ್ರತನಯಾಯೈ~।\\ ಗಿರಿಮಂಡಲಗಾಮಿನ್ಯೈ~।\\ ತಾರಕಾರಾತಿಜನನ್ಯೈ~।\\ ಸಗರಾತ್ಮಜತಾರಕಾಯೈ~।\\ ಸರಸ್ವತೀಸಮಯುಕ್ತಾಯೈ~।\\ ಸುಘೋಷಾಯೈ~।\\ ಸಿಂಧುಗಾಮಿನ್ಯೈ~।\\ ಭಾಗೀರಥ್ಯೈ~।\\ ಭಾಗ್ಯವತ್ಯೈ~।\\ ಭಗೀರಥರಥಾನುಗಾಯೈ~।\\ ತ್ರಿವಿಕ್ರಮಪದೋದ್ಭೂತಾಯೈ~।\\ ತ್ರಿಲೋಕಪಥಗಾಮಿನ್ಯೈ~।\\ ಕ್ಷೀರಶುಭ್ರಾಯೈ~।\\ ಬಹುಕ್ಷೀರಾಯೈ~।\\ ಕ್ಷೀರವೃಕ್ಷಸಮಾಕುಲಾಯೈ~।\\ ತ್ರಿಲೋಚನಜಟಾವಾಸಾಯೈ~।\\ ಋಣತ್ರಯವಿಮೋಚಿನ್ಯೈ~।\\ ತ್ರಿಪುರಾರಿಶಿರಶ್ಚೂಡಾಯೈ~।\\ ಜಾಹ್ನವ್ಯೈ~।\\ ನರಕಭೀತಿಹೃತೇ~।\\ ಅವ್ಯಯಾಯೈ~।\\ ನಯನಾನಂದದಾಯಿನ್ಯೈ~।\\ ನಗಪುತ್ರಿಕಾಯೈ~।\\ ನಿರಂಜನಾಯೈ~।\\ ನಿತ್ಯಶುದ್ಧಾಯೈ~।\\ ನೀರಜಾಲಿಪರಿಷ್ಕೃತಾಯೈ~।\\ ಸಾವಿತ್ರ್ಯೈ~।\\ ಸಲಿಲಾವಾಸಾಯೈ~।\\ ಸಾಗರಾಂಬುಸಮೇಧಿನ್ಯೈ~।\\ ರಮ್ಯಾಯೈ~।\\ ಬಿಂದುಸರಸೇ~।\\ ಅವ್ಯಕ್ತಾಯೈ~।\\ ಅವ್ಯಕ್ತರೂಪಧೃತೇ~।\\ ಉಮಾಸಪತ್ನ್ಯೈ~।\\ ಶುಭ್ರಾಂಗಾಯೈ~।\\ ಶ್ರೀಮತ್ಯೈ~।\\ ಧವಲಾಂಬರಾಯೈ~।\\ ಆಖಂಡಲವನಾವಾಸಾಯೈ~।\\ ಕಂಠೇಂದುಕೃತಶೇಖರಾಯೈ~।\\ ಅಮೃತಾಕಾರಸಲಿಲಾಯೈ~।\\ ಲೀಲಾಲಿಂಗಿತಪರ್ವತಾಯೈ~।\\ ವಿರಿಂಚಿಕಲಶಾವಾಸಾಯೈ~।\\ ತ್ರಿವೇಣ್ಯೈ~।\\ ತ್ರಿಗುಣಾತ್ಮಿಕಾಯೈ~।\\ ಸಂಗತಾಘೌಘಶಮನ್ಯೈ~।\\ ಭೀತಿಹರ್ತ್ರ್ಯೈ~।\\ ಶಂಖದುಂದುಭಿನಿಸ್ವನಾಯೈ~।\\ ಭಾಗ್ಯದಾಯಿನ್ಯೈ~।\\ ಭಾಗ್ಯಜನನ್ಯೈ~।\\ ನಂದಿನ್ಯೈ~।\\ ಶೀಘ್ರಗಾಯೈ~।\\ ಸಿದ್ಧಾಯೈ~।\\ ಶರಣ್ಯೈ~।\\ ಶಶಿಶೇಖರಾಯೈ~।\\ ಶಾಂಕರ್ಯೈ~।\\ ಶಫರೀಪೂರ್ಣಾಯೈ~।\\ ಭರ್ಗಮೂರ್ಧಕೃತಾಲಯಾಯೈ~।\\ ಭವಪ್ರಿಯಾಯೈ~।\\ ಸತ್ಯಸಂಧಪ್ರಿಯಾಯೈ~।\\ ಹಂಸಸ್ವರೂಪಿಣ್ಯೈ~।\\ ಭಗೀರಥಭೃತಾಯೈ~।\\ ಅನಂತಾಯೈ~।\\ ಶರಚ್ಚಂದ್ರನಿಭಾನನಾಯೈ~।\\ ಓಂಕಾರರೂಪಿಣ್ಯೈ~।\\ ಅನಲಾಯೈ~।\\ ಕ್ರೀಡಾಕಲ್ಲೋಲಕಾರಿಣ್ಯೈ~।\\ ಸ್ವರ್ಗಸೋಪಾನಸರಣ್ಯೈ~।\\ ಸರ್ವದೇವಸ್ವರೂಪಿಣ್ಯೈ~।\\ ಅಂಭಃಪ್ರದಾಯೈ~।\\ ದುಃಖಹಂತ್ರ್ಯೈ~।\\ ಶಾಂತಿ ಸಂತಾನಕಾರಿಣ್ಯೈ~।\\ ದಾರಿದ್ರ್ಯಹಂತ್ರ್ಯೈ~।\\ ಶಿವದಾಯೈ~।\\ ಸಂಸಾರವಿಷನಾಶಿನ್ಯೈ~।\\ ಪ್ರಯಾಗನಿಲಯಾಯೈ~।\\ ಶ್ರೀದಾಯೈ~।\\ ತಾಪತ್ರಯವಿಮೋಚಿನ್ಯೈ~।\\ ಶರಣಾಗತದೀನಾರ್ತಪರಿತ್ರಾಣಾಯೈ~।\\ ಸುಮುಕ್ತಿದಾಯೈ~।\\ ಪಾಪಹಂತ್ರ್ಯೈ~।\\ ಪಾವನಾಂಗಾಯೈ~।\\ ಪರಬ್ರಹ್ಮಸ್ವರೂಪಿಣ್ಯೈ~।\\ ಪೂರ್ಣಾಯೈ~।\\ ಪುರಾತನಾಯೈ~।\\ ಪುಣ್ಯಾಯೈ~।\\ ಪುಣ್ಯದಾಯೈ~।\\ ಪುಣ್ಯವಾಹಿನ್ಯೈ~।\\ ಪುಲೋಮಜಾರ್ಚಿತಾಯೈ~।\\ ಭೂದಾಯೈ~।\\ ಪೂತತ್ರಿಭುವನಾಯೈ~।\\ ಜಯಾಯೈ~।\\ ಜಂಗಮಾಯೈ~।\\ ಜಂಗಮಾಧಾರಾಯೈ~।\\ ಜಲರೂಪಾಯೈ~।\\ ಜಗದ್ಧಾತ್ರ್ಯೈ~।\\ ಜಗದ್ಭೂತಾಯೈ~।\\ ಜನಾರ್ಚಿತಾಯೈ~।\\ ಜಹ್ನುಪುತ್ರ್ಯೈ~।\\ ಜಗನ್ಮಾತ್ರೇ~।\\ ಜಂಬೂದ್ವೀಪವಿಹಾರಿಣ್ಯೈ~।\\ ಭವಪತ್ನ್ಯೈ~।\\ ಭೀಷ್ಮಮಾತ್ರೇ~।\\ ಸಿಕ್ತಾಯೈ~।\\ ರಮ್ಯರೂಪಧೃತೇ~।\\ ಉಮಾಸಹೋದರ್ಯೈ~।\\ ಅಜ್ಞಾನತಿಮಿರಾಪಹೃತೇ ನಮಃ।\\{\Large ॥ಇತಿ ಶ್ರೀಗಂಗಾಷ್ಟೋತ್ತರಶತನಾಮಾವಲಿಃ ॥}\end{multicols}
\as{ಭಾಗೀರಥ್ಯೈ ಚ ವಿದ್ಮಹೇ ವಿಷ್ಣುಪತ್ನ್ಯೈ ಚ ಧೀಮಹಿ । ತನ್ನೋ ಗಂಗಾ ಪ್ರಚೋದಯಾತ್ ॥}
\newpage
\section{ಸಾಮಾನ್ಯಾರ್ಘ್ಯ ವಿಧಿಃ}
\addcontentsline{toc}{section}{ಸಾಮಾನ್ಯಾರ್ಘ್ಯ ವಿಧಿಃ}
ಬಿಂದು ತ್ರಿಕೋಣ ಷಟ್ಕೋಣ ವೃತ್ತ ಚತುರಸ್ರಾತ್ಮಕಂ ಮಂಡಲಂ ವಿಧಾಯ , ಚತುರಸ್ರೇ ಆಗ್ನೇಯಾದಿಷು,\\
\as{೪ ಐಂ } ಹೃದಯಾಯ ನಮಃ । ಹೃದಯಶಕ್ತಿ ಶ್ರೀಪಾದುಕಾಂ ಪೂ । ನಮಃ ॥\\
\as{೪ ಕ್ಲೀಂ } ಶಿರಸೇ ಸ್ವಾಹಾ । ಶಿರಃಶಕ್ತಿ ಶ್ರೀಪಾದುಕಾಂ ಪೂ । ನಮಃ ॥\\
\as{೪ ಸೌಃ } ಶಿಖಾಯೈ ವಷಟ್ । ಶಿಖಾಶಕ್ತಿ ಶ್ರೀಪಾದುಕಾಂ ಪೂ । ನಮಃ ॥\\
\as{೪ ಐಂ } ಕವಚಾಯ ಹುಂ । ಕವಚಶಕ್ತಿ ಶ್ರೀಪಾದುಕಾಂ ಪೂ । ನಮಃ ॥\\
\as{೪ ಕ್ಲೀಂ} ನೇತ್ರತ್ರಯಾಯ ವೌಷಟ್ । ನೇತ್ರಶಕ್ತಿ ಶ್ರೀಪಾದುಕಾಂ ಪೂ । ನಮಃ ॥\\
\as{೪ ಸೌಃ } ಅಸ್ತ್ರಾಯ ಫಟ್ । ಅಸ್ತ್ರಶಕ್ತಿ ಶ್ರೀಪಾದುಕಾಂ ಪೂ । ನಮಃ ॥

ಷಟ್ಕೋಣೇ ಸ್ವಾಗ್ರಾದಿಕೋಣೇಷು\\
\as{೪ ಐಂ } ಹೃದಯಾಯ ನಮಃ । ಹೃದಯಶಕ್ತಿ ಶ್ರೀಪಾದುಕಾಂ ಪೂ । ನಮಃ ॥\\
\as{೪ ಕ್ಲೀಂ } ಶಿರಸೇ ಸ್ವಾಹಾ । ಶಿರಃಶಕ್ತಿ ಶ್ರೀಪಾದುಕಾಂ ಪೂ । ನಮಃ ॥\\
\as{೪ ಸೌಃ } ಶಿಖಾಯೈ ವಷಟ್ । ಶಿಖಾಶಕ್ತಿ ಶ್ರೀಪಾದುಕಾಂ ಪೂ । ನಮಃ ॥\\
\as{೪ ಐಂ } ಕವಚಾಯ ಹುಂ । ಕವಚಶಕ್ತಿ ಶ್ರೀಪಾದುಕಾಂ ಪೂ । ನಮಃ ॥\\
\as{೪ ಕ್ಲೀಂ} ನೇತ್ರತ್ರಯಾಯ ವೌಷಟ್ । ನೇತ್ರಶಕ್ತಿ ಶ್ರೀಪಾದುಕಾಂ ಪೂ । ನಮಃ ॥\\
\as{೪ ಸೌಃ } ಅಸ್ತ್ರಾಯ ಫಟ್ । ಅಸ್ತ್ರಶಕ್ತಿ ಶ್ರೀಪಾದುಕಾಂ ಪೂ । ನಮಃ ॥

ಇತಿ ಚ ಷಡಂಗಾನಿ ವಿನ್ಯಸ್ಯ, ತ್ರಿಕೋಣೇ ಸ್ವಾಗ್ರಾದಿಕೋಣೇಷು\\
\as{೪ ಐಂ ಕಏಈಲಹ್ರೀಂ} ನಮಃ ॥\\
\as{೪ ಕ್ಲೀಂ ಹಸಕಹಲಹ್ರೀಂ} ನಮಃ ॥\\
\as{೪ ಸೌಃ ಸಕಲಹ್ರೀಂ} ನಮಃ ॥

ಇತಿ ಸಂಪೂಜ್ಯ, ವಿದ್ಯಯಾ ಬಿಂದುಂ ಸಂಪೂಜ್ಯ, \as{ಅಸ್ತ್ರಾಯ ಫಟ್} ಇತಿ ಸಾಮಾನ್ಯಾರ್ಘ್ಯಪಾತ್ರಾಧಾರಂ ಪ್ರಕ್ಷಾಲ್ಯ\\
\as{೪ ಅಂ} ಅಗ್ನಿಮಂಡಲಾಯ ಧರ್ಮಪ್ರದದಶಕಲಾತ್ಮನೇ ಸಾಮಾನ್ಯಾರ್ಘ್ಯಪಾತ್ರಾಧಾರಾಯ ನಮಃ । ಇತಿ ಮಂಡಲೋಪರಿ ನಿಧಾಯ\\
\as{೪ ಅಗ್ನಿಂ ದೂತಂ +++++ಸುಕ್ರತುಮ್ ॥} ರಾಂರೀಂರೂಂರೈಂರೌಂರಃ ರಮಲವರಯೂಂ ಅಗ್ನಿಮಂಡಲಾಯ ನಮಃ । ಇತಿ ಅಗ್ನಿಮಂಡಲಂ ವಿಭಾವ್ಯ, ದಶ ವಹ್ನಿಕಾಲಾಃ ಪೂಜಯೇತ್ ।

\as{ಓಂಐಂಹ್ರೀಂಶ್ರೀಂ ಯಂ} ಧೂಮ್ರಾರ್ಚಿಷೇ   ನಮಃ~। \as{೪  ರಂ} ಊಷ್ಮಾಯೈ~। \as{೪  ಲಂ} ಜ್ವಲಿನ್ಯೈ~। \as{೪  ವಂ} ಜ್ವಾಲಿನ್ಯೈ~। \as{೪  ಶಂ} ವಿಸ್ಫುಲಿಂಗಿನ್ಯೈ ~। \as{೪  ಷಂ} ಸುಶ್ರಿಯೈ ~। \as{೪  ಸಂ} ಸುರೂಪಾಯೈ ~। \as{೪  ಹಂ} ಕಪಿಲಾಯೈ ~। \as{೪  ಳಂ} ಹವ್ಯವಹಾಯೈ ~। \as{೪  ಕ್ಷಂ} ಕವ್ಯವಹಾಯೈ ನಮಃ~॥

ಅಸ್ತ್ರಮಂತ್ರೇಣ ಶಂಖಂ ಪ್ರಕ್ಷಾಳ್ಯ\\
\as{೪ ಉಂ} ಸೂರ್ಯಮಂಡಲಾಯ ಅರ್ಥಪ್ರದದ್ವಾದಶಕಲಾತ್ಮನೇ ಸಾಮಾನ್ಯಾರ್ಘ್ಯಪಾತ್ರಾಯ ನಮಃ । ಇತಿ ನಿಧಾಯ\\
\as{೪ ಆಕೃಷ್ಣೇನ +++++ನಿ ಪಶ್ಯನ್ ॥} ಹಾಂಹೀಂಹೂಂಹೈಂಹೌಂಹಃ ಹಮಲವರಯೂಂ ಸೂರ್ಯಮಂಡಲಾಯ ನಮಃ । ಇತಿ ಸೂರ್ಯಮಂಡಲಂ ವಿಭಾವ್ಯ, ದ್ವಾದಶ ಸೂರ್ಯಕಾಲಾಃ ಪೂಜಯೇತ್ ।

\as{ಓಂಐಂಹ್ರೀಂಶ್ರೀಂ ಕಂಭಂ} ತಪಿನ್ಯೈ ನಮಃ~। \as{೪ ಖಂಬಂ} ತಾಪಿನ್ಯೈ~। \as{೪ ಗಂಫಂ} ಧೂಮ್ರಾಯೈ~। \as{೪ ಘಂಪಂ} ಮರೀಚ್ಯೈ~। \as{೪ ಙಂನಂ} ಜ್ವಾಲಿನ್ಯೈ~। \as{೪ ಚಂಧಂ} ರುಚ್ಯೈ~। \as{೪ ಛಂದಂ} ಸುಷುಮ್ನಾಯೈ~। \as{೪ ಜಂಥಂ} ಭೋಗದಾಯೈ~। \as{೪ ಝಂತಂ} ವಿಶ್ವಾಯೈ~। \as{೪ ಞಂಣಂ} ಬೋಧಿನ್ಯೈ~। \as{೪ ಟಂಢಂ} ಧಾರಿಣ್ಯೈ~। \as{೪ ಠಂಡಂ} ಕ್ಷಮಾಯೈ ನಮಃ~॥

\as{೪ ಮಂ} ಸೋಮಮಂಡಲಾಯ ಕಾಮಪ್ರದಷೋಡಶಕಲಾತ್ಮನೇ ಸಾಮಾನ್ಯಾರ್ಘ್ಯಪಾತ್ರಾಮೃತಾಯ ನಮಃ । ಇತಿ ಕಲಶೋದಕೇನ ಶಂಖಂ ಪ್ರಪೂರ್ಯ, ಕ್ಷೀರಬಿಂದುಂ ದತ್ವಾ  ಗಂಧಾದಿಭಿರಭ್ಯರ್ಚ್ಯ,\\
\as{೪ ಆಪ್ಯಾಯಸ್ವ +++++ಸಂಗಥೇ ॥} ಸಾಂಸೀಂಸೂಂಸೈಂಸೌಂಸಃ ಸಮಲವರಯೂಂ ಸೋಮಮಂಡಲಾಯ ನಮಃ । ಇತಿ ಸೋಮಮಂಡಲಂ ವಿಭಾವ್ಯ ,ತತ್ರ ಷೋಡಶ ಸೋಮಕಾಲಾಃ ಪೂಜಯೇತ್ ।

\as{ಓಂಐಂಹ್ರೀಂಶ್ರೀಂ ಅಂ} ಅಮೃತಾಯೈ ನಮಃ~। \as{೪ ಆಂ} ಮಾನದಾಯೈ~। \as{೪ ಇಂ} ಪೂಷಾಯೈ~। \as{೪ ಈಂ} ತುಷ್ಟ್ಯೈ~। \as{೪ ಉಂ} ಪುಷ್ಟ್ಯೈ~। \as{೪ ಊಂ} ರತ್ಯೈ~। \as{೪ ಋಂ} ಧೃತ್ಯೈ~। \as{೪ ೠಂ} ಶಶಿನ್ಯೈ~। \as{೪ ಲೃಂ} ಚಂದ್ರಿಕಾಯೈ~। \as{೪ ಲೄಂ} ಕಾಂತ್ಯೈ~। \as{೪ ಏಂ} ಜ್ಯೋತ್ಸ್ನಾಯೈ~। \as{೪ ಐಂ} ಶ್ರಿಯೈ~। \as{೪ ಓಂ} ಪ್ರೀತ್ಯೈ~। \as{೪ ಔಂ} ಅಂಗದಾಯೈ~। \as{೪ ಅಂ} ಪೂರ್ಣಾಯೈ~। \as{೪ ಅಃ} ಪೂರ್ಣಾಮೃತಾಯೈ ನಮಃ~॥

ತತಃ ಆಗ್ನೇಯಾದಿ ದಿಕ್ಷು ಮಧ್ಯೇ ಚ \\
\as{೪ ಐಂ } ಹೃದಯಾಯ ನಮಃ । ಹೃದಯಶಕ್ತಿ ಶ್ರೀಪಾದುಕಾಂ ಪೂ । ನಮಃ ॥\\
\as{೪ ಕ್ಲೀಂ } ಶಿರಸೇ ಸ್ವಾಹಾ । ಶಿರಃಶಕ್ತಿ ಶ್ರೀಪಾದುಕಾಂ ಪೂ । ನಮಃ ॥\\
\as{೪ ಸೌಃ } ಶಿಖಾಯೈ ವಷಟ್ । ಶಿಖಾಶಕ್ತಿ ಶ್ರೀಪಾದುಕಾಂ ಪೂ । ನಮಃ ॥\\
\as{೪ ಐಂ } ಕವಚಾಯ ಹುಂ । ಕವಚಶಕ್ತಿ ಶ್ರೀಪಾದುಕಾಂ ಪೂ । ನಮಃ ॥\\
\as{೪ ಕ್ಲೀಂ} ನೇತ್ರತ್ರಯಾಯ ವೌಷಟ್ । ನೇತ್ರಶಕ್ತಿ ಶ್ರೀಪಾದುಕಾಂ ಪೂ । ನಮಃ ॥\\
\as{೪ ಸೌಃ } ಅಸ್ತ್ರಾಯ ಫಟ್ । ಅಸ್ತ್ರಶಕ್ತಿ ಶ್ರೀಪಾದುಕಾಂ ಪೂ । ನಮಃ ॥\\
ಇತಿ ಚ ಷಡಂಗಾನಿ ವಿನ್ಯಸ್ಯ

\as{ಅಸ್ತ್ರಾಯ ಫಟ್ } ಇತಿ ಅಸ್ತ್ರೇಣ ಸಂರಕ್ಷ್ಯ । \as{ಕವಚಾಯ ಹುಂ } ಇತಿ ಅವಕುಂಠನ ಮುದ್ರಯಾ ಅವಕುಂಠ್ಯ\\
ಪ್ರಣವೇನ ಅಷ್ಟವಾರಮಭಿಮಂತ್ರ್ಯ ಧೇನುಮುದ್ರಾಂ ಯೋನಿಮುದ್ರಾಂ ಚ ಪ್ರದರ್ಶ್ಯ ।  ಪೂಜಾದ್ರವ್ಯಾಣಿ ಆತ್ಮಾನಂ ಚ ಸಂಪ್ರೋಕ್ಷ್ಯ , ಕಲಶೇ ಕಿಂಚಿನ್ನಿಕ್ಷಿಪೇತ್ ।

\section{ವಿಶೇಷಾರ್ಘ್ಯ ವಿಧಿಃ}
\addcontentsline{toc}{section}{ವಿಶೇಷಾರ್ಘ್ಯ ವಿಧಿಃ}
ಬಿಂದು ತ್ರಿಕೋಣ ಷಟ್ಕೋಣ ವೃತ್ತ ಚತುರಸ್ರಾತ್ಮಕಂ ಮಂಡಲಂ ವಿಧಾಯ , ಚತುರಸ್ರೇ ಆಗ್ನೇಯಾದಿಷು,\\
\as{೪ ಐಂ೫ } ಹೃದಯಾಯ ನಮಃ । ಹೃದಯಶಕ್ತಿ ಶ್ರೀಪಾದುಕಾಂ ಪೂ । ನಮಃ ॥\\
\as{೪ ಕ್ಲೀಂ೬ } ಶಿರಸೇ ಸ್ವಾಹಾ । ಶಿರಃಶಕ್ತಿ ಶ್ರೀಪಾದುಕಾಂ ಪೂ । ನಮಃ ॥\\
\as{೪ ಸೌಃ೪ } ಶಿಖಾಯೈ ವಷಟ್ । ಶಿಖಾಶಕ್ತಿ ಶ್ರೀಪಾದುಕಾಂ ಪೂ । ನಮಃ ॥\\
\as{೪ ಐಂ೫ } ಕವಚಾಯ ಹುಂ । ಕವಚಶಕ್ತಿ ಶ್ರೀಪಾದುಕಾಂ ಪೂ । ನಮಃ ॥\\
\as{೪ ಕ್ಲೀಂ೬} ನೇತ್ರತ್ರಯಾಯ ವೌಷಟ್ । ನೇತ್ರಶಕ್ತಿ ಶ್ರೀಪಾದುಕಾಂ ಪೂ । ನಮಃ ॥\\
\as{೪ ಸೌಃ೪ } ಅಸ್ತ್ರಾಯ ಫಟ್ । ಅಸ್ತ್ರಶಕ್ತಿ ಶ್ರೀಪಾದುಕಾಂ ಪೂ । ನಮಃ ॥

ಷಟ್ಕೋಣೇ ಸ್ವಾಗ್ರಾದಿಕೋಣೇಷು\\
\as{೪ ಐಂ೫ } ಹೃದಯಾಯ ನಮಃ । ಹೃದಯಶಕ್ತಿ ಶ್ರೀಪಾದುಕಾಂ ಪೂ । ನಮಃ ॥\\
\as{೪ ಕ್ಲೀಂ೬ } ಶಿರಸೇ ಸ್ವಾಹಾ । ಶಿರಃಶಕ್ತಿ ಶ್ರೀಪಾದುಕಾಂ ಪೂ । ನಮಃ ॥\\
\as{೪ ಸೌಃ೪ } ಶಿಖಾಯೈ ವಷಟ್ । ಶಿಖಾಶಕ್ತಿ ಶ್ರೀಪಾದುಕಾಂ ಪೂ । ನಮಃ ॥\\
\as{೪ ಐಂ೫ } ಕವಚಾಯ ಹುಂ । ಕವಚಶಕ್ತಿ ಶ್ರೀಪಾದುಕಾಂ ಪೂ । ನಮಃ ॥\\
\as{೪ ಕ್ಲೀಂ೬} ನೇತ್ರತ್ರಯಾಯ ವೌಷಟ್ । ನೇತ್ರಶಕ್ತಿ ಶ್ರೀಪಾದುಕಾಂ ಪೂ । ನಮಃ॥\\
\as{೪ ಸೌಃ೪ } ಅಸ್ತ್ರಾಯ ಫಟ್ । ಅಸ್ತ್ರಶಕ್ತಿ ಶ್ರೀಪಾದುಕಾಂ ಪೂ । ನಮಃ ॥

ಇತಿ ಚ ಷಡಂಗಾನಿ ವಿನ್ಯಸ್ಯ, ತ್ರಿಕೋಣೇ ಸ್ವಾಗ್ರಾದಿಕೋಣೇಷು\\
\as{೪ ಐಂ ಕಏಈಲಹ್ರೀಂ} ನಮಃ ॥\\
\as{೪ ಕ್ಲೀಂ ಹಸಕಹಲಹ್ರೀಂ} ನಮಃ ॥\\
\as{೪ ಸೌಃ ಸಕಲಹ್ರೀಂ} ನಮಃ ॥

ಇತಿ ಸಂಪೂಜ್ಯ, ಮೂಲೇನ ಬಿಂದುಂ ಸಂಪೂಜ್ಯ, 

ಅಸ್ತ್ರ ಮಂತ್ರೇಣ ವಿಶೇಷಾರ್ಘ್ಯಪಾತ್ರಾಧಾರಂ ಪ್ರಕ್ಷಾಲ್ಯ\\
\as{೪ ಐಂ೫ ಅಂ} ಅಗ್ನಿಮಂಡಲಾಯ ಧರ್ಮಪ್ರದದಶಕಲಾತ್ಮನೇ ವಿಶೇಷಾರ್ಘ್ಯಪಾತ್ರಾಧಾರಾಯ ನಮಃ । ಇತಿ ಮಂಡಲೋಪರಿ ನಿಧಾಯ\\
\as{೪ ಅಗ್ನಿಂ ದೂತಂ +++++ಸುಕ್ರತುಮ್ ॥} ರಾಂರೀಂರೂಂರೈಂರೌಂರಃ ರಮಲವರಯೂಂ ಅಗ್ನಿಮಂಡಲಾಯ ನಮಃ । ಇತಿ ಅಗ್ನಿಮಂಡಲಂ ವಿಭಾವ್ಯ , ದಶ ವಹ್ನಿಕಾಲಾಃ ಪೂಜಯೇತ್ ।

\as{ಓಂಐಂಹ್ರೀಂಶ್ರೀಂ ಯಂ} ಧೂಮ್ರಾರ್ಚಿಷೇ   ನಮಃ~। \as{೪  ರಂ} ಊಷ್ಮಾಯೈ~। \as{೪  ಲಂ} ಜ್ವಲಿನ್ಯೈ~। \as{೪  ವಂ} ಜ್ವಾಲಿನ್ಯೈ~। \as{೪  ಶಂ} ವಿಸ್ಫುಲಿಂಗಿನ್ಯೈ ~। \as{೪  ಷಂ} ಸುಶ್ರಿಯೈ ~। \as{೪  ಸಂ} ಸುರೂಪಾಯೈ ~। \as{೪  ಹಂ} ಕಪಿಲಾಯೈ ~। \as{೪  ಳಂ} ಹವ್ಯವಹಾಯೈ ~। \as{೪  ಕ್ಷಂ} ಕವ್ಯವಹಾಯೈ ನಮಃ~॥

ಅಸ್ತ್ರ ಮಂತ್ರೇಣ ವಿಶೇಷಾರ್ಘ್ಯಪಾತ್ರಂ ಪ್ರಕ್ಷಾಳ್ಯ
\as{೪ ಕ್ಲೀಂ೬ ಉಂ} ಸೂರ್ಯಮಂಡಲಾಯ ಅರ್ಥಪ್ರದ ದ್ವಾದಶಕಲಾತ್ಮನೇ ವಿಶೇಷಾರ್ಘ್ಯ ಪಾತ್ರಾಯ ನಮಃ । ಇತಿ ಪಾತ್ರಂ ನಿಧಾಯ

\as{೪} ಹ್ರೀಂ ಐಂ ಮಹಾಲಕ್ಷ್ಮೀಶ್ವರಿ ಪರಮಸ್ವಾಮಿನಿ ಊರ್ಧ್ವಶೂನ್ಯಪ್ರವಾಹಿನಿ ಸೋಮಸೂರ್ಯಾಗ್ನಿಭಕ್ಷಿಣಿ ಪರಮಾಕಾಶಭಾಸುರೇ ಆಗಚ್ಛಾಗಚ್ಛ ವಿಶ ವಿಶ ಪಾತ್ರಂ ಗೃಹಾಣ ಗೃಹಾಣ ಹುಂ ಫಟ್ ಸ್ವಾಹಾ ॥ ಇತಿ ಪುಷ್ಪಂ ವಿಕೀರ್ಯ\\
\as{೪ ಆಕೃಷ್ಣೇನ +++++ನಿ ಪಶ್ಯನ್ ॥} ಹಾಂಹೀಂಹೂಂಹೈಂಹೌಂಹಃ ಹಮಲವರಯೂಂ ಸೂರ್ಯಮಂಡಲಾಯ ನಮಃ । ಇತಿ ಸೂರ್ಯಮಂಡಲಂ ವಿಭಾವ್ಯ , ದ್ವಾದಶ ಸೂರ್ಯಕಾಲಾಃ ಪೂಜಯೇತ್ ।

\as{ಓಂಐಂಹ್ರೀಂಶ್ರೀಂ ಕಂಭಂ} ತಪಿನ್ಯೈ ನಮಃ~। \as{೪ ಖಂಬಂ} ತಾಪಿನ್ಯೈ~। \as{೪ ಗಂಫಂ} ಧೂಮ್ರಾಯೈ~। \as{೪ ಘಂಪಂ} ಮರೀಚ್ಯೈ~। \as{೪ ಙಂನಂ} ಜ್ವಾಲಿನ್ಯೈ~। \as{೪ ಚಂಧಂ} ರುಚ್ಯೈ~। \as{೪ ಛಂದಂ} ಸುಷುಮ್ನಾಯೈ~। \as{೪ ಜಂಥಂ} ಭೋಗದಾಯೈ~। \as{೪ ಝಂತಂ} ವಿಶ್ವಾಯೈ~। \as{೪ ಞಂಣಂ} ಬೋಧಿನ್ಯೈ~। \as{೪ ಟಂಢಂ} ಧಾರಿಣ್ಯೈ~। \as{೪ ಠಂಡಂ} ಕ್ಷಮಾಯೈ ನಮಃ~॥

\as{೪ ಸೌಃ೪ ಮಂ} ಸೋಮಮಂಡಲಾಯ ಕಾಮಪ್ರದ ಷೋಡಶಕಲಾತ್ಮನೇ ವಿಶೇಷಾರ್ಘ್ಯ ಪಾತ್ರಾಮೃತಾಯ ನಮಃ ।\\ \as{ಅಂ ಆಂ ಇಂ ಈಂ++++ಳಂ ಕ್ಷಂ ॥\\ ಕ್ಷಂ ಳಂ ++++ಈಂ ಇಂ ಆಂ ಅಂ ॥} ನಮಃ ॥\\
\as{೪ ಆಪ್ಯಾಯಸ್ವ +++++ಸಂಗಥೇ ॥} ಸಾಂಸೀಂಸೂಂಸೈಂಸೌಂಸಃ ಸಮಲವರಯೂಂ ಸೋಮಮಂಡಲಾಯ ನಮಃ । ಇತಿ ಸೋಮಮಂಡಲಂ ವಿಭಾವ್ಯ ,ತತ್ರ ಷೋಡಶ ಸೋಮಕಾಲಾಃ ಪೂಜಯೇತ್ ।

\as{ಓಂಐಂಹ್ರೀಂಶ್ರೀಂ ಅಂ} ಅಮೃತಾಯೈ ನಮಃ~। \as{೪ ಆಂ} ಮಾನದಾಯೈ~। \as{೪ ಇಂ} ಪೂಷಾಯೈ~। \as{೪ ಈಂ} ತುಷ್ಟ್ಯೈ~। \as{೪ ಉಂ} ಪುಷ್ಟ್ಯೈ~। \as{೪ ಊಂ} ರತ್ಯೈ~। \as{೪ ಋಂ} ಧೃತ್ಯೈ~। \as{೪ ೠಂ} ಶಶಿನ್ಯೈ~। \as{೪ ಲೃಂ} ಚಂದ್ರಿಕಾಯೈ~। \as{೪ ಲೄಂ} ಕಾಂತ್ಯೈ~। \as{೪ ಏಂ} ಜ್ಯೋತ್ಸ್ನಾಯೈ~। \as{೪ ಐಂ} ಶ್ರಿಯೈ~। \as{೪ ಓಂ} ಪ್ರೀತ್ಯೈ~। \as{೪ ಔಂ} ಅಂಗದಾಯೈ~। \as{೪ ಅಂ} ಪೂರ್ಣಾಯೈ~। \as{೪ ಅಃ} ಪೂರ್ಣಾಮೃತಾಯೈ ನಮಃ~॥

\as{೪ ಓಂ ಜುಂಸಃ ಸ್ವಾಹಾ ॥} ಇತ್ಯಷ್ಟವಾರಮಭಿಮಂತ್ರ್ಯ ಅರ್ಘ್ಯಾಮೃತೇ ಅಕಾರಾದಾರಭ್ಯ ಕಕಾರಪರ್ಯಂತಂ ಕಕಾರಾದಾರಭ್ಯ ಥಕಾರಪರ್ಯಂತಂ ಥಕಾರಾತ್ ಸಕಾರಪರ್ಯಂತಂ ಚ ವರ್ಣೈರ್ಯುಕ್ತಂ ತ್ರಿಕೋಣಂ ವಿಲಿಖ್ಯ ಸ್ವಾಗ್ರಾದಿಕೋನೇಷು ಹಳಕ್ಷ ಇತಿ ವರ್ಣತ್ರಯಂ ವಿಲಿಖ್ಯ ತ್ರಿಕೋಣಾದ್ಬಹಿಃ ಪ್ರಾದಕ್ಷಿಣ್ಯೇನ ಖಂಡತ್ರಯಂ ವಿಲಿಖ್ಯ ಬಿಂದೌ ಈಂ ಇತಿ, ತತ್ಪಾರ್ಶ್ವಯೋಃ ಹಂಸಃ ಇತಿ ವಿಲಿಖೇತ್ ।

\as{೪ ಹಂಸಃ ನಮಃ ।} ಇತಿ  ಪ್ರಪೂಜ್ಯ ತ್ರಿಕೋನಂ ಪರಿತಃ ವೃತ್ತಂ ತದ್ಬಹಿಃ ಷಟ್ಕೋಣಂ ಚ ವಿರಚ್ಯ ಸ್ವಾಗ್ರಾದಿಕೋನೇಷು ಷಡಂಗೈಃ ಪೂಜಯೇತ್ । 

\as{೪ ಐಂ೫ } ಹೃದಯಾಯ ನಮಃ । ಹೃದಯಶಕ್ತಿ ಶ್ರೀಪಾದುಕಾಂ ಪೂ । ನಮಃ ॥\\
\as{೪ ಕ್ಲೀಂ೬ } ಶಿರಸೇ ಸ್ವಾಹಾ । ಶಿರಃಶಕ್ತಿ ಶ್ರೀಪಾದುಕಾಂ ಪೂ । ನಮಃ ॥\\
\as{೪ ಸೌಃ೪ } ಶಿಖಾಯೈ ವಷಟ್ । ಶಿಖಾಶಕ್ತಿ ಶ್ರೀಪಾದುಕಾಂ ಪೂ । ನಮಃ ॥\\
\as{೪ ಐಂ೫ } ಕವಚಾಯ ಹುಂ । ಕವಚಶಕ್ತಿ ಶ್ರೀಪಾದುಕಾಂ ಪೂ । ನಮಃ ॥\\
\as{೪ ಕ್ಲೀಂ೬} ನೇತ್ರತ್ರಯಾಯ ವೌಷಟ್ । ನೇತ್ರಶಕ್ತಿ ಶ್ರೀಪಾದುಕಾಂ ಪೂ । ನಮಃ ॥\\
\as{೪ ಸೌಃ೪ } ಅಸ್ತ್ರಾಯ ಫಟ್ । ಅಸ್ತ್ರಶಕ್ತಿ ಶ್ರೀಪಾದುಕಾಂ ಪೂ । ನಮಃ ॥

ಮೂಲೇನ ಸಕೃದಭಿಮಂತ್ರ್ಯ\\
ಆಂ ಸೋಹಂ~। ಆಂ ಹ್ರೀಂ ಕ್ರೋಂ ಯರಲವಶಷಸಹೋಂ ॥\\
ಸುಧಾದೇವ್ಯಾಃ ಪ್ರಾಣಾ ಇಹ ಪ್ರಾಣಾಃ~।
ಸುಧಾದೇವ್ಯಾ ಜೀವ ಇಹ ಸ್ಥಿತಃ~।\\
ಸುಧಾದೇವ್ಯಾಃ ಸರ್ವೇಂದ್ರಿಯಾಣಿ~।
ಸುಧಾದೇವ್ಯಾ ವಾಙ್ಮನಸ್ತ್ವಕ್ಚಕ್ಷುಃ ಶ್ರೋತ್ರ ಜಿಹ್ವಾಘ್ರಾಣಪ್ರಾಣಾ
ಇಹೈವಾಗತ್ಯ ಸುಖಂ ಚಿರಂ ತಿಷ್ಠಂತು ಸ್ವಾಹಾ ॥\\
\as{೪ ೧೫} ತಾಂ ಚಿನ್ಮಯೀಂ ಆನಂದಲಕ್ಷಣಾಂ ಅಮೃತಕಲಶ ಪಿಶಿತಹಸ್ತದ್ವಯಾಂ ಪ್ರಸನ್ನಾಂ ದೇವೀಂ ಪೂಜಯಾಮಿ ಸ್ವಾಹಾ ।\\
\as{ಶಂನೋ ದೇವೀರಭಿ+++++++ವಂತು ನಃ ॥}\\
ಗಂಧಾದಿಭಿಃ ಸಮಂತ್ರಮಭ್ಯರ್ಚ್ಯ,\\
\as{೪ ವಷಟ್} ಇತಿ ಕಿಂಚಿದುದ್ಧೃತ್ಯ,\\
\as{೪ ಸ್ವಾಹಾ} ಇತಿ ತತ್ರೈವ ವಿಸೃಜ್ಯ,\\
\as{೪ ಹುಂ} ಇತಿ ಅವಗುಂಠ್ಯ,\\
\as{೪ ವೌಷಟ್} ಇತಿ ಧೇನುಮುದ್ರಾಂ ಪ್ರದರ್ಶ್ಯ,\\
\as{೪ ಫಟ್} ಇತಿ ಸಂರಕ್ಷ್ಯ,\\
\as{೪ ನಮಃ} ಇತಿ ಪುಷ್ಪಂ ವಿಕೀರ್ಯ,\\
\as{೪ ೧೫} ಇತಿ ಗಾಲಿನ್ಯಾ ನಿರೀಕ್ಷ್ಯ\\
\as{೪ ಐಂ} ಇತಿ ಯೋನ್ಯಾ ಪ್ರಣಮ್ಯ,\\
ಮೂಲೇನ ಸಪ್ತವಾರಮಭಿಮಂತ್ರ್ಯ, ಪಾತ್ರಂ ಸ್ಪೃಷ್ಟ್ವಾ

\subsection{ವಹ್ನಿಕಲಾಃ}
\as{ಓಂಐಂಹ್ರೀಂಶ್ರೀಂ ಯಂ} ಧೂಮ್ರಾರ್ಚಿಷೇ   ನಮಃ~। \as{೪  ರಂ} ಊಷ್ಮಾಯೈ~। \as{೪  ಲಂ} ಜ್ವಲಿನ್ಯೈ~। \as{೪  ವಂ} ಜ್ವಾಲಿನ್ಯೈ~। \as{೪  ಶಂ} ವಿಸ್ಫಲಿಂಗಿನ್ಯೈ ~। \as{೪  ಷಂ} ಸುಶ್ರಿಯೈ ~। \as{೪  ಸಂ} ಸುರೂಪಾಯೈ ~। \as{೪  ಹಂ} ಕಪಿಲಾಯೈ ~। \as{೪  ಳಂ} ಹವ್ಯವಹಾಯೈ ~। \as{೪  ಕ್ಷಂ} ಕವ್ಯವಹಾಯೈ ನಮಃ~॥
\subsection{ಸೂರ್ಯಕಲಾಃ}\as{ಓಂಐಂಹ್ರೀಂಶ್ರೀಂ ಕಂಭಂ} ತಪಿನ್ಯೈ ನಮಃ~। \as{೪ ಖಂಬಂ} ತಾಪಿನ್ಯೈ~। \as{೪ ಗಂಫಂ} ಧೂಮ್ರಾಯೈ~। \as{೪ ಘಂಪಂ} ಮರೀಚ್ಯೈ~। \as{೪ ಙಂನಂ} ಜ್ವಾಲಿನ್ಯೈ~। \as{೪ ಚಂಧಂ} ರುಚ್ಯೈ~। \as{೪ ಛಂದಂ} ಸುಷುಮ್ನಾಯೈ~। \as{೪ ಜಂಥಂ} ಭೋಗದಾಯೈ~। \as{೪ ಝಂತಂ} ವಿಶ್ವಾಯೈ~। \as{೪ ಞಂಣಂ} ಬೋಧಿನ್ಯೈ~। \as{೪ ಟಂಢಂ} ಧಾರಿಣ್ಯೈ~। \as{೪ ಠಂಡಂ} ಕ್ಷಮಾಯೈ ನಮಃ~॥
\subsection{ಸೋಮಕಲಾಃ}\as{ಓಂಐಂಹ್ರೀಂಶ್ರೀಂ ಅಂ} ಅಮೃತಾಯೈ ನಮಃ~। \as{೪ ಆಂ} ಮಾನದಾಯೈ~। \as{೪ ಇಂ} ಪೂಷಾಯೈ~। \as{೪ ಈಂ} ತುಷ್ಟ್ಯೈ~। \as{೪ ಉಂ} ಪುಷ್ಟ್ಯೈ~। \as{೪ ಊಂ} ರತ್ಯೈ~। \as{೪ ಋಂ} ಧೃತ್ಯೈ~। \as{೪ ೠಂ} ಶಶಿನ್ಯೈ~। \as{೪ ಲೃಂ} ಚಂದ್ರಿಕಾಯೈ~। \as{೪ ಲೄಂ} ಕಾಂತ್ಯೈ~। \as{೪ ಏಂ} ಜ್ಯೋತ್ಸ್ನಾಯೈ~। \as{೪ ಐಂ} ಶ್ರಿಯೈ~। \as{೪ ಓಂ} ಪ್ರೀತ್ಯೈ~। \as{೪ ಔಂ} ಅಂಗದಾಯೈ~। \as{೪ ಅಂ} ಪೂರ್ಣಾಯೈ~। \as{೪ ಅಃ} ಪೂರ್ಣಾಮೃತಾಯೈ ನಮಃ~॥
\newpage
\subsection{ಬ್ರಹ್ಮಕಲಾಃ}\as{ಓಂಐಂಹ್ರೀಂಶ್ರೀಂ ಕಂ} ಸೃಷ್ಟ್ಯೈ ನಮಃ~। \as{೪ ಖಂ} ಋದ್ಧ್ಯೈ~। \as{೪ ಗಂ} ಸ್ಮೃತ್ಯೈ~। \as{೪ ಘಂ} ಮೇಧಾಯೈ~। \as{೪ ಙಂ} ಕಾಂತ್ಯೈ~। \as{೪ ಚಂ} ಲಕ್ಷ್ಮ್ಯೈ~। \as{೪ ಛಂ} ದ್ಯುತ್ಯೈ~। \as{೪ ಜಂ} ಸ್ಥಿರಾಯೈ~। \as{೪ ಝಂ} ಸ್ಥಿತ್ಯೈ~। \as{೪ ಞಂ} ಸಿದ್ಧ್ಯೈ~॥
\subsection{ವಿಷ್ಣುಕಲಾಃ}\as{ಓಂಐಂಹ್ರೀಂಶ್ರೀಂ ಟಂ} ಜರಾಯೈ ನಮಃ~। \as{೪ ಠಂ} ಪಾಲಿನ್ಯೈ~। \as{೪ ಡಂ} ಶಾಂತ್ಯೈ~। \as{೪ ಢಂ} ಈಶ್ವರ್ಯೈ~। \as{೪ ಣಂ} ರತ್ಯೈ~। \as{೪ ತಂ} ಕಾಮಿಕಾಯೈ~। \as{೪ ಥಂ} ವರದಾಯೈ~। \as{೪ ದಂ} ಆಹ್ಲಾದಿನ್ಯೈ~। \as{೪ ಧಂ} ಪ್ರೀತ್ಯೈ~। \as{೪ ನಂ} ದೀರ್ಘಾಯೈ~॥
\subsection{ರುದ್ರಕಲಾಃ}\as{ಓಂಐಂಹ್ರೀಂಶ್ರೀಂ ಪಂ} ತೀಕ್ಷ್ಣಾಯೈ ನಮಃ~। \as{೪ ಫಂ} ರೌದ್ರ್ಯೈ~। \as{೪ ಬಂ} ಭಯಾಯೈ~। \as{೪ ಭಂ} ನಿದ್ರಾಯೈ~। \as{೪ ಮಂ} ತಂದ್ರ್ಯೈ~। \as{೪ ಯಂ} ಕ್ಷುಧಾಯೈ~। \as{೪ ರಂ} ಕ್ರೋಧಿನ್ಯೈ~। \as{೪ ಲಂ} ಕ್ರಿಯಾಯೈ~। \as{೪ ವಂ} ಉದ್ಗಾರ್ಯೈ~। \as{೪ ಶಂ} ಮೃತ್ಯವೇ~॥
\subsection{ಈಶ್ವರಕಲಾಃ}\as{ಓಂಐಂಹ್ರೀಂಶ್ರೀಂ ಷಂ} ಪೀತಾಯೈ~। \as{೪ ಸಂ} ಶ್ವೇತಾಯೈ~। \as{೪ ಹಂ} ಅರುಣಾಯೈ~। \as{೪ ಕ್ಷಂ} ಅಸಿತಾಯೈ~॥
\subsection{ಸದಾಶಿವಕಲಾಃ}\as{ಓಂಐಂಹ್ರೀಂಶ್ರೀಂ ಅಂ} ನಿವೃತ್ತ್ಯೈ ನಮಃ~। \as{೪ ಆಂ} ಪ್ರತಿಷ್ಠಾಯೈ~। \as{೪ ಇಂ} ವಿದ್ಯಾಯೈ~। \as{೪ ಈಂ} ಶಾಂತ್ಯೈ~। \as{೪ ಉಂ} ಇಂಧಿಕಾಯೈ~। \as{೪ ಊಂ} ದೀಪಿಕಾಯೈ~। \as{೪ ಋಂ} ರೇಚಿಕಾಯೈ~। \as{೪ ೠಂ} ಮೋಚಿಕಾಯೈ~। \as{೪ ಲೃಂ} ಪರಾಯೈ~। \as{೪ ಲೄಂ} ಸೂಕ್ಷ್ಮಾಯೈ~। \as{೪ ಏಂ} ಸೂಕ್ಷ್ಮಾಮೃತಾಯೈ~। \as{೪ ಐಂ} ಜ್ಞಾನಾಯೈ~। \as{೪ ಓಂ} ಜ್ಞಾನಾಮೃತಾಯೈ~। \as{೪ ಔಂ} ಆಪ್ಯಾಯನ್ಯೈ~। \as{೪ ಅಂ} ವ್ಯಾಪಿನ್ಯೈ~। \as{೪ ಅಃ} ವ್ಯೋಮರೂಪಾಯೈ~॥

\as{೪ ಹಂ॒ಸಃ ಶು॑ಚಿ॒ಷದ್ವಸು॑ರಂತರಿಕ್ಷ॒ ಸದ್ಧೋತಾ᳚ವೇ ದಿ॒ಷದತಿ॑ಥಿರ್ದುರೋಣ॒ಸತ್ ।\\ನೃ॒ಷದ್ವ॑ರ॒ಸದೃ॑ತ॒ಸದ್ವ್ಯೋ᳚ಮ॒ ಸದ॒ಬ್ಜಾ ಗೋ॒ಜಾ ಋ॑ತ॒ಜಾ ಅ॑ದ್ರಿ॒ಜಾ ಋ॒ತಮ್ ॥} ನಮಃ ॥

\as{೪ ಪ್ರತದ್ವಿಷ್ಣುಃ॑ ಸ್ತವತೇ ವೀ॒ರ್ಯೇ᳚ಣ ಮೃ॒ಗೋ ನ ಭೀ॒ಮಃ ಕು॑ಚ॒ರೋ ಗಿ॑ರಿ॒ಷ್ಠಾಃ ।\\ಯಸ್ಯೋ॒ರುಷು॑ ತ್ರಿ॒ಷು ವಿ॒ಕ್ರಮ॑ಣೇಷ್ವಧಿಕ್ಷಿ॒ಯಂತಿ॒ ಭುವ॑ನಾನಿ॒ ವಿಶ್ವಾ᳚  ॥}ನಮಃ ॥

\as{೪ ತ್ರ್ಯಂ᳚ಬಕಂ ಯಜಾಮಹೇ ಸು॒ಗಂಧಿಂ᳚ ಪುಷ್ಟಿ॒ವರ್ಧ॑ನಂ ।\\ ಉ॒ರ್ವಾ॒ರು॒ಕಮಿ॑ವ॒ ಬಂಧ॑ನಾನ್ಮೃ॒ತ್ಯೋರ್ಮು॑ಕ್ಷೀಯ॒ ಮಾಮೃತಾ᳚ತ್ ॥}ನಮಃ ॥

\as{೪ ತದ್ವಿಷ್ಣೋಃ᳚ ಪರ॒ಮಂ ಪ॒ದಂ ಸದಾ᳚ ಪಶ್ಯಂತಿ ಸೂ॒ರಯಃ॑ ।\\ದಿ॒ವೀ᳚ವ॒ ಚಕ್ಷು॒ರಾತ॑ತಂ ॥\\ತದ್ವಿಪ್ರಾ᳚ಸೋ ವಿಪ॒ನ್ಯವೋ᳚ ಜಾಗೃ॒ವಾಂಸಃ॒ ಸಮಿಂ᳚ಧತೇ ।\\ ವಿಷ್ಣೋ॒ರ್ಯತ್ಪ॑ರ॒ಮಂ ಪ॒ದಂ॥}ನಮಃ ॥

\as{೪ ವಿಷ್ಣು॒ರ್ಯೋನಿಂ᳚ ಕಲ್ಪಯತು॒ ತ್ವಷ್ಟಾ᳚ ರೂ॒ಪಾಣಿ॑ ಪಿಂಶತು ।\\ ಆಸಿಂ᳚ಚತು ಪ್ರ॒ಜಾಪ॑ತಿರ್ಧಾ॒ತಾ ಗರ್ಭಂ᳚ ದಧಾತು ತೇ ॥\\ಗರ್ಭಂ᳚ ಧೇಹಿ ಸಿನೀವಾಲಿ॒ ಗರ್ಭಂ᳚ ಧೇಹಿ ಸರಸ್ವತಿ ।\\ ಗರ್ಭಂ᳚ ತೇ ಅ॒ಶ್ವಿನೌ᳚ ದೇ॒ವಾವಾ ಧ॑ತ್ತಾಂ॒ ಪುಷ್ಕ॑ರಸ್ರಜಾ ॥}ನಮಃ ॥

\as{೪ ಅಖಂಡೈಕರಸಾನಂದಕರೇ ಪರಸುಧಾತ್ಮನಿ ।\\ ಸ್ವಚ್ಛಂದಸ್ಫುರಣಾಮತ್ರ ನಿಧೇಹಿ ಕುಲನಾಯಿಕೇ ॥}ನಮಃ ॥

\as{೪ ಅಕುಲಸ್ಥಾಮೃತಾಕಾರೇ ಶುದ್ಧಜ್ಞಾನ ಕಲೇಬರೇ ।\\ ಅಮೃತತ್ವಂ ನಿಧೇಹ್ಯಸ್ಮಿನ್ ವಸ್ತುನಿ ಕ್ಲಿನ್ನರೂಪಿಣಿ ॥}ನಮಃ ॥

\as{೪ ತ್ವದ್ರೂಪಿಣ್ಯೈಕರಸ್ಯತ್ವಂ ಕೃತ್ವಾ ಹ್ಯೇತತ್ ಸ್ವರೂಪಿಣಿ ।\\ ಭೂತ್ವಾ ಪರಾಮೃತಾಕಾರೇ ಮಯಿ ಚಿತ್ಸ್ಫುರಣಂ ಕುರು ॥}ನಮಃ ॥

\as{೪ ಐಂ ಬ್ಲೂಂ ಝ್ರೌಂ ಜುಂ ಸಃ ಅಮೃತೇ ಅಮೃತೋದ್ಭವೇ ಅಮೃತೇಶ್ವರಿ\\ ಅಮೃತವರ್ಷಿಣಿ ಅಮೃತಂ ಸ್ರಾವಯ ಸ್ರಾವಯ ಸ್ವಾಹಾ ॥} ನಮಃ ॥

\as{೪ಐಂ ವದ ವದ ವಾಗ್ವಾದಿನಿ ಐಂ ಕ್ಲೀಂ ಕ್ಲಿನ್ನೇ ಕ್ಲೇದಿನಿ ಕ್ಲೇದಯ ಮಹಾಕ್ಷೋಭಂ ಕುರು ಕುರು ಕ್ಲೀಂ ಸೌಃ ಮೋಕ್ಷಂ ಕುರು ಕುರು ಹ್ಸೌಂ ಸೌಃ ॥} ನಮಃ ॥

ಏವಮಭಿಮಂತ್ರ್ಯ  ವಿಶೇಷಾರ್ಘ್ಯಬಿಂದುಭಿಃ ಗುರುತ್ರಯಂ ಸ್ವಶಿರಸಿ ತರ್ಪಯೇತ್ ॥\\
\dhyana{ಓಂಐಂಹ್ರೀಂಶ್ರೀಂಐಂಕ್ಲೀಂಸೌಃ ಹಂಸಃಶಿವಃ ಸೋಽಹಂಹಂಸಃ ಹ್‌ಸ್‌ಖ್‌ಫ್ರೇಂ ಹಸಕ್ಷಮಲವರಯೂಂ ಹ್‌ಸೌಃ ಸಹಕ್ಷಮಲವರಯೀಂ ಸ್‌ಹೌಃ ಹಂಸಃ ಶಿವಃ ಸೋಽಹಂ ಹಂಸಃ॥} ಸ್ವಾತ್ಮಾರಾಮ ಪರಮಾನಂದ ಪಂಜರ ವಿಲೀನ ತೇಜಸೇ ಪರಮೇಷ್ಠಿಗುರವೇ ನಮಃ~।ಶ್ರೀಪಾದುಕಾಂ ಪೂ । ತ । ನಮಃ ॥\\
\dhyana{೭ ಸೋಽಹಂ ಹಂಸಃ ಶಿವಃ ಹ್‌ಸ್‌ಖ್‌ಫ್ರೇಂ ಹಸಕ್ಷಮಲವರಯೂಂ ಹ್‌ಸೌಃ ಸಹಕ್ಷಮಲವರಯೀಂ ಸ್‌ಹೌಃ ಸೋಽಹಂ ಹಂಸಃ ಶಿವಃ॥} ಸ್ವಚ್ಛಪ್ರಕಾಶ ವಿಮರ್ಶಹೇತವೇ ಪರಮಗುರವೇ ನಮಃ।ಶ್ರೀಪಾದುಕಾಂ ಪೂ । ತ । ನಮಃ ॥\\
\dhyana{೭ ಹಂಸಃ ಶಿವಃ ಸೋಽಹಂ ಹ್‌ಸ್‌ಖ್‌ಫ್ರೇಂ ಹಸಕ್ಷಮಲವರಯೂಂ ಹ್‌ಸೌಃ ಸಹಕ್ಷಮಲವರಯೀಂ ಸ್‌ಹೌಃ ಹಂಸಃ ಶಿವಃ ಸೋಽಹಂ~॥} ಸ್ವರೂಪ ನಿರೂಪಣ ಹೇತವೇ ಶ್ರೀಗುರವೇ ನಮಃ~।ಶ್ರೀಪಾದುಕಾಂ ಪೂ । ತ । ನಮಃ ॥

ಪುನಃ ವಿಶೇಷಾರ್ಘ್ಯಂ ಗೃಹೀತ್ವಾ  ಸ್ವಶಿರಸಿ\\
\as{೪} ಕುಂಡಲಿನ್ಯಧಿಷ್ಠಿತ ಚಿದಗ್ನಿಮಂಡಲಾಯ ನಮಃ ॥ ಇತಿ ಮನಸಾ ಸಂಪೂಜ್ಯ 

\as{೪ ೧೫} ಪುಣ್ಯಂ ಜುಹೋಮಿ ಸ್ವಾಹಾ ॥\\
\as{೪ ೧೫} ಪಾಪಂ ಜುಹೋಮಿ ಸ್ವಾಹಾ ॥\\
\as{೪ ೧೫} ಕೃತ್ಯಂ ಜುಹೋಮಿ ಸ್ವಾಹಾ ॥\\
\as{೪ ೧೫} ಅಕೃತ್ಯಂ ಜುಹೋಮಿ ಸ್ವಾಹಾ ॥\\
\as{೪ ೧೫} ಸಂಕಲ್ಪಂ ಜುಹೋಮಿ ಸ್ವಾಹಾ ॥\\
\as{೪ ೧೫} ವಿಕಲ್ಪಂ ಜುಹೋಮಿ ಸ್ವಾಹಾ ॥\\
\as{೪ ೧೫} ಧರ್ಮಂ ಜುಹೋಮಿ ಸ್ವಾಹಾ ॥\\
\as{೪ ೧೫} ಅಧರ್ಮಂ ಜುಹೋಮಿ ಸ್ವಾಹಾ ॥\\
\as{೪ ೧೫} ಅಧರ್ಮಂ ಜುಹೋಮಿ ವೌಷಟ್ ॥

\as{೪} ಇತಃ ಪೂರ್ವಂ ಪ್ರಾಣಬುದ್ಧಿ ದೇಹಧರ್ಮಾಧಿಕಾರತೋ ಜಾಗ್ರತ್ ಸ್ವಪ್ನ ಸುಷುಪ್ತ್ಯವಸ್ಥಾಸು ಮನಸಾ ವಾಚಾ ಕರ್ಮಣಾ ಹಸ್ತಾಭ್ಯಾಂ ಪದ್ಭ್ಯಾಮುದರೇಣ ಶಿಶ್ನಾ ಯತ್ ಸ್ಮೃತಂ ಯದುಕ್ತಂ ಯತ್ಕೃತಂ ತತ್ಸರ್ವಂ ಬ್ರಹ್ಮಾರ್ಪಣಂ ಭವತು ಸ್ವಾಹಾ ॥

\as{೪ ಆರ್ದ್ರಂ॒ ಜ್ವಲ॑ತಿ॒ ಜ್ಯೋತಿರ॒ಹಮ॑ಸ್ಮಿ । ಜ್ಯೋತಿ॒ರ್ಜ್ವಲ॑ತಿ॒ ಬ್ರಹ್ಮಾ॒ಹಮ॑ಸ್ಮಿ । ಯೋ॑ಽಹ॒ಮಸ್ಮಿ ॒ಬ್ರಹ್ಮಾ॒ಹಮ॑ಸ್ಮಿ । ಅ॒ಹಮ॑ಸ್ಮಿ ॒ಬ್ರಹ್ಮಾ॒ಹಮ॑ಸ್ಮಿ । ಅ॒ಹಮೇ॒ವಾಹಂ ಮಾಂ ಜು॑ಹೋಮಿ॒ ಸ್ವಾಹಾ᳚ ॥} ಇತಿ ಕುಂಡಲಿನೀ ರೂಪೇ ಚಿದಗ್ನೌ ಮನಸಾ ಹೋಮಂ ವಿಧಾಯ ಕಲಶೇ ಸಾಮಾನ್ಯಾರ್ಘ್ಯಪಾತ್ರೇ ಚ ವಿಶೇಷಾರ್ಘ್ಯಂ ಕಿಂಚಿನ್ನಿಕ್ಷಿಪೇತ್ ।
‌\chapter*{\center ಶ್ರೀಚಕ್ರಾರ್ಚನಂ}
\section{ಮಂಟಪಪೂಜಾ}
ಓಂ ಐಂಹ್ರೀಂಶ್ರೀಂ ಅಮೃತಾಂಭೋನಿಧಯೇ ನಮಃ\\
೪ ರತ್ನದ್ವೀಪಾಯ ನಮಃ\\
೪ ನಾನಾವೃಕ್ಷಮಹೋದ್ಯಾನಾಯ ನಮಃ\\
೪ ಕಲ್ಪವೃಕ್ಷವಾಟಿಕಾಯೈ ನಮಃ\\
೪ ಸಂತಾನವಾಟಿಕಾಯೈ ನಮಃ\\
೪ ಹರಿಚಂದನವಾಟಿಕಾಯೈ ನಮಃ\\
೪ ಮಂದಾರವಾಟಿಕಾಯೈ ನಮಃ\\
೪ ಪಾರಿಜಾತವಾಟಿಕಾಯೈ ನಮಃ\\
೪ ಕದಂಬವಾಟಿಕಾಯೈ ನಮಃ\\
೪ ಪುಷ್ಯರಾಗರತ್ನಪ್ರಾಕಾರಾಯ ನಮಃ\\
೪ ಪದ್ಮರಾಗರತ್ನಪ್ರಾಕಾರಾಯ ನಮಃ\\
೪ ಗೋಮೇಧಕರತ್ನಪ್ರಾಕಾರಾಯ ನಮಃ\\
೪ ವಜ್ರರತ್ನಪ್ರಾಕಾರಾಯ ನಮಃ\\
೪ ವೈಡೂರ್ಯರತ್ನಪ್ರಾಕಾರಾಯ ನಮಃ\\
೪ ಇಂದ್ರನೀಲರತ್ನಪ್ರಾಕಾರಾಯ ನಮಃ\\
೪ ಮುಕ್ತಾರತ್ನಪ್ರಾಕಾರಾಯ ನಮಃ\\
೪ ಮರಕತರತ್ನಪ್ರಾಕಾರಾಯ ನಮಃ\\
೪ ವಿದ್ರುಮರತ್ನಪ್ರಾಕಾರಾಯ ನಮಃ\\
೪ ಮಾಣಿಕ್ಯಮಂಡಪಾಯ ನಮಃ\\
೪ ಸಹಸ್ರಸ್ತಂಭಮಂಡಪಾಯ ನಮಃ\\
೪ ಅಮೃತವಾಪಿಕಾಯೈ ನಮಃ\\
೪ ಆನಂದವಾಪಿಕಾಯೈ ನಮಃ\\
೪ ವಿಮರ್ಶವಾಪಿಕಾಯೈ ನಮಃ\\
೪ ಬಾಲಾತಪೋದ್ಗಾರಕಕ್ಷಾಯ ನಮಃ\\
೪ ಚಂದ್ರಿಕೋದ್ಗಾರಕಕ್ಷಾಯ ನಮಃ\\
೪ ಮಹಾಶೃಂಗಾರಪರಿಘಾಯೈ ನಮಃ\\
೪ ಮಹಾಪದ್ಮಾಟವ್ಯೈ ನಮಃ\\
೪ ಚಿಂತಾಮಣಿಮಯಗೃಹರಾಜಾಯ ನಮಃ\\
೪ ಪೂರ್ವಾಮ್ನಾಯಮಯಪೂರ್ವದ್ವಾರಾಯ ನಮಃ\\
೪ ದಕ್ಷಿಣಾಮ್ನಾಯಮಯದಕ್ಷಿಣದ್ವಾರಾಯ ನಮಃ\\
೪ ಪಶ್ಚಿಮಾಮ್ನಾಯಮಯಪಶ್ಚಿಮದ್ವಾರಾಯ ನಮಃ\\
೪ ಉತ್ತರಾಮ್ನಾಯಮಯೋತ್ತರದ್ವಾರಾಯ ನಮಃ\\
೪ ರತ್ನಪ್ರದೀಪವಲಯಾಯ ನಮಃ\\
೪ ಮಣಿಮಯಮಹಾಸಿಂಹಾಸನಾಯ ನಮಃ\\
೪ ಬ್ರಹ್ಮಮಯೈಕಮಂಚಪಾದಾಯ ನಮಃ(ಆಗ್ನೇಯ್ಯಾಂ)\\
೪ ವಿಷ್ಣುಮಯೈಕಮಂಚಪಾದಾಯ ನಮಃ(ನೈರೃತ್ಯಾಂ)\\
೪ ರುದ್ರಮಯೈಕಮಂಚಪಾದಾಯ ನಮಃ(ವಾಯವ್ಯಾಂ)\\
೪ ಈಶ್ವರಮಯೈಕಮಂಚಪಾದಾಯ ನಮಃ(ಐಶಾನ್ಯಾಂ)\\
೪ ಸದಾಶಿವಮಯೈಕಮಂಚಫಲಕಾಯ ನಮಃ\\
೪ ಹಂಸತೂಲಿಕಾತಲ್ಪಾಯ ನಮಃ\\
೪ ಹಂಸತೂಲಿಕಾತಲ್ಪಮಹೋಪಧಾನಾಯ ನಮಃ\\
೪ ಕೌಸುಂಭಾಸ್ತರಣಾಯ ನಮಃ\\
೪ ಮಹಾವಿತಾನಕಾಯ ನಮಃ\\
೪ ಮಹಾಮಾಯಾಜವನಿಕಾಯೈ ನಮಃ

೪ ದೀಪದೇವಿ ಮಹಾದೇವಿ ಶುಭಂ ಭವತು ಮೇ ಸದಾ~।\\
ಯಾವತ್ ಪೂಜಾಸಮಾಪ್ತಿಃ ಸ್ಯಾತ್ ತಾವತ್ ಪ್ರಜ್ವಲ ಸುಸ್ಥಿರಾ ॥

\as{೪ ಐಂ ಕಏಈಲಹ್ರೀಂ ಕ್ಲೀಂ ಹಸಕಹಲಹ್ರೀಂ ಸೌಃ ಸಕಲಹ್ರೀಂ} ನಮಃ ॥(ಬಿಂದೌ)\\
\as{೪ ಐಂ ಕಏಈಲಹ್ರೀಂ} ನಮಃ~।(ಅಗ್ರಕೋಣೇ)\\
\as{೪ ಕ್ಲೀಂ ಹಸಕಹಲಹ್ರೀಂ} ನಮಃ~।(ಐಶಾನ್ಯಕೋಣೇ)\\
\as{೪ ಸೌಃ ಸಕಲಹ್ರೀಂ} ನಮಃ~।(ಆಗ್ನೇಯಕೋಣೇ)
\section{ಧ್ಯಾನಂ}
\as{ಓಂ ಐಂಹ್ರೀಂಶ್ರೀಂ ಹ್ರೀಂಶ್ರೀಂಸೌಃ ಲಲಿತಾಯಾ \\ಅಮೃತಚೈತನ್ಯಮೂರ್ತಿಂ ಕಲ್ಪಯಾಮಿ ನಮಃ ॥}\\
ಧ್ಯಾಯೇತ್ ನಿರಾಮಯಂ ವಸ್ತು ಜಗತ್ರಯ ವಿಮೋಹಿನೀಮ್ ।\\
ಅಶೇಷವ್ಯವಹಾರಾಣಾಂ  ಸ್ವಾಮಿನೀಂ ಸಂವಿದಂ ಪರಾಮ್ ॥

ಉದ್ಯತ್ಸೂರ್ಯಸಹಸ್ರಾಭಾಂ ದಾಡಿಮೀಕುಸುಮಪ್ರಭಾಮ್ ।\\
ಜಪಾಕುಸುಮಸಂಕಾಶಾಂ ಪದ್ಮರಾಗಮಣಿಪ್ರಭಾಮ್ ॥

ಸ್ಫುರತ್ಪದ್ಮನಿಭಾಂ ತಪ್ತಕಾಂಚನಾಭಾಂ ಸುರೇಶ್ವರೀಂ ।\\
ರಕ್ತೋತ್ಪಲದಲಾಕಾರ ಪಾದಪಲ್ಲವರಾಜಿತಾಂ ॥

ಅನರ್ಘ್ಯರತ್ನಖಚಿತಮಂಜೀರಚರಣದ್ವಯಾಮ್ ।\\
ಪಾದಾಂಗುಲೀಯಕಕ್ಷಿಪ್ತ ರತ್ನತೇಜೋವಿರಾಜಿತಾಂ ॥

ಕದಲೀಲಲಿತಸ್ತಂಭಸುಕುಮಾರೋರುಕೋಮಲಾಂ ।\\
ನಿತಂಬಬಿಂಬವಿಲಸದ್ ರಕ್ತವಸ್ತ್ರಪರಿಷ್ಕೃತಾಂ ॥

ಮೇಖಲಾಬದ್ಧಮಾಣಿಕ್ಯ ಕಿಂಕಿಣೀನಾದವಿಭ್ರಮಾಂ ।\\
ಅಲಕ್ಷ್ಯಮಧ್ಯಮಾಂ ನಿಮ್ನನಾಭಿಂ ಶಾತೋದರೀಂ ಪರಾಮ್ ॥

ರೋಮರಾಜಿಲತೋದ್ಭೂತಮಹಾಕುಚಫಲಾನ್ವಿತಾಂ ।\\
ಸುವೃತ್ತನಿಬಡೋತ್ತುಂಗಕುಚಮಂಡಲರಾಜಿತಾಂ ॥

ಅನರ್ಘ್ಯಮೌಕ್ತಿಕಸ್ಫಾರಹಾರಭಾರವಿರಾಜಿತಾಂ ।\\
ನವರತ್ನಪ್ರಭಾರಾಜದ್ ಗ್ರೈವೇಯಕವಿಭೂಷಣಾಂ ॥

ಶ್ರುತಿಭೂಷಾಮನೋರಮ್ಯ ಕಪೋಲಸ್ಥಲಮಂಜುಲಾಂ ।\\
ಉದ್ಯದಾದಿತ್ಯಸಂಕಾಶಾಂ ತಾಟಂಕಸುಮುಖಪ್ರಭಾಂ ॥

ಪೂರ್ಣಚಂದ್ರಮುಖೀಂ ಪದ್ಮವದನಾಂ ವರನಾಸಿಕಾಂ ।\\
ಸ್ಫುರನ್ಮದನಕೋದಂಡಸುಭ್ರುವಂ ಪದ್ಮಲೋಚನಾಂ ॥

ಲಲಾಟಪಟ್ಟಸಂರಾಜದ್ ರತ್ನಾಢ್ಯತಿಲಕಾಂಕಿತಾಂ ।\\
ಮುಕ್ತಾಮಾಣಿಕ್ಯಘಟಿತಮುಕುಟಸ್ಥಲಕಿಂಕಿಣೀಂ ॥

ಸ್ಫುರಚ್ಚಂದ್ರಕಲಾರಾಜನ್ಮುಕುಟಾಂ ಚ ತ್ರಿಲೋಚನಾಂ ।\\
ಪ್ರವಾಲವಲ್ಲೀ ವಿಲಸದ್ ಬಾಹುವಲ್ಲೀ ಚತುಷ್ಟಯಾಂ ॥

ಇಕ್ಷುಕೋದಂಡಪುಷ್ಪೇಷು ಪಾಶಾಂಕುಶಚತುರ್ಭುಜಾಂ ।\\
ಸರ್ವದೇವಮಯೀಮಂಬಾಂ ಸರ್ವಸೌಭಾಗ್ಯಸುಂದರೀಂ ॥

ಸರ್ವತೀರ್ಥಮಯೀಂ ದಿವ್ಯಾಂ ಸರ್ವಕಾಮಪ್ರಪೂರಿಣೀಂ ।\\
ಸರ್ವಮಂತ್ರಮಯೀಂ ವಿದ್ಯಾಂ ಸರ್ವಾಗಮವಿಶಾರದಾಂ ॥

ಸರ್ವಕ್ಷೇತ್ರಮಯೀಂ ದೇವೀಂ ಸರ್ವವಿದ್ಯಾಮಯೀಂ ಶಿವಾಂ ।\\
ಸರ್ವಯಾಗಮಯೀಂ ವಿದ್ಯಾಂ ಸರ್ವದೇವಸ್ವರೂಪಿಣೀಂ ॥

ಸರ್ವಶಾಸ್ತ್ರಮಯೀಂ ನಿತ್ಯಾಂ ಸರ್ವಾಗಮನಮಸ್ಕೃತಾಂ ।\\
ಸರ್ವಾಮ್ನಾಯಮಯೀಂ ದೇವೀಂ ಸರ್ವಾಯತನಸೇವಿತಾಮ್ ॥

ಸರ್ವಾನಂದಮಯೀಂ ಜ್ಞಾನಗಹ್ವರಾಂ ಸಂವಿದಂ ಪರಾಂ ।\\
ಏವಂ ಧ್ಯಾಯೇತ್ಪರಾಮಂಬಾಂ ಸಚ್ಚಿದಾನಂದರೂಪಿಣೀಂ ॥\\
\as{೪ ಹ್‌ಸ್‌ರೈಂ ಹ್‌ಸ್‌ಕ್ಲ್ರೀಂ ಹ್‌ಸ್‌ರ್ಸೌಃ}\\
ಜಗನ್ಮಾತರ್ಮಹಾದೇವಿ ಮಹಾತ್ರಿಪುರಸುಂದರಿ ।\\
ಸುಧಾಚೈತನ್ಯ ಮೂರ್ತಿಂ ತೇ ಕಲ್ಪಯಾಮಿ ನಮಶ್ಶಿವೇ ॥

ಮಹಾಪದ್ಮವನಾಂತಸ್ಥೇ ಕಾರಣಾನಂದವಿಗ್ರಹೇ~।\\
ಸರ್ವಭೂತಹಿತೇ ಮಾತಃ ಏಹ್ಯೇಹಿ ಪರಮೇಶ್ವರಿ ॥

ದೇವೇಶಿ ಭಕ್ತಸುಲಭೇ ಸರ್ವಾವರಣಸಂಯುತೇ~।\\
ಯಾವತ್ತ್ವಾಂ ಪೂಜಯಿಷ್ಯಾಮಿ ತಾವತ್ತ್ವಂ ಸುಸ್ಥಿರಾ ಭವ ॥

\as{ಬಿಂದುಪೀಠಗತ ನಿರ್ವಿಶೇಷಬ್ರಹ್ಮಾತ್ಮಕ ಶ್ರೀಮತ್ಕಾಮೇಶ್ವರಾಂಕೇ ಶ್ರೀಲಲಿತಾ ಮಹಾತ್ರಿಪುರಸುಂದರೀ ಪರಾಭಟ್ಟಾರಿಕಾಂ ಆವಾಹಯಾಮಿ ನಮಃ ॥}

ಸುಭಗೇ ನಮಃ~। ಆಂ ಸೋಹಂ~। ಆಂ ಹ್ರೀಂ ಕ್ರೋಂ ಯರಲವಶಷಸಹೋಂ ಶ್ರೀಲಲಿತಾಮಹಾತ್ರಿಪುರಸುಂದರ್ಯಾಃ ಪ್ರಾಣಾ ಇಹ ಪ್ರಾಣಾಃ~।\\ ಶ್ರೀಲಲಿತಾಮಹಾತ್ರಿಪುರಸುಂದರ್ಯಾಃ ಜೀವ ಇಹ ಸ್ಥಿತಃ~।\\ ಶ್ರೀಲಲಿತಾಮಹಾತ್ರಿಪುರಸುಂದರ್ಯಾಃ ಸರ್ವೇಂದ್ರಿಯಾಣಿ~।\\ಶ್ರೀಲಲಿತಾಮಹಾತ್ರಿಪುರಸುಂದರ್ಯಾಃ ವಾಙ್ಮನಸ್ತ್ವಕ್ಚಕ್ಷುಃ ಶ್ರೋತ್ರ\\ಜಿಹ್ವಾಘ್ರಾಣಪ್ರಾಣಾ ಇಹೈವಾಗತ್ಯ ಸುಖಂ ಚಿರಂ ತಿಷ್ಠಂತು ಸ್ವಾಹಾ ॥\\
\as{(೧೫)}ಆವಾಹಿತಾ ಭವ~।\\
\as{(೧೫)} ಸಂಸ್ಥಾಪಿತಾ ಭವ~।\\
\as{(೧೫)} ಸನ್ನಿಹಿತಾ ಭವ~।\\
\as{(೧೫)} ಸನ್ನಿರುದ್ಧಾ ಭವ~।\\
\as{(೧೫)} ಸಮ್ಮುಖಾ ಭವ~।\\
\as{(೧೫)} ಅವಗುಂಠಿತಾ ಭವ~।\\
\as{(೧೫)} ವ್ಯಾಪ್ತಾ ಭವ~।\\
\as{(೧೫)} ಸುಪ್ರಸನ್ನಾ ಭವ~।\\
\as{(೧೫)} ವರದಾ ಭವ ॥

ಓಂ ಐಂಹ್ರೀಂಶ್ರೀಂ ಶ್ರೀಮಲ್ಲಲಿತಾಮಹಾತ್ರಿಪುರಸುಂದರ್ಯೈ\\
ಪಾದ್ಯಂ ಕಲ್ಪಯಮಿ ನಮಃ\\
ಆಭರಣಾವರೋಪಣಂ ಕಲ್ಪಯಾಮಿ ನಮಃ~।\\
ಸುಗಂಧಿತೈಲಾಭ್ಯಂಗಂ ಕಲ್ಪಯಾಮಿ ನಮಃ~।\\
ಮಜ್ಜನಶಾಲಾಪ್ರವೇಶನಂ ಕಲ್ಪಯಾಮಿ ನಮಃ~।\\
ಮಜ್ಜನಶಾಲಾಸ್ಥ ಮಣಿಪೀಠೋಪವೇಶನಂ ಕಲ್ಪಯಾಮಿ ನಮಃ~।\\
ದಿವ್ಯಸ್ನಾನೀಯೋದ್ವರ್ತನಂ ಕಲ್ಪಯಾಮಿ ನಮಃ~।\\
ಉಷ್ಣೋದಕಸ್ನಾನಂ ಕಲ್ಪಯಾಮಿ ನಮಃ~।\\
ಕನಕ ಕಲಶಚ್ಯುತ ಸಕಲ ತೀರ್ಥಾಭಿಷೇಕಂ ಕಲ್ಪಯಾಮಿ ನಮಃ~॥\\
(ಇತಿ ಸೂಕ್ತಾಭಿಷೇಚನಂ ಕುರ್ಯಾತ್)\\
ಓಂ ಐಂ ಹ್ರೀಂ ಶ್ರೀಂ ಶ್ರೀ ಶ್ರೀಲಲಿತಾಮಹಾತ್ರಿಪುರಸುಂದರ್ಯೈ \\ಧೌತವಸ್ತ್ರಪರಿಮಾರ್ಜನಂ ಕಲ್ಪಯಾಮಿ ನಮಃ~।\\
ಅರುಣದುಕೂಲಪರಿಧಾನಂ ಕಲ್ಪಯಾಮಿ ನಮಃ~।\\
ಅರುಣಕುಚೋತ್ತರೀಯಂ ಕಲ್ಪಯಾಮಿ ನಮಃ~।\\
ಆಲೇಪಮಂಟಪ ಪ್ರವೇಶನಂ ಕಲ್ಪಯಾಮಿ ನಮಃ~।\\
ಆಲೇಪಮಂಟಪಸ್ಥ ಮಣಿಪೀಠೋಪವೇಶನಂ ಕಲ್ಪಯಾಮಿ ನಮಃ~।\\
ಚಂದನಾಗರು ಕುಂಕುಮ ಮೃಗಮದ ಕರ್ಪೂರ ಕಸ್ತೂರೀ ಗೋರೋಚನಾದಿ\\ ದಿವ್ಯಗಂಧ ಸರ್ವಾಂಗೀಣವಿಲೇಪನಂ ಕಲ್ಪಯಾಮಿ ನಮಃ~।\\
ಕೇಶಭರಸ್ಯ ಕಾಲಾಗರುಧೂಪಂ ಕಲ್ಪಯಾಮಿ ನಮಃ~।\\
ಮಲ್ಲಿಕಾ ಮಾಲತೀ ಜಾತೀ ಚಂಪಕ ಅಶೋಕ ಶತಪತ್ರ ಪೂಗ ಗುಹಳೀ ಪುನ್ನಾಗ \\ಕಹ್ಲಾರ ಮುಖ್ಯ ಸರ್ವರ್ತುಕುಸುಮ ಮಾಲಾಃ ಕಲ್ಪಯಾಮಿ ನಮಃ~।\\
ಭೂಷಣಮಂಟಪ ಪ್ರವೇಶನಂ ಕಲ್ಪಯಾಮಿ ನಮಃ~।\\
ಭೂಷಣಮಂಟಪಸ್ಥ ರತ್ನಪೀಠೋಪವೇಶನಂ ಕಲ್ಪಯಾಮಿ ನಮಃ~।\\
ನವಮಣಿಮಕುಟಂ ಕಲ್ಪಯಾಮಿ ನಮಃ~।\\
ಚಂದ್ರಶಕಲಂ ಕಲ್ಪಯಾಮಿ ನಮಃ~।\\
ಸೀಮಂತಸಿಂದೂರಂ ಕಲ್ಪಯಾಮಿ ನಮಃ~।\\
ತಿಲಕರತ್ನಂ ಕಲ್ಪಯಾಮಿ ನಮಃ~।\\
ಕಾಲಾಂಜನಂ ಕಲ್ಪಯಾಮಿ ನಮಃ~।\\
ಪಾಳೀಯುಗಲಂ ಕಲ್ಪಯಾಮಿ ನಮಃ~।\\
ಮಣಿಕುಂಡಲಯುಗಲಂ ಕಲ್ಪಯಾಮಿ ನಮಃ~।\\
ನಾಸಾಭರಣಂ ಕಲ್ಪಯಾಮಿ ನಮಃ~।\\
ಅಧರಯಾವಕಂ ಕಲ್ಪಯಾಮಿ ನಮಃ~।\\
ಆದ್ಯಭೂಷಣಂ ಕಲ್ಪಯಾಮಿ ನಮಃ~।\\
ಕನಕಚಿಂತಾಕಂ ಕಲ್ಪಯಾಮಿ ನಮಃ~।\\
ಪದಕಂ ಕಲ್ಪಯಾಮಿ ನಮಃ~।\\
ಮಹಾಪದಕಂ ಕಲ್ಪಯಾಮಿ ನಮಃ~।\\
ಮುಕ್ತಾವಲಿಂ ಕಲ್ಪಯಾಮಿ ನಮಃ~।\\
ಏಕಾವಲಿಂ ಕಲ್ಪಯಾಮಿ ನಮಃ~।\\
ಛನ್ನವೀರಂ ಕಲ್ಪಯಾಮಿ ನಮಃ~।\\
ಕೇಯೂರಯುಗಲ ಚತುಷ್ಟಯಂ ಕಲ್ಪಯಾಮಿ ನಮಃ~।\\
ವಲಯಾವಲಿಂ ಕಲ್ಪಯಾಮಿ ನಮಃ~।\\
ಊರ್ಮಿಕಾವಲಿಂ ಕಲ್ಪಯಾಮಿ ನಮಃ~।\\
ಕಾಂಚೀದಾಮ ಕಲ್ಪಯಾಮಿ ನಮಃ~।\\
ಕಟಿಸೂತ್ರಂ ಕಲ್ಪಯಾಮಿ ನಮಃ~।\\
ಸೌಭಾಗ್ಯಾಭರಣಂ ಕಲ್ಪಯಾಮಿ ನಮಃ~।\\
ಪಾದಕಟಕಂ ಕಲ್ಪಯಾಮಿ ನಮಃ~।\\
ರತ್ನನೂಪುರಂ ಕಲ್ಪಯಾಮಿ ನಮಃ~।\\
ಪಾದಾಂಗುಲೀಯಕಂ ಕಲ್ಪಯಾಮಿ ನಮಃ~।\\
ವಾಮೋರ್ಧ್ವಕರೇ ಪಾಶಂ ಕಲ್ಪಯಾಮಿ ನಮಃ~।\\
ದಕ್ಷಿಣೋರ್ಧ್ವಕರೇ ಅಂಕುಶಂ ಕಲ್ಪಯಾಮಿ ನಮಃ~।\\
ವಾಮಾಧಃಕರೇ ಪುಂಡ್ರೇಕ್ಷುಚಾಪಂ ಕಲ್ಪಯಾಮಿ ನಮಃ~।\\
ದಕ್ಷಿಣಾಧಃಕರೇ ಪುಷ್ಪಬಾಣಾನ್ ಕಲ್ಪಯಾಮಿ ನಮಃ~।\\
ಶ್ರೀಮನ್ಮಾಣಿಕ್ಯಪಾದುಕಾ ಯುಗಲಂ ಕಲ್ಪಯಾಮಿ ನಮಃ~।\\
ಸ್ವಸಮಾನ ವೇಷಾಭಿರಾವರಣ ದೇವತಾಭಿಃ ಸಹ\\ ಶ್ರೀ ಚಕ್ರಾಧಿರೋಹಣಂ ಕಲ್ಪಯಾಮಿ ನಮಃ~।\\
ಕಾಮೇಶ್ವರಾಂಕಪರ್ಯಂಕೋಪವೇಶನಂ ಕಲ್ಪಯಾಮಿ ನಮಃ~।\\
ಅಮೃತಾಸವ ಚಷಕಂ ಕಲ್ಪಯಾಮಿ ನಮಃ~।\\
ಆಚಮನೀಯಕಂ ಕಲ್ಪಯಾಮಿ ನಮಃ~।\\
ಕರ್ಪೂರವೀಟಿಕಾಂ ಕಲ್ಪಯಾಮಿ ನಮಃ~।\\
ಆನಂದೋಲ್ಲಾಸ ವಿಲಾಸಹಾಸಂ ಕಲ್ಪಯಾಮಿ ನಮಃ~।
\section{ಶಿವ ಅಂಗಪೂಜಾ ॥}
ಓಂ ಪಾಪನಾಶನಾಯ ನಮಃ~। ಪಾದೌ ಪೂಜಯಾಮಿ॥\\
ಓಂ ಗುರವೇ ನಮಃ~। ಗುಲ್ಫೌ ಪೂಜಯಾಮಿ॥\\
ಓಂ ಜ್ಞಾನಪ್ರದಾಯ ನಮಃ~। ಜಂಘೇ ಪೂಜಯಾಮಿ॥\\
ಓಂ ಜಾಹ್ನವೀಪತಯೇ ನಮಃ~। ಜಾನುನೀ ಪೂಜಯಾಮಿ॥\\
ಓಂ ಉತ್ತಮೋತ್ತಮಾಯ ನಮಃ~। ಊರೂ ಪೂಜಯಾಮಿ॥\\
ಓಂ ಕಂದರ್ಪನಾಶಾಯ ನಮಃ~। ಕಟಿಂ ಪೂಜಯಾಮಿ॥\\
ಓಂ ಗುಹೇಶ್ವರಾಯ ನಮಃ~। ಗುಹ್ಯಂ ಪೂಜಯಾಮಿ॥\\
ಓಂ ನಂದಿಸೇವ್ಯಾಯ ನಮಃ~। ನಾಭಿಂ ಪೂಜಯಾಮಿ॥\\
ಓಂ ಸ್ಕಂದಗುರವೇ ನಮಃ~। ಸ್ಕಂಧೌ ಪೂಜಯಾಮಿ॥\\
ಓಂ ಹಿರಣ್ಯಬಾಹವೇ ನಮಃ~। ಬಾಹೂನ್ ಪೂಜಯಾಮಿ॥\\
ಓಂ ಹರಾಯ ನಮಃ~। ಹಸ್ತಾನ್ ಪೂಜಯಾಮಿ॥\\
ಓಂ ನೀಲಕಂಠಾಯ ನಮಃ~। ಕಂಠಂ ಪೂಜಯಾಮಿ॥\\
ಓಂ ವೇದಮೂರ್ತಯೇ ನಮಃ~। ಮುಖಂ ಪೂಜಯಾಮಿ॥\\
ಓಂ ನಾಗಹಾರಾಯ ನಮಃ~। ನಾಸಿಕಾಂ ಪೂಜಯಾಮಿ॥\\
ಓಂ ತ್ರಿಣೇತ್ರಾಯ ನಮಃ~। ನೇತ್ರಾಣಿ ಪೂಜಯಾಮಿ॥\\
ಓಂ ಭಸಿತಾಭಾಸಾಯ ನಮಃ~। ಲಲಾಟಂ ಪೂಜಯಾಮಿ॥\\
ಓಂ ಇಂದುಮೌಲಯೇ ನಮಃ~। ಮೌಲಿಂ ಪೂಜಯಾಮಿ॥\\
ಓಂ ಶರ್ವಾಯ ನಮಃ~। ಶಿರಃ ಪೂಜಯಾಮಿ॥\\
ಓಂ ಸರ್ವಾತ್ಮನೇ ನಮಃ~। ಸರ್ವಾಂಗಂ ಪೂಜಯಾಮಿ॥
\section{ಪತ್ರ ಪೂಜಾ ॥}
ಓಂ ಉಮಾಪತಯೇ ನಮಃ~। ಬಿಲ್ವಪತ್ರಂ ಸಮರ್ಪಯಾಮಿ॥\\
ಓಂ ಜಗದ್ಗುರವೇ ನಮಃ~। ತುಲಸೀಪತ್ರಂ ಸಮರ್ಪಯಾಮಿ॥\\
ಓಂ ಆನಂದಾಯ ನಮಃ~। ಅರ್ಕಪತ್ರಂ ಸಮರ್ಪಯಾಮಿ॥\\
ಓಂ ಸರ್ವಬಂಧವಿಮೋಚನಾಯ ನಮಃ~। ಜಂಬೀರಪತ್ರಂ ಸಮರ್ಪಯಾಮಿ॥\\
ಓಂ ಲೋಕನಾಥಾಯ ನಮಃ~। ನಿರ್ಗುಂಡೀಪತ್ರಂ ಸಮರ್ಪಯಾಮಿ॥\\
ಓಂ ಜಗತ್ಕಾರಣಾಯ ನಮಃ~। ದೂರ್ವಾಪತ್ರಂ ಸಮರ್ಪಯಾಮಿ॥\\
ಓಂ ನಾಗಭೂಷಣಾಯ ನಮಃ~। ಕುಶಪತ್ರಂ ಸಮರ್ಪಯಾಮಿ॥\\
ಓಂ ಮೃಗಧರಾಯ ನಮಃ~। ಮರುಗಪತ್ರಂ ಸಮರ್ಪಯಾಮಿ॥\\
ಓಂ ಪಶುಪತಯೇ ನಮಃ~। ಕಾಮಕಸ್ತೂರಿಕಾಪತ್ರಂ ಸಮರ್ಪಯಾಮಿ॥\\
ಓಂ ಮುಕುಂದಪ್ರಿಯಾಯ ನಮಃ~। ಗಿರಿಕರ್ಣಿಕಾಪತ್ರಂ ಸಮರ್ಪಯಾಮಿ॥\\
ಓಂ ತ್ರ್ಯಂಬಕಾಯ ನಮಃ~। ಮಾಚೀಪತ್ರಂ ಸಮರ್ಪಯಾಮಿ॥\\
ಓಂ ಭಕ್ತಜನಪ್ರಿಯಾಯ ನಮಃ~। ಧಾತ್ರೀಪತ್ರಂ ಸಮರ್ಪಯಾಮಿ॥\\
ಓಂ ವರದಾಯ ನಮಃ~। ವಿಷ್ಣುಕ್ರಾಂತಿಪತ್ರಂ ಸಮರ್ಪಯಾಮಿ॥\\
ಓಂ ಶಿವಾಯ ನಮಃ~। ದ್ರೋಣಪತ್ರಂ ಸಮರ್ಪಯಾಮಿ॥\\
ಓಂ ಶಂಕರಾಯ ನಮಃ~। ಧತ್ತೂರಪತ್ರಂ ಸಮರ್ಪಯಾಮಿ॥\\
ಓಂ ಶಮಪ್ರಾಪ್ತಾಯ ನಮಃ~। ಶಮೀಪತ್ರಂ ಸಮರ್ಪಯಾಮಿ॥\\
ಓಂ ಸಾಂಬಶಿವಾಯ ನಮಃ~। ಸೇವಂತಿಕಾಪತ್ರಂ ಸಮರ್ಪಯಾಮಿ॥\\
ಓಂ ಚರ್ಮವಾಸಸೇ ನಮಃ~। ಚಂಪಕಪತ್ರಂ ಸಮರ್ಪಯಾಮಿ॥\\
ಓಂ ಬ್ರಾಹ್ಮಣಪ್ರಿಯಾಯ ನಮಃ~। ಕರವೀರಪತ್ರಂ ಸಮರ್ಪಯಾಮಿ॥\\
ಓಂ ಗಂಗಾಧರಾಯ ನಮಃ~। ಅಶೋಕಪತ್ರಂ ಸಮರ್ಪಯಾಮಿ॥\\
ಓಂ ಪುಣ್ಯಮೂರ್ತಯೇ ನಮಃ~। ಪುನ್ನಾಗಪತ್ರಂ ಸಮರ್ಪಯಾಮಿ॥\\
ಓಂ ಉಮಾಮಹೇಶ್ವರಾಯ ನಮಃ~। ಸರ್ವಾಣಿ ಪತ್ರಾಣಿ ಸಮರ್ಪಯಾಮಿ॥
\section{ಪುಷ್ಪಪೂಜಾ}
ಓಂ ರುದ್ರಾಯ ನಮಃ~। ದ್ರೋಣಪುಷ್ಪಂ ಸಮರ್ಪಯಾಮಿ॥\\
ಓಂ ಪಶುಪತಯೇ ನಮಃ~। ಧತ್ತೂರಪುಷ್ಪಂ ಸಮರ್ಪಯಾಮಿ॥\\
ಓಂ ಸ್ಥಾಣವೇ ನಮಃ~। ಬೃಹತೀಪುಷ್ಪಂ ಸಮರ್ಪಯಾಮಿ॥\\
ಓಂ ನೀಲಕಂಠಾಯ ನಮಃ~। ಅರ್ಕಪುಷ್ಪಂ ಸಮರ್ಪಯಾಮಿ॥\\
ಓಂ ಉಮಾಪತಯೇ ನಮಃ~। ಬಕುಲಪುಷ್ಪಂ ಸಮರ್ಪಯಾಮಿ॥\\
ಓಂ ಕಾಲಕಾಲಾಯ ನಮಃ~। ಜಾತೀಪುಷ್ಪಂ ಸಮರ್ಪಯಾಮಿ॥\\
ಓಂ ಕಾಲಮೂರ್ತಯೇ ನಮಃ~। ಕರವೀರಪುಷ್ಪಂ ಸಮರ್ಪಯಾಮಿ॥\\
ಓಂ ದೇವದೇವಾಯ ನಮಃ~। ಪಂಕಜಪುಷ್ಪಂ ಸಮರ್ಪಯಾಮಿ॥\\
ಓಂ ವಿಶ್ವಪ್ರಿಯಾಯ ನಮಃ~। ಪುನ್ನಾಗಪುಷ್ಪಂ ಸಮರ್ಪಯಾಮಿ॥\\
ಓಂ ವೃಷಭಧ್ವಜಾಯ ನಮಃ~। ವೈಜಯಂತಿಕಾಪುಷ್ಪಂ ಸಮರ್ಪಯಾಮಿ॥\\
ಓಂ ಸದಾಶಿವಾಯ ನಮಃ~। ಗಿರಿಕರ್ಣಿಕಾಪುಷ್ಪಂ ಸಮರ್ಪಯಾಮಿ॥\\
ಓಂ ಶೂಲಿನೇ ನಮಃ~। ಚಂಪಕಪುಷ್ಪಂ ಸಮರ್ಪಯಾಮಿ॥\\
ಓಂ ಸುರೇಶಾಯ ನಮಃ~। ಸೇವಂತಿಕಾಪುಷ್ಪಂ ಸಮರ್ಪಯಾಮಿ॥\\
ಓಂ ನಿರಹಂಕಾರಾಯ ನಮಃ~। ಮಲ್ಲಿಕಾಪುಷ್ಪಂ ಸಮರ್ಪಯಾಮಿ॥\\
ಓಂ ಸತ್ಯವ್ರತಾಯ ನಮಃ~। ಜಪಾಪುಷ್ಪಂ ಸಮರ್ಪಯಾಮಿ॥\\
ಓಂ ಉಮಾಮಹೇಶ್ವರಾಯ ನಮಃ~। ಸರ್ವಾಣಿ ಪುಷ್ಪಾಣಿ ಸಮರ್ಪಯಾಮಿ॥
\section{ಅಥ ಲಲಿತಾ ಅಂಗಪೂಜಾ}
೪ ಶ್ರೀಮಾತ್ರೇ ನಮಃ~। ಪಾದೌ ಪೂಜಯಾಮಿ~।\\
೪ ಭಾವನಾಯೈ ನಮಃ~। ಗುಲ್ಫೌ ಪೂಜಯಾಮಿ~।\\
೪ ಭಾವನಾಗಮ್ಯಾಯೈ ನಮಃ~। ಜಂಘೇ ಪೂಜಯಾಮಿ~।\\
೪ ಭವಾರಣ್ಯಕುಠಾರಿಕಾಯೈ ನಮಃ~। ಜಾನುನೀ ಪೂಜಯಾಮಿ~।\\
೪ ಭದ್ರಪ್ರಿಯಾಯೈ ನಮಃ~। ಊರೂ ಪೂಜಯಾಮಿ~।\\
೪ ಭದ್ರಮೂರ್ತ್ಯೈ ನಮಃ~। ಕಟಿಂ ಪೂಜಯಾಮಿ~।\\
೪ ಭಕ್ತಸೌಭಾಗ್ಯದಾಯಿನ್ಯೈ ನಮಃ~। ನಾಭಿಂ ಪೂಜಯಾಮಿ~।\\
೪ ಭಕ್ತಿಪ್ರಿಯಾಯೈ ನಮಃ~। ಉದರಂ ಪೂಜಯಾಮಿ~।\\
೪ ಭಕ್ತಿಗಮ್ಯಾಯೈ ನಮಃ~। ಸ್ತನೌ ಪೂಜಯಾಮಿ~।\\
೪ ಭಕ್ತಿವಶ್ಯಾಯೈ ನಮಃ~। ವಕ್ಷಸ್ಥಲಂ ಪೂಜಯಾಮಿ~।\\
೪ ಕಲ್ಯಾಣ್ಯೈ ನಮಃ~। ಬಾಹೂನ್ ಪೂಜಯಾಮಿ~।\\
೪ ಚಕ್ರಿಣ್ಯೈ ನಮಃ~। ಹಸ್ತಾನ್ ಪೂಜಯಾಮಿ~।\\
೪ ವಿಶ್ವಾತೀತಾಯೈ ನಮಃ~। ಕಂಠಂ ಪೂಜಯಾಮಿ~।\\
೪ ತ್ರಿಪುರಾಯೈ ನಮಃ~। ಜಿಹ್ವಾಂ ಪೂಜಯಾಮಿ~।\\
೪ ವಿರಾಗಿಣ್ಯೈ ನಮಃ~। ಮುಖಂ ಪೂಜಯಾಮಿ~।\\
೪ ಮಹೀಯಸ್ಯೈ ನಮಃ~। ನೇತ್ರೇ ಪೂಜಯಾಮಿ~।\\
೪ ಮನುವಿದ್ಯಾಯೈ ನಮಃ~। ಕರ್ಣೌ ಪೂಜಯಾಮಿ~।\\
೪ ಶಿವಾಯೈ ನಮಃ~। ಲಲಾಟಂ ಪೂಜಯಾಮಿ~।\\
೪ ಶಿವಶಕ್ತ್ಯೈ ನಮಃ~। ಶಿರಃ ಪೂಜಯಾಮಿ~।\\
೪ ಲಲಿತಾಂಬಿಕಾಯೈ ನಮಃ~। ಸರ್ವಾಂಗಂ ಪೂಜಯಾಮಿ~।
\section{ಅಥ ಪತ್ರಪೂಜಾ}
೪ ಕಲ್ಯಾಣ್ಯೈ ನಮಃ~। ಮಾಚೀಪತ್ರಂ ಸಮರ್ಪಯಾಮಿ~।\\
೪ ಕಮಲಾಕ್ಷ್ಯೈ ನಮಃ~। ಸೇವಂತಿಕಾಪತ್ರಂ ಸಮರ್ಪಯಾಮಿ~।\\
೪ ಏಕಾರರೂಪಾಯೈ ನಮಃ~। ಬಿಲ್ವಪತ್ರಂ ಸಮರ್ಪಯಾಮಿ~।\\
೪ ಏಕಭೋಗಾಯೈ ನಮಃ~। ತುಲಸೀಪತ್ರಂ ಸಮರ್ಪಯಾಮಿ~।\\
೪ ಏಕರಸಾಯೈ ನಮಃ~। ಕಸ್ತೂರಿಕಾಪತ್ರಂ ಸಮರ್ಪಯಾಮಿ~।\\
೪ ಈಶಿತ್ರ್ಯೈ ನಮಃ~। ಮರುವಕಪತ್ರಂ ಸಮರ್ಪಯಾಮಿ~।\\
೪ ಈಶಶಕ್ತ್ಯೈ ನಮಃ~। ಗಿರಿಕರ್ಣಿಕಾಪತ್ರಂ ಸಮರ್ಪಯಾಮಿ~।\\
೪ ಹ್ರೀಂಮತ್ಯೈ ನಮಃ~। ಕರವೀರಪತ್ರಂ ಸಮರ್ಪಯಾಮಿ~।\\
೪ ಕಾಮದಾಯೈ ನಮಃ~। ದಾಡಿಮೀಪತ್ರಂ ಸಮರ್ಪಯಾಮಿ~।\\
೪ ವಿಶ್ವತೋಮುಖ್ಯೈ ನಮಃ~। ವಿಷ್ಣುಕ್ರಾಂತಿಪತ್ರಂ ಸಮರ್ಪಯಾಮಿ~।\\
೪ ಭುವನೇಶ್ವರ್ಯೈ ನಮಃ~। ಜಂಬೀರಪತ್ರಂ ಸಮರ್ಪಯಾಮಿ~।\\
೪ ಶ್ರಿಯೈ ನಮಃ~। ಪದ್ಮಪತ್ರಂ ಸಮರ್ಪಯಾಮಿ~।\\
೪ ರಮಾಯೈ ನಮಃ~। ಚಂಪಕಪತ್ರಂ ಸಮರ್ಪಯಾಮಿ~।\\
೪ ಪುಷ್ಟ್ಯೈ ನಮಃ~। ಅಪಾಮಾರ್ಗಪತ್ರಂ ಸಮರ್ಪಯಾಮಿ~।\\
೪ ಗೌರ್ಯೈ ನಮಃ~। ಮಾಲತೀಪತ್ರಂ ಸಮರ್ಪಯಾಮಿ~।\\
೪ ಸರಸ್ವತ್ಯೈ ನಮಃ~। ಬದರೀಪತ್ರಂ ಸಮರ್ಪಯಾಮಿ~।\\
೪ ದುರ್ಗಾಯೈ ನಮಃ~। ಪಾರಿಜಾತಪತ್ರಂ ಸಮರ್ಪಯಾಮಿ~।\\
೪ ರುದ್ರಾಣ್ಯೈ ನಮಃ~। ಶಮೀಪತ್ರಂ ಸಮರ್ಪಯಾಮಿ~।\\
೪ ಮಾತೃಕಾಯೈ ನಮಃ~। ಧತ್ತೂರಪತ್ರಂ ಸಮರ್ಪಯಾಮಿ~।\\
೪ ಜಗದಂಬಾಯೈ ನಮಃ~। ನಿರ್ಗುಂಡೀಪತ್ರಂ ಸಮರ್ಪಯಾಮಿ~।\\
೪ ಲಲಿತಾಂಬಿಕಾಯೈ ನಮಃ~। ಕದಂಬಪತ್ರಂ ಸಮರ್ಪಯಾಮಿ~।\\
೪ ಶ್ರೀಚಕ್ರವಾಸಿನ್ಯೈ ನಮಃ~। ಸರ್ವಾಣಿ ಪತ್ರಾಣಿ ಸಮರ್ಪಯಾಮಿ~।
\section{ಪುಷ್ಪಪೂಜಾ}
೪ ಕಾಲಶಕ್ತ್ಯೈ ನಮಃ~। ಜಾಜೀಪುಷ್ಪಂ ಸಮರ್ಪಯಾಮಿ~।\\
೪ ಬ್ರಹ್ಮರೂಪಿಣ್ಯೈ ನಮಃ~। ಕೇತಕೀಪುಷ್ಪಂ ಸಮರ್ಪಯಾಮಿ~।\\
೪ ವಿಷ್ಣುರೂಪಿಣ್ಯೈ ನಮಃ~। ಪುನ್ನಾಗಪುಷ್ಪಂ ಸಮರ್ಪಯಾಮಿ~।\\
೪ ರುದ್ರರೂಪಿಣ್ಯೈ ನಮಃ~। ಕಮಲಪುಷ್ಪಂ ಸಮರ್ಪಯಾಮಿ~।\\
೪ ಸದಾಶಿವರೂಪಿಣ್ಯೈ ನಮಃ~। ಮಲ್ಲಿಕಾಪುಷ್ಪಂ ಸಮರ್ಪಯಾಮಿ~।\\
೪ ಚಿತ್ಕಲಾಯೈ ನಮಃ~। ಸೇವಂತಿಕಾಪುಷ್ಪಂ ಸಮರ್ಪಯಾಮಿ~।\\
೪ ವೇದಮಾತ್ರೇ ನಮಃ~। ಬಕುಲಪುಷ್ಪಂ ಸಮರ್ಪಯಾಮಿ~।\\
೪ ದುರ್ಗಾಯೈ ನಮಃ~। ಜಪಾಪುಷ್ಪಂ ಸಮರ್ಪಯಾಮಿ~।\\
೪ ಮಹಾಲಕ್ಷ್ಮ್ಯೈ ನಮಃ~। ಪಾರಿಜಾತಪುಷ್ಪಂ ಸಮರ್ಪಯಾಮಿ~।\\
೪ ಮಹಾಸರಸ್ವತ್ಯೈ ನಮಃ~। ಮಾಲತೀಪುಷ್ಪಂ ಸಮರ್ಪಯಾಮಿ~।\\
೪ ಚಂಪಕವಾಹಿನ್ಯೈ ನಮಃ~। ಚಂಪಕಪುಷ್ಪಂ ಸಮರ್ಪಯಾಮಿ~।\\
೪ ಕಾಮೇಶ್ವರ್ಯೈ ನಮಃ~। ಕದಂಬಪುಷ್ಪಂ ಸಮರ್ಪಯಾಮಿ~।\\
೪ ಕಾತ್ಯಾಯನ್ಯೈ ನಮಃ~। ಕರವೀರಪುಷ್ಪಂ ಸಮರ್ಪಯಾಮಿ~।\\
೪ ಸರ್ವೇಶ್ವರ್ಯೈ ನಮಃ~। ಅಶೋಕಪುಷ್ಪಂ ಸಮರ್ಪಯಾಮಿ~।\\
೪ ತ್ರಿಪುರಸುಂದರ್ಯೈ ನಮಃ~। ಗಿರಿಕರ್ಣಿಕಾಪುಷ್ಪಂ ಸಮರ್ಪಯಾಮಿ~।\\
೪ ರಾಜರಾಜೇಶ್ವರ್ಯೈ ನಮಃ~। ಬದರೀಪುಷ್ಪಂ ಸಮರ್ಪಯಾಮಿ~।\\
೪ ಶ್ರೀಚಕ್ರವಾಸಿನ್ಯೈ ನಮಃ~। ಸರ್ವಾಣಿ ಪುಷ್ಪಾಣಿ ಸಮರ್ಪಯಾಮಿ~।

{ಚತುರ್ಭಿಃ, ಅಷ್ಟಭಿಃ ಷಡ್ಭಿರ್ವಾ ದಲೈರ್ಯುತಂ ಮಂಡಲಂ ರಜತಾದಿ ಪಾತ್ರೇ ವಿಲಿಖ್ಯ ಘೃತವರ್ತಿಯುತಂ ದೀಪದ್ವಯಂ ನಿಧಾಯ, ಹ್ರೀಂ ಇತಿ ಪ್ರಜ್ವಾಲ್ಯ}\\
\as{೪ ಶ್ರೀಂಹ್ರೀಂಗ್ಲೂಂ ಸ್ಲೂಂಮ್ಲೂಂಪ್ಲೂಂ ನ್ಲೂಂಹ್ರೀಂಶ್ರೀಂ ॥}\\( ಇತಿ ರತ್ನೇಶ್ವರೀವಿದ್ಯಯಾ ಅಭಿಮಂತ್ರ್ಯ  ಚಕ್ರಮುದ್ರಾಂ ಪ್ರದರ್ಶ್ಯ ಮೂಲೇನ ಸಂಪೂಜ್ಯ )

\as{೪ ಜಗದ್ಧ್ವನಿಮಂತ್ರಮಾತಃ ಸ್ವಾಹಾ ॥} (ಇತಿ ಘಂಟಾಂ ಪೂಜಯಿತ್ವಾ )\\
ಓಂ ಐಂಹ್ರೀಂಶ್ರೀಂ ಶ್ರೀಮಲ್ಲಲಿತಾಮಹಾತ್ರಿಪುರಸುಂದರ್ಯೈ \\ಮಂಗಲಾರ್ತಿಕಂ ಕಲ್ಪಯಾಮಿ ನಮಃ~।

ಸಮಸ್ತಚಕ್ರಚಕ್ರೇಶಿಯುತೇ ದೇವಿ ನವಾತ್ಮಿಕೇ ।\\
ಆರಾರ್ತಿಕಮಿದಂ ತುಭ್ಯಂ ಗೃಹಾಣ ಮಮ ಸಿದ್ಧಯೇ ॥\\
 (ಇತಿ ನವಕೃತ್ವಃ ಮಂಡಲಾಕಾರೇಣ ಆಪಾದಮಸ್ತಕಂ\\
  ನೀರಾಜನಂ ವಿಧಾಯ ದಕ್ಷಭಾಗೇ ಸ್ಥಾಪಯೇತ್)

ಓಂ ಐಂಹ್ರೀಂಶ್ರೀಂ ಶ್ರೀಮಲ್ಲಲಿತಾಮಹಾತ್ರಿಪುರಸುಂದರ್ಯೈ \\
ಛತ್ರಂ ಕಲ್ಪಯಾಮಿ ನಮಃ~।\\
ಚಾಮರಯುಗಲಂ ಕಲ್ಪಯಾಮಿ ನಮಃ~।\\
ದರ್ಪಣಂ ಕಲ್ಪಯಾಮಿ ನಮಃ~।\\
ತಾಲವೃಂತಂ ಕಲ್ಪಯಾಮಿ ನಮಃ~।\\
ಗಂಧಂ ಕಲ್ಪಯಾಮಿ ನಮಃ~।\\
ಹರಿದ್ರಾ ಚೂರ್ಣಂ ಕಲ್ಪಯಾಮಿ ನಮಃ~।\\
ಕುಂಕುಮ ಚೂರ್ಣಂ ಕಲ್ಪಯಾಮಿ ನಮಃ~।\\
ಸಿಂದೂರ ಚೂರ್ಣಂ ಕಲ್ಪಯಾಮಿ ನಮಃ~।\\
ಪುಷ್ಪಾಣಿ ಕಲ್ಪಯಾಮಿ ನಮಃ~।\\
ಧೂಪಂ ಕಲ್ಪಯಾಮಿ~।\\
ದೀಪಂ ಕಲ್ಪಯಾಮಿ ನಮಃ~।\\
ನೈವೇದ್ಯಂ ಕಲ್ಪಯಾಮಿ ನಮಃ~।\\
ಪಾನೀಯಂ ಕಲ್ಪಯಾಮಿ ನಮಃ~।\\
ಉತ್ತರಾಪೋಶನಂ ಕಲ್ಪಯಾಮಿ ನಮಃ~।\\
ಹಸ್ತಪ್ರಕ್ಷಾಲನಂ ಕಲ್ಪಯಾಮಿ ನಮಃ~।\\
ಗಂಡೂಷಂ ಕಲ್ಪಯಾಮಿ ನಮಃ~।\\
ಪಾದಪ್ರಕ್ಷಾಲನಂ ಕಲ್ಪಯಾಮಿ ನಮಃ~।\\
ಆಚಮನೀಯಂ ಕಲ್ಪಯಾಮಿ ನಮಃ~।\\
ತಾಂಬೂಲಂ ಕಲ್ಪಯಾಮಿ ನಮಃ~।\\
ಮಂಗಲನೀರಾಜನಂ ಕಲ್ಪಯಾಮಿ ನಮಃ~।\\
ಮಂತ್ರಪುಷ್ಪಂ ಕಲ್ಪಯಾಮಿ ನಮಃ~।\\
ಪದಕ್ಷಿಣನಮಸ್ಕಾರಾನ್ ಕಲ್ಪಯಾಮಿ ನಮಃ~।\\
ಪ್ರಸನ್ನಾರ್ಘ್ಯಂ ಕಲ್ಪಯಾಮಿ ನಮಃ ।\\
ಪ್ರಾರ್ಥನಾಂ ಕಲ್ಪಯಾಮಿ ನಮಃ~।
\section{ಅಥ ಚತುಃಷಷ್ಟಿ ಯೋಗಿನೀಪೂಜಾ}
\begin{multicols}{2} ಓಂ ಬ್ರಹ್ಮಾಣ್ಯೈ~।\\ ಚಂಡಿಕಾಯೈ~।\\ ರೌದ್ರ್ಯೈ~।\\ ಗೌರ್ಯೈ~।\\ ಇಂದ್ರಾಣ್ಯೈ~।\\ ಕೌಮಾರ್ಯೈ~।\\ ವೈಷ್ಣವ್ಯೈ~।\\ ದುರ್ಗಾಯೈ~।\\ ನಾರಸಿಂಹ್ಯೈ~।\\ ಕಾಲಿಕಾಯೈ~।\\ ಚಾಮುಂಡಾಯೈ~।\\ ಶಿವದೂತ್ಯೈ~।\\ ವಾರಾಹ್ಯೈ~।\\ ಕೌಶಿಕ್ಯೈ~।\\ ಮಾಹೇಶ್ವರ್ಯೈ~।\\ ಶಾಂಕರ್ಯೈ~।\\ ಜಯಂತ್ಯೈ~।\\ ಸರ್ವಮಂಗಳಾಯೈ~।\\ ಕಾಳ್ಯೈ~।\\ ಕಪಾಲಿನ್ಯೈ~।\\ ಮೇಧಾಯೈ~।\\ ಶಿವಾಯೈ~।\\ ಶಾಕಂಭರ್ಯೈ~।\\ ಭೀಮಾಯೈ~।\\ ಶಾಂತಾಯೈ~।\\ ಭ್ರಾಮರ್ಯೈ~।\\ ರುದ್ರಾಣ್ಯೈ~।\\ ಅಂಬಿಕಾಯೈ~।\\ ಕ್ಷಮಾಯೈ~।\\ ಧಾತ್ರ್ಯೈ~।\\ ಸ್ವಾಹಾಯೈ~।\\ ಸ್ವಧಾಯೈ~।\\ ಅಪರ್ಣಾಯೈ~।\\ ಮಹೋದರ್ಯೈ~।\\ ಘೋರರೂಪಾಯೈ~।\\ ಮಹಾಕಾಳ್ಯೈ~।\\ ಭದ್ರಕಾಳ್ಯೈ~।\\ ಭಯಂಕರ್ಯೈ~।\\ ಕ್ಷೇಮಂಕರ್ಯೈ~।\\ ಉಗ್ರದಂಡಾಯೈ~।\\ ಚಂಡನಾಯಿಕಾಯೈ~।\\ ಚಂಡಾಯೈ~।\\ ಚಂಡವತ್ಯೈ~।\\ ಚಂಡ್ಯೈ~।\\ ಮಹಾಮೋಹಾಯೈ~।\\ ಪ್ರಿಯಂಕರ್ಯೈ~।\\ ಕಲವಿಕರಿಣ್ಯೈ~।\\ ದೇವ್ಯೈ~।\\ ಬಲಪ್ರಮಥಿನ್ಯೈ~।\\ ಮದನೋನ್ಮಥಿನ್ಯೈ~।\\ ಸರ್ವಭೂತದಮನಾಯೈ~।\\ ಉಮಾಯೈ~।\\ ತಾರಾಯೈ~।\\ ಮಹಾನಿದ್ರಾಯೈ~।\\ ವಿಜಯಾಯೈ~।\\ ಜಯಾಯೈ~।\\ ಶೈಲಪುತ್ರ್ಯೈ~।\\ ಚಂಡಘಂಟಾಯೈ~।\\ ಸ್ಕಂದಮಾತ್ರೇ~।\\ ಕಾಲರಾತ್ರ್ಯೈ~।\\ ಚಂಡಿಕಾಯೈ~।\\ ಕೂಷ್ಮಾಂಡಿನ್ಯೈ~।\\ ಕಾತ್ಯಾಯನ್ಯೈ~।\\ ಮಹಾಗೌರ್ಯೈ ನಮಃ~॥
\end{multicols}\authorline{ಇತಿ ಚತುಃಷಷ್ಟಿಯೋಗಿನೀಪೂಜಾ}
\newpage
\section{ ಯೋಗಿನೀಪೂಜಾ}
\begin{multicols}{2} \as{ಅಂ} ಅಮೃತಾಯೈ ನಮಃ~।\\ \as{ಆಂ} ಆಕರ್ಷಿಣ್ಯೈ ನಮಃ~।\\ \as{ಇಂ} ಇಂದ್ರಾಣ್ಯೈ ನಮಃ~।\\ \as{ಈಂ} ಈಶಾನ್ಯೈ ನಮಃ~।\\ \as{ಉಂ} ಉಮಾಯೈ ನಮಃ~।\\ \as{ಊಂ} ಊರ್ಧ್ವಕೇಶ್ಯೈ ನಮಃ~।\\ \as{ಋಂ} ಋದ್ಧಿದಾಯೈ ನಮಃ~।\\ \as{ೠಂ} ೠಕಾರಾಯೈ ನಮಃ~।\\ \as{ಲೃಂ} ಲೃಕಾರಾಯೈ ನಮಃ~।\\ \as{ಲೄಂ} ಲೄಕಾರಾಯೈ ನಮಃ~।\\ \as{ಏಂ} ಏಕಪದಾಯೈ ನಮಃ~।\\ \as{ಐಂ} ಐಶ್ವರ್ಯಾತ್ಮಿಕಾಯೈ ನಮಃ~।\\ \as{ಓಂ} ಓಂಕಾರಾಯೈ ನಮಃ~।\\ \as{ಔಂ} ಔಷಧ್ಯೈ ನಮಃ~।\\ \as{ಅಂ} ಅಂಬಿಕಾಯೈ ನಮಃ~।\\ \as{ಅಃ} ಅಕ್ಷರಾಯೈ ನಮಃ~।\\ \as{ಕಂ} ಕಾಲರಾತ್ರ್ಯೈ ನಮಃ~।\\ \as{ಖಂ} ಖಂಡಿತಾಯೈ ನಮಃ~।\\ \as{ಗಂ} ಗಾಯತ್ರ್ಯೈ ನಮಃ~।\\ \as{ಘಂ} ಘಂಟಾಕರ್ಷಿಣ್ಯೈ ನಮಃ~।\\ \as{ಙಂ} ಙಾರ್ಣಾಯೈ ನಮಃ~।\\ \as{ಚಂ} ಚಂಡಾಯೈ ನಮಃ~।\\ \as{ಛಂ} ಛಾಯಾಯೈ ನಮಃ~।\\ \as{ಜಂ} ಜಯಾಯೈ ನಮಃ~।\\ \as{ಝಂ} ಝಂಕಾರಿಣ್ಯೈ ನಮಃ~।\\ \as{ಞಂ} ಜ್ಞಾನರೂಪಾಯೈ ನಮಃ~।\\ \as{ಟಂ} ಟಂಕಹಸ್ತಾಯೈ ನಮಃ~।\\ \as{ಠಂ} ಠಂಕಾರಿಣ್ಯೈ ನಮಃ~।\\ \as{ಡಂ} ಡಾಮರ್ಯೈ ನಮಃ~।\\ \as{ಢಂ} ಢಂಕಾರಿಣ್ಯೈ ನಮಃ~।\\ \as{ಣಂ} ಣಾರ್ಣಾಯೈ ನಮಃ~।\\ \as{ತಂ} ತಾಮಸ್ಯೈ ನಮಃ~।\\ \as{ಥಂ} ಸ್ಥಾಣ್ವ್ಯೈ ನಮಃ~।\\ \as{ದಂ} ದಾಕ್ಷಾಯಣ್ಯೈ ನಮಃ~।\\ \as{ಧಂ} ಧಾತ್ರ್ಯೈ ನಮಃ~।\\ \as{ನಂ} ನಾರ್ಯೈ ನಮಃ~।\\ \as{ಪಂ} ಪಾರ್ವತ್ಯೈ ನಮಃ~।\\ \as{ಫಂ} ಫಟ್ಕಾರಿಣ್ಯೈ ನಮಃ~।\\ \as{ಬಂ} ಬಂಧಿನ್ಯೈ ನಮಃ ~।\\ \as{ಭಂ} ಭದ್ರಕಾಲ್ಯೈ ನಮಃ ~।\\ \as{ಮಂ} ಮಹಾಮಾಯಾಯೈ ನಮಃ~।\\ \as{ಯಂ} ಯಶಸ್ವಿನ್ಯೈ ನಮಃ~।\\ \as{ರಂ} ರಕ್ತಾಯೈ ನಮಃ~।\\ \as{ಲಂ} ಲಂಬೋಷ್ಠ್ಯೈ ನಮಃ~।\\ \as{ವಂ} ವರದಾಯೈ ನಮಃ~।\\ \as{ಶಂ} ಶ್ರಿಯೈ ನಮಃ~।\\ \as{ಷಂ} ಷಂಡಾಯೈ ನಮಃ~।\\ \as{ಸಂ} ಸರಸ್ವತ್ಯೈ ನಮಃ~।\\ \as{ಹಂ} ಹಂಸವತ್ಯೈ ನಮಃ~।\\ \as{ಕ್ಷಂ} ಕ್ಷಮಾವತ್ಯೈ ನಮಃ ॥
\end{multicols}
\as{ಚತುರಾಯತನ} ಪೂಜಾಂ ಕೃತ್ವಾ \\
೪ ಅಭೀಷ್ಟಸಿದ್ಧಿಂ ಮೇ ದೇಹಿ ಶರಣಾಗತವತ್ಸಲೇ ।\\
ಭಕ್ತ್ಯಾ ಸಮರ್ಪಯೇ ತುಭ್ಯಂ ಚತುರಾಯತನಾರ್ಚನಮ್ ॥\\
ಇತಿ ಸಾಮನ್ಯಾರ್ಘ್ಯೋದಕೇನ ದೇವ್ಯಾ ವಾಮಹಸ್ತೇ ನಿವೇದಯೇತ್ ।

\section{ಶಿವ ಆವರಣಪೂಜಾ}
(ಲಿಂಗ, ಯೋನಿ, ತ್ರಿಶೂಲ, ಅಕ್ಷಮಾಲಾ, ಅಭಯ, ವರ, ಮೃಗ, ಖಟ್ವಾಂಗ, ಕಪಾಲ, ಡಮರು ಮುದ್ರಾಃ ಪ್ರದರ್ಶ್ಯ)
\subsection{ಪ್ರಥಮಾವರಣಮ್}
ಓಂ ಓಂ ಹೃದಯಾಂಗ ದೇವತಾಭ್ಯೋ ನಮಃ । ಓಂ ನಂ ಶಿರೋಽಂಗ \\ದೇವತಾಭ್ಯೋ ನಮಃ । ಓಂ ಮಂ ಶಿಖಾಂಗ ದೇವತಾಭ್ಯೋ ನಮಃ । ಓಂ ಶಿಂ ಕವಚಾಂಗ ದೇವತಾಭ್ಯೋ ನಮಃ । ಓಂ ವಾಂ ನೇತ್ರಾಂಗ ದೇವತಾಭ್ಯೋ ನಮಃ। ಓಂ ಯಂ ಅಸ್ತ್ರಾಂಗ ದೇವತಾಭ್ಯೋ ನಮಃ ॥\\
ಅಭೀಷ್ಟಸಿದ್ಧಿಂ ಮೇ ದೇಹಿ ಶರಣಾಗತ ವತ್ಸಲ~।\\
ಭಕ್ತ್ಯಾ ಸಮರ್ಪಯೇ ತುಭ್ಯಂ ಪ್ರಥಮಾವರಣಾರ್ಚನಂ ॥
\subsection{ದ್ವಿತೀಯಾವರಣಮ್}
ಓಂ ಗಗನಾ ತತ್ಪುರುಷಾಭ್ಯಾಂ ನಮಃ । ಓಂ  ಕರಾಲಿಕಾ ವಾಮದೇವಾಭ್ಯಾಂ ನಮಃ । ಓಂ  ರಕ್ತಾಘೋರಾಭ್ಯಾಂ ನಮಃ । ಓಂ  ಮಹೋಚ್ಛುಷ್ಮಾ \\ಸದ್ಯೋಜಾತಾಭ್ಯಾಂ ನಮಃ । ಓಂ  ಹೃಲ್ಲೇಖೇಶಾನಾಭ್ಯಾಂ ನಮಃ ॥\\
ಅಭೀಷ್ಟಸಿದ್ಧಿಂ******ದ್ವಿತೀಯಾವರಣಾರ್ಚನಂ ॥
\subsection{ತೃತೀಯಾವರಣಮ್}
ಓಂ ಗಾಯತ್ರ್ಯೈ ನಮಃ । ಓಂ  ಸಾವಿತ್ರ್ಯೈ ನಮಃ । ಓಂ ಸರಸ್ವತ್ಯೈ ನಮಃ । ಓಂ  ವಸುಧಾ ಶಂಖನಿಧಿಭ್ಯಾಂ ನಮಃ । ಓಂ  ವಸುಮತೀ ಪುಷ್ಪನಿಧಿಭ್ಯಾಂ ನಮಃ ॥\\
ಅಭೀಷ್ಟಸಿದ್ಧಿಂ******ತೃತೀಯಾವರಣಾರ್ಚನಂ ॥
\subsection{ಚತುರ್ಥಾವರಣಮ್}
ಓಂ ಅನಂತಾಯ ನಮಃ । ಓಂ  ಸೂಕ್ಷ್ಮಾಯ ನಮಃ । ಓಂ  ಶಿವೋತ್ತಮಾಯ ನಮಃ। ಓಂ  ಏಕನೇತ್ರಾಯ ನಮಃ । ಓಂ  ಏಕರುದ್ರಾಯ ನಮಃ । \\ಓಂ   ತ್ರಿಮೂರ್ತಯೇ ನಮಃ । ಓಂ  ಶ್ರೀಕಂಠಾಯ ನಮಃ । ಓಂ  ಶಿಖಂಡಿನೇ ನಮಃ ॥\\
ಅಭೀಷ್ಟಸಿದ್ಧಿಂ******ತುರೀಯಾವರಣಾರ್ಚನಂ ॥
\subsection{ಪಂಚಮಾವರಣಮ್}
ಓಂ ಉಮಾಯೈ ನಮಃ । ಓಂ  ಚಂಡೇಶ್ವರಾಯ ನಮಃ । ಓಂ  ನಂದಿನೇ ನಮಃ । ಓಂ  ಮಹಾಕಾಲಾಯ ನಮಃ । ಓಂ  ಗಣೇಶ್ವರಾಯ ನಮಃ । ಓಂ  ಸ್ಕಂದಾಯ ನಮಃ । ಓಂ  ವೀರಭದ್ರಾಯ ನಮಃ । ಓಂ  ಭೃಂಗಿರಿಟಾಯ ನಮಃ ॥\\
ಅಭೀಷ್ಟಸಿದ್ಧಿಂ******ಪಂಚಮಾವರಣಾರ್ಚನಂ ॥
\subsection{ಷಷ್ಠಾವರಣಮ್}
ಓಂ ಬ್ರಾಹ್ಮ್ಯೈ ನಮಃ । ಓಂ ಮಾಹೇಶ್ವರ್ಯೈ ನಮಃ । ಓಂ ಕೌಮಾರ್ಯೈ ನಮಃ । ಓಂ ವೈಷ್ಣವ್ಯೈ ನಮಃ । ಓಂ ವಾರಾಹ್ಯೈ ನಮಃ । ಓಂ ಮಾಹೇಂದ್ರ್ಯೈ ನಮಃ । ಓಂ ಚಾಮುಂಡಾಯೈ ನಮಃ । ಓಂ ಮಹಾಲಕ್ಷ್ಮ್ಯೈ ನಮಃ ॥\\
ಅಭೀಷ್ಟಸಿದ್ಧಿಂ******ಷಷ್ಠಾಖ್ಯಾವರಣಾರ್ಚನಂ ॥
\subsection{ಸಪ್ತಮಾವರಣಮ್}
ಓಂ ಅಸಿತಾಂಗಭೈರವಾಯ ನಮಃ । ಓಂ ರುರುಭೈರವಾಯ ನಮಃ । ಓಂ ಚಂಡಭೈರವಾಯ ನಮಃ । ಓಂ ಕ್ರೋಧಭೈರವಾಯ ನಮಃ । ಓಂ ಉನ್ಮತ್ತಭೈರವಾಯ ನಮಃ । ಓಂ ಕಪಾಲಭೈರವಾಯ ನಮಃ । ಓಂ ಭೀಷಣಭೈರವಾಯ ನಮಃ । ಓಂ ಸಂಹಾರಭೈರವಾಯ ನಮಃ ॥\\%। ಓಂ ವರಪ್ರದಭೈರವಾಯ ನಮಃ
ಅಭೀಷ್ಟಸಿದ್ಧಿಂ******ಸಪ್ತಮಾವರಣಾರ್ಚನಂ ॥
\subsection{ಅಷ್ಟಮಾವರಣಮ್}
ಓಂ ಲಂ ಇಂದ್ರಾಯ ನಮಃ । ಓಂ ರಂ ಅಗ್ನಯೇ ನಮಃ । ಓಂ ಮಂ ಯಮಾಯ ನಮಃ । ಓಂ ಕ್ಷಂ ನಿರ್ಋತಯೇ ನಮಃ ।ಓಂ ವಂ ವರುಣಾಯ ನಮಃ ।ಓಂ ಯಂ ವಾಯವೇ ನಮಃ । ಓಂ ಕುಂ ಕುಬೇರಾಯ ನಮಃ । ಓಂ ಹಂ ಈಶಾನಾಯ ನಮಃ । ಓಂ ಆಂ ಬ್ರಹ್ಮಣೇ ನಮಃ । ಓಂ ಹ್ರೀಂ ಅನಂತಾಯ ನಮಃ । ಓಂ ನಿಯತ್ಯೈ ನಮಃ । ಓಂ ಕಾಲಾಯ ॥\\
ಅಭೀಷ್ಟಸಿದ್ಧಿಂ******ಅಷ್ಟಮಾವರಣಾರ್ಚನಂ ॥
\subsection{ನವಮಾವರಣಮ್}
ಓಂ ವಂ ವಜ್ರಾಯ ನಮಃ । ಓಂ ಶಂ ಶಕ್ತ್ಯೈ ನಮಃ । ಓಂ ದಂ ದಂಡಾಯ ನಮಃ । ಓಂ ಖಂ ಖಡ್ಗಾಯ ನಮಃ । ಓಂ ಪಾಂ ಪಾಶಾಯ ನಮಃ । ಓಂ ಅಂ ಅಂಕುಶಾಯ ನಮಃ । ಓಂ ಗಂ ಗದಾಯೈ ನಮಃ । ಓಂ  ತ್ರಿಂ  ತ್ರಿಶೂಲಾಯ ನಮಃ । ಓಂ ಪಂ ಪದ್ಮಾಯ ನಮಃ । ಓಂ ಚಂ ಚಕ್ರಾಯ ನಮಃ ॥\\
ಅಭೀಷ್ಟಸಿದ್ಧಿಂ******ನವಮಾವರಣಾರ್ಚನಂ ॥
\section{ಅಥ ಶಿವಾಷ್ಟೋತ್ತರಶತನಾಮಸ್ತೋತ್ರಮ್ }
\dhyana{ವಾಮಾಂಕ ನ್ಯಸ್ತ ವಾಮೇತರಕರಕಮಲಾಯಾಸ್ತಥಾ ವಾಮಬಾಹು\\ ನ್ಯಸ್ತಾರಕ್ತೋತ್ಪಲಾಯಾಃ ಸ್ತನವಿಧೃತಿಲಸದ್ವಾಮಹಸ್ತಃ ಪ್ರಿಯಾಯಾಃ ।\\
ನಾನಾಕಲ್ಪಾಭಿರಾಮೋ ಧೃತಪರಶುಮೃಗಾಭೀಷ್ಟದಃ ಕಾಂಚನಾಭಃ\\
ಧ್ಯೇಯಃ ಪದ್ಮಾಸನಸ್ಥಃ ಸ್ಮರಲಲಿತವಪುಃ ಸಂಪದೇ ಪಾರ್ವತೀಶಃ ॥}\\
ಶಿವೋ ಮಹೇಶ್ವರಃ ಶಂಭುಃ ಪಿನಾಕೀ ಶಶಿಶೇಖರಃ~।\\
ವಾಮದೇವೋ ವಿರೂಪಾಕ್ಷಃ ಕಪರ್ದೀ ನೀಲಲೋಹಿತಃ ॥೧ ॥

ಶಂಕರಃ ಶೂಲಪಾಣಿಶ್ಚ ಖಟ್ವಾಂಗೀ ವಿಷ್ಣುವಲ್ಲಭಃ~।\\
ಶಿಪಿವಿಷ್ಟೋಂಬಿಕಾನಾಥಃ ಶ್ರೀಕಂಠೋ ಭಕ್ತವತ್ಸಲಃ ॥೨ ॥

ಭವಃ ಶರ್ವಸ್ತ್ರಿಲೋಕೇಶಃ ಶಿತಿಕಂಠಃ ಶಿವಾಪ್ರಿಯಃ।\\
ಉಗ್ರಃ ಕಪಾಲೀ ಕಾಮಾರಿರಂಧಕಾಸುರಸೂದನಃ ॥೩ ॥

ಗಂಗಾಧರೋ ಲಲಾಟಾಕ್ಷಃ ಕಾಲಕಾಲಃ ಕೃಪಾನಿಧಿಃ~।\\
ಭೀಮಃ ಪರಶುಹಸ್ತಶ್ಚ ಮೃಗಪಾಣಿರ್ಜಟಾಧರಃ ॥೪ ॥

ಕೈಲಾಸವಾಸೀ ಕವಚೀ ಕಠೋರಸ್ತ್ರಿಪುರಾಂತಕಃ।\\
ವೃಷಾಂಕೋ ವೃಷಭಾರೂಢೋ ಭಸ್ಮೋದ್ಧೂಲಿತವಿಗ್ರಹಃ ॥೫ ॥

ಸಾಮಪ್ರಿಯಃ ಸ್ವರಮಯಸ್ತ್ರಯೀಮೂರ್ತಿರನೀಶ್ವರಃ~।\\
ಸರ್ವಜ್ಞಃ ಪರಮಾತ್ಮಾ ಚ ಸೋಮಸೂರ್ಯಾಗ್ನಿಲೋಚನಃ ॥೬॥

ಹವಿರ್ಯಜ್ಞಮಯಃ ಸೋಮಃ ಪಂಚವಕ್ತ್ರಃ ಸದಾಶಿವಃ।\\
ವಿಶ್ವೇಶ್ವರೋ ವೀರಭದ್ರೋ ಗಣನಾಥಃ ಪ್ರಜಾಪತಿಃ ॥೭॥

ಹಿರಣ್ಯರೇತಾ ದುರ್ಧರ್ಷೋ ಗಿರೀಶೋ ಗಿರಿಶೋನಘಃ।\\
ಭುಜಂಗಭೂಷಣೋ ಭರ್ಗೋ ಗಿರಿಧನ್ವಾ ಗಿರಿಪ್ರಿಯಃ ॥೮॥

ಕೃತ್ತಿವಾಸಾಃ ಪುರಾರಾತಿರ್ಭಗವಾನ್ ಪ್ರಮಥಾಧಿಪಃ।\\
ಮೃತ್ಯುಂಜಯಃ ಸೂಕ್ಷ್ಮತನುರ್ಜಗದ್ವ್ಯಾಪೀ ಜಗದ್ಗುರುಃ ॥೯॥

ವ್ಯೋಮಕೇಶೋ ಮಹಾಸೇನಜನಕಶ್ಚಾರುವಿಕ್ರಮಃ।\\
ರುದ್ರೋ ಭೂತಪತಿಃ ಸ್ಥಾಣುರಹಿರ್ಬುಧ್ನ್ಯೋ ದಿಗಂಬರಃ ॥೧೦॥

ಅಷ್ಟಮೂರ್ತಿರನೇಕಾತ್ಮಾ ಸಾತ್ವಿಕಃ ಶುದ್ಧವಿಗ್ರಹಃ।\\
ಶಾಶ್ವತಃ ಖಂಡಪರಶುರಜಃ ಪಾಶವಿಮೋಚನಃ ॥೧೧॥

ಮೃಡಃ ಪಶುಪತಿರ್ದೇವೋ ಮಹಾದೇವೋಽವ್ಯಯೋ ಹರಿಃ।\\
ಪೂಷದಂತಭಿದವ್ಯಗ್ರೋ ದಕ್ಷಾಧ್ವರಹರೋ ಹರಃ ॥೧೨॥

ಭಗನೇತ್ರಭಿದವ್ಯಕ್ತಃ ಸಹಸ್ರಾಕ್ಷಃ ಸಹಸ್ರಪಾತ್।\\
ಅಪವರ್ಗಪ್ರದೋಽನಂತಸ್ತಾರಕಃ ಪರಮೇಶ್ವರಃ ॥೧೩॥
%\authorline{॥ಇತಿ ಶಿವಾಷ್ಟೋತ್ತರ ಶತನಾಮಸ್ತೋತ್ರಂ ಸಂಪೂರ್ಣಂ॥}

\section{ಶ್ರೀಚಕ್ರಆವರಣಪೂಜಾ}
ಯಥಾಶಕ್ತಿ ಮೂಲಂ ಜಪ್ತ್ವಾ ದಶ ಮುದ್ರಾಃ ಪ್ರದರ್ಶ್ಯ\\
ದಕ್ಷಿಣಾಮೂರ್ತಿಂ ತ್ರಿಃ ಸಂತರ್ಪ್ಯ ತತಃ ಮೂಲೇನ ತ್ರಿಃ ಸಂತರ್ಪ್ಯ
\subsection{ಮೂಲತ್ರಿಕೋಣಾರ್ಚನಂ}
\as{೪ ಹ್‌ಸ್‌ರೈಂ ಹ್‌ಸ್‌ಕ್ಲ್ರೀಂ ಹ್‌ಸ್‌ರ್ಸೌಃ ।} \\ಅತಿರಹಸ್ಯಯೋಗಿನ್ಯಧಿಷ್ಠಿತ ಸರ್ವಸಿದ್ಧಿಪ್ರದ ಚಕ್ರಾಯ ನಮಃ ।\\ಇತಿ ಪುಷ್ಪಾಂಜಲಿಂ ಸಮರ್ಪ್ಯ\\
\as{೪ ಐಂ೫} ಮಹಾಕಾಮೇಶ್ವರೀ ಶ್ರೀಪಾದುಕಾಂ ಪೂ।ತ।ನಮಃ ।೧\\
\as{೪ ಕ್ಲೀಂ೬} ಮಹಾವಜ್ರೇಶ್ವರೀ ಶ್ರೀಪಾದುಕಾಂ ಪೂ।ತ।ನಮಃ ।೨\\
\as{೪ ಸೌಃ೪} ಮಹಾಭಗಮಾಲಿನೀ ಶ್ರೀಪಾದುಕಾಂ ಪೂ।ತ।ನಮಃ ।೩\\
\as{೪ ೧೫} ಮಹಾಶ್ರೀಸುಂದರೀ ಶ್ರೀಪಾದುಕಾಂ ಪೂ।ತ।ನಮಃ । (ಬಿಂದೌ)
\subsection{ಷಡಂಗಾರ್ಚನಂ}
\as{೪ ಐಂ೫ } ಸರ್ವಜ್ಞಾಯೈ ಹೃದಯಾಯ ನಮಃ ।\\ ಹೃದಯಶಕ್ತಿ ಶ್ರೀಪಾದುಕಾಂ ಪೂ।ತ।ನಮಃ ॥\\
\as{೪ ಕ್ಲೀಂ೬ } ನಿತ್ಯತೃಪ್ತಾಯೈ ಶಿರಸೇ ಸ್ವಾಹಾ ।\\ ಶಿರಃಶಕ್ತಿ ಶ್ರೀಪಾದುಕಾಂ ಪೂ।ತ।ನಮಃ ॥\\
\as{೪ ಸೌಃ೪ } ಅನಾದಿಬೋಧಿನ್ಯೈ ಶಿಖಾಯೈ ವಷಟ್ ।\\ ಶಿಖಾಶಕ್ತಿ ಶ್ರೀಪಾದುಕಾಂ ಪೂ।ತ।ನಮಃ ॥\\
\as{೪ ಐಂ೫ } ಸ್ವತಂತ್ರಾಯೈ ಕವಚಾಯ ಹುಂ ।\\ ಕವಚಶಕ್ತಿ ಶ್ರೀಪಾದುಕಾಂ ಪೂ।ತ।ನಮಃ ॥\\
\as{೪ ಕ್ಲೀಂ೬} ನಿತ್ಯಾಲುಪ್ತಾಯೈ ನೇತ್ರತ್ರಯಾಯ ವೌಷಟ್ ।\\ ನೇತ್ರಶಕ್ತಿ ಶ್ರೀಪಾದುಕಾಂ ಪೂ । ನಮಃ॥\\
\as{೪ ಸೌಃ೪ } ಅನಂತಾಯೈ ಅಸ್ತ್ರಾಯ ಫಟ್ ।\\ ಅಸ್ತ್ರಶಕ್ತಿ ಶ್ರೀಪಾದುಕಾಂ ಪೂ।ತ।ನಮಃ ॥
\subsection{ನಿತ್ಯಾದೇವೀಪೂಜಾ}
ಬಿಂದೌ ಮಹಾನಿತ್ಯಯಾ ತ್ರಿಃ ತಿಥಿನಿತ್ಯಯಾ ತ್ರಿಃ ಪುನಃ ಮಹಾನಿತ್ಯಯಾ ತ್ರಿಃ ತರ್ಪಯಿತ್ವಾ ಪಂಚದಶನಿತ್ಯಾಃ ವಾಮಾವರ್ತೇನ ತ್ರಿಕೋಣೇ ತರ್ಪಯಿತ್ವಾ ಮಹಾನಿತ್ಯಾಂ ಬಿಂದೌ ತರ್ಪಯೇತ್ ॥

\as{೪ ಅಂ} ಐಂ ಸಕಲಹ್ರೀಂ ನಿತ್ಯಕ್ಲಿನ್ನೇ ಮದದ್ರವೇ ಸೌಃ \as{ಅಂ~॥} ಕಾಮೇಶ್ವರೀನಿತ್ಯಾ  ಶ್ರೀಪಾ।ಪೂ।ತ।ನಮಃ ॥೧\\
\as{೪ ಆಂ} ಐಂ ಭಗಭುಗೇ ಭಗಿನಿ ಭಗೋದರಿ ಭಗಮಾಲೇ ಭಗಾವಹೇ ಭಗಗುಹ್ಯೇ ಭಗಯೋನಿ ಭಗನಿಪಾತನಿ ಸರ್ವಭಗವಶಂಕರಿ ಭಗರೂಪೇ ನಿತ್ಯಕ್ಲಿನ್ನೇ ಭಗಸ್ವರೂಪೇ ಸರ್ವಾಣಿ ಭಗಾನಿ ಮೇ ಹ್ಯಾನಯ ವರದೇ ರೇತೇ ಸುರೇತೇ ಭಗಕ್ಲಿನ್ನೇ ಕ್ಲಿನ್ನದ್ರವೇ ಕ್ಲೇದಯ ದ್ರಾವಯ ಅಮೋಘೇ ಭಗವಿಚ್ಚೇ ಕ್ಷುಭ ಕ್ಷೋಭಯ ಸರ್ವಸತ್ವಾನ್ ಭಗೇಶ್ವರಿ ಐಂ ಬ್ಲೂಂ ಜಂ ಬ್ಲೂಂ ಭೇಂ ಬ್ಲೂಂ ಮೋಂ ಬ್ಲೂಂ ಹೇಂ ಬ್ಲೂಂ ಹೇಂ ಕ್ಲಿನ್ನೇ ಸರ್ವಾಣಿ ಭಗಾನಿ ಮೇ ವಶಮಾನಯ ಸ್ತ್ರೀಂ ಹ್‌ರ್‌ಬ್ಲೇಂ ಹ್ರೀಂ \as{ಆಂ ॥} ಭಗಮಾಲಿನೀನಿತ್ಯಾ  ಶ್ರೀಪಾ।ಪೂ।ತ।ನಮಃ ॥೨\\
\as{೪ ಇಂ} ಓಂ ಹ್ರೀಂ ನಿತ್ಯಕ್ಲಿನ್ನೇ ಮದದ್ರವೇ ಸ್ವಾಹಾ \as{ಇಂ ॥} ನಿತ್ಯಕ್ಲಿನ್ನಾನಿತ್ಯಾ  ಶ್ರೀಪಾ।ಪೂ।ತ।ನಮಃ ॥೩\\
\as{೪ ಈಂ} ಓಂ ಕ್ರೋಂಭ್ರೋಂಕ್ರೋಂಝ್ರೋಂಛ್ರೋಂಜ್ರೋಂ ಸ್ವಾಹಾ \as{ಈಂ ॥} ಭೇರುಂಡಾನಿತ್ಯಾ  ಶ್ರೀಪಾ।ಪೂ।ತ।ನಮಃ ॥೪\\
\as{೪ ಉಂ} ಓಂ ಹ್ರೀಂ ವಹ್ನಿವಾಸಿನ್ಯೈ ನಮಃ \as{ಉಂ~॥} ವಹ್ನಿವಾಸಿನೀನಿತ್ಯಾ  ಶ್ರೀಪಾ।ಪೂ।ತ।ನಮಃ ॥೫\\
\as{೪ ಊಂ} ಹ್ರೀಂ ಕ್ಲಿನ್ನೇ ಐಂ ಕ್ರೋಂ ನಿತ್ಯಮದದ್ರವೇ ಹ್ರೀಂ \as{ಊಂ~॥} ಮಹಾವಜ್ರೇಶ್ವರೀನಿತ್ಯಾ  ಶ್ರೀಪಾ।ಪೂ।ತ।ನಮಃ ॥೬\\
\as{೪ ಋಂ} ಹ್ರೀಂ ಶಿವಾದೂತ್ಯೈ ನಮಃ \as{ಋಂ~॥} ಶಿವಾದೂತೀನಿತ್ಯಾ  ಶ್ರೀಪಾ।ಪೂ।ತ।ನಮಃ ॥೭\\
\as{೪ ೠಂ} ಓಂ ಹ್ರೀಂ ಹೂಂಖೇಚಛೇಕ್ಷಃಸ್ತ್ರೀಂಹೂಂಕ್ಷೇ ಹ್ರೀಂ ಫಟ್ \as{ೠಂ~॥} ತ್ವರಿತಾನಿತ್ಯಾ  ಶ್ರೀಪಾ।ಪೂ।ತ।ನಮಃ ॥೮\\
\as{೪ ಲೃಂ} ಐಂಕ್ಲೀಂಸೌಃ \as{ಲೃಂ~॥} ಕುಲಸುಂದರೀನಿತ್ಯಾ  ಶ್ರೀಪಾ।ಪೂ।ತ।ನಮಃ ॥೯\\
\as{೪ ಲೄಂ} ಹಸಕಲರಡೈಂ ಹಸಕಲರಡೀಂ ಹಸಕಲರಡೌಃ \as{ಲೄಂ ॥} ನಿತ್ಯಾನಿತ್ಯಾ  ಶ್ರೀಪಾ।ಪೂ।ತ।ನಮಃ ॥೧೦\\
\as{೪ ಏಂ} ಹ್ರೀಂ ಫ್ರೇಂಸ್ರೂಂಓಂಆಂಕ್ಲೀಂಐಂಬ್ಲೂಂ ನಿತ್ಯಮದದ್ರವೇ ಹುಂಫ್ರೇಂ ಹ್ರೀಂ \as{ಏಂ~॥} ನೀಲಪತಾಕಾನಿತ್ಯಾ  ಶ್ರೀಪಾ।ಪೂ।ತ।ನಮಃ ॥೧೧\\
\as{೪ ಐಂ} ಭಮರಯಉಔಂ \as{ಐಂ ॥} ವಿಜಯಾನಿತ್ಯಾ  ಶ್ರೀಪಾ।ಪೂ।ತ।ನಮಃ ॥೧೨\\
\as{೪ ಓಂ} ಸ್ವೌಂ \as{ಓಂ ॥} ಸರ್ವಮಂಗಳಾನಿತ್ಯಾ  ಶ್ರೀಪಾ।ಪೂ।ತ।ನಮಃ ॥೧೩\\
\as{೪ ಔಂ} ಓಂ ನಮೋ ಭಗವತಿ ಜ್ವಾಲಾಮಾಲಿನಿ ದೇವದೇವಿ ಸರ್ವಭೂತಸಂಹಾರಕಾರಿಕೇ ಜಾತವೇದಸಿ ಜ್ವಲಂತಿ ಜ್ವಲ ಜ್ವಲ ಪ್ರಜ್ವಲ ಪ್ರಜ್ವಲ ಹ್ರಾಂ ಹ್ರೀಂ ಹ್ರೂಂ ರರ ರರ ರರರ ಹುಂ ಫಟ್ ಸ್ವಾಹಾ \as{ಔಂ~॥} ಜ್ವಾಲಾಮಾಲಿನೀನಿತ್ಯಾ  ಶ್ರೀಪಾ।ಪೂ।ತ।ನಮಃ ॥೧೪\\
\as{೪ ಅಂ} ಚ್ಕೌಂ \as{ಅಂ~॥} ಚಿತ್ರಾನಿತ್ಯಾ  ಶ್ರೀಪಾ।ಪೂ।ತ।ನಮಃ ॥೧೫\\
\as{೪ ಅಃ} ೧೫ \as{ಅಃ~॥} ಲಲಿತಾ ಮಹಾನಿತ್ಯಾ ಶ್ರೀಪಾ।ಪೂ।ತ।ನಮಃ ॥೧೬

\subsection{ಗುರುಮಂಡಲಪೂಜಾ}
\as{೪ ದಿವ್ಯೌಘಸಿದ್ಧೌಘಮಾನವೌಘೇಭ್ಯೋ ನಮಃ}

\as{೪} ಪರಪ್ರಕಾಶಾನಂದನಾಥ ಶ್ರೀಪಾ।ಪೂ।ತ।ನಮಃ ।\\
\as{೪} ಪರಶಿವಾನಂದನಾಥ ಶ್ರೀಪಾ।ಪೂ।ತ।ನಮಃ ।\\
\as{೪} ಪರಾಶಕ್ತ್ಯಂಬಾ ಶ್ರೀಪಾ।ಪೂ।ತ।ನಮಃ ।\\
\as{೪} ಕೌಲೇಶ್ವರಾನಂದನಾಥ ಶ್ರೀಪಾ।ಪೂ।ತ।ನಮಃ ।\\
\as{೪} ಶುಕ್ಲದೇವ್ಯಂಬಾ ಶ್ರೀಪಾ।ಪೂ।ತ।ನಮಃ ।\\
\as{೪} ಕುಲೇಶ್ವರಾನಂದನಾಥ ಶ್ರೀಪಾ।ಪೂ।ತ।ನಮಃ ।\\
\as{೪} ಕಾಮೇಶ್ವರ್ಯಂಬಾ ಶ್ರೀಪಾ।ಪೂ।ತ।ನಮಃ ।

\as{೪} ಭೋಗಾನಂದನಾಥ ಶ್ರೀಪಾ।ಪೂ।ತ।ನಮಃ ।\\
\as{೪} ಕ್ಲಿನ್ನಾನಂದನಾಥ ಶ್ರೀಪಾ।ಪೂ।ತ।ನಮಃ ।\\
\as{೪} ಸಮಯಾನಂದನಾಥ ಶ್ರೀಪಾ।ಪೂ।ತ।ನಮಃ ।\\
\as{೪} ಸಹಜಾನಂದನಾಥ ಶ್ರೀಪಾ।ಪೂ।ತ।ನಮಃ ।

\as{೪} ಗಗನಾನಂದನಾಥ ಶ್ರೀಪಾ।ಪೂ।ತ।ನಮಃ ।\\
\as{೪} ವಿಶ್ವಾನಂದನಾಥ ಶ್ರೀಪಾ।ಪೂ।ತ।ನಮಃ ।\\
\as{೪} ವಿಮಲಾನಂದನಾಥ ಶ್ರೀಪಾ।ಪೂ।ತ।ನಮಃ ।\\
\as{೪} ಮದನಾನಂದನಾಥ ಶ್ರೀಪಾ।ಪೂ।ತ।ನಮಃ ।\\
\as{೪} ಭುವನಾನಂದನಾಥ ಶ್ರೀಪಾ।ಪೂ।ತ।ನಮಃ ।\\
\as{೪} ಲೀಲಾನಂದನಾಥ ಶ್ರೀಪಾ।ಪೂ।ತ।ನಮಃ ।\\
\as{೪} ಸ್ವಾತ್ಮಾನಂದನಾಥ ಶ್ರೀಪಾ।ಪೂ।ತ।ನಮಃ ।\\
\as{೪} ಪ್ರಿಯಾನಂದನಾಥ ಶ್ರೀಪಾ।ಪೂ।ತ।ನಮಃ ।
\subsection{(ಷೋಡಶ್ಯುಪಾಸಕಾನಾಂ ಕೃತೇ ವಿದ್ಯಾರ್ಣವತಂತ್ರೋಕ್ತ ಗುರುಪರಂಪರಾ)}
\as{೪} ವ್ಯೋಮಾತೀತಾಂಬಾ ಶ್ರೀಪಾ।ಪೂ।ತ।ನಮಃ ।\\
\as{೪} ವ್ಯೋಮೇಶ್ವರ್ಯಂಬಾ ಶ್ರೀಪಾ।ಪೂ।ತ।ನಮಃ ।\\
\as{೪} ವ್ಯೋಮಗಾಂಬಾ ಶ್ರೀಪಾ।ಪೂ।ತ।ನಮಃ ।\\
\as{೪} ವ್ಯೋಮಚಾರಿಣ್ಯಂಬಾ ಶ್ರೀಪಾ।ಪೂ।ತ।ನಮಃ ।\\
\as{೪} ವ್ಯೋಮಸ್ಥಾಂಬಾ ಶ್ರೀಪಾ।ಪೂ।ತ।ನಮಃ ।

\as{೪} ಉನ್ಮನಾಕಾಶಾನಂದನಾಥ ಶ್ರೀಪಾ।ಪೂ।ತ।ನಮಃ ।\\
\as{೪} ಸಮನಾಕಾಶಾನಂದನಾಥ ಶ್ರೀಪಾ।ಪೂ।ತ।ನಮಃ ।\\
\as{೪} ವ್ಯಾಪಕಾಕಾಶಾನಂದನಾಥ ಶ್ರೀಪಾ।ಪೂ।ತ।ನಮಃ ।\\
\as{೪} ಶಕ್ತ್ಯಾಕಾಶಾನಂದನಾಥ ಶ್ರೀಪಾ।ಪೂ।ತ।ನಮಃ ।\\
\as{೪} ಧ್ವನ್ಯಾಕಾಶಾನಂದನಾಥ ಶ್ರೀಪಾ।ಪೂ।ತ।ನಮಃ ।\\
\as{೪} ಧ್ವನಿಮಾತ್ರಾಕಾಶಾನಂದನಾಥ ಶ್ರೀಪಾ।ಪೂ।ತ।ನಮಃ ।\\
\as{೪} ಅನಾಹತಾಕಾಶನಂದನಾಥ ಶ್ರೀಪಾ।ಪೂ।ತ।ನಮಃ ।\\
\as{೪} ಬಿಂದ್ವಾಕಾಶಾನಂದನಾಥ ಶ್ರೀಪಾ।ಪೂ।ತ।ನಮಃ ।\\
\as{೪} ಇಂದ್ವಾಕಾಶಾನಂದನಾಥ ಶ್ರೀಪಾ।ಪೂ।ತ।ನಮಃ ।

\as{೪} ಪರಮಾತ್ಮಾನಂದನಾಥ ಶ್ರೀಪಾ।ಪೂ।ತ।ನಮಃ ।\\
\as{೪} ಶಾಂಭವಾನಂದನಾಥ ಶ್ರೀಪಾ।ಪೂ।ತ।ನಮಃ ।\\
\as{೪} ಚಿನ್ಮುದ್ರಾನಂದನಾಥ ಶ್ರೀಪಾ।ಪೂ।ತ।ನಮಃ ।\\
\as{೪} ವಾಗ್ಭವಾನಂದನಾಥ ಶ್ರೀಪಾ।ಪೂ।ತ।ನಮಃ ।\\
\as{೪} ಲೀಲಾನಂದನಾಥ ಶ್ರೀಪಾ।ಪೂ।ತ।ನಮಃ ।\\
\as{೪} ಸಂಭ್ರಮಾನಂದನಾಥ ಶ್ರೀಪಾ।ಪೂ।ತ।ನಮಃ ।\\
\as{೪} ಚಿದಾನಂದನಾಥ ಶ್ರೀಪಾ।ಪೂ।ತ।ನಮಃ ।\\
\as{೪} ಪ್ರಸನ್ನಾನಂದನಾಥ ಶ್ರೀಪಾ।ಪೂ।ತ।ನಮಃ ।\\
\as{೪} ವಿಶ್ವಾನಂದನಾಥ ಶ್ರೀಪಾ।ಪೂ।ತ।ನಮಃ ।

\as{೪} ಪರಮೇಶ್ವರ ಪರಮೇಶ್ವರೀ ಶ್ರೀಪಾ।ಪೂ।ತ।ನಮಃ\\
\as{೪} ಮಿತ್ರೀಶಮಯೀ ಶ್ರೀಪಾ।ಪೂ।ತ।ನಮಃ\\
\as{೪} ಷಷ್ಠೀಶಮಯೀ ಶ್ರೀಪಾ।ಪೂ।ತ।ನಮಃ\\
\as{೪} ಉಡ್ಡೀಶಮಯೀ ಶ್ರೀಪಾ।ಪೂ।ತ।ನಮಃ\\
\as{೪} ಚರ್ಯಾನಾಥಮಯೀ ಶ್ರೀಪಾ।ಪೂ।ತ।ನಮಃ\\
\as{೪} ಲೋಪಾಮುದ್ರಾಮಯೀ ಶ್ರೀಪಾ।ಪೂ।ತ।ನಮಃ\\
\as{೪} ಅಗಸ್ತ್ಯಮಯೀ ಶ್ರೀಪಾ।ಪೂ।ತ।ನಮಃ

\as{೪} ಕಾಲತಾಪನಮಯೀ ಶ್ರೀಪಾ।ಪೂ।ತ।ನಮಃ\\
\as{೪} ಧರ್ಮಾಚಾರಮಯೀ ಶ್ರೀಪಾ।ಪೂ।ತ।ನಮಃ\\
\as{೪} ಮುಕ್ತಕೇಶೀಶ್ವರಮಯೀ ಶ್ರೀಪಾ।ಪೂ।ತ।ನಮಃ\\
\as{೪} ದೀಪಕಲಾನಾಥಮಯೀ ಶ್ರೀಪಾ।ಪೂ।ತ।ನಮಃ

\as{೪} ವಿಷ್ಣುದೇವಮಯೀ ಶ್ರೀಪಾ।ಪೂ।ತ।ನಮಃ\\
\as{೪} ಪ್ರಭಾಕರದೇವಮಯೀ ಶ್ರೀಪಾ।ಪೂ।ತ।ನಮಃ\\
\as{೪} ತೇಜೋದೇವಮಯೀ ಶ್ರೀಪಾ।ಪೂ।ತ।ನಮಃ\\
\as{೪} ಮನೋಜದೇವಮಯೀ ಶ್ರೀಪಾ।ಪೂ।ತ।ನಮಃ\\
\as{೪} ಕಲ್ಯಾಣದೇವಮಯೀ ಶ್ರೀಪಾ।ಪೂ।ತ।ನಮಃ\\
\as{೪} ರತ್ನದೇವಮಯೀ ಶ್ರೀಪಾ।ಪೂ।ತ।ನಮಃ\\
\as{೪} ವಾಸುದೇವಮಯೀ ಶ್ರೀಪಾ।ಪೂ।ತ।ನಮಃ\\
\as{೪} ಶ್ರೀರಾಮಾನಂದಮಯೀ ಶ್ರೀಪಾ।ಪೂ।ತ।ನಮಃ

\subsection{ಗುರುತ್ರಯಪೂಜಾ}
\as{ಓಂಐಂಹ್ರೀಂಶ್ರೀಂಐಂಕ್ಲೀಂಸೌಃ} ಹಂಸಃ ಶಿವಃ ಸೋಽಹಂ ಹಂಸಃ, ಹ್‌ಸ್‌ಖ್‌ಫ್ರೇಂ ಹಸಕ್ಷಮಲವರಯೂಂ ಹ್‌ಸೌಃ ಸಹಕ್ಷಮಲವರಯೀಂ ಸ್‌ಹೌಃ ,ಐಂ ಕಏಈಲಹ್ರೀಂ ಕ್ಲೀಂ ಹಸಕಹಲಹ್ರೀಂ ಸೌಃ  ಸಕಲಹ್ರೀಂ, ಹಸಕಲಹ್ರೀಂ ಹಸಕಹಲಹ್ರೀಂ ಸಕಲಹ್ರೀಂ, ಹಸಕಲ ಹಸಕಹಲ ಸಕಲಹ್ರೀಂ, ಪ್ರಜ್ಞಾನಂ ಬ್ರಹ್ಮ, ಅಹಂ ಬ್ರಹ್ಮಾಸ್ಮಿ, ತತ್ವಮಸಿ, ಅಯಮಾತ್ಮಾ ಬ್ರಹ್ಮ, ಹಂಸಃ ಶಿವಃ ಸೋಽಹಂ ಹಂಸಃ॥ ಸ್ವಾತ್ಮಾರಾಮ ಪರಮಾನಂದ ಪಂಜರ ವಿಲೀನ ತೇಜಸೇ ಪರಮೇಷ್ಠಿಗುರವೇ ನಮಃ~। ಶ್ರೀಪಾದುಕಾಂ ಪೂಜಯಾಮಿ ನಮಃ॥

\as{೭} ಐಂಕ್ಲೀಂಸೌಃ ಸೋಽಹಂ ಹಂಸಃ ಶಿವಃ, ಹ್‌ಸ್‌ಖ್‌ಫ್ರೇಂ ಹಸಕ್ಷಮಲವರಯೂಂ ಹ್‌ಸೌಃ ಸಹಕ್ಷಮಲವರಯೀಂ ಸ್‌ಹೌಃ,ಐಂ ಕಏಈಲಹ್ರೀಂ ಕ್ಲೀಂ ಹಸಕಹಲಹ್ರೀಂ ಸೌಃ  ಸಕಲಹ್ರೀಂ, ಹಸಕಲಹ್ರೀಂ ಹಸಕಹಲಹ್ರೀಂ ಸಕಲಹ್ರೀಂ, ಹಸಕಲ ಹಸಕಹಲ ಸಕಲಹ್ರೀಂ, ಪ್ರಜ್ಞಾನಂ ಬ್ರಹ್ಮ, ಅಹಂ ಬ್ರಹ್ಮಾಸ್ಮಿ, ತತ್ವಮಸಿ, ಅಯಮಾತ್ಮಾ ಬ್ರಹ್ಮ,  ಸೋಽಹಂ ಹಂಸಃ ಶಿವಃ ॥ ಸ್ವಚ್ಛಪ್ರಕಾಶ ವಿಮರ್ಶಹೇತವೇ ಪರಮಗುರವೇ ನಮಃ । ಶ್ರೀಪಾದುಕಾಂ ಪೂಜಯಾಮಿ ನಮಃ ॥

\as{೭} ಐಂಕ್ಲೀಂಸೌಃ ಹಂಸಃ ಶಿವಃ ಸೋಽಹಂ, ಹ್‌ಸ್‌ಖ್‌ಫ್ರೇಂ ಹಸಕ್ಷಮಲವರಯೂಂ ಹ್‌ಸೌಃ ಸಹಕ್ಷಮಲವರಯೀಂ ಸ್‌ಹೌಃ , ಐಂ ಕಏಈಲಹ್ರೀಂ ಕ್ಲೀಂ ಹಸಕಹಲಹ್ರೀಂ ಸೌಃ  ಸಕಲಹ್ರೀಂ, ಹಸಕಲಹ್ರೀಂ ಹಸಕಹಲಹ್ರೀಂ ಸಕಲಹ್ರೀಂ, ಹಸಕಲ ಹಸಕಹಲ ಸಕಲಹ್ರೀಂ, ಪ್ರಜ್ಞಾನಂ ಬ್ರಹ್ಮ, ಅಹಂ ಬ್ರಹ್ಮಾಸ್ಮಿ, ತತ್ವಮಸಿ, ಅಯಮಾತ್ಮಾ ಬ್ರಹ್ಮ, ಹಂಸಃ ಶಿವಃ ಸೋಽಹಂ~॥ ಸ್ವರೂಪ ನಿರೂಪಣ ಹೇತವೇ ಶ್ರೀಗುರವೇ ನಮಃ~। ಶ್ರೀಪಾದುಕಾಂ ಪೂಜಯಾಮಿ ನಮಃ ॥

೪ ಶ್ರೀಗುರೋ ದಕ್ಷಿಣಾಮೂರ್ತೇ ಭಕ್ತಾನುಗ್ರಹಕಾರಕ~।\\
ಅನುಜ್ಞಾಂ ದೇಹಿ ಭಗವನ್ ಶ್ರೀಚಕ್ರ ಯಜನಾಯ ಮೇ ॥ ಇತಿ ನತ್ವಾ
\section{ಆವರಣಪೂಜಾ}

\as{ ಸಂವಿನ್ಮಯೇ ಪರೇ ದೇವಿ ಪರಾಮೃತರುಚಿಪ್ರಿಯೇ ।\\
 ಅನುಜ್ಞಾಂ ತ್ರಿಪುರೇ ದೇಹಿ ಪರಿವಾರಾರ್ಚನಾಯ ಮೇ ॥
 
 ಪ್ರಕಟಾದ್ಯಾಶ್ಚ ಯೋಗಿನ್ಯೋ ಮಹಾವೀರಪರಂಪರಾಃ।\\
ಸಾನ್ನಿಧ್ಯಂ ಕುಲಯಾಗೇಽಸ್ಮಿನ್ ಪ್ರಕುರ್ವಂತು ಶಿವಾಜ್ಞಯಾ ॥}
\subsection{ಪ್ರಥಮಾವರಣಂ}
{\bfseries ೪ ಅಂ ಆಂ ಸೌಃ ತ್ರೈಲೋಕ್ಯಮೋಹನ ಚಕ್ರಾಯ ನಮಃ }\\
\as{೪ ಅಂ} ಅಣಿಮಾಸಿದ್ಧಿಶ್ರೀಪಾ।ಪೂ।ತ।ನಮಃ ।೧\\
\as{೪ ಲಂ} ಲಘಿಮಾಸಿದ್ಧಿಶ್ರೀಪಾ।ಪೂ।ತ।ನಮಃ ।೨\\
\as{೪ ಮಂ} ಮಹಿಮಾಸಿದ್ಧಿಶ್ರೀಪಾ।ಪೂ।ತ।ನಮಃ ।೩\\
\as{೪ ಈಂ} ಈಶಿತ್ವಸಿದ್ಧಿಶ್ರೀಪಾ।ಪೂ।ತ।ನಮಃ ।೪\\
\as{೪ ವಂ} ವಶಿತ್ವಸಿದ್ಧಿಶ್ರೀಪಾ।ಪೂ।ತ।ನಮಃ ।೫\\
\as{೪ ಪಂ} ಪ್ರಾಕಾಮ್ಯಸಿದ್ಧಿಶ್ರೀಪಾ।ಪೂ।ತ।ನಮಃ ।೬\\
\as{೪ ಭುಂ} ಭುಕ್ತಿಸಿದ್ಧಿಶ್ರೀಪಾ।ಪೂ।ತ।ನಮಃ ।೭\\
\as{೪ ಇಂ} ಇಚ್ಛಾಸಿದ್ಧಿಶ್ರೀಪಾ।ಪೂ।ತ।ನಮಃ ।೮\\
\as{೪ ಪಂ} ಪ್ರಾಪ್ತಿಸಿದ್ಧಿಶ್ರೀಪಾ।ಪೂ।ತ।ನಮಃ ।೯\\
\as{೪ ಸಂ} ಸರ್ವಕಾಮಸಿದ್ಧಿಶ್ರೀಪಾ।ಪೂ।ತ।ನಮಃ ।೧೦

\as{೪ ಆಂ} ಬ್ರಾಹ್ಮೀಮಾತೃಶ್ರೀಪಾ।ಪೂ।ತ।ನಮಃ ।೧\\
\as{೪ ಈಂ} ಮಾಹೇಶ್ವರೀಮಾತೃಶ್ರೀಪಾ।ಪೂ।ತ।ನಮಃ ।೨\\
\as{೪ ಊಂ} ಕೌಮಾರೀಮಾತೃಶ್ರೀಪಾ।ಪೂ।ತ।ನಮಃ ।೩\\
\as{೪ ೠಂ} ವೈಷ್ಣವೀಮಾತೃಶ್ರೀಪಾ।ಪೂ।ತ।ನಮಃ ।೪\\
\as{೪ ಲೄಂ} ವಾರಾಹೀಮಾತೃಶ್ರೀಪಾ।ಪೂ।ತ।ನಮಃ ।೫\\
\as{೪ ಐಂ} ಮಾಹೇಂದ್ರೀಮಾತೃಶ್ರೀಪಾ।ಪೂ।ತ।ನಮಃ ।೬\\
\as{೪ ಔಂ} ಚಾಮುಂಡಾಮಾತೃಶ್ರೀಪಾ।ಪೂ।ತ।ನಮಃ ।೭\\
\as{೪ ಅಃ} ಮಹಾಲಕ್ಷ್ಮೀಮಾತೃಶ್ರೀಪಾ।ಪೂ।ತ।ನಮಃ ।೮

\as{೪ ದ್ರಾಂ} ಸರ್ವಸಂಕ್ಷೋಭಿಣೀ ಮುದ್ರಾಶಕ್ತಿಶ್ರೀಪಾ।ಪೂ।ತ।ನಮಃ ।೧\\
\as{೪ ದ್ರೀಂ} ಸರ್ವವಿದ್ರಾವಿಣೀ ಮುದ್ರಾಶಕ್ತಿಶ್ರೀಪಾ।ಪೂ।ತ।ನಮಃ ।೨\\
\as{೪ ಕ್ಲೀಂ} ಸರ್ವಾಕರ್ಷಿಣೀ ಮುದ್ರಾಶಕ್ತಿಶ್ರೀಪಾ।ಪೂ।ತ।ನಮಃ ।೩\\
\as{೪ ಬ್ಲೂಂ} ಸರ್ವವಶಂಕರೀ ಮುದ್ರಾಶಕ್ತಿಶ್ರೀಪಾ।ಪೂ।ತ।ನಮಃ ।೪\\
\as{೪ ಸಃ} ಸರ್ವೋನ್ಮಾದಿನೀ ಮುದ್ರಾಶಕ್ತಿಶ್ರೀಪಾ।ಪೂ।ತ।ನಮಃ ।೫\\
\as{೪ ಕ್ರೋಂ} ಸರ್ವಮಹಾಂಕುಶಾ ಮುದ್ರಾಶಕ್ತಿಶ್ರೀಪಾ।ಪೂ।ತ।ನಮಃ ।೬\\
\as{೪ ಹ್‌ಸ್‌ಖ್‌ಫ್ರೇಂ} ಸರ್ವಖೇಚರೀ ಮುದ್ರಾಶಕ್ತಿಶ್ರೀಪಾ।ಪೂ।ತ।ನಮಃ ।೭\\
\as{೪ ಹ್ಸೌಃ} ಸರ್ವಬೀಜ ಮುದ್ರಾಶಕ್ತಿಶ್ರೀಪಾ।ಪೂ।ತ।ನಮಃ ।೮\\
\as{೪ ಐಂ} ಸರ್ವಯೋನಿ ಮುದ್ರಾಶಕ್ತಿಶ್ರೀಪಾ।ಪೂ।ತ।ನಮಃ ।೯\\
\as{೪ ಹ್‌ಸ್‌ರೈಂ ಹ್‌ಸ್‌ಕ್ಲ್ರೀಂ ಹ್‌ಸ್‌ರ್ಸೌಃ}\\ ಸರ್ವತ್ರಿಖಂಡಾ ಮುದ್ರಾಶಕ್ತಿಶ್ರೀಪಾ।ಪೂ।ತ।ನಮಃ ।೧೦

\as{೪} ಏತಾಃ ಪ್ರಕಟಯೋಗಿನ್ಯಃ ತ್ರೈಲೋಕಕ್ಯಮೋಹನೇ ಚಕ್ರೇ ಸಮುದ್ರಾಃ ಸಸಿದ್ಧಯಃ ಸಾಯುಧಾಃ ಸಶಕ್ತಯಃ ಸವಾಹನಾಃ ಸಪರಿವಾರಾಃ ಸರ್ವೋಪಚಾರೈಃ ಸಂಪೂಜಿತಾಃ ಸಂತರ್ಪಿತಾಃ ಸಂತುಷ್ಟಾಃ ಸಂತು ನಮಃ ॥

\as{೪ ಅಂ ಆಂ ಸೌಃ॥} ತ್ರಿಪುರಾಚಕ್ರೇಶ್ವರೀ ಶ್ರೀಪಾದುಕಾಂ ಪೂಜಯಾಮಿ ತರ್ಪಯಾಮಿ ನಮಃ ॥\\
\as{೪ ಅಂ} ಅಣಿಮಾಸಿದ್ಧಿ ಶ್ರೀಪಾದುಕಾಂ ಪೂಜಯಾಮಿ ತರ್ಪಯಾಮಿ ನಮಃ ॥\\
\as{೪ ದ್ರಾಂ} ಸರ್ವಸಂಕ್ಷೋಭಿಣೀಮುದ್ರಾಶಕ್ತಿ ಶ್ರೀಪಾದುಕಾಂ ಪೂಜಯಾಮಿ ತರ್ಪಯಾಮಿ ನಮಃ ॥\\
\as{೪ ದ್ರಾಂ ॥}\\
ಮೂಲೇನ ತ್ರಿಃ ಸಂತರ್ಪ್ಯ
\subsection{ನಾಮಾವಲಿಃ}
ನಿತ್ಯಾಯೈ । ಜಗನ್ಮೂರ್ತ್ಯೈ । ದೇವ್ಯೈ । ಭಗವತ್ಯೈ । ಮಹಾಮಾಯಾಯೈ । ಪ್ರಸನ್ನಾಯೈ । ವರದಾಯೈ । ಮುಕ್ತಿದಾಯಿನ್ಯೈ । ಪರಮಾಯೈ । ಮುಕ್ತಿಹೇತವೇ । ಸನಾತನ್ಯೈ । ಸಂಸಾರಬಂಧಹೇತವೇ । ಸರ್ವೇಶ್ವರೇಶ್ವರ್ಯೈ । ಯೋಗನಿದ್ರಾಯೈ । ಹರಿನೇತ್ರಕೃತಾಲಯಾಯೈ । ವಿಶ್ವೇಶ್ವರ್ಯೈ । ಜಗದ್ಧಾತ್ರ್ಯೈ । ಸ್ಥಿತಿಸಂಹಾರಕಾರಿಣ್ಯೈ । ವಿಷ್ಣೋರ್ನಿದ್ರಾಯೈ । ಭಗವತ್ಯೈ । ಅತುಲಾಯೈ । ಸ್ವಾಹಾಯೈ । ಸ್ವಧಾಯೈ । ವಷಟ್ಕಾರಸ್ವರಾತ್ಮಿಕಾಯೈ । ಸುಧಾಯೈ । ಅಕ್ಷರಾಯೈ । ತ್ರಿಧಾಮಾತ್ರಾತ್ಮಿಕಾಯೈ । ಅರ್ಧಮಾತ್ರಾಸ್ಥಿತಾಯೈ । ಅನುಚ್ಚಾರ್ಯಾಯೈ । ಸಂಧ್ಯಾಯೈ । ಸಾವಿತ್ರ್ಯೈ । ಜನನ್ಯೈ । ಪರಾಯೈ । ಸೃಷ್ಟಿರೂಪಾಯೈ । ಸ್ಥಿತಿರೂಪಾಯೈ । ಸಂಹೃತಿರೂಪಾಯೈ । ಜಗನ್ಮಯ್ಯೈ । \\
\as{೪ ತ್ರಿಪುರಾದೇವ್ಯೈ ವಿದ್ಮಹೇ ಕಾಮೇಶ್ವರ್ಯೈ ಧೀಮಹಿ ।\\ತನ್ನಃ ಕ್ಲಿನ್ನಾ ಪ್ರಚೋದಯಾತ್ ॥}

\as{೪} ಅಭೀಷ್ಟಸಿದ್ಧಿಂ ಮೇ ದೇಹಿ ಶರಣಾಗತ ವತ್ಸಲೇ~।\\
ಭಕ್ತ್ಯಾ ಸಮರ್ಪಯೇ ತುಭ್ಯಂ ಪ್ರಥಮಾವರಣಾರ್ಚನಂ ॥

೪ ಪ್ರಕಟಯೋಗಿನೀ ಮಯೂಖಾಯೈ ಪ್ರಥಮಾವರಣದೇವತಾ ಸಹಿತಾಯೈ ಶ್ರೀಲಲಿತಾಮಹಾತ್ರಿಪುರಸುಂದರೀ ಪರಾಭಟ್ಟಾರಿಕಾಯೈ ನಮಃ ॥ ಇತಿ ಯೋನ್ಯಾ ಪ್ರಣಮೇತ್ ।
\subsection{ದ್ವಿತೀಯಾವರಣಂ}
{\bfseries ೪ ಐಂ ಕ್ಲೀಂ ಸೌಃ ಸರ್ವಾಶಾಪರಿಪೂರಕ ಚಕ್ರಾಯ ನಮಃ}\\
\as{೪ ಅಂ} ಕಾಮಾಕರ್ಷಣೀ ನಿತ್ಯಾಕಲಾ ದೇವೀಶ್ರೀಪಾ।ಪೂ।ತ।ನಮಃ ।೧\\
\as{೪ ಆಂ} ಬುದ್ಧ್ಯಾಕರ್ಷಣೀ ನಿತ್ಯಾಕಲಾ ದೇವೀಶ್ರೀಪಾ।ಪೂ।ತ।ನಮಃ ।೨\\
\as{೪ ಇಂ} ಅಹಂಕಾರಾಕರ್ಷಣೀ ನಿತ್ಯಾಕಲಾ ದೇವೀಶ್ರೀಪಾ।ಪೂ।ತ।ನಮಃ ।೩\\
\as{೪ ಈಂ} ಶಬ್ದಾಕರ್ಷಣೀ ನಿತ್ಯಾಕಲಾ ದೇವೀಶ್ರೀಪಾ।ಪೂ।ತ।ನಮಃ ।೪\\
\as{೪ ಉಂ} ಸ್ಪರ್ಶಾಕರ್ಷಣೀ ನಿತ್ಯಾಕಲಾ ದೇವೀಶ್ರೀಪಾ।ಪೂ।ತ।ನಮಃ ।೫\\
\as{೪ ಊಂ} ರೂಪಾಕರ್ಷಣೀ ನಿತ್ಯಾಕಲಾ ದೇವೀಶ್ರೀಪಾ।ಪೂ।ತ।ನಮಃ ।೬\\
\as{೪ ಋಂ} ರಸಾಕರ್ಷಣೀ ನಿತ್ಯಾಕಲಾ ದೇವೀಶ್ರೀಪಾ।ಪೂ।ತ।ನಮಃ ।೭\\
\as{೪ ೠಂ} ಗಂಧಾಕರ್ಷಣೀ ನಿತ್ಯಾಕಲಾ ದೇವೀಶ್ರೀಪಾ।ಪೂ।ತ।ನಮಃ ।೮\\
\as{೪ ಲೃಂ} ಚಿತ್ತಾಕರ್ಷಣೀ ನಿತ್ಯಾಕಲಾ ದೇವೀಶ್ರೀಪಾ।ಪೂ।ತ।ನಮಃ ।೯\\
\as{೪ ಲೄಂ} ಧೈರ್ಯಾಕರ್ಷಣೀ ನಿತ್ಯಾಕಲಾ ದೇವೀಶ್ರೀಪಾ।ಪೂ।ತ।ನಮಃ ।೧೦\\
\as{೪ ಏಂ} ಸ್ಮೃತ್ಯಾಕರ್ಷಣೀ ನಿತ್ಯಾಕಲಾ ದೇವೀಶ್ರೀಪಾ।ಪೂ।ತ।ನಮಃ ।೧೧\\
\as{೪ ಐಂ} ನಾಮಾಕರ್ಷಣೀ ನಿತ್ಯಾಕಲಾ ದೇವೀಶ್ರೀಪಾ।ಪೂ।ತ।ನಮಃ ।೧೨\\
\as{೪ ಓಂ} ಬೀಜಾಕರ್ಷಣೀ ನಿತ್ಯಾಕಲಾ ದೇವೀಶ್ರೀಪಾ।ಪೂ।ತ।ನಮಃ ।೧೩\\
\as{೪ ಔಂ} ಆತ್ಮಾಕರ್ಷಣೀ ನಿತ್ಯಾಕಲಾ ದೇವೀಶ್ರೀಪಾ।ಪೂ।ತ।ನಮಃ ।೧೪\\
\as{೪ ಅಂ} ಅಮೃತಾಕರ್ಷಣೀ ನಿತ್ಯಾಕಲಾ ದೇವೀಶ್ರೀಪಾ।ಪೂ।ತ।ನಮಃ ।೧೫\\
\as{೪ ಅಃ} ಶರೀರಾಕರ್ಷಣೀ ನಿತ್ಯಾಕಲಾ ದೇವೀಶ್ರೀಪಾ।ಪೂ।ತ।ನಮಃ ।೧೬

\as{೪} ಏತಾಃ ಗುಪ್ತಯೋಗಿನ್ಯಃ ಸರ್ವಾಶಾಪರಿಪೂರಕೇ ಚಕ್ರೇ ಸಮುದ್ರಾಃ ಸಸಿದ್ಧಯಃ ಸಾಯುಧಾಃ ಸಶಕ್ತಯಃ ಸವಾಹನಾಃ ಸಪರಿವಾರಾಃ ಸರ್ವೋಪಚಾರೈಃ ಸಂಪೂಜಿತಾಃ ಸಂತರ್ಪಿತಾಃ ಸಂತುಷ್ಟಾಃ ಸಂತು ನಮಃ ॥

\as{೪ ಐಂ ಕ್ಲೀಂ ಸೌಃ॥} ತ್ರಿಪುರೇಶೀಚಕ್ರೇಶ್ವರೀಶ್ರೀಪಾ।ಪೂ।ತ।ನಮಃ॥\\
\as{೪ ಲಂ} ಲಘಿಮಾಸಿದ್ಧಿಶ್ರೀಪಾ।ಪೂ।ತ।ನಮಃ॥\\
\as{೪ ದ್ರೀಂ} ಸರ್ವವಿದ್ರಾವಿಣೀಮುದ್ರಾಶಕ್ತಿಶ್ರೀಪಾ।ಪೂ।ತ।ನಮಃ॥\\
\as{೪ ದ್ರೀಂ ॥}\\
ಮೂಲೇನ ತ್ರಿಃ ಸಂತರ್ಪ್ಯ
\subsection{ನಾಮಾವಲಿಃ}
ಮಹಾವಿದ್ಯಾಯೈ । ಮಹಾಮಾಯಾಯೈ । ಮಹಾಮೇಧಾಯೈ । ಮಹಾಸ್ಮೃತ್ಯೈ । ಮಹಾಮೋಹಾಯೈ । ಮಹಾದೇವ್ಯೈ । ಮಹೇಶ್ವರ್ಯೈ । ಪ್ರಕೃತ್ಯೈ । ಗುಣತ್ರಯವಿಭಾವಿನ್ಯೈ । ಕಾಲರಾತ್ರ್ಯೈ । ಮಹಾರಾತ್ರ್ಯೈ । ಮೋಹರಾತ್ರ್ಯೈ । ಶ್ರಿಯೈ । ಈಶ್ವರ್ಯೈ । ಹ್ರಿಯೈ । ಬೋಧಲಕ್ಞಣಬುದ್ಧ್ಯೈ । ಲಜ್ಜಾಯೈ । ಪುಷ್ಟ್ಯೈ । ತುಷ್ಟ್ಯೈ । ಶಾಂತ್ಯೈ । ಕ್ಷಾಂತ್ಯೈ । ಖಡ್ಗಿನ್ಯೈ । ಶೂಲಿನ್ಯೈ । ಘೋರಾಯೈ । ಗದಿನ್ಯೈ । ಚಕ್ರಿಣ್ಯೈ । ಶಂಖಿನ್ಯೈ । ಚಾಪಿನ್ಯೈ । ಬಾಣಭುಶುಂಡೀಪರಿಘಾಯುಧಾಯೈ । ಸೌಮ್ಯಾಯೈ । ಸೌಮ್ಯತರಾಯೈ । ಅತಿಸುಂದರ್ಯೈ । ಪರಸ್ಯೈ । ಪರಾಣಾಂ ಪರಮಾಯೈ । ಪರಮೇಶ್ವರ್ಯೈ । ಸದಸದ್ವಸ್ತುಶಕ್ತ್ಯೈ । ಅಖಿಲಾತ್ಮಿಕಾಯೈ । ನಾರ್ಯೈ।\\
\as{೪ ತ್ರಿಪುರೇಶ್ವರ್ಯೈ ವಿದ್ಮಹೇ ಕಾಮೇಶ್ವರ್ಯೈ ಧೀಮಹಿ ।\\ತನ್ನಃ ಕ್ಲಿನ್ನಾ ಪ್ರಚೋದಯಾತ್ ॥}

\as{೪} ಅಭೀಷ್ಟಸಿದ್ಧಿಂ******ದ್ವಿತೀಯಾವರಣಾರ್ಚನಂ ॥

೪ ಗುಪ್ತಯೋಗಿನೀ ಮಯೂಖಾಯೈ ದ್ವಿತೀಯಾವರಣದೇವತಾ ಸಹಿತಾಯೈ ಶ್ರೀಲಲಿತಾಮಹಾತ್ರಿಪುರಸುಂದರೀ ಪರಾಭಟ್ಟಾರಿಕಾಯೈ ನಮಃ ॥ ಇತಿ ಯೋನ್ಯಾ ಪ್ರಣಮೇತ್ ।
\subsection{ತೃತೀಯಾವರಣಂ}
{\bfseries ೪ ಹ್ರೀಂ ಕ್ಲೀಂ ಸೌಃ ಸರ್ವಸಂಕ್ಷೋಭಣಚಕ್ರಾಯ ನಮಃ}\\
\as{೪ ಕಂಖಂಗಂಘಂಙಂ} ಅನಂಗಕುಸುಮಾದೇವೀಶ್ರೀಪಾ।ಪೂ।ತ।ನಮಃ ।೧\\
\as{೪ ಚಂಛಂಜಂಝಂಞಂ} ಅನಂಗಮೇಖಲಾದೇವೀಶ್ರೀಪಾ।ಪೂ।ತ।ನಮಃ ।೨\\
\as{೪ ಟಂಠಂಡಂಢಂಣಂ} ಅನಂಗಮದನಾದೇವೀಶ್ರೀಪಾ।ಪೂ।ತ।ನಮಃ ।೩\\
\as{೪ ತಂಥಂದಂಧಂನಂ} ಅನಂಗಮದನಾತುರಾದೇವೀಶ್ರೀಪಾ।ಪೂ।ತ।ನಮಃ ।೪\\
\as{೪ ಪಂಫಂಬಂಭಂಮಂ} ಅನಂಗರೇಖಾದೇವೀಶ್ರೀಪಾ।ಪೂ।ತ।ನಮಃ ।೫\\
\as{೪ ಯಂರಂಲಂವಂ} ಅನಂಗವೇಗಿನೀದೇವೀಶ್ರೀಪಾ।ಪೂ।ತ।ನಮಃ ।೬\\
\as{೪ ಶಂಷಂಸಂಹಂ} ಅನಂಗಾಂಕುಶಾದೇವೀಶ್ರೀಪಾ।ಪೂ।ತ।ನಮಃ ।೭\\
\as{೪ ಳಂಕ್ಷಂ} ಅನಂಗಮಾಲಿನೀದೇವೀಶ್ರೀಪಾ।ಪೂ।ತ।ನಮಃ ।೮

\as{೪} ಏತಾಃ ಗುಪ್ತತರಯೋಗಿನ್ಯಃ ಸರ್ವಸಂಕ್ಷೋಭಣೇ ಚಕ್ರೇ ಸಮುದ್ರಾಃ ಸಸಿದ್ಧಯಃ ಸಾಯುಧಾಃ ಸಶಕ್ತಯಃ ಸವಾಹನಾಃ ಸಪರಿವಾರಾಃ ಸರ್ವೋಪಚಾರೈಃ ಸಂಪೂಜಿತಾಃ ಸಂತರ್ಪಿತಾಃ ಸಂತುಷ್ಟಾಃ ಸಂತು ನಮಃ ॥

\as{೪ ಹ್ರೀಂ ಕ್ಲೀಂ ಸೌಃ॥}  ತ್ರಿಪುರಸುಂದರೀಚಕ್ರೇಶ್ವರೀಶ್ರೀಪಾ।ಪೂ।ತ।ನಮಃ॥\\
\as{೪ ಮಂ} ಮಹಿಮಾಸಿದ್ಧಿಶ್ರೀಪಾ।ಪೂ।ತ।ನಮಃ॥\\
\as{೪ ಕ್ಲೀಂ} ಸರ್ವಾಕರ್ಷಿಣೀಮುದ್ರಾಶಕ್ತಿಶ್ರೀಪಾ।ಪೂ।ತ।ನಮಃ॥\\
\as{೪ ಕ್ಲೀಂ ॥}\\
ಮೂಲೇನ ತ್ರಿಃ ಸಂತರ್ಪ್ಯ

\subsection{ನಾಮಾವಲಿಃ}
ಶಿವಾಯೈ । ಸಿಂಹವಾಹಿನ್ಯೈ । ಅಂಬಿಕಾಯೈ । ಭದ್ರಕಾಲ್ಯೈ । ಚಂಡಿಕಾಯೈ । ಜಗನ್ಮಾತ್ರೇ । ಮಹಿಷಾಸುರಮರ್ದಿನ್ಯೈ । ಜಗದಾತ್ಮಶಕ್ತ್ಯೈ । ಸರ್ವದೇವಮಯ್ಯೈ । ಶ್ರದ್ಧಾಯೈ । ತ್ರಿಗುಣಾತ್ಮಿಕಾಯೈ । ಸರ್ವಾಶ್ರಯಾಯೈ । ಅವ್ಯಾಕೃತಾಯೈ । ಆದ್ಯಪ್ರಕೃತ್ಯೈ । ಶಬ್ದಾತ್ಮಿಕಾಯೈ । ತ್ರಯ್ಯೈ । ಜಗದ್ವಾರ್ತಾಯೈ । ಆರ್ತಿಹಂತ್ರ್ಯೈ । ಮೇಧಾಯೈ । ದುರ್ಗಾಯೈ । ಭವಸಮುದ್ರನೌಕಾಯೈ । ಅಸಂಗಾಯೈ । ಶ್ರಿಯೈ । ಕೈಟಭಾರಿಹೃದಯೈಕ ಕೃತಾಧಿವಾಸಾಯೈ । ಗೌರ್ಯೈ । ಶಶಿಮೌಲಿಕೃತಪ್ರತಿಷ್ಠಾಯೈ । ಆರ್ದ್ರಚಿತ್ತಾಯೈ । ಗೀರ್ವಾಣವರದಾಯಿನ್ಯೈ । ದೇವ್ಯೈ । ಮಹಾದೇವ್ಯೈ । ಶಿವಾಯೈ । ಪ್ರಕೃತ್ಯೈ । ಭದ್ರಾಯೈ । ರೌದ್ರಾಯೈ । ನಿತ್ಯಾಯೈ । ಗೌರ್ಯೈ । ಧಾತ್ರ್ಯೈ । ಜ್ಯೋತ್ಸ್ನಾಯೈ । ಇಂದುರೂಪಿಣ್ಯೈ । ಸುಖಾಯೈ । ಕಲ್ಯಾಣ್ಯೈ । ಪ್ರಣತಾಮೃದ್ಧ್ಯೈ । ಸಿದ್ಧ್ಯೈ । ಕೂರ್ಮ್ಯೈ । ನೈರ್ರೃತ್ಯೈ । ಭೂಭೃತಾಂಲಕ್ಷ್ಮ್ಯೈ ॥\\
\as{೪ ತ್ರಿಪುರಸುಂದರ್ಯೈ ವಿದ್ಮಹೇ ಕಾಮೇಶ್ವರ್ಯೈ ಧೀಮಹಿ ।\\ತನ್ನಃ ಕ್ಲಿನ್ನಾ ಪ್ರಚೋದಯಾತ್ ॥}

\as{೪} ಅಭೀಷ್ಟಸಿದ್ಧಿಂ******ತೃತೀಯಾವರಣಾರ್ಚನಂ ॥

೪ ಗುಪ್ತತರಯೋಗಿನೀ ಮಯೂಖಾಯೈ ತೃತೀಯಾವರಣದೇವತಾ ಸಹಿತಾಯೈ ಶ್ರೀಲಲಿತಾಮಹಾತ್ರಿಪುರಸುಂದರೀ ಪರಾಭಟ್ಟಾರಿಕಾಯೈ ನಮಃ ॥ ಇತಿ ಯೋನ್ಯಾ ಪ್ರಣಮೇತ್ ।
\subsection{ಚತುರ್ಥಾವರಣಂ}
{\bfseries ೪ ಹೈಂ ಹ್‌ಕ್ಲೀಂ ಹ್‌ಸೌಃ ಸರ್ವಸೌಭಾಗ್ಯದಾಯಕ ಚಕ್ರಾಯ ನಮಃ}\\
\as{೪ ಕಂ} ಸರ್ವಸಂಕ್ಷೋಭಿಣೀಶಕ್ತಿಶ್ರೀಪಾ।ಪೂ।ತ।ನಮಃ ।೧\\
\as{೪ ಖಂ} ಸರ್ವವಿದ್ರಾವಿಣೀಶಕ್ತಿಶ್ರೀಪಾ।ಪೂ।ತ।ನಮಃ ।೨\\
\as{೪ ಗಂ} ಸರ್ವಾಕರ್ಷಿಣೀಶಕ್ತಿಶ್ರೀಪಾ।ಪೂ।ತ।ನಮಃ ।೩\\
\as{೪ ಘಂ} ಸರ್ವಾಹ್ಲಾದಿನೀಶಕ್ತಿಶ್ರೀಪಾ।ಪೂ।ತ।ನಮಃ ।೪\\
\as{೪ ಙಂ} ಸರ್ವಸಮ್ಮೋಹಿನೀಶಕ್ತಿಶ್ರೀಪಾ।ಪೂ।ತ।ನಮಃ ।೫\\
\as{೪ ಚಂ} ಸರ್ವಸ್ತಂಭಿನೀಶಕ್ತಿಶ್ರೀಪಾ।ಪೂ।ತ।ನಮಃ ।೬\\
\as{೪ ಛಂ} ಸರ್ವಜೃಂಭಿಣೀಶಕ್ತಿಶ್ರೀಪಾ।ಪೂ।ತ।ನಮಃ ।೭\\
\as{೪ ಜಂ} ಸರ್ವವಶಂಕರೀಶಕ್ತಿಶ್ರೀಪಾ।ಪೂ।ತ।ನಮಃ ।೮\\
\as{೪ ಝಂ} ಸರ್ವರಂಜನೀಶಕ್ತಿಶ್ರೀಪಾ।ಪೂ।ತ।ನಮಃ ।೯\\
\as{೪ ಞಂ} ಸರ್ವೋನ್ಮಾದಿನೀಶಕ್ತಿಶ್ರೀಪಾ।ಪೂ।ತ।ನಮಃ ।೧೦\\
\as{೪ ಟಂ} ಸರ್ವಾರ್ಥಸಾಧಿನೀಶಕ್ತಿಶ್ರೀಪಾ।ಪೂ।ತ।ನಮಃ ।೧೧\\
\as{೪ ಠಂ} ಸರ್ವಸಂಪತ್ತಿಪೂರಣೀಶಕ್ತಿಶ್ರೀಪಾ।ಪೂ।ತ।ನಮಃ ।೧೨\\
\as{೪ ಡಂ} ಸರ್ವಮಂತ್ರಮಯೀಶಕ್ತಿಶ್ರೀಪಾ।ಪೂ।ತ।ನಮಃ ।೧೩\\
\as{೪ ಢಂ} ಸರ್ವದ್ವಂದ್ವಕ್ಷಯಂಕರೀಶಕ್ತಿಶ್ರೀಪಾ।ಪೂ।ತ।ನಮಃ ।೧೪

\as{೪} ಏತಾಃ ಸಂಪ್ರದಾಯಯೋಗಿನ್ಯಃ ಸರ್ವಸೌಭಾಗ್ಯದಾಯಕೇ ಚಕ್ರೇ ಸಮುದ್ರಾಃ ಸಸಿದ್ಧಯಃ ಸಾಯುಧಾಃ ಸಶಕ್ತಯಃ ಸವಾಹನಾಃ ಸಪರಿವಾರಾಃ ಸರ್ವೋಪಚಾರೈಃ ಸಂಪೂಜಿತಾಃ ಸಂತರ್ಪಿತಾಃ ಸಂತುಷ್ಟಾಃ ಸಂತು ನಮಃ ॥

\as{೪ ಹೈಂ ಹ್‌ಕ್ಲೀಂ ಹ್‌ಸೌಃ॥} ತ್ರಿಪುರವಾಸಿನೀಚಕ್ರೇಶ್ವರೀಶ್ರೀಪಾ।ಪೂ।ತ।ನಮಃ॥\\
\as{೪ ಈಂ} ಈಶಿತ್ವಸಿದ್ಧಿಶ್ರೀಪಾ।ಪೂ।ತ।ನಮಃ॥\\
\as{೪ ಬ್ಲೂಂ} ಸರ್ವವಶಂಕರೀಮುದ್ರಾಶಕ್ತಿಶ್ರೀಪಾ।ಪೂ।ತ।ನಮಃ॥\\
\as{೪ ಬ್ಲೂಂ ॥}\\
ಮೂಲೇನ ತ್ರಿಃ ಸಂತರ್ಪ್ಯ
\subsection{ನಾಮಾವಲಿಃ}
ಶರ್ವಾಣ್ಯೈ । ದುರ್ಗಾಯೈ । ದುರ್ಗಪಾರಾಯೈ । ಸಾರಾಯೈ । ಸರ್ವಕಾರಿಣ್ಯೈ । ಖ್ಯಾತ್ಯೈ । ಕೃಷ್ಣಾಯೈ । ಧೂಮ್ರಾಯೈ । ಅತಿಸೌಮ್ಯಾಯೈ । ಅತಿರೌದ್ರಾಯೈ । ಜಗತ್ಪ್ರತಿಷ್ಠಾಯೈ । ದೇವ್ಯೈ । ಕೃತ್ಯೈ । ವಿಷ್ಣುಮಾಯಾಯೈ । ಚೇತನಾಯೈ । ಬುದ್ಧಿರೂಪಿಣ್ಯೈ । ನಿದ್ರಾರೂಪಿಣ್ಯೈ । ಕ್ಷುಧಾಯೈ । ಛಾಯಾರೂಪಿಣ್ಯೈ । ಶಕ್ತಿರೂಪಾಯೈ । ತೃಷ್ಣಾರೂಪಾಯೈ । ಕ್ಷಾಂತಿರೂಪಾಯೈ । ಜಾತಿರೂಪಾಯೈ । ಲಜ್ಜಾರೂಪಾಯೈ । ಶಾಂತಿರೂಪಾಯೈ । ಶ್ರದ್ಧಾರೂಪಾಯೈ । ಕಾಂತಿಸ್ವರೂಪಿಣ್ಯೈ । ಲಕ್ಷ್ಮೀರೂಪಾಯೈ । ವೃತ್ತಿರೂಪಾಯೈ । ಸ್ಮೃತಿರೂಪಾಯೈ । ದಯಾರೂಪಾಯೈ । ತುಷ್ಟಿರೂಪಾಯೈ । ಪುಷ್ಟಿರೂಪಾಯೈ । ಮಾತೃರೂಪಾಯೈ । ಭ್ರಾಂತಿರೂಪಾಯೈ । ಇಂದ್ರಿಯಾಧಿಷ್ಠಾತ್ರ್ಯೈ । ಭೂತೇಷು ವ್ಯಾಪ್ತಾಯೈ । ಚಿತಿರೂಪಾಯೈ । ಶುಭಹೇತವೇ । ಪಾರ್ವತ್ಯೈ । ಕೌಶಿಕ್ಯೈ । ಕಾಲಿಕಾಯೈ । ಉಗ್ರಚಂಡಾಯೈ ॥ \\
\as{೪ ತ್ರಿಪುರವಾಸಿನ್ಯೈ ವಿದ್ಮಹೇ ಕಾಮೇಶ್ವರ್ಯೈ ಧೀಮಹಿ ।\\ತನ್ನಃ ಕ್ಲಿನ್ನಾ ಪ್ರಚೋದಯಾತ್ ॥}

\as{೪} ಅಭೀಷ್ಟಸಿದ್ಧಿಂ******ತುರೀಯಾವರಣಾರ್ಚನಂ ॥

೪ ಸಂಪ್ರದಾಯಯೋಗಿನೀ ಮಯೂಖಾಯೈ ತುರೀಯಾವರಣದೇವತಾ ಸಹಿತಾಯೈ ಶ್ರೀಲಲಿತಾಮಹಾತ್ರಿಪುರಸುಂದರೀ ಪರಾಭಟ್ಟಾರಿಕಾಯೈ ನಮಃ ॥ ಇತಿ ಯೋನ್ಯಾ ಪ್ರಣಮೇತ್ ।
\subsection{ಪಂಚಮಾವರಣಂ}
{\bfseries ೪ ಹ್‌ಸೈಂ ಹ್‌ಸ್‌ಕ್ಲೀಂ ಹ್‌ಸ್ಸೌಃ ಸರ್ವಾರ್ಥಸಾಧಕ ಚಕ್ರಾಯ ನಮಃ}\\
\as{೪ ಣಂ} ಸರ್ವಸಿದ್ಧಿಪ್ರದಾದೇವೀಶ್ರೀಪಾ।ಪೂ।ತ।ನಮಃ ।೧\\
\as{೪ ತಂ} ಸರ್ವಸಂಪತ್ಪ್ರದಾದೇವೀಶ್ರೀಪಾ।ಪೂ।ತ।ನಮಃ ।೨\\
\as{೪ ಥಂ} ಸರ್ವಪ್ರಿಯಂಕರೀದೇವೀಶ್ರೀಪಾ।ಪೂ।ತ।ನಮಃ ।೩\\
\as{೪ ದಂ} ಸರ್ವಮಂಗಳಕಾರಿಣೀದೇವೀಶ್ರೀಪಾ।ಪೂ।ತ।ನಮಃ ।೪\\
\as{೪ ಧಂ} ಸರ್ವಕಾಮಪ್ರದಾದೇವೀಶ್ರೀಪಾ।ಪೂ।ತ।ನಮಃ ।೫\\
\as{೪ ನಂ} ಸರ್ವದುಃಖವಿಮೋಚನೀದೇವೀಶ್ರೀಪಾ।ಪೂ।ತ।ನಮಃ ।೬\\
\as{೪ ಪಂ} ಸರ್ವಮೃತ್ಯುಪ್ರಶಮನೀದೇವೀಶ್ರೀಪಾ।ಪೂ।ತ।ನಮಃ ।೭\\
\as{೪ ಫಂ} ಸರ್ವವಿಘ್ನನಿವಾರಿಣೀದೇವೀಶ್ರೀಪಾ।ಪೂ।ತ।ನಮಃ ।೮\\
\as{೪ ಬಂ} ಸರ್ವಾಂಗಸುಂದರೀದೇವೀಶ್ರೀಪಾ।ಪೂ।ತ।ನಮಃ ।೯\\
\as{೪ ಭಂ} ಸರ್ವಸೌಭಾಗ್ಯದಾಯಿನೀದೇವೀಶ್ರೀಪಾ।ಪೂ।ತ।ನಮಃ ।೧೦

\as{೪} ಏತಾಃ ಕುಲೋತ್ತೀರ್ಣಯೋಗಿನ್ಯಃ ಸರ್ವಾರ್ಥಸಾಧಕೇ ಚಕ್ರೇ ಸಮುದ್ರಾಃ ಸಸಿದ್ಧಯಃ ಸಾಯುಧಾಃ ಸಶಕ್ತಯಃ ಸವಾಹನಾಃ ಸಪರಿವಾರಾಃ ಸರ್ವೋಪಚಾರೈಃ ಸಂಪೂಜಿತಾಃ ಸಂತರ್ಪಿತಾಃ ಸಂತುಷ್ಟಾಃ ಸಂತು ನಮಃ ॥

\as{೪ ಹ್‌ಸೈಂ ಹ್‌ಸ್‌ಕ್ಲೀಂ ಹ್‌ಸ್ಸೌಃ ॥} ತ್ರಿಪುರಾಶ್ರೀಚಕ್ರೇಶ್ವರೀಶ್ರೀಪಾ।ಪೂ।ತ।ನಮಃ॥\\
\as{೪ ವಂ} ವಶಿತ್ವಸಿದ್ಧಿಶ್ರೀಪಾ।ಪೂ।ತ।ನಮಃ॥\\
\as{೪ ಸಃ }ಸರ್ವೋನ್ಮಾದಿನೀಮುದ್ರಾಶಕ್ತಿಶ್ರೀಪಾ।ಪೂ।ತ।ನಮಃ॥\\
\as{೪ ಸಃ॥}\\
ಮೂಲೇನ ತ್ರಿಃ ಸಂತರ್ಪ್ಯ
\subsection{ನಾಮಾವಲಿಃ}
ಕೃಷ್ಣಾಯೈ । ಹಿಮಾಚಲಕೃತಾಶ್ರಯಾಯೈ । ಧೂಮ್ರಲೋಚನಹಂತ್ರ್ಯೈ । ಅಸಿನ್ಯೈ । ಪಾಶಿನ್ಯೈ । ವಿಚಿತ್ರಖಟ್ವಾಂಗಧರಾಯೈ । ನರಮಾಲಾವಿಭೂಷಣಾಯೈ । ದ್ವೀಪಿಚರ್ಮಪರೀಧಾನಾಯೈ । ಶುಷ್ಕಮಾಂಸಾತಿಭೈರವಾಯೈ । ಅತಿವಿಸ್ತಾರವದನಾಯೈ । ಜಿಹ್ವಾಲಲನಭೀಷಣಾಯೈ । ನಿಮಗ್ನಾಯೈ । ರಕ್ತನಯನಾಯೈ । ನಾದಾಪೂರಿತದಿಙ್ಮುಖಾಯೈ । ಭೀಮಾಕ್ಷ್ಯೈ । ಭೈರವನಾದಿನ್ಯೈ । ಚಂಡಮುಂಡವಿನಾಶಿನ್ಯೈ । ಚಾಮುಂಡಾಯೈ । ಲೋಕವಿದ್ಯಾಯೈ । ಬ್ರಹ್ಮಾಣ್ಯೈ । ಬ್ರಹ್ಮವಾದಿನ್ಯೈ । ಮಹೇಶ್ವರ್ಯೈ । ವೃಷಾರೂಢಾಯೈ । ತ್ರಿಶೂಲವರಧಾರಿಣ್ಯೈ । ಮಹಾಹಿವಲಯಾಯೈ । ಚಂದ್ರರೇಖಾವಿಭೂಷಣಾಯೈ । ಕೌಮಾರ್ಯೈ । ಶಕ್ತಿಹಸ್ತಾಯೈ । ಮಯೂರವರವಾಹನಾಯೈ । ಗುಹರೂಪಿಣ್ಯೈ । ವೈಷ್ಣವ್ಯೈ । ಗರುಡೋಪರಿಸಂಸ್ಥಿತಾಯೈ । ಶಂಖಚಕ್ರಗದಾಶಾರ್ಙ್ಗಖಡ್ಗಹಸ್ತಾಯೈ । ವಾರಾಹ್ಯೈ । ನಾರಸಿಂಹ್ಯೈ । ಘೋರಾರಾವಾಯೈ । ಸಟಾಕ್ಷೇಪಕ್ಷಿಪ್ತನಕ್ಷತ್ರಸಂಹತ್ಯೈ । ವಜ್ರಹಸ್ತಾಯೈ । ಐಂದ್ರ್ಯೈ । ಗಜರಾಜೋಪರಿಸ್ಥಿತಾಯೈ । ಅಂಬಾಯೈ ॥\\
\as{೪ ತ್ರಿಪುರಾಶ್ರಿಯೈ ವಿದ್ಮಹೇ ಕಾಮೇಶ್ವರ್ಯೈ ಧೀಮಹಿ ।\\ತನ್ನಃ ಕ್ಲಿನ್ನಾ ಪ್ರಚೋದಯಾತ್ ॥}

\as{೪} ಅಭೀಷ್ಟಸಿದ್ಧಿಂ******ಪಂಚಮಾವರಣಾರ್ಚನಂ ॥

೪ ಕುಲೋತ್ತೀರ್ಣಯೋಗಿನೀ ಮಯೂಖಾಯೈ ಪಂಚಮಾವರಣದೇವತಾ ಸಹಿತಾಯೈ ಶ್ರೀಲಲಿತಾಮಹಾತ್ರಿಪುರಸುಂದರೀ ಪರಾಭಟ್ಟಾರಿಕಾಯೈ ನಮಃ ॥ ಇತಿ ಯೋನ್ಯಾ ಪ್ರಣಮೇತ್ ।
\subsection{ಷಷ್ಠಾವರಣಂ}
{\bfseries ೪ ಹ್ರೀಂ ಕ್ಲೀಂ ಬ್ಲೇಂ ಸರ್ವರಕ್ಷಾಕರ ಚಕ್ರಾಯ ನಮಃ}\\
\as{೪ ಮಂ} ಸರ್ವಜ್ಞಾದೇವೀಶ್ರೀಪಾ।ಪೂ।ತ।ನಮಃ ।೧\\
\as{೪ ಯಂ} ಸರ್ವಶಕ್ತಿದೇವೀಶ್ರೀಪಾ।ಪೂ।ತ।ನಮಃ ।೨\\
\as{೪ ರಂ} ಸರ್ವೈಶ್ವರ್ಯಪ್ರದಾದೇವೀಶ್ರೀಪಾ।ಪೂ।ತ।ನಮಃ ।೩\\
\as{೪ ಲಂ} ಸರ್ವಜ್ಞಾನಮಯೀದೇವೀಶ್ರೀಪಾ।ಪೂ।ತ।ನಮಃ ।೪\\
\as{೪ ವಂ} ಸರ್ವವ್ಯಾಧಿವಿನಾಶಿನೀದೇವೀಶ್ರೀಪಾ।ಪೂ।ತ।ನಮಃ ।೫\\
\as{೪ ಶಂ} ಸರ್ವಾಧಾರಸ್ವರೂಪಾದೇವೀಶ್ರೀಪಾ।ಪೂ।ತ।ನಮಃ ।೬\\
\as{೪ ಷಂ} ಸರ್ವಪಾಪಹರಾದೇವೀಶ್ರೀಪಾ।ಪೂ।ತ।ನಮಃ ।೭\\
\as{೪ ಸಂ} ಸರ್ವಾನಂದಮಯೀದೇವೀಶ್ರೀಪಾ।ಪೂ।ತ।ನಮಃ ।೮\\
\as{೪ ಹಂ} ಸರ್ವರಕ್ಷಾಸ್ವರೂಪಿಣೀದೇವೀಶ್ರೀಪಾ।ಪೂ।ತ।ನಮಃ ।೯\\
\as{೪ ಕ್ಷಂ} ಸರ್ವೇಪ್ಸಿತಫಲಪ್ರದಾದೇವೀಶ್ರೀಪಾ।ಪೂ।ತ।ನಮಃ ।೧೦

\as{೪} ಏತಾಃ ನಿಗರ್ಭಯೋಗಿನ್ಯಃ ಸರ್ವರಕ್ಷಾಕರೇ ಚಕ್ರೇ ಸಮುದ್ರಾಃ ಸಸಿದ್ಧಯಃ ಸಾಯುಧಾಃ ಸಶಕ್ತಯಃ ಸವಾಹನಾಃ ಸಪರಿವಾರಾಃ ಸರ್ವೋಪಚಾರೈಃ ಸಂಪೂಜಿತಾಃ ಸಂತರ್ಪಿತಾಃ ಸಂತುಷ್ಟಾಃ ಸಂತು ನಮಃ ॥

\as{೪ ಹ್ರೀಂ ಕ್ಲೀಂ ಬ್ಲೇಂ॥} ತ್ರಿಪುರಮಾಲಿನೀಚಕ್ರೇಶ್ವರೀಶ್ರೀಪಾ।ಪೂ।ತ।ನಮಃ॥\\
\as{೪ ಪಂ} ಪ್ರಾಕಾಮ್ಯಸಿದ್ಧಿಶ್ರೀಪಾ।ಪೂ।ತ।ನಮಃ॥\\
\as{೪ ಕ್ರೋಂ} ಸರ್ವಮಹಾಂಕುಶಾಮುದ್ರಾಶಕ್ತಿಶ್ರೀಪಾ।ಪೂ।ತ।ನಮಃ॥\\
\as{೪ ಕ್ರೋಂ॥}\\
ಮೂಲೇನ ತ್ರಿಃ ಸಂತರ್ಪ್ಯ
\subsection{ನಾಮಾವಲಿಃ}
ಸಹಸ್ರನಯನಾಯೈ । ಭೀಷಣಾಯೈ । ಚಂಡಿಕಾಶಕ್ತ್ಯೈ । ಅತ್ಯುಗ್ರಾಯೈ । ಶಿವಾಶತನಿನಾದಿನ್ಯೈ । ಅಪರಾಜಿತಾಯೈ । ಶಿವದೂತ್ಯೈ । ಕಾತ್ಯಾಯನ್ಯೈ । ರಕ್ತಬೀಜನಾಶಿನ್ಯೈ । ಚಂಡಘಂಟಿಕಾಯೈ । ಅಷ್ಟಾದಶಭುಜಾಯೈ । ಉಗ್ರಾಯೈ । ನಿಶುಂಭಾಸುರಘಾತಿನ್ಯೈ । ಶುಂಭಹಂತ್ರ್ಯೈ । ಪ್ರಪನ್ನಾರ್ತಿಹರಾಯೈ । ವಿಶ್ವೇಶ್ವರ್ಯೈ । ಜಗದಾಧಾರಭೂತಾಯೈ । ಮಹೀಸ್ವರೂಪಾಯೈ । ಅಪ್ಸ್ವರೂಪಾಯೈ । ಆಪ್ಯಾಯನ್ಯೈ । ಅಲಂಘ್ಯವೀರ್ಯಾಯೈ । ವೈಷ್ಣವೀಶಕ್ತ್ಯೈ । ಅನಂತವೀರ್ಯಾಯೈ । ವಿಶ್ವಬೀಜಸ್ವರೂಪಿಣ್ಯೈ । ಸಮಸ್ತಸಂಮೋಹಿನ್ಯೈ । ವಿದ್ಯಾಯೈ । ಭುಕ್ತಿಮುಕ್ತಿಪ್ರದಾಯಿನ್ಯೈ । ಅಶೇಷಜನಹೃತ್ಸಂಸ್ಥಿತಾಯೈ । ನಾರಾಯಣ್ಯೈ । ಸರ್ವಮಂಗಲಮಾಗಲ್ಯಾಯೈ । ಶಿವಾಯೈ । ಸರ್ವಾರ್ಥಸಾಧಿಕಾಯೈ । ಶರಣ್ಯಾಯೈ । ತ್ರ್ಯಂಬಿಕಾಯೈ । ಗೌರ್ಯೈ । ಕಲಾಕಾಷ್ಠಾದಿರೂಪಿಣ್ಯೈ । ಪರಿಣಾಮಪ್ರದಾಯಿನ್ಯೈ । ಸೃಷ್ಟಿಸ್ಥಿತಿಲಯಶಕ್ತ್ಯೈ ॥\\
\as{೪ ತ್ರಿಪುರಮಾಲಿನ್ಯೈ ವಿದ್ಮಹೇ ಕಾಮೇಶ್ವರ್ಯೈ ಧೀಮಹಿ ।\\ತನ್ನಃ ಕ್ಲಿನ್ನಾ ಪ್ರಚೋದಯಾತ್ ॥}

\as{೪} ಅಭೀಷ್ಟಸಿದ್ಧಿಂ*******ಷಷ್ಠಾಖ್ಯಾವರಣಾರ್ಚನಂ ॥

೪ ನಿಗರ್ಭಯೋಗಿನೀ ಮಯೂಖಾಯೈ ಷಷ್ಠಾವರಣದೇವತಾ ಸಹಿತಾಯೈ ಶ್ರೀಲಲಿತಾಮಹಾತ್ರಿಪುರಸುಂದರೀ ಪರಾಭಟ್ಟಾರಿಕಾಯೈ ನಮಃ ॥ ಇತಿ ಯೋನ್ಯಾ ಪ್ರಣಮೇತ್ ।
\subsection{ಸಪ್ತಮಾವರಣಂ}
{\bfseries ೪ ಹ್ರೀಂ ಶ್ರೀಂ ಸೌಃ ಸರ್ವರೋಗಹರ ಚಕ್ರಾಯ ನಮಃ}\\
\as{೪ ಅಂಆಂ++ಅಃ । ರ್ಬ್ಲೂಂ ॥}\\ ವಶಿನೀವಾಗ್ದೇವತಾಶ್ರೀಪಾ।ಪೂ।ತ।ನಮಃ ।೧\\
\as{೪ ಕಂಖಂಗಂಘಂಙಂ । ಕ್‌ಲ್‌ಹ್ರೀಂ ॥\\} ಕಾಮೇಶ್ವರೀವಾಗ್ದೇವತಾಶ್ರೀಪಾ।ಪೂ।ತ।ನಮಃ ।೨\\
\as{೪ ಚಂಛಂಜಂಝಂಞಂ । ನ್‌ವ್ಲೀಂ ॥\\} ಮೋದಿನೀವಾಗ್ದೇವತಾಶ್ರೀಪಾ।ಪೂ।ತ।ನಮಃ ।೩\\
\as{೪ ಟಂಠಂಡಂಢಂಣಂ । ಯ್ಲೂಂ ॥\\} ವಿಮಲಾವಾಗ್ದೇವತಾಶ್ರೀಪಾ।ಪೂ।ತ।ನಮಃ ।೪\\
\as{೪ ತಂಥಂದಂಧಂನಂ । ಜ್‌ಮ್ರೀಂ ॥\\} ಅರುಣಾವಾಗ್ದೇವತಾಶ್ರೀಪಾ।ಪೂ।ತ।ನಮಃ ।೫\\
\as{೪ ಪಂಫಂಬಂಭಂಮಂ । ಹ್‌ಸ್‌ಲ್‌ವ್ಯೂಂ ॥\\} ಜಯಿನೀವಾಗ್ದೇವತಾಶ್ರೀಪಾ।ಪೂ।ತ।ನಮಃ ।೬\\
\as{೪ ಯಂರಂಲಂವಂ । ಝ್‌ಮ್‌ರ್ಯೂಂ ॥\\} ಸರ್ವೇಶ್ವರೀವಾಗ್ದೇವತಾಶ್ರೀಪಾ।ಪೂ।ತ।ನಮಃ ।೭\\
\as{೪ ಶಂಷಂಸಂಹಂಳಂಕ್ಷಂ । ಕ್ಷ್‌ಮ್ರೀಂ ॥\\} ಕೌಳಿನೀವಾಗ್ದೇವತಾಶ್ರೀಪಾ।ಪೂ।ತ।ನಮಃ ।೮

\as{೪} ಏತಾಃ ರಹಸ್ಯಯೋಗಿನ್ಯಃ ಸರ್ವರೋಗಹರೇ ಚಕ್ರೇ ಸಮುದ್ರಾಃ ಸಸಿದ್ಧಯಃ ಸಾಯುಧಾಃ ಸಶಕ್ತಯಃ ಸವಾಹನಾಃ ಸಪರಿವಾರಾಃ ಸರ್ವೋಪಚಾರೈಃ ಸಂಪೂಜಿತಾಃ ಸಂತರ್ಪಿತಾಃ ಸಂತುಷ್ಟಾಃ ಸಂತು ನಮಃ ॥

\as{೪ ಹ್ರೀಂ ಶ್ರೀಂ ಸೌಃ॥} ತ್ರಿಪುರಾಸಿದ್ಧಾಚಕ್ರೇಶ್ವರೀಶ್ರೀಪಾ।ಪೂ।ತ।ನಮಃ॥\\
\as{೪ ಭುಂ} ಭುಕ್ತಿಸಿದ್ಧಿಶ್ರೀಪಾ।ಪೂ।ತ।ನಮಃ॥\\
\as{೪ ಹ್‌ಸ್‌ಖ್‌ಫ್ರೇಂ} ಸರ್ವಖೇಚರೀಮುದ್ರಾಶಕ್ತಿಶ್ರೀಪಾ।ಪೂ।ತ।ನಮಃ॥\\
\as{೪ ಹ್‌ಸ್‌ಖ್‌ಫ್ರೇಂ ॥}\\
ಮೂಲೇನ ತ್ರಿಃ ಸಂತರ್ಪ್ಯ
\subsection{ನಾಮಾವಲಿಃ}
ಸನಾತನ್ಯೈ । ಗುಣಾಶ್ರಯಾಯೈ । ಗುಣಮಯ್ಯೈ । ನಾರಾಯಣ್ಯೈ । ಶರಣಾಗತದೀನಾರ್ತಪರಿತ್ರಾಣಪರಾಯಣಾಯೈ । ಸರ್ವಸ್ಯಾರ್ತಿಹರಾಯೈ । ದೇವ್ಯೈ । ಹಂಸಯುಕ್ತವಿಮಾನಸ್ಥಾಯೈ । ಬ್ರಹ್ಮಾಣೀರೂಪಧಾರಿಣ್ಯೈ । ಕೌಶಾಂಭಃಕ್ಷರಿಕಾಯೈ । ತ್ರಿಶೂಲಚಂದ್ರಾಹಿಧರಾಯೈ । ಮಹಾವೃಷಭವಾಹಿನ್ಯೈ । ಮಾಹೇಶ್ವರ್ಯೈ । ತ್ರೈಲೋಕ್ಯತ್ರಾಣಸಹಿತಾಯೈ । ಕಿರೀಟವಜ್ರಧಾರಿಣ್ಯೈ । ವೃತ್ರಪ್ರಾಣಹರಾಯೈ । ಶಿವದೂತೀಸ್ವರೂಪಿಣ್ಯೈ । ಹತದೈತ್ಯಾಯೈ । ಮಹಾಬಲಾಯೈ । ಘೋರರೂಪಾಯೈ । ಮಹಾರಾವಾಯೈ । ದಂಷ್ಟ್ರಾಕರಾಲವದನಾಯೈ । ಶಿರೋಮಾಲಾವಿಭೂಷಣಾಯೈ । ಚಾಮುಂಡಾಯೈ । ಮುಂಡಮಥನಾಯೈ । ಲಕ್ಷ್ಮ್ಯೈ । ಲಜ್ಜಾಯೈ । ಮಹಾವಿದ್ಯಾಯೈ । ಶ್ರದ್ಧಾಯೈ । ಪುಷ್ಟ್ಯೈ । ಸ್ವಧಾಯೈ । ಧ್ರುವಾಯೈ । ಮಹಾರಾತ್ರ್ಯೈ । ಮಹಾಮಾಯಾಯೈ । ಮೇಧಾಯೈ । ಸರಸ್ವತ್ಯೈ । ವರಾಯೈ । ಭೂತ್ಯೈ । ತಾಮಸ್ಯೈ । ನಿಯತಾಯೈ ॥\\
\as{೪ ತ್ರಿಪುರಾಸಿದ್ಧಾಯೈ ವಿದ್ಮಹೇ ಕಾಮೇಶ್ವರ್ಯೈ ಧೀಮಹಿ ।\\ತನ್ನಃ ಕ್ಲಿನ್ನಾ ಪ್ರಚೋದಯಾತ್ ॥}

\as{೪} ಅಭೀಷ್ಟಸಿದ್ಧಿಂ********ಸಪ್ತಮಾವರಣಾರ್ಚನಂ ॥

೪ ರಹಸ್ಯಯೋಗಿನೀ ಮಯೂಖಾಯೈ ಸಪ್ತಮಾವರಣದೇವತಾ ಸಹಿತಾಯೈ ಶ್ರೀಲಲಿತಾಮಹಾತ್ರಿಪುರಸುಂದರೀ ಪರಾಭಟ್ಟಾರಿಕಾಯೈ ನಮಃ ॥ ಇತಿ ಯೋನ್ಯಾ ಪ್ರಣಮೇತ್ ।
\subsection{ಅಷ್ಟಮಾವರಣಂ}
\as{೪ ಯಾಂರಾಂಲಾಂವಾಂಸಾಂ ದ್ರಾಂದ್ರೀಂಕ್ಲೀಂಬ್ಲೂಂಸಃ ।} ಸರ್ವಜಂಭನೇಭ್ಯೋ\\ ಕಾಮೇಶ್ವರೀಕಾಮೇಶ್ವರಬಾಣೇಭ್ಯೋ ನಮಃ ।\\ ಬಾಣಶಕ್ತಿಶ್ರೀಪಾ।ಪೂ।ತ।ನಮಃ।೧\\
\as{೪ ಥಂಧಂ ।} ಸರ್ವಸಂಮೋಹನಾಭ್ಯಾಂ ಕಾಮೇಶ್ವರೀಕಾಮೇಶ್ವರ ಧನುರ್ಭ್ಯಾಂ\\ ನಮಃ ।ಧನುಃಶಕ್ತಿಶ್ರೀಪಾ।ಪೂ।ತ।ನಮಃ।೨\\
\as{೪ ಹ್ರೀಂಆಂ ।} ಸರ್ವವಶೀಕರಣಾಭ್ಯಾಂ ಕಾಮೇಶ್ವರೀಕಾಮೇಶ್ವರ ಪಾಶಾಭ್ಯಾಂ\\ ನಮಃ ।ಪಾಶಶಕ್ತಿಶ್ರೀಪಾ।ಪೂ।ತ।ನಮಃ।೩\\
\as{೪ ಕ್ರೋಂಕ್ರೋಂ ।} ಸರ್ವಸ್ತಂಭನಾಭ್ಯಾಂ ಕಾಮೇಶ್ವರೀಕಾಮೇಶ್ವರ\\ಅಂಕುಶಾಭ್ಯಾಂ ನಮಃ ।ಅಂಕುಶಶಕ್ತಿಶ್ರೀಪಾ।ಪೂ।ತ।ನಮಃ।೪

\as{೪ ಹ್‌ಸ್‌ರೈಂ ಹ್‌ಸ್‌ಕ್ಲ್ರೀಂ ಹ್‌ಸ್‌ರ್ಸೌಃ ಸರ್ವಸಿದ್ಧಿಪ್ರದ ಚಕ್ರಾಯ ನಮಃ}\\
{\bfseries ೪ ಐಂ ೫॥} ಸೂರ್ಯಚಕ್ರೇ ಕಾಮಗಿರಿಪೀಠೇ ಮಿತ್ರೀಶನಾಥ ನವಯೋನಿ ಚಕ್ರಾತ್ಮಕ ಆತ್ಮತತ್ವ ಸಂಹಾರಕೃತ್ಯ ಜಾಗ್ರದ್ದಶಾಧಿಷ್ಠಾಯಕ ಇಚ್ಛಾಶಕ್ತಿ ವಾಗ್ಭವಾತ್ಮಕ ಪರಾಪರಶಕ್ತಿ ಸ್ವರೂಪ ರುದ್ರಾತ್ಮಶಕ್ತಿ ಮಹಾಕಾಮೇಶ್ವರೀ ಶ್ರೀಪಾದುಕಾಂ ಪೂ।ತ।ನಮಃ॥೧

{\bfseries೪ ಕ್ಲೀಂ ೬॥} ಸೋಮಚಕ್ರೇ ಪೂರ್ಣಗಿರಿಪೀಠೇ ಉಡ್ಡೀಶನಾಥ  ದಶಾರದ್ವಯ ಚತುರ್ದಶಾರ ಚಕ್ರಾತ್ಮಕ ವಿದ್ಯಾತತ್ವ ಸ್ಥಿತಿಕೃತ್ಯ ಸ್ವಪ್ನದಶಾಧಿಷ್ಠಾಯಕ ಜ್ಞಾನಶಕ್ತಿ ಕಾಮರಾಜಾತ್ಮಕ ಕಾಮಕಲಾ ಸ್ವರೂಪ ವಿಷ್ಣ್ವಾತ್ಮಶಕ್ತಿ ಮಹಾವಜ್ರೇಶ್ವರೀ ಶ್ರೀಪಾದುಕಾಂ ಪೂ।ತ।ನಮಃ॥೨

{\bfseries೪ ಸೌಃ ೪॥} ಅಗ್ನಿಚಕ್ರೇ ಜಾಲಂಧರಪೀಠೇ  ಷಷ್ಠೀಶನಾಥ ಅಷ್ಟದಳ ಷೋಡಶದಳ ಚತುರಸ್ರ ಚಕ್ರಾತ್ಮಕ ಶಿವತತ್ವ ಸೃಷ್ಟಿಕೃತ್ಯ ಸುಷುಪ್ತಿದಶಾಧಿಷ್ಠಾಯಕ ಕ್ರಿಯಾಶಕ್ತಿ ಶಕ್ತಿಬೀಜಾತ್ಮಕ ವಾಗೀಶ್ವರೀ ಸ್ವರೂಪ ಬ್ರಹ್ಮಾತ್ಮಶಕ್ತಿ  ಮಹಾಭಗಮಾಲಿನೀ ಶ್ರೀಪಾದುಕಾಂ ಪೂ।ತ।ನಮಃ॥೩

{\bfseries೪ ೧೫॥} ಪರಬ್ರಹ್ಮಚಕ್ರೇ ಮಹೋಡ್ಯಾಣಪೀಠೇ ಚರ್ಯಾನಂದನಾಥ ಸಮಸ್ತಚಕ್ರಾತ್ಮಕ ಸಪರಿವಾರ ಪರಮತತ್ವ ಸೃಷ್ಟಿಸ್ಥಿತಿಸಂಹಾರಕೃತ್ಯ ತುರೀಯದಶಾಧಿಷ್ಠಾಯಕ ಇಚ್ಛಾಜ್ಞಾನಕ್ರಿಯಾಶಾಂತಶಕ್ತಿ ವಾಗ್ಭವ ಕಾಮರಾಜ ಶಕ್ತಿಬೀಜಾತ್ಮಕ ಪರಮಶಕ್ತಿಸ್ವರೂಪ ಪರಬ್ರಹ್ಮಾತ್ಮಶಕ್ತಿ ಶ್ರೀಮಹಾತ್ರಿಪುರಸುಂದರೀ ಶ್ರೀಪಾದುಕಾಂ ಪೂ।ತ।ನಮಃ॥(ಬಿಂದೌ)

\as{೪} ಏತಾಃ ಅತಿರಹಸ್ಯಯೋಗಿನ್ಯಃ ಸರ್ವಸಿದ್ಧಿಪ್ರದೇ ಚಕ್ರೇ ಸಮುದ್ರಾಃ ಸಸಿದ್ಧಯಃ ಸಾಯುಧಾಃ ಸಶಕ್ತಯಃ ಸವಾಹನಾಃ ಸಪರಿವಾರಾಃ ಸರ್ವೋಪಚಾರೈಃ ಸಂಪೂಜಿತಾಃ ಸಂತರ್ಪಿತಾಃ ಸಂತುಷ್ಟಾಃ ಸಂತು ನಮಃ ॥

\as{೪ ಹ್‌ಸ್‌ರೈಂ ಹ್‌ಸ್‌ಕ್ಲ್ರೀಂ ಹ್‌ಸ್‌ರ್ಸೌಃ ॥}\\ ತ್ರಿಪುರಾಂಬಾ ಚಕ್ರೇಶ್ವರೀಶ್ರೀಪಾ।ಪೂ।ತ।ನಮಃ॥\\
\as{೪ ಇಂ} ಇಚ್ಛಾಸಿದ್ಧಿಶ್ರೀಪಾ।ಪೂ।ತ।ನಮಃ॥\\
\as{೪ ಹ್ಸೌಃ} ಸರ್ವಬೀಜಮುದ್ರಾಶಕ್ತಿಶ್ರೀಪಾ।ಪೂ।ತ।ನಮಃ॥\\
\as{೪ ಹ್ಸೌಃ॥}\\
ಮೂಲೇನ ತ್ರಿಃ ಸಂತರ್ಪ್ಯ
\subsection{ನಾಮಾವಲಿಃ}
ಸರ್ವತಃಪಾಣಿಪಾದಾಂತಾಯೈ । ಸರ್ವತೋಕ್ಷಿಶಿರೋಮುಖಾಯೈ । ಸರ್ವತಃಶ್ರವಣಘ್ರಾಣಾಯೈ । ಸರ್ವಸ್ವರೂಪಾಯೈ । ಸರ್ವೇಶಾಯೈ । ಸರ್ವಶಕ್ತಿಸಮನ್ವಿತಾಯೈ । ಸಮಸ್ತರೋಗಹಂತ್ರ್ಯೈ । ಸಮಸ್ತಾಭೀಷ್ಟದಾಯಿನ್ಯೈ । ವಿಶ್ವಾತ್ಮಿಕಾಯೈ । ವಿಶ್ವೇಶವಂದ್ಯಾಯೈ । ಪಾಪಹಾರಿಣ್ಯೈ । ಉತ್ಪಾತಪಾಕಜನಿತೋಪಸರ್ಗಚಯನಾಶಿನ್ಯೈ । ವಿಶ್ವಾರ್ತಿಹಾರಿಣ್ಯೈ । ತ್ರೈಲೋಕ್ಯವರದಾಯಿನ್ಯೈ । ನಂದಗೋಪಗೃಹಜಾತಾಯೈ । ಪುತ್ರಪೌತ್ರಪ್ರವರ್ತಿನ್ಯೈ । ಯಶೋದಾಗರ್ಭಸಂಭವಾಯೈ । ವಿಂಧ್ಯಾಚಲವಾಸಿನ್ಯೈ । ರೌದ್ರರೂಪಿಣ್ಯೈ । ರಕ್ತದಂತಿಕಾಯೈ । ದಾಡಿಮೀಕುಸುಮಾಭದಂತಾಯೈ । ಅಯೋನಿಜಾಯೈ । ಶತಾಕ್ಷ್ಯೈ । ಭೀಮಾಯೈ । ಶಾಕಂಭರ್ಯೈ । ದುರ್ಗಾಯೈ । ದಾನವೇಂದ್ರವಿನಾಶಿನ್ಯೈ । ಭೀಮಾದೇವ್ಯೈ । ಭ್ರಾಮರ್ಯೈ । ಮಹಾಕಾಲ್ಯೈ । ಮಹಾಮಾರ್ಯೈ । ಅಜಾಯೈ । ವೃದ್ಧಿಪ್ರದಲಕ್ಷ್ಮ್ಯೈ । ನಿತ್ಯಾಯೈ । ಶೈಲಪುತ್ರ್ಯೈ । ಬ್ರಹ್ಮಚಾರಿಣ್ಯೈ । ಚಂಡಘಂಟಾಯೈ । ವಿಶಾಲಾಕ್ಷ್ಯೈ । ಕೂಷ್ಮಾಂಡಾಯೈ । ವೇದಮಾತೃಕಾಯೈ ॥\\
\as{೪ ತ್ರಿಪುರಾಂಬಾಯೈ ವಿದ್ಮಹೇ ಕಾಮೇಶ್ವರ್ಯೈ ಧೀಮಹಿ ।\\ತನ್ನಃ ಕ್ಲಿನ್ನಾ ಪ್ರಚೋದಯಾತ್ ॥}

\as{೪} ಅಭೀಷ್ಟಸಿದ್ಧಿಂ*******ಅಷ್ಟಮಾವರಣಾರ್ಚನಂ ॥

೪ ಅತಿರಹಸ್ಯಯೋಗಿನೀ ಮಯೂಖಾಯೈ ಅಷ್ಟಮಾವರಣದೇವತಾ ಸಹಿತಾಯೈ ಶ್ರೀಲಲಿತಾಮಹಾತ್ರಿಪುರಸುಂದರೀ ಪರಾಭಟ್ಟಾರಿಕಾಯೈ ನಮಃ ॥ ಇತಿ ಯೋನ್ಯಾ ಪ್ರಣಮೇತ್ ।
\subsection{ನವಮಾವರಣಂ}
{\bfseries ೪ (೧೫) ಸರ್ವಾನಂದಮಯ ಚಕ್ರಾಯ ನಮಃ}\\
{\bfseries ೪ (೧೫)} ಲಲಿತಾ ಮಹಾತ್ರಿಪುರಸುಂದರೀ\\ ಪರಾಭಟ್ಟಾರಿಕಾಶ್ರೀಪಾ।ಪೂ।ತ।ನಮಃ॥ಇತಿ ಬಿಂದೌ ತ್ರಿಃ ಸಂತರ್ಪ್ಯ

\as{೪} ಏಷಾ ಪರಾಪರಾತಿರಹಸ್ಯಯೋಗಿನೀ ಸರ್ವಾನಂದಮಯೇ ಚಕ್ರೇ ಸಮುದ್ರಾ ಸಸಿದ್ಧಿಃ ಸಾಯುಧಾ ಸಶಕ್ತಿಃ ಸವಾಹನಾ ಸಪರಿವಾರಾ ಸರ್ವೋಪಚಾರೈಃ ಸಂಪೂಜಿತಾ ಸಂತರ್ಪಿತಾ ಸಂತುಷ್ಟಾ ಅಸ್ತು ನಮಃ ॥

\as{೪ (೧೫) }ಮಹಾತ್ರಿಪುರಸುಂದರೀ ಚಕ್ರೇಶ್ವರೀಶ್ರೀಪಾ।ಪೂ।ತ।ನಮಃ॥\\
\as{೪ ಪಂ} ಪ್ರಾಪ್ತಿಸಿದ್ಧಿಶ್ರೀಪಾ।ಪೂ।ತ।ನಮಃ॥\\
\as{೪ ಐಂ} ಸರ್ವಯೋನಿಮುದ್ರಾಶಕ್ತಿಶ್ರೀಪಾ।ಪೂ।ತ।ನಮಃ॥\\
\as{೪ ಐಂ ॥}\\
ಮೂಲೇನ ತ್ರಿಃ ಸಂತರ್ಪ್ಯ

[\,ಷೋಡಶ್ಯುಪಾಸಕಾನಾಂ ಕೃತೇ:\\ 
\as{೪ ಹಸಕಲ ಹಸಕಹಲ ಸಕಲಹ್ರೀಂ ॥} ತುರೀಯಾಂಬಾಶ್ರೀಪಾ।ಪೂ।ತ।ನಮಃ॥ (ಇತಿ ತ್ರಿಃ ಸಂತರ್ಪ್ಯ)\\
\as{೪} ಸರ್ವಾನಂದಮಯೇ ಚಕ್ರೇ ಮಹೋಡ್ಯಾಣಪೀಠೇ ಚರ್ಯಾನಂದನಾಥಾತ್ಮಕ ತುರೀಯಾತೀತದಶಾಧಿಷ್ಠಾಯಕ ಶಾಂತ್ಯತೀತಕಲಾತ್ಮಕ ಪ್ರಕಾಶ ವಿಮರ್ಶ ಸಾಮರಸ್ಯಾತ್ಮಕ ಪರಬ್ರಹ್ಮಸ್ವರೂಪಿಣೀ ಪರಾಮೃತಶಕ್ತಿಃ ಸರ್ವ ಮಂತ್ರೇಶ್ವರೀ ಸರ್ವಪೀಠೇಶ್ವರೀ ಸರ್ವವೀರೇಶ್ವರೀ ಸಕಲಜಗದುತ್ಪತ್ತಿ ಮಾತೃಕಾ ಸಚಕ್ರಾ ಸದೇವತಾ ಸಾಸನಾ ಸಾಯುಧಾ ಸಶಕ್ತಿಃ ಸವಾಹನಾ ಸಪರಿವಾರಾ ಸಚಕ್ರೇಶೀಕಾ ಪರಯಾ ಅಪರಯಾ ಪರಾಪರಯಾ ಸಪರ್ಯಯಾ ಸರ್ವೋಪಚಾರೈಃ ಸಂಪೂಜಿತಾ ಸಂತರ್ಪಿತಾ ಸಂತುಷ್ಟಾ ಅಸ್ತು ನಮಃ ॥(ಇತಿ ಸಮಷ್ಟ್ಯಂಜಲಿಃ)

\as{೪ ಸಂ} ಸರ್ವಕಾಮಸಿದ್ಧಿಶ್ರೀಪಾ।ಪೂ।ತ।ನಮಃ॥\\
\as{೪ ಹ್‌ಸ್‌ರೈಂ ಹ್‌ಸ್‌ಕ್ಲ್ರೀಂ ಹ್‌ಸ್‌ರ್ಸೌಃ} ಸರ್ವತ್ರಿಖಂಡಾ ಮುದ್ರಾಶಕ್ತಿಶ್ರೀಪಾ।ಪೂ।ತ।ನಮಃ॥\\
\as{೪ ಹ್‌ಸ್‌ರೈಂ ಹ್‌ಸ್‌ಕ್ಲ್ರೀಂ ಹ್‌ಸ್‌ರ್ಸೌಃ ॥}\\
\as{೪ (೧೬)} ಮಹಾತ್ರಿಪುರಸುಂದರೀಶ್ರೀಪಾ।ಪೂ।ತ।ನಮಃ॥(ಇತಿ ತ್ರಿಃ)]\,
\subsection{ನಾಮಾವಲಿಃ}
ಸ್ಕಂದಮಾತ್ರೇ । ಗಣೇಶ್ವರ್ಯೈ । ವಿರೂಪಾಕ್ಷ್ಯೈ । ಅಂಬಿಕಾಯೈ । ಮಹಾಗೌರ್ಯೈ । ಮಹಾವೀರ್ಯಾಯೈ । ಮಹಾಬಲಪರಾಕ್ರಮಾಯೈ । ಮಯೂರಕುಕ್ಕುಟವೃತಾಯೈ । ಮಹಾಶಕ್ತಿಧರಾಯೈ । ಅನಘಾಯೈ । ಬ್ರಾಹ್ಮ್ಯೈ । ಮಾಹೇಶ್ವರ್ಯೈ । ಕೌಮಾರ್ಯೈ । ಶಂಖಚಕ್ರಗದಾಶಾರ್ಙ್ಗಗೃಹೀತಪರಮಾಯುಧಾಯೈ । ವೈಷ್ಣವ್ಯೈ । ಗೃಹೀತೋಗ್ರಮಹಾಚಕ್ರಾಯೈ । ದಂಷ್ಟ್ರೋದ್ಧೃತವಸುಂಧರಾಯೈ । ವರಾಹರೂಪಿಣ್ಯೈ । ಶಿವಾಯೈ । ನೃಸಿಂಹರೂಪಿಣ್ಯೈ । ಐಂದ್ರ್ಯೈ । ಹತದೈತ್ಯಮಹಾಬಲಾಯೈ । ಬಾಭ್ರವ್ಯೈ । ಲೋಚನತ್ರಯಭೂಷಿತಾಯೈ । ಸರ್ವಭೀತಿಹರಾಯೈ । ಕಾತ್ಯಾಯನ್ಯೈ । ತ್ರಿಶೂಲಧಾರಿಣ್ಯೈ । ಭದ್ರಕಾಲ್ಯೈ । ಚಂಡಿಕಾಯೈ । ಭ್ರಾಮರ್ಯೈ । ಭಗವತ್ಯೈ । ಚಂಡವಿಕ್ರಮಾಯೈ । ಸನಾತನ್ಯೈ । ಸ್ವರ್ಗಾಪವರ್ಗದಾಯೈ । ಮಹೀಮಯ್ಯೈ । ಜಗದ್ಧಾತ್ರ್ಯೈ । ಅನೇಕಮೂರ್ತ್ಯೈ । ವಿಶ್ವೇಶ್ವರ್ಯೈ । ವಿಶ್ವಾಶ್ರಯಾಯೈ । ಸಪ್ತಶತೀಶಕ್ತಿದೇವತಾಯೈ ॥\\
\as{೪ ತ್ರಿಪುರಸುಂದರ್ಯೈ ವಿದ್ಮಹೇ ಕಾಮೇಶ್ವರ್ಯೈ ಧೀಮಹಿ ।\\ತನ್ನಃ ಕ್ಲಿನ್ನಾ ಪ್ರಚೋದಯಾತ್ ॥}

 \as{೪} ಅಭೀಷ್ಟಸಿದ್ಧಿಂ******ನವಮಾವರಣಾರ್ಚನಂ ॥\\

೪ ಪರಾಪರಾತಿರಹಸ್ಯಯೋಗಿನೀ ಮಯೂಖಾಯೈ ನವಮಾವರಣದೇವತಾ ಸಹಿತಾಯೈ ಶ್ರೀಲಲಿತಾಮಹಾತ್ರಿಪುರಸುಂದರೀ ಪರಾಭಟ್ಟಾರಿಕಾಯೈ ನಮಃ ॥ ಇತಿ ಯೋನ್ಯಾ ಪ್ರಣಮೇತ್ ।
\authorline{ಇತ್ಯಾವರಣಾರ್ಚನಂ}
\section{ಪಂಚಪಂಚಿಕಾಪೂಜಾ}
\subsection{ಪಂಚಲಕ್ಷ್ಮ್ಯಂಬಾಃ}
{\bfseries ೪ ೧೫ ॥}\\
\as{ ಶ್ರೀಮಹಾಲಕ್ಷ್ಮೀಶ್ವರೀ ಬೃಂದಮಂಡಿತಾಸನ ಸಂಸ್ಥಿತಾ~।\\
ಸರ್ವಸೌಭಾಗ್ಯ ಜನನೀ ಶ್ರೀಮಹಾ ತ್ರಿಪುರಸುಂದರೀ ॥}\\
ಶ್ರೀವಿದ್ಯಾ ಲಕ್ಷ್ಮ್ಯಂಬಾಶ್ರೀಪಾ।ಪೂ।ತ।ನಮಃ॥

\as{೪ ಶ್ರೀಂ ॥}\\
\as{ ಶ್ರೀಮಹಾಲಕ್ಷ್ಮೀಶ್ವರೀ ಬೃಂದಮಂಡಿತಾಸನ ಸಂಸ್ಥಿತಾ~।\\
ಸರ್ವಸೌಭಾಗ್ಯ ಜನನೀ ಶ್ರೀಮಹಾ ತ್ರಿಪುರಸುಂದರೀ ॥}\\
ಲಕ್ಷ್ಮೀಲಕ್ಷ್ಮ್ಯಂಬಾಶ್ರೀಪಾ।ಪೂ।ತ।ನಮಃ॥೧

\as{೪ ಓಂಶ್ರೀಂ ಹ್ರೀಂ ಶ್ರೀಂ ಕಮಲೇ ಕಮಲಾಲಯೇ ಪ್ರಸೀದ ಪ್ರಸೀದ ಶ್ರೀಂ ಹ್ರೀಂ ಶ್ರೀಂ ಓಂ ಮಹಾಲಕ್ಷ್ಮ್ಯೈ ನಮಃ ॥}\\
\as{ ಶ್ರೀಮಹಾಲಕ್ಷ್ಮೀಶ್ವರೀ ಬೃಂದಮಂಡಿತಾಸನ ಸಂಸ್ಥಿತಾ~।\\
ಸರ್ವಸೌಭಾಗ್ಯ ಜನನೀ ಶ್ರೀಮಹಾ ತ್ರಿಪುರಸುಂದರೀ ॥}\\
ಮಹಾಲಕ್ಷ್ಮೀಲಕ್ಷ್ಮ್ಯಂಬಾಶ್ರೀಪಾ।ಪೂ।ತ।ನಮಃ॥೨

\as{೪ ಶ್ರೀಂ ಹ್ರೀಂ ಕ್ಲೀಂ ॥}\\
\as{ ಶ್ರೀಮಹಾಲಕ್ಷ್ಮೀಶ್ವರೀ ಬೃಂದಮಂಡಿತಾಸನ ಸಂಸ್ಥಿತಾ~।\\
ಸರ್ವಸೌಭಾಗ್ಯ ಜನನೀ ಶ್ರೀಮಹಾ ತ್ರಿಪುರಸುಂದರೀ ॥}\\
ತ್ರಿಶಕ್ತಿಲಕ್ಷ್ಮ್ಯಂಬಾಶ್ರೀಪಾ।ಪೂ।ತ।ನಮಃ॥೩

\as{೪ ಶ್ರೀಂ ಸಹಕಲ ಹ್ರೀಂ ಶ್ರೀಂ ॥} \\
\as{ ಶ್ರೀಮಹಾಲಕ್ಷ್ಮೀಶ್ವರೀ ಬೃಂದಮಂಡಿತಾಸನ ಸಂಸ್ಥಿತಾ~।\\
ಸರ್ವಸೌಭಾಗ್ಯ ಜನನೀ ಶ್ರೀಮಹಾ ತ್ರಿಪುರಸುಂದರೀ ॥}\\
ಸರ್ವಸಾಮ್ರಾಜ್ಯಲಕ್ಷ್ಮ್ಯಂಬಾಶ್ರೀಪಾ।ಪೂ।ತ।ನಮಃ॥೪
\subsection{ಪಂಚಕೋಶಾಂಬಾಃ}
{\bfseries ೪ ೧೫ ॥}\\
\as{ ಶ್ರೀಮಹಾಕೋಶೇಶ್ವರೀ ಬೃಂದಮಂಡಿತಾಸನ ಸಂಸ್ಥಿತಾ~।\\
ಸರ್ವಸೌಭಾಗ್ಯ ಜನನೀ ಶ್ರೀಮಹಾ ತ್ರಿಪುರಸುಂದರೀ ॥}\\
ಶ್ರೀವಿದ್ಯಾ ಕೋಶಾಂಬಾಶ್ರೀಪಾ।ಪೂ।ತ।ನಮಃ॥

\as{೪ ಓಂ ಹ್ರೀಂ ಹಂಸಃ ಸೋಹಂ ಸ್ವಾಹಾ ॥}\\
\as{ ಶ್ರೀಮಹಾಕೋಶೇಶ್ವರೀ ಬೃಂದಮಂಡಿತಾಸನ ಸಂಸ್ಥಿತಾ~।\\
ಸರ್ವಸೌಭಾಗ್ಯ ಜನನೀ ಶ್ರೀಮಹಾ ತ್ರಿಪುರಸುಂದರೀ ॥}\\
ಪರಂಜ್ಯೋತಿಃಕೋಶಾಂಬಾಶ್ರೀಪಾ।ಪೂ।ತ।ನಮಃ॥೧

\as{೪ ಓಂ ಹಂಸಃ ॥} \\
\as{ ಶ್ರೀಮಹಾಕೋಶೇಶ್ವರೀ ಬೃಂದಮಂಡಿತಾಸನ ಸಂಸ್ಥಿತಾ~।\\
ಸರ್ವಸೌಭಾಗ್ಯ ಜನನೀ ಶ್ರೀಮಹಾ ತ್ರಿಪುರಸುಂದರೀ ॥}\\
ಪರಾನಿಷ್ಕಲಾ ಕೋಶಾಂಬಾಶ್ರೀಪಾ।ಪೂ।ತ।ನಮಃ॥೨

\as{೪ ಹಂಸಃ ॥}\\
\as{ ಶ್ರೀಮಹಾಕೋಶೇಶ್ವರೀ ಬೃಂದಮಂಡಿತಾಸನ ಸಂಸ್ಥಿತಾ~।\\
ಸರ್ವಸೌಭಾಗ್ಯ ಜನನೀ ಶ್ರೀಮಹಾ ತ್ರಿಪುರಸುಂದರೀ ॥}\\
ಅಜಪಾಕೋಶಾಂಬಾಶ್ರೀಪಾ।ಪೂ।ತ।ನಮಃ॥೩

\as{೪ ಅಂ ಆಂ ಇಂ ಈಂ++++ಳಂ ಕ್ಷಂ ॥}\\
\as{ ಶ್ರೀಮಹಾಕೋಶೇಶ್ವರೀ ಬೃಂದಮಂಡಿತಾಸನ ಸಂಸ್ಥಿತಾ~।\\
ಸರ್ವಸೌಭಾಗ್ಯ ಜನನೀ ಶ್ರೀಮಹಾ ತ್ರಿಪುರಸುಂದರೀ ॥}\\
ಮಾತೃಕಾಕೋಶಾಂಬಾಶ್ರೀಪಾ।ಪೂ।ತ।ನಮಃ॥೪
\subsection{ಪಂಚಕಲ್ಪಲತಾಂಬಾಃ}
{\bfseries ೪ ೧೫ ॥}\\
\as{ಶ್ರೀಮಹಾಕಲ್ಪಲತೇಶ್ವರೀ ಬೃಂದಮಂಡಿತಾಸನ ಸಂಸ್ಥಿತಾ~।\\
ಸರ್ವಸೌಭಾಗ್ಯ ಜನನೀ ಶ್ರೀಮಹಾ ತ್ರಿಪುರಸುಂದರೀ ॥}\\
ಶ್ರೀವಿದ್ಯಾ ಕಲ್ಪಲತಾಂಬಾಶ್ರೀಪಾ।ಪೂ।ತ।ನಮಃ॥

\as{೪ ಹ್ರೀಂ ಕ್ಲೀಂ ಐಂ ಬ್ಲೂಂ ಸ್ತ್ರೀಂ ॥}\\
\as{ಶ್ರೀಮಹಾಕಲ್ಪಲತೇಶ್ವರೀ ಬೃಂದಮಂಡಿತಾಸನ ಸಂಸ್ಥಿತಾ~।\\
ಸರ್ವಸೌಭಾಗ್ಯ ಜನನೀ ಶ್ರೀಮಹಾ ತ್ರಿಪುರಸುಂದರೀ ॥}\\
ತ್ವರಿತಾ ಕಲ್ಪಲತಾಂಬಾಶ್ರೀಪಾ।ಪೂ।ತ।ನಮಃ॥೧

\as{೪ ಓಂ ಹ್ರೀಂ ಹ್ರಾಂ ಹಸಕಲಹ್ರೀಂ ಓಂ ಸರಸ್ವತ್ಯೈ ನಮಃ ಹ್‌ಸ್ರೈಂ ॥}\\
\as{ಶ್ರೀಮಹಾಕಲ್ಪಲತೇಶ್ವರೀ ಬೃಂದಮಂಡಿತಾಸನ ಸಂಸ್ಥಿತಾ~।\\
ಸರ್ವಸೌಭಾಗ್ಯ ಜನನೀ ಶ್ರೀಮಹಾ ತ್ರಿಪುರಸುಂದರೀ ॥}\\
ಪಾರಿಜಾತೇಶ್ವರೀ ಕಲ್ಪಲತಾಂಬಾಶ್ರೀಪಾ।ಪೂ।ತ।ನಮಃ॥೨

\as{೪ ಶ್ರೀಂ ಹ್ರೀಂ ಕ್ಲೀಂ ಐಂ ಕ್ಲೀಂ ಸೌಃ ॥}\\
\as{ಶ್ರೀಮಹಾಕಲ್ಪಲತೇಶ್ವರೀ ಬೃಂದಮಂಡಿತಾಸನ ಸಂಸ್ಥಿತಾ~।\\
ಸರ್ವಸೌಭಾಗ್ಯ ಜನನೀ ಶ್ರೀಮಹಾ ತ್ರಿಪುರಸುಂದರೀ ॥}\\
ತ್ರಿಪುಟಾ ಕಲ್ಪಲತಾಂಬಾಶ್ರೀಪಾ।ಪೂ।ತ।ನಮಃ॥೩

\as{೪ ದ್ರಾಂ ದ್ರೀಂ ಕ್ಲೀಂ ಬ್ಲೂಂ ಸಃ ॥}\\
\as{ಶ್ರೀಮಹಾಕಲ್ಪಲತೇಶ್ವರೀ ಬೃಂದಮಂಡಿತಾಸನ ಸಂಸ್ಥಿತಾ~।\\
ಸರ್ವಸೌಭಾಗ್ಯ ಜನನೀ ಶ್ರೀಮಹಾ ತ್ರಿಪುರಸುಂದರೀ ॥}\\
ಪಂಚಬಾಣೇಶ್ವರೀ ಕಲ್ಪಲತಾಂಬಾಶ್ರೀಪಾ।ಪೂ।ತ।ನಮಃ॥೪
\subsection{ಪಂಚಕಾಮದುಘಾಂಬಾಃ}
{\bfseries ೪ ೧೫ ॥}\\
\as{ ಶ್ರೀಮಹಾಕಾಮದುಘೇಶ್ವರೀ ಬೃಂದಮಂಡಿತಾಸನ ಸಂಸ್ಥಿತಾ~।\\
ಸರ್ವಸೌಭಾಗ್ಯ ಜನನೀ ಶ್ರೀಮಹಾ ತ್ರಿಪುರಸುಂದರೀ ॥}\\
ಶ್ರೀವಿದ್ಯಾ ಕಾಮದುಘಾಂಬಾಶ್ರೀಪಾ।ಪೂ।ತ।ನಮಃ॥

\as{೪ ಓಂ ಹ್ರೀಂ ಹಂಸಃ ಜುಂ ಸಂಜೀವನಿ ಜೀವಂ ಪ್ರಾಣಗ್ರಂಥಿಸ್ಥಂ ಕುರು ಕುರು ಸ್ವಾಹಾ ॥}\\
\as{ ಶ್ರೀಮಹಾಕಾಮದುಘೇಶ್ವರೀ ಬೃಂದಮಂಡಿತಾಸನ ಸಂಸ್ಥಿತಾ~।\\
ಸರ್ವಸೌಭಾಗ್ಯ ಜನನೀ ಶ್ರೀಮಹಾ ತ್ರಿಪುರಸುಂದರೀ ॥}\\
ಅಮೃತಪೀಠೇಶ್ವರೀ ಕಾಮದುಘಾಂಬಾಶ್ರೀಪಾ।ಪೂ।ತ।ನಮಃ॥೧

\as{೪ ಐಂ ವದ ವದ ವಾಗ್ವಾದಿನಿ ಹ್‌ಸ್ರೈಂ ಕ್ಲೀಂ ಕ್ಲಿನ್ನೇ ಕ್ಲೇದಿನಿ ಮಹಾಕ್ಷೋಭಂ ಕುರು ಕುರು ಹ್‌ಸ್‌ಕ್ಲ್ರೀಂ ಸೌಃ ಓಂ ಮೋಕ್ಷಂ ಕುರು ಕುರು ಹ್‌ಸ್‌ರ್ಸೌಃ~।}\\
\as{ ಶ್ರೀಮಹಾಕಾಮದುಘೇಶ್ವರೀ ಬೃಂದಮಂಡಿತಾಸನ ಸಂಸ್ಥಿತಾ~।\\
ಸರ್ವಸೌಭಾಗ್ಯ ಜನನೀ ಶ್ರೀಮಹಾ ತ್ರಿಪುರಸುಂದರೀ ॥}\\
ಸುಧಾಸೂಕಾಮದುಘಾಂಬಾಶ್ರೀಪಾ।ಪೂ।ತ।ನಮಃ॥೨

\as{೪ ಐಂ ಬ್ಲೂಂ ಝ್ರೌಂ ಜುಂ ಸಃ ಅಮೃತೇ ಅಮೃತೋದ್ಭವೇ ಅಮೃತೇಶ್ವರಿ \\ಅಮೃತವರ್ಷಿಣಿ ಅಮೃತಂ ಸ್ರಾವಯ ಸ್ರಾವಯ ಸ್ವಾಹಾ ॥}\\
\as{ ಶ್ರೀಮಹಾಕಾಮದುಘೇಶ್ವರೀ ಬೃಂದಮಂಡಿತಾಸನ ಸಂಸ್ಥಿತಾ~।\\
ಸರ್ವಸೌಭಾಗ್ಯ ಜನನೀ ಶ್ರೀಮಹಾ ತ್ರಿಪುರಸುಂದರೀ ॥}\\
ಅಮೃತೇಶ್ವರೀ ಕಾಮದುಘಾಂಬಾಶ್ರೀಪಾ।ಪೂ।ತ।ನಮಃ॥೩

\as{೪ ಓಂ ಹ್ರೀಂಶ್ರೀಂಕ್ಲೀಂ ನಮೋ ಭಗವತಿ ಮಾಹೇಶ್ವರಿ ಅನ್ನಪೂರ್ಣೇ ಸ್ವಾಹಾ ॥}\\
\as{ ಶ್ರೀಮಹಾಕಾಮದುಘೇಶ್ವರೀ ಬೃಂದಮಂಡಿತಾಸನ ಸಂಸ್ಥಿತಾ~।\\
ಸರ್ವಸೌಭಾಗ್ಯ ಜನನೀ ಶ್ರೀಮಹಾ ತ್ರಿಪುರಸುಂದರೀ ॥}\\
ಅನ್ನಪೂರ್ಣಾಕಾಮದುಘಾಂಬಾಶ್ರೀಪಾ।ಪೂ।ತ।ನಮಃ॥೪
\newpage
\subsection{ಪಂಚರತ್ನಾಂಬಾಃ}
{\bfseries ೪ ೧೫ ॥}\\
\as{ಶ್ರೀಮಹಾರತ್ನೇಶ್ವರೀ ಬೃಂದಮಂಡಿತಾಸನ ಸಂಸ್ಥಿತಾ~।\\
ಸರ್ವಸೌಭಾಗ್ಯ ಜನನೀ ಶ್ರೀಮಹಾ ತ್ರಿಪುರಸುಂದರೀ ॥}\\
ಶ್ರೀವಿದ್ಯಾ ರತ್ನಾಂಬಾಶ್ರೀಪಾ।ಪೂ।ತ।ನಮಃ॥

\as{೪ ಜ್‌ಝ್ರೀಂ ಮಹಾಚಂಡೇ ತೇಜಃಕರ್ಷಿಣಿ ಕಾಲಮಂಥಾನೇ ಹಃ ॥}\\
\as{ಶ್ರೀಮಹಾರತ್ನೇಶ್ವರೀ ಬೃಂದಮಂಡಿತಾಸನ ಸಂಸ್ಥಿತಾ~।\\
ಸರ್ವಸೌಭಾಗ್ಯ ಜನನೀ ಶ್ರೀಮಹಾ ತ್ರಿಪುರಸುಂದರೀ ॥}\\
ಸಿದ್ಧಲಕ್ಷ್ಮೀರತ್ನಾಂಬಾಶ್ರೀಪಾ।ಪೂ।ತ।ನಮಃ॥೧

\as{೪ ಐಂಹ್ರೀಂಶ್ರೀಂ ಐಂಕ್ಲೀಂಸೌಃ ಓಂ ನಮೋ ಭಗವತಿ ರಾಜಮಾತಂಗೀಶ್ವರಿ ಸರ್ವಜನಮನೋಹರಿ ಸರ್ವಮುಖರಂಜನಿ ಕ್ಲೀಂಹ್ರೀಂಶ್ರೀಂ ಸರ್ವರಾಜವಶಂಕರಿ ಸರ್ವಸ್ತ್ರೀಪುರುಷ ವಶಂಕರಿ ಸರ್ವ ದುಷ್ಟಮೃಗವಶಂಕರಿ ಸರ್ವ ಸತ್ವವಶಂಕರಿ ಸರ್ವ ಲೋಕವಶಂಕರಿ ತ್ರೈಲೋಕ್ಯಂ ಮೇ ವಶಮಾನಯ ಸ್ವಾಹಾ ಸೌಃಕ್ಲೀಂಐಂ ಶ್ರೀಂಹ್ರೀಂಐಂ॥}\\
\as{ಶ್ರೀಮಹಾರತ್ನೇಶ್ವರೀ ಬೃಂದಮಂಡಿತಾಸನ ಸಂಸ್ಥಿತಾ~।\\
ಸರ್ವಸೌಭಾಗ್ಯ ಜನನೀ ಶ್ರೀಮಹಾ ತ್ರಿಪುರಸುಂದರೀ ॥}\\
ರಾಜಮಾತಂಗೀಶ್ವರೀರತ್ನಾಂಬಾಶ್ರೀಪಾ।ಪೂ।ತ।ನಮಃ॥೨

\as{೪ ಶ್ರೀಂ ಹ್ರೀಂ ಶ್ರೀಂ ॥}\\
\as{ಶ್ರೀಮಹಾರತ್ನೇಶ್ವರೀ ಬೃಂದಮಂಡಿತಾಸನ ಸಂಸ್ಥಿತಾ~।\\
ಸರ್ವಸೌಭಾಗ್ಯ ಜನನೀ ಶ್ರೀಮಹಾ ತ್ರಿಪುರಸುಂದರೀ ॥}\\
ಭುವನೇಶ್ವರೀರತ್ನಾಂಬಾಶ್ರೀಪಾ।ಪೂ।ತ।ನಮಃ॥೩

\as{೪ ಐಂ ಗ್ಲೌಂ ಐಂ ನಮೋ ಭಗವತಿ ವಾರ್ತಾಲಿ ವಾರ್ತಾಲಿ ವಾರಾಹಿ ವಾರಾಹಿ ವರಾಹಮುಖಿ ವರಾಹಮುಖಿ ಅಂಧೇ ಅಂಧಿನಿ ನಮಃ ರುಂಧೇ ರುಂಧಿನಿ ನಮಃ ಜಂಭೇ ಜಂಭಿನಿ ನಮಃ ಮೋಹೇ ಮೋಹಿನಿ ನಮಃ ಸ್ತಂಭೇ ಸ್ತಂಭಿನಿ ನಮಃ ಸರ್ವದುಷ್ಟಪ್ರದುಷ್ಟಾನಾಂ ಸರ್ವೇಷಾಂ ಸರ್ವವಾಕ್ಚಿತ್ತ ಚಕ್ಷುರ್ಮುಖಗತಿ ಜಿಹ್ವಾ  ಸ್ತಂಭನಂ ಕುರು ಕುರು ಶೀಘ್ರಂ ವಶ್ಯಂ ಐಂ ಗ್ಲೌಂ ಐಂ ಠಃಠಃಠಃಠಃ ಹುಂ ಫಟ್ ಸ್ವಾಹಾ ॥}\\
\as{ಶ್ರೀಮಹಾರತ್ನೇಶ್ವರೀ ಬೃಂದಮಂಡಿತಾಸನ ಸಂಸ್ಥಿತಾ~।\\
ಸರ್ವಸೌಭಾಗ್ಯ ಜನನೀ ಶ್ರೀಮಹಾ ತ್ರಿಪುರಸುಂದರೀ ॥}\\
ವಾರಾಹೀರತ್ನಾಂಬಾಶ್ರೀಪಾ।ಪೂ।ತ।ನಮಃ॥೪
\section{ಆಮ್ನಾಯಸಮಷ್ಟಿಪೂಜಾ}
\as{೪ ಹ್‌ಸ್‌ರೈಂ ಹ್‌ಸ್‌ಕ್ಲ್ರೀಂ ಹ್‌ಸ್‌ರ್ಸೌಃ }॥ ಪೂರ್ವಾಮ್ನಾಯ ಸಮಯವಿದ್ಯೇಶ್ವರೀ ಉನ್ಮೋದಿನೀ ದೇವ್ಯಂಬಾ ಶ್ರೀಪಾ।ಪೂ।ತ।ನಮಃ॥

\as{೪ ೧೫॥}
ಗುರುತ್ರಯ ಗಣಪತಿ ಪೀಠತ್ರಯ ಸಹಿತಾಯೈ ಶುದ್ಧವಿದ್ಯಾದಿ ಸಮಯವಿದ್ಯೇಶ್ವರೀಪರ್ಯಂತ ಚತುರ್ವಿಂಶತಿದೇವತಾ ಪರಿಸೇವಿತಾಯೈ ಕಾಮಗಿರಿಪೀಠಸ್ಥಿತಾಯೈ  ಪೂರ್ವಾಮ್ನಾಯ ಸಮಷ್ಟಿರೂಪಿಣ್ಯೈ ಶ್ರೀಮಹಾತ್ರಿಪುರಸುಂದರ್ಯೈ ನಮಃ । ಶ್ರೀಮಹಾತ್ರಿಪುರಸುಂದರೀ ಶ್ರೀಪಾ।ಪೂ।ತ।ನಮಃ॥

\as{೪ ಓಂ ಹ್ರೀಂ ಐಂ ಕ್ಲಿನ್ನೇ ಕ್ಲಿನ್ನಮದದ್ರವೇ ಕುಲೇ ಹ್ಸೌಃ }॥ ದಕ್ಷಿಣಾಮ್ನಾಯ ಸಮಯವಿದ್ಯೇಶ್ವರೀ ಭೋಗಿನೀ ದೇವ್ಯಂಬಾ ಶ್ರೀಪಾ।ಪೂ।ತ।ನಮಃ॥

\as{೪ ೧೫॥} 
ಭೈರವಾಷ್ಟಕ ನವಸಿದ್ಧೌಘ ವಟುಕತ್ರಯ ಪದಯುಗ ಸಹಿತಾಯೈ ಸೌಭಾಗ್ಯವಿದ್ಯಾದಿ ಸಮಯವಿದ್ಯೇಶ್ವರೀ ಪರ್ಯಂತ ತ್ರಿಂಶತ್ಸಹಸ್ರದೇವತಾ ಪರಿಸೇವಿತಾಯೈ ಪೂರ್ಣಗಿರಿ ಪೀಠಸ್ಥಿತಾಯೈ ದಕ್ಷಿಣಾಮ್ನಾಯ ಸಮಷ್ಟಿರೂಪಿಣ್ಯೈ ಶ್ರೀಮಹಾತ್ರಿಪುರಸುಂದರ್ಯೈ ನಮಃ । ಶ್ರೀಮಹಾತ್ರಿಪುರಸುಂದರೀ ಶ್ರೀಪಾ।ಪೂ।ತ।ನಮಃ॥

\as{೪ ಹ್ಸ್ರೈಂ ಹ್ಸ್ರೀಂ ಹ್ಸ್ರೌಃ ಹ್‌ಸ್‌ಖ್‌ಫ್ರೇಂ ಭಗವತ್ಯಂಬೇ ಹಸಕ್ಷಮಲವರಯೂಂ ಹ್‌ಸ್‌ಖ್‌ಫ್ರೇಂ ಅಘೋರಮುಖಿ ಛ್ರಾಂ ಛ್ರೀಂ ಕಿಣಿ ಕಿಣಿ ವಿಚ್ಚೇ ಹ್‌ಸ್ರೌಃ ಹ್‌ಸ್‌ಖ್‌ಫ್ರೇಂ ಹ್‌ಸ್ರೌಃ }॥ ಪಶ್ಚಿಮಾಮ್ನಾಯ ಸಮಯವಿದ್ಯೇಶ್ವರೀ ಕುಬ್ಜಿಕಾ ದೇವ್ಯಂಬಾ ಶ್ರೀಪಾ।ಪೂ।ತ।ನಮಃ॥

\as{೪ ೧೫॥}
ನವದೂತೀ ಮಂಡಲತ್ರಯ ದಶವೀರ ಚತುಃಷಷ್ಟಿ ಸಿದ್ಧನಾಥಸಹಿತಾಯೈ ಲೋಪಾಮುದ್ರಾದಿ ಸಮಯವಿದ್ಯೇಶ್ವರೀಪರ್ಯಂತ ದ್ವಿಸಹಸ್ರದೇವತಾ ಪರಿಸೇವಿತಾಯೈ ಪಶ್ಚಿಮಾಮ್ನಾಯ ಸಮಷ್ಟಿರೂಪಿಣ್ಯೈ ಶ್ರೀಮಹಾತ್ರಿಪುರಸುಂದರ್ಯೈ ನಮಃ । ಶ್ರೀಮಹಾತ್ರಿಪುರಸುಂದರೀ ಶ್ರೀಪಾ।ಪೂ।ತ।ನಮಃ॥

\as{೪ ಹ್‌ಸ್‌ಖ್‌ಫ್ರೇಂ  ಮಹಾಚಂಡಯೋಗೀಶ್ವರಿ ಕಾಳಿಕೇ ಫಟ್ }॥ ಉತ್ತರಾಮ್ನಾಯ ಸಮಯವಿದ್ಯೇಶ್ವರೀ ಕಾಳಿಕಾ ದೇವ್ಯಂಬಾ ಶ್ರೀಪಾ।ಪೂ।ತ।ನಮಃ॥

\as{೪ ೧೫॥}
ನವಮುದ್ರಾ ಪಂಚವೀರಾವಳೀ ಸಹಿತಾಯೈ ತುರ್ಯಾಂಬಾದಿ ಸಮಯವಿದ್ಯೇಶ್ವರೀಪರ್ಯಂತ ದ್ವಿಸಹಸ್ರದೇವತಾ ಪರಿಸೇವಿತಾಯೈ ಓಡ್ಯಾಣಪೀಠಸ್ಥಿತಾಯೈ ಉತ್ತರಾಮ್ನಾಯ ಸಮಷ್ಟಿರೂಪಿಣ್ಯೈ ಶ್ರೀಮಹಾತ್ರಿಪುರಸುಂದರ್ಯೈ ನಮಃ । ಶ್ರೀಮಹಾತ್ರಿಪುರಸುಂದರೀ ಶ್ರೀಪಾ।ಪೂ।ತ।ನಮಃ॥

 [\,ಷೋಡಶ್ಯುಪಾಸಕಾನಾಂ ಕೃತೇ :\\
 \as{೪ ಮಖಪರಯಘಚ್ ಮಹಿಚನಡಯಙ್ ಗಂಶಫರ್ }॥ ಊರ್ಧ್ವಾಮ್ನಾಯ ಸಮಯವಿದ್ಯೇಶ್ವರೀ ಚೈತನ್ಯಭೈರವ್ಯಂಬಾ ಶ್ರೀಪಾ।ಪೂ।ತ।ನಮಃ॥\\
\as{೪ ೧೬॥}
ಶ್ರೀಮನ್ಮಾಲಿನಿ ಮಂತ್ರರಾಜ ಗುರುಮಂಡಲ ಸಹಿತಾಯೈ ಪರಾಂಬಾದಿ ಸಮಯವಿದ್ಯೇಶ್ವರೀ ಪರ್ಯಂತ ಅಶೀತಿ ಸಹಸ್ರ ದೇವತಾ ಪರಿಸೇವಿತಾಯೈ ಶಾಂಭವಪೀಠಸ್ಥಿತಾಯೈ ಊರ್ಧ್ವಾಮ್ನಾಯ ಸಮಷ್ಟಿರೂಪಿಣ್ಯೈ ಶ್ರೀಮಹಾತ್ರಿಪುರಸುಂದರ್ಯೈ ನಮಃ । ಶ್ರೀಮಹಾತ್ರಿಪುರಸುಂದರೀ ಶ್ರೀಪಾ।ಪೂ।ತ।ನಮಃ॥

\as{೪ ಭಗವತಿ ವಿಚ್ಚೇ ಮಹಾಮಾಯೇ ಮಾತಂಗಿನಿ ಬ್ಲೂಂ ಅನುತ್ತರವಾಗ್ವಾದಿನಿ ಹ್‌ಸ್‌ಖ್‌ಫ್ರೇಂ ಹ್‌ಸ್‌ಖ್‌ಫ್ರೇಂ ಹ್‌ಸ್ರೌಃ }॥ ಅನುತ್ತರಶಾಂಕರ್ಯಂಬಾ ಶ್ರೀಪಾ।ಪೂ।ತ।ನಮಃ॥

\as{೪ ೧೬॥}
ಪರಿಪೂರ್ಣಾನಂದಾದಿ ನವನಾಥಸಹಿತಾಯೈ ಚತುರ್ದಶ ಮೂಲವಿದ್ಯಾದಿ ಅನುತ್ತರಶಾಂಕರ್ಯಂತಾನಂತದೇವತಾ ಪರಿಸೇವಿತಾಯೈ  ಅನುತ್ತರಾಮ್ನಾಯ ಸಮಷ್ಟಿರೂಪಿಣ್ಯೈ ಶ್ರೀಮಹಾತ್ರಿಪುರಸುಂದರ್ಯೈ ನಮಃ । ಶ್ರೀಮಹಾತ್ರಿಪುರಸುಂದರೀ ಶ್ರೀಪಾ।ಪೂ।ತ।ನಮಃ॥]\,

\section{ಷಡ್ದರ್ಶನಪೂಜಾ}
\as{೪ ತಾರೇ ತುತ್ತಾರೇ ತುರೇ ಸ್ವಾಹಾ }। ತಾರಾ ದೇವತಾಧಿಷ್ಠಿತ ಬೌದ್ಧ ದರ್ಶನ ಶ್ರೀಪಾ।ಪೂ।ತ।ನಮಃ॥(ಸ್ಥಿತಿಚಕ್ರೇ)\\
\as{೪ (ಗಾಯತ್ರೀ) }। ಬ್ರಹ್ಮ ದೇವತಾಧಿಷ್ಠಿತ ವೈದಿಕ ದರ್ಶನ ಶ್ರೀಪಾ।ಪೂ।ತ।ನಮಃ॥(ಪ್ರಥಮಭೂಪುರೇ)\\
\as{೪ (ಪಂಚಾಕ್ಷರೀ)} । ರುದ್ರ ದೇವತಾಧಿಷ್ಠಿತ ಶೈವ ದರ್ಶನ ಶ್ರೀಪಾ।ಪೂ।ತ।ನಮಃ॥(ಬಿಂದೌ)\\
\as{೪ (ಆದಿತ್ಯಮಂತ್ರಃ)}। ಸೂರ್ಯ ದೇವತಾಧಿಷ್ಠಿತ ಸೌರ ದರ್ಶನ ಶ್ರೀಪಾ।ಪೂ।ತ।ನಮಃ॥(ಸೃಷ್ಟಿಚಕ್ರೇ)\\
\as{೪ (ಅಷ್ಟಾಕ್ಷರೀ)}। ವಿಷ್ಣು ದೇವತಾಧಿಷ್ಠಿತ ವೈಷ್ಣವ ದರ್ಶನ ಶ್ರೀಪಾ।ಪೂ।ತ।ನಮಃ॥(ಶಿವವಾಮಭಾಗೇ)\\
\as{೪ ಶ್ರೀಂಹ್ರೀಂಶ್ರೀಂ }। ಭುವನೇಶ್ವರೀ ದೇವತಾಧಿಷ್ಠಿತ ಶಾಕ್ತ ದರ್ಶನ ಶ್ರೀಪಾ।ಪೂ।ತ।ನಮಃ॥(ಬಿಂದುಂ ಪರಿತಃ)
\section{ಷಡಾಧಾರಪೂಜಾ}
\as{೪ ಸಾಂ ಹಂಸಃ } ಸ್ವಚ್ಛಾನಂದವಿಭೂತ್ಯೈ ಸ್ವಾಹಾ ।\\ ಮೂಲಾಧಾರಾಧಿಷ್ಠಾನದೇವತಾಯೈ ಸಾಕಿನೀಸಹಿತ ಗಣನಾಥಸ್ವರೂಪಿಣ್ಯೈ ನಮಃ ।  ಮೂಲಾಧಾರದೇವೀ ಶ್ರೀಪಾದುಕಾಂ ಪೂ।ತ।ನಮಃ॥\\
\as{೪ ಕಾಂ ಸೋಹಂ } ಪರಮಹಂಸವಿಭೂತ್ಯೈ ಸ್ವಾಹಾ ।\\ ಸ್ವಾಧಿಷ್ಠಾನಾಧಿಷ್ಠಾನದೇವತಾಯೈ  ಕಾಕಿನೀಸಹಿತ  ಬ್ರಹ್ಮಸ್ವರೂಪಿಣ್ಯೈ ನಮಃ । ಸ್ವಾಧಿಷ್ಠಾನದೇವೀ ಶ್ರೀಪಾದುಕಾಂ ಪೂ।ತ।ನಮಃ॥\\
\as{೪ ಲಾಂ ಹಂಸಸ್ಸೋಹಂ }ಸ್ವಚ್ಛಾನಂದ ಪರಮಹಂಸ ಪರಮಾತ್ಮನೇ ಸ್ವಾಹಾ ।\\ ಮಣಿಪೂರಾಧಿಷ್ಠಾನದೇವತಾಯೈ ಲಾಕಿನೀಸಹಿತ ವಿಷ್ಣುಸ್ವರೂಪಿಣ್ಯೈ ನಮಃ । ಮಣಿಪೂರದೇವೀ ಶ್ರೀಪಾದುಕಾಂ ಪೂ।ತ।ನಮಃ॥\\
\as{೪ ರಾಂ ಹಂಸಶ್ಶಿವಸ್ಸೋಹಂ } ಸ್ವಾತ್ಮಾನಂ ಬೋಧಯ ಬೋಧಯ ಸ್ವಾಹಾ ।\\ ಅನಾಹತಾಧಿಷ್ಠಾನದೇವತಾಯೈ ರಾಕಿಣೀಸಹಿತ ಸದಾಶಿವಸ್ವರೂಪಿಣ್ಯೈ\\ ನಮಃ । ಅನಾಹತದೇವೀ ಶ್ರೀಪಾದುಕಾಂ ಪೂ।ತ।ನಮಃ॥\\
\as{೪ ಡಾಂ ಸೋಹಂಹಂಸಶ್ಶಿವಃ }ಪರಮಾತ್ಮಾನಂ ಬೋಧಯಬೋಧಯ ಸ್ವಾಹಾ ।\\ ವಿಶುಶುದ್ಧ್ಯಧಿಷ್ಠಾನದೇವತಾಯೈ ಡಾಕಿನೀಸಹಿತ ಜೀವೇಶ್ವರಸ್ವರೂಪಿಣ್ಯೈ ನಮಃ । ವಿಶುದ್ಧಿದೇವೀ ಶ್ರೀಪಾದುಕಾಂ ಪೂ।ತ।ನಮಃ॥\\
\as{೪ ಹಾಂ ಹಂಸಶ್ಶಿವಸ್ಸೋಹಂ ಸೋಹಂಹಂಸಶ್ಶಿವಃ }\\ ಸ್ವಚ್ಛಾನಂದ ಚಿತ್ಪ್ರಕಾಶಾಮೃತಹೇತವೇ ಸ್ವಾಹಾ ।\\ ಆಜ್ಞಾಧಿಷ್ಠಾನದೇವತಾಯೈ ಹಾಕಿನೀಸಹಿತ ಪರಮಾತ್ಮಸ್ವರೂಪಿಣ್ಯೈ ನಮಃ । ಆಜ್ಞಾದೇವೀ ಶ್ರೀಪಾದುಕಾಂ ಪೂ।ತ।ನಮಃ॥
\subsection{ಪುನಃ ಷಡಂಗಾರ್ಚನಂ}
\as{೪ ಐಂ೫ } ಸರ್ವಜ್ಞಾಯೈ ಹೃದಯಾಯ ನಮಃ ।\\ ಹೃದಯಶಕ್ತಿ ಶ್ರೀಪಾದುಕಾಂ ಪೂ।ತ।ನಮಃ ॥\\
\as{೪ ಕ್ಲೀಂ೬ } ನಿತ್ಯತೃಪ್ತಾಯೈ ಶಿರಸೇ ಸ್ವಾಹಾ ।\\ ಶಿರಃಶಕ್ತಿ ಶ್ರೀಪಾದುಕಾಂ ಪೂ।ತ।ನಮಃ ॥\\
\as{೪ ಸೌಃ೪ } ಅನಾದಿಬೋಧಿನ್ಯೈ ಶಿಖಾಯೈ ವಷಟ್ ।\\ ಶಿಖಾಶಕ್ತಿ ಶ್ರೀಪಾದುಕಾಂ ಪೂ।ತ।ನಮಃ ॥\\
\as{೪ ಐಂ೫ } ಸ್ವತಂತ್ರಾಯೈ ಕವಚಾಯ ಹುಂ ।\\ ಕವಚಶಕ್ತಿ ಶ್ರೀಪಾದುಕಾಂ ಪೂ।ತ।ನಮಃ ॥\\
\as{೪ ಕ್ಲೀಂ೬} ನಿತ್ಯಾಲುಪ್ತಾಯೈ ನೇತ್ರತ್ರಯಾಯ ವೌಷಟ್ ।\\ ನೇತ್ರಶಕ್ತಿ ಶ್ರೀಪಾದುಕಾಂ ಪೂ । ನಮಃ॥\\
\as{೪ ಸೌಃ೪ } ಅನಂತಾಯೈ ಅಸ್ತ್ರಾಯ ಫಟ್ ।\\ ಅಸ್ತ್ರಶಕ್ತಿ ಶ್ರೀಪಾದುಕಾಂ ಪೂ।ತ।ನಮಃ ॥

ಮೂಲೇನ ಬಿಂದೌ ತ್ರಿಃ ಸಂತರ್ಪ್ಯ, ದಶ ಮುದ್ರಾಃ ಪ್ರದರ್ಶ್ಯ ಉತ್ತರನ್ಯಾಸಂ ವಿಧಾಯ
\section{ದಂಡನಾಥಾ ನಾಮಪೂಜಾ}
ಪಂಚಮ್ಯೈ ನಮಃ~।  ದಂಡನಾಥಾಯೈ~।  ಸಂಕೇತಾಯೈ~।  ಸಮಯೇಶ್ವರ್ಯೈ~।  ಸಮಯಸಂಕೇತಾಯೈ~।  ವಾರಾಹ್ಯೈ~।  ಪೋತ್ರಿಣ್ಯೈ~।  ಶಿವಾಯೈ~।  ವಾರ್ತಾಲ್ಯೈ~।  ಮಹಾಸೇನಾಯೈ~।  ಆಜ್ಞಾಚಕ್ರೇಶ್ವರ್ಯೈ~।  ಅರಿಘ್ನ್ಯೈ ನಮಃ ॥
\newpage
\section{ಮಂತ್ರಿಣೀ ನಾಮಪೂಜಾ}
ಸಂಗೀತಯೋಗಿನ್ಯೈ ನಮಃ~।  ಶ್ಯಾಮಾಯೈ~।  ಶ್ಯಾಮಲಾಯೈ~।  ಮಂತ್ರನಾಯಿಕಾಯೈ~।  ಮಂತ್ರಿಣ್ಯೈ~।  ಸಚಿವೇಶಾನ್ಯೈ~।  ಪ್ರಧಾನೇಶ್ಯೈ~।  ಶುಕಪ್ರಿಯಾಯೈ~।  ವೀಣಾವತ್ಯೈ~।  ವೈಣಿಕ್ಯೈ~।  ಮುದ್ರಿಣ್ಯೈ~।  ಪ್ರಿಯಕಪ್ರಿಯಾಯೈ~।  ನೀಪಪ್ರಿಯಾಯೈ~।  ಕದಂಬೇಶ್ಯೈ~।  ಕದಂಬವನವಾಸಿನ್ಯೈ~।  ಸದಾಮದಾಯೈ ನಮಃ~॥
\section{ಲಲಿತಾ ನಾಮಪೂಜಾ}
ಸಿಂಹಾಸನೇಶ್ಯೈ ನಮಃ~।  ಲಲಿತಾಯೈ~।  ಮಹಾರಾಜ್ಞ್ಯೈ~।  ವರಾಂಕುಶಾಯೈ~।  ಚಾಪಿನ್ಯೈ~।  ತ್ರಿಪುರಾಯೈ~।  ಮಹಾತ್ರಿಪುರಸುಂದರ್ಯೈ~।  ಸುಂದರ್ಯೈ~।  ಚಕ್ರನಾಥಾಯೈ~।  ಸಮ್ರಾಜ್ಞ್ಯೈ~।  ಚಕ್ರಿಣ್ಯೈ~।  ಚಕ್ರೇಶ್ವರ್ಯೈ~।  ಮಹಾದೇವ್ಯೈ~।  ಕಾಮೇಶ್ಯೈ~।  ಪರಮೇಶ್ವರ್ಯೈ~।  ಕಾಮರಾಜಪ್ರಿಯಾಯೈ~।  ಕಾಮಕೋಟಿಕಾಯೈ~।  ಚಕ್ರವರ್ತಿನ್ಯೈ~।  ಮಹಾವಿದ್ಯಾಯೈ~।  ಶಿವಾನಂಗವಲ್ಲಭಾಯೈ~।  ಸರ್ವಪಾಟಲಾಯೈ~।  ಕುಲನಾಥಾಯೈ~।  ಆಮ್ನಾಯನಾಥಾಯೈ~।  ಸರ್ವಾಮ್ನಾಯನಿವಾಸಿನ್ಯೈ~।  ಶೃಂಗಾರನಾಯಿಕಾಯೈ ನಮಃ~॥
\section{ಶ್ರೀಲಲಿತಾತ್ರಿಶತೀ}
{\bfseries ಅತಿಮಧುರಚಾಪಹಸ್ತಾಮಪರಿಮಿತಾಮೋದಬಾಣಸೌಭಾಗ್ಯಾಂ~।\\
ಅರುಣಾಮತಿಶಯಕರುಣಾಮಭಿನವಕುಲಸುಂದರೀಂ ವಂದೇ ॥

ಹಯಗ್ರೀವ ಉವಾಚ\\
ಕಕಾರರೂಪಾ ಕಲ್ಯಾಣೀ ಕಲ್ಯಾಣಗುಣಶಾಲಿನೀ~।\\
ಕಲ್ಯಾಣಶೈಲನಿಲಯಾ ಕಮನೀಯಾ ಕಲಾವತೀ ॥೧॥

ಕಮಲಾಕ್ಷೀ ಕಲ್ಮಷಘ್ನೀ ಕರುಣಾಮೃತ ಸಾಗರಾ~।\\
ಕದಂಬಕಾನನಾವಾಸಾ ಕದಂಬ ಕುಸುಮಪ್ರಿಯಾ ॥೨॥

ಕಂದರ್ಪವಿದ್ಯಾ ಕಂದರ್ಪ ಜನಕಾಪಾಂಗ ವೀಕ್ಷಣಾ~।\\
ಕರ್ಪೂರವೀಟೀ ಸೌರಭ್ಯ ಕಲ್ಲೋಲಿತ ಕಕುಪ್ತಟಾ ॥೩॥

ಕಲಿದೋಷಹರಾ ಕಂಜಲೋಚನಾ ಕಮ್ರವಿಗ್ರಹಾ~।\\
ಕರ್ಮಾದಿ ಸಾಕ್ಷಿಣೀ ಕಾರಯಿತ್ರೀ ಕರ್ಮಫಲಪ್ರದಾ ॥೪॥

ಏಕಾರರೂಪಾ ಚೈಕಾಕ್ಷರ್ಯೇಕಾನೇಕಾಕ್ಷರಾಕೃತಿಃ~।\\
ಏತತ್ತದಿತ್ಯನಿರ್ದೇಶ್ಯಾ ಚೈಕಾನಂದ ಚಿದಾಕೃತಿಃ ॥೫॥

ಏವಮಿತ್ಯಾಗಮಾಬೋಧ್ಯಾ ಚೈಕಭಕ್ತಿ ಮದರ್ಚಿತಾ~।\\
ಏಕಾಗ್ರಚಿತ್ತ ನಿರ್ಧ್ಯಾತಾ ಚೈಷಣಾ ರಹಿತಾದೃತಾ ॥೬॥

ಏಲಾಸುಗಂಧಿಚಿಕುರಾ ಚೈನಃ ಕೂಟ ವಿನಾಶಿನೀ~।\\
ಏಕಭೋಗಾ ಚೈಕರಸಾ ಚೈಕೈಶ್ವರ್ಯ ಪ್ರದಾಯಿನೀ ॥೭॥

ಏಕಾತಪತ್ರ ಸಾಮ್ರಾಜ್ಯ ಪ್ರದಾ ಚೈಕಾಂತಪೂಜಿತಾ~।\\
ಏಧಮಾನಪ್ರಭಾ ಚೈಜದನೇಕಜಗದೀಶ್ವರೀ ॥೮॥

ಏಕವೀರಾದಿ ಸಂಸೇವ್ಯಾ ಚೈಕಪ್ರಾಭವ ಶಾಲಿನೀ~।\\
ಈಕಾರರೂಪಾ ಚೇಶಿತ್ರೀ ಚೇಪ್ಸಿತಾರ್ಥ ಪ್ರದಾಯಿನೀ ॥೯॥

ಈದೃಗಿತ್ಯ ವಿನಿರ್ದೇಶ್ಯಾ ಚೇಶ್ವರತ್ವ ವಿಧಾಯಿನೀ~।\\
ಈಶಾನಾದಿ ಬ್ರಹ್ಮಮಯೀ ಚೇಶಿತ್ವಾದ್ಯಷ್ಟ ಸಿದ್ಧಿದಾ ॥೧೦॥

ಈಕ್ಷಿತ್ರೀಕ್ಷಣ ಸೃಷ್ಟಾಂಡ ಕೋಟಿರೀಶ್ವರ ವಲ್ಲಭಾ~।\\
ಈಡಿತಾ ಚೇಶ್ವರಾರ್ಧಾಂಗ ಶರೀರೇಶಾಧಿ ದೇವತಾ ॥೧೧॥

ಈಶ್ವರ ಪ್ರೇರಣಕರೀ ಚೇಶತಾಂಡವ ಸಾಕ್ಷಿಣೀ~।\\
ಈಶ್ವರೋತ್ಸಂಗ ನಿಲಯಾ ಚೇತಿಬಾಧಾ ವಿನಾಶಿನೀ ॥೧೨॥

ಈಹಾವಿರಹಿತಾ ಚೇಶ ಶಕ್ತಿ ರೀಷತ್ ಸ್ಮಿತಾನನಾ~।\\
ಲಕಾರರೂಪಾ ಲಲಿತಾ ಲಕ್ಷ್ಮೀ ವಾಣೀ ನಿಷೇವಿತಾ ॥೧೩॥

ಲಾಕಿನೀ ಲಲನಾರೂಪಾ ಲಸದ್ದಾಡಿಮ ಪಾಟಲಾ~।\\
ಲಲಂತಿಕಾಲಸತ್ಫಾಲಾ ಲಲಾಟ ನಯನಾರ್ಚಿತಾ ॥೧೪॥

ಲಕ್ಷಣೋಜ್ಜ್ವಲ ದಿವ್ಯಾಂಗೀ ಲಕ್ಷಕೋಟ್ಯಂಡ ನಾಯಿಕಾ~।\\
ಲಕ್ಷ್ಯಾರ್ಥಾ ಲಕ್ಷಣಾಗಮ್ಯಾ ಲಬ್ಧಕಾಮಾ ಲತಾತನುಃ ॥೧೫॥

ಲಲಾಮರಾಜದಲಿಕಾ ಲಂಬಿಮುಕ್ತಾಲತಾಂಚಿತಾ~।\\
ಲಂಬೋದರ ಪ್ರಸೂರ್ಲಭ್ಯಾ ಲಜ್ಜಾಢ್ಯಾ ಲಯವರ್ಜಿತಾ ॥೧೬॥

ಹ್ರೀಂಕಾರ ರೂಪಾ ಹ್ರೀಂಕಾರ ನಿಲಯಾ ಹ್ರೀಂಪದಪ್ರಿಯಾ~।\\
ಹ್ರೀಂಕಾರ ಬೀಜಾ ಹ್ರೀಂಕಾರಮಂತ್ರಾ ಹ್ರೀಂಕಾರಲಕ್ಷಣಾ ॥೧೭॥

ಹ್ರೀಂಕಾರಜಪ ಸುಪ್ರೀತಾ ಹ್ರೀಂಮತೀ ಹ್ರೀಂವಿಭೂಷಣಾ~।\\
ಹ್ರೀಂಶೀಲಾಹ್ರೀಂಪದಾರಾಧ್ಯಾ ಹ್ರೀಂಗರ್ಭಾಹ್ರೀಂಪದಾಭಿಧಾ॥೧೮॥

ಹ್ರೀಂಕಾರವಾಚ್ಯಾ ಹ್ರೀಂಕಾರ ಪೂಜ್ಯಾ ಹ್ರೀಂಕಾರ ಪೀಠಿಕಾ~।\\
ಹ್ರೀಂಕಾರವೇದ್ಯಾ ಹ್ರೀಂಕಾರಚಿಂತ್ಯಾ ಹ್ರೀಂ ಹ್ರೀಂಶರೀರಿಣೀ ॥೧೯॥

ಹಕಾರರೂಪಾ ಹಲಧೃಕ್ಪೂಜಿತಾ ಹರಿಣೇಕ್ಷಣಾ~।\\
ಹರಪ್ರಿಯಾ ಹರಾರಾಧ್ಯಾ ಹರಿಬ್ರಹ್ಮೇಂದ್ರ ವಂದಿತಾ ॥೨೦॥

ಹಯಾರೂಢಾ ಸೇವಿತಾಂಘ್ರಿರ್ಹಯಮೇಧ ಸಮರ್ಚಿತಾ।\\
ಹರ್ಯಕ್ಷವಾಹನಾ ಹಂಸವಾಹನಾ ಹತದಾನವಾ ॥೨೧॥

ಹತ್ಯಾದಿಪಾಪಶಮನೀ ಹರಿದಶ್ವಾದಿ ಸೇವಿತಾ~।\\
ಹಸ್ತಿಕುಂಭೋತ್ತುಂಗ ಕುಚಾ ಹಸ್ತಿಕೃತ್ತಿ ಪ್ರಿಯಾಂಗನಾ ॥೨೨॥

ಹರಿದ್ರಾಕುಂಕುಮಾ ದಿಗ್ಧಾ ಹರ್ಯಶ್ವಾದ್ಯಮರಾರ್ಚಿತಾ~।\\
ಹರಿಕೇಶಸಖೀ ಹಾದಿವಿದ್ಯಾ ಹಾಲಾಮದಾಲಸಾ ॥೨೩॥

ಸಕಾರರೂಪಾ ಸರ್ವಜ್ಞಾ ಸರ್ವೇಶೀ ಸರ್ವಮಂಗಲಾ~।\\
ಸರ್ವಕರ್ತ್ರೀ ಸರ್ವಭರ್ತ್ರೀ ಸರ್ವಹಂತ್ರೀ ಸನಾತನಾ ॥೨೪॥

ಸರ್ವಾನವದ್ಯಾ ಸರ್ವಾಂಗ ಸುಂದರೀ ಸರ್ವಸಾಕ್ಷಿಣೀ~।\\
ಸರ್ವಾತ್ಮಿಕಾ ಸರ್ವಸೌಖ್ಯ ದಾತ್ರೀ ಸರ್ವವಿಮೋಹಿನೀ ॥೨೫॥

ಸರ್ವಾಧಾರಾ ಸರ್ವಗತಾ ಸರ್ವಾವಗುಣವರ್ಜಿತಾ~।\\
ಸರ್ವಾರುಣಾ ಸರ್ವಮಾತಾ ಸರ್ವಭೂಷಣ ಭೂಷಿತಾ ॥೨೬॥

ಕಕಾರಾರ್ಥಾ ಕಾಲಹಂತ್ರೀ ಕಾಮೇಶೀ ಕಾಮಿತಾರ್ಥದಾ~।\\
ಕಾಮಸಂಜೀವನೀ ಕಲ್ಯಾ ಕಠಿನಸ್ತನ ಮಂಡಲಾ ॥೨೭॥

ಕರಭೋರುಃ ಕಲಾನಾಥ ಮುಖೀ ಕಚಜಿತಾಂಬುದಾ~।\\
ಕಟಾಕ್ಷಸ್ಯಂದಿ ಕರುಣಾ ಕಪಾಲಿ ಪ್ರಾಣ ನಾಯಿಕಾ ॥೨೮॥

ಕಾರುಣ್ಯ ವಿಗ್ರಹಾ ಕಾಂತಾ ಕಾಂತಿಧೂತ ಜಪಾವಲಿಃ~।\\
ಕಲಾಲಾಪಾ ಕಂಬುಕಂಠೀ ಕರನಿರ್ಜಿತ ಪಲ್ಲವಾ ॥೨೯॥

ಕಲ್ಪವಲ್ಲೀ ಸಮಭುಜಾ ಕಸ್ತೂರೀ ತಿಲಕಾಂಚಿತಾ~।\\
ಹಕಾರಾರ್ಥಾ ಹಂಸಗತಿರ್ಹಾಟಕಾಭರಣೋಜ್ಜ್ವಲಾ ॥೩೦॥

ಹಾರಹಾರಿ ಕುಚಾಭೋಗಾ ಹಾಕಿನೀ ಹಲ್ಯವರ್ಜಿತಾ~।\\
ಹರಿತ್ಪತಿ ಸಮಾರಾಧ್ಯಾ ಹಠಾತ್ಕಾರ ಹತಾಸುರಾ ॥೩೧॥

ಹರ್ಷಪ್ರದಾ ಹವಿರ್ಭೋಕ್ತ್ರೀ ಹಾರ್ದ ಸಂತಮಸಾಪಹಾ~।\\
ಹಲ್ಲೀಸಲಾಸ್ಯ ಸಂತುಷ್ಟಾ ಹಂಸಮಂತ್ರಾರ್ಥ ರೂಪಿಣೀ ॥೩೨॥

ಹಾನೋಪಾದಾನ ನಿರ್ಮುಕ್ತಾ ಹರ್ಷಿಣೀ ಹರಿಸೋದರೀ~।\\
ಹಾಹಾಹೂಹೂ ಮುಖ ಸ್ತುತ್ಯಾ ಹಾನಿ ವೃದ್ಧಿ ವಿವರ್ಜಿತಾ ॥೩೩॥

ಹಯ್ಯಂಗವೀನ ಹೃದಯಾ ಹರಿಗೋಪಾರುಣಾಂಶುಕಾ~।\\
ಲಕಾರಾಖ್ಯಾ ಲತಾಪೂಜ್ಯಾ ಲಯಸ್ಥಿತ್ಯುದ್ಭವೇಶ್ವರೀ ॥೩೪॥

ಲಾಸ್ಯ ದರ್ಶನ ಸಂತುಷ್ಟಾ ಲಾಭಾಲಾಭ ವಿವರ್ಜಿತಾ~।\\
ಲಂಘ್ಯೇತರಾಜ್ಞಾ ಲಾವಣ್ಯ ಶಾಲಿನೀ ಲಘು ಸಿದ್ಧಿದಾ ॥೩೫॥

ಲಾಕ್ಷಾರಸ ಸವರ್ಣಾಭಾ ಲಕ್ಷ್ಮಣಾಗ್ರಜ ಪೂಜಿತಾ~।\\
ಲಭ್ಯೇತರಾ ಲಬ್ಧ ಭಕ್ತಿ ಸುಲಭಾ ಲಾಂಗಲಾಯುಧಾ ॥೩೬॥

ಲಗ್ನಚಾಮರ ಹಸ್ತ ಶ್ರೀಶಾರದಾ ಪರಿವೀಜಿತಾ~।\\
ಲಜ್ಜಾಪದ ಸಮಾರಾಧ್ಯಾ ಲಂಪಟಾ ಲಕುಲೇಶ್ವರೀ ॥೩೭॥

ಲಬ್ಧಮಾನಾ ಲಬ್ಧರಸಾ ಲಬ್ಧ ಸಂಪತ್ಸಮುನ್ನತಿಃ~।\\
ಹ್ರೀಂಕಾರಿಣೀ ಹ್ರೀಂಕಾರಾದ್ಯಾ ಹ್ರೀಂಮಧ್ಯಾ ಹ್ರೀಂಶಿಖಾಮಣಿಃ ॥೩೮॥

ಹ್ರೀಂಕಾರಕುಂಡಾಗ್ನಿ ಶಿಖಾ ಹ್ರೀಂಕಾರ ಶಶಿಚಂದ್ರಿಕಾ~।\\
ಹ್ರೀಂಕಾರ ಭಾಸ್ಕರರುಚಿರ್ಹ್ರೀಂಕಾರಾಂಭೋದ ಚಂಚಲಾ ॥೩೯॥

ಹ್ರೀಂಕಾರ ಕಂದಾಂಕುರಿಕಾ ಹ್ರೀಂಕಾರೈಕ ಪರಾಯಣಾ~।\\
ಹ್ರೀಂಕಾರ ದೀರ್ಘಿಕಾಹಂಸೀ ಹ್ರೀಂಕಾರೋದ್ಯಾನ ಕೇಕಿನೀ ॥೪೦॥

ಹ್ರೀಂಕಾರಾರಣ್ಯ ಹರಿಣೀ ಹ್ರೀಂಕಾರಾವಾಲ ವಲ್ಲರೀ~।\\
ಹ್ರೀಂಕಾರ ಪಂಜರಶುಕೀ ಹ್ರೀಂಕಾರಾಂಗಣ ದೀಪಿಕಾ ॥೪೧॥

ಹ್ರೀಂಕಾರ ಕಂದರಾ ಸಿಂಹೀ ಹ್ರೀಂಕಾರಾಂಭೋಜ ಭೃಂಗಿಕಾ~।\\
ಹ್ರೀಂಕಾರ ಸುಮನೋ ಮಾಧ್ವೀ ಹ್ರೀಂಕಾರ ತರುಮಂಜರೀ ॥೪೨॥

ಸಕಾರಾಖ್ಯಾ ಸಮರಸಾ ಸಕಲಾಗಮ ಸಂಸ್ತುತಾ~।\\
ಸರ್ವವೇದಾಂತ ತಾತ್ಪರ್ಯಭೂಮಿಃ ಸದಸದಾಶ್ರಯಾ ॥೪೩॥

ಸಕಲಾ ಸಚ್ಚಿದಾನಂದಾ ಸಾಧ್ಯಾ ಸದ್ಗತಿದಾಯಿನೀ~।\\
ಸನಕಾದಿಮುನಿಧ್ಯೇಯಾ ಸದಾಶಿವ ಕುಟುಂಬಿನೀ ॥೪೪॥

ಸಕಾಲಾಧಿಷ್ಠಾನ ರೂಪಾ ಸತ್ಯರೂಪಾ ಸಮಾಕೃತಿಃ~।\\
ಸರ್ವಪ್ರಪಂಚ ನಿರ್ಮಾತ್ರೀ ಸಮಾನಾಧಿಕ ವರ್ಜಿತಾ ॥೪೫॥

ಸರ್ವೋತ್ತುಂಗಾ ಸಂಗಹೀನಾ ಸಗುಣಾ ಸಕಲೇಶ್ವರೀ~।\\
ಕಕಾರಿಣೀ ಕಾವ್ಯಲೋಲಾ ಕಾಮೇಶ್ವರ ಮನೋಹರಾ ॥೪೬॥

ಕಾಮೇಶ್ವರಪ್ರಾಣನಾಡೀ ಕಾಮೇಶೋತ್ಸಂಗ ವಾಸಿನೀ~।\\
ಕಾಮೇಶ್ವರಾಲಿಂಗಿತಾಂಗೀ ಕಾಮೇಶ್ವರ ಸುಖಪ್ರದಾ ॥೪೭॥

ಕಾಮೇಶ್ವರ ಪ್ರಣಯಿನೀ ಕಾಮೇಶ್ವರ ವಿಲಾಸಿನೀ~।\\
ಕಾಮೇಶ್ವರ ತಪಃ ಸಿದ್ಧಿಃ ಕಾಮೇಶ್ವರ ಮನಃ ಪ್ರಿಯಾ ॥೪೮॥

ಕಾಮೇಶ್ವರ ಪ್ರಾಣನಾಥಾ ಕಾಮೇಶ್ವರ ವಿಮೋಹಿನೀ~।\\
ಕಾಮೇಶ್ವರ ಬ್ರಹ್ಮವಿದ್ಯಾ ಕಾಮೇಶ್ವರ ಗೃಹೇಶ್ವರೀ ॥೪೯॥

ಕಾಮೇಶ್ವರಾಹ್ಲಾದಕರೀ ಕಾಮೇಶ್ವರ ಮಹೇಶ್ವರೀ~।\\
ಕಾಮೇಶ್ವರೀ ಕಾಮಕೋಟಿ ನಿಲಯಾ ಕಾಂಕ್ಷಿತಾರ್ಥದಾ ॥೫೦॥

ಲಕಾರಿಣೀ ಲಬ್ಧರೂಪಾ ಲಬ್ಧಧೀರ್ಲಬ್ಧ ವಾಂಛಿತಾ~।\\
ಲಬ್ಧಪಾಪ ಮನೋದೂರಾ ಲಬ್ಧಾಹಂಕಾರ ದುರ್ಗಮಾ ॥೫೧॥

ಲಬ್ಧಶಕ್ತಿರ್ಲಬ್ಧ ದೇಹಾ ಲಬ್ಧೈಶ್ವರ್ಯ ಸಮುನ್ನತಿಃ~।\\
ಲಬ್ಧವೃದ್ಧಿರ್ಲಬ್ಧಲೀಲಾ ಲಬ್ಧಯೌವನ ಶಾಲಿನೀ ॥೫೨॥

ಲಬ್ಧಾತಿಶಯ ಸರ್ವಾಂಗ ಸೌಂದರ್ಯಾ ಲಬ್ಧ ವಿಭ್ರಮಾ~।\\
ಲಬ್ಧರಾಗಾ ಲಬ್ಧಪತಿರ್ಲಬ್ಧ ನಾನಾಗಮಸ್ಥಿತಿಃ ॥೫೩॥

ಲಬ್ಧ ಭೋಗಾ ಲಬ್ಧ ಸುಖಾ ಲಬ್ಧ ಹರ್ಷಾಭಿಪೂರಿತಾ~।\\
ಹ್ರೀಂಕಾರ ಮೂರ್ತಿರ್ಹ್ರೀಂಕಾರ ಸೌಧಶೃಂಗ ಕಪೋತಿಕಾ ॥೫೪॥

ಹ್ರೀಂಕಾರ ದುಗ್ಧಾಬ್ಧಿ ಸುಧಾ ಹ್ರೀಂಕಾರ ಕಮಲೇಂದಿರಾ~।\\
ಹ್ರೀಂಕಾರಮಣಿ ದೀಪಾರ್ಚಿರ್ಹ್ರೀಂಕಾರ ತರುಶಾರಿಕಾ ॥೫೫॥

ಹ್ರೀಂಕಾರ ಪೇಟಕ ಮಣಿರ್ಹ್ರೀಂಕಾರಾದರ್ಶ ಬಿಂಬಿತಾ~।\\
ಹ್ರೀಂಕಾರ ಕೋಶಾಸಿಲತಾ ಹ್ರೀಂಕಾರಾಸ್ಥಾನ ನರ್ತಕೀ ॥೫೬॥

ಹ್ರೀಂಕಾರ ಶುಕ್ತಿಕಾ ಮುಕ್ತಾಮಣಿರ್ಹ್ರೀಂಕಾರ ಬೋಧಿತಾ~।\\
ಹ್ರೀಂಕಾರಮಯ ಸೌವರ್ಣಸ್ತಂಭ ವಿದ್ರುಮ ಪುತ್ರಿಕಾ ॥೫೭॥

ಹ್ರೀಂಕಾರ ವೇದೋಪನಿಷದ್ ಹ್ರೀಂಕಾರಾಧ್ವರ ದಕ್ಷಿಣಾ~।\\
ಹ್ರೀಂಕಾರ ನಂದನಾರಾಮ ನವಕಲ್ಪಕ ವಲ್ಲರೀ ॥೫೮॥

ಹ್ರೀಂಕಾರ ಹಿಮವದ್ಗಂಗಾ ಹ್ರೀಂಕಾರಾರ್ಣವ ಕೌಸ್ತುಭಾ~।\\
ಹ್ರೀಂಕಾರ ಮಂತ್ರ ಸರ್ವಸ್ವಾ ಹ್ರೀಂಕಾರಪರ ಸೌಖ್ಯದಾ ॥೫೯॥}
\authorline {॥ಇತಿ ಶ್ರೀಲಲಿತಾತ್ರಿಶತೀಸ್ತೋತ್ರಂ ಸಂಪೂರ್ಣಂ ॥}
\section{ಸೌಭಾಗ್ಯವಿದ್ಯಾ ಕವಚಂ}
ಅಸ್ಯ ಶ್ರೀ ಮಹಾತ್ರಿಪುರಸುಂದರೀ ಮಂತ್ರವರ್ಣಾತ್ಮಕ ಕವಚ ಮಹಾಮಂತ್ರಸ್ಯ ದಕ್ಷಿಣಾಮೂರ್ತಿರ್ಋಷಿಃ। ಅನುಷ್ಟುಪ್ ಛಂದಃ। ಶ್ರೀಮಹಾತ್ರಿಪುರಸುಂದರೀ ದೇವತಾ~। ಐಂ ಬೀಜಂ~। ಸೌಃ ಶಕ್ತಿಃ~। ಕ್ಲೀಂ ಕೀಲಕಂ~। ಮಮ ಶರೀರರಕ್ಷಣಾರ್ಥೇ ಜಪೇ ವಿನಿಯೋಗಃ~॥\\
\dhyana{ಬಾಲಾರ್ಕಮಂಡಲಾಭಾಸಾಂ ಚತುರ್ಬಾಹುಂ ತ್ರಿಲೋಚನಾಂ~।\\
ಪಾಶಾಂಕುಶ ಧನುರ್ಬಾಣಾನ್ ಧಾರಯಂತೀಂ ಶಿವಾಂ ಭಜೇ ॥}\\
\as{(ಓಂಐಂಹ್ರೀಂಶ್ರೀಂ)}\\
ಕಕಾರಃ ಪಾತು ಮೇ ಶೀರ್ಷಂ ಏಕಾರಃ ಪಾತು ಫಾಲಕಂ।\\
ಈಕಾರಃ ಪಾತು ಮೇ ವಕ್ತ್ರಂ ಲಕಾರಃ ಪಾತು ಕರ್ಣಕಂ॥೧॥

ಹ್ರೀಂಕಾರಃ ಪಾತು ಹೃದಯಂ ವಾಗ್ಭವಶ್ಚ ಸದಾವತು।\\
ಹಕಾರಃ ಪಾತು ಜಠರಂ ಸಕಾರೋ ನಾಭಿದೇಶಕಂ॥೨॥

ಕಕಾರೋವ್ಯಾದ್ವಸ್ತಿಭಾಗಂ ಹಕಾರಃ ಪಾತು ಲಿಂಗಕಂ।\\
ಲಕಾರೋ ಜಾನುನೀ ಪಾತು ಹ್ರೀಂಕಾರೋ ಜಂಘಯುಗ್ಮಕಂ॥೩॥

ಕಾಮರಾಜಃ ಸದಾ ಪಾತು ಜಠರಾದಿ ಪ್ರದೇಶಕಂ।\\
ಸಕಾರಃ ಪಾತು ಮೇ ಜಂಘೇ ಕಕಾರಃ ಪಾತು ಪೃಷ್ಠಕಂ॥೪॥

ಲಕಾರೋವ್ಯಾನ್ನಿತಂಬಂ ಮೇ ಹ್ರೀಂಕಾರಃ ಪಾತು ಮೂಲಕಂ~।\\
ಶಕ್ತಿಬೀಜಃ ಸದಾ ಪಾತು ಮೂಲಾಧಾರಾದಿ ದೇಶಕಂ॥೫॥

ತ್ರಿಪುರಾ ದೇವತಾ ಪಾತು ತ್ರಿಪುರೇಶೀ ಚ ಸರ್ವದಾ।\\
ತ್ರಿಪುರಾ ಸುಂದರೀ ಪಾತು ತ್ರಿಪುರಾಶ್ರೀ ಸ್ತಥಾವತು॥೬॥

ತ್ರಿಪುರಾ ಮಾಲಿನೀ ಪಾತು ತ್ರಿಪುರಾ ಸಿದ್ಧಿದಾ ವತು।\\
ತ್ರಿಪುರಾಂಬಾ ತಥಾ ಪಾತು ಪಾತು ತ್ರಿಪುರಭೈರವೀ॥೭॥

ಅಣಿಮಾದ್ಯಾ ಸ್ತಥಾ ಪಾಂತು ಬ್ರಾಹ್ಮ್ಯಾದ್ಯಾಃ ಪಾಂತು ಮಾಂ ಸದಾ।\\
ದಶಮುದ್ರಾಸ್ತಥಾ ಪಾಂತು ಕಾಮಾಕರ್ಷಣ ಪೂರ್ವಕಾಃ॥೮॥

ಪಾಂತು ಮಾಂ ಷೋಡಶದಲೇ ಯಂತ್ರೇನಂಗ ಕುಮಾರಿಕಾಃ।\\
ಪಾಂತು ಮಾಂ ಪೃಷ್ಠಪತ್ರೇ ತು ಸರ್ವಸಂಕ್ಷೋಭಣಾದಿಕಾಃ॥೯॥

ಪಾಂತು ಮಾಂ ದಶಕೋಣೇ ತು ಸರ್ವಸಿದ್ಧಿ ಪ್ರದಾಯಿಕಾಃ।\\
ಪಾಂತು ಮಾಂ ಬಾಹ್ಯ ದಿಕ್ಕೋಣೇ ಮಧ್ಯ ದಿಕ್ಕೋಣಕೇ ತಥಾ॥೧೦॥

ಸರ್ವಜ್ಞಾ ದ್ಯಾಸ್ತಥಾ ಪಾಂತು ಸರ್ವಾಭೀಷ್ಟ ಪ್ರದಾಯಿಕಾಃ।\\
ವಶಿನ್ಯಾದ್ಯಾಸ್ತಥಾ ಪಾಂತು ವಸು ಪತ್ರಸ್ಯ ದೇವತಾಃ॥೧೧॥

ತ್ರಿಕೋಣ ಸ್ಯಾಂತ ರಾಲೇ ತು ಪಾಂತು ಮಾಮಾಯುಧಾನಿ ಚ।\\
ಕಾಮೇಶ್ವರ್ಯಾದಿಕಾಃ ಪಾಂತು ತ್ರಿಕೋಣೇ ಕೋಣಸಂಸ್ಥಿತಾಃ॥೧೨॥

ಬಿಂದುಚಕ್ರೇ ತಥಾ ಪಾತು ಮಹಾತ್ರಿಪುರಸುಂದರೀ।\as{(ಶ್ರೀಂಹ್ರೀಂಐಂ)}\\
ಇತೀದಂ ಕವಚಂ ದೇವಿ ಕವಚಂ ಮಂತ್ರಸೂಚಕಂ॥೧೩॥

ಯಸ್ಮೈ ಕಸ್ಮೈ ನ ದಾತವ್ಯಂ ನ ಪ್ರಕಾಶ್ಯಂ ಕಥಂಚನ।\\
ಯಸ್ತ್ರಿಸಂಧ್ಯಂ ಪಠೇದ್ದೇವಿ ಲಕ್ಷ್ಮೀಸ್ತಸ್ಯ ಪ್ರಜಾಯತೇ॥೧೪॥

ಅಷ್ಟಮ್ಯಾಂ ಚ ಚತುರ್ದಶ್ಯಾಂ ಯಃ ಪಠೇತ್ ಪ್ರಯತಃ ಸದಾ।\\
ಪ್ರಸನ್ನಾ ಸುಂದರೀ ತಸ್ಯ ಸರ್ವಸಿದ್ಧಿಪ್ರದಾಯಿನೀ॥೧೫॥
\authorline{॥ಇತಿ ಶ್ರೀ ರುದ್ರಯಾಮಲೇ ತಂತ್ರೇ ತ್ರಿಪುರಾ ಹೃದಯೇ ಕವಚರಹಸ್ಯಂ ಸಂಪೂರ್ಣಂ ॥}
\section{ಮಾತೃಕಾಸ್ತೋತ್ರಂ}
ಗಣೇಶ ಗ್ರಹ ನಕ್ಷತ್ರಯೋಗಿನೀ ರಾಶಿ ರೂಪಿಣೀಂ~।\\
ದೇವೀಂ ಮಂತ್ರಮಯೀಂ ನೌಮಿ ಮಾತೃಕಾಂ ಪೀಠ ರೂಪಿಣೀಂ॥೧॥

ಪ್ರಣಮಾಮಿ ಮಹಾದೇವೀಂ ಮಾತೃಕಾಂ ಪರಮೇಶ್ವರೀಂ।\\
ಕಾಲಹಲ್ಲೋಹಲೋಲ್ಲೋಲ ಕಲನಾಶಮಕಾರಿಣೀಂ॥೨॥

ಯದಕ್ಷರೈಕಮಾತ್ರೇಪಿ ಸಂಸಿದ್ಧೇ ಸ್ಪರ್ಧತೇ ನರಃ।\\
ರವಿತಾರ್ಕ್ಷ್ಯೇಂದುಕಂದರ್ಪಶಂಕರಾನಲವಿಷ್ಣುಭಿಃ॥೩॥

ಯದಕ್ಷರ ಶಶಿ ಜ್ಯೋತ್ಸ್ನಾಮಂಡಿತಂ ಭುವನತ್ರಯಂ।\\
ವಂದೇ ಸರ್ವೇಶ್ವರೀಂ ದೇವೀಂ ಮಹಾ ಶ್ರೀ ಸಿದ್ಧಮಾತೃಕಾಂ॥೪॥

ಯದಕ್ಷರಮಹಾಸೂತ್ರಪ್ರೋತಮೇತಜ್ಜಗತ್ತ್ರಯಂ।\\
ಬ್ರಹ್ಮಾಂಡಾದಿಕಟಾಹಾಂತಂ ತಾಂ ವಂದೇ ಸಿದ್ಧಮಾತೃಕಾಂ॥೫॥

ಯದೇಕಾದಶಮಾಧಾರಂ ಬೀಜಂ ಕೋಣತ್ರಯೋದ್ಭವಂ।\\
ಬ್ರಹ್ಮಾಂಡಾದಿ ಕಟಾಹಾಂತಂ ಜಗದದ್ಯಾಪಿ ದೃಶ್ಯತೇ॥೬॥

ಅಕಚಾದಿಟತೋನ್ನದ್ಧಪಯಶಾಕ್ಷರವರ್ಗಿಣೀಂ।\\
ಜ್ಯೇಷ್ಠಾಂಗಬಾಹುಹೃತ್ಪೃಷ್ಠಕಟಿಪಾದನಿವಾಸಿನೀಂ॥೭॥

ತಾಮೀಕರಾಕ್ಷರೋದ್ಧಾರಾಂ ಸಾರಾತ್ಸಾರಾಂ ಪರಾತ್ಪರಾಂ।\\
ಪ್ರಣಮಾಮಿ ಮಹಾದೇವೀಂ ಪರಮಾನಂದರೂಪಿಣೀಂ॥೮॥

ಅದ್ಯಾಪಿ ಯಸ್ಯಾ ಜಾನಂತಿ ನ ಮನಾಗಪಿ ದೇವತಾಃ।\\
ಕೇಯಂ ಕಸ್ಮಾತ್ ಕ್ವ ಕೇನೇತಿ ಸರೂಪಾರೂಪಭಾವನಾಂ॥೯॥

ವಂದೇ ತಾಮಹ ಮಕ್ಷಯ್ಯಾಂ ಕ್ಷಕಾರಾಕ್ಷರರೂಪಿಣೀಂ~।\\
ದೇವೀಂಕುಲಕಲೋಲ್ಲಾಸ ಪ್ರೋಲ್ಲಸಂತೀಂ ಪರಾಂಶಿವಾಂ॥೧೦॥

ವರ್ಗಾನು ಕ್ರಮ ಯೋಗೇನ ಯಸ್ಯಾಂ ಮಾತ್ರಷ್ಟಕಂ ಸ್ಥಿತಂ।\\
ವಂದೇ ತಾಮಷ್ಟವರ್ಗೋತ್ಥಮಹಾಸಿದ್ಧ್ಯಷ್ಟಕೇಶ್ವರೀಂ ॥೧೧॥

ಕಾಮಪೂರ್ಣಜಕಾರಾಖ್ಯ ಶ್ರೀಪೀಠಾಂತರ್ನಿವಾಸಿನೀಂ।\\
ಚತುರಾಜ್ಞಾಕೋಶಮೂಲಾಂ ನೌಮಿ ಶ್ರೀ ತ್ರಿಪುರಾಮಹಂ॥೧೨॥

ಇತಿ ದ್ವಾದಶಭಿಃ ಶ್ಲೋಕೈಃ ಸ್ತವನಂ ಸರ್ವ ಸಿದ್ಧಿ ಕೃತ್।\\
ದೇವ್ಯಾ ಸ್ತ್ವಖಂಡ ರೂಪಾಯಾಃ ಸ್ತವನಂ ತವ ತಥ್ಯತಃ॥೧೩॥
\authorline{ಇತಿ ಸರ್ವಸಿದ್ಧಿಕೃತ್ಸ್ತೋತ್ರಂ ಸಂಪೂರ್ಣಂ॥}
\section{ಬಲಿದಾನವಿಧಿಃ}
\as{೪ ಐಂ ವ್ಯಾಪಕಮಂಡಲಾಯ ನಮಃ ॥}\\ (ಇತಿ ಮಂಡಲಂ ಗಂಧಾಕ್ಷತಪುಷ್ಪೈರರ್ಚಯಿತ್ವಾ ಧೂಪಂ ದೀಪಂ ಚ ಪ್ರಜ್ವಾಲ್ಯ)

೪ ಯಶೋ ದೇಹಿ ಜಯಂ ದೇಹಿ ತ್ರಿಪುರೇ ಭಕ್ತವತ್ಸಲೇ ।\\ ಗೃಹಾಣೇಮಂ ಬಲಿಂ ಮಾತರ್ದೇಹಿ ಸಿದ್ಧಿಮನುತ್ತಮಾಂ ॥

(ಐಶಾನ್ಯಾಮ್)\\
\as{೪ ವಂ ಏಹ್ಯೇಹಿ ವಟುಕ ಕಪಿಲಜಟಾಭಾರಭಾಸುರ ತ್ರಿನೇತ್ರ ಜ್ವಾಲಾಮುಖ ಸರ್ವವಿಘ್ನಾನ್ನಾಶಯ ನಾಶಯ ಸರ್ವೋಪಚಾರಸಹಿತಂ ಬಲಿಂ ಗೃಹ್ಣ ಗೃಹ್ಣ ಸ್ವಾಹಾ ॥}

೪ ಬಲಿದಾನೇನ ಸಂತುಷ್ಟೋ ವಟುಕಃ ಸರ್ವಸಿದ್ಧಿದಃ ।\\
ಶಾಂತಿಂ ಕರೋತು ಮೇ ನಿತ್ಯಂ ಭೂತವೇತಾಳಸೇವಿತಃ ॥

(ಆಗ್ನೇಯ್ಯಾಂ)\\
\as{೪ ಯಾಂ ಸರ್ವಯೋಗಿನೀಭ್ಯೋ ಹುಂ ಫಟ್ ಸ್ವಾಹಾ ॥}

೪ ಊರ್ಧ್ವಂ ಬ್ರಹ್ಮಾಂಡತೋ ವಾ ದಿವಿ ಗಗನತಲೇ ಭೂತಲೇಽನ್ಯಸ್ಥಲೇ ವಾ\\
ಪಾತಾಳೇ ವಾ ಸಲಿಲಪವನಯೋರ್ಯತ್ರಕುತ್ರಸ್ಥಿತಾ ವಾ ।\\
ಕ್ಷೇತ್ರೋಪಕ್ಷೇತ್ರಪೀಠಾದಿಷು ಚ ಕೃತಪದಾ ಧೂಪದೀಪಾದಿಮಾಂಸೈಃ\\
ಪ್ರೀತಾ ದೇವ್ಯಃ ಸದಾ ನಃ ಕೃತಬಲಿವಿಧಿನಾ ಪಾಂತು ವೀರೇಂದ್ರವಂದ್ಯಾಃ ॥

(ನೈರ್ಋತ್ಯಾಂ)\\
\as{೪ ಕ್ಷಾಂ ಕ್ಷಃ ಕ್ಷೇತ್ರಪಾಲ ಧೂಪದೀಪಾದಿಸಹಿತಂ ಬಲಿಂ ಗೃಹ್ಣ ಗೃಹ್ಣ ಸ್ವಾಹಾ ॥}

೪ ಯೋಽಸ್ಮಿನ್ ಕ್ಷೇತ್ರನಿವಾಸೀ ಚ ಕ್ಷೇತ್ರಪಾಲಃ ಸ ಕಿಂಕರಃ ।\\
ಪ್ರೀತೋಽಯಂ ಬಲಿದಾನೇನ ಸರ್ವರಕ್ಷಾಂ ಕರೋತು ಮೇ ॥

(ವಾಯವ್ಯಾಂ)\\
\as{೪ ಗಾಂ ಗೀಂ ಗೂಂ ಗಣಪತಯೇ ವರ ವರದ\\ ಸರ್ವಜನಂ ಮೇ ವಶಮಾನಯ ಬಲಿಂ ಗೃಹ್ಣ ಗೃಹ್ಣ ಸ್ವಾಹಾ ॥}

೪ ಸರ್ವದಾ ಸರ್ವಕಾರ್ಯೇಷು ನಿರ್ವಿಘ್ನಂ ಮಮ ಸಿದ್ಧಯೇ ।\\
ಶಾಂತಿಂ ಕರೋತು ಸತತಂ ವಿಘ್ನರಾಜಃ ಸಶಕ್ತಿಕಃ ॥

(ಮಂಡಲಮಧ್ಯೇ)\\
\as{೪ ಓಂ ಹ್ರೀಂ ಸರ್ವವಿಘ್ನಕೃದ್ಭ್ಯಃ ಸರ್ವಭೂತೇಭ್ಯೋ ಹುಂ ಫಟ್ ಸ್ವಾಹಾ ॥}

ಭೂತಾ ಯೇ ವಿವಿಧಾಕಾರಾ ದಿವ್ಯಾ ಭೂಮ್ಯಂತರಿಕ್ಷಗಾಃ ।\\
ಪಾತಾಳತಳಸಂಸ್ಥಾಶ್ಚ ಶಿವಯೋಗೇನ ಭಾವಿತಾಃ ॥

ಕ್ರೂರಾದ್ಯಾಃ ಸಪ್ತಲೋಕಸ್ಥಾಃ ಇಂದ್ರಾದ್ಯಾಶಾವ್ಯವಸ್ಥಿತಾಃ ।\\
ತೃಪ್ಯಂತು ಪ್ರೀತಮನಸೋ ಭೂತಾ ಗೃಹ್ಣಂತ್ವಿಮಾಂ ಬಲಿಮ್ ॥

ಬಲಿದಾನೇನ ಸಂತುಷ್ಟಾ  ಕ್ಷಮಧ್ವಂ ಬಲಿದೇವತಾಃ ।\\
ಯಥಾಸುಖಂ ವಿಹರ್ತವ್ಯಂ ಯಥೇಷ್ಟಾಸು ದಿಶಾಸು ಚ ॥
\authorline{ಇತಿ ಬಲಿದಾನವಿಧಿಃ}
\newpage
\section{ಪೂಜಾಸಮರ್ಪಣಂ ದೇವತೋದ್ವಾಸನಂ ಚ}
ಸಾಧು ವಾಸಾಧು ವಾ ಕರ್ಮ ಯದ್ಯದಾಚರಿತಂ ಮಯಾ ।\\ ತತ್ಸರ್ವಂ ಕೃಪಯಾ ದೇವಿ ಗೃಹಾಣಾರಾಧನಂ ಮಮ ॥

ದೇವನಾಥ ಗುರೋ ಸ್ವಾಮಿನ್ ದೇಶಿಕ ಸ್ವಾತ್ಮನಾಯಕ ।\\ ತ್ರಾಹಿ ತ್ರಾಹಿ ಕೃಪಾಸಿಂಧೋ ಪೂಜಾಂ ಪೂರ್ಣತರಾಂ ಕುರು ॥

ಜ್ಞಾನತೋಽಜ್ಞಾನತೋ ವಾಪಿ ಯನ್ಮಯಾಚರಿತಂ ಶಿವೇ ।\\ ತವಕೃತ್ಯಮಿತಿ ಜ್ಞಾತ್ವಾ ಕ್ಷಮಸ್ವ ಪರಮೇಶ್ವರಿ ॥

ಹೃತ್ಪದ್ಮಕರ್ಣಿಕಾಮಧ್ಯೇ ಶಿವೇನ ಸಹ ಶಂಕರಿ ।\\ ಪ್ರವಿಶ ತ್ವಂ ಮಹಾದೇವಿ ಸರ್ವೈರಾವರಣೈಃ ಸಹ ।

ಬದ್ಧ್ವಾ ತ್ವಾಂ ಖೇಚರೀಮುದ್ರಾಂ ಕ್ಷಮಸ್ವೋದ್ವಾಸಯಾಮ್ಯಹಂ ।\\ ಹೃಚ್ಚಕ್ರೇ ತಿಷ್ಠ ಮೇ ದೇವಿ ಸರ್ವೈರಾವರಣೈಃ ಸಹ ॥


\section{ಶಾಂತಿಸ್ತವಃ}
ಜಯಂತು ದೇವ್ಯೋ ಹರಪಾದಪಂಕಜಂ \\
ಪ್ರಪನ್ನದಾಮಾಮೃತಮೋಕ್ಷದಾಯಕಂ ।\\
ಅನಂತಸಿದ್ಧಾಂತ ಗತಿಪ್ರಬೋಧಕಂ\\
ನಮಾಮಿ ಚಾಷ್ಟಾಷ್ಟಕಯೋಗಿನೀಕುಲಮ್ ॥

ಯೋಗಿನೀಚಕ್ರಮಧ್ಯಸ್ಥಂ ಮಾತೃಮಂಡಲವೇಷ್ಟಿತಂ ।\\
ನಮಾಮಿ ಶಿರಸಾ ನಾಥಂ ಶ್ರೀಗುರುಂ ಭೈರವೀಪ್ರಿಯಂ ॥

ಅನಾದಿಘೋರಸಂಸಾರವ್ಯಾಧಿಧ್ವಂಸೈಕಹೇತವೇ ।\\
ನಮಃ ಶ್ರೀನಾಥವೈದ್ಯಾಯ ಭವೌಷಧ ವಿಧಾಯಿನೇ ॥

ಆಪದೋ ದುರಿತಂ ರೋಗಾಃ ಸಮಯಾಚಾರಲಂಘನಾತ್ ।\\
ತೇ ಸರ್ವೇ ಮೇ ವಿನಶ್ಯಂತು ದಿವ್ಯಚಕ್ರಸ್ಯ ಮೇಲನಾತ್ ॥

ಸಂಪೂಜಕಾನಾಂ ಪರಿಪಾಲಕಾನಾಂ ಯತೇಂದ್ರಿಯಾಣಾಂ ಚ ತಪೋಧನಾನಾಂ\\
ದೇಶಸ್ಯ ರಾಷ್ಟ್ರಸ್ಯ ಕುಲಸ್ಯ ವೃದ್ಧಿಂ ಕರೋತು ಶಾಂತಿಂ ಭಗವಾನ್ಕುಲೇಶಃ ॥

ಆಯುರಾರೋಗ್ಯಮೈಶ್ವರ್ಯಂ ಕೀರ್ತಿಲಾಭಃ ಸುಖಂ ಜಯಃ ।\\
ಕಾಂತಿರ್ಮನೋಹರಾ ಚಾಸ್ತು ಪಾಂತು ಸರ್ವಾಶ್ಚ ದೇವತಾಃ ॥

ಯಾ ದಿವ್ಯಾಃ ಕುಲಸಂಭವಾಃ ಕ್ಷಿತಿಷು ಯಾ ಯಾ ದೇವತಾಸ್ತೋಯಗಾಃ\\
ಯಾ ನಿತ್ಯಪ್ರಥಿತಪ್ರಭಾವ ವಿಭವಾ ಯಾ ಮಾತರಿಶ್ವಾಶ್ರಯಾಃ ।\\
ಯಾ ವ್ಯೋಮಾಹಿತಮಂಡಲಾಮೃತಮಯಾ ಯಾಃ ಸರ್ವಗಾಃ ಸರ್ವದಾಃ\\
ತಾಃ ಸರ್ವಾಃ ಕುಲಮಾರ್ಗಪಾಲನರತಾಃ ಶಾಂತಿಂ ಪ್ರಯಚ್ಛಂತು ಮೇ ॥

ಯಾಶ್ಚಕ್ರಕ್ರಮಭೂಮಿಕಾವಸತಯೋ ನಾಡೀಷು ಯಾಃ ಸಂಸ್ಥಿತಾಃ\\
ಯಾಃ ಕಾಯೋದ್ಗಮರೋಮಕೂಪನಿಲಯಾ ಯಾಃ ಸಂಸ್ಥಿತಾ ಧಾತುಷು ।\\
ಉಚ್ಛ್ವಾಸೋರ್ಮಿಮರುತ್ತರಂಗನಿಲಯಾ ನಿಶ್ವಾಸವಾತಾಶ್ರಯಾಃ \\
ತಾ ದೇವ್ಯಃ ಪರಿಪಂಥಿಭಕ್ಷಣರತಾಃ ತೃಪ್ಯಂತು ಕೌಲಾರ್ಚಿತಾಃ ॥

ಯಾಸಾಮಾಜ್ಞಾ ಪ್ರಭಾವೇಣ ಸುಸ್ಥಿತಂ ಭುವನತ್ರಯಂ ।\\
ನಮಸ್ತಾಭ್ಯೋ ನಮಸ್ತಾಭ್ಯೋ ಯೋಗಿನೀಭ್ಯೋ ನಿರಂತರಂ ॥

ಯದ್ಯೇಷಾ ಭೈರವೀಶಕ್ತಿಃ ಯದಿ ಭೈರವಶಾಸನಮ್ ।\\
ಯದ್ಯೇಷ ಧರ್ಮಃ ಶ್ರುತಿಷು ನಶ್ಯಂತು ಕುಲದೂಷಕಾಃ ॥

ನಂದಂತು ಸಾಧಕಕುಲಾನ್ಯಣಿಮಾದಿಸಿದ್ಧಾಃ\\
ಶಾಪಾಃ ಪತಂತು ಸಮಯದ್ವಿಷಿ ಯೋಗಿನೀನಾಮ್ ।\\
ಸಾ ಶಾಂಭವೀ ಸ್ಫುರತು ಕಾಪಿ ಮಮಾಪ್ಯವಸ್ಥಾ\\
ಯಸ್ಯಾಂ ಗುರೋಶ್ಚರಣ ಪಂಕಜಮೇವ ಲಭ್ಯಃ ॥

ಶಿವಾದ್ಯವನಿಪರ್ಯಂತಂ ಬ್ರಹ್ಮಾದಿಸ್ತಂಬಸಂಯುತಮ್ ।\\
ಕಾಲಾಗ್ನ್ಯಾದಿ ಶಿವಾಂತಂ ಚ ಜಗದ್ಯಜ್ಞೇನ ತೃಪ್ಯತು ॥
\section{ವಿಶೇಷಾರ್ಘ್ಯೋದ್ವಾಸನಮ್}
ಮೂಲೇನ ವಿಶೇಷಾರ್ಘ್ಯಪಾತ್ರಂ ಆಮಸ್ತಕಮುದ್ಧೃತ್ಯ ತತ್ರಸ್ಥಂ ಕ್ಷೀರಂ ಪಾತ್ರಾಂತರೇಣಾದಾಯ \\
\as{೪ ಆರ್ದ್ರಂ॒ ಜ್ವಲ॑ತಿ॒ ಜ್ಯೋತಿರ॒ಹಮ॑ಸ್ಮಿ । ಜ್ಯೋತಿ॒ರ್ಜ್ವಲ॑ತಿ॒ ಬ್ರಹ್ಮಾ॒ಹಮ॑ಸ್ಮಿ । ಯೋ॑ಽಹ॒ಮಸ್ಮಿ ॒ಬ್ರಹ್ಮಾ॒ಹಮ॑ಸ್ಮಿ । ಅ॒ಹಮ॑ಸ್ಮಿ ॒ಬ್ರಹ್ಮಾ॒ಹಮ॑ಸ್ಮಿ । ಅ॒ಹಮೇ॒ವಾಹಂ ಮಾಂ ಜು॑ಹೋಮಿ॒ ಸ್ವಾಹಾ᳚ ॥} ಇತಿ ಆತ್ಮನಃ ಕುಂಡಲಿನ್ಯಗ್ನೌ ಹೋಮಬುದ್ಧ್ಯಾ  ಪ್ರಾಶ್ನೀಯಾತ್ ॥

\section{ಭಾವನೋಪನಿಷತ್}
ಓಂ ಭದ್ರಂ ಕರ್ಣೇಭಿಃ ಇತಿ ಶಾಂತಿಃ ॥
ಓಂ ಶ್ರೀಗುರುಃ ಸರ್ವಕಾರಣಭೂತಾ ಶಕ್ತಿಃ~। ತೇನ ನವರಂಧ್ರರೂಪೋ ದೇಹಃ~। ನವಚಕ್ರರೂಪಂ ಶ್ರೀಚಕ್ರಂ~। ವಾರಾಹೀ ಪಿತೃರೂಪಾ~। ಕುರುಕುಲ್ಲಾ ಬಲಿದೇವತಾ ಮಾತಾ~। ಪುರುಷಾರ್ಥಾಃ ಸಾಗರಾಃ~। ದೇಹೋ ನವರತ್ನದ್ವೀಪಃ~। ತ್ವಗಾದಿಸಪ್ತಧಾತುರೋಮ ಸಂಯುಕ್ತಃ। ಸಂಕಲ್ಪಾಃ ಕಲ್ಪತರವಸ್ತೇಜಃ ಕಲ್ಪಕೋ\-ದ್ಯಾನಂ~। ರಸನಯಾ ಭಾವ್ಯಮಾನಾ ಮಧುರಾಮ್ಲತಿಕ್ತಕಟುಕಷಾಯಲವಣರಸಾಃ ಷಡೃತವಃ~। ಜ್ಞಾನಮರ್ಘ್ಯಂ। ಜ್ಞೇಯಂ ಹವಿಃ~। ಜ್ಞಾತಾ ಹೋತಾ~। ಜ್ಞಾತೃಜ್ಞಾನಜ್ಞೇಯಾನಾಮಭೇದಭಾವನಂ ಶ್ರೀಚಕ್ರಪೂಜನಂ~। ನಿಯತಿಃ ಶೃಂಗಾರಾದಯೋ ರಸಾ ಅಣಿಮಾದಯಃ~। ಕಾಮಕ್ರೋಧ ಲೋಭಮೋಹ ಮದ ಮಾತ್ಸರ್ಯ ಪುಣ್ಯ ಪಾಪಮಯಾ ಬ್ರಾಹ್ಮ್ಯಾದ್ಯಷ್ಟಶಕ್ತಯಃ~। ಆಧಾರನವಕಂ ಮುದ್ರಾಶಕ್ತಯಃ~। ಪೃಥಿವ್ಯಪ್ತೇಜೋವಾಯ್ವಾಕಾಶ ಶ್ರೋತ್ರತ್ವಕ್ಚಕ್ಷುರ್ಜಿಹ್ವಾ ಘ್ರಾಣ ವಾಕ್ಪಾಣಿ ಪಾದ ಪಾಯೂಪಸ್ಥಾನಿ ಮನೋವಿಕಾರಾಃ ಕಾಮಾಕರ್ಷಣ್ಯಾದಿ ಷೋಡಶ ಶಕ್ತಯಃ~। ವಚನಾದಾನಗಮನವಿಸರ್ಗಾನಂದ ಹಾನೋಪಾದಾನೋಪೇಕ್ಷಾಖ್ಯ ಬುದ್ಧಯೋನಂಗಕುಸುಮಾದ್ಯಷ್ಟೌ~। ಅಲಂಬುಸಾ ಕುಹೂರ್ವಿಶ್ವೋದರಾ ವಾರಣಾ ಹಸ್ತಿಜಿಹ್ವಾ ಯಶೋವತೀ ಪಯಸ್ವಿನೀ ಗಾಂಧಾರೀ ಪೂಷಾ ಶಂಖಿನೀ ಸರಸ್ವತೀ ಇಡಾ ಪಿಂಗಲಾ ಸುಷುಮ್ನಾ ಚೇತಿ ಚತುರ್ದಶ ನಾಡ್ಯಃ ಸರ್ವಸಂಕ್ಷೋಭಿಣ್ಯಾದಿ ಚತುರ್ದಶ ಶಕ್ತಯಃ~। ಪ್ರಾಣಾಪಾನ ವ್ಯಾನೋದಾನ ಸಮಾನ ನಾಗ ಕೂರ್ಮ ಕೃಕರ ದೇವದತ್ತ ಧನಂಜಯಾ ಇತಿ ದಶ ವಾಯವಃ ಸರ್ವಸಿದ್ಧಿಪ್ರದಾದಿ ಬಹಿರ್ದಶಾರಗಾ ದೇವತಾಃ~। ಏತದ್ವಾಯುಸಂಸರ್ಗ ಕೋಪಾಧಿ ಭೇಧೇನ ರೇಚಕಃ ಪಾಚಕಃ ಶೋಷಕೋ ದಾಹಕಃ ಪ್ಲಾವಕ ಇತಿ ಪ್ರಾಣಮುಖ್ಯಶ್ರೀತ್ವೇನ ಪಂಚಧಾ ಜಠರಾಗ್ನಿರ್ಭವತಿ~। ಕ್ಷಾರಕ ಉದ್ಗಾರಕಃ ಕ್ಷೋಭಕೋ ಜೃಂಭಕೋ ಮೋಹಕ ಇತಿ ನಾಗಪ್ರಾಧಾನ್ಯೇನ ಪಂಚವಿಧಾಸ್ತೇ ಮನುಷ್ಯಾಣಾಂ ದೇಹಗಾ ಭಕ್ಷ್ಯಭೋಜ್ಯ ಚೋಷ್ಯ ಲೇಹ್ಯ ಪೇಯಾತ್ಮಕ ಪಂಚವಿಧಮನ್ನಂ ಪಾಚಯಂತಿ। ಏತಾ ದಶ ವಹ್ನಿಕಲಾಃ ಸರ್ವಜ್ಞಾದ್ಯಾ ಅಂತರ್ದಶಾರಗಾ ದೇವತಾಃ~। ಶೀತೋಷ್ಣಸುಖದುಃಖೇಚ್ಛಾಃ ಸತ್ತ್ವಂ ರಜಸ್ತಮೋ ವಶಿನ್ಯಾದಿಶಕ್ತಯೋಽಷ್ಟೌ~। ಶಬ್ದಾದಿ ತನ್ಮಾತ್ರಾಃ ಪಂಚಪುಷ್ಪಬಾಣಾಃ। ಮನ ಇಕ್ಷುಧನುಃ~। ರಾಗಃ ಪಾಶಃ~। ದ್ವೇಷೋಂಕುಶಃ~। ಅವ್ಯಕ್ತಮಹದಹಂಕಾರಾಃ ಕಾಮೇಶ್ವರೀವಜ್ರೇಶ್ವರೀ ಭಗಮಾಲಿನ್ಯೋಂತಸ್ತ್ರಿಕೋಣಗಾ ದೇವತಾಃ~। ನಿರುಪಾಧಿಕ ಸಂವಿದೇವ ಕಾಮೇಶ್ವರಃ। ಸದಾನಂದಪೂರ್ಣ ಸ್ವಾತ್ಮೈವ ಪರದೇವತಾ ಲಲಿತಾ~। ಲೌಹಿತ್ಯಮೇತಸ್ಯ ಸರ್ವಸ್ಯ ವಿಮರ್ಶಃ~। ಅನನ್ಯಚಿತ್ತತ್ವೇನ ಚ ಸಿದ್ಧಿಃ~। ಭಾವನಾಯಾಃ ಕ್ರಿಯಾ ಉಪಚಾರಃ~। ಅಹಂತ್ವಮಸ್ತಿನಾಸ್ತಿಕರ್ತವ್ಯಮಕರ್ತವ್ಯ ಮುಪಾಸಿತವ್ಯಮಿತಿ ವಿಕಲ್ಪಾನಾ ಮಾತ್ಮನಿ ವಿಲಾಪನಂ ಹೋಮಃ~। ಭಾವನಾವಿಷಯಾಣಾ ಮಭೇದಭಾವನಾ ತರ್ಪಣಂ~। ಪಂಚದಶತಿಥಿರೂಪೇಣ ಕಾಲಸ್ಯ ಪರಿಣಾಮಾವಲೋಕನಂ ಪಂಚದಶನಿತ್ಯಾಃ~। ಏವಂ ಮುಹೂರ್ತತ್ರಿತಯಂ ಮುಹೂರ್ತದ್ವಿತಯಂ ಮುಹೂರ್ತಮಾತ್ರಂ ವಾ ಭಾವನಾಪರೋ ಜೀವನ್ಮುಕ್ತೋ ಭವತಿ~। ಸ ಏವ ಶಿವಯೋಗೀತಿ ಗದ್ಯತೇ~। ಕಾದಿಮತೇನಾಂತಶ್ಚಕ್ರ ಭಾವನಾಃ ಪ್ರತಿಪಾದಿತಾಃ। ಯ ಏವಂ ವೇದ। ಸೋಽಥರ್ವಶಿರ್ಷೋಽಧೀತೇ ॥ ಓಂ ಭದ್ರಂ ಕರ್ಣೇಭಿಃ ಇತಿ ಶಾಂತಿಃ ॥
\chapter*{\center ಭಾವನಾಚಕ್ರನ್ಯಾಸಃ}
\thispagestyle{empty}
\section{ಗುರುಪ್ರಾರ್ಥನಾ}
ಓಂ ಆಬ್ರಹ್ಮಲೋಕಾದಾಶೇಷಾತ್ ಆಲೋಕಾಲೋಕಪರ್ವತಾತ್।\\
ಯೇ ವಸಂತಿ ದ್ವಿಜಾ ದೇವಾಃ ತೇಭ್ಯೋ ನಿತ್ಯಂ ನಮಾಮ್ಯಹಂ॥

ಓಂ ನಮೋ ಬ್ರಹ್ಮಾದಿಭ್ಯೋ ಬ್ರಹ್ಮವಿದ್ಯಾಸಂಪ್ರದಾಯಕರ್ತೃಭ್ಯೋ\\ ವಂಶರ್ಷಿಭ್ಯೋ ಮಹದ್ಭ್ಯೋ ನಮೋ ಗುರುಭ್ಯಃ~॥

ಸರ್ವೋಪಪ್ಲವರಹಿತಃ ಪ್ರಜ್ಞಾನಘನಃ ಪ್ರತ್ಯಗರ್ಥೋ ಬ್ರಹ್ಮೈವಾಹಮಸ್ಮಿ ॥

ಸಚ್ಚಿದಾನಂದ ರೂಪಾಯ ಬಿಂದು ನಾದಾಂತರಾತ್ಮನೇ~।\\
ಆದಿಮಧ್ಯಾಂತ ಶೂನ್ಯಾಯ ಗುರೂಣಾಂ ಗುರವೇ ನಮಃ ॥

ಶುದ್ಧಸ್ಫಟಿಕಸಂಕಾಶಂ ದ್ವಿನೇತ್ರಂ ಕರುಣಾನಿಧಿಂ~।\\
ವರಾಭಯಪ್ರದಂ ವಂದೇ ಶ್ರೀಗುರುಂ ಶಿವರೂಪಿಣಂ ॥

ಗುರುರ್ಬ್ರಹ್ಮಾ ಗ್ರುರುರ್ವಿಷ್ಣುಃ ಗುರುರ್ದೇವೋ ಮಹೇಶ್ವರಃ~।\\
ಗುರುಃ ಸಾಕ್ಷಾತ್ ಪರಂ ಬ್ರಹ್ಮ ತಸ್ಮೈ ಶ್ರೀ ಗುರವೇ ನಮಃ ॥

ಶುಕ್ಲಾಂಬರಧರಂ ವಿಷ್ಣುಂ ಶಶಿವರ್ಣಂ ಚತುರ್ಭುಜಂ~।\\
ಪ್ರಸನ್ನವದನಂ ಧ್ಯಾಯೇತ್ ಸರ್ವವಿಘ್ನೋಪಶಾಂತಯೇ ॥

ಮಂಜುಶಿಂಜಿತ ಮಂಜೀರಂ ವಾಮಮರ್ಧಂ ಮಹೇಶಿತುಃ~।\\
ಆಶ್ರಯಾಮಿ ಜಗನ್ಮೂಲಂ ಯನ್ಮೂಲಂ ವಚಸಾಮಪಿ ॥

ಶ್ರೀ ವಿದ್ಯಾಂ ಜಗತಾಂ ಧಾತ್ರೀಂ ಸರ್ಗಸ್ಥಿತಿ ಲಯೇಶ್ವರೀಂ~।\\
ನಮಾಮಿ ಲಲಿತಾಂ ನಿತ್ಯಂ ಭಕ್ತಾನಾಮಿಷ್ಟದಾಯಿನೀಂ ॥

ಶ್ರೀನಾಥಾದಿ ಗುರುತ್ರಯಂ ಗಣಪತಿಂ ಪೀಠತ್ರಯಂ ಭೈರವಂ\\
ಸಿದ್ಧೌಘಂ ಬಟುಕತ್ರಯಂ ಪದಯುಗಂ ದೂತೀಕ್ರಮಂ ಮಂಡಲಂ~।\\
ವೀರದ್ವ್ಯಷ್ಟ ಚತುಷ್ಕಷಷ್ಟಿನವಕಂ ವೀರಾವಲೀಪಂಚಕಂ\\
ಶ್ರೀಮನ್ಮಾಲಿನಿಮಂತ್ರರಾಜಸಹಿತಂ ವಂದೇ ಗುರೋರ್ಮಂಡಲಂ॥

\section{ಗುರುಪಾದುಕಾಮಂತ್ರಃ}
\dhyana{ಓಂಐಂಹ್ರೀಂಶ್ರೀಂಐಂಕ್ಲೀಂಸೌಃ ಹಂಸಃ ಶಿವಃ ಸೋಽಹಂ ಹ್‌ಸ್‌ಖ್‌ಫ್ರೇಂ ಹಸಕ್ಷಮಲವರಯೂಂ ಹ್‌ಸೌಃ ಸಹಕ್ಷಮಲವರಯೀಂ ಸ್‌ಹೌಃ ಹಂಸಃ ಶಿವಃ ಸೋಽಹಂ~॥} ಸ್ವರೂಪ ನಿರೂಪಣ ಹೇತವೇ ಶ್ರೀಗುರವೇ ನಮಃ~। ಶ್ರೀಪಾದುಕಾಂ ಪೂಜಯಾಮಿ ನಮಃ ॥\\
\dhyana{೭ ಸೋಽಹಂ ಹಂಸಃ ಶಿವಃ ಹ್‌ಸ್‌ಖ್‌ಫ್ರೇಂ ಹಸಕ್ಷಮಲವರಯೂಂ ಹ್‌ಸೌಃ ಸಹಕ್ಷಮಲವರಯೀಂ ಸ್‌ಹೌಃ ಸೋಽಹಂ ಹಂಸಃ ಶಿವಃ॥} ಸ್ವಚ್ಛಪ್ರಕಾಶ ವಿಮರ್ಶಹೇತವೇ ಪರಮಗುರವೇ ನಮಃ।ಶ್ರೀಪಾದುಕಾಂ ಪೂಜಯಾಮಿ ನಮಃ॥\\
\dhyana{೭ ಹಂಸಃಶಿವಃ ಸೋಽಹಂಹಂಸಃ ಹ್‌ಸ್‌ಖ್‌ಫ್ರೇಂ ಹಸಕ್ಷಮಲವರಯೂಂ ಹ್‌ಸೌಃ ಸಹಕ್ಷಮಲವರಯೀಂ ಸ್‌ಹೌಃ ಹಂಸಃ ಶಿವಃ ಸೋಽಹಂ ಹಂಸಃ॥} ಸ್ವಾತ್ಮಾರಾಮ ಪರಮಾನಂದ ಪಂಜರ ವಿಲೀನ ತೇಜಸೇ ಪರಮೇಷ್ಠಿಗುರವೇ ನಮಃ~। ಶ್ರೀಪಾದುಕಾಂ ಪೂಜಯಾಮಿ ನಮಃ॥

ಪೃಥ್ವೀತಿ ಮಂತ್ರಸ್ಯ ಮೇರುಪೃಷ್ಠ ಋಷಿಃ~। ಸುತಲಂ ಛಂದಃ~।\\ಆದಿಕೂರ್ಮೋ ದೇವತಾ ॥ ಆಸನೇ ವಿನಿಯೋಗಃ ॥\\
ಪೃಥ್ವಿ ತ್ವಯಾ ಧೃತಾ ಲೋಕಾ ದೇವಿ ತ್ವಂ ವಿಷ್ಣುನಾ ಧೃತಾ~।\\
ತ್ವಂ ಚ ಧಾರಯ ಮಾಂ ದೇವಿ ಪವಿತ್ರಂ ಕುರು ಚಾಸನಂ ॥

ಅಪಸರ್ಪಂತು ತೇ ಭೂತಾಃ ಯೇ ಭೂತಾ ಭೂಮಿ ಸಂಸ್ಥಿತಾಃ~।\\
ಯೇ ಭೂತಾಃ ವಿಘ್ನಕರ್ತಾರಸ್ತೇ ನಶ್ಯಂತು ಶಿವಾಜ್ಞಯಾ ॥

ಅಪಕ್ರಾಮಂತು ಭೂತಾನಿ ಪಿಶಾಚಾಃ ಸರ್ವತೋ ದಿಶಂ।\\
ಸರ್ವೇಷಾಮವಿರೋಧೇನ ಪೂಜಾ ಕರ್ಮಸಮಾರಭೇ ॥

ವಾಮಪಾದಾದಿ ಪೃಷ್ಠೇ ಚ ಧರಣೀತಾಡನತ್ರಯಮ್ ।\\
ವೀರಭದ್ರ ವಿರೂಪಾಕ್ಷ ವಿಶ್ವರೂಪತ್ರಯಂ ಸ್ಮರೇತ್ ॥

ಸ್ಯೋನಾ ಪೃಥಿವೀತ್ಯಸ್ಯ ಮೇಧಾತಿಥಿಃ ಕಾಣ್ವ ಋಷಿಃ । ಗಾಯತ್ರೀ ಛಂದಃ । ಪೃಥಿವೀ ದೇವತಾ । ಭೂಪ್ರಾರ್ಥನೇ ವಿನಿಯೋಗಃ ॥\\
ಸ್ಯೋ॒ನಾ ಪೃ॑ಥಿವಿ ಭವಾನೃಕ್ಷ॒ರಾ ನಿ॒ವೇಶ॑ನೀ~।\\ ಯಚ್ಛಾ᳚ನಃ॒ ಶರ್ಮ॑ ಸ॒ಪ್ರಥಃ॑ ॥

ಧನುರ್ಧರಾಯೈ ಚ ವಿದ್ಮಹೇ ಸರ್ವಸಿದ್ಧ್ಯೈ ಚ ಧೀಮಹಿ~।\\ ತನ್ನೋ ಧರಾ ಪ್ರಚೋದಯಾತ್ ॥

ಲಂ ಪೃಥಿವ್ಯೈ ನಮಃ~। ರಂ ರಕ್ತಾಸನಾಯ ನಮಃ~। ವಿಂ ವಿಮಲಾಸನಾಯ ನಮಃ~। ಯಂ ಯೋಗಾಸನಾಯ ನಮಃ~। ಕೂರ್ಮಾಸನಾಯ ನಮಃ~। ಅನಂತಾಸನಾಯ ನಮಃ~। ವೀರಾಸನಾಯ ನಮಃ~। ಖಡ್ಗಾಸನಾಯ ನಮಃ~। ಶರಾಸನಾಯ ನಮಃ~। ಪಂ ಪದ್ಮಾಸನಾಯ ನಮಃ~। ಪರಮಸುಖಾಸನಾಯ ನಮಃ॥

೪ ರಕ್ತದ್ವಾದಶಶಕ್ತಿಯುಕ್ತಾಯ ದ್ವೀಪನಾಥಾಯ ನಮಃ ॥

೪ ಶ್ರೀಲಲಿತಾಮಹಾತ್ರಿಪುರಸುಂದರಿ ಆತ್ಮಾನಂ ರಕ್ಷ ರಕ್ಷ ॥

ಓಂ ಗುಂ ಗುರುಭ್ಯೋ ನಮಃ~। ಪರಮಗುರುಭ್ಯೋ ನಮಃ~। ಪರಮೇಷ್ಠಿ\\ಗುರುಭ್ಯೋ ನಮಃ~। ಗಂ ಗಣಪತಯೇ ನಮಃ~। ದುಂ ದುರ್ಗಾಯೈ ನಮಃ~। ಸಂ ಸರಸ್ವತ್ಯೈ ನಮಃ~। ವಂ ವಟುಕಾಯ ನಮಃ~। ಕ್ಷಂ ಕ್ಷೇತ್ರಪಾಲಾಯ ನಮಃ~। ಯಾಂ ಯೋಗಿನೀಭ್ಯೋ ನಮಃ~। ಅಂ ಆತ್ಮನೇ ನಮಃ~। ಪಂ ಪರಮಾತ್ಮನೇ ನಮಃ~। ಸಂ ಸರ್ವಾತ್ಮನೇ ನಮಃ ॥

೪ ಓಂ ನಮೋ ಭಗವತಿ ತಿರಸ್ಕರಿಣಿ ಮಹಾಮಾಯೇ ಮಹಾನಿದ್ರೇ ಸಕಲ \\ಪಶುಜನ ಮನಶ್ಚಕ್ಷುಃಶ್ರೋತ್ರತಿರಸ್ಕರಣಂ ಕುರು ಕುರು ಸ್ವಾಹಾ ॥

೪ ಓಂ ಹಸಂತಿ ಹಸಿತಾಲಾಪೇ ಮಾತಂಗಿ ಪರಿಚಾರಿಕೇ~।\\
ಮಮ ವಿಘ್ನಾಪದಾಂ ನಾಶಂ ಕುರು ಕುರು ಠಃಠಃಠಃ ಹುಂ ಫಟ್ ಸ್ವಾಹಾ ॥

೪ ಓಂ ನಮೋ ಭಗವತಿ ಜ್ವಾಲಾಮಾಲಿನಿ ದೇವದೇವಿ ಸರ್ವಭೂತ ಸಂಹಾರ ಕಾರಿಕೇ ಜಾತವೇದಸಿ ಜ್ವಲಂತಿ ಜ್ವಲ ಜ್ವಲ ಪ್ರಜ್ವಲ ಪ್ರಜ್ವಲ ಹ್ರಾಂ ಹ್ರೀಂ ಹ್ರೂಂ ರರ ರರ ರರರ ಹುಂ ಫಟ್ ಸ್ವಾಹಾ~। ಸಹಸ್ರಾರ ಹುಂ ಫಟ್~।\\ ಭೂರ್ಭುವಃಸುವರೋಮಿತಿ ದಿಗ್ಬಂಧಃ ॥

೪ ಸಮಸ್ತ ಪ್ರಕಟ ಗುಪ್ತ ಗುಪ್ತತರ ಸಂಪ್ರದಾಯ ಕುಲೋತ್ತೀರ್ಣ ನಿಗರ್ಭ ರಹಸ್ಯಾ\-ತಿರಹಸ್ಯ ಪರಾಪರಾತಿರಹಸ್ಯ ಯೋಗಿನೀ ದೇವತಾಭ್ಯೋ ನಮಃ ॥

೪ ಐಂ ಹ್ರಃ ಅಸ್ತ್ರಾಯ ಫಟ್ ॥

೪ ಮೂಲಶೃಂಗಾಟಕಾತ್ ಸುಷುಮ್ನಾಪಥೇನ ಜೀವಶಿವಂ ಪರಮಶಿವಪದೇ \\ಯೋಜಯಾಮಿ ಸ್ವಾಹಾ~।\\
ಯಂ ೮ ಸಂಕೋಚಶರೀರಂ ಶೋಷಯ ಶೋಷಯ ಸ್ವಾಹಾ~।\\
ರಂ ೮ ಸಂಕೋಚಶರೀರಂ ದಹ ದಹ ಪಚ ಪಚ ಸ್ವಾಹಾ~।\\
ವಂ ೮ ಪರಮಶಿವಾಮೃತಂ ವರ್ಷಯ ವರ್ಷಯ ಸ್ವಾಹಾ~।\\
ಲಂ ೮ ಶಾಂಭವಶರೀರಮುತ್ಪಾದಯೋತ್ಪಾದಯ ಸ್ವಾಹಾ~।\\
ಹಂಸಃ ಸೋಹಂ ಅವತರ ಅವತರ ಶಿವಪದಾತ್ ಜೀವಶಿವ\\ ಸುಷುಮ್ನಾಪಥೇನ ಪ್ರವಿಶ ಮೂಲಶೃಂಗಾಟಕಂ ಉಲ್ಲಸೋಲ್ಲಸ\\ ಜ್ವಲ ಜ್ವಲ ಪ್ರಜ್ವಲ ಪ್ರಜ್ವಲ ಹಂಸಃ ಸೋಹಂ ಸ್ವಾಹಾ ॥\\
೪ ಆಂ ಸೋಹಂ (ಇತಿ ತ್ರಿಃ ಹೃದಿ) ಇತಿ ಭೂತಶುದ್ಧಿಃ ॥

ತತಃ ಪ್ರಾಣಾನಾಯಮ್ಯ, ದೇಶಕಾಲೌ ಸಂಕೀರ್ತ್ಯ,\\
ಶುಭಪುಣ್ಯತಿಥೌ ಶ್ರೀಮಹಾಭಟ್ಟಾರಕ ಕಾಮೇಶ್ವರಸಹಿತ ಶ್ರೀಶಕ್ತಿಚಕ್ರೈಕನಾಯಿಕಾ ಶ್ರೀಮಹಾಭಟ್ಟಾರಿಕಾ ಶ್ರೀ ಲಲಿತಾಮಹಾತ್ರಿಪುರಸುಂದರೀ ಪ್ರೀತ್ಯರ್ಥೇ ಶ್ರೀಗುರೋರಾಜ್ಞಯಾ ಶ್ರೀಚಕ್ರ ಅಂತರಾರಾಧನಂ ಕರಿಷ್ಯೇ ॥


\section{ಲಘುಪ್ರಾಣಪ್ರತಿಷ್ಠಾ}
ಹೃದಿ ಹಸ್ತಂ ದತ್ವಾ\\ಆಂ ಸೋಹಂ~। ಆಂ ಹ್ರೀಂ ಕ್ರೋಂ ಯರಲವಶಷಸಹೋಂ ॥ ಮಮ ಪ್ರಾಣಾ ಇಹ ಪ್ರಾಣಾಃ~। ಜೀವ ಇಹ ಸ್ಥಿತಃ~। ಸರ್ವೇಂದ್ರಿಯಾಣಿ~। ಮಮ ವಾಙ್ಮನಸ್ತ್ವಕ್ಚಕ್ಷುಃ ಶ್ರೋತ್ರ ಜಿಹ್ವಾಘ್ರಾಣಪ್ರಾಣಾ ಇಹೈವಾಗತ್ಯ ಸುಖಂ ಚಿರಂ ತಿಷ್ಠಂತು ಸ್ವಾಹಾ ॥

ಮೂಲವಿದ್ಯಾನ್ಯಾಸಂ ಷೋಡಶಾಕ್ಷರೀ ನ್ಯಾಸಂ ಚ ವಿಧಾಯ,

ಶರೀರಂ ಚಿಂತಯೇದಾದೌ ನಿಜಂ ಶ್ರೀಚಕ್ರರೂಪಕಮ್ ।\\
ತ್ವಗಾದ್ಯಾಕಾರನಿರ್ಮುಕ್ತಂ ಜ್ವಲತ್ಕಾಲಾಗ್ನಿಸನ್ನಿಭಮ್ ॥\\
ತತಃ ಸ್ವಾತ್ಮಾನಂ ದೇವೀರೂಪಂ ವಿಭಾವಯೇತ್ ॥

\as{ಅ॒ಣೋರಣೀ॑ಯಾನ್ಮಹ॒ತೋ ಮಹೀ॑ಯಾನಾ॒ತ್ಮಾ ಗುಹಾ॑ಯಾಂ॒ ನಿಹಿ॑ತೋಽಸ್ಯ ಜಂ॒ತೋಃ ।
ತಮ॑ಕ್ರತುಂ ಪಶ್ಯತಿ ವೀತಶೋ॒ಕೋ ಧಾ॒ತುಃ ಪ್ರ॒ಸಾದಾ᳚ನ್ಮಹಿ॒ಮಾನ॑ಮೀಶಂ ॥}

ವಿವೇಕವೃತ್ಯವಚ್ಛಿನ್ನ ಚಿಚ್ಛಕ್ತಿರೂಪಸುಷುಮ್ನಾತ್ಮನೇ ಶ್ರೀಗುರವೇ ನಮಃ ।\\(ಇತಿ ಬ್ರಹ್ಮರಂಧ್ರಂ ಸ್ಪೃಷ್ಟ್ವಾ)

\as{ದಿವ್ಯೌಘಾಃ}\\
ದಕ್ಷಶ್ರೋತ್ರರೂಪ ಪಯಸ್ವಿನ್ಯಾತ್ಮನೇ ಪ್ರಕಾಶಾನಂದನಾಥಾಯ ನಮಃ ।\\
ವಾಮಶ್ರೋತ್ರರೂಪ ಶಂಖಿನ್ಯಾತ್ಮನೇ ವಿಮರ್ಶಾನಂದನಾಥಾಯ ನಮಃ ।\\
ಜಿಹ್ವಾರೂಪ ಸರಸ್ವತ್ಯಾತ್ಮನೇ ಆನಂದಾನಂದನಾಥಾಯ ನಮಃ ।

\as{ಸಿದ್ಧೌಘಾಃ}\\
ದಕ್ಷನೇತ್ರರೂಪ ಪೂಷಾತ್ಮನೇ ಜ್ಞಾನಾನಂದನಾಥಾಯ ನಮಃ ।\\
ವಾಮನೇತ್ರರೂಪ ಗಾಂಧಾರ್ಯಾತ್ಮನೇ ಸತ್ಯಾನಂದನಾಥಾಯ ನಮಃ ।\\
ಧ್ವಜರೂಪ ಕುಹ್ವಾತ್ಮನೇ ಪೂರ್ಣಾನಂದನಾಥಾಯ ನಮಃ ।

\as{ಮಾನವೌಘಾಃ}\\
ದಕ್ಷನಾಸಾರೂಪ ಪಿಂಗಲಾತ್ಮನೇ ಸ್ವಭಾವಾನಂದನಾಥಾಯ ನಮಃ ।\\
ವಾಮನಾಸಾರೂಪ ಇಡಾತ್ಮನೇ ಪ್ರತಿಭಾನಂದನಾಥಾಯ ನಮಃ ।\\
ಪಾಯುರೂಪ ಅಲಂಬುಸಾತ್ಮನೇ ಸುಭಗಾನಂದನಾಥಾಯ ನಮಃ ।

\as{ಸ॒ಪ್ತ ಪ್ರಾ॒ಣಾಃ ಪ್ರ॒ಭವಂ॑ತಿ॒ ತಸ್ಮಾ᳚ಥ್ಸ॒ಪ್ತಾರ್ಚಿಷ॑ಸ್ಸ॒ಮಿಧ॑ಸ್ಸ॒ಪ್ತಜಿ॒ಹ್ವಾಃ ।
ಸ॒ಪ್ತ ಇ॒ಮೇ ಲೋ॒ಕಾ ಯೇಷು॒ ಚರಂ॑ತಿ ಪ್ರಾ॒ಣಾ ಗು॒ಹಾಶ॑ಯಾ॒ನ್ನಿಹಿ॑ತಾಃ ಸ॒ಪ್ತ ಸ॑ಪ್ತ ॥}\\
ನವಚಕ್ರರೂಪ ಶ್ರೀಚಕ್ರಾತ್ಮನೇ ದೇಹಾಯ ನಮಃ ।\\
ಪಿತೃರೂಪ ಅಸ್ಥ್ಯಾದ್ಯವಯವಾತ್ಮನೇ ವಾರಾಹ್ಯೈ ನಮಃ ।\\
ಮಾತೃರೂಪ ಮಾಂಸಾದ್ಯವಯವಾತ್ಮನೇ ಬಲಿದೇವತಾಯೈ ಕುರುಕುಲ್ಲಾಯೈ ನಮಃ ।(ಇತಿ ತ್ರಿಃ ವ್ಯಾಪಕಂ ಕೃತ್ವಾ)\\
ದೇಹಪಶ್ಚಾದ್ಭಾಗರೂಪ ಧರ್ಮಾತ್ಮನೇ ಇಕ್ಷುಸಾಗರಾಯ ನಮಃ ।\\
ದೇಹದಕ್ಷಿಣಭಾಗರೂಪ ಅರ್ಥಾತ್ಮನೇ ಸುರಾಸಾಗರಾಯ ನಮಃ ।\\
ದೇಹಪ್ರಾಗ್ಭಾಗರೂಪ ಕಾಮಾತ್ಮನೇ ಘೃತಸಾಗರಾಯ ನಮಃ ।\\
ದೇಹಉದಗ್ಭಾಗರೂಪ ಮೋಕ್ಷಾತ್ಮನೇ ಕ್ಷೀರಸಾಗರಾಯ ನಮಃ ॥\\
ದೇಹಾತ್ಮನೇ ನವರತ್ನದ್ವೀಪಾಯ ನಮಃ ।\\
ಮಾಂಸಾತ್ಮನೇ ಪುಷ್ಯರಾಗರತ್ನಾಯ ನಮಃ ।\\
ರೋಮಾತ್ಮನೇ ನೀಲರತ್ನಾಯ ನಮಃ ।\\
ತ್ವಗಾತ್ಮನೇ ವೈಡೂರ್ಯರತ್ನಾಯ ನಮಃ ।\\
ರುಧಿರಾತ್ಮನೇ ವಿದ್ರುಮರತ್ನಾಯ ನಮಃ ।\\
ಮಜ್ಜಾತ್ಮನೇ ಮರಕತರತ್ನಾಯ ನಮಃ ।\\
ಅಸ್ಥ್ಯಾತ್ಮನೇ ವಜ್ರರತ್ನಾಯ ನಮಃ ।\\
ಶುಕ್ಲಾತ್ಮನೇ ಮೌಕ್ತಿಕರತ್ನಾಯ ನಮಃ ।\\
ಮೇದ ಆತ್ಮನೇ ಗೋಮೇಧಕರತ್ನಾಯ ನಮಃ ।\\
ಓಜ ಆತ್ಮನೇ ಪದ್ಮರಾಗರತ್ನಾಯ ನಮಃ ।

ಮಾಂಸಾಧಿದೇವತಾಯೈ ಕಾಲಚಕ್ರೇಶ್ವರ್ಯೈ ನಮಃ ।\\
ರೋಮಾಧಿದೇವತಾಯೈ ಮುದ್ರಾಚಕ್ರೇಶ್ವರ್ಯೈ ನಮಃ ।\\
ತ್ವಗಧಿದೇವತಾಯೈ ಮಾತೃಕಾಚಕ್ರೇಶ್ವರ್ಯೈ ನಮಃ ।\\
ರುಧಿರಾಧಿದೇವತಾಯೈ ರತ್ನಚಕ್ರೇಶ್ವರ್ಯೈ ನಮಃ ।\\
ಮಜ್ಜಾಧಿದೇವತಾಯೈ ಗುರುಚಕ್ರೇಶ್ವರ್ಯೈ ನಮಃ ।\\
ಅಸ್ಥ್ಯಧಿದೇವತಾಯೈ ತತ್ವಚಕ್ರೇಶ್ವರ್ಯೈ ನಮಃ ।\\
ಶುಕ್ಲಾಧಿದೇವತಾಯೈ ದೇಶಚಕ್ರೇಶ್ವರ್ಯೈ ನಮಃ ।\\
ಮೇದೋಽಧಿದೇವತಾಯೈ ಗ್ರಹಚಕ್ರೇಶ್ವರ್ಯೈ ನಮಃ ।\\
ಓಜೋಽಧಿದೇವತಾಯೈ ಮೂರ್ತಿಚಕ್ರೇಶ್ವರ್ಯೈ ನಮಃ ।\\
ಸಂಕಲ್ಪಾತ್ಮಭ್ಯಃ ಕಲ್ಪತರುಭ್ಯೋ ನಮಃ ।\\
ತೇಜಆತ್ಮನೇ ಕಲ್ಪಕೋದ್ಯಾನಾಯ ನಮಃ ।

ಮಧುರರಸಾತ್ಮನೇ ವಸಂತರ್ತವೇ ನಮಃ ।\\
ಆಮ್ಲರಸಾತ್ಮನೇ ಗ್ರೀಷ್ಮರ್ತವೇ ನಮಃ ।\\
ತಿಕ್ತರಸಾತ್ಮನೇ ವರ್ಷರ್ತವೇ ನಮಃ ।\\
ಕಟುರಸಾತ್ಮನೇ ಶರದೃತವೇ ನಮಃ ।\\
ಕಷಾಯರಸಾತ್ಮನೇ ಹೇಮಂತರ್ತವೇ ನಮಃ ।\\
ಲವಣರಸಾತ್ಮನೇ ಶಿಶಿರರ್ತವೇ ನಮಃ ।

ಇಂದ್ರಿಯಾತ್ಮಭ್ಯೋಽಶ್ವೇಭ್ಯೋ ನಮಃ ।\\
ಇಂದ್ರಿಯಾರ್ಥಾತ್ಮಭ್ಯೋ ಗಜೇಭ್ಯೋ ನಮಃ ।\\
ಕರುಣಾತ್ಮನೇ ತೋಯಪರಿಘಾಯ ನಮಃ ।\\
ಓಜಃಪುಂಜಾತ್ಮನೇ ಮಾಣಿಕ್ಯಮಂಡಪಾಯ ನಮಃ ।\\
ಜ್ಞಾನಾತ್ಮನೇ ವಿಶೇಷಾರ್ಘ್ಯಾಯ ನಮಃ ।\\
ಜ್ಞೇಯಾತ್ಮನೇ ಹವಿಷೇ ನಮಃ ।\\
ಜ್ಞಾತ್ರಾತ್ಮನೇ ಸ್ವಾತ್ಮನೇ ನಮಃ ।\\
ಚಿದಾತ್ಮನೇ ಶ್ರೀಮಹಾತ್ರಿಪುರಸುಂದರ್ಯೈ ನಮಃ ।

(ಇತಿ ತತ್ತದನುಸಂಧಾನಪೂರ್ವಕಂ ಮನಸಾ ನತ್ವಾ, ಜ್ಞಾತೃಜ್ಞಾನಜ್ಞೇಯಾನಾಂ ನಾಮರೂಪವಿಲಾಪನಾನುಸಂಧಾನೇನ ಚಿನ್ಮಾತ್ರರೂಪತಾ ವಿಭಾವನೇನ ಕ್ಷಣಂ ವಿಶ್ರಮ್ಯ)\\
 \as{(ಲಲಿತಾಸಹಸ್ರನಾಮ ೧ - ೧೦೦)\\
  ಓಂ ಐಂಹ್ರೀಂಶ್ರೀಂ}\\
ಶ್ರೀಮಾತಾ ಶ್ರೀಮಹಾರಾಜ್ಞೀ ಶ್ರೀಮತ್ಸಿಂಹಾಸನೇಶ್ವರೀ~।\\
ಚಿದಗ್ನಿ - ಕುಂಡ - ಸಂಭೂತಾ ದೇವಕಾರ್ಯ - ಸಮುದ್ಯತಾ ॥೧॥

ಉದ್ಯದ್ಭಾನು - ಸಹಸ್ರಾಭಾ ಚತುರ್ಬಾಹು - ಸಮನ್ವಿತಾ~।\\
ರಾಗಸ್ವರೂಪ - ಪಾಶಾಢ್ಯಾ ಕ್ರೋಧಾಕಾರಾಂಕುಶೋಜ್ಜ್ವಲಾ ॥೨॥

ಮನೋರೂಪೇಕ್ಷು - ಕೋದಂಡಾ ಪಂಚತನ್ಮಾತ್ರ - ಸಾಯಕಾ~।\\
ನಿಜಾರುಣ - ಪ್ರಭಾಪೂರ - ಮಜ್ಜದ್‍ಬ್ರಹ್ಮಾಂಡ - ಮಂಡಲಾ ॥೩॥

ಚಂಪಕಾಶೋಕ - ಪುನ್ನಾಗ - ಸೌಗಂಧಿಕ - ಲಸತ್ಕಚಾ~।\\
ಕುರುವಿಂದಮಣಿ - ಶ್ರೇಣೀ - ಕನತ್ಕೋಟೀರ - ಮಂಡಿತಾ ॥೪॥

ಅಷ್ಟಮೀಚಂದ್ರ - ವಿಭ್ರಾಜ - ದಲಿಕಸ್ಥಲ - ಶೋಭಿತಾ~।\\
ಮುಖಚಂದ್ರ - ಕಲಂಕಾಭ - ಮೃಗನಾಭಿ - ವಿಶೇಷಕಾ ॥೫॥

ವದನಸ್ಮರ - ಮಾಂಗಲ್ಯ - ಗೃಹತೋರಣ - ಚಿಲ್ಲಿಕಾ~।\\
ವಕ್ತ್ರಲಕ್ಷ್ಮೀ - ಪರೀವಾಹ - ಚಲನ್ಮೀನಾಭ - ಲೋಚನಾ ॥೬॥

ನವಚಂಪಕ - ಪುಷ್ಪಾಭ - ನಾಸಾದಂಡ - ವಿರಾಜಿತಾ~।\\
ತಾರಾಕಾಂತಿ - ತಿರಸ್ಕಾರಿ - ನಾಸಾಭರಣ - ಭಾಸುರಾ ॥೭॥

ಕದಂಬಮಂಜರೀ - ಕ್ಲೃಪ್ತ - ಕರ್ಣಪೂರ - ಮನೋಹರಾ~।\\
ತಾಟಂಕ - ಯುಗಲೀ - ಭೂತ - ತಪನೋಡುಪ - ಮಂಡಲಾ ॥೮॥

ಪದ್ಮರಾಗಶಿಲಾದರ್ಶ - ಪರಿಭಾವಿ - ಕಪೋಲಭೂಃ~।\\
ನವವಿದ್ರುಮ - ಬಿಂಬಶ್ರೀ - ನ್ಯಕ್ಕಾರಿ - ರದನಚ್ಛದಾ ॥೯॥

ಶುದ್ಧವಿದ್ಯಾಂಕುರಾಕಾರ - ದ್ವಿಜಪಂಕ್ತಿ - ದ್ವಯೋಜ್ಜ್ವಲಾ~।\\
ಕರ್ಪೂರವೀಟಿಕಾಮೋದ - ಸಮಾಕರ್ಷದ್ದಿಗಂತರಾ ॥೧೦॥

ನಿಜ - ಸಲ್ಲಾಪ - ಮಾಧುರ್ಯ - ವಿನಿರ್ಭರ್ತ್ಸಿತ - ಕಚ್ಛಪೀ~।\\
ಮಂದಸ್ಮಿತ - ಪ್ರಭಾಪೂರ - ಮಜ್ಜತ್ಕಾಮೇಶ - ಮಾನಸಾ ॥೧೧॥

ಅನಾಕಲಿತ - ಸಾದೃಶ್ಯ - ಚುಬುಕಶ್ರೀ - ವಿರಾಜಿತಾ~।\\
ಕಾಮೇಶ - ಬದ್ಧ - ಮಾಂಗಲ್ಯ - ಸೂತ್ರ - ಶೋಭಿತ - ಕಂಧರಾ ॥೧೨॥

ಕನಕಾಂಗದ - ಕೇಯೂರ - ಕಮನೀಯ - ಭುಜಾನ್ವಿತಾ~।\\
ರತ್ನಗ್ರೈವೇಯ - ಚಿಂತಾಕ - ಲೋಲ - ಮುಕ್ತಾ - ಫಲಾನ್ವಿತಾ ॥೧೩॥

ಕಾಮೇಶ್ವರ - ಪ್ರೇಮರತ್ನ - ಮಣಿ - ಪ್ರತಿಪಣ - ಸ್ತನೀ~।\\
ನಾಭ್ಯಾಲವಾಲ - ರೋಮಾಲಿ - ಲತಾ - ಫಲ - ಕುಚದ್ವಯೀ ॥೧೪॥

ಲಕ್ಷ್ಯರೋಮ - ಲತಾಧಾರತಾ - ಸಮುನ್ನೇಯ - ಮಧ್ಯಮಾ~।\\
ಸ್ತನಭಾರ - ದಲನ್ಮಧ್ಯ - ಪಟ್ಟಬಂಧ - ವಲಿತ್ರಯಾ ॥೧೫॥

ಅರುಣಾರುಣಕೌಸುಂಭ - ವಸ್ತ್ರ - ಭಾಸ್ವತ್ಕಟೀತಟೀ~।\\
ರತ್ನ - ಕಿಂಕಿಣಿಕಾ - ರಮ್ಯ - ರಶನಾ - ದಾಮ - ಭೂಷಿತಾ ॥೧೬॥

ಕಾಮೇಶ - ಜ್ಞಾತ - ಸೌಭಾಗ್ಯ - ಮಾರ್ದವೋರು - ದ್ವಯಾನ್ವಿತಾ~।\\
ಮಾಣಿಕ್ಯ - ಮುಕುಟಾಕಾರ - ಜಾನುದ್ವಯ - ವಿರಾಜಿತಾ ॥೧೭॥

ಇಂದ್ರಗೋಪ - ಪರಿಕ್ಷಿಪ್ತಸ್ಮರತೂಣಾಭ - ಜಂಘಿಕಾ~।\\
ಗೂಢಗುಲ್ಫಾ ಕೂರ್ಮಪೃಷ್ಠ - ಜಯಿಷ್ಣು - ಪ್ರಪದಾನ್ವಿತಾ ॥೧೮॥

ನಖ - ದೀಧಿತಿ - ಸಂಛನ್ನ - ನಮಜ್ಜನ - ತಮೋಗುಣಾ~।\\
ಪದದ್ವಯ - ಪ್ರಭಾಜಾಲ - ಪರಾಕೃತ - ಸರೋರುಹಾ ॥೧೯॥

ಶಿಂಜಾನ - ಮಣಿಮಂಜೀರ - ಮಂಡಿತ - ಶ್ರೀ - ಪದಾಂಬುಜಾ~।\\
ಮರಾಲೀ - ಮಂದಗಮನಾ ಮಹಾಲಾವಣ್ಯ - ಶೇವಧಿಃ ॥೨೦॥

ಸರ್ವಾರುಣಾಽನವದ್ಯಾಂಗೀ ಸರ್ವಾಭರಣ - ಭೂಷಿತಾ~।\\
ಶಿವ - ಕಾಮೇಶ್ವರಾಂಕಸ್ಥಾ ಶಿವಾ ಸ್ವಾಧೀನ - ವಲ್ಲಭಾ ॥೨೧॥

ಸುಮೇರು - ಮಧ್ಯ - ಶೃಂಗಸ್ಥಾ ಶ್ರೀಮನ್ನಗರ - ನಾಯಿಕಾ~।\\
ಚಿಂತಾಮಣಿ - ಗೃಹಾಂತಸ್ಥಾ ಪಂಚ - ಬ್ರಹ್ಮಾಸನ - ಸ್ಥಿತಾ ॥೨೨॥

ಮಹಾಪದ್ಮಾಟವೀ - ಸಂಸ್ಥಾ ಕದಂಬವನ - ವಾಸಿನೀ~।\\
ಸುಧಾಸಾಗರ - ಮಧ್ಯಸ್ಥಾ ಕಾಮಾಕ್ಷೀ ಕಾಮದಾಯಿನೀ ॥೨೩॥

ದೇವರ್ಷಿ - ಗಣ - ಸಂಘಾತ - ಸ್ತೂಯಮಾನಾತ್ಮ - ವೈಭವಾ~।\\
ಭಂಡಾಸುರ - ವಧೋದ್ಯುಕ್ತ - ಶಕ್ತಿಸೇನಾ - ಸಮನ್ವಿತಾ ॥೨೪॥

ಸಂಪತ್ಕರೀ - ಸಮಾರೂಢ - ಸಿಂಧುರ - ವ್ರಜ - ಸೇವಿತಾ~।\\
ಅಶ್ವಾರೂಢಾಧಿಷ್ಠಿತಾಶ್ವ - ಕೋಟಿ - ಕೋಟಿಭಿರಾವೃತಾ ॥೨೫॥

ಚಕ್ರರಾಜ - ರಥಾರೂಢ - ಸರ್ವಾಯುಧ - ಪರಿಷ್ಕೃತಾ~।\\
ಗೇಯಚಕ್ರ - ರಥಾರೂಢ - ಮಂತ್ರಿಣೀ - ಪರಿಸೇವಿತಾ ॥೨೬॥

ಕಿರಿಚಕ್ರ - ರಥಾರೂಢ - ದಂಡನಾಥಾ - ಪುರಸ್ಕೃತಾ~।\\
ಜ್ವಾಲಾ - ಮಾಲಿನಿಕಾಕ್ಷಿಪ್ತ - ವಹ್ನಿಪ್ರಾಕಾರ - ಮಧ್ಯಗಾ ॥೨೭॥

ಭಂಡಸೈನ್ಯ - ವಧೋದ್ಯುಕ್ತ - ಶಕ್ತಿ - ವಿಕ್ರಮ - ಹರ್ಷಿತಾ~।\\
ನಿತ್ಯಾ - ಪರಾಕ್ರಮಾಟೋಪ - ನಿರೀಕ್ಷಣ - ಸಮುತ್ಸುಕಾ ॥೨೮॥

ಭಂಡಪುತ್ರ - ವಧೋದ್ಯುಕ್ತ - ಬಾಲಾ - ವಿಕ್ರಮ - ನಂದಿತಾ~।\\
ಮಂತ್ರಿಣ್ಯಂಬಾ - ವಿರಚಿತ - ವಿಷಂಗ - ವಧ - ತೋಷಿತಾ ॥೨೯॥

ವಿಶುಕ್ರ - ಪ್ರಾಣಹರಣ - ವಾರಾಹೀ - ವೀರ್ಯ - ನಂದಿತಾ~।\\
ಕಾಮೇಶ್ವರ - ಮುಖಾಲೋಕ - ಕಲ್ಪಿತ - ಶ್ರೀಗಣೇಶ್ವರಾ ॥೩೦॥

ಮಹಾಗಣೇಶ - ನಿರ್ಭಿನ್ನ - ವಿಘ್ನಯಂತ್ರ - ಪ್ರಹರ್ಷಿತಾ~।\\
ಭಂಡಾಸುರೇಂದ್ರ - ನಿರ್ಮುಕ್ತ - ಶಸ್ತ್ರ - ಪ್ರತ್ಯಸ್ತ್ರ - ವರ್ಷಿಣೀ ॥೩೧॥

ಕರಾಂಗುಲಿ - ನಖೋತ್ಪನ್ನ - ನಾರಾಯಣ - ದಶಾಕೃತಿಃ~।\\
ಮಹಾ - ಪಾಶುಪತಾಸ್ತ್ರಾಗ್ನಿ - ನಿರ್ದಗ್ಧಾಸುರ - ಸೈನಿಕಾ ॥೩೨॥

ಕಾಮೇಶ್ವರಾಸ್ತ್ರ - ನಿರ್ದಗ್ಧ - ಸಭಂಡಾಸುರ - ಶೂನ್ಯಕಾ~।\\
ಬ್ರಹ್ಮೋಪೇಂದ್ರ - ಮಹೇಂದ್ರಾದಿ - ದೇವ - ಸಂಸ್ತುತ - ವೈಭವಾ ॥೩೩॥

ಹರ - ನೇತ್ರಾಗ್ನಿ - ಸಂದಗ್ಧ - ಕಾಮ - ಸಂಜೀವನೌಷಧಿಃ~।\\
ಶ್ರೀಮದ್ವಾಗ್ಭವ - ಕೂಟೈಕ - ಸ್ವರೂಪ - ಮುಖ - ಪಂಕಜಾ ॥೩೪॥

ಕಂಠಾಧಃ - ಕಟಿ - ಪರ್ಯಂತ - ಮಧ್ಯಕೂಟ - ಸ್ವರೂಪಿಣೀ~।\\
ಶಕ್ತಿ - ಕೂಟೈಕತಾಪನ್ನ - ಕಟ್ಯಧೋಭಾಗ ಧಾರಿಣೀ ॥೩೫॥

ಮೂಲ - ಮಂತ್ರಾತ್ಮಿಕಾ ಮೂಲಕೂಟತ್ರಯ - ಕಲೇಬರಾ~।\\
ಕುಲಾಮೃತೈಕ - ರಸಿಕಾ ಕುಲಸಂಕೇತ - ಪಾಲಿನೀ ॥೩೬॥

ಕುಲಾಂಗನಾ ಕುಲಾಂತಸ್ಥಾ ಕೌಲಿನೀ ಕುಲಯೋಗಿನೀ~।\\
ಅಕುಲಾ ಸಮಯಾಂತಸ್ಥಾ ಸಮಯಾಚಾರ - ತತ್ಪರಾ ॥೩೭॥

ಮೂಲಾಧಾರೈಕ - ನಿಲಯಾ ಬ್ರಹ್ಮಗ್ರಂಥಿ - ವಿಭೇದಿನೀ \as{(೧೦೦)}~।

\as{ (ಲಲಿತಾಷ್ಟೋತ್ತರ ೧ - ೧೦)\\
ಓಂ ಐಂಹ್ರೀಂಶ್ರೀಂ}\\
ರಜತಾಚಲಶೃಂಗಾಗ್ರಮಧ್ಯಸ್ಥಾಯೈ ನಮೋ ನಮಃ ।\\
ಹಿಮಾಚಲಮಹಾವಂಶಪಾವನಾಯೈ ನಮೋ ನಮಃ ।\\
ಶಂಕರಾರ್ಧಾಂಗಸೌಂದರ್ಯಶರೀರಾಯೈ ನಮೋ ನಮಃ ।\\
ಲಸನ್ಮರಕತಸ್ವಚ್ಛವಿಗ್ರಹಾಯೈ ನಮೋ ನಮಃ ।\\
ಮಹಾತಿಶಯಸೌಂದರ್ಯಲಾವಣ್ಯಾಯೈ ನಮೋ ನಮಃ ।\\
ಶಶಾಂಕಶೇಖರಪ್ರಾಣವಲ್ಲಭಾಯೈ ನಮೋ ನಮಃ ।\\
ಸದಾಪಂಚದಶಾತ್ಮೈಕ್ಯಸ್ವರೂಪಾಯೈ ನಮೋ ನಮಃ ।\\
ವಜ್ರಮಾಣಿಕ್ಯಕಟಕಕಿರೀಟಾಯೈ ನಮೋ ನಮಃ ।\\
ಕಸ್ತೂರೀತಿಲಕೋಲ್ಲಾಸನಿಟಿಲಾಯೈ ನಮೋ ನಮಃ ।\\
ಭಸ್ಮರೇಖಾಂಕಿತಲಸನ್ಮಸ್ತಕಾಯೈ ನಮೋ ನಮಃ ।

\as{ಅತ॑ಸ್ಸಮು॒ದ್ರಾ ಗಿ॒ರಯ॑ಶ್ಚ॒ ಸರ್ವೇ॒ಽಸ್ಮಾಥ್ಸ್ಯಂದಂ॑ತೇ॒ ಸಿಂಧ॑ವ॒ಸ್ಸರ್ವ॑ರೂಪಾಃ ।
ಅತ॑ಶ್ಚ॒ ವಿಶ್ವಾ॒ ಓಷ॑ಧಯೋ॒ ರಸಾ᳚ಚ್ಚ॒ ಯೇನೈ॑ಷ ಭೂ॒ತಸ್ತಿ॑ಷ್ಠತ್ಯಂತರಾ॒ತ್ಮಾ ॥}
\section{(ಹೃದಿ ಹಸ್ತಂ ದತ್ವಾ)\\
ಪಂಚದಶನಿತ್ಯಾಭ್ಯೋ ನಮಃ ।}
ಚತ್ವಾರಿಂಶದಧಿಕ ಚತುರ್ದಶಶತಶ್ವಾಸಾತ್ಮನೇ ಪ್ರತಿಪತಿಥಿರೂಪ ಕಾಮೇಶ್ವರೀನಿತ್ಯಾಯೈ ನಮಃ ।\\
ತದುತ್ತರ ಚತ್ವಾರಿಂಶದಧಿಕ ಚತುರ್ದಶಶತಶ್ವಾಸಾತ್ಮನೇ ದ್ವಿತೀಯಾತಿಥಿರೂಪ ಭಗಮಾಲಿನೀನಿತ್ಯಾಯೈ ನಮಃ ।\\
ತದುತ್ತರ ಚತ್ವಾರಿಂಶದಧಿಕ ಚತುರ್ದಶಶತಶ್ವಾಸಾತ್ಮನೇ ತೃತೀಯಾತಿಥಿರೂಪ ನಿತ್ಯಕ್ಲಿನ್ನಾನಿತ್ಯಾಯೈ ನಮಃ ।\\
ತದುತ್ತರ ಚತ್ವಾರಿಂಶದಧಿಕ ಚತುರ್ದಶಶತಶ್ವಾಸಾತ್ಮನೇ ಚತುರ್ಥೀತಿಥಿರೂಪ ಭೇರುಂಡಾನಿತ್ಯಾಯೈ ನಮಃ ।\\
ತದುತ್ತರ ಚತ್ವಾರಿಂಶದಧಿಕ ಚತುರ್ದಶಶತಶ್ವಾಸಾತ್ಮನೇ ಪಂಚಮೀತಿಥಿರೂಪ ವಹ್ನಿವಾಸಿನೀನಿತ್ಯಾಯೈ ನಮಃ ।\\
ತದುತ್ತರ ಚತ್ವಾರಿಂಶದಧಿಕ ಚತುರ್ದಶಶತಶ್ವಾಸಾತ್ಮನೇ ಷಷ್ಠೀತಿಥಿರೂಪ ಮಹಾವಜ್ರೇಶ್ವರೀನಿತ್ಯಾಯೈ ನಮಃ ।\\
ತದುತ್ತರ ಚತ್ವಾರಿಂಶದಧಿಕ ಚತುರ್ದಶಶತಶ್ವಾಸಾತ್ಮನೇ ಸಪ್ತಮೀತಿಥಿರೂಪ ಶಿವಾದೂತೀನಿತ್ಯಾಯೈ ನಮಃ ।\\
ತದುತ್ತರ ಚತ್ವಾರಿಂಶದಧಿಕ ಚತುರ್ದಶಶತಶ್ವಾಸಾತ್ಮನೇ ಅಷ್ಟಮೀತಿಥಿರೂಪ ತ್ವರಿತಾನಿತ್ಯಾಯೈ ನಮಃ ।\\
ತದುತ್ತರ ಚತ್ವಾರಿಂಶದಧಿಕ ಚತುರ್ದಶಶತಶ್ವಾಸಾತ್ಮನೇ ನವಮೀತಿಥಿರೂಪ ಕುಲಸುಂದರೀನಿತ್ಯಾಯೈ ನಮಃ ।\\
ತದುತ್ತರ ಚತ್ವಾರಿಂಶದಧಿಕ ಚತುರ್ದಶಶತಶ್ವಾಸಾತ್ಮನೇ ದಶಮೀತಿಥಿರೂಪ ನಿತ್ಯಾನಿತ್ಯಾಯೈ ನಮಃ ।\\
ತದುತ್ತರ ಚತ್ವಾರಿಂಶದಧಿಕ ಚತುರ್ದಶಶತಶ್ವಾಸಾತ್ಮನೇ ಏಕಾದಶೀತಿಥಿರೂಪ ನೀಲಪತಾಕಾನಿತ್ಯಾಯೈ ನಮಃ ।\\
ತದುತ್ತರ ಚತ್ವಾರಿಂಶದಧಿಕ ಚತುರ್ದಶಶತಶ್ವಾಸಾತ್ಮನೇ ದ್ವಾದಶೀತಿಥಿರೂಪ ವಿಜಯಾನಿತ್ಯಾಯೈ ನಮಃ ।\\
ತದುತ್ತರ ಚತ್ವಾರಿಂಶದಧಿಕ ಚತುರ್ದಶಶತಶ್ವಾಸಾತ್ಮನೇ ತ್ರಯೋದಶೀತಿಥಿರೂಪ ಸರ್ವಮಂಗಳಾನಿತ್ಯಾಯೈ ನಮಃ ।\\
ತದುತ್ತರ ಚತ್ವಾರಿಂಶದಧಿಕ ಚತುರ್ದಶಶತಶ್ವಾಸಾತ್ಮನೇ ಚತುರ್ದಶೀತಿಥಿರೂಪ ಜ್ವಾಲಾಮಾಲಿನೀನಿತ್ಯಾಯೈ ನಮಃ ।\\
ತದುತ್ತರ ಚತ್ವಾರಿಂಶದಧಿಕ ಚತುರ್ದಶಶತಶ್ವಾಸಾತ್ಮನೇ ಪೌರ್ಣಮಾಸೀತಿಥಿರೂಪ ಚಿತ್ರಾನಿತ್ಯಾಯೈ ನಮಃ ।

\as{(ಸಹಸ್ರನಾಮ ೧೦೧ - ೨೦೦)\\}
ಮಣಿ - ಪೂರಾಂತರುದಿತಾ ವಿಷ್ಣುಗ್ರಂಥಿ - ವಿಭೇದಿನೀ ॥೩೮॥

ಆಜ್ಞಾ - ಚಕ್ರಾಂತರಾಲಸ್ಥಾ ರುದ್ರಗ್ರಂಥಿ - ವಿಭೇದಿನೀ~।\\
ಸಹಸ್ರಾರಾಂಬುಜಾರೂಢಾ ಸುಧಾ - ಸಾರಾಭಿವರ್ಷಿಣೀ ॥೩೯॥

ತಡಿಲ್ಲತಾ - ಸಮರುಚಿಃ ಷಟ್‍ಚಕ್ರೋಪರಿ - ಸಂಸ್ಥಿತಾ~।\\
ಮಹಾಸಕ್ತಿಃ ಕುಂಡಲಿನೀ ಬಿಸತಂತು - ತನೀಯಸೀ ॥೪೦॥

ಭವಾನೀ ಭಾವನಾಗಮ್ಯಾ ಭವಾರಣ್ಯ - ಕುಠಾರಿಕಾ~।\\
ಭದ್ರಪ್ರಿಯಾ ಭದ್ರಮೂರ್ತಿರ್ಭಕ್ತ - ಸೌಭಾಗ್ಯದಾಯಿನೀ ॥೪೧॥

ಭಕ್ತಿಪ್ರಿಯಾ ಭಕ್ತಿಗಮ್ಯಾ ಭಕ್ತಿವಶ್ಯಾ ಭಯಾಪಹಾ~।\\
ಶಾಂಭವೀ ಶಾರದಾರಾಧ್ಯಾ ಶರ್ವಾಣೀ ಶರ್ಮದಾಯಿನೀ ॥೪೨॥

ಶಾಂಕರೀ ಶ್ರೀಕರೀ ಸಾಧ್ವೀ ಶರಚ್ಚಂದ್ರ - ನಿಭಾನನಾ~।\\
ಶಾತೋದರೀ ಶಾಂತಿಮತೀ ನಿರಾಧಾರಾ ನಿರಂಜನಾ ॥೪೩॥

ನಿರ್ಲೇಪಾ ನಿರ್ಮಲಾ ನಿತ್ಯಾ ನಿರಾಕಾರಾ ನಿರಾಕುಲಾ~।\\
ನಿರ್ಗುಣಾ ನಿಷ್ಕಲಾ ಶಾಂತಾ ನಿಷ್ಕಾಮಾ ನಿರುಪಪ್ಲವಾ ॥೪೪॥

ನಿತ್ಯಮುಕ್ತಾ ನಿರ್ವಿಕಾರಾ ನಿಷ್ಪ್ರಪಂಚಾ ನಿರಾಶ್ರಯಾ~।\\
ನಿತ್ಯಶುದ್ಧಾ ನಿತ್ಯಬುದ್ಧಾ ನಿರವದ್ಯಾ ನಿರಂತರಾ ॥೪೫॥

ನಿಷ್ಕಾರಣಾ ನಿಷ್ಕಲಂಕಾ ನಿರುಪಾಧಿರ್ನಿರೀಶ್ವರಾ~।\\
ನೀರಾಗಾ ರಾಗಮಥನೀ ನಿರ್ಮದಾ ಮದನಾಶಿನೀ ॥೪೬॥

ನಿಶ್ಚಿಂತಾ ನಿರಹಂಕಾರಾ ನಿರ್ಮೋಹಾ ಮೋಹನಾಶಿನೀ~।\\
ನಿರ್ಮಮಾ ಮಮತಾಹಂತ್ರೀ ನಿಷ್ಪಾಪಾ ಪಾಪನಾಶಿನೀ ॥೪೭॥

ನಿಷ್ಕ್ರೋಧಾ ಕ್ರೋಧಶಮನೀ ನಿರ್ಲೋಭಾ ಲೋಭನಾಶಿನೀ~।\\
ನಿಃಸಂಶಯಾ ಸಂಶಯಘ್ನೀ ನಿರ್ಭವಾ ಭವನಾಶಿನೀ ॥೪೮॥

ನಿರ್ವಿಕಲ್ಪಾ ನಿರಾಬಾಧಾ ನಿರ್ಭೇದಾ ಭೇದನಾಶಿನೀ~।\\
ನಿರ್ನಾಶಾ ಮೃತ್ಯುಮಥಿನೀ ನಿಷ್ಕ್ರಿಯಾ ನಿಷ್ಪರಿಗ್ರಹಾ~।೪೯॥

ನಿಸ್ತುಲಾ ನೀಲಚಿಕುರಾ ನಿರಪಾಯಾ ನಿರತ್ಯಯಾ~।\\
ದುರ್ಲಭಾ ದುರ್ಗಮಾ ದುರ್ಗಾ ದುಃಖಹಂತ್ರೀ ಸುಖಪ್ರದಾ ॥೫೦॥

ದುಷ್ಟದೂರಾ ದುರಾಚಾರ - ಶಮನೀ ದೋಷವರ್ಜಿತಾ~।\\
ಸರ್ವಜ್ಞಾ ಸಾಂದ್ರಕರುಣಾ ಸಮಾನಾಧಿಕ - ವರ್ಜಿತಾ ॥೫೧॥

ಸರ್ವಶಕ್ತಿಮಯೀ ಸರ್ವ - ಮಂಗಲಾ \as{(೨೦೦)}

\as{(ಅಷ್ಟೋತ್ತರ ೧೧ - ೨೦)\\}
ವಿಕಚಾಂಭೋರುಹದಲಲೋಚನಾಯೈ ನಮೋ ನಮಃ ।\\
ಶರಚ್ಚಾಂಪೇಯಪುಷ್ಪಾಭನಾಸಿಕಾಯೈ ನಮೋ ನಮಃ ।\\
ಲಸತ್ಕಾಂಚನತಾಟಂಕಯುಗಲಾಯೈ ನಮೋ ನಮಃ ।\\
ಮಣಿದರ್ಪಣಸಂಕಾಶಕಪೋಲಾಯೈ ನಮೋ ನಮಃ ।\\
ತಾಂಬೂಲಪೂರಿತಸ್ಮೇರವದನಾಯೈ ನಮೋ ನಮಃ ।\\
ಸುಪಕ್ವದಾಡಿಮೀಬೀಜರದನಾಯೈ ನಮೋ ನಮಃ ।\\
ಕಂಬುಪೂಗಸಮಚ್ಛಾಯಕಂಧರಾಯೈ ನಮೋ ನಮಃ ।\\
ಸ್ಥೂಲಮುಕ್ತಾಫಲೋದಾರಸುಹಾರಾಯೈ ನಮೋ ನಮಃ ।\\
ಗಿರೀಶಬದ್ಧಮಾಂಗಲ್ಯಮಂಗಲಾಯೈ ನಮೋ ನಮಃ ।\\
ಪದ್ಮಪಾಶಾಂಕುಶಲಸತ್ಕರಾಬ್ಜಾಯೈ ನಮೋ ನಮಃ ।	

\as{ಬ್ರ॒ಹ್ಮಾ ದೇ॒ವಾನಾಂ᳚ ಪದ॒ವೀಃ ಕ॑ವೀ॒ನಾಮೃಷಿ॒ರ್ವಿಪ್ರಾ॑ಣಾಂ ಮಹಿ॒ಷೋ ಮೃ॒ಗಾಣಾಂ᳚ ।
ಶ್ಯೇ॒ನೋ ಗೃದ್ಧ್ರಾ॑ಣಾ॒ಁ ಸ್ವಧಿ॑ತಿ॒ರ್ವನಾ॑ನಾ॒ಁ ಸೋಮಃ॑ ಪ॒ವಿತ್ರ॒ಮತ್ಯೇ॑ತಿ॒ ರೇಭನ್॑ ॥}
\section{ಚತುರಶ್ರಾದ್ಯರೇಖಾಯೈ ನಮಃ ।\\ (ಇತಿ ವ್ಯಾಪಕಂ ನ್ಯಸ್ಯ)}
ದಕ್ಷಾಂಸಪೃಷ್ಠರೂಪ ಶಾಂತರಸಾತ್ಮನೇ ಅಣಿಮಾಸಿದ್ಧ್ಯೈ ನಮಃ ।\\
ದಕ್ಷಪಾಣ್ಯಂಗುಲ್ಯಗ್ರರೂಪ ಅದ್ಭುತರಸಾತ್ಮನೇ ಲಘಿಮಾಸಿದ್ಧ್ಯೈ ನಮಃ ।\\
ದಕ್ಷಸ್ಫಿಗ್ರೂಪ ಕರುಣರಸಾತ್ಮನೇ ಮಹಿಮಾಸಿದ್ಧ್ಯೈ ನಮಃ ।\\
ದಕ್ಷಪಾದಾಂಗುಲ್ಯಗ್ರರೂಪ ವೀರರಸಾತ್ಮನೇ ಈಶಿತ್ವಸಿದ್ಧ್ಯೈ ನಮಃ ।\\
ವಾಮಪಾದಾಂಗುಲ್ಯಗ್ರರೂಪ ಹಾಸ್ಯರಸಾತ್ಮನೇ ವಶಿತ್ವಸಿದ್ಧ್ಯೈ ನಮಃ ।\\
ವಾಮಸ್ಫಿಗ್ರೂಪ ಬೀಭತ್ಸರಸಾತ್ಮನೇ ಪ್ರಾಕಾಮ್ಯಸಿದ್ಧ್ಯೈ ನಮಃ ।\\
ವಾಮಪಾಣ್ಯಂಗುಲ್ಯಗ್ರರೂಪ ರೌದ್ರರಸಾತ್ಮನೇ ಭುಕ್ತಿಸಿದ್ಧ್ಯೈ ನಮಃ ।\\
ವಾಮಾಂಸಪೃಷ್ಠರೂಪ ಭಯಾನಕರಸಾತ್ಮನೇ ಇಚ್ಛಾಸಿದ್ಧ್ಯೈ ನಮಃ ।\\
ಚೂಲೀಮೂಲರೂಪ ಶೃಂಗಾರರಸಾತ್ಮನೇ ಪ್ರಾಪ್ತಿಸಿದ್ಧ್ಯೈ ನಮಃ ।\\
ಚೂಲೀಪೃಷ್ಠರೂಪ ನಿಯತ್ಯಾತ್ಮನೇ ಸರ್ವಕಾಮಸಿದ್ಧ್ಯೈ ನಮಃ ।

\as{ಅ॒ಜಾಮೇಕಾಂ॒ ಲೋಹಿ॑ತಶುಕ್ಲಕೃ॒ಷ್ಣಾಂ ಬ॒ಹ್ವೀಂ ಪ್ರ॒ಜಾಂ ಜ॒ನಯಂ॑ತೀಁ॒ ಸರೂ॑ಪಾಂ ।
ಅ॒ಜೋ ಹ್ಯೇಕೋ॑ ಜು॒ಷಮಾ॑ಣೋಽನು॒ಶೇತೇ॒ ಜಹಾ᳚ತ್ಯೇನಾಂ ಭು॒ಕ್ತಭೋ॑ಗಾ॒ಮಜೋ᳚ಽನ್ಯಃ ॥}
\section{ಚತುರಶ್ರ ಮಧ್ಯರೇಖಾಯೈ ನಮಃ ।\\ (ಇತಿ ವ್ಯಾಪಕಂ ನ್ಯಸ್ಯ)}
ಪಾದಾಂಗುಷ್ಠದ್ವಯರೂಪ ಕಾಮಾತ್ಮನೇ ಬ್ರಾಹ್ಮ್ಯೈ ನಮಃ ।\\
ದಕ್ಷಪಾರ್ಶ್ವರೂಪ ಕ್ರೋಧಾತ್ಮನೇ ಮಾಹೇಶ್ವರ್ಯೈ ನಮಃ ।\\
ಮೂರ್ಧರೂಪ ಲೋಭಾತ್ಮನೇ ಕೌಮಾರ್ಯೈ ನಮಃ ।\\
ವಾಮಪಾರ್ಶ್ವರೂಪ ಮೋಹಾತ್ಮನೇ ವೈಷ್ಣವ್ಯೈ ನಮಃ ।\\
ವಾಮಜಾನುರೂಪ ಮದಾತ್ಮನೇ ವಾರಾಹ್ಯೈ ನಮಃ ।\\
ದಕ್ಷಜಾನುರೂಪ ಮಾತ್ಸರ್ಯಾತ್ಮನೇ ಮಾಹೇಂದ್ರ್ಯೈ ನಮಃ ।\\
ದಕ್ಷಬಹಿರಂಸರೂಪ ಪುಣ್ಯಾತ್ಮನೇ ಚಾಮುಂಡಾಯೈ ನಮಃ ।\\
ವಾಮಬಹಿರಂಸರೂಪ ಪಾಪಾತ್ಮನೇ ಮಹಾಲಕ್ಷ್ಮ್ಯೈ ನಮಃ ।

\as{ಹಁ॒ಸಶ್ಶು॑ಚಿ॒ಷದ್ವಸು॑ರಂತರಿಕ್ಷ॒ಸದ್ಧೋತಾ॑ ವೇದಿ॒ಷದತಿ॑ಥಿರ್ದುರೋಣ॒ಸತ್ ।
ನೃ॒ಷದ್ವ॑ರ॒ ಸದೃ॑ತ॒ ಸದ್ವ್ಯೋ॑ಮ॒ ಸದ॒ಬ್ಜಾ ಗೋ॒ಜಾ ಋ॑ತ॒ಜಾ ಅ॑ದ್ರಿ॒ಜಾ ಋ॒ತಂ ಬೃ॒ಹತ್ ॥}
\section{ಚತುರಶ್ರಾಂತ್ಯರೇಖಾಯೈ ನಮಃ ।\\ (ಇತಿ ವ್ಯಾಪಕಂ ನ್ಯಸ್ಯ)}
ಪಾದಾಂಗುಷ್ಠದ್ವಯರೂಪ ಸಹಸ್ರದಲಕಮಲಾತ್ಮನೇ ಸರ್ವಸಂಕ್ಷೋಭಿಣೀಮುದ್ರಾಯೈ ನಮಃ ।\\
ದಕ್ಷಪಾರ್ಶ್ವರೂಪ ಮೂಲಾಧಾರಾತ್ಮನೇ ಸರ್ವವಿದ್ರಾವಿಣೀಮುದ್ರಾಯೈ ನಮಃ ।\\
ಮೂರ್ಧರೂಪ ಸ್ವಾಧಿಷ್ಠಾನಾತ್ಮನೇ ಸರ್ವಾಕರ್ಷಿಣೀಮುದ್ರಾಯೈ ನಮಃ ।\\
ವಾಮಪಾರ್ಶ್ವರೂಪ ಮಣಿಪೂರಾತ್ಮನೇ ಸರ್ವವಶಂಕರೀಮುದ್ರಾಯೈ ನಮಃ ।\\
ವಾಮಜಾನುರೂಪ ಅನಾಹತಾತ್ಮನೇ ಸರ್ವೋನ್ಮಾದಿನೀಮುದ್ರಾಯೈ ನಮಃ ।\\
ದಕ್ಷಜಾನುರೂಪ ವಿಶುದ್ಧ್ಯಾತ್ಮನೇ ಸರ್ವಮಹಾಂಕುಶಾಮುದ್ರಾಯೈ ನಮಃ ।\\
ದಕ್ಷಾಂತರಂಸರೂಪ ಇಂದ್ರಯೋನ್ಯಾತ್ಮನೇ ಸರ್ವಖೇಚರೀಮುದ್ರಾಯೈ ನಮಃ ।\\
ವಾಮಾಂತರಂಸರೂಪ ಆಜ್ಞಾತ್ಮನೇ ಸರ್ವಬೀಜಾಮುದ್ರಾಯೈ ನಮಃ ।\\
ದ್ವಾದಶಾಂತರೂಪ ಊರ್ಧ್ವಸಹಸ್ರದಲಕಮಲಾತ್ಮನೇ ಸರ್ವಯೋನಿಮುದ್ರಾಯೈ ನಮಃ ।\\
ಪಾದಾಂಗುಷ್ಠದ್ವಯರೂಪ ಆಧಾರನವಕಾತ್ಮನೇ ಸರ್ವತ್ರಿಖಂಡಾಮುದ್ರಾಯೈ ನಮಃ ।\\
ಹೃದ್ರೂಪ ತ್ರೈಲೋಕ್ಯಮೋಹನಚಕ್ರೇಶ್ವರ್ಯೈ ತ್ರಿಪುರಾಯೈ ನಮಃ ।\\
ಪ್ರಕಟಯೋಗಿನೀರೂಪ ಸ್ವಾತ್ಮಾತ್ಮನೇ ಅಣಿಮಾಸಿದ್ಧ್ಯೈ ನಮಃ ।\\
ಅಪರಿಚ್ಛಿನ್ನರೂಪ ಸ್ವಾತ್ಮಾತ್ಮನೇ ಸರ್ವಸಂಕ್ಷೋಭಿಣೀಮುದ್ರಾಯೈ ನಮಃ ।

\as{(ಸಹಸ್ರನಾಮ ೨೦೧ - ೩೦೦)\\}
ಸದ್ಗತಿಪ್ರದಾ~।\\
ಸರ್ವೇಶ್ವರೀ ಸರ್ವಮಯೀ ಸರ್ವಮಂತ್ರ - ಸ್ವರೂಪಿಣೀ ॥೫೨॥

ಸರ್ವ - ಯಂತ್ರಾತ್ಮಿಕಾ ಸರ್ವ - ತಂತ್ರರೂಪಾ ಮನೋನ್ಮನೀ~।\\
ಮಾಹೇಶ್ವರೀ ಮಹಾದೇವೀ ಮಹಾಲಕ್ಷ್ಮೀರ್ಮೃಡಪ್ರಿಯಾ ॥೫೩॥

ಮಹಾರೂಪಾ ಮಹಾಪೂಜ್ಯಾ ಮಹಾಪಾತಕ - ನಾಶಿನೀ~।\\
ಮಹಾಮಾಯಾ ಮಹಾಸತ್ತ್ವಾ ಮಹಾಶಕ್ತಿರ್ಮಹಾರತಿಃ ॥೫೪॥

ಮಹಾಭೋಗಾ ಮಹೈಶ್ವರ್ಯಾ ಮಹಾವೀರ್ಯಾ ಮಹಾಬಲಾ~।\\
ಮಹಾಬುದ್ಧಿರ್ಮಹಾಸಿದ್ಧಿರ್ಮಹಾಯೋಗೇಶ್ವರೇಶ್ವರೀ ॥೫೫॥

ಮಹಾತಂತ್ರಾ ಮಹಾಮಂತ್ರಾ ಮಹಾಯಂತ್ರಾ ಮಹಾಸನಾ~।\\
ಮಹಾಯಾಗ - ಕ್ರಮಾರಾಧ್ಯಾ ಮಹಾಭೈರವ - ಪೂಜಿತಾ ॥೫೬॥

ಮಹೇಶ್ವರ - ಮಹಾಕಲ್ಪ - ಮಹಾತಾಂಡವ - ಸಾಕ್ಷಿಣೀ~।\\
ಮಹಾಕಾಮೇಶ - ಮಹಿಷೀ ಮಹಾತ್ರಿಪುರ - ಸುಂದರೀ ॥೫೭॥

ಚತುಃಷಷ್ಟ್ಯುಪಚಾರಾಢ್ಯಾ ಚತುಃಷಷ್ಟಿಕಲಾಮಯೀ~।\\
ಮಹಾಚತುಃ - ಷಷ್ಟಿಕೋಟಿ - ಯೋಗಿನೀ - ಗಣಸೇವಿತಾ ॥೫೮॥

ಮನುವಿದ್ಯಾ ಚಂದ್ರವಿದ್ಯಾ ಚಂದ್ರಮಂಡಲ - ಮಧ್ಯಗಾ~।\\
ಚಾರುರೂಪಾ ಚಾರುಹಾಸಾ ಚಾರುಚಂದ್ರ - ಕಲಾಧರಾ ॥೫೯॥

ಚರಾಚರ - ಜಗನ್ನಾಥಾ ಚಕ್ರರಾಜ - ನಿಕೇತನಾ~।\\
ಪಾರ್ವತೀ ಪದ್ಮನಯನಾ ಪದ್ಮರಾಗ - ಸಮಪ್ರಭಾ ॥೬೦॥

ಪಂಚ - ಪ್ರೇತಾಸನಾಸೀನಾ ಪಂಚಬ್ರಹ್ಮ - ಸ್ವರೂಪಿಣೀ~।\\
ಚಿನ್ಮಯೀ ಪರಮಾನಂದಾ ವಿಜ್ಞಾನ - ಘನರೂಪಿಣೀ ॥೬೧॥

ಧ್ಯಾನ - ಧ್ಯಾತೃ - ಧ್ಯೇಯರೂಪಾ ಧರ್ಮಾಧರ್ಮ - ವಿವರ್ಜಿತಾ~।\\
ವಿಶ್ವರೂಪಾ ಜಾಗರಿಣೀ ಸ್ವಪಂತೀ ತೈಜಸಾತ್ಮಿಕಾ ॥೬೨॥

ಸುಪ್ತಾ ಪ್ರಾಜ್ಞಾತ್ಮಿಕಾ ತುರ್ಯಾ ಸರ್ವಾವಸ್ಥಾ - ವಿವರ್ಜಿತಾ~।\\
ಸೃಷ್ಟಿಕರ್ತ್ರೀ ಬ್ರಹ್ಮರೂಪಾ ಗೋಪ್ತ್ರೀ ಗೋವಿಂದರೂಪಿಣೀ ॥೬೩॥

ಸಂಹಾರಿಣೀ ರುದ್ರರೂಪಾ ತಿರೋಧಾನ - ಕರೀಶ್ವರೀ~।\\
ಸದಾಶಿವಾಽನುಗ್ರಹದಾ ಪಂಚಕೃತ್ಯ - ಪರಾಯಣಾ ॥೬೪॥

ಭಾನುಮಂಡಲ - ಮಧ್ಯಸ್ಥಾ ಭೈರವೀ ಭಗಮಾಲಿನೀ~।\\
ಪದ್ಮಾಸನಾ ಭಗವತೀ ಪದ್ಮನಾಭ - ಸಹೋದರೀ ॥೬೫॥

ಉನ್ಮೇಷ - ನಿಮಿಷೋತ್ಪನ್ನ - ವಿಪನ್ನ - ಭುವನಾವಲಿಃ~।\\
ಸಹಸ್ರ - ಶೀರ್ಷವದನಾ ಸಹಸ್ರಾಕ್ಷೀ ಸಹಸ್ರಪಾತ್ ॥೬೬॥

ಆಬ್ರಹ್ಮ - ಕೀಟ - ಜನನೀ ವರ್ಣಾಶ್ರಮ - ವಿಧಾಯಿನೀ~।\\
ನಿಜಾಜ್ಞಾರೂಪ - ನಿಗಮಾ ಪುಣ್ಯಾಪುಣ್ಯ - ಫಲಪ್ರದಾ ॥೬೭॥

ಶ್ರುತಿ - ಸೀಮಂತ - ಸಿಂದೂರೀ - ಕೃತ - ಪಾದಾಬ್ಜ - ಧೂಲಿಕಾ~।\\
ಸಕಲಾಗಮ - ಸಂದೋಹ - ಶುಕ್ತಿ - ಸಂಪುಟ - ಮೌಕ್ತಿಕಾ ॥೬೮॥

ಪುರುಷಾರ್ಥಪ್ರದಾ ಪೂರ್ಣಾ ಭೋಗಿನೀ ಭುವನೇಶ್ವರೀ~।\\
ಅಂಬಿಕಾಽಽನಾದಿ - ನಿಧನಾ ಹರಿಬ್ರಹ್ಮೇಂದ್ರ - ಸೇವಿತಾ ॥೬೯॥

ನಾರಾಯಣೀ ನಾದರೂಪಾ ನಾಮರೂಪ - ವಿವರ್ಜಿತಾ \as{(೩೦೦)}।\\

\as{(ಅಷ್ಟೋತ್ತರ ೨೧ - ೩೦)\\}
ಪದ್ಮಕೈರವಮಂದಾರಸುಮಾಲಿನ್ಯೈ ನಮೋ ನಮಃ ।\\
ಸುವರ್ಣಕುಂಭಯುಗ್ಮಾಭಸುಕುಚಾಯೈ ನಮೋ ನಮಃ ।\\
ರಮಣೀಯಚತುರ್ಬಾಹುಸಂಯುಕ್ತಾಯೈ ನಮೋ ನಮಃ ।\\
ಕನಕಾಂಗದಕೇಯೂರಭೂಷಿತಾಯೈ ನಮೋ ನಮಃ ।\\
ಬೃಹತ್ಸೌವರ್ಣಸೌಂದರ್ಯವಸನಾಯೈ ನಮೋ ನಮಃ ।\\
ಬೃಹನ್ನಿತಂಬವಿಲಸಜ್ಜಘನಾಯೈ ನಮೋ ನಮಃ ।\\
ಸೌಭಾಗ್ಯಜಾತಶೃಂಗಾರಮಧ್ಯಮಾಯೈ ನಮೋ ನಮಃ ।\\
ದಿವ್ಯಭೂಷಣಸಂದೋಹರಂಜಿತಾಯೈ ನಮೋ ನಮಃ ।\\
ಪಾರಿಜಾತಗುಣಾಧಿಕ್ಯಪದಾಬ್ಜಾಯೈ ನಮೋ ನಮಃ ।\\
ಸುಪದ್ಮರಾಗಸಂಕಾಶಚರಣಾಯೈ ನಮೋ ನಮಃ ।

\as{ಘೃ॒ತಂ ಮಿ॑ಮಿಕ್ಷಿರೇ ಘೃ॒ತಮ॑ಸ್ಯ॒ಯೋನಿ॑ರ್ಘೃ॒ತೇಶ್ರಿ॒ತೋ ಘೃ॒ತಮು॑ವಸ್ಯ॒ ಧಾಮ॑ ।
ಅ॒ನು॒ಷ್ವ॒ಧಮಾವ॑ಹ ಮಾ॒ದಯ॑ಸ್ವ॒ ಸ್ವಾಹಾ॑ಕೃತಂ ವೃಷಭ ವಕ್ಷಿ ಹ॒ವ್ಯಂ ॥}
\section{ಷೋಡಶದಲಕಮಲಾಯ ನಮಃ ।\\ (ಇತಿ ವ್ಯಾಪಕಂ ನ್ಯಸ್ಯ)}
ದಕ್ಷಶ್ರೋತ್ರಪೃಷ್ಠರೂಪ ಪೃಥಿವ್ಯಾತ್ಮನೇ ಕಾಮಾಕರ್ಷಣೀ ನಿತ್ಯಾಕಲಾಯೈ ನಮಃ ।\\
ದಕ್ಷಾಂಸರೂಪಾಬಾತ್ಮನೇ ಬುದ್ಧ್ಯಾಕರ್ಷಣೀ ನಿತ್ಯಾಕಲಾಯೈ ನಮಃ ।\\
ದಕ್ಷಕೂರ್ಪರರೂಪ ತೇಜ ಆತ್ಮನೇ ಅಹಂಕಾರಾಕರ್ಷಣೀ ನಿತ್ಯಾಕಲಾಯೈ ನಮಃ ।\\
ದಕ್ಷಕರಪೃಷ್ಠರೂಪ ವಾಯ್ವಾತ್ಮನೇ ಶಬ್ದಾಕರ್ಷಣೀ ನಿತ್ಯಾಕಲಾಯೈ ನಮಃ ।\\
ದಕ್ಷೋರುರೂಪ ಆಕಾಶಾತ್ಮನೇ ಸ್ಪರ್ಶಾಕರ್ಷಣೀ ನಿತ್ಯಾಕಲಾಯೈ ನಮಃ ।\\
ದಕ್ಷಜಾನುರೂಪ ಶ್ರೋತ್ರಾತ್ಮನೇ ರೂಪಾಕರ್ಷಣೀ ನಿತ್ಯಾಕಲಾಯೈ ನಮಃ ।\\
ದಕ್ಷಗುಲ್ಫರೂಪ ತ್ವಗಾತ್ಮನೇ ರಸಾಕರ್ಷಣೀ ನಿತ್ಯಾಕಲಾಯೈ ನಮಃ ।\\
ದಕ್ಷಪಾದತಲರೂಪ ಚಕ್ಷುರಾತ್ಮನೇ ಗಂಧಾಕರ್ಷಣೀ ನಿತ್ಯಾಕಲಾಯೈ ನಮಃ ।\\
ವಾಮಪಾದತಲರೂಪ ಜಿಹ್ವಾತ್ಮನೇ ಚಿತ್ತಾಕರ್ಷಣೀ ನಿತ್ಯಾಕಲಾಯೈ ನಮಃ ।\\
ವಾಮಗುಲ್ಫರೂಪ ಘ್ರಾಣಾತ್ಮನೇ ಧೈರ್ಯಾಕರ್ಷಣೀ ನಿತ್ಯಾಕಲಾಯೈ ನಮಃ ।\\
ವಾಮಜಾನುರೂಪ ವಾಗಾತ್ಮನೇ ಸ್ಮೃತ್ಯಾಕರ್ಷಣೀ ನಿತ್ಯಾಕಲಾಯೈ ನಮಃ ।\\
ವಾಮೋರುರೂಪ ಪಾಣ್ಯಾತ್ಮನೇ ನಾಮಾಕರ್ಷಣೀ ನಿತ್ಯಾಕಲಾಯೈ ನಮಃ ।\\
ವಾಮಕರಪೃಷ್ಠರೂಪ ಪಾದಾತ್ಮನೇ ಬೀಜಾಕರ್ಷಣೀ ನಿತ್ಯಾಕಲಾಯೈ ನಮಃ ।\\
ವಾಮಕೂರ್ಪರರೂಪ ಪಾಯ್ವಾತ್ಮನೇ ಆತ್ಮಾಕರ್ಷಣೀ ನಿತ್ಯಾಕಲಾಯೈ ನಮಃ ।\\
ವಾಮಾಂಸರೂಪ ಉಪಸ್ಥಾತ್ಮನೇ ಅಮೃತಾಕರ್ಷಣೀ ನಿತ್ಯಾಕಲಾಯೈ ನಮಃ ।\\
ವಾಮಶ್ರೋತ್ರಪೃಷ್ಠರೂಪ ವಿಕೃತಮನಆತ್ಮನೇ ಶರೀರಾಕರ್ಷಣೀ ನಿತ್ಯಾಕಲಾಯೈ ನಮಃ ।\\
ಹೃದ್ರೂಪ ಸರ್ವಾಶಾಪರಿಪೂರಕಚಕ್ರೇಶ್ವರ್ಯೈ ತ್ರಿಪುರೇಶ್ಯೈ ನಮಃ ।\\
ಗುಪ್ತಯೋಗಿನೀರೂಪ ಸ್ವಾತ್ಮಾತ್ಮನೇ ಲಘಿಮಾಸಿದ್ಧ್ಯೈ ನಮಃ ।\\
ಅಪರಿಚ್ಛಿನ್ನರೂಪ ಸ್ವಾತ್ಮಾತ್ಮನೇ ಸರ್ವವಿದ್ರಾವಿಣೀಮುದ್ರಾಯೈ ನಮಃ ।

\as{(ಸಹಸ್ರನಾಮ ೩೦೧ - ೪೦೦)\\}
ಹ್ರೀಂಕಾರೀ ಹ್ರೀಂಮತೀ ಹೃದ್ಯಾ ಹೇಯೋಪಾದೇಯ - ವರ್ಜಿತಾ ॥೭೦॥

ರಾಜರಾಜಾರ್ಚಿತಾ ರಾಜ್ಞೀ ರಮ್ಯಾ ರಾಜೀವಲೋಚನಾ~।\\
ರಂಜನೀ ರಮಣೀ ರಸ್ಯಾ ರಣತ್ಕಿಂಕಿಣಿ - ಮೇಖಲಾ ॥೭೧॥

ರಮಾ ರಾಕೇಂದುವದನಾ ರತಿರೂಪಾ ರತಿಪ್ರಿಯಾ~।\\
ರಕ್ಷಾಕರೀ ರಾಕ್ಷಸಘ್ನೀ ರಾಮಾ ರಮಣಲಂಪಟಾ ॥೭೨॥

ಕಾಮ್ಯಾ ಕಾಮಕಲಾರೂಪಾ ಕದಂಬ - ಕುಸುಮ - ಪ್ರಿಯಾ~।\\
ಕಲ್ಯಾಣೀ ಜಗತೀಕಂದಾ ಕರುಣಾ - ರಸ - ಸಾಗರಾ ॥೭೩॥

ಕಲಾವತೀ ಕಲಾಲಾಪಾ ಕಾಂತಾ ಕಾದಂಬರೀಪ್ರಿಯಾ~।\\
ವರದಾ ವಾಮನಯನಾ ವಾರುಣೀ - ಮದ - ವಿಹ್ವಲಾ ॥೭೪॥

ವಿಶ್ವಾಧಿಕಾ ವೇದವೇದ್ಯಾ ವಿಂಧ್ಯಾಚಲ - ನಿವಾಸಿನೀ~।\\
ವಿಧಾತ್ರೀ ವೇದಜನನೀ ವಿಷ್ಣುಮಾಯಾ ವಿಲಾಸಿನೀ ॥೭೫॥

ಕ್ಷೇತ್ರಸ್ವರೂಪಾ ಕ್ಷೇತ್ರೇಶೀ ಕ್ಷೇತ್ರ - ಕ್ಷೇತ್ರಜ್ಞ - ಪಾಲಿನೀ~।\\
ಕ್ಷಯವೃದ್ಧಿ - ವಿನಿರ್ಮುಕ್ತಾ ಕ್ಷೇತ್ರಪಾಲ - ಸಮರ್ಚಿತಾ ॥೭೬॥

ವಿಜಯಾ ವಿಮಲಾ ವಂದ್ಯಾ ವಂದಾರು - ಜನ - ವತ್ಸಲಾ~।\\
ವಾಗ್ವಾದಿನೀ ವಾಮಕೇಶೀ ವಹ್ನಿಮಂಡಲ - ವಾಸಿನೀ ॥೭೭॥

ಭಕ್ತಿಮತ್ - ಕಲ್ಪಲತಿಕಾ ಪಶುಪಾಶ - ವಿಮೋಚಿನೀ~।\\
ಸಂಹೃತಾಶೇಷ - ಪಾಷಂಡಾ ಸದಾಚಾರ - ಪ್ರವರ್ತಿಕಾ ॥೭೮॥

ತಾಪತ್ರಯಾಗ್ನಿ - ಸಂತಪ್ತ - ಸಮಾಹ್ಲಾದನ ಚಂದ್ರಿಕಾ~।\\
ತರುಣೀ ತಾಪಸಾರಾಧ್ಯಾ ತನುಮಧ್ಯಾ ತಮೋಽಪಹಾ ॥೭೯॥

ಚಿತಿಸ್ತತ್ಪದ - ಲಕ್ಷ್ಯಾರ್ಥಾ ಚಿದೇಕರಸ - ರೂಪಿಣೀ~।\\
ಸ್ವಾತ್ಮಾನಂದ - ಲವೀಭೂತ - ಬ್ರಹ್ಮಾದ್ಯಾನಂದ - ಸಂತತಿಃ ॥೮೦॥

ಪರಾ ಪ್ರತ್ಯಕ್ಚಿತೀರೂಪಾ ಪಶ್ಯಂತೀ ಪರದೇವತಾ~।\\
ಮಧ್ಯಮಾ ವೈಖರೀರೂಪಾ ಭಕ್ತ - ಮಾನಸ - ಹಂಸಿಕಾ ॥೮೧॥

ಕಾಮೇಶ್ವರ - ಪ್ರಾಣನಾಡೀ ಕೃತಜ್ಞಾ ಕಾಮಪೂಜಿತಾ~।\\
ಶೃಂಗಾರ - ರಸ - ಸಂಪೂರ್ಣಾ ಜಯಾ ಜಾಲಂಧರ - ಸ್ಥಿತಾ ॥೮೨॥

ಓಡ್ಯಾಣಪೀಠ - ನಿಲಯಾ ಬಿಂದು - ಮಂಡಲವಾಸಿನೀ~।\\
ರಹೋಯಾಗ - ಕ್ರಮಾರಾಧ್ಯಾ ರಹಸ್ತರ್ಪಣ - ತರ್ಪಿತಾ ॥೮೩॥

ಸದ್ಯಃಪ್ರಸಾದಿನೀ ವಿಶ್ವ - ಸಾಕ್ಷಿಣೀ ಸಾಕ್ಷಿವರ್ಜಿತಾ~।\\
ಷಡಂಗದೇವತಾ - ಯುಕ್ತಾ ಷಾಡ್ಗುಣ್ಯ - ಪರಿಪೂರಿತಾ ॥೮೪॥

ನಿತ್ಯಕ್ಲಿನ್ನಾ ನಿರುಪಮಾ ನಿರ್ವಾಣ - ಸುಖ - ದಾಯಿನೀ~।\\
ನಿತ್ಯಾ - ಷೋಡಶಿಕಾ - ರೂಪಾ ಶ್ರೀಕಂಠಾರ್ಧ - ಶರೀರಿಣೀ ॥೮೫॥

ಪ್ರಭಾವತೀ ಪ್ರಭಾರೂಪಾ ಪ್ರಸಿದ್ಧಾ ಪರಮೇಶ್ವರೀ~।\\
ಮೂಲಪ್ರಕೃತಿರವ್ಯಕ್ತಾ ವ್ಯಕ್ತಾವ್ಯಕ್ತ - ಸ್ವರೂಪಿಣೀ ॥೮೬॥

ವ್ಯಾಪಿನೀ \as{(೪೦೦)}

\as{(ಅಷ್ಟೋತ್ತರ ೪೦ - ೫೦)\\}
ಕಾಮಕೋಟಿಮಹಾಪದ್ಮಪೀಠಸ್ಥಾಯೈ ನಮೋ ನಮಃ ।\\
ಶ್ರೀಕಂಠನೇತ್ರಕುಮುದಚಂದ್ರಿಕಾಯೈ ನಮೋ ನಮಃ ।\\
ಸಚಾಮರ ರಮಾವಾಣೀವೀಜಿತಾಯೈ ನಮೋ ನಮಃ ।\\
ಭಕ್ತರಕ್ಷಣದಾಕ್ಷಿಣ್ಯಕಟಾಕ್ಷಾಯೈ ನಮೋ ನಮಃ ।\\
ಭೂತೇಶಾಲಿಂಗನೋದ್ಭೂತಪುಲಕಾಂಗ್ಯೈ ನಮೋ ನಮಃ ।\\
ಅನಂಗಜನಕಾಪಾಂಗವೀಕ್ಷಣಾಯೈ ನಮೋ ನಮಃ ।\\
ಬ್ರಹ್ಮೋಪೇಂದ್ರಶಿರೋರತ್ನರಂಜಿತಾಯೈ ನಮೋ ನಮಃ ।\\
ಶಚೀಮುಖ್ಯಾಮರವಧೂಸೇವಿತಾಯೈ ನಮೋ ನಮಃ ।\\
ಲೀಲಾಕಲ್ಪಿತಬ್ರಹ್ಮಾಂಡಮಂಡಲಾಯೈ ನಮೋ ನಮಃ ।\\
ಅಮೃತಾದಿಮಹಾಶಕ್ತಿಸಂವೃತಾಯೈ ನಮೋ ನಮಃ ।

\as{ಸ॒ಮು॒ದ್ರಾದೂ॒ರ್ಮಿರ್ಮಧು॑ಮಾಁ॒ ಉದಾ॑ರದುಪಾಁ॒ ಶುನಾ॒ ಸಮ॑ಮೃತ॒ತ್ವಮಾ॑ನಟ್ ।
ಘೃ॒ತಸ್ಯ॒ ನಾಮ॒ ಗುಹ್ಯಂ॒ ಯದಸ್ತಿ॑ ಜಿ॒ಹ್ವಾ ದೇ॒ವಾನಾ॑ಮ॒ಮೃತ॑ಸ್ಯ॒ ನಾಭಿಃ॑ ॥}
\section{ಅಷ್ಟದಲಪದ್ಮಾಯ ನಮಃ ।\\ (ಇತಿ ವ್ಯಾಪಕಂ ನ್ಯಸ್ಯ)}
ದಕ್ಷಶಂಖರೂಪ ವಚನಾತ್ಮನೇ ಅನಂಗಕುಸುಮಾದೇವ್ಯೈ ನಮಃ ।\\
ದಕ್ಷಬಾಹುಮೂಲರೂಪಾದಾನಾತ್ಮನೇ ಅನಂಗಮೇಖಲಾದೇವ್ಯೈ ನಮಃ ।\\
ದಕ್ಷೋರುರೂಪ ಗಮನಾತ್ಮನೇ ಅನಂಗಮದನಾದೇವ್ಯೈ ನಮಃ ।\\
ದಕ್ಷಗುಲ್ಫರೂಪ ವಿಸರ್ಗಾತ್ಮನೇ ಅನಂಗಮದನಾತುರಾದೇವ್ಯೈ ನಮಃ ।\\
ವಾಮಗುಲ್ಫರೂಪ ಆನಂದಾತ್ಮನೇ ಅನಂಗರೇಖಾದೇವ್ಯೈ ನಮಃ ।\\
ವಾಮೋರುರೂಪ ಹಾನಾಖ್ಯಬುದ್ಧ್ಯಾತ್ಮನೇ ಅನಂಗವೇಗಿನೀದೇವ್ಯೈ ನಮಃ ।\\
ವಾಮಬಾಹುಮೂಲರೂಪ ಉಪಾದಾನಾಖ್ಯಬುದ್ಧ್ಯಾತ್ಮನೇ ಅನಂಗಾಂಕುಶಾದೇವ್ಯೈ ನಮಃ ।\\
ವಾಮಶಂಖರೂಪ ಉಪೇಕ್ಷಾಖ್ಯಬುದ್ಧ್ಯಾತ್ಮನೇ ಅನಂಗಮಾಲಿನೀದೇವ್ಯೈ ನಮಃ ।\\
ಹೃದ್ರೂಪ ಸರ್ವಸಂಕ್ಷೋಭಣಚಕ್ರೇಶ್ವರ್ಯೈ ತ್ರಿಪುರಸುಂದರ್ಯೈ ನಮಃ ।\\
ಗುಪ್ತತರಯೋಗಿನೀರೂಪ ಸ್ವಾತ್ಮಾತ್ಮನೇ ಮಹಿಮಾಸಿದ್ಧ್ಯೈ ನಮಃ ।\\
ಅಪರಿಚ್ಛಿನ್ನ ಸ್ವಾತ್ಮಾತ್ಮನೇ ಸರ್ವಾಕರ್ಷಿಣೀಮುದ್ರಾಯೈ ನಮಃ ।

\as{(ಸಹಸ್ರನಾಮ ೪೦೧ - ೫೦೦)\\}
ವಿವಿಧಾಕಾರಾ ವಿದ್ಯಾವಿದ್ಯಾ - ಸ್ವರೂಪಿಣೀ~।\\
ಮಹಾಕಾಮೇಶ - ನಯನ - ಕುಮುದಾಹ್ಲಾದ - ಕೌಮುದೀ ॥೮೭॥

ಭಕ್ತ - ಹಾರ್ದ - ತಮೋಭೇದ - ಭಾನುಮದ್ಭಾನು - ಸಂತತಿಃ~।\\
ಶಿವದೂತೀ ಶಿವಾರಾಧ್ಯಾ ಶಿವಮೂರ್ತಿಃ ಶಿವಂಕರೀ ॥೮೮॥

ಶಿವಪ್ರಿಯಾ ಶಿವಪರಾ ಶಿಷ್ಟೇಷ್ಟಾ ಶಿಷ್ಟಪೂಜಿತಾ~।\\
ಅಪ್ರಮೇಯಾ ಸ್ವಪ್ರಕಾಶಾ ಮನೋವಾಚಾಮಗೋಚರಾ ॥೮೯॥

ಚಿಚ್ಛಕ್ತಿಶ್ ಚೇತನಾರೂಪಾ ಜಡಶಕ್ತಿರ್ಜಡಾತ್ಮಿಕಾ~।\\
ಗಾಯತ್ರೀ ವ್ಯಾಹೃತಿಃ ಸಂಧ್ಯಾ ದ್ವಿಜಬೃಂದ - ನಿಷೇವಿತಾ ॥೯೦॥

ತತ್ತ್ವಾಸನಾ ತತ್ತ್ವಮಯೀ ಪಂಚ - ಕೋಶಾಂತರ - ಸ್ಥಿತಾ~।\\
ನಿಃಸೀಮಮಹಿಮಾ ನಿತ್ಯ - ಯೌವನಾ ಮದಶಾಲಿನೀ ॥೯೧॥

ಮದಘೂರ್ಣಿತ - ರಕ್ತಾಕ್ಷೀ ಮದಪಾಟಲ - ಗಂಡಭೂಃ~।\\
ಚಂದನ - ದ್ರವ - ದಿಗ್ಧಾಂಗೀ ಚಾಂಪೇಯ - ಕುಸುಮ - ಪ್ರಿಯಾ ॥೯೨॥

ಕುಶಲಾ ಕೋಮಲಾಕಾರಾ ಕುರುಕುಲ್ಲಾ ಕುಲೇಶ್ವರೀ~।\\
ಕುಲಕುಂಡಾಲಯಾ ಕೌಲ - ಮಾರ್ಗ - ತತ್ಪರ - ಸೇವಿತಾ ॥೯೩॥

ಕುಮಾರ - ಗಣನಾಥಾಂಬಾ ತುಷ್ಟಿಃ ಪುಷ್ಟಿರ್ಮತಿರ್ಧೃತಿಃ~।\\
ಶಾಂತಿಃ ಸ್ವಸ್ತಿಮತೀ ಕಾಂತಿರ್ನಂದಿನೀ ವಿಘ್ನನಾಶಿನೀ ॥೯೪॥

ತೇಜೋವತೀ ತ್ರಿನಯನಾ ಲೋಲಾಕ್ಷೀ - ಕಾಮರೂಪಿಣೀ~।\\
ಮಾಲಿನೀ ಹಂಸಿನೀ ಮಾತಾ ಮಲಯಾಚಲ - ವಾಸಿನೀ ॥೯೫॥

ಸುಮುಖೀ ನಲಿನೀ ಸುಭ್ರೂಃ ಶೋಭನಾ ಸುರನಾಯಿಕಾ~।\\
ಕಾಲಕಂಠೀ ಕಾಂತಿಮತೀ ಕ್ಷೋಭಿಣೀ ಸೂಕ್ಷ್ಮರೂಪಿಣೀ ॥೯೬॥

ವಜ್ರೇಶ್ವರೀ ವಾಮದೇವೀ ವಯೋಽವಸ್ಥಾ - ವಿವರ್ಜಿತಾ~।\\
ಸಿದ್ಧೇಶ್ವರೀ ಸಿದ್ಧವಿದ್ಯಾ ಸಿದ್ಧಮಾತಾ ಯಶಸ್ವಿನೀ ॥೯೭॥

ವಿಶುದ್ಧಿಚಕ್ರ - ನಿಲಯಾಽಽರಕ್ತವರ್ಣಾ ತ್ರಿಲೋಚನಾ~।\\
ಖಟ್‍ವಾಂಗಾದಿ - ಪ್ರಹರಣಾ ವದನೈಕ - ಸಮನ್ವಿತಾ ॥೯೮॥

ಪಾಯಸಾನ್ನಪ್ರಿಯಾ ತ್ವಕ್ಸ್ಥಾ ಪಶುಲೋಕ - ಭಯಂಕರೀ~।\\
ಅಮೃತಾದಿ - ಮಹಾಶಕ್ತಿ - ಸಂವೃತಾ ಡಾಕಿನೀಶ್ವರೀ ॥೯೯॥

ಅನಾಹತಾಬ್ಜ - ನಿಲಯಾ ಶ್ಯಾಮಾಭಾ ವದನದ್ವಯಾ~।\\
ದಂಷ್ಟ್ರೋಜ್ಜ್ವಲಾಽಕ್ಷ - ಮಾಲಾದಿ - ಧರಾ ರುಧಿರಸಂಸ್ಥಿತಾ ॥೧೦೦॥

ಕಾಲರಾತ್ರ್ಯಾದಿ - ಶಕ್ತ್ಯೌಘ - ವೃತಾ ಸ್ನಿಗ್ಧೌದನಪ್ರಿಯಾ~।\\
ಮಹಾವೀರೇಂದ್ರ - ವರದಾ ರಾಕಿಣ್ಯಂಬಾ - ಸ್ವರೂಪಿಣೀ ॥೧೦೧॥

ಮಣಿಪೂರಾಬ್ಜ - ನಿಲಯಾ ವದನತ್ರಯ - ಸಂಯುತಾ~।\\
ವಜ್ರಾದಿಕಾಯುಧೋಪೇತಾ ಡಾಮರ್ಯಾದಿಭಿರಾವೃತಾ ॥೧೦೨॥

ರಕ್ತವರ್ಣಾ ಮಾಂಸನಿಷ್ಠಾ \as{(೫೦೦)}

\as{(ಅಷ್ಟೋತ್ತರ ೪೧ - ೫೦)\\}
ಏಕಾತಪತ್ರಸಾಮ್ರಾಜ್ಯದಾಯಿಕಾಯೈ ನಮೋ ನಮಃ ।\\
ಸನಕಾದಿಸಮಾರಾಧ್ಯಪಾದುಕಾಯೈ ನಮೋ ನಮಃ ।\\
ದೇವರ್ಷಿಭಿಸ್ಸ್ತೂಯಮಾನವೈಭವಾಯೈ ನಮೋ ನಮಃ ।\\
ಕಲಶೋದ್ಭವದುರ್ವಾಸಃಪೂಜಿತಾಯೈ ನಮೋ ನಮಃ ।\\
ಮತ್ತೇಭವಕ್ತ್ರಷಡ್ವಕ್ತ್ರವತ್ಸಲಾಯೈ ನಮೋ ನಮಃ ।\\
ಚಕ್ರರಾಜಮಹಾಯಂತ್ರಮಧ್ಯವರ್ತಿನ್ಯೈ ನಮೋ ನಮಃ ।\\
ಚಿದಗ್ನಿಕುಂಡಸಂಭೂತಸುದೇಹಾಯೈ ನಮೋ ನಮಃ ।\\
ಶಶಾಂಕಖಂಡಸಂಯುಕ್ತಮಕುಟಾಯೈ ನಮೋ ನಮಃ ।\\
ಮತ್ತಹಂಸವಧೂಮಂದಗಮನಾಯೈ ನಮೋ ನಮಃ ।\\
ವಂದಾರುಜನಸಂದೋಹವಂದಿತಾಯೈ ನಮೋ ನಮಃ ।

\as{ಓಂ ॥ ಹಿರ॑ಣ್ಯವರ್ಣಾಂ॒ ಹರಿ॑ಣೀಂ ಸು॒ವರ್ಣ॑ರಜ॒ತಸ್ರ॑ಜಾಂ ।\\
ಚಂ॒ದ್ರಾಂ ಹಿ॒ರಣ್ಮ॑ಯೀಂ ಲ॒ಕ್ಷ್ಮೀಂ ಜಾತ॑ವೇದೋ ಮ॒ ಆವ॑ಹ ॥ ೧॥}

೪ ಕಕಾರರೂಪಾಯೈ ನಮಃ ।\\
೪ ಕಲ್ಯಾಣ್ಯೈ ನಮಃ ।\\
೪ ಕಲ್ಯಾಣಗುಣಶಾಲಿನ್ಯೈ ನಮಃ ।\\
೪ ಕಲ್ಯಾಣಶೈಲನಿಲಯಾಯೈ ನಮಃ ।\\
೪ ಕಮನೀಯಾಯೈ ನಮಃ ।\\
೪ ಕಲಾವತ್ಯೈ ನಮಃ ।\\
೪ ಕಮಲಾಕ್ಷ್ಯೈ ನಮಃ ।\\
೪ ಕಲ್ಮಷಘ್ನ್ಯೈ ನಮಃ ।\\
೪ ಕರುಣಾಮೃತಸಾಗರಾಯೈ ನಮಃ ।\\
೪ ಕದಂಬಕಾನನಾವಾಸಾಯೈ ನಮಃ ।\\
೪ ಕದಂಬಕುಸುಮಪ್ರಿಯಾಯೈ ನಮಃ ।\\
೪ ಕಂದರ್ಪವಿದ್ಯಾಯೈ ನಮಃ ।\\
೪ ಕಂದರ್ಪಜನಕಾಪಾಂಗವೀಕ್ಷಣಾಯೈ ನಮಃ ।\\
೪ ಕರ್ಪೂರವೀಟೀಸೌರಭ್ಯಕಲ್ಲೋಲಿತಕಕುಪ್ತಟಾಯೈ ನಮಃ ।\\
೪ ಕಲಿದೋಷಹರಾಯೈ ನಮಃ ।\\
೪ ಕಂಜಲೋಚನಾಯೈ ನಮಃ ।\\
೪ ಕಮ್ರವಿಗ್ರಹಾಯೈ ನಮಃ ।\\
೪ ಕರ್ಮಾದಿಸಾಕ್ಷಿಣ್ಯೈ ನಮಃ ।\\
೪ ಕಾರಯಿತ್ರ್ಯೈ ನಮಃ ।\\
೪ ಕರ್ಮಫಲಪ್ರದಾಯೈ ನಮಃ ।

\as{ತಾಂ ಮ॒ ಆವ॑ಹ॒ ಜಾತ॑ವೇದೋ ಲ॒ಕ್ಷ್ಮೀಮನ॑ಪಗಾ॒ಮಿನೀಂ᳚ ।\\
ಯಸ್ಯಾಂ॒ ಹಿರ॑ಣ್ಯಂ ವಿಂ॒ದೇಯಂ॒ ಗಾಮಶ್ವಂ॒ ಪುರು॑ಷಾನ॒ಹಂ ॥ ೨॥}

೪ ಏಕಾರರೂಪಾಯೈ ನಮಃ ।\\
೪ ಏಕಾಕ್ಷರ್ಯೈ ನಮಃ ।\\
೪ ಏಕಾನೇಕಾಕ್ಷರಾಕೃತ್ಯೈ ನಮಃ ।\\
೪ ಏತತ್ತದಿತ್ಯನಿರ್ದೇಶ್ಯಾಯೈ ನಮಃ ।\\
೪ ಏಕಾನಂದಚಿದಾಕೃತ್ಯೈ ನಮಃ ।\\
೪ ಏವಮಿತ್ಯಾಗಮಾಬೋಧ್ಯಾಯೈ ನಮಃ ।\\
೪ ಏಕಭಕ್ತಿಮದರ್ಚಿತಾಯೈ ನಮಃ ।\\
೪ ಏಕಾಗ್ರಚಿತ್ತನಿರ್ಧ್ಯಾತಾಯೈ ನಮಃ ।\\
೪ ಏಷಣಾರಹಿತಾದೃತಾಯೈ ನಮಃ ।\\
೪ ಏಲಾಸುಗಂಧಿಚಿಕುರಾಯೈ ನಮಃ ।\\
೪ ಏನಃಕೂಟವಿನಾಶಿನ್ಯೈ ನಮಃ ।\\
೪ ಏಕಭೋಗಾಯೈ ನಮಃ ।\\
೪ ಏಕರಸಾಯೈ ನಮಃ ।\\
೪ ಏಕೈಶ್ವರ್ಯಪ್ರದಾಯಿನ್ಯೈ ನಮಃ ।\\
೪ ಏಕಾತಪತ್ರಸಾಮ್ರಾಜ್ಯಪ್ರದಾಯೈ ನಮಃ ।\\
೪ ಏಕಾಂತಪೂಜಿತಾಯೈ ನಮಃ ।\\
೪ ಏಧಮಾನಪ್ರಭಾಯೈ ನಮಃ ।\\
೪ ಏಜದನೇಕಜಗದೀಶ್ವರ್ಯೈ ನಮಃ ।\\
೪ ಏಕವೀರಾದಿಸಂಸೇವ್ಯಾಯೈ ನಮಃ ।\\
೪ ಏಕಪ್ರಾಭವಶಾಲಿನ್ಯೈ ನಮಃ ।


\as{ಅ॒ಶ್ವ॒ಪೂ॒ರ್ವಾಂ ರ॑ಥಮ॒ಧ್ಯಾಂ ಹ॒ಸ್ತಿನಾ᳚ದಪ್ರ॒ಬೋಧಿ॑ನೀಂ ।\\
ಶ್ರಿಯಂ॑ ದೇ॒ವೀಮುಪ॑ಹ್ವಯೇ॒ ಶ್ರೀರ್ಮಾ᳚ದೇ॒ವೀರ್ಜು॑ಷತಾಂ ॥ ೩॥}

೪ ಈಕಾರರೂಪಾಯೈ ನಮಃ ।\\
೪ ಈಶಿತ್ರ್ಯೈ ನಮಃ ।\\
೪ ಈಪ್ಸಿತಾರ್ಥಪ್ರದಾಯಿನ್ಯೈ ನಮಃ ।\\
೪ ಈದೃಗಿತ್ಯಾವಿನಿರ್ದೇಶ್ಯಾಯೈ ನಮಃ ।\\
೪ ಈಶ್ವರತ್ವವಿಧಾಯಿನ್ಯೈ ನಮಃ ।\\
೪ ಈಶಾನಾದಿಬ್ರಹ್ಮಮಯ್ಯೈ ನಮಃ ।\\
೪ ಈಶಿತ್ವಾದ್ಯಷ್ಟಸಿದ್ಧಿದಾಯೈ ನಮಃ ।\\
೪ ಈಕ್ಷಿತ್ರ್ಯೈ ನಮಃ ।\\
೪ ಈಕ್ಷಣಸೃಷ್ಟಾಂಡಕೋಟ್ಯೈ ನಮಃ ।\\
೪ ಈಶ್ವರವಲ್ಲಭಾಯೈ ನಮಃ ।\\
೪ ಈಡಿತಾಯೈ ನಮಃ ।\\
೪ ಈಶ್ವರಾರ್ಧಾಂಗಶರೀರಾಯೈ ನಮಃ ।\\
೪ ಈಶಾಧಿದೇವತಾಯೈ ನಮಃ ।\\
೪ ಈಶ್ವರಪ್ರೇರಣಕರ್ಯೈ ನಮಃ ।\\
೪ ಈಶತಾಂಡವಸಾಕ್ಷಿಣ್ಯೈ ನಮಃ ।\\
೪ ಈಶ್ವರೋತ್ಸಂಗನಿಲಯಾಯೈ ನಮಃ ।\\
೪ ಈತಿಬಾಧಾವಿನಾಶಿನ್ಯೈ ನಮಃ ।\\
೪ ಈಹಾವಿರಹಿತಾಯೈ ನಮಃ ।\\
೪ ಈಶಶಕ್ತ್ಯೈ ನಮಃ ।\\
೪ ಈಷತ್ಸ್ಮಿತಾನನಾಯೈ ನಮಃ ।

\as{ಕಾಂ॒ ಸೋ॒ಸ್ಮಿ॒ತಾಂ ಹಿರ॑ಣ್ಯಪ್ರಾ॒ಕಾರಾ॑ಮಾ॒ರ್ದ್ರಾಂ ಜ್ವಲಂ॑ತೀಂ ತೃ॒ಪ್ತಾಂ ತ॒ರ್ಪಯಂ॑ತೀಂ ।\\
ಪ॒ದ್ಮೇ॒ ಸ್ಥಿ॒ತಾಂ ಪ॒ದ್ಮವ॑ರ್ಣಾಂ॒ ತಾಮಿ॒ಹೋಪ॑ಹ್ವಯೇ॒ ಶ್ರಿಯಂ ॥ ೪॥}

೪ ಲಕಾರರೂಪಾಯೈ ನಮಃ ।\\
೪ ಲಲಿತಾಯೈ ನಮಃ ।\\
೪ ಲಕ್ಷ್ಮೀವಾಣೀನಿಷೇವಿತಾಯೈ ನಮಃ ।\\
೪ ಲಾಕಿನ್ಯೈ ನಮಃ ।\\
೪ ಲಲನಾರೂಪಾಯೈ ನಮಃ ।\\
೪ ಲಸದ್ದಾಡಿಮಪಾಟಲಾಯೈ ನಮಃ ।\\
೪ ಲಲಂತಿಕಾಲಸತ್ಫಾಲಾಯೈ ನಮಃ ।\\
೪ ಲಲಾಟನಯನಾರ್ಚಿತಾಯೈ ನಮಃ ।\\
೪ ಲಕ್ಷಣೋಜ್ಜ್ವಲದಿವ್ಯಾಂಗ್ಯೈ ನಮಃ ।\\
೪ ಲಕ್ಷಕೋಟ್ಯಂಡನಾಯಿಕಾಯೈ ನಮಃ ।\\
೪ ಲಕ್ಷ್ಯಾರ್ಥಾಯೈ ನಮಃ ।\\
೪ ಲಕ್ಷಣಾಗಮ್ಯಾಯೈ ನಮಃ ।\\
೪ ಲಬ್ಧಕಾಮಾಯೈ ನಮಃ ।\\
೪ ಲತಾತನವೇ ನಮಃ ।\\
೪ ಲಲಾಮರಾಜದಲಿಕಾಯೈ ನಮಃ ।\\
೪ ಲಂಬಿಮುಕ್ತಾಲತಾಂಚಿತಾಯೈ ನಮಃ ।\\
೪ ಲಂಬೋದರಪ್ರಸುವೇ ನಮಃ ।\\
೪ ಲಭ್ಯಾಯೈ ನಮಃ ।\\
೪ ಲಜ್ಜಾಢ್ಯಾಯೈ ನಮಃ ।\\
೪ ಲಯವರ್ಜಿತಾಯೈ ನಮಃ ।

\as{ಚಂ॒ದ್ರಾಂ ಪ್ರ॑ಭಾ॒ಸಾಂ ಯ॒ಶಸಾ॒ ಜ್ವಲಂ॑ತೀಂ॒ ಶ್ರಿಯಂ॑ ಲೋ॒ಕೇ ದೇ॒ವಜು॑ಷ್ಟಾಮುದಾ॒ರಾಂ ।\\
ತಾಂ ಪ॒ದ್ಮಿನೀ॑ಮೀಂ॒ ಶರ॑ಣಮ॒ಹಂ ಪ್ರಪ॑ದ್ಯೇಽಲ॒ಕ್ಷ್ಮೀರ್ಮೇ॑ ನಶ್ಯತಾಂ॒ ತ್ವಾಂ ವೃ॑ಣೇ ॥ ೫॥}

೪ ಹ್ರೀಂಕಾರರೂಪಾಯೈ ನಮಃ ।\\
೪ ಹ್ರೀಂಕಾರನಿಲಯಾಯೈ ನಮಃ ।\\
೪ ಹ್ರೀಂಪದಪ್ರಿಯಾಯೈ ನಮಃ ।\\
೪ ಹ್ರೀಂಕಾರಬೀಜಾಯೈ ನಮಃ ।\\
೪ ಹ್ರೀಂಕಾರಮಂತ್ರಾಯೈ ನಮಃ ।\\
೪ ಹ್ರೀಂಕಾರಲಕ್ಷಣಾಯೈ ನಮಃ ।\\
೪ ಹ್ರೀಂಕಾರಜಪಸುಪ್ರೀತಾಯೈ ನಮಃ ।\\
೪ ಹ್ರೀಂಮತ್ಯೈ ನಮಃ ।\\
೪ ಹ್ರೀಂವಿಭೂಷಣಾಯೈ ನಮಃ ।\\
೪ ಹ್ರೀಂಶೀಲಾಯೈ ನಮಃ ।\\
೪ ಹ್ರೀಂಪದಾರಾಧ್ಯಾಯೈ ನಮಃ ।\\
೪ ಹ್ರೀಂಗರ್ಭಾಯೈ ನಮಃ ।\\
೪ ಹ್ರೀಂಪದಾಭಿಧಾಯೈ ನಮಃ ।\\
೪ ಹ್ರೀಂಕಾರವಾಚ್ಯಾಯೈ ನಮಃ ।\\
೪ ಹ್ರೀಂಕಾರಪೂಜ್ಯಾಯೈ ನಮಃ ।\\
೪ ಹ್ರೀಂಕಾರಪೀಠಿಕಾಯೈ ನಮಃ ।\\
೪ ಹ್ರೀಂಕಾರವೇದ್ಯಾಯೈ ನಮಃ ।\\
೪ ಹ್ರೀಂಕಾರಚಿಂತ್ಯಾಯೈ ನಮಃ ।\\
೪ ಹ್ರೀಂ ನಮಃ ।\\
೪ ಹ್ರೀಂಶರೀರಿಣ್ಯೈ ನಮಃ । 

\as{ವ॒ಯಂ ನಾಮ॒ ಪ್ರಬ್ರ॑ವಾಮಾ ಘೃ॒ತೇನಾ॒ಸ್ಮಿನ್ ಯ॒ಜ್ಞೇ ಧಾ॑ರಯಾಮಾ॒ ನಮೋ॑ಭಿಃ ।
ಉಪ॑ ಬ್ರ॒ಹ್ಮಾ ಶೃ॑ಣವ ಚ್ಛ॒ಸ್ಯಮಾ॑ನಂ॒ ಚತು॑ಶ್ಶೃಂಗೋಽವಮೀದ್ಗೌ॒ರ ಏ॒ತತ್ ॥}
\section{ಚತುರ್ದಶಾರಚಕ್ರಾಯ ನಮಃ ।\\ (ಇತಿ ವ್ಯಾಪಕಂ ನ್ಯಸ್ಯ)}
ಲಲಾಟಮಧ್ಯಭಾಗರೂಪ ಅಲಂಬುಸಾತ್ಮನೇ ಸರ್ವಸಂಕ್ಷೋಭಿಣೀಶಕ್ತ್ಯೈ ನಮಃ ।\\
ಲಲಾಟದಕ್ಷಭಾಗರೂಪ ಕುಹ್ವಾತ್ಮನೇ ಸರ್ವವಿದ್ರಾವಿಣೀಶಕ್ತ್ಯೈ ನಮಃ ।\\
ದಕ್ಷಗಂಡರೂಪ ವಿಶ್ವೋದರಾತ್ಮನೇ ಸರ್ವಾಕರ್ಷಿಣೀಶಕ್ತ್ಯೈ ನಮಃ ।\\
ದಕ್ಷಾಂಸರೂಪ ವಾರಣಾತ್ಮನೇ ಸರ್ವಾಹ್ಲಾದಿನೀಶಕ್ತ್ಯೈ ನಮಃ ।\\
ದಕ್ಷಪಾರ್ಶ್ವರೂಪ ಹಸ್ತಿಜಿಹ್ವಾತ್ಮನೇ ಸರ್ವಸಮ್ಮೋಹಿನೀಶಕ್ತ್ಯೈ ನಮಃ ।\\
ದಕ್ಷೋರುರೂಪ ಯಶೋವತ್ಯಾತ್ಮನೇ ಸರ್ವಸ್ತಂಭಿನೀಶಕ್ತ್ಯೈ ನಮಃ ।\\
ದಕ್ಷಜಂಘಾರೂಪ ಪಯಸ್ವಿನ್ಯಾತ್ಮನೇ ಸರ್ವಜೃಂಭಿಣೀಶಕ್ತ್ಯೈ ನಮಃ ।\\
ವಾಮಜಂಘಾರೂಪ ಗಾಂಧಾರ್ಯತ್ಮನೇ ಸರ್ವವಶಂಕರೀಶಕ್ತ್ಯೈ ನಮಃ ।\\
ವಾಮೋರುರೂಪ ಪೂಷಾತ್ಮನೇ ಸರ್ವರಂಜನೀಶಕ್ತ್ಯೈ ನಮಃ ।\\
ವಾಮಪಾರ್ಶ್ವರೂಪ ಶಂಖಿನ್ಯಾತ್ಮನೇ ಸರ್ವೋನ್ಮಾದಿನೀಶಕ್ತ್ಯೈ ನಮಃ ।\\
ವಾಮಾಂಸರೂಪ ಸರಸ್ವತ್ಯಾತ್ಮನೇ ಸರ್ವಾರ್ಥಸಾಧಿನೀಶಕ್ತ್ಯೈ ನಮಃ ।\\
ವಾಮಗಂಡರೂಪ ಇಡಾತ್ಮನೇ ಸರ್ವಸಂಪತ್ತಿಪೂರಣೀಶಕ್ತ್ಯೈ ನಮಃ ।\\
ಲಲಾಟವಾಮಭಾಗರೂಪ ಪಿಂಗಲಾತ್ಮನೇ ಸರ್ವಮಂತ್ರಮಯೀಶಕ್ತ್ಯೈ ನಮಃ ।\\
ಶಿರಃಪೃಷ್ಠಭಾಗರೂಪ ಸುಷುಮ್ನಾತ್ಮನೇ ಸರ್ವದ್ವಂದ್ವಕ್ಷಯಂಕರೀಶಕ್ತ್ಯೈ ನಮಃ ।\\
ಹೃದ್ರೂಪ ಸರ್ವಸೌಭಾಗ್ಯದಾಯಕಚಕ್ರೇಶ್ವರ್ಯೈ ತ್ರಿಪುರವಾಸಿನ್ಯೈ ನಮಃ ।\\
ಸಂಪ್ರದಾಯ ಯೋಗಿನೀರೂಪ ಸ್ವಾತ್ಮಾತ್ಮನೇ ಈಶಿತ್ವಸಿದ್ಧ್ಯೈ ನಮಃ ।\\
ಅಪರಿಚ್ಛಿನ್ನರೂಪ ಸ್ವಾತ್ಮಾತ್ಮನೇ ಸರ್ವವಶಂಕರೀಮುದ್ರಾಯೈ ನಮಃ ।

\as{(ಸಹಸ್ರನಾಮ ೫೦೧ - ೬೦೦)\\}
 ಗುಡಾನ್ನ - ಪ್ರೀತ - ಮಾನಸಾ~।\\
ಸಮಸ್ತಭಕ್ತ - ಸುಖದಾ ಲಾಕಿನ್ಯಂಬಾ - ಸ್ವರೂಪಿಣೀ ॥೧೦೩॥

ಸ್ವಾಧಿಷ್ಠಾನಾಂಬುಜ - ಗತಾ ಚತುರ್ವಕ್ತ್ರ - ಮನೋಹರಾ~।\\
ಶೂಲಾದ್ಯಾಯುಧ - ಸಂಪನ್ನಾ ಪೀತವರ್ಣಾಽತಿಗರ್ವಿತಾ ॥೧೦೪॥

ಮೇದೋನಿಷ್ಠಾ ಮಧುಪ್ರೀತಾ ಬಂದಿನ್ಯಾದಿ - ಸಮನ್ವಿತಾ~।\\
ದಧ್ಯನ್ನಾಸಕ್ತ - ಹೃದಯಾ ಕಾಕಿನೀ - ರೂಪ - ಧಾರಿಣೀ ॥೧೦೫॥

ಮೂಲಾಧಾರಾಂಬುಜಾರೂಢಾ ಪಂಚ - ವಕ್ತ್ರಾಽಸ್ಥಿ - ಸಂಸ್ಥಿತಾ~।\\
ಅಂಕುಶಾದಿ - ಪ್ರಹರಣಾ ವರದಾದಿ - ನಿಷೇವಿತಾ ॥೧೦೬॥

ಮುದ್ಗೌದನಾಸಕ್ತ - ಚಿತ್ತಾ ಸಾಕಿನ್ಯಂಬಾ - ಸ್ವರೂಪಿಣೀ~।\\
ಆಜ್ಞಾ - ಚಕ್ರಾಬ್ಜ - ನಿಲಯಾ ಶುಕ್ಲವರ್ಣಾ ಷಡಾನನಾ ॥೧೦೭॥

ಮಜ್ಜಾಸಂಸ್ಥಾ ಹಂಸವತೀ - ಮುಖ್ಯ - ಶಕ್ತಿ - ಸಮನ್ವಿತಾ~।\\
ಹರಿದ್ರಾನ್ನೈಕ - ರಸಿಕಾ ಹಾಕಿನೀ - ರೂಪ - ಧಾರಿಣೀ ॥೧೦೮॥

ಸಹಸ್ರದಲ - ಪದ್ಮಸ್ಥಾ ಸರ್ವ - ವರ್ಣೋಪ - ಶೋಭಿತಾ~।\\
ಸರ್ವಾಯುಧಧರಾ ಶುಕ್ಲ - ಸಂಸ್ಥಿತಾ ಸರ್ವತೋಮುಖೀ ॥೧೦೯॥

ಸರ್ವೌದನ - ಪ್ರೀತಚಿತ್ತಾ ಯಾಕಿನ್ಯಂಬಾ - ಸ್ವರೂಪಿಣೀ~।\\
ಸ್ವಾಹಾ ಸ್ವಧಾಽಮತಿರ್ಮೇಧಾ ಶ್ರುತಿಃ ಸ್ಮೃತಿರನುತ್ತಮಾ ॥೧೧೦॥

ಪುಣ್ಯಕೀರ್ತಿಃ ಪುಣ್ಯಲಭ್ಯಾ ಪುಣ್ಯಶ್ರವಣ - ಕೀರ್ತನಾ~।\\
ಪುಲೋಮಜಾರ್ಚಿತಾ ಬಂಧ - ಮೋಚನೀ ಬಂಧುರಾಲಕಾ ॥೧೧೧॥

ವಿಮರ್ಶರೂಪಿಣೀ ವಿದ್ಯಾ ವಿಯದಾದಿ - ಜಗತ್ಪ್ರಸೂಃ~।\\
ಸರ್ವವ್ಯಾಧಿ - ಪ್ರಶಮನೀ ಸರ್ವಮೃತ್ಯು - ನಿವಾರಿಣೀ ॥೧೧೨॥

ಅಗ್ರಗಣ್ಯಾಽಚಿಂತ್ಯರೂಪಾ ಕಲಿಕಲ್ಮಷ - ನಾಶಿನೀ~।\\
ಕಾತ್ಯಾಯನೀ ಕಾಲಹಂತ್ರೀ ಕಮಲಾಕ್ಷ - ನಿಷೇವಿತಾ ॥೧೧೩॥

ತಾಂಬೂಲ - ಪೂರಿತ - ಮುಖೀ ದಾಡಿಮೀ - ಕುಸುಮ - ಪ್ರಭಾ~।\\
ಮೃಗಾಕ್ಷೀ ಮೋಹಿನೀ ಮುಖ್ಯಾ ಮೃಡಾನೀ ಮಿತ್ರರೂಪಿಣೀ ॥೧೧೪॥

ನಿತ್ಯತೃಪ್ತಾ ಭಕ್ತನಿಧಿರ್ನಿಯಂತ್ರೀ ನಿಖಿಲೇಶ್ವರೀ~।\\
ಮೈತ್ರ್ಯಾದಿ - ವಾಸನಾಲಭ್ಯಾ ಮಹಾಪ್ರಲಯ - ಸಾಕ್ಷಿಣೀ ॥೧೧೫॥

ಪರಾ ಶಕ್ತಿಃ ಪರಾ ನಿಷ್ಠಾ ಪ್ರಜ್ಞಾನಘನ - ರೂಪಿಣೀ~।\\
ಮಾಧ್ವೀಪಾನಾಲಸಾ ಮತ್ತಾ ಮಾತೃಕಾ - ವರ್ಣ - ರೂಪಿಣೀ ॥೧೧೬॥

ಮಹಾಕೈಲಾಸ - ನಿಲಯಾ ಮೃಣಾಲ - ಮೃದು - ದೋರ್ಲತಾ~।\\
ಮಹನೀಯಾ ದಯಾಮೂರ್ತಿರ್ಮಹಾಸಾಮ್ರಾಜ್ಯ - ಶಾಲಿನೀ ॥೧೧೭॥

ಆತ್ಮವಿದ್ಯಾ ಮಹಾವಿದ್ಯಾ ಶ್ರೀವಿದ್ಯಾ ಕಾಮಸೇವಿತಾ~।\\
ಶ್ರೀ - ಷೋಡಶಾಕ್ಷರೀ - ವಿದ್ಯಾ ತ್ರಿಕೂಟಾ ಕಾಮಕೋಟಿಕಾ ॥೧೧೮॥

ಕಟಾಕ್ಷ - ಕಿಂಕರೀ - ಭೂತ - ಕಮಲಾ - ಕೋಟಿ - ಸೇವಿತಾ~।\\
ಶಿರಃಸ್ಥಿತಾ ಚಂದ್ರನಿಭಾ ಭಾಲಸ್ಥೇಂದ್ರ - ಧನುಃಪ್ರಭಾ ॥೧೧೯॥

ಹೃದಯಸ್ಥಾ ರವಿಪ್ರಖ್ಯಾ ತ್ರಿಕೋಣಾಂತರ - ದೀಪಿಕಾ~।\\
ದಾಕ್ಷಾಯಣೀ ದೈತ್ಯಹಂತ್ರೀ ದಕ್ಷಯಜ್ಞ - ವಿನಾಶಿನೀ \as{(೬೦೦)} ॥೧೨೦॥

\as{(ಅಷ್ಟೋತ್ತರ ೫೧ - ೬೦)\\}
ಅಂತರ್ಮುಖಜನಾನಂದಫಲದಾಯೈ ನಮೋ ನಮಃ ।\\
ಪತಿವ್ರತಾಂಗನಾಭೀಷ್ಟಫಲದಾಯೈ ನಮೋ ನಮಃ ।\\
ಅವ್ಯಾಜಕರುಣಾಪೂರಪೂರಿತಾಯೈ ನಮೋ ನಮಃ ।\\
ನಿತಾಂತಸಚ್ಚಿದಾನಂದಸಂಯುಕ್ತಾಯೈ ನಮೋ ನಮಃ ।\\
ಸಹಸ್ರಸೂರ್ಯಸಂಯುಕ್ತಪ್ರಕಾಶಾಯೈ ನಮೋ ನಮಃ ।\\
ರತ್ನಚಿಂತಾಮಣಿಗೃಹಮಧ್ಯಸ್ಥಾಯೈ ನಮೋ ನಮಃ ।\\
ಹಾನಿವೃದ್ಧಿಗುಣಾಧಿಕ್ಯರಹಿತಾಯೈ ನಮೋ ನಮಃ ।\\
ಮಹಾಪದ್ಮಾಟವೀಮಧ್ಯನಿವಾಸಾಯೈ ನಮೋ ನಮಃ ।\\
ಜಾಗ್ರತ್ಸ್ವಪ್ನಸುಷುಪ್ತೀನಾಂ ಸಾಕ್ಷಿಭೂತ್ಯೈ ನಮೋ ನಮಃ ।\\
ಮಹಾಪಾಪೌಘಪಾಪಾನಾಂ ವಿನಾಶಿನ್ಯೈ ನಮೋ ನಮಃ ।

\as{ಚ॒ತ್ವಾರಿ॒ ಶೃಂಗಾ॒ ತ್ರಯೋ॑ ಅಸ್ಯ॒ ಪಾದಾ॒ ದ್ವೇ ಶೀ॒ರ್ಷೇ ಸ॒ಪ್ತ ಹಸ್ತಾ॑ಸೋ ಅ॒ಸ್ಯ ।
ತ್ರಿಧಾ॑ಬ॒ದ್ಧೋ ವೃ॑ಷ॒ಭೋ ರೋ॑ರವೀತಿ ಮ॒ಹೋ ದೇ॒ವೋ ಮರ್ತ್ಯಾ॒ँ ಆವಿ॑ವೇಶ ॥}
\section{ಬಹಿರ್ದಶಾರಚಕ್ರಾಯ ನಮಃ ।\\ (ಇತಿ ವ್ಯಾಪಕಂ ನ್ಯಸ್ಯ)}
ದಕ್ಷಾಕ್ಷಿರೂಪ ಪ್ರಾಣಾತ್ಮನೇ ಸರ್ವಸಿದ್ಧಿಪ್ರದಾದೇವ್ಯೈ ನಮಃ ।\\
ನಾಸಾಮೂಲರೂಪ ಅಪಾನಾತ್ಮನೇ ಸರ್ವಸಂಪತ್ಪ್ರದಾದೇವ್ಯೈ ನಮಃ ।\\
ವಾಮನೇತ್ರರೂಪ ವ್ಯಾನಾತ್ಮನೇ ಸರ್ವಪ್ರಿಯಂಕರೀದೇವ್ಯೈ ನಮಃ ।\\
ಕುಕ್ಷೀಶಕೋಣರೂಪ ಉದಾನಾತ್ಮನೇ ಸರ್ವಮಂಗಳಕಾರಿಣೀದೇವ್ಯೈ ನಮಃ ।\\
ಕುಕ್ಷಿವಾಯುರೂಪ ಸಮಾನಾತ್ಮನೇ ಸರ್ವಕಾಮಪ್ರದಾದೇವ್ಯೈ ನಮಃ ।\\
ವಾಮಜಾನುರೂಪ ನಾಗಾತ್ಮನೇ ಸರ್ವದುಃಖವಿಮೋಚನೀದೇವ್ಯೈ ನಮಃ ।\\
ಗುದರೂಪ ಕೂರ್ಮಾತ್ಮನೇ ಸರ್ವಮೃತ್ಯುಪ್ರಶಮನೀದೇವ್ಯೈ ನಮಃ ।\\
ದಕ್ಷಜಾನುರೂಪ ಕೃಕರಾತ್ಮನೇ ಸರ್ವವಿಘ್ನನಿವಾರಿಣೀದೇವ್ಯೈ ನಮಃ ।\\
ಕುಕ್ಷಿನಿರ್ಋತಿಕೋಣರೂಪ ದೇವದತ್ತಾತ್ಮನೇ ಸರ್ವಾಂಗಸುಂದರೀದೇವ್ಯೈ ನಮಃ ।\\
ಕುಕ್ಷುವಹ್ನಿಕೋಣರೂಪ ಧನಂಜಯಾತ್ಮನೇ ಸರ್ವಸೌಭಾಗ್ಯದಾಯಿನೀದೇವ್ಯೈ ನಮಃ ।\\
ಹೃದ್ರೂಪ ಸರ್ವಾರ್ಥಸಾಧಕಚಕ್ರೇಶ್ವರ್ಯೈ ತ್ರಿಪುರಾಶ್ರಿಯೈ ನಮಃ ।\\
ಕುಲೋತ್ತೀರ್ಣಯೋಗಿನೀರೂಪ ಸ್ವಾತ್ಮಾತ್ಮನೇ ವಶಿತ್ವಸಿದ್ಧ್ಯೈ ನಮಃ ।\\
ಅಪರಿಚ್ಛಿನ್ನ ಸ್ವಾತ್ಮಾತ್ಮನೇ ಸರ್ವೋನ್ಮಾದಿನೀ ಮುದ್ರಾಯೈ ನಮಃ ।

\as{(ಸಹಸ್ರನಾಮ ೬೦೧ - ೭೦೦)\\}
ದರಾಂದೋಲಿತ - ದೀರ್ಘಾಕ್ಷೀ ದರ - ಹಾಸೋಜ್ಜ್ವಲನ್ಮುಖೀ~।\\
ಗುರುಮೂರ್ತಿರ್ಗುಣನಿಧಿರ್ಗೋಮಾತಾ ಗುಹಜನ್ಮಭೂಃ ॥೧೨೧॥

ದೇವೇಶೀ ದಂಡನೀತಿಸ್ಥಾ ದಹರಾಕಾಶ - ರೂಪಿಣೀ~।\\
ಪ್ರತಿಪನ್ಮುಖ್ಯ - ರಾಕಾಂತ - ತಿಥಿ - ಮಂಡಲ - ಪೂಜಿತಾ ॥೧೨೨॥

ಕಲಾತ್ಮಿಕಾ ಕಲಾನಾಥಾ ಕಾವ್ಯಾಲಾಪ - ವಿನೋದಿನೀ~।\\
ಸಚಾಮರ - ರಮಾ - ವಾಣೀ - ಸವ್ಯ - ದಕ್ಷಿಣ - ಸೇವಿತಾ ॥೧೨೩॥

ಆದಿಶಕ್ತಿರಮೇಯಾಽಽತ್ಮಾ ಪರಮಾ ಪಾವನಾಕೃತಿಃ~।\\
ಅನೇಕಕೋಟಿ - ಬ್ರಹ್ಮಾಂಡ - ಜನನೀ ದಿವ್ಯವಿಗ್ರಹಾ ॥೧೨೪॥

ಕ್ಲೀಂಕಾರೀ ಕೇವಲಾ ಗುಹ್ಯಾ ಕೈವಲ್ಯ - ಪದದಾಯಿನೀ~।\\
ತ್ರಿಪುರಾ ತ್ರಿಜಗದ್ವಂದ್ಯಾ ತ್ರಿಮೂರ್ತಿಸ್ತ್ರಿದಶೇಶ್ವರೀ ॥೧೨೫॥

ತ್ರ್ಯಕ್ಷರೀ ದಿವ್ಯ - ಗಂಧಾಢ್ಯಾ ಸಿಂದೂರ - ತಿಲಕಾಂಚಿತಾ~।\\
ಉಮಾ ಶೈಲೇಂದ್ರತನಯಾ ಗೌರೀ ಗಂಧರ್ವ - ಸೇವಿತಾ ॥೧೨೬॥

ವಿಶ್ವಗರ್ಭಾ ಸ್ವರ್ಣಗರ್ಭಾ ವರದಾ ವಾಗಧೀಶ್ವರೀ~।\\
ಧ್ಯಾನಗಮ್ಯಾಽಪರಿಚ್ಛೇದ್ಯಾ ಜ್ಞಾನದಾ ಜ್ಞಾನವಿಗ್ರಹಾ ॥೧೨೭॥

ಸರ್ವವೇದಾಂತ - ಸಂವೇದ್ಯಾ ಸತ್ಯಾನಂದ - ಸ್ವರೂಪಿಣೀ~।\\
ಲೋಪಾಮುದ್ರಾರ್ಚಿತಾ ಲೀಲಾ - ಕ್ಲೃಪ್ತ - ಬ್ರಹ್ಮಾಂಡ - ಮಂಡಲಾ ॥೧೨೮॥

ಅದೃಶ್ಯಾ ದೃಶ್ಯರಹಿತಾ ವಿಜ್ಞಾತ್ರೀ ವೇದ್ಯವರ್ಜಿತಾ~।\\
ಯೋಗಿನೀ ಯೋಗದಾ ಯೋಗ್ಯಾ ಯೋಗಾನಂದಾ ಯುಗಂಧರಾ ॥೧೨೯॥

ಇಚ್ಛಾಶಕ್ತಿ - ಜ್ಞಾನಶಕ್ತಿ - ಕ್ರಿಯಾಶಕ್ತಿ - ಸ್ವರೂಪಿಣೀ~।\\
ಸರ್ವಾಧಾರಾ ಸುಪ್ರತಿಷ್ಠಾ ಸದಸದ್ರೂಪ - ಧಾರಿಣೀ ॥೧೩೦॥

ಅಷ್ಟಮೂರ್ತಿರಜಾಜೈತ್ರೀ ಲೋಕಯಾತ್ರಾ - ವಿಧಾಯಿನೀ~।\\
ಏಕಾಕಿನೀ ಭೂಮರೂಪಾ ನಿರ್ದ್ವೈತಾ ದ್ವೈತವರ್ಜಿತಾ ॥೧೩೧॥

ಅನ್ನದಾ ವಸುದಾ ವೃದ್ಧಾ ಬ್ರಹ್ಮಾತ್ಮೈಕ್ಯ - ಸ್ವರೂಪಿಣೀ~।\\
ಬೃಹತೀ ಬ್ರಾಹ್ಮಣೀ ಬ್ರಾಹ್ಮೀ ಬ್ರಹ್ಮಾನಂದಾ ಬಲಿಪ್ರಿಯಾ ॥೧೩೨॥

ಭಾಷಾರೂಪಾ ಬೃಹತ್ಸೇನಾ ಭಾವಾಭಾವ - ವಿವರ್ಜಿತಾ~।\\
ಸುಖಾರಾಧ್ಯಾ ಶುಭಕರೀ ಶೋಭನಾ ಸುಲಭಾ ಗತಿಃ ॥೧೩೩॥

ರಾಜ - ರಾಜೇಶ್ವರೀ ರಾಜ್ಯ - ದಾಯಿನೀ ರಾಜ್ಯ - ವಲ್ಲಭಾ~।\\
ರಾಜತ್ಕೃಪಾ ರಾಜಪೀಠ - ನಿವೇಶಿತ - ನಿಜಾಶ್ರಿತಾ ॥೧೩೪॥

ರಾಜ್ಯಲಕ್ಷ್ಮೀಃ ಕೋಶನಾಥಾ ಚತುರಂಗ - ಬಲೇಶ್ವರೀ~।\\
ಸಾಮ್ರಾಜ್ಯ - ದಾಯಿನೀ ಸತ್ಯಸಂಧಾ ಸಾಗರಮೇಖಲಾ ॥೧೩೫॥

ದೀಕ್ಷಿತಾ ದೈತ್ಯಶಮನೀ ಸರ್ವಲೋಕ - ವಶಂಕರೀ~।\\
ಸರ್ವಾರ್ಥದಾತ್ರೀ ಸಾವಿತ್ರೀ ಸಚ್ಚಿದಾನಂದ - ರೂಪಿಣೀ \as{(೭೦೦)} ॥೧೩೬॥

\as{(ಅಷ್ಟೋತ್ತರ ೬೧ - ೭೦)\\}
ದುಷ್ಟಭೀತಿಮಹಾಭೀತಿಭಂಜನಾಯೈ ನಮೋ ನಮಃ ।\\
ಸಮಸ್ತದೇವದನುಜಪ್ರೇರಕಾಯೈ ನಮೋ ನಮಃ ।\\
ಸಮಸ್ತಹೃದಯಾಂಭೋಜನಿಲಯಾಯೈ ನಮೋ ನಮಃ ।\\
ಅನಾಹತಮಹಾಪದ್ಮಮಂದಿರಾಯೈ ನಮೋ ನಮಃ ।\\
ಸಹಸ್ರಾರಸರೋಜಾತವಾಸಿತಾಯೈ ನಮೋ ನಮಃ ।\\
ಪುನರಾವೃತಿರಹಿತಪುರಸ್ಥಾಯೈ ನಮೋ ನಮಃ ।\\
ವಾಣೀಗಾಯತ್ರೀಸಾವಿತ್ರೀಸನ್ನುತಾಯೈ ನಮೋ ನಮಃ ।\\
ರಮಾಭೂಮಿಸುತಾರಾಧ್ಯಪದಾಬ್ಜಾಯೈ ನಮೋ ನಮಃ ।\\
ಲೋಪಾಮುದ್ರಾರ್ಚಿತಶ್ರೀಮಚ್ಚರಣಾಯೈ ನಮೋ ನಮಃ ।\\
ಸಹಸ್ರರತಿಸೌಂದರ್ಯಶರೀರಾಯೈ ನಮೋ ನಮಃ ।

\as{ತ್ರಿಧಾ॑ಹಿ॒ತಂ ಪ॒ಣಿಭಿ॑ರ್ಗು॒ಹ್ಯಮಾ॑ನಂ॒ ಗವಿ॑ ದೇ॒ವಾಸೋ॑ ಘೃ॒ತಮನ್ವ॑ವಿಂದನ್ ।
ಇಂದ್ರ॒ ಏಕ॒ಁ ಸೂರ್ಯ॒ ಏಕಂ॑ ಜಜಾನ ವೇ॒ನಾದೇಕಁ॑ ಸ್ವ॒ಧಯಾ॒ ನಿಷ್ಟ॑ತಕ್ಷುಃ ॥}
\section{ಅಂತರ್ದಶಾರಚಕ್ರಾಯ ನಮಃ ।\\ (ಇತಿ ವ್ಯಾಪಕಂ ನ್ಯಸ್ಯ)}
ದಕ್ಷನಾಸಾರೂಪ ರೇಚಕಾಗ್ನ್ಯಾತ್ಮನೇ ಸರ್ವಜ್ಞಾದೇವ್ಯೈ ನಮಃ ।\\
ದಕ್ಷಸೃಗ್ರೂಪ ಪಾಚಕಾಗ್ನ್ಯಾತ್ಮನೇ ಸರ್ವಶಕ್ತಿದೇವ್ಯೈ ನಮಃ ।\\
ದಕ್ಷಸ್ತನರೂಪ ಶೋಷಕಾಗ್ನ್ಯಾತ್ಮನೇ ಸರ್ವೈಶ್ವರ್ಯಪ್ರದಾದೇವ್ಯೈ ನಮಃ ।\\
ದಕ್ಷವೃಷಣರೂಪ ದಾಹಕಾಗ್ನ್ಯಾತ್ಮನೇ ಸರ್ವಜ್ಞಾನಮಯೀದೇವ್ಯೈ ನಮಃ ।\\
ಸೀವಿನೀರೂಪ ಪ್ಲಾವಕಾಗ್ನ್ಯಾತ್ಮನೇ ಸರ್ವವ್ಯಾಧಿವಿನಾಶಿನೀದೇವ್ಯೈ ನಮಃ ।\\
ವಾಮವೃಷಣರೂಪ ಕ್ಷಾರಕಾಗ್ನ್ಯಾತ್ಮನೇ ಸರ್ವಾಧಾರಸ್ವರೂಪಾದೇವ್ಯೈ ನಮಃ ।\\
ವಾಮಸ್ತನರೂಪ ಉದ್ಗಾರಕಾಗ್ನ್ಯಾತ್ಮನೇ ಸರ್ವಪಾಪಹರಾದೇವ್ಯೈ ನಮಃ ।\\
ವಾಮಸೃಗ್ರೂಪ ಕ್ಷೋಭಕಾಗ್ನ್ಯಾತ್ಮನೇ ಸರ್ವಾನಂದಮಯೀದೇವ್ಯೈ ನಮಃ ।\\
ವಾಮನಾಸಾರೂಪ ಜೃಂಭಕಾಗ್ನ್ಯಾತ್ಮನೇ ಸರ್ವರಕ್ಷಾಸ್ವರೂಪಿಣೀದೇವ್ಯೈ ನಮಃ ।\\
ನಾಸಾಗ್ರರೂಪ ಮೋಹಕಾಗ್ನ್ಯಾತ್ಮನೇ ಸರ್ವೇಪ್ಸಿತಫಲಪ್ರದಾದೇವ್ಯೈ ನಮಃ ।\\
ಹೃದ್ರೂಪ ಸರ್ವರಕ್ಷಾಕರಚಕ್ರೇಶ್ವರ್ಯೈ ತ್ರಿಪುರಮಾಲಿನ್ಯೈ ನಮಃ ।\\
ನಿಗರ್ಭಯೋಗಿನೀರೂಪ ಸ್ವಾತ್ಮಾತ್ಮನೇ ಪ್ರಾಕಾಮ್ಯಸಿದ್ಧ್ಯೈ ನಮಃ ।\\
ಅಪರಿಚ್ಛಿನ್ನರೂಪ ಸ್ವಾತ್ಮಾತ್ಮನೇ ಸರ್ವಮಹಾಂಕುಶಾಮುದ್ರಾಯೈ ನಮಃ ।

\as{(ಸಹಸ್ರನಾಮ ೭೦೧ - ೮೦೦)\\}
ದೇಶ - ಕಾಲಾಪರಿಚ್ಛಿನ್ನಾ ಸರ್ವಗಾ ಸರ್ವಮೋಹಿನೀ~।\\
ಸರಸ್ವತೀ ಶಾಸ್ತ್ರಮಯೀ ಗುಹಾಂಬಾ ಗುಹ್ಯರೂಪಿಣೀ ॥೧೩೭॥

ಸರ್ವೋಪಾಧಿ - ವಿನಿರ್ಮುಕ್ತಾ ಸದಾಶಿವ - ಪತಿವ್ರತಾ~।\\
ಸಂಪ್ರದಾಯೇಶ್ವರೀ ಸಾಧ್ವೀ ಗುರುಮಂಡಲ - ರೂಪಿಣೀ ॥೧೩೮॥

ಕುಲೋತ್ತೀರ್ಣಾ ಭಗಾರಾಧ್ಯಾ ಮಾಯಾ ಮಧುಮತೀ ಮಹೀ~।\\
ಗಣಾಂಬಾ ಗುಹ್ಯಕಾರಾಧ್ಯಾ ಕೋಮಲಾಂಗೀ ಗುರುಪ್ರಿಯಾ ॥೧೩೯॥

ಸ್ವತಂತ್ರಾ ಸರ್ವತಂತ್ರೇಶೀ ದಕ್ಷಿಣಾಮೂರ್ತಿ - ರೂಪಿಣೀ~।\\
ಸನಕಾದಿ - ಸಮಾರಾಧ್ಯಾ ಶಿವಜ್ಞಾನ - ಪ್ರದಾಯಿನೀ ॥೧೪೦॥

ಚಿತ್ಕಲಾಽಽನಂದ - ಕಲಿಕಾ ಪ್ರೇಮರೂಪಾ ಪ್ರಿಯಂಕರೀ~।\\
ನಾಮಪಾರಾಯಣ - ಪ್ರೀತಾ ನಂದಿವಿದ್ಯಾ ನಟೇಶ್ವರೀ ॥೧೪೧॥

ಮಿಥ್ಯಾ - ಜಗದಧಿಷ್ಠಾನಾ ಮುಕ್ತಿದಾ ಮುಕ್ತಿರೂಪಿಣೀ~।\\
ಲಾಸ್ಯಪ್ರಿಯಾ ಲಯಕರೀ ಲಜ್ಜಾ ರಂಭಾದಿವಂದಿತಾ ॥೧೪೨॥

ಭವದಾವ - ಸುಧಾವೃಷ್ಟಿಃ ಪಾಪಾರಣ್ಯ - ದವಾನಲಾ~।\\
ದೌರ್ಭಾಗ್ಯ - ತೂಲವಾತೂಲಾ ಜರಾಧ್ವಾಂತ - ರವಿಪ್ರಭಾ ॥೧೪೩॥

ಭಾಗ್ಯಾಬ್ಧಿ - ಚಂದ್ರಿಕಾ ಭಕ್ತ - ಚಿತ್ತಕೇಕಿ - ಘನಾಘನಾ~।\\
ರೋಗಪರ್ವತ - ದಂಭೋಲಿರ್ಮೃತ್ಯುದಾರು - ಕುಠಾರಿಕಾ ॥೧೪೪॥

ಮಹೇಶ್ವರೀ ಮಹಾಕಾಲೀ ಮಹಾಗ್ರಾಸಾ ಮಹಾಶನಾ~।\\
ಅಪರ್ಣಾ ಚಂಡಿಕಾ ಚಂಡಮುಂಡಾಸುರ - ನಿಷೂದಿನೀ ॥೧೪೫॥

ಕ್ಷರಾಕ್ಷರಾತ್ಮಿಕಾ ಸರ್ವ - ಲೋಕೇಶೀ ವಿಶ್ವಧಾರಿಣೀ~।\\
ತ್ರಿವರ್ಗದಾತ್ರೀ ಸುಭಗಾ ತ್ರ್ಯಂಬಕಾ ತ್ರಿಗುಣಾತ್ಮಿಕಾ ॥೧೪೬॥

ಸ್ವರ್ಗಾಪವರ್ಗದಾ ಶುದ್ಧಾ ಜಪಾಪುಷ್ಪ - ನಿಭಾಕೃತಿಃ~।\\
ಓಜೋವತೀ ದ್ಯುತಿಧರಾ ಯಜ್ಞರೂಪಾ ಪ್ರಿಯವ್ರತಾ ॥೧೪೭॥

ದುರಾರಾಧ್ಯಾ ದುರಾಧರ್ಷಾ ಪಾಟಲೀ - ಕುಸುಮ - ಪ್ರಿಯಾ~।\\
ಮಹತೀ ಮೇರುನಿಲಯಾ ಮಂದಾರ - ಕುಸುಮ - ಪ್ರಿಯಾ ॥೧೪೮॥

ವೀರಾರಾಧ್ಯಾ ವಿರಾಡ್ರೂಪಾ ವಿರಜಾ ವಿಶ್ವತೋಮುಖೀ~।\\
ಪ್ರತ್ಯಗ್ರೂಪಾ ಪರಾಕಾಶಾ ಪ್ರಾಣದಾ ಪ್ರಾಣರೂಪಿಣೀ ॥೧೪೯॥

ಮಾರ್ತಾಂಡ - ಭೈರವಾರಾಧ್ಯಾ ಮಂತ್ರಿಣೀನ್ಯಸ್ತ - ರಾಜ್ಯಧೂಃ~।\\
ತ್ರಿಪುರೇಶೀ ಜಯತ್ಸೇನಾ ನಿಸ್ತ್ರೈಗುಣ್ಯಾ ಪರಾಪರಾ ॥೧೫೦॥

ಸತ್ಯ - ಜ್ಞಾನಾನಂದ - ರೂಪಾ ಸಾಮರಸ್ಯ - ಪರಾಯಣಾ~।\\
ಕಪರ್ದಿನೀ ಕಲಾಮಾಲಾ ಕಾಮಧುಕ್ ಕಾಮರೂಪಿಣೀ ॥೧೫೧॥

ಕಲಾನಿಧಿಃ ಕಾವ್ಯಕಲಾ ರಸಜ್ಞಾ ರಸಶೇವಧಿಃ \as{(೮೦೦)}~।

\as{ಅಷ್ಟೋತ್ತರ (೭೧ - ೮೦)\\}
ಭಾವನಾಮಾತ್ರಸಂತುಷ್ಟಹೃದಯಾಯೈ ನಮೋ ನಮಃ ।\\
ಸತ್ಯಸಂಪೂರ್ಣವಿಜ್ಞಾನಸಿದ್ಧಿದಾಯೈ ನಮೋ ನಮಃ ।\\
ಶ್ರೀಲೋಚನಕೃತೋಲ್ಲಾಸಫಲದಾಯೈ ನಮೋ ನಮಃ ।\\
ಶ್ರೀಸುಧಾಬ್ಧಿಮಣಿದ್ವೀಪಮಧ್ಯಗಾಯೈ ನಮೋ ನಮಃ ।\\
ದಕ್ಷಾಧ್ವರವಿನಿರ್ಭೇದಸಾಧನಾಯೈ ನಮೋ ನಮಃ ।\\
ಶ್ರೀನಾಥಸೋದರೀಭೂತಶೋಭಿತಾಯೈ ನಮೋ ನಮಃ ।\\
ಚಂದ್ರಶೇಖರಭಕ್ತಾರ್ತಿಭಂಜನಾಯೈ ನಮೋ ನಮಃ ।\\
ಸರ್ವೋಪಾಧಿವಿನಿರ್ಮುಕ್ತಚೈತನ್ಯಾಯೈ ನಮೋ ನಮಃ ।\\
ನಾಮಪಾರಯಣಾಭೀಷ್ಟಫಲದಾಯೈ ನಮೋ ನಮಃ ।\\
ಸೃಷ್ಟಿಸ್ಥಿತಿತಿರೋಧಾನಸಂಕಲ್ಪಾಯೈ ನಮೋ ನಮಃ ।

\as{ಆ॒ದಿ॒ತ್ಯವ॑ರ್ಣೇ॒ ತಪ॒ಸೋಽಧಿ॑ಜಾ॒ತೋ ವನ॒ಸ್ಪತಿ॒ಸ್ತವ॑ ವೃ॒ಕ್ಷೋಽಥ ಬಿ॒ಲ್ವಃ ।\\
ತಸ್ಯ॒ ಫಲಾ᳚ನಿ॒ ತಪ॒ಸಾ ನು॑ದಂತು ಮಾ॒ಯಾಂತ॑ರಾ॒ಯಾಶ್ಚ॑ ಬಾ॒ಹ್ಯಾ ಅ॑ಲ॒ಕ್ಷ್ಮೀಃ ॥ ೬॥}

೪ ಹಕಾರರೂಪಾಯೈ ನಮಃ ।\\
೪ ಹಲಧೃತ್ಪೂಜಿತಾಯೈ ನಮಃ ।\\
೪ ಹರಿಣೇಕ್ಷಣಾಯೈ ನಮಃ ।\\
೪ ಹರಪ್ರಿಯಾಯೈ ನಮಃ ।\\
೪ ಹರಾರಾಧ್ಯಾಯೈ ನಮಃ ।\\
೪ ಹರಿಬ್ರಹ್ಮೇಂದ್ರವಂದಿತಾಯೈ ನಮಃ ।\\
೪ ಹಯಾರೂಢಾಸೇವಿತಾಂಘ್ರ್ಯೈ ನಮಃ ।\\
೪ ಹಯಮೇಧಸಮರ್ಚಿತಾಯೈ ನಮಃ ।\\
೪ ಹರ್ಯಕ್ಷವಾಹನಾಯೈ ನಮಃ ।\\
೪ ಹಂಸವಾಹನಾಯೈ ನಮಃ ।\\
೪ ಹತದಾನವಾಯೈ ನಮಃ ।\\
೪ ಹತ್ತ್ಯಾದಿಪಾಪಶಮನ್ಯೈ ನಮಃ ।\\
೪ ಹರಿದಶ್ವಾದಿಸೇವಿತಾಯೈ ನಮಃ ।\\
೪ ಹಸ್ತಿಕುಂಭೋತ್ತುಂಗಕುಚಾಯೈ ನಮಃ ।\\
೪ ಹಸ್ತಿಕೃತಿಪ್ರಿಯಾಂಗನಾಯೈ ನಮಃ ।\\
೪ ಹರಿದ್ರಾಕುಂಕುಮಾದಿಗ್ಧಾಯೈ ನಮಃ ।\\
೪ ಹರ್ಯಶ್ವಾದ್ಯಮರಾರ್ಚಿತಾಯೈ ನಮಃ ।\\
೪ ಹರಿಕೇಶಸಖ್ಯೈ ನಮಃ ।\\
೪ ಹಾದಿವಿದ್ಯಾಯೈ ನಮಃ ।\\
೪ ಹಾಲಾಮದಾಲಸಾಯೈ ನಮಃ ।

\as{ಉಪೈ॑ತು॒ ಮಾಂ ದೇ॑ವಸ॒ಖಃ ಕೀ॒ರ್ತಿಶ್ಚ॒ ಮಣಿ॑ನಾ ಸ॒ಹ ।\\
ಪ್ರಾ॒ದು॒ರ್ಭೂ॒ತೋಽಸ್ಮಿ॑ ರಾಷ್ಟ್ರೇ॒ಽಸ್ಮಿನ್ ಕೀ॒ರ್ತಿಮೃ॑ದ್ಧಿಂ ದ॒ದಾತು॑ ಮೇ ॥ ೭॥}

೪ ಸಕಾರರೂಪಾಯೈ ನಮಃ ।\\
೪ ಸರ್ವಜ್ಞಾಯೈ ನಮಃ ।\\
೪ ಸರ್ವೇಶ್ಯೈ ನಮಃ ।\\
೪ ಸರ್ವಮಂಗಲಾಯೈ ನಮಃ ।\\
೪ ಸರ್ವಕರ್ತ್ರ್ಯೈ ನಮಃ ।\\
೪ ಸರ್ವಭರ್ತ್ರ್ಯೈ ನಮಃ ।\\
೪ ಸರ್ವಹಂತ್ರ್ಯೈ ನಮಃ ।\\
೪ ಸನಾತನ್ಯೈ ನಮಃ ।\\
೪ ಸರ್ವಾನವದ್ಯಾಯೈ ನಮಃ ।\\
೪ ಸರ್ವಾಂಗಸುಂದರ್ಯೈ ನಮಃ ।\\
೪ ಸರ್ವಸಾಕ್ಷಿಣ್ಯೈ ನಮಃ ।\\
೪ ಸರ್ವಾತ್ಮಿಕಾಯೈ ನಮಃ ।\\
೪ ಸರ್ವಸೌಖ್ಯದಾತ್ರ್ಯೈ ನಮಃ ।\\
೪ ಸರ್ವವಿಮೋಹಿನ್ಯೈ ನಮಃ ।\\
೪ ಸರ್ವಾಧಾರಾಯೈ ನಮಃ ।\\
೪ ಸರ್ವಗತಾಯೈ ನಮಃ ।\\
೪ ಸರ್ವಾವಗುಣವರ್ಜಿತಾಯೈ ನಮಃ ।\\
೪ ಸರ್ವಾರುಣಾಯೈ ನಮಃ ।\\
೪ ಸರ್ವಮಾತ್ರೇ ನಮಃ ।\\
೪ ಸರ್ವಭೂಷಣಭೂಷಿತಾಯೈ ನಮಃ ।

\as{ಕ್ಷುತ್ಪಿ॑ಪಾ॒ಸಾಮ॑ಲಾಂ ಜ್ಯೇ॒ಷ್ಠಾಮ॑ಲ॒ಕ್ಷ್ಮೀಂ ನಾ॑ಶಯಾ॒ಮ್ಯಹಂ ।
ಅಭೂ॑ತಿ॒ಮಸ॑ಮೃದ್ಧಿಂ॒ ಚ ಸರ್ವಾಂ॒ ನಿರ್ಣು॑ದ ಮೇ॒ ಗೃಹಾ᳚ತ್ ॥ ೮॥}

೪ ಕಕಾರಾರ್ಥಾಯೈ ನಮಃ ।\\
೪ ಕಾಲಹಂತ್ರ್ಯೈ ನಮಃ ।\\
೪ ಕಾಮೇಶ್ಯೈ ನಮಃ ।\\
೪ ಕಾಮಿತಾರ್ಥದಾಯೈ ನಮಃ ।\\
೪ ಕಾಮಸಂಜೀವಿನ್ಯೈ ನಮಃ ।\\
೪ ಕಲ್ಯಾಯೈ ನಮಃ ।\\
೪ ಕಠಿನಸ್ತನಮಂಡಲಾಯೈ ನಮಃ ।\\
೪ ಕರಭೋರವೇ ನಮಃ ।\\
೪ ಕಲಾನಾಥಮುಖ್ಯೈ ನಾಮ್ಃ\\
೪ ಕಚಜಿತಾಂಬುದಾಯೈ ನಮಃ ।\\
೪ ಕಟಾಕ್ಷಸ್ಯಂದಿಕರುಣಾಯೈ ನಮಃ ।\\
೪ ಕಪಾಲಿಪ್ರಾಣನಾಯಿಕಾಯೈ ನಮಃ ।\\
೪ ಕಾರುಣ್ಯವಿಗ್ರಹಾಯೈ ನಮಃ ।\\
೪ ಕಾಂತಾಯೈ ನಮಃ ।\\
೪ ಕಾಂತಿಧೂತಜಪಾವಲ್ಯೈ ನಮಃ ।\\
೪ ಕಲಾಲಾಪಾಯೈ ನಮಃ ।\\
೪ ಕಂಬುಕಂಠ್ಯೈ ನಮಃ ।\\
೪ ಕರನಿರ್ಜಿತಪಲ್ಲವಾಯೈ ನಮಃ ।\\
೪ ಕಲ್ಪವಲ್ಲೀಸಮಭುಜಾಯೈ ನಮಃ ।\\
೪ ಕಸ್ತೂರೀತಿಲಕಾಂಚಿತಾಯೈ ನಮಃ ।

\as{ಗಂ॒ಧ॒ದ್ವಾ॒ರಾಂ ದು॑ರಾಧ॒ರ್ಷಾಂ॒ ನಿ॒ತ್ಯಪು॑ಷ್ಟಾಂ ಕರೀ॒ಷಿಣೀಂ᳚ ।\\
ಈ॒ಶ್ವರೀ॑ँ ಸರ್ವ॑ಭೂತಾ॒ನಾಂ॒ ತಾಮಿ॒ಹೋಪ॑ಹ್ವಯೇ॒ ಶ್ರಿಯಂ ॥ ೯॥}

೪ ಹಕಾರಾರ್ಥಾಯೈ ನಮಃ ।\\
೪ ಹಂಸಗತ್ಯೈ ನಮಃ ।\\
೪ ಹಾಟಕಾಭರಣೋಜ್ಜ್ವಲಾಯೈ ನಮಃ ।\\
೪ ಹಾರಹಾರಿಕುಚಾಭೋಗಾಯೈ ನಮಃ ।\\
೪ ಹಾಕಿನ್ಯೈ ನಮಃ ।\\
೪ ಹಲ್ಯವರ್ಜಿತಾಯೈ ನಮಃ ।\\
೪ ಹರಿತ್ಪತಿಸಮಾರಾಧ್ಯಾಯೈ ನಮಃ ।\\
೪ ಹಠಾತ್ಕಾರಹತಾಸುರಾಯೈ ನಮಃ ।\\
೪ ಹರ್ಷಪ್ರದಾಯೈ ನಮಃ ।\\
೪ ಹವಿರ್ಭೋಕ್ತ್ರ್ಯೈ ನಮಃ ।\\
೪ ಹಾರ್ದಸಂತಮಸಾಪಹಾಯೈ ನಮಃ ।\\
೪ ಹಲ್ಲೀಸಲಾಸ್ಯಸಂತುಷ್ಟಾಯೈ ನಮಃ ।\\
೪ ಹಂಸಮಂತ್ರಾರ್ಥರೂಪಿಣ್ಯೈ ನಮಃ ।\\
೪ ಹಾನೋಪಾದಾನನಿರ್ಮುಕ್ತಾಯೈ ನಮಃ ।\\
೪ ಹರ್ಷಿಣ್ಯೈ ನಮಃ ।\\
೪ ಹರಿಸೋದರ್ಯೈ ನಮಃ ।\\
೪ ಹಾಹಾಹೂಹೂಮುಖಸ್ತುತ್ಯಾಯೈ ನಮಃ ।\\
೪ ಹಾನಿವೃದ್ಧಿವಿವರ್ಜಿತಾಯೈ ನಮಃ ।\\
೪ ಹಯ್ಯಂಗವೀನಹೃದಯಾಯೈ ನಮಃ ।\\
೪ ಹರಿಗೋಪಾರುಣಾಂಶುಕಾಯೈ ನಮಃ ।

\as{ಮನ॑ಸಃ॒ ಕಾಮ॒ಮಾಕೂ᳚ತಿಂ ವಾ॒ಚಃ ಸ॒ತ್ಯಮ॑ಶೀಮಹಿ ।\\
ಪ॒ಶೂ॒ನಾಂ ರೂ॒ಪಮನ್ನ॑ಸ್ಯ॒ ಮಯಿ॒ ಶ್ರೀಃ ಶ್ರ॑ಯತಾಂ॒ ಯಶಃ॑ ॥ ೧೦॥}

೪ ಲಕಾರಾಖ್ಯಾಯೈ ನಮಃ ।\\
೪ ಲತಾಪೂಜ್ಯಾಯೈ ನಮಃ ।\\
೪ ಲಯಸ್ಥಿತ್ಯುದ್ಭವೇಶ್ವರ್ಯೈ ನಮಃ ।\\
೪ ಲಾಸ್ಯದರ್ಶನಸಂತುಷ್ಟಾಯೈ ನಮಃ ।\\
೪ ಲಾಭಾಲಾಭವಿವರ್ಜಿತಾಯೈ ನಮಃ ।\\
೪ ಲಂಘ್ಯೇತರಾಜ್ಞಾಯೈ ನಮಃ ।\\
೪ ಲಾವಣ್ಯಶಾಲಿನ್ಯೈ ನಮಃ ।\\
೪ ಲಘುಸಿದ್ಧಿದಾಯೈ ನಮಃ ।\\
೪ ಲಾಕ್ಷಾರಸಸವರ್ಣಾಭಾಯೈ ನಮಃ ।\\
೪ ಲಕ್ಷ್ಮಣಾಗ್ರಜಪೂಜಿತಾಯೈ ನಮಃ ।\\
೪ ಲಭ್ಯೇತರಾಯೈ ನಮಃ ।\\
೪ ಲಬ್ಧಭಕ್ತಿಸುಲಭಾಯೈ ನಮಃ ।\\
೪ ಲಾಂಗಲಾಯುಧಾಯೈ ನಮಃ ।\\
೪ ಲಗ್ನಚಾಮರಹಸ್ತ ಶ್ರೀಶಾರದಾ ಪರಿವೀಜಿತಾಯೈ ನಮಃ ।\\
೪ ಲಜ್ಜಾಪದಸಮಾರಾಧ್ಯಾಯೈ ನಮಃ ।\\
೪ ಲಂಪಟಾಯೈ ನಮಃ ।\\
೪ ಲಕುಲೇಶ್ವರ್ಯೈ ನಮಃ ।\\
೪ ಲಬ್ಧಮಾನಾಯೈ ನಮಃ ।\\
೪ ಲಬ್ಧರಸಾಯೈ ನಮಃ ।\\
೪ ಲಬ್ಧಸಂಪತ್ಸಮುನ್ನತ್ಯೈ ನಮಃ ।

\as{ಕ॒ರ್ದಮೇ॑ನ ಪ್ರ॑ಜಾಭೂ॒ತಾ॒ ಮ॒ಯಿ॒ ಸಂಭ॑ವ ಕ॒ರ್ದಮ ।
ಶ್ರಿಯಂ॑ ವಾ॒ಸಯ॑ ಮೇ ಕು॒ಲೇ ಮಾ॒ತರಂ॑ ಪದ್ಮ॒ಮಾಲಿ॑ನೀಂ ॥ ೧೧॥}

೪ ಹ್ರೀಂಕಾರಿಣ್ಯೈ ನಮಃ ।\\
೪ ಹ್ರೀಂಕಾರಾದ್ಯಾಯೈ ನಮಃ ।\\
೪ ಹ್ರೀಂಮಧ್ಯಾಯೈ ನಮಃ ।\\
೪ ಹ್ರೀಂಶಿಖಾಮಣಯೇ ನಮಃ ।\\
೪ ಹ್ರೀಂಕಾರಕುಂಡಾಗ್ನಿಶಿಖಾಯೈ ನಮಃ ।\\
೪ ಹ್ರೀಂಕಾರಶಶಿಚಂದ್ರಿಕಾಯೈ ನಮಃ ।\\
೪ ಹ್ರೀಂಕಾರಭಾಸ್ಕರರುಚ್ಯೈ ನಮಃ ।\\
೪ ಹ್ರೀಂಕಾರಾಂಭೋದಚಂಚಲಾಯೈ ನಮಃ ।\\
೪ ಹ್ರೀಂಕಾರಕಂದಾಂಕುರಿಕಾಯೈ ನಮಃ ।\\
೪ ಹ್ರೀಂಕಾರೈಕಪರಾಯಣಾಯೈ ನಮಃ ।\\
೪ ಹ್ರೀಂಕಾರದೀರ್ಘಿಕಾಹಂಸ್ಯೈ ನಮಃ ।\\
೪ ಹ್ರೀಂಕಾರೋದ್ಯಾನಕೇಕಿನ್ಯೈ ನಮಃ ।\\
೪ ಹ್ರೀಂಕಾರಾರಣ್ಯಹರಿಣ್ಯೈ ನಮಃ ।\\
೪ ಹ್ರೀಂಕಾರಾವಾಲವಲ್ಲರ್ಯೈ ನಮಃ ।\\
೪ ಹ್ರೀಂಕಾರಪಂಜರಶುಕ್ಯೈ ನಮಃ ।\\
೪ ಹ್ರೀಂಕಾರಾಂಗಣದೀಪಿಕಾಯೈ ನಮಃ ।\\
೪ ಹ್ರೀಂಕಾರಕಂದರಾಸಿಂಹ್ಯೈ ನಮಃ ।\\
೪ ಹ್ರೀಂಕಾರಾಂಭೋಜಭೃಂಗಿಕಾಯೈ ನಮಃ ।\\
೪ ಹ್ರೀಂಕಾರಸುಮನೋಮಾಧ್ವ್ಯೈ ನಮಃ ।\\
೪ ಹ್ರೀಂಕಾರತರುಮಂಜರ್ಯೈ ನಮಃ ।

\as{ಯೋ ದೇ॒ವಾನಾಂ᳚ ಪ್ರಥ॒ಮಂ ಪು॒ರಸ್ತಾ॒ದ್ವಿಶ್ವಾ॒ಧಿಯೋ॑ ರು॒ದ್ರೋ ಮ॒ಹರ್ಷಿಃ॑ ।
ಹಿ॒ರ॒ಣ್ಯ॒ಗ॒ರ್ಭಂ ಪ॑ಶ್ಯತ॒ ಜಾಯ॑ಮಾನಁ॒ ಸನೋ॑ ದೇ॒ವಶ್ಶು॒ಭಯಾ॒ ಸ್ಮೃತ್ಯಾ॒ ಸಂಯು॑ನಕ್ತು ॥}
\section{ಅಷ್ಟಕೋಣಚಕ್ರಾಯ ನಮಃ ।\\ (ಇತಿ ವ್ಯಾಪಕಂ ನ್ಯಸ್ಯ)}
ಚುಬುಕದಕ್ಷಭಾಗರೂಪ ಶೀತಾತ್ಮನೇ ವಶಿನೀವಾಗ್ದೇವತಾಯೈ ನಮಃ ।\\
ಕಂಠದಕ್ಷಭಾಗರೂಪ ಉಷ್ಣಾತ್ಮನೇ ಕಾಮೇಶ್ವರೀವಾಗ್ದೇವತಾಯೈ ನಮಃ ।\\
ಹೃದಯದಕ್ಷಭಾಗರೂಪ ಸುಖಾತ್ಮನೇ ಮೋದಿನೀವಾಗ್ದೇವತಾಯೈ ನಮಃ ।\\
ನಾಭಿದಕ್ಷಭಾಗರೂಪ ದುಃಖಾತ್ಮನೇ ವಿಮಲಾವಾಗ್ದೇವತಾಯೈ ನಮಃ ।\\
ನಾಭಿವಾಮಭಾಗರೂಪ ಇಚ್ಛಾತ್ಮನೇ ಅರುಣಾವಾಗ್ದೇವತಾಯೈ ನಮಃ ।\\
ಹೃದಯವಾಮಭಾಗರೂಪ ಸತ್ವಗುಣಾತ್ಮನೇ ಜಯಿನೀವಾಗ್ದೇವತಾಯೈ ನಮಃ ।\\
ಕಂಠವಾಮಭಾಗರೂಪ ರಜೋಗುಣಾತ್ಮನೇ ಸರ್ವೇಶ್ವರೀವಾಗ್ದೇವತಾಯೈ ನಮಃ ।\\
ಚುಬುಕವಾಮಭಾಗರೂಪ ತಮೋಗುಣಾತ್ಮನೇ ಕೌಳಿನೀವಾಗ್ದೇವತಾಯೈ ನಮಃ ।\\
ಹೃದ್ರೂಪ ಸರ್ವರೋಗಹರ ಚಕ್ರೇಶ್ವರ್ಯೈ ತ್ರಿಪುರಾಸಿದ್ಧಾಯೈ ನಮಃ ।\\
ರಹಸ್ಯಯೋಗಿನೀ ರೂಪ ಸ್ವಾತ್ಮಾತ್ಮನೇ ಭುಕ್ತಿಸಿದ್ಧ್ಯೈ ನಮಃ ।\\
ಅಪರಿಚ್ಛಿನ್ನರೂಪ ಸ್ವಾತ್ಮಾತ್ಮನೇ ಸರ್ವಖೇಚರೀಮುದ್ರಾಯೈ ನಮಃ ।

\as{(ಸಹಸ್ರನಾಮ ೮೦೧ - ೯೦೦)\\}
ಪುಷ್ಟಾ ಪುರಾತನಾ ಪೂಜ್ಯಾ ಪುಷ್ಕರಾ ಪುಷ್ಕರೇಕ್ಷಣಾ ॥೧೫೨॥

ಪರಂಜ್ಯೋತಿಃ ಪರಂಧಾಮ ಪರಮಾಣುಃ ಪರಾತ್ಪರಾ~।\\
ಪಾಶಹಸ್ತಾ ಪಾಶಹಂತ್ರೀ ಪರಮಂತ್ರ - ವಿಭೇದಿನೀ ॥೧೫೩॥

ಮೂರ್ತಾಽಮೂರ್ತಾಽನಿತ್ಯತೃಪ್ತಾ ಮುನಿಮಾನಸ - ಹಂಸಿಕಾ~।\\
ಸತ್ಯವ್ರತಾ ಸತ್ಯರೂಪಾ ಸರ್ವಾಂತರ್ಯಾಮಿನೀ ಸತೀ ॥೧೫೪॥

ಬ್ರಹ್ಮಾಣೀ ಬ್ರಹ್ಮಜನನೀ ಬಹುರೂಪಾ ಬುಧಾರ್ಚಿತಾ~।\\
ಪ್ರಸವಿತ್ರೀ ಪ್ರಚಂಡಾಽಽಜ್ಞಾ ಪ್ರತಿಷ್ಠಾ ಪ್ರಕಟಾಕೃತಿಃ ॥೧೫೫॥

ಪ್ರಾಣೇಶ್ವರೀ ಪ್ರಾಣದಾತ್ರೀ ಪಂಚಾಶತ್ಪೀಠ - ರೂಪಿಣೀ~।\\
ವಿಶೃಂಖಲಾ ವಿವಿಕ್ತಸ್ಥಾ ವೀರಮಾತಾ ವಿಯತ್ಪ್ರಸೂಃ ॥೧೫೬॥

ಮುಕುಂದಾ ಮುಕ್ತಿನಿಲಯಾ ಮೂಲವಿಗ್ರಹ - ರೂಪಿಣೀ~।\\
ಭಾವಜ್ಞಾ ಭವರೋಗಘ್ನೀ ಭವಚಕ್ರ - ಪ್ರವರ್ತಿನೀ ॥೧೫೭॥

ಛಂದಃಸಾರಾ ಶಾಸ್ತ್ರಸಾರಾ ಮಂತ್ರಸಾರಾ ತಲೋದರೀ~।\\
ಉದಾರಕೀರ್ತಿರುದ್ದಾಮವೈಭವಾ ವರ್ಣರೂಪಿಣೀ ॥೧೫೮॥

ಜನ್ಮಮೃತ್ಯು - ಜರಾತಪ್ತ - ಜನವಿಶ್ರಾಂತಿ - ದಾಯಿನೀ~।\\
ಸರ್ವೋಪನಿಷದುದ್ಘುಷ್ಟಾ ಶಾಂತ್ಯತೀತ - ಕಲಾತ್ಮಿಕಾ ॥೧೫೯॥

ಗಂಭೀರಾ ಗಗನಾಂತಸ್ಥಾ ಗರ್ವಿತಾ ಗಾನಲೋಲುಪಾ~।\\
ಕಲ್ಪನಾ - ರಹಿತಾ ಕಾಷ್ಠಾಽಕಾಂತಾ ಕಾಂತಾರ್ಧ - ವಿಗ್ರಹಾ ॥೧೬೦॥

ಕಾರ್ಯಕಾರಣ - ನಿರ್ಮುಕ್ತಾ ಕಾಮಕೇಲಿ - ತರಂಗಿತಾ~।\\
ಕನತ್ಕನಕತಾ - ಟಂಕಾ ಲೀಲಾ - ವಿಗ್ರಹ - ಧಾರಿಣೀ ॥೧೬೧॥

ಅಜಾ ಕ್ಷಯವಿನಿರ್ಮುಕ್ತಾ ಮುಗ್ಧಾ ಕ್ಷಿಪ್ರ - ಪ್ರಸಾದಿನೀ~।\\
ಅಂತರ್ಮುಖ - ಸಮಾರಾಧ್ಯಾ ಬಹಿರ್ಮುಖ - ಸುದುರ್ಲಭಾ ॥೧೬೨॥

ತ್ರಯೀ ತ್ರಿವರ್ಗನಿಲಯಾ ತ್ರಿಸ್ಥಾ ತ್ರಿಪುರಮಾಲಿನೀ~।\\
ನಿರಾಮಯಾ ನಿರಾಲಂಬಾ ಸ್ವಾತ್ಮಾರಾಮಾ ಸುಧಾಸೃತಿಃ ॥೧೬೩॥

ಸಂಸಾರಪಂಕ - ನಿರ್ಮಗ್ನ - ಸಮುದ್ಧರಣ - ಪಂಡಿತಾ~।\\
ಯಜ್ಞಪ್ರಿಯಾ ಯಜ್ಞಕರ್ತ್ರೀ ಯಜಮಾನ - ಸ್ವರೂಪಿಣೀ ॥೧೬೪॥

ಧರ್ಮಾಧಾರಾ ಧನಾಧ್ಯಕ್ಷಾ ಧನಧಾನ್ಯ - ವಿವರ್ಧಿನೀ~।\\
ವಿಪ್ರಪ್ರಿಯಾ ವಿಪ್ರರೂಪಾ ವಿಶ್ವಭ್ರಮಣ - ಕಾರಿಣೀ ॥೧೬೫॥

ವಿಶ್ವಗ್ರಾಸಾ ವಿದ್ರುಮಾಭಾ ವೈಷ್ಣವೀ ವಿಷ್ಣುರೂಪಿಣೀ~।\\
ಅಯೋನಿರ್ಯೋನಿನಿಲಯಾ ಕೂಟಸ್ಥಾ ಕುಲರೂಪಿಣೀ ॥೧೬೬॥

ವೀರಗೋಷ್ಠೀಪ್ರಿಯಾ ವೀರಾ ನೈಷ್ಕರ್ಮ್ಯಾ \as{(೯೦೦)}

\as{(ಅಷ್ಟೋತ್ತರ ೮೦ ೯೦)\\}
ಶ್ರೀಷೋಡಶಾಕ್ಷರೀಮಂತ್ರಮಧ್ಯಗಾಯೈ ನಮೋ ನಮಃ ।\\
ಅನಾದ್ಯಂತಸ್ವಯಂಭೂತದಿವ್ಯಮೂರ್ತ್ಯೈ ನಮೋ ನಮಃ ।\\
ಭಕ್ತಹಂಸಪರೀಮುಖ್ಯವಿಯೋಗಾಯೈ ನಮೋ ನಮಃ ।\\
ಮಾತೃಮಂಡಲಸಂಯುಕ್ತಲಲಿತಾಯೈ ನಮೋ ನಮಃ ।\\
ಭಂಡದೈತ್ಯಮಹಾಸತ್ತ್ವನಾಶನಾಯೈ ನಮೋ ನಮಃ ।\\
ಕ್ರೂರಭಂಡಶಿರಚ್ಛೇದನಿಪುಣಾಯೈ ನಮೋ ನಮಃ ।\\
ಧಾತ್ರಚ್ಯುತಸುರಾಧೀಶಸುಖದಾಯೈ ನಮೋ ನಮಃ ।\\
ಚಂಡಮುಂಡನಿಶುಂಭಾದಿಖಂಡನಾಯೈ ನಮೋ ನಮಃ ।\\
ರಕ್ತಾಕ್ಷರಕ್ತಜಿಹ್ವಾದಿಶಿಕ್ಷಣಾಯೈ ನಮೋ ನಮಃ ।\\
ಮಹಿಷಾಸುರದೋರ್ವೀರ್ಯನಿಗ್ರಹಾಯೈ ನಮೋ ನಮಃ ।

\as{ಯಸ್ಮಾ॒ತ್ಪರಂ॒ ನಾಪ॑ರ॒ಮಸ್ತಿ॒ ಕಿಂಚಿ॒ದ್ಯಸ್ಮಾ॒ನ್ನಾಣೀ॑ಯೋ॒ ನ ಜ್ಯಾಯೋ᳚ಽಸ್ತಿ॒ ಕಶ್ಚಿ॑ತ್ ।
ವೃ॒ಕ್ಷ ಇ॑ವ ಸ್ತಬ್ಧೋ ದಿ॒ವಿ ತಿ॑ಷ್ಠ॒ತ್ಯೇಕ॒ಸ್ತೇನೇ॒ದಂಪೂ॒ರ್ಣಂ ಪುರು॑ಷೇಣ॒ ಸರ್ವಂ᳚ ॥}

ಹೃದಯತ್ರಿಕೋಣಾಧೋಭಾಗರೂಪಪಂಚತನ್ಮಾತ್ರಾತ್ಮಕೇಭ್ಯಃ ಸರ್ವಜಂಭನಬಾಣೇಭ್ಯೋ ನಮಃ ॥\\
ತದ್ದಕ್ಷಭಾಗರೂಪಮನ ಆತ್ಮಕಾಭ್ಯಾಂ ಸರ್ವಮೋಹನಧನುರ್ಭ್ಯಾಂ ನಮಃ ।\\
ತದೂರ್ಧ್ವಭಾಗರೂಪರಾಗಾತ್ಮಕಾಭ್ಯಾಂ ಸರ್ವವಶಂಕರಪಾಶಾಭ್ಯಾಂ ನಮಃ ।\\
ತದ್ವಾಮಭಾಗರೂಪದ್ವೇಷಾತ್ಮಕಾಭ್ಯಾಂ ಸರ್ವಸ್ತಂಭಕರಾಂಕುಶಾಭ್ಯಾಂ ನಮಃ ॥
\section{ತ್ರಿಕೋಣಚಕ್ರಾಯ ನಮಃ ।\\ (ಇತಿ ವ್ಯಾಪಕಂ ನ್ಯಸ್ಯ)}
ಹೃದಯತ್ರಿಕೋಣಾಗ್ರಭಾಗರೂಪ ಮಹತ್ತತ್ವಾತ್ಮನೇ ಕಾಮೇಶ್ವರೀ ದೇವ್ಯೈ ನಮಃ ।\\
ತದ್ದಕ್ಷಕೋಣರೂಪಾಂಹಕಾರಾತ್ಮನೇ ವಜ್ರೇಶ್ವರೀ ದೇವ್ಯೈ ನಮಃ ।\\
ತದ್ವಾಮಕೋಣರೂಪಾವ್ಯಕ್ತಾತ್ಮನೇ ಭಗಮಾಲಿನೀ ದೇವ್ಯೈ ನಮಃ ।\\
ಹೃದ್ರೂಪ ಸರ್ವಸಿದ್ಧಿಪ್ರದಚಕ್ರೇಶ್ವರ್ಯೈ ತ್ರಿಪುರಾಂಬಾಯೈ ನಮಃ ।\\
ಅತಿರಹಸ್ಯಯೋಗಿನೀರೂಪ ಸ್ವಾತ್ಮಾತ್ಮನೇ ಇಚ್ಛಾಸಿದ್ಧ್ಯೈ ನಮಃ ।\\
ಅಪರಿಚ್ಛಿನ್ನರೂಪ ಸ್ವಾತ್ಮಾತ್ಮನೇ ಸರ್ವಬೀಜಮುದ್ರಾಯೈ ನಮಃ ।

\as{(ಸಹಸ್ರನಾಮ ೯೦೧ - ೧೦೦೦)\\}
ನಾದರೂಪಿಣೀ~।\\
ವಿಜ್ಞಾನಕಲನಾ ಕಲ್ಯಾ ವಿದಗ್ಧಾ ಬೈಂದವಾಸನಾ ॥೧೬೭॥

ತತ್ತ್ವಾಧಿಕಾ ತತ್ತ್ವಮಯೀ ತತ್ತ್ವಮರ್ಥ - ಸ್ವರೂಪಿಣೀ~।\\
ಸಾಮಗಾನಪ್ರಿಯಾ ಸೌಮ್ಯಾ ಸದಾಶಿವ - ಕುಟುಂಬಿನೀ ॥೧೬೮॥

ಸವ್ಯಾಪಸವ್ಯ - ಮಾರ್ಗಸ್ಥಾ ಸರ್ವಾಪದ್ವಿನಿವಾರಿಣೀ~।\\
ಸ್ವಸ್ಥಾ ಸ್ವಭಾವಮಧುರಾ ಧೀರಾ ಧೀರಸಮರ್ಚಿತಾ ॥೧೬೯॥

ಚೈತನ್ಯಾರ್ಘ್ಯ - ಸಮಾರಾಧ್ಯಾ ಚೈತನ್ಯ - ಕುಸುಮಪ್ರಿಯಾ~।\\
ಸದೋದಿತಾ ಸದಾತುಷ್ಟಾ ತರುಣಾದಿತ್ಯ - ಪಾಟಲಾ ॥೧೭೦॥

ದಕ್ಷಿಣಾ - ದಕ್ಷಿಣಾರಾಧ್ಯಾ ದರಸ್ಮೇರ - ಮುಖಾಂಬುಜಾ~।\\
ಕೌಲಿನೀ - ಕೇವಲಾಽನರ್ಘ್ಯ - ಕೈವಲ್ಯ - ಪದದಾಯಿನೀ ॥೧೭೧॥

ಸ್ತೋತ್ರಪ್ರಿಯಾ ಸ್ತುತಿಮತೀ ಶ್ರುತಿ - ಸಂಸ್ತುತ - ವೈಭವಾ~।\\
ಮನಸ್ವಿನೀ ಮಾನವತೀ ಮಹೇಶೀ ಮಂಗಲಾಕೃತಿಃ ॥೧೭೨॥

ವಿಶ್ವಮಾತಾ ಜಗದ್ಧಾತ್ರೀ ವಿಶಾಲಾಕ್ಷೀ ವಿರಾಗಿಣೀ~।\\
ಪ್ರಗಲ್ಭಾ ಪರಮೋದಾರಾ ಪರಾಮೋದಾ ಮನೋಮಯೀ ॥೧೭೩॥

ವ್ಯೋಮಕೇಶೀ ವಿಮಾನಸ್ಥಾ ವಜ್ರಿಣೀ ವಾಮಕೇಶ್ವರೀ~।\\
ಪಂಚಯಜ್ಞ - ಪ್ರಿಯಾ ಪಂಚ - ಪ್ರೇತ - ಮಂಚಾಧಿಶಾಯಿನೀ ॥೧೭೪॥

ಪಂಚಮೀ ಪಂಚಭೂತೇಶೀ ಪಂಚ - ಸಂಖ್ಯೋಪಚಾರಿಣೀ~।\\
ಶಾಶ್ವತೀ ಶಾಶ್ವತೈಶ್ವರ್ಯಾ ಶರ್ಮದಾ ಶಂಭುಮೋಹಿನೀ ॥೧೭೫॥

ಧರಾ ಧರಸುತಾ ಧನ್ಯಾ ಧರ್ಮಿಣೀ ಧರ್ಮವರ್ಧಿನೀ~।\\
ಲೋಕಾತೀತಾ ಗುಣಾತೀತಾ ಸರ್ವಾತೀತಾ ಶಮಾತ್ಮಿಕಾ ॥೧೭೬॥

ಬಂಧೂಕ - ಕುಸುಮಪ್ರಖ್ಯಾ ಬಾಲಾ ಲೀಲಾವಿನೋದಿನೀ~।\\
ಸುಮಂಗಲೀ ಸುಖಕರೀ ಸುವೇಷಾಢ್ಯಾ ಸುವಾಸಿನೀ ॥೧೭೭॥

ಸುವಾಸಿನ್ಯರ್ಚನ - ಪ್ರೀತಾಽಽಶೋಭನಾ ಶುದ್ಧಮಾನಸಾ~।\\
ಬಿಂದು - ತರ್ಪಣ - ಸಂತುಷ್ಟಾ ಪೂರ್ವಜಾ ತ್ರಿಪುರಾಂಬಿಕಾ ॥೧೭೮॥

ದಶಮುದ್ರಾ - ಸಮಾರಾಧ್ಯಾ ತ್ರಿಪುರಾಶ್ರೀ - ವಶಂಕರೀ~।\\
ಜ್ಞಾನಮುದ್ರಾ ಜ್ಞಾನಗಮ್ಯಾ ಜ್ಞಾನಜ್ಞೇಯ - ಸ್ವರೂಪಿಣೀ ॥೧೭೯॥

ಯೋನಿಮುದ್ರಾ ತ್ರಿಖಂಡೇಶೀ ತ್ರಿಗುಣಾಂಬಾ ತ್ರಿಕೋಣಗಾ~।\\
ಅನಘಾಽದ್ಭುತ - ಚಾರಿತ್ರಾ ವಾಂಛಿತಾರ್ಥ - ಪ್ರದಾಯಿನೀ ॥೧೮೦॥

ಅಭ್ಯಾಸಾತಿಶಯ - ಜ್ಞಾತಾ ಷಡಧ್ವಾತೀತ - ರೂಪಿಣೀ~।\\
ಅವ್ಯಾಜ - ಕರುಣಾ - ಮೂರ್ತಿರಜ್ಞಾನ - ಧ್ವಾಂತ - ದೀಪಿಕಾ ॥೧೮೧॥

ಆಬಾಲ - ಗೋಪ - ವಿದಿತಾ ಸರ್ವಾನುಲ್ಲಂಘ್ಯ - ಶಾಸನಾ~।\\
ಶ್ರೀಚಕ್ರರಾಜ - ನಿಲಯಾ ಶ್ರೀಮತ್ - ತ್ರಿಪುರಸುಂದರೀ ॥೧೮೨॥

ಶ್ರೀಶಿವಾ ಶಿವಶಕ್ತ್ಯೈಕ್ಯ  ರೂಪಿಣೀ ಲಲಿತಾಂಬಿಕಾ\as{(೧೦೦೦)। ಶ್ರೀಂಹ್ರೀಂಐಂ ಓಂ}

\as{(ಅಷ್ಟೋತ್ತರ ೯೦ - ೧೦೮)\\}
ಅಭ್ರಕೇಶಮಹೋತ್ಸಾಹಕಾರಣಾಯೈ ನಮೋ ನಮಃ ।\\
ಮಹೇಶಯುಕ್ತನಟನತತ್ಪರಾಯೈ ನಮೋ ನಮಃ ।\\
ನಿಜಭರ್ತೃಮುಖಾಂಭೋಜಚಿಂತನಾಯೈ ನಮೋ ನಮಃ ।\\
ವೃಷಭಧ್ವಜವಿಜ್ಞಾನಭಾವನಾಯೈ ನಮೋ ನಮಃ ।\\
ಜನ್ಮಮೃತ್ಯುಜರಾರೋಗಭಂಜನಾಯೈ ನಮೋ ನಮಃ ।\\
ವಿಧೇಯಮುಕ್ತವಿಜ್ಞಾನಸಿದ್ಧಿದಾಯೈ ನಮೋ ನಮಃ ।\\
ಕಾಮಕ್ರೋಧಾದಿಷಡ್ವರ್ಗನಾಶನಾಯೈ ನಮೋ ನಮಃ ।\\
ರಾಜರಾಜಾರ್ಚಿತಪದಸರೋಜಾಯೈ ನಮೋ ನಮಃ ।\\
ಸರ್ವವೇದಾಂತಸಂಸಿದ್ಧಸುತತ್ವಾಯೈ ನಮೋ ನಮಃ ।\\
ಶ್ರೀವೀರಭಕ್ತವಿಜ್ಞಾನವಿಧಾನಾಯೈ ನಮೋ ನಮಃ ।\\
ಅಶೇಷದುಷ್ಟದನುಜಸೂದನಾಯೈ ನಮೋ ನಮಃ ।\\
ಸಾಕ್ಷಾಚ್ಛ್ರೀದಕ್ಷಿಣಾಮೂರ್ತಿಮನೋಜ್ಞಾಯೈ ನಮೋ ನಮಃ ।\\
ಹಯಮೇಧಾಗ್ರಸಂಪೂಜ್ಯಮಹಿಮಾಯೈ ನಮೋ ನಮಃ ।\\
ದಕ್ಷಪ್ರಜಾಪತಿಸುತವೇಷಾಢ್ಯಾಯೈ ನಮೋ ನಮಃ ।\\
ಸುಮಬಾಣೇಕ್ಷುಕೋದಂಡಮಂಡಿತಾಯೈ ನಮೋ ನಮಃ ।\\
ನಿತ್ಯಯೌವನಮಾಂಗಲ್ಯಮಂಗಲಾಯೈ ನಮೋ ನಮಃ ।\\
ಮಹಾದೇವಸಮಾಯುಕ್ತಶರೀರಾಯೈ ನಮೋ ನಮಃ ।\\
ಮಹಾದೇವರತೌತ್ಸುಕ್ಯಮಹಾದೇವ್ಯೈ ನಮೋ ನಮಃ ।

\as{ಆಪಃ॑ ಸೃ॒ಜಂತು॑ ಸ್ನಿ॒ಗ್ಧಾ॒ನಿ॒ ಚಿ॒ಕ್ಲೀ॒ತ ವ॑ಸ ಮೇ॒ ಗೃಹೇ ।\\
ನಿ ಚ॑ ದೇ॒ವೀಂ ಮಾ॒ತರಂ॒ ಶ್ರಿಯಂ॑ ವಾ॒ಸಯ॑ ಮೇ ಕು॒ಲೇ ॥ ೧೨॥}

೪ ಸಕಾರಾಖ್ಯಾಯೈ ನಮಃ ।\\
೪ ಸಮರಸಾಯೈ ನಮಃ ।\\
೪ ಸಕಲಾಗಮಸಂಸ್ತುತಾಯೈ ನಮಃ ।\\
೪ ಸರ್ವವೇದಾಂತ ತಾತ್ಪರ್ಯಭೂಮ್ಯೈ ನಮಃ ।\\
೪ ಸದಸದಾಶ್ರಯಾಯೈ ನಮಃ ।\\
೪ ಸಕಲಾಯೈ ನಮಃ ।\\
೪ ಸಚ್ಚಿದಾನಂದಾಯೈ ನಮಃ ।\\
೪ ಸಾಧ್ಯಾಯೈ ನಮಃ ।\\
೪ ಸದ್ಗತಿದಾಯಿನ್ಯೈ ನಮಃ ।\\
೪ ಸನಕಾದಿಮುನಿಧ್ಯೇಯಾಯೈ ನಮಃ ।\\
೪ ಸದಾಶಿವಕುಟುಂಬಿನ್ಯೈ ನಮಃ ।\\
೪ ಸಕಲಾಧಿಷ್ಠಾನರೂಪಾಯೈ ನಮಃ ।\\
೪ ಸತ್ಯರೂಪಾಯೈ ನಮಃ ।\\
೪ ಸಮಾಕೃತ್ಯೈ ನಮಃ ।\\
೪ ಸರ್ವಪ್ರಪಂಚನಿರ್ಮಾತ್ರ್ಯೈ ನಮಃ ।\\
೪ ಸಮಾನಾಧಿಕವರ್ಜಿತಾಯೈ ನಮಃ ।\\
೪ ಸರ್ವೋತ್ತುಂಗಾಯೈ ನಮಃ ।\\
೪ ಸಂಗಹೀನಾಯೈ ನಮಃ ।\\
೪ ಸಗುಣಾಯೈ ನಮಃ ।\\
೪ ಸಕಲೇಷ್ಟದಾಯೈ ನಮಃ ।


\as{ಆ॒ರ್ದ್ರಾಂ ಪು॒ಷ್ಕರಿ॑ಣೀಂ ಪು॒ಷ್ಟಿಂ॒ ಪಿಂ॒ಗ॒ಲಾಂ ಪ॑ದ್ಮಮಾ॒ಲಿನೀಂ ।\\
ಚಂ॒ದ್ರಾಂ ಹಿ॒ರಣ್ಮ॑ಯೀಂ ಲ॒ಕ್ಷ್ಮೀಂ ಜಾತ॑ವೇದೋ ಮ॒ ಆವ॑ಹ ॥ ೧೩॥}

೪ ಕಕಾರಿಣ್ಯೈ ನಮಃ ।\\
೪ ಕಾವ್ಯಲೋಲಾಯೈ ನಮಃ ।\\
೪ ಕಾಮೇಶ್ವರಮನೋಹರಾಯೈ ನಮಃ ।\\
೪ ಕಾಮೇಶ್ವರಪ್ರಾಣನಾಡ್ಯೈ ನಮಃ ।\\
೪ ಕಾಮೇಶೋತ್ಸಂಗವಾಸಿನ್ಯೈ ನಮಃ ।\\
೪ ಕಾಮೇಶ್ವರಾಲಿಂಗಿತಾಂಗ್ಯೈ ನಮಃ ।\\
೪ ಕಾಮೇಶ್ವರಸುಖಪ್ರದಾಯೈ ನಮಃ ।\\
೪ ಕಾಮೇಶ್ವರಪ್ರಣಯಿನ್ಯೈ ನಮಃ ।\\
೪ ಕಾಮೇಶ್ವರವಿಲಾಸಿನ್ಯೈ ನಮಃ ।\\
೪ ಕಾಮೇಶ್ವರತಪಸ್ಸಿದ್ಧ್ಯೈ ನಮಃ ।\\
೪ ಕಾಮೇಶ್ವರಮನಃಪ್ರಿಯಾಯೈ ನಮಃ ।\\
೪ ಕಾಮೇಶ್ವರಪ್ರಾಣನಾಥಾಯೈ ನಮಃ ।\\
೪ ಕಾಮೇಶ್ವರವಿಮೋಹಿನ್ಯೈ ನಮಃ ।\\
೪ ಕಾಮೇಶ್ವರಬ್ರಹ್ಮವಿದ್ಯಾಯೈ ನಮಃ ।\\
೪ ಕಾಮೇಶ್ವರಗೃಹೇಶ್ವರ್ಯೈ ನಮಃ ।\\
೪ ಕಾಮೇಶ್ವರಾಹ್ಲಾದಕರ್ಯೈ ನಮಃ ।\\
೪ ಕಾಮೇಶ್ವರಮಹೇಶ್ವರ್ಯೈ ನಮಃ ।\\
೪ ಕಾಮೇಶ್ವರ್ಯೈ ನಮಃ ।\\
೪ ಕಾಮಕೋಟಿನಿಲಯಾಯೈ ನಮಃ ।\\
೪ ಕಾಂಕ್ಷಿತಾರ್ಥದಾಯೈ ನಮಃ ।

\as{ಆ॒ರ್ದ್ರಾಂ ಯಃ॒ ಕರಿ॑ಣೀಂ ಯ॒ಷ್ಟಿಂ॒ ಸು॒ವ॒ರ್ಣಾಂ ಹೇ॑ಮಮಾ॒ಲಿನೀಂ ।\\
ಸೂ॒ರ್ಯಾಂ ಹಿ॒ರಣ್ಮ॑ಯೀಂ ಲ॒ಕ್ಷ್ಮೀಂ॒ ಜಾತ॑ವೇದೋ ಮ॒ ಆವಹ ॥ ೧೪॥}

೪ ಲಕಾರಿಣ್ಯೈ ನಮಃ ।\\
೪ ಲಬ್ಧರೂಪಾಯೈ ನಮಃ ।\\
೪ ಲಬ್ಧಧಿಯೇ ನಮಃ ।\\
೪ ಲಬ್ಧವಾಂಛಿತಾಯೈ ನಮಃ ।\\
೪ ಲಬ್ಧಪಾಪಮನೋದೂರಾಯೈ ನಮಃ ।\\
೪ ಲಬ್ಧಾಹಂಕಾರದುರ್ಗಮಾಯೈ ನಮಃ ।\\
೪ ಲಬ್ಧಶಕ್ತ್ಯೈ ನಮಃ ।\\
೪ ಲಬ್ಧದೇಹಾಯೈ ನಮಃ ।\\
೪ ಲಬ್ಧೈಶ್ವರ್ಯಸಮುನ್ನತ್ಯೈ ನಮಃ ।\\
೪ ಲಬ್ಧಬುದ್ಧಯೇ ನಮಃ ।\\
೪ ಲಬ್ಧಲೀಲಾಯೈ ನಮಃ ।\\
೪ ಲಬ್ಧಯೌವನಶಾಲಿನ್ಯೈ ನಮಃ ।\\
೪ ಲಬ್ಧಾತಿಶಯಸರ್ವಾಂಗಸೌಂದರ್ಯಾಯೈ ನಮಃ ।\\
೪ ಲಬ್ಧವಿಭ್ರಮಾಯೈ ನಮಃ ।\\
೪ ಲಬ್ಧರಾಗಾಯೈ ನಮಃ ।\\
೪ ಲಬ್ಧಪತ್ಯೈ ನಮಃ ।\\
೪ ಲಬ್ಧನಾನಾಗಮಸ್ಥಿತ್ಯೈ ನಮಃ ।\\
೪ ಲಬ್ಧಭೋಗಾಯೈ ನಮಃ ।\\
೪ ಲಬ್ಧಸುಖಾಯೈ ನಮಃ ।\\
೪ ಲಬ್ಧಹರ್ಷಾಭಿಪೂರಿತಾಯೈ ನಮಃ ।

\as{ತಾಂ ಮ॒ ಆವ॑ಹ॒ ಜಾತ॑ವೇದೋ ಲ॒ಕ್ಷ್ಮೀಮನ॑ಪಗಾ॒ಮಿನೀಂ᳚ ।\\
ಯಸ್ಯಾಂ॒ ಹಿ॑ರಣ್ಯಂ॒ ಪ್ರಭೂ॑ತಂ॒ ಗಾವೋ॑ ದಾ॒ಸ್ಯೋಽಶ್ವಾ᳚ನ್ವಿಂ॒ದೇಯಂ॒ ಪುರು॑ಷಾನ॒ಹಂ ॥ ೧೫॥}

೪ ಹ್ರೀಂಕಾರಮೂರ್ತಯೇ ನಮಃ ।\\
೪ ಹ್ರೀಂಕಾರಸೌಧಶೃಂಗಕಪೋತಿಕಾಯೈ ನಮಃ ।\\
೪ ಹ್ರೀಂಕಾರದುಗ್ಧಾಬ್ಧಿಸುಧಾಯೈ ನಮಃ ।\\
೪ ಹ್ರೀಂಕಾರಕಮಲೇಂದಿರಾಯೈ ನಮಃ ।\\
೪ ಹ್ರೀಂಕರಮಣಿದೀಪಾರ್ಚಿಷೇ ನಮಃ ।\\
೪ ಹ್ರೀಂಕಾರತರುಶಾರಿಕಾಯೈ ನಮಃ ।\\
೪ ಹ್ರೀಂಕಾರಪೇಟಕಮಣಯೇ ನಮಃ ।\\
೪ ಹ್ರೀಂಕಾರಾದರ್ಶಬಿಂಬಿತಾಯೈ ನಮಃ ।\\
೪ ಹ್ರೀಂಕಾರಕೋಶಾಸಿಲತಾಯೈ ನಮಃ ।\\
೪ ಹ್ರೀಂಕಾರಾಸ್ಥಾನನರ್ತಕ್ಯೈ ನಮಃ ।\\
೪ ಹ್ರೀಂಕಾರಶುಕ್ತಿಕಾ ಮುಕ್ತಾಮಣಯೇ ನಮಃ ।\\
೪ ಹ್ರೀಂಕಾರಬೋಧಿತಾಯೈ ನಮಃ ।\\
೪ ಹ್ರೀಂಕಾರಮಯಸೌವರ್ಣಸ್ತಂಭವಿದ್ರುಮಪುತ್ರಿಕಾಯೈ ನಮಃ ।\\
೪ ಹ್ರೀಂಕಾರವೇದೋಪನಿಷದೇ ನಮಃ ।\\
೪ ಹ್ರೀಂಕಾರಾಧ್ವರದಕ್ಷಿಣಾಯೈ ನಮಃ ।\\
೪ ಹ್ರೀಂಕಾರನಂದನಾರಾಮನವಕಲ್ಪಕ ವಲ್ಲರ್ಯೈ ನಮಃ ।\\
೪ ಹ್ರೀಂಕಾರಹಿಮವದ್ಗಂಗಾಯೈ ನಮಃ ।\\
೪ ಹ್ರೀಂಕಾರಾರ್ಣವಕೌಸ್ತುಭಾಯೈ ನಮಃ ।\\
೪ ಹ್ರೀಂಕಾರಮಂತ್ರಸರ್ವಸ್ವಾಯೈ ನಮಃ ।\\
೪ ಹ್ರೀಂಕಾರಪರಸೌಖ್ಯದಾಯೈ ನಮಃ ।

\as{ನ॒ ಕರ್ಮ॑ಣಾ ನ ಪ್ರ॒ಜಯಾ॒ ಧನೇ॑ನ॒ ತ್ಯಾಗೇ॑ನೈಕೇ ಅಮೃತ॒ತ್ವಮಾ॑ನ॒ಶುಃ ।
ಪರೇ॑ಣ॒ ನಾಕಂ॒ ನಿಹಿ॑ತಂ॒ ಗುಹಾ॑ಯಾಂ ವಿ॒ಭ್ರಾಜ॑ದೇ॒ತದ್ಯತ॑ಯೋ ವಿ॒ಶಂತಿ॑ ।
ವೇ॒ದಾಂ॒ತ॒ವಿ॒ಜ್ಞಾನ॒ಸುನಿ॑ಶ್ಚಿತಾ॒ರ್ಥಾಸ್ಸಂನ್ಯಾ॑ಸಯೋ॒ಗಾದ್ಯತ॑ಯಶ್ಶುದ್ಧ॒ಸತ್ತ್ವಾಃ᳚ ।
ತೇ ಬ್ರ॑ಹ್ಮಲೋ॒ಕೇ ತು॒ ಪರಾಂ᳚ತಕಾಲೇ॒ ಪರಾ॑ಮೃತಾ॒ತ್ಪರಿ॑ಮುಚ್ಯಂತಿ॒ ಸರ್ವೇ᳚ ।
ದ॒ಹ್ರಂ॒ ವಿ॒ಪಾ॒ಪಂ ಪ॒ರಮೇ᳚ಽಶ್ಮಭೂತಂ॒ ಯತ್ಪುಂ॑ಡರೀ॒ಕಂ ಪು॒ರಮ॑ಧ್ಯಸಁ॒ಸ್ಥಂ ।
ತ॒ತ್ರಾ॒ಪಿ॒ ದ॒ಹ್ರಂ ಗ॒ಗನಂ॑ ವಿಶೋಕ॒ಸ್ತಸ್ಮಿ॑ನ್ ಯದಂ॒ತಸ್ತ॒ದುಪಾ॑ಸಿತ॒ವ್ಯಂ ।
ಯೋ ವೇದಾದೌ ಸ್ವ॑ರಃ ಪ್ರೋ॒ಕ್ತೋ॒ ವೇ॒ದಾಂತೇ॑ ಚ ಪ್ರ॒ತಿಷ್ಠಿ॑ತಃ ।
ತಸ್ಯ॑ ಪ್ರ॒ಕೃತಿ॑ಲೀನ॒ಸ್ಯ॒ ಯಃ॒ ಪರ॑ಸ್ಸ ಮ॒ಹೇಶ್ವ॑ರಃ  ॥}
\section{ಬಿಂದುಚಕ್ರಾಯ ನಮಃ ।\\ (ಇತಿ ವ್ಯಾಪಕಂ ನ್ಯಸ್ಯ)}
ಹೃನ್ಮಧ್ಯರೂಪ ನಿರುಪಾಧಿಕ ಸಂವಿನ್ಮಾತ್ರರೂಪ ಕಾಮೇಶ್ವರಾಂಕ ನಿಲಯಾಯೈ ಸಚ್ಚಿದಾನಂದೈಕ ಬ್ರಹ್ಮಾತ್ಮಿಕಾಯೈ ಪರದೇವತಾಯೈ ಲಲಿತಾಯೈ ಮಹಾತ್ರಿಪುರಸುಂದರ್ಯೈ ನಮಃ ॥

ನಿರುಪಾಧಿಕ ಚೈತನ್ಯಮೇವ ಸಚ್ಚಿದಾನಂದಾತ್ಮಕಂ ಅನ್ತಃಕರಣಪ್ರತಿಬಿಂಬಿತಂ ಸತ್ ತದಹಮೇವೇತ್ಯನುಸಂಧಾನಂ ಲಲಿತಾಯಾ ಲೌಹಿತ್ಯಮಿತಿ ವಿಭಾವ್ಯ ॥

ಅಭೇದಸಂಬಂಧೇನ ಸತ್ವಚಿತ್ವಾದಿ ಧರ್ಮವಿಶಿಷ್ಟಸಂವಿದಃ ಕೇವಲಸಂವಿದಶ್ಚ ತಾದಾತ್ಮ್ಯಸಂಬಂಧರೂಪಂ ಕಾಮೇಶ್ವರಾಂಕಯಂತ್ರಣಂ ವಿಶೇಷಣಂ ವಿಭಾವ್ಯ ॥

ಉಪಾಧ್ಯಭಾವರೂಪಶುಕ್ಲತ್ವೋಪಲಕ್ಷಿತಾ ಸತೀ ಶುದ್ಧಸಂವಿದೇವ ಶುಕ್ಲಚರಣಃ ॥

ಚಿತ್ವವಿಶಿಷ್ಟಸಂವಿತ್ ಪ್ರಾಥಮಿಕ ಪರಾಹಂತಾತ್ಮಕ ಮೃತ್ಯುರೂಪೇಣ ರಾಗೇಣೋಪಲಕ್ಷಿತಾ ಸತೀ ರಕ್ತಚರಣಃ ॥

ಅಹಮಾಕಾರವೃತಿನಿರೂಪಿತಾ ವಿಷಯತಾ ಚರಣಯೋರ್ಮಿಥೋ ವಿಶೇಷಣವಿಶೇಷ್ಯಭಾವರೂಪೈವ ತದುಭಯಸಾಮರಸ್ಯಮಿತಿ ವಿಭಾವ್ಯ ॥

ಹೃದ್ರೂಪಸರ್ವಾನಂದಮಯಚಕ್ರೇಶ್ವರ್ಯೈ ಮಹಾತ್ರಿಪುರಸುಂದರ್ಯೈ ನಮಃ ॥

ಪರಾಪರರಹಸ್ಯ ಯೋಗಿನೀರೂಪ ಸ್ವಾತ್ಮಾತ್ಮನೇ ಪ್ರಾಪ್ತಿಸಿದ್ಧ್ಯೈ ನಮಃ ॥

ಅಪರಿಚ್ಛಿನ್ನರೂಪಸ್ವಾತ್ಮಾತ್ಮನೇ ಸರ್ವಯೋನಿಮುದ್ರಾಯೈ ನಮಃ ॥\\
ಇತಿ ತತ್ತತ್ಸ್ಥಾನಸ್ಪರ್ಶಪೂರ್ವಕಂ ಸಮ್ಯಗನುಸಂಧಾಯ ಉಪಚಾರಾನ್\\ ಸಮರ್ಪಯೇತ್ ॥
ತದ್ಯಥಾ  -

ಏವಮಪರಿಚ್ಛಿನ್ನತಯಾ ಭಾವಿತಾಯಾ ಲಲಿತಾಯಾಃ ಸ್ವೇ ಮಹಿಮ್ನ್ಯೇವ ಪ್ರತಿಷ್ಠಿತಮಾಸನಮನುಸಂದಧಾಮಿ॥

ವಿಯದಾದಿ ಸ್ಥೂಲಪ್ರಪಂಚರೂಪ ಪಾದಗತಸ್ಯ ಮಲಸ್ಯ ಸಚ್ಚಿದಾನಂದೈಕ ರೂಪತ್ವ ಭಾವನಾಜಲೇನ ಕ್ಷಾಲನಂ ಪಾದ್ಯಂ ಭಾವಯಾಮಿ ॥

ಸೂಕ್ಷ್ಮ ಪ್ರಪಂಚರೂಪಹಸ್ತಗತಸ್ಯ ತಸ್ಯ ಕ್ಷಾಲನಮರ್ಘ್ಯಂ ಚಿಂತಯಾಮಿ ॥

ಭಾವನಾರೂಪಾಣಾಮಪಾಮಪಿ ಕಬಲೀಕಾರರೂಪಮಾಚಮನಂ ಭಾವಯಾಮಿ ॥

ಸತ್ತ್ವ ಚಿತ್ತ್ವ ಆನಂದತ್ವಾದ್ಯಖಿಲಾವಯವಾವಚ್ಛೇದೇನ ಭಾವನಾಜಲಸಂಪರ್ಕರೂಪಂ ಸ್ನಾನಮನುಚಿಂತಯಾಮಿ ॥

ತೇಷ್ವೇವಾವಯವೇಷು ಪ್ರಸಕ್ತಾಯಾ ಭಾವನಾತ್ಮಕವೃತಿವಿಶೇಷ್ಯತಾಯಾಃ ಪ್ರೋಂಛನಂ ವೃತ್ಯವಿಷಯತ್ವಭಾವನೇನ ವಸ್ತ್ರಂ ಕಲ್ಪಯಾಮಿ ॥

ನಿರ್ವಿಷಯತ್ವ ನಿರಂಜನತ್ವ ಅಶೋಕತ್ವ ಅಮೃತತ್ವಾದ್ಯನೇಕ ಧರ್ಮರೂಪ ಆಭರಣಾನಿ ಧರ್ಮ್ಯಭೇದಭಾವನೇನ ಸಮರ್ಪಯಾಮಿ ॥

ಸ್ವಶರೀರಘಟಕ ಪಾರ್ಥಿವಭಾಗಾನಾಂ ಜಡತಾಪನಯನೇನ ಚಿನ್ಮಾತ್ರಾವಶೇಷರೂಪಂ ಗಂಧಂ ಪ್ರಯಚ್ಛಾಮಿ ॥

ಆಕಾಶಭಾಗಾನಾಂ ತಥಾ ಭಾವನೇನ ಪುಷ್ಪಾಣಿ ಸಮರ್ಪಯಾಮಿ ॥

ವಾಯವ್ಯಭಾಗಾನಾಂ ತಥಾ ಭಾವನಯಾ ಧೂಪಯಾಮಿ ॥

ತೈಜಸಭಾಗಾನಾಂ ತಥಾಕರಣೇನೋದ್ದೀಪಯಾಮಿ ॥

ಅಮೃತಭಾಗಾಂಸ್ತಥಾ ವಿಭಾವ್ಯ ನಿವೇದಯಾಮಿ ॥

ಷೋಡಶಾಂತೇಂದುಮಂಡಲಸ್ಯ ತಥಾ ಭಾವನೇನ ತಾಂಬೂಲಕಲ್ಪನಾಮಾಚರಾಮಿ ॥

ಪರಾಪಶ್ಯಂತ್ಯಾದಿನಿಖಿಲಶಬ್ದಾನಾಂ ನಾದದ್ವಾರಾ ಬ್ರಹ್ಮಣ್ಯುಪಸಂಹಾರಚಿಂತನೇನ ಸ್ತೌಮಿ ॥

ವಿಷಯೇಷು ಧಾವಮಾನಾನಾಂ ಚಿತ್ತವೃತ್ತೀನಾಂ ವಿಷಯಜಡತಾ ನಿರಾಸೇನ ಬ್ರಹ್ಮಣಿ ಪ್ರವಿಲಾಪನೇನ ಪ್ರದಕ್ಷಿಣೀಕರೋಮಿ ॥

ತಾಸಾಂ ವಿಷಯೇಭ್ಯಃ ಪರಾವರ್ತನೇನ ಬ್ರಹ್ಮೈಕಪ್ರವಣತಯಾ ಪ್ರಣಮಾಮಿ॥\\(ಇತ್ಯುಪಚರ್ಯಂ ಜುಹುಯಾತ್ ॥)

ವಿಹಿತಾವಿಹಿತವಿಷಯಾಃ ವೃತ್ತಯಃ ಉತ್ಪನ್ನಾಃ ಅಹಂ ತ್ವಂ ಗುರುರ್ದೇವತೇತ್ಯಾದಯಃ ತಾಃ ಸರ್ವಾಃ ಚಕ್ರರಾಜಸ್ಥ ಅನಂತಶಕ್ತಿಕದಂಬರೂಪಾಃ ತತ್ತತ್ಸೂಕ್ಷ್ಮರೂಪಾಃ ಯೇ ಯೇ ಸಂಸ್ಕಾರಾಃ ತತ್ತತ್ಸರ್ವಂ ಚಿನ್ಮಾತ್ರಮೇವೇತಿ ವಿಭಾವನಯಾ  ನಿರ್ವ್ಯುತ್ಥಾನಂ ಸ್ವಾತ್ಮನಿ ಜುಹೋಮಿ ॥

ಪ್ರಕೃತಭಾವನಾಸು ಯೇ ಗುರುಚರಣಾದಿ ಶಕ್ತಿಕದಂಬಾಂತಾ ವಿಷಯಾಸ್ತೇ ಸರ್ವೇಽಪಿ ಚಿನ್ಮಾತ್ರರೂಪಾಃ, ನ ಪರಸ್ಪರಂ ಭಿದ್ಯಂತೇ ಇತಿ ಭಾವನಯಾ ತರ್ಪಯಾಮಿ ॥

ತಿಥಿಚಕ್ರಮುಕ್ತರೂಪಂ ಕಾಲಚಕ್ರಂ ದೇಶಚಕ್ರಂ ಚ ಸರ್ವಮಸ್ತಿ ಭಾತಿ ಪ್ರಿಯಂ ಚ, ನ ತು ನಾಮರೂಪವತ್ । ಅತಃ ಸರ್ವಂ ಬ್ರಹ್ಮೈವೇತಿ ವಿಭಾವಯಾಮಿ ॥

ಅಥವಾ ಪೂರ್ವಲಿಖಿತಾಂ ನಿತ್ಯಾಭಾವನಾಮಿಹೈವ ಶ್ವಾಸಪ್ರವಿಲಾಪನಫಲಿಕಾಂ ಕುರ್ಯಾತ್ । ತೇನ ಮನಃ ಪವನಾತ್ಮನಾಂ ಐಕ್ಯನಿಭಾಲನೇನ ತ್ರೀನ್ಮುಹೂರ್ತಾನ್ ದ್ವಾವೇಕಂ ವಾ ಮುಹೂರ್ತಮವಿಚ್ಛಿನ್ನಂ ವ್ಯಾಪಯೇತ್ । ತಸ್ಯ ದೇವತಾತ್ಮೈಕ್ಯಸಿದ್ಧಿಃ ಚಿಂತಿತಕಾರ್ಯಾಣ್ಯಯತ್ನೇನ ಸಿದ್ಧ್ಯಂತಿ ॥
\begin{center}{\LARGE\bfseries ಸದ್ಗುರುಚರಣಾರವಿಂದಾರ್ಪಣಮಸ್ತು\\ಓಂ ತತ್ಸತ್\\********}\end{center}
