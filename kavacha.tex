\setcounter{page}{0}
\thispagestyle{empty}
\chapter*{\center ಕವಚಾನಿ}
\section{ಗಾಯತ್ರೀಹೃದಯಂ }
\thispagestyle{empty}
ಅಸ್ಯ ಶ್ರೀ ಗಾಯತ್ರೀಹೃದಯಸ್ಯ ನಾರಾಯಣ ಋಷಿಃ~। ಗಾಯತ್ರೀ ಛಂದಃ~। ಪರಮೇಶ್ವರೀ ದೇವತಾ~। ಜಪೇ ವಿನಿಯೋಗಃ ॥
\begin{center}ನ್ಯಾಸಃ\end{center}
ದ್ಯೌರ್ಮೂರ್ಧ್ನಿ ದೈವತಂ~। ದಂತಪಂಕ್ತಾವಶ್ವಿನೌ~। ಉಭೇ ಸಂಧ್ಯೇ ಚೋಷ್ಠೌ~। ಮುಖಮಗ್ನಿಃ~। ಜಿಹ್ವಾ ಸರಸ್ವತೀ~। ಗ್ರೀವಾಯಾಂ ತು ಬೃಹಸ್ಪತಿಃ~। ಸ್ತನಯೋರ್ವಸವೋಽಷ್ಟೌ~। ಬಾಹ್ವೋರ್ಮರುತಃ~। ಹೃದಯೇ ಪರ್ಜನ್ಯಃ~। ಆಕಾಶಮುದರಂ~। ನಾಭಾವಂತರಿಕ್ಷಂ~। ಕಟ್ಯೋ ರಿಂದ್ರಾಗ್ನೀ~। ಜಘನೇ ವಿಜ್ಞಾನಘನಃ ಪ್ರಜಾಪತಿಃ~। ಕೈಲಾಸ ಮಲಯೇ ಊರೂ~। ವಿಶ್ವೇದೇವಾ ಜಾನ್ವೋಃ~। ಜಂಘಾಯಾಂ ಕೌಶಿಕಃ~। ಗುಹ್ಯಮಯನೇ~। ಊರೂ ಪಿತರಃ~। ಪಾದೌ ಪೃಥಿವೀ~। ವನಸ್ಪತ ಯೋಽಙ್ಗುಲೀಷು~। ಋಷಯೋ ರೋಮಾಣಿ~। ನಖಾನಿ ಮುಹೂರ್ತಾನಿ~। ಅಸ್ಥಿಷು ಗ್ರಹಾಃ~। ಅಸೃಙ್ಮಾಂಸಮೃತವಃ~। ಸಂವತ್ಸರಾ ವೈ ನಿಮಿಷಂ~। ಅಹೋರಾತ್ರಾವಾದಿತ್ಯಶ್ಚಂದ್ರಮಾಃ~। ಪ್ರವರಾಂ ದಿವ್ಯಾಂ ಗಾಯತ್ರೀಂ ಸಹಸ್ರನೇತ್ರಾಂ ಶರಣಮಹಂ ಪ್ರಪದ್ಯೇ~। ಓಂ ತತ್ಸವಿತುರ್ವರೇಣ್ಯಾಯ ನಮಃ~। ಓಂ ತತ್ಪೂರ್ವಾಜಯಾಯ ನಮಃ~। ತತ್ಪ್ರಾತರಾದಿತ್ಯಾಯ ನಮಃ~। ತತ್ಪ್ರಾತರಾದಿತ್ಯಪ್ರತಿಷ್ಠಾಯೈ ನಮಃ~। ಪ್ರಾತರಧೀಯಾನೋ ರಾತ್ರಿಕೃತಂ ಪಾಪಂ ನಾಶಯತಿ~। ಸಾಯಮಧೀಯಾನೋ ದಿವಸಕೃತಂ ಪಾಪಂ ನಾಶಯತಿ~। ಸಾಯಂಪ್ರಾತರಧೀಯಾನೋ ಅಪಾಪೋ ಭವತಿ~। ಸರ್ವತೀರ್ಥೇಷು ಸ್ನಾತೋ ಭವತಿ~। ಸರ್ವೈರ್ದೇವೈರ್ಜ್ಞಾತೋ ಭವತಿ~। ಅವಾಚ್ಯವಚನಾತ್ಪೂತೋ ಭವತಿ~। ಅಭಕ್ಷ್ಯಭಕ್ಷಣಾತ್ಪೂತೋ ಭವತಿ~। ಅಭೋಜ್ಯಭೋಜನಾತ್ಪೂತೋ ಭವತಿ~। ಅಚೋಷ್ಯ ಚೋಷಣಾತ್ಪೂತೋ ಭವತಿ~। ಅಸಾಧ್ಯಸಾಧನಾತ್ಪೂತೋ ಭವತಿ~। ದುಷ್ಪ್ರತಿಗ್ರಹಶತಸಹಸ್ರಾತ್ಪೂತೋ ಭವತಿ~। ಸರ್ವಪ್ರತಿಗ್ರಹಾತ್ಪೂತೋ ಭವತಿ~। ಪಂಕ್ತಿದೂಷಣಾತ್ಪೂತೋ ಭವತಿ~। ಅನೃತವಚನಾತ್ಪೂತೋ ಭವತಿ~। ಅಥಾಬ್ರಹ್ಮಚಾರೀ ಬ್ರಹ್ಮಚಾರೀ ಭವತಿ~। ಅನೇನ ಹೃದಯೇನಾಧೀತೇನ ಕ್ರತುಸಹಸ್ರೇಣೇಷ್ಟಂ ಭವತಿ~। ಷಷ್ಟಿಶತ ಸಹಸ್ರಗಾಯತ್ರ್ಯಾ ಜಪ್ಯಾನಿ ಫಲಾನಿ ಭವಂತಿ~। ಅಷ್ಟೌ ಬ್ರಾಹ್ಮಣಾನ್ಸಮ್ಯಗ್ಗ್ರಾಹಯೇತ್~। ತಸ್ಯ ಸಿದ್ಧಿರ್ಭವತಿ~। ಯ ಇದಂ ನಿತ್ಯಮಧೀಯಾನೋ ಬ್ರಾಹ್ಮಣಃ ಪ್ರಾತಃ ಶುಚಿಃ ಸರ್ವಪಾಪೈಃ ಪ್ರಮುಚ್ಯತ ಇತಿ~। ಬ್ರಹ್ಮಲೋಕೇ ಮಹೀಯತೇ~। ಇತ್ಯಾಹ ಭಗವಾನ್ ಶ್ರೀನಾರಾಯಣಃ ॥
\authorline{ಇತಿ ಶ್ರೀಮದ್ದೇವೀಭಾಗವತೇ ಮಹಾಪುರಾಣೇ ಗಾಯತ್ರೀಹೃದಯಂ ॥}
\section{ಗಾಯತ್ರೀ ಕವಚಂ}
ಅಸ್ಯ ಶ್ರೀ ಗಾಯತ್ತ್ರೀ ಕವಚಸ್ಯ ಬ್ರಹ್ಮವಿಷ್ಣುಮಹೇಶ್ವರಾಃ ಋಷಯಃ~। ಋಗ್ಯಜುಸ್ಸಾಮಾಥರ್ವಶ್ಛಂದಾಂಸಿ~। ಪರಬ್ರಹ್ಮರೂಪಾ ಗಾಯತ್ರೀ ದೇವತಾ~। ತದ್ಬೀಜಂ~। ಭರ್ಗಃ ಶಕ್ತಿಃ~। ಧಿಯಃ ಕೀಲಕಂ~। ಜಪೇ ವಿನಿಯೋಗಃ ॥
\newpage
\dhyana{ಮುಕ್ತಾ ವಿದ್ರುಮ ಹೇಮನೀಲ ಧವಲ ಚ್ಛಾಯೈರ್ಮುಖೈ ಸ್ತ್ರೀಕ್ಷಣೈಃ\\
ಯುಕ್ತಾಮಿಂದು ನಿಬದ್ಧ ರತ್ನ ಮಕುಟಾಂ ತತ್ವಾರ್ಥ ವರ್ಣಾತ್ಮಿಕಾಂ~।\\
ಗಾಯತ್ರೀಂ ವರದಾಭಯಾಂಕುಶಕಶಾಃ ಶುಭ್ರಂ ಕಪಾಲಂ ಗದಾಂ\\
ಶಂಖಂ ಚಕ್ರ ಮಥಾರವಿಂದ ಯುಗಲಂ ಹಸ್ತೈರ್ವಹಂತೀಂ ಭಜೇ॥}

ಗಾಯತ್ರೀ ಪೂರ್ವತಃ ಪಾತು ಸಾವಿತ್ರೀ ಪಾತು ದಕ್ಷಿಣೇ।\\
ಬ್ರಹ್ಮ ಸಂಧ್ಯಾ ತು ಮೇ ಪಶ್ಚಾದುತ್ತರಾಯಾಂ ಸರಸ್ವತೀ॥೧॥

ಪಾರ್ವತೀ ಮೇ ದಿಶಂ ರಾಕ್ಷೇ ತ್ಪಾವಕೀಂ ಜಲಶಾಯಿನೀ।\\
ಯಾತುಧಾನೀಂ ದಿಶಂ ರಕ್ಷೇ ದ್ಯಾತುಧಾನಭಯಂಕರೀ॥೨॥

ಪಾವಮಾನೀಂ ದಿಶಂ ರಕ್ಷೇತ್ಪವಮಾನ ವಿಲಾಸಿನೀ।\\
ದಿಶಂ ರೌದ್ರೀಂಚ ಮೇ ಪಾತು ರುದ್ರಾಣೀ ರುದ್ರ ರೂಪಿಣೀ॥೩॥

ಊರ್ಧ್ವಂ ಬ್ರಹ್ಮಾಣೀ ಮೇ ರಕ್ಷೇ ದಧಸ್ತಾ ದ್ವೈಷ್ಣವೀ ತಥಾ।\\
ಏವಂ ದಶ ದಿಶೋ ರಕ್ಷೇ ತ್ಸರ್ವಾಂಗಂ ಭುವನೇಶ್ವರೀ॥೪॥

ತತ್ಪದಂ ಪಾತು ಮೇ ಪಾದೌ ಜಂಘೇ ಮೇ ಸವಿತುಃ ಪದಂ।\\
ವರೇಣ್ಯಂ ಕಟಿ ದೇಶೇ ತು ನಾಭಿಂ ಭರ್ಗ ಸ್ತಥೈವ ಚ॥೫॥

ದೇವಸ್ಯ ಮೇ ತದ್ಧೃದಯಂ ಧೀಮಹೀತಿ ಚ ಗಲ್ಲಯೋಃ।\\
ಧಿಯಃ ಪದಂ ಚ ಮೇನೇತ್ರೇ ಯಃ ಪದಂ ಮೇ ಲಲಾಟಕಂ॥೬॥

ನಃ ಪಾತು ಮೇ ಪದಂ ಮೂರ್ಧ್ನಿ ಶಿಖಾಯಾಂ ಮೇ ಪ್ರಚೋದಯಾತ್।\\
ತತ್ಪದಂ ಪಾತು ಮೂರ್ಧಾನಂ ಸಕಾರಃ ಪಾತು ಫಾಲಕಂ॥೭॥

ಚಕ್ಷುಷೀ ತು ವಿಕಾರಾರ್ಣಸ್ತುಕಾರಸ್ತು ಕಪೋಲಯೋಃ।\\
ನಾಸಾಪುಟಂ ವಕಾರಾರ್ಣೋ ರೇಕಾರಸ್ತು ಮುಖೇ ತಥಾ ॥೮॥

ಣಿಕಾರ ಊರ್ಧ್ವ ಮೋಷ್ಠಂತು ಯಕಾರಸ್ತ್ವಧರೋಷ್ಠಕಂ।\\
ಆಸ್ಯಮಧ್ಯೇ ಭಕಾರಾರ್ಣೋ ರ್ಗೋಕಾರ ಶ್ಚುಬುಕೇ ತಥಾ ॥೯॥

ದೇಕಾರಃ ಕಂಠ ದೇಶೇತು ವಕಾರಃ ಸ್ಕಂಧ ದೇಶಕಂ।\\
ಸ್ಯಕಾರೋ ದಕ್ಷಿಣಂ ಹಸ್ತಂ ಧೀಕಾರೋ ವಾಮ ಹಸ್ತಕಂ॥೧೦॥

ಮಕಾರೋ ಹೃದಯಂ ರಕ್ಷೇದ್ಧಿಕಾರ ಉದರೇ ತಥಾ।\\
ಧಿಕಾರೋ ನಾಭಿ ದೇಶೇ ತು ಯೋಕಾರಸ್ತು ಕಟಿಂ ತಥಾ॥೧೧॥

ಗುಹ್ಯಂ ರಕ್ಷತು ಯೋಕಾರ ಊರೂ ದ್ವೌ ನಃ ಪದಾಕ್ಷರಂ।\\
ಪ್ರಕಾರೋ ಜಾನುನೀ ರಕ್ಷೇಚ್ಚೋಕಾರೋ ಜಂಘ ದೇಶಕಂ॥೧೨॥

ದಕಾರಂ ಗುಲ್ಫ ದೇಶೇ ತು ಯಾಕಾರಃ ಪದಯುಗ್ಮಕಂ।\\
ತಕಾರೋ ವ್ಯಂಜನಂ ಚೈವ ಸರ್ವಾಂಗಂ ಮೇ ಸದಾವತು ॥೧೩॥

ಇದಂ ತು ಕವಚಂ ದಿವ್ಯಂ ಬಾಧಾ ಶತ ವಿನಾಶನಂ।\\
ಚತುಃಷಷ್ಟಿ ಕಲಾ ವಿದ್ಯಾದಾಯಕಂ ಮೋಕ್ಷಕಾರಕಂ॥೧೪॥

ಮುಚ್ಯತೇ ಸರ್ವ ಪಾಪೇಭ್ಯಃ ಪರಂ ಬ್ರಹ್ಮಾಧಿಗಚ್ಛತಿ~।\\
ಪಠನಾ ಚ್ಛ್ರವಣಾ ದ್ವಾಪಿ ಗೋ ಸಹಸ್ರ ಫಲಂ ಲಭೇತ್ ॥೧೫॥

\authorline{॥ಶ್ರೀ ದೇವೀಭಾಗವತಾಂತರ್ಗತಂ ಗಾಯತ್ರೀ ಕವಚಂ ಸಂಪೂರ್ಣಂ ॥}
\section{ಶ್ರೀ ತ್ರೈಲೋಕ್ಯಮೋಹನಕವಚಂ }
(ಉತ್ಕೀಲನಮಂತ್ರಂ ಪಠಿತ್ವಾ ಕವಚಂ ಪಠೇತ್)\\
ಓಂ ಹಸಕಲಹ್ರೀಂ ಹ್ರೀಂಹ್ರೀಂಹ್ರೀಂ ಸಕಲಹ್ರೀಂ ಕ್ಲೀಂಕ್ಲೀಂಕ್ಲೀಂ  ಶ್ರೀಂಶ್ರೀಂಶ್ರೀಂ ಕ್ರೀಂಕ್ರೀಂಕ್ರೀಂ ರುದ್ರಸೂಚ್ಯಗ್ರೇಣ ಮೂಲವಿದ್ಯಾಶಾಪಂ ಉತ್ಕೀಲಯೋತ್ಕೀಲಯ ಸ್ವಾಹಾ ॥

ಶ್ರೀ ದೇವ್ಯುವಾಚ ॥\\
ದೇವ ದೇವ ಜಗನ್ನಾಥ ಸಚ್ಚಿದಾನಂದವಿಗ್ರಹ~।\\
ಪಂಚಕೃತ್ಯ ಪರೇಶಾನ ಪರಮಾನಂದದಾಯಕ ॥

ಶ್ರೀಮತ್ತ್ರಿಪುರಸುಂದರ್ಯಾ ಯಾ ಯಾ ವಿದ್ಯಾಸ್ತ್ವಯೋದಿತಾಃ~।\\
ಕೃಪಯಾ ಕಥಿತಾಃ ಸರ್ವಾಃ ಶ್ರುತಾಶ್ಚಾಧಿಗತಾ ಮಯಾ ॥

ಪ್ರಾಣನಾಥಾಧುನಾ ಬ್ರೂಹಿ ಕವಚಂ ಮಂತ್ರವಿಗ್ರಹಂ~।\\
ತ್ರೈಲೋಕ್ಯಮೋಹನಂ ಚೇತಿ ನಾಮತಃ ಕಥಿತಂ ಪುರಾ~।\\
ಇದಾನೀಂ ಶ್ರೋತುಮಿಚ್ಛಾಮಿ ಸರ್ವಾರ್ಥಂ ಕವಚಂ ಸ್ಫುಟಂ ॥

ಈಶ್ವರ ಉವಾಚ ॥\\
ಶೃಣು ದೇವಿ ಪ್ರವಕ್ಷ್ಯಾಮಿ ಸುಂದರಿ ಪ್ರಾಣವಲ್ಲಭೇ~।\\
ತ್ರೈಲೋಕ್ಯಮೋಹನಂ ನಾಮ ಸರ್ವವಿದ್ಯೌಘವಿಗ್ರಹಂ ॥

ಯದ್ಧೃತ್ವಾ ದಾನವಾನ್ ವಿಷ್ಣುಃ ನಿಜಘಾನ ಮುಹುರ್ಮುಹುಃ~।\\
ಸೃಷ್ಟಿಂ ವಿತನುತೇ ಬ್ರಹ್ಮಾ ಯದ್ಧೃತ್ವಾ ಪಠನಾದ್ಯತಃ ॥

ಸಂಹರ್ತಾಹಂ ಯತೋ ದೇವಿ ದೇವೇಶೋ ವಾಸವೋ ಯತಃ~।\\
ಧನಾಧಿಪಃ ಕುಬೇರೋಽಪಿ ಯತಃ ಸರ್ವೇ ದಿಗೀಶ್ವರಾಃ ॥

ನ ದೇಯಂ ಯದಶಿಷ್ಯೇಭ್ಯೋ ದೇಯಂ ಶಿಷ್ಯೇಭ್ಯ ಏವ ಚ~।\\
ಅಭಕ್ತೇಭ್ಯೋಽಪಿ ಪುತ್ರೇಭ್ಯೋ ದತ್ವಾ ಮೃತ್ಯುಮವಾಪ್ನುಯಾತ್ ॥

ಶ್ರೀಮತ್ತ್ರಿಪುರಸುಂದರ್ಯಾಃ ಕವಚಸ್ಯ ಋಷಿಃ ಶಿವಃ~।\\
ಛಂದೋ ವಿರಾಡ್ ದೇವತಾ ಚ ಶ್ರೀಮತ್ತ್ರಿಪುರಸುಂದರೀ~।\\
ಧರ್ಮಾರ್ಥಕಾಮಮೋಕ್ಷೇಷು ವಿನಿಯೋಗಃ ಪ್ರಕೀರ್ತಿತಃ ॥\\
\newpage
({\bfseries ಓಂಐಂಹ್ರೀಂಶ್ರೀಂ}) ಶಿರೋ ಮೇ ವಾಗ್ಭವಂ ಪಾತು ಕಏಈಲಹ್ರೀಂ ಸ್ವರೂಪಕಂ।\\
ಹಸಕಲಹ್ರೀಂ ಲಲಾಟಂ ಚ ಪಾತು ಕಾಮೇಶ್ವರೀ ಮಮ~।\\
ಹಕಹಲಹ್ರೀಂ ದೃಶೌ ಚ ಪಾತು ಕಾಮೇಶಮಧ್ಯಮಂ~।\\
ಕಹಯಲಹ್ರೀಂ ಶ್ರುತೀ ತು ಪಾತು ಕಾಮಂ ತುರೀಯಕಂ~।\\
ಹಕಲಸಹ್ರೀಂ ಶಕ್ತ್ಯಾಖ್ಯಂ ಪಾತು ಜಿಹ್ವಾಂ ಚ ಪಂಚಮಂ~॥\\
ಕಏಈಲಹ್ರೀಂ ಹಸಕಲಹ್ರೀಂ ಹಕಹಲಹ್ರೀಂ\\
ಕಹಯಲಹ್ರೀಂ ಹಕಲಸಹ್ರೀಂ  ಪರಮಾತ್ಮರೂಪಿಣೀ~।\\
ವದನಂ ಸಕಲಂ ಪಾತು ಪಂಚಕೂಟೈಸ್ತು ಪಂಚಮೀ ॥

ಕಏಈಲಹ್ರೀಂ ಘ್ರಾಣಂ ಮೇ ಪಾತು ವಾಗ್ಭವಸಂಜ್ಞಕಂ~।\\
ಹಸಕಹಲಹ್ರೀಂ ಕಂಠಂ ಪಾತು ಕಾಮೇಶಸಂಜ್ಞಕಂ~।\\
ಸಕಲಹ್ರೀಂ ಶಕ್ತಿಕೂಟಂ ಸ್ಕಂಧೌ ಪಾತು ಸದಾ ಮಮ~॥\\
ಕಏಈಲಹ್ರೀಂ ಹಸಕಹಲಹ್ರೀಂ ಸಕಲಹ್ರೀಂ ಚ।\\
ಕಾಮೇನೋಪಾಸಿತಾ ವಿದ್ಯಾ ಕಕ್ಷದೇಶೇ ಸದಾವತು ॥

ಕ್ಲೀಂಹ್ರೀಂಶ್ರೀಂಐಂಸೌಃ ಓಂಹ್ರೀಂಶ್ರೀಂ ಐಂಕ್ಲೀಂಸೌಃ\\ ಶ್ರೀಂಹ್ರೀಂಕ್ಲೀಂಐಂಸೌಃ॥\\
ಕಾಮಾದಿಷೋಡಶೀ ಪಾತು ಭುಜೌ ತ್ರಿಪುರಸುಂದರೀ ॥

ಓಂಕ್ಲೀಂಹ್ರೀಂಶ್ರೀಂ ಐಂಕ್ಲೀಂಸೌಃ  ಶ್ರೀಂಹ್ರೀಂಕ್ಲೀಂ\\ ಸ್ತ್ರೀಂಐಂಕ್ರೋಂಕ್ರೀಂಶ್ರೀಂಹೂಂ।\\
ತಾರಾದಿಷೋಡಶೀ ಪಾತು ಮಣಿಬಂಧದ್ವಯಂ ತಥಾ ॥

ಶ್ರೀಂಹ್ರೀಂಐಂಕ್ಲೀಂಸೌಃ ಓಂಹ್ರೀಂಶ್ರೀಂ ಐಂಕ್ಲೀಂಸೌಃ\\ ಸೌಃಐಂಕ್ಲೀಂಹ್ರೀಂಶ್ರೀಂ।\\
ರಮಾದಿಷೋಡಶೀ ಪಾತು ಕರೌ ತ್ರಿಪುರಸುಂದರೀ ॥

ಹ್ರೀಂಹ್ಸೌಃ ಓಂಐಂಹ್ರೀಂಶ್ರೀಂ ಐಂಕ್ಲೀಂಸೌಃ \\ಸ್ತ್ರೀಂಕ್ರೋಂಐಂಹ್ರೀಂ ಹೂಂಹೂಂಶ್ರೀಂ~।\\
ಮಾಯಾದಿಕಾ ತು ಹೃತ್ಪಾತು ಶ್ರೀಮತ್ತ್ರಿಪುರಸುಂದರೀ ॥

ಐಂಹ್ರೀಂಶ್ರೀಂಕ್ಲೀಂಸೌಃ ಐಂಕ್ಲೀಂಸೌಃ ಸೌಃಕ್ಲೀಂಐಂ ಸೌಃಕ್ಲೀಂಶ್ರೀಂಹ್ರೀಂಐಂ~।\\
ವಾಗಾದಿಷೋಡಶೀ ಪಾತು ಸ್ತನೌ ಮೇ ಸುಂದರೀ ಪರಾ ॥

ಕಏಶ್ರೀಂಕಏಈಲಹ್ರೀಂ ಕ್ಲೀಂಹಸಕಹಲಹ್ರೀಂ ಸೌಃಸಕಲಹ್ರೀಂ।\\
ನಖವರ್ಣಾಖ್ಯವಿದ್ಯೇಯಂ ಪಾರ್ಶ್ವೌ ಪಾತ್ವಪರಾಜಿತಾ ॥

ಹ್ರೀಂಹ್ಸೌಂಸ್ಹೌಂ ಹ್ರೀಂಸ್ಹೌಂಹ್ಸೌಂ ಕ್ಲೀಂಹಸಕಹಲಹ್ರೀಂಹ್ರೀಂಹ್ರೀಂ\\ ಸೌಃಸೌಃ ಹಹಸಕಹಲಹ್ರೀಂ ಕ್ಲೀಂಹ್ಸೌಂಸ್ಹೌಂ ಹ್ರೀಂಸ್ಹೌಂಹ್ಸೌಂಹ್ರೀಂ~।\\
ಏಕತ್ರಿಂಶದ್ವರ್ಣರೂಪಾ ಮಹಾಪುರುಷಪೂಜಿತಾ ॥\\
ಮಹಾಗುಹ್ಯಸ್ವರೂಪಾ ಚ ಕೇವಲಾನಂದಚಿನ್ಮಯೀ~।\\
ಕಟಿದೇಶಂ ಸದಾ ಪಾತು ಪರಬ್ರಹ್ಮಸ್ವರೂಪಿಣೀ ॥

ಹಸಕಲಹ್ರೀಂ ಪೃಷ್ಠದೇಶೇ ದೇವೀರಕ್ಷತು ವೈ ಸದಾ~।\\
ಹಸಕಹಲಹ್ರೀಂ ಕುಕ್ಷಿದೇಶಂ ಮಹಾವಿದ್ಯಾ ಚ ಪಾತು ಮಾಂ~।\\
ಸಕಲಹ್ರೀಂ ಶಕ್ತಿಕೂಟಂ ಪಾತು ವಕ್ಷಸ್ಥಲಂ ಮಮ~॥\\
ಹಸಕಲಹ್ರೀಂ ಹಸಕಹಲಹ್ರೀಂ ಸಕಲಹ್ರೀಂ ಮೇ~।\\
ಲೋಪಾಮುದ್ರಾಪಂಚದಶೀ ಮಧ್ಯದೇಶಂ ಸದಾವತು ॥

ಕಹಏಈಲಹ್ರೀಂ ನಾಭಿಂ ಪಾತು ಹಕಏಈಲಹ್ರೀಂ ಕಟಿಂ ಪಾತು~।\\
ಸಕ್ಥಿನೀ ಮೇ ಸದಾ ಪಾತು ಸಕಏಈಲಹ್ರೀಂ ಸದಾ॥\\
ಕಹಏಈಲಹ್ರೀಂ ಹಕಏಈಲಹ್ರೀಂ ಸಕಏಈಲಹ್ರೀಂ ಮೇ।\\
ವಸುಚಂದ್ರಾಮಾನವೀ ಮಾಂ ಸಾ ಸದಾ ಸರ್ವತೋಽವತು ॥

ಸಹಕಏಈಲಹ್ರೀಂ ಮೇ ಊರುಯುಗ್ಮಂ ಸದಾವತು~।\\
ಸಹಕಹಏಈಲಹ್ರೀಂ ಗುಹ್ಯಂ ಪಾತು ವರಪ್ರದಾ~।\\
ಹಸಕಏಈಲಹ್ರೀಂ ತು ಜಾನುನೀ ಪಾತು ಮೇ ಸದಾ~॥\\
ಸಹಕಏಈಲಹ್ರೀಂ ಸಹಕಹಏಈಲಹ್ರೀಂ ಹಸಕಏಈಲಹ್ರೀಂ~।\\
ಚಂದ್ರವಿದ್ಯಾ ಚ ಪಾತು ಮಾಂ ಪಕ್ಷಾದ್ಯಕ್ಷರವರ್ಣಕಾ~।\\
ಜಲಜೇ ಭಯಸಂಘಾತೇ ಸದಾ ಮಾಂ ಪರಿರಕ್ಷತು ॥

ಹಸಕಏಈಲಹ್ರೀಂ ಗುಲ್ಫಯುಗ್ಮಂ ಮಮ ವೈ ಸರ್ವದಾವತು~।\\
ಹಸಕಹಏಈಲಹ್ರೀಂ ಪಾದೌ ಪಾಯಾತ್ ಸನಾತನೀ~।\\
ಸಹಕಏಈಲಹ್ರೀಂ ಮೇ ಪ್ರಪದೌ ಪಾತು ಸರ್ವದಾ~॥\\
ಹಸಕಏಈಲಹ್ರೀಂ ಹಸಕಹಏಈಲಹ್ರೀಂ\\
ಸಹಕಏಈಲಹ್ರೀಂ ಸದಾ ಕುಬೇರೇಣ ಪ್ರಪೂಜಿತಾ~।\\
ದ್ವಾವಿಂಶತ್ಯಕ್ಷರೀ ವಿದ್ಯಾ ಸರ್ವಾಂಗಂ ಮೇ ಸದಾವತು ॥

ಕಏಈಲಹ್ರೀಂ ಪ್ರಾಚ್ಯಾಂ ತು ತ್ರಿಪುರಾ ಪರಿರಕ್ಷತು~।\\
ಹಸಕಹಲಹ್ರೀಂ ಪಾತು ವಹ್ನಿಕೋಣೇ ನಿರಂತರಂ~।\\
ಸಹಸಕಲಹ್ರೀಂ ಯಾಮ್ಯಾಂ ತು ಪಾತು ಮೇ ಸರ್ವಸಿದ್ಧಿದಾ~॥\\
ಕಏಈಲಹ್ರೀಂ ಹಸಕಹಲಹ್ರೀಂ ಸಹಸಕಲಹ್ರೀಂ ತು~।\\
ಅಗಸ್ತ್ಯವಿದ್ಯಾ ಸಾ ಸೇವ್ಯಾ ಚಕ್ರಸ್ಥಾ ಮಾಂ ಸದಾವತು ॥

ಸಏಈಲಹ್ರೀಂ ಚ ನಿತ್ಯಂ ನೈರೃತ್ಯಾಂ ಮಾಂ ಸದಾವತು~।\\
ಸಹಕಹಲಹ್ರೀಂ ಚೈವ ಪ್ರತೀಚ್ಯಾಂ ಪಾತು ಪಾರ್ವತೀ ॥\\
ಸಕಲಹ್ರೀಂ ತು ವಾಯವ್ಯೇ ಸದಾ ಮಾಂ ಪರಿರಕ್ಷತು~।\\
ಸಏಈಲಹ್ರೀಂ ಸಹಕಹಲಹ್ರೀಂ ಸಕಲಹ್ರೀಂ ತು~॥\\
ನಂದ್ಯಾರಾಧಿತವಿದ್ಯೇಯಂ ಸರ್ವಾಂಗೇ ಮಾಂ ಸದಾವತು ॥

ಹಸಕಲಹ್ರೀಂ ಉತ್ತರೇ ಚ ಪಾತು ಮಾಂ ಜಗದೀಶ್ವರೀ~।\\
ಸಹಕಲಹ್ರೀಂ ಈಶದಿಶಿ ಶಿವಪತ್ನೀ ಚ ಪಾತು ಮಾಂ~।\\
ಸಕಹಲಹ್ರೀಂ ಸುಂದರೀ ಊರ್ಧ್ವೇ ಮಾಂ ಪಾತು ಸರ್ವದಾ~॥\\
ಹಸಕಲಹ್ರೀಂ ಸಹಕಲಹ್ರೀಂ ಸಕಹಲಹ್ರೀಂ ಮಾಂ~।\\
ಅಧೋ ರಕ್ಷತು ಮೇ ನಿತ್ಯಂ ಸೂರ್ಯಪೂಜ್ಯಾ ಮಹೋದಯಾ ॥

ಕಏಈಲಹ್ರೀಂ ಹಸಕಹಲಹ್ರೀಂ ಸಕಲಹ್ರೀಂ ಮೇ~।\\
ಸರ್ವಾಂಗಂ ಶಕ್ರಸಂಪೂಜ್ಯಾ ಸತತಂ ಪರಿರಕ್ಷತು ॥

ಕಏಈಲಹ್ರೀಂ ಹಕಹಲಹ್ರೀಂ ಹಸಕಲಹ್ರೀಂ ಚ~।\\
ಬ್ರಹ್ಮಾಣೀ ಮಾಂ ಸದಾ ಪಾಯಾತ್ ಶ್ರೀಮತ್ತ್ರಿಪುರಸುಂದರೀ ॥

ಹಸಕಲಹ್ರೀಂ ಹಸಕಹಲಹ್ರೀಂ ಸಕಲಹ್ರೀಂ \\ಹಸಕಲ ಹಸಕಹಲ ಸಕಲಹ್ರೀಂ ಚ ಶಾಂಕರೀ।\\
ಚತುಃಕೂಟಾ ಮಹಾವಿದ್ಯಾ ಪಾತಾಲೇ ಮಾಂ ಸದಾವತು ॥

ಹಸಕಲಹ್ರೀಂ ಆಧಾರಂ ಹಸಕಹಲಹ್ರೀಂ ಚ ಲಿಂಗಕೇ~।\\
ಸಕಲಹ್ರೀಂ ಪಾತು ನಾಭಿಂ ಸಹಕಲಹ್ರೀಂ ಅನಾಹತಂ ।\\
ಸಹಕಹಲಹ್ರೀಂ ಕಂಠಂ ಸಹಸಕಲಹ್ರೀಂ ತಥಾ~॥\\
ಹಸಕಲಹ್ರೀಂ ಹಸಕಹಲಹ್ರೀಂ ಸಕಲಹ್ರೀಂ \\ಸಹಕಲಹ್ರೀಂ ಸಹಕಹಲಹ್ರೀಂ ಸಹಸಕಲಹ್ರೀಂ~।\\
ಮನೋಭವಾ ಸದಾ ಪಾತು ರದಸಂಖ್ಯಾ ಮಹಾಪ್ರಭಾ~।\\
ಷಟ್ಕೂಟಾ ವೈಷ್ಣವೀ ಸಾ ವೈ ಪಾತು ಮಾಂ ಸುಂದರೀ ಪರಾ ॥

ಕಏಈಲಹರೀ ಹಸಕಹಲರೀ ಸಕಲಹರೀ~।\\
ದುರ್ವಾಸಸಾ ಪ್ರಪೂಜ್ಯಾ ಚ ದಿಕ್ಷು ವಿದ್ಯಾ ಸದಾವತು ॥

ಕಹಏಈಲಹ್ರೀಂ ಹಲಏಈಲಹ್ರೀಂ ಸಕಏಈಲಹ್ರೀಂ~।\\
ಕ್ರೋಧೇನ ಪೂಜಿತಾ ನಿತ್ಯಂ ವಿದಿಕ್ಷು ಪರಿರಕ್ಷತು ॥

ಹಸಕಲಹ್ರೀಂ ಹಸಕಹಹಲಹ್ರೀಂ ಸಕಲಹ್ರೀಂ~।\\
ಮಹಾಜ್ಞಾನಮಯೀ ಪಾತು ನಿತ್ಯಂ ಮಾಂ ಷೋಡಶೀ ಪರಾ ॥

ಶ್ರೀಂಹ್ರೀಂಕ್ಲೀಂಐಂಸೌಃ ಓಂಹ್ರೀಂಶ್ರೀಂ ಕಏಈಲಹ್ರೀಂ\\ ಹಸಕಹಲಹ್ರೀಂ ಸಕಲಹ್ರೀಂ ಸೌಃಐಂಕ್ಲೀಂಹ್ರೀಂಶ್ರೀಂ~।\\
ಸರ್ವಾಂಗಂ ಮೇ ಮಹಾವಿದ್ಯಾ ಬೀಜರೂಪಾ ಚ ಷೋಡಶೀ ॥

ಓಂ ಕ್ಲೀಂಹ್ರೀಂಶ್ರೀಂ ಐಂಕ್ಲೀಂಸೌಃ ಕಏಈಲಹ್ರೀಂ\\ಹಸಕಹಲಹ್ರೀಂ ಸಕಲಹ್ರೀಂ ಸ್ತ್ರೀಂಐಂಕ್ರೋಂಕ್ರೀಂ \\ಈಂ ಹೂಂ ಷೋಡಶಸ್ವರರೂಪಿಣಿ  ಶ್ರೀಮತ್ತ್ರಿಪುರಸುಂದರಿ \\ಹ್ರಾಂಹ್ರೀಂಹ್ರೂಂ ಫಟ್ ಸರ್ವಸಿದ್ಧಿಂ ಪ್ರಯಚ್ಛ ಪ್ರಯಚ್ಛ ಸ್ವಾಹಾ~।\\
ಶ್ರೀಮಹಾಷೋಡಶೀ ವಿದ್ಯಾನಾಖ್ಯಾತಾ ಭುವನತ್ರಯೇ~।\\
ಜ್ಞಾನೇನ ಮೃತ್ಯುಹಾ ಸಾ ಮಾಂ ಶಿರಸ್ಥಾ ಸರ್ವತೋಽವತು ॥\\
ಶ್ರೀಮಹಾಷೋಡಶೀ ಪೂರ್ಣಾ ಮಹಾದೇವೇನ ಪೂಜಿತಾ~।\\
ಯಸ್ಯಾ ವಿಜ್ಞಾನಮಾತ್ರೇಣ ಮೃತ್ಯೋರ್ಮೃತ್ಯುರ್ಭವೇತ್ಸ್ವಯಂ ॥

ಕ್ಲೀಂಐಂಶ್ರೀಂ ಕಏಈಲಹ್ರೀಂ ಹಸಕಹಲಹ್ರೀಂ ಸಕಲಹ್ರೀಂ~।\\
ಕಾಮವಾಗೀಶ್ವರೀ ಲಕ್ಷ್ಮೀಸ್ತ್ರಿಕೂಟಾ ಪರಮೇಶ್ವರೀ ॥

ಐಂ ಕಏಈಲಹ್ರೀಂ ಕ್ಲೀಂ ಹಸಕಹಲಹ್ರೀಂ ಸೌಃ ಸಕಲಹ್ರೀಂ\\ಸೋಹಂ ಹೌಂ ಹಂಸಃ ಹ್ರೀಂ ಸಕಲಹ್ರೀಂ ಸೌಃ \\ ಹಸಕಹಲಹ್ರೀಂ ಕ್ಲೀಂ ಕಏಈಲಹ್ರೀಂ ಐಂ ಬ್ರಹ್ಮಸ್ವರೂಪಿಣೀ~।\\
ನೇತ್ರವೇದಯುತೈರ್ವರ್ಣೈರ್ಯುತಾ ಸಾ ಸರ್ವತೋಽವತು~॥\\
ಬ್ರಹ್ಮಸ್ವರೂಪಿಣೀ ಚೇಯಂ ಪರಮಾನಂದಚಿದ್ಘನಾ~।\\
ಅಷ್ಟಾದಶಾಕ್ಷರೀ ವಿದ್ಯಾ ಸದಾ ಮಾಂ ಪರಿರಕ್ಷತು ॥({\bfseries ಶ್ರೀಂಹ್ರೀಂಐಂ})

ಇತಿ ತೇ ಕಥಿತಂ ದೇವಿ ಬ್ರಹ್ಮವಿದ್ಯಾಕಲೇಬರಂ~।\\
ತ್ರೈಲೋಕ್ಯಮೋಹನಂ ನಾಮ ಕವಚಂ ಬ್ರಹ್ಮರೂಪಕಂ ॥

ಸಪ್ತಲಕ್ಷಮಹಾವಿದ್ಯಾಃ ತಂತ್ರಾದೌ ಕಥಿತಾಃ ಪ್ರಿಯೇ~।\\
ತಾಸಾಂ ಸಾರಾತ್ಸಾರತಯಾ ಯಾ ಯಾ ವಿದ್ಯಾಃ ಸುಗೋಪಿತಾಃ ॥

ಬಹುನಾತ್ರಕಿಮುಕ್ತೇನ ಶ್ರೀಮಹಾಷೋಡಶೀ ಪರಾ~।\\
ಪ್ರಕಾಶಿತಾ ಮಯಾ ದೇವಿ ಯಾಂ ಪೃಚ್ಛಸಿ ಪುನಃ ಪುನಃ ॥

ಮಹಾವಿದ್ಯಾಮಯಂ ಬ್ರಹ್ಮಕವಚಂ ಮನ್ಮುಖೋದಿತಂ~।\\
ಗುರುಮಭ್ಯರ್ಚ್ಯ ವಿಧಿವತ್ ಕವಚಂ ಪ್ರಪಠೇತ್ತತಃ ॥

ದೇವೀಮಭ್ಯರ್ಚ್ಯ ವಿಧಿವತ್ ಪುರಶ್ಚರ್ಯಾಂ ಸಮಾಚರೇತ್~।\\
ಅಷ್ಟೋತ್ತರಶತಂ ಜಪ್ತ್ವಾ ದಶಾಂಶಂ ಹವನಾದಿಕಂ ॥

ತತಃ ಸುಸಿದ್ಧಕವಚಃ ಪುಣ್ಯಾತ್ಮಾ ಮದನೋಪಮಃ~।\\
ಮಂತ್ರಸಿದ್ಧಿರ್ಭವೇತ್ತಸ್ಯ ಪುರಶ್ಚರ್ಯಾಂ ವಿನಾ ತತಃ ॥

ಗದ್ಯಪದ್ಯಮಯೀ ವಾಣೀ ತಸ್ಯ ವಕ್ತ್ರಾತ್ ಪ್ರಜಾಯತೇ~।\\
ವಕ್ತ್ರೇ ತಸ್ಯವಸೇದ್ವಾಣೀ ಕಮಲಾ ನಿಶ್ಚಲಾ ಗೃಹೇ ॥

ಪುಷ್ಪಾಂಜಲ್ಯಷ್ಟಕಂ ದತ್ವಾ ಮೂಲೇನೈವ ಪಠೇತ್ ಸಕೃತ್~।\\
ಅಪಿ ವರ್ಷಸಹಸ್ರಾಣಾಂ ಪೂಜಾಫಲಮವಾಪ್ನುಯಾತ್ ॥

ಆತ್ಮಾನಂ ತನ್ಮಯಂ ಕೃತ್ವಾ ಯಃ ಪಠೇತ್ ಕವಚಂ ಪರಂ~।\\
ಯಂ ಯಂ ಪಶ್ಯತಿ ವೈ ಶೀಘ್ರಂ ಸ ಸ ದಾಸೋ ಭವೇದ್ಧೃವಂ ॥

ವಿಲಿಖ್ಯ ಭೂರ್ಜೇ ಘುಟಿಕಾಂ ಸ್ವರ್ಣಸ್ಥಾಂ ಧಾರಯೇದ್ಯದಿ~।\\
ಕಂಠೇ ವಾ ಯದಿ ವಾ ಬಾಹೌ ಸ ಕುರ್ಯಾದ್ದಾಸವಜ್ಜಗತ್ ॥

ತ್ರಿಲೋಕೀಂ ಕ್ಷೋಭಯತ್ಯೇವ ತ್ರೈಲೋಕ್ಯವಿಜಯೀ ಭವೇತ್~।\\
ತದ್ಗಾತ್ರಂ ಪ್ರಾಪ್ಯ ಶಸ್ತ್ರಾಣಿ ಬ್ರಹ್ಮಾಸ್ತ್ರಾದೀನಿ ಪಾರ್ವತಿ ॥

ಮಾಲ್ಯಾನಿ ಕುಸುಮಾನೀವ ಸುಖದಾನಿ ಭವಂತಿ ಹಿ~।\\
ಸ್ಪರ್ಧಾಂ ನಿರಸ್ಯ ಭವನೇ ಲಕ್ಷ್ಮೀರ್ವಾಣೀ ವಸೇತ್ತತಃ ॥

ಇದಂ ಕವಚಮಜ್ಞಾತ್ವಾ ಯೋ ಜಪೇತ್ಸುಂದರೀಂ ಪರಾಂ~।\\
ನವಲಕ್ಷಂ ಪ್ರಜಪ್ತ್ವಾಽಪಿ ತಸ್ಯ ವಿದ್ಯಾ ನ ಸಿದ್ಧ್ಯತಿ ॥

ಸ ಶಸ್ತ್ರಘಾತಮಾಪ್ನೋತಿ ಸೋಽಚಿರಾನ್ಮೃತ್ಯುಮಾಪ್ನುಯಾತ್~।\\
ಇದಮೇವ ಪರಂ ಯಸ್ಮಾದ್ ಭುಕ್ತಿಮುಕ್ತಿಪ್ರದಾಯಕಂ~।\\
ತಸ್ಮಾತ್ಸರ್ವಪ್ರಯತ್ನೇನ ಪಠನೀಯಂ ಮುುಮುಕ್ಷುಭಿಃ ॥


\authorline{ಇತಿ ಶ್ರೀರುದ್ರಯಾಮಲೇ ಗೌರೀಶ್ವರಸಂವಾದೇ ಶ್ರೀರಾಜರಾಜೇಶ್ವರೀ \\ಮಹಾತ್ರಿಪುರಸುಂದರ್ಯಾಃ ತ್ರೈಲೋಕ್ಯಮೋಹನಂ ನಾಮ ಕವಚಂ ಸಂಪೂರ್ಣಂ॥}
\section{ಶಿವಕವಚಂ }
ಅಸ್ಯ ಶ್ರೀ ಶಿವಕವಚ ಸ್ತೋತ್ರಮಹಾಮಂತ್ರಸ್ಯ ಋಷಭಯೋಗೀಶ್ವರ ಋಷಿಃ~। ಅನುಷ್ಟುಪ್ ಛಂದಃ~। ಶ್ರೀಸದಾಶಿವೋ ದೇವತಾ~। ಓಂ ಬೀಜಂ~। ನಮಃ ಶಕ್ತಿಃ~। ಶಿವಾಯೇತಿ ಕೀಲಕಂ~। ಸದಾಶಿವಪ್ರೀತ್ಯರ್ಥೇ ಜಪೇ ವಿನಿಯೋಗಃ~।\\
ಓಂ ನಮೋ ಭಗವತೇ ಜ್ವಲ ಜ್ವಲ ಮಹಾರುದ್ರಾಯ ಶ್ರೀಂ ಹ್ರೀಂ ಕ್ಲೀಂ(*)\\೧. * ಹ್ರಾಂ ಸರ್ವಶಕ್ತಿ ಧಾಮ್ನೇ
೨. * ನಂ ತೃಪ್ತಿಶಕ್ತಿಧಾಮ್ನೇ\\
೩. * ಮಂ ಅನಾದಿಬೋಧಶಕ್ತಿಧಾಮ್ನೇ
೪. * ಶಿಂ ಸ್ವತಂತ್ರಶಕ್ತಿಧಾಮ್ನೇ\\
೫. * ವಾಂ ಅಲುಪ್ತಶಕ್ತಿಧಾಮ್ನೇ
೬. * ಯಂ ಅನಂತಶಕ್ತಿಧಾಮ್ನೇ ಇತಿ ನ್ಯಾಸಃ\\
\dhyana{ವಜ್ರದಂಷ್ಟ್ರಂ ತ್ರಿನಯನಂ ಕಾಲಕಂಠಮರಿಂದಮಂ~।\\
ಸಹಸ್ರಕರಮತ್ಯುಗ್ರಂ ವಂದೇ ಶಂಭುಂ ಉಮಾಪತಿಂ ॥}\\
ಋಷಭ ಉವಾಚ ॥\\
ನಮಸ್ಕೃತ್ಯ ಮಹಾದೇವಂ ವಿಶ್ವವ್ಯಾಪಿನಮೀಶ್ವರಂ~।\\
ವಕ್ಷ್ಯೇ ಶಿವಮಯಂ ವರ್ಮ ಸರ್ವರಕ್ಷಾಕರಂ ನೃಣಾಂ ॥೧॥\\
ಶುಚೌ ದೇಶೇ ಸಮಾಸೀನೋ ಯಥಾವತ್ಕಲ್ಪಿತಾಸನಃ~।\\
ಜಿತೇಂದ್ರಿಯೋ ಜಿತಪ್ರಾಣಶ್ಚಿಂತಯೇಚ್ಛಿವಮವ್ಯಯಂ ॥೨॥

ಹೃತ್ಪುಂಡರೀಕಾಂತರಸನ್ನಿವಿಷ್ಟಂ\\ಸ್ವತೇಜಸಾ ವ್ಯಾಪ್ತನಭೋಽವಕಾಶಂ~।\\
ಅತೀಂದ್ರಿಯಂ ಸೂಕ್ಷ್ಮಮನಂತಮಾದ್ಯಂ\\ಧ್ಯಾಯೇತ್ ಪರಾನಂದಮಯಂ ಮಹೇಶಂ ॥೩॥

ಧ್ಯಾನಾವಧೂತಾಖಿಲಕರ್ಮಬಂಧ\\ಶ್ಚಿರಂ ಚಿದಾನಂದನಿಮಗ್ನಚೇತಾಃ~।\\
ಷಡಕ್ಷರನ್ಯಾಸ ಸಮಾಹಿತಾತ್ಮಾ\\ಶೈವೇನ ಕುರ್ಯಾತ್ಕವಚೇನ ರಕ್ಷಾಂ ॥೪॥

ಮಾಂ ಪಾತು ದೇವೋಽಖಿಲದೇವತಾತ್ಮಾ\\ಸಂಸಾರಕೂಪೇ ಪತಿತಂ ಗಭೀರೇ~।\\
ತನ್ನಾಮ ದಿವ್ಯಂ ಪರಮಂತ್ರಮೂಲಂ\\ಧುನೋತು ಮೇ ಸರ್ವಮಘಂ ಹೃದಿಸ್ಥಂ ॥೫॥

ಸರ್ವತ್ರ ಮಾಂ ರಕ್ಷತು ವಿಶ್ವಮೂರ್ತಿ\\ರ್ಜ್ಯೋತಿರ್ಮಯಾನಂದಘನಶ್ಚಿದಾತ್ಮಾ~।\\
ಅಣೋರಣೀಯಾನುರುಶಕ್ತಿರೇಕಃ\\ಸ ಈಶ್ವರಃ ಪಾತು ಭಯಾದಶೇಷಾತ್ ॥೬॥

ಯೋ ಭೂಸ್ವರೂಪೇಣ ಬಿಭರ್ತಿ ವಿಶ್ವಂ\\ಪಾಯಾತ್ಸ ಭೂಮೇರ್ಗಿರಿಶೋಽಷ್ಟಮೂರ್ತಿಃ~।\\
ಯೋಽಪಾಂ ಸ್ವರೂಪೇಣ ನೃಣಾಂ ಕರೋತಿ\\ಸಂಜೀವನಂ ಸೋಽವತು ಮಾಂ ಜಲೇಭ್ಯಃ ॥೭॥

ಕಲ್ಪಾವಸಾನೇ ಭುವನಾನಿ ದಗ್ಧ್ವಾ\\ಸರ್ವಾಣಿ ಯೋ ನೃತ್ಯತಿ ಭೂರಿಲೀಲಃ~।\\
ಸ ಕಾಲರುದ್ರೋಽವತು ಮಾಂ ದವಾಗ್ನೇ\\ರ್ವಾತ್ಯಾದಿಭೀತೇರಖಿಲಾಚ್ಚ ತಾಪಾತ್ ॥೮॥

ಪ್ರದೀಪ್ತವಿದ್ಯುತ್ಕನಕಾವಭಾಸೋ\\ವಿದ್ಯಾವರಾಭೀತಿಕುಠಾರಪಾಣಿಃ~।\\
ಚತುರ್ಮುಖಸ್ತತ್ಪುರುಷಸ್ತ್ರಿನೇತ್ರಃ\\ಪ್ರಾಚ್ಯಾಂ ಸ್ಥಿತೋ ರಕ್ಷತು ಮಾಮಜಸ್ರಂ ॥೯॥

ಕುಠಾರಖೇಟಾಂಕುಶಶೂಲಢಕ್ಕಾ\\ಕಪಾಲಪಾಶಾಕ್ಷಗುಣಾನ್ ದಧಾನಃ~।\\
ಚತುರ್ಮುಖೋ ನೀಲರುಚಿಸ್ತ್ರಿನೇತ್ರಃ\\ಪಾಯಾದಘೋರೋ ದಿಶಿ ದಕ್ಷಿಣಸ್ಯಾಂ ॥೧೦॥

ಕುಂದೇಂದುಶಂಖಸ್ಫಟಿಕಾವಭಾಸೋ\\ವೇದಾಕ್ಷಮಾಲಾವರದಾಭಯಾಂಕಃ~।\\
ತ್ರ್ಯಕ್ಷಶ್ಚತುರ್ವಕ್ತ್ರ ಉರುಪ್ರಭಾವಃ\\ಸದ್ಯೋಽಧಿಜಾತೋಽವತು ಮಾಂ ಪ್ರತೀಚ್ಯಾಂ ॥೧೧॥
\newpage
 ವರಾಕ್ಷಮಾಲಾಭಯಟಂಕಹಸ್ತಃ\\ಸರೋಜಕಿಂಜಲ್ಕಸಮಾನವರ್ಣಃ~।\\
ತ್ರಿಲೋಚನಶ್ಚಾರುಚತುರ್ಮುಖೋ ಮಾಂ\\ಪಾಯಾದುದೀಚ್ಯಾಂ ದಿಶಿ ವಾಮದೇವಃ ॥೧೨॥

ವೇದಾಭಯೇಷ್ಟಾಂಕುಶಟಂಕಪಾಶ\\ಕಪಾಲಢಕ್ಕಾಕ್ಷರಶೂಲಪಾಣಿಃ~।\\
ಸಿತದ್ಯುತಿಃ ಪಂಚಮುಖೋಽವತಾನ್ಮಾಂ\\ಈಶಾನ ಊರ್ಧ್ವಂ ಪರಮಪ್ರಕಾಶಃ ॥೧೩॥

ಮೂರ್ಧಾನಮವ್ಯಾನ್ಮಮ ಚಂದ್ರಮೌಲಿ\\ರ್ಭಾಲಂ ಮಮಾವ್ಯಾದಥ ಭಾಲನೇತ್ರಃ~।\\
ನೇತ್ರೇ ಮಮಾವ್ಯಾದ್ಭಗನೇತ್ರಹಾರೀ\\ನಾಸಾಂ ಸದಾ ರಕ್ಷತು ವಿಶ್ವನಾಥಃ ॥೧೪॥

ಪಾಯಾಚ್ಛ್ರುತೀ ಮೇ ಶ್ರುತಿಗೀತಕೀರ್ತಿಃ\\ಕಪೋಲಮವ್ಯಾತ್ಸತತಂ ಕಪಾಲೀ~।\\
ವಕ್ತ್ರಂ ಸದಾ ರಕ್ಷತು ಪಂಚವಕ್ತ್ರೋ\\ಜಿಹ್ವಾಂ ಸದಾ ರಕ್ಷತು ವೇದಜಿಹ್ವಃ ॥೧೫॥

ಕಂಠಂ ಗಿರೀಶೋಽವತು ನೀಲಕಂಠಃ\\ಪಾಣಿದ್ವಯಂ ಪಾತು ಪಿನಾಕಪಾಣಿಃ~।\\
ದೋರ್ಮೂಲಮವ್ಯಾನ್ಮಮ ಧರ್ಮಬಾಹು\\ರ್ವಕ್ಷಃಸ್ಥಲಂ ದಕ್ಷಮಖಾಂತಕೋಽವ್ಯಾತ್ ॥೧೬॥
\newpage
ಮಮೋದರಂ ಪಾತು ಗಿರೀಂದ್ರಧನ್ವಾ\\ಮಧ್ಯಂ ಮಮಾವ್ಯಾನ್ಮದನಾಂತಕಾರೀ~।\\
ಹೇರಂಬತಾತೋ ಮಮ ಪಾತು ನಾಭಿಂ\\ಪಾಯಾತ್ಕಟಿಂ ಧೂರ್ಜಟಿರೀಶ್ವರೋ ಮೇ ॥೧೭॥

ಊರುದ್ವಯಂ ಪಾತು ಕುಬೇರಮಿತ್ರೋ\\ಜಾನುದ್ವಯಂ ಮೇ ಜಗದೀಶ್ವರೋಽವ್ಯಾತ್~।\\
ಜಂಘಾಯುಗಂ ಪುಂಗವಕೇತುರವ್ಯಾತ್\\ಪಾದೌ ಮಮಾವ್ಯಾತ್ಸುರವಂದ್ಯಪಾದಃ ॥೧೮॥

ಮಹೇಶ್ವರಃ ಪಾತು ದಿನಾದಿಯಾಮೇ\\ಮಾಂ ಮಧ್ಯಯಾಮೇಽವತು ವಾಮದೇವಃ~।\\
ತ್ರಿಲೋಚನಃ ಪಾತು ತೃತೀಯಯಾಮೇ\\ವೃಷಧ್ವಜಃ ಪಾತು ದಿನಾಂತ್ಯಯಾಮೇ ॥೨೦॥

ಪಾಯಾನ್ನಿಶಾದೌ ಶಶಿಶೇಖರೋ ಮಾಂ\\ಗಂಗಾಧರೋ ರಕ್ಷತು ಮಾಂ ನಿಶೀಥೇ~।\\
ಗೌರೀಪತಿಃ ಪಾತು ನಿಶಾವಸಾನೇ\\ಮೃತ್ಯುಂಜಯೋ ರಕ್ಷತು ಸರ್ವಕಾಲಂ ॥೨೧॥

ಅಂತಃಸ್ಥಿತಂ ರಕ್ಷತು ಶಂಕರೋ ಮಾಂ\\ಸ್ಥಾಣುಃ ಸದಾ ಪಾತು ಬಹಿಃಸ್ಥಿತಂ ಮಾಂ~।\\
ತದಂತರೇ ಪಾತು ಪತಿಃ ಪಶೂನಾಂ\\ಸದಾಶಿವೋ ರಕ್ಷತು ಮಾಂ ಸಮಂತಾತ್ ॥೨೨॥
\newpage
ತಿಷ್ಠಂತಮವ್ಯಾದ್ ಭುವನೈಕನಾಥಃ\\ಪಾಯಾದ್ವ್ರಜಂತಂ ಪ್ರಮಥಾಧಿನಾಥಃ~।\\
ವೇದಾಂತವೇದ್ಯೋಽವತು ಮಾಂ ನಿಷಣ್ಣಂ\\ ಮಾಮವ್ಯಯಃ ಪಾತು ಶಿವಃ ಶಯಾನಂ ॥೨೩॥

ಮಾರ್ಗೇಷು ಮಾಂ ರಕ್ಷತು ನೀಲಕಂಠಃ\\ಶೈಲಾದಿದುರ್ಗೇಷು ಪುರತ್ರಯಾರಿಃ~।\\
ಅರಣ್ಯವಾಸಾದಿ ಮಹಾಪ್ರವಾಸೇ\\ಪಾಯಾನ್ಮೃಗವ್ಯಾಧ ಉದಾರಶಕ್ತಿಃ ॥೨೪॥

ಕಲ್ಪಾಂತಕಾಲೋಗ್ರಪಟುಪ್ರಕೋಪ\\ಸ್ಫುಟಾಟ್ಟಹಾಸೋಚ್ಚಲಿತಾಂಡಕೋಶಃ~।\\
ಘೋರಾರಿಸೇನಾರ್ಣವದುರ್ನಿವಾರ\\ಮಹಾಭಯಾದ್ರಕ್ಷತು ವೀರಭದ್ರಃ ॥೨೫॥

ಪತ್ತ್ಯಶ್ವಮಾತಂಗರಥಾವರೂಥಿನೀ\\ಸಹಸ್ರಲಕ್ಷಾಯುತ ಕೋಟಿಭೀಷಣಂ~।\\
ಅಕ್ಷೌಹಿಣೀನಾಂ ಶತಮಾತತಾಯಿನಾಂ\\ಛಿಂದ್ಯಾನ್ಮೃಡೋ ಘೋರಕುಠಾರ ಧಾರಯಾ ॥೨೬॥

ನಿಹಂತು ದಸ್ಯೂನ್ಪ್ರಲಯಾನಲಾರ್ಚಿಃ\\ಜ್ವಲತ್ತ್ರಿಶೂಲಂ ತ್ರಿಪುರಾಂತಕಸ್ಯ~।\\
ಶಾರ್ದೂಲಸಿಂಹರ್ಕ್ಷವೃಕಾದಿಹಿಂಸ್ರಾನ್\\ಸಂತ್ರಾಸಯತ್ವೀಶಧನುಃ ಪಿನಾಕಃ ॥೨೭॥
\newpage
ದುಃಸ್ವಪ್ನದುಃಶಕುನದುರ್ಗತಿದೌರ್ಮನಸ್ಯ\\ದುರ್ಭಿಕ್ಷದುರ್ವ್ಯಸನದುಃಸಹದುರ್ಯಶಾಂಸಿ~।\\
ಉತ್ಪಾತತಾಪವಿಷಭೀತಿಮಸದ್ಗ್ರಹಾರ್ತಿಂ\\ವ್ಯಾಧೀಂಶ್ಚ ನಾಶಯತು ಮೇ ಜಗತಾಮಧೀಶಃ ॥೨೮॥

ಓಂ ನಮೋ ಭಗವತೇ ಸದಾಶಿವಾಯ ಸಕಲ ತತ್ವಾತ್ಮಕಾಯ ಸರ್ವಮಂತ್ರ ಸ್ವರೂಪಾಯ ಸರ್ವಯಂತ್ರಾಧಿಷ್ಠಿತಾಯ ಸರ್ವತಂತ್ರ ಸ್ವರೂಪಾಯ ಸರ್ವತತ್ವ ವಿದೂರಾಯ ಬ್ರಹ್ಮ ರುದ್ರಾವತಾರಿಣೇ ನೀಲಕಂಠಾಯ ಪಾರ್ವತೀ ಮನೋಹರ ಪ್ರಿಯಾಯ ಸೋಮ ಸೂರ್ಯಾಗ್ನಿ ಲೋಚನಾಯ ಭಸ್ಮೋದ್ಧೂಲಿತ ವಿಗ್ರಹಾಯ ಮಹಾಮಣಿ ಮುಕುಟ ಧಾರಣಾಯ ಮಾಣಿಕ್ಯ ಭೂಷಣಾಯ ಸೃಷ್ಟಿ ಸ್ಥಿತಿ ಪ್ರಲಯ ಕಾಲ ರೌದ್ರಾವತಾರಾಯ ದಕ್ಷಾಧ್ವರ ಧ್ವಂಸಕಾಯ ಮಹಾಕಾಲ ಭೇದನಾಯ ಮೂಲಾಧಾರೈಕ ನಿಲಯಾಯ ತತ್ವಾತೀತಾಯ ಗಂಗಾಧರಾಯ ಸರ್ವದೇವಾಧಿ ದೇವಾಯ ಷಡಾಶ್ರಯಾಯ ವೇದಾಂತ ಸಾರಾಯ ತ್ರಿವರ್ಗ ಸಾಧನಾಯ ಅನಂತ ಕೋಟಿ ಬ್ರಹ್ಮಾಂಡ ನಾಯಕಾಯ ಅನಂತ ವಾಸುಕಿ ತಕ್ಷಕ ಕರ್ಕೋಟಕ ಶಂಖ ಕುಲಿಕ ಪದ್ಮ ಮಹಾಪದ್ಮೇತಿ ಅಷ್ಟ ಮಹಾನಾಗಕುಲ ಭೂಷಣಾಯ ಪ್ರಣವ ಸ್ವರೂಪಾಯ ಚಿದಾಕಾಶಾಯ ಆಕಾಶ ದಿಕ್ ಸ್ವರೂಪಾಯ ಗ್ರಹ ನಕ್ಷತ್ರ ಮಾಲಿನೇ ಸಕಲಾಯ ಕಲಂಕ ರಹಿತಾಯ ಸಕಲ ಲೋಕೈಕ ಕರ್ತ್ರೇ ಸಕಲ ಲೋಕೈಕ ಭರ್ತ್ರೇ ಸಕಲ ಲೋಕೈಕ ಸಂಹರ್ತ್ರೇ ಸಕಲ ಲೋಕೈಕ ಗುರವೇ ಸಕಲ ಲೋಕೈಕ ಸಾಕ್ಷಿಣೇ ಸಕಲ ನಿಗಮಗುಹ್ಯಾಯ ಸಕಲ ವೇದಾಂತಪಾರಗಾಯ ಸಕಲ ಲೋಕೈಕ ವರಪ್ರದಾಯ ಸಕಲ ಲೋಕೈಕ ಶಂಕರಾಯ ಸಕಲ ದುರಿತಾರ್ತಿ ಭಂಜನಾಯ ಸಕಲ ಜಗದಭಯಂಕರಾಯ ಶಶಾಂಕ ಶೇಖರಾಯ ಶಾಶ್ವತ ನಿಜಾವಾಸಾಯ ನಿರಾಕಾರಾಯ ನಿರಾಭಾಸಾಯ ನಿರಾಮಯಾಯ ನಿರ್ಮಲಾಯ ನಿರ್ಮದಾಯ ನಿಶ್ಚಿಂತಾಯ ನಿರಹಂಕಾರಾಯ ನಿರಂಕುಶಾಯ ನಿಷ್ಕಲಂಕಾಯ ನಿರ್ಗುಣಾಯ ನಿಷ್ಕಾಮಾಯ ನಿರುಪಪ್ಲವಾಯ ನಿರುಪದ್ರವಾಯ ನಿರವದ್ಯಾಯ ನಿರಂತರಾಯ ನಿಷ್ಕಾರಣಾಯ ನಿರಾತಂಕಾಯ ನಿಷ್ಪ್ರಪಂಚಾಯ ನಿಸ್ಸಂಗಾಯ ನಿರ್ದ್ವಂದ್ವಾಯ ನಿರಾಧಾರಾಯ ನೀರಾಗಾಯ ನಿಷ್ಕ್ರೋಧಾಯ ನಿರ್ಲೋಪಾಯ ನಿಷ್ಪಾಪಾಯ ನಿರ್ಭಯಾಯ ನಿರ್ವಿಕಲ್ಪಾಯ ನಿರ್ಭೇದಾಯ ನಿಷ್ಕ್ರಿಯಾಯ ನಿಸ್ತುಲಾಯ ನಿಃಸಂಶಯಾಯ ನಿರಂಜನಾಯ ನಿರುಪಮ ವಿಭವಾಯ ನಿತ್ಯ ಶುದ್ಧ ಬುದ್ಧ ಮುಕ್ತ ಪರಿಪೂರ್ಣ ಸಚ್ಚಿದಾನಂದಾದ್ವಯಾಯ ಪರಮ ಶಾಂತ ಸ್ವರೂಪಾಯ ಪರಮ ಶಾಂತ ಪ್ರಕಾಶಾಯ ತೇಜೋರೂಪಾಯ ತೇಜೋಮಯಾಯ ತೇಜೋಽಧಿಪತಯೇ ಜಯ ಜಯ ರುದ್ರ ಮಹಾರುದ್ರ ಮಹಾರೌದ್ರ ಭದ್ರಾವತಾರ ಮಹಾಭೈರವ ಕಾಲಭೈರವ ಕಲ್ಪಾಂತಭೈರವ ಕಪಾಲ ಮಾಲಾಧರ ಖಟ್ವಾಂಗ ಚರ್ಮಖಡ್ಗಧರ ಪಾಶಾಂಕುಶ ಡಮರುಶೂಲ ಚಾಪ ಬಾಣ ಗದಾ ಶಕ್ತಿ ಭಿಂದಿಪಾಲ ತೋಮರ ಮುಸಲ ಮುದ್ಗರ ಪಾಶ ಪರಿಘ ಭುಶುಂಡೀ ಶತಘ್ನೀ ಚಕ್ರಾದ್ಯಾಯುಧ ಭೀಷಣಾಕಾರ ಸಹಸ್ರಮುಖ ದಂಷ್ಟ್ರಾಕರಾಲವದನ ವಿಕಟಾಟ್ಟಹಾಸ ವಿಸ್ಫಾರಿತ ಬ್ರಹ್ಮಾಂಡಮಂಡಲ ನಾಗೇಂದ್ರಕುಂಡಲ ನಾಗೇಂದ್ರಹಾರ ನಾಗೇಂದ್ರವಲಯ ನಾಗೇಂದ್ರಚರ್ಮಧರ ನಗೇಂದ್ರನಿಕೇತನ ಮೃತ್ಯುಂಜಯ ತ್ರ್ಯಂಬಕ ತ್ರಿಪುರಾಂತಕ ವಿಶ್ವರೂಪ ವಿರೂಪಾಕ್ಷ ವಿಶ್ವೇಶ್ವರ ವೃಷಭವಾಹನ ವಿಷವಿಭೂಷಣ ವಿಶ್ವತೋಮುಖ ಸರ್ವತೋಮುಖ ಮಾಂ ರಕ್ಷ ರಕ್ಷ ಜ್ವಲ ಜ್ವಲ ಪ್ರಜ್ವಲ ಪ್ರಜ್ವಲ ಮಹಾಮೃತ್ಯುಭಯಂ ಶಮಯ ಶಮಯ ಅಪಮೃತ್ಯುಭಯಂ ನಾಶಯ ನಾಶಯ ರೋಗಭಯಂ ಉತ್ಸಾದಯೋತ್ಸಾದಯ ವಿಷಸರ್ಪಭಯಂ ಶಮಯ ಶಮಯ ಚೋರಾನ್ ಮಾರಯ ಮಾರಯ ಮಮ ಶತ್ರೂನ್ ಉಚ್ಚಾಟಯೋಚ್ಚಾಟಯ ತ್ರಿಶೂಲೇನ ವಿದಾರಯ ವಿದಾರಯ ಕುಠಾರೇಣ ಭಿಂಧಿ ಭಿಂಧಿ ಖಡ್ಗೇನ ಛಿಂಧಿ ಛಿಂಧಿ ಖಟ್ವಾಂಗೇನ ವಿಪೋಥಯ ವಿಪೋಥಯ ಮುಸಲೇನ ನಿಷ್ಪೇಷಯ ನಿಷ್ಪೇಷಯ ಬಾಣೈಃ ಸಂತಾಡಯ ಸಂತಾಡಯ ಯಕ್ಷ ರಕ್ಷಾಂಸಿ ಭೀಷಯ ಭೀಷಯ ಅಶೇಷ ಭೂತಾನ್ ವಿದ್ರಾವಯ ವಿದ್ರಾವಯ ಕೂಷ್ಮಾಂಡ ಭೂತ ವೇತಾಲ ಮಾರೀಗಣ ಬ್ರಹ್ಮರಾಕ್ಷಸಗಣಾನ್ ಸಂತ್ರಾಸಯ ಸಂತ್ರಾಸಯ ಮಮ ಅಭಯಂ ಕುರು ಕುರು ಮಮ ಪಾಪಂ ಶೋಧಯ ಶೋಧಯ ವಿತ್ರಸ್ತಂ ಮಾಂ ಆಶ್ವಾಸಯ ಆಶ್ವಾಸಯ ನರಕ ಮಹಾಭಯಾನ್ ಮಾಂ ಉದ್ಧರ ಉದ್ಧರ ಅಮೃತ ಕಟಾಕ್ಷ ವೀಕ್ಷಣೇನ ಮಾಂ ಆಲೋಕಯ ಆಲೋಕಯ ಸಂಜೀವಯ ಸಂಜೀವಯ ಕ್ಷುತ್ತೃಷ್ಣಾರ್ತಂ ಮಾಂ ಆಪ್ಯಾಯಯ ಆಪ್ಯಾಯಯ ದುಃಖಾತುರಂ ಮಾಂ ಆನಂದಯ ಆನಂದಯ ಶಿವಕವಚೇನ ಮಾಂ ಆಚ್ಛಾದಯ ಆಚ್ಛಾದಯ ಹರ ಹರ ಮೃತ್ಯುಂಜಯ ತ್ರ್ಯಂಬಕ ಸದಾಶಿವ ಪರಮಶಿವ ನಮಸ್ತೇ ನಮಸ್ತೇ ನಮಃ॥
\section{ಅನ್ನಪೂರ್ಣಾ ಕವಚಂ}
ದೇವ್ಯುವಾಚ॥\\
ಭವತಾ ತ್ವನ್ನಪೂರ್ಣಾಯಾ ಯಾ ಯಾ ವಿದ್ಯಾಃ ಸುದುರ್ಲಭಾಃ~।\\
ಕೃಪಯಾ ಕಥಿತಾಃ ಸರ್ವಾಃ ಶ್ರುತಾಶ್ಚಾಧಿಗತಾ ಮಯಾ ॥೧ ॥

ಸಾಂಪ್ರತಂ ಶ್ರೋತುಮಿಚ್ಛಾಮಿ ಕವಚಂ ಮಂತ್ರ ವಿಗ್ರಹಂ~।\\
ಈಶ್ವರ ಉವಾಚ॥\\
ಶೃಣು ಪಾರ್ವತಿ ವಕ್ಷ್ಯಾಮಿ ಸಾವಧಾನಾವಧಾರಯ॥೨ ॥

ಬ್ರಹ್ಮವಿದ್ಯಾ ಸ್ವರೂಪಂ ಚ ಮಹದೈಶ್ವರ್ಯದಾಯಕಂ ।\\
ಪಠನಾದ್ಧಾರಣಾನ್ಮರ್ತ್ಯ ಸ್ತ್ರೈಲೋಕ್ಯೈಶ್ವರ್ಯ ಭಾಗ್ಭವೇತ್॥೩ ॥

ತ್ರೈಲೋಕ್ಯ ರಕ್ಷಣಸ್ಯಾಸ್ಯ ಕವಚಸ್ಯ ಋಷಿಃ ಶಿವಃ ।\\
ಛಂದೋ ವಿರಾಡ್ ದೇವತಾ ಸ್ಯಾದನ್ನಪೂರ್ಣಾ ಸಮೃದ್ಧಿದಾ॥೪ ॥

ಧರ್ಮಾರ್ಥಕಾಮಮೋಕ್ಷೇಷು ವಿನಿಯೋಗಃ ಪ್ರಕೀರ್ತಿತಃ।\\
ಹ್ರೀಂ ನಮೋ ಭಗವತ್ಯಂತೇ ಮಾಹೇಶ್ವರಿ ಪದಂ ತತಃ॥೫ ॥

ಅನ್ನಪೂರ್ಣೇ ತತಃ ಸ್ವಾಹಾ ಚೈಷಾ ಸಪ್ತದಶಾಕ್ಷರೀ ।\\
ಪಾತು ಮಾಮನ್ನಪೂರ್ಣಾ ಸಾ ಯಾ ಖ್ಯಾತಾ ಭುವನತ್ರಯೇ ॥೬ ॥

ವಿಮಾಯಾ ಪ್ರಣವಾದ್ಯೈಷಾ ತಥಾ ಸಪ್ತದಶಾಕ್ಷರೀ~।\\
ಪಾತ್ವನ್ನಪೂರ್ಣಾ ಸರ್ವಾಂಗೇ ರತ್ನಕುಂಭಾನ್ನಪಾತ್ರದಾ॥೭ ॥

ಶ್ರೀಬೀಜಾದ್ಯಾ ತಥೈವೈಷಾ ದ್ವಿರಂಧ್ರಾರ್ಣಾ ತಥಾ ಮುಖಂ ।\\
ಪ್ರಣವಾದ್ಯಾ ಭ್ರುವೌ ಪಾತು ಕಂಠಂ ವಾಗ್ಬೀಜಪೂರ್ವಿಕಾ॥೮ ॥

ಕಾಮಬೀಜಾದಿಕಾ ಚೈಷಾ ಹೃದಯಂ ತು ಮಹೇಶ್ವರೀ ।\\
ಐಂಶ್ರೀಂಹ್ರೀಂ ಚ ನಮೋಽಂತೇ ತು ಭಗವತಿ ಪದಂ ತತಃ॥೯ ॥

ಮಾಹೇಶ್ವರಿ ಪದಂ ಚಾನ್ನಪೂರ್ಣೇ ಸ್ವಾಹೇತಿ ಪಾತು ಮೇ ।\\
ನಾಭಿಮೇಕೋನವಿಂಶಾರ್ಣಾ ಪಾಯಾನ್ಮಾಹೇಶ್ವರೀ ಸದಾ॥೧೦ ॥

ತಾರಂ ಮಾಯಾ ರಮಾ ಕಾಮಃ ಷೋಡಶಾರ್ಣಾ ತತಃ ಪರಂ ।\\
ಶಿರಃಸ್ಥಾ ಸರ್ವದಾ ಪಾತು ವಿಂಶತ್ಯರ್ಣಾತ್ಮಿಕಾ ಪರಾ॥೧೧ ॥

ಅನ್ನಪೂರ್ಣಾ ಮಹಾವಿದ್ಯಾ ಹ್ರೀಂ ಪಾತು ಭುವನೇಶ್ವರೀ ।\\
ಶಿರಃ ಶ್ರೀಂ ಹ್ರೀಂ ತಥಾ ಕ್ಲೀಂ ಚ ತ್ರಿಪುಟಾ ಪಾತು ಮೇಗುದಂ॥೧೨॥

ಷಡ್‌ದೀರ್ಘ ಭಾಜಾ ಬೀಜೇನ ಷಡಂಗಾನಿ ಪುನಂತು ಮಾಂ ।\\
ಇಂದ್ರೋ ಮಾಂ ಪಾತು ಪೂರ್ವಂ ಚ ವಹ್ನಿಕೋಣೇನಲೋವತು॥೧೩ ॥

ಯಮೋಮಾಂ ದಕ್ಷಿಣೇ ಪಾತು ನೈರ್ಋತ್ಯಾಂ ನಿರ್ಋತಿಶ್ಚ ಮಾಂ ।\\
ಪಶ್ಚಿಮೇ ವರುಣಃ ಪಾತು ವಾಯವ್ಯಾಂ ಪವನೋವತು॥೧೪ ॥

ಕುಬೇರಶ್ಚೋತ್ತರೇ ಪಾತು ಚೈಶಾನ್ಯಾಂ ಶಂಕರೋಽವತು ।\\
ಊರ್ಧ್ವಾಧಃ ಪಾತು ಸತತಂ ಬ್ರಹ್ಮಾನಂತೋ ಯಥಾಕ್ರಮಾತ್॥೧೫ ॥

ಪಾಂತು ವಜ್ರಾದ್ಯಾಯುಧಾನಿ ದಶದಿಕ್ಷು ಯಥಾಕ್ರಮಂ ।\\
ಇತಿ ತೇ ಕಥಿತಂ ಪುಣ್ಯಂ ತ್ರೈಲೋಕ್ಯ ರಕ್ಷಣಂ ಪರಂ॥೧೬ ॥

ಯದ್ಧೃತ್ವಾ ಪಠನಾದ್ದೇವಾಃ ಸರ್ವೈಶ್ವರ್ಯಮವಾಪ್ನುಯುಃ ।\\
ಬ್ರಹ್ಮಾ ವಿಷ್ಣುಶ್ಚ ರುದ್ರಶ್ಚ ಧಾರಣಾತ್ ಪಠನಾದ್ಯತಃ॥೧೭ ॥

ಸೃಜತ್ಯವತಿ ಹಂತ್ಯೇವ ಕಲ್ಪೇ ಕಲ್ಪೇ ಪೃಥಕ್ ಪೃಥಕ್ ।\\
ಪುಷ್ಪಾಂಜಲ್ಯಷ್ಟಕಂ ದೇವ್ಯೈ ಮೂಲೇನೈವ ಸಮರ್ಪಯೇತ್॥೧೮ ॥

ಕವಚಸ್ಯಾಸ್ಯ ಪಠನಾತ್ ಪೂಜಾಯಾಃ ಫಲಮಾಪ್ನುಯಾತ್ ।\\
ವಾಣೀ ವಕ್ತ್ರೇ ವಸೇತ್ತಸ್ಯ ಸತ್ಯಂ ಸತ್ಯಂ ನ ಸಂಶಯಃ॥೧೯ ॥

ಅಷ್ಟೋತ್ತರ ಶತಂ ಚಾಸ್ಯ ಪುರಶ್ಚರ್ಯಾ ವಿಧಿಃ ಸ್ಮೃತಃ ।\\
ಭೂರ್ಜೇ ವಿಲಿಖ್ಯ ಗುಟಿಕಾಂ ಸ್ವರ್ಣಸ್ಥಾಂ ಧಾರಯೇದ್ಯದಿ॥೨೦ ॥

ಕಂಠೇ ವಾ ದಕ್ಷಿಣೇ ಬಾಹೌ ಸೋಪಿ ಪುಣ್ಯವತಾಂ ವರಃ ।\\
ಸರ್ವಾಣ್ಯಸ್ತ್ರಾಣಿ ಶಸ್ತ್ರಾಣಿ ತದ್ಗಾತ್ರಂ ಪ್ರಾಪ್ಯ ಪಾರ್ವತಿ ।\\
ಮಾಲ್ಯಾನಿ ಕುಸುಮಾನೀವ ಸುಖದಾನಿ ಭವಂತಿ ಹಿ॥೨೧ ॥
\begin{center}{\Large॥ಇತಿ ಭೈರವ ತಂತ್ರೇ ಅನ್ನಪೂರ್ಣಾ ಕವಚಂ॥}\end{center}
\newpage
\section{ಸರಸ್ವತೀ ಕವಚಮ್\\ (ರುದ್ರಯಾಮಲಾಂತರ್ಗತಮ್ }

ಅಸ್ಯ ಶ್ರೀ ಸರಸ್ವತೀ ಕವಚಮಂತ್ರಸ್ಯ ಆಶ್ವಲಾಯನ ಋಷಿಃ । ಅನುಷ್ಟುಪ್ ಛಂದಃ । ಶ್ರೀ ಸರಸ್ವತೀ ದೇವತಾ । ಐಂ ಬೀಜಂ । ಹ್ರೀಂ ಶಕ್ತಿಃ । ಕ್ಲೀಂ ಕೀಲಕಂ । ಸರಸ್ವತೀ ಪ್ರಸಾದಸಿದ್ಧ್ಯರ್ಥೇ ಜಪೇ ವಿನಿಯೋಗಃ ॥

\dhyana{ದೋರ್ಭಿರ್ಯುಕ್ತಾ ಚತುರ್ಭಿಃ ಸ್ಫಟಿಕಮಣಿಮಯೀಮಕ್ಷಮಾಲಾಂದಧಾನಾ\\
ಹಸ್ತೇನೈಕೇನ ಪದ್ಮಂ ಸಿತಮಪಿ ಚ ಶುಕಂ ಪುಸ್ತಕಂ ಚಾಪರೇಣ ।\\
ಭಾಸಾ ಕುಂದೇಂದುಶಂಖಸ್ಫಟಿಕಮಣಿನಿಭಾ ಭಾಸಮಾನಾಽಸಮಾನಾ\\
ಸಾ ಮೇ ವಾಗ್ದೇವತೇಯಂ ನಿವಸತು ವದನೇ ಸರ್ವದಾ ಸುಪ್ರಸನ್ನಾ ॥}

ಸರಸ್ವತೀ ಶಿರಃ ಪಾತು ಫಾಲಂ ಫಾಲಾಕ್ಷಸೋದರೀ ।\\
ಶ್ರುತೀ ಶ್ರುತಿಮಯೀ ಪಾತು ನೇತ್ರೇರ್ಕೇಂದ್ವಗ್ನಿಲೋಚನಾ ॥೧॥

ಘ್ರಾಣಂ ಪ್ರಾಣನಿಧಿಃ ಪಾತು ಕಪೋಲೌ ಕಾಮಿತಾರ್ಥದಾ ।\\
ವಕ್ತ್ರಂ ವಿದ್ಯಾತ್ಮಿಕಾ ಪಾತು ಸ್ಕಂಧೌ ಸ್ಕಂದಸಮರ್ಚಿತಾ ॥೨॥

ಭುಜೌ ಚತುರ್ಭುಜಾ ಪಾತು ಕರೌ ಕಾಂಕ್ಷಿತದಾಯಿಕಾ ।\\
ಪಾರ್ಶ್ವೌ ಮೇ ಪಾತು ದೇವೇಶೀ ವಕ್ಷೋ ಬ್ರಹ್ಮಮುಖಾಸನಾ ॥೩॥

ಕುಕ್ಷಿಮಕ್ಷರರೂಪಾ ಚ ನಾಭಿಂ ನಾಭಿಜವಲ್ಲಭಾ ।\\
ಮಧ್ಯಂ ಸುಮಧ್ಯಮಾ ಪಾತು ಗುಹ್ಯಂ ಸರ್ವಾಂಗ ಸುಂದರೀ ॥೪॥

ಊರೂ ಮೇ ಪಾತು ವಾಗ್ದೇವೀ ಜಾನುನೀ ಜಗದೀಶ್ವರೀ ।\\
ಜಂಘೇ ಪಾತು ಮಹಾದೇವೀ ಗುಲ್ಫೌ ಮೇ ಗುಣರೂಪಿಣೀ॥೫॥

ಪಾದೌ ವೇದಾತ್ಮಿಕಾ ಪಾತು ಸರ್ವಾಂಗಂ ಮಾತೃಕಾತ್ಮಿಕಾ ।\\
ಇತೀದಂ ಕವಚಂ ದಿವ್ಯಂ ವಾಗ್ದೇವ್ಯಾಃ ಪ್ರೀತಿಕಾರಣಂ ॥೬॥

ಫಲಶ್ರುತಿಃ॥\\
ಜಡಾನಾಂ ಬುದ್ಧಿದಂ ಪುಣ್ಯಂ ಮೂಕಾನಾಂ ವಾಕ್ಪ್ರದಾಯಕಂ ।\\
ಅಂಧಾನಾಂ ದೃಷ್ಟಿದಂ ಚೈವ ಸರ್ವಜ್ಞಾನಪ್ರದಾಯಕಂ ॥೭॥

ವಾಗ್ವಶ್ಯಜನಕಂ ನೄಣಾಂ ತಥಾ ಭೂ-ಪಾಲಮೋಹನಂ ।\\
ವಾಕ್‌ಸ್ತಂಭಕಾರಕಂ ಚೈವ ಸಭಾಯಾಂ ಪ್ರತಿವಾದಿನಾಂ ॥೮॥

ಪುತ್ರಪ್ರದಮಪುತ್ರಾಣಾಂ ಧನದಂ ಧನಕಾಮಿನಾಂ ।\\
ಮೋಕ್ಷದಂ ಮೋಕ್ಷಕಾಮಾನಾಂ ಮಂತ್ರಸಿದ್ಧಿ ಪ್ರದಾಯಕಂ ॥೯॥

ಬಹುನಾ ಕಿಮಿಹೋಕ್ತೇನ ಸತ್ಯಂ ಸತ್ಯಂ ಮುನೀಶ್ವರ ।\\
ಆಶ್ವಲಾಯನಸಂಪ್ರೊಕ್ತಂ ಷಣ್ಮಾಸಂ ಜಪತಾಂ ನೃಣಾಂ ॥೧೦॥

ಕವಿತಾವಾಕ್ಪಟುತ್ವಂ ಚ ಜಾಯತೇ ನಾತ್ರ ಸಂಶಯಃ ।\\
ಶುಕ್ರವಾರೇ ವಿಶೇಷೇಣ ಜಪ್ತವ್ಯಂ ದೇವಿ ಸಾಧಕೈಃ ॥೧೧॥

ಸತ್ಯಂ ಸಾರಸ್ವತಂ ಚೈವ ಸ್ಥಿರತಾಮೇತಿ ತತ್ಕುಲೇ ।\\
ಪೌರ್ಣಮಾಸ್ಯಾಮಮಾವಸ್ಯಾಂ ದೇವಿ ಸಾರಸ್ವತೇ ತಥಾ ॥೧೨॥

ಯೋಗೇ ವಿಶೇಷೇ ಜಪ್ತವ್ಯಂ ವಿದ್ಯಾರ್ಥಿಭಿರತಂದ್ರಿತೈಃ ।\\
ಅನೇನ ಕವಚೇನೈವ ನ್ಯಸ್ತಾಂಗೋ ಮೂಲಮಂತ್ರಕಂ ॥೧೩॥

ಅಷ್ಟೋತ್ತರಶತಂ ಜಪ್ತ್ವಾ  ವಾಚಂ ಚೈತಾಂ ಚ ಭಕ್ಷಯೇತ್ ।\\
ಪ್ರಾತಃ ಕಾಲೇ ತು ಮಾಸೈಕಂ ವಾಕ್ಸಿದ್ಧಿರತುಲಾ ಭವೇತ್ ॥೧೪॥

ಗೋಮಯೇನ ಮೃದಾ ವಾಪಿ ನಿರ್ಮಾಯ ಪ್ರತಿವಾದಿನಮ್ ।\\
ವಾಮಪಾದೇನ ಚಾಕ್ರಮ್ಯ ತಜ್ಜಿಹ್ವಾಂ ಕವಚಂ ಜಪೇತ್ ॥೧೫॥

ಮೂಕೋ ವೈ ಜಾಯತೇ ಶೀಘ್ರಮುನ್ಮತ್ತೋ ವಾ ಭವೇದ್ಧ್ರುವಮ್ ।\\
ಯಂ ಯಂ ಕಾಮಯತೇ ಕಾಮಂ ತಂತಮುದ್ದಿಶ್ಯ ಪಾರ್ವತಿ ॥೧೬॥

ಅಷ್ಟೋತ್ತರಶತಂ ಜಪ್ತ್ವಾ ಫಲಂ ವಿಂದತಿ ಮಾನವಃ ।\\
ಅಶ್ವತ್ಥೇ ರಾಜವಶ್ಯಾರ್ಥೀ ತೇಜಸೇಽಭಿಮುಖೋ ರವೇಃ ॥೧೭॥

ಕನ್ಯಾರ್ಥೀ ಚಂಡಿಕಾಗೇಹೇ ಗೇಹೇ ಶತ್ರುಕೃತೇ ಮಮ ।\\
ಶ್ರೀಕಾಮೋ ಬಿಲ್ವಮೂಲೇ ತು ಉದ್ಯಾನೇ ಸ್ತ್ರೀವಶೀ ಭವೇತ್ ।\\
ಪುತ್ರಾರ್ಥೀ ದಕ್ಷಿಣಾಮೂರ್ತೇಃ ಸನ್ನಿಧೌ ಮಮ ಪಾರ್ವತೀ ॥೧೮॥
\authorline{ಇತಿ ಶ್ರೀರುದ್ರಯಾಮಲೇ ಉಮಾಮಹೇಶ್ವರಸಂವಾದೇ \\ಸರಸ್ವತೀಕವಚಂ ಸಂಪೂರ್ಣಂ ॥}\authorline{ಇತಿ ಶ್ರೀರುದ್ರಯಾಮಲೇ ಉಮಾಮಹೇಶ್ವರಸಂವಾದೇ \\ಸರಸ್ವತೀಕವಚಂ ಸಂಪೂರ್ಣಂ ॥}
\section{ಶ್ರೀ ಸರಸ್ವತೀ ಕವಚಂ \\(ಬ್ರಹ್ಮವೈವರ್ತಾಂತರ್ಗತಂ)}
ಬ್ರಹ್ಮೋವಾಚ~।\\
ಶೃಣು ವತ್ಸ ಪ್ರವಕ್ಷ್ಯಾಮಿ ಕವಚಂ ಸರ್ವಕಾಮದಂ~।\\
ಶ್ರುತಿಸಾರಂ ಶ್ರುತಿಸುಖಂ ಶ್ರುತ್ಯುಕ್ತಂ ಶ್ರುತಿಪೂಜಿತಂ ॥೧॥

ಉಕ್ತಂ ಕೃಷ್ಣೇನ ಗೋಲೋಕೇ ಮಹ್ಯಂ ವೃಂದಾವನೇ ವನೇ~।\\
ರಾಸೇಶ್ವರೇಣ ವಿಭುನಾ ರಾಸೇ ವೈ ರಾಸಮಂಡಲೇ ॥೨॥

ಅತೀವ ಗೋಪನೀಯಂ ಚ ಕಲ್ಪವೃಕ್ಷಸಮಂ ಪರಂ~।\\
ಅಶ್ರುತಾದ್ಭುತಮಂತ್ರಾಣಾಂ ಸಮೂಹೈಶ್ಚ ಸಮನ್ವಿತಂ ॥೩॥

ಯದ್ಧೃತ್ವಾ ಪಠನಾದ್ಬ್ರಹ್ಮನ್ಬುದ್ಧಿಮಾಂಶ್ಚ ಬೃಹಸ್ಪತಿಃ~।\\
ಯದ್ಧೃತ್ವಾ ಭಗವಾಂಛುಕ್ರಃ ಸರ್ವದೈತ್ಯೇಷು ಪೂಜಿತಃ ॥೪॥

ಪಠನಾದ್ಧಾರಣಾದ್ವಾಗ್ಮೀ ಕವೀಂದ್ರೋ ವಾಲ್ಮಿಕೋ ಮುನಿಃ~।\\
ಸ್ವಾಯಂಭುವೋ ಮನುಶ್ಚೈವ ಯದ್ಧೃತ್ವಾ ಸರ್ವಪೂಜಿತಃ ॥೫॥

ಕಣಾದೋ ಗೌತಮಃ ಕಣ್ವಃ ಪಾಣಿನಿಃ ಶಾಕಟಾಯನಃ~।\\
ಗ್ರಂಥಂ ಚಕಾರ ಯದ್ಧೃತ್ವಾ ದಕ್ಷಃ ಕಾತ್ಯಾಯನಃ ಸ್ವಯಂ ॥೬॥

ಧೃತ್ವಾ ವೇದವಿಭಾಗಂ ಚ ಪುರಾಣಾನ್ಯಖಿಲಾನಿ ಚ~।\\
ಚಕಾರ ಲೀಲಾಮಾತ್ರೇಣ ಕೃಷ್ಣದ್ವೈಪಾಯನಃ ಸ್ವಯಂ ॥೭॥

ಶಾತಾತಪಶ್ಚ ಸಂವರ್ತೋ ವಸಿಷ್ಠಶ್ಚ ಪರಾಶರಃ~।\\
ಯದ್ಧೃತ್ವಾ ಪಠನಾದ್ಗ್ರಂಥಂ ಯಾಜ್ಞವಲ್ಕ್ಯಶ್ಚಕಾರ ಸಃ ॥೮॥

ಋಷ್ಯಶೃಂಗೋ ಭರದ್ವಾಜಶ್ಚಾಽಽಸ್ತೀಕೋ ದೇವಲಸ್ತಥಾ~।\\
ಜೈಗೀಷವ್ಯೋಽಥ ಜಾಬಾಲಿರ್ಯದ್ಧೃತ್ವಾ ಸರ್ವಪೂಜಿತಃ ॥೯॥

ಕವಚಸ್ಯಾಸ್ಯ ವಿಪ್ರೇಂದ್ರ ಋಷಿರೇಷ ಪ್ರಜಾಪತಿಃ~।\\
ಸ್ವಯಂ ಬೃಹಸ್ಪತಿಶ್ಛಂದೋ ದೇವೋ ರಾಸೇಶ್ವರಃ ಪ್ರಭುಃ ॥೧೦॥

ಸರ್ವತತ್ತ್ವಪರಿಜ್ಞಾನೇ ಸರ್ವಾರ್ಥೇಽಪಿ ಚ ಸಾಧನೇ~।\\
ಕವಿತಾಸು ಚ ಸರ್ವಾಸು ವಿನಿಯೋಗಃ ಪ್ರಕೀರ್ತಿತಃ ॥೧೧॥

ಓಂ ಹ್ರೀಂ ಸರಸ್ವತ್ಯೈ ಸ್ವಾಹಾ ಶಿರೋ ಮೇ ಪಾತು ಸರ್ವತಃ~।\\
ಶ್ರೀಂ ವಾಗ್ದೇವತಾಯೈ ಸ್ವಾಹಾ ಭಾಲಂ ಮೇ ಸರ್ವದಾಽವತು ॥೧೨॥

ಓಂ ಸರಸ್ವತ್ಯೈ ಸ್ವಾಹೇತಿ ಶ್ರೋತ್ರಂ ಪಾತು ನಿರಂತರಂ~।\\
ಓಂ ಶ್ರೀಂ ಹ್ರೀಂ ಭಾರತ್ಯೈ ಸ್ವಾಹಾ ನೇತ್ರಯುಗ್ಮಂ ಸದಾಽವತು ॥೧೩॥

ಓಂ ಹ್ರೀಂ ವಾಗ್ವಾದಿನ್ಯೈ ಸ್ವಾಹಾ ನಾಸಾಂ ಮೇ ಸರ್ವತೋಽವತು~।\\
ಹ್ರೀಂ ವಿದ್ಯಾಧಿಷ್ಠಾತೃದೇವ್ಯೈ ಸ್ವಾಹಾ ಶ್ರೋತ್ರಂ ಸದಾಽವತು ॥೧೪॥

ಓಂ ಶ್ರೀಂ ಹ್ರೀಂ ಬ್ರಾಹ್ಮ್ಯೈ ಸ್ವಾಹೇತಿ ದಂತಪಂಕ್ತೀಃ ಸದಾಽವತು~।\\
ಐಮಿತ್ಯೇಕಾಕ್ಷರೋ ಮಂತ್ರೋ ಮಮ ಕಂಠಂ ಸದಾಽವತು ॥೧೫॥

ಓಂಶ್ರೀಂಹ್ರೀಂ ಪಾತುಮೇ ಗ್ರೀವಾಂ ಸ್ಕಂಧಂಮೇ ಶ್ರೀಂ ಸದಾಽವತು~।\\
ಶ್ರೀಂ ವಿದ್ಯಾಧಿಷ್ಠಾತೃದೇವ್ಯೈ ಸ್ವಾಹಾ ವಕ್ಷಃ ಸದಾಽವತು ॥೧೬॥

ಓಂ ಹ್ರೀಂ ವಿದ್ಯಾಸ್ವರೂಪಾಯೈ ಸ್ವಾಹಾ ಮೇ ಪಾತು ನಾಭಿಕಾಂ~।\\
ಓಂ ಹ್ರೀಂ ಹ್ರೀಂ ವಾಣ್ಯೈ ಸ್ವಾಹೇತಿ ಮಮ ಪೃಷ್ಠಂ ಸದಾಽವತು ॥೧೭॥

ಓಂ ಸರ್ವವರ್ಣಾತ್ಮಿಕಾಯೈ ಪಾದಯುಗ್ಮಂ ಸದಾಽವತು~।\\
ಓಂ ರಾಗಧಿಷ್ಠಾತೃದೇವ್ಯೈ ಸರ್ವಾಂಗಂ ಮೇ ಸದಾಽವತು ॥೧೮॥

ಓಂ ಸರ್ವಕಂಠವಾಸಿನ್ಯೈ ಸ್ವಾಹಾ ಪ್ರಾಚ್ಯಾಂ ಸದಾಽವತು~।\\
ಓಂ ಹ್ರೀಂ ಜಿಹ್ವಾಗ್ರವಾಸಿನ್ಯೈ ಸ್ವಾಹಾಽಗ್ನಿದಿಶಿ ರಕ್ಷತು ॥೧೯॥

ಓಂ ಐಂ ಶ್ರೀಂ ಹ್ರೀಂ ಸರಸ್ವತ್ಯೈ ಬುಧಜನನ್ಯೈ ಸ್ವಾಹಾ~।\\
ಸತತಂ ಮಂತ್ರರಾಜೋಽಯಂ ದಕ್ಷಿಣೇ ಮಾಂ ಸದಾಽವತು ॥೨೦॥

ಓಂ ಹ್ರೀಂ ಶ್ರೀಂ ತ್ರ್ಯಕ್ಷರೋ ಮಂತ್ರೋ ನೈರೃತ್ಯಾಂ ಮೇ ಸದಾಽವತು~।\\
ಕವಿಜಿಹ್ವಾಗ್ರವಾಸಿನ್ಯೈ ಸ್ವಾಹಾ ಮಾಂ ವಾರುಣೇಽವತು ॥೨೧॥

ಓಂ ಸದಂಬಿಕಾಯೈ ಸ್ವಾಹಾ ವಾಯವ್ಯೇ ಮಾಂ ಸದಾಽವತು~।\\
ಓಂ ಗದ್ಯಪದ್ಯವಾಸಿನ್ಯೈ ಸ್ವಾಹಾ ಮಾಮುತ್ತರೇಽವತು ॥೨೨॥

ಓಂ ಸರ್ವಶಾಸ್ತ್ರವಾಸಿನ್ಯೈ ಸ್ವಾಹೈಶಾನ್ಯಾಂ ಸದಾಽವತು~।\\
ಓಂ ಹ್ರೀಂ ಸರ್ವಪೂಜಿತಾಯೈ ಸ್ವಾಹಾ ಚೋರ್ಧ್ವಂ ಸದಾಽವತು ॥೨೩॥

ಐಂ ಹ್ರೀಂ ಪುಸ್ತಕವಾಸಿನ್ಯೈ ಸ್ವಾಹಾಽಧೋ ಮಾಂ ಸದಾಽವತು~।\\
ಓಂ ಗ್ರಂಥಬೀಜರೂಪಾಯೈ ಸ್ವಾಹಾ ಮಾಂ ಸರ್ವತೋಽವತು ॥೨೪॥

ಇತಿ ತೇ ಕಥಿತಂ ವಿಪ್ರ ಸರ್ವಮಂತ್ರೌಘವಿಗ್ರಹಂ~।\\
ಇದಂ ವಿಶ್ವಜಯಂ ನಾಮ ಕವಚಂ ಬ್ರಹ್ಮರೂಪಕಂ ॥೨೫॥

ಪುರಾ ಶ್ರುತಂ ಧರ್ಮವಕ್ತ್ರಾತ್ಪರ್ವತೇ ಗಂಧಮಾದನೇ~।\\
ತವ ಸ್ನೇಹಾನ್ಮಯಾಽಽಖ್ಯಾತಂ ಪ್ರವಕ್ತವ್ಯಂ ನ ಕಸ್ಯಚಿತ್ ॥೨೬॥

ಗುರುಮಭ್ಯರ್ಚ್ಯ ವಿಧಿವದ್ವಸ್ತ್ರಾಲಂಕಾರಚಂದನೈಃ~।\\
ಪ್ರಣಮ್ಯ ದಂಡವದ್ಭೂಮೌ ಕವಚಂ ಧಾರಯೇತ್ಸುಧೀಃ ॥೨೭॥

ಪಂಚಲಕ್ಷಜಪೇನೈವ ಸಿದ್ಧಂ ತು ಕವಚಂ ಭವೇತ್~।\\
ಯದಿ ಸ್ಯಾತ್ಸಿದ್ಧಕವಚೋ ಬೃಹಸ್ಪತಿಸಮೋ ಭವೇತ್ ॥೨೮॥

ಮಹಾವಾಗ್ಮೀ ಕವೀಂದ್ರಶ್ಚ ತ್ರೈಲೋಕ್ಯವಿಜಯೀ ಭವೇತ್~।\\
ಶಕ್ನೋತಿ ಸರ್ವಂ ಜೇತುಂ ಸ ಕವಚಸ್ಯ ಪ್ರಭಾವತಃ ॥೨೯॥

ಇದಂ ತೇ ಕಾಣ್ವಶಾಖೋಕ್ತಂ ಕಥಿತಂ ಕವಚಂ ಮುನೇ~।\\
ಸ್ತೋತ್ರಂ ಪೂಜಾವಿಧಾನಂ ಚ ಧ್ಯಾನಂ ವೈ ವಂದನಂ ತಥಾ ॥೩೦॥

\authorline{॥ಇತಿ ಶ್ರೀ ಬ್ರಹ್ಮವೈವರ್ತೇ ಮಹಾಪುರಾಣೇ ಪ್ರಕೃತಿಖಂಡೇ ನಾರದನಾರಾಯಣಸಂವಾದೇ \\ಸರಸ್ವತೀಕವಚಂ ನಾಮ ಚತುರ್ಥೋಽಧ್ಯಾಯಃ ॥}
\section{ಶ್ರೀಗಣೇಶಕವಚಂ }
ಗೌರ್ಯುವಾಚ~॥\\
ಏಷೋಽತಿಚಪಲೋ ದೈತ್ಯಾನ್ಬಾಲ್ಯೇಽಪಿ ನಾಶಯತ್ಯಹೋ~।\\
ಅಗ್ರೇ ಕಿಂ ಕರ್ಮ ಕರ್ತೇತಿ ನ ಜಾನೇ ಮುನಿಸತ್ತಮ ॥೧॥

ದೈತ್ಯಾ ನಾನಾವಿಧಾ ದುಷ್ಟಾಃ ಸಾಧುದೇವದ್ರುಹಃ ಖಲಾಃ~।\\
ಅತೋಽಸ್ಯ ಕಂಠೇ ಕಿಂಚಿತ್ತ್ವಂ ರಕ್ಷಾರ್ಥಂ ಬದ್ಧುಮರ್ಹಸಿ ॥೨॥

ಮುನಿರುವಾಚ~॥\\
\dhyana{ಧ್ಯಾಯೇತ್ಸಿಂಹಗತಂ ವಿನಾಯಕಮಮುಂ ದಿಗ್ಬಾಹುಮಾದ್ಯೇ ಯುಗೇ\\
ತ್ರೇತಾಯಾಂ ತು ಮಯೂರವಾಹನಮಮುಂ ಷಡ್ಬಾಹುಕಂ ಸಿದ್ಧಿದಂ~।\\
ದ್ವಾಪಾರೇ ತು ಗಜಾನನಂ ಯುಗಭುಜಂ ರಕ್ತಾಂಗರಾಗಂ ವಿಭುಂ\\
ತುರ್ಯೇ ತು ದ್ವಿಭುಜಂ ಸಿತಾಂಗರುಚಿರಂ ಸರ್ವಾರ್ಥದಂ ಸರ್ವದಾ ॥೩॥}

ವಿನಾಯಕಃ ಶಿಖಾಂ ಪಾತು ಪರಮಾತ್ಮಾ ಪರಾತ್ಪರಃ~।\\
ಅತಿಸುಂದರಕಾಯಸ್ತು ಮಸ್ತಕಂ ಸುಮಹೋತ್ಕಟಃ ॥೪॥

ಲಲಾಟಂ ಕಶ್ಯಪಃ ಪಾತು ಭ್ರೂಯುಗಂ ತು ಮಹೋದರಃ~।\\
ನಯನೇ ಭಾಲಚಂದ್ರಸ್ತು ಗಜಾಸ್ಯಸ್ತ್ವೋಷ್ಠಪಲ್ಲವೌ ॥೫॥

ಜಿಹ್ವಾಂ ಪಾತು ಗಣಕ್ರೀಡಶ್ಚಿಬುಕಂ ಗಿರಿಜಾಸುತಃ~।\\
ವಾಚಂ ವಿನಾಯಕಃ ಪಾತು ದಂತಾನ್ ರಕ್ಷತು ವಿಘ್ನಹಾ ॥೬॥

ಶ್ರವಣೌ ಪಾಶಪಾಣಿಸ್ತು ನಾಸಿಕಾಂ ಚಿಂತಿತಾರ್ಥದಃ~।\\
ಗಣೇಶಸ್ತು ಮುಖಂ ಕಂಠಂ ಪಾತು ದೇವೋ ಗಣಂಜಯಃ ॥೭॥

ಸ್ಕಂಧೌ ಪಾತು ಗಜಸ್ಕಂಧಃ ಸ್ತನೌ ವಿಘ್ನವಿನಾಶನಃ~।\\
ಹೃದಯಂ ಗಣನಾಥಸ್ತು ಹೇರಂಬೋ ಜಠರಂ ಮಹಾನ್ ॥೮॥

ಧರಾಧರಃ ಪಾತು ಪಾರ್ಶ್ವೌ ಪೃಷ್ಠಂ ವಿಘ್ನಹರಃ ಶುಭಃ~।\\
ಲಿಂಗಂ ಗುಹ್ಯಂ ಸದಾ ಪಾತು ವಕ್ರತುಂಡೋ ಮಹಾಬಲಃ ॥೯॥

ಗಣಕ್ರೀಡೋ ಜಾನುಜಂಘೇ ಊರೂ ಮಂಗಲಮೂರ್ತಿಮಾನ್~।\\
ಏಕದಂತೋ ಮಹಾಬುದ್ಧಿಃ ಪಾದೌ ಗುಲ್ಫೌ ಸದಾಽವತು ॥೧೦॥

ಕ್ಷಿಪ್ರಪ್ರಸಾದನೋ ಬಾಹೂ ಪಾಣೀ ಆಶಾಪ್ರಪೂರಕಃ~।\\
ಅಂಗುಲೀಶ್ಚ ನಖಾನ್ಪಾತು ಪದ್ಮಹಸ್ತೋಽರಿನಾಶನಃ ॥೧೧॥

ಸರ್ವಾಂಗಾನಿ ಮಯೂರೇಶೋ ವಿಶ್ವವ್ಯಾಪೀ ಸದಾಽವತು~।\\
ಅನುಕ್ತಮಪಿ ಯತ್ಸ್ಥಾನಂ ಧೂಮ್ರಕೇತುಃ ಸದಾಽವತು ॥೧೨॥

ಆಮೋದಸ್ತ್ವಗ್ರತಃ ಪಾತು ಪ್ರಮೋದಃ ಪೃಷ್ಠತೋಽವತು~।\\
ಪ್ರಾಚ್ಯಾಂ ರಕ್ಷತು ಬುದ್ಧೀಶ ಆಗ್ನೇಯ್ಯಾಂ ಸಿದ್ಧಿದಾಯಕಃ ॥೧೩॥

ದಕ್ಷಿಣಾಸ್ಯಾಮುಮಾಪುತ್ರೋ ನೈರೃತ್ಯಾಂ ತು ಗಣೇಶ್ವರಃ~।\\
ಪ್ರತೀಚ್ಯಾಂ ವಿಘ್ನಹರ್ತಾಽವ್ಯಾದ್ವಾಯವ್ಯಾಂ ಗಜಕರ್ಣಕಃ ॥೧೪॥

ಕೌಬೇರ್ಯಾಂ ನಿಧಿಪಃ ಪಾಯಾದೀಶಾನ್ಯಾಮೀಶನಂದನಃ~।\\
ದಿವಾಽವ್ಯಾದೇಕದಂತಸ್ತು ರಾತ್ರೌ ಸಂಧ್ಯಾಸು ವಿಘ್ನಹೃತ್ ॥೧೫॥

ರಾಕ್ಷಸಾಸುರವೇತಾಲಗ್ರಹಭೂತಪಿಶಾಚತಃ~।\\
ಪಾಶಾಂಕುಶಧರಃ ಪಾತು ರಜಃಸತ್ತ್ವತಮಃಸ್ಮೃತೀಃ ॥೧೬॥

ಜ್ಞಾನಂ ಧರ್ಮಂ ಚ ಲಕ್ಷ್ಮೀಂ ಚ ಲಜ್ಜಾಂ ಕೀರ್ತಿಂ ತಥಾ ಕುಲಂ~।\\
ವಪುರ್ಧನಂ ಚ ಧಾನ್ಯಂ ಚ ಗೃಹಾನ್ ದಾರಾನ್ಸುತಾನ್ಸಖೀನ್ ॥೧೭॥

ಸರ್ವಾಯುಧಧರಃ ಪೌತ್ರಾನ್ ಮಯೂರೇಶೋಽವತಾತ್ಸದಾ~।\\
ಕಪಿಲೋಽಜಾದಿಕಂ ಪಾತು ಗಜಾಶ್ವಾನ್ವಿಕಟೋಽವತು ॥೧೮॥

ಭೂರ್ಜಪತ್ರೇ ಲಿಖಿತ್ವೇದಂ ಯಃ ಕಂಠೇ ಧಾರಯೇತ್ಸುಧೀಃ~।\\
ನ ಭಯಂ ಜಾಯತೇ ತಸ್ಯ ಯಕ್ಷರಕ್ಷಃಪಿಶಾಚತಃ ॥೧೯॥

ತ್ರಿಸಂಧ್ಯಂ ಜಪತೇ ಯಸ್ತು ವಜ್ರಸಾರತನುರ್ಭವೇತ್~।\\
ಯಾತ್ರಾಕಾಲೇ ಪಠೇದ್ಯಸ್ತು ನಿರ್ವಿಘ್ನೇನ ಫಲಂ ಲಭೇತ್ ॥೨೦॥

ಯುದ್ಧಕಾಲೇ ಪಠೇದ್ಯಸ್ತು ವಿಜಯಂ ಚಾಪ್ನುಯಾದ್ದ್ರುತಂ~।\\
ಮಾರಣೋಚ್ಚಾಟನಾಕರ್ಷಸ್ತಂಭಮೋಹನಕರ್ಮಣಿ ॥೨೧॥

ಸಪ್ತವಾರಂ ಜಪೇದೇತದ್ದಿನಾನಾಮೇಕವಿಂಶತಿಂ~।\\
ತತ್ತತ್ಫಲಮವಾಪ್ನೋತಿ ಸಾಧಕೋ ನಾತ್ರಸಂಶಯಃ ॥೨೨॥

ಏಕವಿಂಶತಿವಾರಂ ಚ ಪಠೇತ್ತಾವದ್ದಿನಾನಿ ಯಃ~।\\
ಕಾರಾಗೃಹಗತಂ ಸದ್ಯೋ ರಾಜ್ಞಾ ವಧ್ಯಂ ಚ ಮೋಚಯೇತ್ ॥೨೩॥

ರಾಜದರ್ಶನವೇಲಾಯಾಂ ಪಠೇದೇತತ್ತ್ರಿವಾರತಃ~।\\
ಸ ರಾಜಾನಂ ವಶಂ ನೀತ್ವಾ ಪ್ರಕೃತೀಶ್ಚ ಸಭಾಂ ಜಯೇತ್ ॥೨೪॥

ಇದಂ ಗಣೇಶಕವಚಂ ಕಶ್ಯಪೇನ ಸಮೀರಿತಂ~।\\
ಮುದ್ಗಲಾಯ ಚ ತೇ ನಾಥ ಮಾಂಡವ್ಯಾಯ ಮಹರ್ಷಯೇ ॥೨೫॥

ಮಹ್ಯಂ ಸ ಪ್ರಾಹ ಕೃಪಯಾ ಕವಚಂ ಸರ್ವಸಿದ್ಧಿದಂ~।\\
ನ ದೇಯಂ ಭಕ್ತಿಹೀನಾಯ ದೇಯಂ ಶ್ರದ್ಧಾವತೇ ಶುಭಂ ॥೨೬॥

ಯಸ್ಯಾನೇನ ಕೃತಾ ರಕ್ಷಾ ನ ಬಾಧಾಸ್ಯ ಭವೇತ್ಕ್ವಚಿತ್~।\\
ರಾಕ್ಷಸಾಸುರವೇತಾಲದೈತ್ಯದಾನವಸಂಭವಾ ॥೨೭॥

\authorline {ಇತಿ ಶ್ರೀಗಣೇಶಪುರಾಣೇ ಗಣೇಶಕವಚಂ ಸಂಪೂರ್ಣಂ ॥}
\section{ಶ್ರೀಬಾಲಾತ್ರಿಪುರಸುಂದರೀ ಕವಚಂ}
ಶ್ರೀ ಪಾರ್ವತ್ಯುವಾಚ ॥\\
ದೇವ ದೇವ ಮಹಾದೇವ ಶಂಕರ ಪ್ರಾಣ ವಲ್ಲಭ~।\\
ಕವಚಂ ಶ್ರೋತುಮಿಚ್ಛಾಮಿ ಬಾಲಾಯಾ ವದ ಮೇ ಪ್ರಭೋ ॥೧॥

ಶ್ರೀ ಮಹೇಶ್ವರ ಉವಾಚ ॥\\
ಶ್ರೀಬಾಲಾಕವಚಂ ದೇವಿ ಮಹಾಪ್ರಾಣಾಧಿಕಂ ಪರಂ~।\\
ವಕ್ಷ್ಯಾಮಿ ಸಾವಧಾನಾ ತ್ವಂ ಶೃಣುಷ್ವಾವಹಿತಾ ಪ್ರಿಯೇ ॥೨॥

ಅಸ್ಯ ಶ್ರೀಬಾಲಾಕವಚಸ್ತೋತ್ರ ಮಹಾಮಂತ್ರಸ್ಯ ಶ್ರೀ ದಕ್ಷಿಣಾಮುರ್ತಿರ್ಋಷಿಃ~। ಪಂಕ್ತಿಶ್ಛಂದಃ। ಬಾಲಾತ್ರಿಪುರಸುಂದರೀ ದೇವತಾ। ಐಂ ಬೀಜಂ। ಸೌಃ ಶಕ್ತಿಃ। ಕ್ಲೀಂ ಕೀಲಕಂ। ಶ್ರೀಬಾಲಾತ್ರಿಪುರಸುಂದರೀದೇವತಾ ಪ್ರಸಾದಸಿದ್ಧ್ಯರ್ಥೇ ಜಪೇ ವಿನಿಯೋಗಃ॥

\dhyana{ಅರುಣ ಕಿರಣ ಜಾಲೈರಂಚಿತಾಶಾವಕಾಶಾ\\
ವಿಧೃತಜಪವಟೀಕಾ ಪುಸ್ತಕಾಭೀತಿಹಸ್ತಾ~।\\
ಇತರಕರವರಾಢ್ಯಾ ಫುಲ್ಲಕಹ್ಲಾರಸಂಸ್ಥಾ\\
ನಿವಸತು ಹೃದಿ ಬಾಲಾ ನಿತ್ಯಕಲ್ಯಾಣಶೀಲಾ }॥೩॥

(ಐಂ)ವಾಗ್ಭವಃ ಪಾತು ಶೀರ್ಷೇ (ಕ್ಲೀಂ) ಕಾಮರಾಜಸ್ತಥಾ ಹೃದಿ~।\\
ಸೌಃ ಶಕ್ತಿಬೀಜಂ ಮಾಂ ಪಾತು ನಾಭೌ ಗುಹ್ಯೇ ಚ ಪಾದಯೋಃ ॥೪॥

ಐಂ ಕ್ಲೀಂ ಸೌಃ ವದನೇ ಪಾತು ಬಾಲಾ ಮಾಂ ಸರ್ವಸಿದ್ಧಯೇ~।\\
ಹ್‌ಸ್‌ರೈಂ ಹ್‌ಸ್‌ಕ್ಲ್ರೀಂ ಹ್‌ಸ್‌ರ್‌ಸೌಃ ಪಾತು ಸ್ಕಂಧೇ ಭೈರವೀ ಕಂಠದೇಶತಃ ॥೫॥

ಸುಂದರೀ ನಾಸದೇಶೇವ್ಯಾಚ್ಚರ್ಚೇ ಕಾಮಕಲಾ ಸದಾ~।\\
ಭ್ರೂನಾಸಯೋರಂತರಾಲೇ ಮಹಾತ್ರಿಪುರಸುಂದರೀ ॥೬॥

ಲಲಾಟೇ ಸುಭಗಾ ಪಾತು ಭಗಾ ಮಾಂ ಕಂಠದೇಶತಃ~।\\
ಭಗೋದಯಾ ತು ಹೃದಯೇ ಉದರೇ ಭಗಸರ್ಪಿಣೀ ॥೭॥

ಭಗಮಾಲಾ ನಾಭಿದೇಶೇ ಲಿಂಗೇ ಪಾತು ಮನೋಭವಾ~।\\
ಗುಹ್ಯೇ ಪಾತು ಮಹಾವೀರಾ ರಾಜರಾಜೇಶ್ವರೀ ಶಿವಾ ॥೮॥

ಚೈತನ್ಯರೂಪಿಣೀ ಪಾತು ಪಾದಯೋರ್ಜಗದಂಬಿಕಾ~।\\
ನಾರಾಯಣೀ ಸರ್ವಗಾತ್ರೇ ಸರ್ವಕಾರ್ಯಶುಭಂಕರೀ ॥೯॥

ಬ್ರಹ್ಮಾಣೀ ಪಾತು ಮಾಂ ಪೂರ್ವೇ ದಕ್ಷಿಣೇ ವೈಷ್ಣವೀ ತಥಾ~।\\
ಪಶ್ಚಿಮೇ ಪಾತು ವಾರಾಹೀ ಹ್ಯುತ್ತರೇ ತು ಮಹೇಶ್ವರೀ ॥೧೦॥

ಆಗ್ನೇಯ್ಯಾಂ ಪಾತು ಕೌಮಾರೀ ಮಹಾಲಕ್ಷ್ಮೀಶ್ಚ ನೈರ್ಋತೇ~।\\
ವಾಯವ್ಯೇ ಪಾತು ಚಾಮುಂಡಾ ಚೇಂದ್ರಾಣೀ ಪಾತು ಚೇಶಕೇ ॥೧೧॥

ಜಲೇ ಪಾತು ಮಹಾಮಾಯಾ ಪೃಥಿವ್ಯಾಂ ಸರ್ವಮಂಗಲಾ~।\\
ಸ್ಕ್ಲೀಂ ಮಾಂ ಸರ್ವತಃ ಪಾತು ಸಕಲಹ್ರೀಂ ಪಾತು ಸಂಧಿಷು ॥೧೨॥

ಜಲೇ ಸ್ಥಲೇ ತಥಾಕಾಶೇ ದಿಕ್ಷು ರಾಜಗೃಹೇ ತಥಾ~।\\
ಕ್ಷೂಂಕ್ಷೇಂಮಾಂ ತ್ವರಿತಾಪಾತು ಸಹ್ರೀಂಸಕ್ಲೀಂ ಮನೋಭವಾ ॥೧೩॥

ಹಂಸಃ ಪಾಯಾನ್ಮಹಾದೇವೀ ಪರಂ ನಿಷ್ಕಲದೇವತಾ~।\\
ವಿಜಯಾ ಮಂಗಲಾ ದೂತೀ ಕಲ್ಯಾಣೀ ಭಗಮಾಲಿನೀ ॥೧೪॥

ಜ್ವಾಲಾಮಾಲಿನಿನಿತ್ಯಾ ಚ ಸರ್ವದಾ ಪಾತು ಮಾಂ ಶಿವಾ~।\\
ಇತ್ಯೇತತ್ಕವಚಂ ದೇವಿ ಬಾಲಾದೇವ್ಯಾಃ ಪ್ರಕೀರ್ತಿತಂ~।\\
ಸರ್ವಸ್ವಂ ಮೇ ತವ ಪ್ರೀತ್ಯಾ ಪ್ರಾಣವದ್ರಕ್ಷಿತಂ ಕುರು॥೧೫॥

\authorline{ಇತಿ ಶ್ರೀರುದ್ರಯಾಮಲೇ ಶ್ರೀಶಿವಪಾರ್ವತಿಸಂವಾದೇ\\ ಶ್ರೀಬಾಲಾತ್ರಿಪುರಸುಂದರೀ ಮಂತ್ರಕವಚಂ ಸಂಪೂರ್ಣಂ ॥}
(((edited
\section{ಶ್ರೀಬಾಲಾತ್ರಿಪುರಸುಂದರೀ ಕವಚಂ}

ಅಸ್ಯ ಶ್ರೀಬಾಲಾಕವಚಸ್ತೋತ್ರ ಮಹಾಮಂತ್ರಸ್ಯ ಶ್ರೀ ದಕ್ಷಿಣಾಮುರ್ತಿರ್ಋಷಿಃ~। ಪಂಕ್ತಿಶ್ಛಂದಃ~। ಬಾಲಾತ್ರಿಪುರಸುಂದರೀ ದೇವತಾ~। ಐಂ ಬೀಜಂ~। ಸೌಃ ಶಕ್ತಿಃ~। ಕ್ಲೀಂ ಕೀಲಕಂ~। ಶ್ರೀಬಾಲಾತ್ರಿಪುರಸುಂದರೀದೇವತಾ ಪ್ರಸಾದಸಿದ್ಧ್ಯರ್ಥೇ ಜಪೇ ವಿನಿಯೋಗಃ॥

\dhyana{ಮುಕ್ತಾಕಂಕಣಕುಂಡಲಾಂಗದಮಣಿಗ್ರೈವೇಯಹಾರೋರ್ಮಿಕಾ-\\
ವಿದ್ಯೋತದ್ವಲಯಾದಿಕಂಕಣಕಟೀಸೂತ್ರಾಂ ಸ್ಫುರನ್ನೂಪುರಾಮ್ ।\\
ಮಾಣಿಕ್ಯೋದರಬಂಧಕಂಬುಕಬರೀಮಿಂದೋಃ ಕಲಾಂ ಬಿಭ್ರತೀಂ\\
ಪಾಶಂ ಚಾಂಕುಶಪುಸ್ತಕಾಕ್ಷವಲಯಂ ದಕ್ಷೋರ್ಧ್ವಬಾಹ್ವಾದಿತಃ }॥೩॥

ಐಂ ವಾಗ್ಭವಂ ಪಾತು ಶೀರ್ಷಂ ಕ್ಲೀಂ ಕಾಮಸ್ತು ತಥಾ ಹೃದಿ~।\\
ಸೌಃ ಶಕ್ತಿಬೀಜಂ ಚ ಪಾತು ನಾಭೌ ಗುಹ್ಯೇ ಚ ಪಾದಯೋಃ ॥೪॥

ಐಂ ಕ್ಲೀಂ ಸೌಃ ವದನೇ ಪಾತು ಬಾಲಾ ಮಾಂ ಸರ್ವಸಿದ್ಧಯೇ~।\\
ಹಸಕಲಹ್ರೀಂ ಸೌಃ ಪಾತು  ಭೈರವೀ ಕಂಠದೇಶತಃ ॥೫॥

ಸುಂದರೀ ನಾಭಿದೇಶೇವ್ಯಾಚ್ಛೀರ್ಷಿಕಸ ಸಕಲಾ ಸದಾ~।\\
ಭ್ರೂನಾಸಯೋರಂತರಾಲೇ ಮಹಾತ್ರಿಪುರಸುಂದರೀ ॥೬॥

ಲಲಾಟೇ ಸುಭಗಾ ಪಾತು ಭಗಾ ಮಾಂ ಕಂಠದೇಶತಃ~।\\
ಭಗೋದೇವೀ ತು ಹೃದಯೇ ಉದರೇ ಭಗಸರ್ಪಿಣೀ ॥೭॥

ಭಗಮಾಲಾ ನಾಭಿದೇಶೇ ಲಿಂಗೇ ಪಾತು ಮನೋಭವಾ~।\\
ಗುಹ್ಯೇ ಪಾತು ಮಹಾವೀರಾ ರಾಜರಾಜೇಶ್ವರೀ ಶಿವಾ ॥೮॥

ಚೈತನ್ಯರೂಪಿಣೀ ಪಾತು ಪಾದಯೋರ್ಜಗದಂಬಿಕಾ~।\\
ನಾರಾಯಣೀ ಸರ್ವರಾತ್ರೇ ಸರ್ವಕಾರ್ಯೇ ಶುಭಂಕರೀ ॥೯॥

ಬ್ರಹ್ಮಾಣೀ ಪಾತು ಮಾಂ ಪೂರ್ವೇ ದಕ್ಷಿಣೇ ವೈಷ್ಣವೀ ತಥಾ~।\\
ಪಶ್ಚಿಮೇ ಪಾತು ವಾರಾಹೀ ಉತ್ತರೇ ತು ಮಹೇಶ್ವರೀ ॥೧೦॥

ಆಗ್ನೇಯ್ಯಾಂ ಪಾತು ಕೌಮಾರೀ ಮಹಾಲಕ್ಷ್ಮೀಶ್ಚ ನೈರ್ಋತೇ~।\\
ವಾಯವ್ಯೇ ಪಾತು ಚಾಮುಂಡಾ ಇಂದ್ರಾಣೀ ಪಾತು ಚೇಶಕೇ ॥೧೧॥

ಜಲೇ ಪಾತು ಮಹಾಮಾಯಾ ಪೃಥಿವ್ಯಾಂ ಸರ್ವಮಂಗಲಾ~।\\
ಆಕಾಶೇ ಪಾತು ವರದಾ ಸರ್ವತೋ ಭುವನೇಶ್ವರೀ ॥೧೨॥

ಇದಂತು ಕವಚಂ ನಾಮ ದೇವಾನಾಮಪಿ ದುರ್ಲಭಮ್ ।\\
ಪಠೇತ್ ಪ್ರಾತಃ ಸಮುತ್ಥಾಯ ಶುಚಿಃ ಪ್ರಯತಮಾನಸಃ ॥೧೩॥

ನಾಮಯೋ ವ್ಯಾಧಯಸ್ತಸ್ಯ ನ ಭಯಂ ನ ಕ್ವಚಿದ್ಭವೇತ್ ।\\
ನ ಚ ಮಾರೀಭಯಂ ತಸ್ಯ ಪಾತಕಾನಾಂ ಭಯಂ ತಥಾ ॥೧೪॥

ನ ದಾರಿದ್ರ್ಯವಶಂ ಗಚ್ಛೇತ್ತಿಷ್ಠೇನ್ಮೃತ್ಯುವಶೇ ನ ಚ ।\\
ಗಚ್ಛೇಚ್ಛಿವಪುರೇ ದೇವಿ ಸತ್ಯಂ ಸತ್ಯಂ ವದಾಮ್ಯಹಮ್ ॥೧೫॥

ಇದಂ ಕವಚಮಜ್ಞಾತ್ವಾ ಶ್ರೀವಿದ್ಯಾಂ ಯೋ ಜಪೇಚ್ಛಿವೇ ।\\
ಸ ನಾಪ್ನೋತಿ ಫಲಂ ತಸ್ಯ ಪ್ರಾಪ್ನುಯಾಚ್ಛಕ್ರಘಾತಕಮ್ ॥೧೬॥
\authorline{ಇತಿ ಶ್ರೀರುದ್ರಯಾಮಲೇ ಶ್ರೀಭೈರವೀಸಂವಾದೇ\\ ಶ್ರೀಬಾಲಾತ್ರಿಪುರಸುಂದರೀ ಕವಚಂ ಸಂಪೂರ್ಣಂ ॥}
)))))

\section{ಶ್ರೀ ದಕ್ಷಿಣಾಮೂರ್ತಿ ಕವಚಂ}
ಪಾರ್ವತ್ಯುವಾಚ ॥\\
ನಮಸ್ತೇಸ್ತು ತ್ರಯೀನಾಥ ಪರಮಾನಂದ ಕಾರಕ~।\\
ಕವಚಂ ದಕ್ಷಿಣಾಮೂರ್ತೇಃ ಕೃಪಯಾ ವದ ಮೇ ಪ್ರಭೋ ॥

ಈಶ್ವರ ಉವಾಚ ॥\\
ವಕ್ಷ್ಯೇಹಂ ದೇವ ದೇವೇಶಿ ದಕ್ಷಿಣಾಮೂರ್ತಿರವ್ಯಯಂ।\\
ಕವಚಂ ಸರ್ವಪಾಪಘ್ನಂ ವೇದಾನಾಂ ಜ್ಞಾನಗೋಚರಂ॥

ಅಣಿಮಾದಿ ಮಹಾಸಿದ್ಧಿವಿಧಾನಚತುರಂ ಶುಭಂ।\\
ವೇದಶಾಸ್ತ್ರಪುರಾಣಾನಿ ಕವಿತಾತರ್ಕ ಏವ ಚ ॥

ಬಹುಧಾ ದೇವಿ ಜಾಯಂತೇ ಕವಚಸ್ಯ ಪ್ರಭಾವತಃ~।\\
ಋಷಿರ್ಬ್ರಹ್ಮಾ ಸಮುದ್ದಿಷ್ಟಶ್ಛಂದೋಽನುಷ್ಟುಬುದಾಹೃತಂ॥

ದೇವತಾ ದಕ್ಷಿಣಾಮೂರ್ತಿಃ ಪರಮಾತ್ಮಾ ಸದಾಶಿವಃ।\\
ಬೀಜಂ ವೇದಾದಿಕಂ ಚೈವ ಸ್ವಾಹಾ ಶಕ್ತಿರುದಾಹೃತಾ।\\
ಸರ್ವಜ್ಞತ್ವೇಪಿ ದೇವೇಶಿ ವಿನಿಯೋಗಃ ಪ್ರಚಕ್ಷ್ಯತೇ॥

\dhyana{ಅದ್ವಂದ್ವನೇತ್ರಮಮಲೇಂದು ಕಲಾವತಂಸಂ\\
ಹಂಸಾವಲಂಬಿತಸಮಾನಜಟಾಕಲಾಪಂ~।\\
ಆನೀಲಕಂಠಮುಪಕಂಠಮುನಿಪ್ರವೀರಾನ್\\
ಅಧ್ಯಾಪಯಂತಮವಲೋಕಯ ಲೋಕನಾಥಂ॥}

ಶಿರೋ ಮೇ ದಕ್ಷಿಣಾಮೂರ್ತಿರವ್ಯಾತ್ ಫಾಲಂ ಮಹೇಶ್ವರಃ।\\
ದೃಶೌ ಪಾತು ಮಹಾದೇವಃ ಶ್ರವಣೇ ಚಂದ್ರಶೇಖರಃ॥೧॥

ಕಪೋಲೌ ಪಾತು ಮೇ ರುದ್ರೋ ನಾಸಾಂ ಪಾತು ಜಗದ್ಗುರುಃ।\\
ಮುಖಂ ಗೌರೀಪತಿಃ ಪಾತು ರಸನಾಂ ವೇದರೂಪಧೃತ್।\\
ದಶನಾನ್ ತ್ರಿಪುರಧ್ವಂಸೀ ಚೌಷ್ಠಂ ಪನ್ನಗಭೂಷಣಃ ॥೨॥

ಅಧರಂ ಪಾತು ವಿಶ್ವಾತ್ಮಾ ಹನೂ ಪಾತು ಜಗನ್ಮಯಃ।\\
ಚುಬುಕಂ ದೇವದೇವಸ್ತು ಪಾತು ಕಂಠಂ ಜಟಾಧರಃ॥೩॥

ಸ್ಕಂಧೌ ಮೇ ಪಾತು ಶುದ್ಧಾತ್ಮಾ ಕರೌ ಪಾತು ಯಮಾಂತಕಃ।\\
ಕುಚಾಗ್ರಂ ಕರಮಧ್ಯಂ ಚ ನಖರಾನ್ ಶಂಕರಃ ಸ್ವಯಂ ॥೪॥

ಹೃನ್ಮೇ ಪಶುಪತಿಃ ಪಾತು ಪಾರ್ಶ್ವೇ ಪರಮಪೂರುಷಃ~।\\
ಮಧ್ಯಮಂ ಪಾತು ಶರ್ವೋ ಮೇ ನಾಭಿಂ ನಾರಾಯಣಪ್ರಿಯಃ ॥೫॥

ಕಟಿಂ ಪಾತು ಜಗದ್ಭರ್ತಾ ಸಕ್ಥಿನೀ ಚ ಮೃಡಃ ಸ್ವಯಂ।\\
ಕೃತ್ತಿವಾಸಾಃ ಸ್ವಯಂ ಗುಹ್ಯಮೂರೂ ಪಾತು ಪಿನಾಕಧೃತ್॥೬॥

ಜಾನುನೀ ತ್ರ್ಯಂಬಕಃ ಪಾತು ಜಂಘೇ ಪಾತು ಸದಾಶಿವಃ।\\
ಸ್ಮರಾರಿಃ ಪಾತು ಮೇ ಪಾದೌ ಪಾತು ಸರ್ವಾಂಗಮೀಶ್ವರಃ ॥೭॥

ಇತೀದಂ ಕವಚಂ ದೇವಿ ಪರಮಾನಂದ ದಾಯಕಂ।\\
ಪ್ರಾತಃ ಕಾಲೇ ಶುಚಿರ್ಭೂತ್ವಾ ತ್ರಿವಾರಂ ಸರ್ವದಾ ಪಠೇತ್ ॥೮॥

ನಿತ್ಯಂ ಪೂಜಾ ಸಮಾಯುಕ್ತಃ ಸಂವತ್ಸರಮತಂದ್ರಿತಃ~।\\
ಜಪೇತ್ ತ್ರಿಸಂಧ್ಯಂ ಯೋ ವಿದ್ವಾನ್ ವೇದಶಾಸ್ತ್ರಾರ್ಥಪಾರಗಃ॥೯॥

ಗದ್ಯಪದ್ಯೈಸ್ತಥಾ ಚಾಪಿ ನಾಟಕಾಃ ಸ್ವಯಮೇವ ಹಿ~।\\
ನಿರ್ಗಚ್ಛಂತಿ ಮುಖಾಂಭೋಜಾತ್ ಸತ್ಯಮೇತನ್ನ ಸಂಶಯಃ ॥೧೦॥
\authorline{ಇತಿ ಶ್ರೀ ಬ್ರಹ್ಮವೈವರ್ತಮಹಾಪುರಾಣೇ  ಶ್ರೀ ದಕ್ಷಿಣಾಮೂರ್ತಿಕವಚಂ ಸಂಪೂರ್ಣಂ}
\section{ಶ್ರೀ ಲಕ್ಷ್ಮೀನಾರಾಯಣ ಕವಚಂ }
ಶ್ರೀ ಭೈರವ ಉವಾಚ ॥\\
ಅಧುನಾ ದೇವಿ ವಕ್ಷ್ಯಾಮಿ ಲಕ್ಷ್ಮೀನಾರಾಯಣಸ್ಯ ತೇ~।\\
ಕವಚಂ ಮಂತ್ರಗರ್ಭಂ ಚ ವಜ್ರಪಂಜರಕಾಖ್ಯಯಾ ॥೧॥

ಶ್ರೀವಜ್ರಪಂಜರಂ ನಾಮ ಕವಚಂ ಪರಮಾದ್ಭುತಂ~।\\
ರಹಸ್ಯಂ ಸರ್ವದೇವಾನಾಂ ಸಾಧಕಾನಾಂ ವಿಶೇಷತಃ ॥೨॥

ಯಂ ಧೃತ್ವಾ ಭಗವಾನ್ ದೇವಃ ಪ್ರಸೀದತಿ ಪರಃ ಪುಮಾನ್~।\\
ಯಸ್ಯ ಧಾರಣಮಾತ್ರೇಣ ಬ್ರಹ್ಮಾ ಲೋಕಪಿತಾಮಹಃ ॥೩॥

ಈಶ್ವರೋಽಹಂ ಶಿವೋ ಭೀಮೋ ವಾಸವೋಽಪಿ ದಿವಸ್ಪತಿಃ~।\\
ಸೂರ್ಯಸ್ತೇಜೋನಿಧಿರ್ದೇವಿ ಚಂದ್ರಮಾಸ್ತಾರಕೇಶ್ವರಃ ॥೪॥

ವಾಯುಶ್ಚ ಬಲವಾಂಲ್ಲೋಕೇ ವರುಣೋ ಯಾದಸಾಂಪತಿಃ~।\\
ಕುಬೇರೋಽಪಿ ಧನಾಧ್ಯಕ್ಷೋ ಧರ್ಮರಾಜೋ ಯಮಃ ಸ್ಮೃತಃ ॥೫॥

ಯಂ ಧೃತ್ವಾ ಸಹಸಾ ವಿಷ್ಣುಃ ಸಂಹರಿಷ್ಯತಿ ದಾನವಾನ್~।\\
ಜಘಾನ ರಾವಣಾದೀಂಶ್ಚ ಕಿಂ ವಕ್ಷ್ಯೇಽಹಮತಃ ಪರಂ ॥೬॥
\newpage
ಕವಚಸ್ಯಾಸ್ಯ ಸುಭಗೇ ಕಥಿತೋಽಯಂ ಮುನಿಃ ಶಿವಃ~।\\
ತ್ರಿಷ್ಟುಪ್ ಛಂದೋ ದೇವತಾ ಚ ಲಕ್ಷ್ಮೀನಾರಾಯಣೋ ಮತಃ ॥೭॥

ರಮಾ(ಶ್ರೀಂ) ಬೀಜಂ ಪರಾ(ಹ್ರೀಂ) ಶಕ್ತಿಸ್ತಾರಂ(ಓಂ) ಕೀಲಕಮೀಶ್ವರಿ~।\\
ಭೋಗಾಪವರ್ಗಸಿದ್ಧ್ಯರ್ಥಂ ವಿನಿಯೋಗ ಇತಿ ಸ್ಮೃತಃ ॥೮॥

\dhyana{ಪೂರ್ಣೇಂದುವದನಂ ಪೀತವಸನಂ ಕಮಲಾಸನಂ~।\\
ಲಕ್ಷ್ಮ್ಯಾ ಶ್ರಿತಂ ಚತುರ್ಬಾಹುಂ ಲಕ್ಷ್ಮೀನಾರಾಯಣಂ ಭಜೇ }॥೯॥

ಓಂ ವಾಸುದೇವೋಽವತು ಮೇ ಮಸ್ತಕಂ ಸಶಿರೋರುಹಂ~।\\
ಹ್ರೀಂ ಲಲಾಟಂ ಸದಾ ಪಾತು ಲಕ್ಷ್ಮೀವಿಷ್ಣುಃ ಸಮಂತತಃ ॥೧೦॥

ಹ್ಸೌಃ ನೇತ್ರೇಽವತಾಲ್ಲಕ್ಷ್ಮೀಗೋವಿಂದೋ ಜಗತಾಂ ಪತಿಃ~।\\
ಹ್ರೀಂ ನಾಸಾಂ ಸರ್ವದಾ ಪಾತು ಲಕ್ಷ್ಮೀದಾಮೋದರಃ ಪ್ರಭುಃ ॥೧೧॥

ಶ್ರೀಂ ಮುಖಂ ಸತತಂ ಪಾತು ದೇವೋ ಲಕ್ಷ್ಮೀತ್ರಿವಿಕ್ರಮಃ~।\\
ಲಕ್ಷ್ಮೀ ಕಂಠಂ ಸದಾ ಪಾತು ದೇವೋ ಲಕ್ಷ್ಮೀಜನಾರ್ದನಃ ॥೧೨॥

ನಾರಾಯಣಾಯ ಬಾಹೂ ಮೇ ಪಾತು ಲಕ್ಷ್ಮೀಗದಾಗ್ರಜಃ~।\\
ನಮಃ ಪಾರ್ಶ್ವೌ ಸದಾ ಪಾತು ಲಕ್ಷ್ಮೀನಂದೈಕನಂದನಃ ॥೧೩॥

ಅಂ ಆಂ ಇಂ ಈಂ ಪಾತು ವಕ್ಷೋ ಓಂ ಲಕ್ಷ್ಮೀತ್ರಿಪುರೇಶ್ವರಃ~।\\
ಉಂ ಊಂ ಋಂ ೠಂ ಪಾತು ಕುಕ್ಷಿಂ ಹ್ರೀಂ ಲಕ್ಷ್ಮೀಗರುಡಧ್ವಜಃ ॥೧೪॥

ಲೃಂ ಲೄಂ ಏಂ ಐಂ ಪಾತು ಪೃಷ್ಠಂ ಹ್ಸೌಃ ಲಕ್ಷ್ಮೀನೃಸಿಂಹಕಃ~।\\
ಓಂ ಔಂ ಅಂ ಅಃ ಪಾತು ನಾಭಿಂ ಹ್ರೀಂ ಲಕ್ಷ್ಮೀವಿಷ್ಟರಶ್ರವಾಃ ॥೧೫॥

ಕಂ ಖಂ ಗಂ ಘಂ ಗುದಂ ಪಾತು ಶ್ರೀಂ ಲಕ್ಷ್ಮೀಕೈಟಭಾಂತಕಃ~।\\
ಚಂ ಛಂ ಜಂ ಝಂ ಪಾತು ಶಿಶ್ನಂ ಲಕ್ಷ್ಮೀ ಲಕ್ಷ್ಮೀಶ್ವರಃ ಪ್ರಭುಃ ॥೧೬॥

ಟಂ ಠಂ ಡಂ ಢಂ ಕಟಿಂ ಪಾತು ನಾರಾಯಣಾಯ ನಾಯಕಃ~।\\
ತಂ ಥಂ ದಂ ಧಂ ಪಾತು ಚೋರೂ ನಮೋ ಲಕ್ಷ್ಮೀಜಗತ್ಪತಿಃ ॥೧೭॥

ಪಂ ಫಂ ಬಂ ಭಂ ಪಾತು ಜಾನೂ ಓಂ ಹ್ರೀಂ ಲಕ್ಷ್ಮೀಚತುರ್ಭುಜಃ~।\\
ಯಂ ರಂ ಲಂ ವಂ ಪಾತು ಜಂಘೇ ಹ್ಸೌಃ ಲಕ್ಷ್ಮೀಗದಾಧರಃ ॥೧೮॥

ಶಂ ಷಂ ಸಂ ಹಂ ಪಾತು ಗುಲ್ಫೌ ಹ್ರೀಂ ಶ್ರೀಂ ಲಕ್ಷ್ಮೀರಥಾಂಗಭೃತ್~।\\
ಳಂ ಕ್ಷಃ ಪಾದೌ ಸದಾ ಪಾತು ಮೂಲಂ ಲಕ್ಷ್ಮೀಸಹಸ್ರಪಾತ್ ॥೧೯॥

ಙಂ ಞಂ ಣಂ ನಂ ಮಂ ಮೇ ಪಾತು ಲಕ್ಷ್ಮೀಶಃ ಸಕಲಂ ವಪುಃ~।\\
ಇಂದ್ರೋ ಮಾಂ ಪೂರ್ವತಃ ಪಾತು ವಹ್ನಿರ್ವಹ್ನೌ ಸದಾವತು ॥೨೦॥

ಯಮೋ ಮಾಂ ದಕ್ಷಿಣೇ ಪಾತು ನೈರೃತ್ಯಾಂ ನಿರೃತಿಶ್ಚ ಮಾಂ~।\\
ವರುಣಃ ಪಶ್ಚಿಮೇಽವ್ಯಾನ್ಮಾಂ ವಾಯವ್ಯೇಽವತು ಮಾಂ ಮರುತ್ ॥೨೧॥

ಉತ್ತರೇ ಧನದಃ ಪಾಯಾದೈಶಾನ್ಯಾಮೀಶ್ವರೋಽವತು~।\\
ವಜ್ರ ಶಕ್ತಿ ದಂಡ ಖಡ್ಗ ಪಾಶ ಯಷ್ಟಿ ಧ್ವಜಾಂಕಿತಾಃ ॥೨೨॥

ಸಶೂಲಾಃ ಸರ್ವದಾ ಪಾಂತು ದಿಗೀಶಾಃ ಪರಮಾರ್ಥದಾಃ~।\\
ಅನಂತಃ ಪಾತ್ವಧೋ ನಿತ್ಯಮೂರ್ಧ್ವೇ ಬ್ರಹ್ಮಾವತಾಚ್ಚ ಮಾಂ ॥೨೩॥

ದಶದಿಕ್ಷು ಸದಾ ಪಾತು ಲಕ್ಷ್ಮೀನಾರಾಯಣಃ ಪ್ರಭುಃ~।\\
ಪ್ರಭಾತೇ ಪಾತು ಮಾಂ ವಿಷ್ಣುರ್ಮಧ್ಯಾಹ್ನೇ ವಾಸುದೇವಕಃ ॥೨೪॥

ದಾಮೋದರೋಽವತಾತ್ ಸಾಯಂ ನಿಶಾದೌ ನರಸಿಂಹಕಃ~।\\
ಸಂಕರ್ಷಣೋಽರ್ಧರಾತ್ರೇಽವ್ಯಾತ್ ಪ್ರಭಾತೇಽವ್ಯಾತ್ ತ್ರಿವಿಕ್ರಮಃ ॥೨೫॥

ಅನಿರುದ್ಧಃ ಸರ್ವಕಾಲಂ ವಿಷ್ವಕ್ಸೇನಶ್ಚ ಸರ್ವತಃ~।\\
ರಣೇ ರಾಜಕುಲೇ ದ್ಯೂತೇ ವಿವಾದೇ ಶತ್ರುಸಂಕಟೇ~।\\
ಓಂ ಹ್ರೀಂ ಹ್ಸೌಃ ಹ್ರೀಂ ಶ್ರೀಂ ಮೂಲಂ ಲಕ್ಷ್ಮೀನಾರಾಯಣೋಽವತು ॥೨೬॥

ಓಂಓಂಓಂರಣರಾಜಚೌರರಿಪುತಃ ಪಾಯಾಚ್ಚ ಮಾಂ ಕೇಶವಃ\\
ಹ್ರೀಂಹ್ರೀಂಹ್ರೀಂಹಹಹಾ ಹಸೌಃ ಹಸಹಸೌಃ ವಹ್ನೇರ್ವತಾನ್ಮಾಧವಃ~।\\
ಹ್ರೀಂಹ್ರೀಂಹ್ರೀಂಜಲಪರ್ವತಾಗ್ನಿಭಯತಃ ಪಾಯಾದನಂತೋ ವಿಭುಃ\\
ಶ್ರೀಂಶ್ರೀಂಶ್ರೀಂಶಶಶಾಲಲಂ ಪ್ರತಿದಿನಂ ಲಕ್ಷ್ಮೀಧವಃ ಪಾತು ಮಾಂ ॥೨೭॥

ಇತೀದಂ ಕವಚಂ ದಿವ್ಯಂ ವಜ್ರಪಂಜರಕಾಭಿಧಂ~।\\
ಲಕ್ಷ್ಮೀನಾರಾಯಣಸ್ಯೇಷ್ಟಂ ಚತುರ್ವರ್ಗಫಲಪ್ರದಂ ॥೨೮॥

ಸರ್ವಸೌಭಾಗ್ಯನಿಲಯಂ ಸರ್ವಸಾರಸ್ವತಪ್ರದಂ~।\\
ಲಕ್ಷ್ಮೀಸಂವನನಂ ತತ್ವಂ ಪರಮಾರ್ಥರಸಾಯನಂ ॥೨೯॥

ಮಂತ್ರಗರ್ಭಂ ಜಗತ್ಸಾರಂ ರಹಸ್ಯಂ ತ್ರಿದಿವೌಕಸಾಂ~।\\
ದಶವಾರಂ ಪಠೇದ್ರಾತ್ರೌ ರತಾಂತೇ ವೈಷ್ಣವೋತ್ತಮಃ ॥೩೦॥

ಸ್ವಪ್ನೇ ವರಪ್ರದಂ ಪಶ್ಯೇಲ್ಲಕ್ಷ್ಮೀನಾರಾಯಣಂ ಸುಧೀಃ~।\\
ತ್ರಿಸಂಧ್ಯಂ ಯಃ ಪಠೇನ್ನಿತ್ಯಂ ಕವಚಂ ಮನ್ಮುಖೋದಿತಂ ॥೩೧॥

ಸ ಯಾತಿ ಪರಮಂ ಧಾಮ ವೈಷ್ಣವಂ ವೈಷ್ಣವೇಶ್ವರಃ~।\\
ಮಹಾಚೀನಪದಸ್ಥೋಽಪಿ ಯಃ ಪಠೇದಾತ್ಮಚಿಂತಕಃ ॥೩೨॥

ಆನಂದಪೂರಿತಸ್ತೂರ್ಣಂ ಲಭೇದ್ ಮೋಕ್ಷಂ ಸ ಸಾಧಕಃ~।\\
ಗಂಧಾಷ್ಟಕೇನ ವಿಲಿಖೇದ್ರವೌ ಭೂರ್ಜೇ ಜಪನ್ಮನುಂ ॥೩೩॥

ಪೀತಸೂತ್ರೇಣ ಸಂವೇಷ್ಟ್ಯ ಸೌವರ್ಣೇನಾಥ ವೇಷ್ಟಯೇತ್~।\\
ಧಾರಯೇದ್ಗುಟಿಕಾಂ ಮೂರ್ಧ್ನಿ ಲಕ್ಷ್ಮೀನಾರಾಯಣಂ ಸ್ಮರನ್ ॥೩೪॥
\newpage
ರಣೇ ರಿಪೂನ್ ವಿಜಿತ್ಯಾಶು ಕಲ್ಯಾಣೀ ಗೃಹಮಾವಿಶೇತ್~।\\
ವಂಧ್ಯಾ ವಾ ಕಾಕವಂಧ್ಯಾ ವಾ ಮೃತವತ್ಸಾ ಚ ಯಾಂಗನಾ ॥೩೫॥

ಸಾ ಬಧ್ನೀಯಾತ್ ಕಂಠದೇಶೇ ಲಭೇತ್ ಪುತ್ರಾಂಶ್ಚಿರಾಯುಷಃ~।\\
ಗುರೂಪದೇಶತೋ ಧೃತ್ವಾ ಗುರುಂ ಧ್ಯಾತ್ವಾ ಮನುಂ ಜಪನ್ ॥೩೬॥

ವರ್ಣಲಕ್ಷಪುರಶ್ಚರ್ಯಾ ಫಲಮಾಪ್ನೋತಿ ಸಾಧಕಃ~।\\
ಬಹುನೋಕ್ತೇನ ಕಿಂ ದೇವಿ ಕವಚಸ್ಯಾಸ್ಯ ಪಾರ್ವತಿ ॥೩೭॥

ವಿನಾನೇನ ನ ಸಿದ್ಧಿಃ ಸ್ಯಾನ್ಮಂತ್ರಸ್ಯಾಸ್ಯ ಮಹೇಶ್ವರಿ~।\\
ಸರ್ವಾಗಮರಹಸ್ಯಾಢ್ಯಂ ತತ್ವಾತ್ ತತ್ವಂ ಪರಾತ್ ಪರಂ ॥೩೮॥

ಅಭಕ್ತಾಯ ನ ದಾತವ್ಯಂ ಕುಚೈಲಾಯ ದುರಾತ್ಮನೇ~।\\
ದೀಕ್ಷಿತಾಯ ಕುಲೀನಾಯ ಸ್ವಶಿಷ್ಯಾಯ ಮಹಾತ್ಮನೇ ॥೩೯॥

ಮಹಾಚೀನಪದಸ್ಥಾಯ ದಾತವ್ಯಂ ಕವಚೋತ್ತಮಂ~।\\
ಗುಹ್ಯಂ ಗೋಪ್ಯಂ ಮಹಾದೇವಿ ಲಕ್ಷ್ಮೀನಾರಾಯಣಪ್ರಿಯಂ~।\\
ವಜ್ರಪಂಜರಕಂ ವರ್ಮ ಗೋಪನೀಯಂ ಸ್ವಯೋನಿವತ್ ॥೪೦॥
\authorline{॥ಇತಿ ಶ್ರೀ ಲಕ್ಷ್ಮೀನಾರಾಯಣ ಕವಚಂ ಸಂಪೂರ್ಣಂ ॥}
\section{॥ ಚಂಡೀಕವಚಮ್ ॥}
ಅಸ್ಯ ಶ್ರೀಚಂಡೀಕವಚಸ್ಯ~। ಬ್ರಹ್ಮಾ ಋಷಿಃ~। ಅನುಷ್ಟುಪ್ ಛಂದಃ~।\\ ಚಾಮುಂಡಾ ದೇವತಾ~। ಅಂಗನ್ಯಾಸೋಕ್ತಮಾತರೋ ಬೀಜಮ್~।\\ದಿಗ್ಬಂಧದೇವತಾಸ್ತತ್ವಮ್~। ಶ್ರೀಜಗದಂಬಾಪ್ರೀತ್ಯರ್ಥೇ ಜಪೇ ವಿನಿಯೋಗಃ ॥
\newpage
ಓಂ ನಮಶ್ಚಂಡಿಕಾಯೈ ॥ ಮಾರ್ಕಂಡೇಯ ಉವಾಚ ॥\\
ಯದ್ಗುಹ್ಯಂ ಪರಮಂ ಲೋಕೇ ಸರ್ವರಕ್ಷಾಕರಂ ನೃಣಾಮ್~।\\
ಯನ್ನ ಕಸ್ಯಚಿದಾಖ್ಯಾತಂ ತನ್ಮೇ ಬ್ರೂಹಿ ಪಿತಾಮಹ ॥೧॥

ಬ್ರಹ್ಮೋವಾಚ ॥\\
ಅಸ್ತಿ ಗುಹ್ಯತಮಂ ವಿಪ್ರ ಸರ್ವಭೂತೋಪಕಾರಕಮ್~।\\
ದೇವ್ಯಾಸ್ತು ಕವಚಂ ಪುಣ್ಯಂ ತಚ್ಛೃಣುಷ್ವ ಮಹಾಮುನೇ ॥೨॥

ಪ್ರಥಮಂ ಶೈಲಪುತ್ರೀತಿ ದ್ವಿತೀಯಂ ಬ್ರಹ್ಮಚಾರಿಣೀ~।\\
ತೃತೀಯಂ ಚಂದ್ರಘಂಟೇತಿ ಕೂಷ್ಮಾಂಡೇತಿ ಚತುರ್ಥಕಮ್ ॥೩॥

ಪಂಚಮಂ ಸ್ಕಂದಮಾತೇತಿ ಷಷ್ಠಂ ಕಾತ್ಯಾಯನೀತಿ ಚ~।\\
ಸಪ್ತಮಂ ಕಾಲರಾತ್ರಿಶ್ಚ ಮಹಾಗೌರೀತಿ ಚಾಷ್ಟಮಮ್ ॥೪॥

ನವಮಂ ಸಿದ್ಧಿದಾತ್ರೀ ಚ ನವದುರ್ಗಾಃ ಪ್ರಕೀರ್ತಿತಾಃ~।\\
ಉಕ್ತಾನ್ಯೇತಾನಿ ನಾಮಾನಿ ಬ್ರಹ್ಮಣೈವ ಮಹಾತ್ಮನಾ ॥೫॥

ಅಗ್ನಿನಾ ದಹ್ಯಮಾನಸ್ತು ಶತ್ರುಮಧ್ಯೇ ಗತೋ ರಣೇ~।\\
ವಿಷಮೇ ದುರ್ಗಮೇ ಚೈವ ಭಯಾರ್ತಾಃ ಶರಣಂ ಗತಾಃ ॥೬॥

ನ ತೇಷಾಂ ಜಾಯತೇ ಕಿಂಚಿದಶುಭಂ ರಣಸಂಕಟೇ~।\\
ನಾಪದಂ ತಸ್ಯ ಪಶ್ಯಾಮಿ ಶೋಕದುಃಖಭಯಂ ನಹಿ ॥೭॥

ಯೈಸ್ತು ಭಕ್ತ್ಯಾ ಸ್ಮೃತಾ ನೂನಂ ತೇಷಾಂ ಸಿದ್ಧಿಃ ಪ್ರಜಾಯತೇ~।\\
ಪ್ರೇತಸಂಸ್ಥಾ ತು ಚಾಮುಂಡಾ ವಾರಾಹೀ ಮಹಿಷಾಸನಾ ॥೮॥

ಐಂದ್ರೀ ಗಜಸಮಾರೂಢಾ ವೈಷ್ಣವೀ ಗರುಡಾಸನಾ~।\\
ಮಾಹೇಶ್ವರೀ ವೃಷಾರೂಢಾ ಕೌಮಾರೀ ಶಿಖಿವಾಹನಾ ॥೯॥

ಬ್ರಾಹ್ಮೀ ಹಂಸಸಮಾರೂಢಾ ಸರ್ವಾಭರಣಭೂಷಿತಾ~।\\
ನಾನಾಽಭರಣಶೋಭಾಢ್ಯಾ ನಾನಾರತ್ನೋಪಶೋಭಿತಾಃ ॥೧೦॥

ದೃಶ್ಯಂತೇ ರಥಮಾರೂಢಾ ದೇವ್ಯಃ ಕ್ರೋಧಸಮಾಕುಲಾಃ~।\\
ಶಂಖಂ ಚಕ್ರಂ ಗದಾಂ ಶಕ್ತಿಂ ಹಲಂ ಚ ಮುಸಲಾಯುಧಮ್ ॥೧೧॥

ಖೇಟಕಂ ತೋಮರಂ ಚೈವ ಪರಶುಂ ಪಾಶಮೇವ ಚ~।\\
ಕುಂತಾಯುಧಂ ತ್ರಿಶೂಲಂ ಚ ಶಾರ್ಙ್ಗಮಾಯುಧಮುತ್ತಮಮ್ ॥೧೨॥

ದೈತ್ಯಾನಾಂ ದೇಹನಾಶಾಯ ಭಕ್ತಾನಾಮಭಯಾಯ ಚ~।\\
ಧಾರಯಂತ್ಯಾಯುಧಾನೀತ್ಥಂ ದೇವಾನಾಂ ಚ ಹಿತಾಯ ವೈ ॥೧೩॥

ಮಹಾಬಲೇ ಮಹೋತ್ಸಾಹೇ ಮಹಾಭಯವಿನಾಶಿನಿ~।\\
ತ್ರಾಹಿ ಮಾಂ ದೇವಿ ದುಷ್ಪ್ರೇಕ್ಷ್ಯೇ ಶತ್ರೂಣಾಂ ಭಯವರ್ಧಿನಿ ॥೧೪॥

ಪ್ರಾಚ್ಯಾಂ ರಕ್ಷತು ಮಾಮೈಂದ್ರೀ ಆಗ್ನೇಯ್ಯಾಮಗ್ನಿದೇವತಾ~।\\
ದಕ್ಷಿಣೇಽವತು ವಾರಾಹೀ ನೈರ್‌ಋತ್ಯಾಂ ಖಡ್ಗಧಾರಿಣೀ ॥೧೫॥

ಪ್ರತೀಚ್ಯಾಂ ವಾರುಣೀ ರಕ್ಷೇದ್ವಾಯವ್ಯಾಂ ಮೃಗವಾಹಿನೀ~।\\
ಉದೀಚ್ಯಾಂ ರಕ್ಷ ಕೌಬೇರಿ ಈಶಾನ್ಯಾಂ ಶೂಲಧಾರಿಣಿ ॥೧೬॥

ಊರ್ಧ್ವಂ ಬ್ರಹ್ಮಾಣೀ ಮೇ ರಕ್ಷೇದಧಸ್ತಾದ್ವೈಷ್ಣವೀ ತಥಾ~।\\
ಏವಂ ದಶ ದಿಶೋ ರಕ್ಷೇಚ್ಚಾಮುಂಡಾ ಶವವಾಹನಾ ॥೧೭॥

ಜಯಾ ಮೇ ಚಾಗ್ರತಃ ಸ್ಥಾತು ವಿಜಯಾ ಸ್ಥಾತು ಪೃಷ್ಠತಃ~।\\
ಅಜಿತಾ ವಾಮಪಾರ್ಶ್ವೇ ತು ದಕ್ಷಿಣೇ ಚಾಪರಾಜಿತಾ ॥೧೮॥

ಶಿಖಾಂ ಮೇ ದ್ಯೋತಿನೀ ರಕ್ಷೇದುಮಾ ಮೂರ್ಧ್ನಿ ವ್ಯವಸ್ಥಿತಾ~।\\
ಮಾಲಾಧರೀ ಲಲಾಟೇ ಚ ಭ್ರುವೌ ರಕ್ಷೇದ್ಯಶಸ್ವಿನೀ ॥೧೯॥

ತ್ರಿನೇತ್ರಾ ಚ ಭ್ರುವೋರ್ಮಧ್ಯೇ ಯಮಘಂಟಾ ಚ ನಾಸಿಕೇ~।\\
ಶಂಖಿನೀ ಚಕ್ಷುಷೋರ್ಮಧ್ಯೇ ಶ್ರೋತ್ರಯೋರ್ದ್ವಾರವಾಸಿನೀ ॥೨೦॥

ಕಪೋಲೌ ಕಾಲಿಕಾ ರಕ್ಷೇತ್ಕರ್ಣಮೂಲೇ ತು ಶಾಂಕರೀ~।\\
ನಾಸಿಕಾಯಾಂ ಸುಗಂಧಾ ಚ ಉತ್ತರೋಷ್ಠೇ ಚ ಚರ್ಚಿಕಾ ॥೨೧॥

ಅಧರೇ ಚಾಮೃತಕಲಾ ಜಿಹ್ವಾಯಾಂ ಚ ಸರಸ್ವತೀ~।\\
ದಂತಾನ್ ರಕ್ಷತು ಕೌಮಾರೀ ಕಂಠಮಧ್ಯೇ ತು ಚಂಡಿಕಾ ॥೨೨॥

ಘಂಟಿಕಾಂ ಚಿತ್ರಘಂಟಾ ಚ ಮಹಾಮಾಯಾ ಚ ತಾಲುಕೇ~।\\
ಕಾಮಾಕ್ಷೀ ಚಿಬುಕಂ ರಕ್ಷೇದ್ವಾಚಂ ಮೇ ಸರ್ವಮಂಗಲಾ ॥೨೩॥

ಗ್ರೀವಾಯಾಂ ಭದ್ರಕಾಲೀ ಚ ಪೃಷ್ಠವಂಶೇ ಧನುರ್ಧರೀ~।\\
ನೀಲಗ್ರೀವಾ ಬಹಿಃಕಂಠೇ ನಲಿಕಾಂ ನಲಕೂಬರೀ ॥೨೪॥

ಖಡ್ಗಧಾರಿಣ್ಯುಭೌ ಸ್ಕಂಧೌ ಬಾಹೂ ಮೇ ವಜ್ರಧಾರಿಣೀ~।\\
ಹಸ್ತಯೋರ್ದಂಡಿನೀ ರಕ್ಷೇದಂಬಿಕಾ ಚಾಂಗುಲೀಸ್ತಥಾ ॥೨೫॥

ನಖಾಂಛೂಲೇಶ್ವರೀ ರಕ್ಷೇತ್ ಕುಕ್ಷೌ ರಕ್ಷೇನ್ನಲೇಶ್ವರೀ~।\\
ಸ್ತನೌ ರಕ್ಷೇನ್ಮಹಾಲಕ್ಷ್ಮೀರ್ಮನಃ ಶೋಕವಿನಾಶಿನೀ ॥೨೬॥

ಹೃದಯಂ ಲಲಿತಾದೇವೀ ಉದರಂ ಶೂಲಧಾರಿಣೀ~।\\
ನಾಭೌ ಚ ಕಾಮಿನೀ ರಕ್ಷೇದ್ಗುಹ್ಯಂ ಗುಹ್ಯೇಶ್ವರೀ ತಥಾ ॥೨೭॥

ಕಟ್ಯಾಂ ಭಗವತೀ ರಕ್ಷೇಜ್ಜಾನುನೀ ವಿಂಧ್ಯವಾಸಿನೀ~।\\
ಭೂತನಾಥಾ ಚ ಮೇಢ್ರಂ ಮೇ ಊರೂ ಮಹಿಷವಾಹಿನೀ ॥೨೮॥

ಜಂಘೇ ಮಹಾಬಲಾ ಪ್ರೋಕ್ತಾ ಸರ್ವಕಾಮಪ್ರದಾಯಿನೀ~।\\
ಗುಲ್ಫಯೋರ್ನಾರಸಿಂಹೀ ಚ ಪಾದೌ ಚಾಮಿತತೇಜಸೀ ॥೨೯॥

ಪಾದಾಂಗುಲೀಃ ಶ್ರೀರ್ಮೇ ರಕ್ಷೇತ್ಪಾದಾಧಸ್ತಲವಾಸಿನೀ~।\\
ನಖಾಂದಂಷ್ಟ್ರಾಃ ಕರಾಲೀ ಚ ಕೇಶಾಂಶ್ಚೈವೋರ್ಧ್ವಕೇಶಿನೀ ॥೩೦॥

ರೋಮಕೂಪೇಷು ಕೌಬೇರೀ ತ್ವಚಂ ವಾಗೀಶ್ವರೀ ತಥಾ~।\\
ರಕ್ತಮಜ್ಜಾವಸಾಮಾಂಸಾನ್ಯಸ್ಥಿಮೇದಾಂಸಿ ಪಾರ್ವತೀ ॥೩೧॥

ಅಂತ್ರಾಣಿ ಕಾಲರಾತ್ರಿಶ್ಚ ಪಿತ್ತಂ ಚ ಮುಕುಟೇಶ್ವರೀ~।\\
ಪದ್ಮಾವತೀ ಪದ್ಮಕೋಶೇ ಕಫೇ ಚೂಡಾಮಣಿಸ್ತಥಾ ॥೩೨॥

ಜ್ವಾಲಾಮುಖೀ ನಖಜ್ವಾಲಾ ಅಭೇದ್ಯಾ ಸರ್ವಸಂಧಿಷು~।\\
ಶುಕ್ರಂ ಬ್ರಹ್ಮಾಣೀ ಮೇ ರಕ್ಷೇಚ್ಛಾಯಾಂ ಛತ್ರೇಶ್ವರೀ ತಥಾ ॥೩೩॥

ಅಹಂಕಾರಂ ಮನೋ ಬುದ್ಧಿಂ ರಕ್ಷ ಮೇ ಧರ್ಮಚಾರಿಣಿ~।\\
ಪ್ರಾಣಾಪಾನೌ ತಥಾ ವ್ಯಾನಂ ಸಮಾನೋದಾನಮೇವ ಚ ॥೩೪॥

ಯಶಃ ಕೀರ್ತಿಂ ಚ ಲಕ್ಷ್ಮೀಂ ಚ ಸದಾ ರಕ್ಷತು ವೈಷ್ಣವೀ~।\\
ಗೋತ್ರಮಿಂದ್ರಾಣೀ ಮೇ ರಕ್ಷೇತ್ಪಶೂನ್ಮೇ ರಕ್ಷ ಚಂಡಿಕೇ ॥೩೫॥

ಪುತ್ರಾನ್ ರಕ್ಷೇನ್ಮಹಾಲಕ್ಷ್ಮೀರ್ಭಾರ್ಯಾಂ ರಕ್ಷತು ಭೈರವೀ~।\\
ಮಾರ್ಗಂ ಕ್ಷೇಮಕರೀ ರಕ್ಷೇದ್ವಿಜಯಾ ಸರ್ವತಃ ಸ್ಥಿತಾ ॥೩೬॥

ರಕ್ಷಾಹೀನಂ ತು ಯತ್ಸ್ಥಾನಂ ವರ್ಜಿತಂ ಕವಚೇನ ತು~।\\
ತತ್ಸರ್ವಂ ರಕ್ಷ ಮೇ ದೇವಿ ಜಯಂತೀ ಪಾಪನಾಶಿನೀ ॥೩೭॥

ಪದಮೇಕಂ ನ ಗಚ್ಛೇತ್ತು ಯದೀಚ್ಛೇಚ್ಛುಭಮಾತ್ಮನಃ~।\\
ಕವಚೇನಾವೃತೋ ನಿತ್ಯಂ ಯತ್ರ ಯತ್ರಾಧಿಗಚ್ಛತಿ ॥೩೮॥

ತತ್ರ ತತ್ರಾರ್ಥ ಲಾಭಶ್ಚ ವಿಜಯಃ ಸಾರ್ವಕಾಮಿಕಃ~।\\
ಯಂ ಯಂ ಕಾಮಯತೇ ಕಾಮಂ ತಂ ತಂ ಪ್ರಾಪ್ನೋತಿ ನಿಶ್ಚಿತಮ್ ॥೩೯॥

ಪರಮೈಶ್ವರ್ಯಮತುಲಂ ಪ್ರಾಪ್ಸ್ಯತೇ ಭೂತಲೇ ಪುಮಾನ್~।\\
ನಿರ್ಭಯೋ ಜಾಯತೇ ಮರ್ತ್ಯಃ ಸಂಗ್ರಾಮೇಷ್ವ ಪರಾಜಿತಃ ॥೪೦॥

ತ್ರೈಲೋಕ್ಯೇ ತು ಭವೇತ್ಪೂಜ್ಯಃ ಕವಚೇನಾವೃತಃ ಪುಮಾನ್~।\\
ಇದಂ ತು ದೇವ್ಯಾಃ ಕವಚಂ ದೇವಾನಾಮಪಿ ದುರ್ಲಭಮ್ ॥೪೧॥

ಯಃ ಪಠೇತ್ಪ್ರಯತೋ ನಿತ್ಯಂ ತ್ರಿಸಂಧ್ಯಂ ಶ್ರದ್ಧಯಾನ್ವಿತಃ~।\\
ದೈವೀ ಕಲಾ ಭವೇತ್ತಸ್ಯ ತ್ರೈಲೋಕ್ಯೇಪ್ಯಪರಾಜಿತಃ ॥೪೨॥

ಜೀವೇದ್ವರ್ಷಶತಂ ಸಾಗ್ರಮಪಮೃತ್ಯು ವಿವರ್ಜಿತಃ~।\\
ನಶ್ಯಂತಿ ವ್ಯಾಧಯಃ ಸರ್ವೇ ಲೂತಾವಿಸ್ಫೋಟಕಾದಯಃ ॥೪೩॥

ಸ್ಥಾವರಂ ಜಂಗಮಂ ವಾಪಿ ಕೃತ್ರಿಮಂ ಚಾಪಿ ಯದ್ವಿಷಮ್~।\\
ಆಭಿಚಾರಾಣಿ ಸರ್ವಾಣಿ ಮಂತ್ರಯಂತ್ರಾಣಿ ಭೂತಲೇ ॥೪೪॥

ಭೂಚರಾಃ ಖೇಚರಾಶ್ಚೈವ ಜಲಜಾಶ್ಚೋಪದೇಶಿಕಾಃ~।\\
ಸಹಜಾಃ ಕುಲಜಾ ಮಾಲಾಃ ಶಾಕಿನೀ ಡಾಕಿನೀ ತಥಾ ॥೪೫॥

ಅಂತರಿಕ್ಷಚರಾ ಘೋರಾ ಡಾಕಿನ್ಯಶ್ಚ ಮಹಾಬಲಾಃ~।\\
ಗ್ರಹಭೂತಪಿಶಾಚಾಶ್ಚ ಯಕ್ಷಗಂಧರ್ವರಾಕ್ಷಸಾಃ ॥೪೬॥

ಬ್ರಹ್ಮರಾಕ್ಷಸವೇತಾಲಾಃ ಕೂಷ್ಮಾಂಡಾ ಭೈರವಾದಯಃ~।\\
ನಶ್ಯಂತಿ ದರ್ಶನಾತ್ತಸ್ಯ ಕವಚೇ ಹೃದಿ ಸಂಸ್ಥಿತೇ ॥೪೭॥

ಮಾನೋನ್ನತಿರ್ಭವೇದ್ರಾಜ್ಞಸ್ತೇಜೋವೃದ್ಧಿಕರಂ ಪರಮ್~।\\
ಯಶಸಾ ವರ್ಧತೇ ಸೋಽಪಿ ಕೀರ್ತಿಮಂಡಿತಭೂತಲೇ ॥೪೮॥

ಜಪೇತ್ಸಪ್ತಶತೀಂ ಚಂಡೀಂ ಕೃತ್ವಾ ತು ಕವಚಂ ಪುರಾ~।\\
ಯಾವದ್ಭೂಮಂಡಲಂ ಧತ್ತೇ ಸಶೈಲವನಕಾನನಮ್ ॥೪೯॥

ತಾವತ್ತಿಷ್ಠತಿ ಮೇದಿನ್ಯಾಂ ಸಂತತಿಃ ಪುತ್ರಪೌತ್ರಿಕೀ~।\\
ದೇಹಾಂತೇ ಪರಮಂ ಸ್ಥಾನಂ ಯತ್ಸುರೈರಪಿ ದುರ್ಲಭಮ್~।\\
ಪ್ರಾಪ್ನೋತಿ ಪುರುಷೋ ನಿತ್ಯಂ ಮಹಾಮಾಯಾಪ್ರಸಾದತಃ ॥೫೦॥
\authorline{ಇತಿ ಶ್ರೀವಾರಾಹಪುರಾಣೇ ಹರಿಹರಬ್ರಹ್ಮವಿರಚಿತಂ ದೇವ್ಯಾಃ ಕವಚಂ ಸಂಪೂರ್ಣಮ್ ॥}
\section{ಅರ್ಗಲಾ ಸ್ತೋತ್ರಮ್}
ಅಸ್ಯ ಶ್ರೀಅರ್ಗಲಾ ಸ್ತೋತ್ರಮಂತ್ರಸ್ಯ ವಿಷ್ಣುರ್ಋಷಿಃ । ಅನುಷ್ಟುಪ್ಛಂದಃ ।  ಶ್ರೀಮಹಾಲಕ್ಷ್ಮೀರ್ದೇವತಾ ।  ಶ್ರೀಜಗದಂಬಾ ಪ್ರೀತ್ಯರ್ಥೇ ಜಪೇ ವಿನಿಯೋಗಃ ॥

ಓಂ ನಮಶ್ಚಂಡಿಕಾಯೈ ॥\\
ಜಯಂತೀ ಮಂಗಲಾ ಕಾಲೀ ಭದ್ರಕಾಲೀ ಕಪಾಲಿನೀ ।\\
ದುರ್ಗಾ ಕ್ಷಮಾ ಶಿವಾ ಧಾತ್ರೀ ಸ್ವಾಹಾ ಸ್ವಧಾ ನಮೋಽಸ್ತು ತೇ ॥೧॥

ಮಧುಕೈಟಭವಿದ್ರಾವಿ ವಿಧಾತೃವರದೇ ನಮಃ ।\\
ರೂಪಂ ದೇಹಿ ಜಯಂ ದೇಹಿ ಯಶೋ ದೇಹಿ ದ್ವಿಷೋ ಜಹಿ ॥೨॥

ಮಹಿಷಾಸುರನಿರ್ನಾಶವಿಧಾತ್ರಿ ವರದೇ ನಮಃ ।\\
ರೂಪಂ ದೇಹಿ ಜಯಂ ದೇಹಿ ಯಶೋ ದೇಹಿ ದ್ವಿಷೋ ಜಹಿ ॥೩॥

ವಂದಿತಾಂಘ್ರಿಯುಗೇ ದೇವಿ ಸರ್ವಸೌಭಾಗ್ಯದಾಯಿನಿ ।\\
ರೂಪಂ ದೇಹಿ ಜಯಂ ದೇಹಿ ಯಶೋ ದೇಹಿ ದ್ವಿಷೋ ಜಹಿ ॥೪॥

ರಕ್ತಬೀಜವಧೇ ದೇವಿ ಚಂಡಮುಂಡವಿನಾಶಿನಿ ।\\
ರೂಪಂ ದೇಹಿ ಜಯಂ ದೇಹಿ ಯಶೋ ದೇಹಿ ದ್ವಿಷೋ ಜಹಿ ॥೫॥

ಅಚಿಂತ್ಯರೂಪಚರಿತೇ ಸರ್ವಶತ್ರುವಿನಾಶಿನಿ ।\\
ರೂಪಂ ದೇಹಿ ಜಯಂ ದೇಹಿ ಯಶೋ ದೇಹಿ ದ್ವಿಷೋ ಜಹಿ ॥೬॥

ನತೇಭ್ಯಃ ಸರ್ವದಾ ಭಕ್ತ್ಯಾ ಚಂಡಿಕೇ ದುರಿತಾಪಹೇ ।\\
ರೂಪಂ ದೇಹಿ ಜಯಂ ದೇಹಿ ಯಶೋ ದೇಹಿ ದ್ವಿಷೋ ಜಹಿ ॥೭॥

ಸ್ತುವದ್ಭ್ಯೋ ಭಕ್ತಿಪೂರ್ವಂ ತ್ವಾಂ ಚಂಡಿಕೇ ವ್ಯಾಧಿನಾಶಿನಿ ।\\
ರೂಪಂ ದೇಹಿ ಜಯಂ ದೇಹಿ ಯಶೋ ದೇಹಿ ದ್ವಿಷೋ ಜಹಿ ॥೮॥

ಚಂಡಿಕೇ ಸತತಂ ಯೇ ತ್ವಾಮರ್ಚಯಂತೀಹ ಭಕ್ತಿತಃ ।\\
ರೂಪಂ ದೇಹಿ ಜಯಂ ದೇಹಿ ಯಶೋ ದೇಹಿ ದ್ವಿಷೋ ಜಹಿ ॥೯॥

ದೇಹಿ ಸೌಭಾಗ್ಯಮಾರೋಗ್ಯಂ ದೇಹಿ ದೇವಿ ಪರಂ ಸುಖಂ ।\\
ರೂಪಂ ದೇಹಿ ಜಯಂ ದೇಹಿ ಯಶೋ ದೇಹಿ ದ್ವಿಷೋ ಜಹಿ ॥೧೦॥

ವಿಧೇಹಿ ದ್ವಿಷತಾಂ ನಾಶಂ ವಿಧೇಹಿ ಬಲಮುಚ್ಚಕೈಃ ।\\
ರೂಪಂ ದೇಹಿ ಜಯಂ ದೇಹಿ ಯಶೋ ದೇಹಿ ದ್ವಿಷೋ ಜಹಿ ॥೧೧॥

ವಿಧೇಹಿ ದೇವಿ ಕಲ್ಯಾಣಂ ವಿಧೇಹಿ ಪರಮಾಂ ಶ್ರಿಯಂ ।\\
ರೂಪಂ ದೇಹಿ ಜಯಂ ದೇಹಿ ಯಶೋ ದೇಹಿ ದ್ವಿಷೋ ಜಹಿ ॥೧೨॥

ವಿದ್ಯಾವಂತಂ ಯಶಸ್ವಂತಂ ಲಕ್ಷ್ಮೀವಂತಂ ಜನಂ ಕುರು ।\\
ರೂಪಂ ದೇಹಿ ಜಯಂ ದೇಹಿ ಯಶೋ ದೇಹಿ ದ್ವಿಷೋ ಜಹಿ ॥೧೩॥

ಪ್ರಚಂಡದೈತ್ಯದರ್ಪಘ್ನೇ ಚಂಡಿಕೇ ಪ್ರಣತಾಯ ಮೇ ।\\
ರೂಪಂ ದೇಹಿ ಜಯಂ ದೇಹಿ ಯಶೋ ದೇಹಿ ದ್ವಿಷೋ ಜಹಿ ॥೧೪॥

ಚತುರ್ಭುಜೇ ಚತುರ್ವಕ್ತ್ರಸಂಸ್ತುತೇ ಪರಮೇಶ್ವರಿ ।\\
ರೂಪಂ ದೇಹಿ ಜಯಂ ದೇಹಿ ಯಶೋ ದೇಹಿ ದ್ವಿಷೋ ಜಹಿ ॥೧೫॥

ಕೃಷ್ಣೇನ ಸಂಸ್ತುತೇ ದೇವಿ ಶಶ್ವದ್ಭಕ್ತ್ಯಾ ತ್ವಮಂಬಿಕೇ ।\\
ರೂಪಂ ದೇಹಿ ಜಯಂ ದೇಹಿ ಯಶೋ ದೇಹಿ ದ್ವಿಷೋ ಜಹಿ ॥೧೬॥

ಹಿಮಾಚಲಸುತಾನಾಥಸಂಸ್ತುತೇ ಪರಮೇಶ್ವರಿ ।\\
ರೂಪಂ ದೇಹಿ ಜಯಂ ದೇಹಿ ಯಶೋ ದೇಹಿ ದ್ವಿಷೋ ಜಹಿ ॥೧೭॥

ಸುರಾಸುರಶಿರೋರತ್ನನಿಘೃಷ್ಟಚರಣೇಽಂಬಿಕೇ ।\\
ರೂಪಂ ದೇಹಿ ಜಯಂ ದೇಹಿ ಯಶೋ ದೇಹಿ ದ್ವಿಷೋ ಜಹಿ ॥೧೮॥

ಇಂದ್ರಾಣೀಪತಿಸದ್ಭಾವಪೂಜಿತೇ ಪರಮೇಶ್ವರಿ ।\\
ರೂಪಂ ದೇಹಿ ಜಯಂ ದೇಹಿ ಯಶೋ ದೇಹಿ ದ್ವಿಷೋ ಜಹಿ ॥೧೯॥

ದೇವಿ ಭಕ್ತಜನೋದ್ದಾಮದತ್ತಾನಂದೋದಯೇಂಬಿಕೇ ।\\
ರೂಪಂ ದೇಹಿ ಜಯಂ ದೇಹಿ ಯಶೋ ದೇಹಿ ದ್ವಿಷೋ ಜಹಿ ॥೨೦॥

ಪುತ್ರಾನ್ ದೇಹಿ ಧನಂ ದೇಹಿ ಸರ್ವಕಾಮಾಂಶ್ಚ ದೇಹಿ ಮೇ ।\\
ರೂಪಂ ದೇಹಿ ಜಯಂ ದೇಹಿ ಯಶೋ ದೇಹಿ ದ್ವಿಷೋ ಜಹಿ ॥೨೧॥

ಪತ್ನೀಂ ಮನೋರಮಾಂ ದೇಹಿ ಮನೋವೃತ್ತಾನು ಸಾರಿಣೀಂ।\\
ತಾರಿಣೀಂ ದುರ್ಗಸಂಸಾರಸಾಗರಸ್ಯ ಕುಲೋದ್ಭವಾಂ॥೨೨॥

ಇದಂ ಸ್ತೋತ್ರಂ ಪಠಿತ್ವಾ ತು ಮಹಾಸ್ತೋತ್ರಂ ಪಠೇನ್ನರಃ ।\\
ಸ ತು ಸಪ್ತಶತೀ ಸಂಖ್ಯಾವರಮಾಪ್ನೋತಿ ಸಂಪದಾಂ ॥೨೩॥
\authorline{॥ ಮಾರ್ಕಂಡೇಯಪುರಾಣೇ ಅರ್ಗಲಾ ಸ್ತೋತ್ರಂ ॥}
\section{ಕೀಲಕಮ್}
ಓಂ ಅಸ್ಯ ಶ್ರೀಕೀಲಕಮಂತ್ರಸ್ಯ ಶಿವಋಷಿಃ । ಅನುಷ್ಟುಪ್ ಛಂದಃ । ಶ್ರೀಮಹಾಸರಸ್ವತೀ ದೇವತಾ । ಶ್ರೀಜಗದಂಬಾಪ್ರೀತ್ಯರ್ಥಂ ಜಪೇ ವಿನಿಯೋಗಃ ॥

ಓಂ ನಮಶ್ಚಂಡಿಕಾಯೈ ॥ ಮಾರ್ಕಂಡೇಯ ಉವಾಚ ॥\\
ವಿಶುದ್ಧಜ್ಞಾನದೇಹಾಯ ತ್ರಿವೇದೀದಿವ್ಯಚಕ್ಷುಷೇ ।\\
ಶ್ರೇಯಃಪ್ರಾಪ್ತಿನಿಮಿತ್ತಾಯ ನಮಃ ಸೋಮಾರ್ಧಧಾರಿಣೇ ॥೧॥

ಸರ್ವಮೇತದ್ವಿನಾ ಯಸ್ತು ಮಂತ್ರಾಣಾಮಪಿ ಕೀಲಕಂ ।\\
ಸೋಽಪಿ ಕ್ಷೇಮಮವಾಪ್ನೋತಿ ಸತತಂ ಜಾಪ್ಯತತ್ಪರಃ ॥೨॥

ಸಿದ್ಧ್ಯಂತ್ಯುಚ್ಚಾಟನಾದೀನಿ ವಸ್ತೂನಿ ಸಕಲಾನ್ಯಪಿ ।\\
ಏತೇನ ಸ್ತುವತಾಂ ನಿತ್ಯಂ ಸ್ತೋತ್ರಮಾತ್ರೇಣ ಸಿದ್ಧ್ಯತಿ ॥೩॥

ನ ಮಂತ್ರೋ ನೌಷಧಂ ತತ್ರ ನ ಕಿಂಚಿದಪಿ ವಿದ್ಯತೇ ।\\
ವಿನಾ ಜಾಪ್ಯೇನ ಸಿದ್ಧ್ಯೇತ ಸರ್ವಮುಚ್ಚಾಟನಾದಿಕಂ ॥೪॥

ಸಮಗ್ರಾಣ್ಯಪಿ ಸಿದ್ಧ್ಯಂತಿ ಲೋಕಶಂಕಾಮಿಮಾಂ ಹರಃ ।\\
ಕೃತ್ವಾ ನಿಮಂತ್ರಯಾಮಾಸ ಸರ್ವಮೇವಮಿದಂ ಶುಭಂ ॥೫॥

ಸ್ತೋತ್ರಂ ವೈ ಚಂಡಿಕಾಯಾಸ್ತು ತಚ್ಚ ಗುಹ್ಯಂ ಚಕಾರ ಸಃ ।\\
ಸಮಾಪ್ತಿರ್ನ ಚ ಪುಣ್ಯಸ್ಯ ತಾಂ ಯಥಾವನ್ನಿಯಂತ್ರಣಾಂ ॥೬॥

ಸೋಽಪಿ ಕ್ಷೇಮಮವಾಪ್ನೋತಿ ಸರ್ವಮೇವ ನ ಸಂಶಯಃ ।\\
ಕೃಷ್ಣಾಯಾಂ ವಾ ಚತುರ್ದಶ್ಯಾಮಷ್ಟಮ್ಯಾಂ ವಾ ಸಮಾಹಿತಃ ॥೭॥

ದದಾತಿ ಪ್ರತಿಗೃಹ್ಣಾತಿ ನಾನ್ಯಥೈಷಾ ಪ್ರಸೀದತಿ ।\\
ಇತ್ಥಂ ರೂಪೇಣ ಕೀಲೇನ ಮಹಾದೇವೇನ ಕೀಲಿತಂ ॥೮॥

ಯೋ ನಿಷ್ಕೀಲಾಂ ವಿಧಾಯೈನಾಂ ನಿತ್ಯಂ ಜಪತಿ ಸುಸ್ಫುಟಂ ।\\
ಸಸಿದ್ಧಃ ಸಗಣಃ ಸೋಽಪಿ ಗಂಧರ್ವೋ ಜಾಯತೇ ವನೇ ॥೯॥

ನ ಚೈವಾಪ್ಯಟತಸ್ತಸ್ಯ ಭಯಂ ಕ್ವಾಪಿ ಹಿ ಜಾಯತೇ ।\\
ನಾಪಮೃತ್ಯುವಶಂ ಯಾತಿ ಮೃತೋ ಮೋಕ್ಷಮವಾಪ್ನುಯಾತ್ ॥೧೦॥

ಜ್ಞಾತ್ವಾ ಪ್ರಾರಭ್ಯ ಕುರ್ವೀತ ಹ್ಯಕುರ್ವಾಣೋ ವಿನಶ್ಯತಿ ।\\
ತತೋ ಜ್ಞಾತ್ವೈವ ಸಂಪನ್ನಮಿದಂ ಪ್ರಾರಭ್ಯತೇ ಬುಧೈಃ ॥೧೧॥

ಸೌಭಾಗ್ಯಾದಿ ಚ ಯತ್ಕಿಂಚಿದ್ ದೃಶ್ಯತೇ ಲಲನಾಜನೇ ।\\
ತತ್ಸರ್ವಂ ತತ್ಪ್ರಸಾದೇನ ತೇನ ಜಾಪ್ಯಮಿದಂ ಶುಭಂ ॥೧೨॥

ಶನೈಸ್ತು ಜಪ್ಯಮಾನೇಽಸ್ಮಿನ್ ಸ್ತೋತ್ರೇ ಸಂಪತ್ತಿರುಚ್ಚಕೈಃ ।\\
ಭವತ್ಯೇವ ಸಮಗ್ರಾಪಿ ತತಃ ಪ್ರಾರಭ್ಯಮೇವ ತತ್ ॥೧೩॥

ಐಶ್ವರ್ಯಂ ಯತ್ಪ್ರಸಾದೇನ ಸೌಭಾಗ್ಯಾರೋಗ್ಯಸಂಪದಃ ।\\
ಶತ್ರುಹಾನಿಃ ಪರೋ ಮೋಕ್ಷಃ ಸ್ತೂಯತೇ ಸಾ ನ ಕಿಂ ಜನೈಃ ॥೧೪॥
\authorline{॥ಭಗವತ್ಯಾಃ ಕೀಲಕಸ್ತೋತ್ರಂ ॥}
\section{ಬ್ರಹ್ಮಸ್ತುತಿ}
ಬ್ರಹ್ಮೋವಾಚ ॥\\
ತ್ವಂ ಸ್ವಾಹಾ ತ್ವಂ ಸ್ವಧಾ ತ್ವಂ ಹಿ ವಷಟ್ಕಾರ ಸ್ವರಾತ್ಮಿಕಾ ।\\
ಸುಧಾ ತ್ವಮಕ್ಷರೇ ನಿತ್ಯೇ ತ್ರಿಧಾ ಮಾತ್ರಾತ್ಮಿಕಾ ಸ್ಥಿತಾ ॥೨॥

ಅರ್ಧಮಾತ್ರಾ ಸ್ಥಿತಾ ನಿತ್ಯಾ ಯಾನುಚ್ಚಾರ್ಯಾ ವಿಶೇಷತಃ ।\\
ತ್ವಮೇವ ಸಂಧ್ಯಾ ಸಾವಿತ್ರೀ ತ್ವಂ ದೇವಿ ಜನನೀ ಪರಾ ॥೩॥

ತ್ವಯೈತದ್ಧಾರ್ಯತೇ ವಿಶ್ವಂ ತ್ವಯೈತತ್ಸೃಜ್ಯತೇ ಜಗತ್ ।\\
ತ್ವಯೈತತ್ಪಾಲ್ಯತೇ ದೇವಿ ತ್ವಮತ್ಸ್ಯಂತೇ ಚ ಸರ್ವದಾ ॥೪॥

ವಿಸೃಷ್ಟೌ ಸೃಷ್ಟಿರೂಪಾ ತ್ವಂ ಸ್ಥಿತಿರೂಪಾ ಚ ಪಾಲನೇ ।\\
ತಥಾ ಸಂಹೃತಿರೂಪಾಂತೇ ಜಗತೋಽಸ್ಯ ಜಗನ್ಮಯೇ ॥೫॥

ಮಹಾವಿದ್ಯಾ ಮಹಾಮಾಯಾ ಮಹಾಮೇಧಾ ಮಹಾಸ್ಮೃತಿಃ ।\\
ಮಹಾಮೋಹಾ ಚ ಭವತೀ ಮಹಾದೇವೀ ಮಹೇಶ್ವರೀ ॥೬॥

ಪ್ರಕೃತಿಸ್ತ್ವಂ ಚ ಸರ್ವಸ್ಯ ಗುಣತ್ರಯ ವಿಭಾವಿನೀ ।\\
ಕಾಲರಾತ್ರಿರ್ಮಹಾರಾತ್ರಿರ್ಮೋಹರಾತ್ರಿಶ್ಚ ದಾರುಣಾ ॥೭॥

ತ್ವಂ ಶ್ರೀಸ್ತ್ವಮೀಶ್ವರೀ ತ್ವಂ ಹ್ರೀಸ್ತ್ವಂ ಬುದ್ಧಿರ್ಬೋಧಲಕ್ಷಣಾ ।\\
ಲಜ್ಜಾ ಪುಷ್ಟಿಸ್ತಥಾ ತುಷ್ಟಿಸ್ತ್ವಂ ಶಾಂತಿಃ ಕ್ಷಾಂತಿರೇವ ಚ ॥೮॥

ಖಡ್ಗಿನೀ ಶೂಲಿನೀ ಘೋರಾ ಗದಿನೀ ಚಕ್ರಿಣೀ ತಥಾ ।\\
ಶಂಖಿನೀ ಚಾಪಿನೀ ಬಾಣಭುಶುಂಡೀಪರಿಘಾಯುಧಾ ॥೯॥

ಸೌಮ್ಯಾ ಸೌಮ್ಯತರಾಶೇಷ ಸೌಮ್ಯೇಭ್ಯಸ್ತ್ವತಿ ಸುಂದರೀ ।\\
ಪರಾಪರಾಣಾಂ ಪರಮಾ ತ್ವಮೇವ ಪರಮೇಶ್ವರೀ ॥೧೦॥

ಯಚ್ಚ ಕಿಂಚಿತ್ ಕ್ವಚಿದ್ವಸ್ತು ಸದಸದ್ವಾಖಿಲಾತ್ಮಿಕೇ ।\\
ತಸ್ಯ ಸರ್ವಸ್ಯ ಯಾ ಶಕ್ತಿಃ ಸಾ ತ್ವಂ ಕಿಂ ಸ್ತೂಯಸೇ ಮಯಾ ॥೧೧॥

ಯಯಾ ತ್ವಯಾ ಜಗತ್‌ಸ್ರಷ್ಟಾ ಜಗತ್ಪಾತ್ಯತ್ತಿ ಯೋ ಜಗತ್ ।\\
ಸೋಽಪಿ ನಿದ್ರಾವಶಂ ನೀತಃ ಕಸ್ತ್ವಾಂ ಸ್ತೋತುಮಿಹೇಶ್ವರಃ ॥೧೨॥

ವಿಷ್ಣುಃ ಶರೀರಗ್ರಹಣಮಹಮೀಶಾನ ಏವ ಚ ।\\
ಕಾರಿತಾಸ್ತೇ ಯತೋಽತಸ್ತ್ವಾಂ ಕಃ ಸ್ತೋತುಂ ಶಕ್ತಿಮಾನ್ಭವೇತ್ ॥೧೩॥

ಸಾ ತ್ವಮಿತ್ಥಂ ಪ್ರಭಾವೈಃ ಸ್ವೈರುದಾರೈರ್ದೇವಿ ಸಂಸ್ತುತಾ ।\\
ಮೋಹಯೈತೌ ದುರಾಧರ್ಷಾವಸುರೌ ಮಧುಕೈಟಭೌ ॥೧೪॥

ಪ್ರಬೋಧಂ ಚ ಜಗತ್ಸ್ವಾಮೀ ನೀಯತಾಮಚ್ಯುತೋ ಲಘು ।\\
ಬೋಧಶ್ಚ ಕ್ರಿಯತಾಮಸ್ಯ ಹಂತುಮೇತೌ ಮಹಾಸುರೌ ॥೧೫॥


\section{ಶಕ್ರಾದಿಸ್ತುತಿ}
          ಋಷಿರುವಾಚ ॥೧॥\\
     ಶಕ್ರಾದಯಃ ಸುರಗಣಾ ನಿಹತೇಽತಿವೀರ್ಯೇ\\
ತಸ್ಮಿನ್ ದುರಾತ್ಮನಿ ಸುರಾರಿಬಲೇ ಚ ದೇವ್ಯಾ ।\\
     ತಾಂ ತುಷ್ಟುವುಃ ಪ್ರಣತಿನಮ್ರಶಿರೋಧರಾಂಸಾ\\
ವಾಗ್ಭಿಃ ಪ್ರಹರ್ಷಪುಲಕೋದ್ಗಮಚಾರುದೇಹಾಃ ॥೨॥

     ದೇವ್ಯಾ ಯಯಾ ತತಮಿದಂ ಜಗದಾತ್ಮಶಕ್ತ್ಯಾ\\
ನಿಶ್ಶೇಷದೇವಗಣಶಕ್ತಿಸಮೂಹಮೂರ್ತ್ಯಾ ।\\
     ತಾಮಂಬಿಕಾಮಖಿಲದೇವಮಹರ್ಷಿಪೂಜ್ಯಾಂ\\
ಭಕ್ತ್ಯಾ ನತಾಃ ಸ್ಮ ವಿದಧಾತು ಶುಭಾನಿ ಸಾ ನಃ ॥೩॥

     ಯಸ್ಯಾಃ ಪ್ರಭಾವಮತುಲಂ ಭಗವಾನನಂತೋ\\
ಬ್ರಹ್ಮಾ ಹರಶ್ಚ ನ ಹಿ ವಕ್ತುಮಲಂ ಬಲಂ ಚ ।\\
     ಸಾ ಚಂಡಿಕಾಖಿಲಜಗತ್ಪರಿಪಾಲನಾಯ\\
ನಾಶಾಯ ಚಾಶುಭಭಯಸ್ಯ ಮತಿಂ ಕರೋತು ॥೪॥

     ಯಾ ಶ್ರೀಃ ಸ್ವಯಂ ಸುಕೃತಿನಾಂ ಭವನೇಷ್ವಲಕ್ಷ್ಮೀಃ\\
ಪಾಪಾತ್ಮನಾಂ ಕೃತಧಿಯಾಂ ಹೃದಯೇಷು ಬುದ್ಧಿಃ ।\\
     ಶ್ರದ್ಧಾ ಸತಾಂ ಕುಲಜನಪ್ರಭವಸ್ಯ ಲಜ್ಜಾ\\
ತಾಂ ತ್ವಾಂ ನತಾಃ ಸ್ಮ ಪರಿಪಾಲಯ ದೇವಿ ವಿಶ್ವಂ ॥೫॥

     ಕಿಂ ವರ್ಣಯಾಮ ತವ ರೂಪಮಚಿಂತ್ಯಮೇತತ್\\
ಕಿಂ ಚಾತಿವೀರ್ಯಮಸುರ ಕ್ಷಯಕಾರಿ ಭೂರಿ ।\\
     ಕಿಂ ಚಾಹವೇಷು ಚರಿತಾನಿ ತವಾದ್ಭುತಾನಿ\\
ಸರ್ವೇಷು ದೇವ್ಯಸುರ ದೇವ ಗಣಾದಿಕೇಷು ॥೬॥

     ಹೇತುಃ ಸಮಸ್ತ ಜಗತಾಂ ತ್ರಿಗುಣಾಪಿ ದೋಷೈ-\\
ರ್ನ ಜ್ಞಾಯಸೇ ಹರಿಹರಾದಿಭಿರಪ್ಯಪಾರಾ ।\\
     ಸರ್ವಾಶ್ರಯಾಖಿಲಮಿದಂ ಜಗದಂಶಭೂತ-\\
ಮವ್ಯಾಕೃತಾ ಹಿ ಪರಮಾ ಪ್ರಕೃತಿಸ್ತ್ವಮಾದ್ಯಾ ॥೭॥

     ಯಸ್ಯಾಃ ಸಮಸ್ತ ಸುರತಾ ಸಮುದೀರಣೇನ\\
ತೃಪ್ತಿಂ ಪ್ರಯಾತಿ ಸಕಲೇಷು ಮಖೇಷು ದೇವಿ ।\\
     ಸ್ವಾಹಾಸಿ ವೈ ಪಿತೃಗಣಸ್ಯ ಚ ತೃಪ್ತಿಹೇತು-\\
ರುಚ್ಚಾರ್ಯಸೇ ತ್ವಮತ ಏವ ಜನೈಃ ಸ್ವಧಾ ಚ ॥೮॥

     ಯಾ ಮುಕ್ತಿಹೇತುರವಿಚಿಂತ್ಯಮಹಾವ್ರತಾ ತ್ವಂ\\
ಅಭ್ಯಸ್ಯಸೇ ಸುನಿಯತೇಂದ್ರಿಯ ತತ್ತ್ವಸಾರೈಃ ।\\
     ಮೋಕ್ಷಾರ್ಥಿಭಿರ್ಮುನಿಭಿರಸ್ತಸಮಸ್ತದೋಷೈ-\\
ರ್ವಿದ್ಯಾಸಿ ಸಾ ಭಗವತೀ ಪರಮಾ ಹಿ ದೇವಿ ॥೯॥

     ಶಬ್ದಾತ್ಮಿಕಾ ಸುವಿಮಲರ್ಗ್ಯಜುಷಾಂ ನಿಧಾನ-\\
ಮುದ್ಗೀಥರಮ್ಯಪದಪಾಠವತಾಂ ಚ ಸಾಮ್ನಾಂ ।\\
     ದೇವಿ ತ್ರಯೀ ಭಗವತೀ ಭವಭಾವನಾಯ\\
ವಾರ್ತಾಸಿ ಸರ್ವಜಗತಾಂ ಪರಮಾರ್ತಿ ಹಂತ್ರೀ ॥೧೦॥

     ಮೇಧಾಸಿ ದೇವಿ ವಿದಿತಾಖಿಲಶಾಸ್ತ್ರಸಾರಾ\\
ದುರ್ಗಾಸಿ ದುರ್ಗಭವಸಾಗರ ನೌರಸಂಗಾ ।\\
     ಶ್ರೀಃ ಕೈಟಭಾರಿ ಹೃದಯೈಕ ಕೃತಾಧಿವಾಸಾ\\
ಗೌರೀ ತ್ವಮೇವ ಶಶಿಮೌಲಿಕೃತಪ್ರತಿಷ್ಠಾ ॥೧೧॥

     ಈಷತ್ಸಹಾಸಮಮಲಂ ಪರಿಪೂರ್ಣಚಂದ್ರ-\\
ಬಿಂಬಾನುಕಾರಿ ಕನಕೋತ್ತಮ ಕಾಂತಿಕಾಂತಂ ।\\
     ಅತ್ಯದ್ಭುತಂ ಪ್ರಹೃತಮಾತ್ತರುಷಾ ತಥಾಪಿ\\
ವಕ್ತ್ರಂ ವಿಲೋಕ್ಯ ಸಹಸಾ ಮಹಿಷಾಸುರೇಣ ॥೧೨॥

     ದೃಷ್ಟ್ವಾ ತು ದೇವಿ ಕುಪಿತಂ ಭ್ರುಕುಟೀಕರಾಲ-\\
ಮುದ್ಯಚ್ಛಶಾಂಕ ಸದೃಶಚ್ಛವಿ ಯನ್ನ ಸದ್ಯಃ ।\\
     ಪ್ರಾಣಾನ್ಮುಮೋಚ ಮಹಿಷಸ್ತದತೀವ ಚಿತ್ರಂ\\
ಕೈರ್ಜೀವ್ಯತೇ ಹಿ ಕುಪಿತಾಂತಕ ದರ್ಶನೇನ ॥೧೩॥

     ದೇವಿ ಪ್ರಸೀದ ಪರಮಾ ಭವತೀ ಭವಾಯ\\
ಸದ್ಯೋ ವಿನಾಶಯಸಿ ಕೋಪವತೀ ಕುಲಾನಿ ।\\
     ವಿಜ್ಞಾತಮೇತದಧುನೈವ ಯದಸ್ತಮೇತ-\\
ನ್ನೀತಂ ಬಲಂ ಸುವಿಪುಲಂ ಮಹಿಷಾಸುರಸ್ಯ ॥೧೪॥

     ತೇ ಸಮ್ಮತಾ ಜನಪದೇಷು ಧನಾನಿ ತೇಷಾಂ\\
ತೇಷಾಂ ಯಶಾಂಸಿ ನ ಚ ಸೀದತಿ ಬಂಧುವರ್ಗಃ ।\\
     ಧನ್ಯಾಸ್ತ ಏವ ನಿಭೃತಾತ್ಮಜ ಭೃತ್ಯ ದಾರಾ\\
ಯೇಷಾಂ ಸದಾಭ್ಯುದಯದಾ ಭವತೀ ಪ್ರಸನ್ನಾ ॥೧೫॥

     ಧರ್ಮ್ಯಾಣಿ ದೇವಿ ಸಕಲಾನಿ ಸದೈವ ಕರ್ಮಾ-\\
ಣ್ಯತ್ಯಾದೃತಃ ಪ್ರತಿದಿನಂ ಸುಕೃತೀ ಕರೋತಿ ।\\
     ಸ್ವರ್ಗಂ ಪ್ರಯಾತಿ ಚ ತತೋ ಭವತೀಪ್ರಸಾದಾ-\\
ಲ್ಲೋಕತ್ರಯೇಽಪಿ ಫಲದಾ ನನು ದೇವಿ ತೇನ ॥೧೬॥

     ದುರ್ಗೇ ಸ್ಮೃತಾ ಹರಸಿ ಭೀತಿಮಶೇಷಜಂತೋಃ\\
ಸ್ವಸ್ಥೈಃ ಸ್ಮೃತಾ ಮತಿಮತೀವ ಶುಭಾಂ ದದಾಸಿ ।\\
     ದಾರಿದ್ರ್ಯ ದುಃಖ ಭಯಹಾರಿಣಿ ಕಾ ತ್ವದನ್ಯಾ\\
ಸರ್ವೋಪಕಾರ ಕರಣಾಯ ಸದಾಽಽರ್ದ್ರಚಿತ್ತಾ ॥೧೭॥

     ಏಭಿರ್ಹತೈರ್ಜಗದುಪೈತಿ ಸುಖಂ ತಥೈತೇ\\
ಕುರ್ವಂತು ನಾಮ ನರಕಾಯ ಚಿರಾಯ ಪಾಪಂ ।\\
     ಸಂಗ್ರಾಮಮೃತ್ಯುಮಧಿಗಮ್ಯ ದಿವಂ ಪ್ರಯಾಂತು\\
ಮತ್ವೇತಿ ನೂನಮಹಿತಾನ್ವಿನಿಹಂಸಿ ದೇವಿ ॥೧೮॥

     ದೃಷ್ಟ್ವೈವ ಕಿಂ ನ ಭವತೀ ಪ್ರಕರೋತಿ ಭಸ್ಮ\\
ಸರ್ವಾಸುರಾನರಿಷು ಯತ್ಪ್ರಹಿಣೋಷಿ ಶಸ್ತ್ರಂ ।\\
     ಲೋಕಾನ್ಪ್ರಯಾಂತು ರಿಪವೋಽಪಿ ಹಿ ಶಸ್ತ್ರಪೂತಾ\\
ಇತ್ಥಂ ಮತಿರ್ಭವತಿ ತೇಷ್ವಹಿತೇಷು ಸಾಧ್ವೀ ॥೧೯॥

     ಖಡ್ಗಪ್ರಭಾ ನಿಕರ ವಿಸ್ಫುರಣೈಸ್ತಥೋಗ್ರೈಃ\\
ಶೂಲಾಗ್ರಕಾಂತಿನಿವಹೇನ ದೃಶೋಽಸುರಾಣಾಂ ।\\
     ಯನ್ನಾಗತಾ ವಿಲಯಮಂಶುಮದಿಂದುಖಂಡ-\\
ಯೋಗ್ಯಾನನಂ ತವ ವಿಲೋಕಯತಾಂ ತದೇತತ್ ॥೨೦॥

     ದುರ್ವೃತ್ತವೃತ್ತಶಮನಂ ತವ ದೇವಿ ಶೀಲಂ\\
ರೂಪಂ ತಥೈತದವಿಚಿಂತ್ಯಮತುಲ್ಯಮನ್ಯೈಃ ।\\
     ವೀರ್ಯಂ ಚ ಹಂತೃಹೃತದೇವಪರಾಕ್ರಮಾಣಾಂ\\
ವೈರಿಷ್ವಪಿ ಪ್ರಕಟಿತೈವ ದಯಾ ತ್ವಯೇತ್ಥಂ ॥೨೧॥

     ಕೇನೋಪಮಾ ಭವತು ತೇಽಸ್ಯ ಪರಾಕ್ರಮಸ್ಯ\\
ರೂಪಂ ಚ ಶತ್ರುಭಯಕಾರ್ಯತಿಹಾರಿ ಕುತ್ರ ।\\
     ಚಿತ್ತೇ ಕೃಪಾ ಸಮರನಿಷ್ಠುರತಾ ಚ ದೃಷ್ಟಾ\\
ತ್ವಯ್ಯೇವ ದೇವಿ ವರದೇ ಭುವನತ್ರಯೇಽಪಿ ॥೨೨॥

     ತ್ರೈಲೋಕ್ಯಮೇತದಖಿಲಂ ರಿಪುನಾಶನೇನ\\
ತ್ರಾತಂ ತ್ವಯಾ ಸಮರಮೂರ್ಧನಿ ತೇಽಪಿ ಹತ್ವಾ ।\\
     ನೀತಾ ದಿವಂ ರಿಪುಗಣಾ ಭಯಮಪ್ಯಪಾಸ್ತ-\\
ಮಸ್ಮಾಕಮುನ್ಮದಸುರಾರಿಭವಂ ನಮಸ್ತೇ ॥೨೩॥

ಶೂಲೇನ ಪಾಹಿ ನೋ ದೇವಿ ಪಾಹಿ ಖಡ್ಗೇನ ಚಾಂಬಿಕೇ ।\\
ಘಂಟಾಸ್ವನೇನ ನಃ ಪಾಹಿ ಚಾಪಜ್ಯಾನಿಃಸ್ವನೇನ ಚ ॥೨೪॥

ಪ್ರಾಚ್ಯಾಂ ರಕ್ಷ ಪ್ರತೀಚ್ಯಾಂ ಚ ಚಂಡಿಕೇ ರಕ್ಷ ದಕ್ಷಿಣೇ ।\\
ಭ್ರಾಮಣೇನಾತ್ಮಶೂಲಸ್ಯ ಉತ್ತರಸ್ಯಾಂ ತಥೇಶ್ವರಿ ॥೨೫॥

ಸೌಮ್ಯಾನಿ ಯಾನಿ ರೂಪಾಣಿ ತ್ರೈಲೋಕ್ಯೇ ವಿಚರಂತಿ ತೇ ।\\
ಯಾನಿ ಚಾತ್ಯಂತ ಘೋರಾಣಿ ತೈ ರಕ್ಷಾಸ್ಮಾಂಸ್ತಥಾ ಭುವಂ ॥೨೬॥

ಖಡ್ಗಶೂಲಗದಾದೀನಿ ಯಾನಿ ಚಾಸ್ತ್ರಾಣಿ ತೇಽಮ್ಬಿಕೇ ।\\
ಕರಪಲ್ಲವಸಂಗೀನಿ ತೈರಸ್ಮಾನ್ರಕ್ಷ ಸರ್ವತಃ ॥೨೭॥

 ಋಷಿರುವಾಚ ॥೨೮॥\\
ಏವಂ ಸ್ತುತಾ ಸುರೈರ್ದಿವ್ಯೈಃ ಕುಸುಮೈರ್ನಂದನೋದ್ಭವೈಃ ।\\
ಅರ್ಚಿತಾ ಜಗತಾಂ ಧಾತ್ರೀ ತಥಾ ಗಂಧಾನುಲೇಪನೈಃ ॥೨೯॥

ಭಕ್ತ್ಯಾ ಸಮಸ್ತೈಸ್ತ್ರಿದಶೈರ್ದಿವ್ಯೈರ್ಧೂಪೈಃ ಸುಧೂಪಿತಾ ।\\
ಪ್ರಾಹ ಪ್ರಸಾದಸುಮುಖೀ ಸಮಸ್ತಾನ್ ಪ್ರಣತಾನ್ ಸುರಾನ್ ॥೩೦॥

ದೇವ್ಯುವಾಚ ॥೩೧॥\\
ವ್ರಿಯತಾಂ ತ್ರಿದಶಾಃ ಸರ್ವೇ ಯದಸ್ಮತ್ತೋಽಭಿವಾಂಛಿತಂ ॥೩೨॥

ದೇವಾ ಊಚುಃ ॥೩೩॥\\
ಭಗವತ್ಯಾ ಕೃತಂ ಸರ್ವಂ ನ ಕಿಂಚಿದವಶಿಷ್ಯತೇ ।\\
ಯದಯಂ ನಿಹತಃ ಶತ್ರುರಸ್ಮಾಕಂ ಮಹಿಷಾಸುರಃ ॥೩೪॥

ಯದಿ ಚಾಪಿ ವರೋ ದೇಯಸ್ತ್ವಯಾಽಸ್ಮಾಕಂ ಮಹೇಶ್ವರಿ ।\\
ಸಂಸ್ಮೃತಾಽಸಂಸ್ಮೃತಾ ತ್ವಂ ನೋ ಹಿಂಸೇಥಾಃ ಪರಮಾಪದಃ ॥೩೫॥

ಯಶ್ಚ ಮರ್ತ್ಯಃ ಸ್ತವೈರೇಭಿಸ್ತ್ವಾಂ ಸ್ತೋಷ್ಯತ್ಯಮಲಾನನೇ ॥೩೬॥

ತಸ್ಯ ವಿತ್ತರ್ದ್ಧಿವಿಭವೈರ್ಧನದಾರಾದಿ ಸಂಪದಾಂ ।\\
ವೃದ್ಧಯೇಽಸ್ಮತ್ಪ್ರಸನ್ನಾ ತ್ವಂ ಭವೇಥಾಃ ಸರ್ವದಾಂಬಿಕೇ ॥೩೭॥


\section{ನಾರಾಯಣೀಸ್ತುತಿ}
ಓಂ ಋಷಿರುವಾಚ ॥೧॥\\
ದೇವ್ಯಾ ಹತೇ ತತ್ರ ಮಹಾಸುರೇಂದ್ರೇ\\
        ಸೇಂದ್ರಾಃ ಸುರಾ ವಹ್ನಿಪುರೋಗಮಾಸ್ತಾಂ ।\\
ಕಾತ್ಯಾಯನೀಂ ತುಷ್ಟುವುರಿಷ್ಟಲಾಭಾದ್\\
      ವಿಕಾಶಿವಕ್ತ್ರಾಬ್ಜವಿಕಾಶಿತಾಶಾಃ ॥೨॥

ದೇವಿ ಪ್ರಪನ್ನಾರ್ತಿಹರೇ ಪ್ರಸೀದ\\
        ಪ್ರಸೀದ ಮಾತರ್ಜಗತೋಽಖಿಲಸ್ಯ ।\\
ಪ್ರಸೀದ ವಿಶ್ವೇಶ್ವರಿ ಪಾಹಿ ವಿಶ್ವಂ\\
        ತ್ವಮೀಶ್ವರೀ ದೇವಿ ಚರಾಚರಸ್ಯ ॥೩॥

ಆಧಾರಭೂತಾ ಜಗತಸ್ತ್ವಮೇಕಾ\\
        ಮಹೀಸ್ವರೂಪೇಣ ಯತಃ ಸ್ಥಿತಾಸಿ ।\\
ಅಪಾಂ ಸ್ವರೂಪಸ್ಥಿತಯಾ ತ್ವಯೈತ-\\
      ದಾಪ್ಯಾಯತೇ ಕೃತ್ಸ್ನಮಲಂಘ್ಯವೀರ್ಯೇ ॥೪॥

ತ್ವಂ ವೈಷ್ಣವೀಶಕ್ತಿರನಂತವೀರ್ಯಾ\\
      ವಿಶ್ವಸ್ಯ ಬೀಜಂ ಪರಮಾಸಿ ಮಾಯಾ ।\\
ಸಮ್ಮೋಹಿತಂ ದೇವಿ ಸಮಸ್ತಮೇತತ್\\
      ತ್ವಂ ವೈ ಪ್ರಸನ್ನಾ ಭುವಿ ಮುಕ್ತಿಹೇತುಃ ॥೫॥

ವಿದ್ಯಾಃ ಸಮಸ್ತಾಸ್ತವ ದೇವಿ ಭೇದಾಃ\\
        ಸ್ತ್ರಿಯಃ ಸಮಸ್ತಾಃ ಸಕಲಾ ಜಗತ್ಸು ।\\
ತ್ವಯೈಕಯಾ ಪೂರಿತಮಂಬಯೈತತ್\\
        ಕಾ ತೇ ಸ್ತುತಿಃ ಸ್ತವ್ಯಪರಾಪರೋಕ್ತಿಃ ॥೬॥

ಸರ್ವಭೂತಾ ಯದಾ ದೇವೀ ಭುಕ್ತಿಮುಕ್ತಿಪ್ರದಾಯಿನೀ ।\\
ತ್ವಂ ಸ್ತುತಾ ಸ್ತುತಯೇ ಕಾ ವಾ ಭವಂತು ಪರಮೋಕ್ತಯಃ ॥೭॥

ಸರ್ವಸ್ಯ ಬುದ್ಧಿರೂಪೇಣ ಜನಸ್ಯ ಹೃದಿ ಸಂಸ್ಥಿತೇ ।\\
ಸ್ವರ್ಗಾಪವರ್ಗದೇ ದೇವಿ ನಾರಾಯಣಿ ನಮೋಽಸ್ತು ತೇ ॥೮॥

ಕಲಾಕಾಷ್ಠಾದಿರೂಪೇಣ ಪರಿಣಾಮಪ್ರದಾಯಿನಿ ।\\
ವಿಶ್ವಸ್ಯೋಪರತೌ ಶಕ್ತೇ ನಾರಾಯಣಿ ನಮೋಽಸ್ತು ತೇ ॥೯॥

ಸರ್ವಮಂಗಲಮಾಂಗಲ್ಯೇ ಶಿವೇ ಸರ್ವಾರ್ಥಸಾಧಿಕೇ ।\\
ಶರಣ್ಯೇ ತ್ರ್ಯಂಬಕೇ ಗೌರಿ ನಾರಾಯಣಿ ನಮೋಽಸ್ತು ತೇ ॥೧೦॥

ಸೃಷ್ಟಿಸ್ಥಿತಿವಿನಾಶಾನಾಂ ಶಕ್ತಿಭೂತೇ ಸನಾತನಿ ।\\
ಗುಣಾಶ್ರಯೇ ಗುಣಮಯೇ ನಾರಾಯಣಿ ನಮೋಽಸ್ತು ತೇ ॥೧೧॥

ಶರಣಾಗತದೀನಾರ್ತಪರಿತ್ರಾಣಪರಾಯಣೇ ।\\
ಸರ್ವಸ್ಯಾರ್ತಿಹರೇ ದೇವಿ ನಾರಾಯಣಿ ನಮೋಽಸ್ತು ತೇ ॥೧೨॥

ಹಂಸಯುಕ್ತವಿಮಾನಸ್ಥೇ ಬ್ರಹ್ಮಾಣೀರೂಪಧಾರಿಣಿ ।\\
ಕೌಶಾಂಭಃಕ್ಷರಿಕೇ ದೇವಿ ನಾರಾಯಣಿ ನಮೋಽಸ್ತು ತೇ ॥೧೩॥

ತ್ರಿಶೂಲಚಂದ್ರಾಹಿಧರೇ ಮಹಾವೃಷಭವಾಹಿನಿ ।\\
ಮಾಹೇಶ್ವರೀಸ್ವರೂಪೇಣ ನಾರಾಯಣಿ ನಮೋಽಸ್ತುತೇ ॥೧೪॥

ಮಯೂರಕುಕ್ಕುಟವೃತೇ ಮಹಾಶಕ್ತಿಧರೇಽನಘೇ ।\\
ಕೌಮಾರೀರೂಪಸಂಸ್ಥಾನೇ ನಾರಾಯಣಿ ನಮೋಽಸ್ತು ತೇ ॥೧೫॥

ಶಂಖಚಕ್ರಗದಾಶಾರ್ಙ್ಗಗೃಹೀತಪರಮಾಯುಧೇ ।\\
ಪ್ರಸೀದ ವೈಷ್ಣವೀರೂಪೇ ನಾರಾಯಣಿ ನಮೋಽಸ್ತು ತೇ ॥೧೬॥

ಗೃಹೀತೋಗ್ರಮಹಾಚಕ್ರೇ ದಂಷ್ಟ್ರೋದ್ಧೃತವಸುಂಧರೇ ।\\
ವರಾಹರೂಪಿಣಿ ಶಿವೇ ನಾರಾಯಣಿ ನಮೋಽಸ್ತು ತೇ ॥೧೭॥

ನೃಸಿಂಹರೂಪೇಣೋಗ್ರೇಣ ಹಂತುಂ ದೈತ್ಯಾನ್ ಕೃತೋದ್ಯಮೇ ।\\
ತ್ರೈಲೋಕ್ಯತ್ರಾಣಸಹಿತೇ ನಾರಾಯಣಿ ನಮೋಽಸ್ತು ತೇ ॥೧೮॥

ಕಿರೀಟಿನಿ ಮಹಾವಜ್ರೇ ಸಹಸ್ರನಯನೋಜ್ಜ್ವಲೇ ।\\
ವೃತ್ರಪ್ರಾಣಹರೇ ಚೈಂದ್ರಿ ನಾರಾಯಣಿ ನಮೋಽಸ್ತು ತೇ ॥೧೯॥

ಶಿವದೂತೀ ಸ್ವರೂಪೇಣ ಹತದೈತ್ಯ ಮಹಾಬಲೇ ।\\
ಘೋರರೂಪೇ ಮಹಾರಾವೇ ನಾರಾಯಣಿ ನಮೋಽಸ್ತು ತೇ ॥೨೦॥

ದಂಷ್ಟ್ರಾ ಕರಾಲವದನೇ ಶಿರೋಮಾಲಾವಿಭೂಷಣೇ ।\\
ಚಾಮುಂಡೇ ಮುಂಡಮಥನೇ ನಾರಾಯಣಿ ನಮೋಽಸ್ತು ತೇ ॥೨೧॥

ಲಕ್ಷ್ಮಿ ಲಜ್ಜೇ ಮಹಾವಿದ್ಯೇ ಶ್ರದ್ಧೇ ಪುಷ್ಟಿ ಸ್ವಧೇ ಧ್ರುವೇ ।\\
ಮಹಾರಾತ್ರಿ ಮಹಾಮಾಯೇ ನಾರಾಯಣಿ ನಮೋಽಸ್ತು ತೇ ॥೨೨॥

ಮೇಧೇ ಸರಸ್ವತಿ ವರೇ ಭೂತಿ ಬಾಭ್ರವಿ ತಾಮಸಿ ।\\
ನಿಯತೇ ತ್ವಂ ಪ್ರಸೀದೇಶೇ ನಾರಾಯಣಿ ನಮೋಽಸ್ತುತೇ ॥೨೩॥

ಸರ್ವಸ್ವರೂಪೇ ಸರ್ವೇಶೇ ಸರ್ವಶಕ್ತಿಸಮನ್ವಿತೇ ।\\
ಭಯೇಭ್ಯಸ್ತ್ರಾಹಿ ನೋ ದೇವಿ ದುರ್ಗೇ ದೇವಿ ನಮೋಽಸ್ತು ತೇ ॥೨೪॥

ಏತತ್ತೇ ವದನಂ ಸೌಮ್ಯಂ ಲೋಚನತ್ರಯಭೂಷಿತಂ ।\\
ಪಾತು ನಃ ಸರ್ವಭೂತೇಭ್ಯಃ ಕಾತ್ಯಾಯನಿ ನಮೋಽಸ್ತು ತೇ ॥೨೫॥

ಜ್ವಾಲಾ ಕರಾಲಮತ್ಯುಗ್ರಮಶೇಷಾಸುರ ಸೂದನಂ ।\\
ತ್ರಿಶೂಲಂ ಪಾತು ನೋ ಭೀತೇರ್ಭದ್ರಕಾಲಿ ನಮೋಽಸ್ತು ತೇ ॥೨೬॥

ಹಿನಸ್ತಿ ದೈತ್ಯತೇಜಾಂಸಿ ಸ್ವನೇನಾಪೂರ್ಯ ಯಾ ಜಗತ್ ।\\
ಸಾ ಘಂಟಾ ಪಾತು ನೋ ದೇವಿ ಪಾಪೇಭ್ಯೋ ನಃ ಸುತಾನಿವ ॥೨೭॥

ಅಸುರಾಸೃಗ್ವಸಾ ಪಂಕ ಚರ್ಚಿತಸ್ತೇ ಕರೋಜ್ಜ್ವಲಃ ।\\
ಶುಭಾಯ ಖಡ್ಗೋ ಭವತು ಚಂಡಿಕೇ ತ್ವಾಂ ನತಾ ವಯಂ ॥೨೮॥

ರೋಗಾನಶೇಷಾನಪಹಂಸಿ ತುಷ್ಟಾ\\
        ರುಷ್ಟಾ ತು ಕಾಮಾನ್ ಸಕಲಾನಭೀಷ್ಟಾನ್ ।\\
ತ್ವಾಮಾಶ್ರಿತಾನಾಂ ನ ವಿಪನ್ನರಾಣಾಂ\\
        ತ್ವಾಮಾಶ್ರಿತಾ ಹ್ಯಾಶ್ರಯತಾಂ ಪ್ರಯಾಂತಿ ॥೨೯॥

ಏತತ್ಕೃತಂ ಯತ್ಕದನಂ ತ್ವಯಾದ್ಯ\\
        ಧರ್ಮದ್ವಿಷಾಂ ದೇವಿ ಮಹಾಸುರಾಣಾಂ ।\\
ರೂಪೈರನೇಕೈರ್ಬಹುಧಾತ್ಮಮೂರ್ತಿಂ\\
        ಕೃತ್ವಾಂಬಿಕೇ ತತ್ಪ್ರಕರೋತಿ ಕಾನ್ಯಾ ॥೩೦॥

ವಿದ್ಯಾಸು ಶಾಸ್ತ್ರೇಷು ವಿವೇಕದೀಪೇ-\\
      ಷ್ವಾದ್ಯೇಷು ವಾಕ್ಯೇಷು ಚ ಕಾ ತ್ವದನ್ಯಾ ।\\
ಮಮತ್ವಗರ್ತೇಽತಿಮಹಾಂಧಕಾರೇ\\
      ವಿಭ್ರಾಮಯಸ್ಯೇತದತೀವ ವಿಶ್ವಂ ॥೩೧॥

ರಕ್ಷಾಂಸಿ ಯತ್ರೋಗ್ರವಿಷಾಶ್ಚ ನಾಗಾ\\
        ಯತ್ರಾರಯೋ ದಸ್ಯುಬಲಾನಿ ಯತ್ರ ।\\
ದಾವಾನಲೋ ಯತ್ರ ತಥಾಬ್ಧಿಮಧ್ಯೇ\\
        ತತ್ರ ಸ್ಥಿತಾ ತ್ವಂ ಪರಿಪಾಸಿ ವಿಶ್ವಂ ॥೩೨॥

ವಿಶ್ವೇಶ್ವರಿ ತ್ವಂ ಪರಿಪಾಸಿ ವಿಶ್ವಂ\\
        ವಿಶ್ವಾತ್ಮಿಕಾ ಧಾರಯಸೀಹ ವಿಶ್ವಂ ।\\
ವಿಶ್ವೇಶವಂದ್ಯಾ ಭವತೀ ಭವಂತಿ\\
        ವಿಶ್ವಾಶ್ರಯಾ ಯೇ ತ್ವಯಿ ಭಕ್ತಿನಮ್ರಾಃ ॥೩೩॥

ದೇವಿ ಪ್ರಸೀದ ಪರಿಪಾಲಯ ನೋಽರಿಭೀತೇ-\\
      ರ್ನಿತ್ಯಂ ಯಥಾಸುರವಧಾದಧುನೈವ ಸದ್ಯಃ ।\\
ಪಾಪಾನಿ ಸರ್ವಜಗತಾಂ ಪ್ರಶಮಂ ನಯಾಶು\\
        ಉತ್ಪಾತಪಾಕಜನಿತಾಂಶ್ಚ ಮಹೋಪಸರ್ಗಾನ್ ॥೩೪॥

ಪ್ರಣತಾನಾಂ ಪ್ರಸೀದ ತ್ವಂ ದೇವಿ ವಿಶ್ವಾರ್ತಿಹಾರಿಣಿ ।\\
ತ್ರೈಲೋಕ್ಯವಾಸಿನಾಮೀಡ್ಯೇ ಲೋಕಾನಾಂ ವರದಾ ಭವ ॥೩೫॥

ದೇವ್ಯುವಾಚ ॥೩೬॥\\
ವರದಾಹಂ ಸುರಗಣಾ ವರಂ ಯಂ ಮನಸೇಚ್ಛಥ ।\\
ತಂ ವೃಣುಧ್ವಂ ಪ್ರಯಚ್ಛಾಮಿ ಜಗತಾಮುಪಕಾರಕಂ ॥೩೭॥

ದೇವಾ ಊಚುಃ ॥೩೮॥\\
ಸರ್ವಾಬಾಧಾಪ್ರಶಮನಂ ತ್ರೈಲೋಕ್ಯಸ್ಯಾಖಿಲೇಶ್ವರಿ ।\\
ಏವಮೇವ ತ್ವಯಾ ಕಾರ್ಯಮಸ್ಮದ್ವೈರಿವಿನಾಶನಂ ॥೩೯॥
\section{ ಶ್ರೀ ಅನಘಾಕವಚಾಷ್ಟಕಂ }
ಶಿರೋ ಮೇ ಅನಘಾ ಪಾತು ಭಾಲಂ ಮೇ ದತ್ತಭಾಮಿನೀ।\\
ಭ್ರೂಮಧ್ಯಂ ಯೋಗಿನೀ ಪಾತು ನೇತ್ರೇ ಪಾತು ಸುದರ್ಶಿನೀ॥೧॥

ನಾಸಾರಂಧ್ರದ್ವಯಂ ಪಾತು ಯೋಗೇಶೀ ಭಕ್ತವತ್ಸಲಾ~।\\
ಮುಖಂ ಮೇ ಮಧುವಾಕ್ಪಾತು ದತ್ತಚಿತ್ತವಿಹಾರಿಣೀ॥೨॥

ತ್ರಿಕಂಠೀ ಪಾತು ಮೇ ಕಂಠಂ ವಾಚಂ ವಾಚಸ್ಪತಿಪ್ರಿಯಾ।\\
ಸ್ಕಂಧೌ ಮೇ ತ್ರಿಗುಣಾ ಪಾತು ಭುಜೌ ಕಮಲಧಾರಿಣೀ॥೩॥

ಕರೌ ಸೇವಾರತಾ ಪಾತು ಹೃದಯಂ ಮಂದಹಾಸಿನೀ।\\
ಉದರಂ ಅನ್ನದಾ ಪಾತು ಸ್ವಯಂಜಾ ನಾಭಿಮಂಡಲಂ॥೪॥

ಕಮನೀಯಾ ಕಟಿಂ ಪಾತು ಗುಹ್ಯಂ ಗುಹ್ಯೇಶ್ವರೀ ಸದಾ।\\
ಊರೂ ಮೇ ಪಾತು ಜಂಭಘ್ನೀ ಜಾನುನೀ ರೇಣುಕೇಷ್ಟದಾ ॥೫॥

ಪಾದೌ ಪಾದಸ್ಥಿತಾ ಪಾತು ಪುತ್ರದಾ ವೈ ಖಿಲಂ ವಪುಃ।\\
ವಾಮಗಾ ಪಾತು ವಾಮಾಂಗಂ ದಕ್ಷಾಂಗಂ ಗುರುಗಾಮಿನೀ॥೬॥

ಗೃಹಂ ಮೇ ದತ್ತಗೃಹಿಣೀ ಬಾಹ್ಯೇ ಸರ್ವಾತ್ಮಿಕಾವತು।\\
ತ್ರಿಕಾಲೇ ಸರ್ವದಾ ರಕ್ಷೇತ್ ಪತಿಶುಶ್ರೂಷಣೋತ್ಸುಕಾ ॥೭॥
\newpage
ಜಾಯಾಂ ಮೇ ದತ್ತವಾಮಾಂಗೀ ಅಷ್ಟಪುತ್ರಾ ಸುತೋವತು।\\
ಗೋತ್ರಮತ್ರಿಸ್ನುಷಾ ರಕ್ಷೇದ್ ಅನಘಾ ಭಕ್ತರಕ್ಷಣೀ॥೮॥

ಯಃ ಪಠೇದನಘಾಕವಚಂ ನಿತ್ಯಂ ಭಕ್ತಿಯುತೋ ನರಃ।\\
ತಸ್ಮೈ ಭವತ್ಯನಘಾಂಬಾ ವರದಾ ಸರ್ವಭಾಗ್ಯದಾ॥೯॥

\section{ಶೀತಲಾ ಕವಚಮ್}
ಅಸ್ಯ ಶ್ರೀ ಶೀತಲಾಕವಚಸ್ಯ ಮಹೇಶ್ವರಃ ಋಷಿಃ । ಅನುಷ್ಟುಪ್ ಛನ್ದಃ ।
ಶೀತಲಾ ದೇವತಾ। ಲಕ್ಷ್ಮೀಬೀಜಂ । ರಮಾ ಶಕ್ತಿಃ ।ತಾರಂ ಕೀಲಕಮ್।
ಲೂತಾವಿಸ್ಫೋಟಕಾದೀನಾಂ ಶಾಂತ್ಯರ್ಥೇ ಜಪೇ ವಿನಿಯೋಗಃ ।

\dhyana{ಉದ್ಯತ್ಸೂರ್ಯನಿಭಾಂ ನವೇಂದುಮುಕುಟಾಂ ಸೂರ್ಯಾಗ್ನಿನೇತ್ರೋಜ್ಜ್ವಲಾಂ\\
ನಾನಾಗಂಧ ವಿಲೇಪನಾಂ ಮೃದುತನುಂ ದಿವ್ಯಾಂಬರಾಲಂಕೃತಾಮ್ ।\\
ದೋರ್ಭ್ಯಾಂ ಸಂದಧತೀಂ ವರಾಭಯಯುಗಂ ವಾಹೇ ಸ್ಥಿತಾಂ ರಾಸಭೇ\\
ಭಕ್ತಾಭೀಷ್ಟ-ಫಲ-ಪ್ರದಾಂ ಭಗವತೀಂ ಶ್ರೀಶೀತಲಾಂ ತ್ವಾಂ ಭಜೇ ॥}

ಶೀತಲಾ ಪಾತು ಮೇ ಪ್ರಾಣೇ ರುನುಕೀ ಪಾತು ಚಾಪಾನೇ ।\\
ಸಮಾನೇ ಝುನುಕೀ ಪಾತು ಉದಾನೇ ಪಾತು ಮನ್ದಲಾ ॥೧॥

ವ್ಯಾನೇ ಚ ಸೇಢಲಾ ಪಾತು ಮನೋ ಮೇ ಶಾಂಕರೀ ತಥಾ ।\\
ಪಾತು ಮಾಮಿಂದ್ರಿಯಾನ್ ಸರ್ವಾನ್ ಶ್ರೀದುರ್ಗಾ ವಿನ್ಧ್ಯವಾಸಿನೀ ॥೨॥

ಮಮ ಪಾತು ಶಿರೋ ದುರ್ಗಾ ಕಮಲಾ ಪಾತು ಮಸ್ತಕಮ್ ।\\
ಹ್ರೀಂ ಮೇ ಪಾತು ಭ್ರುವೋರ್ಮಧ್ಯೇ ಭವಾನೀ ಭುವನೇಶ್ವರೀ ॥\\
ಪಾತು ಮೇ ಮಧುಮತೀ ದೇವೀ ಓಂಕಾರಂ ಭೃಕುಟಿದ್ವಯಮ್ ॥೩॥

ನಾಸಿಕಾಂ ಶಾರದಾ ಪಾತು ತಮಸಾ ವರ್ತ್ಮಸಂಯುತಮ್ ।\\
ನೇತ್ರೇ ಜ್ವಾಲಾಮುಖೀ ಪಾತು ಭೀಷಣಾ ಪಾತು ಶ್ರುತೀ ಮೇ ॥೪॥

ಕಪೋಲೌ ಕಾಲಿಕಾ ಪಾತು ಸುಮುಖೀ ಪಾತು ಚೋಷ್ಠಯೋಃ ।\\
ಸನ್ಧ್ಯಯೋಃ ತ್ರಿಪುರಾ ಪಾತು ದನ್ತೇ ಚ ರಕ್ತದನ್ತಿಕಾ ॥೫॥

ಜಿಹ್ವಾಂ ಸರಸ್ವತೀ ಪಾತು ತಾಲುಕೇ ಚ ವಾಗ್ವಾದಿನೀ ।\\
ಕಣ್ಠೇ ಪಾತು ತು ಮಾತಂಗೀ ಗ್ರೀವಾಯಾಂ ಭದ್ರಕಾಲಿಕಾ ॥೬॥

ಸ್ಕನ್ಧೌ ಚ ಪಾತು ಮೇ ಛಿನ್ನಾ ಕಕುದಂ ಸ್ಕನ್ದ-ಮಾತರಃ ।\\
ಬಾಹುಯುಗ್ಮೌ ಚ ಮೇ ಪಾತು ಶ್ರೀದೇವೀ ಬಗಲಾಮುಖೀ ॥೭॥

ಕರೌ ಮೇ ಭೈರವೀ ಪಾತು ಪೃಷ್ಠೇ ಪಾತು ಧನುರ್ಧರೀ।\\
ವಕ್ಷಃಸ್ಥಲೇ ಚ ಮೇ ಪಾತು ದುರ್ಗಾ ಮಹಿಷಮರ್ದಿನೀ ॥೮॥

ಹೃದಯೇ ಲಲಿತಾ ಪಾತು ಕುಕ್ಷೌ ಪಾತು ಮಹೇಶ್ವರೀ ।\\
ಪಾರ್ಶ್ವೌ ಚ ಗಿರಿಜಾ ಪಾತು ಚಾನ್ನಪೂರ್ಣಾ ತು ಚೋದರಮ್ ॥೯॥

ನಾಭಿಂ ನಾರಾಯಣೀ ಪಾತು ಕಟಿಂ ಮೇ ಸರ್ವಮಂಗಲಾ ।\\
ಜಂಘಯೋ ರ್ಮೇ ಸದಾ ಪಾತು ದೇವೀ ಕಾತ್ಯಾಯನೀ ಪರಾ ॥೧೦॥

ಬ್ರಹ್ಮಾಣೀ ಶಿಶ್ನಂ ಪಾತು ವೃಷಣಂ ಪಾತು ಕಪಾಲಿನೀ ।\\
ಗುಹ್ಯಂ ಗುಹ್ಯೇಶ್ವರೀ ಪಾತು ಜಾನುನೋರ್ಜಗದೀಶ್ವರೀ ॥೧೧॥

ಪಾತು ಗುಲ್ಫೌ ತು ಕೌಮಾರೀ ಪಾದಪೃಷ್ಠಂ ತು ವೈಷ್ಣವೀ।\\
ವಾರಾಹೀ ಪಾತು ಪಾದಾಗ್ರೇ ಐಂದ್ರಾಣೀ ಸರ್ವಮರ್ಮಸು ॥೧೨॥

ಮಾರ್ಗೇ ರಕ್ಷತು ಚಾಮುಂಡಾ ವನೇ ತು ವನವಾಸಿನೀ ।\\
ಜಲೇ ಚ ವಿಜಯಾ ರಕ್ಷೇತ್ ವಹ್ನೌ ಮೇ ಚಾಪರಾಜಿತಾ ॥೧೩॥

ರಣೇ ಕ್ಷೇಮಂಕರೀ ರಕ್ಷೇತ್ ಸರ್ವತ್ರ ಸರ್ವಮಂಗಲಾ ।\\
ಭವಾನೀ ಪಾತು ಬಂಧೂನ್ ಮೇ ಭಾರ್ಯಾಂ ರಕ್ಷತು ಚಾಂಬಿಕಾ ॥೧೪॥

ಪುತ್ರಾನ್ ರಕ್ಷತು ಮಾಹೇಂದ್ರೀ ಕನ್ಯಕಾಂ ಪಾತು ಶಾಂಭವೀ ।\\
ಗೃಹೇಷು ಸರ್ವಕಲ್ಯಾಣೀ ಪಾತು ನಿತ್ಯಂ ಮಹೇಶ್ವರೀ ॥೧೫॥

ಪೂರ್ವೇ ಕಾದಂಬರೀ ಪಾತು ವಹ್ನೌ ಶುಕ್ಲೇಶ್ವರೀ ತಥಾ ।\\
ದಕ್ಷಿಣೇ ಕರಾಲಿನೀ ಪಾತು, ಪ್ರೇತಾರೂಢಾ ತು ನೈರ್ಋತೇ ॥೧೬॥

ಪಾಶಹಸ್ತಾ ಪಶ್ಚಿಮೇ ಪಾಯಾತ್ ವಾಯವ್ಯೇ ಮೃಗವಾಹಿನೀ ।\\
ಪಾತು ಮೇ ಚೋತ್ತರೇ ದೇವೀ ಯಕ್ಷಿಣೀ ಸಿಂಹವಾಹಿನೀ ।\\
ಈಶಾನೇ ಶೂಲಿನೀ ಪಾತು ಊರ್ಧ್ವೇ ಚ ಖಗಗಾಮಿನೀ ॥೧೭॥

ಅಧಸ್ತಾತ್ ವೈಷ್ಣವೀ ಪಾತು, ಸರ್ವತ್ರ ನಾರಸಿಂಹಿಕಾ ।\\
ಪ್ರಭಾತೇ ಸುಂದರೀ ಪಾತು ಮಧ್ಯಾಹ್ನೇ ಜಗದಮ್ಬಿಕಾ ॥೧೮॥

ಸಾಯಾಹ್ನೇ ಚಂಡಿಕಾ ಪಾತು ನಿಶೀಥೇಽತ್ರ ನಿಶಾಚರೀ ।\\
ನಿಶಾಂತೇ ಖೇಚರೀ ಪಾತು ಸರ್ವದಾ ದಿವ್ಯಯೋಗಿನೀ ॥೧೯॥

ವಾಯೌ ಮಾಂ ಪಾತು ವೇತಾಲೀ ವಾಹನೇ ವಜ್ರಧಾರಿಣೀ ।\\
ಸಿಂಹಾ ಸಿಂಹಾಸನೇ ಪಾತು ಶಯ್ಯಾಂ ಚ ಭಗಮಾಲಿನೀ ॥೨೦॥

ಸರ್ವರೋಗೇಷು ಮಾಂ ಪಾತು ಕಾಲರಾತ್ರಿಸ್ವರೂಪಿಣೀ ।\\
ಯಕ್ಷೇಭ್ಯೋ ಯಾಕಿನೀ ಪಾತು ರಾಕ್ಷಸೇ ಡಾಕಿನೀ ತಥಾ ॥೨೧॥

ಭೂತಪ್ರೇತಪಿಶಾಚೇಭ್ಯೋ ಹಾಕಿನೀ ಪಾತು ಮಾಂ ಸದಾ ।\\
ಮಂತ್ರಂ ಮಂತ್ರಾಭಿಚಾರೇಷು ಶಾಕಿನೀ ಪಾತು ಮಾಂ ಸದಾ ॥೨೨॥

ಸರ್ವತ್ರ ಸರ್ವದಾ ಪಾತು ಶ್ರೀದೇವೀ ಗಿರಿಜಾತ್ಮಜಾ ।\\
ಇತ್ಯೇತತ್ ಕಥಿತಂ ಗುಹ್ಯಂ ಶೀತಲಾಕವಚಮುತ್ತಮಮ್ ॥೨೩॥

\chapter*{\center ಸ್ತೋತ್ರಾಣಿ}
\thispagestyle{empty}
\section{ಶ್ರೀದಕ್ಷಿಣಾಮೂರ್ತಿ ಅಷ್ಟೋತ್ತರ ಶತನಾಮಸ್ತೋತ್ರಂ }
\dhyana{ವಟವೃಕ್ಷ ತಟಾಸೀನಂ ಯೋಗಿ ಧ್ಯೇಯಾಂಘ್ರಿ ಪಂಕಜಂ।\\
ಶರಚ್ಚಂದ್ರ ನಿಭಂ ಪೂಜ್ಯಂ ಜಟಾಮುಕುಟ ಮಂಡಿತಂ ॥

ಗಂಗಾಧರಂ ಲಲಾಟಾಕ್ಷಂ ವ್ಯಾಘ್ರ ಚರ್ಮಾಂಬರಾವೃತಂ।\\
ನಾಗಭೂಷಂ ಪರಂಬ್ರಹ್ಮ ದ್ವಿಜರಾಜಾವತಂಸಕಂ ॥

ಅಕ್ಷಮಾಲಾ ಜ್ಞಾನಮುದ್ರಾ ವೀಣಾ ಪುಸ್ತಕ ಶೋಭಿತಂ।\\
ಶುಕಾದಿ ವೃದ್ಧ ಶಿಷ್ಯಾಢ್ಯಂ ವೇದ ವೇದಾಂತಗೋಚರಂ॥\\
ಯುವಾನಂ ಮನ್ಮಥಾರಾತಿಂ ದಕ್ಷಿಣಾಮೂರ್ತಿಮಾಶ್ರಯೇ॥}

ಓಂ ವಿದ್ಯಾರೂಪೀ ಮಹಾಯೋಗೀ ಶುದ್ಧ ಜ್ಞಾನೀ ಪಿನಾಕಧೃತ್~।\\
ರತ್ನಾಲಂಕೃತ ಸರ್ವಾಂಗೀ ರತ್ನಮೌಳಿರ್ಜಟಾಧರಃ ॥೧॥

ಗಂಗಾಧಾರ್ಯಚಲಾವಾಸೀ ಮಹಾಜ್ಞಾನೀ ಸಮಾಧಿಕೃತ್।\\
ಅಪ್ರಮೇಯೋ ಯೋಗನಿಧಿಸ್ತಾರಕೋ ಭಕ್ತವತ್ಸಲಃ॥೨॥

ಬ್ರಹ್ಮರೂಪೀ ಜಗದ್ವ್ಯಾಪೀ ವಿಷ್ಣುಮೂರ್ತಿಃ ಪುರಾತನಃ~।\\
ಉಕ್ಷವಾಹಶ್ಚರ್ಮವಾಸಾಃ ಪೀತಾಂಬರ ವಿಭೂಷಣಃ॥೩॥

ಮೋಕ್ಷದಾಯೀ ಮೋಕ್ಷ ನಿಧಿಶ್ಚಾಂಧಕಾರಿರ್ಜಗತ್ಪತಿಃ।\\
ವಿದ್ಯಾಧಾರೀ ಶುಕ್ಲ ತನುಃ ವಿದ್ಯಾದಾಯೀ ಗಣಾಧಿಪಃ॥೪॥

ಪ್ರೌಢಾಪಸ್ಮೃತಿ ಸಂಹರ್ತಾ ಶಶಿಮೌಳಿರ್ಮಹಾಸ್ವನಃ~।\\
ಸಾಮ ಪ್ರಿಯೋಽವ್ಯಯಃ ಸಾಧುಃ ಸರ್ವ ವೇದೈರಲಂಕೃತಃ ॥೫॥

ಹಸ್ತೇ ವಹ್ನಿಧರಃ ಶ್ರೀಮಾನ್ ಮೃಗಧಾರೀ ವಶಂಕರಃ~।\\
ಯಜ್ಞನಾಥಃ ಕ್ರತುಧ್ವಂಸೀ ಯಜ್ಞಭೋಕ್ತಾ ಯಮಾಂತಕಃ॥೬॥

ಭಕ್ತಾನುಗ್ರಹ ಮೂರ್ತಿಶ್ಚ ಭಕ್ತಸೇವ್ಯೋ ವೃಷಧ್ವಜಃ~।\\
ಭಸ್ಮೋದ್ಧೂಲಿತ ಸರ್ವಾಂಗಃ ಚಾಕ್ಷಮಾಲಾಧರೋ ಮಹಾನ್ ॥೭॥

ತ್ರಯೀಮೂರ್ತಿಃ ಪರಂಬ್ರಹ್ಮ ನಾಗರಾಜೈರಲಂಕೃತಃ~।\\
ಶಾಂತರೂಪೋ ಮಹಾಜ್ಞಾನೀ ಸರ್ವ ಲೋಕ ವಿಭೂಷಣಃ ॥೮॥

ಅರ್ಧನಾರೀಶ್ವರೋ ದೇವೋ ಮುನಿಸೇವ್ಯಸ್ಸುರೋತ್ತಮಃ~।\\
ವ್ಯಾಖ್ಯಾನದೇವೋ ಭಗವಾನ್ ರವಿ ಚಂದ್ರಾಗ್ನಿ ಲೋಚನಃ ॥೯॥

ಜಗದ್ಗುರುರ್ಮಹಾದೇವೋ ಮಹಾನಂದ ಪರಾಯಣಃ~।\\
ಜಟಾಧಾರೀ ಮಹಾಯೋಗೀ ಜ್ಞಾನಮಾಲೈರಲಂಕೃತಃ ॥೧೦॥

ವ್ಯೋಮಗಂಗಾ ಜಲ ಸ್ಥಾನಃ ವಿಶುದ್ಧೋ ಯತಿರೂರ್ಜಿತಃ~।\\
ತತ್ತ್ವಮೂರ್ತಿರ್ಮಹಾಯೋಗೀ ಮಹಾಸಾರಸ್ವತಪ್ರದಃ ॥೧೧॥

ವ್ಯೋಮಮೂರ್ತಿಶ್ಚ ಭಕ್ತಾನಾಂ ಇಷ್ಟಃ ಕಾಮಫಲಪ್ರದಃ~।\\
ಪರಮೂರ್ತಿಃ ಚಿತ್ಸ್ವರೂಪೀ ತೇಜೋಮೂರ್ತಿರನಾಮಯಃ ॥೧೨॥

ವೇದವೇದಾಂಗ ತತ್ತ್ವಜ್ಞಃ ಚತುಃಷಷ್ಟಿ ಕಲಾನಿಧಿಃ~।\\
ಭವರೋಗ ಭಯಧ್ವಂಸೀ ಭಕ್ತಾನಾಮಭಯಪ್ರದಃ ॥೧೩॥

ನೀಲಗ್ರೀವೋ ಲಲಾಟಾಕ್ಷೋ ಗಜ ಚರ್ಮಾಗತಿಪ್ರದಃ~।\\
ಅರಾಗೀ ಕಾಮದಶ್ಚಾಥ ತಪಸ್ವೀ ವಿಷ್ಣುವಲ್ಲಭಃ ॥೧೪॥

ಬ್ರಹ್ಮಚಾರೀ ಚ ಸನ್ಯಾಸೀ ಗೃಹಸ್ಥಾಶ್ರಮ ಕಾರಣಃ~।\\
ದಾಂತಃ ಶಮವತಾಂ ಶ್ರೇಷ್ಠೋ ಸತ್ಯರೂಪೋ ದಯಾಪರಃ ॥೧೫॥

ಯೋಗಪಟ್ಟಾಭಿರಾಮಶ್ಚ ವೀಣಾಧಾರೀ ವಿಚೇತನಃ~।\\
ಮತಿಪ್ರಜ್ಞಾ ಸುಧಾಧಾರೀ ಮುದ್ರಾಪುಸ್ತಕ ಧಾರಣಃ ॥೧೬॥

ವೇತಾಲಾದಿ ಪಿಶಾಚೌಘ ರಾಕ್ಷಸೌಘ ವಿನಾಶನಃ~।\\
ರಾಜ ಯಕ್ಷ್ಮಾದಿ ರೋಗಾಣಾಂ ವಿನಿಹಂತಾ ಸುರೇಶ್ವರಃ ॥೧೭॥

\authorline{॥ಇತಿ ಶ್ರೀ ದಕ್ಷಿಣಾಮೂರ್ತಿ ಅಷ್ಟೋತ್ತರ ಶತನಾಮ ಸ್ತೋತ್ರಂ ಸಂಪೂರ್ಣಂ ॥}
\section{ಶ್ರೀವಿಘ್ನೇಶ್ವರಾಷ್ಟೋತ್ತರ ಶತನಾಮಸ್ತೋತ್ರಂ}
\dhyana{ಗಜವದನಮಚಿಂತ್ಯಂ ತೀಕ್ಷ್ಣದಂಷ್ಟ್ರಂ ತ್ರಿಣೇತ್ರಂ\\
ಬೃಹದುದರಮಶೇಷಂ ಭೂತಿರೂಪಂ ಪುರಾಣಮ್~।\\
ಅಮರವರಸುಪೂಜ್ಯಂ ರಕ್ತವರ್ಣಂ ಸುರೇಶಂ\\
 ಪಶುಪತಿಸುತಮೀಶಂ ವಿಘ್ನರಾಜಂ ನಮಾಮಿ ॥}

ವಿನಾಯಕೋ ವಿಘ್ನರಾಜೋ ಗೌರೀಪುತ್ರೋ ಗಣೇಶ್ವರಃ~।\\
ಸ್ಕಂದಾಗ್ರಜೋಽವ್ಯಯಃ ಪೂತೋ ದಕ್ಷೋಽಧ್ಯಕ್ಷೋ ದ್ವಿಜಪ್ರಿಯಃ ॥೧॥

ಅಗ್ನಿಗರ್ವಚ್ಛಿದಿಂದ್ರಶ್ರೀಪ್ರದೋ ವಾಣೀಬಲಪ್ರದಃ~।\\
ಸರ್ವಸಿದ್ಧಿಪ್ರದಶ್ಶರ್ವತನಯಃ ಶರ್ವರೀಪ್ರಿಯಃ ॥೨॥

ಸರ್ವಾತ್ಮಕಃ ಸೃಷ್ಟಿಕರ್ತಾ ದೇವಾನೀಕಾರ್ಚಿತಶ್ಶಿವಃ~।\\
ಶುದ್ಧೋ ಬುದ್ಧಿಪ್ರಿಯಶ್ಶಾಂತೋ ಬ್ರಹ್ಮಚಾರೀ ಗಜಾನನಃ ॥೩॥

ದ್ವೈಮಾತ್ರೇಯೋ ಮುನಿಸ್ತುತ್ಯೋ ಭಕ್ತವಿಘ್ನವಿನಾಶನಃ~।\\
ಏಕದಂತಶ್ಚತುರ್ಬಾಹುಶ್ಚತುರಶ್ಶಕ್ತಿಸಂಯುತಃ ॥೪॥

ಲಂಬೋದರಶ್ಶೂರ್ಪಕರ್ಣೋ ಹರಿರ್ಬ್ರಹ್ಮ ವಿದುತ್ತಮಃ~।\\
ಕಾಲೋ ಗ್ರಹಪತಿಃ ಕಾಮೀ ಸೋಮಸೂರ್ಯಾಗ್ನಿಲೋಚನಃ ॥೫॥

ಪಾಶಾಂಕುಶಧರಶ್ಚಂಡೋ ಗುಣಾತೀತೋ ನಿರಂಜನಃ~।\\
ಅಕಲ್ಮಷಸ್ಸ್ವಯಂಸಿದ್ಧಸ್ಸಿದ್ಧಾರ್ಚಿತಪದಾಂಬುಜಃ ॥೬॥

ಬೀಜಪೂರಫಲಾಸಕ್ತೋ ವರದಶ್ಶಾಶ್ವತಃ ಕೃತೀ~।\\
ದ್ವಿಜಪ್ರಿಯೋ ವೀತಭಯೋ ಗದೀ ಚಕ್ರೀಕ್ಷುಚಾಪಧೃತ್ ॥೭॥

ಶ್ರೀದೋಽಜ ಉತ್ಪಲಕರಃ ಶ್ರೀಪತಿಃ ಸ್ತುತಿಹರ್ಷಿತಃ~।\\
ಕುಲಾದ್ರಿಭೇತ್ತಾ ಜಟಿಲಃ ಕಲಿಕಲ್ಮಷನಾಶನಃ ॥೮॥

ಚಂದ್ರಚೂಡಾಮಣಿಃ ಕಾಂತಃ ಪಾಪಹಾರೀ ಸಮಾಹಿತಃ~।\\
ಆಶ್ರಿತಶ್ಶ್ರೀಕರಸ್ಸೌಮ್ಯೋ ಭಕ್ತವಾಂಛಿತದಾಯಕಃ ॥೯॥

ಶಾಂತಃ ಕೈವಲ್ಯಸುಖದಸ್ಸಚ್ಚಿದಾನಂದವಿಗ್ರಹಃ~।\\
ಜ್ಞಾನೀ ದಯಾಯುತೋ ದಾಂತೋ ಬ್ರಹ್ಮ ದ್ವೇಷವಿವರ್ಜಿತಃ ॥೧೦॥

ಪ್ರಮತ್ತದೈತ್ಯಭಯದಃ ಶ್ರೀಕಂಠೋ ವಿಬುಧೇಶ್ವರಃ~।\\
ರಮಾರ್ಚಿತೋವಿಧಿರ್ನಾಗರಾಜಯಜ್ಞೋಪವೀತಕಃ ॥೧೧॥

ಸ್ಥೂಲಕಂಠಃ ಸ್ವಯಂಕರ್ತಾ ಸಾಮಘೋಷಪ್ರಿಯಃ ಪರಃ~।\\
ಸ್ಥೂಲತುಂಡೋಽಗ್ರಣೀರ್ಧೀರೋ ವಾಗೀಶಸ್ಸಿದ್ಧಿದಾಯಕಃ ॥೧೨॥

ದೂರ್ವಾಬಿಲ್ವಪ್ರಿಯೋಽವ್ಯಕ್ತಮೂರ್ತಿರದ್ಭುತಮೂರ್ತಿಮಾನ್~।\\
ಶೈಲೇಂದ್ರತನುಜೋತ್ಸಂಗಖೇಲನೋತ್ಸುಕಮಾನಸಃ ॥೧೩॥

ಸ್ವಲಾವಣ್ಯಸುಧಾಸಾರೋ ಜಿತಮನ್ಮಥವಿಗ್ರಹಃ~।\\
ಸಮಸ್ತಜಗದಾಧಾರೋ ಮಾಯೀ ಮೂಷಕವಾಹನಃ ॥೧೪॥

ಹೃಷ್ಟಸ್ತುಷ್ಟಃ ಪ್ರಸನ್ನಾತ್ಮಾ ಸರ್ವಸಿದ್ಧಿಪ್ರದಾಯಕಃ~।\\
ಅಷ್ಟೋತ್ತರಶತೇನೈವಂ ನಾಮ್ನಾಂ ವಿಘ್ನೇಶ್ವರಂ ವಿಭುಂ ॥೧೫॥

ತುಷ್ಟಾವ ಶಂಕರಃ ಪುತ್ರಂ ತ್ರಿಪುರಂ ಹಂತುಮುದ್ಯತಃ~।\\
ಯಃ ಪೂಜಯೇದನೇನೈವ ಭಕ್ತ್ಯಾ ಸಿದ್ಧಿವಿನಾಯಕಂ ॥೧೬॥

ದೂರ್ವಾದಲೈರ್ಬಿಲ್ವಪತ್ರೈಃ ಪುಷ್ಪೈರ್ವಾ ಚಂದನಾಕ್ಷತೈಃ~।\\
ಸರ್ವಾನ್ಕಾಮಾನವಾಪ್ನೋತಿ ಸರ್ವವಿಘ್ನೈಃ ಪ್ರಮುಚ್ಯತೇ ॥

\authorline{॥ಇತಿ ಶ್ರೀವಿಘ್ನೇಶ್ವರಾಷ್ಟೋತ್ತರಶತನಾಮಸ್ತೋತ್ರಂ ಸಂಪೂರ್ಣಂ ॥}
\section{ ಶ್ರೀಶಂಕರಭಗವತ್ಪಾದಾಷ್ಟೋತ್ತರಶತನಾಮಸ್ತೋತ್ರಂ}
ಶ್ರೀಶಂಕರಾಚಾರ್ಯವರ್ಯೋ  ಬ್ರಹ್ಮಾನಂದಪ್ರದಾಯಕಃ~।\\
ಅಜ್ಞಾನತಿಮಿರಾದಿತ್ಯಃ  ಸುಜ್ಞಾನಾಂಬುಧಿಚಂದ್ರಮಾಃ~॥೧॥

ವರ್ಣಾಶ್ರಮಪ್ರತಿಷ್ಠಾತಾ  ಶ್ರೀಮಾನ್ಮುಕ್ತಿಪ್ರದಾಯಕಃ ।\\
ಶಿಷ್ಯೋಪದೇಶನಿರತೋ ಭಕ್ತಾಭೀಷ್ಟಪ್ರದಾಯಕಃ ॥೨॥

ಸೂಕ್ಷ್ಮತತ್ತ್ವರಹಸ್ಯಜ್ಞಃ ಕಾರ್ಯಾಕಾರ್ಯಪ್ರಬೋಧಕಃ ।\\
ಜ್ಞಾನಮುದ್ರಾಂಚಿತಕರಃ  ಶಿಷ್ಯಹೃತ್ತಾಪಹಾರಕಃ ॥೩॥

ಪರಿವ್ರಾಜಾಶ್ರಮೋದ್ಧರ್ತಾ  ಸರ್ವತಂತ್ರಸ್ವತಂತ್ರಧೀಃ।\\
ಅದ್ವೈತಸ್ಥಾಪನಾಚಾರ್ಯಃ ಸಾಕ್ಷಾಚ್ಛಂಕರರೂಪಧೃತ್~॥೪॥
 
ಷಣ್ಮತಸ್ಥಾಪನಾಚಾರ್ಯಃ  ತ್ರಯೀಮಾರ್ಗಪ್ರಕಾಶಕಃ~।\\
ವೇದವೇದಾಂತತತ್ತ್ವಜ್ಞೋ ದುರ್ವಾದಿಮತಖಂಡನಃ ॥೫॥

ವೈರಾಗ್ಯನಿರತಃ ಶಾಂತಃ ಸಂಸಾರಾರ್ಣವತಾರಕಃ~।\\
ಪ್ರಸನ್ನವದನಾಂಭೋಜಃ ಪರಮಾರ್ಥಪ್ರಕಾಶಕಃ~॥೬॥

ಪುರಾಣಸ್ಮೃತಿಸಾರಜ್ಞೋ ನಿತ್ಯತೃಪ್ತೋ ಮಹಾನ್ ಶುಚಿಃ।\\
ನಿತ್ಯಾನಂದೋ ನಿರಾತಂಕೋ ನಿಃಸಂಗೋ ನಿರ್ಮಲಾತ್ಮಕಃ ॥೭॥

ನಿರ್ಮಮೋ ನಿರಹಂಕಾರೋ ವಿಶ್ವವಂದ್ಯಪದಾಂಬುಜಃ ।\\
ಸತ್ತ್ವಪ್ರಧಾನಃ ಸದ್ಭಾವಃ ಸಂಖ್ಯಾತೀತಗುಣೋಜ್ವಲಃ ॥೮॥

ಅನಘಃ ಸಾರಹೃದಯಃ ಸುಧೀಃ ಸಾರಸ್ವತಪ್ರದಃ।\\
ಸತ್ಯಾತ್ಮಾ ಪುಣ್ಯಶೀಲಶ್ಚ ಸಾಂಖ್ಯಯೋಗವಿಚಕ್ಷಣಃ ॥೯॥

ತಪೋರಾಶಿರ್ಮಹಾತೇಜಾ ಗುಣತ್ರಯವಿಭಾಗವಿತ್~।\\
ಕಲಿಘ್ನಃ ಕಾಲಕರ್ಮಜ್ಞಃ ತಮೋಗುಣನಿವಾರಕಃ॥೧೦॥

ಭಗವಾನ್ ಭಾರತೀಜೇತಾ ಶಾರದಾಹ್ವಾನಪಂಡಿತಃ ।\\
ಧರ್ಮಾಧರ್ಮವಿಭಾಗಜ್ಞೋ ಲಕ್ಷ್ಯಭೇದಪ್ರದರ್ಶಕಃ ॥೧೧॥

ನಾದಬಿಂದುಕಲಾಭಿಜ್ಞೋ ಯೋಗಿಹೃತ್ಪದ್ಮಭಾಸ್ಕರಃ।\\
ಅತೀಂದ್ರಿಯಜ್ಞಾನನಿಧಿಃ ನಿತ್ಯಾನಿತ್ಯವಿವೇಕವಾನ್~॥೧೨॥

ಚಿದಾನಂದಃ ಚಿನ್ಮಯಾತ್ಮಾ ಪರಕಾಯಪ್ರವೇಶಕೃತ್~।\\
ಅಮಾನುಷಚರಿತ್ರಾಢ್ಯಃ ಕ್ಷೇಮದಾಯೀ ಕ್ಷಮಾಕರಃ॥೧೩॥

ಭವ್ಯೋ ಭದ್ರಪ್ರದೋ ಭೂರಿಮಹಿಮಾ ವಿಶ್ವರಂಜಕಃ ।\\
ಸ್ವಪ್ರಕಾಶಃ ಸದಾಧಾರೋ  ವಿಶ್ವಬಂಧುಃ ಶುಭೋದಯಃ ॥೧೪॥

ವಿಶಾಲಕೀರ್ತಿಃ ವಾಗೀಶಃ ಸರ್ವಲೋಕಹಿತೋತ್ಸುಕಃ।\\
ಕೈಲಾಸಯಾತ್ರಾಸಂಪ್ರಾಪ್ತ ಚಂದ್ರಮೌಲಿಪ್ರಪೂಜಕಃ॥೧೫॥
 
ಕಾಂಚ್ಯಾಂ ಶ್ರೀಚಕ್ರರಾಜಾಖ್ಯ ಯಂತ್ರಸ್ಥಾಪನದೀಕ್ಷಿತಃ।\\
 ಶ್ರೀಚಕ್ರಾತ್ಮಕತಾಟಂಕ ತೋಷಿತಾಂಬಾಮನೋರಥಃ॥೧೬॥

 ಶ್ರೀಬ್ರಹ್ಮಸೂತ್ರೋಪನಿಷದ್ಭಾಷ್ಯಾದಿ ಗ್ರಂಥಕಲ್ಪಕಃ।\\
ಚತುರ್ದಿಕ್ಚತುರಾಮ್ನಾಯ ಪ್ರತಿಷ್ಠಾತಾ ಮಹಾಮತಿಃ॥೧೭॥

ದ್ವಿಸಪ್ತತಿಮತೋಚ್ಛೇತ್ತಾ ಸರ್ವದಿಗ್ವಿಜಯಪ್ರಭುಃ~।\\
ಕಾಷಾಯವಸನೋಪೇತೋ ಭಸ್ಮೋದ್ಧೂಲಿತವಿಗ್ರಹಃ॥೧೮॥

ಜ್ಞಾನಾತ್ಮಕೈಕದಂಡಾಢ್ಯಃ ಕಮಂಡಲುಲಸತ್ಕರಃ।\\
ಗುರುಭೂಮಂಡಲಾಚಾರ್ಯೋ ಭಗವತ್ಪಾದಸಂಜ್ಞಕಃ॥೧೯॥

ವ್ಯಾಸಸಂದರ್ಶನಪ್ರೀತಃ ಋಷ್ಯಶೃಂಗಪುರೇಶ್ವರಃ।\\
ಸೌಂದರ್ಯಲಹರೀಮುಖ್ಯ ಬಹುಸ್ತೋತ್ರವಿಧಾಯಕಃ॥೨೦॥

ಚತುಷ್ಷಷ್ಟಿಕಲಾಭಿಜ್ಞೋ ಬ್ರಹ್ಮರಾಕ್ಷಸಮೋಕ್ಷದಃ।\\
 ಶ್ರೀಮನ್ಮಂಡನಮಿಶ್ರಾಖ್ಯ ಸ್ವಯಂಭೂಜಯಸನ್ನುತಃ॥೨೧॥

 ತೋಟಕಾಚಾರ್ಯಸಂಪೂಜ್ಯಃ  ಪದ್ಮಪಾದಾರ್ಚಿತಾಂಘ್ರಿಕಃ।\\
 ಹಸ್ತಾಮಲಕಯೋಗೀಂದ್ರ ಬ್ರಹ್ಮಜ್ಞಾನಪ್ರದಾಯಕಃ॥೨೨॥

 ಸುರೇಶ್ವರಾಖ್ಯಸಚ್ಛಿಷ್ಯ ಸನ್ನ್ಯಾಸಾಶ್ರಮದಾಯಕಃ।\\
 ನೃಸಿಂಹಭಕ್ತಃ  ಸದ್ರತ್ನಗರ್ಭಹೇರಂಬಪೂಜಕಃ ।\\
 ವ್ಯಾಖ್ಯಾಸಿಂಹಾಸನಾಧೀಶೋ ಜಗತ್ಪೂಜ್ಯೋ ಜಗದ್ಗುರುಃ~॥೨೩॥

\authorline{ಇತಿ ಶ್ರೀಶಂಕರಭಗವತ್ಪಾದಾಷ್ಟೋತ್ತರಶತನಾಮಸ್ತೋತ್ರಂ ॥}
\section{ಶ್ರೀದತ್ತಾತ್ರೇಯಾಷ್ಟೋತ್ತರಶತನಾಮಸ್ತೋತ್ರಂ}
ಅಸ್ಯ ದತ್ತಾತ್ರೇಯಾಷ್ಟೋತ್ತರ ಶತನಾಮಸ್ತೋತ್ರ ಮಹಾಮಂತ್ರಸ್ಯ, ಬ್ರಹ್ಮವಿಷ್ಣುಮಹೇಶ್ವರಾ ಋಷಯಃ~। ಶ್ರೀದತ್ತಾತ್ರೇಯೋ ದೇವತಾ~। ಅನುಷ್ಟುಪ್ಛಂದಃ~। ಶ್ರೀದತ್ತಾತ್ರೇಯಪ್ರೀತ್ಯರ್ಥೇ ಜಪೇ ವಿನಿಯೋಗಃ~।\\
\dhyana{ದಿಗಂಬರಂ ಭಸ್ಮವಿಲೇಪಿತಾಂಗಂ ಚಕ್ರಂ ತ್ರಿಶೂಲಂ ಡಮರುಂ ಗದಾಂ ಚ~।\\
ಪದ್ಮಾನನಂಯೋಗಿಮುನೀಂದ್ರವಂದ್ಯಂ~ಧ್ಯಾಯಾಮಿತಂದತ್ತಮಭೀಷ್ಟಸಿದ್ಧ್ಯೈ॥}

ಓಂ ಅನಸೂಯಾಸುತೋ ದತ್ತೋ ಹ್ಯತ್ರಿಪುತ್ರೋ ಮಹಾಮುನಿಃ~।\\
ಯೋಗೀಂದ್ರಃ ಪುಣ್ಯಪುರುಷೋ ದೇವೇಶೋ ಜಗದೀಶ್ವರಃ ॥೧॥

ಪರಮಾತ್ಮಾ ಪರಂ ಬ್ರಹ್ಮ ಸದಾನಂದೋ ಜಗದ್ಗುರುಃ~।\\
ನಿತ್ಯತೃಪ್ತೋ ನಿರ್ವಿಕಾರೋ ನಿರ್ವಿಕಲ್ಪೋ ನಿರಂಜನಃ ॥೨॥

ಗುಣಾತ್ಮಕೋ ಗುಣಾತೀತೋ ಬ್ರಹ್ಮವಿಷ್ಣುಶಿವಾತ್ಮಕಃ~।\\
ನಾನಾರೂಪಧರೋ ನಿತ್ಯಃ ಶಾಂತೋ ದಾಂತಃ ಕೃಪಾನಿಧಿಃ ॥೩॥

ಭಕ್ತಪ್ರಿಯೋ ಭವಹರೋ ಭಗವಾನ್ಭವನಾಶನಃ~।\\
ಆದಿದೇವೋ ಮಹಾದೇವಃ ಸರ್ವೇಶೋ ಭುವನೇಶ್ವರಃ ॥೪॥

ವೇದಾಂತವೇದ್ಯೋ ವರದೋ ವಿಶ್ವರೂಪೋಽವ್ಯಯೋ ಹರಿಃ~।\\
ಸಚ್ಚಿದಾನಂದಃ ಸರ್ವೇಶೋ ಯೋಗೀಶೋ ಭಕ್ತವತ್ಸಲಃ ॥೫॥

ದಿಗಂಬರೋ ದಿವ್ಯಮೂರ್ತಿರ್ದಿವ್ಯಭೂತಿವಿಭೂಷಣಃ~।\\
ಅನಾದಿಃ ಸಿದ್ಧಸುಲಭೋ ಭಕ್ತವಾಂಛಿತದಾಯಕಃ ॥೬॥

ಏಕೋಽನೇಕೋ ಹ್ಯದ್ವಿತೀಯೋ ನಿಗಮಾಗಮಪಂಡಿತಃ~।\\
ಭುಕ್ತಿಮುಕ್ತಿಪ್ರದಾತಾ ಚ ಕಾರ್ತವೀರ್ಯವರಪ್ರದಃ ॥೭॥

ಶಾಶ್ವತಾಂಗೋ ವಿಶುದ್ಧಾತ್ಮಾ ವಿಶ್ವಾತ್ಮಾ ವಿಶ್ವತೋ ಮುಖಃ~।\\
ಸರ್ವೇಶ್ವರಃ ಸದಾತುಷ್ಟಃ ಸರ್ವಮಂಗಲದಾಯಕಃ ॥೮॥

ನಿಷ್ಕಲಂಕೋ ನಿರಾಭಾಸೋ ನಿರ್ವಿಕಲ್ಪೋ ನಿರಾಶ್ರಯಃ~।\\
ಪುರುಷೋತ್ತಮೋ ಲೋಕನಾಥಃ ಪುರಾಣಪುರುಷೋಽನಘಃ ॥೯॥

ಅಪಾರಮಹಿಮಾಽನಂತೋ ಹ್ಯಾದ್ಯಂತರಹಿತಾಕೃತಿಃ~।\\
ಸಂಸಾರವನದಾವಾಗ್ನಿರ್ಭವಸಾಗರತಾರಕಃ ॥೧೦॥

ಶ್ರೀನಿವಾಸೋ ವಿಶಾಲಾಕ್ಷಃ ಕ್ಷೀರಾಬ್ಧಿಶಯನೋಽಚ್ಯುತಃ~।\\
ಸರ್ವಪಾಪಕ್ಷಯಕರಸ್ತಾಪತ್ರಯನಿವಾರಣಃ ॥೧೧॥

ಲೋಕೇಶಃ ಸರ್ವಭೂತೇಶೋ ವ್ಯಾಪಕಃ ಕರುಣಾಮಯಃ~।\\
ಬ್ರಹ್ಮಾದಿವಂದಿತಪದೋ ಮುನಿವಂದ್ಯಃ ಸ್ತುತಿಪ್ರಿಯಃ ॥೧೨॥

ನಾಮರೂಪಕ್ರಿಯಾತೀತೋ ನಿಃಸ್ಪೃಹೋ ನಿರ್ಮಲಾತ್ಮಕಃ~।\\
ಮಾಯಾಧೀಶೋ ಮಹಾತ್ಮಾ ಚ ಮಹಾದೇವೋ ಮಹೇಶ್ವರಃ ॥೧೩॥

ವ್ಯಾಘ್ರಚರ್ಮಾಂಬರಧರೋ ನಾಗಕುಂಡಲಭೂಷಣಃ~।\\
ಸರ್ವಲಕ್ಷಣಸಂಪೂರ್ಣಃ ಸರ್ವಸಿದ್ಧಿಪ್ರದಾಯಕಃ ॥೧೪॥

ಸರ್ವಜ್ಞಃ ಕರುಣಾಸಿಂಧುಃ ಸರ್ಪಹಾರಃ ಸದಾಶಿವಃ~।\\
ಸಹ್ಯಾದ್ರಿವಾಸಃ ಸರ್ವಾತ್ಮಾ ಭವಬಂಧವಿಮೋಚನಃ ॥೧೫॥

ವಿಶ್ವಂಭರೋ ವಿಶ್ವನಾಥೋ ಜಗನ್ನಾಥೋ ಜಗತ್ಪ್ರಭುಃ~।\\
ನಿತ್ಯಂ ಪಠತಿ ಯೋ ಭಕ್ತ್ಯಾ ಸರ್ವಪಾಪೈಃ ಪ್ರಮುಚ್ಯತೇ ॥೧೬॥

ಸರ್ವದುಃಖಪ್ರಶಮನಂ ಸರ್ವಾರಿಷ್ಟನಿವಾರಣಂ~।\\
ಭೋಗಮೋಕ್ಷಪ್ರದಂ ನೄಣಾಂ ದತ್ತಸಾಯುಜ್ಯದಾಯಕಂ~।\\
ಪಠಂತಿ ಯೇ ಪ್ರಯತ್ನೇನ ಸತ್ಯಂ ಸತ್ಯಂ ವದಾಮ್ಯಹಂ ॥೧೭॥

\authorline{॥ಇತಿ ಬ್ರಹ್ಮಾಂಡಪುರಾಣೇ ಬ್ರಹ್ಮನಾರದಸಂವಾದೇ ಶ್ರೀದತ್ತಾತ್ರೇಯಾಷ್ಟೋತ್ತರ ಶತನಾಮಸ್ತೋತ್ರಂ ॥}
\section{ಲಕ್ಷ್ಮೀನರಸಿಂಹ ಅಷ್ಟೋತ್ತರಶತನಾಮಸ್ತೋತ್ರಂ}
ನಾರಸಿಂಹೋ ಮಹಾಸಿಂಹೋ ದಿವ್ಯಸಿಂಹೋ ಮಹಾಬಲಃ~।\\
ಉಗ್ರಸಿಂಹೋ ಮಹಾದೇವಃ ಸ್ತಂಭಜಶ್ಚೋಗ್ರಲೋಚನಃ ॥೧ ॥

ರೌದ್ರಃ ಸರ್ವಾದ್ಭುತಃ ಶ್ರೀಮಾನ್ ಯೋಗಾನಂದ ಸ್ತ್ರಿವಿಕ್ರಮಃ।\\
ಹರಿಃ ಕೋಲಾಹಲಃ ಚಕ್ರೀ ವಿಜಯೋ ಜಯವರ್ಧನಃ ॥೨ ॥

ಪಂಚಾನನಃ ಪರಂಬ್ರಹ್ಮ ಚಾಘೋರೋ ಘೋರವಿಕ್ರಮಃ~।\\
ಜ್ವಲನ್ಮುಖೋ ಜ್ವಾಲಮಾಲೀ ಮಹಾಜ್ವಾಲೋ ಮಹಾಪ್ರಭುಃ ॥೩ ॥

ನಿಟಿಲಾಕ್ಷಃ ಸಹಸ್ರಾಕ್ಷೋ ದುರ್ನಿರೀಕ್ಷಃ ಪ್ರತಾಪನಃ~।\\
ಮಹಾದಂಷ್ಟ್ರಾಯುಧಃ ಪ್ರಾಜ್ಞಶ್ಚಂಡಕೋಪೀ ಸದಾಶಿವಃ ॥೪ ॥

ಹಿರಣ್ಯಕಶಿಪುಧ್ವಂಸೀ ದೈತ್ಯದಾನವ ಭಂಜನಃ~।\\
ಗುಣಭದ್ರೋ ಮಹಾಭದ್ರೋ ಬಲಭದ್ರಃ ಸುಭದ್ರಕಃ ॥೫ ॥

ಕರಾಳೋ ವಿಕರಾಳಶ್ಚ ವಿಕರ್ತಾ ಸರ್ವಕರ್ತೃಕಃ~।\\
ಶಿಂಶುಮಾರಸ್ತ್ರಿಲೋಕಾತ್ಮಾ ಈಶಃ ಸರ್ವೇಶ್ವರೋ ವಿಭುಃ ॥೬ ॥

ಭೈರವಾಡಂಬರೋ ದಿವ್ಯಶ್ಚಾಚ್ಯುತಃ ಕವಿ ಮಾಧವಃ~।\\
ಅಧೋಕ್ಷಜೋಽಕ್ಷರಃ ಸರ್ವೋ ವನಮಾಲೀ ವರಪ್ರದಃ ॥೭ ॥

ವಿಶ್ವಂಭರೋದ್ಭುತೋ ಭವ್ಯಃ ಶ್ರೀವಿಷ್ಣುಃ ಪುರುಷೋತ್ತಮಃ~।\\
ಅನಘಾಸ್ತ್ರೋ ನಖಾಸ್ತ್ರಶ್ಚ ಸೂರ್ಯ ಜ್ಯೋತಿಃ ಸುರೇಶ್ವರಃ ॥೮ ॥

ಸಹಸ್ರಬಾಹುಃ ಸರ್ವಜ್ಞಃ ಸರ್ವಸಿದ್ಧಿ ಪ್ರದಾಯಕಃ~।\\
ವಜ್ರದಂಷ್ಟ್ರೋ ವಜ್ರನಖೋ ಮಹಾನಂದಃ ಪರಂತಪಃ ॥೯ ॥

ಸರ್ವಮಂತ್ರೈಕ ರೂಪಶ್ಚ ಸರ್ವಯಂತ್ರ ವಿದಾರಣಃ~।\\
ಸರ್ವತಂತ್ರಾತ್ಮಕೋ ಽವ್ಯಕ್ತಃ ಸುವ್ಯಕ್ತೋ ಭಕ್ತವತ್ಸಲಃ ॥೧೦ ॥

ವೈಶಾಖಶುಕ್ಲ ಭೂತೋತ್ಥಃ ಶರಣಾಗತವತ್ಸಲಃ~।\\
ಉದಾರಕೀರ್ತಿಃ ಪುಣ್ಯಾತ್ಮಾ ಮಹಾತ್ಮಾ ಚಂಡ ವಿಕ್ರಮಃ ॥೧೧ ॥

ವೇದತ್ರಯ ಪ್ರಪೂಜ್ಯಶ್ಚ ಭಗವಾನ್ ಪರಮೇಶ್ವರಃ~।\\
ಶ್ರೀವತ್ಸಾಂಕಃ ಶ್ರೀನಿವಾಸೋ ಜಗದ್ವ್ಯಾಪೀ ಜಗನ್ಮಯಃ ॥೧೨ ॥

ಜಗತ್ಪಾಲೋ ಜಗನ್ನಾಥೋ ಮಹಾಕಾಯೋ ದ್ವಿರೂಪಭೃತ್~।\\
ಪರಮಾತ್ಮಾ ಪರಂಜ್ಯೋತಿಃ ನಿರ್ಗುಣಶ್ಚ ನೃಕೇಸರೀ ॥೧೩ ॥

ಪರತತ್ತ್ವಂ ಪರಂಧಾಮ ಸಚ್ಚಿದಾನಂದವಿಗ್ರಹಃ~।\\
ಲಕ್ಷ್ಮೀನೃಸಿಂಹಃ ಸರ್ವಾತ್ಮಾ ಧೀರಃ ಪ್ರಹ್ಲಾದಪಾಲಕಃ ॥೧೪ ॥

ಇದಂ ಶ್ರೀಮನ್ನೃಸಿಂಹಸ್ಯ ನಾಮ್ನಾಮಷ್ಟೋತ್ತರಂ ಶತಂ~।\\
ತ್ರಿಸಂಧ್ಯಂ ಯಃಪಠೇತ್ ಭಕ್ತ್ಯಾ ಸರ್ವಾಭೀಷ್ಟಮವಾಪ್ನುಯಾತ್ ॥೧೫ ॥

\authorline{॥ಇತಿ ಲಕ್ಷ್ಮೀನರಸಿಂಹ ಅಷ್ಟೋತ್ತರ ಶತನಾಮಸ್ತೋತ್ರಂ॥}
\section{ಅಥ ಶಿವಾಷ್ಟೋತ್ತರಶತನಾಮಸ್ತೋತ್ರಮ್ }
ಶಿವೋ ಮಹೇಶ್ವರಃ ಶಂಭುಃ ಪಿನಾಕೀ ಶಶಿಶೇಖರಃ~।\\
ವಾಮದೇವೋ ವಿರೂಪಾಕ್ಷಃ ಕಪರ್ದೀ ನೀಲಲೋಹಿತಃ ॥೧ ॥

ಶಂಕರಃ ಶೂಲಪಾಣಿಶ್ಚ ಖಟ್ವಾಂಗೀ ವಿಷ್ಣುವಲ್ಲಭಃ~।\\
ಶಿಪಿವಿಷ್ಟೋಂಬಿಕಾನಾಥಃ ಶ್ರೀಕಂಠೋ ಭಕ್ತವತ್ಸಲಃ ॥೨ ॥

ಭವಃ ಶರ್ವಸ್ತ್ರಿಲೋಕೇಶಃ ಶಿತಿಕಂಠಃ ಶಿವಾಪ್ರಿಯಃ।\\
ಉಗ್ರಃ ಕಪಾಲೀ ಕಾಮಾರಿರಂಧಕಾಸುರಸೂದನಃ ॥೩ ॥

ಗಂಗಾಧರೋ ಲಲಾಟಾಕ್ಷಃ ಕಾಲಕಾಲಃ ಕೃಪಾನಿಧಿಃ~।\\
ಭೀಮಃ ಪರಶುಹಸ್ತಶ್ಚ ಮೃಗಪಾಣಿರ್ಜಟಾಧರಃ ॥೪ ॥

ಕೈಲಾಸವಾಸೀ ಕವಚೀ ಕಠೋರಸ್ತ್ರಿಪುರಾಂತಕಃ।\\
ವೃಷಾಂಕೋ ವೃಷಭಾರೂಢೋ ಭಸ್ಮೋದ್ಧೂಲಿತವಿಗ್ರಹಃ ॥೫ ॥

ಸಾಮಪ್ರಿಯಃ ಸ್ವರಮಯಸ್ತ್ರಯೀಮೂರ್ತಿರನೀಶ್ವರಃ~।\\
ಸರ್ವಜ್ಞಃ ಪರಮಾತ್ಮಾ ಚ ಸೋಮಸೂರ್ಯಾಗ್ನಿಲೋಚನಃ ॥೬॥

ಹವಿರ್ಯಜ್ಞಮಯಃ ಸೋಮಃ ಪಂಚವಕ್ತ್ರಃ ಸದಾಶಿವಃ।\\
ವಿಶ್ವೇಶ್ವರೋ ವೀರಭದ್ರೋ ಗಣನಾಥಃ ಪ್ರಜಾಪತಿಃ ॥೭॥

ಹಿರಣ್ಯರೇತಾ ದುರ್ಧರ್ಷೋ ಗಿರೀಶೋ ಗಿರಿಶೋನಘಃ।\\
ಭುಜಂಗಭೂಷಣೋ ಭರ್ಗೋ ಗಿರಿಧನ್ವಾ ಗಿರಿಪ್ರಿಯಃ ॥೮॥

ಕೃತ್ತಿವಾಸಾಃ ಪುರಾರಾತಿರ್ಭಗವಾನ್ ಪ್ರಮಥಾಧಿಪಃ।\\
ಮೃತ್ಯುಂಜಯಃ ಸೂಕ್ಷ್ಮತನುರ್ಜಗದ್ವ್ಯಾಪೀ ಜಗದ್ಗುರುಃ ॥೯॥

ವ್ಯೋಮಕೇಶೋ ಮಹಾಸೇನಜನಕಶ್ಚಾರುವಿಕ್ರಮಃ।\\
ರುದ್ರೋ ಭೂತಪತಿಃ ಸ್ಥಾಣುರಹಿರ್ಬುಧ್ನ್ಯೋ ದಿಗಂಬರಃ ॥೧೦॥

ಅಷ್ಟಮೂರ್ತಿರನೇಕಾತ್ಮಾ ಸಾತ್ವಿಕಃ ಶುದ್ಧವಿಗ್ರಹಃ।\\
ಶಾಶ್ವತಃ ಖಂಡಪರಶುರಜಃ ಪಾಶವಿಮೋಚನಃ ॥೧೧॥

ಮೃಡಃ ಪಶುಪತಿರ್ದೇವೋ ಮಹಾದೇವೋಽವ್ಯಯೋ ಹರಿಃ।\\
ಪೂಷದಂತಭಿದವ್ಯಗ್ರೋ ದಕ್ಷಾಧ್ವರಹರೋ ಹರಃ ॥೧೨॥

ಭಗನೇತ್ರಭಿದವ್ಯಕ್ತಃ ಸಹಸ್ರಾಕ್ಷಃ ಸಹಸ್ರಪಾತ್।\\
ಅಪವರ್ಗಪ್ರದೋಽನಂತಸ್ತಾರಕಃ ಪರಮೇಶ್ವರಃ ॥೧೩॥

\authorline{॥ಇತಿ ಶಿವಾಷ್ಟೋತ್ತರ ಶತನಾಮಸ್ತೋತ್ರಂ ಸಂಪೂರ್ಣಂ॥}
\section{ಶ್ರೀಬಾಲಾಷ್ಟೋತ್ತರಶತನಾಮಸ್ತೋತ್ರಂ}
ಅಸ್ಯ ಶ್ರೀಬಾಲಾಷ್ಟೋತ್ತರ ಶತನಾಮಸ್ತೋತ್ರ ಮಹಾಮಂತ್ರಸ್ಯ  ದಕ್ಷಿಣಾಮೂರ್ತಿಃ ಋಷಿಃ~। ಅನುಷ್ಟುಪ್ ಛಂದಃ~। ಬಾಲಾತ್ರಿಪುರಸುಂದರೀ ದೇವತಾ~। ಐಂ ಬೀಜಂ~। ಕ್ಲೀಂ ಶಕ್ತಿಃ~। ಸೌಃ ಕೀಲಕಂ~।  ಶ್ರೀಬಾಲಾಪ್ರಿತ್ಯರ್ಥೇ ನಾಮಪಾರಾಯಣೇ ವಿನಿಯೋಗಃ ॥

\dhyana{ಪಾಶಾಂಕುಶೇ ಪುಸ್ತಕಾಕ್ಷಸೂತ್ರೇ ಚ ದಧತೀ ಕರೈಃ~।\\
ರಕ್ತಾ ತ್ರ್ಯಕ್ಷಾ ಚಂದ್ರಫಾಲಾ ಪಾತು ಬಾಲಾ ಸುರಾರ್ಚಿತಾ ॥}

ಕಲ್ಯಾಣೀ ತ್ರಿಪುರಾ ಬಾಲಾ ಮಾಯಾ ತ್ರಿಪುರಸುಂದರೀ~।\\
ಸುಂದರೀ ಸೌಭಾಗ್ಯವತೀ ಕ್ಲೀಂಕಾರೀ ಸರ್ವಮಂಗಲಾ ॥೧॥

ಹ್ರೀಂಕಾರೀ ಸ್ಕಂದಜನನೀ ಪರಾ ಪಂಚದಶಾಕ್ಷರೀ~।\\
ತ್ರಿಲೋಕೀ ಮೋಹನಾಧೀಶಾ ಸರ್ವೇಶೀ ಸರ್ವರೂಪಿಣೀ ॥೨॥

ಸರ್ವಸಂಕ್ಷೋಭಿಣೀ ಪೂರ್ಣಾ ನವಮುದ್ರೇಶ್ವರೀ ಶಿವಾ~।\\
ಅನಂಗಕುಸುಮಾ ಖ್ಯಾತಾ ಅನಂಗಾ ಭುವನೇಶ್ವರೀ ॥೩॥

ಜಪ್ಯಾ ಸ್ತವ್ಯಾ ಶ್ರುತಿರ್ನಿತ್ಯಾ ನಿತ್ಯಕ್ಲಿನ್ನಾಽಮೃತೋದ್ಭವಾ~।\\
ಮೋಹಿನೀ ಪರಮಾಽಽನಂದಾ ಕಾಮೇಶತರುಣಾ ಕಲಾ ॥೪॥

ಕಲಾವತೀ ಭಗವತೀ ಪದ್ಮರಾಗಕಿರೀಟಿನೀ~।\\
ಸೌಗಂಧಿನೀ ಸರಿದ್ವೇಣೀ ಮಂತ್ರಿಣೀ ಮಂತ್ರರೂಪಿಣೀ ॥೫॥

ತತ್ತ್ವತ್ರಯೀ ತತ್ತ್ವಮಯೀ ಸಿದ್ಧಾ ತ್ರಿಪುರವಾಸಿನೀ~।\\
ಶ್ರೀರ್ಮತಿಶ್ಚ ಮಹಾದೇವೀ ಕೌಲಿನೀ ಪರದೇವತಾ ॥೬॥

ಕೈವಲ್ಯರೇಖಾ ವಶಿನೀ ಸರ್ವೇಶೀ ಸರ್ವಮಾತೃಕಾ~।\\
ವಿಷ್ಣುಸ್ವಸಾ ದೇವಮಾತಾ ಸರ್ವಸಂಪತ್ಪ್ರದಾಯಿನೀ ॥೭॥

ಕಿಂಕರೀ ಮಾತಾ ಗೀರ್ವಾಣೀ ಸುರಾಪಾನಾನುಮೋದಿನೀ~।\\
ಆಧಾರಾಹಿತಪತ್ನೀಕಾ ಸ್ವಾಧಿಷ್ಠಾನಸಮಾಶ್ರಯಾ ॥೮॥

ಅನಾಹತಾಬ್ಜನಿಲಯಾ ಮಣಿಪೂರಸಮಾಶ್ರಯಾ~।\\
ಆಜ್ಞಾ ಪದ್ಮಾಸನಾಸೀನಾ ವಿಶುದ್ಧಸ್ಥಲಸಂಸ್ಥಿತಾ ॥೯॥

ಅಷ್ಟಾತ್ರಿಂಶತ್ಕಲಾಮೂರ್ತಿ ಸ್ಸುಷುಮ್ನಾ ಚಾರುಮಧ್ಯಮಾ~।\\
ಯೋಗೇಶ್ವರೀ ಮುನಿಧ್ಯೇಯಾ ಪರಬ್ರಹ್ಮಸ್ವರೂಪಿಣೀ ॥೧೦॥

ಚತುರ್ಭುಜಾ ಚಂದ್ರಚೂಡಾ ಪುರಾಣಾಗಮರೂಪಿಣೀ~।\\
ಐಂಕಾರಾದಿರ್ಮಹಾವಿದ್ಯಾ ಪಂಚಪ್ರಣವರೂಪಿಣೀ ॥೧೧॥

ಭೂತೇಶ್ವರೀ ಭೂತಮಯೀ ಪಂಚಾಶದ್ವರ್ಣರೂಪಿಣೀ~।\\
ಷೋಢಾನ್ಯಾಸ ಮಹಾಭೂಷಾ ಕಾಮಾಕ್ಷೀ ದಶಮಾತೃಕಾ ॥೧೨॥

ಆಧಾರಶಕ್ತಿಃ ತರುಣೀ ಲಕ್ಷ್ಮೀಃ ತ್ರಿಪುರಭೈರವೀ~।\\
ಶಾಂಭವೀ ಸಚ್ಚಿದಾನಂದಾ ಸಚ್ಚಿದಾನಂದರೂಪಿಣೀ ॥೧೩॥

ಮಾಂಗಲ್ಯ ದಾಯಿನೀ ಮಾನ್ಯಾ ಸರ್ವಮಂಗಲಕಾರಿಣೀ~।\\
ಯೋಗಲಕ್ಷ್ಮೀಃ ಭೋಗಲಕ್ಷ್ಮೀಃ ರಾಜ್ಯಲಕ್ಷ್ಮೀಃ ತ್ರಿಕೋಣಗಾ ॥೧೪॥

ಸರ್ವಸೌಭಾಗ್ಯಸಂಪನ್ನಾ ಸರ್ವಸಂಪತ್ತಿದಾಯಿನೀ~।\\
ನವಕೋಣಪುರಾವಾಸಾ ಬಿಂದುತ್ರಯಸಮನ್ವಿತಾ ॥೧೫॥

ನಾಮ್ನಾಮಷ್ಟೋತ್ತರಶತಂ ಪಠೇನ್ನ್ಯಾಸಸಮನ್ವಿತಂ~।\\
ಸರ್ವಸಿದ್ಧಿಮವಾಪ್ನೋತಿ ಸಾಧಕೋಽಭೀಷ್ಟಮಾಪ್ನುಯಾತ್ ॥೧೬॥
\authorline{ಇತಿ ಶ್ರೀ ರುದ್ರಯಾಮಲತಂತ್ರೇ ಉಮಾಮಹೇಶ್ವರಸಂವಾದೇ\\ ಶ್ರೀ ಬಾಲಾ ಅಷ್ಟೋತ್ತರ ಶತನಾಮಸ್ತೋತ್ರಂ ಸಂಪೂರ್ಣಂ ॥}
\newpage
\section{ಶ್ರೀಅನ್ನಪೂರ್ಣಾಷ್ಟೋತ್ತರಶತನಾಮಸ್ತೋತ್ರಂ }
ಅಸ್ಯ ಶ್ರೀಅನ್ನಪೂರ್ಣಾಷ್ಟೋತ್ತರ ಶತನಾಮಸ್ತೋತ್ರಮಂತ್ರಸ್ಯ ಭಗವಾನ್ ಶ್ರೀಬ್ರಹ್ಮಾ ಋಷಿಃ~। ಅನುಷ್ಟುಪ್ಛಂದಃ~। ಶ್ರೀಅನ್ನಪೂರ್ಣೇಶ್ವರೀ ದೇವತಾ~। ಸ್ವಧಾ ಬೀಜಂ~। ಸ್ವಾಹಾ ಶಕ್ತಿಃ~। ಓಂ ಕೀಲಕಂ~।

ಅನ್ನಪೂರ್ಣಾ ಶಿವಾ ದೇವೀ ಭೀಮಾ ಪುಷ್ಟಿಸ್ಸರಸ್ವತೀ~।\\
ಸರ್ವಜ್ಞಾ ಪಾರ್ವತೀ ದುರ್ಗಾ ಶರ್ವಾಣೀ ಶಿವವಲ್ಲಭಾ ॥೧॥

ವೇದವೇದ್ಯಾ ಮಹಾವಿದ್ಯಾ ವಿದ್ಯಾದಾತ್ರೀ ವಿಶಾರದಾ~।\\
ಕುಮಾರೀ ತ್ರಿಪುರಾ ಬಾಲಾ ಲಕ್ಷ್ಮೀಶ್ಶ್ರೀರ್ಭಯಹಾರಿಣೀ ॥೨॥

ಭವಾನೀ ವಿಷ್ಣುಜನನೀ ಬ್ರಹ್ಮಾದಿಜನನೀ ತಥಾ~।\\
ಗಣೇಶಜನನೀ ಶಕ್ತಿಃ ಕುಮಾರಜನನೀ ಶುಭಾ ॥೩॥

ಭೋಗಪ್ರದಾ ಭಗವತೀ ಭಕ್ತಾಭೀಷ್ಟಪ್ರದಾಯಿನೀ~।\\
ಭವರೋಗಹರಾ ಭವ್ಯಾ ಶುಭ್ರಾ ಪರಮಮಂಗಲಾ ॥೪॥

ಭವಾನೀ ಚಂಚಲಾ ಗೌರೀ ಚಾರುಚಂದ್ರಕಲಾಧರಾ~।\\
ವಿಶಾಲಾಕ್ಷೀ ವಿಶ್ವಮಾತಾ ವಿಶ್ವವಂದ್ಯಾ ವಿಲಾಸಿನೀ ॥೫॥

ಆರ್ಯಾ ಕಲ್ಯಾಣನಿಲಯಾ ರುದ್ರಾಣೀ ಕಮಲಾಸನಾ~।\\
ಶುಭಪ್ರದಾ ಶುಭಾವರ್ತಾ ವೃತ್ತಪೀನಪಯೋಧರಾ ॥೬॥

ಅಂಬಾ ಸಂಸಾರಮಥಿನೀ ಮೃಡಾನೀ ಸರ್ವಮಂಗಲಾ~।\\
ವಿಷ್ಣುಸಂಸೇವಿತಾ ಸಿದ್ಧಾ ಬ್ರಹ್ಮಾಣೀ ಸುರಸೇವಿತಾ ॥೭॥

ಪರಮಾನಂದದಾ ಶಾಂತಿಃ ಪರಮಾನಂದರೂಪಿಣೀ~।\\
ಪರಮಾನಂದಜನನೀ ಪರಾನಂದಪ್ರದಾಯಿನೀ ॥೮॥

ಪರೋಪಕಾರನಿರತಾ ಪರಮಾ ಭಕ್ತವತ್ಸಲಾ~।\\
ಪೂರ್ಣಚಂದ್ರಾಭವದನಾ ಪೂರ್ಣಚಂದ್ರನಿಭಾಂಶುಕಾ ॥೯॥

ಶುಭಲಕ್ಷಣಸಂಪನ್ನಾ ಶುಭಾನಂದಗುಣಾರ್ಣವಾ~।\\
ಶುಭಸೌಭಾಗ್ಯನಿಲಯಾ ಶುಭದಾ ಚ ರತಿಪ್ರಿಯಾ ॥೧೦॥

ಚಂಡಿಕಾ ಚಂಡಮಥನೀ ಚಂಡದರ್ಪನಿವಾರಿಣೀ~।\\
ಮಾರ್ತಂಡನಯನಾ ಸಾಧ್ವೀ ಚಂದ್ರಾಗ್ನಿನಯನಾ ಸತೀ ॥೧೧॥

ಪುಂಡರೀಕಕರಾ ಪೂರ್ಣಾ ಪುಣ್ಯದಾ ಪುಣ್ಯರೂಪಿಣೀ~।\\
ಮಾಯಾತೀತಾ ಶ್ರೇಷ್ಠಮಾಯಾ ಶ್ರೇಷ್ಠಧರ್ಮಾತ್ಮವಂದಿತಾ ॥೧೨॥

ಅಸೃಷ್ಟಿಸ್ಸಂಗರಹಿತಾ ಸೃಷ್ಟಿಹೇತುಃ ಕಪರ್ದಿನೀ~।\\
ವೃಷಾರೂಢಾ ಶೂಲಹಸ್ತಾ ಸ್ಥಿತಿಸಂಹಾರಕಾರಿಣೀ ॥೧೩॥

ಮಂದಸ್ಮಿತಾ ಸ್ಕಂದಮಾತಾ ಶುದ್ಧಚಿತ್ತಾ ಮುನಿಸ್ತುತಾ~।\\
ಮಹಾಭಗವತೀ ದಕ್ಷಾ ದಕ್ಷಾಧ್ವರವಿನಾಶಿನೀ ॥೧೪॥

ಸರ್ವಾರ್ಥದಾತ್ರೀ ಸಾವಿತ್ರೀ ಸದಾಶಿವಕುಟುಂಬಿನೀ~।\\
ನಿತ್ಯಸುಂದರಸರ್ವಾಂಗೀ ಸಚ್ಚಿದಾನಂದಲಕ್ಷಣಾ ॥೧೫॥

ನಾಮ್ನಾಮಷ್ಟೋತ್ತರಶತಮಂಬಾಯಾಃ ಪುಣ್ಯಕಾರಣಂ~।\\
ಸರ್ವಸೌಭಾಗ್ಯಸಿದ್ಧ್ಯರ್ಥಂ ಜಪನೀಯಂ ಪ್ರಯತ್ನತಃ ॥೧೬॥

ಏತಾನಿ ದಿವ್ಯನಾಮಾನಿ ಶ್ರುತ್ವಾ ಧ್ಯಾತ್ವಾ ನಿರಂತರಂ~।\\
ಸ್ತುತ್ವಾ ದೇವೀಂಚ ಸತತಂ ಸರ್ವಾನ್ಕಾಮಾನವಾಪ್ನುಯಾತ್ ॥೧೭॥
\authorline{॥ಇತಿ ಶ್ರೀಬ್ರಹ್ಮೋತ್ತರಖಂಡೇ ಆಗಮಪ್ರಖ್ಯಾತಿಶಿವರಹಸ್ಯೇ\\
ಶ್ರೀಅನ್ನಪೂರ್ಣಾಷ್ಟೋತ್ತರಶತನಾಮಸ್ತೋತ್ರಂ ಸಂಪೂರ್ಣಂ ॥}
\section{ಸೌಭಾಗ್ಯಾಷ್ಟೋತ್ತರಶತನಾಮಸ್ತೋತ್ರಂ }
ಸೌಭಾಗ್ಯಾಷ್ಟೋತ್ತರಶತನಾಮಸ್ತೋತ್ರಸ್ಯ ಶಿವ ಋಷಿಃ । ಅನುಷ್ಟುಪ್ಛಂದಃ । ಶ್ರೀಲಲಿತಾಂಬಿಕಾ  ದೇವತಾ ॥
ಕೂಟತ್ರಯೇಣ ನ್ಯಾಸಃ॥

\as{ಓಂ ಐಂಹ್ರೀಂ ಶ್ರೀಂ}\\
ಕಾಮೇಶ್ವರೀ ಕಾಮಶಕ್ತಿಃ ಕಾಮಸೌಭಾಗ್ಯದಾಯಿನೀ।\\
ಕಾಮರೂಪಾ ಕಾಮಕಲಾ ಕಾಮಿನೀ ಕಮಲಾಸನಾ ॥೧॥

ಕಮಲಾ ಕಲ್ಪನಾಹೀನಾ ಕಮನೀಯಾ ಕಲಾವತೀ~।\\
ಕಮಲಾ ಭಾರತೀಸೇವ್ಯಾ ಕಲ್ಪಿತಾಶೇಷಸಂಸೃತಿಃ ॥೨॥

ಅನುತ್ತರಾಽನಘಾಽನಂತಾಽದ್ಭುತರೂಪಾಽನಲೋದ್ಭವಾ~।\\
ಅತಿಲೋಕಚರಿತ್ರಾಽತಿಸುಂದರ್ಯತಿಶುಭಪ್ರದಾ ॥೩॥

ಅಘಹಂತ್ರ್ಯತಿವಿಸ್ತಾರಾಽರ್ಚನತುಷ್ಟಾಽಮಿತಪ್ರಭಾ~।\\
ಏಕರೂಪೈಕವೀರೈಕನಾಥೈಕಾಂತಾಽರ್ಚನಪ್ರಿಯಾ ॥೪॥

ಏಕೈಕಭಾವತುಷ್ಟೈಕರಸೈಕಾಂತಜನಪ್ರಿಯಾ~।\\
ಏಧಮಾನಪ್ರಭಾವೈಧದ್ಭಕ್ತಪಾತಕನಾಶಿನೀ ॥೫॥

ಏಲಾಮೋದಮುಖೈನೋಽದ್ರಿಶಕ್ರಾಯುಧಸಮಸ್ಥಿತಿಃ~।\\
ಈಹಾಶೂನ್ಯೇಪ್ಸಿತೇಶಾದಿಸೇವ್ಯೇಶಾನವರಾಂಗನಾ ॥೬॥

ಈಶ್ವರಾಽಽಜ್ಞಾಪಿಕೇಕಾರಭಾವ್ಯೇಪ್ಸಿತಫಲಪ್ರದಾ~।\\
ಈಶಾನೇತಿಹರೇಕ್ಷೇಷದರುಣಾಕ್ಷೀಶ್ವರೇಶ್ವರೀ ॥೭॥

ಲಲಿತಾ ಲಲನಾರೂಪಾ ಲಯಹೀನಾ ಲಸತ್ತನುಃ~।\\
ಲಯಸರ್ವಾ ಲಯಕ್ಷೋಣಿರ್ಲಯಕರ್ಣೀ ಲಯಾತ್ಮಿಕಾ ॥೮॥

ಲಘಿಮಾ ಲಘುಮಧ್ಯಾಽಽಢ್ಯಾ ಲಲಮಾನಾ ಲಘುದ್ರುತಾ~।\\
ಹಯಾಽಽರೂಢಾ ಹತಾಽಮಿತ್ರಾ ಹರಕಾಂತಾ ಹರಿಸ್ತುತಾ ॥೯॥

ಹಯಗ್ರೀವೇಷ್ಟದಾ ಹಾಲಾಪ್ರಿಯಾ ಹರ್ಷಸಮುದ್ಧತಾ~।\\
ಹರ್ಷಣಾ ಹಲ್ಲಕಾಭಾಂಗೀ ಹಸ್ತ್ಯಂತೈಶ್ವರ್ಯದಾಯಿನೀ ॥೧೦॥

ಹಲಹಸ್ತಾಽರ್ಚಿತಪದಾ ಹವಿರ್ದಾನಪ್ರಸಾದಿನೀ~।\\
ರಾಮರಾಮಾಽರ್ಚಿತಾ ರಾಜ್ಞೀ ರಮ್ಯಾ ರವಮಯೀ ರತಿಃ ॥೧೧॥

ರಕ್ಷಿಣೀರಮಣೀರಾಕಾ ರಮಣೀಮಂಡಲಪ್ರಿಯಾ~।\\
ರಕ್ಷಿತಾಽಖಿಲಲೋಕೇಶಾ ರಕ್ಷೋಗಣನಿಷೂದಿನೀ ॥೧೨॥

ಅಂಬಾಂತಕಾರಿಣ್ಯಂಭೋಜಪ್ರಿಯಾಂತಕಭಯಂಕರೀ~।\\
ಅಂಬುರೂಪಾಂಬುಜಕರಾಂಬುಜಜಾತವರಪ್ರದಾ ॥೧೩॥

ಅಂತಃಪೂಜಾಪ್ರಿಯಾಂತಃಸ್ವರೂಪಿಣ್ಯಂತರ್ವಚೋಮಯೀ~।\\
ಅಂತಕಾರಾತಿವಾಮಾಂಕಸ್ಥಿತಾಂತಃಸುಖರೂಪಿಣೀ ॥೧೪॥

ಸರ್ವಜ್ಞಾ ಸರ್ವಗಾ ಸಾರಾ ಸಮಾ ಸಮಸುಖಾ ಸತೀ~।\\
ಸಂತತಿಃ ಸಂತತಾ ಸೋಮಾ ಸರ್ವಾ ಸಾಂಖ್ಯಾ ಸನಾತನೀ ॥೧೫॥\as{ಶ್ರೀಂಹ್ರೀಂಐಂ}
\authorline{॥ಇತಿ ಸೌಭಾಗ್ಯಾಷ್ಟೋತ್ತರಶತನಾಮಸ್ತೋತ್ರಂ ॥}

\section{ಲಲಿತಾ ದ್ವಾದಶ ನಾಮ ಸ್ತೋತ್ರಮ್~॥}
ಶೃಣು ದ್ವಾದಶ ನಾಮಾನಿ ತಸ್ಯಾ ದೇವ್ಯಾ ಘಟೋದ್ಭವ ~।\\
ಯೇಷಾಮಾಕರ್ಣನಾಮಾತ್ರಾತ್ ಪ್ರಸನ್ನಾ ಸಾ ಭವಿಷ್ಯತಿ~॥

ಪಂಚಮೀ ದಂಡನಾಥಾ ಚ ಸಂಕೇತಾ ಸಮಯೇಶ್ವರೀ~।\\
ತಥಾ ಸಮಯಸಂಕೇತಾ ವಾರಾಹೀ ಪೋತ್ರಿಣೀ ಶಿವಾ~॥

ವಾರ್ತಾಲೀ ಚ ಮಹಾಸೇನಾ ಆಜ್ಞಾಚಕ್ರೇಶ್ವರೀ ತಥಾ~।\\
ಅರಿಘ್ನೀ ಚೇತಿ ಸಂಪ್ರೋಕ್ತಮ್ ನಾಮ ದ್ವಾದಶಕಂ ಮುನೇ~॥

ನಾಮದ್ವಾದಶಕಾಭಿಖ್ಯ ವಜ್ರಪಂಜರ ಮಧ್ಯಗಃ~।\\
ಸಂಕಟೇ ದುಃಖಮಾಪ್ನೋತಿ ನ ಕದಾಚನ ಮಾನವಃ~॥
\authorline{ಇತಿ ಶ್ರೀ ಲಲಿತಾ ದ್ವಾದಶ ನಾಮ ಸ್ತೋತ್ರಂ}
\section{ಲಲಿತಾ ಷೋಡಶನಾಮ ಸ್ತೋತ್ರಮ್~॥}
ಸಂಗೀತಯೋಗಿನೀ ಶ್ಯಾಮಾ ಶ್ಯಾಮಲಾ ಮಂತ್ರನಾಯಿಕಾ ।\\
ಮಂತ್ರಿಣೀ ಸಚಿವೇಶಾನೀ ಪ್ರಧಾನೇಶೀ ಶುಕಪ್ರಿಯಾ~॥

ವೀಣಾವತೀ ವೈಣಿಕೀ ಚ ಮುದ್ರಿಣೀ ಪ್ರಿಯಕಪ್ರಿಯಾ~।\\
ನೀಪಪ್ರಿಯಾ ಕದಂಬೇಶೀ ಕದಂಬವನವಾಸಿನೀ~॥

ಸದಾಮದಾ ಚ ನಾಮಾನಿ ಷೋಡಶೈತಾನಿ ಕುಂಭಜ~।\\
ಏತೈರ್ಯಃ ಸಚಿವೇಶಾನೀಂ ಸಕೃತ್ ಸ್ತೌತಿ ಶರೀರವಾನ್ ।\\
ತಸ್ಯ ತ್ರೈಲೋಕ್ಯಮಖಿಲಂ ವಶೇ ತಿಷ್ಠತ್ಯಸಂಶಯಃ~॥
\authorline{ಇತಿ ಶ್ರೀ ಲಲಿತಾ ಷೋಡಶನಾಮ ಸ್ತೋತ್ರಮ್}
\section{ಲಲಿತಾ ಪಂಚವಿಂಶತಿನಾಮ ಸ್ತೋತ್ರಮ್~॥}
ಸಿಂಹಾಸನೇಶೀ ಲಲಿತಾ ಮಹಾರಾಜ್ಞೀ ವರಾಂಕುಶಾ~।\\
ಚಾಪಿನೀ ತ್ರಿಪುರಾ ಚೈವ ಮಹಾತ್ರಿಪುರಸುಂದರೀ~॥

ಸುಂದರೀ ಚಕ್ರನಾಥಾ ಚ ಸಮ್ರಾಜ್ಞೀ ಚಕ್ರಿಣೀ ತಥಾ~।\\
ಚಕ್ರೇಶ್ವರೀ ಮಹಾದೇವೀ ಕಾಮೇಶೀ ಪರಮೇಶ್ವರೀ ~॥

ಕಾಮರಾಜಪ್ರಿಯಾ ಕಾಮಕೋಟಿಕಾ ಚಕ್ರವರ್ತಿನೀ~।\\
ಮಹಾವಿದ್ಯಾ ಶಿವಾನಂಗವಲ್ಲಭಾ ಸರ್ವಪಾಟಲಾ~॥

ಕುಲನಾಥಾಮ್ನಾಯನಾಥಾ ಸರ್ವಾಮ್ನಾಯ ನಿವಾಸಿನೀ~।\\
ಶೃಂಗಾರ ನಾಯಿಕಾ ಚೇತಿ ಪಂಚವಿಂಶತಿ ನಾಮಭಿಃ~॥

ಸ್ತುವಂತಿ ಯೇ ಮಹಾಭಾಗಾಂ ಲಲಿತಾಂ ಪರಮೇಶ್ವರೀಮ್~।\\
ತೇ ಪ್ರಾಪ್ನುವಂತಿ ಸೌಭಾಗ್ಯಮಷ್ಟೌ ಸಿದ್ಧೀರ್ಮಹದ್ಯಶಃ~॥
\authorline{ಇತಿ ಶ್ರೀ ಲಲಿತಾ ಪಂಚವಿಂಶತಿನಾಮ ಸ್ತೋತ್ರಮ್~॥}


\section{ ಶ್ರೀ ಲಲಿತಾಸಹಸ್ರನಾಮ ಸ್ತೋತ್ರಂ }
ಅಸ್ಯ ಶ್ರೀಲಲಿತಾಸಹಸ್ರನಾಮಸ್ತೋತ್ರಮಾಲಾ ಮಂತ್ರಸ್ಯ~। ವಶಿನ್ಯಾದಿವಾಗ್ದೇವತಾ ಋಷಯಃ~। ಅನುಷ್ಟುಪ್ ಛಂದಃ~। ಶ್ರೀಲಲಿತಾಪರಮೇಶ್ವರೀ ದೇವತಾ। ಶ್ರೀಮದ್ವಾಗ್ಭವಕೂಟೇತಿ ಬೀಜಂ। ಮಧ್ಯಕೂಟೇತಿ ಶಕ್ತಿಃ। ಶಕ್ತಿಕೂಟೇತಿ ಕೀಲಕಂ। ವಾಕ್ಸಿದ್ಧ್ಯರ್ಥೇ ಜಪೇ ವಿನಿಯೋಗಃ~।

\dhyana{ಸಿಂದೂರಾರುಣ ವಿಗ್ರಹಾಂ ತ್ರಿನಯನಾಂ ಮಾಣಿಕ್ಯಮೌಲಿ ಸ್ಫುರತ್\\
ತಾರಾ ನಾಯಕ ಶೇಖರಾಂ ಸ್ಮಿತಮುಖೀಮಾಪೀನವಕ್ಷೋರುಹಾಂ~।\\
ಪಾಣಿಭ್ಯಾಮಲಿಪೂರ್ಣ ರತ್ನ ಚಷಕಂ ರಕ್ತೋತ್ಪಲಂ ಬಿಭ್ರತೀಂ\\
ಸೌಮ್ಯಾಂ ರತ್ನ ಘಟಸ್ಥ ರಕ್ತಚರಣಾಂ ಧ್ಯಾಯೇತ್ ಪರಾಮಂಬಿಕಾಂ ॥

ಅರುಣಾಂ ಕರುಣಾ ತರಂಗಿತಾಕ್ಷೀಂ ಧೃತ ಪಾಶಾಂಕುಶ ಪುಷ್ಪ ಬಾಣಚಾಪಾಂ~।\\
ಅಣಿಮಾದಿಭಿರಾವೃತಾಂ ಮಯೂಖೈರಹಮಿತ್ಯೇವ ವಿಭಾವಯೇ ಭವಾನೀಂ ॥
\newpage
ಧ್ಯಾಯೇತ್ ಪದ್ಮಾಸನಸ್ಥಾಂ ವಿಕಸಿತವದನಾಂ ಪದ್ಮಪತ್ರಾಯತಾಕ್ಷೀಂ\\
ಹೇಮಾಭಾಂ ಪೀತವಸ್ತ್ರಾಂ ಕರಕಲಿತಲಸದ್ಧೇಮಪದ್ಮಾಂ ವರಾಂಗೀಂ~।\\
ಸರ್ವಾಲಂಕಾರಯುಕ್ತಾಂ ಸತತಮಭಯದಾಂ ಭಕ್ತನಮ್ರಾಂ ಭವಾನೀಂ\\
ಶ್ರೀವಿದ್ಯಾಂ ಶಾಂತಮೂರ್ತಿಂ ಸಕಲ ಸುರನುತಾಂ ಸರ್ವ ಸಂಪತ್ಪ್ರದಾತ್ರೀಂ ॥

ಸಕುಂಕುಮವಿಲೇಪನಾಮಲಿಕಚುಂಬಿಕಸ್ತೂರಿಕಾಂ\\
ಸಮಂದಹಸಿತೇಕ್ಷಣಾಂ ಸಶರಚಾಪಪಾಶಾಂಕುಶಾಂ~।\\
ಅಶೇಷಜನಮೋಹಿನೀಂ ಅರುಣಮಾಲ್ಯಭೂಷಾಂಬರಾಂ\\
ಜಪಾಕುಸುಮಭಾಸುರಾಂ ಜಪವಿಧೌ ಸ್ಮರೇದಂಬಿಕಾಂ ॥}

{\bfseries ಓಂ ಐಂಹ್ರೀಂಶ್ರೀಂ}\\
ಶ್ರೀಮಾತಾ ಶ್ರೀಮಹಾರಾಜ್ಞೀ ಶ್ರೀಮತ್ಸಿಂಹಾಸನೇಶ್ವರೀ~।\\
ಚಿದಗ್ನಿ-ಕುಂಡ-ಸಂಭೂತಾ ದೇವಕಾರ್ಯ-ಸಮುದ್ಯತಾ ॥೧॥

ಉದ್ಯದ್ಭಾನು-ಸಹಸ್ರಾಭಾ ಚತುರ್ಬಾಹು-ಸಮನ್ವಿತಾ~।\\
ರಾಗಸ್ವರೂಪ-ಪಾಶಾಢ್ಯಾ ಕ್ರೋಧಾಕಾರಾಂಕುಶೋಜ್ಜ್ವಲಾ ॥೨॥

ಮನೋರೂಪೇಕ್ಷು-ಕೋದಂಡಾ ಪಂಚತನ್ಮಾತ್ರ-ಸಾಯಕಾ~।\\
ನಿಜಾರುಣ-ಪ್ರಭಾಪೂರ-ಮಜ್ಜದ್‍ಬ್ರಹ್ಮಾಂಡ-ಮಂಡಲಾ ॥೩॥

ಚಂಪಕಾಶೋಕ-ಪುನ್ನಾಗ-ಸೌಗಂಧಿಕ-ಲಸತ್ಕಚಾ~।\\
ಕುರುವಿಂದಮಣಿ-ಶ್ರೇಣೀ-ಕನತ್ಕೋಟೀರ-ಮಂಡಿತಾ ॥೪॥

ಅಷ್ಟಮೀಚಂದ್ರ-ವಿಭ್ರಾಜ-ದಲಿಕಸ್ಥಲ-ಶೋಭಿತಾ~।\\
ಮುಖಚಂದ್ರ-ಕಲಂಕಾಭ-ಮೃಗನಾಭಿ-ವಿಶೇಷಕಾ ॥೫॥

ವದನಸ್ಮರ-ಮಾಂಗಲ್ಯ-ಗೃಹತೋರಣ-ಚಿಲ್ಲಿಕಾ~।\\
ವಕ್ತ್ರಲಕ್ಷ್ಮೀ-ಪರೀವಾಹ-ಚಲನ್ಮೀನಾಭ-ಲೋಚನಾ ॥೬॥

ನವಚಂಪಕ-ಪುಷ್ಪಾಭ-ನಾಸಾದಂಡ-ವಿರಾಜಿತಾ~।\\
ತಾರಾಕಾಂತಿ-ತಿರಸ್ಕಾರಿ-ನಾಸಾಭರಣ-ಭಾಸುರಾ ॥೭॥

ಕದಂಬಮಂಜರೀ-ಕ್ಲೃಪ್ತ-ಕರ್ಣಪೂರ-ಮನೋಹರಾ~।\\
ತಾಟಂಕ-ಯುಗಲೀ-ಭೂತ-ತಪನೋಡುಪ-ಮಂಡಲಾ ॥೮॥

ಪದ್ಮರಾಗಶಿಲಾದರ್ಶ-ಪರಿಭಾವಿ-ಕಪೋಲಭೂಃ~।\\
ನವವಿದ್ರುಮ-ಬಿಂಬಶ್ರೀ-ನ್ಯಕ್ಕಾರಿ-ರದನಚ್ಛದಾ ॥೯॥

ಶುದ್ಧವಿದ್ಯಾಂಕುರಾಕಾರ-ದ್ವಿಜಪಂಕ್ತಿ-ದ್ವಯೋಜ್ಜ್ವಲಾ~।\\
ಕರ್ಪೂರವೀಟಿಕಾಮೋದ-ಸಮಾಕರ್ಷದ್ದಿಗಂತರಾ ॥೧೦॥

ನಿಜ-ಸಲ್ಲಾಪ-ಮಾಧುರ್ಯ-ವಿನಿರ್ಭರ್ತ್ಸಿತ-ಕಚ್ಛಪೀ~।\\
ಮಂದಸ್ಮಿತ-ಪ್ರಭಾಪೂರ-ಮಜ್ಜತ್ಕಾಮೇಶ-ಮಾನಸಾ ॥೧೧॥

ಅನಾಕಲಿತ-ಸಾದೃಶ್ಯ-ಚುಬುಕಶ್ರೀ-ವಿರಾಜಿತಾ~।\\
ಕಾಮೇಶ-ಬದ್ಧ-ಮಾಂಗಲ್ಯ-ಸೂತ್ರ-ಶೋಭಿತ-ಕಂಧರಾ ॥೧೨॥

ಕನಕಾಂಗದ-ಕೇಯೂರ-ಕಮನೀಯ-ಭುಜಾನ್ವಿತಾ~।\\
ರತ್ನಗ್ರೈವೇಯ-ಚಿಂತಾಕ-ಲೋಲ-ಮುಕ್ತಾ-ಫಲಾನ್ವಿತಾ ॥೧೩॥

ಕಾಮೇಶ್ವರ-ಪ್ರೇಮರತ್ನ-ಮಣಿ-ಪ್ರತಿಪಣ-ಸ್ತನೀ~।\\
ನಾಭ್ಯಾಲವಾಲ-ರೋಮಾಲಿ-ಲತಾ-ಫಲ-ಕುಚದ್ವಯೀ ॥೧೪॥

ಲಕ್ಷ್ಯರೋಮ-ಲತಾಧಾರತಾ-ಸಮುನ್ನೇಯ-ಮಧ್ಯಮಾ~।\\
ಸ್ತನಭಾರ-ದಲನ್ಮಧ್ಯ-ಪಟ್ಟಬಂಧ-ವಲಿತ್ರಯಾ ॥೧೫॥

ಅರುಣಾರುಣಕೌಸುಂಭ-ವಸ್ತ್ರ-ಭಾಸ್ವತ್ಕಟೀತಟೀ~।\\
ರತ್ನ-ಕಿಂಕಿಣಿಕಾ-ರಮ್ಯ-ರಶನಾ-ದಾಮ-ಭೂಷಿತಾ ॥೧೬॥

ಕಾಮೇಶ-ಜ್ಞಾತ-ಸೌಭಾಗ್ಯ-ಮಾರ್ದವೋರು-ದ್ವಯಾನ್ವಿತಾ~।\\
ಮಾಣಿಕ್ಯ-ಮುಕುಟಾಕಾರ-ಜಾನುದ್ವಯ-ವಿರಾಜಿತಾ ॥೧೭॥

ಇಂದ್ರಗೋಪ-ಪರಿಕ್ಷಿಪ್ತಸ್ಮರತೂಣಾಭ-ಜಂಘಿಕಾ~।\\
ಗೂಢಗುಲ್ಫಾ ಕೂರ್ಮಪೃಷ್ಠ-ಜಯಿಷ್ಣು-ಪ್ರಪದಾನ್ವಿತಾ ॥೧೮॥

ನಖ-ದೀಧಿತಿ-ಸಂಛನ್ನ-ನಮಜ್ಜನ-ತಮೋಗುಣಾ~।\\
ಪದದ್ವಯ-ಪ್ರಭಾಜಾಲ-ಪರಾಕೃತ-ಸರೋರುಹಾ ॥೧೯॥

ಶಿಂಜಾನ-ಮಣಿಮಂಜೀರ-ಮಂಡಿತ-ಶ್ರೀ-ಪದಾಂಬುಜಾ~।\\
ಮರಾಲೀ-ಮಂದಗಮನಾ ಮಹಾಲಾವಣ್ಯ-ಶೇವಧಿಃ ॥೨೦॥

ಸರ್ವಾರುಣಾಽನವದ್ಯಾಂಗೀ ಸರ್ವಾಭರಣ-ಭೂಷಿತಾ~।\\
ಶಿವ-ಕಾಮೇಶ್ವರಾಂಕಸ್ಥಾ ಶಿವಾ ಸ್ವಾಧೀನ-ವಲ್ಲಭಾ ॥೨೧॥

ಸುಮೇರು-ಮಧ್ಯ-ಶೃಂಗಸ್ಥಾ ಶ್ರೀಮನ್ನಗರ-ನಾಯಿಕಾ~।\\
ಚಿಂತಾಮಣಿ-ಗೃಹಾಂತಸ್ಥಾ ಪಂಚ-ಬ್ರಹ್ಮಾಸನ-ಸ್ಥಿತಾ ॥೨೨॥

ಮಹಾಪದ್ಮಾಟವೀ-ಸಂಸ್ಥಾ ಕದಂಬವನ-ವಾಸಿನೀ~।\\
ಸುಧಾಸಾಗರ-ಮಧ್ಯಸ್ಥಾ ಕಾಮಾಕ್ಷೀ ಕಾಮದಾಯಿನೀ ॥೨೩॥

ದೇವರ್ಷಿ-ಗಣ-ಸಂಘಾತ-ಸ್ತೂಯಮಾನಾತ್ಮ-ವೈಭವಾ~।\\
ಭಂಡಾಸುರ-ವಧೋದ್ಯುಕ್ತ-ಶಕ್ತಿಸೇನಾ-ಸಮನ್ವಿತಾ ॥೨೪॥

ಸಂಪತ್ಕರೀ-ಸಮಾರೂಢ-ಸಿಂಧುರ-ವ್ರಜ-ಸೇವಿತಾ~।\\
ಅಶ್ವಾರೂಢಾಧಿಷ್ಠಿತಾಶ್ವ-ಕೋಟಿ-ಕೋಟಿಭಿರಾವೃತಾ ॥೨೫॥

ಚಕ್ರರಾಜ-ರಥಾರೂಢ-ಸರ್ವಾಯುಧ-ಪರಿಷ್ಕೃತಾ~।\\
ಗೇಯಚಕ್ರ-ರಥಾರೂಢ-ಮಂತ್ರಿಣೀ-ಪರಿಸೇವಿತಾ ॥೨೬॥

ಕಿರಿಚಕ್ರ-ರಥಾರೂಢ-ದಂಡನಾಥಾ-ಪುರಸ್ಕೃತಾ~।\\
ಜ್ವಾಲಾ-ಮಾಲಿನಿಕಾಕ್ಷಿಪ್ತ-ವಹ್ನಿಪ್ರಾಕಾರ-ಮಧ್ಯಗಾ ॥೨೭॥

ಭಂಡಸೈನ್ಯ-ವಧೋದ್ಯುಕ್ತ-ಶಕ್ತಿ-ವಿಕ್ರಮ-ಹರ್ಷಿತಾ~।\\
ನಿತ್ಯಾ-ಪರಾಕ್ರಮಾಟೋಪ-ನಿರೀಕ್ಷಣ-ಸಮುತ್ಸುಕಾ ॥೨೮॥

ಭಂಡಪುತ್ರ-ವಧೋದ್ಯುಕ್ತ-ಬಾಲಾ-ವಿಕ್ರಮ-ನಂದಿತಾ~।\\
ಮಂತ್ರಿಣ್ಯಂಬಾ-ವಿರಚಿತ-ವಿಷಂಗ-ವಧ-ತೋಷಿತಾ ॥೨೯॥

ವಿಶುಕ್ರ-ಪ್ರಾಣಹರಣ-ವಾರಾಹೀ-ವೀರ್ಯ-ನಂದಿತಾ~।\\
ಕಾಮೇಶ್ವರ-ಮುಖಾಲೋಕ-ಕಲ್ಪಿತ-ಶ್ರೀಗಣೇಶ್ವರಾ ॥೩೦॥

ಮಹಾಗಣೇಶ-ನಿರ್ಭಿನ್ನ-ವಿಘ್ನಯಂತ್ರ-ಪ್ರಹರ್ಷಿತಾ~।\\
ಭಂಡಾಸುರೇಂದ್ರ-ನಿರ್ಮುಕ್ತ-ಶಸ್ತ್ರ-ಪ್ರತ್ಯಸ್ತ್ರ-ವರ್ಷಿಣೀ ॥೩೧॥

ಕರಾಂಗುಲಿ-ನಖೋತ್ಪನ್ನ-ನಾರಾಯಣ-ದಶಾಕೃತಿಃ~।\\
ಮಹಾ-ಪಾಶುಪತಾಸ್ತ್ರಾಗ್ನಿ-ನಿರ್ದಗ್ಧಾಸುರ-ಸೈನಿಕಾ ॥೩೨॥

ಕಾಮೇಶ್ವರಾಸ್ತ್ರ-ನಿರ್ದಗ್ಧ-ಸಭಂಡಾಸುರ-ಶೂನ್ಯಕಾ~।\\
ಬ್ರಹ್ಮೋಪೇಂದ್ರ-ಮಹೇಂದ್ರಾದಿ-ದೇವ-ಸಂಸ್ತುತ-ವೈಭವಾ ॥೩೩॥

ಹರ-ನೇತ್ರಾಗ್ನಿ-ಸಂದಗ್ಧ-ಕಾಮ-ಸಂಜೀವನೌಷಧಿಃ~।\\
ಶ್ರೀಮದ್ವಾಗ್ಭವ-ಕೂಟೈಕ-ಸ್ವರೂಪ-ಮುಖ-ಪಂಕಜಾ ॥೩೪॥

ಕಂಠಾಧಃ-ಕಟಿ-ಪರ್ಯಂತ-ಮಧ್ಯಕೂಟ-ಸ್ವರೂಪಿಣೀ।\\
ಶಕ್ತಿ-ಕೂಟೈಕತಾಪನ್ನ-ಕಟ್ಯಧೋಭಾಗ ಧಾರಿಣೀ ॥೩೫॥

ಮೂಲ-ಮಂತ್ರಾತ್ಮಿಕಾ ಮೂಲಕೂಟತ್ರಯ-ಕಲೇಬರಾ~।\\
ಕುಲಾಮೃತೈಕ-ರಸಿಕಾ ಕುಲಸಂಕೇತ-ಪಾಲಿನೀ ॥೩೬॥

ಕುಲಾಂಗನಾ ಕುಲಾಂತಸ್ಥಾ ಕೌಲಿನೀ ಕುಲಯೋಗಿನೀ~।\\
ಅಕುಲಾ ಸಮಯಾಂತಸ್ಥಾ ಸಮಯಾಚಾರ-ತತ್ಪರಾ ॥೩೭॥

ಮೂಲಾಧಾರೈಕ-ನಿಲಯಾ ಬ್ರಹ್ಮಗ್ರಂಥಿ-ವಿಭೇದಿನೀ \as{(೧೦೦)}~।\\
ಮಣಿ-ಪೂರಾಂತರುದಿತಾ ವಿಷ್ಣುಗ್ರಂಥಿ-ವಿಭೇದಿನೀ ॥೩೮॥

ಆಜ್ಞಾ-ಚಕ್ರಾಂತರಾಲಸ್ಥಾ ರುದ್ರಗ್ರಂಥಿ-ವಿಭೇದಿನೀ~।\\
ಸಹಸ್ರಾರಾಂಬುಜಾರೂಢಾ ಸುಧಾ-ಸಾರಾಭಿವರ್ಷಿಣೀ ॥೩೯॥

ತಡಿಲ್ಲತಾ-ಸಮರುಚಿಃ ಷಟ್‍ಚಕ್ರೋಪರಿ-ಸಂಸ್ಥಿತಾ~।\\
ಮಹಾಸಕ್ತಿಃ ಕುಂಡಲಿನೀ ಬಿಸತಂತು-ತನೀಯಸೀ ॥೪೦॥

ಭವಾನೀ ಭಾವನಾಗಮ್ಯಾ ಭವಾರಣ್ಯ-ಕುಠಾರಿಕಾ~।\\
ಭದ್ರಪ್ರಿಯಾ ಭದ್ರಮೂರ್ತಿರ್ಭಕ್ತ-ಸೌಭಾಗ್ಯದಾಯಿನೀ ॥೪೧॥

ಭಕ್ತಿಪ್ರಿಯಾ ಭಕ್ತಿಗಮ್ಯಾ ಭಕ್ತಿವಶ್ಯಾ ಭಯಾಪಹಾ~।\\
ಶಾಂಭವೀ ಶಾರದಾರಾಧ್ಯಾ ಶರ್ವಾಣೀ ಶರ್ಮದಾಯಿನೀ ॥೪೨॥

ಶಾಂಕರೀ ಶ್ರೀಕರೀ ಸಾಧ್ವೀ ಶರಚ್ಚಂದ್ರ-ನಿಭಾನನಾ~।\\
ಶಾತೋದರೀ ಶಾಂತಿಮತೀ ನಿರಾಧಾರಾ ನಿರಂಜನಾ ॥೪೩॥

ನಿರ್ಲೇಪಾ ನಿರ್ಮಲಾ ನಿತ್ಯಾ ನಿರಾಕಾರಾ ನಿರಾಕುಲಾ~।\\
ನಿರ್ಗುಣಾ ನಿಷ್ಕಲಾ ಶಾಂತಾ ನಿಷ್ಕಾಮಾ ನಿರುಪಪ್ಲವಾ ॥೪೪॥

ನಿತ್ಯಮುಕ್ತಾ ನಿರ್ವಿಕಾರಾ ನಿಷ್ಪ್ರಪಂಚಾ ನಿರಾಶ್ರಯಾ~।\\
ನಿತ್ಯಶುದ್ಧಾ ನಿತ್ಯಬುದ್ಧಾ ನಿರವದ್ಯಾ ನಿರಂತರಾ ॥೪೫॥

ನಿಷ್ಕಾರಣಾ ನಿಷ್ಕಲಂಕಾ ನಿರುಪಾಧಿರ್ನಿರೀಶ್ವರಾ~।\\
ನೀರಾಗಾ ರಾಗಮಥನೀ ನಿರ್ಮದಾ ಮದನಾಶಿನೀ ॥೪೬॥

ನಿಶ್ಚಿಂತಾ ನಿರಹಂಕಾರಾ ನಿರ್ಮೋಹಾ ಮೋಹನಾಶಿನೀ~।\\
ನಿರ್ಮಮಾ ಮಮತಾಹಂತ್ರೀ ನಿಷ್ಪಾಪಾ ಪಾಪನಾಶಿನೀ ॥೪೭॥

ನಿಷ್ಕ್ರೋಧಾ ಕ್ರೋಧಶಮನೀ ನಿರ್ಲೋಭಾ ಲೋಭನಾಶಿನೀ~।\\
ನಿಃಸಂಶಯಾ ಸಂಶಯಘ್ನೀ ನಿರ್ಭವಾ ಭವನಾಶಿನೀ ॥೪೮॥

ನಿರ್ವಿಕಲ್ಪಾ ನಿರಾಬಾಧಾ ನಿರ್ಭೇದಾ ಭೇದನಾಶಿನೀ~।\\
ನಿರ್ನಾಶಾ ಮೃತ್ಯುಮಥಿನೀ ನಿಷ್ಕ್ರಿಯಾ ನಿಷ್ಪರಿಗ್ರಹಾ~।೪೯॥

ನಿಸ್ತುಲಾ ನೀಲಚಿಕುರಾ ನಿರಪಾಯಾ ನಿರತ್ಯಯಾ~।\\
ದುರ್ಲಭಾ ದುರ್ಗಮಾ ದುರ್ಗಾ ದುಃಖಹಂತ್ರೀ ಸುಖಪ್ರದಾ ॥೫೦॥

ದುಷ್ಟದೂರಾ ದುರಾಚಾರ-ಶಮನೀ ದೋಷವರ್ಜಿತಾ~।\\
ಸರ್ವಜ್ಞಾ ಸಾಂದ್ರಕರುಣಾ ಸಮಾನಾಧಿಕ-ವರ್ಜಿತಾ ॥೫೧॥

ಸರ್ವಶಕ್ತಿಮಯೀ ಸರ್ವ-ಮಂಗಲಾ \as{(೨೦೦)} ಸದ್ಗತಿಪ್ರದಾ~।\\
ಸರ್ವೇಶ್ವರೀ ಸರ್ವಮಯೀ ಸರ್ವಮಂತ್ರ-ಸ್ವರೂಪಿಣೀ ॥೫೨॥

ಸರ್ವ-ಯಂತ್ರಾತ್ಮಿಕಾ ಸರ್ವ-ತಂತ್ರರೂಪಾ ಮನೋನ್ಮನೀ~।\\
ಮಾಹೇಶ್ವರೀ ಮಹಾದೇವೀ ಮಹಾಲಕ್ಷ್ಮೀರ್ಮೃಡಪ್ರಿಯಾ ॥೫೩॥

ಮಹಾರೂಪಾ ಮಹಾಪೂಜ್ಯಾ ಮಹಾಪಾತಕ-ನಾಶಿನೀ~।\\
ಮಹಾಮಾಯಾ ಮಹಾಸತ್ತ್ವಾ ಮಹಾಶಕ್ತಿರ್ಮಹಾರತಿಃ ॥೫೪॥

ಮಹಾಭೋಗಾ ಮಹೈಶ್ವರ್ಯಾ ಮಹಾವೀರ್ಯಾ ಮಹಾಬಲಾ~।\\
ಮಹಾಬುದ್ಧಿರ್ಮಹಾಸಿದ್ಧಿರ್ಮಹಾಯೋಗೇಶ್ವರೇಶ್ವರೀ ॥೫೫॥

ಮಹಾತಂತ್ರಾ ಮಹಾಮಂತ್ರಾ ಮಹಾಯಂತ್ರಾ ಮಹಾಸನಾ~।\\
ಮಹಾಯಾಗ-ಕ್ರಮಾರಾಧ್ಯಾ ಮಹಾಭೈರವ-ಪೂಜಿತಾ ॥೫೬॥

ಮಹೇಶ್ವರ-ಮಹಾಕಲ್ಪ-ಮಹಾತಾಂಡವ-ಸಾಕ್ಷಿಣೀ~।\\
ಮಹಾಕಾಮೇಶ-ಮಹಿಷೀ ಮಹಾತ್ರಿಪುರ-ಸುಂದರೀ ॥೫೭॥

ಚತುಃಷಷ್ಟ್ಯುಪಚಾರಾಢ್ಯಾ ಚತುಃಷಷ್ಟಿಕಲಾಮಯೀ~।\\
ಮಹಾಚತುಃ-ಷಷ್ಟಿಕೋಟಿ-ಯೋಗಿನೀ-ಗಣಸೇವಿತಾ ॥೫೮॥

ಮನುವಿದ್ಯಾ ಚಂದ್ರವಿದ್ಯಾ ಚಂದ್ರಮಂಡಲ-ಮಧ್ಯಗಾ~।\\
ಚಾರುರೂಪಾ ಚಾರುಹಾಸಾ ಚಾರುಚಂದ್ರ-ಕಲಾಧರಾ ॥೫೯॥

ಚರಾಚರ-ಜಗನ್ನಾಥಾ ಚಕ್ರರಾಜ-ನಿಕೇತನಾ~।\\
ಪಾರ್ವತೀ ಪದ್ಮನಯನಾ ಪದ್ಮರಾಗ-ಸಮಪ್ರಭಾ ॥೬೦॥

ಪಂಚ-ಪ್ರೇತಾಸನಾಸೀನಾ ಪಂಚಬ್ರಹ್ಮ-ಸ್ವರೂಪಿಣೀ~।\\
ಚಿನ್ಮಯೀ ಪರಮಾನಂದಾ ವಿಜ್ಞಾನ-ಘನರೂಪಿಣೀ ॥೬೧॥

ಧ್ಯಾನ-ಧ್ಯಾತೃ-ಧ್ಯೇಯರೂಪಾ ಧರ್ಮಾಧರ್ಮ-ವಿವರ್ಜಿತಾ~।\\
ವಿಶ್ವರೂಪಾ ಜಾಗರಿಣೀ ಸ್ವಪಂತೀ ತೈಜಸಾತ್ಮಿಕಾ ॥೬೨॥

ಸುಪ್ತಾ ಪ್ರಾಜ್ಞಾತ್ಮಿಕಾ ತುರ್ಯಾ ಸರ್ವಾವಸ್ಥಾ-ವಿವರ್ಜಿತಾ~।\\
ಸೃಷ್ಟಿಕರ್ತ್ರೀ ಬ್ರಹ್ಮರೂಪಾ ಗೋಪ್ತ್ರೀ ಗೋವಿಂದರೂಪಿಣೀ ॥೬೩॥

ಸಂಹಾರಿಣೀ ರುದ್ರರೂಪಾ ತಿರೋಧಾನ-ಕರೀಶ್ವರೀ~।\\
ಸದಾಶಿವಾಽನುಗ್ರಹದಾ ಪಂಚಕೃತ್ಯ-ಪರಾಯಣಾ ॥೬೪॥

ಭಾನುಮಂಡಲ-ಮಧ್ಯಸ್ಥಾ ಭೈರವೀ ಭಗಮಾಲಿನೀ~।\\
ಪದ್ಮಾಸನಾ ಭಗವತೀ ಪದ್ಮನಾಭ-ಸಹೋದರೀ ॥೬೫॥

ಉನ್ಮೇಷ-ನಿಮಿಷೋತ್ಪನ್ನ-ವಿಪನ್ನ-ಭುವನಾವಲಿಃ~।\\
ಸಹಸ್ರ-ಶೀರ್ಷವದನಾ ಸಹಸ್ರಾಕ್ಷೀ ಸಹಸ್ರಪಾತ್ ॥೬೬॥

ಆಬ್ರಹ್ಮ-ಕೀಟ-ಜನನೀ ವರ್ಣಾಶ್ರಮ-ವಿಧಾಯಿನೀ~।\\
ನಿಜಾಜ್ಞಾರೂಪ-ನಿಗಮಾ ಪುಣ್ಯಾಪುಣ್ಯ-ಫಲಪ್ರದಾ ॥೬೭॥

ಶ್ರುತಿ-ಸೀಮಂತ-ಸಿಂದೂರೀ-ಕೃತ-ಪಾದಾಬ್ಜ-ಧೂಲಿಕಾ~।\\
ಸಕಲಾಗಮ-ಸಂದೋಹ-ಶುಕ್ತಿ-ಸಂಪುಟ-ಮೌಕ್ತಿಕಾ ॥೬೮॥

ಪುರುಷಾರ್ಥಪ್ರದಾ ಪೂರ್ಣಾ ಭೋಗಿನೀ ಭುವನೇಶ್ವರೀ~।\\
ಅಂಬಿಕಾಽಽನಾದಿ-ನಿಧನಾ ಹರಿಬ್ರಹ್ಮೇಂದ್ರ-ಸೇವಿತಾ ॥೬೯॥

ನಾರಾಯಣೀ ನಾದರೂಪಾ ನಾಮರೂಪ-ವಿವರ್ಜಿತಾ \as{(೩೦೦)}।\\
ಹ್ರೀಂಕಾರೀ ಹ್ರೀಂಮತೀ ಹೃದ್ಯಾ ಹೇಯೋಪಾದೇಯ-ವರ್ಜಿತಾ ॥೭೦॥

ರಾಜರಾಜಾರ್ಚಿತಾ ರಾಜ್ಞೀ ರಮ್ಯಾ ರಾಜೀವಲೋಚನಾ~।\\
ರಂಜನೀ ರಮಣೀ ರಸ್ಯಾ ರಣತ್ಕಿಂಕಿಣಿ-ಮೇಖಲಾ ॥೭೧॥

ರಮಾ ರಾಕೇಂದುವದನಾ ರತಿರೂಪಾ ರತಿಪ್ರಿಯಾ~।\\
ರಕ್ಷಾಕರೀ ರಾಕ್ಷಸಘ್ನೀ ರಾಮಾ ರಮಣಲಂಪಟಾ ॥೭೨॥

ಕಾಮ್ಯಾ ಕಾಮಕಲಾರೂಪಾ ಕದಂಬ-ಕುಸುಮ-ಪ್ರಿಯಾ~।\\
ಕಲ್ಯಾಣೀ ಜಗತೀಕಂದಾ ಕರುಣಾ-ರಸ-ಸಾಗರಾ ॥೭೩॥

ಕಲಾವತೀ ಕಲಾಲಾಪಾ ಕಾಂತಾ ಕಾದಂಬರೀಪ್ರಿಯಾ~।\\
ವರದಾ ವಾಮನಯನಾ ವಾರುಣೀ-ಮದ-ವಿಹ್ವಲಾ ॥೭೪॥

ವಿಶ್ವಾಧಿಕಾ ವೇದವೇದ್ಯಾ ವಿಂಧ್ಯಾಚಲ-ನಿವಾಸಿನೀ~।\\
ವಿಧಾತ್ರೀ ವೇದಜನನೀ ವಿಷ್ಣುಮಾಯಾ ವಿಲಾಸಿನೀ ॥೭೫॥

ಕ್ಷೇತ್ರಸ್ವರೂಪಾ ಕ್ಷೇತ್ರೇಶೀ ಕ್ಷೇತ್ರ-ಕ್ಷೇತ್ರಜ್ಞ-ಪಾಲಿನೀ~।\\
ಕ್ಷಯವೃದ್ಧಿ-ವಿನಿರ್ಮುಕ್ತಾ ಕ್ಷೇತ್ರಪಾಲ-ಸಮರ್ಚಿತಾ ॥೭೬॥

ವಿಜಯಾ ವಿಮಲಾ ವಂದ್ಯಾ ವಂದಾರು-ಜನ-ವತ್ಸಲಾ~।\\
ವಾಗ್ವಾದಿನೀ ವಾಮಕೇಶೀ ವಹ್ನಿಮಂಡಲ-ವಾಸಿನೀ ॥೭೭॥

ಭಕ್ತಿಮತ್-ಕಲ್ಪಲತಿಕಾ ಪಶುಪಾಶ-ವಿಮೋಚಿನೀ~।\\
ಸಂಹೃತಾಶೇಷ-ಪಾಷಂಡಾ ಸದಾಚಾರ-ಪ್ರವರ್ತಿಕಾ ॥೭೮॥

ತಾಪತ್ರಯಾಗ್ನಿ-ಸಂತಪ್ತ-ಸಮಾಹ್ಲಾದನ ಚಂದ್ರಿಕಾ~।\\
ತರುಣೀ ತಾಪಸಾರಾಧ್ಯಾ ತನುಮಧ್ಯಾ ತಮೋಽಪಹಾ ॥೭೯॥

ಚಿತಿಸ್ತತ್ಪದ-ಲಕ್ಷ್ಯಾರ್ಥಾ ಚಿದೇಕರಸ-ರೂಪಿಣೀ~।\\
ಸ್ವಾತ್ಮಾನಂದ-ಲವೀಭೂತ-ಬ್ರಹ್ಮಾದ್ಯಾನಂದ-ಸಂತತಿಃ ॥೮೦॥

ಪರಾ ಪ್ರತ್ಯಕ್ಚಿತೀರೂಪಾ ಪಶ್ಯಂತೀ ಪರದೇವತಾ~।\\
ಮಧ್ಯಮಾ ವೈಖರೀರೂಪಾ ಭಕ್ತ-ಮಾನಸ-ಹಂಸಿಕಾ ॥೮೧॥

ಕಾಮೇಶ್ವರ-ಪ್ರಾಣನಾಡೀ ಕೃತಜ್ಞಾ ಕಾಮಪೂಜಿತಾ~।\\
ಶೃಂಗಾರ-ರಸ-ಸಂಪೂರ್ಣಾ ಜಯಾ ಜಾಲಂಧರ-ಸ್ಥಿತಾ ॥೮೨॥

ಓಡ್ಯಾಣಪೀಠ-ನಿಲಯಾ ಬಿಂದು-ಮಂಡಲವಾಸಿನೀ~।\\
ರಹೋಯಾಗ-ಕ್ರಮಾರಾಧ್ಯಾ ರಹಸ್ತರ್ಪಣ-ತರ್ಪಿತಾ ॥೮೩॥

ಸದ್ಯಃಪ್ರಸಾದಿನೀ ವಿಶ್ವ-ಸಾಕ್ಷಿಣೀ ಸಾಕ್ಷಿವರ್ಜಿತಾ~।\\
ಷಡಂಗದೇವತಾ-ಯುಕ್ತಾ ಷಾಡ್ಗುಣ್ಯ-ಪರಿಪೂರಿತಾ ॥೮೪॥

ನಿತ್ಯಕ್ಲಿನ್ನಾ ನಿರುಪಮಾ ನಿರ್ವಾಣ-ಸುಖ-ದಾಯಿನೀ~।\\
ನಿತ್ಯಾ-ಷೋಡಶಿಕಾ-ರೂಪಾ ಶ್ರೀಕಂಠಾರ್ಧ-ಶರೀರಿಣೀ ॥೮೫॥

ಪ್ರಭಾವತೀ ಪ್ರಭಾರೂಪಾ ಪ್ರಸಿದ್ಧಾ ಪರಮೇಶ್ವರೀ~।\\
ಮೂಲಪ್ರಕೃತಿರವ್ಯಕ್ತಾ ವ್ಯಕ್ತಾವ್ಯಕ್ತ-ಸ್ವರೂಪಿಣೀ ॥೮೬॥

ವ್ಯಾಪಿನೀ \as{(೪೦೦)} ವಿವಿಧಾಕಾರಾ ವಿದ್ಯಾವಿದ್ಯಾ-ಸ್ವರೂಪಿಣೀ~।\\
ಮಹಾಕಾಮೇಶ-ನಯನ-ಕುಮುದಾಹ್ಲಾದ-ಕೌಮುದೀ ॥೮೭॥

ಭಕ್ತ-ಹಾರ್ದ-ತಮೋಭೇದ-ಭಾನುಮದ್ಭಾನು-ಸಂತತಿಃ~।\\
ಶಿವದೂತೀ ಶಿವಾರಾಧ್ಯಾ ಶಿವಮೂರ್ತಿಃ ಶಿವಂಕರೀ ॥೮೮॥

ಶಿವಪ್ರಿಯಾ ಶಿವಪರಾ ಶಿಷ್ಟೇಷ್ಟಾ ಶಿಷ್ಟಪೂಜಿತಾ~।\\
ಅಪ್ರಮೇಯಾ ಸ್ವಪ್ರಕಾಶಾ ಮನೋವಾಚಾಮಗೋಚರಾ ॥೮೯॥

ಚಿಚ್ಛಕ್ತಿಶ್ ಚೇತನಾರೂಪಾ ಜಡಶಕ್ತಿರ್ಜಡಾತ್ಮಿಕಾ~।\\
ಗಾಯತ್ರೀ ವ್ಯಾಹೃತಿಃ ಸಂಧ್ಯಾ ದ್ವಿಜಬೃಂದ-ನಿಷೇವಿತಾ ॥೯೦॥

ತತ್ತ್ವಾಸನಾ ತತ್ತ್ವಮಯೀ ಪಂಚ-ಕೋಶಾಂತರ-ಸ್ಥಿತಾ~।\\
ನಿಃಸೀಮಮಹಿಮಾ ನಿತ್ಯ-ಯೌವನಾ ಮದಶಾಲಿನೀ ॥೯೧॥

ಮದಘೂರ್ಣಿತ-ರಕ್ತಾಕ್ಷೀ ಮದಪಾಟಲ-ಗಂಡಭೂಃ~।\\
ಚಂದನ-ದ್ರವ-ದಿಗ್ಧಾಂಗೀ ಚಾಂಪೇಯ-ಕುಸುಮ-ಪ್ರಿಯಾ ॥೯೨॥

ಕುಶಲಾ ಕೋಮಲಾಕಾರಾ ಕುರುಕುಲ್ಲಾ ಕುಲೇಶ್ವರೀ~।\\
ಕುಲಕುಂಡಾಲಯಾ ಕೌಲ-ಮಾರ್ಗ-ತತ್ಪರ-ಸೇವಿತಾ ॥೯೩॥

ಕುಮಾರ-ಗಣನಾಥಾಂಬಾ ತುಷ್ಟಿಃ ಪುಷ್ಟಿರ್ಮತಿರ್ಧೃತಿಃ~।\\
ಶಾಂತಿಃ ಸ್ವಸ್ತಿಮತೀ ಕಾಂತಿರ್ನಂದಿನೀ ವಿಘ್ನನಾಶಿನೀ ॥೯೪॥

ತೇಜೋವತೀ ತ್ರಿನಯನಾ ಲೋಲಾಕ್ಷೀ-ಕಾಮರೂಪಿಣೀ~।\\
ಮಾಲಿನೀ ಹಂಸಿನೀ ಮಾತಾ ಮಲಯಾಚಲ-ವಾಸಿನೀ ॥೯೫॥

ಸುಮುಖೀ ನಲಿನೀ ಸುಭ್ರೂಃ ಶೋಭನಾ ಸುರನಾಯಿಕಾ~।\\
ಕಾಲಕಂಠೀ ಕಾಂತಿಮತೀ ಕ್ಷೋಭಿಣೀ ಸೂಕ್ಷ್ಮರೂಪಿಣೀ ॥೯೬॥

ವಜ್ರೇಶ್ವರೀ ವಾಮದೇವೀ ವಯೋಽವಸ್ಥಾ-ವಿವರ್ಜಿತಾ~।\\
ಸಿದ್ಧೇಶ್ವರೀ ಸಿದ್ಧವಿದ್ಯಾ ಸಿದ್ಧಮಾತಾ ಯಶಸ್ವಿನೀ ॥೯೭॥

ವಿಶುದ್ಧಿಚಕ್ರ-ನಿಲಯಾಽಽರಕ್ತವರ್ಣಾ ತ್ರಿಲೋಚನಾ~।\\
ಖಟ್‍ವಾಂಗಾದಿ-ಪ್ರಹರಣಾ ವದನೈಕ-ಸಮನ್ವಿತಾ ॥೯೮॥

ಪಾಯಸಾನ್ನಪ್ರಿಯಾ ತ್ವಕ್ಸ್ಥಾ ಪಶುಲೋಕ-ಭಯಂಕರೀ~।\\
ಅಮೃತಾದಿ-ಮಹಾಶಕ್ತಿ-ಸಂವೃತಾ ಡಾಕಿನೀಶ್ವರೀ ॥೯೯॥

ಅನಾಹತಾಬ್ಜ-ನಿಲಯಾ ಶ್ಯಾಮಾಭಾ ವದನದ್ವಯಾ~।\\
ದಂಷ್ಟ್ರೋಜ್ಜ್ವಲಾಽಕ್ಷ-ಮಾಲಾದಿ-ಧರಾ ರುಧಿರಸಂಸ್ಥಿತಾ ॥೧೦೦॥

ಕಾಲರಾತ್ರ್ಯಾದಿ-ಶಕ್ತ್ಯೌಘ-ವೃತಾ ಸ್ನಿಗ್ಧೌದನಪ್ರಿಯಾ~।\\
ಮಹಾವೀರೇಂದ್ರ-ವರದಾ ರಾಕಿಣ್ಯಂಬಾ-ಸ್ವರೂಪಿಣೀ ॥೧೦೧॥

ಮಣಿಪೂರಾಬ್ಜ-ನಿಲಯಾ ವದನತ್ರಯ-ಸಂಯುತಾ~।\\
ವಜ್ರಾದಿಕಾಯುಧೋಪೇತಾ ಡಾಮರ್ಯಾದಿಭಿರಾವೃತಾ ॥೧೦೨॥

ರಕ್ತವರ್ಣಾ ಮಾಂಸನಿಷ್ಠಾ \as{(೫೦೦)} ಗುಡಾನ್ನ-ಪ್ರೀತ-ಮಾನಸಾ~।\\
ಸಮಸ್ತಭಕ್ತ-ಸುಖದಾ ಲಾಕಿನ್ಯಂಬಾ-ಸ್ವರೂಪಿಣೀ ॥೧೦೩॥

ಸ್ವಾಧಿಷ್ಠಾನಾಂಬುಜ-ಗತಾ ಚತುರ್ವಕ್ತ್ರ-ಮನೋಹರಾ~।\\
ಶೂಲಾದ್ಯಾಯುಧ-ಸಂಪನ್ನಾ ಪೀತವರ್ಣಾಽತಿಗರ್ವಿತಾ ॥೧೦೪॥

ಮೇದೋನಿಷ್ಠಾ ಮಧುಪ್ರೀತಾ ಬಂದಿನ್ಯಾದಿ-ಸಮನ್ವಿತಾ~।\\
ದಧ್ಯನ್ನಾಸಕ್ತ-ಹೃದಯಾ ಕಾಕಿನೀ-ರೂಪ-ಧಾರಿಣೀ ॥೧೦೫॥

ಮೂಲಾಧಾರಾಂಬುಜಾರೂಢಾ ಪಂಚ-ವಕ್ತ್ರಾಽಸ್ಥಿ-ಸಂಸ್ಥಿತಾ~।\\
ಅಂಕುಶಾದಿ-ಪ್ರಹರಣಾ ವರದಾದಿ-ನಿಷೇವಿತಾ ॥೧೦೬॥

ಮುದ್ಗೌದನಾಸಕ್ತ-ಚಿತ್ತಾ ಸಾಕಿನ್ಯಂಬಾ-ಸ್ವರೂಪಿಣೀ~।\\
ಆಜ್ಞಾ-ಚಕ್ರಾಬ್ಜ-ನಿಲಯಾ ಶುಕ್ಲವರ್ಣಾ ಷಡಾನನಾ ॥೧೦೭॥

ಮಜ್ಜಾಸಂಸ್ಥಾ ಹಂಸವತೀ-ಮುಖ್ಯ-ಶಕ್ತಿ-ಸಮನ್ವಿತಾ~।\\
ಹರಿದ್ರಾನ್ನೈಕ-ರಸಿಕಾ ಹಾಕಿನೀ-ರೂಪ-ಧಾರಿಣೀ ॥೧೦೮॥

ಸಹಸ್ರದಲ-ಪದ್ಮಸ್ಥಾ ಸರ್ವ-ವರ್ಣೋಪ-ಶೋಭಿತಾ~।\\
ಸರ್ವಾಯುಧಧರಾ ಶುಕ್ಲ-ಸಂಸ್ಥಿತಾ ಸರ್ವತೋಮುಖೀ ॥೧೦೯॥

ಸರ್ವೌದನ-ಪ್ರೀತಚಿತ್ತಾ ಯಾಕಿನ್ಯಂಬಾ-ಸ್ವರೂಪಿಣೀ~।\\
ಸ್ವಾಹಾ ಸ್ವಧಾಽಮತಿರ್ಮೇಧಾ ಶ್ರುತಿಃ ಸ್ಮೃತಿರನುತ್ತಮಾ ॥೧೧೦॥

ಪುಣ್ಯಕೀರ್ತಿಃ ಪುಣ್ಯಲಭ್ಯಾ ಪುಣ್ಯಶ್ರವಣ-ಕೀರ್ತನಾ~।\\
ಪುಲೋಮಜಾರ್ಚಿತಾ ಬಂಧ-ಮೋಚನೀ ಬಂಧುರಾಲಕಾ ॥೧೧೧॥

ವಿಮರ್ಶರೂಪಿಣೀ ವಿದ್ಯಾ ವಿಯದಾದಿ-ಜಗತ್ಪ್ರಸೂಃ~।\\
ಸರ್ವವ್ಯಾಧಿ-ಪ್ರಶಮನೀ ಸರ್ವಮೃತ್ಯು-ನಿವಾರಿಣೀ ॥೧೧೨॥

ಅಗ್ರಗಣ್ಯಾಽಚಿಂತ್ಯರೂಪಾ ಕಲಿಕಲ್ಮಷ-ನಾಶಿನೀ~।\\
ಕಾತ್ಯಾಯನೀ ಕಾಲಹಂತ್ರೀ ಕಮಲಾಕ್ಷ-ನಿಷೇವಿತಾ ॥೧೧೩॥

ತಾಂಬೂಲ-ಪೂರಿತ-ಮುಖೀ ದಾಡಿಮೀ-ಕುಸುಮ-ಪ್ರಭಾ~।\\
ಮೃಗಾಕ್ಷೀ ಮೋಹಿನೀ ಮುಖ್ಯಾ ಮೃಡಾನೀ ಮಿತ್ರರೂಪಿಣೀ ॥೧೧೪॥

ನಿತ್ಯತೃಪ್ತಾ ಭಕ್ತನಿಧಿರ್ನಿಯಂತ್ರೀ ನಿಖಿಲೇಶ್ವರೀ~।\\
ಮೈತ್ರ್ಯಾದಿ-ವಾಸನಾಲಭ್ಯಾ ಮಹಾಪ್ರಲಯ-ಸಾಕ್ಷಿಣೀ ॥೧೧೫॥

ಪರಾ ಶಕ್ತಿಃ ಪರಾ ನಿಷ್ಠಾ ಪ್ರಜ್ಞಾನಘನ-ರೂಪಿಣೀ~।\\
ಮಾಧ್ವೀಪಾನಾಲಸಾ ಮತ್ತಾ ಮಾತೃಕಾ-ವರ್ಣ-ರೂಪಿಣೀ ॥೧೧೬॥

ಮಹಾಕೈಲಾಸ-ನಿಲಯಾ ಮೃಣಾಲ-ಮೃದು-ದೋರ್ಲತಾ~।\\
ಮಹನೀಯಾ ದಯಾಮೂರ್ತಿರ್ಮಹಾಸಾಮ್ರಾಜ್ಯ-ಶಾಲಿನೀ ॥೧೧೭॥

ಆತ್ಮವಿದ್ಯಾ ಮಹಾವಿದ್ಯಾ ಶ್ರೀವಿದ್ಯಾ ಕಾಮಸೇವಿತಾ~।\\
ಶ್ರೀ-ಷೋಡಶಾಕ್ಷರೀ-ವಿದ್ಯಾ ತ್ರಿಕೂಟಾ ಕಾಮಕೋಟಿಕಾ ॥೧೧೮॥

ಕಟಾಕ್ಷ-ಕಿಂಕರೀ-ಭೂತ-ಕಮಲಾ-ಕೋಟಿ-ಸೇವಿತಾ~।\\
ಶಿರಃಸ್ಥಿತಾ ಚಂದ್ರನಿಭಾ ಭಾಲಸ್ಥೇಂದ್ರ-ಧನುಃಪ್ರಭಾ ॥೧೧೯॥

ಹೃದಯಸ್ಥಾ ರವಿಪ್ರಖ್ಯಾ ತ್ರಿಕೋಣಾಂತರ-ದೀಪಿಕಾ~।\\
ದಾಕ್ಷಾಯಣೀ ದೈತ್ಯಹಂತ್ರೀ ದಕ್ಷಯಜ್ಞ-ವಿನಾಶಿನೀ \as{(೬೦೦)} ॥೧೨೦॥

ದರಾಂದೋಲಿತ-ದೀರ್ಘಾಕ್ಷೀ ದರ-ಹಾಸೋಜ್ಜ್ವಲನ್ಮುಖೀ~।\\
ಗುರುಮೂರ್ತಿರ್ಗುಣನಿಧಿರ್ಗೋಮಾತಾ ಗುಹಜನ್ಮಭೂಃ ॥೧೨೧॥

ದೇವೇಶೀ ದಂಡನೀತಿಸ್ಥಾ ದಹರಾಕಾಶ-ರೂಪಿಣೀ~।\\
ಪ್ರತಿಪನ್ಮುಖ್ಯ-ರಾಕಾಂತ-ತಿಥಿ-ಮಂಡಲ-ಪೂಜಿತಾ ॥೧೨೨॥

ಕಲಾತ್ಮಿಕಾ ಕಲಾನಾಥಾ ಕಾವ್ಯಾಲಾಪ-ವಿನೋದಿನೀ~।\\
ಸಚಾಮರ-ರಮಾ-ವಾಣೀ-ಸವ್ಯ-ದಕ್ಷಿಣ-ಸೇವಿತಾ ॥೧೨೩॥

ಆದಿಶಕ್ತಿರಮೇಯಾಽಽತ್ಮಾ ಪರಮಾ ಪಾವನಾಕೃತಿಃ~।\\
ಅನೇಕಕೋಟಿ-ಬ್ರಹ್ಮಾಂಡ-ಜನನೀ ದಿವ್ಯವಿಗ್ರಹಾ ॥೧೨೪॥

ಕ್ಲೀಂಕಾರೀ ಕೇವಲಾ ಗುಹ್ಯಾ ಕೈವಲ್ಯ-ಪದದಾಯಿನೀ~।\\
ತ್ರಿಪುರಾ ತ್ರಿಜಗದ್ವಂದ್ಯಾ ತ್ರಿಮೂರ್ತಿಸ್ತ್ರಿದಶೇಶ್ವರೀ ॥೧೨೫॥

ತ್ರ್ಯಕ್ಷರೀ ದಿವ್ಯ-ಗಂಧಾಢ್ಯಾ ಸಿಂದೂರ-ತಿಲಕಾಂಚಿತಾ~।\\
ಉಮಾ ಶೈಲೇಂದ್ರತನಯಾ ಗೌರೀ ಗಂಧರ್ವ-ಸೇವಿತಾ ॥೧೨೬॥

ವಿಶ್ವಗರ್ಭಾ ಸ್ವರ್ಣಗರ್ಭಾ ವರದಾ ವಾಗಧೀಶ್ವರೀ~।\\
ಧ್ಯಾನಗಮ್ಯಾಽಪರಿಚ್ಛೇದ್ಯಾ ಜ್ಞಾನದಾ ಜ್ಞಾನವಿಗ್ರಹಾ ॥೧೨೭॥

ಸರ್ವವೇದಾಂತ-ಸಂವೇದ್ಯಾ ಸತ್ಯಾನಂದ-ಸ್ವರೂಪಿಣೀ~।\\
ಲೋಪಾಮುದ್ರಾರ್ಚಿತಾ ಲೀಲಾ-ಕ್ಲೃಪ್ತ-ಬ್ರಹ್ಮಾಂಡ-ಮಂಡಲಾ ॥೧೨೮॥

ಅದೃಶ್ಯಾ ದೃಶ್ಯರಹಿತಾ ವಿಜ್ಞಾತ್ರೀ ವೇದ್ಯವರ್ಜಿತಾ~।\\
ಯೋಗಿನೀ ಯೋಗದಾ ಯೋಗ್ಯಾ ಯೋಗಾನಂದಾ ಯುಗಂಧರಾ ॥೧೨೯॥

ಇಚ್ಛಾಶಕ್ತಿ-ಜ್ಞಾನಶಕ್ತಿ-ಕ್ರಿಯಾಶಕ್ತಿ-ಸ್ವರೂಪಿಣೀ~।\\
ಸರ್ವಾಧಾರಾ ಸುಪ್ರತಿಷ್ಠಾ ಸದಸದ್ರೂಪ-ಧಾರಿಣೀ ॥೧೩೦॥

ಅಷ್ಟಮೂರ್ತಿರಜಾಜೈತ್ರೀ ಲೋಕಯಾತ್ರಾ-ವಿಧಾಯಿನೀ~।\\
ಏಕಾಕಿನೀ ಭೂಮರೂಪಾ ನಿರ್ದ್ವೈತಾ ದ್ವೈತವರ್ಜಿತಾ ॥೧೩೧॥

ಅನ್ನದಾ ವಸುದಾ ವೃದ್ಧಾ ಬ್ರಹ್ಮಾತ್ಮೈಕ್ಯ-ಸ್ವರೂಪಿಣೀ~।\\
ಬೃಹತೀ ಬ್ರಾಹ್ಮಣೀ ಬ್ರಾಹ್ಮೀ ಬ್ರಹ್ಮಾನಂದಾ ಬಲಿಪ್ರಿಯಾ ॥೧೩೨॥

ಭಾಷಾರೂಪಾ ಬೃಹತ್ಸೇನಾ ಭಾವಾಭಾವ-ವಿವರ್ಜಿತಾ~।\\
ಸುಖಾರಾಧ್ಯಾ ಶುಭಕರೀ ಶೋಭನಾ ಸುಲಭಾ ಗತಿಃ ॥೧೩೩॥

ರಾಜ-ರಾಜೇಶ್ವರೀ ರಾಜ್ಯ-ದಾಯಿನೀ ರಾಜ್ಯ-ವಲ್ಲಭಾ~।\\
ರಾಜತ್ಕೃಪಾ ರಾಜಪೀಠ-ನಿವೇಶಿತ-ನಿಜಾಶ್ರಿತಾ ॥೧೩೪॥

ರಾಜ್ಯಲಕ್ಷ್ಮೀಃ ಕೋಶನಾಥಾ ಚತುರಂಗ-ಬಲೇಶ್ವರೀ~।\\
ಸಾಮ್ರಾಜ್ಯ-ದಾಯಿನೀ ಸತ್ಯಸಂಧಾ ಸಾಗರಮೇಖಲಾ ॥೧೩೫॥

ದೀಕ್ಷಿತಾ ದೈತ್ಯಶಮನೀ ಸರ್ವಲೋಕ-ವಶಂಕರೀ~।\\
ಸರ್ವಾರ್ಥದಾತ್ರೀ ಸಾವಿತ್ರೀ ಸಚ್ಚಿದಾನಂದ-ರೂಪಿಣೀ \as{(೭೦೦)} ॥೧೩೬॥

ದೇಶ-ಕಾಲಾಪರಿಚ್ಛಿನ್ನಾ ಸರ್ವಗಾ ಸರ್ವಮೋಹಿನೀ~।\\
ಸರಸ್ವತೀ ಶಾಸ್ತ್ರಮಯೀ ಗುಹಾಂಬಾ ಗುಹ್ಯರೂಪಿಣೀ ॥೧೩೭॥

ಸರ್ವೋಪಾಧಿ-ವಿನಿರ್ಮುಕ್ತಾ ಸದಾಶಿವ-ಪತಿವ್ರತಾ~।\\
ಸಂಪ್ರದಾಯೇಶ್ವರೀ ಸಾಧ್ವೀ ಗುರುಮಂಡಲ-ರೂಪಿಣೀ ॥೧೩೮॥

ಕುಲೋತ್ತೀರ್ಣಾ ಭಗಾರಾಧ್ಯಾ ಮಾಯಾ ಮಧುಮತೀ ಮಹೀ~।\\
ಗಣಾಂಬಾ ಗುಹ್ಯಕಾರಾಧ್ಯಾ ಕೋಮಲಾಂಗೀ ಗುರುಪ್ರಿಯಾ ॥೧೩೯॥

ಸ್ವತಂತ್ರಾ ಸರ್ವತಂತ್ರೇಶೀ ದಕ್ಷಿಣಾಮೂರ್ತಿ-ರೂಪಿಣೀ~।\\
ಸನಕಾದಿ-ಸಮಾರಾಧ್ಯಾ ಶಿವಜ್ಞಾನ-ಪ್ರದಾಯಿನೀ ॥೧೪೦॥

ಚಿತ್ಕಲಾಽಽನಂದ-ಕಲಿಕಾ ಪ್ರೇಮರೂಪಾ ಪ್ರಿಯಂಕರೀ~।\\
ನಾಮಪಾರಾಯಣ-ಪ್ರೀತಾ ನಂದಿವಿದ್ಯಾ ನಟೇಶ್ವರೀ ॥೧೪೧॥

ಮಿಥ್ಯಾ-ಜಗದಧಿಷ್ಠಾನಾ ಮುಕ್ತಿದಾ ಮುಕ್ತಿರೂಪಿಣೀ~।\\
ಲಾಸ್ಯಪ್ರಿಯಾ ಲಯಕರೀ ಲಜ್ಜಾ ರಂಭಾದಿವಂದಿತಾ ॥೧೪೨॥

ಭವದಾವ-ಸುಧಾವೃಷ್ಟಿಃ ಪಾಪಾರಣ್ಯ-ದವಾನಲಾ~।\\
ದೌರ್ಭಾಗ್ಯ-ತೂಲವಾತೂಲಾ ಜರಾಧ್ವಾಂತ-ರವಿಪ್ರಭಾ ॥೧೪೩॥

ಭಾಗ್ಯಾಬ್ಧಿ-ಚಂದ್ರಿಕಾ ಭಕ್ತ-ಚಿತ್ತಕೇಕಿ-ಘನಾಘನಾ~।\\
ರೋಗಪರ್ವತ-ದಂಭೋಲಿರ್ಮೃತ್ಯುದಾರು-ಕುಠಾರಿಕಾ ॥೧೪೪॥

ಮಹೇಶ್ವರೀ ಮಹಾಕಾಲೀ ಮಹಾಗ್ರಾಸಾ ಮಹಾಶನಾ~।\\
ಅಪರ್ಣಾ ಚಂಡಿಕಾ ಚಂಡಮುಂಡಾಸುರ-ನಿಷೂದಿನೀ ॥೧೪೫॥

ಕ್ಷರಾಕ್ಷರಾತ್ಮಿಕಾ ಸರ್ವ-ಲೋಕೇಶೀ ವಿಶ್ವಧಾರಿಣೀ~।\\
ತ್ರಿವರ್ಗದಾತ್ರೀ ಸುಭಗಾ ತ್ರ್ಯಂಬಕಾ ತ್ರಿಗುಣಾತ್ಮಿಕಾ ॥೧೪೬॥

ಸ್ವರ್ಗಾಪವರ್ಗದಾ ಶುದ್ಧಾ ಜಪಾಪುಷ್ಪ-ನಿಭಾಕೃತಿಃ~।\\
ಓಜೋವತೀ ದ್ಯುತಿಧರಾ ಯಜ್ಞರೂಪಾ ಪ್ರಿಯವ್ರತಾ ॥೧೪೭॥

ದುರಾರಾಧ್ಯಾ ದುರಾಧರ್ಷಾ ಪಾಟಲೀ-ಕುಸುಮ-ಪ್ರಿಯಾ~।\\
ಮಹತೀ ಮೇರುನಿಲಯಾ ಮಂದಾರ-ಕುಸುಮ-ಪ್ರಿಯಾ ॥೧೪೮॥

ವೀರಾರಾಧ್ಯಾ ವಿರಾಡ್ರೂಪಾ ವಿರಜಾ ವಿಶ್ವತೋಮುಖೀ~।\\
ಪ್ರತ್ಯಗ್ರೂಪಾ ಪರಾಕಾಶಾ ಪ್ರಾಣದಾ ಪ್ರಾಣರೂಪಿಣೀ ॥೧೪೯॥

ಮಾರ್ತಾಂಡ-ಭೈರವಾರಾಧ್ಯಾ ಮಂತ್ರಿಣೀನ್ಯಸ್ತ-ರಾಜ್ಯಧೂಃ~।\\
ತ್ರಿಪುರೇಶೀ ಜಯತ್ಸೇನಾ ನಿಸ್ತ್ರೈಗುಣ್ಯಾ ಪರಾಪರಾ ॥೧೫೦॥

ಸತ್ಯ-ಜ್ಞಾನಾನಂದ-ರೂಪಾ ಸಾಮರಸ್ಯ-ಪರಾಯಣಾ~।\\
ಕಪರ್ದಿನೀ ಕಲಾಮಾಲಾ ಕಾಮಧುಕ್ ಕಾಮರೂಪಿಣೀ ॥೧೫೧॥

ಕಲಾನಿಧಿಃ ಕಾವ್ಯಕಲಾ ರಸಜ್ಞಾ ರಸಶೇವಧಿಃ \as{(೮೦೦)}~।\\
ಪುಷ್ಟಾ ಪುರಾತನಾ ಪೂಜ್ಯಾ ಪುಷ್ಕರಾ ಪುಷ್ಕರೇಕ್ಷಣಾ ॥೧೫೨॥

ಪರಂಜ್ಯೋತಿಃ ಪರಂಧಾಮ ಪರಮಾಣುಃ ಪರಾತ್ಪರಾ~।\\
ಪಾಶಹಸ್ತಾ ಪಾಶಹಂತ್ರೀ ಪರಮಂತ್ರ-ವಿಭೇದಿನೀ ॥೧೫೩॥

ಮೂರ್ತಾಽಮೂರ್ತಾಽನಿತ್ಯತೃಪ್ತಾ ಮುನಿಮಾನಸ-ಹಂಸಿಕಾ~।\\
ಸತ್ಯವ್ರತಾ ಸತ್ಯರೂಪಾ ಸರ್ವಾಂತರ್ಯಾಮಿನೀ ಸತೀ ॥೧೫೪॥

ಬ್ರಹ್ಮಾಣೀ ಬ್ರಹ್ಮಜನನೀ ಬಹುರೂಪಾ ಬುಧಾರ್ಚಿತಾ~।\\
ಪ್ರಸವಿತ್ರೀ ಪ್ರಚಂಡಾಽಽಜ್ಞಾ ಪ್ರತಿಷ್ಠಾ ಪ್ರಕಟಾಕೃತಿಃ ॥೧೫೫॥

ಪ್ರಾಣೇಶ್ವರೀ ಪ್ರಾಣದಾತ್ರೀ ಪಂಚಾಶತ್ಪೀಠ-ರೂಪಿಣೀ~।\\
ವಿಶೃಂಖಲಾ ವಿವಿಕ್ತಸ್ಥಾ ವೀರಮಾತಾ ವಿಯತ್ಪ್ರಸೂಃ ॥೧೫೬॥

ಮುಕುಂದಾ ಮುಕ್ತಿನಿಲಯಾ ಮೂಲವಿಗ್ರಹ-ರೂಪಿಣೀ~।\\
ಭಾವಜ್ಞಾ ಭವರೋಗಘ್ನೀ ಭವಚಕ್ರ-ಪ್ರವರ್ತಿನೀ ॥೧೫೭॥

ಛಂದಃಸಾರಾ ಶಾಸ್ತ್ರಸಾರಾ ಮಂತ್ರಸಾರಾ ತಲೋದರೀ~।\\
ಉದಾರಕೀರ್ತಿರುದ್ದಾಮವೈಭವಾ ವರ್ಣರೂಪಿಣೀ ॥೧೫೮॥

ಜನ್ಮಮೃತ್ಯು-ಜರಾತಪ್ತ-ಜನವಿಶ್ರಾಂತಿ-ದಾಯಿನೀ~।\\
ಸರ್ವೋಪನಿಷದುದ್ಘುಷ್ಟಾ ಶಾಂತ್ಯತೀತ-ಕಲಾತ್ಮಿಕಾ ॥೧೫೯॥

ಗಂಭೀರಾ ಗಗನಾಂತಸ್ಥಾ ಗರ್ವಿತಾ ಗಾನಲೋಲುಪಾ~।\\
ಕಲ್ಪನಾ-ರಹಿತಾ ಕಾಷ್ಠಾಽಕಾಂತಾ ಕಾಂತಾರ್ಧ-ವಿಗ್ರಹಾ ॥೧೬೦॥

ಕಾರ್ಯಕಾರಣ-ನಿರ್ಮುಕ್ತಾ ಕಾಮಕೇಲಿ-ತರಂಗಿತಾ~।\\
ಕನತ್ಕನಕತಾ-ಟಂಕಾ ಲೀಲಾ-ವಿಗ್ರಹ-ಧಾರಿಣೀ ॥೧೬೧॥

ಅಜಾ ಕ್ಷಯವಿನಿರ್ಮುಕ್ತಾ ಮುಗ್ಧಾ ಕ್ಷಿಪ್ರ-ಪ್ರಸಾದಿನೀ~।\\
ಅಂತರ್ಮುಖ-ಸಮಾರಾಧ್ಯಾ ಬಹಿರ್ಮುಖ-ಸುದುರ್ಲಭಾ ॥೧೬೨॥

ತ್ರಯೀ ತ್ರಿವರ್ಗನಿಲಯಾ ತ್ರಿಸ್ಥಾ ತ್ರಿಪುರಮಾಲಿನೀ~।\\
ನಿರಾಮಯಾ ನಿರಾಲಂಬಾ ಸ್ವಾತ್ಮಾರಾಮಾ ಸುಧಾಸೃತಿಃ ॥೧೬೩॥

ಸಂಸಾರಪಂಕ-ನಿರ್ಮಗ್ನ-ಸಮುದ್ಧರಣ-ಪಂಡಿತಾ~।\\
ಯಜ್ಞಪ್ರಿಯಾ ಯಜ್ಞಕರ್ತ್ರೀ ಯಜಮಾನ-ಸ್ವರೂಪಿಣೀ ॥೧೬೪॥

ಧರ್ಮಾಧಾರಾ ಧನಾಧ್ಯಕ್ಷಾ ಧನಧಾನ್ಯ-ವಿವರ್ಧಿನೀ~।\\
ವಿಪ್ರಪ್ರಿಯಾ ವಿಪ್ರರೂಪಾ ವಿಶ್ವಭ್ರಮಣ-ಕಾರಿಣೀ ॥೧೬೫॥

ವಿಶ್ವಗ್ರಾಸಾ ವಿದ್ರುಮಾಭಾ ವೈಷ್ಣವೀ ವಿಷ್ಣುರೂಪಿಣೀ~।\\
ಅಯೋನಿರ್ಯೋನಿನಿಲಯಾ ಕೂಟಸ್ಥಾ ಕುಲರೂಪಿಣೀ ॥೧೬೬॥

ವೀರಗೋಷ್ಠೀಪ್ರಿಯಾ ವೀರಾ ನೈಷ್ಕರ್ಮ್ಯಾ \as{(೯೦೦)} ನಾದರೂಪಿಣೀ~।\\
ವಿಜ್ಞಾನಕಲನಾ ಕಲ್ಯಾ ವಿದಗ್ಧಾ ಬೈಂದವಾಸನಾ ॥೧೬೭॥

ತತ್ತ್ವಾಧಿಕಾ ತತ್ತ್ವಮಯೀ ತತ್ತ್ವಮರ್ಥ-ಸ್ವರೂಪಿಣೀ~।\\
ಸಾಮಗಾನಪ್ರಿಯಾ ಸೌಮ್ಯಾ ಸದಾಶಿವ-ಕುಟುಂಬಿನೀ ॥೧೬೮॥

ಸವ್ಯಾಪಸವ್ಯ-ಮಾರ್ಗಸ್ಥಾ ಸರ್ವಾಪದ್ವಿನಿವಾರಿಣೀ~।\\
ಸ್ವಸ್ಥಾ ಸ್ವಭಾವಮಧುರಾ ಧೀರಾ ಧೀರಸಮರ್ಚಿತಾ ॥೧೬೯॥

ಚೈತನ್ಯಾರ್ಘ್ಯ-ಸಮಾರಾಧ್ಯಾ ಚೈತನ್ಯ-ಕುಸುಮಪ್ರಿಯಾ~।\\
ಸದೋದಿತಾ ಸದಾತುಷ್ಟಾ ತರುಣಾದಿತ್ಯ-ಪಾಟಲಾ ॥೧೭೦॥

ದಕ್ಷಿಣಾ-ದಕ್ಷಿಣಾರಾಧ್ಯಾ ದರಸ್ಮೇರ-ಮುಖಾಂಬುಜಾ~।\\
ಕೌಲಿನೀ-ಕೇವಲಾಽನರ್ಘ್ಯ-ಕೈವಲ್ಯ-ಪದದಾಯಿನೀ ॥೧೭೧॥

ಸ್ತೋತ್ರಪ್ರಿಯಾ ಸ್ತುತಿಮತೀ ಶ್ರುತಿ-ಸಂಸ್ತುತ-ವೈಭವಾ~।\\
ಮನಸ್ವಿನೀ ಮಾನವತೀ ಮಹೇಶೀ ಮಂಗಲಾಕೃತಿಃ ॥೧೭೨॥

ವಿಶ್ವಮಾತಾ ಜಗದ್ಧಾತ್ರೀ ವಿಶಾಲಾಕ್ಷೀ ವಿರಾಗಿಣೀ~।\\
ಪ್ರಗಲ್ಭಾ ಪರಮೋದಾರಾ ಪರಾಮೋದಾ ಮನೋಮಯೀ ॥೧೭೩॥

ವ್ಯೋಮಕೇಶೀ ವಿಮಾನಸ್ಥಾ ವಜ್ರಿಣೀ ವಾಮಕೇಶ್ವರೀ~।\\
ಪಂಚಯಜ್ಞ-ಪ್ರಿಯಾ ಪಂಚ-ಪ್ರೇತ-ಮಂಚಾಧಿಶಾಯಿನೀ ॥೧೭೪॥

ಪಂಚಮೀ ಪಂಚಭೂತೇಶೀ ಪಂಚ-ಸಂಖ್ಯೋಪಚಾರಿಣೀ~।\\
ಶಾಶ್ವತೀ ಶಾಶ್ವತೈಶ್ವರ್ಯಾ ಶರ್ಮದಾ ಶಂಭುಮೋಹಿನೀ ॥೧೭೫॥

ಧರಾ ಧರಸುತಾ ಧನ್ಯಾ ಧರ್ಮಿಣೀ ಧರ್ಮವರ್ಧಿನೀ~।\\
ಲೋಕಾತೀತಾ ಗುಣಾತೀತಾ ಸರ್ವಾತೀತಾ ಶಮಾತ್ಮಿಕಾ ॥೧೭೬॥

ಬಂಧೂಕ-ಕುಸುಮಪ್ರಖ್ಯಾ ಬಾಲಾ ಲೀಲಾವಿನೋದಿನೀ~।\\
ಸುಮಂಗಲೀ ಸುಖಕರೀ ಸುವೇಷಾಢ್ಯಾ ಸುವಾಸಿನೀ ॥೧೭೭॥

ಸುವಾಸಿನ್ಯರ್ಚನ-ಪ್ರೀತಾಽಽಶೋಭನಾ ಶುದ್ಧಮಾನಸಾ~।\\
ಬಿಂದು-ತರ್ಪಣ-ಸಂತುಷ್ಟಾ ಪೂರ್ವಜಾ ತ್ರಿಪುರಾಂಬಿಕಾ ॥೧೭೮॥

ದಶಮುದ್ರಾ-ಸಮಾರಾಧ್ಯಾ ತ್ರಿಪುರಾಶ್ರೀ-ವಶಂಕರೀ~।\\
ಜ್ಞಾನಮುದ್ರಾ ಜ್ಞಾನಗಮ್ಯಾ ಜ್ಞಾನಜ್ಞೇಯ-ಸ್ವರೂಪಿಣೀ ॥೧೭೯॥

ಯೋನಿಮುದ್ರಾ ತ್ರಿಖಂಡೇಶೀ ತ್ರಿಗುಣಾಂಬಾ ತ್ರಿಕೋಣಗಾ~।\\
ಅನಘಾಽದ್ಭುತ-ಚಾರಿತ್ರಾ ವಾಂಛಿತಾರ್ಥ-ಪ್ರದಾಯಿನೀ ॥೧೮೦॥

ಅಭ್ಯಾಸಾತಿಶಯ-ಜ್ಞಾತಾ ಷಡಧ್ವಾತೀತ-ರೂಪಿಣೀ~।\\
ಅವ್ಯಾಜ-ಕರುಣಾ-ಮೂರ್ತಿರಜ್ಞಾನ-ಧ್ವಾಂತ-ದೀಪಿಕಾ ॥೧೮೧॥

ಆಬಾಲ-ಗೋಪ-ವಿದಿತಾ ಸರ್ವಾನುಲ್ಲಂಘ್ಯ-ಶಾಸನಾ~।\\
ಶ್ರೀಚಕ್ರರಾಜ-ನಿಲಯಾ ಶ್ರೀಮತ್-ತ್ರಿಪುರಸುಂದರೀ ॥೧೮೨॥

ಶ್ರೀಶಿವಾ ಶಿವ-ಶಕ್ತ್ಯೈಕ್ಯ-ರೂಪಿಣೀ ಲಲಿತಾಂಬಿಕಾ\as{(೧೦೦೦)}।{\bfseries ಶ್ರೀಂಹ್ರೀಂಐಂ ಓಂ}\\
ಏವಂ ಶ್ರೀಲಲಿತಾ ದೇವ್ಯಾ ನಾಮ್ನಾಂ ಸಾಹಸ್ರಕಂ ಜಗುಃ ॥೧೮೩॥
\section{ಶ್ರೀಮೇಧಾದಕ್ಷಿಣಾಮೂರ್ತಿಸಹಸ್ರನಾಮಸ್ತೋತ್ರಂ}
ಅಸ್ಯ ಶ್ರೀ ಮೇಧಾದಕ್ಷಿಣಾಮೂರ್ತಿಸಹಸ್ರನಾಮಸ್ತೋತ್ರಸ್ಯ ಬ್ರಹ್ಮಾ ಋಷಿಃ~। ಗಾಯತ್ರೀ ಛಂದಃ~। ದಕ್ಷಿಣಾಮೂರ್ತಿರ್ದೇವತಾ~। ಓಂ ಬೀಜಂ~। ಸ್ವಾಹಾ ಶಕ್ತಿಃ~। ನಮಃ ಕೀಲಕಂ~। ದಕ್ಷಿಣಾಮೂರ್ತಿಪ್ರೀತ್ಯರ್ಥೇ ಜಪೇ ವಿನಿಯೋಗಃ~।

\dhyana{ಸಿದ್ಧಿತೋಯನಿಧೇರ್ಮಧ್ಯೇ ರತ್ನಗ್ರೀವೇ ಮನೋರಮೇ~।\\
ಕದಂಬವನಿಕಾಮಧ್ಯೇ ಶ್ರೀಮದ್ವಟತರೋರಧಃ ॥೧॥

	ಆಸೀನಮಾದ್ಯಂ ಪುರುಷಮಾದಿಮಧ್ಯಾಂತವರ್ಜಿತಂ~।\\
	ಶುದ್ಧಸ್ಫಟಿಕ ಗೋಕ್ಷೀರ ಶರತ್ಪೂರ್ಣೇಂದು ಶೇಖರಂ ॥೨॥

ದಕ್ಷಿಣೇ ಚಾಕ್ಷಮಾಲಾಂ ಚ ವಹ್ನಿಂ ವೈ ವಾಮಹಸ್ತಕೇ~।\\
ಜಟಾಮಂಡಲ ಸಂಲಗ್ನ ಶೀತಾಂಶುಕರ ಮಂಡಿತಂ ॥೩॥

	ನಾಗಹಾರಧರಂ ಚಾರುಕಂಕಣೈಃ ಕಟಿಸೂತ್ರಕೈಃ~।\\
	ವಿರಾಜಮಾನ ವೃಷಭಂ ವ್ಯಾಘ್ರ ಚರ್ಮಾಂಬರಾವೃತಂ ॥೪॥

ಚಿಂತಾಮಣಿ ಮಹಾಬೃಂದೈಃ ಕಲ್ಪಕೈಃ ಕಾಮಧೇನುಭಿಃ~।\\
ಚತುಷ್ಷಷ್ಟಿ ಕಲಾವಿದ್ಯಾ ಮೂರ್ತಿಭಿಃ ಶ್ರುತಿಮಸ್ತಕೈಃ ॥೫॥

	ರತ್ನಸಿಂಹಾಸನೇ ಸಾಧುದ್ವೀಪಿಚರ್ಮ ಸಮಾಯುತಂ~।\\
	ತತ್ರಾಷ್ಟದಲಪದ್ಮಸ್ಯ ಕರ್ಣಿಕಾಯಾಂ ಸುಶೋಭನೇ ॥೬॥

ವೀರಾಸನೇ ಸಮಾಸೀನಂ ಲಂಬದಕ್ಷಪದಾಂಬುಜಂ~।\\
ಜ್ಞಾನಮುದ್ರಾಂ ಪುಸ್ತಕಂ ಚ ವರಾಭೀತಿಧರಂ ಹರಂ ॥೭॥

	ಪಾದಮೂಲ ಸಮಾಕ್ರಾಂತ ಮಹಾಪಸ್ಮಾರ ವೈಭವಂ~।\\
	ರುದ್ರಾಕ್ಷಮಾಲಾಭರಣ ಭೂಷಿತಂ ಭೂತಿಭಾಸುರಂ ॥೮॥

ಗಜಚರ್ಮೋತ್ತರೀಯಂ ಚ ಮಂದಸ್ಮಿತ ಮುಖಾಂಬುಜಂ~।\\
ಸಿದ್ಧವೃಂದೈ ರ್ಯೋಗಿವೃಂದೈ ರ್ಮುನಿವೃಂದೈ ರ್ನಿಷೇವಿತಂ ॥೯॥

	ಆರಾಧ್ಯಮಾನವೃಷಭಂ ಅಗ್ನೀಂದುರವಿಲೋಚನಂ~।\\
	ಪೂರಯಂತಂ ಕೃಪಾದೃಷ್ಟ್ಯಾ  ಪುಮರ್ಥಾನಾಶ್ರಿತೇ ಜನೇ~।\\
ಏವಂ ವಿಭಾವಯೇದೀಶಂ ಸರ್ವವಿದ್ಯಾ ಕಲಾನಿಧಿಂ ॥೧೦॥}
\newpage
	ದೇವದೇವೋ ಮಹಾದೇವೋ ದೇವಾನಾಮಪಿ ದೇಶಿಕಃ~।\\
	ದಕ್ಷಿಣಾಮೂರ್ತಿರೀಶಾನೋ ದಯಾಪೂರಿತ ದಿಙ್ಮುಖಃ ॥೧॥

ಕೈಲಾಸಶಿಖರೋತ್ತುಂಗ ಕಮನೀಯನಿಜಾಕೃತಿಃ~।\\
ವಟದ್ರುಮತಟೀ ದಿವ್ಯಕನಕಾಸನ ಸಂಸ್ಥಿತಃ ॥೨॥

	ಕಟೀತಟ ಪಟೀಭೂತ ಕರಿಚರ್ಮೋಜ್ಜ್ವಲಾಕೃತಿಃ~।\\
	ಪಾಟೀರಾಪಾಂಡುರಾಕಾರ ಪರಿಪೂರ್ಣಸುಧಾಧಿಪಃ~।೩॥

ಜಟಾಕೋಟೀರಘಟಿತ ಸುಧಾಕರ ಸುಧಾಪ್ಲುತಃ~।\\
ಪಶ್ಯಲ್ಲಲಾಟಸುಭಗ ಸುಂದರ ಭ್ರೂವಿಲಾಸವಾನ್ ॥೪॥

	ಕಟಾಕ್ಷಸರಣೀ ನಿರ್ಯತ್ಕರುಣಾಪೂರ್ಣ ಲೋಚನಃ~।\\
	ಕರ್ಣಾಲೋಲ ತಟಿದ್ವರ್ಣ ಕುಂಡಲೋಜ್ಜ್ವಲ ಗಂಡಭೂಃ ॥೫॥

ತಿಲಪ್ರಸೂನ ಸಂಕಾಶ ನಾಸಿಕಾಪುಟ ಭಾಸುರಃ~।\\
ಮಂದಸ್ಮಿತ ಸ್ಫುರನ್ಮುಗ್ಧ ಮಹನೀಯ ಮುಖಾಂಬುಜಃ ॥೬॥

	ಕುಂದಕುಡ್ಮಲ ಸಂಸ್ಪರ್ಧಿ ದಂತಪಂಕ್ತಿ ವಿರಾಜಿತಃ~।\\
	ಸಿಂದೂರಾರುಣ ಸುಸ್ನಿಗ್ಧ ಕೋಮಲಾಧರ ಪಲ್ಲವಃ ॥೭॥

ಶಂಖಾಟೋಪ ಗಲದ್ದಿವ್ಯ ಗಳವೈಭವಮಂಜುಲಃ~।\\
ಕರಕಂದಲಿತ ಜ್ಞಾನಮುದ್ರಾ ರುದ್ರಾಕ್ಷಮಾಲಿಕಃ ॥೮॥

	ಅನ್ಯಹಸ್ತ ತಲನ್ಯಸ್ತ ವೀಣಾ ಪುಸ್ತೋಲ್ಲಸದ್ವಪುಃ~।\\
	ವಿಶಾಲ ರುಚಿರೋರಸ್ಕ ವಲಿಮತ್ಪಲ್ಲವೋದರಃ ॥೯॥

ಬೃಹತ್ಕಟಿ ನಿತಂಬಾಢ್ಯಃ ಪೀವರೋರು ದ್ವಯಾನ್ವಿತಃ~।\\
ಜಂಘಾವಿಜಿತ ತೂಣೀರಸ್ತುಂಗಗುಲ್ಫ ಯುಗೋಜ್ಜ್ವಲಃ ॥೧೦॥

	ಮೃದು ಪಾಟಲ ಪಾದಾಬ್ಜಶ್ಚಂದ್ರಾಭ ನಖದೀಧಿತಿಃ~।\\
	ಅಪಸವ್ಯೋರು ವಿನ್ಯಸ್ತ ಸವ್ಯಪಾದ ಸರೋರುಹಃ ॥೧೧॥

ಘೋರಾಪಸ್ಮಾರ ನಿಕ್ಷಿಪ್ತ ಧೀರದಕ್ಷ ಪದಾಂಬುಜಃ~।\\
ಸನಕಾದಿ ಮುನಿಧ್ಯೇಯಃ ಸರ್ವಾಭರಣ ಭೂಷಿತಃ ॥೧೨॥

	ದಿವ್ಯಚಂದನ ಲಿಪ್ತಾಂಗಶ್ಚಾರುಹಾಸ ಪರಿಷ್ಕೃತಃ~।\\
	ಕರ್ಪೂರ ಧವಲಾಕಾರಃ ಕಂದರ್ಪಶತ ಸುಂದರಃ ॥೧೩॥

ಕಾತ್ಯಾಯನೀ ಪ್ರೇಮನಿಧಿಃ ಕರುಣಾರಸ ವಾರಿಧಿಃ~।\\
ಕಾಮಿತಾರ್ಥಪ್ರದಃ ಶ್ರೀಮತ್ಕಮಲಾ ವಲ್ಲಭಪ್ರಿಯಃ ॥೧೪॥

	ಕಟಾಕ್ಷಿತಾತ್ಮವಿಜ್ಞಾನಃ ಕೈವಲ್ಯಾನಂದಕಂದಲಃ~।\\
	ಮಂದಹಾಸ ಸಮಾನೇಂದುಃ ಛಿನ್ನಾಜ್ಞಾನ ತಮಸ್ತತಿಃ ॥೧೫॥

ಸಂಸಾರಾನಲ ಸಂತಪ್ತಜನತಾಮೃತ ಸಾಗರಃ~।\\
ಗಂಭೀರ ಹೃದಯಾಂಭೋಜ ನಭೋಮಣಿ ನಿಭಾಕೃತಿಃ ॥೧೬॥

	ನಿಶಾಕರಕರಾಕಾರ ವಶೀಕೃತಜಗತ್ತ್ರಯಃ~।\\
	ತಾಪಸಾರಾಧ್ಯ ಪಾದಾಬ್ಜಸ್ತರುಣಾನಂದ ವಿಗ್ರಹಃ ॥೧೭॥

ಭೂತಿ ಭೂಷಿತ ಸರ್ವಾಂಗೋ ಭೂತಾಧಿಪತಿರೀಶ್ವರಃ~।\\
ವದನೇಂದು ಸ್ಮಿತಜ್ಯೋತ್ಸ್ನಾನಿಲೀನ ತ್ರಿಪುರಾಕೃತಿಃ ॥೧೮॥

	ತಾಪತ್ರಯತ ಮೋಭಾನುಃ ಪಾಪಾರಣ್ಯ ದವಾನಲಃ~।\\
	ಸಂಸಾರ ಸಾಗರೋದ್ಧರ್ತಾ ಹಂಸಾಗ್ರ್ಯೋಪಾಸ್ಯ ವಿಗ್ರಹಃ ॥೧೯॥

ಲಲಾಟ ಹುತಭುಗ್ದಗ್ಧ ಮನೋಭವ ಶುಭಾಕೃತಿಃ~।\\
ತುಚ್ಛೀಕೃತ ಜಗಜ್ಜಾಲ ಸ್ತುಷಾರಕರ ಶೀತಲಃ ॥೨೦॥

	ಅಸ್ತಂಗತ ಸಮಸ್ತೇಚ್ಛೋ ನಿಸ್ತುಲಾನಂದ ಮಂಥರಃ~।\\
	ಧೀರೋದಾತ್ತ ಗುಣಾಧಾರ ಉದಾರ ವರವೈಭವಃ ॥೨೧॥

ಅಪಾರ ಕರುಣಾಮೂರ್ತಿರಜ್ಞಾನ ಧ್ವಾಂತಭಾಸ್ಕರಃ~।\\
ಭಕ್ತಮಾನಸ ಹಂಸಾಗ್ರ್ಯ ಭವಾಮಯ ಭಿಷಕ್ತಮಃ ॥೨೨॥

	ಯೋಗೀಂದ್ರಪೂಜ್ಯಪಾದಾಬ್ಜೋ ಯೋಗಪಟ್ಟೋಲ್ಲಸತ್ಕಟಿಃ~।\\
	ಶುದ್ಧಸ್ಫಟಿಕಸಂಕಾಶೋ ಬದ್ಧಪನ್ನಗಭೂಷಣಃ ॥೨೩॥

ನಾನಾಮುನಿ ಸಮಾಕೀರ್ಣೋ ನಾಸಾಗ್ರನ್ಯಸ್ತಲೋಚನಃ~।\\
ವೇದಮೂರ್ಧೈಕಸಂವೇದ್ಯೋ ನಾದಧ್ಯಾನಪರಾಯಣಃ ॥೨೪॥

	ಧರಾಧರೇಂದು ರಾನಂದಸಂದೋಹ ರಸಸಾಗರಃ~।\\
	ದ್ವೈತವೃಂದವಿಮೋಹಾಂಧ್ಯ ಪರಾಕೃತ ದೃಗದ್ಭುತಃ ॥೨೫॥

ಪ್ರತ್ಯಗಾತ್ಮಾ ಪರಂಜ್ಯೋತಿಃ ಪುರಾಣಃ ಪರಮೇಶ್ವರಃ~।\\
ಪ್ರಪಂಚೋಪಶಮಃ ಪ್ರಾಜ್ಞಃ ಪುಣ್ಯಕೀರ್ತಿಃ ಪುರಾತನಃ ॥೨೬॥

	ಸರ್ವಾಧಿಷ್ಠಾನ ಸನ್ಮಾತ್ರಸ್ಸ್ವಾತ್ಮ ಬಂಧಹರೋ ಹರಃ~।\\
	ಸರ್ವಪ್ರೇಮನಿಜಾಹಾಸಃ ಸರ್ವಾನುಗ್ರಹಕೃತ್ ಶಿವಃ ॥೨೭॥

ಸರ್ವೇಂದ್ರಿಯಗುಣಾಭಾಸಃ ಸರ್ವಭೂತಗುಣಾಶ್ರಯಃ~।\\
ಸಚ್ಚಿದಾನಂದಪೂರ್ಣಾತ್ಮಾ ಸ್ವೇ ಮಹಿಮ್ನಿ ಪ್ರತಿಷ್ಠಿತಃ ॥೨೮॥

	ಸರ್ವಭೂತಾಂತರಸ್ಸಾಕ್ಷೀ ಸರ್ವಜ್ಞಸ್ಸರ್ವಕಾಮದಃ~।\\
	ಸನಕಾದಿಮಹಾಯೋಗಿಸಮಾರಾಧಿತಪಾದುಕಃ ॥೨೯॥

ಆದಿದೇವೋ ದಯಾಸಿಂಧುಃ ಶಿಕ್ಷಿತಾಸುರವಿಗ್ರಹಃ~।\\
ಯಕ್ಷಕಿನ್ನರಗಂಧರ್ವಸ್ತೂಯಮಾನಾತ್ಮವೈಭವಃ ॥೩೦॥

	ಬ್ರಹ್ಮಾದಿದೇವವಿನುತೋ ಯೋಗಮಾಯಾನಿಯೋಜಕಃ~।\\
	ಶಿವಯೋಗೀ ಶಿವಾನಂದಃ ಶಿವಭಕ್ತಸಮುದ್ಧರಃ ॥೩೧॥

ವೇದಾಂತಸಾರಸಂದೋಹಃ ಸರ್ವಸತ್ತ್ವಾವಲಂಬನಃ~।\\
ವಟಮೂಲಾಶ್ರಯೋ ವಾಗ್ಮೀ ಮಾನ್ಯೋ ಮಲಯಜಪ್ರಿಯಃ ॥೩೨॥

	ಸುಶೀಲೋ ವಾಂಛಿತಾರ್ಥಜ್ಞಃ ಪ್ರಸನ್ನವದನೇಕ್ಷಣಃ ।\\
	ನೃತ್ತಗೀತಕಲಾಭಿಜ್ಞಃ ಕರ್ಮವಿತ್ ಕರ್ಮಮೋಚಕಃ ॥೩೩॥

ಕರ್ಮಸಾಕ್ಷೀ ಕರ್ಮಮಯಃ ಕರ್ಮಣಾಂ ಚ ಫಲಪ್ರದಃ~।\\
ಜ್ಞಾನದಾತಾ ಸದಾಚಾರಃ ಸರ್ವೋಪದ್ರವಮೋಚಕಃ ॥೩೪॥

	ಅನಾಥನಾಥೋ ಭಗವಾನಾಶ್ರಿತಾಮರಪಾದಪಃ~।\\
	ವರಪ್ರದಃ ಪ್ರಕಾಶಾತ್ಮಾ ಸರ್ವಭೂತಹಿತೇ ರತಃ ॥೩೫॥

ವ್ಯಾಘ್ರಚರ್ಮಾಸನಾಸೀನ ಆದಿಕರ್ತಾ ಮಹೇಶ್ವರಃ~।\\
ಸುವಿಕ್ರಮಃ ಸರ್ವಗತೋ ವಿಶಿಷ್ಟಜನವತ್ಸಲಃ ॥೩೬॥

	ಚಿಂತಾಶೋಕಪ್ರಶಮನೋ ಜಗದಾನಂದಕಾರಕಃ~।\\
	ರಶ್ಮಿಮಾನ್ ಭುವನೇಶಶ್ಚ ದೇವಾಸುರಸುಪೂಜಿತಃ ॥೩೭॥

ಮೃತ್ಯುಂಜಯೋ ವ್ಯೋಮಕೇಶಃ ಷಟ್ತ್ರಿಂಶತ್ತತ್ತ್ವಸಂಗ್ರಹಃ~।\\
ಅಜ್ಞಾತಸಂಭವೋ ಭಿಕ್ಷುರದ್ವಿತೀಯೋ ದಿಗಂಬರಃ ॥೩೮॥

	ಸಮಸ್ತದೇವತಾಮೂರ್ತಿಃ ಸೋಮಸೂರ್ಯಾಗ್ನಿಲೋಚನಃ~।\\
	ಸರ್ವಸಾಮ್ರಾಜ್ಯನಿಪುಣೋ ಧರ್ಮಮಾರ್ಗಪ್ರವರ್ತಕಃ ॥೩೯॥

ವಿಶ್ವಾಧಿಕಃ ಪಶುಪತಿಃ ಪಶುಪಾಶವಿಮೋಚಕಃ~।\\
ಅಷ್ಟಮೂರ್ತಿರ್ದೀಪ್ತಮೂರ್ತಿಃ ನಾಮೋಚ್ಚಾರಣಮುಕ್ತಿದಃ ॥೪೦॥

	ಸಹಸ್ರಾದಿತ್ಯಸಂಕಾಶಃ ಸದಾಷೋಡಶವಾರ್ಷಿಕಃ~।\\
	ದಿವ್ಯಕೇಲೀಸಮಾಯುಕ್ತೋ ದಿವ್ಯಮಾಲ್ಯಾಂಬರಾವೃತಃ ॥೪೧॥

ಅನರ್ಘರತ್ನಸಂಪೂರ್ಣೋ ಮಲ್ಲಿಕಾಕುಸುಮಪ್ರಿಯಃ~।\\
ತಪ್ತಚಾಮೀಕರಾಕಾರೋ ಜಿತದಾವಾನಲಾಕೃತಿಃ ॥೪೨॥

	ನಿರಂಜನೋ ನಿರ್ವಿಕಾರೋ ನಿಜಾವಾಸೋ ನಿರಾಕೃತಿಃ~।\\
	ಜಗದ್ಗುರುರ್ಜಗತ್ಕರ್ತಾ ಜಗದೀಶೋ ಜಗತ್ಪತಿಃ ॥೪೩॥

ಕಾಮಹಂತಾ ಕಾಮಮೂರ್ತಿಃ ಕಲ್ಯಾಣವೃಷವಾಹನಃ~।\\
ಗಂಗಾಧರೋ ಮಹಾದೇವೋ ದೀನಬಂಧವಿಮೋಚಕಃ ॥೪೪॥

	ಧೂರ್ಜಟಿಃ ಖಂಡಪರಶುಃ ಸದ್ಗುಣೋ ಗಿರಿಜಾಸಖಃ~।\\
	ಅವ್ಯಯೋ ಭೂತಸೇನೇಶಃ ಪಾಪಘ್ನಃ ಪುಣ್ಯದಾಯಕಃ ॥೪೫॥

ಉಪದೇಷ್ಟಾ ದೃಢಪ್ರಜ್ಞೋ ರುದ್ರೋ ರೋಗವಿನಾಶನಃ~।\\
ನಿತ್ಯಾನಂದೋ ನಿರಾಧಾರೋ ಹರೋ ದೇವಶಿಖಾಮಣಿಃ ॥೪೬॥

	ಪ್ರಣತಾರ್ತಿಹರಃ ಸೋಮಃ ಸಾಂದ್ರಾನಂದೋ ಮಹಾಮತಿಃ~।\\
	ಆಶ್ಚರ್ಯವೈಭವೋ ದೇವಃ ಸಂಸಾರಾರ್ಣವತಾರಕಃ ॥೪೭॥

ಯಜ್ಞೇಶೋ ರಾಜರಾಜೇಶೋ ಭಸ್ಮರುದ್ರಾಕ್ಷಲಾಂಛನಃ~।\\
ಅನಂತಸ್ತಾರಕಃ ಸ್ಥಾಣುಃ ಸರ್ವವಿದ್ಯೇಶ್ವರೋ ಹರಿಃ ॥೪೮॥

	ವಿಶ್ವರೂಪೋ ವಿರೂಪಾಕ್ಷಃ ಪ್ರಭುಃ ಪರಿವೃಢೋ ದೃಢಃ~।\\
	ಭವ್ಯೋ ಜಿತಾರಿಷಡ್ವರ್ಗೋ ಮಹೋದಾರೋ ವಿಷಾಶನಃ ॥೪೯॥

ಸುಕೀರ್ತಿರಾದಿಪುರುಷೋ ಜರಾಮರಣವರ್ಜಿತಃ~।\\
ಪ್ರಮಾಣಭೂತೋ ದುರ್ಜ್ಞೇಯಃ ಪುಣ್ಯಃ ಪರಪುರಂಜಯಃ ॥೫೦॥

	ಗುಣಾಕಾರೋ ಗುಣಶ್ರೇಷ್ಠಃ ಸಚ್ಚಿದಾನಂದವಿಗ್ರಹಃ~।\\
	ಸುಖದಃ ಕಾರಣಂ ಕರ್ತಾ ಭವಬಂಧವಿಮೋಚಕಃ ॥೫೧॥

ಅನಿರ್ವಿಣ್ಣೋ ಗುಣಗ್ರಾಹೀ ನಿಷ್ಕಲಂಕಃ ಕಲಂಕಹಾ~।\\
ಪುರುಷಃ ಶಾಶ್ವತೋ ಯೋಗೀ ವ್ಯಕ್ತಾವ್ಯಕ್ತಃ ಸನಾತನಃ ॥೫೨॥

	ಚರಾಚರಾತ್ಮಾ ಸೂಕ್ಷ್ಮಾತ್ಮಾ ವಿಶ್ವಕರ್ಮಾ ತಮೋಽಪಹೃತ್~।\\
	ಭುಜಂಗಭೂಷಣೋ ಭರ್ಗಸ್ತರುಣಃ ಕರುಣಾಲಯಃ ॥೫೩॥

ಅಣಿಮಾದಿಗುಣೋಪೇತೋ ಲೋಕವಶ್ಯವಿಧಾಯಕಃ~।\\
ಯೋಗಪಟ್ಟಧರೋ ಮುಕ್ತೋ ಮುಕ್ತಾನಾಂ ಪರಮಾ ಗತಿಃ ॥೫೪॥

	ಗುರುರೂಪಧರಃ ಶ್ರೀಮತ್ಪರಮಾನಂದಸಾಗರಃ~।\\
	ಸಹಸ್ರಬಾಹುಃ ಸರ್ವೇಶಃ ಸಹಸ್ರಾವಯವಾನ್ವಿತಃ ॥೫೫॥

ಸಹಸ್ರಮೂರ್ಧಾ ಸರ್ವಾತ್ಮಾ ಸಹಸ್ರಾಕ್ಷಃ ಸಹಸ್ರಪಾತ್~।\\
ನಿರಾಭಾಸಃ ಸೂಕ್ಷ್ಮತನುರ್ಹೃದಿ ಜ್ಞಾತಃ ಪರಾತ್ಪರಃ ॥೫೬॥

	ಸರ್ವಾತ್ಮಗಃ ಸರ್ವಸಾಕ್ಷೀ ನಿಃಸಂಗೋ ನಿರುಪದ್ರವಃ~।\\
	ನಿಷ್ಕಲಃ ಸಕಲಾಧ್ಯಕ್ಷಶ್ಚಿನ್ಮಯಸ್ತಮಸಃ ಪರಃ ॥೫೭॥

ಜ್ಞಾನವೈರಾಗ್ಯಸಂಪನ್ನೋ ಯೋಗಾನಂದಮಯಃ ಶಿವಃ~।\\
ಶಾಶ್ವತೈಶ್ವರ್ಯಸಂಪೂರ್ಣೋ ಮಹಾಯೋಗೀಶ್ವರೇಶ್ವರಃ ॥೫೮॥

	ಸಹಸ್ರಶಕ್ತಿಸಂಯುಕ್ತಃ ಪುಣ್ಯಕಾಯೋ ದುರಾಸದಃ~।\\
	ತಾರಕಬ್ರಹ್ಮಸಂಪೂರ್ಣಸ್ತಪಸ್ವಿಜನಸಂವೃತಃ ॥೫೯॥

ವಿಧೀಂದ್ರಾಮರಸಂಪೂಜ್ಯೋ ಜ್ಯೋತಿಷಾಂ ಜ್ಯೋತಿರುತ್ತಮಃ~।\\
ನಿರಕ್ಷರೋ ನಿರಾಲಂಬಃ ಸ್ವಾತ್ಮಾರಾಮೋ ವಿಕರ್ತನಃ ॥೬೦॥

	ನಿರವದ್ಯೋ ನಿರಾತಂಕೋ ಭೀಮೋ ಭೀಮಪರಾಕ್ರಮಃ~।\\
	ವೀರಭದ್ರಃ ಪುರಾರಾತಿರ್ಜಲಂಧರಶಿರೋಹರಃ ॥೬೧॥

ಅಂಧಕಾಸುರಸಂಹರ್ತಾ ಭಗನೇತ್ರಭಿದದ್ಭುತಃ~।\\
ವಿಶ್ವಗ್ರಾಸೋಽಧರ್ಮಶತ್ರುರ್ಬ್ರಹ್ಮಜ್ಞಾನೈಕಮಂಥರಃ ॥೬೨॥

	ಅಗ್ರೇಸರಸ್ತೀರ್ಥಭೂತಃ ಸಿತಭಸ್ಮಾವಕುಂಠನಃ~।\\
	ಅಕುಂಠಮೇಧಾಃ ಶ್ರೀಕಂಠೋ ವೈಕುಂಠಪರಮಪ್ರಿಯಃ ॥೬೩॥

ಲಲಾಟೋಜ್ಜ್ವಲನೇತ್ರಾಬ್ಜಸ್ತುಷಾರಕರಶೇಖರಃ~।\\
ಗಜಾಸುರಶಿರಶ್ಛೇತ್ತಾ ಗಂಗೋದ್ಭಾಸಿತಮೂರ್ಧಜಃ ॥೬೪॥

	ಕಲ್ಯಾಣಾಚಲಕೋದಂಡಃ ಕಮಲಾಪತಿಸಾಯಕಃ~।\\
	ವಾರಾಂ ಶೇವಧಿ ತೂಣೀರಃ ಸರೋಜಾಸನ ಸಾರಥಿಃ ॥೬೫॥

ತ್ರಯೀತುರಂಗಸಂಕ್ರಾಂತೋ ವಾಸುಕಿಜ್ಯಾವಿರಾಜಿತಃ~।\\
ರವೀಂದುಚರಣಾಚಾರಿಧರಾರಥವಿರಾಜಿತಃ ॥೬೬॥

	ತ್ರಯ್ಯಂತಪ್ರಗ್ರಹೋದಾರಚಾರುಘಂಟಾರವೋಜ್ಜ್ವಲಃ~।\\
	ಉತ್ತಾನಪರ್ವಲೋಮಾಢ್ಯೋ ಲೀಲಾವಿಜಿತಮನ್ಮಥಃ ॥೬೭॥

ಜಾತುಪ್ರಪನ್ನಜನತಾ ಜೀವನೋಪಾಯನೋತ್ಸುಕಃ~।\\
ಸಂಸಾರಾರ್ಣವನಿರ್ಮಗ್ನ ಸಮುದ್ಧರಣಪಂಡಿತಃ ॥೬೮॥

	ಮದದ್ವಿರದಧಿಕ್ಕಾರಿಗತಿಮಂಜುಲವೈಭವಃ~।\\
	ಮತ್ತಕೋಕಿಲಮಾಧುರ್ಯರಸನಿರ್ಭರಗೀರ್ಗಣಃ ॥೬೯॥

ಕೈವಲ್ಯೋದಧಿಕಲ್ಲೋಲಲೀಲಾತಾಂಡವಪಂಡಿತಃ~।\\
ವಿಷ್ಣುರ್ಜಿಷ್ಣುರ್ವಾಸುದೇವಃ ಪ್ರಭವಿಷ್ಣುಃ ಪುರಾತನಃ ॥೭೦॥

	ವರ್ಧಿಷ್ಣುರ್ವರದೋ ವೈದ್ಯೋ ಹರಿರ್ನಾರಾಯಣೋಽಚ್ಯುತಃ~।\\
	ಅಜ್ಞಾನವನದಾವಾಗ್ನಿಃ ಪ್ರಜ್ಞಾಪ್ರಾಸಾದಭೂಪತಿಃ ॥೭೧॥

ಸರ್ಪಭೂಷಿತಸರ್ವಾಂಗಃ ಕರ್ಪೂರೋಜ್ಜ್ವಲಿತಾಕೃತಿಃ~।\\
ಅನಾದಿಮಧ್ಯನಿಧನೋ ಗಿರೀಶೋ ಗಿರಿಜಾಪತಿಃ ॥೭೨॥

	ವೀತರಾಗೋ ವಿನೀತಾತ್ಮಾ ತಪಸ್ವೀ ಭೂತಭಾವನಃ~।\\
	ದೇವಾಸುರಗುರುಧ್ಯೇಯೋ ದೇವಾಸುರನಮಸ್ಕೃತಃ ॥೭೩॥

ದೇವಾದಿದೇವೋ ದೇವರ್ಷಿರ್ದೇವಾಸುರವರಪ್ರದಃ~।\\
ಸರ್ವದೇವಮಯೋಽಚಿಂತ್ಯೋ ದೇವಾತ್ಮಾ ಚಾತ್ಮಸಂಭವಃ ॥೭೪॥

	ನಿರ್ಲೇಪೋ ನಿಷ್ಪ್ರಪಂಚಾತ್ಮಾ ನಿರ್ವಿಘ್ನೋ ವಿಘ್ನನಾಶಕಃ~।\\
	ಏಕಜ್ಯೋತಿರ್ನಿರಾತಂಕೋ ವ್ಯಾಪ್ತಮೂರ್ತಿರನಾಕುಲಃ ॥೭೫॥

ನಿರವದ್ಯಪದೋಪಾಧಿರ್ವಿದ್ಯಾರಾಶಿರನುತ್ತಮಃ~।\\
ನಿತ್ಯಾನಂದಃ ಸುರಾಧ್ಯಕ್ಷೋ ನಿಃಸಂಕಲ್ಪೋ ನಿರಂಜನಃ ॥೭೬॥

	ನಿಷ್ಕಲಂಕೋ ನಿರಾಕಾರೋ ನಿಷ್ಪ್ರಪಂಚೋ ನಿರಾಮಯಃ~।\\
	ವಿದ್ಯಾಧರೋ ವಿಯತ್ಕೇಶೋ ಮಾರ್ಕಂಡೇಯವರಪ್ರದಃ ॥೭೭॥

ಭೈರವೋ ಭೈರವೀನಾಥಃ ಕಾಮದಃ ಕಮಲಾಸನಃ~।\\
ವೇದವೇದ್ಯಃ ಸುರಾನಂದೋ ಲಸಜ್ಜ್ಯೋತಿಃ ಪ್ರಭಾಕರಃ ॥೭೮॥

	ಚೂಡಾಮಣಿಃ ಸುರಾಧೀಶೋ ಯಜ್ಞಗೇಯೋ ಹರಿಪ್ರಿಯಃ~।\\
	ನಿರ್ಲೇಪೋ ನೀತಿಮಾನ್ ಸೂತ್ರೀ ಶ್ರೀಹಾಲಾಹಲಸುಂದರಃ ॥೭೯॥

ಧರ್ಮದಕ್ಷೋ ಮಹಾರಾಜಃ ಕಿರೀಟೀ ವಂದಿತೋ ಗುಹಃ~।\\
ಮಾಧವೋ ಯಾಮಿನೀನಾಥಃ ಶಂಬರಃ ಶಬರೀಪ್ರಿಯಃ ॥೮೦॥

	ಸಂಗೀತವೇತ್ತಾ ಲೋಕಜ್ಞಃ ಶಾಂತಃ ಕಲಶಸಂಭವಃ~।\\
	ಬ್ರಹ್ಮಣ್ಯೋ ವರದೋ ನಿತ್ಯಃ ಶೂಲೀ ಗುರುವರೋ ಹರಃ ॥೮೧॥

ಮಾರ್ತಾಂಡಃ ಪುಂಡರೀಕಾಕ್ಷೋ ಲೋಕನಾಯಕವಿಕ್ರಮಃ~।\\
ಮುಕುಂದಾರ್ಚ್ಯೋ ವೈದ್ಯನಾಥಃ ಪುರಂದರವರಪ್ರದಃ ॥೮೨॥

	ಭಾಷಾವಿಹೀನೋ ಭಾಷಾಜ್ಞೋ ವಿಘ್ನೇಶೋ ವಿಘ್ನನಾಶನಃ~।\\
	ಕಿನ್ನರೇಶೋ ಬೃಹದ್ಭಾನುಃ ಶ್ರೀನಿವಾಸಃ ಕಪಾಲಭೃತ್ ॥೮೩॥

ವಿಜಯೋ ಭೂತಭಾವಜ್ಞೋ ಭೀಮಸೇನೋ ದಿವಾಕರಃ~।\\
ಬಿಲ್ವಪ್ರಿಯೋ ವಸಿಷ್ಠೇಶಃ ಸರ್ವಮಾರ್ಗಪ್ರವರ್ತಕಃ ॥೮೪॥

	ಓಷಧೀಶೋ ವಾಮದೇವೋ ಗೋವಿಂದೋ ನೀಲಲೋಹಿತಃ~।\\
	ಷಡರ್ಧನಯನಃ ಶ್ರೀಮನ್ಮಹಾದೇವೋ ವೃಷಧ್ವಜಃ ॥೮೫॥

ಕರ್ಪೂರದೀಪಿಕಾಲೋಲಃ ಕರ್ಪೂರರಸಚರ್ಚಿತಃ~।\\
ಅವ್ಯಾಜಕರುಣಾಮೂರ್ತಿಸ್ತ್ಯಾಗರಾಜಃ ಕ್ಷಪಾಕರಃ ॥೮೬॥

	ಆಶ್ಚರ್ಯವಿಗ್ರಹಃ ಸೂಕ್ಷ್ಮಃ ಸಿದ್ಧೇಶಃ ಸ್ವರ್ಣಭೈರವಃ~।\\
	ದೇವರಾಜಃ ಕೃಪಾಸಿಂಧುರದ್ವಯೋಽಮಿತವಿಕ್ರಮಃ ॥೮೭॥

ನಿರ್ಭೇದೋ ನಿತ್ಯಸತ್ವಸ್ಥೋ ನಿರ್ಯೋಗಕ್ಷೇಮ ಆತ್ಮವಾನ್~।\\
ನಿರಪಾಯೋ ನಿರಾಸಂಗೋ ನಿಃಶಬ್ದೋ ನಿರುಪಾಧಿಕಃ ॥೮೮॥

	ಭವಃ ಸರ್ವೇಶ್ವರಃ ಸ್ವಾಮೀ ಭವಭೀತಿವಿಭಂಜನಃ~।\\
	ದಾರಿದ್ರ್ಯತೃಣಕೂಟಾಗ್ನಿರ್ದಾರಿತಾಸುರಸಂತತಿಃ ॥೮೯॥

ಮುಕ್ತಿದೋ ಮುದಿತೋಽಕುಬ್ಜೋ ಧಾರ್ಮಿಕೋ ಭಕ್ತವತ್ಸಲಃ~।\\
ಅಭ್ಯಾಸಾತಿಶಯಜ್ಞೇಯಶ್ಚಂದ್ರಮೌಲಿಃ ಕಲಾಧರಃ ॥೯೦॥

	ಮಹಾಬಲೋ ಮಹಾವೀರ್ಯೋ ವಿಭುಃ ಶ್ರೀಶಃ ಶುಭಪ್ರದಃ~।\\
	ಸಿದ್ಧಃ ಪುರಾಣಪುರುಷೋ ರಣಮಂಡಲಭೈರವಃ ॥೯೧॥

ಸದ್ಯೋಜಾತೋ ವಟಾರಣ್ಯವಾಸೀ ಪುರುಷವಲ್ಲಭಃ~।\\
ಹರಿಕೇಶೋ ಮಹಾತ್ರಾತಾ ನೀಲಗ್ರೀವಸ್ಸುಮಂಗಲಃ ॥೯೨॥

	ಹಿರಣ್ಯಬಾಹುಸ್ತೀಕ್ಷ್ಣಾಂಶುಃ ಕಾಮೇಶಃ ಸೋಮವಿಗ್ರಹಃ~।\\
	ಸರ್ವಾತ್ಮಾ ಸರ್ವಕರ್ತಾ ಚ ತಾಂಡವೋ ಮುಂಡಮಾಲಿಕಃ ॥೯೩॥

ಅಗ್ರಗಣ್ಯಃ ಸುಗಂಭೀರೋ ದೇಶಿಕೋ ವೈದಿಕೋತ್ತಮಃ~।\\
ಪ್ರಸನ್ನದೇವೋ ವಾಗೀಶಶ್ಚಿಂತಾತಿಮಿರಭಾಸ್ಕರಃ ॥೯೪॥

	ಗೌರೀಪತಿಸ್ತುಂಗಮೌಲಿರ್ಮಖರಾಜೋ ಮಹಾಕವಿಃ~।\\
	ಶ್ರೀಧರಸ್ಸರ್ವಸಿದ್ಧೇಶೋ ವಿಶ್ವನಾಥೋ ದಯಾನಿಧಿಃ ॥೯೫॥

ಅಂತರ್ಮುಖೋ ಬಹಿರ್ದೃಷ್ಟಿಃ ಸಿದ್ಧವೇಷಮನೋಹರಃ~।\\
ಕೃತ್ತಿವಾಸಾಃ ಕೃಪಾಸಿಂಧುರ್ಮಂತ್ರಸಿದ್ಧೋ ಮತಿಪ್ರದಃ ॥೯೬॥

	ಮಹೋತ್ಕೃಷ್ಟಃ ಪುಣ್ಯಕರೋ ಜಗತ್ಸಾಕ್ಷೀ ಸದಾಶಿವಃ~।\\
	ಮಹಾಕ್ರತುರ್ಮಹಾಯಜ್ವಾ ವಿಶ್ವಕರ್ಮಾ ತಪೋನಿಧಿಃ ॥೯೭॥

ಛಂದೋಮಯೋ ಮಹಾಜ್ಞಾನೀ ಸರ್ವಜ್ಞೋ ದೇವವಂದಿತಃ~।\\
ಸಾರ್ವಭೌಮಸ್ಸದಾನಂದಃ ಕರುಣಾಮೃತವಾರಿಧಿಃ ॥೯೮॥

	ಕಾಲಕಾಲಃ ಕಲಿಧ್ವಂಸೀ ಜರಾಮರಣನಾಶಕಃ~।\\
	ಶಿತಿಕಂಠಶ್ಚಿದಾನಂದೋ ಯೋಗಿನೀಗಣಸೇವಿತಃ ॥೯೯॥

ಚಂಡೀಶಃ ಶುಕಸಂವೇದ್ಯಃ ಪುಣ್ಯಶ್ಲೋಕೋ ದಿವಸ್ಪತಿಃ~।\\
ಸ್ಥಾಯೀ ಸಕಲತತ್ತ್ವಾತ್ಮಾ ಸದಾಸೇವಕವರ್ಧನಃ ॥೧೦೦॥

	ರೋಹಿತಾಶ್ವಃ ಕ್ಷಮಾರೂಪೀ ತಪ್ತಚಾಮೀಕರಪ್ರಭಃ~।\\
	ತ್ರಿಯಂಬಕೋ ವರರುಚಿರ್ದೇವದೇವಶ್ಚತುರ್ಭುಜಃ ॥೧೦೧।

ವಿಶ್ವಂಭರೋ ವಿಚಿತ್ರಾಂಗೋ ವಿಧಾತಾ ಪುರಶಾಸನಃ~।\\
ಸುಬ್ರಹ್ಮಣ್ಯೋ ಜಗತ್ಸ್ವಾಮೀ ರೋಹಿತಾಕ್ಷಃ ಶಿವೋತ್ತಮಃ ॥೧೦೨॥

	ನಕ್ಷತ್ರಮಾಲಾಭರಣೋ ಮಘವಾನ್ ಅಘನಾಶನಃ~।\\
	ವಿಧಿಕರ್ತಾ ವಿಧಾನಜ್ಞಃ ಪ್ರಧಾನಪುರುಷೇಶ್ವರಃ ॥೧೦೩॥

ಚಿಂತಾಮಣಿಃ ಸುರಗುರುರ್ಧ್ಯೇಯೋ ನೀರಾಜನಪ್ರಿಯಃ~।\\
ಗೋವಿಂದೋ ರಾಜರಾಜೇಶೋ ಬಹುಪುಷ್ಪಾರ್ಚನಪ್ರಿಯಃ ॥೧೦೪॥।

ಸರ್ವಾನಂದೋ ದಯಾರೂಪೀ ಶೈಲಜಾಸುಮನೋಹರಃ~।\\
ಸುವಿಕ್ರಮಃ ಸರ್ವಗತೋ ಹೇತುಸಾಧನವರ್ಜಿತಃ ॥೧೦೫॥

	ವೃಷಾಂಕೋ ರಮಣೀಯಾಂಗಃ ಸದಂಘ್ರಿಃ ಸಾಮಪಾರಗಃ~।\\
	ಮಂತ್ರಾತ್ಮಾ ಕೋಟಿಕಂದರ್ಪಸೌಂದರ್ಯರಸವಾರಿಧಿಃ ॥೧೦೬ ॥

ಯಜ್ಞೇಶೋ ಯಜ್ಞಪುರುಷಃ ಸೃಷ್ಟಿಸ್ಥಿತ್ಯಂತಕಾರಣಂ~।\\
ಪರಹಂಸೈಕಜಿಜ್ಞಾಸ್ಯಃ ಸ್ವಪ್ರಕಾಶಸ್ವರೂಪವಾನ್ ॥೧೦೭॥

	ಮುನಿಮೃಗ್ಯೋ ದೇವಮೃಗ್ಯೋ ಮೃಗಹಸ್ತೋ ಮೃಗೇಶ್ವರಃ~।\\
	ಮೃಗೇಂದ್ರಚರ್ಮವಸನೋ ನರಸಿಂಹನಿಪಾತನಃ ॥೧೦೮॥

ಮುನಿವಂದ್ಯೋ ಮುನಿಶ್ರೇಷ್ಠೋ ಮುನಿಬೃಂದನಿಷೇವಿತಃ~।\\
ದುಷ್ಟಮೃತ್ಯುರದುಷ್ಟೇಹೋ ಮೃತ್ಯುಹಾ ಮೃತ್ಯುಪೂಜಿತಃ ॥೧೦೯॥

	ಅವ್ಯಕ್ತೋಽಮ್ಬುಜಜನ್ಮಾದಿಕೋಟಿಕೋಟಿಸುಪೂಜಿತಃ~।\\
	ಲಿಂಗಮೂರ್ತಿರಲಿಂಗಾತ್ಮಾ ಲಿಂಗಾತ್ಮಾ ಲಿಂಗವಿಗ್ರಹಃ ॥೧೧೦॥

ಯಜುರ್ಮೂರ್ತಿಃ ಸಾಮಮೂರ್ತಿರೃಙ್ಮೂರ್ತಿರ್ಮೂರ್ತಿವರ್ಜಿತಃ~।\\
ವಿಶ್ವೇಶೋ ಗಜಚರ್ಮೈಕಚೇಲಾಂಚಿತಕಟೀತಟಃ ॥೧೧೧॥

	ಪಾವನಾಂತೇವಸದ್ಯೋಗಿಜನಸಾರ್ಥಸುಧಾಕರಃ~।\\
	ಅನಂತಸೋಮಸೂರ್ಯಾಗ್ನಿಮಂಡಲಪ್ರತಿಮಪ್ರಭಃ ॥೧೧೨॥

ಚಿಂತಾಶೋಕಪ್ರಶಮನಃ ಸರ್ವವಿದ್ಯಾವಿಶಾರದಃ~।\\
ಭಕ್ತವಿಜ್ಞಪ್ತಿಸಂಧಾತಾ ಕರ್ತಾ ಗಿರಿವರಾಕೃತಿಃ ॥೧೧೩॥

	ಜ್ಞಾನಪ್ರದೋ ಮನೋವಾಸಃ ಕ್ಷೇಮ್ಯೋ ಮೋಹವಿನಾಶನಃ~।\\
	ಸುರೋತ್ತಮಶ್ಚಿತ್ರಭಾನುಃ ಸದಾವೈಭವತತ್ಪರಃ ॥೧೧೪॥

ಸುಹೃದಗ್ರೇಸರಃ ಸಿದ್ಧಜ್ಞಾನಮುದ್ರೋ ಗಣಾಧಿಪಃ~।\\
ಆಗಮಶ್ಚರ್ಮವಸನೋ ವಾಂಛಿತಾರ್ಥಫಲಪ್ರದಃ ॥೧೧೫॥

	ಅಂತರ್ಹಿತೋಽಸಮಾನಶ್ಚ ದೇವಸಿಂಹಾಸನಾಧಿಪಃ~।\\
	ವಿವಾದಹಂತಾ ಸರ್ವಾತ್ಮಾ ಕಾಲಃ ಕಾಲವಿವರ್ಜಿತಃ ॥೧೧೬॥

ವಿಶ್ವಾತೀತೋ ವಿಶ್ವಕರ್ತಾ ವಿಶ್ವೇಶೋ ವಿಶ್ವಕಾರಣಂ~।\\
ಯೋಗಿಧ್ಯೇಯೋ ಯೋಗನಿಷ್ಠೋ ಯೋಗಾತ್ಮಾ ಯೋಗವಿತ್ತಮಃ ॥೧೧೭॥

	ಓಂಕಾರರೂಪೋ ಭಗವಾನ್ ಬಿಂದುನಾದಮಯಃ ಶಿವಃ~।\\
	ಚತುರ್ಮುಖಾದಿಸಂಸ್ತುತ್ಯಶ್ಚತುರ್ವರ್ಗಫಲಪ್ರದಃ ॥೧೧೮॥

ಸಹ್ಯಾಚಲಗುಹಾವಾಸೀ ಸಾಕ್ಷಾನ್ಮೋಕ್ಷರಸಾಮೃತಃ~।\\
ದಕ್ಷಾಧ್ವರಸಮುಚ್ಛೇತ್ತಾ ಪಕ್ಷಪಾತವಿವರ್ಜಿತಃ ॥೧೧೯॥

	ಓಂಕಾರವಾಚಕಃ ಶಂಭುಃ ಶಂಕರಃ ಶಶಿಶೀತಲಃ~।\\
	ಪಂಕಜಾಸನಸಂಸೇವ್ಯಃ ಕಿಂಕರಾಮರವತ್ಸಲಃ ॥೧೨೦॥

ನತದೌರ್ಭಾಗ್ಯತೂಲಾಗ್ನಿಃ ಕೃತಕೌತುಕಮಂಗಲಃ~।\\
ತ್ರಿಲೋಕಮೋಹನಃ ಶ್ರೀಮತ್ತ್ರಿಪುಂಡ್ರಾಂಕಿತಮಸ್ತಕಃ ॥೧೨೧॥

	ಕ್ರೌಂಚಾರಿಜನಕಃ ಶ್ರೀಮದ್ಗಣನಾಥಸುತಾನ್ವಿತಃ~।\\
	ಅದ್ಭುತಾನಂತವರದೋಽಪರಿಚ್ಛಿನ್ನಾತ್ಮವೈಭವಃ ॥೧೨೨॥

ಇಷ್ಟಾಪೂರ್ತಪ್ರಿಯಃ ಶರ್ವ ಏಕವೀರಃ ಪ್ರಿಯಂವದಃ~।\\
ಊಹಾಪೋಹವಿನಿರ್ಮುಕ್ತ ಓಂಕಾರೇಶ್ವರಪೂಜಿತಃ ॥೧೨೩॥

	ರುದ್ರಾಕ್ಷವಕ್ಷಾ ರುದ್ರಾಕ್ಷರೂಪೋ ರುದ್ರಾಕ್ಷಪಕ್ಷಕಃ~।\\
	ಭುಜಗೇಂದ್ರಲಸತ್ಕಂಠೋ ಭುಜಂಗಾಭರಣಪ್ರಿಯಃ ॥೧೨೪॥

ಕಲ್ಯಾಣರೂಪಃ ಕಲ್ಯಾಣಃ ಕಲ್ಯಾಣಗುಣಸಂಶ್ರಯಃ~।\\
ಸುಂದರಭ್ರೂಃ ಸುನಯನಃ ಸುಲಲಾಟಃ ಸುಕಂಧರಃ ॥೧೨೫॥

	ವಿದ್ವಜ್ಜನಾಶ್ರಯೋ ವಿದ್ವಜ್ಜನಸ್ತವ್ಯಪರಾಕ್ರಮಃ~।\\
	ವಿನೀತವತ್ಸಲೋ ನೀತಿಸ್ವರೂಪೋ ನೀತಿಸಂಶ್ರಯಃ ॥೧೨೬॥

ಅತಿರಾಗೀ ವೀತರಾಗೀ ರಾಗಹೇತುರ್ವಿರಾಗವಿತ್~।\\
ರಾಗಹಾ ರಾಗಶಮನೋ ರಾಗದೋ ರಾಗಿರಾಗವಿತ್ ॥೧೨೭॥

	ಮನೋನ್ಮನೋ ಮನೋರೂಪೋ ಬಲಪ್ರಮಥನೋ ಬಲಃ~।\\
	ವಿದ್ಯಾಕರೋ ಮಹಾವಿದ್ಯೋ ವಿದ್ಯಾವಿದ್ಯಾವಿಶಾರದಃ ॥೧೨೮॥

ವಸಂತಕೃದ್ವಸಂತಾತ್ಮಾ ವಸಂತೇಶೋ ವಸಂತದಃ~।\\
ಪ್ರಾವೃಟ್ಕೃತ್ ಪ್ರಾವೃಡಾಕಾರಃ ಪ್ರಾವೃಟ್ಕಾಲಪ್ರವರ್ತಕಃ ॥೧೨೯॥

	ಶರನ್ನಾಥೋ ಶರತ್ಕಾಲನಾಶಕಃ ಶರದಾಶ್ರಯಃ~।\\
	ಕುಂದಮಂದಾರಪುಷ್ಪೌಘಲಸದ್ವಾಯುನಿಷೇವಿತಃ ॥೧೩೦॥

ದಿವ್ಯದೇಹಪ್ರಭಾಕೂಟಸಂದೀಪಿತದಿಗಂತರಃ~।\\
ದೇವಾಸುರಗುರುಸ್ತವ್ಯೋ ದೇವಾಸುರನಮಸ್ಕೃತಃ ॥೧೩೧॥

	ವಾಮಾಂಗಭಾಗವಿಲಸಚ್ಛ್ಯಾಮಲಾವೀಕ್ಷಣಪ್ರಿಯಃ~।\\
	ಕೀರ್ತ್ಯಾಧಾರಃ ಕೀರ್ತಿಕರಃ ಕೀರ್ತಿಹೇತುರಹೇತುಕಃ ॥೧೩೨॥

ಶರಣಾಗತದೀನಾರ್ತಪರಿತ್ರಾಣಪರಾಯಣಃ~।\\
ಮಹಾಪ್ರೇತಾಸನಾಸೀನೋ ಜಿತಸರ್ವಪಿತಾಮಹಃ ॥೧೩೩॥

	ಮುಕ್ತಾದಾಮಪರೀತಾಂಗೋ ನಾನಾಗಾನವಿಶಾರದಃ~।\\
	ವಿಷ್ಣುಬ್ರಹ್ಮಾದಿವಂದ್ಯಾಂಘ್ರಿರ್ನಾನಾದೇಶೈಕನಾಯಕಃ ॥೧೩೪॥

ಧೀರೋದಾತ್ತೋ ಮಹಾಧೀರೋ ಧೈರ್ಯದೋ ಧೈರ್ಯವರ್ಧಕಃ~।\\
ವಿಜ್ಞಾನಮಯ ಆನಂದಮಯಃ ಪ್ರಾಣಮಯೋಽನ್ನದಃ ॥೧೩೫॥

	ಭವಾಬ್ಧಿತರಣೋಪಾಯಃ ಕವಿರ್ದುಃಸ್ವಪ್ನನಾಶನಃ~।\\
	ಗೌರೀವಿಲಾಸಸದನಃ ಪಿಶಾಚಾನುಚರಾವೃತಃ ॥೧೩೬॥

ದಕ್ಷಿಣಾಪ್ರೇಮಸಂತುಷ್ಟೋ ದಾರಿದ್ರ್ಯವಡವಾನಲಃ~।\\
ಅದ್ಭುತಾನಂತ ಸಂಗ್ರಾಮೋ ಢಕ್ಕಾವಾದನ ತತ್ಪರಃ ॥೧೩೭॥

	ಪ್ರಾಚ್ಯಾತ್ಮಾ ದಕ್ಷಿಣಾಕಾರಃ ಪ್ರತೀಚ್ಯಾತ್ಮೋತ್ತರಾಕೃತಿಃ~।\\
	ಊರ್ಧ್ವಾದ್ಯನ್ಯದಿಗಾಕಾರೋ ಮರ್ಮಜ್ಞಃ ಸರ್ವಶಿಕ್ಷಕಃ ॥೧೩೮॥

ಯುಗಾವಹೋ ಯುಗಾಧೀಶೋ ಯುಗಾತ್ಮಾ ಯುಗನಾಯಕಃ~।\\
ಜಂಗಮಃ ಸ್ಥಾವರಾಕಾರಃ ಕೈಲಾಸಶಿಖರಪ್ರಿಯಃ ॥೧೩೯॥

	ಹಸ್ತರಾಜತ್ಪುಂಡರೀಕಃ ಪುಂಡರೀಕನಿಭೇಕ್ಷಣಃ~।\\
	ಲೀಲಾವಿಡಂಬಿತವಪುರ್ಭಕ್ತಮಾನಸಮಂಡಿತಃ ॥೧೪೦॥

ಬೃಂದಾರಕಪ್ರಿಯತಮೋ ಬೃಂದಾರಕವರಾರ್ಚಿತಃ~।\\
ನಾನಾವಿಧಾನೇಕರತ್ನಲಸತ್ಕುಂಡಲಮಂಡಿತಃ ॥೧೪೧॥

	ನಿಃಸೀಮಮಹಿಮಾ ನಿತ್ಯಲೀಲಾವಿಗ್ರಹರೂಪಧೃತ್~।\\
	ಚಂದನದ್ರವದಿಗ್ಧಾಂಗಶ್ಚಾಂಪೇಯಕುಸುಮಾರ್ಚಿತಃ ॥೧೪೨॥

ಸಮಸ್ತಭಕ್ತಸುಖದಃ ಪರಮಾಣುರ್ಮಹಾಹ್ರದಃ~।\\
ಅಲೌಕಿಕೋ ದುಷ್ಪ್ರಧರ್ಷಃ ಕಪಿಲಃ ಕಾಲಕಂಧರಃ ॥೧೪೩॥

	ಕರ್ಪೂರಗೌರಃ ಕುಶಲಃ ಸತ್ಯಸಂಧೋ ಜಿತೇಂದ್ರಿಯಃ~।\\
	ಶಾಶ್ವತೈಶ್ವರ್ಯವಿಭವಃ ಪೋಷಕಃ ಸುಸಮಾಹಿತಃ ॥೧೪೪॥

ಮಹರ್ಷಿನಾಥಿತೋ ಬ್ರಹ್ಮಯೋನಿಃ ಸರ್ವೋತ್ತಮೋತ್ತಮಃ~।\\
ಭೂಮಿಭಾರಾರ್ತಿಸಂಹರ್ತಾ ಷಡೂರ್ಮಿರಹಿತೋ ಮೃಡಃ ॥೧೪೫॥

	ತ್ರಿವಿಷ್ಟಪೇಶ್ವರಃ ಸರ್ವಹೃದಯಾಂಬುಜಮಧ್ಯಗಃ~।\\
	ಸಹಸ್ರದಲಪದ್ಮಸ್ಥಃ ಸರ್ವವರ್ಣೋಪಶೋಭಿತಃ ॥೧೪೬॥

ಪುಣ್ಯಮೂರ್ತಿಃ ಪುಣ್ಯಲಭ್ಯಃ ಪುಣ್ಯಶ್ರವಣಕೀರ್ತನಃ~।\\
ಸೂರ್ಯಮಂಡಲಮಧ್ಯಸ್ಥಶ್ಚಂದ್ರಮಂಡಲಮಧ್ಯಗಃ ॥೧೪೭॥

	ಸದ್ಭಕ್ತಧ್ಯಾನನಿಗಲಃ ಶರಣಾಗತಪಾಲಕಃ~।\\
	ಶ್ವೇತಾತಪತ್ರರುಚಿರಃ ಶ್ವೇತಚಾಮರವೀಜಿತಃ ॥೧೪೮॥

ಸರ್ವಾವಯವಸಂಪೂರ್ಣಃ ಸರ್ವಲಕ್ಷಣಲಕ್ಷಿತಃ~।\\
ಸರ್ವಮಂಗಲಮಾಂಗಲ್ಯಃ ಸರ್ವಕಾರಣಕಾರಣಃ ॥೧೪೯॥

	ಅಮೋದೋ ಮೋದಜನಕಃ ಸರ್ಪರಾಜೋತ್ತರೀಯಕಃ~।\\
	ಕಪಾಲೀ ಕೋವಿದಃ ಸಿದ್ಧಕಾಂತಿಸಂವಲಿತಾನನಃ ॥೧೫೦॥

ಸರ್ವಸದ್ಗುರುಸಂಸೇವ್ಯೋ ದಿವ್ಯಚಂದನಚರ್ಚಿತಃ~।\\
ವಿಲಾಸಿನೀಕೃತೋಲ್ಲಾಸ ಇಚ್ಛಾಶಕ್ತಿನಿಷೇವಿತಃ ॥೧೫೧॥

	ಅನಂತಾನಂದಸುಖದೋ ನಂದನಃ ಶ್ರೀನಿಕೇತನಃ~।\\
	ಅಮೃತಾಬ್ಧಿಕೃತಾವಾಸೋ ನಿತ್ಯಕ್ಲೀಬೋ ನಿರಾಮಯಃ ॥೧೫೨॥

ಅನಪಾಯೋಽನಂತದೃಷ್ಟಿರಪ್ರಮೇಯೋಽಜರೋಽಮರಃ~।\\
ತಮೋಮೋಹಪ್ರತಿಹತಿರಪ್ರತರ್ಕ್ಯೋಽಮೃತೋಽಕ್ಷರಃ ॥೧೫೩॥

	ಅಮೋಘಬುದ್ಧಿರಾಧಾರ ಆಧಾರಾಧೇಯವರ್ಜಿತಃ~।\\
	ಈಷಣಾತ್ರಯನಿರ್ಮುಕ್ತ ಇಹಾಮುತ್ರವಿವರ್ಜಿತಃ ॥೧೫೪॥

ಋಗ್ಯಜುಃಸಾಮನಯನೋ ಬುದ್ಧಿಸಿದ್ಧಿಸಮೃದ್ಧಿದಃ~।\\
ಔದಾರ್ಯನಿಧಿರಾಪೂರ್ಣ ಐಹಿಕಾಮುಷ್ಮಿಕಪ್ರದಃ ॥೧೫೫॥

	ಶುದ್ಧಸನ್ಮಾತ್ರಸಂವಿದ್ಧೀಸ್ವರೂಪಃ ಸುಖವಿಗ್ರಹಃ~।\\
	ದರ್ಶನಪ್ರಥಮಾಭಾಸೋ ದೃಷ್ಟಿದೃಶ್ಯವಿವರ್ಜಿತಃ ॥೧೫೬॥

ಅಗ್ರಗಣ್ಯೋಽಚಿಂತ್ಯರೂಪಃ ಕಲಿಕಲ್ಮಷನಾಶನಃ~।\\
ವಿಮರ್ಶರೂಪೋ ವಿಮಲೋ ನಿತ್ಯರೂಪೋ ನಿರಾಶ್ರಯಃ ॥೧೫೭॥

ನಿತ್ಯಶುದ್ಧೋ ನಿತ್ಯಬುದ್ಧೋ ನಿತ್ಯಮುಕ್ತೋಽಪರಾಕೃತಃ~।\\
ಮೈತ್ರ್ಯಾದಿವಾಸನಾಲಭ್ಯೋ ಮಹಾಪ್ರಲಯಸಂಸ್ಥಿತಃ ॥೧೫೮॥

ಮಹಾಕೈಲಾಸನಿಲಯಃ ಪ್ರಜ್ಞಾನಘನವಿಗ್ರಹಃ~।\\
ಶ್ರೀಮಾನ್ ವ್ಯಾಘ್ರಪುರಾವಾಸೋ ಭುಕ್ತಿಮುಕ್ತಿಪ್ರದಾಯಕಃ ॥೧೫೯॥

	ಜಗದ್ಯೋನಿರ್ಜಗತ್ಸಾಕ್ಷೀ ಜಗದೀಶೋ ಜಗನ್ಮಯಃ~।\\
	ಜಪೋ ಜಪಪರೋ ಜಪ್ಯೋ ವಿದ್ಯಾಸಿಂಹಾಸನಪ್ರಭುಃ ॥೧೬೦॥

ತತ್ತ್ವಾನಾಂ ಪ್ರಕೃತಿಸ್ತತ್ತ್ವಂ ತತ್ತ್ವಂಪದನಿರೂಪಿತಃ~।\\
ದಿಕ್ಕಾಲಾದ್ಯನವಚ್ಛಿನ್ನಃ ಸಹಜಾನಂದಸಾಗರಃ ॥೧೬೧॥

	ಪ್ರಕೃತಿಃ ಪ್ರಾಕೃತಾತೀತೋ ವಿಜ್ಞಾನೈಕರಸಾಕೃತಿಃ~।\\
	ನಿಃಶಂಕಮತಿದೂರಸ್ಥಶ್ಚೈತ್ಯಚೇತನಚಿಂತನಃ ॥೧೬೨॥

ತಾರಕಾನಾಂ ಹೃದಂತಸ್ಥಸ್ತಾರಕಸ್ತಾರಕಾಂತಕಃ~।\\
ಧ್ಯಾನೈಕಪ್ರಕಟೋ ಧ್ಯೇಯೋ ಧ್ಯಾನೀ ಧ್ಯಾನವಿಭೂಷಣಃ ॥೧೬೩॥

	ಪರಂ ವ್ಯೋಮ ಪರಂ ಧಾಮ ಪರಮಾತ್ಮಾ ಪರಂ ಪದಂ~।\\
	ಪೂರ್ಣಾನಂದಃ ಸದಾನಂದೋ ನಾದಮಧ್ಯಪ್ರತಿಷ್ಠಿತಃ ॥೧೬೪॥

ಪ್ರಮಾವಿಪರ್ಯಯಾತೀತಃ ಪ್ರಣತಾಜ್ಞಾನನಾಶಕಃ~।\\
ಬಾಣಾರ್ಚಿತಾಂಘ್ರಿರ್ಬಹುದೋ ಬಾಲಕೇಲಿಕುತೂಹಲೀ ॥೧೬೫॥

	ಬ್ರಹ್ಮರೂಪೀ ಬ್ರಹ್ಮಪದಂ ಬ್ರಹ್ಮವಿದ್ ಬ್ರಾಹ್ಮಣಪ್ರಿಯಃ~।\\
	ಭ್ರೂಕ್ಷೇಪದತ್ತಲಕ್ಷ್ಮೀಕೋ ಭ್ರೂಮಧ್ಯಧ್ಯಾನಲಕ್ಷಿತಃ ॥೧೬೬॥

ಯಶಸ್ಕರೋ ರತ್ನಗರ್ಭೋ ಮಹಾರಾಜ್ಯಸುಖಪ್ರದಃ~।\\
ಶಬ್ದಬ್ರಹ್ಮ ಶಮಪ್ರಾಪ್ಯೋ ಲಾಭಕೃಲ್ಲೋಕವಿಶ್ರುತಃ ॥೧೬೭॥

	ಶಾಸ್ತಾ ಶಿವಾದ್ರಿನಿಲಯಃ ಶರಣ್ಯೋ ಯಾಜಕಪ್ರಿಯಃ~।\\
	ಸಂಸಾರವೈದ್ಯಃ ಸರ್ವಜ್ಞಃ ಸಭೇಷಜವಿಭೇಷಜಃ ॥೧೬೮॥

ಮನೋವಚೋಭಿರಗ್ರಾಹ್ಯಃ ಪಂಚಕೋಶವಿಲಕ್ಷಣಃ~।\\
ಅವಸ್ಥಾತ್ರಯನಿರ್ಮುಕ್ತಸ್ತ್ವವಸ್ಥಾಸಾಕ್ಷಿತುರ್ಯಕಃ ॥೧೬೯॥

	ಪಂಚಭೂತಾದಿದೂರಸ್ಥಃ ಪ್ರತ್ಯಗೇಕರಸೋಽವ್ಯಯಃ~।\\
	ಷಟ್ಚಕ್ರಾಂತರ್ಗತೋಲ್ಲಾಸೀ ಷಡ್ವಿಕಾರವಿವರ್ಜಿತಃ ॥೧೭೦॥

ವಿಜ್ಞಾನಘನಸಂಪೂರ್ಣೋ ವೀಣಾವಾದನತತ್ಪರಃ~।\\
ನೀಹಾರಾಕಾರಗೌರಾಂಗೋ ಮಹಾಲಾವಣ್ಯವಾರಿಧಿಃ ॥೧೭೧॥

	ಪರಾಭಿಚಾರಶಮನಃ ಷಡಧ್ವೋಪರಿಸಂಸ್ಥಿತಃ~।\\
	ಸುಷುಮ್ನಾಮಾರ್ಗಸಂಚಾರೀ ಬಿಸತಂತುನಿಭಾಕೃತಿಃ ॥೧೭೨॥

ಪಿನಾಕೀ ಲಿಂಗರೂಪಶ್ರೀಃ ಮಂಗಲಾವಯವೋಜ್ಜ್ವಲಃ~।\\
ಕ್ಷೇತ್ರಾಧಿಪಃ ಸುಸಂವೇದ್ಯಃ ಶ್ರೀಪ್ರದೋ ವಿಭವಪ್ರದಃ ॥೧೭೩॥

	ಸರ್ವವಶ್ಯಕರಃ ಸರ್ವದೋಷಹಾ ಪುತ್ರಪೌತ್ರದಃ~।\\
	ತೈಲದೀಪಪ್ರಿಯಸ್ತೈಲಪಕ್ವಾನ್ನಪ್ರೀತಮಾನಸಃ ॥೧೭೪॥

ತೈಲಾಭಿಷೇಕಸಂತುಷ್ಟಸ್ತಿಲಭಕ್ಷಣತತ್ಪರಃ~।\\
ಆಪಾದಕನಿಕಾಮುಕ್ತಾಭೂಷಾಶತಮನೋಹರಃ ॥೧೭೫॥

	ಶಾಣೋಲ್ಲೀಢಮಣಿಶ್ರೇಣೀರಮ್ಯಾಂಘ್ರಿನಖಮಂಡಲಃ~।\\
	ಮಣಿಮಂಜೀರಕಿರಣಕಿಂಜಲ್ಕಿತಪದಾಂಬುಜಃ ॥೧೭೬॥

ಅಪಸ್ಮಾರೋಪರಿನ್ಯಸ್ತಸವ್ಯಪಾದಸರೋರುಹಃ~।\\
ಕಂದರ್ಪತೂಣಾಭಜಂಘೋ ಗುಲ್ಫೋದಂಚಿತನೂಪುರಃ ॥೧೭೭॥

	ಕರಿಹಸ್ತೋಪಮೇಯೋರುರಾದರ್ಶೋಜ್ಜ್ವಲಜಾನುಭೃತ್~।\\
	ವಿಶಂಕಟಕಟಿನ್ಯಸ್ತವಾಚಾಲಮಣಿಮೇಖಲಃ ॥೧೭೮॥

ಆವರ್ತನಾಭಿರೋಮಾಲಿವಲಿಮತ್ಪಲ್ಲವೋದರಃ~।\\
ಮುಕ್ತಾಹಾರಲಸತ್ತುಂಗವಿಪುಲೋರಸ್ಕರಂಜಿತಃ ॥೧೭೯॥

	ವೀರಾಸನಸಮಾಸೀನೋ ವೀಣಾಪುಸ್ತೋಲ್ಲಸತ್ಕರಃ~।\\
	ಅಕ್ಷಮಾಲಾಲಸತ್ಪಾಣಿಶ್ಚಿನ್ಮುದ್ರಿತಕರಾಂಬುಜಃ ॥೧೮೦॥

ಮಾಣಿಕ್ಯಕಂಕಣೋಲ್ಲಾಸಿ ಕರಾಂಬುಜವಿರಾಜಿತಃ~।\\
ಅನರ್ಘ್ಯರತ್ನ ಗ್ರೈವೇಯ ವಿಲಸತ್ಕಂಬುಕಂಧರಃ ॥೧೮೧॥

	ಅನಾಕಲಿತಸಾದೃಶ್ಯಚಿಬುಕಶ್ರೀವಿರಾಜಿತಃ~।\\
	ಮುಗ್ಧಸ್ಮಿತಪರೀಪಾಕಪ್ರಕಾಶಿತರದಾಂಕುರಃ ॥೧೮೨॥

ಚಾರುಚಾಂಪೇಯಪುಷ್ಪಾಭನಾಸಿಕಾಪುಟರಂಜಿತಃ~।\\
ವರವಜ್ರಶಿಲಾದರ್ಶಪರಿಭಾವಿಕಪೋಲಭೂಃ ॥೧೮೩॥

	ಕರ್ಣದ್ವಯೋಲ್ಲಸದ್ದಿವ್ಯಮಣಿಕುಂಡಲಮಂಡಿತಃ~।\\
	ಕರುಣಾಲಹರೀಪೂರ್ಣಕರ್ಣಾಂತಾಯತಲೋಚನಃ ॥೧೮೪॥

ಅರ್ಧಚಂದ್ರಾಭನಿಟಿಲಪಾಟೀರತಿಲಕೋಜ್ಜ್ವಲಃ~।\\
ಚಾರುಚಾಮೀಕರಾಕಾರಜಟಾಚರ್ಚಿತಚಂದನಃ~।\\
ಕೈಲಾಸಶಿಖರಸ್ಫರ್ಧಿಕಮನೀಯನಿಜಾಕೃತಿಃ ॥೧೮೫॥
\authorline{ಇತಿ ಶ್ರೀದಕ್ಷಿಣಾಮೂರ್ತಿಸಹಸ್ರನಾಮಸ್ತೋತ್ರಂ ಸಂಪೂರ್ಣಂ ॥}
\section{ ಶ್ರೀ ಶಿವಕಾಮಸುಂದರೀ ಸಹಸ್ರನಾಮಸ್ತೋತ್ರಂ }
ಅಸ್ಯ ಶ್ರೀಶಿವಕಾಮಸುಂದರೀಸಹಸ್ರನಾಮ ಸ್ತೋತ್ರಮಹಾಮಂತ್ರಸ್ಯ~।
ಸದಾಶಿವ ಋಷಿಃ~। ಅನುಷ್ಟುಪ್ ಛಂದಃ~। ಶ್ರೀಮಚ್ಛಿವಕಾಮಸುಂದರೀ ದೇವತಾ~। ವಾಗ್ಭವಸ್ವರೂಪಂ ಐಂ ಬೀಜಂ~। ಚಿದಾನಂದಾತ್ಮಕಂ ಹ್ರೀಂ ಶಕ್ತಿಃ~। ಕಾಮರಾಜಾತ್ಮಕಂ ಕ್ಲೀಂ ಕೀಲಕಂ~। ಜಪೇ ವಿನಿಯೋಗಃ ॥\\
ಷೋಡಶರ್ಣಾಮೂಲೇನ ನ್ಯಾಸಃ ॥

\dhyana{ಸಿದ್ಧಸಿದ್ಧನವರತ್ನಭೂಮಿಕೇ ಕಲ್ಪವೃಕ್ಷನವವಾಟಿಸಂವೃತೇ~।\\
	ರತ್ನಸಾಲವನಸಂಭೃತೇಽನಿಶಂ ತತ್ರ ವಾಪಿಶತಕೇನ ಸಂವೃತೇ ॥೭॥

ರತ್ನವಾಟಿಮಣಿಮಂಡಪೇಽರುಣೇ ಚಂಡಭಾನುಶತಕೋಟಿಭಾಸುರೇ~।\\
ಆದಿಶೈವಮಣಿಮಂಚಕೇ ಪರೇ ಶಂಕರಾಂಕಮಣಿಪೀಠಕೋಪರಿ ॥\\
	ಕಾದಿಹಾಂತಮನುರೂಪಿಣೀಂ ಶಿವಾಂ ಸಂಸ್ಮರೇಚ್ಚ ಶಿವಕಾಮಸುಂದರೀಂ॥೮॥}

	ಶ್ರೀಶಿವಾ ಶಿವಕಾಮೀ ಚ ಸುಂದರೀ ಭುವನೇಶ್ವರೀ~।\\
	ಆನಂದಸಿಂಧುರಾನಂದಾನಂದಮೂರ್ತಿರ್ವಿನೋದಿನೀ ॥೧॥

ತ್ರೈಪುರೀ ಸುಂದರೀ ಪ್ರೇಮಪಾಥೋನಿಧಿರನುತ್ತಮಾ~।\\
ರಾಮೋಲ್ಲಾಸಾ ಪರಾ ಭೂತಿಃ ವಿಭೂತಿಶ್ಶಂಕರಪ್ರಿಯಾ ॥೨॥

	ಶೃಂಗಾರಮೂರ್ತಿರ್ವಿರತಾ ರಸಾನುಭವರೋಚನಾ~।\\
	ಪರಮಾನಂದಲಹರೀ ರತಿರಂಗವತೀ ಸತೀ ॥೩॥

ರಂಗಮಾಲಾನಂಗಕಲಾಕೇಲಿಃ ಕೈವಲ್ಯದಾ ಕಲಾ~।\\
ರಸಕಲ್ಪಾ ಕಲ್ಪಲತಾ ಕುತೂಹಲವತೀ ಗತಿಃ ॥೪॥

	ವಿನೋದದುಗ್ಧಾ ಸುಸ್ನಿಗ್ಧಾ ಮುಗ್ಧಮೂರ್ತಿರ್ಮನೋಹರಾ~।\\
	ಬಾಲಾರ್ಕಕೋಟಿಕಿರಣಾ ಚಂದ್ರಕೋಟಿಸುಶೀತಲಾ ॥೫॥

ಸ್ರವತ್ಪೀಯೂಷದಿಗ್ಧಾಂಗೀ ಸಂಗೀತ ನಟಿಕಾ ಶಿವಾ~।\\
ಕುರಂಗನಯನಾ ಕಾಂತಾ ಸುಖಸಂತತಿರಿಂದಿರಾ ॥೬॥

	ಮಂಗಲಾ ಮಧುರಾಪಾಂಗಾ ರಂಜನೀ ರಮಣೀ ರತಿಃ~।\\
	ರಾಜರಾಜೇಶ್ವರೀ ರಾಜ್ಞೀ ಮಹೇಂದ್ರಪರಿವಂದಿತಾ ॥೭॥

ಪ್ರಪಂಚಗತಿರೀಶಾನೀ ಸಾಮರಸ್ಯಪರಾಯಣಾ~।\\
ಹಂಸೋಲ್ಲಾಸಾ ಹಂಸಗತಿಃ ಶಿಂಜತ್ಕನಕನೂಪುರಾ ॥೮॥
\newpage
	ಮೇರುಮಂದರವಕ್ಷೋಜಾ ಸೃಣಿಪಾಶವರಾಯುಧಾ~।\\
	ಶಂಖಕೋದಂಡಪುಸ್ತಾಬ್ಜಪಾಣಿದ್ವಯವಿರಾಜಿತಾ ॥೯॥

ಚಂದ್ರಬಿಂಬಾನನಾ ಚಾರುಮಕುಟೋತ್ತಂಸಚಂದ್ರಿಕಾ~।\\
ಸಿಂದೂರತಿಲಕಾ ಚಾರುಧಮ್ಮಿಲ್ಲಾಮಲಮಾಲಿಕಾ ॥೧೦॥

	ಮಂದಾರದಾಮಮುದಿತಾ ರಕ್ತಪುಷ್ಪವಿಭೂಷಿತಾ~।\\
	ಸುವರ್ಣಾಭರಣಪ್ರೀತಾ ಮುಕ್ತಾದಾಮಮನೋಹರಾ ॥೧೧॥

ತಾಂಬೂಲಪೂರವದನಾ ಮದನಾನಂದಮಾನಸಾ~।\\
ಸುಖಾರಾಧ್ಯಾ ತಪಸ್ಸಾರಾ ಕೃಪಾವಾರಿಧಿರೀಶ್ವರೀ ॥೧೨॥

	ವಕ್ಷಃಸ್ಥಲಲಸನ್ಮಗ್ನಾ ಪ್ರಭಾ ಮಧುರಸೋನ್ಮುಖಾ~।\\
	ಬಿಂದುನಾದಾತ್ಮಿಕಾ ಚಾರುರಸಿತಾ ತುರ್ಯರೂಪಿಣೀ ॥೧೩॥

ಕಮನೀಯಾಕೃತಿಃ ಧನ್ಯಾ ಶಂಕರಪ್ರೀತಿಮಂಜರೀ~।\\
ಕನ್ಯಾ ಕಲಾವತೀ ಮಾತಾ ಗಜೇಂದ್ರಗಮನಾ ಶುಭಾ ॥೧೪॥

	ಕುಮಾರೀ ಕರಭೋರುಶ್ರೀಃ ರೂಪಲಕ್ಷ್ಮೀಃ ಸುರಾಜಿತಾ~।\\
	ಸಂತೋಷಸೀಮಾ ಸಂಪತ್ತಿಃ ಶಾತಕುಂಭಪ್ರಿಯಾ ದ್ಯುತಿಃ ॥೧೫॥

ಪರಿಪೂರ್ಣಾ ಜಗದ್ಧಾತ್ರೀ ವಿಧಾತ್ರೀ ಬಲವರ್ಧಿನೀ~।\\
ಸಾರ್ವಭೌಮನೃಪಶ್ರೀಶ್ಚ ಸಾಮ್ರಾಜ್ಯಗತಿರಾಸಿಕಾ ॥೧೬॥

	ಸರೋಜಾಕ್ಷೀ ದೀರ್ಘದೃಷ್ಟಿಃ ಸೌಚಕ್ಷಣವಿಚಕ್ಷಣಾ~।\\
	ರಂಗಸ್ರವಂತೀ ರಸಿಕಾ ಪ್ರಧಾನರಸರೂಪಿಣೀ ॥೧೭॥

ರಸಸಿಂಧುಃ ಸುಗಾತ್ರೀ ಚ ಯುವತಿಃ ಮೈಥುನೋನ್ಮುಖೀ~।\\
ನಿರಂತರಾ ರಸಾಸಕ್ತಾ ಶಕ್ತಿಸ್ತ್ರಿಭುವನಾತ್ಮಿಕಾ ॥೧೮॥

ಕಾಮಾಕ್ಷೀ ಕಾಮನಿಷ್ಠಾ ಚ ಕಾಮೇಶೀ ಭಗಮಂಗಲಾ~।\\
ಸುಭಗಾ ಭಗಿನೀ ಭೋಗ್ಯಾ ಭಾಗ್ಯದಾ ಭಯದಾ ಭಗಾ~।\\
ಭಗಲಿಂಗಾನಂದಕಲಾ ಭಗಮಧ್ಯನಿವಾಸಿನೀ ॥೧೯॥

	ಭಗರೂಪಾ ಭಗಮಯೀ ಭಗಯಂತ್ರಾ ಭಗೋತ್ತಮಾ~।\\
	ಯೋನಿರ್ಜಯಾ ಕಾಮಕಲಾ ಕುಲಾಮೃತಪರಾಯಣಾ ॥೨೦॥

ಕುಲಕುಂಡಾಲಯಾ ಸೂಕ್ಷ್ಮಜೀವಸ್ಫುಲಿಂಗರೂಪಿಣೀ~।\\
ಮೂಲಸ್ಥಿತಾ ಕೇಲಿರತಾ ವಲಯಾಕೃತಿರೀಡಿತಾ ॥೨೧॥

	ಸುಷುಮ್ನಾ ಕಮಲಾನಂದಾ ಚಿತ್ರಾ ಕೂರ್ಮಗತಿರ್ಗಿರಿಃ~।\\
	ಸಿತಾರುಣಾ ಸಿಂಧುರೂಪಾ ಪ್ರವೇಗಾ ನಿರ್ಧನೀ ಕ್ಷಮಾ ॥೨೨॥

ಘಂಟಾಕೋಟಿರಸಾರಾವಾ ರವಿಬಿಂಬೋತ್ಥಿತಾದ್ಭುತಾ~।\\
ನಾದಾಂತಲೀನಾ ಸಂಪೂರ್ಣಾ ಪ್ರಣವಾ ಬಹುರೂಪಿಣೀ ॥೨೩॥

	ಭೃಂಗಾರಾವಾ ವಶಗತಿಃ ವಾಗೀಶೀ ಮಧುರಧ್ವನಿಃ~।\\
	ವರ್ಣಮಾಲಾ ಸಿದ್ಧಿಕಲಾ ಷಟ್ಚಕ್ರಕ್ರಮವಾಸಿನೀ ॥೨೪॥

ಮಣಿಪೂರಸ್ಥಿತಾ ಸ್ನಿಗ್ಧಾ ಕೂರ್ಮಚಕ್ರಪರಾಯಣಾ~।\\
ಮೂಲಕೇಲಿರತಾ ಸಾಧ್ವೀ ಸ್ವಾಧಿಷ್ಠಾನನಿವಾಸಿನೀ ॥೨೫॥

	ಅನಾಹತಗತಿರ್ದೀಪಾ ಶಿವಾನಂದಮಯದ್ಯುತಿಃ~।\\
	ವಿರುದ್ಧರುಧಾ ಸಂಬುದ್ಧಾ ಜೀವಭೋಕ್ತ್ರೀ ಸ್ಥಲೀರತಾ ॥೨೬॥

ಆಜ್ಞಾಚಕ್ರೋಜ್ಜ್ವಲಸ್ಫಾರಸ್ಫುರಂತೀ ನಿರ್ಗತದ್ವಿಷಾ~।\\
ಚಂದ್ರಿಕಾ ಚಂದ್ರಕೋಟೀಶೀ ಸೂರ್ಯಕೋಟಿಪ್ರಭಾಮಯೀ ॥೨೭॥
\newpage
	ಪದ್ಮರಾಗಾರುಣಚ್ಛಾಯಾ ನಿತ್ಯಾಹ್ಲಾದಮಯೀಪ್ರಭಾ~।\\
	ಮಹಾಶೂನ್ಯಾಲಯಾ ಚಂದ್ರಮಂಡಲಾಮೃತನಂದಿತಾ ॥೨೮॥

ಕಾಂತಾಂಗಸಂಗಮುದಿತಾ ಸುಧಾಮಾಧುರ್ಯಸಂಭೃತಾ~।\\
ಮಹಾಚಂದ್ರಸ್ಮಿತಾಲಸಾ ಮೃತ್ಪಾತ್ರಸ್ಥಾ ಸುಧಾದ್ಯುತಿಃ ॥೨೯॥

	ಸ್ರವತ್ಪೀಯೂಷಸಂಸಕ್ತಾ ಶಶ್ವತ್ಕುಂಡಾಲಯಾ ಭವಾ~।\\
	ಶ್ರೇಯೋ ದ್ಯುತಿಃ ಪ್ರತ್ಯಗರ್ಥಾ ಸೇವಾ ಫಲವತೀ ಮಹೀ ॥೩೦॥

ಶಿವಾ ಶಿವಪ್ರಿಯಾ ಶೈವಾ ಶಂಕರೀ ಶಾಂಭವೀ ವಿಭುಃ~।\\
ಸ್ವಯಂಭೂಃ ಸ್ವಪ್ರಿಯಾ ಸ್ವೀಯಾ ಸ್ವಕೀಯಾ ಜನಮಾತೃಕಾ ॥

	ಸುರಾಮಾ ಸ್ವಪ್ರಿಯಾ ಶ್ರೇಯಃ ಸ್ವಾಧಿಕಾರಾಧಿನಾಯಿಕಾ~।\\
	ಮಂಡಲಾ ಜನನೀ ಮಾನ್ಯಾ ಸರ್ವಮಂಗಲಸಂತತಿಃ ॥೩೨॥

ಭದ್ರಾ ಭಗವತೀ ಭಾವ್ಯಾ ಕಲಿತಾರ್ಧೇಂದುಭಾಸುರಾ~।\\
ಕಲ್ಯಾಣಲಲಿತಾ ಕಾಮ್ಯಾ ಕುಕರ್ಮಕುಮತಿಪ್ರದಾ ॥೩೩॥

	ಕುರಂಗಾಕ್ಷೀ ಕ್ಷೀರನೇತ್ರಾ ಕ್ಷೀರಾ ಮಧುರಸೋನ್ಮದಾ~।\\
	ವಾರುಣೀಪಾನಮುದಿತಾ ಮದಿರಾಮುದಿತಾ ಸ್ಥಿರಾ ॥೩೪॥

ಕಾದಂಬರೀಪಾನರುಚಿಃ ವಿಪಾಶಾ ಪಶುಭಾವನಾ~।\\
ಮುದಿತಾ ಲಲಿತಾಪಾಂಗಾ ದರಾಂದೋಲಿತದೀರ್ಘದೃಕ್ ॥೩೫॥

	ದೈತ್ಯಕುಲಾನಲಶಿಖಾ ಮನೋರಥಸುಧಾದ್ಯುತಿಃ~।\\
	ಸುವಾಸಿನೀ ಪೀತಗಾತ್ರೀ ಪೀನಶ್ರೋಣಿಪಯೋಧರಾ ॥೩೬॥

ಸುಚಾರುಕಬರೀ ದಧ್ಯುದಧ್ಯುತ್ಥಿತ ಸುಮೌಕ್ತಿಕಾ~।\\
ಬಿಂಬಾಧರದ್ಯುತಿಃ ಮುಗ್ಧಾ ಪ್ರವಾಲೋತ್ತಮದೀಧಿತಿಃ ॥೩೭॥

	ತಿಲಪ್ರಸೂನನಾಸಾಗ್ರಾ ಹೇಮಮೌಕ್ತಿಕಕೋರಕಾ~।\\
	ನಿಷ್ಕಲಂಕೇಂದುವದನಾ ಬಾಲೇಂದುವದನೋಜ್ವಲಾ ॥೩೮॥

ನೃತ್ಯಂತ್ಯಂಜನನೇತ್ರಾಂತಾ ಪ್ರಸ್ಫುರತ್ಕರ್ಣಶಷ್ಕುಲೀ~।\\
ಭಾಲಚಂದ್ರಾತಪೋನ್ನದ್ಧಾ ಮಣಿಸೂರ್ಯಕಿರೀಟಿನೀ ॥೩೯॥

	ಕಚೌಘಚಂಪಕಶ್ರೇಣೀ ಮಾಲಿನೀದಾಮಮಂಡಿತಾ~।\\
	ಹೇಮಮಾಣಿಕ್ಯ ತಾಟಂಕಾ ಮಣಿಕಾಂಚನ ಕುಂಡಲಾ ॥೪೦॥

ಸುಚಾರುಚುಬುಕಾ ಕಂಬುಕಂಠೀ ಕುಂಡಾವಲೀ ರಮಾ~।\\
ಗಂಗಾತರಂಗಹಾರೋರ್ಮಿಃ ಮತ್ತಕೋಕಿಲನಿಸ್ವನಾ ॥೪೧॥

	ಮೃಣಾಲವಿಲಸದ್ಬಾಹು ಪಾಶಾಂಕುಶ ಧನುರ್ಧರಾ~।\\
	ಕೇಯೂರಕಂಕಣಶ್ರೇಣೀ ನಾನಾಮಣಿಮನೋರಮಾ ॥೪೨॥

ತಾಮ್ರಪಂಕಜಪಾಣಿಶ್ರೀಃ ನವರತ್ನಪ್ರಭಾವತೀ~।\\
ಅಂಗುಲೀಯಮಣಿಶ್ರೇಣೀ ಕಾಂತಿಮಂಗಲಸಂತತಿಃ ॥೪೩॥

	ಮಂದರದ್ವಂದ್ವಸುಕುಚಾ ರೋಮರಾಜಿಭುಜಂಗಿಕಾ~।\\
	ಗಂಭೀರನಾಭಿಸ್ತ್ರಿವಲೀಭಂಗುರಾ ಕ್ಷಣಿಮಧ್ಯಮಾ ॥೪೪॥

ರಣತ್ಕಾಂಚೀಗುಣಾನದ್ಧಾ ಪಟ್ಟಾಂಶುಕನಿತಂಬಿಕಾ~।\\
ಮೇರುಸಂಧಿನಿತಂಬಾಢ್ಯಾ ಗಜಶುಂಡೋರುಯುಗ್ಮಯುಕ್ ॥೪೫॥

	ಸುಜಾನುರ್ಮದನಾನಂದಮಯಜಂಘಾದ್ವಯಾನ್ವಿತಾ~।\\
	ಗೂಢಗುಲ್ಫಾ ಮಂಜುಶಿಂಜನ್ಮಣಿನೂಪುರಮಂಡಿತಾ ॥೪೬॥

ಪದದ್ವಂದ್ವಜಿತಾಂಭೋಜಾ ನಖಚಂದ್ರಾವಲೀಪ್ರಭಾ~।\\
ಸುಸೀಮಪ್ರಪದಾ ರಾಜಂಹಸಮತ್ತೇಭಮಂದಗಾ ॥೪೭॥

	ಯೋಗಿಧ್ಯೇಯಪದದ್ವಂದ್ವಾ ಸೌಂದರ್ಯಾಮೃತಸಾರಿಣೀ~।\\
	ಲಾವಣ್ಯಸಿಂಧುಃ ಸಿಂದೂರತಿಲಕಾ ಕುಟಿಲಾಲಕಾ ॥೪೮॥

ಸಾಧುಸೀಮಂತಿನೀ ಸಿದ್ಧಬುದ್ಧವೃಂದಾರಕೋದಯಾ~।\\
ಬಾಲಾರ್ಕಕಿರಣಶ್ರೇಣಿಶೋಣಶ್ರೀಃ ಪ್ರೇಮಕಾಮಧುಕ್ ॥೪೯॥

	ರಸಗಂಭೀರಸರಸೀ ಪದ್ಮಿನೀ ರಸಸಾರಸಾ~।\\
	ಪ್ರಸನ್ನಾಸನ್ನವರದಾ ಶಾರದಾ ಭುವಿ ಭಾಗ್ಯದಾ ॥೫೦॥

ನಟರಾಜಪ್ರಿಯಾ ವಿಶ್ವಾನಾದ್ಯಾ ನರ್ತಕನರ್ತಕೀ~।\\
ಚಿತ್ರಯಂತ್ರಾ ಚಿತ್ರತಂತ್ರಾ ಚಿತ್ರವಿದ್ಯಾವಲೀಯತಿಃ ॥೫೧॥

	ಚಿತ್ರಕೂಟಾ ತ್ರಿಕೂಟಾ ಚ ಪಂಚಕೂಟಾ ಚ ಪಂಚಮೀ~।\\
	ಚತುಷ್ಕೂಟಾ ಶಂಭುವಿದ್ಯಾ ಷಟ್ಕೂಟಾ ವಿಷ್ಣುಪೂಜಿತಾ ॥೫೨॥

ಕೂಟಷೋಡಶಸಂಪನ್ನಾ ತುರೀಯಾ ಪರಮಾ ಕಲಾ~।\\
ಷೋಡಶೀ ಮಂತ್ರಯಂತ್ರಾಣಾಂ ಈಶ್ವರೀ ಮೇರುಮಂಡಲಾ ॥೫೩॥

	ಷೋಡಶಾರ್ಣಾ ತ್ರಿವರ್ಣಾ ಚ ಬಿಂದುನಾದಸ್ವರೂಪಿಣೀ~।\\
	ವರ್ಣಾತೀತಾ ವರ್ಣಮತಾ ಶಬ್ದಬ್ರಹ್ಮಮಯೀ ಸುಖಾ ॥೫೪॥

ಸುಖಜ್ಯೋತ್ಸ್ನಾನಂದವಿದ್ಯುದಂತರಾಕಾಶದೇವತಾ~।\\
ಚೈತನ್ಯಾ ವಿಧಿಕೂಟಾತ್ಮಾ ಕಾಮೇಶೀ ಸ್ವಪ್ನದರ್ಶನಾ ॥೫೫॥

	ಸ್ವಪ್ನರೂಪಾ ಬೋಧಕರೀ ಜಾಗ್ರತೀ ಜಾಗರಾಶ್ರಯಾ~।\\
	ಸ್ವಪ್ನಾಶ್ರಯಾ ಸುಷುಪ್ತಿಸ್ಥಾ ತಂತ್ರಮೂರ್ತಿಶ್ಚ ಮಾಧವೀ ॥೫೬॥

ಲೋಪಾಮುದ್ರಾ ಕಾಮರಾಜ್ಞೀ ಮಾಧವೀ ಮಿತ್ರರೂಪಿಣೀ~।\\
ಶಾಂಕರೀ ನಂದಿವಿದ್ಯಾ ಚ ಭಾಸ್ವನ್ಮಂಡಲಮಧ್ಯಗಾ ॥೫೭॥

	ಮಾಹೇಂದ್ರಸ್ವರ್ಗಸಂಪತ್ತಿಃ ದುರ್ವಾಸಸ್ಸೇವಿತಾ ಶ್ರುತಿಃ~।\\
	ಸಾಧಕೇಂದ್ರಗತಿಸ್ಸಾಧ್ವೀ ಸುಲಿಪ್ತಾ ಸಿದ್ಧಿಕಂಧರಾ ॥೫೮॥

ಪುರತ್ರಯೇಶೀ ಪುರಕೃತ್ ಷಷ್ಠೀ ಚ ಪರದೇವತಾ~।\\
ವಿಘ್ನದೂರೀ ಭೂರಿಗುಣಾ ಪುಷ್ಟಿಃ ಪೂಜಿತಕಾಮಧುಕ್ ॥೫೯॥

	ಹೇರಂಬಮಾತಾ ಗಣಪಾ ಗುಹಾಂಬಾಽಽರ್ಯಾ ನಿತಂಬಿನೀ~।\\
	ಏಷಾ ಸೀಮಂತಿನೀ ಮೋಕ್ಷದಕ್ಷಾ ದೀಕ್ಷಿತಮಾತೃಕಾ ॥೬೦॥

ಸಾಧಕಾಂಬಾ ಸಿದ್ಧಮಾತಾ ಸಾಧಕೇಂದ್ರಮನೋರಮಾ~।\\
ಯೌವನೋನ್ಮಾದಿನೀ ತುಂಗಸ್ತನೀ ಸುಶ್ರೋಣಿಮಂಡಿತಾ ॥೬೧॥

	ಪದ್ಮರಕ್ತೋತ್ಪಲವತೀ ರಕ್ತಮಾಲ್ಯಾನುಲೇಪನಾ~।\\
	ರಕ್ತಮಾಲ್ಯರುಚಿರ್ದಕ್ಷಾ ಶಿಖಂಡಿನ್ಯತಿಸುಂದರೀ ॥೬೨॥

ಶಿಖಂಡಿನೃತ್ಯಸಂತುಷ್ಟಾ ಶಿಖಂಡಿಕುಲಪಾಲಿನೀ~।\\
ವಸುಂಧರಾ ಚ ಸುರಭಿಃ ಕಮನೀಯತನುಶ್ಶುಭಾ ॥೬೩॥

	ನಂದಿನೀ ತ್ರೀಕ್ಷಣವತೀ ವಸಿಷ್ಠಾಲಯದೇವತಾ~।\\
	ಗೋಲಕೇಶೀ ಚ ಲೋಕೇಂದ್ರಾ ನೃಲೋಕಪರಿಪಾಲಿಕಾ ॥೬೪॥

ಹವಿರ್ಧಾತ್ರೀ ದೇವಮಾತಾ ವೃಂದಾರಕಪರಾತ್ಮಯುಕ್~।\\
ರುದ್ರಮಾತಾ ರುದ್ರಪತ್ನೀ ಮದೋದ್ಗಾರಭರಾ ಕ್ಷಿತಿಃ ॥೬೫॥

	ದಕ್ಷಿಣಾ ಯಜ್ಞಸಂಪತ್ತಿಃ ಸ್ವಬಲಾ ಧೀರನಂದಿತಾ~।\\
	ಕ್ಷೀರಪೂರ್ಣಾರ್ಣವಗತಿಃ ಸುಧಾಯೋನಿಃ ಸುಲೋಚನಾ ॥೬೬॥

ರಮಾ ತುಂಗಾ ಸದಾಸೇವ್ಯಾ ಸುರಸಂಘದಯಾ ಉಮಾ~।\\
ಸುಚರಿತ್ರಾ ಚಿತ್ರವರಾ ಸುಸ್ತನೀ ವತ್ಸವತ್ಸಲಾ ॥೬೭॥

	ರಜಸ್ವಲಾ ರಜೋಯುಕ್ತಾ ರಂಜಿತಾ ರಂಗಮಾಲಿಕಾ~।\\
	ರಕ್ತಪ್ರಿಯಾ ಸುರಕ್ತಾ ಚ ರತಿರಂಗಸ್ವರೂಪಿಣೀ ॥೬೮॥

ರಜಶ್ಶುಕ್ಲಾಕ್ಷಿಕಾ ನಿಷ್ಠಾ ಋತುಸ್ನಾತಾ ರತಿಪ್ರಿಯಾ~।\\
ಭಾವ್ಯಾಭಾವ್ಯಾ ಕಾಮಕೇಲಿಃ ಸ್ಮರಭೂಃ ಸ್ಮರಜೀವಿಕಾ ॥೬೯॥

	ಸಮಾಧಿಕುಸುಮಾನಂದಾ ಸ್ವಯಂಭುಕುಸುಮಪ್ರಿಯಾ~।\\
	ಸ್ವಯಂಭುಪ್ರೇಮಸಂತುಷ್ಟಾ ಸ್ವಯಂಭೂನಿಂದಕಾಂತಕಾ ॥೭೦॥

ಸ್ವಯಂಭುಸ್ಥಾ ಶಕ್ತಿಪುಟಾ ರವಿಃ ಸರ್ವಸ್ವಪೇಟಿಕಾ~।\\
ಅತ್ಯಂತರಸಿಕಾ ದೂತಿಃ ವಿದಗ್ಧಾ ಪ್ರೀತಿಪೂಜಿತಾ ॥೭೧॥

	ತೂಲಿಕಾಯಂತ್ರನಿಲಯಾ ಯೋಗಪೀಠನಿವಾಸಿನೀ~।\\
	ಸುಲಕ್ಷಣಾ ದೃಶ್ಯರೂಪಾ ಸರ್ವ ಲಕ್ಷಣಲಕ್ಷಿತಾ ॥೭೨॥

ನಾನಾಲಂಕಾರಸುಭಗಾ ಪಂಚಕಾಮಶರಾರ್ಚಿತಾ~।\\
ಊರ್ಧ್ವತ್ರಿಕೋಣಯಂತ್ರಸ್ಥಾ ಬಾಲಾ ಕಾಮೇಶ್ವರೀ ತಥಾ ॥೭೩॥

	ಗುಣಾಧ್ಯಕ್ಷಾ ಕುಲಾಧ್ಯಕ್ಷಾ ಲಕ್ಷ್ಮೀಶ್ಚೈವ ಸರಸ್ವತೀ~।\\
	ವಸಂತಮದನೋತ್ತುಂಗ ಸ್ತನೀ ಕುಚಭರೋನ್ನತಾ ॥೭೪॥

ಕಲಾಧರಮುಖೀ ಮೂರ್ಧಪಾಥೋಧಿಶ್ಚ ಕಲಾವತೀ~।\\
ದಕ್ಷಪಾದಾದಿಶೀರ್ಷಾಂತಷೋಡಶಸ್ವರಸಂಯುತಾ ॥೭೫॥

	ಶ್ರದ್ಧಾ ಪೂರ್ತಿಃ ರತಿಶ್ಚೈವ ಭೂತಿಃ ಕಾಂತಿರ್ಮನೋರಮಾ~।\\
	ವಿಮಲಾ ಯೋಗಿನೀ ಘೋರಾ ಮದನೋನ್ಮಾದಿನೀ ಮದಾ ॥೭೬॥

ಮೋದಿನೀ ದೀಪಿನೀ ಚೈವ ಶೋಷಿಣೀ ಚ ವಶಂಕರೀ~।\\
ರಜನ್ಯಂತಾ ಕಾಮಕಲಾ ಲಸತ್ಕಮಲಧಾರಿಣೀ ॥೭೭॥

	ವಾಮಮೂರ್ಧಾದಿಪಾದಾಂತಷೋಡಶಸ್ವರಸಂಯುತಾ~।\\
	ಪೂಷರೂಪಾ ಸುಮನಸಾಂ ಸೇವ್ಯಾ ಪ್ರೀತಿಃ ದ್ಯುತಿಸ್ತಥಾ ॥೭೮॥

ಋದ್ಧಿಃ ಸೌದಾಮಿನೀ ಚಿಚ್ಚ ಹಂಸಮಾಲಾವೃತಾ ತಥಾ~।\\
ಶಶಿನೀ ಚೈವ ಚ ಸ್ವಸ್ಥಾ ಸಂಪೂರ್ಣಮಂಡಲೋದಯಾ ॥೭೯॥

	ಪುಷ್ಟಿಶ್ಚಾಮೃತಪೂರ್ಣಾ ಚ ಭಗಮಾಲಾ ಸ್ವರೂಪಿಣೀ~।\\
	ಭಗಯಂತ್ರಾಶ್ರಯಾ ಶಂಭುರೂಪಾ ಸಂಯೋಗಯೋಗಿನೀ ॥೮೦॥

ದ್ರಾವಿಣೀ ಬೀಜರೂಪಾ ಚ ಹ್ಯಕ್ಷುಬ್ಧಾ ಸಾಧಕಪ್ರಿಯಾ~।\\
ರಜಃ ಪೀಠಮಯೀ ನಾದ್ಯಾ ಸುಖದಾ ವಾಂಛಿತಪ್ರದಾ ॥೮೧॥

	ರಜಸ್ಸವಿತ್ ರಜಶ್ಶಕ್ತಿಃ ಶುಕ್ಲಬಿಂದು ಸ್ವರೂಪಿಣೀ~।\\
	ಸರ್ವಸಾಕ್ಷೀ ಸಾಮರಸ್ಯಾ ಶಿವಶಕ್ತಿಮಯೀ ಪ್ರಭಾ ॥೮೨॥

ಸಂಯೋಗಾನಂದನಿಲಯಾ ಸಂಯೋಗಪ್ರೀತಿಮಾತೃಕಾ~।\\
ಸಂಯೋಗಕುಸುಮಾನಂದಾ ಸಂಯೋಗಯೋಗಪದ್ಧತಿಃ ॥೮೩॥

	ಸಂಯೋಗಸುಖದಾವಸ್ಥಾ ಚಿದಾನಂದಾರ್ಘ್ಯಸೇವಿತಾ~।\\
	ಅರ್ಘ್ಯಪೂಜ್ಯಾ ಚ ಸಂಪತ್ತಿಃ ಅರ್ಘ್ಯದಾಭಿನ್ನರೂಪಿಣೀ ॥೮೪॥

ಸಾಮರಸ್ಯಪರಾ ಪ್ರೀತಾ ಪ್ರಿಯಸಂಗಮರಂಗಿಣೀ~।\\
ಜ್ಞಾನದೂತೀ ಜ್ಞಾನಗಮ್ಯಾ ಜ್ಞಾನಯೋನಿಶ್ಶಿವಾಲಯಾ ॥೮೫॥

	ಚಿತ್ಕಲಾ ಸತ್ಕಲಾ ಜ್ಞಾನಕಲಾ ಸಂವಿತ್ಕಲಾತ್ಮಿಕಾ~।\\
	ಕಲಾಚತುಷ್ಟಯೀ ಪದ್ಮವಾಸಿನೀ ಸೂಕ್ಷ್ಮರೂಪಿಣೀ ॥೮೬॥

ಹಂಸಕೇಲಿಸ್ಥಲಸ್ವಸ್ಥಾ ಹಂಸದ್ವಯವಿಕಾಸಿನೀ~।\\
ವಿರಾಗಿತಾ ಮೋಕ್ಷಕಲಾ ಪರಮಾತ್ಮಕಲಾವತೀ ॥೮೭॥

	ವಿದ್ಯಾಕಲಾಂತರಾತ್ಮಸ್ಥಾ ಚತುಷ್ಟಯಕಲಾವತೀ~।\\
	ವಿದ್ಯಾಸಂತೋಷಣಾ ತೃಪ್ತಿ ಪರಬ್ರಹ್ಮಪ್ರಕಾಶಿನೀ ॥೮೮॥

ಪರಮಾತ್ಮಪರಾ ವಸ್ತುಲೀನಾ ಶಕ್ತಿಚತುಷ್ಟಯೀ~।\\
ಶಾಂತಿರ್ಬೋಧಕಲಾ ವ್ಯಾಪ್ತಿಃ ಪರಜ್ಞಾನಾತ್ಮಿಕಾ ಕಲಾ ॥೮೯॥

	ಪಶ್ಯಂತೀ ಪರಮಾತ್ಮಸ್ಥಾ ಚಾಂತರಾತ್ಮಕಲಾ ಶಿವಾ~।\\
	ಮಧ್ಯಮಾ ವೈಖರೀ ಚಾತ್ಮ ಕಲಾಽಽನಂದಕಲಾವತೀ ॥೯೦॥

ತರುಣೀ ತಾರಕಾ ತಾರಾ ಶಿವಲಿಂಗಾಲಯಾತ್ಮವಿತ್~।\\
ಪರಸ್ಪರಸ್ವಭಾವಾ ಚ ಬ್ರಹ್ಮಜ್ಞಾನವಿನೋದಿನೀ ॥೯೧॥

	ರಾಮೋಲ್ಲಾಸಾ ಚ ದುರ್ಧರ್ಷಾ ಪರಮಾರ್ಘ್ಯಪ್ರಿಯಾ ರಮಾ~।\\
	ಜಾತ್ಯಾದಿರಹಿತಾ ಯೋಗಿನ್ಯಾನಂದಮಾತ್ರಪದ್ಧತಿಃ ॥೯೨॥

ಕಾಂತಾ ಶಾಂತಾ ದಾಂತಯಾತಿಃ ಕಲಿತಾ ಹೋಮಪದ್ಧತಿಃ~।\\
ದಿವ್ಯಭಾವಪ್ರದಾ ದಿವ್ಯಾ ವೀರಸೂರ್ವೀರಭಾವದಾ ॥೯೩॥

	ಪಶುದೇಹಾ ವೀರಗತಿಃ ವೀರಹಂಸಮನೋದಯಾ~।\\
	ಮೂರ್ಧಾಭಿಷಿಕ್ತಾ ರಾಜಶ್ರೀಃ ಕ್ಷತ್ರಿಯೋತ್ತಮಮಾತೃಕಾ ॥೯೪॥

ಶಸ್ತ್ರಾಸ್ತ್ರಕುಶಲಾ ಶೋಭಾ ರಥಸ್ಥಾ ಯುದ್ಧಜೀವಿಕಾ~।\\
ಅಶ್ವಾರೂಢಾ ಗಜಾರೂಢಾ ಭೂತೋಕ್ತಿಃ ಸುರಸುಶ್ರಯಾ ॥೯೫॥

	ರಾಜನೀತಿಶ್ಶಾಂತಿಕರ್ತ್ರೀ ಚತುರಂಗಬಲಾಶ್ರಯಾ~।\\
	ಪೋಷಿಣೀ ಶರಣಾ ಪದ್ಮಪಾಲಿಕಾ ಜಯಪಾಲಿಕಾ ॥೯೬॥

ವಿಜಯಾ ಯೋಗಿನೀ ಯಾತ್ರಾ ಪರಸೈನ್ಯವಿಮರ್ದಿನೀ~।\\
ಪೂರ್ಣವಿತ್ತಾ ವಿತ್ತಗಮ್ಯಾ ವಿತ್ತಸಂಚಯ ಶಾಲಿನೀ ॥೯೭॥

	ಮಹೇಶೀ ರಾಜ್ಯಭೋಗಾ ಚ ಗಣಿಕಾಗಣಭೋಗಭೃತ್~।\\
	ಉಕಾರಿಣೀ ರಮಾ ಯೋಗ್ಯಾ ಮಂದಸೇವ್ಯಾ ಪದಾತ್ಮಿಕಾ ॥

ಸೈನ್ಯಶ್ರೇಣೀ ಶೌರ್ಯರತಾ ಪತಾಕಾಧ್ವಜಮಾಲಿನೀ~।\\
ಸುಚ್ಛತ್ರ ಚಾಮರಶ್ರೇಣಿಃ ಯುವರಾಜವಿವರ್ಧಿನೀ ॥೯೯॥

	ಪೂಜಾ ಸರ್ವಸ್ವಸಂಭಾರಾ ಪೂಜಾಪಾಲನಲಾಲಸಾ~।\\
	ಪೂಜಾಭಿಪೂಜನೀಯಾ ಚ ರಾಜಕಾರ್ಯಪರಾಯಣಾ ॥೧೦೦॥

ಬ್ರಹ್ಮಕ್ಷತ್ರಮಯೀ ಸೋಮಸೂರ್ಯವಹ್ನಿಸ್ವರೂಪಿಣೀ~।\\
ಪೌರೋಹಿತ್ಯಪ್ರಿಯಾ ಸಾಧ್ವೀ ಬ್ರಹ್ಮಾಣೀ ಯಂತ್ರಸಂತತಿಃ ॥

	ಸೋಮಪಾನಜನಾಪ್ರೀತಾ ಯೋಜನಾಧ್ವಗತಿಕ್ಷಮಾ~।\\
	ಪ್ರೀತಿಗ್ರಹಾ ಪರಾ ದಾತ್ರೀ ಶ್ರೇಷ್ಠಜಾತಿಃ ಸತಾಂಗತಿಃ ॥೧೦೨॥

ಗಾಯತ್ರೀ ವೇದವಿದ್ಧ್ಯೇಯಾ ದೀಕ್ಷಾ ಸಂತೋಷತರ್ಪಣಾ~।\\
ರತ್ನದೀಧಿತಿವಿದ್ಯುತ್ಸಹಸನಾ ವೈಶ್ಯಜೀವಿಕಾ ॥೧೦೩॥

	ಕೃಷಿರ್ವಾಣಿಜ್ಯಭೂತಿಶ್ಚ ವೃದ್ಧಿದಾ ವೃದ್ಧಸೇವಿತಾ~।\\
	ತುಲಾಧಾರಾ ಸ್ವಪ್ನಕಾಮಾ ಮಾನೋನ್ಮಾನಪರಾಯಣಾ ॥೧೦೪॥

ಶ್ರದ್ಧಾ ವಿಪ್ರಗತಿಃ ಕರ್ಮಕರೀ ಕೌತುಕಪೂಜಿತಾ~।\\
ನಾನಾಭಿಚಾರಚತುರಾ ವಾರಸ್ತ್ರೀಶ್ರೀಃ ಕಲಾಮಯೀ ॥೧೦೫॥

	ಸುಕರ್ಣಧಾರಾ ನೌಪಾರಾ ಸರ್ವಾಶಾ ರತಿಮೋಹಿನೀ~।\\
	ದುರ್ಗಾ ವಿಂಧ್ಯವನಸ್ಥಾ ಚ ಕಾಲದರ್ಪನಿಷೂದಿನೀ ॥೧೦೬॥

ಭೂಭಾರಶಮನೀ ಕೃಷ್ಣಾ ರಕ್ಷೋರಾಕ್ಷಸಸಾಹಸಾ~।\\
ವಿವಿಧೋತ್ಪಾತಶಮನೀ ಸಮಯಾ ಸುರಸೇವಿತಾ ॥೧೦೭॥

	ಪಂಚಾವಯವವಾಕ್ಯಶ್ರೀಃ ಪ್ರಪಂಚೋದ್ಯಾನಚಂದ್ರಿಕಾ~।\\
	ಸಿದ್ಧಿಸಂದೋಹಸಂಸಿದ್ಧಯೋಗಿನೀವೃಂದಸೇವಿತಾ ॥೧೦೮॥

ನಿತ್ಯಾ ಷೋಡಶಿಕಾರೂಪಾ ಕಾಮೇಶೀ ಭಗಮಾಲಿನೀ~।\\
ನಿತ್ಯಕ್ಲಿನ್ನಾ ನಿರಾಧಾರಾ ವಹ್ನಿಮಂಡಲವಾಸಿನೀ ॥೧೦೯॥

	ಮಹಾವಜ್ರೇಶ್ವರೀ ನಿತ್ಯಾ ಶಿವದೂತೀತಿ ವಿಶ್ರುತಾ~।\\
	ತ್ವರಿತಾ ಪ್ರಥಿತಾ ಖ್ಯಾತಾ ವಿಖ್ಯಾತಾ ಕುಲಸುಂದರೀ ॥೧೧೦॥

ನಿತ್ಯಾ ನೀಲಪತಾಕಾ ಚ ವಿಜಯಾ ಸರ್ವಮಂಗಲಾ~।\\
ಜ್ವಾಲಾಮಾಲಾ ವಿಚಿತ್ರಾ ಚ ಮಹಾತ್ರಿಪುರಸುಂದರೀ ॥೧೧೧॥

	ಗುರುವೃಂದಾ ಪರಗುರುಃ ಪ್ರಕಾಶಾನಂದದಾಯಿನೀ~।\\
	ಶಿವಾನಂದಾ ನಾದರೂಪಾ ಶಕ್ರಾನಂದಸ್ವರೂಪಿಣೀ ॥೧೧೨॥

ದೇವ್ಯಾನಂದಾ ನಾದಮಯೀ ಕೌಲೇಶಾನಂದನಾಥಿನೀ~।\\
ಶುಕ್ಲದೇವ್ಯಾನಂದನಾಥಾ ಕುಲೇಶಾನಂದದಾಯಿನೀ ॥೧೧೩॥

	ದಿವ್ಯೌಘಸೇವಿತಾ ದಿವ್ಯಭೋಗದಾನಪರಾಯಣಾ~।\\
	ಕ್ರೀಡಾನಂದಾ ಕ್ರೀಡಮಾನಾ ಸಮಯಾನಂದದಾಯಿನೀ ॥೧೧೪॥

ವೇದಾನಂದಾ ಪಾರ್ವತೀ ಚ ಸಹಜಾನಂದದಾಯಿನೀ~।\\
ಸಿದ್ಧೌಘಗುರುರೂಪಾ ಚಾಪ್ಯಪರಾ ಗುರುರೂಪಿಣೀ ॥೧೧೫॥

	ಗಗನಾನಂದನಾಥಾ ಚ ವಿಶ್ವಾದ್ಯಾನಂದದಾಯಿನೀ~।\\
	ವಿಮಲಾನಂದನಾಥಾ ಚ ಮದನಾನಂದದಾಯಿನೀ ॥೧೧೬॥

ಭುವನಾನಂದನಾಥಾ ಚ ಲೀಲೋದ್ಯಾನಪ್ರಿಯಾ ಗತಿಃ~।\\
ಸ್ವಾತ್ಮಾನಂದವಿನೋದಾ ಚ ಪ್ರಿಯಾದ್ಯಾನಂದನಾಥಿನೀ ॥೧೧೭॥

	ಮಾನವಾದ್ಯಾ ಗುರುಶ್ರೇಷ್ಠಾ ಪರಮೇಷ್ಠಿ ಗುರುಪ್ರಭಾ~।\\
	ಪರಮಾದ್ಯಾ ಗುರುಶ್ಶಕ್ತಿಃ (ಕೀರ್ತಿತಾ)ಕಿರ್ತನಪ್ರಿಯಾ ॥೧೧೮॥

ತ್ರೈಲೋಕ್ಯಮೋಹನಾಖ್ಯಾ ಚ ಸರ್ವಾಶಾಪರಿಪೂರಕಾ~।\\
ಸರ್ವಸಂಕ್ಷೋಭಿಣೀ ಪೂರ್ವಾಮ್ನಾಯಾ ಚಕ್ರತ್ರಯಾಲಯಾ ॥೧೧೯॥

	ಸರ್ವಸೌಭಾಗ್ಯದಾತ್ರೀ ಚ ಸರ್ವಾರ್ಥಸಾಧಕಪ್ರಿಯಾ~।\\
	ಸರ್ವರಕ್ಷಾಕರೀ ಸಾಧುರ್ದಕ್ಷಿಣಾಮ್ನಾಯದೇವತಾ ॥೧೨೦॥

ಮಧ್ಯಚಕ್ರೈಕನಿಲಯಾ ಪಶ್ಚಿಮಾಮ್ನಾಯದೇವತಾ~।\\
ನವಚಕ್ರಕೃತಾವಾಸಾ ಕೌಬೇರಾಮ್ನಾಯದೇವತಾ ॥೧೨೧॥

	ಬಿಂದುಚಕ್ರಕೃತಾಯಾಸಾ ಮಧ್ಯಸಿಂಹಾಸನೇಶ್ವರೀ~।\\
	ಶ್ರೀವಿದ್ಯಾ ನವದುರ್ಗಾ ಚ ಮಹಿಷಾಸುರಮರ್ದಿನೀ ॥೧೨೨॥

ಸರ್ವಸಾಮ್ರಾಜ್ಯಲಕ್ಷ್ಮೀಶ್ಚ ಅಷ್ಟಲಕ್ಷ್ಮೀಶ್ಚ ಸಂಶ್ರುತಾ~।\\
ಶೈಲೇಂದ್ರತನಯಾ ಜ್ಯೋತಿಃ ನಿಷ್ಕಲಾ ಶಾಂಭವೀ ಉಮಾ ॥೧೨೩॥

	ಅಜಪಾ ಮಾತೃಕಾ ಚೇತಿ ಶುಕ್ಲವರ್ಣಾ ಷಡಾನನಾ~।\\
	ಪಾರಿಜಾತೇಶ್ವರೀ ಚೈವ ತ್ರಿಕೂಟಾ ಪಂಚಬಾಣದಾ ॥೧೧೪॥

ಪಂಚಕಲ್ಪಲತಾ ಚೈವ ತ್ರ್ಯಕ್ಷರೀ ಮೂಲಪೀಠಿಕಾ~।\\
ಸುಧಾಶ್ರೀರಮೃತೇಶಾನೀ ಹ್ಯನ್ನಪೂರ್ಣಾ ಚ ಕಾಮಧುಕ್ ॥೧೨೫॥

	ಪಾಶಹಸ್ತಾ ಸಿದ್ಧಲಕ್ಷ್ಮೀಃ ಮಾತಂಗೀ ಭುವನೇಶ್ವರೀ~।\\
	ವಾರಾಹೀ ನವರತ್ನಾನಾಮೀಶ್ವರೀ ಚ ಪ್ರಕೀರ್ತಿತಾ ॥೧೨೬॥

ಪರಂ ಜ್ಯೋತಿಃ ಕೋಶರೂಪಾ ಸೈಂಧವೀ ಶಿವದರ್ಶನಾ~।\\
ಪರಾಪರಾ ಸ್ವಾಮಿನೀ ಚ ಶಾಕ್ತದರ್ಶನವಿಶ್ರುತಾ ॥೧೨೭॥

	ಬ್ರಹ್ಮದರ್ಶನರೂಪಾ ಚ ಶಿವದರ್ಶನರೂಪಿಣೀ~।\\
	ವಿಷ್ಣುದರ್ಶನರೂಪಾ ಚ ಸ್ರಷ್ಟೃದರ್ಶನರೂಪಿಣೀ ॥೧೨೮॥

ಸೌರದರ್ಶನರೂಪಾ ಚ ಸ್ಥಿತಿಚಕ್ರಕೃತಾಶ್ರಯಾ~।\\
ಬೌದ್ಧದರ್ಶನರೂಪಾ ಚ ತುರೀಯಾ ಬಹುರೂಪಿಣೀ ॥೧೨೯॥

	ತತ್ವಮುದ್ರಾಸ್ವರೂಪಾ ಚ ಪ್ರಸನ್ನಾ ಜ್ಞಾನಮಾತೃಕಾ~।\\
	ಸರ್ವೋಪಚಾರಸಂತುಷ್ಟಾ ಹೃನ್ಮಯೀ ಶೀರ್ಷದೇವತಾ ॥೧೩೦॥

ಶಿಖಾಸ್ಥಿತಾ ವರ್ಮಮಯೀ ನೇತ್ರತ್ರಯವಿಲಾಸಿನೀ~।\\
ಅಸ್ತ್ರಸ್ಥಾ ಚತುರಸ್ರಸ್ಥಾ ದ್ವಾರಸ್ಥಾ ದ್ವಾರದೇವತಾ ॥೧೩೧॥

	ಅಣಿಮಾ ಪಶ್ಚಿಮಸ್ಥಾ ಚ ದಕ್ಷಿಣದ್ವಾರದೇವತಾ~।\\
	ವಶಿತ್ವಾ ವಾಯುಕೋಣಸ್ಥಾ ಪ್ರಾಕಾಮ್ಯೇಶಾನದೇವತಾ ॥೧೩೨॥

ಮಹಿಮಾಪೂರ್ವನಾಥಾ ಚ ಲಘಿಮೋತ್ತರದೇವತಾ~।\\
ಅಗ್ನಿಕೋಣಸ್ಥಗರಿಮಾ ಪ್ರಾಪ್ತಿರ್ನೈಋತಿವಾಸಿನೀ ॥೧೩೩॥

	ಈಶಿತ್ವಸಿದ್ಧಿಸುರಥಾ ಸರ್ವಕಾಮೋರ್ಧ್ವವಾಸಿನೀ~।\\
	ಬ್ರಾಹ್ಮೀ ಮಾಹೇಶ್ವರೀ ಚೈವ ಕೌಮಾರೀ ವೈಷ್ಣವೀ ತಥಾ ॥೧೩೪॥

ವಾರಾಹ್ಯೈಂದ್ರೀ ಚ ಚಾಮುಂಡಾ ವಾಮಾ ಜ್ಯೇಷ್ಠಾ ಸರಸ್ವತೀ~।\\
ಕ್ಷೋಭಿಣೀ ದ್ರಾವಿಣೀ ರೌದ್ರೀ ಕಾಲ್ಯುನ್ಮಾದನಕಾರಿಣೀ ॥೧೩೫॥

	ಖೇಚರಾ ಕಾಲಕರಣೀ ಚ ಬಲಾನಾಂ ವಿಕರಣೀ ತಥಾ~।\\
	ಮನೋನ್ಮನೀ ಸರ್ವಭೂತದಮನೀ ಸರ್ವಸಿದ್ಧಿದಾ ॥೧೩೬॥

ಬಲಪ್ರಮಥಿನೀ ಶಕ್ತಿಃ ಬುದ್ಧ್ಯಾಕರ್ಷಣರೂಪಿಣೀ~।\\
ಅಹಂಕಾರಾಕರ್ಷಣೀ ಚ ಶಬ್ದಾಕರ್ಷಣರೂಪಿಣೀ ॥೧೩೭॥

	ಸ್ಪರ್ಶಾಕರ್ಷಣರೂಪಾ ಚ ರೂಪಾಕರ್ಷಣರೂಪಿಣೀ~।\\
	ರಸಾಕರ್ಷಣರೂಪಾ ಚ ಗಂಧಾಕರ್ಷಣರೂಪಿಣೀ ॥೧೩೮॥

ಚಿತ್ರಾಕರ್ಷಣರೂಪಾ ಚ ಧೈರ್ಯಾಕರ್ಷಣರೂಪಿಣೀ~।\\
ಸ್ಮೃತ್ಯಾಕರ್ಷಣರೂಪಾ ಚ ನಾಮಾಕರ್ಷಣರೂಪಿಣೀ ॥೧೩೯॥

	ಬೀಜಾಕರ್ಷಣರೂಪಾ ಚ ಹ್ಯಾತ್ಮಾಕರ್ಷಣರೂಪಿಣೀ~।\\
	ಅಮೃತಾಕರ್ಷಣೀ ಚೈವ ಶರೀರಾಕರ್ಷಣೀ ತಥಾ ॥೧೪೦॥

ಷೋಡಶಸ್ವರಸಂಪನ್ನಾ ಸ್ರವತ್ಪೀಯೂಷಮಂಡಿತಾ~।\\
ತ್ರಿಪುರೇಶೀ ಸಿದ್ಧಿದಾತ್ರೀ ಕಲಾದರ್ಶನವಾಸಿನೀ ॥೧೪೧॥

	ಸರ್ವಸಂಕ್ಷೋಭಚಕ್ರೇಶೀ ಶಕ್ತಿರ್ಗುಹ್ಯತರಾಭಿಧಾ~।\\
	ಅನಂಗಕುಸುಮಾಶಕ್ತಿಃ ತಥೈವಾನಂಗಮೇಖಲಾ ॥೧೪೨॥

ಅನಂಗಮದನಾಽನಂಗಮದನಾತುರರೂಪಿಣೀ~।\\
ಅನಂಗರೇಖಾ ಚಾನಂಗವೇಗಾನಂಗಾಕುಶಾಭಿಧಾ ॥೧೪೩॥

	ಅನಂಗಮಾಲಿನೀ ಚೈವ ಹ್ಯಷ್ಟವರ್ಗಾಧಿಗಾಮಿನೀ~।\\
	ವಸ್ವಷ್ಟಕಕೃತಾವಾಸಾ ಶ್ರೀಮತ್ತ್ರಿಪುರಸುಂದರೀ ॥೧೪೪॥

ಸರ್ವಸಾಮ್ರಾಜ್ಯಸುಭಗಾ ಸರ್ವಭಾಗ್ಯಪ್ರದೇಶ್ವರೀ~।\\
ಸಂಪ್ರದಾಯೇಶ್ವರೀ ಸರ್ವಸಂಕ್ಷೋಭಣಕರೀ ತಥಾ ॥೧೪೫॥

	ಸರ್ವವಿದ್ರಾವಿಣೀ ಸರ್ವಾಕರ್ಷಿಣೀ ರೂಪಕಾರಿಣೀ~।\\
	ಸರ್ವಾಹ್ಲಾದನಶಕ್ತಿಶ್ಚ ಸರ್ವಸಮ್ಮೋಹಿನೀ ತಥಾ ॥೧೪೬॥

ಸರ್ವಸ್ತಂಭನಶಕ್ತಿಶ್ಚ ಸರ್ವಜೃಂಭಣಕಾರಿಣೀ~।\\
ಸರ್ವವಶ್ಯಕಶಕ್ತಿಶ್ಚ ತಥಾ ಸರ್ವಾನುರಂಜನೀ ॥೧೪೭॥

	ಸರ್ವೋನ್ಮಾದನಶಕ್ತಿಶ್ಚ ತಥಾ ಸರ್ವಾರ್ಥಸಾಧಿಕಾ~।\\
	ಸರ್ವಸಂಪತ್ತಿದಾ ಚೈವ ಸರ್ವಮಾತೃಮಯೀ ತಥಾ ॥೧೪೮॥

ಸರ್ವದ್ವಂದ್ವಕ್ಷಯಕರೀ ಸಿದ್ಧಿಸ್ತ್ರಿಪುರವಾಸಿನೀ~।\\
ಚತುರ್ದಶಾರಚಕ್ರೇಶೀ ಕುಲಯೋಗಸಮನ್ವಯಾ ॥೧೪೯॥

	ಸರ್ವಸಿದ್ಧಿಪ್ರದಾ ಚೈವ ಸರ್ವಸಂಪತ್ಪ್ರದಾ ತಥಾ~।\\
	ಸರ್ವಪ್ರಿಯಕರೀ ಚೈವ ಸರ್ವಮಂಗಲಕಾರಿಣೀ ॥೧೫೦॥

ಸರ್ವಕಾಮಪ್ರಪೂರ್ಣಾ ಚ ಸರ್ವದುಃಖವಿಮೋಚಿನೀ~।\\
ಸರ್ವಮೃತ್ಯುಪ್ರಶಮನೀ ಸರ್ವ ವಿಘ್ನವಿನಾಶಿನೀ ॥೧೫೧॥

	ಸರ್ವಾಂಗಸುಂದರೀ ಚೈವ ಸರ್ವಸೌಭಾಗ್ಯದಾಯಿನೀ~।\\
	ತ್ರಿಪುರಾ ಶ್ರೀಶ್ಚ ಸರ್ವಾರ್ಥಸಾಧಿಕಾ ದಶಕೋಣಗಾ ॥೧೫೨॥

ಸರ್ವರಕ್ಷಾಕರೀ ಚೈವ ಈಶ್ವರೀ ಯೋಗಿನೀ ತಥಾ~।\\
ಸರ್ವಜ್ಞಾ ಸರ್ವಶಕ್ತಿಶ್ಚ ಸರ್ವೈಶ್ವರ್ಯಪ್ರದಾ ತಥಾ ॥೧೫೩॥

	ಸರ್ವಜ್ಞಾನಮಯೀ ಚೈವ ಸರ್ವವ್ಯಾಧಿವಿನಾಶಿನೀ~।\\
	ಸರ್ವಾಧಾರಸ್ವರೂಪಾ ಚ ಸರ್ವಪಾಪಹರಾ ತಥಾ ॥೧೫೪॥

ಸರ್ವಾನಂದಮಯೀ ಚೈವ ಸರ್ವರಕ್ಷಾಸ್ವರೂಪಿಣೀ~।\\
ತಥೈವ ಚ ಮಹಾಶಕ್ತಿಃ ಸರ್ವೇಪ್ಸಿತಫಲಪ್ರದಾ ॥೧೫೫॥

	ಅಂತರ್ದಶಾರಚಕ್ರಸ್ಥಾ ತಥಾ ತ್ರಿಪುರಮಾಲಿನೀ~।\\
	ಸರ್ವರೋಗಹರಾ ಚೈವ ರಹಸ್ಯಯೋಗಿನೀ ತಥಾ ॥೧೫೬॥

ವಾಗ್ದೇವೀ ವಶಿನೀ ಚೈವ ತಥಾ ಕಾಮೇಶ್ವರೀ ತಥಾ~।\\
ಮೋದಿನೀ ವಿಮಲಾ ಚೈವ ಹ್ಯರುಣಾ ಜಯಿನೀ ತಥಾ ॥೧೫೭॥

	ಶಿವಕಾಮಪ್ರದಾ ದೇವೀ ಶಿವಕಾಮಸ್ಯ ಸುಂದರೀ~।\\
	ಲಲಿತಾ ಲಲಿತಾಧ್ಯಾನಫಲದಾ ಶುಭಕಾರಿಣೀ ॥೧೫೮॥

ಸರ್ವೇಶ್ವರೀ ಕೌಲಿನೀ ಚ ವಸುವಂಶಾಭಿವರ್ದ್ಧಿನೀ~।\\
ಸರ್ವಕಾಮಪ್ರದಾ ಚೈವ ಪರಾಪರರಹಸ್ಯವಿತ್ ॥೧೫೯॥

	ತ್ರಿಕೋಣಚತುರಶ್ರಸ್ಥ ಕಾಮೇಶ್ವರ್ಯಾಯುಧಾತ್ಮಿಕಾ~।\\
	ಕಾಮೇಶ್ವರೀಬಾಣರೂಪಾ ಕಾಮೇಶೀ ಚಾಪರೂಪಿಣೀ ॥೧೬೦॥

ಕಾಮೇಶೀ ಪಾಶಹಸ್ತಾ ಚ ಕಾಮೇಶ್ಯಂಕುಶರೂಪಿಣೀ~।\\
ಕಾಮೇಶ್ವರೀ ರುದ್ರಶಕ್ತಿಃ ಅಗ್ನಿಚಕ್ರಕೃತಾಲಯಾ ॥೧೬೧॥

	ಕಾಮಾಭಿಂತ್ರಾ ಕಾಮದೋಗ್ಧ್ರೀ ಕಾಮದಾ ಚ ತ್ರಿಕೋಣಗಾ~।\\
	ದಕ್ಷಕೋಣೇಶ್ವರೀ ವಿಷ್ಣುಶಕ್ತಿರ್ಜಾಲಂಧರಾಲಯಾ ॥೧೬೨॥

ಸೂರ್ಯಚಕ್ರಾಲಯಾ ವಾಮಕೋಣಗಾ ಸೋಮಚಕ್ರಗಾ~।\\
ಭಗಮಾಲಾ ಬೃಹಚ್ಛಕ್ತಿ ಪೂರ್ಣಾ ಪೂರ್ವಾಸ್ರರಾಗಿಣೀ ॥೧೬೩॥

	ಶ್ರೀಮತ್ತ್ರಿಕೋಣಭುವನಾ ತ್ರಿಪುರಾಖ್ಯಾ ಮಹೇಶ್ವರೀ~।\\
	ಸರ್ವಾನಂದಮಯೀಶಾನೀ ಬಿಂದುಗಾತಿರಹಸ್ಯಗಾ ॥೧೬೪॥

ಪರಬ್ರಹ್ಮಸ್ವರೂಪಾ ಚ ಮಹಾತ್ರಿಪುರಸುಂದರೀ~।\\
ಸರ್ವಚಕ್ರಾಂತರಸ್ಥಾ ಚ ಸರ್ವಚಕ್ರಾಧಿದೇವತಾ ॥೧೬೫॥

	ಸರ್ವಚಕ್ರೇಶ್ವರೀ ಸರ್ವಮಂತ್ರಾಣಾಮೀಶ್ವರೀ ತಥಾ~।\\
	ಸರ್ವವಿದ್ಯೇಶ್ವರೀ ಚೈವ ಸರ್ವವಾಗೀಶ್ವರೀ ತಥಾ ॥೧೬೬॥

ಸರ್ವಯೋಗೇಶ್ವರೀ ಸರ್ವಪೀಠೇಶ್ವರ್ಯಖಿಲೇಶ್ವರೀ~।\\
ಸರ್ವಕಾಮೇಶ್ವರೀ ಸರ್ವತತ್ವೇಶ್ವರ್ಯಾಗಮೇಶ್ವರೀ ॥೧೬೭॥

	ಶಕ್ತಿಃ ಶಕ್ತಿಭೃದುಲ್ಲಾಸಾ ನಿರ್ದ್ವಂದ್ವಾದ್ವೈತಗರ್ಭಿಣೀ~।\\
	ನಿಷ್ಪ್ರಪಂಚಾ ಪ್ರಪಂಚಾಭಾ ಮಹಾಮಾಯಾ ಪ್ರಪಂಚಸೂಃ ॥೧೬೮॥

ಸರ್ವವಿಶ್ವೋತ್ಪತ್ತಿಧಾತ್ರೀ ಪರಮಾನಂದಕಾರಣಾ~।\\
ಲಾವಣ್ಯಸಿಂಧುಲಹರೀ ಸುಂದರೀತೋಷಮಂದಿರಾ ॥೧೬೯॥

	ಶಿವಕಾಮಸುಂದರೀ ದೇವೀ ಸರ್ವಮಂಗಲದಾಯಿನೀ~।\\
	ಇತಿನಾಮ್ನಾಂ ಸಹಸ್ರಂ ಚ ಗದಿತಂ ಇಷ್ಟದಾಯಕಂ ॥೧೭೦॥
\authorline{ಇತಿ ಶ್ರೀರುದ್ರಯಾಮಲೇ ಉಮಾಮಹೇಶಸಂವಾದೇ ಶ್ರೀಶಿವಕಾಮಸುಂದರ್ಯಾಃ ಶ್ರೀಮತ್ತ್ರಿಪುರಸುಂದರ್ಯಾಃ ಷೋಡಶಾರ್ಣಾಯಾಃ ತುರೀಯಸಹಸ್ರನಾಮಸ್ತೋತ್ರಂ ಸಂಪೂರ್ಣಂ ॥}
\section{ ಶ್ರೀಅನ್ನಪೂರ್ಣಾಸಹಸ್ರನಾಮಸ್ತೋತ್ರಂ }
ಓಂ ಅಸ್ಯ ಶ್ರೀಮದನ್ನಪೂರ್ಣಾಸಹಸ್ರನಾಮಸ್ತೋತ್ರಮಾಲಾಮಂತ್ರಸ್ಯ\\ ಶ್ರೀಭಗವಾನ್ ಋಷಿಃ । ಅನುಷ್ಟುಪ್ಛಂದಃ । ಪ್ರಕಟಗುಪ್ತಗುಪ್ತತರಸಂಪ್ರದಾಯಕುಲೋತ್ತೀರ್ಣನಿಗರ್ಭರಹಸ್ಯಾತಿರಹಸ್ಯಪರಾಪರಾತಿರಹಸ್ಯಾತಿಪೂರ್ವಾಚಿಂತ್ಯ\\ಪ್ರಭಾವಾ ಭಗವತೀ ಶ್ರೀಮದನ್ನಪೂರ್ಣಾ ದೇವತಾ । ಹಲೋ ಬೀಜಾನಿ । \\ಸ್ವರಾಃ ಶಕ್ತಯಃ । ಜೀವೋ ಬೀಜಂ । ಬುದ್ಧಿಃ ಶಕ್ತಿಃ । ಉದಾನೋ ಬೀಜಂ ।\\ ಸುಷುಮ್ನಾ ನಾಡೀ ಸರಸ್ವತೀ ಶಕ್ತಿಃ । ಜಪೇ ವಿನಿಯೋಗಃ ॥

\dhyana{ಅರ್ಕೋನ್ಮುಕ್ತಶಶಾಂಕಕೋಟಿವದನಾಮಾಪೀನತುಂಗಸ್ತನೀಂ\\
ಚಂದ್ರಾರ್ಧಾಂಕಿತಮಸ್ತಕಾಂ ಮಧುಮದಾಮಾಲೋಲನೇತ್ರತ್ರಯೀಂ~।\\
ಬಿಭ್ರಾಣಾಮನಿಶಂ ವರಂ ಜಪವಟೀಂ ಶೂಲಂ ಕಪಾಲಂ ಕರೈಃ\\
ಆದ್ಯಾಂ ಯೌವನಗರ್ವಿತಾಂ ಲಿಪಿತನುಂ ವಾಗೀಶ್ವರೀಮಾಶ್ರಯೇ ॥}
\newpage
ಅನ್ನಪೂರ್ಣಾ ಅನ್ನದಾತ್ರೀ ಅನ್ನರಾಶಿಕೃತಾಲಯಾ~।\\
ಅನ್ನದಾ ಅನ್ನರೂಪಾ ಚ ಅನ್ನದಾನರತೋತ್ಸವಾ ॥ ೧॥

ಅನಂತಾ ಚ ಅನಂತಾಕ್ಷೀ ಅನಂತಗುಣಶಾಲಿನೀ~।\\
ಅಚ್ಯುತಾ ಅಚ್ಯುತಪ್ರಾಣಾ ಅಚ್ಯುತಾನಂದಕಾರಿಣೀ ॥ ೨॥

ಅವ್ಯಕ್ತಾಽನಂತಮಹಿಮಾ ಅನಂತಸ್ಯ ಕುಲೇಶ್ವರೀ~।\\
ಅಬ್ಧಿಸ್ಥಾ ಅಬ್ಧಿಶಯನಾ ಅಬ್ಧಿಜಾ ಅಬ್ಧಿನಂದಿನೀ ॥ ೩॥

ಅಬ್ಜಸ್ಥಾ ಅಬ್ಜನಿಲಯಾ ಅಬ್ಜಜಾ ಅಬ್ಜಭೂಷಣಾ~।\\
ಅಬ್ಜಾಭಾ ಅಬ್ಜಹಸ್ತಾ ಚ ಅಬ್ಜಪತ್ರಶುಭೇಕ್ಷಣಾ ॥ ೪॥

ಅಬ್ಜಾನನಾ ಅನಂತಾತ್ಮಾ ಅಗ್ನಿಸ್ಥಾ ಅಗ್ನಿರೂಪಿಣೀ~।\\
ಅಗ್ನಿಜಾಯಾ ಅಗ್ನಿಮುಖೀ ಅಗ್ನಿಕುಂಡಕೃತಾಲಯಾ ॥ ೫॥

ಅಕಾರಾ ಅಗ್ನಿಮಾತಾ ಚ ಅಜಯಾಽದಿತಿನಂದಿನೀ~।\\
ಆದ್ಯಾ ಆದಿತ್ಯಸಂಕಾಶಾ ಆತ್ಮಜ್ಞಾ ಆತ್ಮಗೋಚರಾ ॥ ೬॥

ಆತ್ಮಸೂರಾತ್ಮದಯಿತಾ ಆಧಾರಾ ಆತ್ಮರೂಪಿಣೀ~।\\
ಆಶಾ ಆಕಾಶಪದ್ಮಸ್ಥಾ ಅವಕಾಶಸ್ವರೂಪಿಣೀ ॥ ೭॥

ಆಶಾಪುರೀ ಅಗಾಧಾ ಚ ಅಣಿಮಾದಿಸುಸೇವಿತಾ~।\\
ಅಂಬಿಕಾ ಅಬಲಾ ಅಂಬಾ ಅನಾದ್ಯಾ ಚ ಅಯೋನಿಜಾ ॥ ೮॥

ಅನೀಶಾ ಈಶಿಕಾ ಈಶಾ ಈಶಾನೀ ಈಶ್ವರಪ್ರಿಯಾ~।\\
ಈಶ್ವರೀ ಈಶ್ವರಪ್ರಾಣಾ ಈಶ್ವರಾನಂದದಾಯಿನೀ ॥ ೯॥

ಇಂದ್ರಾಣೀ ಇಂದ್ರದಯಿತಾ ಇಂದ್ರಸೂರಿಂದ್ರಪಾಲಿನೀ~।\\
ಇಂದಿರಾ ಇಂದ್ರಭಗಿನೀ ಇಂದ್ರಿಯಾ ಇಂದುಭೂಷಣಾ ॥ ೧೦॥

ಇಂದುಮಾತಾ ಇಂದುಮುಖೀ ಇಂದ್ರಿಯಾಣಾಂ ವಶಂಕರೀ~।\\
ಉಮಾ ಉಮಾಪತೇಃ ಪ್ರಾಣಾ ಓಡ್ಯಾಣಪೀಠವಾಸಿನೀ ॥ ೧೧॥

ಉತ್ತರಜ್ಞಾ ಉತ್ತರಾಖ್ಯಾ ಉಕಾರಾ ಉತ್ತರಾತ್ಮಿಕಾ~।\\
ಋಮಾತಾ ಋಭವಾ ಋಸ್ಥಾ ೠಌೃಕಾರಸ್ವರೂಪಿಣೀ ॥ ೧೨॥

ಋಕಾರಾ ಚ ಌಕಾರಾ ಚ ಌತಕಪ್ರೀತಿದಾಯಿನೀ~।\\
ಏಕಾ ಚ ಏಕವೀರಾ ಚ ಏಕಾರೈಕಾರರೂಪಿಣೀ ॥ ೧೩॥

ಓಕಾರೀ ಓಘರೂಪಾ ಚ ಓಘತ್ರಯಸುಪೂಜಿತಾ~।\\
ಓಘಸ್ಥಾ ಓಘಸಂಭೂತಾ ಓಘಧಾತ್ರೀ ಚ ಓಘಸೂಃ ॥ ೧೪॥

ಷೋಡಶಸ್ವರಸಂಭೂತಾ ಷೋಡಶಸ್ವರರೂಪಿಣೀ~।\\
ವರ್ಣಾತ್ಮಾ ವರ್ಣನಿಲಯಾ ಶೂಲಿನೀ ವರ್ಣಮಾಲಿನೀ ॥ ೧೫॥

ಕಾಲರಾತ್ರಿರ್ಮಹಾರಾತ್ರಿರ್ಮೋಹರಾತ್ರಿಃ ಸುಲೋಚನಾ~।\\
ಕಾಲೀ ಕಪಾಲಿನೀ ಕೃತ್ಯಾ ಕಾಲಿಕಾ ಸಿಂಹಗಾಮಿನೀ ॥ ೧೬॥

ಕಾತ್ಯಾಯನೀ ಕಲಾಧಾರಾ ಕಾಲದೈತ್ಯನಿಕೃಂತನೀ~।\\
ಕಾಮಿನೀ ಕಾಮವಂದ್ಯಾ ಚ ಕಮನೀಯಾ ವಿನೋದಿನೀ ॥ ೧೭॥

ಕಾಮಸೂಃ ಕಾಮವನಿತಾ ಕಾಮಧುಕ್ ಕಮಲಾವತೀ~।\\
ಕಾಮದಾತ್ರೀ ಕರಾಲೀ ಚ ಕಾಮಕೇಲಿವಿನೋದಿನೀ ॥ ೧೮॥

ಕಾಮನಾ ಕಾಮದಾ ಕಾಮ್ಯಾ ಕಮಲಾ ಕಮಲಾರ್ಚಿತಾ~।\\
ಕಾಶ್ಮೀರಲಿಪ್ತವಕ್ಷೋಜಾ ಕಾಶ್ಮೀರದ್ರವಚರ್ಚಿತಾ ॥ ೧೯॥

ಕನಕಾ ಕನಕಪ್ರಾಣಾ ಕನಕಾಚಲವಾಸಿನೀ~।\\
ಕನಕಾಭಾ ಕಾನನಸ್ಥಾ ಕಾಮಾಖ್ಯಾ ಕನಕಪ್ರದಾ ॥ ೨೦॥

ಕಾಮಪೀಠಸ್ಥಿತಾ ನಿತ್ಯಾ ಕಾಮಧಾಮನಿವಾಸಿನೀ~।\\
ಕಂಬುಕಂಠೀ ಕರಾಲಾಕ್ಷೀ ಕಿಶೋರೀ ಚ ಕಲಾಪಿನೀ ॥ ೨೧॥

ಕಲಾ ಕಾಷ್ಠಾ ನಿಮೇಷಾ ಚ ಕಾಲಸ್ಥಾ ಕಾಲರೂಪಿಣೀ~।\\
ಕಾಲಜ್ಞಾ ಕಾಲಮಾತಾ ಚ ಕಾಲಧಾತ್ರೀ ಕಲಾವತೀ ॥ ೨೨॥

ಕಾಲದಾ ಕಾಲಹಾ ಕುಲ್ಯಾ ಕುರುಕುಲ್ಲಾ ಕುಲಾಂಗನಾ~।\\
ಕೀರ್ತಿದಾ ಕೀರ್ತಿಹಾ ಕೀರ್ತಿಃ ಕೀರ್ತಿಸ್ಥಾ ಕೀರ್ತಿವರ್ಧನೀ ॥ ೨೩॥

ಕೀರ್ತಿಜ್ಞಾ ಕೀರ್ತಿತಪದಾ ಕೃತ್ತಿಕಾ ಕೇಶವಪ್ರಿಯಾ~।\\
ಕೇಶಿಹಾ ಕೇಲಿಕಾರೀ ಚ ಕೇಶವಾನಂದಕಾರಿಣೀ ॥ ೨೪॥

ಕುಮುದಾಭಾ ಕುಮಾರೀ ಚ ಕರ್ಮದಾ ಕಮಲೇಕ್ಷಣಾ~।\\
ಕೌಮುದೀ ಕುಮುದಾನಂದಾ ಕೌಲಿನೀ ಚ ಕುಮುದ್ವತೀ ॥ ೨೫॥

ಕೋದಂಡಧಾರಿಣೀ ಕ್ರೋಧಾ ಕೂಟಸ್ಥಾ ಕೋಟರಾಶ್ರಯಾ~।\\
ಕಾಲಕಂಠೀ ಕರಾಲಾಂಗೀ ಕಾಲಾಂಗೀ ಕಾಲಭೂಷಣಾ ॥ ೨೬॥

ಕಂಕಾಲೀ ಕಾಮದಾಮಾ ಚ ಕಂಕಾಲಕೃತಭೂಷಣಾ~।\\
ಕಪಾಲಕರ್ತ್ರಿಕಕರಾ ಕರವೀರಸ್ವರೂಪಿಣೀ ॥ ೨೭॥

ಕಪರ್ದಿನೀ ಕೋಮಲಾಂಗೀ ಕೃಪಾಸಿಂಧುಃ ಕೃಪಾಮಯೀ~।\\
ಕುಶಾವತೀ ಕುಂಡಸಂಸ್ಥಾ ಕೌಬೇರೀ ಕೌಶಿಕೀ ತಥಾ ॥ ೨೮॥

ಕಾಶ್ಯಪೀ ಕದ್ರುತನಯಾ ಕಲಿಕಲ್ಮಷನಾಶಿನೀ~।\\
ಕಂಜಸ್ಥಾ ಕಂಜವದನಾ ಕಂಜಕಿಂಜಲ್ಕಚರ್ಚಿತಾ ॥ ೨೯॥

ಕಂಜಾಭಾ ಕಂಜಮಧ್ಯಸ್ಥಾ ಕಂಜನೇತ್ರಾ ಕಚೋದ್ಭವಾ~।\\
ಕಾಮರೂಪಾ ಚ ಹ್ರೀಂಕಾರೀ ಕಶ್ಯಪಾನ್ವಯವರ್ಧಿನೀ ॥ ೩೦॥

ಖರ್ವಾ ಚ ಖಂಜನದ್ವಂದ್ವಲೋಚನಾ ಖರ್ವವಾಹಿನೀ~।\\
ಖಡ್ಗಿನೀ ಖಡ್ಗಹಸ್ತಾ ಚ ಖೇಚರೀ ಖಡ್ಗರೂಪಿಣೀ ॥ ೩೧॥

ಖಗಸ್ಥಾ ಖಗರೂಪಾ ಚ ಖಗಗಾ ಖಗಸಂಭವಾ~।\\
ಖಗಧಾತ್ರೀ ಖಗಾನಂದಾ ಖಗಯೋನಿಸ್ವರೂಪಿಣೀ ॥ ೩೨॥

ಖಗೇಶೀ ಖೇಟಕಕರಾ ಖಗಾನಂದವಿವರ್ಧಿನೀ~।\\
ಖಗಮಾನ್ಯಾ ಖಗಾಧಾರಾ ಖಗಗರ್ವವಿಮೋಚಿನೀ ॥ ೩೩॥

ಗಂಗಾ ಗೋದಾವರೀ ಗೀತಿರ್ಗಾಯತ್ರೀ ಗಗನಾಲಯಾ~।\\
ಗೀರ್ವಾಣಸುಂದರೀ ಗೌಶ್ಚ ಗಾಧಾ ಗೀರ್ವಾಣಪೂಜಿತಾ ॥ ೩೪॥

ಗೀರ್ವಾಣಚರ್ಚಿತಪದಾ ಗಾಂಧಾರೀ ಗೋಮತೀ ತಥಾ~।\\
ಗರ್ವಿಣೀ ಗರ್ವಹಂತ್ರೀ ಚ ಗರ್ಭಸ್ಥಾ ಗರ್ಭಧಾರಿಣೀ ॥ ೩೫॥

ಗರ್ಭದಾ ಗರ್ಭಹಂತ್ರೀ ಚ ಗಂಧರ್ವಕುಲಪೂಜಿತಾ~।\\
ಗಯಾ ಗೌರೀ ಚ ಗಿರಿಜಾ ಗಿರಿಸ್ಥಾ ಗಿರಿಸಂಭವಾ ॥ ೩೬॥

ಗಿರಿಗಹ್ವರಮಧ್ಯಸ್ಥಾ ಕುಂಜರೇಶ್ವರಗಾಮಿನೀ~।\\
ಕಿರೀಟಿನೀ ಚ ಗದಿನೀ ಗುಂಜಾಹಾರವಿಭೂಷಣಾ ॥ ೩೭॥

ಗಣಪಾ ಗಣಕಾ ಗಣ್ಯಾ ಗಣಕಾನಂದಕಾರಿಣೀ~।\\
ಗಣಪೂಜ್ಯಾ ಚ ಗೀರ್ವಾಣೀ ಗಣಪಾನಂದಕಾರಿಣೀ ॥ ೩೮॥

ಗುರುಮಾತಾ ಗುರುರತಾ ಗುರುಭಕ್ತಿಪರಾಯಣಾ~।\\
ಗೋತ್ರಾ ಗೌಃ ಕೃಷ್ಣಭಗಿನೀ ಕೃಷ್ಣಸೂಃ ಕೃಷ್ಣನಂದಿನೀ ॥ ೩೯॥

ಗೋವರ್ಧನೀ ಗೋತ್ರಧರಾ ಗೋವರ್ಧನಕೃತಾಲಯಾ~।\\
ಗೋವರ್ಧನಧರಾ ಗೋದಾ ಗೌರಾಂಗೀ ಗೌತಮಾತ್ಮಜಾ ॥ ೪೦॥

ಘರ್ಘರಾ ಘೋರರೂಪಾ ಚ ಘೋರಾ ಘರ್ಘರನಾದಿನೀ~।\\
ಶ್ಯಾಮಾ ಘನರವಾಽಘೋರಾ ಘನಾ ಘೋರಾರ್ತಿನಾಶಿನೀ ॥ ೪೧॥

ಘನಸ್ಥಾ ಚ ಘನಾನಂದಾ ದಾರಿದ್ರ್ಯಘನನಾಶಿನೀ~।\\
ಚಿತ್ತಜ್ಞಾ ಚಿಂತಿತಪದಾ ಚಿತ್ತಸ್ಥಾ ಚಿತ್ತರೂಪಿಣೀ ॥ ೪೨॥

ಚಕ್ರಿಣೀ ಚಾರುಚಂಪಾಭಾ ಚಾರುಚಂಪಕಮಾಲಿನೀ~।\\
ಚಂದ್ರಿಕಾ ಚಂದ್ರಕಾಂತಿಶ್ಚ ಚಾಪಿನೀ ಚಂದ್ರಶೇಖರಾ ॥ ೪೩॥

ಚಂಡಿಕಾ ಚಂಡದೈತ್ಯಘ್ನೀ ಚಂದ್ರಶೇಖರವಲ್ಲಭಾ~।\\
ಚಾಂಡಾಲಿನೀ ಚ ಚಾಮುಂಡಾ ಚಂಡಮುಂಡವಧೋದ್ಯತಾ ॥ ೪೪॥

ಚೈತನ್ಯಭೈರವೀ ಚಂಡಾ ಚೈತನ್ಯಘನಗೇಹಿನೀ~।\\
ಚಿತ್ಸ್ವರೂಪಾ ಚಿದಾಧಾರಾ ಚಂಡವೇಗಾ ಚಿದಾಲಯಾ ॥ ೪೫॥

ಚಂದ್ರಮಂಡಲಮಧ್ಯಸ್ಥಾ ಚಂದ್ರಕೋಟಿಸುಶೀತಲಾ~।\\
ಚಪಲಾ ಚಂದ್ರಭಗಿನೀ ಚಂದ್ರಕೋಟಿನಿಭಾನನಾ ॥ ೪೬॥

ಚಿಂತಾಮಣಿಗುಣಾಧಾರಾ ಚಿಂತಾಮಣಿವಿಭೂಷಣಾ~।\\
ಭಕ್ತಚಿಂತಾಮಣಿಲತಾ ಚಿಂತಾಮಣಿಕೃತಾಲಯಾ ॥ ೪೭॥

ಚಾರುಚಂದನಲಿಪ್ತಾಂಗೀ ಚತುರಾ ಚ ಚತುರ್ಮುಖೀ~।\\
ಚೈತನ್ಯದಾ ಚಿದಾನಂದಾ ಚಾರುಚಾಮರವೀಜಿತಾ ॥ ೪೮॥

ಛತ್ರದಾ ಛತ್ರಧಾರೀ ಚ ಛಲಚ್ಛದ್ಮವಿನಾಶಿನೀ~।\\
ಛತ್ರಹಾ ಛತ್ರರೂಪಾ ಚ ಛತ್ರಚ್ಛಾಯಾಕೃತಾಲಯಾ ॥ ೪೯॥

ಜಗಜ್ಜೀವಾ ಜಗದ್ಧಾತ್ರೀ ಜಗದಾನಂದಕಾರಿಣೀ~।\\
ಯಜ್ಞಪ್ರಿಯಾ ಯಜ್ಞರತಾ ಜಪಯಜ್ಞಪರಾಯಣಾ ॥ ೫೦॥

ಜನನೀ ಜಾನಕೀ ಯಜ್ವಾ ಯಜ್ಞಹಾ ಯಜ್ಞನಂದಿನೀ~।\\
ಯಜ್ಞದಾ ಯಜ್ಞಫಲದಾ ಯಜ್ಞಸ್ಥಾನಕೃತಾಲಯಾ ॥ ೫೧॥

ಯಜ್ಞಭೋಕ್ತ್ರೀ ಯಜ್ಞರೂಪಾ ಯಜ್ಞವಿಘ್ನವಿನಾಶಿನೀ~।\\
ಜಪಾಕುಸುಮಸಂಕಾಶಾ ಜಪಾಕುಸುಮಶೋಭಿತಾ ॥ ೫೨॥

ಜಾಲಂಧರೀ ಜಯಾ ಜೈತ್ರೀ ಜೀಮೂತಚಯಭಾಷಿಣೀ~।\\
ಜಯದಾ ಜಯರೂಪಾ ಚ ಜಯಸ್ಥಾ ಜಯಕಾರಿಣೀ ॥ ೫೩॥

ಜಗದೀಶಪ್ರಿಯಾ ಜೀವಾ ಜಲಸ್ಥಾ ಜಲಜೇಕ್ಷಣಾ~।\\
ಜಲರೂಪಾ ಜಹ್ನುಕನ್ಯಾ ಯಮುನಾ ಜಲಜೋದರೀ ॥ ೫೪॥

ಜಲಜಾಸ್ಯಾ ಜಾಹ್ನವೀ ಚ ಜಲಜಾಭಾ ಜಲೋದರೀ~।\\
ಯದುವಂಶೋದ್ಭವಾ ಜೀವಾ ಯಾದವಾನಂದಕಾರಿಣೀ ॥ ೫೫॥

ಯಶೋದಾ ಯಶಸಾಂ ರಾಶಿರ್ಯಶೋದಾನಂದಕಾರಿಣೀ~।\\
ಜ್ವಲಿನೀ ಜ್ವಾಲಿನೀ ಜ್ವಾಲಾ ಜ್ವಲತ್ಪಾವಕಸನ್ನಿಭಾ ॥ ೫೬॥

ಜ್ವಾಲಾಮುಖೀ ಜಗನ್ಮಾತಾ ಯಮಲಾರ್ಜುನಭಂಜನೀ~।\\
ಜನ್ಮದಾ ಜನ್ಮಹಾ ಜನ್ಯಾ ಜನ್ಮಭೂರ್ಜನಕಾತ್ಮಜಾ ॥ ೫೭॥

ಜನಾನಂದಾ ಜಾಂಬವತೀ ಜಂಬೂದ್ವೀಪಕೃತಾಲಯಾ~।\\
ಜಾಂಬೂನದಸಮಾನಾಭಾ ಜಾಂಬೂನದವಿಭೂಷಣಾ ॥ ೫೮॥

ಜಂಭಹಾ ಜಾತಿದಾ ಜಾತಿರ್ಜ್ಞಾನದಾ ಜ್ಞಾನಗೋಚರಾ~।\\
ಜ್ಞಾನರೂಪಾಽಜ್ಞಾನಹಾ ಚ ಜ್ಞಾನವಿಜ್ಞಾನಶಾಲಿನೀ ॥ ೫೯॥

ಜಿನಜೈತ್ರೀ ಜಿನಾಧಾರಾ ಜಿನಮಾತಾ ಜಿನೇಶ್ವರೀ~।\\
ಜಿತೇಂದ್ರಿಯಾ ಜನಾಧಾರಾ ಅಜಿನಾಂಬರಧಾರಿಣೀ ॥ ೬೦॥

ಶಂಭುಕೋಟಿದುರಾಧರ್ಷಾ ವಿಷ್ಣುಕೋಟಿವಿಮರ್ದಿನೀ~।\\
ಸಮುದ್ರಕೋಟಿಗಂಭೀರಾ ವಾಯುಕೋಟಿಮಹಾಬಲಾ ॥ ೬೧॥

ಸೂರ್ಯಕೋಟಿಪ್ರತೀಕಾಶಾ ಯಮಕೋಟಿದುರಾಪಹಾ~।\\
ಕಾಮಧುಕ್ಕೋಟಿಫಲದಾ ಶಕ್ರಕೋಟಿಸುರಾಜ್ಯದಾ ॥ ೬೨॥

ಕಂದರ್ಪಕೋಟಿಲಾವಣ್ಯಾ ಪದ್ಮಕೋಟಿನಿಭಾನನಾ~।\\
ಪೃಥ್ವೀಕೋಟಿಜನಾಧಾರಾ ಅಗ್ನಿಕೋಟಿಭಯಂಕರೀ ॥ ೬೩॥

ಅಣಿಮಾ ಮಹಿಮಾ ಪ್ರಾಪ್ತಿರ್ಗರಿಮಾ ಲಘಿಮಾ ತಥಾ~।\\
ಪ್ರಾಕಾಮ್ಯದಾ ವಶಕರೀ ಈಶಿಕಾ ಸಿದ್ಧಿದಾ ತಥಾ ॥ ೬೪॥

ಮಹಿಮಾದಿಗುಣೋಪೇತಾ ಅಣಿಮಾದ್ಯಷ್ಟಸಿದ್ಧಿದಾ~।\\
ಜವನಘ್ನೀ ಜನಾಧೀನಾ ಜಾಮಿನೀ ಚ ಜರಾಪಹಾ ॥ ೬೫॥

ತಾರಿಣೀ ತಾರಿಕಾ ತಾರಾ ತೋತಲಾ ತುಲಸೀಪ್ರಿಯಾ~।\\
ತಂತ್ರಿಣೀ ತಂತ್ರರೂಪಾ ಚ ತಂತ್ರಜ್ಞಾ ತಂತ್ರಧಾರಿಣೀ ॥ ೬೬॥

ತಾರಹಾರಾ ಚ ತುಲಜಾ ಡಾಕಿನೀತಂತ್ರಗೋಚರಾ~।\\
ತ್ರಿಪುರಾ ತ್ರಿದಶಾ ತ್ರಿಸ್ಥಾ ತ್ರಿಪುರಾಸುರಘಾತಿನೀ ॥ ೬೭॥

ತ್ರಿಗುಣಾ ಚ ತ್ರಿಕೋಣಸ್ಥಾ ತ್ರಿಮಾತ್ರಾ ತ್ರಿತನುಸ್ಥಿತಾ~।\\
ತ್ರೈವಿದ್ಯಾ ಚ ತ್ರಯೀ ತ್ರಿಘ್ನೀ ತುರೀಯಾ ತ್ರಿಪುರೇಶ್ವರೀ ॥ ೬೮॥

ತ್ರಿಕೋದರಸ್ಥಾ ತ್ರಿವಿಧಾ ತ್ರೈಲೋಕ್ಯಾ ತ್ರಿಪುರಾತ್ಮಿಕಾ~।\\
ತ್ರಿಧಾಮ್ನೀ ತ್ರಿದಶಾರಾಧ್ಯಾ ತ್ರ್ಯಕ್ಷಾ ತ್ರಿಪುರವಾಸಿನೀ ॥ ೬೯॥

ತ್ರಿವರ್ಣೀ ತ್ರಿಪದೀ ತಾರಾ ತ್ರಿಮೂರ್ತಿಜನನೀ ತ್ವರಾ~।\\
ತ್ರಿದಿವಾ ತ್ರಿದಿವೇಶಾಽಽದಿದೇವೀ ತ್ರೈಲೋಕ್ಯಧಾರಿಣೀ ॥ ೭೦॥

ತ್ರಿಮೂರ್ತಿಶ್ಚ ತ್ರಿಜನನೀ ತ್ರೀಭೂಸ್ತ್ರೀಪುರಸುಂದರೀ~।\\
ತಪಸ್ವಿನೀ ತಪೋನಿಷ್ಠಾ ತರುಣೀ ತಾರರೂಪಿಣೀ ॥ ೭೧॥

ತಾಮಸೀ ತಾಪಸೀ ಚೈವ ತಾಪಘ್ನೀ ಚ ತಮೋಪಹಾ~।\\
ತರುಣಾರ್ಕಪ್ರತೀಕಾಶಾ ತಪ್ತಕಾಂಚನಸನ್ನಿಭಾ ॥ ೭೨॥

ಉನ್ಮಾದಿನೀ ತಂತುರೂಪಾ ತ್ರೈಲೋಕ್ಯವ್ಯಾಪಿನೀಶ್ವರೀ~।\\
ತಾರ್ಕಿಕೀ ತರ್ಕವಿದ್ಯಾ ಚ ತಾಪತ್ರಯವಿನಾಶಿನೀ ॥ ೭೩॥

ತ್ರಿಪುಷ್ಕರಾ ತ್ರಿಕಾಲಜ್ಞಾ ತ್ರಿಸಂಧ್ಯಾ ಚ ತ್ರಿಲೋಚನಾ~।\\
ತ್ರಿವರ್ಗಾ ಚ ತ್ರಿವರ್ಗಸ್ಥಾ ತಪಸಸ್ಸಿದ್ಧಿದಾಯಿನೀ ॥ ೭೪॥

ಅಧೋಕ್ಷಜಾ ಅಯೋಧ್ಯಾ ಚ ಅಪರ್ಣಾ ಚ ಅವಂತಿಕಾ~।\\
ಕಾರಿಕಾ ತೀರ್ಥರೂಪಾ ಚ ತೀರಾ ತೀರ್ಥಕರೀ ತಥಾ ॥ ೭೫॥

ದಾರಿದ್ರ್ಯದುಃಖದಲಿನೀ ಅದೀನಾ ದೀನವತ್ಸಲಾ~।\\
ದೀನನಾಥಪ್ರಿಯಾ ದೀರ್ಘಾ ದಯಾಪೂರ್ಣಾ ದಯಾತ್ಮಿಕಾ ॥ ೭೬॥

ದೇವದಾನವಸಂಪೂಜ್ಯಾ ದೇವಾನಾಂ ಪ್ರಿಯಕಾರಿಣೀ~।\\
ದಕ್ಷಪುತ್ರೀ ದಕ್ಷಮಾತಾ ದಕ್ಷಯಜ್ಞವಿನಾಶಿನೀ ॥ ೭೭॥

ದೇವಸೂರ್ದಕ್ಷಿಣಾ ದಕ್ಷಾ ದುರ್ಗಾ ದುರ್ಗತಿನಾಶಿನೀ~।\\
ದೇವಕೀಗರ್ಭಸಂಭೂತಾ ದುರ್ಗದೈತ್ಯವಿನಾಶಿನೀ ॥ ೭೮॥

ಅಟ್ಟಾಽಟ್ಟಹಾಸಿನೀ ದೋಲಾ ದೋಲಾಕರ್ಮಾಭಿನಂದಿನೀ~।\\
ದೇವಕೀ ದೇವಿಕಾ ದೇವೀ ದುರಿತಘ್ನೀ ತಟಿತ್ತಥಾ ॥ ೭೯॥

ಗಂಡಕೀ ಗಲ್ಲಕೀ ಕ್ಷಿಪ್ರಾ ದ್ವಾರಾ ದ್ವಾರವತೀ ತಥಾ~।\\
ಆನಂದೋದಧಿಮಧ್ಯಸ್ಥಾ ಕಟಿಸೂತ್ರೈರಲಂಕೃತಾ ॥ ೮೦॥

ಘೋರಾಗ್ನಿದಾಹದಮನೀ ದುಃಖದುಸ್ಸ್ವಪ್ನನಾಶಿನೀ~।\\
ಶ್ರೀಮಯೀ ಶ್ರೀಮತೀ ಶ್ರೇಷ್ಠಾ ಶ್ರೀಕರೀ ಶ್ರೀವಿಭಾವಿನೀ ॥ ೮೧॥

ಶ್ರೀದಾ ಶ್ರೀಶಾ ಶ್ರೀನಿವಾಸಾ ಶ್ರೀಮತೀ ಶ್ರೀರ್ಮತಿರ್ಗತಿಃ~।\\
ಧನದಾ ದಾಮಿನೀ ದಾಂತಾ ಧರ್ಮದಾ ಧನಶಾಲಿನೀ ॥ ೮೨॥

ದಾಡಿಮೀಪುಷ್ಪಸಂಕಾಶಾ ಧನಾಗಾರಾ ಧನಂಜಯಾ~।\\
ಧೂಮ್ರಾಭಾ ಧೂಮ್ರದೈತ್ಯಘ್ನೀ ಧವಲಾ ಧವಲಪ್ರಿಯಾ ॥ ೮೩॥

ಧೂಮ್ರವಕ್ತ್ರಾ ಧೂಮ್ರನೇತ್ರಾ ಧೂಮ್ರಕೇಶೀ ಚ ಧೂಸರಾ~।\\
ಧರಣೀ ಧಾರಿಣೀ ಧೈರ್ಯಾ ಧರಾ ಧಾತ್ರೀ ಚ ಧೈರ್ಯದಾ ॥ ೮೪॥

ದಮನೀ ಧರ್ಮಿಣೀ ಧೂಶ್ಚ ದಯಾ ದೋಗ್ಧ್ರೀ ದುರಾಸದಾ~।\\
ನಾರಾಯಣೀ ನಾರಸಿಂಹೀ ನೃಸಿಂಹಹೃದಯಾಲಯಾ ॥ ೮೫॥

ನಾಗಿನೀ ನಾಗಕನ್ಯಾ ಚ ನಾಗಸೂರ್ನಾಗನಾಯಿಕಾ~।\\
ನಾನಾರತ್ನವಿಚಿತ್ರಾಂಗೀ ನಾನಾಭರಣಮಂಡಿತಾ ॥ ೮೬॥

ದುರ್ಗಸ್ಥಾ ದುರ್ಗರೂಪಾ ಚ ದುಃಖದುಷ್ಕೃತನಾಶಿನೀ~।\\
ಹ್ರೀಂಕಾರೀ ಚೈವ ಶ್ರೀಂಕಾರೀ ಹುಂಕಾರೀ ಕ್ಲೇಶನಾಶಿನೀ ॥ ೮೭॥

ನಗಾತ್ಮಜಾ ನಾಗರೀ ಚ ನವೀನಾ ನೂತನಪ್ರಿಯಾ~।\\
ನೀರಜಾಸ್ಯಾ ನೀರದಾಭಾ ನವಲಾವಣ್ಯಸುಂದರೀ ॥ ೮೮॥

ನೀತಿಜ್ಞಾ ನೀತಿದಾ ನೀತಿರ್ನಿಮ್ನಾಭಿರ್ನಗೇಶ್ವರೀ~।\\
ನಿಷ್ಠಾ ನಿತ್ಯಾ ನಿರಾತಂಕಾ ನಾಗಯಜ್ಞೋಪವೀತಿನೀ ॥ ೮೯॥

ನಿಧಿದಾ ನಿಧಿರೂಪಾ ಚ ನಿರ್ಗುಣಾ ನರವಾಹಿನೀ~।\\
ನರಮಾಂಸರತಾ ನಾರೀ ನರಮುಂಡವಿಭೂಷಣಾ ॥ ೯೦॥

ನಿರಾಧಾರಾ ನಿರ್ವಿಕಾರಾ ನುತಿರ್ನಿರ್ವಾಣಸುಂದರೀ~।\\
ನರಾಸೃಕ್ಪಾನಮತ್ತಾ ಚ ನಿರ್ವೈರಾ ನಾಗಗಾಮಿನೀ ॥ ೯೧॥

ಪರಮಾ ಪ್ರಮಿತಾ ಪ್ರಾಜ್ಞಾ ಪಾರ್ವತೀ ಪರ್ವತಾತ್ಮಜಾ~।\\
ಪರ್ವಪ್ರಿಯಾ ಪರ್ವರತಾ ಪರ್ವಪಾವನಪಾವನೀ ॥ ೯೨॥

ಪರಾತ್ಪರತರಾ ಪೂರ್ವಾ ಪಶ್ಚಿಮಾ ಪಾಪನಾಶಿನೀ~।\\
ಪಶೂನಾಂ ಪತಿಪತ್ನೀ ಚ ಪತಿಭಕ್ತಿಪರಾಯಣಾ ॥ ೯೩॥

ಪರೇಶೀ ಪಾರಗಾ ಪಾರಾ ಪರಂಜ್ಯೋತಿಸ್ವರೂಪಿಣೀ~।\\
ನಿಷ್ಠುರಾ ಕ್ರೂರಹೃದಯಾ ಪರಾಸಿದ್ಧಿಃ ಪರಾಗತಿಃ ॥ ೯೪॥

ಪಶುಘ್ನೀ ಪಶುರೂಪಾ ಚ ಪಶುಹಾ ಪಶುವಾಹಿನೀ~।\\
ಪಿತಾ ಮಾತಾ ಚ ಯಂತ್ರೀ ಚ ಪಶುಪಾಶವಿನಾಶಿನೀ ॥ ೯೫॥

ಪದ್ಮಿನೀ ಪದ್ಮಹಸ್ತಾ ಚ ಪದ್ಮಕಿಂಜಲ್ಕವಾಸಿನೀ~।\\
ಪದ್ಮವಕ್ತ್ರಾ ಚ ಪದ್ಮಾಕ್ಷೀ ಪದ್ಮಸ್ಥಾ ಪದ್ಮಸಂಭವಾ ॥ ೯೬॥

ಪದ್ಮಾಸ್ಯಾ ಪಂಚಮೀ ಪೂರ್ಣಾ ಪೂರ್ಣಪೀಠನಿವಾಸಿನೀ~।\\
ಪದ್ಮರಾಗಪ್ರತೀಕಾಶಾ ಪಾಂಚಾಲೀ ಪಂಚಮಪ್ರಿಯಾ ॥ ೯೭॥

ಪರಬ್ರಹ್ಮಸ್ವರೂಪಾ ಚ ಪರಬ್ರಹ್ಮನಿವಾಸಿನೀ~।\\
ಪರಮಾನಂದಮುದಿತಾ ಪರಚಕ್ರನಿವಾಸಿನೀ ॥ ೯೮॥

ಪರೇಶೀ ಪರಮಾ ಪೃಥ್ವೀ ಪೀನತುಂಗಪಯೋಧರಾ~।\\
ಪರಾಪರಾ ಪರಾವಿದ್ಯಾ ಪರಮಾನಂದದಾಯಿನೀ ॥ ೯೯॥

ಪೂಜ್ಯಾ ಪ್ರಜ್ಞಾವತೀ ಪುಷ್ಟಿಃ ಪಿನಾಕಿಪರಿಕೀರ್ತಿತಾ~।\\
ಪ್ರಾಣಘ್ನೀ ಪ್ರಾಣರೂಪಾ ಚ ಪ್ರಾಣದಾ ಚ ಪ್ರಿಯಂವದಾ ॥ ೧೦೦॥

ಫಣಿಭೂಷಾ ಫಣಾವೇಶೀ ಫಕಾರಾಕುಂಠಮಾಲಿನೀ~।\\
ಫಣಿರಾಡ್ವೃತಸರ್ವಾಂಗೀ ಫಲಭಾಗನಿವಾಸಿನೀ ॥ ೧೦೧॥

ಬಲಭದ್ರಸ್ಯ ಭಗಿನೀ ಬಾಲಾ ಬಾಲಪ್ರದಾಯಿನೀ~।\\
ಫಲ್ಗುರುಪಾ ಪ್ರಲಂಬಧ್ನೀ ಫಲ್ಗೂತ್ಸವವಿನೋದಿನೀ ॥ ೧೦೨॥

ಭವಾನೀ ಭವಪತ್ನೀ ಚ ಭವಭೀತಿಹರಾ ಭವಾ~।\\
ಭವೇಶ್ವರೀ ಭವಾರಾಧ್ಯಾ ಭವೇಶೀ ಭವನಾಯಿಕಾ ॥ ೧೦೩॥

ಭವಮಾತಾ ಭವಾಗಮ್ಯಾ ಭವಕಂಟಕನಾಶಿನೀ~।\\
ಭವಪ್ರಿಯಾ ಭವಾನಂದಾ ಭವ್ಯಾ ಚ ಭವಮೋಚನೀ ॥ ೧೦೪॥

ಭಾವನೀಯಾ ಭಗವತೀ ಭವಭಾರವಿನಾಶಿನೀ~।\\
ಭೂತಧಾತ್ರೀ ಚ ಭೂತೇಶೀ ಭೂತಸ್ಥಾ ಭೂತರೂಪಿಣೀ ॥ ೧೦೫॥

ಭೂತಮಾತಾ ಚ ಭೂತಘ್ನೀ ಭೂತಪಂಚಕವಾಸಿನೀ~।\\
ಭೋಗೋಪಚಾರಕುಶಲಾ ಭಿಸ್ಸಾಧಾತ್ರೀ ಚ ಭೂಚರೀ ॥ ೧೦೬॥

ಭೀತಿಘ್ನೀ ಭಕ್ತಿಗಮ್ಯಾ ಚ ಭಕ್ತಾನಾಮಾರ್ತಿನಾಶಿನೀ~।\\
ಭಕ್ತಾನುಕಂಪಿನೀ ಭೀಮಾ ಭಗಿನೀ ಭಗನಾಯಿಕಾ ॥ ೧೦೭॥

ಭಗವಿದ್ಯಾ ಭಗಕ್ಲಿನ್ನಾ ಭಗಯೋನಿರ್ಭಗಪ್ರದಾ~।\\
ಭಗೇಶೀ ಭಗರೂಪಾ ಚ ಭಗಗುಹ್ಯಾ ಭಗಾಪಹಾ ॥ ೧೦೮॥

ಭಗೋದರೀ ಭಗಾನಂದಾ ಭಗಾದ್ಯಾ ಭಗಮಾಲಿನೀ~।\\
ಭೋಗಪ್ರದಾ ಭೋಗವಾಸಾ ಭೋಗಮೂಲಾ ಚ ಭೋಗಿನೀ ॥ ೧೦೯॥

ಭೇರುಂಡಾ ಭೇದಿನೀ ಭೀಮಾ ಭದ್ರಕಾಲೀ ಭಿದೋಜ್ಝಿತಾ~।\\
ಭೈರವೀ ಭುವನೇಶಾನೀ ಭುವನಾ ಭುವನೇಶ್ವರೀ ॥ ೧೧೦॥

ಭೀಮಾಕ್ಷೀ ಭಾರತೀ ಚೈವ ಭೈರವಾಷ್ಟಕಸೇವಿತಾ~।\\
ಭಾಸ್ವರಾ ಭಾಸ್ವತೀ ಭೀತಿರ್ಭಾಸ್ವದುತ್ಥಾನಶಾಲಿನೀ ॥ ೧೧೧॥

ಭಾಗೀರಥೀ ಭೋಗವತೀ ಭವಘ್ನೀ ಭುವನಾತ್ಮಿಕಾ~।\\
ಭೂತಿದಾ ಭೂತಿರೂಪಾ ಚ ಭೂತಸ್ಥಾ ಭೂತವರ್ಧಿನೀ ॥ ೧೧೨॥

ಮಾಹೇಶ್ವರೀ ಮಹಾಮಾಯಾ ಮಹಾತೇಜಾ ಮಹಾಸುರೀ~।\\
ಮಹಾಜಿಹ್ವಾ ಮಹಾಲೋಲಾ ಮಹಾದಂಷ್ಟ್ರಾ ಮಹಾಭುಜಾ ॥ ೧೧೩॥

ಮಹಾಮೋಹಾಂಧಕಾರಘ್ನೀ ಮಹಾಮೋಕ್ಷಪ್ರದಾಯಿನೀ~।\\
ಮಹಾದಾರಿದ್ರ್ಯಶಮನೀ ಮಹಾಶತ್ರುವಿಮರ್ದಿನೀ ॥ ೧೧೪॥

ಮಹಾಶಕ್ತಿರ್ಮಹಾಜ್ಯೋತಿರ್ಮಹಾಸುರವಿಮರ್ದಿನೀ~।\\
ಮಹಾಕಾಯಾ ಮಹಾವೀರ್ಯಾ ಮಹಾಪಾತಕನಾಶಿನೀ ॥ ೧೧೫॥

ಮಹಾರವಾ ಮಂತ್ರಮಯೀ ಮಣಿಪೂರನಿವಾಸಿನೀ~।\\
ಮಾನಸೀ ಮಾನದಾ ಮಾನ್ಯಾ ಮನಶ್ಚಕ್ಷುರಗೋಚರಾ ॥ ೧೧೬॥

ಮಾಹೇಂದ್ರೀ ಮಧುರಾ ಮಾಯಾ ಮಹಿಷಾಸುರಮರ್ದಿನೀ~।\\
ಮಹಾಕುಂಡಲಿನೀ ಶಕ್ತಿರ್ಮಹಾವಿಭವವರ್ಧಿನೀ ॥ ೧೧೭॥

ಮಾನಸೀ ಮಾಧವೀ ಮೇಧಾ ಮತಿದಾ ಮತಿಧಾರಿಣೀ~।\\
ಮೇನಕಾಗರ್ಭಸಂಭೂತಾ ಮೇನಕಾಭಗಿನೀ ಮತಿಃ ॥ ೧೧೮॥

ಮಹೋದರೀ ಮುಕ್ತಕೇಶೀ ಮುಕ್ತಿಕಾಮ್ಯಾರ್ಥಸಿದ್ಧಿದಾ~।\\
ಮಾಹೇಶೀ ಮಹಿಷಾರೂಢಾ ಮಧುದೈತ್ಯವಿಮರ್ದಿನೀ ॥ ೧೧೯॥

ಮಹಾವ್ರತಾ ಮಹಾಮೂರ್ಧಾ ಮಹಾಭಯವಿನಾಶಿನೀ~।\\
ಮಾತಂಗೀ ಮತ್ತಮಾತಂಗೀ ಮಾತಂಗಕುಲಮಂಡಿತಾ ॥ ೧೨೦॥

ಮಹಾಘೋರಾ ಮಾನನೀಯಾ ಮತ್ತಮಾತಂಗಗಾಮಿನೀ~।\\
ಮುಕ್ತಾಹಾರಲತೋಪೇತಾ ಮದಧೂರ್ಣಿತಲೋಚನಾ ॥ ೧೨೧॥

ಮಹಾಪರಾಧರಾಶಿಘ್ನೀ ಮಹಾಚೋರಭಯಾಪಹಾ~।\\
ಮಹಾಚಿಂತ್ಯಸ್ವರೂಪಾ ಚ ಮಣಿಮಂತ್ರಮಹೌಷಧೀ ॥ ೧೨೨॥

ಮಣಿಮಂಡಪಮಧ್ಯಸ್ಥಾ ಮಣಿಮಾಲಾವಿರಾಜಿತಾ~।\\
ಮಂತ್ರಾತ್ಮಿಕಾ ಮಂತ್ರಗಮ್ಯಾ ಮಂತ್ರಮಾತಾ ಸುಮಂತ್ರಿಣೀ ॥ ೧೨೩॥

ಮೇರುಮಂದರಮಧ್ಯಸ್ಥಾ ಮಕರಾಕೃತಿಕುಂಡಲಾ~।\\
ಮಂಥರಾ ಚ ಮಹಾಸೂಕ್ಷ್ಮಾ ಮಹಾದೂತೀ ಮಹೇಶ್ವರೀ ॥ ೧೨೪॥

ಮಾಲಿನೀ ಮಾನವೀ ಮಾಧ್ವೀ ಮದರೂಪಾ ಮದೋತ್ಕಟಾ~।\\
ಮದಿರಾ ಮಧುರಾ ಚೈವ ಮೋದಿನೀ ಚ ಮದೋದ್ಧತಾ ॥ ೧೨೫॥

ಮಂಗಲಾಂಗೀ ಮಧುಮಯೀ ಮಧುಪಾನಪರಾಯಣಾ~।\\
ಮನೋರಮಾ ರಮಾಮಾತಾ ರಾಜರಾಜೇಶ್ವರೀ ರಮಾ ॥ ೧೨೬॥

ರಾಜಮಾನ್ಯಾ ರಾಜಪೂಜ್ಯಾ ರಕ್ತೋತ್ಪಲವಿಭೂಷಣಾ~।\\
ರಾಜೀವಲೋಚನಾ ರಾಮಾ ರಾಧಿಕಾ ರಾಮವಲ್ಲಭಾ ॥ ೧೨೭॥

ಶಾಕಿನೀ ಡಾಕಿನೀ ಚೈವ ಲಾವಣ್ಯಾಂಬುಧಿವೀಚಿಕಾ~।\\
ರುದ್ರಾಣೀ ರುದ್ರರೂಪಾ ಚ ರೌದ್ರಾ ರುದ್ರಾರ್ತಿನಾಶಿನೀ ॥ ೧೨೮॥

ರಕ್ತಪ್ರಿಯಾ ರಕ್ತವಸ್ತ್ರಾ ರಕ್ತಾಕ್ಷೀ ರಕ್ತಲೋಚನಾ~।\\
ರಕ್ತಕೇಶೀ ರಕ್ತದಂಷ್ಟ್ರಾ ರಕ್ತಚಂದನಚರ್ಚಿತಾ ॥ ೧೨೯॥

ರಕ್ತಾಂಗೀ ರಕ್ತಭೂಷಾ ಚ ರಕ್ತಬೀಜನಿಪಾತಿನೀ~।\\
ರಾಗಾದಿದೋಷರಹಿತಾ ರತಿಜಾ ರತಿದಾಯಿನೀ ॥ ೧೩೦॥

ವಿಶ್ವೇಶ್ವರೀ ವಿಶಾಲಾಕ್ಷೀ ವಿಂಧ್ಯಪೀಠನಿವಾಸಿನೀ~।\\
ವಿಶ್ವಭೂರ್ವೀರವಿದ್ಯಾ ಚ ವೀರಸೂರ್ವೀರನಂದಿನೀ ॥ ೧೩೧॥

ವೀರೇಶ್ವರೀ ವಿಶಾಲಾಕ್ಷೀ ವಿಷ್ಣುಮಾಯಾ ವಿಮೋಹಿನೀ~।\\
ವಿದ್ಯಾವತೀ ವಿಷ್ಣುರೂಪಾ ವಿಶಾಲನಯನೋಜ್ಜ್ವಲಾ ॥ ೧೩೨॥

ವಿಷ್ಣುಮಾತಾ ಚ ವಿಶ್ವಾತ್ಮಾ ವಿಷ್ಣುಜಾಯಾಸ್ವರೂಪಿಣೀ~।\\
ವಾರಾಹೀ ವರದಾ ವಂದ್ಯಾ ವಿಖ್ಯಾತಾ ವಿಲಸಲ್ಕಚಾ ॥ ೧೩೩॥

ಬ್ರಹ್ಮೇಶೀ ಬ್ರಹ್ಮದಾ ಬ್ರಾಹ್ಮೀ ಬ್ರಹ್ಮಾಣೀ ಬ್ರಹ್ಮರೂಪಿಣೀ~।\\
ದ್ವಾರಕಾ ವಿಶ್ವವಂದ್ಯಾ ಚ ವಿಶ್ವಪಾಶವಿಮೋಚನೀ~।\\
ವಿಶ್ವಾಸಕಾರಿಣೀ ವಿಶ್ವಾ ವಿಶ್ವಶಕ್ತಿರ್ವಿಚಕ್ಷಣಾ ॥ ೧೩೪॥

ಬಾಣಚಾಪಧರಾ ವೀರಾ ಬಿಂದುಸ್ಥಾ ಬಿಂದುಮಾಲಿನೀ~।\\
ಷಟ್ಚಕ್ರಭೇದಿನೀ ಷೋಢಾ ಷೋಡಶಾರನಿವಾಸಿನೀ ॥ ೧೩೫॥

ಶಿತಿಕಂಠಪ್ರಿಯಾ ಶಾಂತಾ ಶಾಕಿನೀ ವಾತರೂಪಿಣೀ~।\\
ಶಾಶ್ವತೀ ಶಂಭುವನಿತಾ ಶಾಂಭವೀ ಶಿವರೂಪಿಣೀ ॥ ೧೩೬॥

ಶಿವಮಾತಾ ಚ ಶಿವದಾ ಶಿವಾ ಶಿವಹೃದಾಸನಾ~।\\
ಶುಕ್ಲಾಂಬರಾ ಶೀತಲಾ ಚ ಶೀಲಾ ಶೀಲಪ್ರದಾಯಿನೀ ॥ ೧೩೭॥

ಶಿಶುಪ್ರಿಯಾ ವೈದ್ಯವಿದ್ಯಾ ಸಾಲಗ್ರಾಮಶಿಲಾ ಶುಚಿಃ~।\\
ಹರಿಪ್ರಿಯಾ ಹರಮೂರ್ತಿರ್ಹರಿನೇತ್ರಕೃತಾಲಯಾ ॥ ೧೩೮॥

ಹರಿವಕ್ತ್ರೋದ್ಭವಾ ಹಾಲಾ ಹರಿವಕ್ಷಃಸ್ಥಲಸ್ಥಿತಾ~।\\
ಕ್ಷೇಮಂಕರೀ ಕ್ಷಿತಿಃ ಕ್ಷೇತ್ರಾ ಕ್ಷುಧಿತಸ್ಯ ಪ್ರಪೂರಣೀ ॥ ೧೩೯॥

ವೈಶ್ಯಾ ಚ ಕ್ಷತ್ರಿಯಾ ಶೂದ್ರೀ ಕ್ಷತ್ರಿಯಾಣಾಂ ಕುಲೇಶ್ವರೀ~।\\
ಹರಪತ್ನೀ ಹರಾರಾಧ್ಯಾ ಹರಸೂರ್ಹರರೂಪಿಣೀ ॥ ೧೪೦॥

ಸರ್ವಾನಂದಮಯೀ ಸಿದ್ಧಿಸ್ಸರ್ವರಕ್ಷಾಸ್ವರೂಪಿಣೀ~।\\
ಸರ್ವದುಷ್ಟಪ್ರಶಮನೀ ಸರ್ವೇಪ್ಸಿತಫಲಪ್ರದಾ ॥ ೧೪೧॥

ಸರ್ವಸಿದ್ಧೇಶ್ವರಾರಾಧ್ಯಾ ಸರ್ವಮಂಗಲಮಂಗಲಾ ॥

ಫಲಶ್ರುತಿಃ~।\\
ಪುಣ್ಯಂ ಸಹಸ್ರನಾಮೇದಂ ತವ ಪ್ರೀತ್ಯಾ ಪ್ರಕಾಶಿತಂ ॥ ೧೪೨॥

ಗೋಪನೀಯಂ ಪ್ರಯತ್ನೇನ ಪಠನೀಯಂ ಪ್ರಯತ್ನತಃ~।\\
ನಾತಃ ಪರತರಂ ಪುಣ್ಯಂ ನಾತಃ ಪರತರಂ ತಪಃ ॥ ೧೪೩॥

ನಾತಃ ಪರತರಂ ಸ್ತೋತ್ರಂ ನಾತಃ ಪರತರಾ ಗತಿಃ~।\\
ಸ್ತೋತ್ರಂ ನಾಮಸಹಸ್ರಾಖ್ಯಂ ಮಮ ವಕ್ತ್ರಾದ್ವಿನಿರ್ಗತಂ ॥ ೧೪೪॥

ಯಃ ಪಠೇತ್ಪರಯಾ ಭಕ್ತ್ಯಾ ಶೃಣುಯಾದ್ವಾ ಸಮಾಹಿತಃ~।\\
ಮೋಕ್ಷಾರ್ಥೀ ಲಭತೇ ಮೋಕ್ಷಂ ಸ್ವರ್ಗಾರ್ಥೀ ಸ್ವರ್ಗಮಾಪ್ನುಯಾತ್ ॥ ೧೪೫॥

ಕಾಮಾರ್ಥೀ ಲಭತೇ ಕಾಮಂ ಧನಾರ್ಥೀ ಲಭತೇ ಧನಂ~।\\
ವಿದ್ಯಾರ್ಥೀ ಲಭತೇ ವಿದ್ಯಾಂ ಯಶೋಽರ್ಥೀ ಲಭತೇ ಯಶಃ ॥ ೧೪೬॥

ಕನ್ಯಾರ್ಥೀ ಲಭತೇ ಕನ್ಯಾಂ ಸುತಾರ್ಥೀ ಲಭತೇ ಸುತಾನ್~।\\
ಮೂರ್ಖೋಽಪಿ ಲಭತೇ ಶಾಸ್ತ್ರಂ ಚೋರೋಽಪಿ ಲಭತೇ ಗತಿಂ ॥ ೧೪೭॥

ಗುರ್ವಿಣೀ ಜನಯೇತ್ಪುತ್ರಂ ಕನ್ಯಾ ವಿಂದತಿ ಸತ್ಪತಿಂ~।\\
ಸಂಕ್ರಾಂತ್ಯಾಂ ಚ ಚತುರ್ದಶ್ಯಾಮಷ್ಟಮ್ಯಾಂ ಚ ವಿಶೇಷತಃ ॥ ೧೪೮॥

ಪೌರ್ಣಮಾಸ್ಯಾಮಮಾವಸ್ಯಾಂ ನವಮ್ಯಾಂ ಭೌಮವಾಸರೇ~।\\
ಪಠೇದ್ವಾ ಪಾಠಯೇದ್ವಾಪಿ ಪೂಜಯೇದ್ವಾಪಿ ಪುಸ್ತಕಂ ॥ ೧೪೯॥

ಸ ಮುಕ್ತಸ್ಸರ್ವಪಾಪೇಭ್ಯಃ ಕಾಮೇಶ್ವರಸಮೋ ಭವೇತ್~।\\
ಲಕ್ಷ್ಮೀವಾನ್ ಸುತವಾಂಶ್ಚೈವ ವಲ್ಲಭಸ್ಸರ್ವಯೋಷಿತಾಂ ॥ ೧೫೦॥

ತಸ್ಯ ವಶ್ಯಂ ಭವೇದಾಶು ತ್ರೈಲೋಕ್ಯಂ ಸಚರಾಚರಂ~।\\
ವಿದ್ಯಾನಾಂ ಪಾರಗೋ ವಿಪ್ರಃ ಕ್ಷತ್ರಿಯೋ ವಿಜಯೀ ರಣೇ ॥ ೧೫೧॥

ವೈಶ್ಯೋ ಧನಸಮೃದ್ಧಸ್ಸ್ಯಾಚ್ಛೂದ್ರಸ್ಸುಖಮವಾಪ್ನುಯಾತ್~।\\
ಕ್ಷೇತ್ರೇ ಚ ಬಹುಸಸ್ಯಂ ಸ್ಯಾದ್ಗಾವಶ್ಚ ಬಹುದುಗ್ಧದಾಃ ॥ ೧೫೨॥

ನಾಶುಭಂ ನಾಪದಸ್ತಸ್ಯ ನ ಭಯಂ ನೃಪಶತ್ರುತಃ~।\\
ಜಾಯತೇ ನಾಶುಭಾ ಬುದ್ಧಿರ್ಲಭತೇ ಕುಲಪೂಜ್ಯತಾಂ ॥ ೧೫೩॥

ನ ಬಾಧಂತೇ ಗ್ರಹಾಸ್ತಸ್ಯ ನ ರಕ್ಷಾಂಸಿ ನ ಪನ್ನಗಾಃ~।\\
ನ ಪಿಶಾಚಾ ನ ಡಾಕಿನ್ಯೋ ಭೂತಭೇತಾಲಡಂಭಕಾಃ ॥ ೧೫೪॥

ಬಾಲಗ್ರಹಾಭಿಭೂತಾನಾಂ ಬಾಲಾನಾಂ ಶಾಂತಿಕಾರಕಂ~।\\
ದ್ವಂದ್ವಾನಾಂ ಪ್ರತಿಭೇದೇ ಚ ಮೈತ್ರೀಕರಣಮುತ್ತಮಂ ॥ ೧೫೫॥

ಲೋಹಪಾಶೈದೃಢೈರ್ಬದ್ಧೋ ಬಂಧೀ ವೇಶ್ಮನಿ ದುರ್ಗಮೇ~।\\
ತಿಷ್ಠಂಛೃಣ್ವನ್ಪಠನ್ಮರ್ತ್ಯೋ ಮುಚ್ಯತೇ ನಾತ್ರ ಸಂಶಯಃ ॥ ೧೫೬॥

ಪಶ್ಯಂತಿ ನಹಿ ತೇ ಶೋಕಂ ವಿಯೋಗಂ ಚಿರಜೀವಿನಃ~।\\
ಶೃಣ್ವತೀ ಬದ್ಧಗರ್ಭಾ ಚ ಸುಖಂ ಚೈವ ಪ್ರಸೂಯತೇ ॥ ೧೫೭॥

ಏಕದಾ ಪಠನಾದೇವ ಸರ್ವಪಾಪಕ್ಷಯೋ ಭವೇತ್~।\\
ನಶ್ಯಂತಿ ಚ ಮಹಾರೋಗಾ ದಶಧಾವರ್ತನೇನ ಚ ॥ ೧೫೮॥

ಶತಧಾವರ್ತನೇ ಚೈವ ವಾಚಾಂ ಸಿದ್ಧಿಃ ಪ್ರಜಾಯತೇ~।\\
ನವರಾತ್ರೇ ಜಿತಾಹಾರೋ ದೃಢಬುದ್ಧಿರ್ಜಿತೇಂದ್ರಿಯಃ ॥ ೧೫೯॥

ಅಂಬಿಕಾಯತನೇ ವಿದ್ವಾನ್ ಶುಚಿಷ್ಮಾನ್ ಮೂರ್ತಿಸನ್ನಿಧೌ~।\\
ಏಕಾಕೀ ಚ ದಶಾವರ್ತಂ ಪಠಂಧೀರಶ್ಚ ನಿರ್ಭಯಃ ॥ ೧೬೦॥

ಸಾಕ್ಷಾತ್ತ್ವಗವತೀ ತಸ್ಮೈ ಪ್ರಯಚ್ಛೇದೀಪ್ಸಿತಂ ಫಲಂ~।\\
ಸಿದ್ಧಪೀಠೇ ಗಿರೌ ರಮ್ಯೇ ಸಿದ್ಧಕ್ಷೇತ್ರೇ ಸುರಾಲಯೇ ॥ ೧೬೧॥

ಪಠನಾತ್ಸಾಧಕಸ್ಯಾಶು ಸಿದ್ಧಿರ್ಭವತಿ ವಾಂಛಿತಾ~।\\
ದಶಾವರ್ತಂ ಪಠೇನ್ನಿತ್ಯಂ ಭೂಮಿಶಾಯೀ ನರಶ್ಶುಚಿಃ ॥ ೧೬೨॥

ಸ್ವಪ್ನೇ ಮೂರ್ತಿಮಯೀಂ ದೇವೀಂ ವರದಾಂ ಸೋಽಪಿ ಪಶ್ಯತಿ~।\\
ಆವರ್ತನಸಹಸ್ರೈರ್ಯೇ ಜಪಂತಿ ಪುರುಷೋತ್ತಮಾಃ ॥ ೧೬೩॥

ತೇ ಸಿದ್ಧಾ ಸಿದ್ಧಿದಾ ಲೋಕೇ ಶಾಪಾನುಗ್ರಹಣಕ್ಷಮಾಃ~।\\
ಪ್ರಯಚ್ಛಂತಶ್ಚ ಸರ್ವಸ್ವಂ ಸೇವಂತೇ ತಾನ್ಮಹೀಶ್ವರಾಃ ॥ ೧೬೪॥

ಭೂರ್ಜಪತ್ರೇಽಷ್ಟಗಂಧೇನ ಲಿಖಿತ್ವಾ ತು ಶುಭೇ ದಿನೇ~।\\
ಧಾರಯೇದ್ಯಂತ್ರಿತಂ ಶೀರ್ಷೇ ಪೂಜಯಿತ್ವಾ ಕುಮಾರಿಕಾಂ ॥ ೧೬೫॥

ಬ್ರಾಹ್ಮಣಾನ್ ವರನಾರೀಶ್ಚ ಧೂಪೈಃ ಕುಸುಮಚಂದನೈಃ~।\\
ಕ್ಷೀರಖಂಡಾದಿಭೋಜ್ಯಾಂಶ್ಚ ಭೋಜಯಿತ್ವಾ ಸುಭಕ್ತಿತಃ ॥ ೧೬೬॥

ಬಧ್ನಂತಿ ಯೇ ಮಹಾರಕ್ಷಾಂ ಬಾಲಾನಾಂ ಚ ವಿಶೇಷತಃ~।\\
ರುದ್ರಂ ದೃಷ್ಟ್ವಾ ಯಥಾ ದೇವಂ ವಿಷ್ಣುಂ ದೃಷ್ಟ್ವಾ ಚ ದಾನವಾಃ ॥ ೧೬೭॥

ಪನ್ನಗಾ ಗರುಡಂ ದೃಷ್ಟ್ವಾ ಸಿಂಹಂ ದೃಷ್ಟ್ವಾ ಯಥಾ ಗಜಾಃ~।\\
ಮಂಡೂಕಾ ಭೋಗಿನಂ ದೃಷ್ಟ್ವಾ ಮಾರ್ಜಾರಂ ಮೂಷಿಕಾಸ್ತಥಾ ॥ ೧೬೮॥

ವಿಘ್ನಭೂತಾಃ ಪಲಾಯಂತೇ ತಸ್ಯ ವಕ್ತ್ರವಿಲೋಕನಾತ್~।\\
ಅಗ್ನಿಚೋರಭಯಂ ತಸ್ಯ ಕದಾಚಿನ್ನೈವ ಸಂಭವೇತ್ ॥ ೧೬೯॥

ಪಾತಕಾನ್ವಿವಿಧಾನ್ಸೋಽಪಿ ಮೇರುಮಂದರಸನ್ನಿಭಾನ್~।\\
ಭಸ್ಮಿತಾನ್ಕುರುತೇ ಕ್ಷಿಪ್ರಂ ತೃಣಂ ವಹ್ನಿಹುತಂ ಯಥಾ ॥ ೧೭೦॥

ನೃಪಾಶ್ಚ ವಶ್ಯತಾಂ ಯಾಂತಿ ನೃಪಪೂಜ್ಯಾಶ್ಚ ತೇ ನರಾಃ~।\\
ಮಹಾರ್ಣವೇ ಮಹಾನದ್ಯಾಂ ಪೋತಸ್ಥೇ ಚ ನ ಭೀಃ ಕ್ವಚಿತ್ ॥ ೧೭೧॥

ರಣೇ ದ್ಯೂತೇ ವಿವಾದೇ ಚ ವಿಜಯಂ ಪ್ರಾಪ್ನುವಂತಿ ತೇ~।\\
ಸರ್ವತ್ರ ಪೂಜಿತೋ ಲೋಕೈರ್ಬಹುಮಾನಪುರಸ್ಸರೈಃ ॥ ೧೭೨॥

ರತಿರಾಗವಿವೃದ್ಧಾಶ್ಚ ವಿಹ್ವಲಾಃ ಕಾಮಪೀಡಿತಾಃ~।\\
ಯೌವನಾಕ್ರಾಂತದೇಹಾಸ್ತಾನ್ ಶ್ರಯಂತೇ ವಾಮಲೋಚನಾಃ ॥ ೧೭೩॥

ಸಹಸ್ರಂ ಜಪತೇ ಯಸ್ತು ಖೇಚರೀ ಜಾಯತೇ ನರಃ~।\\
ಸಹಸ್ರದಶಕಂ ದೇವಿ ಯಃ ಪಠೇದ್ಭಕ್ತಿಮಾನ್ನರಃ ॥ ೧೭೪॥

ಸಾ ತಸ್ಯ ಜಗತಾಂ ಧಾತ್ರೀ ಪ್ರತ್ಯಕ್ಷಾ ಭವತಿ ಧ್ರುವಂ~।\\
ಲಕ್ಷಂ ಪೂರ್ಣಂ ಯದಾ ದೇವಿ ಸ್ತೋತ್ರರಾಜಂ ಜಪೇತ್ಸುಧೀಃ ॥ ೧೭೫॥

ಭವಪಾಶವಿನಿರ್ಮುಕ್ತೋ ಮಮ ತುಲ್ಯೋ ನ ಸಂಶಯಃ~।\\
ಸರ್ವತೀರ್ಥೇಷು ಯತ್ಪುಣ್ಯಂ ಸರ್ವತೀರ್ಥೇಷು ಯತ್ಫಲಂ ॥ ೧೭೬॥

ಸರ್ವಧರ್ಮೇಷು ಯಜ್ಞೇಷು ಸರ್ವದಾನೇಷು ಯತ್ಫಲಂ~।\\
ಸರ್ವವೇದೇಷು ಪ್ರೋಕ್ತೇಷು ಯತ್ಫಲಂ ಪರಿಕೀರ್ತಿತಂ ॥ ೧೭೭॥

ತತ್ಪುಣ್ಯಂ ಕೋಟಿಗುಣಿತಂ ಸಕೃಜ್ಜಪ್ತ್ವಾ ಲಭೇನ್ನರಃ~।\\
ದೇಹಾಂತೇ ಪರಮಂ ಸ್ಥಾನಂ ಯತ್ಸುರೈರಪಿ ದುರ್ಲಭಂ~।\\
ಸ ಯಾಸ್ಯತಿ ನ ಸಂದೇಹಸ್ಸ್ತವರಾಜಸ್ಯ ಕೀರ್ತನಾತ್ ॥ ೧೭೮॥
\authorline{॥ ಇತಿ ರುದ್ರಯಾಮಲೇ ಶ್ರೀಅನ್ನಪೂರ್ಣಾಸಹಸ್ರನಾಮಸ್ತೋತ್ರಂ ಸಂಪೂರ್ಣಂ ॥}
\section{ಶ್ರೀಶೀತಲಾಸ್ತೋತ್ರ}
ಅಸ್ಯ ಶ್ರೀಶೀತಲಾಸ್ತೋತ್ರಸ್ಯ ಮಹಾದೇವ ಋಷಿಃ~। ಅನುಷ್ಟುಪ್ ಛಂದಃ। ಶೀತಲಾ ದೇವತಾ। ಲಕ್ಷ್ಮೀ ಬೀಜಂ। ಭವಾನೀ ಶಕ್ತಿಃ~। ಜಪೇ ವಿನಿಯೋಗಃ ॥

ಈಶ್ವರ ಉವಾಚ ॥\\
ವಂದೇಹಂ ಶೀತಲಾಂ ದೇವೀಂ ರಾಸಭಸ್ಥಾಂ ದಿಗಂಬರಾಂ।\\
ಮಾರ್ಜನೀಕಲಶೋಪೇತಾಂ ಶೂರ್ಪಾಲಂಕೃತಮಸ್ತಕಾಂ॥೧॥

ವಂದೇಹಂ ಶೀತಲಾಂ ದೇವೀಂ ಸರ್ವರೋಗಭಯಾಪಹಾಂ।\\
ಯಾಮಾಸಾದ್ಯ ನಿವರ್ತೇತ ವಿಸ್ಫೋಟಕಭಯಂ ಮಹತ್॥೨॥

ಶೀತಲೇ ಶೀತಲೇ ಚೇತಿ ಯೋ ಬ್ರೂಯಾದ್ದಾಹಪೀಡಿತಃ।\\
ವಿಸ್ಫೋಟಕಭಯಂ ಘೋರಂ ಕ್ಷಿಪ್ರಂ ತಸ್ಯ ಪ್ರಣಶ್ಯತಿ॥೩॥

ಯಸ್ತ್ವಾಮುದಕಮಧ್ಯೇ ತು ಸ್ಥಿತ್ವಾ ಪೂಜಯತೇ ನರಃ।\\
ವಿಸ್ಫೋಟಕಭಯಂ ಘೋರಂ ಗೃಹೇ ತಸ್ಯ ನ ಜಾಯತೇ॥೪॥

ಶೀತಲೇ ಜ್ವರದಗ್ಧಸ್ಯ ಪೂತಿಗಂಧಯುತಸ್ಯ ಚ~।\\
ಪ್ರಣಷ್ಟಚಕ್ಷುಷಃ ಪುಂಸಸ್ತ್ವಾಮಾಹುರ್ಜೀವನೌಷಧಂ॥೫॥

ಶೀತಲೇ ತನುಜಾನ್ ರೋಗಾನ್ನೃಣಾಂ ಹರಸಿ ದುಸ್ತ್ಯಜಾನ್।\\
ವಿಸ್ಫೋಟಕವಿದೀರ್ಣಾನಾಂ ತ್ವಮೇಕಾಮೃತವರ್ಷಿಣೀ॥೬॥

ಗಲಗಂಡಗ್ರಹಾ ರೋಗಾ ಯೇ ಚಾನ್ಯೇ ದಾರುಣಾ ನೃಣಾಂ।\\
ತ್ವದನುಧ್ಯಾನಮಾತ್ರೇಣ ಶೀತಲೇ ಯಾಂತಿ ಸಂಕ್ಷಯಂ॥೭॥

ನ ಮಂತ್ರೋ ನೌಷಧಂ ತಸ್ಯ ಪಾಪರೋಗಸ್ಯ ವಿದ್ಯತೇ।\\
ತ್ವಾಮೇಕಾಂ ಶೀತಲೇ ಧಾತ್ರೀಂ ನಾನ್ಯಾಂ ಪಶ್ಯಾಮಿ ದೇವತಾಂ॥೮॥

ಮೃಣಾಲತಂತುಸದೃಶೀಂ ನಾಭಿಹೃನ್ಮಧ್ಯಸಂಸ್ಥಿತಾಂ।\\
ಯಸ್ತ್ವಾಂ ಸಂಚಿಂತಯೇದ್ದೇವಿ ತಸ್ಯ ಮೃತ್ಯುರ್ನ ಜಾಯತೇ॥೯॥

ಅಷ್ಟಕಂ ಶೀತಲಾದೇವ್ಯಾ ಯೋ ನರಃ ಪ್ರಪಠೇತ್ಸದಾ।\\
ವಿಸ್ಫೋಟಕಭಯಂ ಘೋರಂ ಗೃಹೇ ತಸ್ಯ ನ ಜಾಯತೇ॥೧೦॥

ಶ್ರೋತವ್ಯಂ ಪಠಿತವ್ಯಂ ಚ ಶ್ರದ್ಧಾಭಕ್ತಿಸಮನ್ವಿತೈಃ।\\
ಉಪಸರ್ಗವಿನಾಶಾಯ ಪರಂ ಸ್ವಸ್ತ್ಯಯನಂ ಮಹತ್॥೧೧॥

ಶೀತಲೇ ತ್ವಂ ಜಗನ್ಮಾತಾ ಶೀತಲೇ ತ್ವಂ ಜಗತ್ಪಿತಾ।\\
ಶೀತಲೇ ತ್ವಂ ಜಗದ್ಧಾತ್ರೀ ಶೀತಲಾಯೈ ನಮೋ ನಮಃ॥೧೨॥

ರಾಸಭೋ ಗರ್ದಭಶ್ಚೈವ ಖರೋ ವೈಶಾಖನಂದನಃ।\\
ಶೀತಲಾವಾಹನಶ್ಚೈವ ದೂರ್ವಾಕಂದನಿಕೃಂತನಃ॥೧೩॥

ಏತಾನಿ ಖರನಾಮಾನಿ ಶೀತಲಾಗ್ರೇ ತು ಯಃ ಪಠೇತ್।\\
ತಸ್ಯ ಗೇಹೇ ಶಿಶೂನಾಂ ಚ ಶೀತಲಾರುಡ್ ನ ಜಾಯತೇ॥೧೪॥

ಶೀತಲಾಷ್ಟಕಮೇವೇದಂ ನ ದೇಯಂ ಯಸ್ಯ ಕಸ್ಯಚಿತ್।\\
ದಾತವ್ಯಂ ಚ ಸದಾ ತಸ್ಮೈ ಶ್ರದ್ಧಾಭಕ್ತಿಯುತಾಯ ವೈ॥೧೫॥

\authorline{॥ಇತಿ ಶ್ರೀಸ್ಕಂದಮಹಾಪುರಾಣೇ ಶೀತಲಾಷ್ಟಕಂ ಸಂಪೂರ್ಣಂ॥}
\section{ ಶ್ರೀ ಮಹಾಲಕ್ಷ್ಮ್ಯಷ್ಟಕಂ}
ನಮಸ್ತೇಸ್ತು ಮಹಾಮಾಯೇ ಶ್ರೀಪೀಠೇ ಸುರಪೂಜಿತೇ।\\
ಶಂಖಚಕ್ರಗದಾಹಸ್ತೇ ಮಹಾಲಕ್ಷ್ಮೀ ನಮೋಸ್ತು ತೇ ॥೧॥

	ನಮಸ್ತೇ ಗರುಡಾರೂಢೇ ಕೋಲಾಸುರಭಯಂಕರಿ।\\
	ಸರ್ವಪಾಪಹರೇ ದೇವಿ ಮಹಾಲಕ್ಷ್ಮಿ ನಮೋಸ್ತು ತೇ ॥೨॥

ಸರ್ವಜ್ಞೇ ಸರ್ವವರದೇ ಸರ್ವದುಷ್ಟಭಯಂಕರಿ।\\
ಸರ್ವದುಃಖಹರೇ ದೇವಿ ಮಹಾಲಕ್ಷ್ಮಿ ನಮೋಸ್ತು ತೇ ॥೩॥

	ಸಿದ್ಧಿಬುದ್ಧಿಪ್ರದೇ ದೇವಿ ಭುಕ್ತಿಮುಕ್ತಿಪ್ರದಾಯಿನಿ।\\
	ಮಂತ್ರಮೂರ್ತೇ ಸದಾ ದೇವಿ ಮಹಾಲಕ್ಷ್ಮಿ ನಮೋಸ್ತು ತೇ ॥೪॥

ಆದ್ಯಂತರಹಿತೇ ದೇವಿ ಆದ್ಯಶಕ್ತಿಮಹೇಶ್ವರಿ।\\
ಯೋಗಜ್ಞೇ ಯೋಗಸಂಭೂತೇ ಮಹಾಲಕ್ಷ್ಮಿ ನಮೋಸ್ತು ತೇ ॥೫॥

	ಸ್ಥೂಲಸೂಕ್ಷ್ಮಮಹಾರೌದ್ರೇ ಮಹಾಶಕ್ತಿಮಹೋದರೇ।\\
	ಮಹಾಪಾಪಹರೇ ದೇವಿ ಮಹಾಲಕ್ಷ್ಮಿ ನಮೋಸ್ತು ತೇ ॥೬॥

ಪದ್ಮಾಸನಸ್ಥಿತೇ ದೇವಿ ಪರಬ್ರಹ್ಮಸ್ವರೂಪಿಣಿ।\\
ಪರಮೇಶಿ ಜಗನ್ಮಾತರ್ಮಹಾಲಕ್ಷ್ಮಿ ನಮೋಸ್ತು ತೇ ॥೭॥

	ಶ್ವೇತಾಂಬರಧರೇ ದೇವಿ ನಾನಾಲಂಕಾರಭೂಷಿತೇ।\\
	ಜಗತ್ಸ್ಥಿತೇ ಜಗನ್ಮಾತರ್ಮಹಾಲಕ್ಷ್ಮಿ ನಮೋಸ್ತು ತೇ ॥೮॥

ಮಹಾಲಕ್ಷ್ಮ್ಯಷ್ಟಕಂ ಸ್ತೋತ್ರಂ ಯಃ ಪಠೇದ್ಭಕ್ತಿಮಾನ್ನರಃ।\\
ಸರ್ವಸಿದ್ಧಿಮವಾಪ್ನೋತಿ ರಾಜ್ಯಂ ಪ್ರಾಪ್ನೋತಿ ಸರ್ವದಾ ॥೯॥

	ಏಕಕಾಲೇ ಪಠೇನ್ನಿತ್ಯಂ ಮಹಾಪಾಪವಿನಾಶನಂ।\\
	ದ್ವಿಕಾಲಂ ಯಃ ಪಠೇನ್ನಿತ್ಯಂ ಧನಧಾನ್ಯಸಮನ್ವಿತಃ॥೧೦॥

ತ್ರಿಕಾಲಂ ಯಃ ಪಠೇನ್ನಿತ್ಯಂ ಮಹಾಶತ್ರುವಿನಾಶನಂ।\\
ಮಹಾಲಕ್ಷ್ಮೀರ್ಭವೇನ್ನಿತ್ಯಂ ಪ್ರಸನ್ನಾ ವರದಾ ಶುಭಾ ॥೧೧॥
\authorline{\bfseries ಸದ್ಗುರುಚರಣಾರವಿಂದಾರ್ಪಣಮಸ್ತು\\ಓಂ ತತ್ಸತ್\\********}


ಗಂಗಾಷ್ಟೋತ್ತರಶತನಾಮಸ್ತೋತ್ರಂ
ಗಂಗಾ ವಿಷ್ಣುಪಾದಾಬ್ಜಸಂಭೂತಾ ಸುರನಿಮ್ನಗಾ ।
ಹಿಮಾಚಲೇಂದುನಯಾ ಗಿರಿಮಂಡಲಗಾಮಿನೀ ॥೧॥
	ತಾರಕಾರಾತಿ ಜನನೀ ಸಗರಾತ್ಮಜತಾರಿಕಾ ।
	ಸರಸ್ವತೀಸಮಾಯುಕ್ತಾ ಸುಘೋಷಾ ಸಿಂಧುಗಾಮಿನೀ ॥೨॥
ಭಾಗೀರಥೀ ಭಗವತೀ ಭಗೀರಥರಥಾನುಗಾ ।
ತ್ರಿವಿಕ್ರಮಪದೋದ್ಭೂತಾ ತ್ರಿಲೋಕಪಥಗಾಮಿನೀ ॥೩॥
	ಕ್ಷೀರಶುಭ್ರಾ ಬಹುಕ್ಷೀರಾ ಕ್ಷೀರವೃಕ್ಷಸಮಾಕುಲಾ।
	ತ್ರಿಲೋಚನಜಟಾವಾಸಾ ಋಣತ್ರಯವಿಮೋಚಿನೀ ॥೪॥
ತ್ರಿಪುರಾರಿಶಿರಶ್ಚೂಡಾ ಅವ್ಯಯಾ ಜಾಹ್ನವೀ ತಥಾ।
ನಿರಂಜನಾ ನಿತ್ಯಶುದ್ಧಾ ನಯನಾನಂದದಾಯಿನೀ॥೫॥
	ಸಾವಿತ್ರೀ ಸಲಿಲಾವಾಸಾ ನರಭೀತಿಹರಾ ತಥಾ।
	ಉಮಾಸಪತ್ನೀ ರಮ್ಯಾ ಚ ನೀರಜಾಲಿಪರಿಷ್ಕೃತಾ॥೬॥
ಅವ್ಯಕ್ತಾ ಬಿಂದುಸರಸಾ ಸಾಗರಾಂಬುಸಮೇಧಿನೀ।
ಶುಭ್ರಗಂಧೀ ಶುಭ್ರವಸ್ತ್ರಾ ಬೃಂದಾರಕಸಮಾಶ್ರಿತಾ॥೭॥
	ಆಖಂಡಲವನಾವಾಸಾ ಖಂಡೇಂದುಕೃತಶೇಖರಾ।
	ಅಮೃತಾಕಾರಸಲಿಲಾ ಲೀಲಾಲಿಂಗಿತಪರ್ವತಾ॥೮॥
ವಿರಿಂಚಿಕಲಶಾವಾಸಾ ತ್ರಿವೇಣೀ ತ್ರಿಗುಣಾತ್ಮಿಕಾ ।
ಸಂಗತಾಘೌಘಶಮನೀ ಶ್ರೀಮತೀಶೀಘ್ರಗಾಮಿನೀ॥೯॥
	ಭೀತಿಹಂತ್ರೀ ಭಾಗ್ಯಮಾತಾ ಶಂಖದುಂದುಭಿನಿಸ್ವನಾ।
	ಆನಂದಿನೀ ಶರಣ್ಯಾ ಚ ಭಿನ್ನಬ್ರಹ್ಮಾಂಡದರ್ಪಿಣೀ ॥೧೦॥
ಸಿದ್ಧಾ ಚ ಶಫರೀಪೂರ್ಣಾ ಅನಂತಾ ಶಶಿಶೇಖರಾ ।
ಭವಪ್ರಿಯಾ ಶಾಂಕರೀ ಚ ಭಗಮೂರ್ಧ್ನಿ ಕೃತಾಲಯಾ॥ ೧೧॥
	ಸತ್ಯಸಂಧಪ್ರಿಯಾ ಹಂಸರೂಪಿಣೀ ಅತುಲಾ ತಥಾ ।
	ಓಂಕಾರರೂಪಿಣೀ ಶೀತಾ ಶರಚ್ಚಂದ್ರನಿಭಾನನಾ ॥೧೨॥
ಭಗೀರಥಭೃತಾ ಪೂತಾ ಕ್ರೀಡಾಕಲ್ಲೋಲಕಾರಿಣೀ।
ಸ್ವರ್ಗಸೋಪಾನಸರಣಿಃ ಸರ್ವದೇವಸ್ವರೂಪಿಣೀ ॥೧೩॥
	ಅಂಭಃಪ್ರದಾ ದುಃಖಹಂತ್ರೀ ಶಾಂತಿಸಂತಾನಕಾರಿಣೀ ।
	ದಾರಿದ್ರ್ಯಹಂತ್ರೀ ಶಿವದಾ ಸಂಸಾರವಿಷನಾಶಿನೀ ॥೧೪॥
ಪ್ರಯಾಗನಿಲಯಾ ಪಾಪಹಂತ್ರೀ ಪುಣ್ಯಾ ಪುರಾತನಾ ।
ಶರಣಾಗತದೀನಾರ್ತ ಪರಿತ್ರಾಣಪರಾಯಣಾ ॥೧೫॥
	ಸುಮುಕ್ತಿದಾ ಪಾವನಾಂಗೀ ಸಿದ್ಧಯೋಗಿನಿಷೇವಿತಾ ।
	ಜಗದ್ಧಿತಾ ಜಹ್ನುಪುತ್ರೀ ಜಂಗಮಾ ಪುಣ್ಯವಾಹಿನೀ ॥೧೬॥
ಜಯಾ ಪೂತತ್ರಿಭುವನಾ ಜಂಬೂದ್ವೀಪವಿಹಾರಿಣೀ ।
ಭವಪತ್ನೀ ಜಗನ್ಮಾತಾ ಅಜ್ಞಾನತಿಮಿರಾಪಹಾ ॥೧೭॥
	ಭೀಷ್ಮಮಾತಾ ಪುಣ್ಯದಾ ಚ ಪೂರ್ಣಾರ್ಥಾ ನಗಪುತ್ರಿಕಾ ।
	ಜಲರೂಪಾ ಚ ಪ್ರಥಿತಾ ತಾಪತ್ರಯ ವಿಮೋಚಿನೀ ॥೧೮॥
ಪುಲೋಮಜಾರ್ಚಿತಾ ಸಿದ್ಧಾ ರಮ್ಯರೂಪಧೃತಾ ತಥಾ ।
ವಿಮಲಾ ಜಂಗಮಾಧಾರಾ ಉಮಾಕರಸಮುದ್ಭವಾ ॥೧೯॥
॥ಇತಿ ಶ್ರೀ ಗಂಗಾಷ್ಟೋತ್ತರಶತನಾಮಸ್ತೋತ್ರಂ ಸಂಪೂರ್ಣಮ್॥
