अस्य श्री कामेश्वरीनित्यामहामन्त्रस्य सम्मोहन ऋषिः~। गायत्रीच्छन्दः~। श्रीकामेश्वरी देवता~। क्लीं इति बीजम्~। सौः इति शक्तिः। ऐं कीलकम्~।

ॐ ऐं क्लीं सौः अङ्गुष्ठाभ्यां नमः-हृदयाय नमः~। 
ॐ ॐ नमः कामेश्वरि तर्जनीभ्यां नमः-शिरसे स्वाहा~।
ॐ इच्छाकामफलप्रदे मध्यमाभ्यां नमः-शिखायै वषट्~।
ॐ सर्वसत्ववशङ्करि अनामिकाभ्यां नमः-कवचाय हुं~।
ॐ सर्वजगत्क्षोभणकरि कनिष्ठिकाभ्यां नमः-नेत्रत्रयाय वौषट्~।
ॐ सौः क्लीं ऐं करतलकरपृष्ठाभ्यां नमः-अस्त्राय फट्~।
भूर्भुवःसुवरोमिति दिग्बन्धः ।

ध्यानम्
देवीं ध्यायेज्जगद्धात्रीं जपाकुसुमसन्निभाम् ।
बालभानुप्रतीकाशां शातकुम्भसमप्रभाम् ॥
रक्तवस्त्रपरीधानां सम्पद्विद्यावशङ्करीम् ।
नमामि वरदां देवीं कामेशीमभयप्रदाम् ॥
पञ्चोपचारैः सम्पूज्य-

मनुः-ॐ ऐं क्लीं सौः ॐ नमः कामेश्वरि इच्छाकामफलप्रदे सर्वसत्ववशङ्करि सर्वजगत्क्षोभणकरि हुं हुं हुं द्रां द्रीं क्लीं ब्लूं सः सौः क्लीं ऐं ॥


मातृकान्यासः।
ॐ ऐं ह्रीं श्रीं अं कामेश्वर्यै नमः हंसः शिरसि~। 
ॐ ऐं ह्रीं श्रीं आं महामायायै नमः हंसः मुखवृत्ते~। 
ॐ ऐं ह्रीं श्रीं इं वागीश्यै नमः हंसः दक्षनेत्रे~। 
ॐ ऐं ह्रीं श्रीं ईं ब्रह्मसञ्ज्ञितायै नमः हंसः वामनेत्रे~। 
ॐ ऐं ह्रीं श्रीं उं अक्षरायै नमः हंसः दक्षकर्णे~। 
ॐ ऐं ह्रीं श्रीं ऊं त्रिमात्रे नमः हंसः वामकर्णे~। 
ॐ ऐं ह्रीं श्रीं ऋं त्रिपदायै नमः हंसः दक्षनासापुटे~। 
ॐ ऐं ह्रीं श्रीं ॠं त्रिगुणात्मिकायै नमः हंसः वामनासापुटे~। 
ॐ ऐं ह्रीं श्रीं ऌं सुरसिद्ध्यै नमः हंसः दक्षगण्डे~। 
ॐ ऐं ह्रीं श्रीं ऌृं गणाध्यक्षायै नमः हंसः वामगण्डे~। 
ॐ ऐं ह्रीं श्रीं एं गणमात्रे नमः हंसः ओष्ठे~। 
ॐ ऐं ह्रीं श्रीं ऐं गणेश्वर्यै नमः हंसः अधरे~। 
ॐ ऐं ह्रीं श्रीं ओं चण्डिकायै नमः हंसः ऊर्ध्वदन्तपङ्क्तौ~। 
ॐ ऐं ह्रीं श्रीं औं चण्डमुण्डायै नमः हंसः अधोदन्तपङ्क्तौ~। 
ॐ ऐं ह्रीं श्रीं अं चामुण्डायै नमः हंसः जिह्वाग्रे~। 
ॐ ऐं ह्रीं श्रीं अः इष्टदायै नमः हंसः कण्ठदेशे~। 
ॐ ऐं ह्रीं श्रीं कं स्वराणांशक्त्यै नमः हंसः दक्षबाहुमूले~। 
ॐ ऐं ह्रीं श्रीं खं सर्वसिद्धिप्रदायिकायै नमः हंसः दक्षकूर्परे~। 
ॐ ऐं ह्रीं श्रीं गं विश्वम्भरायै नमः हंसः दक्षमणिबन्धे~। 
ॐ ऐं ह्रीं श्रीं घं विश्वयोन्यै नमः हंसः दक्षहस्ताङ्गुलिमूले~। 
ॐ ऐं ह्रीं श्रीं ङं विश्वमात्रे नमः हंसः दक्षहस्ताङ्गुल्यग्रे~। 
ॐ ऐं ह्रीं श्रीं चं वसुप्रदायै नमः हंसः वामबाहुमूले~। 
ॐ ऐं ह्रीं श्रीं छं स्वाहायै नमः हंसः वामकूर्परे~। 
ॐ ऐं ह्रीं श्रीं जं स्वधायै नमः हंसः वाममणिबन्धे~। 
ॐ ऐं ह्रीं श्रीं झं तुष्ट्यै नमः हंसः वामहस्ताङ्गुलिमूले~। 
ॐ ऐं ह्रीं श्रीं ञं ऋद्ध्यै नमः हंसः वामहस्ताङ्गुल्यग्रे~। 
ॐ ऐं ह्रीं श्रीं टं गायत्र्यै नमः हंसः दक्षोरुमूले~। 
ॐ ऐं ह्रीं श्रीं ठं गोगणाख्यगायै नमः हंसः दक्षजानौ~। 
ॐ ऐं ह्रीं श्रीं डं वेदमात्रे नमः हंसः दक्षपाङ्गुलिमूले~। 
ॐ ऐं ह्रीं श्रीं ढं वरिष्ठायै नमः हंसः दक्षगुल्फे~। 
ॐ ऐं ह्रीं श्रीं णं सुप्रभायै नमः हंसः दक्षपाङ्गुल्यग्रे~। 
ॐ ऐं ह्रीं श्रीं तं सिद्धवाहिन्यै नमः हंसः वामोरुमूले~।
ॐ ऐं ह्रीं श्रीं थं अादित्यहृदयायै नमः हंसः वामजानुनि~। 
ॐ ऐं ह्रीं श्रीं दं चन्द्रायै नमः हंसः वामगुल्फे~। 
ॐ ऐं ह्रीं श्रीं धं चन्द्रभावानुमण्डलायै नमः हंसः वामपादाङ्गुलिमूले~। 
ॐ ऐं ह्रीं श्रीं नं ज्योत्स्नायै नमः हंसः वामपादाङ्गुल्यग्रे~। 
ॐ ऐं ह्रीं श्रीं पं हिरण्मय्यै नमः हंसः दक्षपार्श्वे~। 
ॐ ऐं ह्रीं श्रीं फं भव्यायै नमः हंसः वामपार्श्वे~।
ॐ ऐं ह्रीं श्रीं बं भवदुःखभयापहायै नमः हंसः पृष्ठे~।
ॐ ऐं ह्रीं श्रीं भं शिवतत्वायै नमः हंसः नाभौ~। 
ॐ ऐं ह्रीं श्रीं मं शिवायै नमः हंसः उदरे ।
ॐ ऐं ह्रीं श्रीं यं शान्तायै नमः हंसः हृदि~। 
ॐ ऐं ह्रीं श्रीं रं शान्तिदायै नमः हंसः दक्षांसे~। 
ॐ ऐं ह्रीं श्रीं लं शान्तरूपिण्यै नमः हंसः ककुदि~। 
ॐ ऐं ह्रीं श्रीं वं सौभाग्यदायै नमः हंसः वामांसे~। 
ॐ ऐं ह्रीं श्रीं शं शुभायै नमः हंसः हृदयादिदक्षहस्तान्तम्~।
ॐ ऐं ह्रीं श्रीं षं गौर्यै नमः हंसः हृदयादिवामहस्तान्तम्~। 
ॐ ऐं ह्रीं श्रीं सं उमायै नमः हंसः हृदयादिदक्षपादन्तम्~। 
ॐ ऐं ह्रीं श्रीं हं हैमवत्यै नमः हंसः हृदयादिवामपादान्तम्। 
ॐ ऐं ह्रीं श्रीं ळं प्रियायै नमः हंसः हृदयादिपादान्तम् ~। 
ॐ ऐं ह्रीं श्रीं क्षं दक्षायै नमः हंसः हृदयादिब्रह्मरन्ध्रान्तम् ~।

सम्मोहनऋषये नमः (शिरसि)। गायत्रीच्छन्दसे नमः (मुखे)~। श्रीकामेश्वरीदेवतायै नमः (हृदये)। कंबीजाय नमः (गुह्ये)। इंशक्तये नमः(पादयोः)।
लंकीलकाय नमः॥\\
ॐ ऐं हृदयाय नमः~। ॐ सकलह्रीं शिरसे स्वाहा~। ॐ नित्य शिखायै वषट्~। ॐ क्लिन्ने कवचाय हुं~। ॐ मदद्रवे नेत्रत्रयाय वौषट्~। ॐ सौः अस्त्राय फट्~।
ॐ ऐं दक्षनेत्रे ॐ सकलह्रीं वामनेत्रे । ॐ नि दक्षकर्णे । ॐ त्य वामकर्णे । ॐ क्लि दक्षनासायाम् । ॐ न्ने वामनासायाम् । ॐ म जिह्वायाम् । ॐ द हृदये । ॐ द्र नाभौ । ॐ वे गुह्ये । ॐ सौः सर्वाङ्गे ।
मनुः- ऐं सकलह्रीं नित्यक्लिन्ने मदद्रवे सौः ॥

अग्नीशासुरवायुकोणेषु चतुर्दिक्षु च -
ॐ ऐं हृदयाय नमः~।हृदयशक्तिपादुकां पूजयामि नमः
ॐ सकलह्रीं शिरसे स्वाहा~।शिरःशक्तिपादुकां पूजयामि नमः
ॐ नित्य शिखायै वषट्~।शिखाशक्तिपादुकां पूजयामि नमः
ॐ क्लिन्ने कवचाय हुं~।कवचशक्तिपादुकां पूजयामि नमः
ॐ मदद्रवे नेत्रत्रयाय वौषट्~।नेत्रशक्तिपादुकां पूजयामि नमः
ॐ सौः अस्त्राय फट्~।अस्त्रशक्तिपादुकां पूजयामि नमः
बालार्ककोटिसङ्काशां माणिक्यमुकुटोज्वलाम् ।
हारग्रैवेयकाञ्चीभिरूर्मिकानूपुरादिभिः ॥
मण्डितां रक्तवसनां रत्नाभरणशोभिताम् ।
षड्भुजां त्रीक्षणामिन्दुकलाकलितमौलिकाम् ॥
पञ्चाष्ट-षोडश-द्वन्द्व-षट्कोण-चतुरस्रगाम् ।
मन्दस्मितोल्लसद्वक्त्रां दयामन्थरवीक्षणाम् ॥
पाशाङ्कुशौ च पुण्ड्रेक्षुचापं पुष्पशिलीमुखम् ।
रत्नपात्रं सुधापूर्णं वरदं बिभ्रतीं करैः ॥

देव्याः पृष्ठभागे पञ्चदलाष्टदलयोरन्तराले गुरुत्रयं सम्पूज्य,
पञ्चदलेषु देव्यग्रतः प्रादक्षिण्येन -	
ॐ ह्रीं श्रीं द्रां मदनबाणाय नमः ।
ॐ ह्रीं श्रीं द्रीं उन्मादनबाणाय नमः ।
ॐ ह्रीं श्रीं क्लीं दीपनबाणाय नमः ।
ॐ ह्रीं श्रीं ब्लूं मोहनबाणाय नमः ।
ॐ ह्रीं श्रीं सः शोषणबाणाय नमः ।

तद्बहिः अष्टदलेषु देव्यग्रतः प्रादक्षिण्येन -
ॐ ह्रीं श्रीं अनङ्गकुसुमापादुकां पूजयामि नमः
ॐ ह्रीं श्रीं अनङ्गमेखलापादुकां पूजयामि नमः
ॐ ह्रीं श्रीं अनङ्गमदनापादुकां पूजयामि नमः
ॐ ह्रीं श्रीं अनङ्गमदनातुरापादुकां पूजयामि नमः
ॐ ह्रीं श्रीं अनङ्गवेगिनीपादुकां पूजयामि नमः
ॐ ह्रीं श्रीं अनङ्गुवनपालापादुकां पूजयामि नमः
ॐ ह्रीं श्रीं अनङ्गशशिरेखापादुकां पूजयामि नमः
ॐ ह्रीं श्रीं अनङ्गगगनरेखापादुकां पूजयामि नमः

तद्बहिः षोडशदलेषु देव्यग्रतः प्रादक्षिण्येन -
ॐ ह्रीं श्रीं अं श्रद्धापादुकां पूजयामि नमः
ॐ ह्रीं श्रीं आं प्रीतिपादुकां पूजयामि नमः
ॐ ह्रीं श्रीं इं रतिपादुकां पूजयामि नमः
ॐ ह्रीं श्रीं ईं धृतिपादुकां पूजयामि नमः
ॐ ह्रीं श्रीं उं कान्तिपादुकां पूजयामि नमः
ॐ ह्रीं श्रीं ऊं मनोरमापादुकां पूजयामि नमः
ॐ ह्रीं श्रीं ऋं मनोहरापादुकां पूजयामि नमः
ॐ ह्रीं श्रीं ॠं मनोरथापादुकां पूजयामि नमः
ॐ ह्रीं श्रीं ऌं मदनापादुकां पूजयामि नमः
ॐ ह्रीं श्रीं ऌृं उन्मादिनीपादुकां पूजयामि नमः
ॐ ह्रीं श्रीं एं मोहिनीपादुकां पूजयामि नमः
ॐ ह्रीं श्रीं ऐं शङ्खिनीपादुकां पूजयामि नमः
ॐ ह्रीं श्रीं ओं शोषिणीपादुकां पूजयामि नमः
ॐ ह्रीं श्रीं औं वशङ्करीपादुकां पूजयामि नमः
ॐ ह्रीं श्रीं अं शिञ्जिनीपादुकां पूजयामि नमः
ॐ ह्रीं श्रीं अः सुभगापादुकां पूजयामि नमः

षोडशदलाग्रेषु -
ॐ ह्रीं श्रीं अं पूषा पादुकां पूजयामि नमः
ॐ ह्रीं श्रीं आं इद्धा पादुकां पूजयामि नमः
ॐ ह्रीं श्रीं इं सुमनसा पादुकां पूजयामि नमः
ॐ ह्रीं श्रीं ईं रति पादुकां पूजयामि नमः
ॐ ह्रीं श्रीं उं प्रीति पादुकां पूजयामि नमः
ॐ ह्रीं श्रीं ऊं धृति पादुकां पूजयामि नमः
ॐ ह्रीं श्रीं ऋं ऋद्धि पादुकां पूजयामि नमः
ॐ ह्रीं श्रीं ॠं सौम्या पादुकां पूजयामि नमः
ॐ ह्रीं श्रीं ऌं मरीचि पादुकां पूजयामि नमः
ॐ ह्रीं श्रीं ऌृं अंशुमालिनी पादुकां पूजयामि नमः
ॐ ह्रीं श्रीं एं शशिनी पादुकां पूजयामि नमः
ॐ ह्रीं श्रीं ऐं अङ्गिरा पादुकां पूजयामि नमः
ॐ ह्रीं श्रीं ओं छाया पादुकां पूजयामि नमः
ॐ ह्रीं श्रीं औं सम्पूर्णमणडला पादुकां पूजयामि नमः
ॐ ह्रीं श्रीं अं तुष्टि पादुकां पूजयामि नमः
ॐ ह्रीं श्रीं अः अमृता पादुकां पूजयामि नमः

ॐ ह्रीं श्रीं ऐं डाकिनी पादुकां पूजयामि नमः
ॐ ह्रीं श्रीं ऐं राकिणी पादुकां पूजयामि नमः
ॐ ह्रीं श्रीं ऐं लाकिनी पादुकां पूजयामि नमः
ॐ ह्रीं श्रीं ऐं काकिनी पादुकां पूजयामि नमः
ॐ ह्रीं श्रीं ऐं शाकिनी पादुकां पूजयामि नमः
ॐ ह्रीं श्रीं ऐं हाकिनी पादुकां पूजयामि नमः

षट्कोणाद्बहिः चतुरस्राभ्यन्तरे 
ॐ ह्रीं श्रीं ऐं वं वटुक पादुकां पूजयामि नमः
ॐ ह्रीं श्रीं ऐं गं गणपति पादुकां पूजयामि नमः
ॐ ह्रीं श्रीं ऐं दुं दुर्गा पादुकां पूजयामि नमः
ॐ ह्रीं श्रीं ऐं क्षं क्षेत्रपाल पादुकां पूजयामि नमः

तद्बहिः देव्याः पृष्ठभागमारभ्य पूर्वादिदशदिक्षु -
ॐ ह्रीं श्रीं ऐं लं इन्द्रशक्तिपादुकां पूजयामि नमः
ॐ ह्रीं श्रीं ऐं रं अग्नशक्तिपादुकां पूजयामि नमः
ॐ ह्रीं श्रीं ऐं मं यमशक्तिपादुकां पूजयामि नमः
ॐ ह्रीं श्रीं ऐं क्षं निर्ऋतशक्तिपादुकां पूजयामि नमः
ॐ ह्रीं श्रीं ऐं वं वरुणशक्तिपादुकां पूजयामि नमः
ॐ ह्रीं श्रीं ऐं यं वायुशक्तिपादुकां पूजयामि नमः
ॐ ह्रीं श्रीं ऐं कुं कुबेरशक्तिपादुकां पूजयामि नमः
ॐ ह्रीं श्रीं ऐं हं ईशानशक्तिपादुकां पूजयामि नमः
ॐ ह्रीं श्रीं ऐं आं ब्रह्मशक्तिपादुकां पूजयामि नमः
ॐ ह्रीं श्रीं ऐं ह्रीं अनंतशक्तिपादुकां पूजयामि नमः

एवमेव तत्रैव तदायुधानि पूजयेत् -
ॐ ह्रीं श्रीं ऐं वं वज्रशक्तिपादुकां पूजयामि नमः
ॐ ह्रीं श्रीं ऐं शं शक्तिशक्तिपादुकां पूजयामि नमः
ॐ ह्रीं श्रीं ऐं दं दंडशक्तिपादुकां पूजयामि नमः
ॐ ह्रीं श्रीं ऐं खं खड्गशक्तिपादुकां पूजयामि नमः
ॐ ह्रीं श्रीं ऐं पां पाशशक्तिपादुकां पूजयामि नमः
ॐ ह्रीं श्रीं ऐं अं अंङ्कुशशक्तिपादुकां पूजयामि नमः
ॐ ह्रीं श्रीं ऐं गं गदाशक्तिपादुकां पूजयामि नमः
ॐ ह्रीं श्रीं ऐं त्रिं त्रिशूलशक्तिपादुकां पूजयामि नमः
ॐ ह्रीं श्रीं ऐं पं पद्मशक्तिपादुकां पूजयामि नमः
ॐ ह्रीं श्रीं ऐं चं चक्रशक्तिपादुकां पूजयामि नमः
कामेश्वर्यै विद्महे नित्यक्लिन्नायै धीमहि तन्नो नित्या प्रचोदयात् ॥
इति कामेश्वरी-अावरणपूजा ।

भगमालिनीनित्या 
ऐं भगभुगे भगिनि भगोदरि भगमाले भगावहे भगगुह्ये भगयोनि भगनिपातिनि सर्वभगवशङ्करि भगरुपे नित्यक्लिन्ने भगस्वरुपे सर्वभगानि मे ह्यानय वरदे रेते सुरेते भगक्लिन्ने क्लिन्नद्रवे क्लेदय द्रावय अमोघे भगविच्चे क्षुभ क्षोभय सर्वसत्वान् भगेश्वरि ऐं ब्लूं जं ब्लूं भें ब्लूं मों ब्लूं हें क्लिन्ने सर्वाणि भगानि मे वशमानय स्त्रीं हर ब्लें ह्रीं ॥

अस्य श्रीभगमालिनीनित्यामहामन्त्रस्य मदनन ऋषिः~। त्रिष्टुप्च्छन्दः~। श्रीभगमालिनी देवता~। ऐं बीजम्~। ह्रीं शक्तिः। ब्लूं कीलकम्~।
ब्लां इत्यादिन्यासः । ध्यानम्-
भगरूपां भगमयां दुकूलवसनां शिवाम् ।
सर्वालङ्कारसंयुक्तां सर्वलोकवशङ्करीम् ॥
भगोदरीं महादेवीं रक्तोत्पलसमप्रभाम् ।
कामेश्वराङ्कनिलयां वन्दे श्रीभगमालिनीम् ॥

न्यासः
ॐ ऐं ह्रीं श्रीं अं भगमालिन्यै नमः हंसः शिरसि~। 
ॐ ऐं ह्रीं श्रीं आं भगायै नमः हंसः मुखवृत्ते~। 
ॐ ऐं ह्रीं श्रीं इं भाग्यायै नमः हंसः दक्षनेत्रे~। 
ॐ ऐं ह्रीं श्रीं ईं भगोदर्यै नमः हंसः वामनेत्रे~। 
ॐ ऐं ह्रीं श्रीं उं गुह्यायै नमः हंसः दक्षकर्णे~। 
ॐ ऐं ह्रीं श्रीं ऊं दाक्षायण्यै नमः हंसः वामकर्णे~। 
ॐ ऐं ह्रीं श्रीं ऋं कन्यायै नमः हंसः दक्षनासापुटे~। 
ॐ ऐं ह्रीं श्रीं ॠं दक्षयज्ञविनाशिन्यै नमः हंसः वामनासापुटे~। 
ॐ ऐं ह्रीं श्रीं ऌं जयायै नमः हंसः दक्षगण्डे~। 
ॐ ऐं ह्रीं श्रीं ऌृं विजयायै नमः हंसः वामगण्डे~। 
ॐ ऐं ह्रीं श्रीं एं अजितायै नमः हंसः ओष्ठे~। 
ॐ ऐं ह्रीं श्रीं ऐं अपराजितायै नमः हंसः अधरे~। 
ॐ ऐं ह्रीं श्रीं ओं सुदीप्तायै नमः हंसः ऊर्ध्वदन्तपङ्क्तौ~। 
ॐ ऐं ह्रीं श्रीं औं लेलिहानायै नमः हंसः अधोदन्तपङ्क्तौ~। 
ॐ ऐं ह्रीं श्रीं अं करालायै नमः हंसः जिह्वाग्रे~। 
ॐ ऐं ह्रीं श्रीं अः स्वरशक्त्यै नमः हंसः कण्ठदेशे~। 
ॐ ऐं ह्रीं श्रीं कं अाकाशनिलयायै नमः हंसः दक्षबाहुमूले~। 
ॐ ऐं ह्रीं श्रीं खं ब्राह्म्यै नमः हंसः दक्षकूर्परे~। 
ॐ ऐं ह्रीं श्रीं गं बालायै नमः हंसः दक्षमणिबन्धे~। 
ॐ ऐं ह्रीं श्रीं घं ब्रह्मचारिण्यै नमः हंसः दक्षहस्ताङ्गुलिमूले~। 
ॐ ऐं ह्रीं श्रीं ङं बाह्यास्यरतायै नमः हंसः दक्षहस्ताङ्गुल्यग्रे~। 
ॐ ऐं ह्रीं श्रीं चं प्रह्व्यै नमः हंसः वामबाहुमूले~। 
ॐ ऐं ह्रीं श्रीं छं सावित्र्यै नमः हंसः वामकूर्परे~। 
ॐ ऐं ह्रीं श्रीं जं ब्रह्मपूजितायै नमः हंसः वाममणिबन्धे~। 
ॐ ऐं ह्रीं श्रीं झं प्राज्ञायय नमः हंसः वामहस्ताङ्गुलिमूले~। 
ॐ ऐं ह्रीं श्रीं ञं मात्रे नमः हंसः वामहस्ताङ्गुल्यग्रे~। 
ॐ ऐं ह्रीं श्रीं टं परायै नमः हंसः दक्षोरुमूले~। 
ॐ ऐं ह्रीं श्रीं ठं बुद्ध्यै नमः हंसः दक्षजानौ~। 
ॐ ऐं ह्रीं श्रीं डं विश्वमात्रे नमः हंसः दक्षपाङ्गुलिमूले~। 
ॐ ऐं ह्रीं श्रीं ढं शाश्वत्यै नमः हंसः दक्षगुल्फे~। 
ॐ ऐं ह्रीं श्रीं णं मैत्र्यै नमः हंसः दक्षपाङ्गुल्यग्रे~। 
ॐ ऐं ह्रीं श्रीं तं कात्यायन्यै नमः हंसः वामोरुमूले~।
ॐ ऐं ह्रीं श्रीं थं दुर्गायै नमः हंसः वामजानुनि~। 
ॐ ऐं ह्रीं श्रीं दं दुर्गसन्तारिण्यै नमः हंसः वामगुल्फे~। 
ॐ ऐं ह्रीं श्रीं धं परायै नमः हंसः वामपादाङ्गुलिमूले~। 
ॐ ऐं ह्रीं श्रीं नं मूलप्रकृत्यै नमः हंसः वामपादाङ्गुल्यग्रे~। 
ॐ ऐं ह्रीं श्रीं पं ईशानायै नमः हंसः दक्षपार्श्वे~। 
ॐ ऐं ह्रीं श्रीं फं पुंस्प्रदानेश्वर्यै नमः हंसः वामपार्श्वे~।
ॐ ऐं ह्रीं श्रीं बं अाप्यायिन्यै नमः हंसः पृष्ठे~।
ॐ ऐं ह्रीं श्रीं भं पावन्यै नमः हंसः नाभौ~। 
ॐ ऐं ह्रीं श्रीं मं पवित्रायै नमः हंसः उदरे ।
ॐ ऐं ह्रीं श्रीं यं मङ्गलायै नमः हंसः हृदि~। 
ॐ ऐं ह्रीं श्रीं रं यमायै नमः हंसः दक्षांसे~। 
ॐ ऐं ह्रीं श्रीं लं ज्योतिष्मत्यै नमः हंसः ककुदि~। 
ॐ ऐं ह्रीं श्रीं वं संहारिण्यै नमः हंसः वामांसे~। 
ॐ ऐं ह्रीं श्रीं शं सृष्ट्यै नमः हंसः हृदयादिदक्षहस्तान्तम्~।
ॐ ऐं ह्रीं श्रीं षं स्थित्यै नमः हंसः हृदयादिवामहस्तान्तम्~। 
ॐ ऐं ह्रीं श्रीं सं अन्तकारिण्यै नमः हंसः हृदयादिदक्षपादन्तम्~। 
ॐ ऐं ह्रीं श्रीं हं अघोरायै नमः हंसः हृदयादिवामपादान्तम्। 
ॐ ऐं ह्रीं श्रीं ळं घोररूपायै नमः हंसः हृदयादिपादान्तम् ~। 
ॐ ऐं ह्रीं श्रीं क्षं शक्त्यै नमः हंसः हृदयादिमस्तकान्तम् ~।

सुभगाय ऋषये नमः । गायत्रीच्छन्दसे नमः । श्रीभगमालिनीदेवतयै नमः । हरब्लें बीजाय नमः । श्रीं शक्तये नमः । क्लीं कीलकाय नमः ।


ॐ ऐं अङ्गुष्ठाभ्यां नमः-हृदयाय नमः~। 
ॐ भगभुगे तर्जनीभ्यां नमः-शिरसे स्वाहा~।
ॐ भगिनि मध्यमाभ्यां नमः-शिखायै वषट्~।
ॐ भगोदरि अनामिकाभ्यां नमः-कवचाय हुं~।
ॐ भगमाले कनिष्ठिकाभ्यां नमः-नेत्रत्रयाय वौषट्~।
ॐ भगावहे करतलकरपृष्ठाभ्यां नमः-अस्त्राय फट्~।
मूलेन व्यापकं कुर्यात् ।
ध्यानम्
अरुणामरुणाकल्पां सुंदरीं सुस्मिताननाम् ।
त्रिनेत्रां बाहुभिः षड्भिरुपेतां कमलासनाम् ॥
कह्लारपाशपुंड्रेक्षुकोदंडान्वामबाहुभिः ।
दधानां दक्षिणैः पद्ममङ्कुशं पुष्पसायकम् ॥
तथाविधाभिः परितो युतां शक्तिगणैः स्तुतैः ।
अक्षरोक्ताभिरन्याभिः स्मरोन्मादमदात्मभिः ॥
पञ्चत्रिंशच्छतार्णैस्तैः रूपिणी शक्तिपञ्चकम् ।
सप्ताक्षरीं च संयोज्य शक्तीस्तत्सङ्ख्यका यजेत् ॥

अग्नीशासुरवायुकोणेषु चतुर्दिक्षु च षडङ्गपूजनम् -
ॐ ऐं हृदयाय नमः~।हृदयशक्तिपादुकां पूजयामि नमः
ॐ भगभुगे शिरसे स्वाहा~।शिरःशक्तिपादुकां पूजयामि नमः
ॐ भगिनि शिखायै वषट्~।शिखाशक्तिपादुकां पूजयामि नमः
ॐ भगोदरि कवचाय हुं~।कवचशक्तिपादुकां पूजयामि नमः
ॐ भगमाले नेत्रत्रयाय वौषट्~।नेत्रशक्तिपादुकां पूजयामि नमः
ॐ भगावहे अस्त्राय फट्~।अस्त्रशक्तिपादुकां पूजयामि नमः

ॐऐं रागशक्तिपादुकां पूजयामि नमः
ॐऐं द्वेषशक्तिपादुकां पूजयामि नमः
ॐ ऐं मदनापादुकां पूजयामि नमः
ॐ ऐं मोहिनीपादुकां पूजयामि नमः
ॐ ऐं लोलापादुकां पूजयामि नमः
ॐ ऐं जम्भिनीपादुकां पूजयामि नमः
ॐ ऐं उद्यमापादुकां पूजयामि नमः
ॐ ऐं शुभापादुकां पूजयामि नमः
ॐ ऐं ह्लादिनीपादुकां पूजयामि नमः
ॐ ह्रीं श्रीं द्राविणीपादुकां पूजयामि नमः
ॐ ह्रीं श्रीं प्रीतिपादुकां पूजयामि नमः
ॐ ह्रीं श्रीं रतिपादुकां पूजयामि नमः
ॐ ह्रीं श्रीं रक्तापादुकां पूजयामि नमः
ॐ ह्रीं श्रीं मनोरमापादुकां पूजयामि नमः
ॐ ह्रीं श्रीं सर्वोन्मादापादुकां पूजयामि नमः
ॐ ह्रीं श्रीं सर्वसुखापादुकां पूजयामि नमः
ॐ ह्रीं श्रीं अनङ्गापादुकां पूजयामि नमः
ॐ ह्रीं श्रीं अभितोद्यमापादुकां पूजयामि नमः
ॐ ह्रीं श्रीं अनल्पापादुकां पूजयामि नमः
ॐ ह्रीं श्रीं व्यक्तविभवापादुकां पूजयामि नमः
ॐ ह्रीं श्रीं विविधविग्रहापादुकां पूजयामि नमः
ॐ ह्रीं श्रीं क्षोभविग्रहापादुकां पूजयामि नमः
ॐ ह्रीं श्रीं इक्षुकोदण्डायपादुकां पूजयामि नमः
ॐ ह्रीं श्रीं पाशपादुकां पूजयामि नमः
ॐ ह्रीं श्रीं कह्लारपादुकां पूजयामि नमः
ॐ ह्रीं श्रीं पद्मपादुकां पूजयामि नमः
ॐ ह्रीं श्रीं अङ्कुशपादुकां पूजयामि नमः
ॐ ह्रीं श्रीं पुष्पसायकपादुकां पूजयामि नमः

नित्यक्लिन्ना 
अरुणामरुणाकल्पामरुणांशुकधारिणीम् । अरुणासृग्विलेपां तां चारुस्मेरमुखाम्बुजाम् ॥
नेत्रत्रयोल्लसद्वक्त्रां भाले घर्माम्बुमौक्तिकैः । विराजमानां मन्दारलसदर्धेन्दुशेखराम् ॥
चतुर्भिर्बाहुभिः पाशमङ्कुशं पानपात्रकम् । अभयं बिभ्रतीं पद्ममध्यासीनां मदालसाम् ॥

अस्य श्रीनित्यक्लिन्नामहामन्त्रस्य ब्रह्मा ऋषिः । विराट् छन्दः । श्रीनित्यक्लिन्ना देवता । ह्रां बीजम् । स्वाहा शक्तिः । क्लीं कीलकम् ।
ह्रां इत्यादिः न्यासः ॥
पाद्मरागमणिप्रख्यां हेमताटङ्कसंयुताम् । रक्तवस्त्रधरां देवीं रक्तमाल्यानुलेपनाम् ॥
अञ्जनाञ्चितनेत्रां तां पद्मपत्रनिभेक्षणाम् । नित्यक्लिन्नां नमस्यामि चतुर्भुजविराजिताम् ॥

ॐ ऐं ह्रीं श्रीं अं नित्यक्लिन्नायै नमः हंसः शिरसि~। 
ॐ ऐं ह्रीं श्रीं आं नित्यमदद्रवायै नमः हंसः मुखवृत्ते~। 
ॐ ऐं ह्रीं श्रीं इं विश्वरूपिण्यै नमः हंसः दक्षनेत्रे~। 
ॐ ऐं ह्रीं श्रीं ईं योगेश्वर्यै नमः हंसः वामनेत्रे~। 
ॐ ऐं ह्रीं श्रीं उं योगगम्यायै नमः हंसः दक्षकर्णे~। 
ॐ ऐं ह्रीं श्रीं ऊं योगमात्रे नमः हंसः वामकर्णे~। 
ॐ ऐं ह्रीं श्रीं ऋं वसुन्धरायै नमः हंसः दक्षनासापुटे~। 
ॐ ऐं ह्रीं श्रीं ॠं धन्यायै नमः हंसः वामनासापुटे~। 
ॐ ऐं ह्रीं श्रीं ऌं धनेश्वर्यै नमः हंसः दक्षगण्डे~। 
ॐ ऐं ह्रीं श्रीं ऌृं धान्यरत्नदायै नमः हंसः वामगण्डे~। 
ॐ ऐं ह्रीं श्रीं एं पशुवर्धिन्यै नमः हंसः ओष्ठे~। 
ॐ ऐं ह्रीं श्रीं ऐं कूष्माण्डिन्यै नमः हंसः अधरे~। 
ॐ ऐं ह्रीं श्रीं ओं दारुण्यै नमः हंसः ऊर्ध्वदन्तपङ्क्तौ~। 
ॐ ऐं ह्रीं श्रीं औं चण्ड्यै नमः हंसः अधोदन्तपङ्क्तौ~। 
ॐ ऐं ह्रीं श्रीं अं स्वराणांशक्त्यै नमः हंसः जिह्वाग्रे~। 
ॐ ऐं ह्रीं श्रीं अः घोरायै नमः हंसः कण्ठदेशे~। 
ॐ ऐं ह्रीं श्रीं कं घोरस्वरूपायै नमः हंसः दक्षबाहुमूले~। 
ॐ ऐं ह्रीं श्रीं खं मातृकायै नमः हंसः दक्षकूर्परे~। 
ॐ ऐं ह्रीं श्रीं गं माध्व्यै नमः हंसः दक्षमणिबन्धे~। 
ॐ ऐं ह्रीं श्रीं घं दक्षायै नमः हंसः दक्षहस्ताङ्गुलिमूले~। 
ॐ ऐं ह्रीं श्रीं ङं एकाक्षरायै नमः हंसः दक्षहस्ताङ्गुल्यग्रे~। 
ॐ ऐं ह्रीं श्रीं चं विश्वमूर्त्यै नमः हंसः वामबाहुमूले~। 
ॐ ऐं ह्रीं श्रीं छं विश्वायै नमः हंसः वामकूर्परे~। 
ॐ ऐं ह्रीं श्रीं जं विश्वेश्वर्यै नमः हंसः वाममणिबन्धे~। 
ॐ ऐं ह्रीं श्रीं झं ध्रुवायै नमः हंसः वामहस्ताङ्गुलिमूले~। 
ॐ ऐं ह्रीं श्रीं ञं शर्वायै नमः हंसः वामहस्ताङ्गुल्यग्रे~। 
ॐ ऐं ह्रीं श्रीं टं क्ष्मादिभूतात्मने नमः हंसः दक्षोरुमूले~। 
ॐ ऐं ह्रीं श्रीं ठं भूतिदायै नमः हंसः दक्षजानौ~। 
ॐ ऐं ह्रीं श्रीं डं भूतवर्धिन्यै नमः हंसः दक्षपाङ्गुलिमूले~। 
ॐ ऐं ह्रीं श्रीं ढं भूतेश्वरप्रियायै नमः हंसः दक्षगुल्फे~। 
ॐ ऐं ह्रीं श्रीं णं भूत्यै नमः हंसः दक्षपाङ्गुल्यग्रे~। 
ॐ ऐं ह्रीं श्रीं तं भूतमालायै नमः हंसः वामोरुमूले~।
ॐ ऐं ह्रीं श्रीं थं यौवन्यै नमः हंसः वामजानुनि~। 
ॐ ऐं ह्रीं श्रीं दं वैदेहीपूजितायै नमः हंसः वामगुल्फे~। 
ॐ ऐं ह्रीं श्रीं धं सीतायै नमः हंसः वामपादाङ्गुलिमूले~। 
ॐ ऐं ह्रीं श्रीं नं मायाव्यै नमः हंसः वामपादाङ्गुल्यग्रे~। 
ॐ ऐं ह्रीं श्रीं पं भववाहिन्यै नमः हंसः दक्षपार्श्वे~। 
ॐ ऐं ह्रीं श्रीं फं सत्वस्थायै नमः हंसः वामपार्श्वे~।
ॐ ऐं ह्रीं श्रीं बं सत्वनिलायायै नमः हंसः पृष्ठे~।
ॐ ऐं ह्रीं श्रीं भं सत्वासत्वचिकीर्षणायै नमः हंसः नाभौ~। 
ॐ ऐं ह्रीं श्रीं मं विश्वस्थायै नमः हंसः उदरे ।
ॐ ऐं ह्रीं श्रीं यं विश्वनिलयायै नमः हंसः हृदि~। 
ॐ ऐं ह्रीं श्रीं रं श्रीफलायै नमः हंसः दक्षांसे~। 
ॐ ऐं ह्रीं श्रीं लं श्रीनिकेतनायै नमः हंसः ककुदि~। 
ॐ ऐं ह्रीं श्रीं वं श्रियै नमः हंसः वामांसे~। 
ॐ ऐं ह्रीं श्रीं शं शशाङ्कधरायै नमः हंसः हृदयादिदक्षहस्तान्तम्~।
ॐ ऐं ह्रीं श्रीं षं नन्दायै नमः हंसः हृदयादिवामहस्तान्तम्~। 
ॐ ऐं ह्रीं श्रीं सं अम्बिकायै नमः हंसः हृदयादिदक्षपादन्तम्~। 
ॐ ऐं ह्रीं श्रीं हं अनन्दायै नमः हंसः हृदयादिवामपादान्तम्। 
ॐ ऐं ह्रीं श्रीं ळं अनादिनिधनायै नमः हंसः हृदयादिपादान्तम् ~। 
ॐ ऐं ह्रीं श्रीं क्षं अानन्दकलिकायै नमः हंसः हृदयादिमस्तकान्तम् ~।

ब्रह्मर्षये नमः (शिरसि)। विराट्-छन्दसे नमः (मुखे)~। श्रीनित्यक्लिन्नानित्यादेवतायै नमः (हृदये)। ह्रीं बीजाय नमः (गुह्ये)। स्वाहा शक्तये नमः(पादयोः)।
न्ने कीलकाय नमः॥\\
ह्रीं नित्य क्लिन्ने मद द्रवे स्वाहा 


ॐ ह्रीं दक्षनेत्रे । ॐ नि वामनेत्रे । ॐ त्य दक्षकर्णे । ॐ क्लि वामकर्णे । ॐ न्ने दक्षनासायाम् । ॐ म वामनासायाम् । ॐ द जिह्वायाम् । ॐ द्र हृदये । ॐ वे नाभौ । ॐ स्वा गुह्ये । ॐ हा सर्वाङ्गे ।
चतुरस्रे पश्चिमादारभ्य दशदिक्षु -
ॐ ह्रीं श्रीं मदाविलायैपादुकां पूजयामि नमः
ॐ ह्रीं श्रीं मङ्गलायैपादुकां पूजयामि नमः
ॐ ह्रीं श्रीं मन्मथार्तायैपादुकां पूजयामि नमः
ॐ ह्रीं श्रीं मनस्विन्यैपादुकां पूजयामि नमः
ॐ ह्रीं श्रीं मोहायैपादुकां पूजयामि नमः
ॐ ह्रीं श्रीं अामोदायैपादुकां पूजयामि नमः
ॐ ह्रीं श्रीं मानमय्यैपादुकां पूजयामि नमः
ॐ ह्रीं श्रीं मायायैपादुकां पूजयामि नमः
ॐ ह्रीं श्रीं मन्दायैपादुकां पूजयामि नमः
ॐ ह्रीं श्रीं मनोवत्यैपादुकां पूजयामि नमः

तदन्तः अष्टदलपद्मे पश्चिमादारभ्य प्रादक्षिण्येन -
ॐ ह्रीं श्रीं नित्यायैपादुकां पूजयामि नमः
ॐ ह्रीं श्रीं निरञ्जनायैपादुकां पूजयामि नमः
ॐ ह्रीं श्रीं क्लिन्नायैपादुकां पूजयामि नमः
ॐ ह्रीं श्रीं क्लेदिन्यैपादुकां पूजयामि नमः
ॐ ह्रीं श्रीं मदनातुरायैपादुकां पूजयामि नमः
ॐ ह्रीं श्रीं मदद्रवायैपादुकां पूजयामि नमः
ॐ ह्रीं श्रीं द्राविण्यैपादुकां पूजयामि नमः
ॐ ह्रीं श्रीं द्रविणायैपादुकां पूजयामि नमः

तदन्तः त्रिकोणे स्वाग्रात्प्रादक्षिण्येन -
ॐ ह्रीं श्रीं क्षोभिण्यैपादुकां पूजयामि नमः
ॐ ह्रीं श्रीं मोहिन्यैपादुकां पूजयामि नमः
ॐ ह्रीं श्रीं लोलायैपादुकां पूजयामि नमः

वाय्वादिकोणेषु चतुर्दिक्षु च षडङ्गार्चनम् ।

भेरुण्डा
तप्तकाञ्चनसङ्काशदेहां नेत्रत्रयान्विताम्  ।
चारुस्मिताञ्चितमुखीं दिव्यालङ्कारभूषिताम्  ॥
ताटङ्कहारकेयूररत्नवस्त्रकमण्डिताम्  ।
रसनानूपुरोर्म्यादिभूषणैरतिसुन्दरीम्  ॥
पाशाङ्कुशौ चर्मखङ्गौ गदावज्रधनुःशरान्  ।
करैर्दधानामासीना पूजायामन्यदास्थिताम् ॥
शक्तीश्च तत्समाकारतेजोहेतिभिरन्विताः ।
पूजयेत्तद्वदभितः स्मितसौम्यमुखः सदा ॥

अस्य श्रीभेरुणडानित्यामहामन्त्रस्य महाविष्णुर्ऋषिः। गायत्री छन्दः । श्रीभेरुण्डानित्या देवता । भ्रों बीजम् । स्वाहा शक्तिः । क्रों कीलकम् ॥
भ्रां इत्यादिः न्यासः ॥
भूर्भुवःसुवरोमिति दिग्बन्धः ।
शुद्धस्फटिकसङ्काशां पद्मपत्रसमप्रभाम् । मध्याह्नादित्यसङ्काशां शुभ्रवस्त्रसमन्विताम् ॥
श्वेतचन्दनलिप्ताङ्गीं शुभ्रमाल्यविभूषिताम् । बिभ्रतीं चिन्मयां मुद्रामक्षमालां च पुस्तकम् ॥
सहस्रपद्मकमले समासीनां शुचिस्मिताम् । सर्वविद्याप्रदां देवीं भेरुण्डां प्रणमाम्यहम् ॥
पञ्चोपचारैः सम्पूज्य -

ॐ ऐं ह्रीं श्रीं अं भेरुण्डायै नमः हंसः शिरसि~। 
ॐ ऐं ह्रीं श्रीं आं भैरव्यै नमः हंसः मुखवृत्ते~। 
ॐ ऐं ह्रीं श्रीं इं साध्व्यै नमः हंसः दक्षनेत्रे~। 
ॐ ऐं ह्रीं श्रीं ईं नताख्यानन्तसम्भवा नमः हंसः वामनेत्रे~। 
ॐ ऐं ह्रीं श्रीं उं त्रिगुण्यै नमः हंसः दक्षकर्णे~। 
ॐ ऐं ह्रीं श्रीं ऊं घोषिण्यै नमः हंसः वामकर्णे~। 
ॐ ऐं ह्रीं श्रीं ऋं घोषायै नमः हंसः दक्षनासापुटे~। 
ॐ ऐं ह्रीं श्रीं ॠं लक्ष्म्यै नमः हंसः वामनासापुटे~। 
ॐ ऐं ह्रीं श्रीं ऌं पुष्टायै नमः हंसः दक्षगण्डे~। 
ॐ ऐं ह्रीं श्रीं ऌृं शुभायै नमः हंसः वामगण्डे~। 
ॐ ऐं ह्रीं श्रीं एं लयायै नमः हंसः ओष्ठे~। 
ॐ ऐं ह्रीं श्रीं ऐं धर्मायै नमः हंसः अधरे~। 
ॐ ऐं ह्रीं श्रीं ओं दयायै नमः हंसः ऊर्ध्वदन्तपङ्क्तौ~। 
ॐ ऐं ह्रीं श्रीं औं धर्मबुद्ध्यै नमः हंसः अधोदन्तपङ्क्तौ~। 
ॐ ऐं ह्रीं श्रीं अं धर्माधर्मपुटच्छयायै नमः हंसः जिह्वाग्रे~। 
ॐ ऐं ह्रीं श्रीं अः ज्येष्ठायै नमः हंसः कण्ठदेशे~। 
ॐ ऐं ह्रीं श्रीं कं यमस्य भगिन्यै नमः हंसः दक्षबाहुमूले~। 
ॐ ऐं ह्रीं श्रीं खं चैलकौशेयवासिन्यै नमः हंसः दक्षकूर्परे~। 
ॐ ऐं ह्रीं श्रीं गं भ्रमणायै नमः हंसः दक्षमणिबन्धे~। 
ॐ ऐं ह्रीं श्रीं घं भ्रामिण्यै नमः हंसः दक्षहस्ताङ्गुलिमूले~। 
ॐ ऐं ह्रीं श्रीं ङं भ्राम्यायै नमः हंसः दक्षहस्ताङ्गुल्यग्रे~। 
ॐ ऐं ह्रीं श्रीं चं भ्रमायै नमः हंसः वामबाहुमूले~। 
ॐ ऐं ह्रीं श्रीं छं ज्ञानापहारिण्यै नमः हंसः वामकूर्परे~। 
ॐ ऐं ह्रीं श्रीं जं माहेन्द्र्यै नमः हंसः वाममणिबन्धे~। 
ॐ ऐं ह्रीं श्रीं झं वारुण्यै नमः हंसः वामहस्ताङ्गुलिमूले~। 
ॐ ऐं ह्रीं श्रीं ञं सौम्यायै नमः हंसः वामहस्ताङ्गुल्यग्रे~। 
ॐ ऐं ह्रीं श्रीं टं कौबेर्यै नमः हंसः दक्षोरुमूले~। 
ॐ ऐं ह्रीं श्रीं ठं हव्यवाहिन्यै नमः हंसः दक्षजानौ~। 
ॐ ऐं ह्रीं श्रीं डं वायव्यै नमः हंसः दक्षपाङ्गुलिमूले~। 
ॐ ऐं ह्रीं श्रीं ढं नैर्ऋत्यायै नमः हंसः दक्षगुल्फे~। 
ॐ ऐं ह्रीं श्रीं णं ईशान्यै नमः हंसः दक्षपाङ्गुल्यग्रे~। 
ॐ ऐं ह्रीं श्रीं तं लोकपालायै नमः हंसः वामोरुमूले~।
ॐ ऐं ह्रीं श्रीं थं एकरूपिण्यै नमः हंसः वामजानुनि~। 
ॐ ऐं ह्रीं श्रीं दं मोहिन्यै नमः हंसः वामगुल्फे~। 
ॐ ऐं ह्रीं श्रीं धं मोहजनन्यै नमः हंसः वामपादाङ्गुलिमूले~। 
ॐ ऐं ह्रीं श्रीं नं स्मृत्यै नमः हंसः वामपादाङ्गुल्यग्रे~। 
ॐ ऐं ह्रीं श्रीं पं वृत्तान्तबाधिन्यै नमः हंसः दक्षपार्श्वे~। 
ॐ ऐं ह्रीं श्रीं फं यक्षाणां जनन्यै नमः हंसः वामपार्श्वे~।
ॐ ऐं ह्रीं श्रीं बं यक्ष्यै नमः हंसः पृष्ठे~।
ॐ ऐं ह्रीं श्रीं भं सिद्ध्यै नमः हंसः नाभौ~। 
ॐ ऐं ह्रीं श्रीं मं वैश्रवणालयायै नमः हंसः उदरे ।
ॐ ऐं ह्रीं श्रीं यं मेधायै नमः हंसः हृदि~। 
ॐ ऐं ह्रीं श्रीं रं श्रद्धायै नमः हंसः दक्षांसे~। 
ॐ ऐं ह्रीं श्रीं लं धृत्यै नमः हंसः ककुदि~। 
ॐ ऐं ह्रीं श्रीं वं प्रज्ञायै नमः हंसः वामांसे~। 
ॐ ऐं ह्रीं श्रीं शं सर्वदेवनमस्कृतायै नमः हंसः हृदयादिदक्षहस्तान्तम्~।
ॐ ऐं ह्रीं श्रीं षं अाशायै नमः हंसः हृदयादिवामहस्तान्तम्~। 
ॐ ऐं ह्रीं श्रीं सं वाञ्छायै नमः हंसः हृदयादिदक्षपादन्तम्~। 
ॐ ऐं ह्रीं श्रीं हं निरीहायै नमः हंसः हृदयादिवामपादान्तम्। 
ॐ ऐं ह्रीं श्रीं ळं इच्छायै नमः हंसः हृदयादिपादान्तम् ~। 
ॐ ऐं ह्रीं श्रीं क्षं भूतानुवर्तिनी नमः हंसः हृदयादिमस्तकान्तम् ~।

महाविष्णवे ऋषये नमः (शिरसि)। गायत्री-छन्दसे नमः (मुखे)~। श्रीभेरुण्डानित्यादेवतायै नमः (हृदये)। भ्रां बीजाय नमः (गुह्ये)। स्वाहा शक्तये नमः(पादयोः)।
क्रों कीलकाय नमः॥
क्रों भ्रों क्रों झ्रों छ्रों ज्रों 

ॐ नमः क्रों नमः भ्रों नमः क्रों नमः झ्रों नमः छ्रों नमः ज्रों नमः स्वां नमः हां नमः ॥
मूलेन व्यापकं विन्यस्य ध्यात्वा चतुरस्रद्वयपार्श्वयोः -
ॐ ह्रीं श्रीं ब्राह्मी पादुकां पूजयामि नमः
ॐ ह्रीं श्रीं माहेश्वरी पादुकां पूजयामि नमः
ॐ ह्रीं श्रीं कौमारी पादुकां पूजयामि नमः
ॐ ह्रीं श्रीं वैष्णवी पादुकां पूजयामि नमः
ॐ ह्रीं श्रीं वाराही पादुकां पूजयामि नमः
ॐ ह्रीं श्रीं इन्द्राणी पादुकां पूजयामि नमः
ॐ ह्रीं श्रीं चामुण्डा पादुकां पूजयामि नमः
ॐ ह्रीं श्रीं महालक्ष्मी पादुकां पूजयामि नमः
ॐ ह्रीं श्रीं कृतयुगशक्ति पादुकां पूजयामि नमः
ॐ ह्रीं श्रीं त्रेतायुगशक्ति पादुकां पूजयामि नमः
ॐ ह्रीं श्रीं द्वापरयुगशक्ति पादुकां पूजयामि नमः
ॐ ह्रीं श्रीं कलियुगशक्ति पादुकां पूजयामि नमः
ॐ ह्रीं श्रीं विजया पादुकां पूजयामि नमः
ॐ ह्रीं श्रीं विमला पादुकां पूजयामि नमः
ॐ ह्रीं श्रीं शुभा पादुकां पूजयामि नमः
ॐ ह्रीं श्रीं विश्वा पादुकां पूजयामि नमः
ॐ ह्रीं श्रीं विभूति पादुकां पूजयामि नमः
ॐ ह्रीं श्रीं विनता पादुकां पूजयामि नमः
ॐ ह्रीं श्रीं विविधा पादुकां पूजयामि नमः
ॐ ह्रीं श्रीं विरता पादुकां पूजयामि नमः
ॐ ह्रीं श्रीं कमला पादुकां पूजयामि नमः
ॐ ह्रीं श्रीं कामिनी पादुकां पूजयामि नमः
ॐ ह्रीं श्रीं किराती पादुकां पूजयामि नमः
ॐ ह्रीं श्रीं कीर्ति पादुकां पूजयामि नमः
ॐ ह्रीं श्रीं कुर्दिनी पादुकां पूजयामि नमः
ॐ ह्रीं श्रीं कुलसुन्दरी पादुकां पूजयामि नमः
ॐ ह्रीं श्रीं कल्याणी पादुकां पूजयामि नमः
ॐ ह्रीं श्रीं कालकोला पादुकां पूजयामि नमः
ॐ ह्रीं श्रीं डाकिनी पादुकां पूजयामि नमः
ॐ ह्रीं श्रीं राकिनी पादुकां पूजयामि नमः
ॐ ह्रीं श्रीं लाकिनी पादुकां पूजयामि नमः
ॐ ह्रीं श्रीं काकिनी पादुकां पूजयामि नमः
ॐ ह्रीं श्रीं साकिनी पादुकां पूजयामि नमः
ॐ ह्रीं श्रीं लाकिनी पादुकां पूजयामि नमः
ॐ ह्रीं श्रीं इच्छाशक्ति पादुकां पूजयामि नमः
ॐ ह्रीं श्रीं ज्ञानशक्ति पादुकां पूजयामि नमः
ॐ ह्रीं श्रीं क्रियाशक्ति पादुकां पूजयामि नमः


ॐ शरेभ्यो नमः ।
ॐ खड्गाय नमः ।
ॐ अङ्कुशाय नमः ।
ॐ पाशाय नमः ।
ॐ गदायै नमः ।
ॐ  चर्मणे नमः ।
ॐ धनुषे नमः ।
ॐ वज्राय नमः ।
 
ॐ ह्रीं श्रीं ऐं लं इन्द्रशक्तिपादुकां पूजयामि नमः
ॐ ह्रीं श्रीं ऐं रं अग्नशक्तिपादुकां पूजयामि नमः
ॐ ह्रीं श्रीं ऐं मं यमशक्तिपादुकां पूजयामि नमः
ॐ ह्रीं श्रीं ऐं क्षं निर्ऋतशक्तिपादुकां पूजयामि नमः
ॐ ह्रीं श्रीं ऐं वं वरुणशक्तिपादुकां पूजयामि नमः
ॐ ह्रीं श्रीं ऐं यं वायुशक्तिपादुकां पूजयामि नमः
ॐ ह्रीं श्रीं ऐं कुं कुबेरशक्तिपादुकां पूजयामि नमः
ॐ ह्रीं श्रीं ऐं हं ईशानशक्तिपादुकां पूजयामि नमः
ॐ ह्रीं श्रीं ऐं आं ब्रह्मशक्तिपादुकां पूजयामि नमः
ॐ ह्रीं श्रीं ऐं ह्रीं अनंतशक्तिपादुकां पूजयामि नमः

एवमेव तत्रैव तदायुधानि पूजयेत् -
ॐ ह्रीं श्रीं ऐं वं वज्रशक्तिपादुकां पूजयामि नमः
ॐ ह्रीं श्रीं ऐं शं शक्तिशक्तिपादुकां पूजयामि नमः
ॐ ह्रीं श्रीं ऐं दं दंडशक्तिपादुकां पूजयामि नमः
ॐ ह्रीं श्रीं ऐं खं खड्गशक्तिपादुकां पूजयामि नमः
ॐ ह्रीं श्रीं ऐं पां पाशशक्तिपादुकां पूजयामि नमः
ॐ ह्रीं श्रीं ऐं अं अंङ्कुशशक्तिपादुकां पूजयामि नमः
ॐ ह्रीं श्रीं ऐं गं गदाशक्तिपादुकां पूजयामि नमः
ॐ ह्रीं श्रीं ऐं त्रिं त्रिशूलशक्तिपादुकां पूजयामि नमः
ॐ ह्रीं श्रीं ऐं पं पद्मशक्तिपादुकां पूजयामि नमः
ॐ ह्रीं श्रीं ऐं चं चक्रशक्तिपादुकां पूजयामि नमः
कामेश्वर्यै विद्महे नित्यक्लिन्नायै धीमहि तन्नो नित्या प्रचोदयात् ॥
इति भेरुण्डा-अावरणपूजा ।

वह्निवासिनीनित्या
तप्तकाञ्चनसङ्काशां नवयौवनसुन्दरीम् । चारुस्मेरमुखाम्भोजां विलसन्नयनत्रयाम् ॥
अष्टाभिर्बाहुभिर्युक्तां माणिक्याभरणोज्वलाम् । पद्मरागकिरीटांशुसम्भेदारुणिताम्बराम् ॥
पीतकौशेयवसनां रत्नमञ्जीरमेखलाम् । रत्नमौक्तिकसम्भिन्नस्तबकाभरणोज्वलाम् ॥
रक्ताब्जकम्बुपुण्ड्रेक्षुचापपूर्णेन्दुमण्डलाम् । दधानां बाहुभिर्वामैः कल्हारं हेमशृङ्गकम् ॥
पुष्पेषुं मातुलुङ्गं च दधानां दक्षिणैः करैः । स्वसमानभिरभितः शक्तिभिः परिवारिताम् ॥

अस्य श्रीवह्निवासिनीनित्यामहामन्त्रस्य वसिष्ठ ऋषिः। गायत्री छन्दः । श्रीवह्निवासिनीनित्या देवता । ह्रीं बीजम् । नमः शक्तिः । ॐ कीलकम् ॥
ह्रां इत्यादिः न्यासः ॥
वह्निवासिन्यै वह्निनिलयायै वह्निरूपिण्यै यज्ञविद्यायै महाविद्यायै ब्रह्मविद्यायै गुहालयायै भूतेश्वर्यै ब्रह्मधात्र्यै विमलायै कनकप्रभायै विरूपाक्षायै विशालाक्ष्यै हिरण्याक्ष्यै शान्ताननायै त्र्यक्षायै कमलायै विद्यायै सिद्धविद्यायै धराधिपायै देवमात्रे दित्यै पुण्यायै दन्वै कद्र्वै सुपर्णिकायै अपांनिधये महावेगायै महोभिर्वरुणालयायै इष्टायै तुष्टिकर्यै छायायै सामगायै रुचिरायै परायै ऋग्यजुःसानविलयायै देवोत्पत्यै स्तुतिप्रियायै प्रद्युम्नदयितायै साध्व्यै सुखसौभाग्यसिद्धिदायै सर्वकामप्रदायै भद्रायै सुभद्रायै सर्वमङ्गलायै धामिन्यै धमन्यै माध्व्यै मधुकैटभमर्दिनी अबाणप्रहरिण्यै

वसिष्ठ-ऋषये नमः (शिरसि)। गायत्री-छन्दसे नमः (मुखे)~। श्रीवह्निवासिनीनित्यादेवतायै नमः (हृदये)। ह्रीं बीजाय नमः (गुह्ये)। नमः शक्तये नमः(पादयोः)।
वह्निवासिन्यै कीलकाय नमः॥
ह्रां इत्यादिन्यासः ।
ॐ ॐ नमः दक्षनेत्रे । ॐ ह्रीं नमः वामनेत्रे । ॐ वं नमः दक्षकर्णे । ॐ ह्निं वामकर्णे । ॐ वां दक्षनासायाम् । ॐ सिं नमः वामनासायाम् । ॐ न्यैं नमः मुखे। ॐ नं नमः लिङ्गे । ॐ मं नमः गुदे । (मूलम्) सर्वाङ्गे ।
ज्वालिनी विस्फुलिङ्गिनी मङ्गला मनोहरा कनका कितवा विश्वा विविधा मेषा वृषा मिथुना कर्कटा सिंहा कन्या तुला कीटा चापा मकरा कुम्भा मीना घस्मरा सर्वभक्षा विश्वा विविधोद्भवा चित्रा निःसपत्ना पावनी रक्ता निरातङ्का अचिन्त्यवैभवा 

महावज्रेश्वरी ६
रक्तां रक्ताम्बरां रक्तगन्धमाल्यविभूषणाम् । चतुर्भुजां त्रिणयनां माणिक्यमुकुटोज्वलाम् ॥
पाशाङ्कुशाविक्षुचापं दाडिमीसायकं तथा । दधानां बाहुभिर्भिन्नैः दयादमसुशीतलैः ॥
पश्यन्ती साधकं त्र्यश्रषट्कोणाब्जमहीपुरे । चक्रमध्ये सुखासीनां स्मेरवक्त्रसरोरुहाम् ॥
शक्तिभिः स्वस्वरूपाभिरावृत्तां पोतमध्यगे । सिंहासनेऽभितः प्रेङ्खत्पोतस्थाभिः स्वशक्तिभिः ॥
वृत्तान्ताभिर्विनोदानि यातायातादिभिः सदा । कुर्वाणामरुणाम्भोधौ चिन्तयेद्वज्रनायिकाम् ॥

अस्य श्रीमहावज्रेश्वरीनित्यामहामन्त्रस्य ब्रह्मर्ऋषिः। गायत्री छन्दः । श्रीमहावज्रेश्वरीनित्या देवता । ॐ बीजम् । फ्रें शक्तिः । ह्रीं कीलकम् ॥
ॐ,ह्रीं,फ्रेंसः,नित्यक्लिन्ने, मदद्रवे, स्वाहा इति   न्यासः ॥
तप्तकाञ्चनसङ्काशां कनकाभरणान्विताम् । हेमताटङ्कसंयुक्तां कस्तूरीतिलकान्विताम् ॥
हेमचिन्ताकसंयुक्तां पूर्णचन्द्रमुखाम्बुजाम् । पीराम्बरसमोपेतां पुण्यमाल्यविभूषिताम् ॥
मुक्ताहारसमोपेतां मुकुटेन विराजिताम् । महावज्रेश्वरीं वन्दे सर्वैश्वर्यफलप्रदाम् ॥

महावज्रेश्वर्यै नित्यायै विधिस्थायै चारुहासिन्यै उषायै अनिरुद्धपत्न्यै रेवत्यै रैवतात्मजायै हलायुधायै प्रियायै मायायै गोकुलायै गोकुलालयायै कृष्णानुजायै कृष्णरजायै नन्ददुहित्रे कंसविद्राविण्यै क्रुद्धायै सिद्धचारणसेवितायै गोक्षीराङ्गायै धृतवत्यै भव्यायै गोपजनप्रियायै शाकम्भर्यै सिद्धविद्यायै वृद्धायै सिद्धकर्यै क्रियायै दावाग्नये विश्वरूपायै विश्वेश्यै दितिसम्भवायै अाधारचक्रनिलयायै द्वारशालावगाहिन्यै सूक्ष्मायै सूक्ष्मतरायै स्थूलायै सप्रपञ्चायै निष्प्रपञ्चायै क्रियातीतायै क्रियारूपायै फलप्रदायै प्राणाख्यायै मन्त्रमात्रे सोमसूर्यामृतप्रदायै छन्दःख्यातायै चिद्रूपायै परमानन्ददायिन्यै
निरानन्दायै

ब्रह्मणे ऋषये नमः (शिरसि)। गायत्री-छन्दसे नमः (मुखे)~। श्रीमहावज्रेश्वरीनित्यादेवतायै नमः (हृदये)। ह्रीं बीजाय नमः (गुह्ये)।  ह्रीं शक्तये नमः(पादयोः)।
ऐं कीलकाय नमः॥
ह्रीं क्लिन्ने ऐं क्रों नित्यमदद्रवे ह्रीं ।
ह्रीं क्लिन्ने ह्रीं ,ह्रींऐं ह्रीं,ह्रींक्रोंह्रीं, ह्रींनित्यह्रीं,ह्रींमदह्रीं,ह्रींद्रवेह्रीं  । इति न्यासः ।
ह्रीं क्लिन्ने ह्रीं, ह्रीं ऐं ह्रीं, ह्रीं क्रों ह्रीं, ह्रींनिह्रीं, ह्रींत्यह्रीं ,ह्रींमह्रीं, ह्रींदह्रीं ,ह्रींद्रह्रीं ,ह्रींवेह्रीं । 
ह्रींशोणसमुद्राय नमः ह्रीं कनकपोताय नमः ह्रींरत्नसिंहासनाय नमः ।

हृल्लेखा क्लेदिनी क्लिन्ना क्षोभिणी मदना मदनातुरा निरञ्जना रागवती मदनावती मेखला द्राविणी वेगवती 

कमला कल्पा कलां कलिता कौतुका किराता काला कदना कौशिकी कम्बुवाहिनी कातरा कपटा कीर्ति कुमारी कुङ्कुमा भञ्जिनी वेगिनी भोगा चपला पेशला सती रति श्रद्धा भोगलोला मदा उन्मत्ता मनस्विनी 
त्वरितानित्या
अस्य श्रीत्वरितानित्यामहामन्त्रस्य ईश्वर ऋषिःब्रह्मर्ऋषिः। त्वरिता छन्दः । श्रीत्वरितानित्या देवता । हुं बीजम् । स्त्रीं शक्तिः । ह्रीं कीलकम् ॥
ह्रां इत्यादिन्यासः



नित्यानित्या १०
उद्यद्भास्करबिम्बास्यां माणिक्यमुकुटोज्वलाम् । पद्मरागकृताकल्पामरुणांशुकधारिणीम् ॥
चारुस्मितलसद्वक्त्रषट्सरोजविराजिताम् । प्रतिवक्त्रं त्रिणयनां भुजैर्द्वादशभिर्युताम् ॥
पाशेक्षुगुणपुण्ड्रेक्षुचापाखेटत्रिशूलकान् । वहन्तीं वरदां वामैरङ्कुशं पुस्तकं तथा ॥
पुष्पेषुमण्डलाग्रञ्च नृकपालाभये तथा । दधानां दक्षिणैर्हस्तैर्ध्यायेद्देवीमनन्यधीः ॥


अस्य श्रीनित्यानित्यामहामन्त्रस्य जमदग्निर्ऋषिः। पङ्क्तिश्छन्दः । श्रीनित्यानित्या देवता । ऐं बीजम् । सौः शक्तिः । क्लीं कीलकम् ॥
ऐं क्लीं सौः सौः क्लीं ऐं  इति न्यासः ॥
उद्यत्प्रद्योतननिभां जपाकुसुमसन्निभाम् । हरिचन्दनलिप्ताङ्गीं रक्तमाल्या विभूषिताम् ॥
रत्नाभरणभूषाङ्गीं रक्तवस्त्रसुशोभिताम् । जगदम्बां नमस्यामि नित्यां श्रीपरमेश्वरीम् ॥

नित्यायै भैरव्यै सूक्ष्मायै प्रचण्डायै सद्गतिप्रदायै प्रियायै शुद्धायै शुष्कायै रक्ताङ्ग्यै रक्तलोचनायै खट्वाङ्गधारिण्यै शङ्खायै कङ्कालायै कालबर्हिण्यै हिमघ्न्यनावृत्तायै स्वरशक्त्यै 
भूतनाथायै भूतभव्यायै दुर्वृत्तजनसम्पदायै पुण्योत्सवायै पुण्यगन्धायै पुण्यपापविवेचिन्यै दिग्वाससे क्षौमवसनायै एकवस्त्रायै जटाधरायै कपालमालिन्यै घण्टाधरायै कपालमालिन्यै घण्टाधरायै धनुरधरायै धनुर्धरायै टङ्कहस्तायै चलायै ब्राह्म्यै हाकिन्यै शाकिन्यै रमायै ब्रह्माण्डपालितमुखायै विष्णुमायायै चतुर्भुजायै अष्टादशभुजायै भीमायै विचित्रायै चित्ररूपायै पद्मासनायै पद्मवहायै स्फुरत्कान्त्यै शुभावहायै मौनिन्यै मौलिन्यै मान्या मानदा मानवर्धिन्यै जगत्प्रियायै विष्णुगर्भायै मङ्गलामङ्गलप्रियायै भूत्यै भूतिकर्यै भोगीन्द्रशयनायै अमितायै तमवामीकर्यै कृत्यायै अार्यायै वंशविवर्धिन्यै अघौघशोषिण्यै श्राव्यायै कृतान्तशक्त्यै ॥

दक्षिणामूर्ति-ऋषये नमः (शिरसि)। पङ्क्तिच्छन्दसे नमः (मुखे)~। श्रीनित्यानित्यादेवतायै नमः (हृदये)। ऐं बीजाय नमः (गुह्ये)। ॐ शक्तये नमः(पादयोः)।
ईं  कीलकाय नमः॥ह्सां ह्सीं इत्यादिन्यासः ॥

हं सं कं लं रं डं 
डाकिनी लाकिनी शाकिनी राकिणी काकिनी हाकिनी अन्तशक्ति नियतिशक्ति ब्रह्मशक्ति कालशक्ति अभयशक्ति खड्गशक्ति पुस्तकशक्ति पाशशक्ति इक्षुचापशक्ति त्रिशूलशक्ति कपालशक्ति इषुशक्ति अङ्कुशशक्ति अक्षगुणशक्ति खेटकशक्ति वरशक्ति 

श्रीनीलपताकानित्या

इन्द्रनीलनिभां भास्वन्मणिमौलिविराजिताम् । पञ्चवक्त्रां त्रिणयनामरुणांशुकधारिणीम् ॥
दशहस्तां लसन्मुक्ताप्रायाभरणमण्डिताम् । रत्नस्तबकसम्भिन्नदेहां चारुस्मिताननाम् ॥
पाशं पताकां चर्माणि शार्ङ्गचापं वरं करैः । दधानां वामपार्श्वस्थैः सर्वाभरणभूषितैः ॥
अङ्कुशं च ततः शक्तिं खड्गं बाणं तथाभायम् । दधानां दक्षिणैर्हस्तैरासीनां पद्मविष्टरे ॥
स्वाकारवर्णवेषास्यपाण्यायुधविभूषणैः । शक्तिवृन्दैर्वृतां ध्यायेद्देवीं नित्यार्चनक्रमे ॥

अस्य श्रीनीलपताकानित्यामहामन्त्रस्य परमशिव ऋषिः । जगती च्छन्दः । श्रीनीलपताकानित्या देवता । ह्रीं बीजम् । फ्रें शक्तिः । स्रूं कीलकम् ।
ॐ ह्रीं , फ्रें स्रूं , अां ह्रीं क्रों, नित्यमदद्रवे, हुं ,क्रों  इति न्यासः । दिग्बन्धः ।
ध्यानं 
पञ्चवक्त्रां त्रिणयनामरुणांशुकधारिणीम् ॥ दशहस्तां लसन्मुक्ताप्रायाभरणमण्डिताम् ।
नीलमेघसमप्रख्यां धूम्रार्चिसदृशप्रभाम् । नीलपुष्पस्रजोपेतां ध्यायेन्नीलपताकिनीम् ॥
ॐ ह्रीं फ्रें स्रूं अां ह्रीं क्रों, नित्यमदद्रवे, हुं ,क्रों  ।

नीलपताकायै नीलायै मायायै जगत्प्रियायै सहस्रवज्रायै पद्माक्ष्यै पद्मिन्यै श्रियै अनुत्तमायै दिव्यक्रमायै दिव्यभोगायै दिव्यमाल्यानुलेपिन्यै शुक्लाच्छवसनायै सौम्यायै सर्वर्तुकुसुमोचितायै स्वराणांशक्त्यै साधकाभीष्टदायिकायै सर्वैश्वर्यगुणोपेतायै प्रणवाग्रसम्भवायै व्यञ्जनायै व्यञ्जकायै व्यक्तायै सर्वावरणानुवर्तिन्यै जगन्मात्रे भयङ्कर्यै भूतदात्र्यै सुदुर्लभायै कामिन्यै दण्डिन्यै दण्डघायै खड्गमुद्गरपाण्यै शस्त्रास्त्रदर्शिन्यै बीजायै विबीजायै बिजिन्यै परायै वाचस्पतिप्रियायै दीक्षायै परीक्षायै विश्वसम्भवायै राजस्यै तामस्यै सत्वायै सत्वोद्रिक्तायै विमोहिन्यै अतीतानागतज्ञानायै वर्तमानोपदेशिन्यै अाप्तोपदेशिन्यै संवित्सत्वबोधायै धराधरायै प्रकृत्यै विकृत्यै गङ्गायै धूर्जट्यै विकृताननायै योगप्रियायै योगगम्यायै योगिध्येयायै परापरायै वैष्णव्यै त्रिपद्यै इष्टरक्षायै कादिशक्त्यै ।

अस्य श्रीनीलपताकानित्यामहामन्त्रस्य सम्मोहन ऋषिः । गायत्री च्छन्दः । श्रीनीलपताकानित्या देवता । ह्रीं बीजम् । ह्रीं शक्तिः । क्लीं कीलकम् ।

ॐ ह्रीं फ्रें , स्रूं ओं अां क्लीं, ऐं ब्लूं  नित्यमद, द्र, वे, हुं  इति न्यासः ।

ॐ ह्रीं , ॐफ्रें , ॐस्रूं , ॐओं, ॐअां, ॐऐं , ॐब्लूं , ॐनिं, ॐत्यं, ॐमं, ॐदं, ॐद्रं, ॐवें, ॐहुं , ॐक्रों, ॐफ्रें, ह्रीं ।

अभयाय नमः बाणाय खड्गाय शक्तये  अङ्कुशाय पाशाय पताकाय चर्मणे शार्ङ्गचापाय वराय

ह्रीं इच्छाशक्तिपादुकां पूजयामि नमः । ज्ञानशक्तिपादुकां पूजयामि नमः । क्रियाशक्तिपादुकां पूजयामि नमः ।

ह्रीं हाकिनीपादुकां पूजयामि नमः । शाकिनी , काकिनी , लाकिनी , राकिणी , डाकिनी ।

ह्रीं ब्राह्मीपादुकां पूजयामि नमः । माहेश्वरीपादुकां । कौमारी । वैष्णवी, वाराही माहेन्द्री चामुण्डा ।

ॐ ह्रीं सुमुखीपादुकां । सुन्दरी, सारा सुमना सरस्वती समया सर्वगा सिद्धा 

विह्वला लोला मदना विनोदा पुण्या अाकर्षिणी नित्या मालिनी कौतुका  पुराणा ।

इन्द्रशक्तिपादुकां अग्निशक्तिपादुकां यम निर्ऋति वरुण वायु कुबेर ईशान 								

अनन्त ब्रह्म नियति काल 				
																		
वज्रशक्ति, शक्तिशक्ति, दण्डशक्ति खड्गशक्ति पाशशक्ति अङ्कुशशक्ति गदाशक्ति त्रिशूलशक्ति पद्मशक्ति चक्रशक्ति ।



विजया नित्या 
