‌\chapter*{\center ಗುರುಗೀತಾ}
ಓಂ ಅಸ್ಯ ಶ್ರೀಗುರುಗೀತಾಸ್ತೋತ್ರಮಂತ್ರಸ್ಯ ಭಗವಾನ್ ಸದಾಶಿವಋಷಿಃ। ವಿರಾಟ್ ಛಂದಃ। ಶ್ರೀ ಗುರುಪರಮಾತ್ಮಾ ದೇವತಾ। ಹಂ ಬೀಜಂ। ಸಃ ಶಕ್ತಿಃ। ಸೋಽಹಂ ಕೀಲಕಂ। ಶ್ರೀಗುರುಪ್ರಸಾದಸಿದ್ಧಯರ್ಥೇ ಜಪೇ ವಿನಿಯೋಗಃ ॥\\

ಅಥ ಕರನ್ಯಾಸಃ -\\
ಓಂ ಹಂ ಸಾಂ ಸೂರ್ಯಾತ್ಮನೇ ಅಂಗುಷ್ಠಾಭ್ಯಾಂ ನಮಃ।\\
ಓಂ ಹಂ ಸೀಂ ಸೋಮಾತ್ಮನೇ ತರ್ಜನೀಭ್ಯಾಂ ನಮಃ।\\
ಓಂ ಹಂ ಸೂಂ ನಿರಂಜನಾತ್ಮನೇ ಮಧ್ಯಮಾಭ್ಯಾಂ ನಮಃ।\\
ಓಂ ಹಂ ಸೈಂ ನಿರಾಭಾಸಾತ್ಮನೇ ಅನಾಮಿಕಾಭ್ಯಾಂ ನಮಃ।\\
ಓಂ ಹಂ ಸೌಂ ಅತನುಸೂಕ್ಷ್ಮಾತ್ಮನೇ ಕನಿಷ್ಠಿಕಾಭ್ಯಾಂ ನಮಃ।\\
ಓಂ ಹಂ ಸಃ ಅವ್ಯಕ್ತಾತ್ಮನೇ ಕರತಲಕರಪೃಷ್ಠಾಭ್ಯಾಂ ನಮಃ।\\
ಏವಂ ಹೃದಯಾದಿಷಡಂಗನ್ಯಾಸಂ ವಿಧಾಯ ಧ್ಯಾಯೇತ್\-\\
\section{ಅಥ ಪ್ರಥಮೋಽಧ್ಯಾಯಃ}
ಅಚಿಂತ್ಯಾವ್ಯಕ್ತರೂಪಾಯ ನಿರ್ಗುಣಾಯ ಗುಣಾತ್ಮನೇ~।\\
ಸಮಸ್ತಜಗದಾಧಾರಮೂರ್ತಯೇ ಬ್ರಹ್ಮಣೇ ನಮಃ~॥೧॥

ಋಷಯ ಊಚುಃ~।\\
ಸೂತ ಸೂತ ಮಹಾಪ್ರಾಜ್ಞ ನಿಗಮಾಗಮಪಾರಗ~।\\
ಗುರುಸ್ವರೂಪಮಸ್ಮಾಕಂ ಬ್ರೂಹಿ ಸರ್ವಮಲಾಪಹಂ~॥೨॥

ಯಸ್ಯ ಶ್ರವಣಮಾತ್ರೇಣ ದೇಹೀ ದುಃಖಾದ್ವಿಮುಚ್ಯತೇ~।\\
ಯೇನ ಮಾರ್ಗೇಣ ಮುನಯಃ ಸರ್ವಜ್ಞತ್ವಂ ಪ್ರಪೇದಿರೇ~॥೩॥

ಯತ್ಪ್ರಾಪ್ಯ ನ ಪುನರ್ಯಾತಿ ನರಃ ಸಂಸಾರಬಂಧನಂ~।\\
ತಥಾವಿಧಂ ಪರಂ ತತ್ತ್ವಂ ವಕ್ತವ್ಯಮಧುನಾ ತ್ವಯಾ~॥೪॥

ಗುಹ್ಯಾದ್ಗುಹ್ಯತಮಂ ಸಾರಂ ಗುರುಗೀತಾ ವಿಶೇಷತಃ~।\\
ತ್ವತ್ಪ್ರಸಾದಾಚ್ಚ ಶ್ರೋತವ್ಯಾ ತತ್ಸರ್ವಂ ಬ್ರೂಹಿ ಸೂತ ನಃ~॥೫॥

ಇತಿ ಸಂಪ್ರಾರ್ಥಿತಃ ಸೂತೋ ಮುನಿಸಂಘೈರ್ಮುಹುರ್ಮುಹುಃ~।\\
ಕುತೂಹಲೇನ ಮಹತಾ ಪ್ರೋವಾಚ ಮಧುರಂ ವಚಃ~॥೬॥

ಸೂತ ಉವಾಚ~।\\
ಶ್ರುಣುಧ್ವಂ ಮುನಯಃ ಸರ್ವೇ ಶ್ರದ್ಧಯಾ ಪರಯಾ ಮುದಾ~।\\
ವದಾಮಿ ಭವರೋಗಘ್ನೀಂ ಗೀತಾಂ ಮಾತೃಸ್ವರೂಪಿಣೀಂ~॥೭॥

ಪುರಾ ಕೈಲಾಸಶಿಖರೇ ಸಿದ್ಧಗಂಧರ್ವಸೇವಿತೇ~।\\
ತತ್ರ ಕಲ್ಪಲತಾಪುಷ್ಪಮಂದಿರೇಽತ್ಯಂತಸುಂದರೇ~॥೮॥

ವ್ಯಾಘ್ರಾಜಿನೇ ಸಮಾಸೀನಂ ಶುಕಾದಿಮುನಿವಂದಿತಂ~।\\
ಬೋಧಯಂತಂ ಪರಂ ತತ್ತ್ವಂ ಮಧ್ಯೇ ಮುನಿಗಣೇ ಕ್ವಚಿತ್~॥೯॥

ಪ್ರಣಮ್ರವದನಾ ಶಶ್ವನ್ನಮಸ್ಕುರ್ವಂತಮಾದರಾತ್~।\\
ದೃಷ್ಟ್ವಾ ವಿಸ್ಮಯಮಾಪನ್ನಾ ಪಾರ್ವತೀ ಪರಿಪೃಚ್ಛತಿ~॥೧೦॥

ಪಾರ್ವತ್ಯುವಾಚ~।\\
ಓಂ ನಮೋ ದೇವ ದೇವೇಶ ಪರಾತ್ಪರ ಜಗದ್ಗುರೋ~।\\
ತ್ವಾಂ ನಮಸ್ಕುರ್ವತೇ ಭಕ್ತ್ಯಾ ಸುರಾಸುರನರಾಃ ಸದಾ~॥೧೧॥

ವಿಧಿವಿಷ್ಣುಮಹೇಂದ್ರಾದ್ಯೈರ್ವಂದ್ಯಃ ಖಲು ಸದಾ ಭವಾನ್~।\\
ನಮಸ್ಕರೋಷಿ ಕಸ್ಮೈ ತ್ವಂ ನಮಸ್ಕಾರಾಶ್ರಯಃ ಕಿಲ~॥೧೨॥

ದೃಷ್ಟ್ವೈತತ್ಕರ್ಮ ವಿಪುಲಮಾಶ್ಚರ್ಯಂ ಪ್ರತಿಭಾತಿ ಮೇ~।\\
ಕಿಮೇತನ್ನ ವಿಜಾನೇಽಹಂ ಕೃಪಯಾ ವದ ಮೇ ಪ್ರಭೋ~॥೧೩॥

ಭಗವನ್ ಸರ್ವಧರ್ಮಜ್ಞ ವ್ರತಾನಾಂ ವ್ರತನಾಯಕಂ~।\\
ಬ್ರೂಹಿ ಮೇ ಕೃಪಯಾ ಶಂಭೋ ಗುರುಮಾಹಾತ್ಮ್ಯಮುತ್ತಮಂ~॥೧೪॥

ಕೇನ ಮಾರ್ಗೇಣ ಭೋ ಸ್ವಾಮಿನ್ ದೇಹೀ ಬ್ರಹ್ಮಮಯೋ ಭವೇತ್~।\\
ತತ್ಕೃಪಾಂ ಕುರು ಮೇ ಸ್ವಾಮಿನ್ನಮಾಮಿ ಚರಣೌ ತವ~॥೧೫॥

ಇತಿ ಸಂಪ್ರಾರ್ಥಿತಃ ಶಶ್ವನ್ಮಹಾದೇವೋ ಮಹೇಶ್ವರಃ~।\\
ಆನಂದಭರಿತಃ ಸ್ವಾಂತೇ ಪಾರ್ವತೀಮಿದಮಬ್ರವೀತ್~॥೧೬॥

ಶ್ರೀ ಮಹಾದೇವ ಉವಾಚ~।\\
ನ ವಕ್ತವ್ಯಮಿದಂ ದೇವಿ ರಹಸ್ಯಾತಿರಹಸ್ಯಕಂ~।\\
ನ ಕಸ್ಯಾಪಿ ಪುರಾ ಪ್ರೋಕ್ತಂ ತ್ವದ್ಭಕ್ತ್ಯರ್ಥಂ ವದಾಮಿ ತತ್~॥೧೭॥

ಮಮ ರೂಪಾಸಿ ದೇವಿ ತ್ವಮತಸ್ತತ್ಕಥಯಾಮಿ ತೇ~।\\
ಲೋಕೋಪಕಾರಕಃ ಪ್ರಶ್ನೋ ನ ಕೇನಾಪಿ ಕೃತಃ ಪುರಾ~॥೧೮॥

ಯಸ್ಯ ದೇವೇ ಪರಾ ಭಕ್ತಿರ್ಯಥಾ ದೇವೇ ತಥಾ ಗುರೌ~।\\
ತಸ್ಯೈತೇ ಕಥಿತಾ ಹ್ಯರ್ಥಾಃ ಪ್ರಕಾಶಂತೇ ಮಹಾತ್ಮನಃ~॥೧೯॥

ಯೋ ಗುರುಃ ಸ ಶಿವಃ ಪ್ರೋಕ್ತೋ ಯಃ ಶಿವಃ ಸ ಗುರುಃ ಸ್ಮೃತಃ~।\\
ವಿಕಲ್ಪಂ ಯಸ್ತು ಕುರ್ವೀತ ಸ ನರೋ ಗುರುತಲ್ಪಗಃ~॥೨೦॥

ದುರ್ಲಭಂ ತ್ರಿಷು ಲೋಕೇಷು ತಚ್ಛೃಣುಶ್ವ ವದಾಮ್ಯಹಂ~।\\
ಗುರುರ್ಬ್ರಹ್ಮ ವಿನಾ ನಾನ್ಯಃ ಸತ್ಯಂ ಸತ್ಯಂ ವರಾನನೇ~॥೨೧॥

ವೇದಶಾಸ್ತ್ರಪುರಾಣಾನಿ ಚೇತಿಹಾಸಾದಿಕಾನಿ ಚ~।\\
ಮಂತ್ರಯಂತ್ರಾದಿವಿದ್ಯಾನಾಂ ಮೋಹನೋಚ್ಚಾಟನಾದಿಕಂ~॥೨೨॥

ಶೈವಶಾಕ್ತಾಗಮಾದೀನಿ ಹ್ಯನ್ಯೇ ಚ ಬಹವೋ ಮತಾಃ~।\\
ಅಪಭ್ರಂಶಾಃ ಸಮಸ್ತಾನಾಂ ಜೀವಾನಾಂ ಭ್ರಾಂತಚೇತಸಾಂ~॥೨೩॥

ಜಪಸ್ತಪೋ ವ್ರತಂ ತೀರ್ಥಂ ಯಜ್ಞೋ ದಾನಂ ತಥೈವ ಚ~।\\
ಗುರುತತ್ತ್ವಮವಿಜ್ಞಾಯ ಸರ್ವಂ ವ್ಯರ್ಥಂ ಭವೇತ್ಪ್ರಿಯೇ~॥೨೪॥

ಗುರುಬುದ್ಧ್ಯಾತ್ಮನೋ ನಾನ್ಯತ್ ಸತ್ಯಂ ಸತ್ಯಂ ವರಾನನೇ~।\\
ತಲ್ಲಾಭಾರ್ಥಂ ಪ್ರಯತ್ನಸ್ತು ಕರ್ತವ್ಯಶ್ಚ ಮನೀಷಿಭಿಃ~॥೨೫॥

ಗೂಢಾವಿದ್ಯಾ ಜಗನ್ಮಾಯಾ ದೇಹಶ್ಚಾಜ್ಞಾನಸಂಭವಃ~।\\
ವಿಜ್ಞಾನಂ ಯತ್ಪ್ರಸಾದೇನ ಗುರುಶಬ್ದೇನ ಕಥ್ಯತೇ~॥೨೬॥

ಯದಂಘ್ರಿಕಮಲದ್ವಂದ್ವಂ ದ್ವಂದ್ವತಾಪನಿವಾರಕಂ~।\\
ತಾರಕಂ ಭವಸಿಂಧೋಶ್ಚ ತಂ ಗುರುಂ ಪ್ರಣಮಾಮ್ಯಹಂ~॥೨೭॥

ದೇಹೀ ಬ್ರಹ್ಮ ಭವೇದ್ಯಸ್ಮಾತ್ ತ್ವತ್‌ಕೃಪಾರ್ಥಂ ವದಾಮಿ ತತ್~।\\
ಸರ್ವಪಾಪವಿಶುದ್ಧಾತ್ಮಾ ಶ್ರೀಗುರೋಃ ಪಾದಸೇವನಾತ್~॥೨೮॥

ಸರ್ವತೀರ್ಥಾವಗಾಹಸ್ಯ ಸಂಪ್ರಾಪ್ನೋತಿ ಫಲಂ ನರಃ~।\\
ಗುರೋಃ ಪಾದೋದಕಂ ಪೀತ್ವಾ ಶೇಷಂ ಶಿರಸಿ ಧಾರಯನ್~॥೨೯॥

ಶೋಷಣಂ ಪಾಪಪಂಕಸ್ಯ ದೀಪನಂ ಜ್ಞಾನತೇಜಸಃ~।\\
ಗುರೋಃ ಪಾದೋದಕಂ ಸಮ್ಯಕ್ ಸಂಸಾರಾರ್ಣವತಾರಕಂ~॥೩೦॥

ಅಜ್ಞಾನಮೂಲಹರಣಂ ಜನ್ಮಕರ್ಮನಿವಾರಕಂ~।\\
ಜ್ಞಾನವಿಜ್ಞಾನಸಿದ್ಧ್ಯರ್ಥಂ ಗುರುಪಾದೋದಕಂ ಪಿಬೇತ್~॥೩೧॥

ಗುರುಪಾದೋದಕಂ ಪಾನಂ ಗುರೋರುಚ್ಛಿಷ್ಟಭೋಜನಂ~।\\
ಗುರುಮೂರ್ತೇಃ ಸದಾ ಧ್ಯಾನಂ ಗುರೋರ್ನಾಮ್ನಃ ಸದಾ ಜಪಃ~॥೩೨॥

ಸ್ವದೇಶಿಕಸ್ಯೈವ ಚ ನಾಮಕೀರ್ತನಂ ಭವೇದನಂತಸ್ಯ ಶಿವಸ್ಯ ಕೀರ್ತನಂ~।\\
ಸ್ವದೇಶಿಕಸ್ಯೈವ ಚ ನಾಮಚಿಂತನಂ ಭವೇದನಂತಸ್ಯ ಶಿವಸ್ಯ ಚಿಂತನಂ~॥೩೩॥

ಯತ್ಪಾದರೇಣುರ್ವೈ ನಿತ್ಯಂ ಕೋಽಪಿ ಸಂಸಾರವಾರಿಧೌ~।\\
ಸೇತುಬಂಧಾಯತೇ ನಾಥಂ ದೇಶಿಕಂ ತಮುಪಾಸ್ಮಹೇ~॥೩೪॥

ಯದನುಗ್ರಹಮಾತ್ರೇಣ ಶೋಕಮೋಹೌ ವಿನಶ್ಯತಃ~।\\
ತಸ್ಮೈ ಶ್ರೀದೇಶಿಕೇಂದ್ರಾಯ ನಮೋಽಸ್ತು ಪರಮಾತ್ಮನೇ~॥೩೫॥

ಯಸ್ಮಾದನುಗ್ರಹಂ ಲಬ್ಧ್ವಾ ಮಹದಜ್ಞಾನಮುತ್ಸೃಜೇತ್~।\\
ತಸ್ಮೈ ಶ್ರೀದೇಶಿಕೇಂದ್ರಾಯ ನಮಶ್ಚಾಭೀಷ್ಟಸಿದ್ಧಯೇ~॥೩೬॥

ಕಾಶೀಕ್ಷೇತ್ರ ನಿವಾಸಶ್ಚ ಜಾಹ್ನವೀ ಚರಣೋದಕಂ~।\\
ಗುರುರ್ವಿಶ್ವೇಶ್ವರಃ ಸಾಕ್ಷಾತ್ ತಾರಕಂ ಬ್ರಹ್ಮ ನಿಶ್ಚಯಃ~॥೩೭॥

ಗುರುಸೇವಾ ಗಯಾ ಪ್ರೋಕ್ತಾ ದೇಹಃ ಸ್ಯಾದಕ್ಷಯೋ ವಟಃ~।\\
ತತ್ಪಾದಂ ವಿಷ್ಣುಪಾದಂ ಸ್ಯಾತ್ ತತ್ರ ದತ್ತಮನಂತಕಂ~॥೩೮॥

ಗುರುಮೂರ್ತಿಂ ಸ್ಮರೇನ್ನಿತ್ಯಂ ಗುರೋರ್ನಾಮ ಸದಾ ಜಪೇತ್~।\\
ಗುರೋರಾಜ್ಞಾಂ ಪ್ರಕುರ್ವೀತ ಗುರೋರನ್ಯಂ ನ ಭಾವಯೇತ್~॥೩೯॥

ಗುರುವಕ್ತ್ರೇ ಸ್ಥಿತಂ ಬ್ರಹ್ಮ ಪ್ರಾಪ್ಯತೇ ತತ್ಪ್ರಸಾದತಃ~।\\
ಗುರೋರ್ಧ್ಯಾನಂ ಸದಾ ಕುರ್ಯಾತ್ ಕುಲಸ್ತ್ರೀ ಸ್ವಪತಿಂ ಯಥಾ~॥೪೦॥

ಸ್ವಾಶ್ರಮಂ ಚ ಸ್ವಜಾತಿಂ ಚ ಸ್ವಕೀರ್ತಿಂ ಪುಷ್ಟಿವರ್ಧನಂ~।\\
ಏತತ್ಸರ್ವಂ ಪರಿತ್ಯಜ್ಯ ಗುರುಮೇವ ಸಮಾಶ್ರಯೇತ್~॥೪೧॥

ಅನನ್ಯಾಶ್ಚಿಂತಯಂತೋ ಯೇ ಸುಲಭಂ ಪರಮಂ ಸುಖಂ~।\\
ತಸ್ಮಾತ್ಸರ್ವಪ್ರಯತ್ನೇನ ಗುರೋರಾರಾಧನಂ ಕುರು~॥೪೨॥

ಗುರುವಕ್ತ್ರೇ ಸ್ಥಿತಾ ವಿದ್ಯಾ ಗುರುಭಕ್ತ್ಯಾ ಚ ಲಭ್ಯತೇ~।\\
ತ್ರೈಲೋಕ್ಯೇ ಸ್ಫುಟವಕ್ತಾರೋ ದೇವರ್ಷಿಪಿತೃಮಾನವಾಃ~॥೪೩॥

ಗುಕಾರಶ್ಚಾಂಧಕಾರೋ ಹಿ ರುಕಾರಸ್ತೇಜ ಉಚ್ಯತೇ~।\\
ಅಜ್ಞಾನಗ್ರಾಸಕಂ ಬ್ರಹ್ಮ ಗುರುರೇವ ನ ಸಂಶಯಃ~॥೪೪॥

ಗುಕಾರೋ ಭವರೋಗಃ ಸ್ಯಾತ್ ರುಕಾರಸ್ತನ್ನಿರೋಧಕೃತ್~।\\
ಭವರೋಗಹರತ್ವಾಚ್ಚ ಗುರುರಿತ್ಯಭಿಧೀಯತೇ~॥೪೫॥

ಗುಕಾರಶ್ಚ ಗುಣಾತೀತೋ ರೂಪಾತೀತೋ ರುಕಾರಕಃ~।\\
ಗುಣರೂಪವಿಹೀನತ್ವಾತ್ ಗುರುರಿತ್ಯಭಿಧೀಯತೇ~॥೪೬॥

ಗುಕಾರಃ ಪ್ರಥಮೋ ವರ್ಣೋ ಮಾಯಾದಿಗುಣಭಾಸಕಃ~।\\
ರುಕಾರೋಽಸ್ತಿ ಪರಂ ಬ್ರಹ್ಮ ಮಾಯಾಭ್ರಾಂತಿವಿಮೋಚನಂ~॥೪೭॥

ಏವಂ ಗುರುಪದಂ ಶ್ರೇಷ್ಠಂ ದೇವಾನಾಮಪಿ ದುರ್ಲಭಂ~।\\
ಗರುಡೋರಗಗಂಧರ್ವಸಿದ್ಧಾದಿಸುರಪೂಜಿತಂ~॥೪೮॥

ಧ್ರುವಂ ದೇವಿ ಮುಮುಕ್ಷೂಣಾಂ ನಾಸ್ತಿ ತತ್ತ್ವಂ ಗುರೋಃ ಪರಂ~।\\
ಗುರೋರಾರಾಧನಂ ಕುರ್ಯಾತ್ ಸ್ವಜೀವಿತ್ವಂ ನಿವೇದಯೇತ್~॥೪೯॥

ಆಸನಂ ಶಯನಂ ವಸ್ತ್ರಂ ವಾಹನಂ ಭೂಷಣಾದಿಕಂ~।\\
ಸಾಧಕೇನ ಪ್ರದಾತವ್ಯಂ ಗುರುಸಂತೋಷಕಾರಣಂ~॥೫೦॥

ಕರ್ಮಣಾ ಮನಸಾ ವಾಚಾ ಸರ್ವದಾಽಽರಾಧಯೇದ್ಗುರುಂ~।\\
ದೀರ್ಘದಂಡಂ ನಮಸ್ಕೃತ್ಯ ನಿರ್ಲಜ್ಜೋ ಗುರುಸನ್ನಿಧೌ~॥೫೧॥

ಶರೀರಮಿಂದ್ರಿಯಂ ಪ್ರಾಣಮರ್ಥಸ್ವಜನಬಾಂಧವಾನ್~।\\
ಆತ್ಮದಾರಾದಿಕಂ ಸರ್ವಂ ಸದ್ಗುರುಭ್ಯೋ ನಿವೇದಯೇತ್~॥೫೨॥

ಗುರುರೇಕೋ ಜಗತ್ಸರ್ವಂ ಬ್ರಹ್ಮವಿಷ್ಣುಶಿವಾತ್ಮಕಂ~।\\
ಗುರೋಃ ಪರತರಂ ನಾಸ್ತಿ ತಸ್ಮಾತ್ಸಂಪೂಜಯೇದ್ಗುರುಂ~॥೫೩॥

ಸರ್ವಶ್ರುತಿಶಿರೋರತ್ನವಿರಾಜಿತಪದಾಂಬುಜಂ~।\\
ವೇದಾಂತಾರ್ಥಪ್ರವಕ್ತಾರಂ ತಸ್ಮಾತ್ ಸಂಪೂಜಯೇದ್ಗುರುಂ~॥೫೪॥

ಯಸ್ಯ ಸ್ಮರಣಮಾತ್ರೇಣ ಜ್ಞಾನಮುತ್ಪದ್ಯತೇ ಸ್ವಯಂ~।\\
ಸ ಏವ ಸರ್ವಸಂಪತ್ತಿಃ ತಸ್ಮಾತ್ಸಂಪೂಜಯೇದ್ಗುರುಂ~॥೫೫॥

ಕೃಮಿಕೋಟಿಭಿರಾವಿಷ್ಟಂ ದುರ್ಗಂಧಕುಲದೂಷಿತಂ~।\\
ಅನಿತ್ಯಂ ದುಃಖನಿಲಯಂ ದೇಹಂ ವಿದ್ಧಿ ವರಾನನೇ~॥೫೬॥

ಸಂಸಾರವೃಕ್ಷಮಾರೂಢಾಃ ಪತಂತಿ ನರಕಾರ್ಣವೇ~।\\
ಯಸ್ತಾನುದ್ಧರತೇ ಸರ್ವಾನ್ ತಸ್ಮೈ ಶ್ರೀಗುರವೇ ನಮಃ~॥೫೭॥

ಗುರುರ್ಬ್ರಹ್ಮಾ ಗುರುರ್ವಿಷ್ಣುರ್ಗುರುರ್ದೇವೋ ಮಹೇಶ್ವರಃ~।\\
ಗುರುರೇವ ಪರಂ ಬ್ರಹ್ಮ ತಸ್ಮೈ ಶ್ರೀಗುರವೇ ನಮಃ~॥೫೮॥

ಅಜ್ಞಾನತಿಮಿರಾಂಧಸ್ಯ ಜ್ಞಾನಾಂಜನಶಲಾಕಯಾ~।\\
ಚಕ್ಷುರುನ್ಮೀಲಿತಂ ಯೇನ ತಸ್ಮೈ ಶ್ರೀಗುರವೇ ನಮಃ~॥೫೯॥

ಅಖಂಡಮಂಡಲಾಕಾರಂ ವ್ಯಾಪ್ತಂ ಯೇನ ಚರಾಚರಂ~।\\
ತತ್ಪದಂ ದರ್ಶಿತಂ ಯೇನ ತಸ್ಮೈ ಶ್ರೀಗುರವೇ ನಮಃ~॥೬೦॥

ಸ್ಥಾವರಂ ಜಂಗಮಂ ವ್ಯಾಪ್ತಂ ಯತ್ಕಿಂಚಿತ್ಸಚರಾಚರಂ~।\\
ತ್ವಂಪದಂ ದರ್ಶಿತಂ ಯೇನ ತಸ್ಮೈ ಶ್ರೀಗುರವೇ ನಮಃ~॥೬೧॥

ಚಿನ್ಮಯಂ ವ್ಯಾಪಿತಂ ಸರ್ವಂ ತ್ರೈಲೋಕ್ಯಂ ಸಚರಾಚರಂ~।\\
ಅಸಿತ್ವಂ ದರ್ಶಿತಂ ಯೇನ ತಸ್ಮೈ ಶ್ರೀಗುರವೇ ನಮಃ~॥೬೨॥

ನಿಮಿಷಾನ್ನಿಮಿಷಾರ್ಧಾದ್ವಾ ಯದ್ವಾಕ್ಯಾದ್ವೈ ವಿಮುಚ್ಯತೇ~।\\
ಸ್ವಾತ್ಮಾನಂ ಶಿವಮಾಲೋಕ್ಯ ತಸ್ಮೈ ಶ್ರೀಗುರವೇ ನಮಃ~॥೬೩॥

ಚೈತನ್ಯಂ ಶಾಶ್ವತಂ ಶಾಂತಂ ವ್ಯೋಮಾತೀತಂ ನಿರಂಜನಂ~।\\
ನಾದಬಿಂದುಕಲಾತೀತಂ ತಸ್ಮೈ ಶ್ರೀಗುರವೇ ನಮಃ~॥೬೪॥

ನಿರ್ಗುಣಂ ನಿರ್ಮಲಂ ಶಾಂತಂ ಜಂಗಮಂ ಸ್ಥಿರಮೇವ ಚ~।\\
ವ್ಯಾಪ್ತಂ ಯೇನ ಜಗತ್ಸರ್ವಂ ತಸ್ಮೈ ಶ್ರೀಗುರವೇ ನಮಃ~॥೬೫॥

ಸ ಪಿತಾ ಸ ಚ ಮೇ ಮಾತಾ ಸ ಬಂಧುಃ ಸ ಚ ದೇವತಾ~।\\
ಸಂಸಾರಮೋಹನಾಶಾಯ ತಸ್ಮೈ ಶ್ರೀಗುರವೇ ನಮಃ~॥೬೬॥

ಯತ್ಸತ್ತ್ವೇನ ಜಗತ್ಸತ್ವಂ ಯತ್ಪ್ರಕಾಶೇನ ಭಾತಿ ತತ್~।\\
ಯದಾನಂದೇನ ನಂದಂತಿ ತಸ್ಮೈ ಶ್ರೀಗುರವೇ ನಮಃ~॥೬೭॥

ಯಸ್ಮಿನ್‌ಸ್ಥಿತಮಿದಂ ಸರ್ವಂ ಭಾತಿ ಯದ್ಭಾನರೂಪತಃ~।\\
ಪ್ರಿಯಂ ಪುತ್ರಾದಿ ಯತ್ಪ್ರೀತ್ಯಾ ತಸ್ಮೈ ಶ್ರೀಗುರವೇ ನಮಃ~॥೬೮॥

ಯೇನೇದಂ ದರ್ಶಿತಂ ತತ್ತ್ವಂ ಚಿತ್ತಚೈತ್ಯಾದಿಕಂ ತಥಾ~।\\
ಜಾಗ್ರತ್ಸ್ವಪ್ನಸುಷುಪ್ತ್ಯಾದಿ ತಸ್ಮೈ ಶ್ರೀಗುರವೇ ನಮಃ~॥೬೯॥

ಯಸ್ಯ ಜ್ಞಾನಮಿದಂ ವಿಶ್ವಂ ನ ದೃಶ್ಯಂ ಭಿನ್ನಭೇದತಃ~।\\
ಸದೈಕರೂಪರೂಪಾಯ ತಸ್ಮೈ ಶ್ರೀಗುರವೇ ನಮಃ~॥೭೦॥

ಯಸ್ಯ ಜ್ಞಾತಂ ಮತಂ ತಸ್ಯ ಮತಂ ಯಸ್ಯ ನ ವೇದ ಸಃ~।\\
ಅನನ್ಯಭಾವಭಾವಾಯ ತಸ್ಮೈ ಶ್ರೀಗುರವೇ ನಮಃ~॥೭೧॥

ಯಸ್ಮೈ ಕಾರಣರೂಪಾಯ ಕಾರ್ಯರೂಪೇಣ ಭಾತಿ ಯತ್~।\\
ಕಾರ್ಯಕಾರಣರೂಪಾಯ ತಸ್ಮೈ ಶ್ರೀಗುರವೇ ನಮಃ~॥೭೨॥

ನಾನಾರೂಪಮಿದಂ ವಿಶ್ವಂ ನ ಕೇನಾಪ್ಯಸ್ತಿ ಭಿನ್ನತಾ~।\\
ಕಾರ್ಯಕಾರಣರೂಪಾಯ ತಸ್ಮೈ ಶ್ರೀಗುರವೇ ನಮಃ~॥೭೩॥

ಜ್ಞಾನಶಕ್ತಿಸಮಾರೂಢತತ್ತ್ವಮಾಲಾವಿಭೂಷಿಣೇ~।\\
ಭುಕ್ತಿಮುಕ್ತಿಪ್ರದಾತ್ರೇ ಚ ತಸ್ಮೈ ಶ್ರೀಗುರವೇ ನಮಃ~॥೭೪॥

ಅನೇಕಜನ್ಮಸಂಪ್ರಾಪ್ತಕರ್ಮಬಂಧವಿದಾಹಿನೇ~।\\
ಜ್ಞಾನಾನಲಪ್ರಭಾವೇನ ತಸ್ಮೈ ಶ್ರೀಗುರವೇ ನಮಃ~॥೭೫॥

ಶೋಷಣಂ ಭವಸಿಂಧೋಶ್ಚ ದೀಪನಂ ಕ್ಷರಸಂಪದಾಂ~।\\
ಗುರೋಃ ಪಾದೋದಕಂ ಯಸ್ಯ ತಸ್ಮೈ ಶ್ರೀಗುರವೇ ನಮಃ~॥೭೬॥

ನ ಗುರೋರಧಿಕಂ ತತ್ತ್ವಂ ನ ಗುರೋರಧಿಕಂ ತಪಃ~।\\
ನ ಗುರೋರಧಿಕಂ ಜ್ಞಾನಂ ತಸ್ಮೈ ಶ್ರೀಗುರವೇ ನಮಃ~॥೭೭॥

ಮನ್ನಾಥಃ ಶ್ರೀಜಗನ್ನಾಥೋ ಮದ್ಗುರುಃ ಶ್ರೀಜಗದ್ಗುರುಃ~।\\
ಮಮಾತ್ಮಾ ಸರ್ವಭೂತಾತ್ಮಾ ತಸ್ಮೈ ಶ್ರೀಗುರವೇ ನಮಃ~॥೭೮॥

ಗುರುರಾದಿರನಾದಿಶ್ಚ ಗುರುಃ ಪರಮದೈವತಂ~।\\
ಗುರುಮಂತ್ರಸಮೋ ನಾಸ್ತಿ ತಸ್ಮೈ ಶ್ರೀಗುರವೇ ನಮಃ~॥೭೯॥

ಏಕ ಏವ ಪರೋ ಬಂಧುರ್ವಿಷಮೇ ಸಮುಪಸ್ಥಿತೇ~।\\
ಗುರುಃ ಸಕಲಧರ್ಮಾತ್ಮಾ ತಸ್ಮೈ ಶ್ರೀಗುರವೇ ನಮಃ~॥೮೦॥

ಗುರುಮಧ್ಯೇ ಸ್ಥಿತಂ ವಿಶ್ವಂ ವಿಶ್ವಮಧ್ಯೇ ಸ್ಥಿತೋ ಗುರುಃ~।\\
ಗುರುರ್ವಿಶ್ವಂ ನ ಚಾನ್ಯೋಽಸ್ತಿ ತಸ್ಮೈ ಶ್ರೀಗುರವೇ ನಮಃ~॥೮೧॥

ಭವಾರಣ್ಯಪ್ರವಿಷ್ಟಸ್ಯ ದಿಙ್ಮೋಹಭ್ರಾಂತಚೇತಸಃ~।\\
ಯೇನ ಸಂದರ್ಶಿತಃ ಪಂಥಾಃ ತಸ್ಮೈ ಶ್ರೀಗುರವೇ ನಮಃ~॥೮೨॥

ತಾಪತ್ರಯಾಗ್ನಿತಪ್ತನಾಮಶಾಂತಪ್ರಾಣಿನಾಂ ಭುವಿ~।\\
ಯಸ್ಯ ಪಾದೋದಕಂ ಗಂಗಾ ತಸ್ಮೈ ಶ್ರೀಗುರವೇ ನಮಃ~॥೮೩॥

ಅಜ್ಞಾನಸರ್ಪದಷ್ಟಾನಾಂ ಪ್ರಾಣಿನಾಂ ಕಶ್ಚಿಕಿತ್ಸಕಃ~।\\
ಸಮ್ಯಗ್‌ಜ್ಞಾನಮಹಾಮಂತ್ರವೇದಿನಂ ಸದ್ಗುರುಂ ವಿನಾ~॥೮೪॥

ಹೇತವೇ ಜಗತಾಮೇವ ಸಂಸಾರಾರ್ಣವಸೇತವೇ~।\\
ಪ್ರಭವೇ ಸರ್ವವಿದ್ಯಾನಾಂ ಶಂಭವೇ ಗುರವೇ ನಮಃ~॥೮೫॥

ಧ್ಯಾನಮೂಲಂ ಗುರೋರ್ಮೂತಿಃ ಪೂಜಾಮೂಲಂ ಗುರೋಃ ಪದಂ~।\\
ಮಂತ್ರಮೂಲಂ ಗುರೋರ್ವಾಕ್ಯಂ ಮುಕ್ತಿಮೂಲಂ ಗುರೋಃ ಕೃಪಾ~॥೮೬॥

ಸಪ್ತಸಾಗರಪರ್ಯಂತಂ ತೀರ್ಥಸ್ನಾನಫಲಂ ತು ಯತ್~।\\
ಗುರುಪಾದಪಯೋಬಿಂದೋಃ ಸಹಸ್ರಾಂಶೇನ ತತ್ಫಲಂ~॥೮೭॥

ಶಿವೇ ರುಷ್ಟೇ ಗುರುಸ್ತ್ರಾತಾ ಗುರೌ ರುಷ್ಟೇ ನ ಕಶ್ಚನ~।\\
ಲಬ್ಧ್ವಾ ಕುಲಗುರುಂ ಸಮ್ಯಗ್ಗುರುಮೇವ ಸಮಾಶ್ರಯೇತ್~॥೮೮॥

ಮಧುಲುಬ್ಧೋ ಯಥಾ ಭೃಂಗಃ ಪುಷ್ಪಾತ್ಪುಷ್ಪಾಂತರಂ ವ್ರಜೇತ್~।\\
ಜ್ಞಾನಲುಬ್ಧಸ್ತಥಾ ಶಿಷ್ಯೋ ಗುರೋರ್ಗುರ್ವನ್ತರಂ ವ್ರಜೇತ್~॥೮೯॥

ವಂದೇ ಗುರುಪದದ್ವಂದ್ವಂ ವಾಙ್ಮನೋತೀತಗೋಚರಂ~।\\
ಶ್ವೇತರಕ್ತಪ್ರಭಾಭಿನ್ನಂ ಶಿವಶಕ್ತ್ಯಾತ್ಮಕಂ ಪರಂ~॥೯೦॥

ಗುಕಾರಂ ಚ ಗುಣಾತೀತಂ ರುಕಾರಂ ರೂಪವರ್ಜಿತಂ~।\\
ಗುಣಾತೀತಮರೂಪಂ ಚ ಯೋ ದದ್ಯಾತ್ ಸ ಗುರುಃ ಸ್ಮೃತಃ~॥೯೧॥

ಅತ್ರಿನೇತ್ರಃ ಶಿವಃ ಸಾಕ್ಷಾತ್ ದ್ವಿಬಾಹುಶ್ಚ ಹರಿಃ ಸ್ಮೃತಃ~।\\
ಯೋಽಚತುರ್ವದನೋ ಬ್ರಹ್ಮಾ ಶ್ರೀಗುರುಃ ಕಥಿತಃ ಪ್ರಿಯೇ~॥೯೨॥

ಅಯಂ ಮಯಾಂಜಲಿರ್ಬದ್ಧೋ ದಯಾಸಾಗರಸಿದ್ಧಯೇ~।\\
ಯದನುಗ್ರಹತೋ ಜಂತುಶ್ಚಿತ್ರಸಂಸಾರಮುಕ್ತಿಭಾಕ್~॥೯೩॥

ಶ್ರೀಗುರೋಃ ಪರಮಂ ರೂಪಂ ವಿವೇಕಚಕ್ಷುರಗ್ರತಃ~।\\
ಮಂದಭಾಗ್ಯಾ ನ ಪಶ್ಯಂತಿ ಅಂಧಾಃ ಸೂರ್ಯೋದಯಂ ಯಥಾ~॥೯೪॥

ಕುಲಾನಾಂ ಕುಲಕೋಟೀನಾಂ ತಾರಕಸ್ತತ್ರ ತತ್‌ಕ್ಷಣಾತ್~।\\
ಅತಸ್ತಂ ಸದ್ಗುರುಂ ಜ್ಞಾತ್ವಾ ತ್ರಿಕಾಲಮಭಿವಾದಯೇತ್~॥೯೫॥

ಶ್ರೀನಾಥಚರಣದ್ವಂದ್ವಂ ಯಸ್ಯಾಂ ದಿಶಿ ವಿರಾಜತೇ~।\\
ತಸ್ಯಾಂ ದಿಶಿ ನಮಸ್ಕುರ್ಯಾದ್ ಭಕ್ತ್ಯಾ ಪ್ರತಿದಿನಂ ಪ್ರಿಯೇ~॥೯೬॥

ಸಾಷ್ಟಾಂಗಪ್ರಣಿಪಾತೇನ ಸ್ತುವನ್ನಿತ್ಯಂ ಗುರುಂ ಭಜೇತ್~।\\
ಭಜನಾತ್ಸ್ಥೈರ್ಯಮಾಪ್ನೋತಿ ಸ್ವಸ್ವರೂಪಮಯೋ ಭವೇತ್~॥೯೭॥

ದೋರ್ಭ್ಯಾಂ ಪದ್ಭ್ಯಾಂ ಚ ಜಾನುಭ್ಯಾಮುರಸಾ ಶಿರಸಾ ದೃಶಾ~।\\
ಮನಸಾ ವಚಸಾ ಚೇತಿ ಪ್ರಣಾಮೋಷ್ಟಾಂಗ ಉಚ್ಯತೇ~॥೯೮॥

ತಸ್ಯೈ ದಿಶೇ ಸತತಮಞ್ಜಲಿರೇಷ ನಿತ್ಯಂ\\
ಪ್ರಕ್ಷಿಪ್ಯತಾಂ ಮುಖರಿತೈರ್ಮಧುರೈಃ ಪ್ರಸೂನೈಃ~।\\
ಜಾಗರ್ತಿ ಯತ್ರ ಭಗವಾನ್ ಗುರುಚಕ್ರವರ್ತೀ\\
ವಿಶ್ವಸ್ಥಿತಿಪ್ರಲಯನಾಟಕನಿತ್ಯಸಾಕ್ಷೀ~॥೯೯॥

ಅಭ್ಯಸ್ತೈಃ ಕಿಮು ದೀರ್ಘಕಾಲವಿಮಲೈರ್ವ್ಯಾಧಿಪ್ರದೈರ್ದುಷ್ಕರೈಃ\\
ಪ್ರಾಣಾಯಾಮಶತೈರನೇಕಕರಣೈರ್ದುಃಖಾತ್ಮಕೈರ್ದುರ್ಜಯೈಃ~।\\
ಯಸ್ಮಿನ್ನಭ್ಯುದಿತೇ ವಿನಶ್ಯತಿ ಬಲೀ ವಾಯುಃ ಸ್ವಯಂ ತತ್‌ಕ್ಷಣಾತ್\\
ಪ್ರಾಪ್ತುಂ ತತ್ಸಹಜಸ್ವಭಾವಮನಿಶಂ ಸೇವೇತ ಚೈಕಂ ಗುರುಂ~॥೧೦೦॥

ಜ್ಞಾನಂ ವಿನಾ ಮುಕ್ತಿಪದಂ ಲಭ್ಯತೇ ಗುರುಭಕ್ತಿತಃ~।\\
ಗುರೋಃ ಪ್ರಸಾದತೋ ನಾನ್ಯತ್ ಸಾಧನಂ ಗುರುಮಾರ್ಗಿಣಾಂ~॥೧೦೧॥

ಯಸ್ಮಾತ್ಪರತರಂ ನಾಸ್ತಿ ನೇತಿ ನೇತೀತಿ ವೈ ಶ್ರುತಿಃ~।\\
ಮನಸಾ ವಚಸಾ ಚೈವ ಸತ್ಯಮಾರಾಧಯೇದ್ಗುರುಂ~॥೧೦೨॥

ಗುರೋಃ ಕೃಪಾಪ್ರಸಾದೇನ ಬ್ರಹ್ಮವಿಷ್ಣುಶಿವಾದಯಃ~।\\
ಸಾಮರ್ಥ್ಯಮಭಜನ್ ಸರ್ವೇ ಸೃಷ್ಟಿಸ್ಥಿತ್ಯಂತಕರ್ಮಣಿ~॥೧೦೩॥

ದೇವಕಿನ್ನರಗಂಧರ್ವಾಃ ಪಿತೃಯಕ್ಷಾಸ್ತು ತುಂಬುರುಃ~।\\
ಮುನಯೋಽಪಿ ನ ಜಾನಂತಿ ಗುರುಶುಶ್ರೂಷಣೇ ವಿಧಿಂ~॥೧೦೪॥

ತಾರ್ಕಿಕಾಶ್ಛಾಂದಸಾಶ್ಚೈವ ದೈವಜ್ಞಾಃ ಕರ್ಮಠಾಃ ಪ್ರಿಯೇ~।\\
ಲೌಕಿಕಾಸ್ತೇ ನ ಜಾನಂತಿ ಗುರುತತ್ತ್ವಂ ನಿರಾಕುಲಂ~॥೧೦೫॥

ಮಹಾಹಂಕಾರಗರ್ವೇಣ ತತೋ ವಿದ್ಯಾಬಲೇನ ಚ~।\\
ಭ್ರಮಂತ್ಯೇತಸ್ಮಿನ್ ಸಂಸಾರೇ ಘಟೀಯಂತ್ರಂ ಯಥಾ ಪುನಃ~॥೧೦೬॥

ಯಜ್ಞಿನೋಽಪಿ ನ ಮುಕ್ತಾಃ ಸ್ಯುಃ ನ ಮುಕ್ತಾ ಯೋಗಿನಸ್ತಥಾ~।\\
ತಾಪಸಾ ಅಪಿ ನೋ ಮುಕ್ತಾ ಗುರುತತ್ತ್ವಾತ್ಪರಾಙ್ಮುಖಾಃ~॥೧೦೭॥

ನ ಮುಕ್ತಾ ವಸುಗಂಧರ್ವಾಃ ಪಿತೃಯಕ್ಷಾಸ್ತು ಚಾರಣಾಃ~।\\
ಋಷಯಃ ಸಿದ್ಧದೇವಾದ್ಯಾ ಗುರುಸೇವಾಪರಾಙ್ಮುಖಾಃ~॥೧೦೮॥
\authorline{॥ಇತಿ ಶ್ರೀಸ್ಕಂದಪುರಾಣೇ ಉತ್ತರಖಂಡೇ ಉಮಾಮಹೇಶ್ವರ ಸಂವಾದೇ ಶ್ರೀ ಗುರುಗೀತಾಯಾಂ ಪ್ರಥಮೋಽಧ್ಯಾಯಃ॥}

\section{॥ಅಥ ದ್ವಿತೀಯೋಽಧ್ಯಾಯಃ॥}

ಧ್ಯಾನಂ ಶ್ರುಣು ಮಹಾದೇವಿ ಸರ್ವಾನಂದಪ್ರದಾಯಕಂ~।\\
ಸರ್ವಸೌಖ್ಯಕರಂ ಚೈವ ಭುಕ್ತಿಮುಕ್ತಿಪ್ರದಾಯಕಂ~॥೧೦೯॥

ಶ್ರೀಮತ್ಪರಂ ಬ್ರಹ್ಮ ಗುರುಂ ಸ್ಮರಾಮಿ\\ ಶ್ರೀಮತ್ಪರಂ ಬ್ರಹ್ಮ ಗುರುಂ ಭಜಾಮಿ~।\\
ಶ್ರೀಮತ್ಪರಂ ಬ್ರಹ್ಮ ಗುರುಂ ವದಾಮಿ\\ ಶ್ರೀಮತ್ಪರಂ ಬ್ರಹ್ಮ ಗುರುಂ ನಮಾಮಿ~॥೧೧೦॥

ಬ್ರಹ್ಮಾನಂದಂ ಪರಮಸುಖದಂ ಕೇವಲಂ ಜ್ಞಾನಮೂರ್ತಿಂ\\
ದ್ವಂದ್ವಾತೀತಂ ಗಗನಸದೃಶಂ ತತ್ತ್ವಮಸ್ಯಾದಿಲಕ್ಷ್ಯಂ~।\\
ಏಕಂ ನಿತ್ಯಂ ವಿಮಲಮಚಲಂ ಸರ್ವಧೀಸಾಕ್ಷಿಭೂತಂ\\
ಭಾವಾತೀತಂ ತ್ರಿಗುಣರಹಿತಂ ಸದ್ಗುರುಂ ತಂ ನಮಾಮಿ~॥೧೧೧॥

ಹೃದಂಬುಜೇ ಕರ್ಣಿಕಮಧ್ಯಸಂಸ್ಥೇ\\ ಸಿಂಹಾಸನೇ ಸಂಸ್ಥಿತದಿವ್ಯಮೂರ್ತಿಂ~।\\
ಧ್ಯಾಯೇದ್ಗುರುಂ ಚಂದ್ರಕಲಾಪ್ರಕಾಶಂ\\ ಸಚ್ಚಿತ್ಸುಖಾಭೀಷ್ಟವರಂ ದಧಾನಂ~॥೧೧೨॥

ಶ್ವೇತಾಂಬರಂ ಶ್ವೇತವಿಲೇಪಪುಷ್ಪಂ\\ ಮುಕ್ತಾವಿಭೂಷಂ ಮುದಿತಂ ದ್ವಿನೇತ್ರಂ~।\\
ವಾಮಾಂಕಪೀಠಸ್ಥಿತದಿವ್ಯಶಕ್ತಿಂ\\ ಮಂದಸ್ಮಿತಂ ಪೂರ್ಣಕೃಪಾನಿಧಾನಂ~॥೧೧೩॥

ಜ್ಞಾನಸ್ವರೂಪಂ ನಿಜಭಾವಯುಕ್ತಂ\\ ಆನಂದಮಾನಂದಕರಂ ಪ್ರಸನ್ನಂ~।\\
ಯೋಗೀಂದ್ರಮೀಡ್ಯಂ ಭವರೋಗವೈದ್ಯಂ\\ ಶ್ರೀಮದ್ಗುರುಂ ನಿತ್ಯಮಹಂ ನಮಾಮಿ~॥೧೧೪॥

ವಂದೇ ಗುರೂಣಾಂ ಚರಣಾರವಿಂದಂ\\ ಸಂದರ್ಶಿತಸ್ವಾತ್ಮಸುಖಾವಬೋಧಮ್~।\\
ಜನಸ್ಯ ಯೋ ಜಾಂಗಲಿಕಾಯಮಾನಂ \\ಸಂಸಾರಹಾಲಾಹಲಮೋಹಶಾಂತ್ಯೈ~॥೧೧೫॥

ಯಸ್ಮಿನ್ ಸೃಷ್ಟಿಸ್ಥಿತಿಧ್ವಂಸನಿಗ್ರಹಾನುಗ್ರಹಾತ್ಮಕಂ~।\\
ಕೃತ್ಯಂ ಪಂಚವಿಧಂ ಶಶ್ವತ್ ಭಾಸತೇ ತಂ ಗುರುಂ ಭಜೇತ್~॥೧೧೬॥

ಪಾದಾಬ್ಜೇ ಸರ್ವಸಂಸಾರದಾವಕಾಲಾನಲಂ ಸ್ವಕೇ~।\\
ಬ್ರಹ್ಮರಂಧ್ರೇ ಸ್ಥಿತಾಂಭೋಜಮಧ್ಯಸ್ಥಂ ಚಂದ್ರಮಂಡಲಂ~॥೧೧೭॥

ಅಕಥಾದಿತ್ರಿರೇಖಾಬ್ಜೇ ಸಹಸ್ರದಲಮಂಡಲೇ~।\\
ಹಂಸಪಾರ್ಶ್ವತ್ರಿಕೋಣೇ ಚ ಸ್ಮರೇತ್ತನ್ಮಧ್ಯಗಂ ಗುರುಂ~॥೧೧೮॥

ನಿತ್ಯಂ ಶುದ್ಧಂ ನಿರಾಭಾಸಂ ನಿರಾಕಾರಂ ನಿರಂಜನಂ~।\\
ನಿತ್ಯಬೋಧಂ ಚಿದಾನಂದಂ ಗುರುಂ ಬ್ರಹ್ಮ ನಮಾಮ್ಯಹಂ~॥೧೧೯॥

ಸಕಲಭುವನಸೃಷ್ಟಿಃ ಕಲ್ಪಿತಾಶೇಷಸೃಷ್ಟಿಃ\\
ನಿಖಿಲನಿಗಮದೃಷ್ಟಿಃ ಸತ್ಪದಾರ್ಥೈಕಸೃಷ್ಟಿಃ~।\\
ಅಥಗಣಪರಮೇಷ್ಠೀ ಸತ್ಪದಾರ್ಥೈಕದೃಷ್ಟಿಃ\\
ಭವಗುಣಪರಮೇಷ್ಠೀ ಮೋಕ್ಷಮಾರ್ಗೈಕದೃಷ್ಟಿಃ~॥೧೨೦॥

ಸಕಲಭುವನರಂಗಸ್ಥಾಪನಾಸ್ತಂಭಯಷ್ಟಿಃ\\
ಸಕರುಣರಸವೃಷ್ಟಿಸ್ತತ್ತ್ವಮಾಲಾಸಮಷ್ಟಿಃ~।\\
ಸಕಲಸಮಯಸೃಷ್ಟಿಸ್ಸಚ್ಚಿದಾನಂದದೃಷ್ಟಿಃ\\
ನಿವಸತು ಮಯಿ ನಿತ್ಯಂ ಶ್ರೀಗುರೋರ್ದಿವ್ಯದೃಷ್ಟಿಃ~॥೧೨೧॥

ನ ಗುರೋರಧಿಕಂ ನ ಗುರೋರಧಿಕಂ\\ ನ ಗುರೋರಧಿಕಂ ನ ಗುರೋರಧಿಕಂ~।\\
ಶಿವಶಾಸನತಃ ಶಿವಶಾಸನತಃ \\ಶಿವಶಾಸನತಃ ಶಿವಶಾಸನತಃ~॥೧೨೨॥

ಇದಮೇವ ಶಿವಮಿದಮೇವ ಶಿವಂ\\ ಇದಮೇವ ಶಿವಮಿದಮೇವ ಶಿವಂ~।\\
ಹರಿಶಾಸನತೋ ಹರಿಶಾಸನತೋ\\ ಹರಿಶಾಸನತೋ ಹರಿಶಾಸನತಃ~॥೧೨೩॥

ವಿದಿತಂ ವಿದಿತಂ ವಿದಿತಂ ವಿದಿತಂ ವಿಜನಂ ವಿಜನಂ ವಿಜನಂ ವಿಜನಂ~।\\
ವಿಧಿಶಾಸನತೋ ವಿಧಿಶಾಸನತೋ ವಿಧಿಶಾಸನತೋ ವಿಧಿಶಾಸನತಃ~॥೧೨೪॥

ಏವಂವಿಧಂ ಗುರುಂ ಧ್ಯಾತ್ವಾ ಜ್ಞಾನಮುತ್ಪದ್ಯತೇ ಸ್ವಯಂ~।\\
ತದಾ ಗುರೂಪದೇಶೇನ ಮುಕ್ತೋಽಹಮಿತಿ ಭಾವಯೇತ್~॥೧೨೫॥

ಗುರೂಪದಿಷ್ಟಮಾರ್ಗೇಣ ಮನಃಶುದ್ಧಿಂ ತು ಕಾರಯೇತ್~।\\
ಅನಿತ್ಯಂ ಖಂಡಯೇತ್ಸರ್ವಂ ಯತ್ಕಿಂಚಿದಾತ್ಮಗೋಚರಂ~॥೧೨೬॥

ಜ್ಞೇಯಂ ಸರ್ವಂ ಪ್ರತೀತಂ ಚ ಜ್ಞಾನಂ ಚ ಮನ ಉಚ್ಯತೇ~।\\
ಜ್ಞಾನಂ ಜ್ಞೇಯಂ ಸಮಂ ಕುರ್ಯಾನ್ನಾನ್ಯಃ ಪಂಥಾ ದ್ವಿತೀಯಕಃ~॥೧೨೭॥

ಕಿಮತ್ರ ಬಹುನೋಕ್ತೇನ ಶಾಸ್ತ್ರಕೋಟಿಶತೈರಪಿ~।\\
ದುರ್ಲಭಾ ಚಿತ್ತವಿಶ್ರಾಂತಿಃ ವಿನಾ ಗುರುಕೃಪಾಂ ಪರಾಂ~॥೧೨೮॥

ಕರುಣಾಖಡ್ಗಪಾತೇನ ಛಿತ್ವಾ ಪಾಶಾಷ್ಟಕಂ ಶಿಶೋಃ~।\\
ಸಮ್ಯಗಾನಂದಜನಕಃ ಸದ್ಗುರುಃ ಸೋಽಭಿಧೀಯತೇ~॥೧೨೯॥

ಏವಂ ಶ್ರುತ್ವಾ ಮಹಾದೇವಿ ಗುರುನಿಂದಾಂ ಕರೋತಿ ಯಃ~।\\
ಸ ಯಾತಿ ನರಕಾನ್ ಘೋರಾನ್ ಯಾವಚ್ಚಂದ್ರದಿವಾಕರೌ~॥೧೩೦॥

ಯಾವತ್ಕಲ್ಪಾಂತಕೋ ದೇಹಸ್ತಾವದ್ದೇವಿ ಗುರುಂ ಸ್ಮರೇತ್~।\\
ಗುರುಲೋಪೋ ನ ಕರ್ತವ್ಯಃ ಸ್ವಚ್ಛಂದೋ ಯದಿ ವಾ ಭವೇತ್~॥೧೩೧॥

ಹುಂಕಾರೇಣ ನ ವಕ್ತವ್ಯಂ ಪ್ರಾಜ್ಞಶಿಷ್ಯೈಃ ಕದಾಚನ~।\\
ಗುರೋರಗ್ರೇ ನ ವಕ್ತವ್ಯಮಸತ್ಯಂ ತು ಕದಾಚನ~॥೧೩೨॥

ಗುರುಂ ತ್ವಂಕೃತ್ಯ ಹುಂಕೃತ್ಯ ಗುರುಸಾನ್ನಿಧ್ಯಭಾಷಣಃ~।\\
ಅರಣ್ಯೇ ನಿರ್ಜಲೇ ದೇಶೇ ಸಂಭವೇದ್ ಬ್ರಹ್ಮರಾಕ್ಷಸಃ~॥೧೩೩॥

ಅದ್ವೈತಂ ಭಾವಯೇನ್ನಿತ್ಯಂ ಸರ್ವಾವಸ್ಥಾಸು ಸರ್ವದಾ~।\\
ಕದಾಚಿದಪಿ ನೋ ಕುರ್ಯಾದದ್ವೈತಂ ಗುರುಸನ್ನಿಧೌ~॥೧೩೪॥

ದೃಶ್ಯವಿಸ್ಮೃತಿಪರ್ಯಂತಂ ಕುರ್ಯಾದ್ ಗುರುಪದಾರ್ಚನಂ~।\\
ತಾದೃಶಸ್ಯೈವ ಕೈವಲ್ಯಂ ನ ಚ ತದ್ವ್ಯತಿರೇಕಿಣಃ~॥೧೩೫॥

ಅಪಿ ಸಂಪೂರ್ಣತತ್ತ್ವಜ್ಞೋ ಗುರುತ್ಯಾಗಿ ಭವೇದ್ಯದಾ~।\\
ಭವತ್ಯೇವ ಹಿ ತಸ್ಯಾಂತಕಾಲೇ ವಿಕ್ಷೇಪಮುತ್ಕಟಂ~॥೧೩೬॥

ಗುರುಕಾರ್ಯಂ ನ ಲಂಘೇತ ನಾಪೃಷ್ಟ್ವಾ ಕಾರ್ಯಮಾಚರೇತ್~।\\
ನ ಹ್ಯುತ್ತಿಷ್ಠೇದ್ದಿಶೇಽನತ್ವಾ ಗುರುಸದ್ಭಾವಶೋಭಿತಃ~॥೧೩೭॥

ಗುರೌ ಸತಿ ಸ್ವಯಂ ದೇವಿ ಪರೇಷಾಂ ತು ಕದಾಚನ~।\\
ಉಪದೇಶಂ ನ ವೈ ಕುರ್ಯಾತ್ ತಥಾ ಚೇದ್ರಾಕ್ಷಸೋ ಭವೇತ್~॥೧೩೮॥

ನ ಗುರೋರಾಶ್ರಮೇ ಕುರ್ಯಾತ್ ದುಷ್ಪಾನಂ ಪರಿಸರ್ಪಣಂ~।\\
ದೀಕ್ಷಾ ವ್ಯಾಖ್ಯಾ ಪ್ರಭುತ್ವಾದಿ ಗುರೋರಾಜ್ಞಾಂ ನ ಕಾರಯೇತ್~॥೧೩೯॥

ನೋಪಾಶ್ರಮಂ ಚ ಪರ್ಯಕಂ ನ ಚ ಪಾದಪ್ರಸಾರಣಂ~।\\
ನಾಂಗಭೋಗಾದಿಕಂ ಕುರ್ಯಾನ್ನ ಲೀಲಾಮಪರಾಮಪಿ~॥೧೪೦॥

ಗುರೂಣಾಂ ಸದಸದ್ವಾಪಿ ಯದುಕ್ತಂ ತನ್ನ ಲಂಘಯೇತ್~।\\
ಕುರ್ವನ್ನಾಜ್ಞಾಂ ದಿವಾ ರಾತ್ರೌ ದಾಸವನ್ನಿವಸೇದ್ಗುರೌ~॥೧೪೧॥

ಅದತ್ತಂ ನ ಗುರೋರ್ದ್ರವ್ಯಮುಪಭುಂಜೀತ ಕರ್ಹಿಚಿತ್~।\\
ದತ್ತೇ ಚ ರಂಕವದ್ಗ್ರಾಹ್ಯಂ ಪ್ರಾಣೋಽಪ್ಯೇತೇನ ಲಭ್ಯತೇ~॥೧೪೨॥

ಪಾದುಕಾಸನಶಯ್ಯಾದಿ ಗುರುಣಾ ಯದಧಿಷ್ಠಿತಂ~।\\
ನಮಸ್ಕುರ್ವೀತ ತತ್ಸರ್ವಂ ಪಾದಾಭ್ಯಾಂ ನ ಸ್ಪೃಶೇತ್ ಕ್ವಚಿತ್~॥೧೪೩॥

ಗಚ್ಛತಃ ಪೃಷ್ಠತೋ ಗಚ್ಛೇತ್ ಗುರುಚ್ಛಾಯಾಂ ನ ಲಂಘಯೇತ್~।\\
ನೋಲ್ಬಣಂ ಧಾರಯೇದ್ವೇಷಂ ನಾಲಂಕಾರಾಂಸ್ತಥೋಲ್ಬಣಾನ್~॥೧೪೪॥

ಗುರುನಿಂದಾಕರಂ ದೃಷ್ಟ್ವಾ ಧಾವಯೇದಥ ವಾರಯೇತ್~।\\
ಸ್ಥಾನಂ ವಾ ತತ್ಪರಿತ್ಯಾಜ್ಯಂ ಜಿಹ್ವಾಚ್ಛೇದಾಕ್ಷಮೋ ಯದಿ~॥೧೪೫॥

ನೋಚ್ಛಿಷ್ಟಂ ಕಸ್ಯಚಿದ್ದೇಯಂ ಗುರೋರಾಜ್ಞಾಂ ನ ಚ ತ್ಯಜೇತ್~।\\
ಕೃತ್ಸ್ನಮುಚ್ಛಿಷ್ಟಮಾದಾಯ ಹವಿರ್ವದ್ಭಕ್ಷಯೇತ್ಸ್ವಯಂ~॥೧೪೬॥

ನಾನೃತಂ ನಾಪ್ರಿಯಂ ಚೈವ ನ ಗರ್ವಂ ನಾಪಿ ವಾ ಬಹು~।\\
ನ ನಿಯೋಗಧರಂ ಬ್ರೂಯಾತ್ ಗುರೋರಾಜ್ಞಾಂ ವಿಭಾವಯೇತ್~॥೧೪೭॥

ಪ್ರಭೋ ದೇವಕುಲೇಶಾನ ಸ್ವಾಮಿನ್ ರಾಜನ್ ಕುಲೇಶ್ವರ~।\\
ಇತಿ ಸಂಬೋಧನೈರ್ಭೀತೋ ಸಂಚರೇದ್ಗುರುಸನ್ನಿಧೌ~॥೧೪೮॥

ಮುನಿಭಿಃ ಪನ್ನಗೈರ್ವಾಪಿ ಸುರೈರ್ವಾ ಶಾಪಿತೋ ಯದಿ~।\\
ಕಾಲಮೃತ್ಯುಭಯಾದ್ವಾಪಿ ಗುರುಃ ಸಂತ್ರಾತಿ ಪಾರ್ವತಿ~॥೧೪೯॥

ಅಶಕ್ತಾ ಹಿ ಸುರಾದ್ಯಾಶ್ಚ ಹ್ಯಶಕ್ತಾಃ ಮುನಯಸ್ತಥಾ~।\\
ಗುರುಶಾಪೋಪಪನ್ನಸ್ಯ ರಕ್ಷಣಾಯ ಚ ಕುತ್ರಚಿತ್~॥೧೫೦॥

ಮಂತ್ರರಾಜಮಿದಂ ದೇವಿ ಗುರುರಿತ್ಯಕ್ಷರದ್ವಯಂ~।\\
ಸ್ಮೃತಿವೇದಪುರಾಣಾನಾಂ ಸಾರಮೇವ ನ ಸಂಶಯಃ~॥೧೫೧॥

ಸತ್ಕಾರಮಾನಪೂಜಾರ್ಥಂ ದಂಡಕಾಷಾಯಧಾರಣಃ~।\\
ಸ ಸಂನ್ಯಾಸೀ ನ ವಕ್ತವ್ಯಃ ಸಂನ್ಯಾಸೀ ಜ್ಞಾನತತ್ಪರಃ~॥೧೫೨॥

ವಿಜಾನಂತಿ ಮಹಾವಾಕ್ಯಂ ಗುರೋಶ್ಚರಣ ಸೇವಯಾ~।\\
ತೇ ವೈ ಸಂನ್ಯಾಸಿನಃ ಪ್ರೋಕ್ತಾ ಇತರೇ ವೇಷಧಾರಿಣಃ~॥೧೫೩॥

ನಿತ್ಯಂ ಬ್ರಹ್ಮ ನಿರಾಕಾರಂ ನಿರ್ಗುಣಂ ಸತ್ಯಚಿದ್ಘನಂ~।\\
ಯಃ ಸಾಕ್ಷಾತ್ಕುರುತೇ ಲೋಕೇ ಗುರುತ್ವಂ ತಸ್ಯ ಶೋಭತೇ~॥೧೫೪॥

ಗುರುಪ್ರಸಾದತಃ ಸ್ವಾತ್ಮನ್ಯಾತ್ಮಾರಾಮನಿರೀಕ್ಷಣಾತ್~।\\
ಸಮತಾ ಮುಕ್ತಿಮಾರ್ಗೇಣ ಸ್ವಾತ್ಮಜ್ಞಾನಂ ಪ್ರವರ್ತತೇ~॥೧೫೫॥

ಆಬ್ರಹ್ಮಸ್ತಂಭಪರ್ಯಂತಂ ಪರಮಾತ್ಮಸ್ವರೂಪಕಂ~।\\
ಸ್ಥಾವರಂ ಜಂಗಮಂ ಚೈವ ಪ್ರಣಮಾಮಿ ಜಗನ್ಮಯಂ~॥೧೫೬॥

ವಂದೇಹಂ ಸಚ್ಚಿದಾನಂದಂ ಭಾವಾತೀತಂ ಜಗದ್ಗುರುಂ~।\\
ನಿತ್ಯಂ ಪೂರ್ಣಂ ನಿರಾಕಾರಂ ನಿರ್ಗುಣಂ ಸ್ವಾತ್ಮಸಂಸ್ಥಿತಂ~॥೧೫೭॥

ಪರಾತ್ಪರತರಂ ಧ್ಯಾಯೇನ್ನಿತ್ಯಮಾನಂದಕಾರಕಂ~।\\
ಹೃದಯಾಕಾಶಮಧ್ಯಸ್ಥಂ ಶುದ್ಧಸ್ಫಟಿಕಸನ್ನಿಭಂ~॥೧೫೮॥

ಸ್ಫಾಟಿಕೇ ಸ್ಫಾಟಿಕಂ ರೂಪಂ ದರ್ಪಣೇ ದರ್ಪಣೋ ಯಥಾ~।\\
ತಥಾತ್ಮನಿ ಚಿದಾಕಾರಮಾನಂದಂ ಸೋಽಹಮಿತ್ಯುತ~॥೧೫೯॥

ಅಂಗುಷ್ಠಮಾತ್ರಂ ಪುರುಷಂ ಧ್ಯಾಯೇಚ್ಚ ಚಿನ್ಮಯಂ ಹೃದಿ~।\\
ತತ್ರ ಸ್ಫುರತಿ ಯೋ ಭಾವಃ ಶ್ರುಣು ತತ್ಕಥಯಾಮಿ ತೇ~॥೧೬೦॥

ಅಜೋಽಹಮಮರೋಽಹಂ ಚ ಹ್ಯನಾದಿನಿಧನೋ ಹ್ಯಹಂ~।\\
ಅವಿಕಾರಶ್ಚಿದಾನಂದೋ ಹ್ಯಣೀಯಾನ್ಮಹತೋ ಮಹಾನ್~॥೧೬೧॥

ಅಪೂರ್ವಮಪರಂ ನಿತ್ಯಂ ಸ್ವಯಂಜ್ಯೋತಿರ್ನಿರಾಮಯಂ~।\\
ವಿರಜಂ ಪರಮಾಕಾಶಂ ಧ್ರುವಮಾನಂದಮವ್ಯಯಂ~॥೧೬೨॥

ಅಗೋಚರಂ ತಥಾಽಗಮ್ಯಂ ನಾಮರೂಪವಿವರ್ಜಿತಂ~।\\
ನಿಃಶಬ್ದಂ ತು ವಿಜಾನೀಯಾತ್ಸ್ವಭಾವಾದ್ಬ್ರಹ್ಮ ಪಾರ್ವತಿ~॥೧೬೩॥

ಯಥಾ ಗಂಧಸ್ವಭಾವತ್ವಂ ಕರ್ಪೂರಕುಸುಮಾದಿಷು~।\\
ಶೀತೋಷ್ಣತ್ವಸ್ವಭಾವತ್ವಂ ತಥಾ ಬ್ರಹ್ಮಣಿ ಶಾಶ್ವತಂ~॥೧೬೪॥

ಯಥಾ ನಿಜಸ್ವಭಾವೇನ ಕುಂಡಲೇ ಕಟಕಾದಯಃ~।\\
ಸುವರ್ಣತ್ವೇನ ತಿಷ್ಠಂತಿ ತಥಾಽಹಂ ಬ್ರಹ್ಮ ಶಾಶ್ವತಂ~॥೧೬೫॥

ಸ್ವಯಂ ತಥಾವಿಧೋ ಭೂತ್ವಾ ಸ್ಥಾತವ್ಯಂ ಯತ್ರಕುತ್ರಚಿತ್~।\\
ಕೀಟೋ ಭೃಂಗ ಇವ ಧ್ಯಾನಾದ್ಯಥಾ ಭವತಿ ತಾದೃಶಃ~॥೧೬೬॥

ಗುರುಧ್ಯಾನಂ ತಥಾ ಕೃತ್ವಾ ಸ್ವಯಂ ಬ್ರಹ್ಮಮಯೋ ಭವೇತ್~।\\
ಪಿಂಡೇ ಪದೇ ತಥಾ ರೂಪೇ ಮುಕ್ತಾಸ್ತೇ ನಾತ್ರ ಸಂಶಯಃ~॥೧೬೭॥

ಶ್ರೀಪಾರ್ವತೀ ಉವಾಚ~।\\
ಪಿಂಡಂ ಕಿಂ ತು ಮಹಾದೇವ ಪದಂ ಕಿಂ ಸಮುದಾಹೃತಂ~।\\
ರೂಪಾತೀತಂ ಚ ರೂಪಂ ಕಿಮೇತದಾಖ್ಯಾಹಿ ಶಂಕರ~॥೧೬೮॥

ಶ್ರೀಮಹಾದೇವ ಉವಾಚ~।\\
ಪಿಂಡಂ ಕುಂಡಲಿನೀ ಶಕ್ತಿಃ ಪದಂ ಹಂಸಮುದಾಹೃತಂ~।\\
ರೂಪಂ ಬಿಂದುರಿತಿ ಜ್ಞೇಯಂ ರೂಪಾತೀತಂ ನಿರಂಜನಂ~॥೧೬೯॥

ಪಿಂಡೇ ಮುಕ್ತಾಃ ಪದೇ ಮುಕ್ತಾ ರೂಪೇ ಮುಕ್ತಾ ವರಾನನೇ~।\\
ರೂಪಾತೀತೇ ತು ಯೇ ಮುಕ್ತಾಸ್ತೇ ಮುಕ್ತಾ ನಾತ್ರ ಸಂಶಯಃ~॥೧೭೦॥

ಗುರುಧ್ಯಾನೇನೈವ ನಿತ್ಯಂ ದೇಹೀ ಬ್ರಹ್ಮಮಯೋ ಭವೇತ್~।\\
ಸ್ಥಿತಶ್ಚ ಯತ್ರ ಕುತ್ರಾಪಿ ಮುಕ್ತೋಽಸೌ ನಾತ್ರ ಸಂಶಯಃ~॥೧೭೧॥

ಜ್ಞಾನಂ ಸ್ವಾನುಭವಃ ಶಾಂತಿರ್ವೈರಾಗ್ಯಂ ವಕ್ತೃತಾ ಧೃತಿಃ~।\\
ಷಡ್ಗುಣೈಶ್ವರ್ಯಯುಕ್ತೋ ಹಿ ಭಗವಾನ್ ಶ್ರೀಗುರುಃ ಪ್ರಿಯೇ~॥೧೭೨॥

ಗುರುಃ ಶಿವೋ ಗುರುರ್ದೇವೋ ಗುರುರ್ಬಂಧುಃ ಶರೀರಿಣಾಂ~।\\
ಗುರುರಾತ್ಮಾ ಗುರುರ್ಜೀವೋ ಗುರೋರನ್ಯನ್ನ ವಿದ್ಯತೇ~॥೧೭೩॥

ಏಕಾಕೀ ನಿಸ್ಪೃಹಃ ಶಾಂತಶ್ಚಿಂತಾಸೂಯಾದಿವರ್ಜಿತಃ~।\\
ಬಾಲ್ಯಭಾವೇನ ಯೋ ಭಾತಿ ಬ್ರಹ್ಮಜ್ಞಾನೀ ಸ ಉಚ್ಯತೇ~॥೧೭೪॥

ನ ಸುಖಂ ವೇದಶಾಸ್ತ್ರೇಷು ನ ಸುಖಂ ಮಂತ್ರಯಂತ್ರಕೇ~।\\
ಗುರೋಃ ಪ್ರಸಾದಾದನ್ಯತ್ರ ಸುಖಂ ನಾಸ್ತಿ ಮಹೀತಲೇ~॥೧೭೫॥

ಚಾರ್ವಾಕವೈಷ್ಣವಮತೇ ಸುಖಂ ಪ್ರಾಭಾಕರೇ ನ ಹಿ~।\\
ಗುರೋಃ ಪಾದಾಂತಿಕೇ ಯದ್ವತ್ಸುಖಂ ವೇದಾಂತಸಮ್ಮತಂ~॥೧೭೬॥

ನ ತತ್ಸುಖಂ ಸುರೇಂದ್ರಸ್ಯ ನ ಸುಖಂ ಚಕ್ರವರ್ತಿನಾಂ~।\\
ಯತ್ಸುಖಂ ವೀತರಾಗಸ್ಯ ಮುನೇರೇಕಾಂತವಾಸಿನಃ~॥೧೭೭॥

ನಿತ್ಯಂ ಬ್ರಹ್ಮರಸಂ ಪೀತ್ವಾ ತೃಪ್ತೋ ಯಃ ಪರಮಾತ್ಮನಿ~।\\
ಇಂದ್ರಂ ಚ ಮನ್ಯತೇ ತುಚ್ಛಂ ನೃಪಾಣಾಂ ತತ್ರ ಕಾ ಕಥಾ~॥೧೭೮॥

ಯತಃ ಪರಮಕೈವಲ್ಯಂ ಗುರುಮಾರ್ಗೇಣ ವೈ ಭವೇತ್~।\\
ಗುರುಭಕ್ತಿರತಃ ಕಾರ್ಯಾ ಸರ್ವದಾ ಮೋಕ್ಷಕಾಂಕ್ಷಿಭಿಃ~॥೧೭೯॥

ಏಕ ಏವಾದ್ವಿತೀಯೋಽಹಂ ಗುರುವಾಕ್ಯೇನ ನಿಶ್ಚಿತಃ~।\\
ಏವಮಭ್ಯಸ್ಯತಾ ನಿತ್ಯಂ ನ ಸೇವ್ಯಂ ವೈ ವನಾಂತರಂ~॥೧೮೦॥

ಅಭ್ಯಾಸಾನ್ನಿಮಿಷೇಣೈವ ಸಮಾಧಿಮಧಿಗಚ್ಛತಿ~।\\
ಆಜನ್ಮಜನಿತಂ ಪಾಪಂ ತತ್ಕ್ಷಣಾದೇವ ನಶ್ಯತಿ~॥೧೮೧॥

ಕಿಮಾವಾಹನಮವ್ಯಕ್ತೇ ವ್ಯಾಪಕೇ ಕಿಂ ವಿಸರ್ಜನಂ~।\\
ಅಮೂರ್ತೌ ಚ ಕಥಂ ಪೂಜಾ ಕಥಂ ಧ್ಯಾನಂ ನಿರಾಮಯೇ~॥೧೮೨॥

ಗುರುರ್ವಿಷ್ಣುಃ ಸತ್ತ್ವಮಯೋ ರಾಜಸಶ್ಚತುರಾನನಃ~।\\
ತಾಮಸೋ ರುದ್ರರೂಪೇಣ ಸೃಜತ್ಯವತಿ ಹಂತಿ ಚ~॥೧೮೩॥

ಸ್ವಯಂ ಬ್ರಹ್ಮಮಯೋ ಭೂತ್ವಾ ತತ್ಪರಂ ನಾವಲೋಕಯೇತ್~।\\
ಪರಾತ್ಪರತರಂ ನಾನ್ಯತ್ ಸರ್ವಗಂ ಚ ನಿರಾಮಯಂ~॥೧೮೪॥

ತಸ್ಯಾವಲೋಕನಂ ಪ್ರಾಪ್ಯ ಸರ್ವಸಂಗವಿವರ್ಜಿತಃ~।\\
ಏಕಾಕೀ ನಿಸ್ಪೃಹಃ ಶಾಂತಃ ಸ್ಥಾತವ್ಯಂ ತತ್ಪ್ರಸಾದತಃ~॥೧೮೫॥

ಲಬ್ಧಂ ವಾಽಥ ನ ಲಬ್ಧಂ ವಾ ಸ್ವಲ್ಪಂ ವಾ ಬಹುಲಂ ತಥಾ~।\\
ನಿಷ್ಕಾಮೇನೈವ ಭೋಕ್ತವ್ಯಂ ಸದಾ ಸಂತುಷ್ಟಮಾನಸಃ~॥೧೮೬॥

ಸರ್ವಜ್ಞಪದಮಿತ್ಯಾಹುರ್ದೇಹೀ ಸರ್ವಮಯೋ ಭುವಿ~।\\
ಸದಾಽನಂದಃ ಸದಾ ಶಾಂತೋ ರಮತೇ ಯತ್ರಕುತ್ರಚಿತ್~॥೧೮೭॥

ಯತ್ರೈವ ತಿಷ್ಠತೇ ಸೋಽಪಿ ಸ ದೇಶಃ ಪುಣ್ಯಭಾಜನಃ~।\\
ಮುಕ್ತಸ್ಯ ಲಕ್ಷಣಂ ದೇವಿ ತವಾಗ್ರೇ ಕಥಿತಂ ಮಯಾ~॥೧೮೮॥

ಉಪದೇಶಸ್ತ್ವಯಂ ದೇವಿ ಗುರುಮಾರ್ಗೇಣ ಮುಕ್ತಿದಃ~।\\
ಗುರುಭಕ್ತಿಸ್ತಥಾತ್ಯಂತಾ ಕರ್ತವ್ಯಾ ವೈ ಮನೀಷಿಭಿಃ~॥೧೮೯॥

ನಿತ್ಯಯುಕ್ತಾಶ್ರಯಃ ಸರ್ವೋ ವೇದಕೃತ್ಸರ್ವವೇದಕೃತ್~।\\
ಸ್ವಪರಜ್ಞಾನದಾತಾ ಚ ತಂ ವಂದೇ ಗುರುಮೀಶ್ವರಂ~॥೧೯೦॥

ಯದ್ಯಪ್ಯಧೀತಾ ನಿಗಮಾಃ ಷಡಂಗಾ ಆಗಮಾಃ ಪ್ರಿಯೇ~।\\
ಅಧ್ಯಾತ್ಮಾದೀನಿ ಶಾಸ್ತ್ರಾಣಿ ಜ್ಞಾನಂ ನಾಸ್ತಿ ಗುರುಂ ವಿನಾ~॥೧೯೧॥

ಶಿವಪೂಜಾರತೋ ವಾಪಿ ವಿಷ್ಣುಪೂಜಾರತೋಽಥವಾ~।\\
ಗುರುತತ್ತ್ವವಿಹೀನಶ್ಚೇತ್ತತ್ಸರ್ವಂ ವ್ಯರ್ಥಮೇವ ಹಿ~॥೧೯೨॥

ಶಿವಸ್ವರೂಪಮಜ್ಞಾತ್ವಾ ಶಿವಪೂಜಾ ಕೃತಾ ಯದಿ~।\\
ಸಾ ಪೂಜಾ ನಾಮಮಾತ್ರಂ ಸ್ಯಾಚ್ಚಿತ್ರದೀಪ ಇವ ಪ್ರಿಯೇ~॥೧೯೩॥

ಸರ್ವಂ ಸ್ಯಾತ್ಸಫಲಂ ಕರ್ಮ ಗುರುದೀಕ್ಷಾಪ್ರಭಾವತಃ~।\\
ಗುರುಲಾಭಾತ್ಸರ್ವಲಾಭೋ ಗುರುಹೀನಸ್ತು ಬಾಲಿಶಃ~॥೧೯೪॥

ಗುರುಹೀನಃ ಪಶುಃ ಕೀಟಃ ಪತಂಗೋ ವಕ್ತುಮರ್ಹತಿ~।\\
ಶಿವರೂಪಂ ಸ್ವರೂಪಂ ಚ ನ ಜಾನಾತಿ ಯತಸ್ಸ್ವಯಂ~॥೧೯೫॥

ತಸ್ಮಾತ್ಸರ್ವಪ್ರಯತ್ನೇನ ಸರ್ವಸಂಗವಿವರ್ಜಿತಃ~।\\
ವಿಹಾಯ ಶಾಸ್ತ್ರಜಾಲಾನಿ ಗುರುಮೇವ ಸಮಾಶ್ರಯೇತ್~॥೧೯೬॥

ನಿರಸ್ತಸರ್ವಸಂದೇಹೋ ಏಕೀಕೃತ್ಯ ಸುದರ್ಶನಂ~।\\
ರಹಸ್ಯಂ ಯೋ ದರ್ಶಯತಿ ಭಜಾಮಿ ಗುರುಮೀಶ್ವರಂ~॥೧೯೭॥

ಜ್ಞಾನಹೀನೋ ಗುರುಸ್ತ್ಯಾಜ್ಯೋ ಮಿಥ್ಯಾವಾದೀ ವಿಡಂಬಕಃ~।\\
ಸ್ವವಿಶ್ರಾಂತಿಂ ನ ಜಾನಾತಿ ಪರಶಾಂತಿಂ ಕರೋತಿ ಕಿಂ~॥೧೯೮॥

ಶಿಲಾಯಾಃ ಕಿಂ ಪರಂ ಜ್ಞಾನಂ ಶಿಲಾಸಂಘಪ್ರತಾರಣೇ~।\\
ಸ್ವಯಂ ತರ್ತುಂ ನ ಜಾನಾತಿ ಪರಂ ನಿಸ್ತಾರಯೇತ್ ಕಥಂ~॥೧೯೯॥

ನ ವಂದನೀಯಾಸ್ತೇ ಕಷ್ಟಂ ದರ್ಶನಾದ್ಭ್ರಾಂತಿಕಾರಕಾಃ~।\\
ವರ್ಜಯೇತ್ತಾನ್ ಗುರೂನ್ ದೂರೇ ಧೀರಾನೇವ ಸಮಾಶ್ರಯೇತ್~॥೨೦೦॥

ಪಾಷಂಡಿನಃ ಪಾಪರತಾಃ ನಾಸ್ತಿಕಾ ಭೇದಬುದ್ಧಯಃ~।\\
ಸ್ತ್ರೀಲಂಪಟಾ ದುರಾಚಾರಾಃ ಕೃತಘ್ನಾ ಬಕವೃತ್ತಯಃ~॥೨೦೧॥

ಕರ್ಮಭ್ರಷ್ಟಾಃ ಕ್ಷಮಾನಷ್ಟಾ ನಿಂದ್ಯತರ್ಕೈಶ್ಚ ವಾದಿನಃ~।\\
ಕಾಮಿನಃ ಕ್ರೋಧಿನಶ್ಚೈವ ಹಿಂಸ್ರಾಶ್ಚಂಡಾಃ ಶಠಾಸ್ತಥಾ~॥೨೦೨॥

ಜ್ಞಾನಲುಪ್ತಾ ನ ಕರ್ತವ್ಯಾ ಮಹಾಪಾಪಾಸ್ತಥಾ ಪ್ರಿಯೇ~।\\
ಏಭ್ಯೋ ಭಿನ್ನೋ ಗುರುಃ ಸೇವ್ಯಃ ಏಕಭಕ್ತ್ಯಾ ವಿಚಾರ್ಯ ಚ~॥೨೦೩॥

ಶಿಷ್ಯಾದನ್ಯತ್ರ ದೇವೇಶಿ ನ ವದೇದ್ಯಸ್ಯ ಕಸ್ಯಚಿತ್~।\\
ನರಾಣಾಂ ಚ ಫಲಪ್ರಾಪ್ತೌ ಭಕ್ತಿರೇವ ಹಿ ಕಾರಣಂ~॥೨೦೪॥

ಗೂಢೋ ದೃಢಶ್ಚ ಪ್ರೀತಶ್ಚ ಮೌನೇನ ಸುಸಮಾಹಿತಃ~।\\
ಸಕೃತ್ಕಾಮಗತೌ ವಾಪಿ ಪಂಚಧಾ ಗುರುರೀರಿತಃ~॥೨೦೫॥

ಸರ್ವಂ ಗುರುಮುಖಾಲ್ಲಬ್ಧಂ ಸಫಲಂ ಪಾಪನಾಶನಂ~।\\
ಯದ್ಯದಾತ್ಮಹಿತಂ ವಸ್ತು ತತ್ತದ್ದ್ರವ್ಯಂ ನ ವಂಚಯೇತ್~॥೨೦೬॥

ಗುರುದೇವಾರ್ಪಣಂ ವಸ್ತು ತೇನ ತುಷ್ಟೋಽಸ್ಮಿ ಸುವ್ರತೇ~।\\
ಶ್ರೀಗುರೋಃಪಾದುಕಾಂಮುದ್ರಾಂ ಮೂಲಮಂತ್ರಂಚಗೋಪಯೇತ್~॥೨೦೭॥

ನತಾಸ್ಮಿ ತೇ ನಾಥ ಪದಾರವಿಂದಂ\\ ಬುದ್ಧೀಂದ್ರಿಯಪ್ರಾಣಮನೋವಚೋಭಿಃ~।\\
ಯಚ್ಚಿಂತ್ಯತೇ ಭಾವಿತ ಆತ್ಮಯುಕ್ತೌ\\ ಮುಮುಕ್ಷಿಭಿಃ ಕರ್ಮಮಯೋಪಶಾಂತಯೇ~॥೨೦೮॥

ಅನೇನ ಯದ್ಭವೇತ್ಕಾರ್ಯಂ ತದ್ವದಾಮಿ ತವ ಪ್ರಿಯೇ~।\\
ಲೋಕೋಪಕಾರಕಂ ದೇವಿ ಲೌಕಿಕಂ ತು ವಿವರ್ಜಯೇತ್~॥೨೦೯॥

ಲೌಕಿಕಾದ್ಧರ್ಮತೋ ಯಾತಿ ಜ್ಞಾನಹೀನೋ ಭವಾರ್ಣವೇ~।\\
ಜ್ಞಾನಭಾವೇ ಚ ಯತ್ಸರ್ವಂ ಕರ್ಮ ನಿಷ್ಕರ್ಮ ಶಾಮ್ಯತಿ~॥೨೧೦॥

ಇಮಾಂ ತು ಭಕ್ತಿಭಾವೇನ ಪಠೇದ್ವೈ ಶ್ರುಣುಯಾದಪಿ~।\\
ಲಿಖಿತ್ವಾ ಯತ್ಪ್ರದಾನೇನ ತತ್ಸರ್ವಂ ಫಲಮಶ್ನುತೇ~॥೨೧೧॥

ಗುರುಗೀತಾಮಿಮಾಂ ದೇವಿ ಹೃದಿ ನಿತ್ಯಂ ವಿಭಾವಯ~।\\
ಮಹಾವ್ಯಾಧಿಗತೈರ್ದುಃಖೈಃ ಸರ್ವದಾ ಪ್ರಜಪೇನ್ಮುದಾ~॥೨೧೨॥

ಗುರುಗೀತಾಕ್ಷರೈಕೈಕಂ ಮಂತ್ರರಾಜಮಿದಂ ಪ್ರಿಯೇ~।\\
ಅನ್ಯೇ ಚ ವಿವಿಧಾ ಮಂತ್ರಾಃ ಕಲಾಂ ನಾರ್ಹನ್ತಿ ಷೋಡಶೀಂ~॥೨೧೩॥

ಅನಂತ ಫಲಮಾಪ್ನೋತಿ ಗುರುಗೀತಾ ಜಪೇನ ತು~।\\
ಸರ್ವಪಾಪಹರಾ ದೇವಿ ಸರ್ವದಾರಿದ್ರ್ಯನಾಶಿನೀ~॥೨೧೪॥

ಅಕಾಲಮೃತ್ಯುಹರ್ತ್ರೀ ಚ ಸರ್ವಸಂಕಟನಾಶಿನೀ~।\\
ಯಕ್ಷರಾಕ್ಷಸಭೂತಾದಿಚೋರವ್ಯಾಘ್ರವಿಘಾತಿನೀ~॥೨೧೫॥

ಸರ್ವೋಪದ್ರವಕುಷ್ಠಾದಿದುಷ್ಟದೋಷನಿವಾರಿಣೀ~।\\
ಯತ್ಫಲಂ ಗುರುಸಾನ್ನಿಧ್ಯಾತ್ತತ್ಫಲಂ ಪಠನಾದ್ಭವೇತ್~॥೨೧೬॥

ಮಹಾವ್ಯಾಧಿಹರಾ ಸರ್ವವಿಭೂತೇಃ ಸಿದ್ಧಿದಾ ಭವೇತ್~।\\
ಅಥವಾ ಮೋಹನೇ ವಶ್ಯೇ ಸ್ವಯಮೇವ ಜಪೇತ್ಸದಾ~॥೨೧೭॥

ಕುಶದೂರ್ವಾಸನೇ ದೇವಿ ಹ್ಯಾಸನೇ ಶುಭ್ರಕಂಬಲೇ~।\\
ಉಪವಿಶ್ಯ ತತೋ ದೇವಿ ಜಪೇದೇಕಾಗ್ರಮಾನಸಃ~॥೨೧೮॥

ಶುಕ್ಲಂ ಸರ್ವತ್ರ ವೈ ಪ್ರೋಕ್ತಂ ವಶ್ಯೇ ರಕ್ತಾಸನಂ ಪ್ರಿಯೇ~।\\
ಪದ್ಮಾಸನೇ ಜಪೇನ್ನಿತ್ಯಂ ಶಾಂತಿವಶ್ಯಕರಂ ಪರಂ~॥೨೧೯॥

ವಸ್ತ್ರಾಸನೇ ಚ ದಾರಿದ್ರ್ಯಂ ಪಾಷಾಣೇ ರೋಗಸಂಭವಃ~।\\
ಮೇದಿನ್ಯಾಂ ದುಃಖಮಾಪ್ನೋತಿ ಕಾಷ್ಠೇ ಭವತಿ ನಿಷ್ಫಲಂ~॥೨೨೦॥

ಕೃಷ್ಣಾಜಿನೇ ಜ್ಞಾನಸಿದ್ಧಿರ್ಮೋಕ್ಷಶ್ರೀರ್ವ್ಯಾಘ್ರಚರ್ಮಣಿ~।\\
ಕುಶಾಸನೇ ಜ್ಞಾನಸಿದ್ಧಿಃ ಸರ್ವಸಿದ್ಧಿಸ್ತು ಕಂಬಲೇ~॥೨೨೧॥

ಆಗ್ನೇಯ್ಯಾಂ ಕರ್ಷಣಂ ಚೈವ ವಾಯವ್ಯಾಂ ಶತ್ರುನಾಶನಂ~।\\
ನೈರ್ಋತ್ಯಾಂ ದರ್ಶನಂ ಚೈವ ಐಶಾನ್ಯಾಂ ಜ್ಞಾನಮೇವ ಚ~॥೨೨೨॥

ಉದಙ್ಮುಖಃ ಶಾಂತಿಜಪ್ಯೇ ವಶ್ಯೇ ಪೂರ್ವಮುಖಸ್ತಥಾ~।\\
ಯಾಮ್ಯೇ ತು ಮಾರಣಂ ಪ್ರೋಕ್ತಂ ಪಶ್ಚಿಮೇ ಚ ಧನಾಗಮಃ~॥೨೨೩॥

ಮೋಹನಂ ಸರ್ವಭೂತಾನಾಂ ಬಂಧಮೋಕ್ಷಕರಂ ಪರಂ~।\\
ದೇವರಾಜ್ಞಾಂ ಪ್ರಿಯಕರಂ ರಾಜಾನಂ ವಶಮಾನಯೇತ್~॥೨೨೪॥

ಮುಖಸ್ತಂಭಕರಂ ಚೈವ ಗುಣಾನಾಂ ಚ ವಿವರ್ಧನಂ~।\\
ದುಷ್ಕರ್ಮನಾಶನಂ ಚೈವ ತಥಾ ಸತ್ಕರ್ಮಸಿದ್ಧಿದಂ~॥೨೨೫॥

ಪ್ರಸಿದ್ಧಂ ಸಾಧಯೇತ್ಕಾರ್ಯಂ ನವಗ್ರಹಭಯಾಪಹಂ~।\\
ದುಃಸ್ವಪ್ನನಾಶನಂ ಚೈವ ಸುಸ್ವಪ್ನಫಲದಾಯಕಂ~॥೨೨೬॥

ಮೋಹಶಾಂತಿಕರಂ ಚೈವ ಬಂಧಮೋಕ್ಷಕರಂ ಪರಂ~।\\
ಸ್ವರೂಪಜ್ಞಾನನಿಲಯಂ ಗೀತಾಶಾಸ್ತ್ರಮಿದಂ ಶಿವೇ~॥೨೨೭॥

ಯಂ ಯಂ ಚಿಂತಯತೇ ಕಾಮಂ ತಂ ತಂ ಪ್ರಾಪ್ನೋತಿ ನಿಶ್ಚಯಃ~।\\
ನಿತ್ಯಂ ಸೌಭಾಗ್ಯದಂ ಪುಣ್ಯಂ ತಾಪತ್ರಯಕುಲಾಪಹಂ~॥೨೨೮॥

ಸರ್ವಶಾಂತಿಕರಂ ನಿತ್ಯಂ ತಥಾ ವಂಧ್ಯಾ ಸುಪುತ್ರದಂ~।\\
ಅವೈಧವ್ಯಕರಂ ಸ್ತ್ರೀಣಾಂ ಸೌಭಾಗ್ಯಸ್ಯ ವಿವರ್ಧನಂ~॥೨೨೯॥

ಆಯುರಾರೋಗ್ಯಮೈಶ್ವರ್ಯಂ ಪುತ್ರಪೌತ್ರಪ್ರವರ್ಧನಂ~।\\
ನಿಷ್ಕಾಮಜಾಪೀ ವಿಧವಾ ಪಠೇನ್ಮೋಕ್ಷಮವಾಪ್ನುಯಾತ್~॥೨೩೦॥

ಅವೈಧವ್ಯಂ ಸಕಾಮಾ ತು ಲಭತೇ ಚಾನ್ಯಜನ್ಮನಿ~।\\
ಸರ್ವದುಃಖಮಯಂ ವಿಘ್ನಂ ನಾಶಯೇತ್ತಾಪಹಾರಕಂ~॥೨೩೧॥

ಸರ್ವಪಾಪಪ್ರಶಮನಂ ಧರ್ಮಕಾಮಾರ್ಥಮೋಕ್ಷದಂ~।\\
ಯಂ ಯಂ ಚಿಂತಯತೇ ಕಾಮಂ ತಂ ತಂ ಪ್ರಾಪ್ನೋತಿ ನಿಶ್ಚಿತಂ~॥೨೩೨॥

ಕಾಮ್ಯಾನಾಂ ಕಾಮಧೇನುರ್ವೈ ಕಲ್ಪಿತೇ ಕಲ್ಪಪಾದಪಃ~।\\
ಚಿಂತಾಮಣಿಶ್ಚಿಂತಿತಸ್ಯ ಸರ್ವಮಂಗಲಕಾರಕಂ~॥೨೩೩॥

ಲಿಖಿತ್ವಾ ಪೂಜಯೇದ್ಯಸ್ತು ಮೋಕ್ಷಶ್ರಿಯಮವಾಪ್ನುಯಾತ್~।\\
ಗುರುಭಕ್ತಿರ್ವಿಶೇಷೇಣ ಜಾಯತೇ ಹೃದಿ ಸರ್ವದಾ~॥೨೩೪॥

ಜಪಂತಿ ಶಾಕ್ತಾಃ ಸೌರಾಶ್ಚ ಗಾಣಪತ್ಯಾಶ್ಚ ವೈಷ್ಣವಾಃ~।\\
ಶೈವಾಃ ಪಾಶುಪತಾಃ ಸರ್ವೇ ಸತ್ಯಂ ಸತ್ಯಂ ನ ಸಂಶಯಃ~॥೨೩೫॥

\authorline{॥ಇತಿ ಶ್ರೀಸ್ಕಂದಪುರಾಣೇ ಉತ್ತರಖಂಡೇ ಉಮಾಮಹೇಶ್ವರ ಸಂವಾದೇ ಶ್ರೀ ಗುರುಗೀತಾಯಾಂ ದ್ವಿತೀಯೋಽಧ್ಯಾಯಃ॥}

\section{॥ಅಥ ತೃತೀಯಃ ಅಧ್ಯಾಯಃ॥}

ಅಥ ಕಾಮ್ಯಜಪಸ್ಥಾನಂ ಕಥಯಾಮಿ ವರಾನನೇ~।\\
ಸಾಗರಾಂತೇ ಸರಿತ್ತೀರೇ ತೀರ್ಥೇ ಹರಿಹರಾಲಯೇ~॥೨೩೬॥

ಶಕ್ತಿದೇವಾಲಯೇ ಗೋಷ್ಠೇ ಸರ್ವದೇವಾಲಯೇ ಶುಭೇ~।\\
ವಟಸ್ಯ ಧಾತ್ರ್ಯಾ ಮೂಲೇ ವಾ ಮಠೇ ವೃಂದಾವನೇ ತಥಾ~॥೨೩೭॥

ಪವಿತ್ರೇ ನಿರ್ಮಲೇ ದೇಶೇ ನಿತ್ಯಾನುಷ್ಠಾನತೋಽಪಿ ವಾ~।\\
ನಿರ್ವೇದನೇನ ಮೌನೇನ ಜಪಮೇತತ್ ಸಮಾರಭೇತ್~॥೨೩೮॥

ಜಾಪ್ಯೇನ ಜಯಮಾಪ್ನೋತಿ ಜಪಸಿದ್ಧಿಂ ಫಲಂ ತಥಾ~।\\
ಹೀನಂ ಕರ್ಮ ತ್ಯಜೇತ್ಸರ್ವಂ ಗಹಿರ್ತಸ್ಥಾನಮೇವ ಚ~॥೨೩೯॥

ಶ್ಮಶಾನೇ ಬಿಲ್ವಮೂಲೇ ವಾ ವಟಮೂಲಾಂತಿಕೇ ತಥಾ~।\\
ಸಿದ್ಧ್ಯಂತಿ ಕಾನನೇ ಮೂಲೇ ಚೂತವೃಕ್ಷಸ್ಯ ಸನ್ನಿಧೌ~॥೨೪೦॥

ಪೀತಾಸನಂ ಮೋಹನೇ ತು ಹ್ಯಸಿತಂ ಚಾಭಿಚಾರಿಕೇ~।\\
ಜ್ಞೇಯಂ ಶುಕ್ಲಂ ಚ ಶಾಂತ್ಯರ್ಥಂ ವಶ್ಯೇ ರಕ್ತಂ ಪ್ರಕೀರ್ತಿತಂ~॥೨೪೧॥

ಜಪಂ ಹೀನಾಸನಂ ಕುರ್ವನ್ ಹೀನಕರ್ಮಫಲಪ್ರದಂ~।\\
ಗುರುಗೀತಾಂ ಪ್ರಯಾಣೇ ವಾ ಸಂಗ್ರಾಮೇ ರಿಪುಸಂಕಟೇ~॥೨೪೨॥

ಜಪನ್ ಜಯಮವಾಪ್ನೋತಿ ಮರಣೇ ಮುಕ್ತಿದಾಯಿಕಾ~।\\
ಸರ್ವಕರ್ಮಾಣಿ ಸಿದ್ಧ್ಯಂತಿ ಗುರುಪುತ್ರೇ ನ ಸಂಶಯಃ~॥೨೪೩॥

ಗುರುಮಂತ್ರೋ ಮುಖೇ ಯಸ್ಯ ತಸ್ಯ ಸಿದ್ಧ್ಯಂತಿ ನಾನ್ಯಥಾ~।\\
ದೀಕ್ಷಯಾ ಸರ್ವಕರ್ಮಾಣಿ ಸಿದ್ಧ್ಯಂತಿ ಗುರುಪುತ್ರಕೇ~॥೨೪೪॥

ಭವಮೂಲವಿನಾಶಾಯ ಚಾಷ್ಟಪಾಶನಿವೃತ್ತಯೇ~।\\
ಗುರುಗೀತಾಂಭಸಿ ಸ್ನಾನಂ ತತ್ತ್ವಜ್ಞಃ ಕುರುತೇ ಸದಾ~॥೨೪೫॥

ಸ ಏವಂ ಸದ್ಗುರುಃ ಸಾಕ್ಷಾತ್ ಸದಸದ್ಬ್ರಹ್ಮವಿತ್ತಮಃ~।\\
ತಸ್ಯ ಸ್ಥಾನಾನಿ ಸರ್ವಾಣಿ ಪವಿತ್ರಾಣಿ ನ ಸಂಶಯಃ~॥೨೪೬॥

ಸರ್ವಶುದ್ಧಃ ಪವಿತ್ರೋಽಸೌ ಸ್ವಭಾವಾದ್ಯತ್ರ ತಿಷ್ಠತಿ~।\\
ತತ್ರ ದೇವಗಣಾಃ ಸರ್ವೇ ಕ್ಷೇತ್ರಪೀಠೇ ಚರಂತಿ ಚ~॥೨೪೭॥

ಆಸನಸ್ಥಾಃ ಶಯಾನಾ ವಾ ಗಚ್ಛಂತಸ್ತಿಷ್ಠಂತೋಽಪಿ ವಾ~।\\
ಅಶ್ವಾರೂಢಾ ಗಜಾರೂಢಾಃ ಸುಷುಪ್ತಾ ಜಾಗ್ರತೋಽಪಿ ವಾ~॥೨೪೮॥

ಶುಚಿರ್ಭೂತಾ ಜ್ಞಾನವಂತೋ ಗುರುಗೀತಾಂ ಜಪಂತಿ ಯೇ~।\\
ತೇಷಾಂ ದರ್ಶನಸಂಸ್ಪರ್ಶಾತ್ ದಿವ್ಯಜ್ಞಾನಂ ಪ್ರಜಾಯತೇ~॥೨೪೯॥

ಸಮುದ್ರೇ ವೈ ಯಥಾ ತೋಯಂ ಕ್ಷೀರೇ ಕ್ಷೀರಂ ಜಲೇ ಜಲಂ~।\\
ಭಿನ್ನೇ ಕುಂಭೇ ಯಥಾಕಾಶಂ ತಥಾಽಽತ್ಮಾ ಪರಮಾತ್ಮನಿ~॥೨೫೦॥

ತಥೈವ ಜ್ಞಾನವಾನ್ ಜೀವಃ ಪರಮಾತ್ಮನಿ ಸರ್ವದಾ~।\\
ಐಕ್ಯೇನ ರಮತೇ ಜ್ಞಾನೀ ಯತ್ರ ಕುತ್ರ ದಿವಾನಿಶಂ~॥೨೫೧॥

ಏವಂವಿಧೋ ಮಹಾಯುಕ್ತಃ ಸರ್ವತ್ರ ವರ್ತತೇ ಸದಾ~।\\
ತಸ್ಮಾತ್ಸರ್ವಪ್ರಕಾರೇಣ ಗುರುಭಕ್ತಿಂ ಸಮಾಚರೇತ್~॥೨೫೨॥

ಗುರುಸಂತೋಷಣಾದೇವ ಮುಕ್ತೋ ಭವತಿ ಪಾರ್ವತಿ~।\\
ಅಣಿಮಾದಿಷು ಭೋಕ್ತೃತ್ವಂ ಕೃಪಯಾ ದೇವಿ ಜಾಯತೇ~॥೨೫೩॥

ಸಾಮ್ಯೇನ ರಮತೇ ಜ್ಞಾನೀ ದಿವಾ ವಾ ಯದಿ ವಾ ನಿಶಿ~।\\
ಏವಂವಿಧೋ ಮಹಾಮೌನೀ ತ್ರೈಲೋಕ್ಯಸಮತಾಂ ವ್ರಜೇತ್~॥೨೫೪॥

ಅಥ ಸಂಸಾರಿಣಃ ಸರ್ವೇ ಗುರುಗೀತಾಜಪೇನ ತು~।\\
ಸರ್ವಾನ್ ಕಾಮಾಂಸ್ತು ಭುಂಜಂತಿ ತ್ರಿಸತ್ಯಂ ಮಮ ಭಾಷಿತಂ~॥೨೫೫॥

ಸತ್ಯಂ ಸತ್ಯಂ ಪುನಃ ಸತ್ಯಂ ಧರ್ಮಸಾರಂ ಮಯೋದಿತಂ~।\\
ಗುರುಗೀತಾಸಮಂ ಸ್ತೋತ್ರಂ ನಾಸ್ತಿ ತತ್ತ್ವಂ ಗುರೋಃ ಪರಂ~॥೨೫೬॥

ಗುರುರ್ದೇವೋ ಗುರುರ್ಧರ್ಮೋ ಗುರೌ ನಿಷ್ಠಾ ಪರಂ ತಪಃ~।\\
ಗುರೋಃ ಪರತರಂ ನಾಸ್ತಿ ತ್ರಿವಾರಂ ಕಥಯಾಮಿ ತೇ~॥೨೫೭॥

ಧನ್ಯಾ ಮಾತಾ ಪಿತಾ ಧನ್ಯೋ ಗೋತ್ರಂ ಧನ್ಯಂ ಕುಲೋದ್ಭವಃ~।\\
ಧನ್ಯಾ ಚ ವಸುಧಾ ದೇವಿ ಯತ್ರ ಸ್ಯಾದ್ಗುರುಭಕ್ತತಾ~॥೨೫೮॥

ಆಕಲ್ಪಜನ್ಮ ಕೋಟೀನಾಂ ಯಜ್ಞವ್ರತತಪಃಕ್ರಿಯಾಃ~।\\
ತಾಃ ಸರ್ವಾಃ ಸಫಲಾ ದೇವಿ ಗುರಸಂತೋಷಮಾತ್ರತಃ~॥೨೫೯॥

ಶರೀರಮಿಂದ್ರಿಯಂ ಪ್ರಾಣಶ್ಚಾರ್ಥಃ ಸ್ವಜನಬಂಧುತಾ~।\\
ಮಾತೃಕುಲಂ ಪಿತೃಕುಲಂ ಗುರುರೇವ ನ ಸಂಶಯಃ~॥೨೬೦॥

ಮಂದಭಾಗ್ಯಾ ಹ್ಯಶಕ್ತಾಶ್ಚ ಯೇ ಜನಾ ನಾನುಮನ್ವತೇ~।\\
ಗುರುಸೇವಾಸು ವಿಮುಖಾಃ ಪಚ್ಯಂತೇ ನರಕೇಶುಚೌ~॥೨೬೧॥

ವಿದ್ಯಾ ಧನಂ ಬಲಂ ಚೈವ ತೇಷಾಂ ಭಾಗ್ಯಂ ನಿರರ್ಥಕಂ~।\\
ಯೇಷಾಂ ಗುರುಕೃಪಾ ನಾಸ್ತಿ ಅಧೋ ಗಚ್ಛಂತಿ ಪಾರ್ವತಿ~॥೨೬೨॥

ಬ್ರಹ್ಮಾ ವಿಷ್ಣುಶ್ಚ ರುದ್ರಶ್ಚ ದೇವತಾಃ ಪಿತೃಕಿನ್ನರಾಃ~।\\
ಸಿದ್ಧಚಾರಣಯಕ್ಷಾಶ್ಚ ಅನ್ಯೇ ಚ ಮುನಯೋ ಜನಾಃ~॥೨೬೩॥

ಗುರುಭಾವಃ ಪರಂ ತೀರ್ಥಮನ್ಯತ್ತೀರ್ಥಂ ನಿರರ್ಥಕಂ~।\\
ಸರ್ವತೀರ್ಥಮಯಂ ದೇವಿ ಶ್ರೀಗುರೋಶ್ಚರಣಾಂಬುಜಂ~॥೨೬೪॥

ಕನ್ಯಾಭೋಗರತಾ ಮಂದಾಃ ಸ್ವಕಾಂತಾಯಾಃ ಪರಾಙ್ಮುಖಾಃ~।\\
ಅತಃ ಪರಂ ಮಯಾ ದೇವಿ ಕಥಿತನ್ನ ಮಮ ಪ್ರಿಯೇ~॥೨೬೫॥

ಇದಂ ರಹಸ್ಯಮಸ್ಪಷ್ಟಂ ವಕ್ತವ್ಯಂ ಚ ವರಾನನೇ~।\\
ಸುಗೋಪ್ಯಂ ಚ ತವಾಗ್ರೇ ತು ಮಮಾತ್ಮಪ್ರೀತಯೇ ಸತಿ~॥೨೬೬॥

ಸ್ವಾಮಿಮುಖ್ಯಗಣೇಶಾದ್ಯಾನ್ ವೈಷ್ಣವಾದೀಂಶ್ಚ ಪಾರ್ವತಿ~।\\
ನ ವಕ್ತವ್ಯಂ ಮಹಾಮಾಯೇ ಪಾದಸ್ಪರ್ಶಂ ಕುರುಷ್ವ ಮೇ~॥೨೬೭॥

ಅಭಕ್ತೇ ವಂಚಕೇ ಧೂರ್ತೇ ಪಾಷಂಡೇ ನಾಸ್ತಿಕಾದಿಷು~।\\
ಮನಸಾಽಪಿ ನ ವಕ್ತವ್ಯಾ ಗುರುಗೀತಾ ಕದಾಚನ~॥೨೬೮॥

ಗುರವೋ ಬಹವಃ ಸಂತಿ ಶಿಷ್ಯವಿತ್ತಾಪಹಾರಕಾಃ~।\\
ತಮೇಕಂ ದುರ್ಲಭಂ ಮನ್ಯೇ ಶಿಷ್ಯಹೃತ್ತಾಪಹಾರಕಂ~॥೨೬೯॥

ಚಾತುರ್ಯವಾನ್ ವಿವೇಕೀ ಚ ಅಧ್ಯಾತ್ಮಜ್ಞಾನವಾನ್ ಶುಚಿಃ~।\\
ಮಾನಸಂ ನಿರ್ಮಲಂ ಯಸ್ಯ ಗುರುತ್ವಂ ತಸ್ಯ ಶೋಭತೇ~॥೨೭೦॥

ಗುರವೋ ನಿರ್ಮಲಾಃ ಶಾಂತಾಃ ಸಾಧವೋ ಮಿತಭಾಷಿಣಃ~।\\
ಕಾಮಕ್ರೋಧವಿನಿರ್ಮುಕ್ತಾಃ ಸದಾಚಾರಾಃ ಜಿತೇಂದ್ರಿಯಾಃ~॥೨೭೧॥

ಸೂಚಕಾದಿಪ್ರಭೇದೇನ ಗುರವೋ ಬಹುಧಾ ಸ್ಮೃತಾಃ~।\\
ಸ್ವಯಂ ಸಮ್ಯಕ್ ಪರೀಕ್ಷ್ಯಾಥ ತತ್ತ್ವನಿಷ್ಠಂ ಭಜೇತ್ಸುಧೀಃ~॥೨೭೨॥

ವರ್ಣಜಾಲಮಿದಂ ತದ್ವದ್ಬಾಹ್ಯಶಾಸ್ತ್ರಂ ತು ಲೌಕಿಕಂ~।\\
ಯಸ್ಮಿನ್ ದೇವಿ ಸಮಭ್ಯಸ್ತಂ ಸ ಗುರುಃ ಸೂಚಕಃ ಸ್ಮೃತಃ~॥೨೭೩॥

ವರ್ಣಾಶ್ರಮೋಚಿತಾಂ ವಿದ್ಯಾಂ ಧರ್ಮಾಧರ್ಮವಿಧಾಯಿನೀಂ~।\\
ಪ್ರವಕ್ತಾರಂ ಗುರುಂ ವಿದ್ಧಿ ವಾಚಕಂ ತ್ವಿತಿ ಪಾರ್ವತಿ~॥೨೭೪॥

ಪಂಚಾಕ್ಷರ್ಯಾದಿಮಂತ್ರಾಣಾಮುಪದೇಷ್ಟಾ ತು ಪಾರ್ವತಿ~।\\
ಸ ಗುರುರ್ಬೋಧಕೋ ಭೂಯಾದುಭಯೋರಯಮುತ್ತಮಃ~॥೨೭೫॥

ಮೋಹಮಾರಣವಶ್ಯಾದಿತುಚ್ಛಮಂತ್ರೋಪದರ್ಶಿನಂ~।\\
ನಿಷಿದ್ಧಗುರುರಿತ್ಯಾಹುಃ ಪಂಡಿತಾಸ್ತತ್ತ್ವದರ್ಶಿನಃ~॥೨೭೬॥

ಅನಿತ್ಯಮಿತಿ ನಿರ್ದಿಶ್ಯ ಸಂಸಾರಂ ಸಂಕಟಾಲಯಂ~।\\
ವೈರಾಗ್ಯಪಥದರ್ಶೀ ಯಃ ಸ ಗುರುರ್ವಿಹಿತಃ ಪ್ರಿಯೇ~॥೨೭೭॥

ತತ್ತ್ವಮಸ್ಯಾದಿವಾಕ್ಯಾನಾಮುಪದೇಷ್ಟಾ ತು ಪಾರ್ವತಿ~।\\
ಕಾರಣಾಖ್ಯೋ ಗುರುಃ ಪ್ರೋಕ್ತೋ ಭವರೋಗನಿವಾರಕಃ~॥೨೭೮॥

ಸರ್ವಸಂದೇಹಸಂದೋಹನಿರ್ಮೂಲನವಿಚಕ್ಷಣಃ~।\\
ಜನ್ಮಮೃತ್ಯುಭಯಘ್ನೋ ಯಃ ಸ ಗುರುಃ ಪರಮೋ ಮತಃ~॥೨೭೯॥

ಬಹುಜನ್ಮಕೃತಾತ್ ಪುಣ್ಯಾಲ್ಲಭ್ಯತೇಽಸೌ ಮಹಾಗುರುಃ~।\\
ಲಬ್ಧ್ವಾಽಮುಂ ನ ಪುನರ್ಯಾತಿ ಶಿಷ್ಯಃ ಸಂಸಾರಬಂಧನಂ~॥೨೮೦॥

ಏವಂ ಬಹುವಿಧಾ ಲೋಕೇ ಗುರವಃ ಸಂತಿ ಪಾರ್ವತಿ~।\\
ತೇಷು ಸರ್ವಪ್ರಯತ್ನೇನ ಸೇವ್ಯೋ ಹಿ ಪರಮೋ ಗುರುಃ~॥೨೮೧॥

ನಿಷಿದ್ಧಗುರುಶಿಷ್ಯಸ್ತು ದುಷ್ಟಸಂಕಲ್ಪದೂಷಿತಃ~।\\
ಬ್ರಹ್ಮಪ್ರಲಯಪರ್ಯಂತಂ ನ ಪುನರ್ಯಾತಿ ಮರ್ತ್ಯತಾಂ~॥೨೮೨॥

ಏವಂ ಶ್ರುತ್ವಾ ಮಹಾದೇವೀ ಮಹಾದೇವವಚಸ್ತಥಾ~।\\
ಅತ್ಯಂತವಿಹ್ವಲಮನಾ ಶಂಕರಂ ಪರಿಪೃಚ್ಛತಿ~॥೨೮೩॥

ಪಾರ್ವತ್ಯುವಾಚ~।\\
ನಮಸ್ತೇ ದೇವದೇವಾತ್ರ ಶ್ರೋತವ್ಯಂ ಕಿಂಚಿದಸ್ತಿ ಮೇ~।\\
ಶ್ರುತ್ವಾ ತ್ವದ್ವಾಕ್ಯಮಧುನಾ ಭೃಶಂ ಸ್ಯಾದ್ವಿಹ್ವಲಂ ಮನಃ~॥೨೮೪॥

ಸ್ವಯಂ ಮೂಢಾ ಮೃತ್ಯುಭೀತಾಃ ಸುಕೃತಾದ್ವಿರತಿಂ ಗತಾಃ~।\\
ದೈವಾನ್ನಿಷಿದ್ಧಗುರುಗಾ ಯದಿ ತೇಷಾಂ ತು ಕಾ ಗತಿಃ~॥೨೮೫॥

ಶ್ರೀ ಮಹಾದೇವ ಉವಾಚ~।\\
ಶ್ರುಣು ತತ್ತ್ವಮಿದಂ ದೇವಿ ಯದಾ ಸ್ಯಾದ್ವಿರತೋ ನರಃ~।\\
ತದಾಽಸಾವಧಿಕಾರೀತಿ ಪ್ರೋಚ್ಯತೇ ಶ್ರುತಿಮಸ್ತಕೈಃ~॥೨೮೬॥

ಅಖಂಡೈಕರಸಂ ಬ್ರಹ್ಮ ನಿತ್ಯಮುಕ್ತಂ ನಿರಾಮಯಂ~।\\
ಸ್ವಸ್ಮಿನ್ ಸಂದರ್ಶಿತಂ ಯೇನ ಸ ಭವೇದಸ್ಯ ದೇಶಿಕಃ~॥೨೮೭॥

ಜಲಾನಾಂ ಸಾಗರೋ ರಾಜಾ ಯಥಾ ಭವತಿ ಪಾರ್ವತಿ~।\\
ಗುರೂಣಾಂ ತತ್ರ ಸರ್ವೇಷಾಂ ರಾಜಾಯಂ ಪರಮೋ ಗುರುಃ~॥೨೮೮॥

ಮೋಹಾದಿರಹಿತಃ ಶಾಂತೋ ನಿತ್ಯತೃಪ್ತೋ ನಿರಾಶ್ರಯಃ~।\\
ತೃಣೀಕೃತಬ್ರಹ್ಮವಿಷ್ಣುವೈಭವಃ ಪರಮೋ ಗುರುಃ~॥೨೮೯॥

ಸರ್ವಕಾಲವಿದೇಶೇಷು ಸ್ವತಂತ್ರೋ ನಿಶ್ಚಲಸ್ಸುಖೀ~।\\
ಅಖಂಡೈಕರಸಾಸ್ವಾದತೃಪ್ತೋ ಹಿ ಪರಮೋ ಗುರುಃ~॥೨೯೦॥

ದ್ವೈತಾದ್ವೈತವಿನಿರ್ಮುಕ್ತಃ ಸ್ವಾನುಭೂತಿಪ್ರಕಾಶವಾನ್~।\\
ಅಜ್ಞಾನಾಂಧತಮಶ್ಛೇತ್ತಾ ಸರ್ವಜ್ಞಃ ಪರಮೋ ಗುರುಃ~॥೨೯೧॥

ಯಸ್ಯ ದರ್ಶನಮಾತ್ರೇಣ ಮನಸಃ ಸ್ಯಾತ್ ಪ್ರಸನ್ನತಾ~।\\
ಸ್ವಯಂ ಭೂಯಾತ್ ಧೃತಿಶ್ಶಾಂತಿಃ ಸ ಭವೇತ್ ಪರಮೋ ಗುರುಃ~॥೨೯೨॥

ಸಿದ್ಧಿಜಾಲಂ ಸಮಾಲೋಕ್ಯ ಯೋಗಿನಾಂ ಮಂತ್ರವಾದಿನಾಂ~।\\
ತುಚ್ಛಾಕಾರಮನೋವೃತ್ತಿರ್ಯಸ್ಯಾಸೌ ಪರಮೋ ಗುರುಃ~॥೨೯೩॥

ಸ್ವಶರೀರಂ ಶವಂ ಪಶ್ಯನ್ ತಥಾ ಸ್ವಾತ್ಮಾನಮದ್ವಯಂ~।\\
ಯಃ ಸ್ತ್ರೀಕನಕಮೋಹಘ್ನಃ ಸ ಭವೇತ್ ಪರಮೋ ಗುರುಃ~॥೨೯೪॥

ಮೌನೀ ವಾಗ್ಮೀತಿ ತತ್ತ್ವಜ್ಞೋ ದ್ವಿಧಾಭೂಚ್ಛೃಣು ಪಾರ್ವತಿ~।\\
ನ ಕಶ್ಚಿನ್ಮೌನಿನಾ ಲಾಭೋ ಲೋಕೇಽಸ್ಮಿನ್ಭವತಿ ಪ್ರಿಯೇ~॥೨೯೫॥

ವಾಗ್ಮೀ ತೂತ್ಕಟಸಂಸಾರಸಾಗರೋತ್ತಾರಣಕ್ಷಮಃ~।\\
ಯತೋಸೌ ಸಂಶಯಚ್ಛೇತ್ತಾ ಶಾಸ್ತ್ರಯುಕ್ತ್ಯನುಭೂತಿಭಿಃ~॥೨೯೬॥

ಗುರುನಾಮಜಪಾದ್ದೇವಿ ಬಹುಜನ್ಮಾರ್ಜಿತಾನ್ಯಪಿ~।\\
ಪಾಪಾನಿ ವಿಲಯಂ ಯಾಂತಿ ನಾಸ್ತಿ ಸಂದೇಹಮಣ್ವಪಿ~॥೨೯೭॥

ಶ್ರೀಗುರೋಸ್ಸದೃಶಂ ದೈವಂ ಶ್ರೀಗುರೋಃ ಸದೃಶಃ ಪಿತಾ~।\\
ಗುರುಧ್ಯಾನಸಮಂ ಕರ್ಮ ನಾಸ್ತಿ ನಾಸ್ತಿ ಮಹೀತಲೇ~॥೨೯೮॥

ಕುಲಂ ಧನಂ ಬಲಂ ಶಾಸ್ತ್ರಂ ಬಾಂಧವಾಸ್ಸೋದರಾ ಇಮೇ~।\\
ಮರಣೇ ನೋಪಯುಜ್ಯಂತೇ ಗುರುರೇಕೋ ಹಿ ತಾರಕಃ~॥೨೯೯॥

ಕುಲಮೇವ ಪವಿತ್ರಂ ಸ್ಯಾತ್ ಸತ್ಯಂ ಸ್ವಗುರುಸೇವಯಾ~।\\
ತೃಪ್ತಾಃ ಸ್ಯುಸ್ಸಕಲಾ ದೇವಾ ಬ್ರಹ್ಮಾದ್ಯಾ ಗುರುತರ್ಪಣಾತ್~॥೩೦೦॥

ಗುರುರೇಕೋ ಹಿ ಜಾನಾತಿ ಸ್ವರೂಪಂ ದೇವಮವ್ಯಯಂ~।\\
ತಜ‌್‌ಜ್ಞಾನಂ ಯತ್ಪ್ರಸಾದೇನ ನಾನ್ಯಥಾ ಶಾಸ್ತ್ರಕೋಟಿಭಿಃ~॥೩೦೧॥

ಸ್ವರೂಪಜ್ಞಾನಶೂನ್ಯೇನ ಕೃತಮಪ್ಯಕೃತಂ ಭವೇತ್~।\\
ತಪೋಜಪಾದಿಕಂ ದೇವಿ ಸಕಲಂ ಬಾಲಜಲ್ಪವತ್~॥೩೦೨॥

ಶಿವಂ ಕೇಚಿದ್ಧರಿಂ ಕೇಚಿದ್ವಿಧಿಂ ಕೇಚಿತ್ತು ಕಾಂಶ್ಚನ~।\\
ಶಕ್ತಿಂ ದೇವಮಿತಿ ಜ್ಞಾತ್ವಾ ವಿವದಂತಿ ವೃಥಾ ನರಾಃ~॥೩೦೩॥

ನ ಜಾನಂತಿ ಪರಂ ತತ್ತ್ವಂ ಗುರುದೀಕ್ಷಾಪರಾಙ್ಮುಖಾಃ~।\\
ಭ್ರಾಂತಾಃ ಪಶುಸಮಾ ಹ್ಯೇತೇ ಸ್ವಪರಿಜ್ಞಾನವರ್ಜಿತಾಃ~॥೩೦೪॥

ತಸ್ಮಾತ್ಕೈವಲ್ಯಸಿದ್ಧ್ಯರ್ಥಂ ಗುರುಮೇವ ಭಜೇತ್ಪ್ರಿಯೇ~।\\
ಗುರುಂ ವಿನಾ ನ ಜಾನಂತಿ ಮೂಢಾಸ್ತತ್ಪರಮಂ ಪದಂ~॥೩೦೫॥

ಭಿದ್ಯತೇ ಹೃದಯಗ್ರಂಥಿಶ್ಛಿದ್ಯಂತೇ ಸರ್ವಸಂಶಯಾಃ~।\\
ಕ್ಷೀಯಂತೇ ಸರ್ವಕರ್ಮಾಣಿ ಗುರೋಃ ಕರುಣಯಾ ಶಿವೇ~॥೩೦೬॥

ಕೃತಾಯಾ ಗುರುಭಕ್ತೇಸ್ತು ವೇದಶಾಸ್ತ್ರಾನುಸಾರತಃ~।\\
ಮುಚ್ಯತೇ ಪಾತಕಾದ್ಘೋರಾದ್ಗುರುಭಕ್ತೋ ವಿಶೇಷತಃ~॥೩೦೭॥

ದುಃಸಂಗಂ ಚ ಪರಿತ್ಯಜ್ಯ ಪಾಪಕರ್ಮ ಪರಿತ್ಯಜೇತ್~।\\
ಚಿತ್ತಚಿಹ್ನಮಿದಂ ಯಸ್ಯ ತಸ್ಯ ದೀಕ್ಷಾ ವಿಧೀಯತೇ~॥೩೦೮॥

ಚಿತ್ತತ್ಯಾಗನಿಯುಕ್ತಶ್ಚ ಕ್ರೋಧಗರ್ವವಿವರ್ಜಿತಃ~।\\
ದ್ವೈತಭಾವಪರಿತ್ಯಾಗೀ ತಸ್ಯ ದೀಕ್ಷಾ ವಿಧೀಯತೇ~॥೩೦೯॥

ಏತಲ್ಲಕ್ಷಣ ಸಂಯುಕ್ತಂ ಸರ್ವಭೂತಹಿತೇ ರತಂ~।\\
ನಿರ್ಮಲಂ ಜೀವಿತಂ ಯಸ್ಯ ತಸ್ಯ ದೀಕ್ಷಾ ವಿಧೀಯತೇ~॥೩೧೦॥

ಕ್ರಿಯಯಾ ಚಾನ್ವಿತಂ ಪೂರ್ವಂ ದೀಕ್ಷಾಜಾಲಂ ನಿರೂಪಿತಂ~।\\
ಮಂತ್ರದೀಕ್ಷಾಭಿಧಂ ಸರ್ವಂ  ಸಾಂಗೋಪಾಂಗಂ ಶಿವೋದಿತಂ~॥೩೧೧॥

ಕ್ರಿಯಯಾ ಸ್ಯಾದ್ವಿರಹಿತಾಂ ಗುರುಸಾಯುಜ್ಯದಾಯಿನೀಂ~।\\
ಗುರುದೀಕ್ಷಾಂ ವಿನಾ ಕೋ ವಾ ಗುರುತ್ವಾಚಾರಪಾಲಕಃ~॥೩೧೨॥

ಶಕ್ತೋ ನ ಚಾಪಿ ಶಕ್ತೋ ವಾ ದೈಶಿಕಾಂಘ್ರಿಸಮಾಶ್ರಯಾತ್~।\\
ತಸ್ಯ ಜನ್ಮಾಸ್ತಿ ಸಫಲಂ ಭೋಗಮೋಕ್ಷಫಲಪ್ರದಂ~॥೩೧೩॥

ಅತ್ಯಂತಚಿತ್ತಪಕ್ವಸ್ಯ ಶ್ರದ್ಧಾಭಕ್ತಿಯುತಸ್ಯ ಚ~।\\
ಪ್ರವಕ್ತವ್ಯಮಿದಂ ದೇವಿ ಮಮಾತ್ಮಪ್ರೀತಯೇ ಸದಾ~॥೩೧೪॥

ರಹಸ್ಯಂ ಸರ್ವಶಾಸ್ತ್ರೇಷು ಗೀತಾಶಾಸ್ತ್ರಮಿದಂ ಶಿವೇ~।\\
ಸಮ್ಯಕ್ಪರೀಕ್ಷ್ಯ ವಕ್ತವ್ಯಂ ಸಾಧಕಸ್ಯ ಮಹಾತ್ಮನಃ~॥೩೧೫॥

ಸತ್ಕರ್ಮಪರಿಪಾಕಾಚ್ಚ ಚಿತ್ತಶುದ್ಧಸ್ಯ ಧೀಮತಃ~।\\
ಸಾಧಕಸ್ಯೈವ ವಕ್ತವ್ಯಾ ಗುರುಗೀತಾ ಪ್ರಯತ್ನತಃ~॥೩೧೬॥

ನಾಸ್ತಿಕಾಯ ಕೃತಘ್ನಾಯ ದಾಂಭಿಕಾಯ ಶಠಾಯ ಚ~।\\
ಅಭಕ್ತಾಯ ವಿಭಕ್ತಾಯ ನ ವಾಚ್ಯೇಯಂ ಕದಾಚನ~॥೩೧೭॥

ಸ್ತ್ರೀಲೋಲುಪಾಯ ಮೂರ್ಖಾಯ ಕಾಮೋಪಹತಚೇತಸೇ~।\\
ನಿಂದಕಾಯ ನ ವಕ್ತವ್ಯಾ ಗುರುಗೀತಾ ಸ್ವಭಾವತಃ~॥೩೧೮॥

ಸರ್ವ ಪಾಪಪ್ರಶಮನಂ ಸರ್ವೋಪದ್ರವವಾರಕಂ~।\\
ಜನ್ಮಮೃತ್ಯುಹರಂ ದೇವಿ ಗೀತಾಶಾಸ್ತ್ರಮಿದಂ ಶಿವೇ~॥೩೧೯॥

ಶ್ರುತಿಸಾರಮಿದಂ ದೇವಿ ಸರ್ವಮುಕ್ತಂ ಸಮಾಸತಃ~।\\
ನಾನ್ಯಥಾ ಸದ್ಗತಿಃ ಪುಂಸಾಂ ವಿನಾ ಗುರುಪದಂ ಶಿವೇ~॥೩೨೦॥

ಬಹುಜನ್ಮಕೃತಾತ್ಪಾಪಾದಯಮರ್ಥೋ ನ ರೋಚತೇ~।\\
ಜನ್ಮಬಂಧನಿವೃತ್ಯರ್ಥಂ ಗುರುಮೇವ ಭಜೇತ್ಸದಾ~॥೩೨೧॥

ಅಹಮೇವ ಜಗತ್ಸರ್ವಮಹಮೇವ ಪರಂ ಪದಂ~।\\
ಏತಜ್‌ಜ್ಞಾನಂ ಯತೋ ಭೂಯಾತ್ತಂ ಗುರುಂ ಪ್ರಣಮಾಮ್ಯಹಂ~॥೩೨೨॥

ಅಲಂ ವಿಕಲ್ಪೈರಹಮೇವ ಕೇವಲೋ\\ ಮಯಿ ಸ್ಥಿತಂ ವಿಶ್ವಮಿದಂ ಚರಾಚರಂ~।\\
ಇದಂ ರಹಸ್ಯಂ ಮಮ ಯೇನ ದರ್ಶಿತಂ\\ ಸ ವಂದನೀಯೋ ಗುರುರೇವ ಕೇವಲಂ~॥೩೨೩॥

ಯಸ್ಯಾಂತಂ ನಾದಿಮಧ್ಯಂ ನ ಹಿ ಕರಚರಣಂ \\ನಾಮಗೋತ್ರಂ ನ ಸೂತ್ರಂ~।\\
ನೋ ಜಾತಿರ್ನೈವ ವರ್ಣೋ ನ ಭವತಿ ಪುರುಷೋ\\ ನೋ ನಪುಂಸಂ ನ ಚ ಸ್ತ್ರೀ~॥೩೨೪॥

ನಾಕಾರಂ ನೋ ವಿಕಾರಂ ನ ಹಿ ಜನಿಮರಣಂ \\ನಾಸ್ತಿ ಪುಣ್ಯಂ ನ ಪಾಪಂ~।\\
ನೋಽತತ್ತ್ವಂ ತತ್ತ್ವಮೇಕಂ ಸಹಜಸಮರಸಂ \\ಸದ್ಗುರುಂ ತಂ ನಮಾಮಿ~॥೩೨೫॥

ನಿತ್ಯಾಯ ಸತ್ಯಾಯ ಚಿದಾತ್ಮಕಾಯ \\ನವ್ಯಾಯ ಭವ್ಯಾಯ ಪರಾತ್ಪರಾಯ~।\\
ಶುದ್ಧಾಯ ಬುದ್ಧಾಯ ನಿರಂಜನಾಯ\\ ನಮೋಽಸ್ಯ ನಿತ್ಯಂ ಗುರುಶೇಖರಾಯ~॥೩೨೬॥

ಸಚ್ಚಿದಾನಂದರೂಪಾಯ ವ್ಯಾಪಿನೇ ಪರಮಾತ್ಮನೇ~।\\
ನಮಃ ಶ್ರೀಗುರುನಾಥಾಯ ಪ್ರಕಾಶಾನಂದಮೂರ್ತಯೇ~॥೩೨೭॥

ಸತ್ಯಾನಂದಸ್ವರೂಪಾಯ ಬೋಧೈಕಸುಖಕಾರಿಣೇ~।\\
ನಮೋ ವೇದಾಂತವೇದ್ಯಾಯ ಗುರವೇ ಬುದ್ಧಿಸಾಕ್ಷಿಣೇ~॥೩೨೮॥

ನಮಸ್ತೇ ನಾಥ ಭಗವನ್ ಶಿವಾಯ ಗುರುರೂಪಿಣೇ~।\\
ವಿದ್ಯಾವತಾರಸಂಸಿದ್ಧ್ಯೈ ಸ್ವೀಕೃತಾನೇಕವಿಗ್ರಹ~॥೩೨೯॥

ನವಾಯ ನವರೂಪಾಯ ಪರಮಾರ್ಥೈಕರೂಪಿಣೇ~।\\
ಸರ್ವಾಜ್ಞಾನತಮೋಭೇದಭಾನವೇ ಚಿದ್ಘನಾಯ ತೇ~॥೩೩೦॥

ಸ್ವತಂತ್ರಾಯ ದಯಾಕ್ಲೃಪ್ತವಿಗ್ರಹಾಯ ಶಿವಾತ್ಮನೇ~।\\
ಪರತಂತ್ರಾಯ ಭಕ್ತಾನಾಂ ಭವ್ಯಾನಾಂ ಭವ್ಯರೂಪಿಣೇ~॥೩೩೧॥

ವಿವೇಕಿನಾಂ ವಿವೇಕಾಯ ವಿಮರ್ಶಾಯ ವಿಮರ್ಶಿನಾಂ~।\\
ಪ್ರಕಾಶಿನಾಂ ಪ್ರಕಾಶಾಯ ಜ್ಞಾನಿನಾಂ ಜ್ಞಾನರೂಪಿಣೇ~॥೩೩೨॥

ಪುರಸ್ತತ್ಪಾರ್ಶ್ವಯೋಃ ಪೃಷ್ಠೇ ನಮಸ್ಕುರ್ಯಾದುಪರ್ಯಧಃ~।\\
ಸದಾ ಮಚ್ಚಿತ್ತರೂಪೇಣ ವಿಧೇಹಿ ಭವದಾಸನಂ~॥೩೩೩॥

ಶ್ರೀಗುರುಂ ಪರಮಾನಂದಂ ವಂದೇ ಹ್ಯಾನಂದವಿಗ್ರಹಂ~।\\
ಯಸ್ಯ ಸನ್ನಿಧಿಮಾತ್ರೇಣ ಚಿದಾನಂದಾಯತೇ ನಮಃ~॥೩೩೪॥

ನಮೋಽಸ್ತು ಗುರವೇ ತುಭ್ಯಂ ಸಹಜಾನಂದರೂಪಿಣೇ~।\\
ಯಸ್ಯ ವಾಗಮೃತಂ ಹಂತಿ ವಿಷಂ ಸಂಸಾರಸಂಜ್ಞಕಂ~॥೩೩೫॥

ನಾನಾಯುಕ್ತೋಪದೇಶೇನ ತಾರಿತಾ ಶಿಷ್ಯಸಂತತಿಃ~।\\
ತತ್ಕೃಪಾಸಾರವೇದೇನ ಗುರುಚಿತ್ಪದಮಚ್ಯುತಂ~॥೩೩೬॥

ಅಚ್ಯುತಾಯ ನಮಸ್ತುಭ್ಯಂ ಗುರವೇ ಪರಮಾತ್ಮನೇ~।\\
ಸರ್ವತಂತ್ರಸ್ವತಂತ್ರಾಯ ಚಿದ್ಘನಾನಂದಮೂರ್ತಯೇ~॥೩೩೭॥

ನಮೋಚ್ಯುತಾಯ ಗುರವೇ ವಿದ್ಯಾವಿದ್ಯಾಸ್ವರೂಪಿಣೇ~।\\
ಶಿಷ್ಯಸನ್ಮಾರ್ಗಪಟವೇ ಕೃಪಾಪೀಯೂಷಸಿಂಧವೇ~॥೩೩೮॥

ಓಮಚ್ಯುತಾಯ ಗುರವೇ ಶಿಷ್ಯಸಂಸಾರಸೇತವೇ~।\\
ಭಕ್ತಕಾರ್ಯೈಕಸಿಂಹಾಯ ನಮಸ್ತೇ ಚಿತ್ಸುಖಾತ್ಮನೇ~॥೩೩೯॥

ಗುರುನಾಮಸಮಂ ದೈವಂ ನ ಪಿತಾ ನ ಚ ಬಾಂಧವಾಃ~।\\
ಗುರುನಾಮಸಮಃ ಸ್ವಾಮೀ ನೇದೃಶಂ ಪರಮಂ ಪದಂ~॥೩೪೦॥

ಏಕಾಕ್ಷರಪ್ರದಾತಾರಂ ಯೋ ಗುರುಂ ನೈವ ಮನ್ಯತೇ~।\\
ಶ್ವಾನಯೋನಿಶತಂ ಗತ್ವಾ ಚಾಂಡಾಲೇಷ್ವಪಿ ಜಾಯತೇ~॥೩೪೧॥

ಗುರುತ್ಯಾಗಾದ್ಭವೇನ್ಮೃತ್ಯುರ್ಮಂತ್ರತ್ಯಾಗಾದ್ದರಿದ್ರತಾ~।\\
ಗುರುಮಂತ್ರಪರಿತ್ಯಾಗೀ ರೌರವಂ ನರಕಂ ವ್ರಜೇತ್~॥೩೪೨॥

ಶಿವಕ್ರೋಧಾದ್ಗುರುಸ್ತ್ರಾತಾ ಗುರುಕ್ರೋಧಾಚ್ಛಿವೋ ನ ಹಿ~।\\
ತಸ್ಮಾತ್ಸರ್ವಪ್ರಯತ್ನೇನ ಗುರೋರಾಜ್ಞಾಂ ನ ಲಂಘಯೇತ್~॥೩೪೩॥

ಸಂಸಾರಸಾಗರಸಮುದ್ಧರಣೈಕಮಂತ್ರಂ \\ಬ್ರಹ್ಮಾದಿದೇವಮುನಿಪೂಜಿತಸಿದ್ಧಮಂತ್ರಂ~।\\
ದಾರಿದ್ರ್ಯದುಃಖಭವರೋಗವಿನಾಶಮಂತ್ರಂ \\ವಂದೇ ಮಹಾಭಯಹರಂ ಗುರುರಾಜಮಂತ್ರಂ~॥೩೪೪॥

ಸಪ್ತಕೋಟಿಮಹಾಮಂತ್ರಾಶ್ಚಿತ್ತವಿಭ್ರಂಶಕಾರಕಾಃ~।\\
ಏಕ ಏವ ಮಹಾಮಂತ್ರೋ ಗುರುರಿತ್ಯಕ್ಷರದ್ವಯಂ~॥೩೪೫॥

ಏವಮುಕ್ತ್ವಾ ಮಹಾದೇವಃ ಪಾರ್ವತೀಂ ಪುನರಬ್ರವೀತ್~।\\
ಇದಮೇವ ಪರಂ ತತ್ತ್ವಂ ಶೃಣು ದೇವಿ ಸುಖಾವಹಂ~॥೩೪೬॥

ಗುರುತತ್ತ್ವಮಿದಂ ದೇವಿ ಸರ್ವಮುಕ್ತಂ ಸಮಾಸತಃ~।\\
ರಹಸ್ಯಮಿದಮವ್ಯಕ್ತನ್ನ ವದೇದ್ಯಸ್ಯ ಕಸ್ಯಚಿತ್~॥೩೪೭॥

ನ ಮೃಷಾ ಸ್ಯಾದಿಯಂ ದೇವಿ ಮದುಕ್ತಿಃ ಸತ್ಯರೂಪಿಣೀ~।\\
ಗುರುಗೀತಾಸಮಂ ಸ್ತೋತ್ರಂ ನಾಸ್ತಿ ನಾಸ್ತಿ ಮಹೀತಲೇ~॥೩೪೮॥

ಗುರುಗೀತಾಮಿಮಾಂ ದೇವಿ ಭವದುಃಖವಿನಾಶಿನೀಂ~।\\
ಗುರುದೀಕ್ಷಾವಿಹೀನಸ್ಯ ಪುರತೋ ನ ಪಠೇತ್ ಕ್ವಚಿತ್~॥೩೪೯॥

ರಹಸ್ಯಮತ್ಯಂತರಹಸ್ಯಮೇತತ್ \\ನ ಪಾಪಿನಾ ಲಭ್ಯಮಿದಂ ಮಹೇಶ್ವರಿ~।\\
ಅನೇಕಜನ್ಮಾರ್ಜಿತಪುಣ್ಯಪಾಕಾತ್ \\ಗುರೋಸ್ತು ತತ್ತ್ವಂ ಲಭತೇ ಮನುಷ್ಯಃ~॥೩೫೦॥

ಯಸ್ಯ ಪ್ರಸಾದಾದಹಮೇವ ಸರ್ವಂ\\ ಮಯ್ಯೇವ ಸರ್ವಂ ಪರಿಕಲ್ಪಿತಂ ಚ~।\\
ಇತ್ಥಂ ವಿಜಾನಾಮಿ ಸದಾತ್ಮರೂಪಂ \\ತಸ್ಯಾಂಘ್ರಿಪದ್ಮಂ ಪ್ರಣತೋಽಸ್ಮಿ ನಿತ್ಯಂ~॥೩೫೧॥

ಅಜ್ಞಾನತಿಮಿರಾಂಧಸ್ಯ ವಿಷಯಾಕ್ರಾಂತಚೇತಸಃ~।\\
ಜ್ಞಾನಪ್ರಭಾಪ್ರದಾನೇನ ಪ್ರಸಾದಂ ಕುರು ಮೇ ಪ್ರಭೋ~॥೩೫೨॥

\authorline{॥ಇತಿ ಶ್ರೀಗುರುಗೀತಾಯಾಂ ತೃತೀಯೋಽಧ್ಯಾಯಃ॥\\
॥ಇತಿ ಶ್ರೀಸ್ಕಂದಪುರಾಣೇ ಉತ್ತರಖಂಡೇ ಈಶ್ವರಪಾರ್ವತೀ ಸಂವಾದೇ ಗುರುಗೀತಾ ಸಮಾಪ್ತಾ॥\\
{\bfseries ॥ಶ್ರೀಗುರುದತ್ತಾತ್ರೇಯಾರ್ಪಣಮಸ್ತು॥}}
