\chapter*{\center ದುರ್ಗಾ ಕಲ್ಪಃ \\ \Large ಶ್ರೀಗುರುಭ್ಯೋ ನಮಃ\\
ಶ್ರೀ ಗಣೇಶಾಯ ನಮಃ}

ಆಚಮ್ಯ ಪವಿತ್ರಂ ಧೃತ್ವಾ ಗುರುಗಣಪತಿ ಸ್ಮರಣಪೂರ್ವಕಂ ಘಂಟಾನಾದಂ ಕೃತ್ವಾ , ಪ್ರಾಣಾನಾಯಮ್ಯ, ದೇಶಕಾಲೌ ಸಂಕೀರ್ತ್ಯ ನಕ್ಷತ್ರ ರಾಶ್ಯಾದ್ಯುಕ್ತ್ವಾ ಸ್ಕಂದ ಪುರಾಣೋಕ್ತಪ್ರಕಾರೇಣ  ಆಶ್ವಯುಜ ಶುಕ್ಲ ಪ್ರತಿಪದಾದಿ ನವಮೀಪರ್ಯಂತ ಕರ್ತವ್ಯ ಶ್ರೀದುರ್ಗಾವ್ರತ ಕಲ್ಪೋಕ್ತಾನುಷ್ಠಾನ ಸಿದ್ಧ ಸಾಧ್ಯ ಆಯುರಾರೋಗ್ಯೈಶ್ವರ್ಯಧನ ಧಾನ್ಯ ವಿಜಯ ಪ್ರಾಪ್ತ್ಯರ್ಥಂ ಪುತ್ರಪೌತ್ರಾದಿ ವೃದ್ಧಿ ವಸ್ತ್ರಮಾಲ್ಯಾನುಲೇಪನಾದಿ ನಾನಾ ಸೌಖ್ಯಪ್ರಾಪ್ತ್ಯರ್ಥಂ ಸಕಲ ಸೌಭಾಗ್ಯ ಪ್ರಾಪ್ತ್ಯರ್ಥಂ ಅಕಂಟಕ ರಾಜ್ಯ ಪ್ರಾಪ್ತ್ಯರ್ಥಂ ಸಕಲ ದುಃಖ ನಿವೃತ್ಯರ್ಥಂ ಸಕಲ ಸೌಭಾಗ್ಯ ಪ್ರಾಪ್ತ್ಯರ್ಥಂ ರಾಜಾಮಾತ್ಯಾದಿ ಸರ್ವಜನ ವಶೀಕರಣ ಸಿದ್ಧ್ಯರ್ಥಂ ಸಕಲ ಶತ್ರುಕ್ಷಯಾರ್ಥಂ ಅಶ್ವಪಶ್ವಾದಿ ಕೋಶಾಭಿವೃದ್ಧ್ಯರ್ಥಂ ಧನಕನಕವಸ್ತುವಾಹನ ಸಮೃದ್ಧ್ಯರ್ಥಂ ಸಕಲಮನೋರಥಾವಾಪ್ತ್ಯರ್ಥಂ ಭೀಕರಭೂತಪ್ರೇತಪಿಶಾಚ ಶಾಕಿನೀ ಡಾಕಿನೀ ಬ್ರಹ್ಮರಾಕ್ಷಸ ವೇತಾಳ ಗ್ರಹನಿವೃತ್ತಿ ಕ್ಷಾಮರಡಾಮರ ಮಹೋತ್ಪಾತಾಗ್ನಿ ಚೋರವ್ಯಾಘ್ರ ರಾಕ್ಷಸ ಸಮುದ್ಭವ ಸಮಸ್ತದುಶ್ಚರಿತ ನಿವಾರಣಾರ್ಥಂ ಬ್ರಹ್ಮ ವಿಷ್ಣುಮಹೇಶ್ವರ ಇಂದ್ರಾದಿ ಲೋಕಪಾಲ ನವಗ್ರಹ ಅಷ್ಟ ವಸು ಏಕಾದಶರುದ್ರ ದ್ವಾದಶಾದಿತ್ಯ ಪಿತೃಮುಖಾದಿ ಸಮಸ್ತದೇವತಾ ಪೂಜಾದ್ವಾರಾ ಮಹಾಕಾಳೀ ಮಹಾಲಕ್ಷ್ಮೀ ಮಹಾಸರಸ್ವತೀ ಪ್ರೀತ್ಯರ್ಥಂ ಮಹಾಕಾಳೀ ಮಹಾಲಕ್ಷ್ಮೀ ಮಹಾಸರಸ್ವತೀ ಮುದ್ದಿಶ್ಯ ಮಹಾನವಮೀ ವ್ರತಕಲ್ಪೋಕ್ತ ಧ್ಯಾನಾವಾಹನಾದಿ ಷೋಡಶೋಪಚಾರಪೂಜಾಂ ಕರಿಷ್ಯೇ ॥

ತದಂಗ ಗಣಪತಿ ಕಲಶ ಶಂಖಪೂಜಾಃ ಕೃತ್ವಾ (ಉಪದೇಶೇ ಸತಿ ನವಾರ್ಣಜಪಂ ಕೃತ್ವಾ) ಆತ್ಮಾರ್ಚನಂ ವಿಧಾಯ ಮಂಟಪಂ ಧ್ಯಾಯೇತ್ ।ಕಲಶಂ  ಪ್ರತಿಷ್ಠಾಪ್ಯ ॥\\
(ದೇವತಾಸ್ಥಾನಮಾಗತ್ಯ ಮಂಟಪಾರ್ಚನಪೂರ್ವಕಂ-)\\
\section{ಮಂಟಪಾರ್ಚನಂ}
ಮಹಾಮಾಣಿಕ್ಯ ವೈಡೂರ್ಯ ಕಾಂಚನಸ್ತಂಭ ಪೂರ್ವಕಂ ।\\
ಮುಕ್ತಾದಾಮ ಸಮಾಕೀರ್ಣಂ ರತ್ನವೈಡೂರ್ಯ ಮಂಟಪಮ್ ॥

ಮಂದವಾಯುಗತಿಕ್ರಾಂತಂ ಗಂಧಪುಷ್ಪ ಮನೋಹರಮ್ ।\\
ರತ್ನಚಾಮರಘಂಟಾದಿ ವಿತಾನೈರುಪಶೋಭಿತಮ್ ॥

ಜಾತೀ ಚಂಪಕ ಕಹ್ಲಾರೈಃ ಕೇತಕೀ ಮಲ್ಲಿಕಾದಿಭಿಃ ।\\
ಬದ್ಧಾಭಿಶ್ಚಿತ್ರಮಾಲಾಭಿಃ ಸುವೃತಾಭಿರಲಂಕೃತಮ್ ।\\
ಚತುರ್ದ್ವಾರ ಸಮಾಯುಕ್ತಂ ಧ್ಯಾಯೇತ್ ಸೌಭಾಗ್ಯ ಮಂಟಪಮ್॥

ಚಿತ್ರಮಂಟಪಾಯ ನಮಃ\\
ಯೋಗಮಂಟಪಾಯ ನಮಃ\\
ಭೋಗಮಂಟಪಾಯ ನಮಃ\\
ರತ್ನಮಂಟಪಾಯ ನಮಃ\\
ಪ್ರವಾಲಮಂಟಪಾಯ ನಮಃ\\
ಮೌಕ್ತಿಕಮಂಟಪಾಯ ನಮಃ\\
ಪುಷ್ಪಮಂಟಪಾಯ ನಮಃ\\
ನವರತ್ನಖಚಿತಸೌಭಾಗ್ಯಮಂಟಪಾಯ ನಮಃ\\
ಮೋಕ್ಷಲಕ್ಷ್ಮೀವಿಲಾಸಮಂಟಪಾಯ ನಮಃ\\
ತನ್ಮಧ್ಯೇ ಶ್ರೀಮಹಾಕಾಳೀ ಮಹಾಲಕ್ಷ್ಮೀ ಮಹಾಸರಸ್ವತೀ ಸ್ವರೂಪಿಣೀ ದುರ್ಗಾಪರಮೇಶ್ವರ್ಯೈ ನಮಃ ॥\\
ನವರತ್ನಖಚಿತಸೌಭಾಗ್ಯ ಮಂಟಪಂ ಕಲ್ಪಯಾಮಿ॥
\section{ದ್ವಾರಪೂಜಾ}
\as{ಓಂ ಪೂರ್ವ}ದ್ವಾರೇ ಶ್ರಿಯೈ  ನಮಃ ।\\
ಭದ್ರಾಯೈ ನಮಃ ।
ಸುಭದ್ರಾಯೈ ನಮಃ ।\\
\as{ಓಂ ದಕ್ಷಿಣ}ದ್ವಾರೇ ಶ್ರಿಯೈ  ನಮಃ ।\\
ಬಲಾಯ ನಮಃ ।
ಪ್ರಬಲಾಯ ನಮಃ ।\\
\as{ಓಂ ಪಶ್ಚಿಮ}ದ್ವಾರೇ ಶ್ರಿಯೈ  ನಮಃ ।\\
ಜಯಾ ನಮಃ ।
ವಿಜಯಾಯೈ ನಮಃ ।\\
\as{ಓಂ ಉತ್ತರ}ದ್ವಾರೇ ಶ್ರಿಯೈ  ನಮಃ ।\\
ಚಂಡಾಯೈ ನಮಃ ।
ಪ್ರಚಂಡಾಯೈ ನಮಃ ।\\ದ್ವಾರಪಾಲಪೂಜಾಂ ಸಮರ್ಪಯಾಮಿ॥

ಪೀಠಪೂಜಾಂ ಕರಿಷ್ಯೇ\\
 ಓಂ ಆಧಾರಶಕ್ತ್ಯೈ ನಮಃ । ಓಂ ಮೂಲಪ್ರಕೃತ್ಯೈ ನಮಃ । ಓಂ ಆದಿಕೂರ್ಮಾಯ ನಮಃ । ಓಂ ಅನಂತಾಯ ನಮಃ । ಓಂ ಪೃಥಿವ್ಯೈ ನಮಃ । ಓಂ ಕ್ಷೀರಸಮುದ್ರಾಯ ನಮಃ । ಓಂ ಶ್ವೇತದ್ವೀಪಾಯ ನಮಃ । ಓಂ ರತ್ನಮಂಡಪಾಯ ನಮಃ । ಓಂ ಕಲ್ಪವೃಕ್ಷಾಯ ನಮಃ । ಓಂ ಶ್ವೇತಚ್ಛತ್ರಾಯ ನಮಃ । ಓಂ ಸಿತಚಾಮರಾಭ್ಯಾಂ ನಮಃ । ಓಂ ರತ್ನ ಸಿಂಹಾಸನಾಯ ನಮಃ । ಓಂ ಧರ್ಮಾಯ ನಮಃ । ಓಂ ಜ್ಞಾನಾಯ ನಮಃ । ಓಂ ವೈರಾಗ್ಯಾಯ ನಮಃ । ಓಂ ಐಶ್ವರ್ಯಾಯ ನಮಃ । ಓಂ ಅಧರ್ಮಾಯ ನಮಃ । ಓಂ ಅಜ್ಞಾನಾಯ ನಮಃ । ಓಂ ಅವೈರಾಗ್ಯಾಯ ನಮಃ ।ಓಂ ಆನೈಶ್ವರ್ಯಾಯ ನಮಃ । ಓಂ ಸಂ ಸತ್ವಾಯ ನಮಃ । ಓಂ ರಂ ರಜಸೇ ನಮಃ । ಓಂ ತಂ ತಮಸೇ ನಮಃ । ಓಂ ಮಂ ಮಾಯಾಯೈ ನಮಃ । ವಿಂ ವಿದ್ಯಾಯೈ ನಮಃ । ಓಂ ಅಂ ಅನಂತಾಯ ನಮಃ । ಓಂ ಪಂ ಪದ್ಮಾಯ ನಮಃ ।  ಓಂ ಅಂ ಸೂರ್ಯಮಂಡಲಾಯ ವಸುಪ್ರದ ದ್ವಾದಶಕಲಾತ್ಮನೇ ನಮಃ । ಓಂ ಉಂ ಸೋಮಮಂಡಲಾಯ ಕಾಮಪ್ರದ ಷೋಡಶ ನಮಃ । ಓಂ ಮಂ ವಹ್ನಿಮಂಡಲಾಯ ಧರ್ಮಪ್ರದ ದಶಕಲಾತ್ಮನೇ ನಮಃ । ಓಂ  ಬ್ರಹ್ಮಣೇ ನಮಃ ।  ಓಂ ವಿಷ್ಣವೇ ನಮಃ ।  ಓಂ ಈಶ್ವರಾಯ ನಮಃ । ಓಂ ಅಂ ಆತ್ಮನೇ ನಮಃ । ಓಂ ಉಂ ಅಂತರಾತ್ಮನೇ ನಮಃ । ಓಂ ಮಂ ಪರಮಾತ್ಮನೇ ನಮಃ । ಓಂ ಹ್ರೀಂ ಜ್ಞಾನಾತ್ಮನೇ ನಮಃ ।

\section{ದಿಕ್ಪಾಲಕಪೂಜಾ}
\as{ಓಂ ಲಂ} ಇಂದ್ರಾಯ ಸುರಾಧಿಪತಯೇ ವಜ್ರಹಸ್ತಾಯ ಶ್ವೇತವರ್ಣಾಯ ಐರಾವತವಾಹನಾಯ ಸಾಂಗಾಯ ಸಾಯುಧಾಯ ಸವಾಹನಾಯ ಸಪರಿವಾರಾಯ ಸಶಕ್ತಿಕಾಯ  ಶ್ರೀದುರ್ಗಾಪರಮೇಶ್ವರೀಪಾರ್ಷದಾಯ ನಮಃ~॥\\
\as{ಓಂ ರಂ} ಅಗ್ನಯೇ ತೇಜೋಽಧಿಪತಯೇ ಶಕ್ತಿಹಸ್ತಾಯ ರಕ್ತವರ್ಣಾಯ ಮೇಷವಾಹನಾಯ ಸಾಂಗಾಯ\\
\as{ಓಂ ಟಂ} ಯಮಾಯ ಪ್ರೇತಾಧಿಪತಯೇ ದಂಡಹಸ್ತಾಯ ನೀಲವರ್ಣಾಯ ಮಹಿಷವಾಹನಾಯ ಸಾಂಗಾಯ ******** ನಮಃ~॥\\
\as{ಓಂ ಕ್ಷಂ} ನಿರ್ಋತಯೇ ರಕ್ಷೋಧಿಪತಯೇ ಖಡ್ಗಹಸ್ತಾಯ ಹಿರಣ್ಯವರ್ಣಾಯ ನರವಾಹನಾಯ ಸಾಂಗಾಯ ******** ನಮಃ~॥\\
\as{ಓಂ ವಂ} ವರುಣಾಯ ಜಲಾಧಿಪತಯೇ ಪಾಶಹಸ್ತಾಯ ನೀಲಶುಭ್ರವರ್ಣಾಯ ಮಕರವಾಹನಾಯ ಸಾಂಗಾಯ ****** ನಮಃ~॥\\
\as{ಓಂ ಯಂ} ವಾಯವೇ ಸರ್ವಪ್ರಾಣಾಧಿಪತಯೇ ಅಂಕುಶಹಸ್ತಾಯ ಮೇಘವರ್ಣಾಯ ಹರಿಣವಾಹನಾಯ ಸಾಂಗಾಯ  ***** ನಮಃ~॥\\
\as{ಓಂ ಸಂ} ಸೋಮಾಯ ನಕ್ಷತ್ರಾಧಿಪತಯೇ ಗದಾಹಸ್ತಾಯ ಶುಕ್ಲವರ್ಣಾಯ ಅಶ್ವವಾಹನಾಯ ಸಾಂಗಾಯ ****** ನಮಃ~॥\\
\as{ಓಂ ಯಂ} ಈಶಾನಾಯ ವಿದ್ಯಾಧಿಪತಯೇ ತ್ರಿಶೂಲಹಸ್ತಾಯ ಹೇಮವರ್ಣಾಯ ವೃಷಭವಾಹನಾಯ ಸಾಂಗಾಯ **** ನಮಃ~॥\\
\as{ಓಂ ಐಂ} ಬ್ರಹ್ಮಣೇ ಸರ್ವಲೋಕಾಧಿಪತಯೇ ಪದ್ಮಹಸ್ತಾಯ ಶುಭ್ರವರ್ಣಾಯ ಹಂಸವಾಹನಾಯ  ಸಾಂಗಾಯ ********* ನಮಃ~॥\\
\as{ಓಂ ವಂ}  ಅನಂತಾಯ ಪಾತಾಲಾ(ನಾಗಾ)ಧಿಪತಯೇ ಚಕ್ರಹಸ್ತಾಯ ನೀಲವರ್ಣಾಯ ಗರುಡವಾಹನಾಯ  ಸಾಂಗಾಯ**** ನಮಃ~॥\\
ಪೀಠಪೂಜಾಂ ಸಮರ್ಪಯಾಮಿ~॥
\section{ನವಶಕ್ತಿಪೂಜಾ}
ಓಂ ಪ್ರಭಾಯೈ ನಮಃ । ಓಂ  ಮಾಯಾಯೈ ನಮಃ । ಓಂ  ಜಯಾಯೈ ನಮಃ । ಓಂ  ಸೂಕ್ಷ್ಮಾಯೈ ನಮಃ । ಓಂ  ವಿಶುದ್ಧಾಯೈ ನಮಃ । ಓಂ  ನಂದಿನ್ಯೈ ನಮಃ । ಓಂ  ಸುಪ್ರಭಾಯೈ ನಮಃ । ಓಂ  ವಿಜಯಾಯೈ ನಮಃ । ಓಂ  ಸರ್ವಸಿದ್ಧಿಪ್ರದಾಯೈ ನಮಃ ।
\newpage
ಪಾದುಕಾದಿ ಪೀಠಾಂತಪೂಜಾಂ ಕರಿಷ್ಯೇ ॥\\
ಮಂತ್ರೀಶಾನಂದನಾಥಾಯೈ ನಮಃ ಪಾದುಕಾಂ ಕಲ್ಪಯಾಮಿ ತರ್ಪಯಾಮಿ ಪೂಜಯಾಮಿ ॥\\
ಪುಷ್ಟೀಶಾನಂದನಾಥಾಯೈ ನಮಃ ಪಾದುಕಾಂ ಕ । ತ । ಪೂಜಯಾಮಿ ॥\\
ಚರ್ಚೀಶಾನಂದನಾಥಾಯೈ ನಮಃ ಪಾದುಕಾಂ ಕ । ತ । ಪೂಜಯಾಮಿ ॥\\
ಪ್ರಕಾಶಾನಂದನಾಥಾಯೈ ನಮಃ ಪಾದುಕಾಂ ಕ । ತ । ಪೂಜಯಾಮಿ ॥\\
ವರ್ಮೀಶಾನಂದನಾಥಾಯೈ ನಮಃ ಪಾದುಕಾಂ ಕ । ತ । ಪೂಜಯಾಮಿ ॥\\
ಆನಂದಾನಂದನಾಥಾಯೈ ನಮಃ ಪಾದುಕಾಂ ಕ । ತ । ಪೂಜಯಾಮಿ ॥\\
ಜ್ಞಾನಾನಂದನಾಥಾಯೈ ನಮಃ ಪಾದುಕಾಂ ಕಲ್ಪಯಾಮಿ ತರ್ಪಯಾಮಿ ಪೂಜಯಾಮಿ ॥

ಗಂ ಗಣಪತಯೇ  ನಮಃ ॥\\
ವಂ ವಟುಕ ಭೈರವಾಯ ನಮಃ ॥\\
ಕಾಮರೂಪ ಪೀಠಾಯೈ ನಮಃ ॥\\
ಪೂರ್ಣಗಿರಿ ಪೀಠಾಯೈ ನಮಃ ॥\\
ಜಾಲಂಧರ ಪೀಠಾಯೈ ನಮಃ ॥\\
ಓಡ್ಯಾಣ ಪೀಠಾಯೈ ನಮಃ ॥\\
ಕಲ್ಪವೃಕ್ಷಾಯ ನಮಃ ॥\\
ಕದಂಬವನವಾಟಿಕಾಯೈ ನಮಃ ॥\\
ಹ್ರಾಂ ನಮಃ ॥
ಹ್ರೀಂ ನಮಃ ॥
ಹ್ರೂಂ ನಮಃ ॥
ಹ್ರೈಂ ನಮಃ ॥
ಹ್ರೌಂ ನಮಃ ॥
ಹ್ರಃ ನಮಃ ॥

ಓಂ ಬ್ರಹ್ಮಣೇ ಪೃಥಿವ್ಯಧಿಪತಯೇ ನಮಃ ॥\\
ವಿಂ ವಿಷ್ಣವೇ ಪಾತಾಳಾಧಿಪತಯೇ ನಮಃ ॥\\
ರುಂ ರುದ್ರಾಯ ತೇಜೋಽಧಿಪತಯೇ ನಮಃ ॥\\
ಹ್ರೀಂ ಈಶಾನಾಯ ವಿದ್ಯಾಧಿಪತಯೇ ನಮಃ ॥\\
ಪಂಚವಕ್ತ್ರಾಯ ನಮಃ ॥
ಸದಾಶಿವಾಯ ನಮಃ ॥\\
ಮಹಾಪದ್ಮಾಸನಾಯ ನಮಃ ॥\\
ನಮೋ ಭಗವತ್ಯೈ ಸರ್ವೇಶ್ವರ್ಯೈ ಸರ್ವಭೂತಾತ್ಮಿಕಾಯೈ ಸಂಯೋಗಪೀಠಾತ್ಮಿಕಾಯೈ ನಮಃ ॥\\
(ಐಂ ಹ್ರೀಂ ಶ್ರೀಂ ಕ್ಲೀಂ ವಾಗೀಶ್ವರೀಂ  ಪರಮವಾಗೀಶ್ವರೀಮೂರ್ತಿಮಾವಾಹಯಾಮಿ ॥ ಆವಾಹಿತಾ ಭವ~। ಸಂಸ್ಥಾಪಿತಾ ಭವ~। ಸನ್ನಿಹಿತಾ ಭವ~। ಸನ್ನಿರುದ್ಧಾ ಭವ~। ಸಮ್ಮುಖಾ ಭವ~। ಅವಗುಂಠಿತಾ ಭವ~। ವ್ಯಾಪ್ತಾ ಭವ~। ಸುಪ್ರಸನ್ನಾ ಭವ~।)

ತತ್ತದ್ದಿನಾನುಸಾರೇಣ ಧ್ಯಾಯೇತ್ ॥
(ಪ್ರತಿಪದಿ)\\
\as{ಹಂಸಾರೂಢಾಂ ಶುಕ್ಲವರ್ಣಾಂ ಶುಕ್ಲಮಾಲ್ಯಾದ್ಯಲಂಕೃತಾಂ ।\\
ಚತುರ್ಭುಜಾಂ ಸೃಕ್ ಸೃವೌ ಚ ಕಮಂಡಲ್ವಕ್ಷಮಾಲಿಕಾಂ ।\\
ಬಿಭ್ರತೀಂ ಚಿಂತಯೇದ್ದೇವೀಂ ಪ್ರತಿಪದ್ದಿವಸೇ ಸುಧೀಃ ॥

ಶುದ್ಧಸ್ಫಟಿಕಸಂಕಾಶಾಂ ರವಿಬಿಂಬಸಮಪ್ರಭಾಂ ।\\
ಕುಂದೇಂದುತುಹಿನಾಭಾಸಾಂ ಹಂಸಾಸನಗತಾಂ ಶುಭಾಂ ॥

ಅಂಕುಶಂ ಚಾಕ್ಷಸೂತ್ರಂ ಚ ಪಾಶಪುಸ್ತಕಧಾರಿಣೀಂ ।\\
ಮುಕ್ತಾಹಾರೈಃ ಸಮಾಯುಕ್ತಾಂ ದೇವೀಂ ಧ್ಯಾಯೇಚ್ಚತುರ್ಭುಜಾಂ ॥

ಪುಸ್ತಕಜಪವಟಹಸ್ತೇ ವರದಾಭಯಚಿಹ್ನಚಾರು ಬಾಹುಲತೇ ।\\
ಕರ್ಪೂರಾಮಲದೇಹೇ ವಾಗೀಶ್ವರಿ ಶೋಧಯಾಶು ಮಮ ಚೇತಃ ॥}\\
ಪ್ರಥಮದಿವಸೇ ಮಹಾಕಾಳೀ ಮಹಾಲಕ್ಷ್ಮೀ ಮಹಾಸರಸ್ವತೀ ಸ್ವರೂಪಿಣೀಂ ಹಂಸವಾಹಿನೀಂ ಶೈಲಪುತ್ರೀಮಾವಾಹಯಾಮಿ ॥

(ದ್ವಿತೀಯಾಯಾಂ)\\
\as{ರಕ್ತವರ್ಣಾಂ ಗಜಾರೂಢಾಂ ರಕ್ತಮಾಲ್ಯವಿಭೂಷಿತಾಂ ।\\
ಚತುರ್ಭುಜಾಂ ಶಕ್ತಿಶೂಲೇ ಗದಾಮಭಯಧಾರಿಣೀಂ ।\\
ದೇವೀಂ ಚ ಪೂಜಯೇದ್ಧೀಮಾನ್ ದ್ವಿತೀಯಾಯಾಂ ಸದಾಶಿವಾಂ ॥}\\
ದ್ವಿತೀಯದಿವಸೇ ಮಹಾಕಾಳೀ ಮಹಾಲಕ್ಷ್ಮೀ ಮಹಾಸರಸ್ವತೀ ಸ್ವರೂಪಿಣೀಂ ಗಜವಾಹಿನೀಂ ಬ್ರಹ್ಮಚಾರಿಣೀಮಾವಾಹಯಾಮಿ ॥

(ತೃತೀಯಾಯಾಂ)\\
\as{ಗೌರಾಂಗೀಂ ದ್ವಿಭುಜಾಂ ದೇವೀಂ ಕಹ್ಲಾರವರಧಾರಿಣೀಂ ।\\
ಚಂಡೀಂ ಚ ಪೂಜಯೇದ್ಧೀಮಾನ್ ಸಿಂಹಾರೂಢಾಂ ತೃತೀಯಕೇ ॥\\}
ತೃತೀಯದಿವಸೇ  ಮಹಾಕಾಳೀ ಮಹಾಲಕ್ಷ್ಮೀ ಮಹಾಸರಸ್ವತೀ ಸ್ವರೂಪಿಣೀಂ ಸಿಂಹವಾಹಿನೀಂ ಚಂಡೀಮಾವಾಹಯಾಮಿ ॥

(ಚತುರ್ಥ್ಯಾಂ)\\
\as{ಗೌರವರ್ಣಾದ್ಯಲಂಕಾರೈಃ ಕ್ಷೌಮವರ್ಣಾಂಬರಾನ್ವಿತಾಂ ।\\
ಪಾಶಾಂಕುಶ ಧರಾಂ ದೇವೀಂ ತಥಾ ಚ ಮೃಗವಾಹಿನೀಂ ॥\\
ಚತುರ್ಭುಜಾಂ ಸುವರ್ಣಾಭಾಂ ಚತುರ್ಥ್ಯಾಂ ಪೂಜಯೇನ್ನೃಪ ॥}\\
ಚತುರ್ಥದಿವಸೇ ಮಹಾಕಾಳೀ ಮಹಾಲಕ್ಷ್ಮೀ ಮಹಾಸರಸ್ವತೀ ಸ್ವರೂಪಿಣೀಂ ಮೃಗವಾಹಿನೀಂ ಕೂಷ್ಮಾಂಡಾಮಾವಾಹಯಾಮಿ ॥

(ಪಂಚಮ್ಯಾಂ)\\
\as{ಸ್ವರ್ಣವಸ್ತ್ರಾದ್ಯಲಂಕಾರೈಃ ಸ್ವರ್ಣವರ್ಣೋಪಶೋಭಿತಾಂ ।\\
ಚತುರ್ಭುಜಾಂ ಸುವರ್ಣಾಭಾಂ ಶಂಖಚಕ್ರಗದಾಂಬುಜಾಂ ॥\\
ಪಂಚಮ್ಯಾಂ ಪೂಜಯೇದ್ದೇವೀಂ ಸದಾ ಮಕರವಾಹಿನೀಂ ॥}\\
ಪಂಚಮದಿವಸೇ ಮಹಾಕಾಳೀ ಮಹಾಲಕ್ಷ್ಮೀ ಮಹಾಸರಸ್ವತೀ ಸ್ವರೂಪಿಣೀಂ ಮಕರವಾಹಿನೀಂ ಸ್ಕಂದಮಾತರಮಾವಾಹಯಾಮಿ ॥

(ಷಷ್ಠ್ಯಾಂ)\\
\as{ಶ್ವೇತಾಂಬರಾಮಾಭರಣೈಃ ಸ್ಫಾಟಿಕೈರುಪಶೋಭಿತಾಂ ।\\
ಬಾಣಕೋದಂಡಖೇಟಾಂಶ್ಚ ಕಮಂಡಲುಧರಾಂ ಶಿವಾಂ ॥\\
ಮಯೂರವಾಹಿನೀಂ ದೇವೀಂ ಷಷ್ಠ್ಯಾಂ ಸಂಪೂಜಯೇನ್ನೃಪ ॥}\\
ಷಷ್ಠದಿವಸೇ ಮಹಾಕಾಳೀ ಮಹಾಲಕ್ಷ್ಮೀ ಮಹಾಸರಸ್ವತೀ ಸ್ವರೂಪಿಣೀಂ ಮಯೂರವಾಹಿನೀಂ ಕತ್ಯಾಯನೀಮಾವಾಹಯಾಮಿ ॥

(ಸಪ್ತಮ್ಯಾಂ)\\
\as{ಶ್ವೇತವರ್ಣಾದ್ಯಲಂಕಾರೈರ್ದ್ವಿಭುಜಾಂ ಪದ್ಮಧಾರಿಣೀಂ ।\\
ಸಪ್ತಮ್ಯಾಂ ಪೂಜಯೇದ್ ಗೌರೀಂ ಅಶ್ವಾರೂಢಾಂ ಸದಾ ನೃಪ ॥}\\
ಸಪ್ತಮದಿವಸೇ ಮಹಾಕಾಳೀ ಮಹಾಲಕ್ಷ್ಮೀ ಮಹಾಸರಸ್ವತೀ ಸ್ವರೂಪಿಣೀಂ ಅಶ್ವಾರೂಢಾಂ ಗೌರೀಮಾವಾಹಯಾಮಿ ॥

(ಅಷ್ಟಮ್ಯಾಂ)\\
\as{ರೌದ್ರವರ್ಣಾದ್ಯಲಂಕಾರೈರ್ಭೂಷಿತಾಂ ಚ ಚತುರ್ಭುಜಾಂ ।\\
ತ್ರಿಶೂಲಂ ಡಮರುಂ ಚೈವ ಚರ್ಮಖಟ್ವಾಂಗಧಾರಿಣೀಂ ।\\
ಅಷ್ಟಮ್ಯಾಂ ಪೂಜಯೇನ್ಮೂರ್ತಿಂ ವೃಷಾರೂಢಾಂ ಪ್ರಯತ್ನತಃ ॥}\\
ಅಷ್ಟಮದಿವಸೇ ಮಹಾಕಾಳೀ ಮಹಾಲಕ್ಷ್ಮೀ ಮಹಾಸರಸ್ವತೀ ಸ್ವರೂಪಿಣೀಂ\\ ವೃಷಭವಾಹಿನೀಂ ತ್ರಿಮೂರ್ತಿಮಾವಾಹಯಾಮಿ ॥

(ನವಮ್ಯಾಂ)\\
\as{ರಕ್ತವರ್ಣಾಂ ರಕ್ತಭೂಷಾಂ ಅಷ್ಟಾದಶಭುಜೈರ್ಯುತಾಂ ।\\
ಸಾಕ್ಷಮಾಲಾಂ ಮಾರ್ಗಣಂ ಚ  ಅರವಿಂದಂ ತಥಾ ಧನುಃ ॥

ಭಿದುರಂ ಚ  ಗದಾಂ ಚೈವ ರಥಾಂಗಂ ಶೂಲಮೇವ ಚ ।\\
ಕುಠಾರಂ ಶಂಖಪೂರ್ಣಂ ಚ ಪಾಶಂ ಶಕ್ತಿಂ ಚ ಯಷ್ಟಿಕಾಂ ॥

ಖೇಟಂ ಶರಾಸನಂ ಪಾತ್ರಂ ಕಮಂಡಲುಧೃತಾಂ ಶಿವಾಂ ।\\
ತ್ರಿಣೇತ್ರಾಂ ಚಂದ್ರಚೂಡಾಂ ಚ ಮಕುಟೇನೋಪಶೋಭಿತಾಂ ॥

ಸೂರ್ಯಕೋಟಿಪ್ರತೀಕಾಶಾಂ ಮಹಿಷಾಸುರಮರ್ದಿನೀಂ ।\\
ಸಿಂಹಾರೂಢಾಂ ಮಹಾದುರ್ಗಾಂ ತ್ರಿದಶೈರಭಿಪೂಜಿತಾಂ ।\\
ನವಮ್ಯಾಂ ಪೂಜಯೇದ್ದೇವೀಂ ಸರ್ವಾಭೀಷ್ಟಪ್ರದಾಯಿನೀಂ ॥}\\
ನವಮದಿವಸೇ ಮಹಾಕಾಳೀ ಮಹಾಲಕ್ಷ್ಮೀ ಮಹಾಸರಸ್ವತೀ ಸ್ವರೂಪಿಣೀಂ ಮಹಾಮಾಯಾಂ ತ್ರಿಗುಣಾತ್ಮಿಕಾಂ ದುರ್ಗಾದೇವೀಮಾವಾಹಯಾಮಿ ॥


\section{ಸಮಷ್ಟಿಧ್ಯಾನಂ}
\as{ಅರುಣಕಮಲಸಂಸ್ಥಾ ತದ್ರಜಃಪುಂಜವರ್ಣಾ\\
ಕರಕಮಲಧೃತೇಷ್ಟಾಭೀತಿಯುಗ್ಮಾಂಬುಜಾ ಚ ।\\
ಮಣಿಮಕುಟವಿಚಿತ್ರಾಲಂಕೃತಾ ಪದ್ಮಹಸ್ತಾ\\
ಸಕಲಭುವನಮಾತಾ ಸಂತತಂ ಶ್ರೀಃ ಶ್ರಿಯೈ ನಃ ॥}\\
ಶ್ರೀ ದುರ್ಗಾಪರಮೇಶ್ವರ್ಯೈ ನಮಃ ॥ ಧ್ಯಾಯಾಮಿ॥

ಸುಭಗೇ ಸುಂದರೇ ಸೌಮ್ಯೇ ಸುಷುಮ್ನೇ ಸುಮನಾವತಿ ।\\
ಮನೋನ್ಮಾಯೇ ಮಹಾಮಾಯೇ ಶೋಭನೇ ಲಲಿತೇ ಶುಭೇ ॥

ಮಹಾಪದ್ಮವನಾಂತಸ್ಥೇ ಕಾರಣಾನಂದ ವಿಗ್ರಹೇ ।\\
ಸರ್ವಭೂತಹಿತೇ ಮಾತರೇಹ್ಯೇಹಿ ಪರಮೇಶ್ವರಿ ॥\\
\as{ಹಿರಣ್ಯವರ್ಣಾಂ ++++++++++ ಮ ಆವಹ ॥} ಆವಾಹನಮ್ ॥

ಅಥ ಕಲ್ಪತರೋ ನಿತ್ಯಂ ಭವತ್ಯಾ ರತ್ನಮಂದಿರೇ ।\\
ರತ್ನಸಿಂಹಾಸನಂ ದೇವಿ ಸ್ವೀಕುರುಷ್ವ ಸುರೇಶ್ವರಿ ॥\\
\as{ತಾಂ ಮ+++++++++++ ಪುರುಷಾನಹಮ್॥}ಆಸನಮ್ ॥

ಸರ್ವರಂಜನಶಕ್ತಿಶ್ಚ ಸರ್ವೋನ್ಮಾದಕರೂಪಿಣಿ ।\\
ಪಾದ್ಯಂ ಗೃಹಾಣ ದೇವೇಶಿ ಶುಂಭಾಸುರ ನಿಬರ್ಹಿಣಿ ॥\\
\as{ಅಶ್ವಪೂರ್ವಾಂ +++++++++ಜುಷತಾಮ್॥}ಪಾದ್ಯಮ್ ॥


ಸೂರ್ಯೇಂದುವಹ್ನಿಪೀಠೇ ತು ಬಿಂದು ಚಕ್ರನಿವಾಸಿನಿ ।\\
ಬ್ರಹ್ಮಸ್ವರೂಪಿಣಿ ದೇವಿ ಗೃಹಾಣಾರ್ಘ್ಯಂ ನಮೋಽಸ್ತು ತೇ ॥\\
\as{ಕಾಂ ಸೋಸ್ಮಿತಾಂ +++++++++ ಶ್ರಿಯಮ್ ॥} ಅರ್ಘ್ಯಮ್ ॥


ನಿಶುಂಭಮಥಿನೀ ದೇವಿ ಶುಂಭಾಸುರ ನಿಬರ್ಹಿಣಿ ।\\
ಅನಯಾ ವ್ಯಾಪ್ತಮಖಿಲಂ ಜಗತ್ಸ್ಥಾವರ ಜಂಗಮಮ್ ॥ ಆಚಮನೀಯಮ್॥\\
\as{ಚನ್ದ್ರಾಂ ಪ್ರಭಾಸಾಂ ++++++++ತ್ವಾಂ ವೃಣೇ ॥} ಆಚಮನಮ್ ॥


ಅಣಿಮಾದಿಗುಣಾಧಾರೇ ಅಕಾರಾದ್ಯಕ್ಷರಾತ್ಮಿಕೇ ।\\
ಅನಂತರೂಪಭೇದಾತ್ಮೇ ಮಧುಪರ್ಕಂ ದದಾಮ್ಯಹಮ್ ॥\\
\as{ಮಧುವಾತಾ+++++ತ್ವೋಷಧೀಃ ॥} ಮಧುಪರ್ಕಃ ॥

ಮಧುಪರ್ಕಾಂತೇ ಆಚಮನೀಯಂ ॥\\
ಮಲಾಪಕರ್ಷಣ ಸ್ನಾನಮ್॥

ಕ್ಷೀರಂ ದಧಿ ಘೃತಂ ಚೈವ ಶರ್ಕರಾಮಧುಸಂಯುತಮ್ ।\\
ಪಂಚಾಮೃತಂ ಪ್ರದಾಸ್ಯಾಮಿ ಶಂಕರಪ್ರಿಯವಲ್ಲಭೇ ॥ ಪಂಚಾಮೃತಮ್ ॥

ಪರಮಾಮೃತನಿಷ್ಯಂದೇ ಪರಮಾರ್ಥಸ್ವರೂಪಿಣಿ ।\\
ರತ್ನಕುಂಭಸ್ಥಿತಂ ತೋಯಂ ಗಂಧಪುಷ್ಪೈಶ್ಚ ಸಂಯುತಮ್ ॥\\
\as{ಆದಿತ್ಯವರ್ಣೇ +++++++++ಬಾಹ್ಯಾ ಅಲಕ್ಷ್ಮೀಃ ॥}ಶುದ್ಧೋದಕಸ್ನಾನಂ॥

ಮದಸ್ವಿನಿ ಮದೋದ್ಗೀರ್ಣೇ ಮಾಲಿನೀ ಮದವರ್ಧಿನಿ ।\\
ಕ್ಷೌಮವಸ್ತ್ರ ಪರೀಧಾನಂ ಜಗನ್ಮಾತರ್ದದಾಮಿ ತೇ ॥\\
\as{ಉಪೈತು ಮಾಂ ++++++++ ದದಾತು ಮೇ ॥\\
ಯುವಂ ವಸ್ತ್ರಾಣಿ ++++ಸಚೇಥೇ ॥}ವಸ್ತ್ರಯುಗ್ಮಮ್ ॥

ಗುಣತ್ರಯ ಸಮಾಯುಕ್ತಂ ಬ್ರಹ್ಮಸೂತ್ರಮನುತ್ತಮಮ್ ।\\
ದದಾಮಿ ತವ ದೇವೇಶಿ ಗೃಹಾಣ ಪರಮೇಶ್ವರಿ ॥

\as{ಕ್ಷುತ್ಪಿಪಾಸಾಮ್+++++++++++ ಗೃಹಾತ್॥\\
ಯಜ್ಞೋಪವೀತಮ್ ++++ತೇಜಃ ॥} ಉಪವೀತಮ್ ॥

ವಜ್ರವೈಡೂರ್ಯ ಮಾಣಿಕ್ಯ ರಚಿತಂ ಹೇಮ ಕಂಚುಕಮ್ ।\\
ಗೃಹಾಣ ತ್ವಂ ಮಹಾದೇವಿ ಜಗನ್ಮಾತರ್ನಮೋಽಸ್ತು ತೇ ॥ಕಂಚುಕಾಭರಣಮ್॥

ಮೂಲ ಭೂತಾತ್ಮಿಕೇ ದೇವಿ ಮೂಲಾಧಾರ ನಿವಾಸಿನಿ ।\\
ನಾನಾರತ್ನೈಶ್ಚ ಖಚಿತಂ ಸರ್ವಾಭರಣಮುತ್ತಮಮ್ ॥\\
\as{ಹಿರಣ್ಯರೂಪಃ+++++ತ್ಯನ್ನಮಸ್ಮೈ ॥ }ಆಭರಣಮ್॥

ಶಿವೇಶ್ವರಿ ಶಿವಾರಾಧ್ಯೇ ಶಿವಾನಂದೇ ಶಿವಾತ್ಮಕೇ ।\\
ಕರ್ಪೂರಕುಂಕುಮಾದ್ಯೈಶ್ಚ ಮರ್ದಿತಂ ಗಂಧಮುತ್ತಮಮ್ ॥\\
\as{ಗಂಧದ್ವಾರಾಂ+++++++++ ಶ್ರಿಯಮ್ ॥\\
ಗಂಧದ್ವಾರಾಂ+++++++++ ಶ್ರಿಯಮ್ ॥}ಗಂಧಃ ॥

ಶಾಂಭವಾನಂದಜನನಿ ಪರಮಾನಂದ ರೂಪಿಣಿ ।\\
ಅಕ್ಷತಾಂಶ್ಚ ಮಯಾನೀತಾನ್ ಗೃಹಾಣ ಪರಮೇಶ್ವರಿ ॥\\
\as{ಅರ್ಚತಪ್ರಾರ್ಚತ+++++ ಧೃಷ್ಣ್ವರದಚತ ॥}ಅಕ್ಷತಾಃ ॥

ಮಾಲ್ಯಾನಿ ಚ ಸುಗಂಧೀನಿ ಮಾಲತ್ಯಾದೀನಿ ಚ ಪ್ರಿಯೇ ।\\
ಮಯಾಹೃತಾನಿ ಪೂಜಾರ್ಥಂ ಪುಷ್ಪಾಣಿ ಪ್ರತಿಗೃಹ್ಯತಾಮ್ ॥\\
\as{ಮನಸಃ +++++++++++ ಯಶಃ ॥\\
ಆಯನೇತೇ++++++ಗೃಹಾ ಇಮೇ ॥}ಪುಷ್ಪಾಣಿ ॥
\subsection{ಬಿಲ್ವಪತ್ರಪೂಜಾ}
ಅಮೃತೋದ್ಭವಂ ಚ ಶ್ರೀವೃಕ್ಷಂ ಮಹಾದೇವೀ ಪ್ರಿಯಂ ಸದಾ ।\\
ಬಿಲ್ವಪತ್ರಂ ಪ್ರಯಚ್ಛಾಮಿ ಪವಿತ್ರಂ ತೇ ಸುರೇಶ್ವರಿ ॥

ಆರ್ಯಾಯೈ ನಮಃ । ಬಿಲ್ವಪತ್ರಂ ಸಮರ್ಪಯಾಮಿ ॥\\
ಕಾತ್ಯಾಯನ್ಯೈ ನಮಃ । ಬಿಲ್ವಪತ್ರಂ ಸಮರ್ಪಯಾಮಿ ॥\\
ಗೌರ್ಯೈ ನಮಃ । ಬಿಲ್ವಪತ್ರಂ ಸಮರ್ಪಯಾಮಿ ॥\\
ಕಾಲ್ಯೈ ನಮಃ । ಬಿಲ್ವಪತ್ರಂ ಸಮರ್ಪಯಾಮಿ ॥\\
ಹೈಮವತ್ಯೈ ನಮಃ । ಬಿಲ್ವಪತ್ರಂ ಸಮರ್ಪಯಾಮಿ ॥\\
ಈಶ್ವರ್ಯೈ ನಮಃ । ಬಿಲ್ವಪತ್ರಂ ಸಮರ್ಪಯಾಮಿ ॥\\
ಶಿವಾಯೈ ನಮಃ । ಬಿಲ್ವಪತ್ರಂ ಸಮರ್ಪಯಾಮಿ ॥\\
ಭವಾನ್ಯೈ ನಮಃ । ಬಿಲ್ವಪತ್ರಂ ಸಮರ್ಪಯಾಮಿ ॥\\
ರುದ್ರಾಣ್ಯೈ  ನಮಃ । ಬಿಲ್ವಪತ್ರಂ ಸಮರ್ಪಯಾಮಿ ॥
\subsection{ಪಲ್ಲವಪೂಜಾ}
ಶಾಂತಾಗೀಂ ಶ್ಯಾಮಲಾಂ ಸರ್ವಾಂ ದುಷ್ಟಾಸುರನಿಬರ್ಹಿಣೀಮ್ ।\\
ಪೂಜಾಂ ಕರೋಮಿ ಚಾರ್ವಂಗೀಂ ಪಲ್ಲವೈರ್ನಂದನೋದ್ಭವೈಃ ॥\\
ಚಾರ್ವಂಗ್ಯೈ ನಮಃ । ಪಲ್ಲವಪೂಜಾಂ ಸಮರ್ಪಯಾಮಿ ॥

ಮಂತ್ರಾಕ್ಷರಮಯೀಂ ಲಕ್ಷ್ಮೀಂ ಸಾಕ್ಷಾತ್ ಸಂಪತ್ಪ್ರದಾಯಿನೀಮ್ ।\\
ದೂರ್ವಾಂಕುರಾನ್  ಪ್ರದಾಸ್ಯಾಮಿ ಅಷ್ಟಗಂಧೇನ ಸಂಯುತಾನ್ ॥\\
ಶ್ರೀದುರ್ಗಾಪರಮೇಶ್ವರ್ಯೈ ನಮಃ। ದುರ್ವಾಂಕುರಾನ್ ಸಮರ್ಪಯಾಮಿ ॥

ಶರತ್ಕಾಲೇ ಸಮುದ್ಭೂತಾನ್ಯಾಹೃತಾನ್ಯಮಲಾನನೇ ।\\
ಫಲಾನೀಮಾನಿ ರಮ್ಯಾಣಿ ಗೃಹೀಷ್ವ ಸುರಪೂಜಿತೇ ॥\\
(ಕದಳೀ ನಾರಿಕೇಳಂ ಚ ಚೂತಜಂಬೀರದ್ರಾಕ್ಷಿಕಾಃ ।\\
ಉರ್ವಾರುಕಾದಿ ಸರ್ವಾಣಿ ತತ್ಕಾಲಫಲಸಂಭವಂ॥)\\
ಶ್ರೀ ದುರ್ಗಾಪರಮೇಶ್ವರ್ಯೈ ನಮಃ । ಫಲಂ ಸಮರ್ಪಯಾಮಿ ।

\subsection{ಅಂಗಪೂಜಾ}
ವಾರಾಹ್ಯೈ ನಮಃ । ಪಾದೌ  ಪೂಜಯಾಮಿ ॥\\
ನಾರಸಿಂಹ್ಯೈ ನಮಃ । ಗುಲ್ಫೌ  ಪೂಜಯಾಮಿ ॥\\
ರಕ್ತಬೀಜನಿಪಾತಿನ್ಯೈ ನಮಃ । ಜಂಘೇ  ಪೂಜಯಾಮಿ ॥\\
ಬ್ರಾಹ್ಮ್ಯೈ ನಮಃ । ಜಾನುನೀ  ಪೂಜಯಾಮಿ ॥\\
ಶಿವದೂತ್ಯೈ ನಮಃ । ಊರೂ  ಪೂಜಯಾಮಿ ॥\\
ಕೌಮಾರ್ಯೈ ನಮಃ । ಜಘನದ್ವಯಂ  ಪೂಜಯಾಮಿ ॥\\
ಕಾಲ್ಯೈ ನಮಃ । ಕಟಿಂ  ಪೂಜಯಾಮಿ ॥\\
ಗೌರ್ಯೈ ನಮಃ । ಗುಹ್ಯಂ  ಪೂಜಯಾಮಿ ॥\\
ಭವಾನ್ಯೈ ನಮಃ । ನಾಭಿಂ  ಪೂಜಯಾಮಿ ॥\\
ಶರ್ವಾಣ್ಯೈ ನಮಃ । ಉದರಂ  ಪೂಜಯಾಮಿ ॥\\
ಹೈಮವತ್ಯೈ ನಮಃ । ಹೃದಯಂ  ಪೂಜಯಾಮಿ ॥\\
ಮನ್ಮಥವಾಸಿನ್ಯೈ ನಮಃ । ಸ್ತನದ್ವಯಂ  ಪೂಜಯಾಮಿ ॥\\
ಚಂಡಿಕಾಯೈ ನಮಃ । ಕಕ್ಷೌ  ಪೂಜಯಾಮಿ ॥\\
ಶಾಕಂಭರ್ಯೈ ನಮಃ । ಕಂಠಂ  ಪೂಜಯಾಮಿ ॥\\
ಕಾಲ್ಯೈ ನಮಃ । ಸ್ಕಂಧೌ  ಪೂಜಯಾಮಿ ॥\\
ಸರ್ವಾಸ್ತ್ರಧಾರಿಣ್ಯೈ ನಮಃ । ಹಸ್ತಾನ್  ಪೂಜಯಾಮಿ ॥\\
ಸರ್ವಮಂಗಲಾಯೈ ನಮಃ । ಮುಖಂ  ಪೂಜಯಾಮಿ ॥\\
ವೇದಸ್ವರೂಪಣ್ಯೈ ನಮಃ । ನಾಸಿಕಾಂ  ಪೂಜಯಾಮಿ ॥\\
ರಕ್ತದಂತಿಕಾಯೈ ನಮಃ । ದಂತಾನ್  ಪೂಜಯಾಮಿ ॥\\
ಬಿಂದುನಾದಸ್ವರೂಪಿಣ್ಯೈ ನಮಃ । ವಕ್ತ್ರಂ  ಪೂಜಯಾಮಿ ॥\\
ಕರುಣಾರವಿಂದಲೋಚನಾಯೈ ನಮಃ । ನೇತ್ರತ್ರಯಂ  ಪೂಜಯಾಮಿ ॥\\
ಕಾರುಣ್ಯಮೂರ್ತ್ಯೈ ನಮಃ । ಕರ್ಣೌ  ಪೂಜಯಾಮಿ ॥\\
ಚಂದ್ರಚೂಡಾಯೈ ನಮಃ । ಲಲಾಟಂ  ಪೂಜಯಾಮಿ ॥\\
ಸರ್ವೇಶ್ವರ್ಯೈ ನಮಃ । ಶಿರಃ  ಪೂಜಯಾಮಿ ॥\\
ಸರ್ವಾಭೀಷ್ಟಪ್ರಾದಾಯಿನ್ಯೈ ನಮಃ । ಸರ್ವಾಂಗಾನಿ ಪೂಜಯಾಮಿ ॥
\subsection{ಅಥ ಪತ್ರಪೂಜಾ}
ಕಾಮಾಕ್ಷ್ಯೈ ನಮಃ । ಧತ್ತೂರ ಪತ್ರಂ ಸಮರ್ಪಯಾಮಿ ॥\\
ಮಂಗಲಾಯೈ ನಮಃ । ಬಿಲ್ವ ಪತ್ರಂ ಸಮರ್ಪಯಾಮಿ ॥\\
ಮಾಧವ್ಯೈ ನಮಃ । ಮರುಗ ಪತ್ರಂ ಸಮರ್ಪಯಾಮಿ ॥\\
ಗಿರಿಜಾಯೈ ನಮಃ । ಅಪಾಮಾರ್ಗ ಪತ್ರಂ ಸಮರ್ಪಯಾಮಿ ॥\\
ನಿರ್ಗುಣಾಯೈ ನಮಃ । ನಿರ್ಗುಂಡೀ ಪತ್ರಂ ಸಮರ್ಪಯಾಮಿ ॥\\
ನಾಗಹಾರಾಯೈ ನಮಃ । ವೇಣು ಪತ್ರಂ ಸಮರ್ಪಯಾಮಿ ॥\\
ವಿಷ್ಣುಪ್ರಿಯಾಯೈ ನಮಃ । ವಿಷ್ಣುಕ್ರಾಂತಿ ಪತ್ರಂ ಸಮರ್ಪಯಾಮಿ ॥\\
ಸರ್ವದಾಯೈ ನಮಃ । ಶತಪತ್ರ ಪತ್ರಂ ಸಮರ್ಪಯಾಮಿ ॥\\
ಯಶಸ್ವಿನ್ಯೈ ನಮಃ । ದೂರ್ವಾ ಪತ್ರಂ ಸಮರ್ಪಯಾಮಿ ॥\\
ಸರ್ವಮಂತ್ರಾತ್ಮಿಕಾಯೈ ನಮಃ । ಸಮಸ್ತಪತ್ರಾಣಿ  ಸಮರ್ಪಯಾಮಿ ॥
\newpage
\subsection{ಪುಷ್ಪಪೂಜಾ}
ಉಮಾಯೈ ನಮಃ । ಪುನ್ನಾಗ ಪುಷ್ಪಂ ಸಮರ್ಪಯಾಮಿ ॥\\
ಕಾತ್ಯಾಯನ್ಯೈ ನಮಃ । ಚಂಪಕ ಪುಷ್ಪಂ ಸಮರ್ಪಯಾಮಿ ॥\\
ಗೌರ್ಯೈ ನಮಃ । ಜಾತೀ ಪುಷ್ಪಂ ಸಮರ್ಪಯಾಮಿ ॥\\
ಕಾಲ್ಯೈ ನಮಃ । ಕೇತಕೀ ಪುಷ್ಪಂ ಸಮರ್ಪಯಾಮಿ ॥\\
ಹೈಮವತ್ಯೈ ನಮಃ । ಕರವೀರ ಪುಷ್ಪಂ ಸಮರ್ಪಯಾಮಿ ॥\\
ಈಶ್ವರ್ಯೈ ನಮಃ । ಉತ್ಪಲ ಪುಷ್ಪಂ ಸಮರ್ಪಯಾಮಿ ॥\\
ಭವಾನ್ಯೈ ನಮಃ । ಮಲ್ಲಿಕಾ ಪುಷ್ಪಂ ಸಮರ್ಪಯಾಮಿ ॥\\
ರುದ್ರಾಣ್ಯೈ ನಮಃ । ಯೂಥಿಕಾ ಪುಷ್ಪಂ ಸಮರ್ಪಯಾಮಿ ॥\\
ಲೋಕಮಾತ್ರೇ ನಮಃ । ಕಮಲ ಪುಷ್ಪಂ ಸಮರ್ಪಯಾಮಿ ॥\\
ಸರ್ವಮಂಗಲಾಯೈ ನಮಃ । ಸಮಸ್ತಪುಷ್ಪಾಣಿ ಸಮರ್ಪಯಾಮಿ ॥

\section{ಆವರಣಪೂಜಾ}
\subsection{ಪ್ರಥಮಾವರಣಮ್}
ನಂದಾಯೈ ನಮಃ।
ರಕ್ತದಂತಿಕಾಯೈ ನಮಃ।
ಶಾಕಂಭರ್ಯೈ ನಮಃ।
ಭೀಮಾಯೈ ನಮಃ।
ಭ್ರಾಮರ್ಯೈ ನಮಃ।
ಶಿವದೂತ್ಯೈ ನಮಃ।\\
\as{ಚಂಡಿಕೇಶ್ವರಿ ಚಾಮುಂಡೇ ದುರ್ಗೇ ದುರ್ಗತಿನಾಶಿನಿ ।\\
ಭಕ್ತ್ತ್ಯಾ ಸಮರ್ಪಯೇ ತುಭ್ಯಂ  ಪ್ರಥಮಾವರಣಾರ್ಚನಮ್ ॥}
\subsection{ದ್ವಿತೀಯಾವರಣಮ್}
ಓಂ ಇಂದ್ರಾಯ ನಮಃ । ಓಂ ಅಗ್ನಯೇ ನಮಃ । ಓಂ ಯಮಾಯ ನಮಃ । ಓಂ ನಿರ್ಋತಯೇ ನಮಃ ।ಓಂ ವರುಣಾಯ ನಮಃ ।ಓಂ ವಾಯವೇ ನಮಃ । ಓಂ ಕುಬೇರಾಯ ನಮಃ । ಓಂ ಈಶಾನಾಯ ನಮಃ । ಬ್ರಹ್ಮಣೇ ನಮಃ । ಅನಂತಾಯ ನಮಃ ॥\\
\as{ಚಂಡಿಕೇಶ್ವರಿ ಚಾಮುಂಡೇ ದುರ್ಗೇ ದುರ್ಗತಿನಾಶಿನಿ ।\\
ಭಕ್ತ್ತ್ಯಾ ಸಮರ್ಪಯೇ ತುಭ್ಯಂ ದ್ವಿತೀಯಾವರಣಾರ್ಚನಮ್ ॥}
\subsection{ತೃತೀಯಾವರಣಮ್}
ಓಂ ಬ್ರಾಹ್ಮ್ಯೈ ನಮಃ । ಓಂ ಮಾಹೇಶ್ವರ್ಯೈ ನಮಃ । ಓಂ ಕೌಮಾರ್ಯೈ ನಮಃ । ಓಂ ವೈಷ್ಣವ್ಯೈ ನಮಃ । ಓಂ ವಾರಾಹ್ಯೈ ನಮಃ । ಓಂ ಇಂದ್ರಾಣ್ಯೈ ನಮಃ । ಓಂ ನಾರಸಿಂಹ್ಯೈ ನಮಃ । ಓಂ ಚಾಮುಂಡಾಯೈ ನಮಃ ।\\

\as{ಚಂಡಿಕೇಶ್ವರಿ ಚಾಮುಂಡೇ ದುರ್ಗೇ ದುರ್ಗತಿನಾಶಿನಿ ।\\
ಭಕ್ತ್ತ್ಯಾ ಸಮರ್ಪಯೇ ತುಭ್ಯಂ ತೃತೀಯಾವರಣಾರ್ಚನಮ್ ॥}
\subsection{ಚತುರ್ಥಾವರಣಮ್}
ಓಂ ಅಸಿತಾಂಗಭೈರವಾಯ ನಮಃ । ಓಂ ರುರುಭೈರವಾಯ ನಮಃ । ಓಂ ಚಂಡಭೈರವಾಯ ನಮಃ । ಓಂ ಕ್ರೋಧಭೈರವಾಯ ನಮಃ । ಓಂ ಉನ್ಮತ್ತಭೈರವಾಯ ನಮಃ । ಓಂ ಕಪಾಲಭೈರವಾಯ ನಮಃ । ಓಂ ಭೀಷಣಭೈರವಾಯ ನಮಃ । ಓಂ ಸಂಹಾರಭೈರವಾಯ ನಮಃ  ॥\\
\as{ಚಂಡಿಕೇಶ್ವರಿ ಚಾಮುಂಡೇ ದುರ್ಗೇ ದುರ್ಗತಿನಾಶಿನಿ ।\\
ಭಕ್ತ್ತ್ಯಾ ಸಮರ್ಪಯೇ ತುಭ್ಯಂ ತೃತೀಯಾವರಣಾರ್ಚನಮ್ ॥}
\subsection{ಪಂಚಮಾವರಣಮ್}
ಅಕ್ಷಮಾಲಾಯೈ ನಮಃ ।
ಅರವಿಂದಾಯ ನಮಃ । ಕಮಲಾಯ ನಮಃ । ಬಾಣಾಯ ನಮಃ । ಅಸಯೇ ನಮಃ । ಕುಲಿಶಾಯ ನಮಃ । ಗದಾಯೈ ನಮಃ । ತ್ರಿಶೂಲಾಯ ನಮಃ । ಪದ್ಮಾಯ ನಮಃ । ಚಕ್ರಾಯ ನಮಃ । ಪರಶವೇ ನಮಃ । ಶಂಖಾಯ ನಮಃ । ಘಂಟಾಯೈ ನಮಃ । ಪಾಶಾಯ ನಮಃ । ಶಕ್ತಯೇ ನಮಃ । ದಂಡಾಯ ನಮಃ । ಪಾನಪಾತ್ರಾಯ ನಮಃ । ಚಾಪಾಯ ನಮಃ । ಧೂಮ್ರಾಚಿಷೇ ನಮಃ । ಜುಷ್ಟ್ಯೈ ನಮಃ\\
\as{ಚಂಡಿಕೇಶ್ವರಿ ಚಾಮುಂಡೇ ದುರ್ಗೇ ದುರ್ಗತಿನಾಶಿನಿ ।\\
ಭಕ್ತ್ತ್ಯಾ ಸಮರ್ಪಯೇ ತುಭ್ಯಂ  ಪಂಚಮಾವರಣಾರ್ಚನಮ್ ॥}\\
ಆವರಣದೇವತಾಭ್ಯೋ ನಮಃ । ಪಂಚೋಪಚಾರ ಪೂಜಾ ॥

\section{ಕಲಾವಾಹನಪೂಜಾ}
ಮಂತ್ರಾಕ್ಷರಮಯೀಂ ಲಕ್ಷ್ಮೀಂ ಮಾತೃಕಾರೂಪಧಾರಿಣೀಮ್ ।\\
ನವದುರ್ಗಾತ್ಮಿಕಾಂ ಸಾಕ್ಷಾತ್ ಪರಬ್ರಹ್ಮಸ್ವರೂಪಿಣೀಮ್ ॥

(ನಾಭೌ)\\
ಧೂಮ್ರಾರ್ಚಿಷೇ   ನಮಃ~। ಊಷ್ಮಾಯೈ~। ಜ್ವಲಿನ್ಯೈ~। ಜ್ವಾಲಿನ್ಯೈ~। ವಿಸ್ಫಲಿಂಗಿನ್ಯೈ ~। ಸುಶ್ರಿಯೈ ~। ಸುರೂಪಾಯೈ ~। ಕಪಿಲಾಯೈ ~। ಹವ್ಯವಾಹಾಯೈ ~। ಕವ್ಯವಾಹಾಯೈ ನಮಃ~॥\\
ದಶಕಲಾತ್ಮನೇ ವಹ್ನಿಮಂಡಲಾಯ ನಮಃ । ದೇವ್ಯಾಃ ನಾಭಿಸ್ಥಿತ ದಶಕಲಾಃ ಪೂಜಯಾಮಿ ॥

(ಹೃದಯೇ)\\
ತಪಿನ್ಯೈ ನಮಃ~। ತಾಪಿನ್ಯೈ~। ಧೂಮ್ರಾಯೈ~। ಮರೀಚ್ಯೈ~। ಜ್ವಾಲಿನ್ಯೈ~। ರುಚ್ಯೈ~। ಸುಷುಮ್ನಾಯೈ~। ಭೋಗದಾಯೈ~। ವಿಶ್ವಾಯೈ~। ಬೋಧಿನ್ಯೈ~। ಧಾರಿಣ್ಯೈ~। ಕ್ಷಮಾಯೈ ನಮಃ~॥\\
ದ್ವಾದಶಕಲಾತ್ಮನೇ ಸೂರ್ಯಮಂಡಲಾಯ ನಮಃ । ದೇವ್ಯಾಃ ಹೃದಿಸ್ಥಿತ ದ್ವಾದಶಕಲಾಃ ಪೂಜಯಾಮಿ ॥

(ಕಂಠೇ)\\
ಅಮೃತಾಯೈ ನಮಃ~। ಮಾನದಾಯೈ~। ತುಷ್ಟ್ಯೈ~। ಪುಷ್ಟ್ಯೈ~। ಶಶಿನ್ಯೈ~। ಚಂದ್ರಿಕಾಯೈ~। ಜ್ಯೋತ್ಸ್ನಾಯೈ~। ಪ್ರೀತ್ಯೈ~। ಅಂಗದಾಯೈ~। ಪೂರ್ಣಾಯೈ~। ಪೂರ್ಣಾಮೃತಾಯೈ ನಮಃ~॥\\
ಷೋಡಶಕಲಾತ್ಮನೇ ಸೋಮಮಂಡಲಾಯ ನಮಃ । ದೇವ್ಯಾಃ ಕಂಠಸ್ಥಿತ ಷೋಡಶಕಲಾಃ ಪೂಜಯಾಮಿ ॥

ಮುಖೋದ್ಭವ ಚತುಃಷಷ್ಟಿಯೋಗಿನೀ ಪೂಜಾಂ ಕರಿಷ್ಯೇ ।
\section{ಚತುಃಷಷ್ಟಿಯೋಗಿನೀ ಪೂಜಾ}
\begin{multicols}{2}
ದಿವ್ಯಯೋಗಿನ್ಯೈ ನಮಃ\\
ಮಹಾಯೋಗಿನ್ಯೈ ನಮಃ\\
ಸಿದ್ಧಯೋಗಿನ್ಯೈ ನಮಃ\\
ಗಣೇಶ್ವರೀಯೋಗಿನ್ಯೈ ನಮಃ\\
ಪ್ರೇತಾಶಿನೀಯೋಗಿನ್ಯೈ ನಮಃ\\
ಡಾಕಿನಿಯೋಗಿನ್ಯೈ ನಮಃ\\
ಕಾಳೀಯೋಗಿನ್ಯೈ ನಮಃ\\
ಕಾಲರಾತ್ರಿಯೋಗಿನ್ಯೈ ನಮಃ\\
ನಿಶಾಚರೀಯೋಗಿನ್ಯೈ ನಮಃ\\
ಝಂಕಾರಿಯೋಗಿನ್ಯೈ ನಮಃ\\
ಊರ್ಧ್ವವೇತಾಳೀಯೋಗಿನ್ಯೈ ನಮಃ\\
ಪಿಶಾಚೀಯೋಗಿನ್ಯೈ ನಮಃ\\
ಭೂತಡಾಮರೀಯೋಗಿನ್ಯೈ ನಮಃ\\
ಊರ್ಧ್ವಕೇಶೀಯೋಗಿನ್ಯೈ ನಮಃ\\
ವಿರೂಪಾಕ್ಷೀಯೋಗಿನ್ಯೈ ನಮಃ\\
ಶುಷ್ಕಾಂಗೀಯೋಗಿನ್ಯೈ ನಮಃ\\
ನರಭೋಜಿನೀ ಯೋಗಿನ್ಯೈ ನಮಃ\\
ರಾಕ್ಷಸಿಯೋಗಿನ್ಯೈ ನಮಃ\\
ಘೋರರಕ್ತಾಕ್ಷಿಯೋಗಿನ್ಯೈ ನಮಃ\\
ವಿಶ್ವರೂಪೀಯೋಗಿನ್ಯೈ ನಮಃ\\
ಭಯಂಕರೀಯೋಗಿನ್ಯೈ ನಮಃ\\
ಚಾಮುಂಡೀಯೋಗಿನ್ಯೈ ನಮಃ\\
ವೀರಕೌಮಾರೀಯೋಗಿನ್ಯೈ ನಮಃ\\
ವಾರಾಹೀಯೋಗಿನ್ಯೈ ನಮಃ\\
ಮುಂಡಧಾರಿಣೀಯೋಗಿನ್ಯೈ ನಮಃ\\
ಭ್ರಾಮರೀಯೋಗಿನ್ಯೈ ನಮಃ\\
ರುದ್ರವೇತಾಲೀಯೋಗಿನ್ಯೈ ನಮಃ\\
ಭೀಷ್ಮರೀ ಯೋಗಿನ್ಯೈ ನಮಃ\\
ತ್ರಿಪುರಾಂತಕಿಯೋಗಿನ್ಯೈ ನಮಃ\\
ಭೈರವ್ಯೈ ನಮಃ\\
ಧ್ವಂಸಿನ್ಯೈ ನಮಃ\\
ಕ್ರೋಧಿನ್ಯೈ ನಮಃ\\
ದುರ್ಮುಖೀಯೋಗಿನ್ಯೈ ನಮಃ\\
ಪ್ರೇತವಾಹಿನೀಯೋಗಿನ್ಯೈ ನಮಃ\\
ಖಟ್ವಾಂಗ್ಯೈ ನಮಃ\\
ದೀರ್ಘಲಂಬೋಷ್ಠ್ಯೈ ನಮಃ\\
ಮಾಲಿನ್ಯೈ ನಮಃ\\
ಮಂತ್ರಯೋಗಿನ್ಯೈ ನಮಃ\\
ಕೌಶಿಕೀಯೋಗಿನ್ಯೈ ನಮಃ\\
ಮರ್ದಿನೀಯೋಗಿನ್ಯೈ ನಮಃ\\
ಯಕ್ಷಿಣೀಯೋಗಿನ್ಯೈ ನಮಃ\\
ರೋಮಜಂಘಾಯೋಗಿನ್ಯೈ ನಮಃ\\
ಪ್ರಹಾರಿಣೀಯೋಗಿನ್ಯೈ ನಮಃ\\
ಕಾಲಾಗ್ನಿಯೋಗಿನ್ಯೈ ನಮಃ\\
ಗ್ರಾಮಣೀಯೋಗಿನ್ಯೈ ನಮಃ\\
ಚಕ್ರಿಣ್ಯೈ ನಮಃ\\
ಕಂಕಾಲ್ಯೈ ನಮಃ\\
ಭುವನೇಶ್ವರ್ಯೈ ನಮಃ\\
ಫಟ್ಕಾರೀಯೋಗಿನ್ಯೈ ನಮಃ\\
ವೀರಭದ್ರೇಶೀಯೋಗಿನ್ಯೈ ನಮಃ\\
ಧೂಮ್ರಾಕ್ಷೀಯೋಗಿನ್ಯೈ ನಮಃ\\
ಕಲಹಪ್ರಿಯಾಯೋಗಿನ್ಯೈ ನಮಃ\\
ಕಂಟಕೀಯೋಗಿನ್ಯೈ ನಮಃ\\
ನಾಟಕೀಯೋಗಿನ್ಯೈ ನಮಃ\\
ಮಾರೀಯೋಗಿನ್ಯೈ ನಮಃ\\
ಯಮದೂತೀಯೋಗಿನ್ಯೈ ನಮಃ\\
ಕರಾಲಿನೀಯೋಗಿನ್ಯೈ ನಮಃ\\
ಸಹಸ್ರಾಕ್ಷೀಯೋಗಿನ್ಯೈ ನಮಃ\\
ಕಾಮಲೋಲಾಯೋಗಿನ್ಯೈ ನಮಃ\\
ಕಾಕದಂಷ್ಟ್ರಾಯೋಗಿನ್ಯೈ ನಮಃ\\
ಅಧೋಮುಖೀಯೋಗಿನ್ಯೈ ನಮಃ\\
ಧೂರ್ಜಟೀಯೈ ನಮಃ\\
ವಿಟಕ್ಯೈ ನಮಃ\\
ಘೋರೀಯೋಗಿನ್ಯೈ ನಮಃ\\
ಕಪಾಲೀಯೋಗಿನ್ಯೈ ನಮಃ\\
ವಿಷಲಂಘಿನೀಯೋಗಿನ್ಯೈ ನಮಃ
\end{multicols}
ಅಸ್ಯ ಶ್ರೀ ದುರ್ಗಾಷ್ಟೋತ್ತರಶತ ದಿವ್ಯನಾಮಸ್ತೋತ್ರಮಹಾಮಂತ್ರಸ್ಯ ಧೌಮ್ಯೋ ಭಗವಾನ್ ಋಷಿಃ । ಶ್ರೀ ನವದುರ್ಗಾ ದೇವತಾ । ಅನುಷ್ಟುಪ್ ಛಂದಃ । ಶ್ರೀ ದುರ್ಗಾಪರಮೇಶ್ವರೀ ಪ್ರೀತ್ಯರ್ಥಂ ಪೂಜನೇ ವಿನಿಯೋಗಃ ॥\\
ಮಾತರ್ಮೇ ಮಧುಕೈಟಭಘ್ನಿ ಮಹಿಷಪ್ರಾಣಾಪಹಾರೋದ್ಯಮೇ \\
ಹೇಲಾನಿರ್ಜಿತಧೂಮ್ರಲೋಚನವಧೇ ಹೇ ಚಂಡಮುಂಡಾರ್ದಿನಿ ।\\
ನಿಶ್ಶೇಷೀಕೃತರಕ್ತಬೀಜದನುಜೇ ನಿತ್ಯೇ ನಿಶುಂಭಾಪಹೇ ।\\
ಶುಮ್ಭಧ್ವಂಸಿನಿ ಸಂಹರಾಶು ದುರಿತಂ ದುರ್ಗೇ ನಮಸ್ತೇಽಂಬಿಕೇ ॥
\begin{multicols}{2} ಓಂ ದುರ್ಗಾಯೈ ನಮಃ ।\\
ಓಂ ದಾರಿದ್ರ್ಯಶಮನ್ಯೈ ನಮಃ ।\\
ಓಂ ದುರಿತಘ್ನ್ಯೈ ನಮಃ ।\\
ಓಂ ಲಕ್ಷ್ಮ್ಯೈ ನಮಃ ।\\
ಓಂ ಲಜ್ಜಾಯೈ ನಮಃ ।\\
ಓಂ ಮಹಾವಿದ್ಯಾಯೈ ನಮಃ ।\\
ಓಂ ಶ್ರದ್ಧಾಯೈ ನಮಃ ।\\
ಓಂ ಪುಷ್ಟ್ಯೈ ನಮಃ ।\\
ಓಂ ಸ್ವಧಾಯೈ ನಮಃ ।\\
ಓಂ ಧ್ರುವಾಯೈ ನಮಃ । ೧೦\\
ಓಂ ಮಹಾರಾತ್ರ್ಯೈ ನಮಃ ।\\
ಓಂ ಮಹಾಮಾಯಾಯೈ ನಮಃ ।\\
ಓಂ ಮೇಧಾಯೈ ನಮಃ ।\\
ಓಂ ಮಾತ್ರೇ ನಮಃ ।\\
ಓಂ ಸರಸ್ವತ್ಯೈ ನಮಃ ।\\
ಓಂ ಶಿವಾಯೈ ನಮಃ ।\\
ಓಂ ಶಶಿಧರಾಯೈ ನಮಃ ।\\
ಓಂ ಶಾಂತಾಯೈ ನಮಃ ।\\
ಓಂ ಶಾಂಭವ್ಯೈ ನಮಃ ।\\
ಓಂ ಭೂತಿದಾಯಿನ್ಯೈ ನಮಃ । ೨೦\\
ಓಂ ತಾಮಸ್ಯೈ ನಮಃ ।\\
ಓಂ ನಿಯತಾಯೈ ನಮಃ ।\\
ಓಂ ನಾರ್ಯೈ ನಮಃ ।\\
ಓಂ ಕಾಲ್ಯೈ ನಮಃ ।\\
ಓಂ ನಾರಾಯಣ್ಯೈ ನಮಃ ।\\
ಓಂ ಕಲಾಯೈ ನಮಃ ।\\
ಓಂ ಬ್ರಾಹ್ಮ್ಯೈ ನಮಃ ।\\
ಓಂ ವೀಣಾಧರಾಯೈ ನಮಃ ।\\
ಓಂ ವಾಣ್ಯೈ ನಮಃ ।\\
ಓಂ ಶಾರದಾಯೈ ನಮಃ । ೩೦\\
ಓಂ ಹಂಸವಾಹಿನ್ಯೈ ನಮಃ ।\\
ಓಂ ತ್ರಿಶೂಲಿನ್ಯೈ ನಮಃ ।\\
ಓಂ ತ್ರಿನೇತ್ರಾಯೈ ನಮಃ ।\\
ಓಂ ಈಶಾನಾಯೈ ನಮಃ ।\\
ಓಂ ತ್ರಯ್ಯೈ ನಮಃ ।\\
ಓಂ ತ್ರಯತಮಾಯೈ ನಮಃ ।\\
ಓಂ ಶುಭಾಯೈ ನಮಃ ।\\
ಓಂ ಶಂಖಿನ್ಯೈ ನಮಃ ।\\
ಓಂ ಚಕ್ರಿಣ್ಯೈ ನಮಃ ।\\
ಓಂ ಘೋರಾಯೈ ನಮಃ । ೪೦\\
ಓಂ ಕರಾಲ್ಯೈ ನಮಃ ।\\
ಓಂ ಮಾಲಿನ್ಯೈ ನಮಃ ।\\
ಓಂ ಮತ್ಯೈ ನಮಃ ।\\
ಓಂ ಮಾಹೇಶ್ವರ್ಯೈ ನಮಃ ।\\
ಓಂ ಮಹೇಷ್ವಾಸಾಯೈ ನಮಃ ।\\
ಓಂ ಮಹಿಷಘ್ನ್ಯೈ ನಮಃ ।\\
ಓಂ ಮಧುವ್ರತಾಯೈ ನಮಃ ।\\
ಓಂ ಮಯೂರವಾಹಿನ್ಯೈ ನಮಃ ।\\
ಓಂ ನೀಲಾಯೈ ನಮಃ ।\\
ಓಂ ಭಾರತ್ಯೈ ನಮಃ । ೫೦\\
ಓಂ ಭಾಸ್ವರಾಂಬರಾಯೈ ನಮಃ ।\\
ಓಂ ಪೀತಾಂಬರಧರಾಯೈ ನಮಃ ।\\
ಓಂ ಪೀತಾಯೈ ನಮಃ ।\\
ಓಂ ಕೌಮಾರ್ಯೈ ನಮಃ ।\\
ಓಂ ಪೀವರಸ್ತನ್ಯೈ ನಮಃ ।\\
ಓಂ ರಜನ್ಯೈ ನಮಃ ।\\
ಓಂ ರಾಧಿನ್ಯೈ ನಮಃ ।\\
ಓಂ ರಕ್ತಾಯೈ ನಮಃ ।\\
ಓಂ ಗದಿನ್ಯೈ ನಮಃ ।\\
ಓಂ ಘಂಟಿನ್ಯೈ ನಮಃ । ೬೦\\
ಓಂ ಪ್ರಭಾಯೈ ನಮಃ ।\\
ಓಂ ಶುಂಭಘ್ನ್ಯೈ ನಮಃ ।\\
ಓಂ ಸುಭಗಾಯೈ ನಮಃ ।\\
ಓಂ ಸುಭ್ರುವೇ ನಮಃ ।\\
ಓಂ ನಿಶುಂಭಪ್ರಾಣಹಾರಿಣ್ಯೈ ನಮಃ ।\\
ಓಂ ಕಾಮಾಕ್ಷ್ಯೈ ನಮಃ ।\\
ಓಂ ಕಾಮುಕಾಯೈ ನಮಃ ।\\
ಓಂ ಕನ್ಯಾಯೈ ನಮಃ ।\\
ಓಂ ರಕ್ತಬೀಜನಿಪಾತಿನ್ಯೈ ನಮಃ ।\\
ಓಂ ಸಹಸ್ರವದನಾಯೈ ನಮಃ । ೭೦\\
ಓಂ ಸಂಧ್ಯಾಯೈ ನಮಃ ।\\
ಓಂ ಸಾಕ್ಷಿಣ್ಯೈ ನಮಃ ।\\
ಓಂ ಶಾಂಕರ್ಯೈ ನಮಃ ।\\
ಓಂ ದ್ಯುತಯೇ ನಮಃ ।\\
ಓಂ ಭಾರ್ಗವ್ಯೈ ನಮಃ ।\\
ಓಂ ವಾರುಣ್ಯೈ ನಮಃ ।\\
ಓಂ ವಿದ್ಯಾಯೈ ನಮಃ ।\\
ಓಂ ಧರಾಯೈ ನಮಃ ।\\
ಓಂ ಧರಾಸುರಾರ್ಚಿತಾಯೈ ನಮಃ ।\\
ಓಂ ಗಾಯತ್ರ್ಯೈ ನಮಃ । ೮೦\\
ಓಂ ಗಾಯಕ್ಯೈ ನಮಃ ।\\
ಓಂ ಗಂಗಾಯೈ ನಮಃ ।\\
ಓಂ ದುರ್ಗಾಯೈ ನಮಃ ।\\
ಓಂ ಗೀತಘನಸ್ವನಾಯೈ ನಮಃ ।\\
ಓಂ ಛಂದೋಮಯಾಯೈ ನಮಃ ।\\
ಓಂ ಮಹ್ಯೈ ನಮಃ ।\\
ಓಂ ಛಾಯಾಯೈ ನಮಃ ।\\
ಓಂ ಚಾರ್ವಾಂಗ್ಯೈ ನಮಃ ।\\
ಓಂ ಚಂದನಪ್ರಿಯಾಯೈ ನಮಃ ।\\
ಓಂ ಜನನ್ಯೈ ನಮಃ । ೯೦\\
ಓಂ ಜಾಹ್ನವ್ಯೈ ನಮಃ ।\\
ಓಂ ಜಾತಾಯೈ ನಮಃ ।\\
ಓಂ ಶಾನ್ಙ್ಕರ್ಯೈ ನಮಃ ।\\
ಓಂ ಹತರಾಕ್ಷಸ್ಯೈ ನಮಃ ।\\
ಓಂ ವಲ್ಲರ್ಯೈ ನಮಃ ।\\
ಓಂ ವಲ್ಲಭಾಯೈ ನಮಃ ।\\
ಓಂ ವಲ್ಲ್ಯೈ ನಮಃ ।\\
ಓಂ ವಲ್ಲ್ಯಲಂಕೃತಮಧ್ಯಮಾಯೈ ನಮಃ ।\\
ಓಂ ಹರೀತಕ್ಯೈ ನಮಃ ।\\
ಓಂ ಹಯಾರೂಢಾಯೈ ನಮಃ । ೧೦೦\\
ಓಂ ಭೂತ್ಯೈ ನಮಃ ।\\
ಓಂ ಹರಿಹರಪ್ರಿಯಾಯೈ ನಮಃ ।\\
ಓಂ ವಜ್ರಹಸ್ತಾಯೈ ನಮಃ ।\\
ಓಂ ವರಾರೋಹಾಯೈ ನಮಃ ।\\
ಓಂ ಸರ್ವಸಿದ್ಧ್ಯೈ ನಮಃ ।\\
ಓಂ ವರಪ್ರದಾಯೈ ನಮಃ ।\\
ಓಂ ಸಿಂದೂರವರ್ಣಾಯೈ ನಮಃ ।\\
ಓಂ ಶ್ರೀದುರ್ಗಾದೇವ್ಯೈ ನಮಃ । ೧೦೮
\end{multicols}
ಇತಿ ಶ್ರೀದುರ್ಗಾಷ್ಟೋತ್ತರಶತನಾಮಾವಲಿಃ ಸಮಾಪ್ತಾ ।

ಅಸ್ಯ ಶ್ರೀ ಆರ್ಯಾಷ್ಟೋತ್ತರಶತ ದಿವ್ಯನಾಮಸ್ತೋತ್ರಮಹಾಮಂತ್ರಸ್ಯ ಪೃಥ್ವೀಧರ ಋಷಿಃ । ಶ್ರೀ ಆರ್ಯಾಪರಮೇಶ್ವರೀದೇವತಾ । ಅನುಷ್ಟುಪ್ ಛಂದಃ । ಶ್ರೀ ಆರ್ಯಾಪರಮೇಶ್ವರೀ ಪ್ರೀತ್ಯರ್ಥಂ ಪೂಜನೇ ವಿನಿಯೋಗಃ ॥\\
ಧ್ಯಾನಮ್\\
ಹೇಮಪ್ರಖ್ಯಾಮಿಂದು ಖಂಡಾಂತಮೌಲಿಂ\\
ಶಂಖಾರಿಷ್ಟಾಭೀತಿ ಹಸ್ತಾಂ ತ್ರಿನೇತ್ರಾಂ ।\\
ಹೇಮಾಬ್ಜಸ್ಥಾಂ ಪೀತವರ್ಣಾಂ ಪ್ರಸನ್ನಾಂ\\
ದೇವೀಂ ದುರ್ಗಾಂ ದಿವ್ಯರೂಪಾಂ ನಮಾಮಿ ॥

\begin{multicols}{2} ಓಂ ಆರ್ಯಾಯೈ ನಮಃ ।\\
ಓಂ ಕಾತ್ಯಾಯನ್ಯೈ ನಮಃ ।\\
ಓಂ ಗೌರ್ಯೈ ನಮಃ ।\\
ಓಂ ಕುಮಾರ್ಯೈ ನಮಃ ।\\
ಓಂ ವಿಂಧ್ಯವಾಸಿನ್ಯೈ ನಮಃ ।\\
ಓಂ ವಾಗೀಶ್ವರ್ಯೈ ನಮಃ ।\\
ಓಂ ಮಹಾದೇವ್ಯೈ ನಮಃ ।\\
ಓಂ ಕಾಲ್ಯೈ ನಮಃ ।\\
ಓಂ ಕಂಕಾಲಧಾರಿಣ್ಯೈ ನಮಃ ।\\
ಓಂ ಘೋಣಸಾಭರಣಾಯೈ ನಮಃ । 10\\
ಓಂ ಉಗ್ರಾಯೈ ನಮಃ ।\\
ಓಂ ಸ್ಥೂಲಜಂಘಾಯೈ ನಮಃ ।\\
ಓಂ ಮಹೇಶ್ವರ್ಯೈ ನಮಃ ।\\
ಓಂ ಖಟ್ವಾಂಗಧಾರಿಣ್ಯೈ ನಮಃ ।\\
ಓಂ ಚಂಡ್ಯೈ ನಮಃ ।\\
ಓಂ ಭೀಷಣಾಯೈ ನಮಃ ।\\
ಓಂ ಮಹಿಷಾಂತಕಾಯೈ ನಮಃ ।\\
ಓಂ ರಕ್ಷಿಣ್ಯೈ ನಮಃ ।\\
ಓಂ ರಮಣ್ಯೈ ನಮಃ ।\\
ಓಂ ರಾಜ್ಞ್ಯೈ ನಮಃ । 20\\
ಓಂ ರಜನ್ಯೈ ನಮಃ ।\\
ಓಂ ಶೋಷಿಣ್ಯೈ ನಮಃ ।\\
ಓಂ ರತ್ಯೈ ನಮಃ ।\\
ಓಂ ಗಭಸ್ತಿನ್ಯೈ ನಮಃ ।\\
ಓಂ ಗಂಧಿನ್ಯೈ ನಮಃ ।\\
ಓಂ ದುರ್ಗಾಯೈ ನಮಃ ।\\
ಓಂ ಗಾಂಧಾರ್ಯೈ ನಮಃ ।\\
ಓಂ ಕಲಹಪ್ರಿಯಾಯೈ ನಮಃ ।\\
ಓಂ ವಿಕರಾಲ್ಯೈ ನಮಃ ।\\
ಓಂ ಮಹಾಕಾಲ್ಯೈ ನಮಃ । 30\\
ಓಂ ಭದ್ರಕಾಲ್ಯೈ ನಮಃ ।\\
ಓಂ ತರಂಗಿಣ್ಯೈ ನಮಃ ।\\
ಓಂ ಮಾಲಿನ್ಯೈ ನಮಃ ।\\
ಓಂ ದಾಹಿನ್ಯೈ ನಮಃ ।\\
ಓಂ ಕೃಷ್ಣಾಯೈ ನಮಃ ।\\
ಓಂ ಛೇದಿನ್ಯೈ ನಮಃ ।\\
ಓಂ ಭೇದಿನ್ಯೈ ನಮಃ ।\\
ಓಂ ಅಗ್ರಣ್ಯೈ ನಮಃ ।\\
ಓಂ ಗ್ರಾಮಣ್ಯೈ ನಮಃ ।\\
ಓಂ ನಿದ್ರಾಯೈ ನಮಃ । 40\\
ಓಂ ವಿಮಾನಿನ್ಯೈ ನಮಃ ।\\
ಓಂ ಶೀಘ್ರಗಾಮಿನ್ಯೈ ನಮಃ ।\\
ಓಂ ಚಂಡವೇಗಾಯೈ ನಮಃ ।\\
ಓಂ ಮಹಾನಾದಾಯೈ ನಮಃ ।\\
ಓಂ ವಜ್ರಿಣ್ಯೈ ನಮಃ ।\\
ಓಂ ಭದ್ರಾಯೈ ನಮಃ ।\\
ಓಂ ಪ್ರಜೇಶ್ವರ್ಯೈ ನಮಃ ।\\
ಓಂ ಕರಾಲ್ಯೈ ನಮಃ ।\\
ಓಂ ಭೈರವ್ಯೈ ನಮಃ ।\\
ಓಂ ರೌದ್ರ್ಯೈ ನಮಃ । 50\\
ಓಂ ಅಟ್ಟಹಾಸಿನ್ಯೈ ನಮಃ ।\\
ಓಂ ಕಪಾಲಿನ್ಯೈ ಚಾಮುಂಡಾಯೈ ನಮಃ ।\\
ಓಂ ರಕ್ತಚಾಮುಂಡಾಯೈ ನಮಃ ।\\
ಓಂ ಅಘೋರಾಯೈ ನಮಃ ।\\
ಓಂ ಘೋರರೂಪಿಣ್ಯೈ ನಮಃ ।\\
ಓಂ ವಿರೂಪಾಯೈ ನಮಃ ।\\
ಓಂ ಮಹಾರೂಪಾಯೈ ನಮಃ ।\\
ಓಂ ಸ್ವರೂಪಾಯೈ ನಮಃ ।\\
ಓಂ ಸುಪ್ರತೇಜಸ್ವಿನ್ಯೈ ನಮಃ ।\\
ಓಂ ಅಜಾಯೈ ನಮಃ । 60\\
ಓಂ ವಿಜಯಾಯೈ ನಮಃ ।\\
ಓಂ ಚಿತ್ರಾಯೈ ನಮಃ ।\\
ಓಂ ಅಜಿತಾಯೈ ನಮಃ ।\\
ಓಂ ಅಪರಾಜಿತಾಯೈ ನಮಃ ।\\
ಓಂ ಧರಣ್ಯೈ ನಮಃ ।\\
ಓಂ ಧಾತ್ರ್ಯೈ ನಮಃ ।\\
ಓಂ ಪವಮಾನ್ಯೈ ನಮಃ ।\\
ಓಂ ವಸುಂಧರಾಯೈ ನಮಃ ।\\
ಓಂ ಸುವರ್ಣಾಯೈ ನಮಃ ।\\
ಓಂ ರಕ್ತಾಕ್ಷ್ಯೈ ನಮಃ । 70\\
ಓಂ ಕಪರ್ದಿನ್ಯೈ ನಮಃ ।\\
ಓಂ ಸಿಂಹವಾಹಿನ್ಯೈ ನಮಃ ।\\
ಓಂ ಕದ್ರವೇ ನಮಃ ।\\
ಓಂ ವಿಜಿತಾಯೈ ನಮಃ ।\\
ಓಂ ಸತ್ಯವಾಣ್ಯೈ ನಮಃ ।\\
ಓಂ ಅರುಂಧತ್ಯೈ ನಮಃ ।\\
ಓಂ ಕೌಶಿಕ್ಯೈ ನಮಃ ।\\
ಓಂ ಮಹಾಲಕ್ಷ್ಮ್ಯೈ ನಮಃ ।\\
ಓಂ ವಿದ್ಯಾಯೈ ನಮಃ ।\\
ಓಂ ಮೇಧಾಯೈ ನಮಃ । 80\\
ಓಂ ಸರಸ್ವತ್ಯೈ ನಮಃ ।\\
ಓಂ ತ್ರ್ಯಂಬಕಾಯೈ ನಮಃ ।\\
ಓಂ ತ್ರಿಸನ್ಖ್ಯಾಯೈ ನಮಃ ।\\
ಓಂ ತ್ರಿಮೂರ್ತ್ಯೈ ನಮಃ ।\\
ಓಂ ತ್ರಿಪುರಾಂತಕಾಯೈ ನಮಃ ।\\
ಓಂ ಬ್ರಾಹ್ಮ್ಯೈ ನಮಃ ।\\
ಓಂ ನಾರಸಿಂಹ್ಯೈ ನಮಃ ।\\
ಓಂ ವಾರಾಹ್ಯೈ ನಮಃ ।\\
ಓಂ ಇಂದ್ರಾಣ್ಯೈ ನಮಃ ।\\
ಓಂ ವೇದಮಾತೃಕಾಯೈ ನಮಃ । 90\\
ಓಂ ಪಾರ್ವತ್ಯೈ ನಮಃ ।\\
ಓಂ ತಾಮಸ್ಯೈ ನಮಃ ।\\
ಓಂ ಸಿದ್ಧಾಯೈ ನಮಃ ।\\
ಓಂ ಗುಹ್ಯಾಯೈ ನಮಃ ।\\
ಓಂ ಇಜ್ಯಾಯೈ ನಮಃ ।\\
ಓಂ ಉಷಾಯೈ ನಮಃ ।\\
ಓಂ ಉಮಾಯೈ ನಮಃ ।\\
ಓಂ ಅಂಬಿಕಾಯೈ ನಮಃ ।\\
ಓಂ ಭ್ರಾಮರ್ಯೈ ನಮಃ ।\\
ಓಂ ವೀರಾಯೈ ನಮಃ । 100\\
ಓಂ ಹಾಹಾಹುಂಕಾರನಾದಿನ್ಯೈ ನಮಃ ।\\
ಓಂ ನಾರಾಯಣ್ಯೈ ನಮಃ ।\\
ಓಂ ವಿಶ್ವರೂಪಾಯೈ ನಮಃ ।\\
ಓಂ ಮೇರುಮಂದಿರವಾಸಿನ್ಯೈ ನಮಃ ।\\
ಓಂ ಶರಣಾಗತದೀನಾರ್ತಪರಿತ್ರಾಣಪರಾಯಣಾಯೈ ನಮಃ ।\\
ಓಂ ತ್ರಿನೇತ್ರಾಯೈ ನಮಃ ।\\
ಓಂ ಶಶಿಧರಾಯೈ ನಮಃ ।\\
ಓಂ ಆರ್ಯಾಯೈ ನಮಃ । 108
\end{multicols}

ಅಥ ಸರಸ್ವತ್ಯಷ್ಟೋತ್ತರ ಶತನಾಮಪೂಜಾಂ ಕರಿಷ್ಯೇ ।
\begin{multicols}{2} ಯೋಗನಿದ್ರಾಯೈ\\
ದೇವಜಾತ್ಯೈ\\
ಶುಂಭಾಯೈ\\
ನಿಶುಂಭಾಯೈ\\
ಶೈಲಜಾತ್ಯೈ\\
ಧೂಮ್ರಾಕ್ಷ್ಯೈ\\
ಚಾಮುಂಡ್ಯೈ\\
ದುರ್ಗಾಯೈ\\
ನವಶಕ್ತ್ಯಾತ್ಮಿಕಾಯೈ\\
ತ್ರಿಗುಣಾತ್ಮಕದುರ್ಗಾಯೈ\\
ಧನಂಜಯಾಯೈ\\
ಸುರಶ್ರೇಷ್ಠಾಯೈ\\
ರಕ್ತದಂತಾಯೈ\\
ಮೃಡಾಯೈ\\
ದ್ವಿರದವಾಸಿನ್ಯೈ\\
ದ್ಯುತ್ಯೈ\\
ರೌದ್ರ್ಯೈ\\
ಮಹಾದೇವ್ಯೈ\\
ಶಂಕರವಲ್ಲಭಾಯೈ\\
ಬ್ರಾಹ್ಮ್ಯೈ\\
ಭೀಮಾಯೈ\\
ಶಿವದೂತ್ಯೈ\\
ಕೌಶಿಕ್ಯೈ\\
ಸರ್ವಶಕ್ತಿಸಮನ್ವಿತಾಯೈ\\
ಕುಮಾರ್ಯೈ\\
ತ್ರಿಮೂರ್ತ್ಯೈ\\
ಕಲ್ಪೋಪಮಾಯೈ\\
ಕಲ್ಪ್ಯೈ\\
ಚಂಡಿಕಾಯೈ\\
ಸುಭದ್ರಾಯೈ\\
ಗಜಾರೂಢಾಯೈ\\
ಸಿಂಹಾರೂಢಾಯೈ\\
ಖಟ್ವಾಂಗಚರ್ಮತ್ರಿಶೂಲಾರವಿಂದಾಯೈ\\
ಡಮರುಗಗದಾಭಯಧಾರಿಣ್ಯೈ\\
ಅಕ್ಷಮಾಲಾರವಿಂದಾಯೈ\\
ಧೃತ್ಯೈ\\
ಕಾಲ್ಯೈ\\
ವೇದಗರ್ಭಾಯೈ\\
ಸ್ವರ್ಣಧಾರಿಣ್ಯೈ\\
ಖಡ್ಗಾಯೈ\\
ವಜ್ರಾಯೈ\\
ತ್ರಿಶೂಲಾಯೈ\\
ಸರ್ಪಾಯೈ\\
ಅಂಬಿಕಾಯೈ\\
ಭವಾಯೈ\\
ಕೌಮಾರ್ಯೈ\\
ವಿದ್ಯಾಯೈ\\
ಸುವಾಸಿನ್ಯೈ\\
ಶುಂಭನಿಶುಂಭನಿಪಾತಿನ್ಯೈ\\
ಮಧುಕೈಟಭಮಹಿಷಾಸುರಮರ್ದಿನ್ಯೈ\\
ಚಂಡಮುಂಡರಕ್ತಬೀಜನಿಹಂತ್ರ್ಯೈ\\
ಜಲದುರ್ಗಾಯೈ\\
ಸ್ಥಲದುರ್ಗಾಯೈ\\
ಗರಿದುರ್ಗಾಯೈ\\
ಅಗ್ನಿದುರ್ಗಾಯೈ\\
ನವದುರ್ಗಾಯೈ\\
ಶ್ರೀದುರ್ಗಾಯೈ\\
ಅನೆಕಾಭೇದ್ಯದುರ್ಗಾಯೈ\\
ಮೂಷಿಕವಾಹಿನ್ಯೈ\\
ಮಯೂರವಾಹಿನ್ಯೈ\\
ರಥಾರೂಢಾಯೈ\\
ಪಂಚಾನನವಾಹಿನ್ಯೈ\\
ರಕ್ತವರ್ಣಾಯೈ\\
ಶುಕ್ಲವರ್ಣಾಯೈ\\
ರೌದ್ರವರ್ಣಾಯೈ\\
ಕ್ರೋಢವರ್ಣಾಯೈ\\
ಪೀತವರ್ಣಾಯೈ\\
ಶೋಣಿತವರ್ಣಾಯೈ\\
ಗೌರವರ್ಣಾಯೈ\\
ಸುವರ್ಣವರ್ಣಾಯೈ\\
ಶಕ್ತಿಶೂಲಗದಾಭಯಧಾರಿಣ್ಯೈ\\
ಶಂಖಚಕ್ರಮಾತುಲುಂಗಧಾರಿಣ್ಯೈ\\
ಬಾಣಕೋದಂಡಖಡ್ಗಧಾರಿಣ್ಯೈ\\
ಅಂಬುಜಧ್ವಜಧಾರಿಣ್ಯೈ\\
ಅಷ್ಟಾದಶಭುಜಾವಲಂಬಿನ್ಯೈ\\
ತಾಪತ್ರಯವಿಧ್ವ್ವಂಸಿನ್ಯೈ\\
ಶತ್ರುಭಂಜಿನ್ಯೈ\\
ಸರ್ವಾರಿಷ್ಟವಿನಾಶಿನ್ಯೈ\\
ತ್ರಿದಶವಂದಿತಾಯೈ\\
ಸರ್ವದೇವಸ್ವರೂಪಿಣ್ಯೈ\\
ಶಾಂಕರ್ಯೈ\\
ರಕ್ತದಂತಿಕಾಯೈ\\
ಅಷ್ಟಕೌಮಾರ್ಯೈ\\
ಜಗತ್ಪ್ರತಿಷ್ಠಿತಾಯೈ\\
ಶತಾಕ್ಷಿಣ್ಯೈ\\
ಪ್ರಕೃತ್ಯೈ\\
ಮಾನ್ಯೈ\\
ತನುಮಧ್ಯಾಯೈ\\
ವಿಷ್ಣುಪ್ರಿಯಾಯೈ\\
ಚೇತನಾಧಿಷ್ಠಾನಾಯೈ\\
ಜಗದಾತ್ಮಿಕಾಯೈ\\
ಅಂಬಿಕಾಯೈ\\
ಶ್ರೇಯೋರೂಪಿಣ್ಯೈ\\
ಸರ್ವಾಶ್ರಯಾಯೈ\\
ಸ್ವಧಾಯೈ\\
ಸದಾತ್ಮಿಕಾಯೈ\\
ಮೇಧಾಯೈ\\
ಪತತ್ರಿಹಂತ್ರಿಣ್ಯೈ\\
ಸ್ವರೂಪಿಣ್ಯೈ\\
ಕ್ಲೇಶಾಯೈ\\
ಕ್ಲೇಶಹಾರಿಣ್ಯೈ\\
ದಾರಿದ್ರ್ಯದುಃಖಭಯಹಾರಿಣ್ಯೈ\\
ಮಹಾಮಾಯಾಯೈ\\
ಮಹಾಲಕ್ಷ್ಮ್ಯೈ\\
ಸರ್ವಾಭರಣಭೂಷಿತಾಯೈ
\end{multicols}
 ಶ್ರೀದುರ್ಗಾಪರಮೇಶ್ವರ್ಯೈ ನಮಃ ।  ಸರಸ್ವತ್ಯಷ್ಟೋತ್ತರ ಶತನಾಮಪೂಜಾಂ ಸಮರ್ಪಯಾಮಿ ॥\\

ಗುಗ್ಗುಲಂ ಘೃತಸಂಯುಕ್ತಂ ನಾನಾಗಂಧೈಃ ಸಮನ್ವಿತಮ್ ।\\
ಧೂಪಂ ಗೃಹಾಣ ದೇವೇಶಿ ಜ್ಞಾನದಾತ್ರಿ ನಮೋಽಸ್ತು ತೇ ॥
\as{ಕರ್ದಮೇನ++++++++ಪದ್ಮಮಾಲಿನೀಮ್ ॥} ಧೂಪಃ ॥

ಷಟ್ಸೂತ್ರ ದಶ ಸಂಯುಕ್ತಂ ಗೋಘೃತೇನ ಸಮನ್ವಿತಂ ।\\
ದೀಪಂ ಗೃಹಾಣ ದೇವೇಶಿ ಭದ್ರಕಾಳಿ ನಮೋಽಸ್ತು ತೇ ॥\\
\as{ಆಪಃ ಸೃಜನ್ತು ++++++++ಮೇ ಕುಲೇ॥}ದೀಪಃ ॥

ಗಾಯತ್ರ್ಯಾ ಸಂಪ್ರೋಕ್ಷ್ಯ -\\
ಪಾಯಸಂ ಸಘೃತಂ ಕ್ಷೌದ್ರಂ ಮಾಷಾಪೂಪಾದಿ ಸಂಯುತಂ ।\\
ಷಡ್ರಸಾನಿ ಚ ಭೋಜ್ಯಾನಿ ವ್ಯಂಜನಾನಿ ಬಹೂನಿ ಚ ॥

ಕದಲೀ ನಾಲಿಕೇರಾಢ್ಯಂ ಸೋಪದಂಶಂ ಸಶರ್ಕರಮ್।\\
ನೈವೇದ್ಯಂ ಗೃಹ್ಯತಾಂ ದೇವಿ ಸುಪ್ರಸನ್ನಾಖಿಲೇಶ್ವರಿ ॥\\
ಮಯಾ ನಿವೇದಿತಂ ಸರ್ವಂ ವಿಶ್ವಮೂರ್ತೇ ನಮೋಽಸ್ತು ತೇ ॥\\
\as{ಆರ್ದ್ರಾಂ ಪುಷ್ಕರಿಣೀಂ ++++ ಮ ಆವಹ॥} ನೈವೇದ್ಯಮ್ ॥

ಗಂಗಾಜಲಂ ಸಮಾನೀತಂ ಸುವರ್ಣಕಲಶೇ ಸ್ಥಿತಂ ।\\
ಪಾನೀಯಂ ಗೃಹ್ಯತಾಂ ದೇವಿ ಪ್ರಸನ್ನಾ ವರದಾ ಭವ ॥\\
\as{ಸ್ವಾದುಃ ಪವಸ್ವ ++++++ಅದಾಭ್ಯಃ ॥}\\
ಮಧ್ಯೇ ಸ್ವಾದೂದಕಂ । ಉತ್ತರಾಪೋಶನಂ । ಬಲಿಹರಣಂ । ಹಸ್ತಪ್ರಕ್ಷಾಳನಂ । ಗಂಡೂಷಂ । ಪುನರಾಮನಂ ।

ತಾಂಬೂಲಂ ಪೂಗಶಕಲಂ ಮೃಗನಾಭಿ ಸಮನ್ವಿತಮ್ ।\\
ಕರ್ಪೂರಚೂರ್ಣಸಂಯುಕ್ತಂ ತಾಂಬೂಲಂ ಪ್ರತಿಗೃಹ್ಯತಾಮ್ ॥

\section{ಅಥ ಅಷ್ಟಲಕ್ಷ್ಮೀಪೂಜಾಂ ಕರಿಷ್ಯೇ ।}
ರಾಜಲಕ್ಷ್ಮ್ಯೈ ನಮಃ । ಭೋಗಲಕ್ಷ್ಮ್ಯೈ ನಮಃ । ವೀರಲಕ್ಷ್ಮ್ಯೈ ನಮಃ । ಆನಂದಲಕ್ಷ್ಮ್ಯೈ ನಮಃ । ಗಜಲಕ್ಷ್ಮ್ಯೈ ನಮಃ । ಸಿಂಧೂರಲಕ್ಷ್ಮ್ಯೈ ನಮಃ । ಅಮೃತಲಕ್ಷ್ಮ್ಯೈ ನಮಃ । ಸತ್ಯಲಕ್ಷ್ಮ್ಯೈ ನಮಃ ।

ದೇವೀಸಮೀಪೇ ಪ್ರಾಙ್ಮುಖಮಂಡಲೇ ಮಾತೃಕಾಬಲಿಂ ದದ್ಯಾತ್ ।\\
ಅಧುನಾ ಶೃಣು ರಾಜೇಂದ್ರ ಬಲಿಪಾತ್ರಂ ಸಮುದ್ಧರೇತ್ ।\\
ಸರ್ವಪೀಠೋಪಪೀಠಾದಿ ದ್ವಾರೋಪಸ್ಥಾನ ಏವ ಚ ॥

ಕ್ಷೇತ್ರೇ ಕ್ಷೇತ್ರಜ್ಞ ಸಂದೋಹ ಸರ್ವದಿಗ್ಭಾಗ ಸಂಸ್ಥಿತಾಃ ।\\
ಯೋಗಿನೀ ಯೋಗ ತತ್ವೇಂದ್ರಾಃ ಸರ್ವತತ್ವೇ ಚ ಕಾರಕಾಃ ॥

ಸಿದ್ಧಪುತ್ರಕ ಆಚಾರ್ಯಾ ಅನ್ಯತ್ರ ಸಮತಿಷ್ಠತ ।\\
ನಗರೇ ವಾ ತತೋ ಗ್ರಾಮೇ ಕಾಂತಾರೇ ಚ ಸರಿತ್ತಟೇ ॥

ವಾಪೀಕೂಪೇಷು ವೃಕ್ಷೇಷು ಸ್ಮಶಾನೇ ಮಾತೃಮಂಡಲೇ ।\\
ನಾನಾರೂಪಧರಾ ಯೇ ಚ ಬಹುರೂಪ ಧರಾ ಅಪಿ ॥

ತೇ ಸರ್ವೇ ಚೈವ ಸಂತುಷ್ಟಾ ಬಲಿಂ ಗೃಹ್ಣಂತು ಸರ್ವದಾ ।\\
ಶರಣಂ ಗತೋಽಸ್ಮ್ಯಹಂ ತೇಷಾಂ ಸಂತು ತೇ ಮೇ ಶುಭಪ್ರದಾಃ ॥

ಸರ್ವೇಭ್ಯೋ ದೇವೇಭ್ಯಃ ಸರ್ವೇಭ್ಯೋ ಭೂತೇಭ್ಯಃ ಇಮಂ ಸದೀಪ ಬಲಿಂ ಗೃಹ್ಣ ಗೃಹ್ಣ ಸ್ವಾಹಾ ॥ ಇತಿ ಬಲಿಂ ದತ್ವಾ -

(ದುರ್ಗೇ ಕಾತ್ಯಾಯನೀ ಚಂಡೀ ಮಹಿಷಾಸುರ ಮರ್ದಿನೀ ।\\
ದೀಪಮಾಲಾಂ ಪ್ರಯಚ್ಛಾಮಿ ಸುಪ್ರೀತಾ ವರದಾ ಭವ ॥)

ನೀರಾಜನಂ ಪ್ರಯಚ್ಛಾಮಿ ವಿಶ್ವದೇವಿ ವರಪ್ರದೇ ।\\
ತ್ರಾಹಿ ಮಾಂ ನರಕಾದ್ಘೋರಾದ್ ಜ್ಞಾನಮೂರ್ತೇ ನಮೋಽಸ್ತು ತೇ ॥\\
\as{ಆರ್ದ್ರಾಂ ಯಃ ಕರಿಣೀಂ+++++ಮ ಆವಹ॥}ನೀರಾಜನಮ್ ॥

ದುರ್ಗೇ ದೇವಿ ನಮಸ್ತುಭ್ಯಂ ನಮಸ್ತ್ರೈಲೋಕ್ಯಪೂಜಿತೇ ।\\
ದಿವ್ಯಪುಷ್ಪ ಸಮಾಯುಕ್ತಂ ಗೃಹಾಣ ಪರಮೇಶ್ವರಿ ॥\\
\as{ಜಾತವೇದಸೇ++++++ತ್ಯಗ್ನಿಃ ॥\\
ದೇವೀಂ ವಾಚ+++++++ಸುಷ್ಟುತೈತು ॥\\
ಪಾವಕಾ ನಃ +++++ರಾಜತಿ ॥\\
ರಾಜಾಧಿರಾಜಾಯ+++++ಮಹೇಶ್ವರಃ ॥}ಮಂತ್ರಪುಷ್ಪಮ್॥

ಶಂಖಚಕ್ರಗದಾಶಾರ್ಙ್ಗಗೃಹೀತಪರಮಾಯುಧೇ ।\\
ಪ್ರಸೀದ ವೈಷ್ಣವೀರೂಪೇ ನಾರಾಯಣಿ ನಮೋಽಸ್ತು ತೇ ॥\\
\as{ತಾಂ ಮ ಆವಹ ++++++ಪುರುಷಾನಹಮ್ ॥}ಪ್ರದಕ್ಷಿಣಮ್॥

ಸರ್ವಸ್ಯ ಬುದ್ಧಿರೂಪೇಣ ಜನಸ್ಯ ಹೃದಿ ಸಂಸ್ಥಿತೇ ।\\
ಸ್ವರ್ಗಾಪವರ್ಗದೇ ದೇವಿ ನಾರಾಯಣಿ ನಮೋಽಸ್ತು ತೇ ॥

ಕಲಾಕಾಷ್ಠಾದಿರೂಪೇಣ ಪರಿಣಾಮಪ್ರದಾಯಿನಿ ।\\
ವಿಶ್ವಸ್ಯೋಪರತೌ ಶಕ್ತೇ ನಾರಾಯಣಿ ನಮೋಽಸ್ತು ತೇ ॥

ಸರ್ವಮಂಗಲಮಾಂಗಲ್ಯೇ ಶಿವೇ ಸರ್ವಾರ್ಥಸಾಧಿಕೇ ।\\
ಶರಣ್ಯೇ ತ್ರ್ಯಂಬಕೇ ಗೌರಿ ನಾರಾಯಣಿ ನಮೋಽಸ್ತು ತೇ ॥

ನಮೋ ದೇವ್ಯೈ ಮಹಾದೇವ್ಯೈ ಶಿವಾಯೈ ಸತತಂ ನಮಃ ।\\
ನಮಃ ಪ್ರಕೃತ್ಯೈ ಭದ್ರಾಯೈ ನಿಯತಾಃ ಪ್ರಣತಾಃ ಸ್ಮ ತಾಮ್। ॥

ರೌದ್ರಾಯೈ ನಮೋ ನಿತ್ಯಾಯೈ ಗೌರ್ಯೈ ಧಾತ್ರ್ಯೈ ನಮೋ ನಮಃ ।\\
ಜ್ಯೋತ್ಸ್ನಾಯೈ ಚೇಂದುರುಪಿಣ್ಯೈ ಸುಖಾಯೈ ಸತತಂ ನಮಃ ॥

ಕಲ್ಯಾಣ್ಯೈ ಪ್ರಣತಾಂ ವೃದ್ಧ್ಯೈ ಸಿದ್ಧ್ಯೈ ಕುರ್ಮೋ ನಮೋ ನಮಃ ।\\
ನೈರ್ಋತ್ಯೈ ಭೂಭೃತಾಂ ಲಕ್ಷ್ಮ್ಯೈ ಶರ್ವಾಣ್ಯೈ ತೇ ನಮೋ ನಮಃ ॥

ದುರ್ಗಾಯೈ ದುರ್ಗಪಾರಾಯೈ ಸಾರಾಯೈ ಸರ್ವಕಾರಿಣ್ಯೈ ।\\
ಖ್ಯಾತ್ಯೈ ತಥೈವ ಕೃಷ್ಣಾಯೈ ಧೂಮ್ರಾಯೈ ಸತತಂ ನಮಃ ॥

ಅತಿಸೌಮ್ಯಾತಿರೌದ್ರಾಯೈ ನತಾಸ್ತಸ್ಯೈ ನಮೋ ನಮಃ ।\\
ನಮೋ ಜಗತ್ಪ್ರತಿಷ್ಠಾಯೈ ದೇವ್ಯೈ ನಮೋ ನಮಃ ॥

ಚತುರ್ಭುಜೇ ಚತುರ್ವಕ್ತ್ರಸಂಸ್ತುತೇ ಪರಮೇಶ್ವರಿ ।\\
ರೂಪಂ ದೇಹಿ ಜಯಂ ದೇಹಿ ಯಶೋ ದೇಹಿ ದ್ವಿಷೋ ಜಹಿ ॥

ಯಾ ದೇವೀ ಸರ್ವಭೂತೇಷು ಬುದ್ಧಿರೂಪೇಣ ಸಂಸ್ಥಿತಾ ।\\
ನಮಸ್ತಸ್ಯೈ ನಮಸ್ತಸ್ಯೈ ನಮಸ್ತಸ್ಯೈ ನಮೋ ನಮಃ ॥

ಯಾ ದೇವೀ ಸರ್ವಭೂತೇಷು ಮಾತೃರೂಪೇಣ ಸಂಸ್ಥಿತಾ ।\\
ನಮಸ್ತಸ್ಯೈ ನಮಸ್ತಸ್ಯೈ ನಮಸ್ತಸ್ಯೈ ನಮೋ ನಮಃ ॥

ಮಾತರ್ಮೇ ಮಧುಕೈಟಭಘ್ನಿ ಮಹಿಷಪ್ರಾಣಾಪಹಾರೋದ್ಯಮೇ \\
ಹೇಲಾನಿರ್ಮಿತಧೂಮ್ರಲೋಚನವಧೇ ಹೇ ಚಂಡಮುಂಡಾರ್ದಿನಿ ।\\
\newpage
ನಿಶ್ಶೇಷೀಕೃತರಕ್ತಬೀಜದನುಜೇ ನಿತ್ಯೇ ನಿಶುಂಭಾಪಹೇ \\
ಶುಮ್ಭಧ್ವಂಸಿನಿ ಸಂಹರಾಶು ದುರಿತಂ ದುರ್ಗೇ ನಮಸ್ತೇಽಂಬಿಕೇ ॥\\
\as{ಯಃ ಶುಚಿಃ ++++++++ಸತತಂ ಜಪೇತ್ ॥}ನಮಸ್ಕಾರಾಃ ॥

ಕುಂಕುಮೇನ ಸಮಾಯುಕ್ತಂ ಚಂದನೇನ ವಿಮಿಶ್ರಿತಂ ।\\
ಬಿಲ್ವಪತ್ರೇಣ ಸಹಿತಂ ಗೃಹಾಣಾರ್ಘ್ಯಂ ನಮೋಽಸ್ತು ತೇ ॥\\
\as{ಕಾತ್ಯಾಯನಾಯ ವಿದ್ಮಹೇ ಕನ್ಯಕುಮಾರಿ ಧೀಮಹಿ ।\\ ತನ್ನೋ ದುರ್ಗಿಃ ಪ್ರಚೋದಯಾತ್ ॥}\\
ಮಹಾಕಾಲ್ಯೈ ದುರ್ಗ್ಯೈ ನಮಃ । ಇದಮರ್ಘ್ಯಮರ್ಘ್ಯಮ್ ।

ಕ್ಷೀರವಾರಿಧಿಜೇ ತುಭ್ಯಂ ನಮೋ ಭಾರ್ಗವ ನಂದಿನಿ ।\\
ಸರ್ವದೇವಸ್ವರೂಪಿಣ್ಯೈ ಮಹಾಲಕ್ಷ್ಮ್ಯೈ ನಮೋ ನಮಃ ॥ \\
\as{ಮಹಾದೇವ್ಯೈ ಚ ವಿದ್ಮಹೇ ವಿಷ್ಣುಪತ್ನ್ಯೈ ಚ ಧೀಮಹಿ ।\\ ತನ್ನೋ ಲಕ್ಷ್ಮೀಃ ಪ್ರಚೋದಯಾತ್ ॥}\\
ಮಹಾಲಕ್ಷ್ಮ್ಯೈ ದುರ್ಗ್ಯೈ ನಮಃ । ಇದಮರ್ಘ್ಯಮರ್ಘ್ಯಮ್ ।

ದಾಮೋದರಿ ನಮಸ್ತುಭ್ಯಂ ನಮಸ್ತ್ರೈಲೋಕ್ಯನಾಯಕಿ ।\\
ನಮಸ್ತೇಽಸ್ತು ಮಹಾ ದೇವಿ ತ್ರಾಹಿ ಮಾಂ ಚ ಸರಸ್ವತಿ ॥\\
\as{ಬ್ರಹ್ಮಪತ್ನ್ಯೈ ಚ ವಿದ್ಮಹೇ ವೇದಮಾತ್ರೇ ಚ ಧೀಮಹಿ ।\\ ತನ್ನಃ ಸರಸ್ವತೀ ಪ್ರಚೋದಯಾತ್ ॥}\\
ಮಹಾಸರಸ್ವತ್ಯೈ ದುರ್ಗ್ಯೈ ನಮಃ । ಇದಮರ್ಘ್ಯಮರ್ಘ್ಯಮ್ ।\\
ಪ್ರಸನ್ನಾರ್ಘ್ಯಮ್॥

ನೃತ್ಯಂ ಗೀತಂ ಚ ವಾದ್ಯಂ ಚ ಛತ್ರಚಾಮರಮೇವ ಚ ।\\
ಆಂದೋಲನಾದಿಕಂ ಕೃತ್ವಾ ಉಪಚಾರಾನ್ ಪ್ರಕಲ್ಪಯೇತ್ ॥

ಬ್ರಾಹ್ಮಣಪೂಜಾ - ಅಥೋಪಾಯನದಾನಂ ।

ಮಹಾಕಾಳೀ ಮಹಾಲಕ್ಷ್ಮೀ ಮಹಾಮಾಯೇ ಸರಸ್ವತಿ ।\\
ದುರ್ಗೇ ಮಾತರ್ನಮಸ್ತೇಽಸ್ತು ಸಂಪೂರ್ಣಂ ಕುರು ಮೇ ಶುಭೇ ॥

ಸ್ತವೇನ ಸ್ತೋತ್ರಮುಖ್ಯೇನ ಸಾಷ್ಟಾಂಗಂ ಪ್ರಣಮೇತ್ತತಃ ।\\
ಕೃತಾಂಜಲಿಪುಟಸ್ತಿಷ್ಠನ್ಪ್ರಾರ್ಥಯೇಚ್ಚಂಡಿಕಾಂ ಶಿವಾಂ ॥

ಸಪ್ತದ್ವೀಪ ಸಮುದ್ರಾಂತ ಕುಲಾಚಲ ನಿವಾಸಿನಿ ।\\
ಯದ್ಗಾಂ ಪ್ರದಕ್ಷಿಣೀಕೃತ್ಯ ತತ್ಪುಣ್ಯಂ ಕುರು ಮೇ ಶುಭೇ ॥

ಜಯ ರುದ್ರೇ ವಿರೂಪಾಕ್ಷಿ ಜಯ ನಿತ್ಯೇ ನಿರಂಜನೇ ।\\ಜಯ ಕಲ್ಯಾಣಶುಭದೇ ಜಯ ಹೇಮವಿಭಾಸಿತೇ ॥

ಜಯ ಸಿದ್ಧಮುನೀನ್ದ್ರಾದಿ ವನ್ದಿತಾಂಘ್ರಿಸರೋರುಹೇ ।\\ಜಯ ರುದ್ರಪ್ರಿಯೇ ಭೂಮೇ ಜಯ ದುಃಖ ನಿಬರ್ಹಿಣಿ ॥

ಜಯ ವಿಷ್ಣುಪ್ರಿಯೇ ದೇವಿ ಜಯಲೋಕವಿಭೂಷಿತೇ ।\\ ಜಯ ರತ್ನಪ್ರದೀಪಾಭೇ ಜಯ ಹೇಮವಿಭಾಸಿತೇ ।

ಜಯ ಸರ್ವೇಶ್ವರಿ ಜಯ ಜ್ಞಾನಗಮ್ಯೇ ಜಿತೇಂದ್ರಿಯೇ ।\\ ಜಯ ಬಾಲೇನ್ದುತಿಲಕೇ ತ್ರ್ಯಮ್ಬಕೇ ಜಯ ವೃದ್ಧಿದೇ ॥

ಸರ್ವಲಕ್ಷ್ಮೀಪ್ರದೇ ದೇವಿ ಸರ್ವರಕ್ಷಾಪ್ರದಾ ಭವ ।\\ಧರ್ಮಾರ್ಥಕಾಮಮೋಕ್ಷಾಖ್ಯ ಚತುರ್ವರ್ಗಫಲಪ್ರದೇ ॥

ಶೈಲಪುತ್ರಿ ನಮಸ್ತೇಽಸ್ತು ಬ್ರಹ್ಮಚಾರಿಣಿ ತೇ ನಮಃ ।\\ಮಧುಕೈಟಭಹಾರಿಣ್ಯೈ ನಮೋ ಮಹಿಷಮರ್ದಿನಿ ॥
\newpage
ಧೂಮ್ರಲೋಚನನಿರ್ನಾಶೇ ಚಂಡಮುಂಡನಿಪಾತಿನಿ ॥\\ ರಕ್ತಬೀಜವಧೇ ದೇವಿ ಶುಂಭಾಸುರ ನಿಬರ್ಹಿಣಿ ॥\\
ನಮೋ ನಿಶುಂಭ ಸಂಹಾರೇ ತ್ರೈಲೋಕ್ಯವರದಾ ಭವ ॥

ದೇಹಿ ದೇವಿ  ಪರಂ ಜ್ಞಾನಂ ದೇವಿ ದೇಹಿ ಪರಂ ಸುಖಮ್ ।\\ಧರ್ಮಂ ದೇಹಿ ಧನಂ ದೇಹಿ ಸರ್ವಕಾಮಾಂಶ್ಚ ದೇಹಿ ಮೇ ॥

ಸುಪುತ್ರಾಂಶ್ಚ ಬಹೂನ್ ಕೋಶಾನ್ ಸುಕ್ಷೇತ್ರಾಣಿ ಸುಖಾನಿ ಚ ।\\ದೇವಿ ದೇಹಿ ಪರಂ ರೂಪಮಿಹ ಭುಕ್ತಿಂ ಸುಖಂ ಕುರು ॥

ಯದ್ಯದಿಚ್ಛಾಮ್ಯಹಂ ನಿತ್ಯಂ ತತ್ತತ್ಸರ್ವಂ ಪ್ರಯಚ್ಛ ಮೇ ।\\ನಾಮ್ನಾಮಷ್ಟೋತ್ತರಶತೈಃ ಸಹಸ್ರೈರ್ವಾ ಯಜೇತ್ಸುಧೀಃ ॥

ಲಕ್ಷ್ಮಿ ತ್ವಾಂ ಪೂಜಯೇ ನಿತ್ಯಂ ಕೃತಾ ಪೂಜಾ ತವಾಜ್ಞಯಾ । \\ಸ್ಥಿರಾ ಭವ ಗೃಹೇ ನಿತ್ಯಂ ಮಮ ಸಂತಾನಕಾರಿಣೀ ॥

ನ್ಯೂನಂ ವಾಪ್ಯಧಿಕಂ ವಾಪಿ ಯನ್ಮಯಾ ಮೋಹತಃ ಕೃತಂ ।\\ಸರ್ವಂ ತದಸ್ತು ಸಂಪೂರ್ಣ ತ್ವತ್ಪ್ರಸಾದಾತ್ ಮಹೇಶ್ವರಿ ॥

ಶವಾಧಿರೂಢಾಂನೃಕಪಾಲ ಹಸ್ತಾಂ\\ ಶೂಲಾಯುಧಾಂ ಭೂತಿಸಿತಾಂಗಯಷ್ಟಿಂ ।\\
ಉಲೂಕಚಿಹ್ನಾಂ ನರರುಂಡಮಾಲಾಂ\\ ನಮಾಮಿ ದೇವೀಂ ರುಧಿರಂ ಪಿಬಂತೀಮ್ ॥

ಪೂಜಾಜಪಾಗ್ನಿಕಾರ್ಯಾದ್ಯೈಃ ಸುಕೃತಂ ಯನ್ಮಯಾರ್ಜಿತಂ ।\\ತತ್ಸರ್ವಂ ಸಫಲಂ ಮೇಽಸ್ತು ಭುಕ್ತಿಂ ಮುಕ್ತಿಂ ಚ ಸಾಧಯ ॥
\newpage
\section{ಪುನಃ ಪ್ರಸನ್ನಪೂಜಾಂ ಕರಿಷ್ಯೇ}
ಉಮಾಯೈ ನಮಃ । \as{ಧ್ಯಾನಂ}\\
ಕಾತ್ಯಾಯನ್ಯೈ ನಮಃ । \as{ಆವಾಹನಮ್}\\
ಮಹಿಷಮರ್ದಿನ್ಯೈ ನಮಃ । \as{ಆಸನಮ್}\\
ದೇವ್ಯೈ ನಮಃ । \as{ಪಾದ್ಯಮ್}\\
ಚಂಡಿಕಾಯೈ ನಮಃ । \as{ಅರ್ಘ್ಯಮ್}\\
ಶಿವದೂತ್ಯೈ ನಮಃ । \as{ಆಚಮನಮ್}\\
ದೇವ್ಯೈ ನಮಃ । \as{ಪಂಚಾಮೃತಮ್}\\
ಕಾಲ್ಯೈ ನಮಃ । \as{ಸ್ನಾನಮ್}\\
ಈಶ್ವರ್ಯೈ ನಮಃ । \as{ವಸ್ತ್ರಮ್}\\
ವಿಭೂತ್ಯೈ ನಮಃ । \as{ಉಪವೇತಮ್}\\
ಶ್ರಿಯೈ ನಮಃ । \as{ಕಂಚುಕಾವರಣಮ್}\\
ಮಂಗಲಾಯೈ ನಮಃ । \as{ಆಭರಣಮ್}\\
ಕಾಂತ್ಯೈ ನಮಃ । \as{ಗಂಧಮ್}\\
ವರದಾಯೈ ನಮಃ । \as{ಅಕ್ಷತಾನ್}\\
ಪದ್ಮಾವತ್ಯೈ ನಮಃ । \as{ಪುಷ್ಪಮ್}\\
ಕಾಲ್ಯೈ ನಮಃ । \as{ಧೂಪಮ್}\\
ಲಕ್ಷ್ಮ್ಯೈ ನಮಃ । \as{ದೀಪಮ್}\\
ಸರ್ವೇಶ್ವರ್ಯೈ ನಮಃ । \as{ನೈವೇದ್ಯಮ್}\\
ಸರಸ್ವತ್ಯೈ ನಮಃ । \as{ಹಸ್ತಪ್ರಕ್ಷಾಳನಮ್}\\
ಜ್ಞಾನದಾತ್ರ್ಯೈ ನಮಃ । \as{ಪುನರಾಚಮನಮ್}\\
ಚಾಮುಂಡಾಯೈ ನಮಃ । \as{ತಾಂಬೂಲಮ್}\\
ಶತಾಕ್ಷ್ಯೈ ನಮಃ । \as{ನೀರಾಜನಮ್}\\
ಹೈಮವತ್ಯೈ ನಮಃ । \as{ಮಂತ್ರಪುಷ್ಪಮ್}\\
ಪಾರ್ವತ್ಯೈ ನಮಃ । \as{ಪ್ರದಕ್ಷಿಣಮ್}\\
ಶಾರದಾಯೈ ನಮಃ । \as{ನಮಸ್ಕಾರಾನ್}\\
ದಾಕ್ಷಾಯಣ್ಯೈ ನಮಃ । \as{ಛತ್ರಮ್}\\
ಭವಾನ್ಯೈ ನಮಃ । \as{ಚಾಮರಮ್}\\
ಮಾಲಿನ್ಯೈ ನಮಃ । \as{ಪಾದುಕಾಮ್}\\
ಸುಂದರ್ಯೈ ನಮಃ । \as{ವ್ಯಜನಮ್}\\
ಮಧುರಾಯೈ ನಮಃ । \as{ನಾಟ್ಯಮ್}\\
ಈಶ್ವರ್ಯೈ ನಮಃ । \as{ನೃತ್ಯಮ್}\\
ಆರ್ಯಾಯೈ ನಮಃ । \as{ಗೀತಮ್}\\
ದಿವ್ಯಚಕ್ಷುಷೇ ನಮಃ । \as{ದರ್ಪಣಮ್}\\
ಭವಾನ್ಯೈ ನಮಃ । \as{ಆಂದೋಲನಮ್}\\
ಸರ್ವೇಶ್ವರ್ಯೈ ನಮಃ । \as{ಸರ್ವೋಪಚಾರಪೂಜಾಂ} ಸಮರ್ಪಯಾಮಿ

ಪೂಜಾದಕ್ಷಿಣಾ॥\\
ಯಸ್ಯ ಸ್ಮೃತ್ಯಾ ಚ ನಮೋಕ್ತ್ಯಾ ತಪಃ ಪೂಜಾ ಕ್ರಿಯಾದಿಷು ।\\
ನ್ಯೂನಂ ಸಂಪೂರ್ಣತಾಂ ಯಾತಿ ಸದ್ಯೋ ವಂದೇ ತಮಚ್ಯುತಮ್ ॥

ಮಂತ್ರಹೀನಂ ಕ್ರಿಯಾಹೀನಂ ಭಕ್ತಿಹೀನಂ ಮಹೇಶ್ವರಿ ।\\
ಯತ್ಪೂಜಿತಂ ಮಯಾ ದೇವಿ ಪರಿಪೂರ್ಣಂ ತದಸ್ತು ಮೇ ॥

ವರ್ತಮಾನೇ ವ್ಯಾವಹಾರಿಕೇ****** ಶುಭತಿಥೌ ** ಕಾಲೇ 
ಮಯಾ ಕೃತ ಸ್ಕಾಂದಪುರಾಣೋಕ್ತ ದುರ್ಗಾವ್ರತ ಕಲ್ಪೋಕ್ತ ಧ್ಯಾನಾವಾಹನಾದಿ\\ 
ಪೂಜಾರಾಧನೇನ ಶ್ರೀ ಮಹಾಕಾಳೀ ಮಹಾಲಕ್ಷ್ಮೀ ಮಹಾಸರಸ್ವತ್ಯಾತ್ಮಿಕಾ\\ ಶ್ರೀದುರ್ಗಾಪರಮೇಶ್ವರೀ ಪ್ರೀಯತಾಮ್ ॥

ಮಧ್ಯೇ ಮಂತ್ರ ತಂತ್ರೇತ್ಯಾದಿ ಪ್ರಸಾದಗ್ರಹಣಾಚಮನಾಂತಮ್ ॥

\authorline{ಓಂತತ್ಸತ್}
