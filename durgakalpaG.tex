\part*{ಗುರುಪೂಜಾಕಲ್ಪಃ\\ಶರನ್ನವರಾತ್ರಪೂಜಾಕಲ್ಪಃ}
\chapter*{\center ಗುರುಪೂಜಾ  \\ \Large ಶ್ರೀಗುರುಭ್ಯೋ ನಮಃ\\
\thispagestyle{empty}
ಶ್ರೀ ಗಣೇಶಾಯ ನಮಃ}

ಪ್ರತಿಪದಾದಿ ನವಮೀ ಪರ್ಯಂತಂ ಕ್ರಿಯಮಾಣ ತ್ರಿಶಕ್ತ್ಯಾತ್ಮಕ ಶ್ರೀಮಹಾಕಾಳೀ ಮಹಾಲಕ್ಷ್ಮೀ ಮಹಾಸರಸ್ವತ್ಯಾತ್ಮಕ ಶ್ರೀ ದುರ್ಗಾಪರಮೇಶ್ವರೀ ಪೂಜಾಂಗತಯಾ  ಶ್ರೀ ಗುರು ಪೂಜಾಂ ಗಣಪತಿ ಪೂಜಾಂ ಚ ಕರಿಷ್ಯೇ । ಇತಿ ಸಂಕಲ್ಪ್ಯ\\
ತದಂಗತಯಾ ಕಲಶಪೂಜಾಂ ಶಂಖಪೂಜಾಂ ಚ ಕೃತ್ವಾ \\ಆತ್ಮಾರ್ಚನಾದಿ ನವಶಕ್ತಿ ಪೂಜಾಂತೇ

(ಉಪದೇಶೇ ಸತಿ ಹಂಸಗಾಯತ್ರೀಂ ಪ್ರಜಪ್ಯ)\\
ಅಥ ಯಾವಚ್ಛಕ್ತಿ ಧ್ಯಾನಾವಾಹನಾದಿ ಷೋಡಶೋಪಚಾರಪೂಜಾಂ ಕರಿಷ್ಯೇ ॥

ಗುರುರ್ಬ್ರಹ್ಮಾ ಗ್ರುರುರ್ವಿಷ್ಣುಃ ಗುರುರ್ದೇವೋ ಮಹೇಶ್ವರಃ~।\\
ಗುರುಃ ಸಾಕ್ಷಾತ್ ಪರಂ ಬ್ರಹ್ಮ ತಸ್ಮೈ ಶ್ರೀ ಗುರವೇ ನಮಃ ॥

ಸಚ್ಚಿದಾನಂದ ರೂಪಾಯ ಬಿಂದು ನಾದಾಂತರಾತ್ಮನೇ~।\\
ಆದಿಮಧ್ಯಾಂತ ಶೂನ್ಯಾಯ ಗುರೂಣಾಂ ಗುರವೇ ನಮಃ ॥
\newpage
ಶುದ್ಧಸ್ಫಟಿಕಸಂಕಾಶಂ ದ್ವಿನೇತ್ರಂ ಕರುಣಾನಿಧಿಂ~।\\
ವರಾಭಯಪ್ರದಂ ವಂದೇ ಶ್ರೀಗುರುಂ ಶಿವರೂಪಿಣಂ ॥

ಸಹಸ್ರದಲಪಂಕಜೇ ಸಕಲಶೀತರಶ್ಮಿಪ್ರಭಂ\\
ವರಾಭಯಕರಾಂಬುಜಂ ವಿಮಲಗಂಧಪುಷ್ಪಾಂಬರಂ~।\\
ಪ್ರಸನ್ನವದನೇಕ್ಷಣಂ ಸಕಲದೇವತಾರೂಪಿಣಂ\\
ಸ್ಮರೇಚ್ಛಿರಸಿ ಹಂಸಗಂ ತದಭಿಧಾನಪೂರ್ವಂ ಗುರುಂ ॥
 
ಹೃದಂಬುಜೇ ಕರ್ಣಿಕಮಧ್ಯಸಂಸ್ಥೇ\\
ಸಿಂಹಾಸನೇ ಸಂಸ್ಥಿತದಿವ್ಯಮೂರ್ತಿಂ ।\\
ಧ್ಯಾಯೇದ್ಗುರುಂ ಚಂದ್ರಕಲಾಪ್ರಕಾಶಂ\\
ಚಿತ್ಪುಸ್ತಕಾಭೀಷ್ಟವರಂ ದಧಾನಂ ॥\as{ಧ್ಯಾನಮ್॥}

 ಆಗಚ್ಛ ದೇಶಿಕೇಂದ್ರ ತ್ವಂ ಸಹಸ್ರದಲಪಂಕಜೇ ।\\
 ಸ್ವಯಂ ಪ್ರಕಾಶ ಚಿನ್ಮೂರ್ತೇ ಪೂಜಯಾಮಿ ಗುರೂತ್ತಮ ॥\as{ಆವಾಹನಮ್॥}

ಸಹಸ್ರಾರ ಸರೋಜಾತ ಮಧ್ಯೇ ಮಣಿವಿರಾಜಿತಮ್ ।\\
ರತ್ನಸಿಂಹಾಸನಂ ಚಾರು ಗೃಹ್ಯತಾಂ ಗುರುಪುಂಗವ ॥\as{ಆಸನಂ॥}

ಪದ್ಮ ಪಲ್ಲವ ಸಂಕಾಶ ಪಾದಪದ್ಮ ವಿರಾಜಿತ ।\\
ಪುರುಷೋತ್ತಮ ಹೇ ನಾಥ ಪಾದ್ಯಂ ತೇ ಕಲ್ಪಯಾಮ್ಯಹಮ್ ॥\as{ಪಾದ್ಯಂ॥} 

ಗಂಗಾದಿ ಸರ್ವತೀರ್ಥೇಭ್ಯೋ ಗಂಧಾಕ್ಷತ ವಿಮಿಶ್ರಿತೈಃ ।\\
ಆಹೃತೈಃ ಸಲಿಲೈ ರರ್ಘ್ಯಂ ಕಲ್ಪಯಾಮಿ ಗೃಹಾಣ ಭೋ ॥\as{ಅರ್ಘ್ಯಂ॥}

ಶೀತಲಂ ವಿಮಲಂ ತೋಯಂ ಕರ್ಪೂರಾದಿ ಸುವಾಸಿತಂ ।\\
ಆಚಮ್ಯತಾಂ ಗುರುಶ್ರೇಷ್ಠ ಮಯಾ ದತ್ತಂ ಚ ಭಕ್ತಿತಃ ॥\as{ಆಚಮನಮ್॥}
\newpage
ಗಾಂಗವಾರಿ ಮನೋಹಾರಿ ಸರ್ವಪಾಪಹರಂ ಶುಭಂ ।\\
ನಿರ್ಮಲಾಯ ಗುರೋ ತುಭ್ಯಂ ಸ್ನಾನಾರ್ಥಂ ಪ್ರದದಾಮ್ಯಹಮ್ ॥\as{ಸ್ನಾನಮ್॥}

ಪಂಚಭೂತವಿಕಾರಾಭಂ ಪಂಚಕೋಶ ಪ್ರಪೂರಿತಮ್ ।\\
ಪಂಚಾಮೃತಂ ಪ್ರದಾಸ್ಯಾಮಿ ಪ್ರಪಂಚ ಭ್ರಾಂತಿಶಾಂತಯೇ ॥\\\as{ಪಂಚಾಮೃತಮ್॥}

ಕಾಮಧೇನು ಸಮುತ್ಪನ್ನಂ ಸರ್ವೇಷಾಂ ಜೀವನಂ ಪರಂ ।\\
ಪಾವನಂ ಯಜ್ಞಹೇತುಶ್ಚ ಪಯಃ ಸ್ನಾನಾರ್ಥಮರ್ಪಿತಮ್ ॥\\
ಯದಂಘ್ರಿಕಮಲದ್ವಂದ್ವಂ ದ್ವಂದ್ವತಾಪನಿವಾರಕಮ್ ।\\
ತಾರಕಂ ಭವಸಿಂಧೋಶ್ಚ ತಂ ಗುರುಂ ಪ್ರಣಮಾಮ್ಯಹಮ್ ॥\as{ಕ್ಷೀರಸ್ನಾನಮ್॥}

ಪಯಸಾ ತು ಸಮುತ್ಪನ್ನಂ ಮಧುರಾಮ್ಲ ಶಶಿಪ್ರಭಂ ।\\
ದಧ್ಯಾನೀತಂ ಮಯಾ ಸ್ವಾಮಿನ್ ಸ್ನಾನಾರ್ಥಂ ಪ್ರತಿಗೃಹ್ಯತಾಮ್ ॥\\
ಶೋಷಣಂ ಪಾಪಪಂಕಸ್ಯ ದೀಪನಂ ಜ್ಞಾನತೇಜಸಃ ।\\
ಗುರೋಃ ಪಾದೋದಕಂ ಸಮ್ಯಕ್ ಸಂಸಾರಾರ್ಣವತಾರಕಮ್ ॥\as{ದಧಿಸ್ನಾನಮ್॥}

ನವನೀತ ಸಮುತ್ಪನ್ನಂ ಆಯುರಾರೋಗ್ಯ ವರ್ಧನಂ ।\\
ಘೃತಂ ತುಭ್ಯಂ ಪ್ರದಾಸ್ಯಾಮಿ ಸ್ನಾನಾರ್ಥಂ ಪ್ರತಿಗೃಹ್ಯತಾಂ ॥\\
ತಾಪತ್ರಯಾಗ್ನಿ ತಪ್ತಾನಾಮಶಾಂತ ಪ್ರಾಣಿನಾಂ ಮುದೇ ।\\
ಗುರುರೇವ ಪರಾ ಗಂಗಾ ತಸ್ಮೈ ಶ್ರೀಗುರವೇ ನಮಃ ॥\as{ಘೃತಸ್ನಾನಮ್॥}

ತರುಪುಷ್ಪಸಮಾಕೃಷ್ಟಂ ಸುಸ್ವಾದು ಮಧುರಂ ಮಧು ।\\
ತೇಜಃಪುಷ್ಟಿಕರಂ ದಿವ್ಯಂ ಸ್ನಾನಾರ್ಥಂ ಪ್ರತಿಗೃಹ್ಯತಾಂ ॥ \\
ಅಜ್ಞಾನತಿಮಿರಾಂಧಸ್ಯ ಜ್ಞಾನಾಂಜನಶಲಾಕಯಾ ।\\
ಚಕ್ಷುರುನ್ಮೀಲಿತಂ ಯೇನ ತಸ್ಮೈ ಶ್ರೀಗುರವೇ ನಮಃ ॥\as{ಮಧುಸ್ನಾನಮ್॥}

ಇಕ್ಷಸಾರಸಮುದ್ಭೂತಾ ಶರ್ಕರಾ ಪುಷ್ಟಿಕಾರಿಕಾ ।\\
ಮಲಾಪಹಾರಿಕಾ ದಿವ್ಯಾ ಸ್ನಾನಾರ್ಥಂ ಪ್ರತಿಗೃಹ್ಯತಾಂ ॥\\
ಯತ್ಪಾದಾಂಬುಜರೇಣುರ್ವೈ ಕೋಽಪಿ ಸಂಸಾರವಾರಿಧೌ ।\\
ಸೇತುಬಂಧಾಯತೇ ನಾಥಂ ದೇಶಿಕಂ ತಮುಪಾಸ್ಮಹೇ ॥\as{ಶರ್ಕರಾಸ್ನಾನಮ್॥}

ಸರ್ವಸಾರಸಮುದ್ಭೂತಂ ಶಕ್ತಿಪುಷ್ಟಿಕರಂ ದೃಢಂ ।\\
ಸುಫಲಂ ಕಾರ್ಯಸಿದ್ಧ್ಯರ್ಥಂ ಸ್ನಾನಾರ್ಥಂ ಪ್ರತಿಗೃಹ್ಯತಾಂ ॥\\
ಅಖಂಡಮಂಡಲಾಕಾರಂ ವ್ಯಾಪ್ತಂ ಯೇನ ಚರಾಚರಮ್ ।\\
ತತ್ಪದಂ ದರ್ಶಿತಂ ಯೇನ ತಸ್ಮೈ ಶ್ರೀಗುರವೇ ನಮಃ ॥\as{ಫಲಸ್ನಾನಮ್॥}

ಮಲಯಾಚಲಸಂಭೂತ ಸುಗಂಧೈರ್ವಾಸಿತಂ ಜಲಮ್ ।\\
ಸುಕಾಂತಿದಾಯಕಂ ದಿವ್ಯಂ ಸ್ನಾನಾರ್ಥಂ ಪ್ರತಿಗೃಹ್ಯತಾಂ ॥\\\as{ಗಂಧೋದಕಸ್ನಾನಮ್॥}

ಪೃಥಿವೀಗರ್ಭ ಸಂಭೂತ ಸುವರ್ಣೇನ ಯುತಂ ಜಲಂ ।\\
ಆತ್ಮತೇಜಃ ಸ್ವರೂಪಾಯ ಗುರವೇ ಪ್ರದದಾಮ್ಯಹಮ್ ॥\\\as{ಸುವರ್ಣೋದಕಸ್ನಾನಮ್॥}

ಅಕ್ಷತಾನಿ ಚ ಕರ್ಮಾಣಿ ಕ್ಷತಾನಿ ತವ ಸೇವಯಾ ।\\
ಅಕ್ಷಯಂ ದೇಹಿ ಮೇ ಜ್ಞಾನಂ ಅಕ್ಷತಾಯ ನಮೋ ನಮಃ ॥\\
ನಿರ್ಗುಣಂ ನಿರ್ಮಲಂ ಶಾಂತಂ ಜಂಗಮಂ ಸ್ಥಿರಮೇವ ಚ ।\\
ವ್ಯಾಪ್ತಂ ಯೇನ ಜಗತ್ಸರ್ವಂ ತಸ್ಮೈ ಶ್ರೀಗುರವೇ ನಮಃ ॥\\\as{ಅಕ್ಷತೋದಕಸ್ನಾನಮ್॥}

ಯದ್ದರ್ಶನಸ್ಪರ್ಶನತೋ ಶುದ್ಧ್ಯಂತ್ಯಜ್ಞಾನ ಚೇತಸಃ ।\\
ತಸ್ಯ ಶುದ್ಧಿಕರಂ ಸ್ನಾನಂ ಕಿಂ ಕೇ ಕುರ್ವಂತಿ ಕ್ಷಾಲನಮ್ ॥\\
ಗಂಗಾ ಸರಸ್ವತೀ ರೇವಾ ಪಯೋಷ್ಣೀ ಯಮುನಾಜಲೈಃ~।\\
ತದಿದಂ ಕಲ್ಪಿತಂ ಸ್ವಾಮಿನ್ ಸ್ನಾನಂ ಚ ಪ್ರತಿಗೃಹ್ಯತಾಮ್ ॥\\\as{ಶುದ್ಧೋದಕಸ್ನಾನಮ್॥}

ನಾನಾತೀರ್ಥೈರಾಹೃತಂ ಚ ಶೈತ್ಯವಾರಣಹೇತವೇ ।\\
ಉಷ್ಣೋದಕಂ ಪ್ರದಾಸ್ಯಾಮಿ ಸ್ನಾನಾರ್ಥಂ ಪ್ರತಿಗೃಹ್ಯತಾಮ್ ॥\\\as{ಉಷ್ಣೋದಕಸ್ನಾನಮ್॥}

ಶ್ವೇತವಸ್ತ್ರದ್ವಯಂ ದೇವ ನಾನಾ ಚಿತ್ರ ಸುಶೋಭಿತಮ್ ।\\
ಭಕ್ತ್ಯಾ ಸಮರ್ಪಯೇ ತುಭ್ಯಂ ಸ್ವೀಕುರುಷ್ವ ದಯಾನಿಧೇ ॥\\
ಸಂಸಾರವೃಕ್ಷಮಾರೂಢಾಃ ಪತಂತಿ ನರಕಾರ್ಣವೇ ।\\
ಯಸ್ತಾನುದ್ಧರತೇ ಸರ್ವಾನ್ ತಸ್ಮೈ ಶ್ರೀಗುರವೇ ನಮಃ ॥\as{ವಸ್ತ್ರಮ್॥}

ಉಪವೀತಮಿದಂ ದೇವ ಸ್ವರ್ಣತಂತು ವಿನಿರ್ಮಿತಂ ।\\
ಧಾರಯಾಮಿ ಗುಣಾತೀತ ಸದ್ಗುರೋ ಪ್ರತಿಗೃಹ್ಯತಾಮ್ ॥\\
ಯತ್ಸತ್ತ್ವೇನ ಜಗತ್ಸತ್ತ್ವಂ ಯತ್ಪ್ರಕಾಶೇನ ಭಾತಿ ತತ್ ।\\
ಯದಾನಂದೇನ ನಂದಂತಿ ತಸ್ಮೈ ಶ್ರೀಗುರವೇ ನಮಃ ॥\as{ಉಪವೀತಮ್॥}

ಮಾಣಿಕ್ಯ ಮುಕ್ತಾಫಲ ವಿದ್ರುಮೈಶ್ಚ \\
ಗೋಮೇಧ ವೈಡೂರ್ಯಕ ಪುಷ್ಯರಾಗೈಃ ।\\
ಪ್ರವಾಲ ನೀಲೈಶ್ಚ ಕೃತಂ ಗೃಹಾಣ\\
 ದಿವ್ಯಂ ಹಿ ರತ್ನಾಭರಣಂ ಚ ದೇವ ॥\\
ಜ್ಞಾನಶಕ್ತಿಸಮಾರೂಢತತ್ತ್ವಮಾಲಾವಿಭೂಷಿಣೇ ।\\
ಭುಕ್ತಿಮುಕ್ತಿಪ್ರದಾತ್ರೇ ಚ ತಸ್ಮೈ ಶ್ರೀಗುರವೇ ನಮಃ ॥\as{ಆಭರಣಮ್॥}
\newpage
ಅಕ್ಷತಾನಿ ಚ ಕರ್ಮಾಣಿ ಕ್ಷತಾನಿ ತವ ಸೇವಯಾ ।\\
ಅಕ್ಷಯಂ ದೇಹಿ ಮೇ ಜ್ಞಾನಂ ಅಕ್ಷತಾಯ ನಮೋ ನಮಃ ॥\\
ಅನೇಕಜನ್ಮಸಂಪ್ರಾಪ್ತಕರ್ಮಬಂಧವಿದಾಹಿನೇ ।\\
ಜ್ಞಾನಾನಲಪ್ರಭಾವೇನ ತಸ್ಮೈ ಶ್ರೀಗುರವೇ ನಮಃ ॥\as{ಅಕ್ಷತಾಃ॥}

ಮಾಲ್ಯಾದೀನಿ ಸುಗಂಧೀನಿ ಮಾಲತ್ಯಾದೀನಿ ಚ ಪ್ರಭೋ~।\\
ಮಯಾಹೃತಾನಿ ಪೂಜಾರ್ಥಂ ಪುಷ್ಪಾಣಿ ಸ್ವೀಕುರು ಪ್ರಭೋ ॥\\
ಮಲ್ಲಿಕಾಜಾತೀಕುಸುಮೈಶ್ಚಂಪಕೈರ್ಬಕುಲೈಃ ಶುಭೈಃ ।\\
ಶತಪತ್ರೈಶ್ಚ ಕಹ್ಲಾರೈರರ್ಚಯೇ ಪರಮೇಶ್ವರ ॥\\
ಮನ್ನಾಥಃ ಶ್ರೀಜಗನ್ನಾಥೋ ಮದ್ಗುರುಃ ಶ್ರೀಜಗದ್ಗುರುಃ ।\\
ಮಮಾಽಽತ್ಮಾ ಸರ್ವಭೂತಾತ್ಮಾ ತಸ್ಮೈ ಶ್ರೀಗುರವೇ ನಮಃ ॥\as{ಪುಷ್ಪಾಣಿ॥}

ಹರಿದ್ರಾಕುಂಕುಮಂ ದಿವ್ಯಂ ಸರ್ವಸೌಭಾಗ್ಯದಾಯಕಂ ।\\
ಮಯಾರ್ಪಿತಂ ಚ ತೇ ಭಕ್ತ್ಯಾ ಸ್ವೀಕುರುಷ್ವ ದಯಾನಿಧೇ ॥\\
ಹೇತವೇ ಜಗತಾಮೇವ ಸಂಸಾರಾರ್ಣವಸೇತವೇ ।\\
ಪ್ರಭವೇ ಸರ್ವವಿದ್ಯಾನಾಂ ಶಂಭವೇ ಗುರವೇ ನಮಃ ॥\as{ಹರಿದ್ರಾಕುಂಕುಮಮ್॥}


ಸಿಂದೂರಂ ನಾಗಸಂಭೂತಂ ಫಾಲಶೋಭಾವಿವರ್ಧನಂ ।\\
ಪೂರಣಂ ಭೂಷಣಾನಾಂ ಚ ಶ್ರೀಗುರೋ ಪ್ರತಿಗೃಹ್ಯತಾಮ್ ॥\\
ಏಕ ಏವ ಪರೋ ಬಂಧುರ್ವಿಷಮೇ ಸಮುಪಸ್ಥಿತೇ ।\\
ನಿಸ್ಪೃಹಃ ಕರುಣಾಸಿಂಧುಃ ತಸ್ಮೈ ಶ್ರೀ ಗುರವೇ ನಮಃ ॥\as{ಸಿಂದೂರಮ್॥}
\newpage
\section{ಅಂಗಪೂಜಾ}
ಸಂಸಾರಾರ್ಣವತಾರಕಾಯ ನಮಃ । ಪಾದೌ ಪೂಜಯಾಮಿ॥\\
ಗುರವೇ ನಮಃ । ಗುಲ್ಫೌ ಪೂಜಯಾಮಿ॥\\
ಅಜ್ಞಾನತಿಮಿರಭಾಸ್ಕರಾಯ ನಮಃ । ಜಂಘೇ ಪೂಜಯಾಮಿ॥\\
ವ್ಯಾಪಕಾಯ ನಮಃ । ಜಾನುನೀ ಪೂಜಯಾಮಿ॥\\
ಕಲಿದೋಷವಿದೂರಾಯ ನಮಃ । ಊರೂ ಪೂಜಯಾಮಿ॥\\
ವಿಶ್ವಂಭರಾಯ ನಮಃ । ಗುಹ್ಯಂ ಪೂಜಯಾಮಿ॥\\
ನಾದಬಿಂದುಕಲಾತೀತಾಯ ನಮಃ । ಕಟಿಂ ಪೂಜಯಾಮಿ॥\\
ಶಾಂತಸ್ವರೂಪಾಯ ನಮಃ । ಉದರ ಪೂಜಯಾಮಿ॥\\
ಸತ್ಯಧರ್ಮಸ್ವರೂಪಾಯ ನಮಃ । ನಾಭಿಂ ಪೂಜಯಾಮಿ॥\\
ಆನಂದಸಾಗರಾಯ ನಮಃ । ವಕ್ಷಸ್ಥಲಂ ಪೂಜಯಾಮಿ॥\\
ದಿವ್ಯಚಕ್ಷುಷೇ ನಮಃ । ಬಾಹೂನ್ ಪೂಜಯಾಮಿ॥\\
ಮಾಯಾಮುಕ್ತಾಯ ನಮಃ । ಹಸ್ತಾನ್ ಪೂಜಯಾಮಿ॥\\
ಚಿನ್ಮಯಾಯ ನಮಃ । ಕಂಠಂ ಪೂಜಯಾಮಿ॥\\
ಕರುಣಾಸಾಂದ್ರಾಯ ನಮಃ । ಮುಖಂ ಪೂಜಯಾಮಿ॥\\
ಸ್ಥಿತಪ್ರಜ್ಞಾಯ ನಮಃ । ದಂತಾನ್ ಪೂಜಯಾಮಿ॥\\
ಜ್ಞಾನತೇಜಸೇ ನಮಃ । ಸ್ಮಿತಂ ಪೂಜಯಾಮಿ॥\\
ನಾನಾರೂಪಾಯ ನಮಃ । ನಾಸಿಕಾಂ ಪೂಜಯಾಮಿ॥\\
ಸಹಸ್ರಾಕ್ಷಾಯ ನಮಃ । ನೇತ್ರಾಣಿ ಪೂಜಯಾಮಿ॥\\
ಓಂಕಾರರೂಪಾಯ ನಮಃ । ಕರ್ಣೌ ಪೂಜಯಾಮಿ॥\\
ದಯಾಸಾಗರಾಯ ನಮಃ । ಲಲಾಟಂ ಪೂಜಯಾಮಿ॥\\
ಸ್ವಪ್ರಕಾಶಾಯ ನಮಃ । ಶಿರಃ ಪೂಜಯಾಮಿ॥\\
ಸದ್ಗುರವೇ ನಮಃ । ಸರ್ವಾಂಗಂ ಪೂಜಯಾಮಿ॥
\section{ಪತ್ರ ಪೂಜಾ }
ಶ್ರೀಗುರವೇ ನಮಃ । ದೂರ್ವಾಪತ್ರಂ ಸಮರ್ಪಯಾಮಿ ।\\
ಪರಮಗುರವೇ ನಮಃ । ಬಿಲ್ವಪತ್ರಂ ಸಮರ್ಪಯಾಮಿ ।\\
ಪರಮಮೇಷ್ಠಿಗುರವೇ ನಮಃ । ತುಲಸೀಪತ್ರಂ ಸಮರ್ಪಯಾಮಿ ।\\
ಜ್ಞಾನಸಾಗರಾಯ ನಮಃ । ಮಲ್ಲಿಕಾಪತ್ರಂ ಸಮರ್ಪಯಾಮಿ ।\\
ಸ್ವಪ್ರಕಾಶಾಯ ನಮಃ । ಸೇವಂತಿಕಾಪತ್ರಂ ಸಮರ್ಪಯಾಮಿ ।\\
ಬ್ರಹ್ಮಸ್ವರೂಪಾಯ ನಮಃ । ಕಸ್ತೂರಿಕಾಪತ್ರಂ ಸಮರ್ಪಯಾಮಿ ।\\
ಶಿವಸ್ವರೂಪಾಯ ನಮಃ । ಕಮಲಪತ್ರಂ ಸಮರ್ಪಯಾಮಿ ।\\
ವಿಷ್ಣುಸ್ವರೂಪಾಯ ನಮಃ । ಕುಶಪತ್ರಂ ಸಮರ್ಪಯಾಮಿ ।\\
ಚಿನ್ಮಯಾಯ ನಮಃ । ವಿಷ್ಣುಕ್ರಾಂತಿಪತ್ರಂ ಸಮರ್ಪಯಾಮಿ ।\\
ಮೋಕ್ಷಪಾಣಯೇ ನಮಃ । ಗಿರಿಕರ್ಣಿಕಾಪತ್ರಂ ಸಮರ್ಪಯಾಮಿ ।\\
ಕರುಣಾಕರಾಯ ನಮಃ । ಚಂಪಕಾಪತ್ರಂ ಸಮರ್ಪಯಾಮಿ ।\\
ಶೋಕರಹಿತಾಯ ನಮಃ । ಕರವೀರಪತ್ರಂ ಸಮರ್ಪಯಾಮಿ ।\\
ಪ್ರಣವಸ್ವರೂಪಾಯ ನಮಃ । ಅಶೋಕಪತ್ರಂ ಸಮರ್ಪಯಾಮಿ ।\\
ಅಜ್ಞಾನಹರಣಾಯ ನಮಃ । ಪುನ್ನಾಗಪತ್ರಂ ಸಮರ್ಪಯಾಮಿ ।\\
ನಿರ್ಗುಣಾಯ ನಮಃ । ಅತಸೀಪತ್ರಂ ಸಮರ್ಪಯಾಮಿ ।\\
ಜಗನ್ನಾಥಾಯ ನಮಃ । ನಿರ್ಗುಂಡೀಪತ್ರಂ ಸಮರ್ಪಯಾಮಿ ।\\
ಶರಣಾಗತರಕ್ಷಕಾಯ ನಮಃ । ಜಾಜೀಪತ್ರಂ ಸಮರ್ಪಯಾಮಿ ।\\
ಭಕ್ತಾಭೀಷ್ಟಪ್ರದಾಯ ನಮಃ । ಶಮೀಪತ್ರಂ ಸಮರ್ಪಯಾಮಿ ।\\
ತ್ರಿಗುಣಾತ್ಮಕಾಯ ನಮಃ । ಮರುಗಪತ್ರಂ ಸಮರ್ಪಯಾಮಿ ।\\
ಮಂಗಲಾಯ ನಮಃ । ಮಂದಾರಪತ್ರಂ ಸಮರ್ಪಯಾಮಿ ।\\
ಜ್ಞಾನನಿಧಯೇ ನಮಃ । ಆಮಲಕಪತ್ರಂ ಸಮರ್ಪಯಾಮಿ ।
\section{ಪುಷ್ಪಪೂಜಾ}
ಶ್ರೀಗುರವೇ ನಮಃ । ದ್ರೋಣಪುಷ್ಪಂ ಸಮರ್ಪಯಾಮಿ ।\\
ಪರಮಪರುಷಾಯ ನಮಃ । ಪಾಟಲೀಪುಷ್ಪಂ ಸಮರ್ಪಯಾಮಿ ।\\
ಮಂಗಲದಾಯಕಾಯ ನಮಃ । ಮಲ್ಲಿಕಾಪುಷ್ಪಂ ಸಮರ್ಪಯಾಮಿ ।\\
ಚಿನ್ಮಯಾಯ ನಮಃ । ಗಿರಿಕರ್ಣಿಕಾಪುಷ್ಪಂ ಸಮರ್ಪಯಾಮಿ ।\\
ಪರಮಾತ್ಮನೇ ನಮಃ । ಪೂಗಪುಷ್ಪಂ ಸಮರ್ಪಯಾಮಿ ।\\
ಪರಮಹಂಸಾಯ ನಮಃ । ಪುನ್ನಾಗಪುಷ್ಪಂ ಸಮರ್ಪಯಾಮಿ ।\\
ಪರಂಜ್ಯೋತಿಷೇ ನಮಃ । ಸೇವಂತಿಕಾಪುಷ್ಪಂ ಸಮರ್ಪಯಾಮಿ ।\\
ಭಕ್ತಪ್ರಿಯಾಯ ನಮಃ । ಮಾಲತೀಪುಷ್ಪಂ ಸಮರ್ಪಯಾಮಿ ।\\
ಮೋಕ್ಷದಾಯಕಾಯ ನಮಃ । ಮಂದಾರಪುಷ್ಪಂ ಸಮರ್ಪಯಾಮಿ ।\\
ಬ್ರಹ್ಮಸ್ವರೂಪಾಯ ನಮಃ । ಬಕುಲಪುಷ್ಪಂ ಸಮರ್ಪಯಾಮಿ ।\\
ಆಕಾರರಹಿತಾಯ ನಮಃ । ಅತಸೀಪುಷ್ಪಂ ಸಮರ್ಪಯಾಮಿ ।\\
ಜಗತ್ಪ್ರಭವೇ ನಮಃ । ಜಾಜೀಪುಷ್ಪಂ ಸಮರ್ಪಯಾಮಿ ।\\
ಕರುಣಾಕರಾಯ ನಮಃ । ಕರವೀರಪುಷ್ಪಂ ಸಮರ್ಪಯಾಮಿ ।\\
ವಿಷ್ಣುರೂಪಿಣೇ ನಮಃ । ತುಲಸೀಪುಷ್ಪಂ ಸಮರ್ಪಯಾಮಿ ।\\
ಸುಗುಣಾಯ ನಮಃ । ಸೌಗಂಧಿಕಾಪುಷ್ಪಂ ಸಮರ್ಪಯಾಮಿ ।\\
ಆನಂದರೂಪಾಯ ನಮಃ । ಜಪಾಪುಷ್ಪಂ ಸಮರ್ಪಯಾಮಿ ।\\
ಅನುಗ್ರಹಪ್ರದಾಯ ನಮಃ । ಚಂಪಕಪುಷ್ಪಂ ಸಮರ್ಪಯಾಮಿ ।\\
ಶಿವರೂಪಾಯ ನಮಃ । ಶ್ವೇತಪುಷ್ಪಂ ಸಮರ್ಪಯಾಮಿ ।\\
ದಯಾನಿಧಯೇ ನಮಃ । ದಾಡಿಮೀಪುಷ್ಪಂ ಸಮರ್ಪಯಾಮಿ ।\\
ಜಗನ್ನಾಥಾಯ ನಮಃ । ವಿಷ್ಣುಕ್ರಾಂತಿಪುಷ್ಪಂ ಸಮರ್ಪಯಾಮಿ ।
\newpage
\section{ಆವರಣಪೂಜಾ}
\addcontentsline{toc}{section}{ಶಿವ ಆವರಣಪೂಜಾ}
\subsection{ಪ್ರಥಮಾವರಣಮ್}
ಓಂ ಅಂ ಅಣಿಮಾಸಿದ್ಧ್ಯೈ ನಮಃ ।
ಓಂ ಲಂ ಲಘಿಮಾಸಿದ್ಧ್ಯೈ ನಮಃ ।
ಓಂ ಮಂ ಮಹಿಮಾಸಿದ್ಧ್ಯೈ ನಮಃ ।
ಓಂ ಈಂ ಈಶಿತ್ವಸಿದ್ಧ್ಯೈ ನಮಃ ।
ಓಂ ವಂ ವಶಿತ್ವಸಿದ್ಧ್ಯೈ ನಮಃ ।
ಓಂ ಪಂ ಪ್ರಾಕಾಮ್ಯಸಿದ್ಧ್ಯೈ ನಮಃ ।
ಓಂ ಭುಂ ಭುಕ್ತಿಸಿದ್ಧ್ಯೈ ನಮಃ ।
ಓಂ ಇಂ ಇಚ್ಛಾಸಿದ್ಧ್ಯೈ ನಮಃ ।
ಓಂ ಪಂ ಪ್ರಾಪ್ತಿಸಿದ್ಧ್ಯೈ ನಮಃ ।
ಓಂ ಸಂ ಸರ್ವಕಾಮಸಿದ್ಧ್ಯೈ ನಮಃ ।\\
ಅಭೀಷ್ಟಸಿದ್ಧಿಂ ಮೇ ದೇಹಿ ಶರಣಾಗತ ವತ್ಸಲ~।\\
ಭಕ್ತ್ಯಾ ಸಮರ್ಪಯೇ ತುಭ್ಯಂ ಪ್ರಥಮಾವರಣಾರ್ಚನಂ ॥
\subsection{ದ್ವಿತೀಯಾವರಣಮ್}
ಓಂ ಮೇಷಾಯ ನಮಃ ।
ಓಂ ವೃಷಭಾಯ ನಮಃ ।
ಓಂ ಮಿಥುನಾಯ ನಮಃ ।
ಓಂ ಕರ್ಕಟಾಯ ನಮಃ ।
ಓಂ ಸಿಂಹಾಯ ನಮಃ ।
ಓಂ ಕನ್ಯಾಯೈ ನಮಃ ।
ಓಂ ತುಲಾಯೈ ನಮಃ ।
ಓಂ ವೃಶ್ಚಕಾಯ ನಮಃ ।
ಓಂ ಧನುಷೇ ನಮಃ ।
ಓಂ ಮಕರಾಯ ನಮಃ ।
ಓಂ ಕುಂಭಾಯ ನಮಃ ।
ಓಂ ಮೀನಾಯ ನಮಃ ।

ಅಭೀಷ್ಟಸಿದ್ಧಿಂ******ದ್ವಿತೀಯಾವರಣಾರ್ಚನಂ ॥
\subsection{ತೃತೀಯಾವರಣಮ್}
ಓಂ ಆದಿತ್ಯಾಯ ನಮಃ ।
ಓಂ ಚಂದ್ರಾಯ ನಮಃ ।
ಓಂ ಮಂಗಳಾಯ ನಮಃ ।
ಓಂ ಬುಧಾಯ ನಮಃ ।
ಓಂ ಗುರವೇ ನಮಃ ।
ಓಂ ಶುಕ್ರಾಯ ನಮಃ ।
ಓಂ ಶನೈಶ್ಚರಾಯ ನಮಃ ।
ಓಂ ರಾಹವೇ ನಮಃ ।
ಓಂ ಕೇತವೇ ನಮಃ ।\\
ಅಭೀಷ್ಟಸಿದ್ಧಿಂ******ತೃತೀಯಾವರಣಾರ್ಚನಂ ॥
\newpage
\subsection{ಚತುರ್ಥಾವರಣಮ್}
ಸನಕಾಯ ನಮಃ ।
ಭೃಗವೇ ನಮಃ ।
ಸನಂದನಾಯ ನಮಃ ।
ಭರದ್ವಾಜಾಯ ನಮಃ ।
ವಸಿಷ್ಠಾಯ ನಮಃ ।
ಅತ್ರಯೇ ನಮಃ ।
ಸನಾತನಾಯ ನಮಃ ।
ಸನತ್ಕುಮಾರಾಯ ನಮಃ ।
ನಾರದಾಯ ನಮಃ ।
ಋಭವೇ ನಮಃ ।\\
ಅಭೀಷ್ಟಸಿದ್ಧಿಂ******ತುರೀಯಾವರಣಾರ್ಚನಂ ॥
\subsection{ಪಂಚಮಾವರಣಮ್}
ಓಂ ಮತಸ್ಯಾಯ ನಮಃ ।
ಓಂ ಕೂರ್ಮಾಯ ನಮಃ ।
ಓಂ ವರಾಹಾಯ ನಮಃ ।
ಓಂ ನರಸಿಂಹಾಯ ನಮಃ ।
ಓಂ ವಾಮನಾಯ ನಮಃ ।
ಓಂ ಭಾರ್ಗವಾಯ ನಮಃ ।
ಓಂ ರಾಮಾಯ ನಮಃ ।
ಓಂ ಕೃಷ್ಣಾಯ ನಮಃ ।
ಓಂ ಬುದ್ಧಾಯ ನಮಃ ।
ಓಂ ಕಲ್ಕಿನೇ ನಮಃ ।\\
ಅಭೀಷ್ಟಸಿದ್ಧಿಂ******ಪಂಚಮಾವರಣಾರ್ಚನಂ ॥
\subsection{ಷಷ್ಠಾವರಣಮ್}
ಓಂ ಚಿಂತಾಮಣಯೇ ನಮಃ ।
ಓಂ ಕಲ್ಪವೃಕ್ಷಾಯ ನಮಃ ।
ಓಂ ಕಾಮಧೇನವೇ ನಮಃ ।
ಓಂ ತ್ರಿಶೂಲಾಯ ನಮಃ ।
ಓಂ ಡಮರುಗಾಯ ನಮಃ ।
ಓಂ ಜಪಮಾಲಾಯೈ ನಮಃ ।
ಓಂ ಪುಸ್ತಕಾಯ ನಮಃ ।
ಓಂ ಶಂಖಾಯ ನಮಃ ।
ಓಂ ಚಕ್ರಾಯ ನಮಃ ।
ಓಂ ಗದಾಯೈ ನಮಃ ।
ಓಂ ಖಡ್ಗಾಯ ನಮಃ ।
ಓಂ ಕಮಲಾಯ ನಮಃ ।\\
ಅಭೀಷ್ಟಸಿದ್ಧಿಂ******ಷಷ್ಠಾಖ್ಯಾವರಣಾರ್ಚನಂ ॥
\subsection{ಸಪ್ತಮಾವರಣಮ್}
ಓಂ ಲಂ ಇಂದ್ರಾಯ ನಮಃ । ಓಂ ರಂ ಅಗ್ನಯೇ ನಮಃ । ಓಂ ಮಂ ಯಮಾಯ ನಮಃ । ಓಂ ಕ್ಷಂ ನಿರ್ಋತಯೇ ನಮಃ ।ಓಂ ವಂ ವರುಣಾಯ ನಮಃ ।ಓಂ ಯಂ ವಾಯವೇ ನಮಃ । ಓಂ ಕುಂ ಕುಬೇರಾಯ ನಮಃ । ಓಂ ಹಂ ಈಶಾನಾಯ ನಮಃ । ಓಂ ಆಂ ಬ್ರಹ್ಮಣೇ ನಮಃ । ಓಂ ಹ್ರೀಂ ಅನಂತಾಯ ನಮಃ । ಓಂ ನಿಯತ್ಯೈ ನಮಃ । ಓಂ ಕಾಲಾಯ ॥\\
ಅಭೀಷ್ಟಸಿದ್ಧಿಂ******ಸಪ್ತಮಾವರಣಾರ್ಚನಂ ॥
\subsection{ಅಷ್ಟಮಾವರಣಮ್}
ಓಂ ಭವಾಯ ನಮಃ ।
ಓಂ ಶರ್ವಾಯ ನಮಃ ।
ಓಂ ಈಶಾನಾಯ ನಮಃ ।
ಓಂ ಪಶುಪತಯೇ ನಮಃ ।
ಓಂ ರುದ್ರಾಯ ನಮಃ ।
ಓಂ ಉಗ್ರಾಯ ನಮಃ ।
ಓಂ ಭೀಮಾಯ ನಮಃ ।
ಓಂ ಮಹತೇ ನಮಃ ।\\
ಓಂ ಭವಸ್ಯ ಪತ್ನ್ಯೈ ನಮಃ ।
ಓಂ ಶರ್ವಸ್ಯ ಪತ್ನ್ಯೈ ನಮಃ ।
ಓಂ ಈಶಾನಸ್ಯ ಪತ್ನ್ಯೈ ನಮಃ ।
ಓಂ ಪಶುಪತೇಃ ಪತ್ನ್ಯೈ ನಮಃ ।
ಓಂ ರುದ್ರಸ್ಯ ಪತ್ನ್ಯೈ ನಮಃ ।
ಓಂ ಉಗ್ರಸ್ಯ ಪತ್ನ್ಯೈ ನಮಃ ।
ಓಂ ಭೀಮಸ್ಯ ಪತ್ನ್ಯೈ ನಮಃ ।
ಓಂ ಮಹತಃ ಪತ್ನ್ಯೈ ನಮಃ ।\\
ಅಭೀಷ್ಟಸಿದ್ಧಿಂ******ಅಷ್ಟಮಾವರಣಾರ್ಚನಂ ॥
\subsection{ನವಮಾವರಣಮ್}
ಓಂ ಬ್ರಾಹ್ಮ್ಯೈ ನಮಃ । ಓಂ ಮಾಹೇಶ್ವರ್ಯೈ ನಮಃ । ಓಂ ಕೌಮಾರ್ಯೈ ನಮಃ । ಓಂ ವೈಷ್ಣವ್ಯೈ ನಮಃ । ಓಂ ವಾರಾಹ್ಯೈ ನಮಃ । ಓಂ ಮಾಹೇಂದ್ರ್ಯೈ ನಮಃ । ಓಂ ಚಾಮುಂಡಾಯೈ ನಮಃ । ಓಂ ಮಹಾಲಕ್ಷ್ಮ್ಯೈ ನಮಃ ॥\\
ಅಭೀಷ್ಟಸಿದ್ಧಿಂ******ನವಮಾವರಣಾರ್ಚನಂ ॥
\newpage
\section{ಪಾದುಕಾಪೂಜಾ}
\as{ದಿವ್ಯೌಘಸಿದ್ಧೌಘಮಾನವೌಘೇಭ್ಯೋ ನಮಃ}

ಪರಪ್ರಕಾಶಾನಂದನಾಥ ಶ್ರೀಪಾ।ಪೂ।ತ।ನಮಃ ।\\
ಪರಶಿವಾನಂದನಾಥ ಶ್ರೀಪಾ।ಪೂ।ತ।ನಮಃ ।\\
ಪರಾಶಕ್ತ್ಯಂಬಾ ಶ್ರೀಪಾ।ಪೂ।ತ।ನಮಃ ।\\
ಕೌಲೇಶ್ವರಾನಂದನಾಥ ಶ್ರೀಪಾ।ಪೂ।ತ।ನಮಃ ।\\
ಶುಕ್ಲದೇವ್ಯಂಬಾ ಶ್ರೀಪಾ।ಪೂ।ತ।ನಮಃ ।\\
ಕುಲೇಶ್ವರಾನಂದನಾಥ ಶ್ರೀಪಾ।ಪೂ।ತ।ನಮಃ ।\\
ಕಾಮೇಶ್ವರ್ಯಂಬಾ ಶ್ರೀಪಾ।ಪೂ।ತ।ನಮಃ ।

ಭೋಗಾನಂದನಾಥ ಶ್ರೀಪಾ।ಪೂ।ತ।ನಮಃ ।\\
ಕ್ಲಿನ್ನಾನಂದನಾಥ ಶ್ರೀಪಾ।ಪೂ।ತ।ನಮಃ ।\\
ಸಮಯಾನಂದನಾಥ ಶ್ರೀಪಾ।ಪೂ।ತ।ನಮಃ ।\\
ಸಹಜಾನಂದನಾಥ ಶ್ರೀಪಾ।ಪೂ।ತ।ನಮಃ ।

ಗಗನಾನಂದನಾಥ ಶ್ರೀಪಾ।ಪೂ।ತ।ನಮಃ ।\\
ವಿಶ್ವಾನಂದನಾಥ ಶ್ರೀಪಾ।ಪೂ।ತ।ನಮಃ ।\\
ವಿಮಲಾನಂದನಾಥ ಶ್ರೀಪಾ।ಪೂ।ತ।ನಮಃ ।\\
ಮದನಾನಂದನಾಥ ಶ್ರೀಪಾ।ಪೂ।ತ।ನಮಃ ।\\
ಭುವನಾನಂದನಾಥ ಶ್ರೀಪಾ।ಪೂ।ತ।ನಮಃ ।\\
ಲೀಲಾನಂದನಾಥ ಶ್ರೀಪಾ।ಪೂ।ತ।ನಮಃ ।\\
ಸ್ವಾತ್ಮಾನಂದನಾಥ ಶ್ರೀಪಾ।ಪೂ।ತ।ನಮಃ ।\\
ಪ್ರಿಯಾನಂದನಾಥ ಶ್ರೀಪಾ।ಪೂ।ತ।ನಮಃ ।
\newpage
ಶ್ರೀಗುರು ಶ್ರೀಪಾ।ಪೂ।ತ।ನಮಃ ।\\
ಪರಮಗುರು ಶ್ರೀಪಾ।ಪೂ।ತ।ನಮಃ ।\\
ಪರಮೇಷ್ಠಿಗುರು ಶ್ರೀಪಾ।ಪೂ।ತ।ನಮಃ ।

ಶ್ರೀಗುರವೇ ನಮಃ ।
ಶಿವಾಯ ನಮಃ ।
ಹೃಷೀಕೇಶಾಯ ನಮಃ ।
ಬ್ರಹ್ಮಣೇ ನಮಃ ।
ಪ್ರಭವೇ ನಮಃ ।
ಈಶ್ವರಾಯ ನಮಃ ।
ತ್ರಿಮೂರ್ತಯೇ ನಮಃ ।
ಜ್ಞಾನಸಾಗರಾಯ ನಮಃ ।
ಭಕ್ತಪ್ರಿಯಾಯ ನಮಃ ।
ಕರುಣಾಕರಾಯ ನಮಃ ।
ಅಜ್ಞಾನನಾಶಕಾಯ ನಮಃ ।
ಸರ್ವಸಿದ್ಧಿದಾಯ ನಮಃ ।
ಮೋಕ್ಷದಾಯ ನಮಃ।
\section{ಅಷ್ಟೋತ್ತರಶತನಾಮಪೂಜಾ}
ಪರಾನಂದನಾಥಾಯ ನಮಃ ।\\
ಭಾನಂದನಾಥಾಯ ನಮಃ ।\\
ಚಿದಾನಂದನಾಥಾಯ ನಮಃ ।\\
ಮಹಾಮಾಯಾನಂದನಾಥಾಯ ನಮಃ ।\\
ಇಚ್ಛಾನಂದನಾಥಾಯ ನಮಃ ।\\
ಸೃಷ್ಟ್ಯಾನಂದನಾಥಾಯ ನಮಃ ।\\
ಸ್ಥಿತ್ಯಾನಂದನಾಥಾಯ ನಮಃ ।\\
ತಿರೋಧಾನಾನಂದನಾಥಾಯ ನಮಃ ।\\
ಮುಕ್ತ್ಯಾನಂದನಾಥಾಯ ನಮಃ ।\\
ಜ್ಞಾನಾನಂದನಾಥಾಯ ನಮಃ ।\\
ಸತ್ಯಾನಂದನಾಥಾಯ ನಮಃ ।\\
ಅಸತ್ಯಾನಂದನಾಥಾಯ ನಮಃ ।\\
ಸದಸತ್ಯಾನಂದನಾಥಾಯ ನಮಃ ।\\
ಕ್ರಿಯಾನಂದನಾಥಾಯ ನಮಃ ।\\
ಆತ್ಮವತ್ಯಾನಂದನಾಥಾಯ ನಮಃ ।\\
ಇಂದ್ರಿಯಾನಂದನಾಥಾಯ ನಮಃ ।\\
ಗೋಚರಾನಂದನಾಥಾಯ ನಮಃ ।\\
ಲೋಕಮುಖ್ಯಾನಂದನಾಥಾಯ ನಮಃ ।\\
ವೇದಾನಂದನಾಥಾಯ ನಮಃ ।\\
ಸಂವಿದಾನಂದನಾಥಾಯ ನಮಃ ।\\
ಕುಂಡಲ್ಯಾನಂದನಾಥಾಯ ನಮಃ ।\\
ಸುಷುಮ್ನಾನಂದನಾಥಾಯ ನಮಃ ।\\
ಪ್ರಾಣಸೂತ್ರಾನಂದನಾಥಾಯ ನಮಃ ।\\
ಸ್ಪಂದಾನಂದನಾಥಾಯ ನಮಃ ।\\
ಮಾತೃಕಾನಂದನಾಥಾಯ ನಮಃ ।\\
ಶಬ್ದಾನಂದನಾಥಾಯ ನಮಃ ।\\
ಮಂತ್ರವಿಗ್ರಹಾನಂದನಾಥಾಯ ನಮಃ ।\\
ವರ್ಣಾನಂದನಾಥಾಯ ನಮಃ ।\\
ಸ್ವರೋದವಾನಂದನಾಥಾಯ ನಮಃ ।\\
ವರ್ಣಜಾನಂದನಾಥಾಯ ನಮಃ ।\\
ವರ್ಗಜಾನಂದನಾಥಾಯ ನಮಃ ।\\
ಸಂಯೋಗಜಾನಂದನಾಥಾಯ ನಮಃ ।\\
ವಿಚ್ಚೇಶ್ವರಾನಂದನಾಥಾಯ ನಮಃ ।\\
ಧರಾನಂದನಾಥಾಯ ನಮಃ ।\\
ಭಾವಾನಂದನಾಥಾಯ ನಮಃ ।\\
ದ್ರವಾನಂದನಾಥಾಯ ನಮಃ ।\\
ವಸನಾನಂದನಾಥಾಯ ನಮಃ ।\\
ಮೋಹಾನಂದನಾಥಾಯ ನಮಃ ।\\
ಮನೋಭವಾನಂದನಾಥಾಯ ನಮಃ ।\\
ಸೇನಾನಂದನಾಥಾಯ ನಮಃ ।\\
ಖೇಟಾನಂದನಾಥಾಯ ನಮಃ ।\\
ಜ್ವಾಲಾನಂದನಾಥಾಯ ನಮಃ ।\\
ಮಹಾಂಕುಶಾನಂದನಾಥಾಯ ನಮಃ ।\\
ತೈಜಸಾನಂದನಾಥಾಯ ನಮಃ ।\\
ಮೂರ್ಧಾನಂದನಾಥಾಯ ನಮಃ ।\\
ಕುಲಾನಂದನಾಥಾಯ ನಮಃ ।\\
ಸಂಹಾರಾನಂದನಾಥಾಯ ನಮಃ ।\\
ವಿಶ್ವಂಭರಾನಂದನಾಥಾಯ ನಮಃ ।\\
ಕುಟಿಲಾನಂದನಾಥಾಯ ನಮಃ ।\\
ಗಾಲವಾನಂದನಾಥಾಯ ನಮಃ ।\\
ವ್ಯೋಮಾನಂದನಾಥಾಯ ನಮಃ ।\\
ಶ್ವಾಸಾನಂದನಾಥಾಯ ನಮಃ ।\\
ಖೇಚರಾನಂದನಾಥಾಯ ನಮಃ ।\\
ವಿಶ್ವಾನಂದನಾಥಾಯ ನಮಃ ।\\
ಅಜ್ಞಾನಾನಂದನಾಥಾಯ ನಮಃ ।\\
ಶಂಖಾನಂದನಾಥಾಯ ನಮಃ ।\\
ವರಾನಂದನಾಥಾಯ ನಮಃ ।\\
ಹಂಸೇಶ್ವರಾನಂದನಾಥಾಯ ನಮಃ ।\\
ಖಗೇಶ್ವರಾನಂದನಾಥಾಯ ನಮಃ ।\\
ಕೂರ್ಮಾನಂದನಾಥಾಯ ನಮಃ ।\\
ಮೇಷಾನಂದನಾಥಾಯ ನಮಃ ।\\
ಮೀನಾನಂದನಾಥಾಯ ನಮಃ ।\\
ತೀವ್ರಾನಂದನಾಥಾಯ ನಮಃ ।\\
ಆಧಾರೇಶಾನಂದನಾಥಾಯ ನಮಃ ।\\
ಚಕ್ರೇಶಾನಂದನಾಥಾಯ ನಮಃ ।\\
ಕುರಂಗೀಶಾನಂದನಾಥಾಯ ನಮಃ ।\\
ಹೃದೀಶಾನಂದನಾಥಾಯ ನಮಃ ।\\
ಶೀರ್ಷೇಶಾನಂದನಾಥಾಯ ನಮಃ ।\\
ಶಿಖೀಶಾನಂದನಾಥಾಯ ನಮಃ ।\\
ವರ್ಮೇಶಾನಂದನಾಥಾಯ ನಮಃ ।\\
ಅಸ್ತ್ರೇಶಾನಂದನಾಥಾಯ ನಮಃ ।\\
ಪರಗುರ್ವಾನಂದನಾಥಾಯ ನಮಃ ।\\
ಪೂಜ್ಯಗುರ್ವಾನಂದನಾಥಾಯ ನಮಃ ।\\
ಸಂವರ್ತೇಶ್ವರಾನಂದನಾಥಾಯ ನಮಃ ।\\
ಪರಾಪರಾನಂದನಾಥಾಯ ನಮಃ ।\\
ಅಘೋರಾನಂದನಾಥಾಯ ನಮಃ ।\\
ಲಲಿತಾನಂದನಾಥಾಯ ನಮಃ ।\\
ಸ್ವಚ್ಛಾನಂದನಾಥಾಯ ನಮಃ ।\\
ಭೂತೇಶ್ವರಾನಂದನಾಥಾಯ ನಮಃ ।\\
ಆನಂದಾನಂದನಾಥಾಯ ನಮಃ ।\\
ಯೋಗಾನಂದನಾಥಾಯ ನಮಃ ।\\
ಅತೀತಾನಂದನಾಥಾಯ ನಮಃ ।\\
ಮದನಾನಂದನಾಥಾಯ ನಮಃ ।\\
ಯೋಗೇಶ್ವರಾನಂದನಾಥಾಯ ನಮಃ ।\\
ಪೀಠೇಶ್ವರಾನಂದನಾಥಾಯ ನಮಃ ।\\
ಕುಲಕೌಲೇಶ್ವರಾನಂದನಾಥಾಯ ನಮಃ ।\\
ಅಜಕುಬ್ಜೇಶ್ವರಾನಂದನಾಥಾಯ ನಮಃ ।\\
ಶ್ರೀಕಂಠಾನಂದನಾಥಾಯ ನಮಃ ।\\
ಅನಂತಾನಂದನಾಥಾಯ ನಮಃ ।\\
ಕಾಲಗುರ್ವಾನಂದನಾಥಾಯ ನಮಃ ।\\
ಸಿದ್ಧಗುರ್ವಾನಂದನಾಥಾಯ ನಮಃ ।\\
ರತ್ನಗುರ್ವಾನಂದನಾಥಾಯ ನಮಃ ।\\
ಶಿವಗುರ್ವಾನಂದನಾಥಾಯ ನಮಃ ।\\
ಮೈನಾಕಗುರ್ವಾನಂದನಾಥಾಯ ನಮಃ ।\\
ಸಮಯಗುರ್ವಾನಂದನಾಥಾಯ ನಮಃ ।\\
ದ್ವೀಪೇಶ್ವರಾನಂದನಾಥಾಯ ನಮಃ ।\\
ಪುರುಷಾನಂದನಾಥಾಯ ನಮಃ ।\\
ಅನಾಶ್ರಿತಾನಂದನಾಥಾಯ ನಮಃ ।\\
ಅಚಿಂತ್ಯಾನಂದನಾಥಾಯ ನಮಃ ।\\
ಮಣಿವಾಹನಾನಂದನಾಥಾಯ ನಮಃ ।\\
ಅಂಬುವಾಹನಾನಂದನಾಥಾಯ ನಮಃ ।\\
ವಿದ್ಯಾವಾಗೀಶ್ವರಾನಂದನಾಥಾಯ ನಮಃ ।\\
ಚತುರ್ವಿಶ್ವೇಂದ್ರಾನಂದನಾಥಾಯ ನಮಃ ।\\
ಉಮಾಗಂಗೇಶ್ವರಾನಂದನಾಥಾಯ ನಮಃ ।\\
ಕೃಷ್ಣೇಶ್ವರಾನಂದನಾಥಾಯ ನಮಃ ।\\
ಪಿಂಗಲಾನಂದನಾಥಾಯ ನಮಃ ।\\
ಪರವಿದ್ಯೌಘಾನಂದನಾಥಾಯ ನಮಃ ।\\
ಪೀಠೌಘಾನಂದನಾಥಾಯ ನಮಃ ॥\\
ಅಷ್ಟೋತ್ತರಶತನಾಮಪೂಜಾಂ ಸಮರ್ಪಯಾಮಿ ॥

ದಶಾಂಗಂ ಗುಗ್ಗುಲಂ ಧೂಪಂ ಸುಗಂಧಂ ಚ ಮನೋಹರಮ್ ।\\
ಕಪಿಲಾಘೃತ ಸಂಯುಕ್ತಂ ಗೃಹಾಣ ಗುರುಪುಂಗವ ॥\\
ಅಖಂಡಮಂಡಲಾಕಾರಂ ವ್ಯಾಪ್ತಂ ಯೇನ ಚರಾಚರಮ್ ।\\
ತತ್ಪದಂ ದರ್ಶಿತಂ ಯೇನ ತಸ್ಮೈ ಶ್ರೀಗುರವೇ ನಮಃ ॥\as{ಧೂಪಃ॥}

ಅಜ್ಞಾನಧ್ವಾಂತ ವಿತ್ರಸ್ತಂ ಪ್ರಕಾಶಯತಿ ಯೋಽನಿಶಂ ।\\
ಲಘುದೀಪಂ ಪ್ರದಾಸ್ಯಾಮಿ ಜ್ಞಾನದೀಪಂ ಚ ದೇಹಿ ಮೇ ॥\\
ಸಾಜ್ಯಂ ತ್ರಿವರ್ತಿಸಂಯುಕ್ತಂ ವಹ್ನಿನಾ ಯೋಜಿತಂ ಮಯಾ ।\\
ಗೃಹಾಣ ಮಂಗಲಂ ದೀಪಂ ತ್ರೈಲೋಕ್ಯ ತಿಮಿರಾಪಹಮ್ ॥\\
ಅನೇಕಜನ್ಮಸಂಪ್ರಾಪ್ತಕರ್ಮಬಂಧವಿದಾಹಿನೇ ।\\
ಜ್ಞಾನಾನಲಪ್ರಭಾವೇನ ತಸ್ಮೈ ಶ್ರೀಗುರವೇ ನಮಃ ॥\as{ದೀಪಃ॥}

ಶರ್ಕರಾ ಘೃತಸಂಯುಕ್ತಂ ವ್ಯಂಜನಾದಿ ಸಮನ್ವಿತಮ್ ।\\
ಭಕ್ಷ್ಯ ಭೋಜ್ಯ ಸಮೋಪೇತಂ ನೈವೇದ್ಯಂ ಪ್ರತಿಗೃಹ್ಯತಾಮ್ ॥\as{ನೈವೇದ್ಯಮ್॥}

ಶೀತೋದಕಂ ಚ ಸುಸ್ವಾದು ಏಲಾಚೂರ್ಣಾದಿ ಮಿಶ್ರಿತಂ ।\\
ಮಧ್ಯೇ ಮಧ್ಯೇ ಪ್ರದಾಸ್ಯಾ,ಮಿ ಪಿಪಾಸಾ ವಿನಿವೃತ್ತಯೇ ॥\as{ಸ್ವಾದೂದಕಮ್॥}

ಗಂಗಾದಿ ಸರ್ವತೀರ್ಥೋದಬಿಂದುಭಿಃ ಪರಿಪೂರಿತಮ್ ।\\
ಉತ್ತರಾಪೋಶನಾರ್ಥಂ ತು ಗೃಹಾಣ ಜಲಮುತ್ತಮಮ್ ॥\\\as{ಉತ್ತರಾಪೋಶನಮ್॥}
\newpage
ಹಸ್ತಪ್ರಕ್ಷಾಲನಾರ್ಥೇ ಶುದ್ಧಜಲಂ ಸಮರ್ಪಯಾಮಿ ॥\\
ಗಂಡೂಷಾರ್ಥೇ ನಿರ್ಮಲೋದಕಂ ಸಮರ್ಪಯಾಮಿ ॥\\
ಮುಖಶುದ್ಧ್ಯರ್ಥೇ ಪರಿಮಲೋದಕಂ ಸಮರ್ಪಯಾಮಿ ॥\\
ಪಾದಪ್ರಕ್ಷಾಲನಾರ್ಥೇ ಗಂಗೋದಕಂ ಸಮರ್ಪಯಾಮಿ ॥\\
ಕರಶುದ್ಧ್ಯರ್ಥೇ ಕರ್ಪೂರಚಂದನಮಿಶ್ರಿತ ಜಲಂ ಸಮರ್ಪಯಾಮಿ ॥\\
ಆಚಮನಾರ್ಥೇ ಸರ್ವಸರಿತ್ತೀರ್ಥೋದಕಂ ಸಮರ್ಪಯಾಮಿ ॥

ಪೂರ್ಣಫಲಂ ತೇ ದಾಸ್ಯಾಮಿ ಪರಿಪೂರ್ಣಫಲ ಪ್ರದ ।\\
ಕುರು ಮೇ ಸಫಲಂ ಜನ್ಮ ಹ್ಯಖಂಡಫಲದಾನತಃ ॥\\
ನಾರಿಕೇಲಂ ಚ ದ್ರಾಕ್ಷಾ ಚ ದಾಡಿಮಾಂಜೂರಕಾನಿ ಚ ।\\
ಸುಫಲಾನಿ ಪ್ರದಾಸ್ಯಾಮಿ ಸ್ವೀಕುರುಷ್ವ ಜಗದ್ಗುರೋ ॥\\
ಇದಂ ಫಲಂ ಮಯಾ ದೇವ ಸ್ಥಾಪಿತಂ ಪುರತಸ್ತವ ।\\
ತೇನ ಮೇ ಸುಫಲಾವಾಪ್ತಿಃ ಭವೇಜ್ಜನ್ಮನಿ ಜನ್ಮನಿ ॥\as{ಪೂರ್ಣಫಲಮ್॥}

ದಕ್ಷಿಣಾಮೂರ್ತಿ ರೂಪಾಯ ಧನದಾದಿ ನಿಷೇವಿತ ।\\
ದಕ್ಷಿಣಾಂ ತೇ ಪ್ರದಾಸ್ಯಾಮಿ ಧನೈಷಣನಿವೃತ್ತಯೇ ॥\\
ಹಿರಣ್ಯಗರ್ಭಗರ್ಭಸ್ಥಂ ಹೇಮಬೀಜಂ ವಿಭಾವಸೋಃ ।\\
ಅನಂತಪುಣ್ಯ ಫಲದಂ ಅತಃ ಶಾಂತಿಂ ಪ್ರಯಚ್ಛ ಮೇ ॥\as{ಸುವರ್ಣಪುಷ್ಪದಕ್ಷಿಣಾ॥}

ಪೂಗೀಫಲಸಮಾಯುಕ್ತಂ ನಾಗವಲ್ಲೀ ದಲೈರ್ಯುತಂ~।\\
ಕರ್ಪೂರಚೂರ್ಣಸಂಯುಕ್ತಂ ತಾಂಬೂಲಂ ಪ್ರತಿಗೃಹ್ಯತಾಂ ॥\as{ತಾಂಬೂಲಮ್॥}

ಚಂದ್ರಾದಿತ್ಯೌ ಚ ಧರಣೀ ವಿದ್ಯುದಗ್ನಿಸ್ತಥೈವ ಚ ।\\
ತ್ವಮೇವ ಸರ್ವಜ್ಯೋತೀಂಷಿ ಆರ್ತಿಕ್ಯಂ ಪ್ರತಿಗೃಹ್ಯತಾಮ್ ॥\as{ನೀರಾಜನಮ್॥}
\newpage
ತುಲಸೀ ಬಿಲ್ವ ಪತ್ರಾಬ್ಜ ದೂರ್ವಾಂಕುರ ಕುಶೈಸ್ತಥಾ ।\\
ಮಲ್ಲಿಕಾ ಚಂಪಕಾ ಜಾತೀ ದ್ರೋಣಪುಷ್ಪೈಶ್ಚ ಸಂಯುತಮ್ ॥\\
ನಾನಾವಿಧ ಸುಪುಷ್ಪೈಶ್ಚ ನಾನಾ ಪತ್ರೈಶ್ಚ ಸಂಯುತಮ್ ।\\
ಮಂತ್ರಪುಷ್ಪಂ ಮಯಾ ದತ್ತಂ ಪ್ರೀತ್ಯಾ ಸ್ವೀಕುರು ಸದ್ಗುರೋ ॥\\
ಆಜನ್ಮ ಕೃತ ಕರ್ಮಾಣಿ ಸುಖದುಃಖಕರಾಣಿ ಚ ।\\
ಸಮರ್ಪಯಾಮಿ ಪುಷ್ಪಾಣಿ ಪುನರ್ಭವ ನಿವೃತ್ತಯೇ ॥\as{ಪ್ರದಕ್ಷಿಣಮ್॥}


\section{ನಮಸ್ಕಾರಾಃ }
ಗುರುರ್ಬ್ರಹ್ಮಾ ಗುರುರ್ವಿಷ್ಣುರ್ಗುರುರ್ದೇವೋ ಮಹೇಶ್ವರಃ ।\\
ಗುರುಸ್ಸಾಕ್ಷಾತ್ ಪರಬ್ರಹ್ಮ ತಸ್ಮೈ ಶ್ರೀಗುರವೇ ನಮಃ ॥

ಅಖಂಡಮಂಡಲಾಕಾರಂ ವ್ಯಾಪ್ತಂ ಯೇನ ಚರಾಚರಮ್ ।\\
ತತ್ಪದಂ ದರ್ಶಿತಂ ಯೇನ ತಸ್ಮೈ ಶ್ರೀಗುರವೇ ನಮಃ ॥

ಅಜ್ಞಾನತಿಮಿರಾಂಧಸ್ಯ ಜ್ಞಾನಾಂಜನಶಲಾಕಯಾ ।\\
ಚಕ್ಷುರುನ್ಮೀಲಿತಂ ಯೇನ ತಸ್ಮೈ ಶ್ರೀಗುರವೇ ನಮಃ ॥

ಸ್ಥಾವರಂ ಜಂಗಮಂ ವ್ಯಾಪ್ತಂ ಯತ್ಕಿಂಚಿತ್ಸಚರಾಚರಮ್ ।\\
ತ್ವಂ ಪದಂ ದರ್ಶಿತಂ ಯೇನ ತಸ್ಮೈ ಶ್ರೀಗುರವೇ ನಮಃ ॥

ಚಿನ್ಮಯವ್ಯಾಪಿತಂ ಸರ್ವಂ ತ್ರೈಲೋಕ್ಯಂ ಸಚರಾಚರಮ್ ।\\
ಅಸಿತ್ವಂ ದರ್ಶಿತಂ ಯೇನ ತಸ್ಮೈ ಶ್ರೀಗುರವೇ ನಮಃ ॥ 

ಸರ್ವಶ್ರುತಿಶಿರೋರತ್ನ ಸಮುದ್ಭಾಸಿತ ಮೂರ್ತಯೇ ।\\
ವೇದಾಂತಾಂಬುಜಸೂರ್ಯಾಯ ತಸ್ಮೈ ಶ್ರೀಗುರವೇ ನಮಃ ॥

ಚೈತನ್ಯಂ ಶಾಶ್ವತಂ ಶಾಂತಂ ವ್ಯೋಮಾತೀತಂ ನಿರಂಜನಮ್ ।\\
ನಾದಬಿಂದುಕಳಾತೀತಂ ತಸ್ಮೈ ಶ್ರೀಗುರವೇ ನಮಃ ॥

ಜ್ಞಾನಶಕ್ತಿಸಮಾರೂಢತತ್ತ್ವಮಾಲಾವಿಭೂಷಿಣೇ ।\\
ಭುಕ್ತಿಮುಕ್ತಿಪ್ರದಾತ್ರೇ ಚ ತಸ್ಮೈ ಶ್ರೀಗುರವೇ ನಮಃ ॥

ಅನೇಕಜನ್ಮಸಂಪ್ರಾಪ್ತಕರ್ಮಬಂಧವಿದಾಹಿನೇ ।\\
ಜ್ಞಾನಾನಲಪ್ರಭಾವೇನ ತಸ್ಮೈ ಶ್ರೀಗುರವೇ ನಮಃ ॥ 

ಶೋಷಣಂ ಭವಸಿಂಧೋಶ್ಚ ದೀಪನಂ ಕ್ಷರಸಂಪದಾಮ್ ।\\
ಗುರೋಃ ಪಾದೋದಕಂ ಯಸ್ಯ ತಸ್ಮೈ ಶ್ರೀಗುರವೇ ನಮಃ ॥ 

ನ ಗುರೋರಧಿಕಂ ತತ್ತ್ವಂ ನ ಗುರೋರಧಿಕಂ ತಪಃ ।\\
ನ ಗುರೋರಧಿಕಂ ಜ್ಞಾನಂ ತಸ್ಮೈ ಶ್ರೀಗುರವೇ ನಮಃ ॥ 

ಮನ್ನಾಥಃ ಶ್ರೀಜಗನ್ನಾಥೋ ಮದ್ಗುರುಃ ಶ್ರೀಜಗದ್ಗುರುಃ ।\\
ಮಮಾಽಽತ್ಮಾ ಸರ್ವಭೂತಾತ್ಮಾ ತಸ್ಮೈ ಶ್ರೀಗುರವೇ ನಮಃ ॥

ಗುರುರಾದಿರನಾದಿಶ್ಚ ಗುರುಃ ಪರಮದೈವತಮ್ ।\\
ಗುರುಮಂತ್ರಸಮೋ ನಾಸ್ತಿ ತಸ್ಮೈ ಶ್ರೀಗುರವೇ ನಮಃ ॥

ಬ್ರಹ್ಮಾನಂದಂ ಪರಮಸುಖದಂ ಕೇವಲಂ ಜ್ಞಾನಮೂರ್ತಿಂ\\
ದ್ವಂದ್ವಾತೀತಂ ಗಗನಸದೃಶಂ ತತ್ತ್ವಮಸ್ಯಾದಿಲಕ್ಷ್ಯಂ~।\\
ಏಕಂ ನಿತ್ಯಂ ವಿಮಲಮಚಲಂ ಸರ್ವಧೀಸಾಕ್ಷಿಭೂತಂ\\
ಭಾವಾತೀತಂ ತ್ರಿಗುಣರಹಿತಂ ಸದ್ಗುರುಂ ತಂ ನಮಾಮಿ~॥

ಶ್ರೀನಂದನಾಥಾಯ ಮಮ ಗುರವೇ ನಮಃ ॥\\
ಶ್ರೀನಂದನಾಥಾಯ ಮಮ ಪರಮಗುರವೇ ನಮಃ ॥\\
ಶ್ರೀನಂದನಾಥಾಯ ಮಮ ಪರಮೇಷ್ಠಿಗುರವೇ ನಮಃ ॥\as{ನಮಸ್ಕಾರಾಃ॥}
\newpage
ತುಲಸೀ ಬಿಲ್ವಪತ್ರೈಶ್ಚ ಕುಶದೂರ್ವಾಕ್ಷತೈಃ ಸಹ ।\\
ಪ್ರಸನ್ನಾರ್ಘ್ಯಂ ಪ್ರದಾಸ್ಯಾಮಿ ಗೃಹ್ಯತಾಂ ಗುರುಪುಂಗವ ॥\\
\as{ಹಂಸಹಂಸಾಯ ವಿದ್ಮಹೇ ಪರಮಹಂಸಾಯ ಧೀಮಹಿ ।\\
ತನ್ನೋ ಹಂಸಃ ಪ್ರಚೋದಯಾತ್ ॥ಪ್ರಸನ್ನಾರ್ಘ್ಯಮ್॥}

\section{ಪ್ರಾರ್ಥನಾ}
ಸಹಸ್ರದಲಪಂಕಜೇ ಸಕಲಶೀತರಶ್ಮಿಪ್ರಭಂ\\
ವರಾಭಯಕರಾಂಬುಜಂ ವಿಮಲಗಂಧಪುಷ್ಪಾಂಬರಂ~।\\
ಪ್ರಸನ್ನವದನೇಕ್ಷಣಂ ಸಕಲದೇವತಾರೂಪಿಣಂ\\
ಸ್ಮರೇಚ್ಛಿರಸಿ ಹಂಸಗಂ ತದಭಿಧಾನಪೂರ್ವಂ ಗುರುಂ ॥

ಬ್ರಹ್ಮಾನಂದಂ ಪರಮಸುಖದಂ ಕೇವಲಂ ಜ್ಞಾನಮೂರ್ತಿಂ\\
ದ್ವಂದ್ವಾತೀತಂ ಗಗನಸದೃಶಂ ತತ್ತ್ವಮಸ್ಯಾದಿಲಕ್ಷ್ಯಂ~।\\
ಏಕಂ ನಿತ್ಯಂ ವಿಮಲಮಚಲಂ ಸರ್ವಧೀಸಾಕ್ಷಿಭೂತಂ\\
ಭಾವಾತೀತಂ ತ್ರಿಗುಣರಹಿತಂ ಸದ್ಗುರುಂ ತಂ ನಮಾಮಿ~॥

ಹೃದಂಬುಜೇ ಕರ್ಣಿಕಮಧ್ಯಸಂಸ್ಥೇ\\ ಸಿಂಹಾಸನೇ ಸಂಸ್ಥಿತದಿವ್ಯಮೂರ್ತಿಂ~।\\
ಧ್ಯಾಯೇದ್ಗುರುಂ ಚಂದ್ರಕಲಾಪ್ರಕಾಶಂ\\ ಸಚ್ಚಿತ್ಸುಖಾಭೀಷ್ಟವರಂ ದಧಾನಂ~॥

ಶ್ವೇತಾಂಬರಂ ಶ್ವೇತವಿಲೇಪಪುಷ್ಪಂ\\ ಮುಕ್ತಾವಿಭೂಷಂ ಮುದಿತಂ ದ್ವಿನೇತ್ರಂ~।\\
ವಾಮಾಂಕಪೀಠಸ್ಥಿತದಿವ್ಯಶಕ್ತಿಂ\\ ಮಂದಸ್ಮಿತಂ ಪೂರ್ಣಕೃಪಾನಿಧಾನಂ~॥

ಜ್ಞಾನಸ್ವರೂಪಂ ನಿಜಭಾವಯುಕ್ತಂ\\ ಆನಂದಮಾನಂದಕರಂ ಪ್ರಸನ್ನಂ~।\\
ಯೋಗೀಂದ್ರಮೀಡ್ಯಂ ಭವರೋಗವೈದ್ಯಂ\\ ಶ್ರೀಮದ್ಗುರುಂ ನಿತ್ಯಮಹಂ ನಮಾಮಿ~॥

ವಂದೇ ಗುರೂಣಾಂ ಚರಣಾರವಿಂದಂ\\ ಸಂದರ್ಶಿತಸ್ವಾತ್ಮಸುಖಾವಬೋಧಮ್~।\\
ಜನಸ್ಯ ಯೋ ಜಾಂಗಲಿಕಾಯಮಾನಂ \\ಸಂಸಾರಹಾಲಾಹಲಮೋಹಶಾಂತ್ಯೈ~॥

ಗುರುರ್ಬ್ರಹ್ಮಾ ಗುರುರ್ವಿಷ್ಣುರ್ಗುರುರ್ದೇವೋ ಮಹೇಶ್ವರಃ ।\\
ಗುರುಸ್ಸಾಕ್ಷಾತ್ ಪರಬ್ರಹ್ಮ ತಸ್ಮೈ ಶ್ರೀಗುರವೇ ನಮಃ ॥
\section{ಪುನಃ ಪ್ರಸನ್ನಪೂಜಾ}
ಸದ್ಗುರವೇ ನಮಃ । \as{ಧ್ಯಾನಂ}\\
ಜ್ಞಾನಸಾಗರಾಯ ನಮಃ । \as{ಆವಾಹನಮ್}\\
ಭಕ್ತವತ್ಸಲಾಯ ನಮಃ । \as{ಆಸನಮ್}\\
ಸೌಭಾಗ್ಯದಾಯಕಾಯ ನಮಃ । \as{ಪಾದ್ಯಮ್}\\
ನಿರ್ಮಲಾಯ ನಮಃ ।\as{ಅರ್ಘ್ಯಮ್}\\
ಗೂಢವಿದ್ಯಾಯ ನಮಃ ।\as{ಆಚಮನಮ್}\\
ಆಕಾಶರೂಪಾಯ ನಮಃ ।\as{ಸ್ನಾನಮ್}\\
ನಿರ್ವಿಕಾರಾಯ ನಮಃ ।\as{ವಸ್ತ್ರಮ್}\\
ಅವ್ಯಕ್ತಾಯ ನಮಃ ।\as{ಉಪವೀತಮ್}\\
ವ್ಯಕ್ತಾಯ ನಮಃ ।\as{ಆಭರಣಮ್}\\
ನಿರಂಜನಾಯ ನಮಃ ।\as{ಗಂಧಮ್}\\
ತೇಜೋರೂಪಾಯ ನಮಃ ।\as{ಅಕ್ಷತಾಃ}\\
ಜ್ಯೋತೀರೂಪಾಯ ನಮಃ ।\as{ಪುಷ್ಪಮ್}\\
ಪ್ರಕಾಶಾಯ ನಮಃ ।\as{ಧೂಪಃ}\\
ಜ್ಞಾನಮೂರ್ತಯೇ ನಮಃ ।\as{ದೀಪಃ}\\
ಜಗದ್ಗುರವೇ ನಮಃ ।\as{ನೈವೇದ್ಯಮ್}\\
ಅಕಲ್ಮಷಾಯ ನಮಃ ।\as{ತಾಂಬೂಲಮ್}\\
ಬ್ರಹ್ಮಾನಂದಾಯ ನಮಃ ।\as{ನೀರಾಜನಮ್}\\
ಅಜ್ಞಾನಹರಣಾಯ ನಮಃ ।\as{ಮಂತ್ರಪುಷ್ಪಮ್}\\
ವೇದಮೂರ್ತಯೇ ನಮಃ ।\as{ಪ್ರದಕ್ಷಿಣನಮಸ್ಕಾರಾಃ}\\
ಪರಾತ್ಪರಾಯ ನಮಃ ।\as{ಪ್ರಾರ್ಥನಾ}\\
ಪ್ರಣವಸ್ವರೂಪಾಯ ನಮಃ ।\as{ಛತ್ರಮ್}\\
ಮಂಗಲಾಯ ನಮಃ ।\as{ಚಾಮರಯುಗಲಮ್}\\
ಕರುಣಾಕರಾಯ ನಮಃ ।\as{ದರ್ಪಣಮ್}\\
ಮೋಕ್ಷದಾಯಕಾಯ ನಮಃ ।\as{ಅಶ್ವಾರೋಹಣಮ್}\\
ರೂಪಾತೀತಾಯ ನಮಃ ।\as{ಗಜಾರೋಹಣಮ್}\\
ಗುಣಾತೀತಾಯ ನಮಃ ।\as{ರಥಾರೋಹಣಮ್}\\
ಚರಾಚರಾತ್ಮಕಾಯ ನಮಃ । \as{ಸರ್ವೋಪಚಾರ ಪೂಜಾಂ ಸಮರ್ಪಯಾಮಿ ॥}
\begin{center}{\LARGE\bfseries ಓಂ ತತ್ಸತ್}\end{center}
\chapter*{\center  ಬಾಣಂತಿ ಅಮ್ಮನವರ ಪೂಜೆ}
ಜಗತ್ಪ್ರಸೂತಿಕೇ ದೇವಿ ಸರ್ವತ್ರ ಸುರಪೂಜಿತೇ ।\\
ಬ್ರಹ್ಮವಿಷ್ಣ್ವೀಶ ಸಂಸ್ತುತ್ಯೇ ಲೋಕಮಾತರ್ಮಹೇಶ್ವರಿ ॥\as{ಧ್ಯಾನಮ್॥}

ಸುಮಧ್ಯಮಾಂ ಸುವಸನಾಂ ಚಂದ್ರಬಿಂಬಾಧರಾನ್ವಿತಾಮ್ ।\\
ಆವಾಹಯಾಮಿ ದೇವಿ ತ್ವಾಂ ಸರ್ವದಾ ಶುಭಕಾರಿಣಿ ॥\as{ಆವಾಹನಮ್॥}

ರತ್ನಸಿಂಹಾಸನಮಿದಂ ವಿಶ್ವಕರ್ಮ ವಿನಿರ್ಮಿತಮ್ ।\\
ದಿವ್ಯಸಿಂಹಾಸನಂ ತುಭ್ಯಂ ದಾಸ್ಯಾಮಿ ಶಿವಶಕ್ತಯೇ ॥\as{ಆಸನಮ್॥}

ಮಹಾದೇವಿ ಜಗನ್ಮಾತಃ ಅರುಣಾಂಬರ ಭೂಷಿತೇ ।\\
ಪಾದ್ಯಂ ಗೃಹಾಣ ದೇವಿ ತ್ವಂ ಗಂಧಪುಷ್ಪಾಕ್ಷತೈರ್ಯುತಮ್ ॥\as{ಪಾದ್ಯಮ್॥}

ಶ್ರೀಪಾರ್ವತಿ ಮಹಾಭಾಗೇ ಕಾಮೇಶ ಪ್ರಿಯಭಾಮಿನಿ ।\\
ಅರ್ಘ್ಯಂ ಗೃಹಾಣ ದೇವೇಶಿ ಭರ್ತ್ರಾ ಸಹ ಶಿವಂಕರಿ ॥\as{ಅರ್ಘ್ಯಮ್॥}

ಆಚಮ್ಯತಾಂ ಜಲಂ ದಿವ್ಯಂ ಪುಣ್ಯತೀರ್ಥಸಮುದ್ಭವಮ್ ।\\
ಶಿವೇ ಸುಶೀಲೇ ಕಲ್ಯಾಣಿ ಪಾಹಿ ಮಾಂ ಶರಣಾಗತಮ್ ॥\as{ಆಚಮನಮ್॥}

ದಧಿಮಧ್ವಾಜ್ಯ ಸಂಯುಕ್ತಂ ಶರ್ಕರಾಕ್ಷೀರ ಸಂಯುತಮ್ ।\\
ಮಧುಪರ್ಕಂ ಗೃಹಾಣೇಮಂ ಅರ್ಪಯಾಮಿ ಶಿವಪ್ರಿಯೇ ॥\as{ಮಧುಪರ್ಕಃ॥}

ಉಷ್ಣೋದಕಮಿದಂ ದೇವಿ ನಾನಾತೀರ್ಥ ಸಮನ್ವಿತಮ್ ।\\
ದಿವ್ಯೌಷಧಿ ಸಮಾಯುಕ್ತಂ ಸ್ನಾನಾರ್ಥಂ ಪ್ರತಿಗೃಹ್ಯತಾಮ್ ॥\\\as{ಉಷ್ಣೋದಕಸ್ನಾನಮ್॥}

ಜಾಹ್ನವೀ ತೋಯಮಾನೀತಂ ಕರ್ಪೂರೇಣ ಸಮನ್ವಿತಮ್ ।\\
ಸ್ನಪಯಾಮಿ ಸುರಶ್ರೇಷ್ಠೇ ತ್ವಾಂ ಪುತ್ರಾದಿ ಫಲಪ್ರದೇ ॥\\\as{ಮಲಾಪಕರ್ಷಣಸ್ನಾನಮ್॥}

ಕಾಲಾಂಬುದಸಮಾನಾಭಂ ಕಾಲಾಂತಕಕುಟುಂಬಿನಿ ।\\
ಕಾಲಾಗರುಜ ಧೂಪೈಶ್ಚ ಧೂಪಯೇ ಕೇಶಪಾಶಕಮ್ ॥\as{ಧೂಪನಮ್॥}

ಕೇಸರೀವಾಸಿತಂ ಕ್ಷೀರಂ ಕಪಿಲಾದೋಹನೋದ್ಭವಮ್ ।\\
ಮಂದೋಷ್ಣಂ ಮದನಾರಾತಿ ಕಾಮಿನಿ ಪ್ರತಿಗೃಹ್ಯತಾಮ್ ॥\as{ಕ್ಷೀರಮ್॥}

ವೈಡೂರ್ಯಾದಿಸುರತ್ನಾಢ್ಯೇ ಹಂಸತೂಲ ಸುಶೋಭಿತೇ ।\\
ದಿವ್ಯಾಂಬರಸಮಾಯುಕ್ತೇ ಮಂಚೇ ತ್ವಂ ಶಯನಂ ಕುರು ॥\\\as{ಹಂಸತೂಲಿಕಾತಲ್ಪಃ॥}

ವಸ್ತ್ರಂ ಚ ಸೋಮದೈವತ್ಯಂ ಲಜ್ಜಾಯಾಸ್ತು ನಿವಾರಕಮ್ ।\\
ಪ್ರಯಚ್ಛಾಮಿ ತವ ಪ್ರೀತ್ಯೈ ಮಮಾಭೀಷ್ಟಪ್ರದಾ ಭವ ॥\as{ವಸ್ತ್ರಮ್॥}

ಬ್ರಹ್ಮಸೂತ್ರಂ ಸ್ವರ್ಣಮಯಂ ತ್ರಿವೃತಂ ರತ್ನಸಂಯುತಮ್ ।\\
ಉಪವೀತಂ ತು ದಾಸ್ಯಾಮಿ ನಾರಾಯಣ ಸಹೋದರಿ ॥\as{ಉಪವೀತಮ್॥}

ಕಚೋರಾದಿಸುಗಂಧಾಢ್ಯಂ ಉತ್ತಮಂ ದಿವ್ಯಚಂದನಮ್ ।\\
ವಿಲೇಪನಂ ಸುರಾಧೀಶೇ ಪ್ರೀತ್ಯರ್ಥಂ ಪ್ರತಿಗೃಹ್ಯತಾಮ್ ॥\as{ಗಂಧಃ॥}

ಸೌಭಾಗ್ಯಶುಭದೇ ದೇವಿ ಸರ್ವಮಂಗಳದಾಯಿನಿ ।\\
ಹರಿದ್ರಾಂ ತೇ ಪ್ರದಾಸ್ಯಾಮಿ ಗೃಹಾಣ ಪರಮೇಶ್ವರಿ ॥\as{ಹರಿದ್ರಾ॥}

ಕುಂಕುಮಂ ಕಾಂತಿದಂ ದಿವ್ಯಂ ಸರ್ವಕಾರ್ಯ ಫಲಪ್ರದಂ ।\\
ಕುಂಕುಮೇನಾರ್ಚಿತೇ ದೇವಿ ಗೃಹಾಣ ವರದಾ ಭವ ॥\as{ಕುಂಕುಮಮ್॥}

ಶ್ರುತ್ಯಂತರತ್ನ ಸಂದೋಹ ನೀರಾಜಿತ ಪದಾಂಬುಜೇ ।\\
ಅಲಂಕರೋಮಿ ತ್ವಾಂ ಭಕ್ತ್ಯಾ ಕೇಯೂರ ಕಟಕಾದಿಭಿಃ ॥\as{ಆಭರಣಮ್॥}

ಶಾಲೇಯನ್ ಚಂದ್ರಸಂಕಾಶಾನ್ ಹರಿದ್ರಾಮಿಶ್ರಿತಾನ್ ಶುಭಾನ್ ।\\
ಅಕ್ಷತಾನ್ ಅರ್ಪಯೇ ತುಭ್ಯಂ ಗೃಹಾಣ ಪರಮೇಶ್ವರಿ ॥\as{ಅಕ್ಷತಾಃ॥}

ಜಾಜೀ ಪುನ್ನಾಗ ಮಂದಾರ ಕೇತಕೀ ಚಂಪಕಾನಿ ಚ ।\\
ಪುಷ್ಪಾಣಿ ತವ ಪೂಜಾರ್ಥಂ ಅರ್ಪಯೇ ಪರಮೇಶ್ವರಿ ॥\as{ಪುಷ್ಪಾಣಿ॥}

\section{ಅಂಗಪೂಜಾ}
ಉಮಾಯೈ ನಮಃ । ಪಾದೌ ಪೂಜಯಾಮಿ~।\\
ಗೌರ್ಯೈ ನಮಃ । ಗುಲ್ಫೌ ಪೂಜಯಾಮಿ~।\\
ಪಾರ್ವತ್ಯೈ ನಮಃ । ಜಂಘೇ ಪೂಜಯಾಮಿ~।\\
ಜಗತ್ಪ್ರತಿಷ್ಠಾಯೈ ನಮಃ । ಜಾನುನೀ ಪೂಜಯಾಮಿ~।\\
ಮೂಲಪ್ರಕೃತ್ಯೈ ನಮಃ । ಊರೂ ಪೂಜಯಾಮಿ~।\\
ಅಂಬಿಕಾಯೈ ನಮಃ । ಕಟಿಂ ಪೂಜಯಾಮಿ~।\\
ಅನ್ನಪೂರ್ಣಾಯೈ ನಮಃ । ನಾಭಿಂ ಪೂಜಯಾಮಿ~।\\
ವಸುಂಧರಾಯೈ ನಮಃ । ಉದರಂ ಪೂಜಯಾಮಿ~।\\
ಮಹಾಬಲಾಯೈ ನಮಃ । ಸ್ತನೌ ಪೂಜಯಾಮಿ~।\\
ವರಪ್ರದಾಯೈ ನಮಃ । ವಕ್ಷಸ್ಥಲಂ ಪೂಜಯಾಮಿ~।\\
ಕಂಬುಕಂಠ್ಯೈ ನಮಃ । ಬಾಹೂನ್ ಪೂಜಯಾಮಿ~।\\
ಕಾತ್ಯಾಯನ್ನಯೈ ನಮಃ । ಹಸ್ತಾನ್ ಪೂಜಯಾಮಿ~।\\
ಬ್ರಹ್ಮವಿದ್ಯಾಯೈ ನಮಃ । ಕಂಠಂ ಪೂಜಯಾಮಿ~।\\
ಶಾಂಕರ್ಯೈ ನಮಃ । ಜಿಹ್ವಾಂ ಪೂಜಯಾಮಿ~।\\
ಶಿವಾಯೈ ನಮಃ । ಮುಖಂ ಪೂಜಯಾಮಿ~।\\
ಶುಭದಾಯೈ ನಮಃ । ನೇತ್ರೇ ಪೂಜಯಾಮಿ~।\\
ರುದ್ರಾಣ್ಯೈ ನಮಃ । ಕರ್ಣೌ ಪೂಜಯಾಮಿ~।\\
ಸರ್ವಮಂಗಳಾಯೈ ನಮಃ । ಲಲಾಟಂ ಪೂಜಯಾಮಿ~।\\
ಸರ್ವೇಶ್ವರ್ಯೈ ನಮಃ । ಶಿರಃ ಪೂಜಯಾಮಿ~।\\
ಶ್ರೀದೇವ್ಯೈ ನಮಃ । ಸರ್ವಾಂಗಂ ಪೂಜಯಾಮಿ~।

ಶ್ರೀಖಂಡಲಾಕ್ಷಾಗರು ಗುಗ್ಗುಲಾದ್ಯೈಃ \\
ಕಚೋರಶ್ರೀವಾಸ ಮಧೂಕಯುಕ್ತೈಃ ।\\
ಸುವಾಸಿತಂ ಧೂಪಮಿದಂ ಗೃಹಾಣ \\
ಭಕ್ತೌಘ ಸಂಪೂಜಿತ ಪಾದಪದ್ಮೇ ॥\as{ಧೂಪಃ॥}

ಸಾಜ್ಯಂ ತ್ರಿವರ್ತಿಸಂಯುಕ್ತಂ ವಹ್ನಿನಾ ಯೋಜಿತಂ ಮಯಾ~।\\
ಗೃಹಾಣ ಮಂಗಳಂ ದೀಪಂ ತ್ರೈಲೋಕ್ಯತಿಮಿರಾಪಹಂ ॥\as{ದೀಪಃ॥}

ಅನ್ನಂ ಚತುರ್ವಿಧಂ ಸ್ವಾದು ರಸೈಃ ಷಡ್ಭಿಃ ಸಮನ್ವಿತಮ್ ।\\
ಭಕ್ಷ್ಯ ಭೋಜ್ಯ ಸಮಾಯುಕ್ತಂ ನೈವೇದ್ಯಂ ಪ್ರತಿಗೃಹ್ಯತಾಮ್ ॥\as{ನೈವೇದ್ಯಮ್॥}

ಇದಂ ಫಲಂ ಮಯಾದೇವ ಸ್ಥಾಪಿತಂ ಪುರತಸ್ತವ~।\\
ತೇನ ಮೇ ಸುಫಲಾವಾಪ್ತಿಃ ಭವೇಜ್ಜನ್ಮನಿ ಜನ್ಮನಿ ॥\as{ಪೂರ್ಣಫಲಮ್॥}

ಹಿರಣ್ಯಗರ್ಭಗರ್ಭಸ್ಥಂ ಹೇಮಬೀಜಂ ವಿಭಾವಸೋಃ~।\\
ಅನಂತಪುಣ್ಯಫಲದಂ ಅತಃ ಶಾಂತಿಂ ಪ್ರಯಚ್ಛ ಮೇ ॥\as{ಸುವರ್ಣಪುಷ್ಪದಕ್ಷಿಣಾ॥}
\newpage
ಕ್ರಮುಕೈರ್ನಾಗಪತ್ರೈಶ್ಚ ಕರ್ಪೂರಮಿಶ್ರ ಚೂರ್ಣಕೈಃ ।\\
ಸಂಯುಕ್ತಂ ದೇವಿ ತಾಂಬೂಲಂ ಗೃಹಾಣ ಶಿವವಲ್ಲಭೇ ॥\as{ತಾಂಬೂಲಮ್॥}

ನೀರಾಜನಂ ಸಕರ್ಪೂರಂ ಘೃತವರ್ತಿ ಸಮನ್ವಿತಮ್ ।\\
ಗೃಹಾಣ ಮಂಗಳಂ ದೀಪಂ ಮಹಾದೇವಿ ನಮೋಽಸ್ತು ತೇ ॥\as{ನೀರಾಜನಮ್॥}

ಏಲಾ ಲವಂಗ ಪುನ್ನಾಗ ದ್ರೋಣ ಪುಷ್ಪ ಸಮನ್ವಿತಮ್ ।\\
ಪಾದಪದ್ಮೇ ಮಂತ್ರಪುಷ್ಪಂ ಅರ್ಪಯಾಮಿ ಮಹಾಶಿವೇ ॥\as{ಮಂತ್ರಪುಷ್ಪಮ್॥}

ಪ್ರದಕ್ಷಿಣತ್ರಯಂ ದೇವಿ ಪ್ರಯತ್ನೇನ ಮಯಾ ಕೃತಮ್ ।\\
ತೇನ ಪಾಪಾನಿ ಸರ್ವಾಣಿ ವ್ಯಪೋಹಂತು ಸದಾ ಮಮ ॥

ಯಾನಿ ಕಾನಿ ಚ ಪಾಪಾನಿ ಜನ್ಮಾಂತರಕೃತಾನಿ ಚ ।\\
ತಾನಿ ತಾನಿ ವಿನಶ್ಯಂತಿ ಪ್ರದಕ್ಷಿಣಪದೇ ಪದೇ ॥\as{ನಮಸ್ಕಾರಾಃ॥}

ಪಾಪೋಽಹಂ ಪಾಪಕರ್ಮಾಹಂ ಪಾಪಾತ್ಮಾ ಪಾಪಸಂಭವ ।\\
ತ್ರಾಹಿ ಮಾ ಕೃಪಯಾ ದೇವಿ ಶರಣಾಗತವತ್ಸಲೇ ॥

ಅನ್ಯಥಾ ಶರಣಂ ನಾಸ್ತಿ ತ್ವಮೇವ ಶರಣಂ ಮಮ ।\\
ತಸ್ಮಾತ್ಕಾರುಣ್ಯಭಾವೇನ ರಕ್ಷ ಮಾಮ್ ಜಗದೀಶ್ವರಿ ॥\as{ಪ್ರದಕ್ಷಿಣಮ್॥}

ನಮಸ್ತೇ ದೇವ ದೇವೇಶಿ ನಮಸ್ತೇ ಕರುಣಾರ್ಣವೇ ।\\
ಪ್ರಸನ್ನಾರ್ಘ್ಯಂ ಮಯಾ ದತ್ತಂ ಗೃಹಾಣ ಪರಮೇಶ್ವರಿ ॥

ನಮೋನಮಸ್ತೇ ಜಗದೇಕನಾಥೇ\\ ನಮೋನಮಃ ಶ್ರೀತ್ರಿಪುರಾಭಿಧಾನೇ ।\\
ನಮೋನಮೋ ಭಂಡಮಹಾಸುರಘ್ನೇ\\ ನಮೋಽಸ್ತು ಕಾಮೇಶ್ವರಿ ವಾಮಕೇಶಿ ॥

ಚಿಂತಾಮಣೇ ಚಿಂತಿತದಾನದಕ್ಷೇ\\ಽಚಿಂತ್ಯೇ ಚಿರಾಕಾರತರಂಗಮಾಲೇ ।\\
ಚಿತ್ರಾಂಬರೇ ಚಿತ್ರಜಗತ್ಪ್ರಸೂತೇ\\ ಚಿತ್ರಾಖ್ಯನಿತ್ಯೇ ಸುಖದೇ ನಮಸ್ತೇ ॥

ಸವಾರುಣೇ ಸಾಂದ್ರಸುಧಾಂಶುಶೀತೇ\\ ಸಾರಂಗಶಾವಾಕ್ಷಿ ಸರೋಜವಕ್ತ್ರೇ ।\\
ಸಾರಸ್ಯ ಸಾರಸ್ಯ ಸದೈಕಭೂಮೇ\\ ಸಮಸ್ತವಿದ್ಯೇಶ್ವರಿ ಸಂನತಿಸ್ತೇ ॥

ಪುತ್ರಾನ್ ದೇಹಿ ಧನಂ ದೇಹಿ ಸೌಭಾಗ್ಯಂ ದೇಹಿ ಸುವ್ರತೇ ।\\
ಅನ್ಯಾಂಶ್ಚ ಸರ್ವಕಾಮಾಂಶ್ಚ ದೇಹಿ ದೇವಿ ನಮೋಽಸ್ತು ತೇ ॥\as{ನಮಸ್ಕಾರಾಃ॥}

(ಉಡಿತುಂಬುವುದು)\\
ತಂಡುಲಂ ನಾರಿಕೇಲಂ ಚ ಗುಡವಸ್ತ್ರ ಸಮನ್ವಿತಮ್ ।\\
ಮಯಾ ದತ್ತಂ ಗೃಹಾಣೇದಂ ಪ್ರಸೀದ ಪರಮೇಶ್ವರಿ ॥

\chapter*{\center ದುರ್ಗಾಕಲ್ಪಃ}
ಗುರುರ್ಬ್ರಹ್ಮಾ ಗ್ರುರುರ್ವಿಷ್ಣುಃ ಗುರುರ್ದೇವೋ ಮಹೇಶ್ವರಃ~।\\
ಗುರುಃ ಸಾಕ್ಷಾತ್ ಪರಂ ಬ್ರಹ್ಮ ತಸ್ಮೈ ಶ್ರೀ ಗುರವೇ ನಮಃ ॥

ಸಚ್ಚಿದಾನಂದ ರೂಪಾಯ ಬಿಂದು ನಾದಾಂತರಾತ್ಮನೇ~।\\
ಆದಿಮಧ್ಯಾಂತ ಶೂನ್ಯಾಯ ಗುರೂಣಾಂ ಗುರವೇ ನಮಃ ॥

ಶ್ರೀಗುರವೇ ನಮಃ ।  ಶ್ರೀಪಾ।ಪೂ।ನಮಃ ।\\
ಪರಮಗುರವೇ ನಮಃ ।  ಶ್ರೀಪಾ।ಪೂ।ನಮಃ ।\\
ಪರಮೇಷ್ಠಿಗುರವೇ ನಮಃ ।  ಶ್ರೀಪಾ।ಪೂ।ನಮಃ ।

\section{ದ್ವಾರಪೂಜಾ}
\as{ಓಂ ಐಂಹ್ರೀಂಶ್ರೀಂ ವಂ} ವಟುಕಾಯ ನಮಃ(ದ್ವಾರಸ್ಯ ಅಧಃ)\\
\as{೪ ಭಂ} ಭದ್ರಕಾಲ್ಯೈ ನಮಃ (ದ್ವಾರಸ್ಯ ದಕ್ಷಭಾಗೇ)\\
\as{೪ ಭಂ} ಭೈರವಾಯ ನಮಃ(ದ್ವಾರಸ್ಯ ವಾಮಭಾಗೇ)\\
\as{೪ ಲಂ} ಲಂಬೋದರಾಯ ನಮಃ(ದ್ವಾರಸ್ಯ ಊರ್ಧ್ವಭಾಗೇ)

\dhyana{ಓಂ ಐಂ} ಆತ್ಮತತ್ವಂ ಶೋಧಯಾಮಿ ನಮಃ ಸ್ವಾಹಾ।\\
\dhyana{ಓಂ ಹ್ರೀಂ} ವಿದ್ಯಾತತ್ವಂ ಶೋಧಯಾಮಿ ನಮಃ ಸ್ವಾಹಾ।\\
\dhyana{ಓಂ ಕ್ಲೀಂ} ಶಿವತತ್ವಂ ಶೋಧಯಾಮಿ ನಮಃ ಸ್ವಾಹಾ।\\
\dhyana{ಓಂ ಐಂ ಹ್ರೀಂ ಕ್ಲೀಂ ಚಾಮುಂಡಾಯೈ ವಿಚ್ಚೇ}\\ ಸರ್ವತತ್ವಂ ಶೋಧಯಾಮಿ ನಮಃ ಸ್ವಾಹಾ।

೪ ಹೇ ಘಂಟೇ ಸುಸ್ವರೇ ರಮ್ಯೇ ಘಂಟಾಧ್ವನಿವಿಭೂಷಿತೇ।\\
ವಾದಯಂತಿ ಪರಾನಂದೇ ಘಂಟಾದೇವಂ ಪ್ರಪೂಜಯೇ॥\\
\as{ಓಂ ಜಗದ್ಧ್ವನಿಮಂತ್ರಮಾತಃ ಸ್ವಾಹಾ ॥}\\
ಆಗಮಾರ್ಥಂ ಚ ದೇವಾನಾಂ ಗಮನಾರ್ಥಂ ಚ ರಕ್ಷಸಾಂ।\\
ಕುರ್ಯಾತ್ ಘಂಟಾರವಂ ತತ್ರ ದೇವತಾಹ್ವಾನ ಲಾಂಛನಂ॥

\as{ಓಂ ಯೇಭ್ಯೋ᳚ ಮಾ॒ತಾ ಮಧು॑ಮ॒ತ್ಪಿನ್ವ॑ತೇ॒ ಪಯಃ॑ ಪೀ॒ಯೂಷಂ॒ ದ್ಯೌರದಿ॑ತಿ॒ರದ್ರಿ॑ ಬರ್ಹಾಃ । ಉ॒ಕ್ಥ ಶು॑ಷ್ಮಾನ್ ವೃಷಭ॒ರಾನ್ತ್ಸ್ವಪ್ನ॑ ಸ॒ಸ್ತಾँ ಆ᳚ದಿ॒ತ್ಯಾँ ಅನು॑ಮದಾ ಸ್ವ॒ಸ್ತಯೇ᳚ ॥\\ ಏ॒ವಾ ಪಿ॒ತ್ರೇ ವಿ॒ಶ್ವದೇ᳚ವಾಯ॒ ವೃಷ್ಣೇ᳚
ಯ॒ಜ್ಞೈರ್ವಿ॑ಧೇಮ॒ ನಮ॑ಸಾ ಹ॒ವಿರ್ಭಿಃ॑ । ಬೃಹ॑ಸ್ಪತೇಸುಪ್ರ॒ಜಾ
ವೀ॒ರವ᳚ನ್ತೋ ವ॒ಯಂ ಸ್ಯಾ᳚ಮ॒ ಪತ॑ಯೋ ರಯೀ॒ಣಾಮ್॥}

ವಿಷ್ಣುಶಕ್ತಿಸಮೋಪೇತೇ ಸರ್ವವರ್ಣೇ ಮಹೀತಲೇ~।\\
ಅನೇಕರತ್ನಸಂಭೂತೇ ಭೂಮಿದೇವಿ ನಮೋಽಸ್ತು ತೇ॥

ಪೃಥ್ವೀತಿ ಮಂತ್ರಸ್ಯ ಮೇರುಪೃಷ್ಠ ಋಷಿಃ~। ಸುತಲಂ ಛಂದಃ~।\\ಆದಿಕೂರ್ಮೋ ದೇವತಾ ॥ ಆಸನೇ ವಿನಿಯೋಗಃ ॥\\
ಪೃಥ್ವಿ ತ್ವಯಾ ಧೃತಾ ಲೋಕಾ ದೇವಿ ತ್ವಂ ವಿಷ್ಣುನಾ ಧೃತಾ~।\\
ತ್ವಂ ಚ ಧಾರಯ ಮಾಂ ದೇವಿ ಪವಿತ್ರಂ ಕುರು ಚಾಸನಂ ॥

ಅಪಸರ್ಪಂತು ತೇ ಭೂತಾಃ ಯೇ ಭೂತಾ ಭೂಮಿ ಸಂಸ್ಥಿತಾಃ~।\\
ಯೇ ಭೂತಾಃ ವಿಘ್ನಕರ್ತಾರಸ್ತೇನಶ್ಯಂತು ಶಿವಾಜ್ಞಯಾ ॥

ಅಪಕ್ರಾಮಂತು ಭೂತಾನಿ ಪಿಶಾಚಾಃ ಸರ್ವತೋ ದಿಶಂ।\\
ಸರ್ವೇಷಾಮವಿರೋಧೇನ ಪೂಜಾ ಕರ್ಮಸಮಾರಭೇ ॥

ಸ್ಯೋನಾ ಪೃಥಿವೀತ್ಯಸ್ಯ ಮೇಧಾತಿಥಿಃ ಕಾಣ್ವ ಋಷಿಃ । ಗಾಯತ್ರೀ ಛಂದಃ । ಪೃಥಿವೀ ದೇವತಾ । ಭೂಪ್ರಾರ್ಥನೇ ವಿನಿಯೋಗಃ ॥\\
\as{ಸ್ಯೋ॒ನಾ ಪೃ॑ಥಿವಿ ಭವಾನೃಕ್ಷ॒ರಾ ನಿ॒ವೇಶ॑ನೀ~।\\ ಯಚ್ಛಾ᳚ನಃ॒ ಶರ್ಮ॑ ಸ॒ಪ್ರಥಃ॑ ॥}

ಧನುರ್ಧರಾಯೈ ಚ ವಿದ್ಮಹೇ ಸರ್ವಸಿದ್ಧ್ಯೈ ಚ ಧೀಮಹಿ~।\\ ತನ್ನೋ ಧರಾ ಪ್ರಚೋದಯಾತ್ ॥

ಲಂ ಪೃಥಿವ್ಯೈ ನಮಃ~। ರಂ ರಕ್ತಾಸನಾಯ ನಮಃ~। ವಿಂ ವಿಮಲಾಸನಾಯ ನಮಃ~। ಯಂ ಯೋಗಾಸನಾಯ ನಮಃ~। ಕೂರ್ಮಾಸನಾಯ ನಮಃ~। ಅನಂತಾಸನಾಯ ನಮಃ~। ವೀರಾಸನಾಯ ನಮಃ~। ಖಡ್ಗಾಸನಾಯ ನಮಃ~। ಶರಾಸನಾಯ ನಮಃ~। ಪಂ ಪದ್ಮಾಸನಾಯ ನಮಃ~। ಪರಮಸುಖಾಸನಾಯ ನಮಃ॥

೪ ರಕ್ತದ್ವಾದಶಶಕ್ತಿಯುಕ್ತಾಯ ದ್ವೀಪನಾಥಾಯ ನಮಃ ॥

೪ ಶ್ರೀಲಲಿತಾಮಹಾತ್ರಿಪುರಸುಂದರಿ ಆತ್ಮಾನಂ ರಕ್ಷ ರಕ್ಷ ॥

ಓಂ ಗುಂ ಗುರುಭ್ಯೋ ನಮಃ~। ಪರಮಗುರುಭ್ಯೋ ನಮಃ~। ಪರಮೇಷ್ಠಿ\\ಗುರುಭ್ಯೋ ನಮಃ~। ಗಂ ಗಣಪತಯೇ ನಮಃ~। ದುಂ ದುರ್ಗಾಯೈ ನಮಃ~। ಸಂ ಸರಸ್ವತ್ಯೈ ನಮಃ~। ವಂ ವಟುಕಾಯ ನಮಃ~। ಕ್ಷಂ ಕ್ಷೇತ್ರಪಾಲಾಯ ನಮಃ~। ಯಾಂ ಯೋಗಿನೀಭ್ಯೋ ನಮಃ~। ಅಂ ಆತ್ಮನೇ ನಮಃ~। ಪಂ ಪರಮಾತ್ಮನೇ ನಮಃ~। ಸಂ ಸರ್ವಾತ್ಮನೇ ನಮಃ ॥
\newpage
೪ ಓಂ ನಮೋ ಭಗವತಿ ತಿರಸ್ಕರಿಣಿ ಮಹಾಮಾಯೇ ಮಹಾನಿದ್ರೇ ಸಕಲ \\ಪಶುಜನ ಮನಶ್ಚಕ್ಷುಃಶ್ರೋತ್ರತಿರಸ್ಕರಣಂ ಕುರು ಕುರು ಸ್ವಾಹಾ ॥

೪ ಓಂ ಹಸಂತಿ ಹಸಿತಾಲಾಪೇ ಮಾತಂಗಿ ಪರಿಚಾರಿಕೇ~।\\
ಮಮ ವಿಘ್ನಾಪದಾಂ ನಾಶಂ ಕುರು ಕುರು ಠಃಠಃಠಃ ಹುಂ ಫಟ್ ಸ್ವಾಹಾ ॥

೪ ಓಂ ನಮೋ ಭಗವತಿ ಜ್ವಾಲಾಮಾಲಿನಿ ದೇವದೇವಿ ಸರ್ವಭೂತ ಸಂಹಾರ ಕಾರಿಕೇ ಜಾತವೇದಸಿ ಜ್ವಲಂತಿ ಜ್ವಲ ಜ್ವಲ ಪ್ರಜ್ವಲ ಪ್ರಜ್ವಲ ಹ್ರಾಂ ಹ್ರೀಂ ಹ್ರೂಂ ರರ ರರ ರರರ ಹುಂ ಫಟ್ ಸ್ವಾಹಾ~। ಸಹಸ್ರಾರ ಹುಂ ಫಟ್~।\\ ಭೂರ್ಭುವಃಸುವರೋಮಿತಿ ದಿಗ್ಬಂಧಃ ॥

೪ ಸಮಸ್ತ ಪ್ರಕಟ ಗುಪ್ತ ಗುಪ್ತತರ ಸಂಪ್ರದಾಯ ಕುಲೋತ್ತೀರ್ಣ ನಿಗರ್ಭ ರಹಸ್ಯಾ\-ತಿರಹಸ್ಯ ಪರಾಪರಾತಿರಹಸ್ಯ ಯೋಗಿನೀ ದೇವತಾಭ್ಯೋ ನಮಃ ॥

೪ ಐಂ ಹ್ರಃ ಅಸ್ತ್ರಾಯ ಫಟ್ ॥

೪ ಶ್ರೀಗುರೋ ದಕ್ಷಿಣಾಮೂರ್ತೇ ಭಕ್ತಾನುಗ್ರಹಕಾರಕ~।\\
ಅನುಜ್ಞಾಂ ದೇಹಿ ಭಗವನ್ ಶ್ರೀಚಕ್ರ ಯಜನಾಯ ಮೇ ॥

೪ ಅತಿಕ್ರೂರ ಮಹಾಕಾಯ ಕಲ್ಪಾಂತದಹನೋಪಮ~।\\
ಭೈರವಾಯ ನಮಸ್ತುಭ್ಯಮನುಜ್ಞಾಂ ದಾತುಮರ್ಹಸಿ ॥

೪ ಮೂಲಶೃಂಗಾಟಕಾತ್ ಸುಷುಮ್ನಾಪಥೇನ ಜೀವಶಿವಂ ಪರಮಶಿವಪದೇ \\ಯೋಜಯಾಮಿ ಸ್ವಾಹಾ~।\\
ಯಂ ೮ ಸಂಕೋಚಶರೀರಂ ಶೋಷಯ ಶೋಷಯ ಸ್ವಾಹಾ~।\\
ರಂ ೮ ಸಂಕೋಚಶರೀರಂ ದಹ ದಹ ಪಚ ಪಚ ಸ್ವಾಹಾ~।\\
ವಂ ೮ ಪರಮಶಿವಾಮೃತಂ ವರ್ಷಯ ವರ್ಷಯ ಸ್ವಾಹಾ~।\\
ಲಂ ೮ ಶಾಂಭವಶರೀರಮುತ್ಪಾದಯೋತ್ಪಾದಯ ಸ್ವಾಹಾ~।\\
ಹಂಸಃ ಸೋಹಂ ಅವತರ ಅವತರ ಶಿವಪದಾತ್ ಜೀವಶಿವ\\ ಸುಷುಮ್ನಾಪಥೇನ ಪ್ರವಿಶ ಮೂಲಶೃಂಗಾಟಕಂ ಉಲ್ಲಸೋಲ್ಲಸ\\ ಜ್ವಲ ಜ್ವಲ ಪ್ರಜ್ವಲ ಪ್ರಜ್ವಲ ಹಂಸಃ ಸೋಹಂ ಸ್ವಾಹಾ ॥\\
೪ ಆಂ ಸೋಹಂ (ಇತಿ ತ್ರಿಃ ಹೃದಿ) ಇತಿ ಭೂತಶುದ್ಧಿಃ ॥\\
ತತಃ ಪ್ರಾಣಾನಾಯಮ್ಯ, ದೇಶಕಾಲೌ ಸಂಕೀರ್ತ್ಯ 

ಪ್ರತಿಪದಾದಿ ನವಮೀಪರ್ಯಂತ ಶ್ರೀದುರ್ಗಾಪರಮೇಶ್ವರೀ ಪೂಜನೇನ ಸಕುಟುಂಬಸ್ಯ ಆಯುರಾರೋಗ್ಯ ಧನ ಧಾನ್ಯ ಪುತ್ರ ಪೌತ್ರ ಪಶ್ವಾದಿ ಸಮೃದ್ಧ್ಯರ್ಥಂ ವಿಜಯಲಾಭ ಭೂಪ್ರಾಪ್ತಿ ಭೀಕರ ಭೂತ ಪ್ರೇತ ಪಿಶಾಚ ಶಾಕಿನೀ ಡಾಕಿನೀ ಬ್ರಹ್ಮರಾಕ್ಷಸ ವೇತಾಳ ದುಷ್ಟಗ್ರಹನಿವೃತ್ತಿ ಕ್ಷಾಮರಡಾಮರ ಮಹೋತ್ಪಾತಾಗ್ನಿ ಚೋರವ್ಯಾಘ್ರ ರಾಕ್ಷಸ ಸಮುದ್ಭವ ಸಮಸ್ತದುಶ್ಚರಿತ ನಿವಾರಣಾರ್ಥಂ ಬ್ರಹ್ಮ ವಿಷ್ಣು ಮಹೇಶ್ವರ ಇಂದ್ರಾದಿ ಲೋಕಪಾಲ ನವಗ್ರಹ ಅಷ್ಟ ವಸು ಏಕಾದಶರುದ್ರ ದ್ವಾದಶಾದಿತ್ಯ ಪಿತೃಮುಖಾದಿ ಸಮಸ್ತದೇವತಾ ಪೂಜಾಫಲ ಪ್ರಾಪ್ತ್ಯರ್ಥಂ ರಾಜಾಮಾತ್ಯಾದಿ ಸರ್ವಜನ ವಶೀಕರಣ ಸಿದ್ಧ್ಯರ್ಥಂ ಮಹಾಕಾಳೀ ಮಹಾಲಕ್ಷ್ಮೀ ಮಹಾಸರಸ್ವತ್ಯಾತ್ಮಕ ನವದುರ್ಗಾ  ಪ್ರೀತ್ಯರ್ಥಂ ಸರ್ವೇಷಾಂ ಮಹಾಜನಾನಾಂ ಕ್ಷೇಮ ಸ್ಥೈರ್ಯ ವೀರ್ಯ ವಿಜಯ ಅಭಯ ಆಯುಃ ಆರೋಗ್ಯ ಐಶ್ವರ್ಯ ಅಭಿವೃದ್ಧ್ಯರ್ಥಂ , ಆಧ್ಯಾತ್ಮಿಕ ಆಧಿದೈವಿಕ ಆಧಿಭೌತಿಕ ತಾಪತ್ರಯ ನಿವೃತ್ಯರ್ಥಂ, ಭೌಮಾಂತರಿಕ್ಷ ದಿವ್ಯ ಮಹೋತ್ಪಾತ ಆಗಾಮಿ ಸಂಚಿತ ದುಷ್ಟಾರಿಷ್ಟ ದೋಷಪರಿಹಾರಾರ್ಥಂ ನಾನಾ ವಿಧ ರೋಗನಿವಾರಣಾರ್ಥಂ, ಸ್ವಕೀಯೈಃ ಪರಕೀಯೈಃ ಸ್ವಗ್ರಾಮಸ್ಥೈಃ  ಅನ್ಯಗ್ರಾಮಸ್ಥೈಃ ಕೃತ ಕ್ರಿಯಮಾಣ ದುಷ್ಟಮಂತ್ರ ತಂತ್ರ ವಿಷಚೂರ್ಣ ಪ್ರತಿಮಾಸ್ಥಾಪನಾದಿ ಅಭಿಚಾರ ಕರ್ಮ ಪಾಣಿ ಹವನ ದುರ್ನಿರೀಕ್ಷಣ ದುರ್ದೇವತಾ ಔಪಾಸನಾದಿ ಕ್ಷುದ್ರ ಕರ್ಮ ದೋಷ ಪರಿಹಾರಾರ್ಥಂ ಅಲಕ್ಷ್ಮೀ ಪರಿಹಾರದ್ವಾರಾ ಮಹಾಲಕ್ಷ್ಮೀ ಕೃಪಾಕಟಾಕ್ಷ ಸಿದ್ಧ್ಯರ್ಥಂ ಆದಿತ್ಯಾದಿ ನವಗ್ರಹ ಪೀಡಾ ಪರಿಹಾರಾರ್ಥಂ ದೇಹರಕ್ಷಣಾರ್ಥಂ ಮಂತ್ರರಕ್ಷಣಾರ್ಥಂ ಗೃಹದಾರಾದಿ ಗೋ ಸಂರಕ್ಷಣಾರ್ಥಂ ಧರ್ಮಾರ್ಥಕಾಮಮೋಕ್ಷ ಚತುರ್ವಿಧ ಪುರುಷಾರ್ಥ ಸಿದ್ಧ್ಯರ್ಥಂ ತ್ರಿಶಕ್ತ್ಯಾತ್ಮಕ ಶ್ರೀ ದುರ್ಗಾಪರಮೇಶ್ವರೀ ಪ್ರೀತ್ಯರ್ಥಂ ನವರಾತ್ರಾಂಗತಯಾ ಯಾವಚ್ಛಕ್ತಿ ಕಲ್ಪೋಕ್ತ ಧ್ಯಾನಾವಾಹನಾದಿ ಷೋಡಶೋಪಚಾರಪೂಜಾಂ ಕರಿಷ್ಯೇ ॥

ಆದೌ ಪೂಜಾನಿರ್ವಿಘ್ನತಾ ಸಿದ್ಧ್ಯರ್ಥಂ ಗುರುಗಣಪತಿಪೂಜಾಂ ಚ ಕರಿಷ್ಯೇ ।

\section{ಅಥ ಕಲಶಾರ್ಚನಂ}
\addcontentsline{toc}{section}{ಸಾಮಾನ್ಯಾರ್ಘ್ಯ ವಿಧಿಃ}
ಬಿಂದು ತ್ರಿಕೋಣ ವೃತ್ತ ಚತುರಸ್ರಾತ್ಮಕಂ ಮಂಡಲಂ ವಿಧಾಯ , ಚತುರಸ್ರೇ ಆಗ್ನೇಯಾದಿಷು,\\
\as{ಓಂ ಐಂ } ಹೃದಯಾಯ ನಮಃ । ಹೃದಯಶಕ್ತಿ ಶ್ರೀಪಾದುಕಾಂ ಪೂ । ನಮಃ ॥\\
\as{ಓಂ ಹ್ರೀಂ } ಶಿರಸೇ ಸ್ವಾಹಾ । ಶಿರಃಶಕ್ತಿ ಶ್ರೀಪಾದುಕಾಂ ಪೂ । ನಮಃ ॥\\
\as{ಓಂ ಕ್ಲೀಂ } ಶಿಖಾಯೈ ವಷಟ್ । ಶಿಖಾಶಕ್ತಿ ಶ್ರೀಪಾದುಕಾಂ ಪೂ । ನಮಃ ॥\\
\as{ಓಂ ಚಾಮುಂಡಾಯೈ } ಕವಚಾಯ ಹುಂ । ಕವಚಶಕ್ತಿ ಶ್ರೀಪಾದುಕಾಂ ಪೂ । ನಮಃ ॥\\
\as{ಓಂ ವಿಚ್ಚೇ} ನೇತ್ರತ್ರಯಾಯ ವೌಷಟ್ । ನೇತ್ರಶಕ್ತಿ ಶ್ರೀಪಾದುಕಾಂ ಪೂ । ನಮಃ ॥\\
\as{ಓಂ ಐಂಹ್ರೀಂಕ್ಲೀಂ ಚಾಮುಂಡಾಯೈ ವಿಚ್ಛೇ } ಅಸ್ತ್ರಾಯ ಫಟ್ । ಅಸ್ತ್ರಶಕ್ತಿ ಶ್ರೀಪಾದುಕಾಂ ಪೂ । ನಮಃ ॥\\
ಇತಿ ಚ ಷಡಂಗಾನಿ ವಿನ್ಯಸ್ಯ, ತ್ರಿಕೋಣೇ ಸ್ವಾಗ್ರಾದಿಕೋಣೇಷು\\
\as{ಓಂ ಐಂ } ನಮಃ ॥\\
\as{ಓಂ ಹ್ರೀಂ} ನಮಃ ॥\\
\as{ಓಂ ಕ್ಲೀಂ } ನಮಃ ॥\\
ಇತಿ ಸಂಪೂಜ್ಯ, \as{ಮೂಲೇ}ನ ಬಿಂದುಂ ಸಂಪೂಜ್ಯ,

\as{ಓಂ ಐಂ} ಅಂ ಅಗ್ನಿಮಂಡಲಾಯ ಧರ್ಮಪ್ರದದಶಕಲಾತ್ಮನೇ ವರ್ಧಿನೀಕಲಶಾಧಾರಾಯ ನಮಃ । ಇತಿ ಪಾತ್ರಾಧಾರಂ\\
\as{ಓಂ ಹ್ರೀಂ} ಉಂ ಅರ್ಕಮಂಡಲಾಯ ಅರ್ಥಪ್ರದದ್ವಾದಶಕಲಾತ್ಮನೇ ವರ್ಧಿನೀಕಲಶಾಯ ನಮಃ । ಇತಿ ಪಾತ್ರಂ ಚ ನಿಧಾಯ\\
\as{ಓಂ ಕ್ಲೀಂ} ಮಂ ಸೋಮಮಂಡಲಾಯ ಕಾಮಪ್ರದಷೋಡಶಕಲಾತ್ಮನೇ ವರ್ಧಿನೀಕಲಶಾಮೃತಾಯ ನಮಃ । ಇತಿ ಜಲಮಾಪೂರ್ಯ

ಕಲಶಸ್ಯ ಮುಖೇ ವಿಷ್ಣುಃ ಕಂಠೇ ರುದ್ರ ಸಮಾಶ್ರಿತಾಃ  ।\\
ಮೂಲೇ ತತ್ರ ಸ್ಥಿತೋಬ್ರಹ್ಮಾ, ಮಧ್ಯೇ ಮಾತೃ ಗಣಾಃ ಸ್ಮೃತಾಃ ॥

ಕುಕ್ಷೌ ತು ಸಾಗರಾಃ ಸರ್ವೇ ಸಪ್ತ ದ್ವೀಪ ವಸುಂಧರಾ।\\
ಋಗ್ವೇದೋಽಥ ಯಜುರ್ವೇದಃ ಸಾಮವೇದೋ ಹ್ಯಥರ್ವಣಃ।\\
ಅಂಗೈಶ್ಚ ಸಹಿತಾಃ ಸರ್ವೇ ಕಲಶಾಂಬು ಸಮಾಶ್ರಿತಾಃ ॥

ಅತ್ರ ಗಾಯತ್ರಿ ಸಾವಿತ್ರೀ ಶಾಂತಿಃ ಪುಷ್ಟಿಕರೀ ತಥಾ ।\\
ಆಯಾಂತು ದೇವ ಪೂಜಾರ್ಥಂ ದುರಿತಕ್ಷಯ ಕಾರಕಾಃ ॥

ಸರ್ವೇ ಸಮುದ್ರಾಃ ಸರಿತಾ ತೀರ್ಥಾನಿ ಜಲದಾ ನದಾಃ ।\\
ಗಂಗೇ ಚ ಯಮುನೇ ಚೈವ ಗೋದಾವರಿ ಸರಸ್ವತಿ ।\\
ನರ್ಮದೇ ಸಿಂಧು ಕಾವೇರಿ ಜಲೇಽಸ್ಮಿನ್ ಸನ್ನಿಧಿಂ ಕುರು ॥


\as{ಓಂ ಆಪೋ॒ ವಾ ಇ॒ದँ ಸರ್ವಂ॒ ವಿಶ್ವಾ॑ ಭೂ॒ತಾನ್ಯಾಪಃ॑ ಪ್ರಾ॒ಣಾ ವಾ ಆಪಃ॑
ಪ॒ಶವ॒ ಆಪೋಽನ್ನ॒ಮಾಪೋ॑ಽಮೃ॒ತಮಾಪಃ॑ ಸ॒ಮ್ರಾಡಾಪೋ॑ ವಿ॒ರಾಡಾಪಃ॑
ಸ್ವ॒ರಾಡಾಪ॒ಶ್ಛಂದಾँ॒॒ಸ್ಯಾಪೋ॒ ಜ್ಯೋತೀँ॒॒ಷ್ಯಾಪೋ॒
ಯಜೂँ॒॒ಷ್ಯಾಪಃ॑ ಸ॒ತ್ಯಮಾಪ॒ಸ್ಸರ್ವಾ॑ ದೇ॒ವತಾ॒ ಆಪೋ॒
ಭೂರ್ಭುವ॒ಸ್ಸುವ॒ರಾಪ॒ ಓಂ ॥

ಓಂ ಇ॒ಮಂ ಮೇ᳚ ಗಂಗೇ ಯಮುನೇ ಸರ॑ಸ್ವತಿ॒ ಶುತು॑ದ್ರಿ॒ಸ್ತೋಮಂ᳚ ಸಚತಾ॒ ಪರು॒ಷ್ಣ್ಯಾ ।
ಅ॒ಸಿ॒ಕ್ನಿ॒ಯಾ ಮ॑ರುಧ್ವೃದೇ ವಿ॒ತಸ್ತ॒ಯಾರ್ಜೀ᳚ಕೀಯೇ  ಶೃಣು॒ಹ್ಯಾ ಸು॒ಶೋಮ॑ಯಾ ॥
ಸಿತಾ᳚ಸಿತೇ ಸ॒ರಿತೇ॒ ಯತ್ರ॑ ಸಂಗ॒ತೇ ತತ್ರಾ᳚ ಪ್ಲು॒ತೋಸೋ॒ದಿವ॒ ಮುತ್ಪ॑ತಂತಿ ।
ಏವೈ᳚ ತ॒ನ್ವಾ\nicefrac{೧}{೨}(ಅಂ) ವಿಸೃ॑ಜಂತಿ॒ ಧೀರಾ॒ಸ್ತೇಜನಾ᳚ಸೋ ಅಮೃತ॒ತ್ವಂ ಭ॑ಜಂತೇ ॥}

\as{ಸಿತಮಕರನಿಶಣ್ಣಾಂ ಶುಭ್ರವರ್ಣಾಂ ತ್ರಿನೇತ್ರಾಂ\\ಕರಧೃತ ಕಲಶೋದ್ಭ್ಯ ತ್ಪಂಕಜಾ ಭೀತ್ಯಭೀಷ್ಟಾಂ ।\\
ವಿಧಿ ಹರಿಹರ ರೂಪಾಂ ಸೇಂದು ಕೋಟೀರಚೂಡಾಂ\\ಭಸಿತ ಸಿತ ದುಕೂಲಾಂ ಜಾಹ್ನವೀಂ ತಾಂ ನಮಾಮಿ ॥}\\
ಗಂಗಾದಿ ಸರ್ವ ತೀರ್ಥೇಭ್ಯೋ ನಮಃ । ಇತಿ ಪಠಿತ್ವಾ,\\
೧೦ ಗಾಯತ್ರೀ ಹಂಸಗಾಯತ್ರೀ ನವಾರ್ಣ ಕಾಮಧೇನು ಗಂಗಾ ಮಣಿಕರ್ಣಿಕಾ ಅಷ್ಟಾಕ್ಷರೀ ಪಂಚಾಕ್ಷರೀ ಮಂತ್ರೈರಭಿಮಂತ್ರಯೇತ್ ॥\\
ಯಥಾವಕಾಶಂ ಗಂಗಾಪೂಜಾ ವಿಧೇಯಾ ।\\
\section{ಗಂಗಾಷ್ಟೋತ್ತರಶತನಾಮಾವಲಿಃ}
\begin{multicols}{2}ಗಂಗಾಯೈ~।\\ ವಿಷ್ಣುಪಾದಸಂಭೂತಾಯೈ~।\\ ಹರವಲ್ಲಭಾಯೈ~।\\ ಹಿಮಾಚಲೇಂದ್ರತನಯಾಯೈ~।\\ ಗಿರಿಮಂಡಲಗಾಮಿನ್ಯೈ~।\\ ತಾರಕಾರಾತಿಜನನ್ಯೈ~।\\ ಸಗರಾತ್ಮಜತಾರಕಾಯೈ~।\\ ಸರಸ್ವತೀಸಮಯುಕ್ತಾಯೈ~।\\ ಸುಘೋಷಾಯೈ~।\\ ಸಿಂಧುಗಾಮಿನ್ಯೈ~।\\ ಭಾಗೀರಥ್ಯೈ~।\\ ಭಾಗ್ಯವತ್ಯೈ~।\\ ಭಗೀರಥರಥಾನುಗಾಯೈ~।\\ ತ್ರಿವಿಕ್ರಮಪದೋದ್ಭೂತಾಯೈ~।\\ ತ್ರಿಲೋಕಪಥಗಾಮಿನ್ಯೈ~।\\ ಕ್ಷೀರಶುಭ್ರಾಯೈ~।\\ ಬಹುಕ್ಷೀರಾಯೈ~।\\ ಕ್ಷೀರವೃಕ್ಷಸಮಾಕುಲಾಯೈ~।\\ ತ್ರಿಲೋಚನಜಟಾವಾಸಾಯೈ~।\\ ಋಣತ್ರಯವಿಮೋಚಿನ್ಯೈ~।\\ ತ್ರಿಪುರಾರಿಶಿರಶ್ಚೂಡಾಯೈ~।\\ ಜಾಹ್ನವ್ಯೈ~।\\ ನರಕಭೀತಿಹೃತೇ~।\\ ಅವ್ಯಯಾಯೈ~।\\ ನಯನಾನಂದದಾಯಿನ್ಯೈ~।\\ ನಗಪುತ್ರಿಕಾಯೈ~।\\ ನಿರಂಜನಾಯೈ~।\\ ನಿತ್ಯಶುದ್ಧಾಯೈ~।\\ ನೀರಜಾಲಿಪರಿಷ್ಕೃತಾಯೈ~।\\ ಸಾವಿತ್ರ್ಯೈ~।\\ ಸಲಿಲಾವಾಸಾಯೈ~।\\ ಸಾಗರಾಂಬುಸಮೇಧಿನ್ಯೈ~।\\ ರಮ್ಯಾಯೈ~।\\ ಬಿಂದುಸರಸೇ~।\\ ಅವ್ಯಕ್ತಾಯೈ~।\\ ಅವ್ಯಕ್ತರೂಪಧೃತೇ~।\\ ಉಮಾಸಪತ್ನ್ಯೈ~।\\ ಶುಭ್ರಾಂಗಾಯೈ~।\\ ಶ್ರೀಮತ್ಯೈ~।\\ ಧವಲಾಂಬರಾಯೈ~।\\ ಆಖಂಡಲವನಾವಾಸಾಯೈ~।\\ ಕಂಠೇಂದುಕೃತಶೇಖರಾಯೈ~।\\ ಅಮೃತಾಕಾರಸಲಿಲಾಯೈ~।\\ ಲೀಲಾಲಿಂಗಿತಪರ್ವತಾಯೈ~।\\ ವಿರಿಂಚಿಕಲಶಾವಾಸಾಯೈ~।\\ ತ್ರಿವೇಣ್ಯೈ~।\\ ತ್ರಿಗುಣಾತ್ಮಿಕಾಯೈ~।\\ ಸಂಗತಾಘೌಘಶಮನ್ಯೈ~।\\ ಭೀತಿಹರ್ತ್ರ್ಯೈ~।\\ ಶಂಖದುಂದುಭಿನಿಸ್ವನಾಯೈ~।\\ ಭಾಗ್ಯದಾಯಿನ್ಯೈ~।\\ ಭಾಗ್ಯಜನನ್ಯೈ~।\\ ನಂದಿನ್ಯೈ~।\\ ಶೀಘ್ರಗಾಯೈ~।\\ ಸಿದ್ಧಾಯೈ~।\\ ಶರಣ್ಯೈ~।\\ ಶಶಿಶೇಖರಾಯೈ~।\\ ಶಾಂಕರ್ಯೈ~।\\ ಶಫರೀಪೂರ್ಣಾಯೈ~।\\ ಭರ್ಗಮೂರ್ಧಕೃತಾಲಯಾಯೈ~।\\ ಭವಪ್ರಿಯಾಯೈ~।\\ ಸತ್ಯಸಂಧಪ್ರಿಯಾಯೈ~।\\ ಹಂಸಸ್ವರೂಪಿಣ್ಯೈ~।\\ ಭಗೀರಥಭೃತಾಯೈ~।\\ ಅನಂತಾಯೈ~।\\ ಶರಚ್ಚಂದ್ರನಿಭಾನನಾಯೈ~।\\ ಓಂಕಾರರೂಪಿಣ್ಯೈ~।\\ ಅನಲಾಯೈ~।\\ ಕ್ರೀಡಾಕಲ್ಲೋಲಕಾರಿಣ್ಯೈ~।\\ ಸ್ವರ್ಗಸೋಪಾನಸರಣ್ಯೈ~।\\ ಸರ್ವದೇವಸ್ವರೂಪಿಣ್ಯೈ~।\\ ಅಂಭಃಪ್ರದಾಯೈ~।\\ ದುಃಖಹಂತ್ರ್ಯೈ~।\\ ಶಾಂತಿ ಸಂತಾನಕಾರಿಣ್ಯೈ~।\\ ದಾರಿದ್ರ್ಯಹಂತ್ರ್ಯೈ~।\\ ಶಿವದಾಯೈ~।\\ ಸಂಸಾರವಿಷನಾಶಿನ್ಯೈ~।\\ ಪ್ರಯಾಗನಿಲಯಾಯೈ~।\\ ಶ್ರೀದಾಯೈ~।\\ ತಾಪತ್ರಯವಿಮೋಚಿನ್ಯೈ~।\\ ಶರಣಾಗತದೀನಾರ್ತಪರಿತ್ರಾಣಾಯೈ~।\\ ಸುಮುಕ್ತಿದಾಯೈ~।\\ ಪಾಪಹಂತ್ರ್ಯೈ~।\\ ಪಾವನಾಂಗಾಯೈ~।\\ ಪರಬ್ರಹ್ಮಸ್ವರೂಪಿಣ್ಯೈ~।\\ ಪೂರ್ಣಾಯೈ~।\\ ಪುರಾತನಾಯೈ~।\\ ಪುಣ್ಯಾಯೈ~।\\ ಪುಣ್ಯದಾಯೈ~।\\ ಪುಣ್ಯವಾಹಿನ್ಯೈ~।\\ ಪುಲೋಮಜಾರ್ಚಿತಾಯೈ~।\\ ಭೂದಾಯೈ~।\\ ಪೂತತ್ರಿಭುವನಾಯೈ~।\\ ಜಯಾಯೈ~।\\ ಜಂಗಮಾಯೈ~।\\ ಜಂಗಮಾಧಾರಾಯೈ~।\\ ಜಲರೂಪಾಯೈ~।\\ ಜಗದ್ಧಾತ್ರ್ಯೈ~।\\ ಜಗದ್ಭೂತಾಯೈ~।\\ ಜನಾರ್ಚಿತಾಯೈ~।\\ ಜಹ್ನುಪುತ್ರ್ಯೈ~।\\ ಜಗನ್ಮಾತ್ರೇ~।\\ ಜಂಬೂದ್ವೀಪವಿಹಾರಿಣ್ಯೈ~।\\ ಭವಪತ್ನ್ಯೈ~।\\ ಭೀಷ್ಮಮಾತ್ರೇ~।\\ ಸಿಕ್ತಾಯೈ~।\\ ರಮ್ಯರೂಪಧೃತೇ~।\\ ಉಮಾಸಹೋದರ್ಯೈ~।\\ ಅಜ್ಞಾನತಿಮಿರಾಪಹೃತೇ ನಮಃ।\\{\Large ॥ಇತಿ ಶ್ರೀಗಂಗಾಷ್ಟೋತ್ತರಶತನಾಮಾವಲಿಃ ॥}\end{multicols}
\as{ಭಾಗೀರಥ್ಯೈ ಚ ವಿದ್ಮಹೇ ವಿಷ್ಣುಪತ್ನ್ಯೈ ಚ ಧೀಮಹಿ । ತನ್ನೋ ಗಂಗಾ ಪ್ರಚೋದಯಾತ್ ॥}

\section{ಸಾಮಾನ್ಯಾರ್ಘ್ಯ ವಿಧಿಃ}
\addcontentsline{toc}{section}{ಸಾಮಾನ್ಯಾರ್ಘ್ಯ ವಿಧಿಃ}
ಬಿಂದು ತ್ರಿಕೋಣ ಷಟ್ಕೋಣ ವೃತ್ತ ಚತುರಸ್ರಾತ್ಮಕಂ ಮಂಡಲಂ ವಿಧಾಯ , ಚತುರಸ್ರೇ ಆಗ್ನೇಯಾದಿಷು,\\
\as{ಓಂ ಐಂ } ಹೃದಯಾಯ ನಮಃ । ಹೃದಯಶಕ್ತಿ ಶ್ರೀಪಾದುಕಾಂ ಪೂ । ನಮಃ ॥\\
\as{ಓಂ ಹ್ರೀಂ } ಶಿರಸೇ ಸ್ವಾಹಾ । ಶಿರಃಶಕ್ತಿ ಶ್ರೀಪಾದುಕಾಂ ಪೂ । ನಮಃ ॥\\
\as{ಓಂ ಕ್ಲೀಂ } ಶಿಖಾಯೈ ವಷಟ್ । ಶಿಖಾಶಕ್ತಿ ಶ್ರೀಪಾದುಕಾಂ ಪೂ । ನಮಃ ॥\\
\as{ಓಂ ಚಾಮುಂಡಾಯೈ } ಕವಚಾಯ ಹುಂ । ಕವಚಶಕ್ತಿ ಶ್ರೀಪಾದುಕಾಂ ಪೂ । ನಮಃ ॥\\
\as{ಓಂ ವಿಚ್ಚೇ} ನೇತ್ರತ್ರಯಾಯ ವೌಷಟ್ । ನೇತ್ರಶಕ್ತಿ ಶ್ರೀಪಾದುಕಾಂ ಪೂ । ನಮಃ ॥\\
\as{ಓಂ ಐಂಹ್ರೀಂಕ್ಲೀಂ ಚಾಮುಂಡಾಯೈ ವಿಚ್ಛೇ } ಅಸ್ತ್ರಾಯ ಫಟ್ । ಅಸ್ತ್ರಶಕ್ತಿ ಶ್ರೀಪಾದುಕಾಂ ಪೂ । ನಮಃ ॥

ಷಟ್ಕೋಣೇ ಸ್ವಾಗ್ರಾದಿಕೋಣೇಷು\\
\as{ಓಂ ಐಂ } ಹೃದಯಾಯ ನಮಃ । ಹೃದಯಶಕ್ತಿ ಶ್ರೀಪಾದುಕಾಂ ಪೂ । ನಮಃ ॥\\
\as{ಓಂ ಹ್ರೀಂ } ಶಿರಸೇ ಸ್ವಾಹಾ । ಶಿರಃಶಕ್ತಿ ಶ್ರೀಪಾದುಕಾಂ ಪೂ । ನಮಃ ॥\\
\as{ಓಂ ಕ್ಲೀಂ } ಶಿಖಾಯೈ ವಷಟ್ । ಶಿಖಾಶಕ್ತಿ ಶ್ರೀಪಾದುಕಾಂ ಪೂ । ನಮಃ ॥\\
\as{ಓಂ ಚಾಮುಂಡಾಯೈ } ಕವಚಾಯ ಹುಂ । ಕವಚಶಕ್ತಿ ಶ್ರೀಪಾದುಕಾಂ ಪೂ । ನಮಃ ॥\\
\as{ಓಂ ವಿಚ್ಚೇ} ನೇತ್ರತ್ರಯಾಯ ವೌಷಟ್ । ನೇತ್ರಶಕ್ತಿ ಶ್ರೀಪಾದುಕಾಂ ಪೂ । ನಮಃ ॥\\
\as{ಓಂ ಐಂಹ್ರೀಂಕ್ಲೀಂ ಚಾಮುಂಡಾಯೈ ವಿಚ್ಛೇ } ಅಸ್ತ್ರಾಯ ಫಟ್ । ಅಸ್ತ್ರಶಕ್ತಿ ಶ್ರೀಪಾದುಕಾಂ ಪೂ । ನಮಃ ॥

ಇತಿ ಚ ಷಡಂಗಾನಿ ವಿನ್ಯಸ್ಯ, ತ್ರಿಕೋಣೇ ಸ್ವಾಗ್ರಾದಿಕೋಣೇಷು\\
\as{ಓಂ ಐಂ } ನಮಃ ॥\\
\as{ಓಂ ಹ್ರೀಂ} ನಮಃ ॥\\
\as{ಓಂ ಕ್ಲೀಂ } ನಮಃ ॥\\
ಇತಿ ಸಂಪೂಜ್ಯ, ಮೂಲೇನ ಬಿಂದುಂ ಸಂಪೂಜ್ಯ, \as{ಅಸ್ತ್ರಾಯ ಫಟ್} ಇತಿ ಸಾಮಾನ್ಯಾರ್ಘ್ಯಪಾತ್ರಾಧಾರಂ ಪ್ರಕ್ಷಾಲ್ಯ\\
\as{ಓಂ ಅಂ} ಅಗ್ನಿಮಂಡಲಾಯ ಧರ್ಮಪ್ರದದಶಕಲಾತ್ಮನೇ ಸಾಮಾನ್ಯಾರ್ಘ್ಯಪಾತ್ರಾಧಾರಾಯ ನಮಃ । ಇತಿ ಮಂಡಲೋಪರಿ ನಿಧಾಯ\\
\as{ಓಂ ಅಗ್ನಿಂ ದೂತಂ +++++ಸುಕ್ರತುಮ್ ॥} ರಾಂರೀಂರೂಂರೈಂರೌಂರಃ ರಮಲವರಯೂಂ ಅಗ್ನಿಮಂಡಲಾಯ ನಮಃ । ಇತಿ ಅಗ್ನಿಮಂಡಲಂ ವಿಭಾವ್ಯ, ದಶ ವಹ್ನಿಕಾಲಾಃ ಪೂಜಯೇತ್ ।

\as{ಓಂಐಂಹ್ರೀಂಶ್ರೀಂ ಯಂ} ಧೂಮ್ರಾರ್ಚಿಷೇ   ನಮಃ~। \as{೪  ರಂ} ಊಷ್ಮಾಯೈ~। \as{೪ ಲಂ} ಜ್ವಲಿನ್ಯೈ~। \as{೪  ವಂ} ಜ್ವಾಲಿನ್ಯೈ~। \as{೪  ಶಂ} ವಿಸ್ಫುಲಿಂಗಿನ್ಯೈ ~। \as{೪  ಷಂ} ಸುಶ್ರಿಯೈ ~। \as{೪  ಸಂ} ಸುರೂಪಾಯೈ ~। \as{೪  ಹಂ} ಕಪಿಲಾಯೈ ~। \as{೪  ಳಂ} ಹವ್ಯವಹಾಯೈ ~। \as{೪  ಕ್ಷಂ} ಕವ್ಯವಹಾಯೈ ನಮಃ~॥

ಅಸ್ತ್ರಮಂತ್ರೇಣ ಶಂಖಂ ಪ್ರಕ್ಷಾಳ್ಯ\\
\as{೪ ಉಂ} ಸೂರ್ಯಮಂಡಲಾಯ ಅರ್ಥಪ್ರದದ್ವಾದಶಕಲಾತ್ಮನೇ ಸಾಮಾನ್ಯಾರ್ಘ್ಯಪಾತ್ರಾಯ ನಮಃ । ಇತಿ ನಿಧಾಯ\\
\as{೪ ಆಕೃಷ್ಣೇನ +++++ನಿ ಪಶ್ಯನ್ ॥} ಹಾಂಹೀಂಹೂಂಹೈಂಹೌಂಹಃ ಹಮಲವರಯೂಂ ಸೂರ್ಯಮಂಡಲಾಯ ನಮಃ । ಇತಿ ಸೂರ್ಯಮಂಡಲಂ ವಿಭಾವ್ಯ, ದ್ವಾದಶ ಸೂರ್ಯಕಾಲಾಃ ಪೂಜಯೇತ್ ।

\as{ಓಂಐಂಹ್ರೀಂಶ್ರೀಂ ಕಂಭಂ} ತಪಿನ್ಯೈ ನಮಃ~। \as{೪ ಖಂಬಂ} ತಾಪಿನ್ಯೈ~। \as{೪ ಗಂಫಂ} ಧೂಮ್ರಾಯೈ~। \as{೪ ಘಂಪಂ} ಮರೀಚ್ಯೈ~। \as{೪ ಙಂನಂ} ಜ್ವಾಲಿನ್ಯೈ~। \as{೪ ಚಂಧಂ} ರುಚ್ಯೈ~। \as{೪ ಛಂದಂ} ಸುಷುಮ್ನಾಯೈ~। \as{೪ ಜಂಥಂ} ಭೋಗದಾಯೈ~। \as{೪ ಝಂತಂ} ವಿಶ್ವಾಯೈ~। \as{೪ ಞಂಣಂ} ಬೋಧಿನ್ಯೈ~। \as{೪ ಟಂಢಂ} ಧಾರಿಣ್ಯೈ~। \as{೪ ಠಂಡಂ} ಕ್ಷಮಾಯೈ ನಮಃ~॥

\as{೪ ಮಂ} ಸೋಮಮಂಡಲಾಯ ಕಾಮಪ್ರದಷೋಡಶಕಲಾತ್ಮನೇ ಸಾಮಾನ್ಯಾರ್ಘ್ಯಪಾತ್ರಾಮೃತಾಯ ನಮಃ । ಇತಿ ಕಲಶೋದಕೇನ ಶಂಖಂ ಪ್ರಪೂರ್ಯ, ಕ್ಷೀರಬಿಂದುಂ ದತ್ವಾ  ಗಂಧಾದಿಭಿರಭ್ಯರ್ಚ್ಯ,\\
\as{೪ ಆಪ್ಯಾಯಸ್ವ +++++ಸಂಗಥೇ ॥} ಸಾಂಸೀಂಸೂಂಸೈಂಸೌಂಸಃ ಸಮಲವರಯೂಂ ಸೋಮಮಂಡಲಾಯ ನಮಃ । ಇತಿ ಸೋಮಮಂಡಲಂ ವಿಭಾವ್ಯ ,ತತ್ರ ಷೋಡಶ ಸೋಮಕಾಲಾಃ ಪೂಜಯೇತ್ ।

\as{ಓಂಐಂಹ್ರೀಂಶ್ರೀಂ ಅಂ} ಅಮೃತಾಯೈ ನಮಃ~। \as{೪ ಆಂ} ಮಾನದಾಯೈ~। \as{೪ ಇಂ} ಪೂಷಾಯೈ~। \as{೪ ಈಂ} ತುಷ್ಟ್ಯೈ~। \as{೪ ಉಂ} ಪುಷ್ಟ್ಯೈ~। \as{೪ ಊಂ} ರತ್ಯೈ~। \as{೪ ಋಂ} ಧೃತ್ಯೈ~। \as{೪ ೠಂ} ಶಶಿನ್ಯೈ~। \as{೪ ಲೃಂ} ಚಂದ್ರಿಕಾಯೈ~। \as{೪ ಲೄಂ} ಕಾಂತ್ಯೈ~। \as{೪ ಏಂ} ಜ್ಯೋತ್ಸ್ನಾಯೈ~। \as{೪ ಐಂ} ಶ್ರಿಯೈ~। \as{೪ ಓಂ} ಪ್ರೀತ್ಯೈ~। \as{೪ ಔಂ} ಅಂಗದಾಯೈ~। \as{೪ ಅಂ} ಪೂರ್ಣಾಯೈ~। \as{೪ ಅಃ} ಪೂರ್ಣಾಮೃತಾಯೈ ನಮಃ~॥

ತತಃ ಆಗ್ನೇಯ್ಯಾದಿ ದಿಕ್ಷು ಮಧ್ಯೇ ಚ \\
\as{ಓಂ ಐಂ } ಹೃದಯಾಯ ನಮಃ । ಹೃದಯಶಕ್ತಿ ಶ್ರೀಪಾದುಕಾಂ ಪೂ । ನಮಃ ॥\\
\as{ಓಂ ಹ್ರೀಂ } ಶಿರಸೇ ಸ್ವಾಹಾ । ಶಿರಃಶಕ್ತಿ ಶ್ರೀಪಾದುಕಾಂ ಪೂ । ನಮಃ ॥\\
\as{ಓಂ ಕ್ಲೀಂ } ಶಿಖಾಯೈ ವಷಟ್ । ಶಿಖಾಶಕ್ತಿ ಶ್ರೀಪಾದುಕಾಂ ಪೂ । ನಮಃ ॥\\
\as{ಓಂ ಚಾಮುಂಡಾಯೈ } ಕವಚಾಯ ಹುಂ ।\\ ಕವಚಶಕ್ತಿ ಶ್ರೀಪಾದುಕಾಂ ಪೂ । ನಮಃ ॥\\
\as{ಓಂ ವಿಚ್ಚೇ} ನೇತ್ರತ್ರಯಾಯ ವೌಷಟ್ । \\ನೇತ್ರಶಕ್ತಿ ಶ್ರೀಪಾದುಕಾಂ ಪೂ । ನಮಃ ॥\\
\as{ಓಂ ಐಂಹ್ರೀಂಕ್ಲೀಂ ಚಾಮುಂಡಾಯೈ ವಿಚ್ಛೇ } ಅಸ್ತ್ರಾಯ ಫಟ್ ।\\ ಅಸ್ತ್ರಶಕ್ತಿ ಶ್ರೀಪಾದುಕಾಂ ಪೂ । ನಮಃ ॥ಇತಿ ಚ ಷಡಂಗಾನಿ ವಿನ್ಯಸ್ಯ

\as{ಅಸ್ತ್ರಾಯ ಫಟ್ } ಇತಿ ಅಸ್ತ್ರೇಣ ಸಂರಕ್ಷ್ಯ ।\\ \as{ಕವಚಾಯ ಹುಂ } ಇತಿ ಅವಕುಂಠನ ಮುದ್ರಯಾ ಅವಕುಂಠ್ಯ

ಶಂಖಂ ಚಂದ್ರಾರ್ಕದೈವತ್ಯಂ ವಾರುಣಂ ಚಾಧಿದೈವತ್ಯಂ ।\\
ಪೃಷ್ಠೇ ಪ್ರಜಾಪತಿಸ್ತತ್ರ ಅಗ್ರೇ ಗಂಗಾಸರಸ್ವತೀ ॥

ತ್ರೈಲೋಕ್ಯೇ ಯಾನಿ ತೀರ್ಥಾನಿ ವಾಸುದೇವಸ್ಯ ಚಾಜ್ಞಯಾ ।\\
ಶಂಖೇ ತಿಷ್ಠಂತಿ ವಿಪ್ರೇಂದ್ರ ತಸ್ಮಾಚ್ಛಂಖಂ ಪ್ರಪೂಜಯೇತ್ ॥

ವಿಲಯಂ ಯಾಂತಿ ಪಾಪಾನಿ ಹಿಮವದ್ಭಾಸ್ಕರೋದಯೇ ।\\
ದರ್ಶನೇನಾಪಿ ಶಂಖಸ್ಯ ಕಿಂ ಪುನಃ ಸ್ಪರ್ಶನೇ ಭವೇತ್ ॥

ಪಾಂಚಜನ್ಯಂ ಮಹಾತ್ಮಾನಂ ಪಾಪಘ್ನಂ ತು ಪವಿತ್ರಕಂ ।\\
ನತ್ವಾ ಶಂಖಂ ಕರೇ ಧೃತ್ವಾ ಮಂತ್ರೈ ರೇತೈಸ್ತು ವೈಷ್ಣವೈಃ ।\\
ಯಃ ಸ್ನಾಪಯತಿ ಗೋವಿಂದಂ ತಸ್ಯ ಪುಣ್ಯಮನಂತಕಂ ॥

ಶಂಖಮಧ್ಯಸ್ಥಿತಂ ತೋಯಂ ಭ್ರಾಮಿತಂ ಕೇಶವೋಪರಿ ।\\
ಅಂಗಲಗ್ನಂ ಮನುಷ್ಯಾಣಾಂ ಬ್ರಹ್ಮಹತ್ಯಾಯುತಂ ದಹೇತ್ ॥

ಗರ್ಭಾ ದೇವಾರಿ ನಾರೀಣಾಂ ವಿಶೀರ್ಯಂತೇ ಸಹಸ್ರಧಾ ।\\
ತವ ನಾದೇನ ಪಾತಾಳೇ ಪಾಂಚಜನ್ಯ ನಮೋಸ್ತುತೇ ॥

ತ್ವಂ ಪುರಾ ಸಾಗರೋತ್ಪನ್ನೋ ವಿಷ್ಣುನಾ ವಿಧೃತಃ ಕರೇ ।\\
ನಿರ್ಮಿತಃ ಸರ್ವದೇವೈಸ್ತು ಪಾಂಚಜನ್ಯ ನಮೋಸ್ತುತೇ ॥

ಪವನಾಯ ನಮಃ । ಪಾಂಚಜನ್ನ್ಯಾಯ ನಮಃ । ಪದ್ಮ ಗರ್ಭಾಯ ನಮಃ । ಅಂಬುರಾಜಾಯ ನಮಃ । ಕಂಬುರಾಜಾಯ ನಮಃ । ಧವಳಾಯ ನಮಃ ।\\
\as{ಪಾಂಚಜನ್ಯಾಯ ವಿದ್ಮಹೇ ಪದ್ಮಗರ್ಭಾಯ ಧೀಮಹಿ ।\\
ತನ್ನಃ  ಶಂಖಃ ಪ್ರಚೋದಯಾತ್ ॥}\\
ಪ್ರಣವೇನ ಅಷ್ಟವಾರಮಭಿಮಂತ್ರ್ಯ ಧೇನುಮುದ್ರಾಂ ಯೋನಿಮುದ್ರಾಂ ಚ ಪ್ರದರ್ಶ್ಯ ।  ಪೂಜಾದ್ರವ್ಯಾಣಿ ಆತ್ಮಾನಂ ಚ ಸಂಪ್ರೋಕ್ಷ್ಯ , ಕಲಶೇ ಕಿಂಚಿನ್ನಿಕ್ಷಿಪೇತ್ ।

\section{ಆತ್ಮಾರ್ಚನಂ}
ಹೃದಿ ಸ್ಥಿತಂ ಪಂಕಜಮಷ್ಟಪತ್ರಂ\\ ಸಕೇಸರಂ ಕರ್ಣಿಕಮಧ್ಯನಾಲಂ ।\\
ಅಂಗುಷ್ಠಮಾತ್ರಂ ಮುನಯೋ ವದಂತಿ\\ ಧ್ಯಾಯೇಚ್ಚ ವಿಷ್ಣುಂ ಪುರುಷಂ ಪುರಾಣಮ್ ॥

ಹೃದಯಕಮಲಮಧ್ಯೇ ಸೂರ್ಯಬಿಂಬಾಸನಸ್ಥಂ\\
ಸಕಲಭುವನಬೀಜಂ ಸೃಷ್ಟಿಸಂಹಾರಹೇತುಮ್ ।\\
ನಿರತಿಶಯಸುಖಾತ್ಮ ಜ್ಯೋತಿಷಂ ಹಂಸರೂಪಂ\\
ಪರಮಪುರುಷಮಾದ್ಯಂ ಚಿಂತಯೇದಾತ್ಮಮೂರ್ತಿಮ್ ॥

ಆರಾಧಯಾಮಿ ಮಣಿಸನ್ನಿಭಮಾತ್ಮಲಿಂಗಂ\\
ಮಾಯಾಪುರೀ ಹೃದಯಪಂಕಜ ಸನ್ನಿವಿಷ್ಟಮ್ ।\\
ಶ್ರದ್ಧಾನದೀ ವಿಮಲಚಿತ್ತ ಜಾಲಾಭಿಷೇಕೈಃ\\
ನಿತ್ಯಂ ಸಮಾಧಿ ಕುಸುಮೈರಪುನರ್ಭವಾಯ ॥

ದೇಹೋ ದೇವಾಲಯಃ ಪ್ರೋಕ್ತೋ ಜೀವೋ ಹಂಸಃ ಸದಾಶಿವಃ ।\\
ತ್ಯಜೇದಜ್ಞಾನನಿರ್ಮಾಲ್ಯಂ ಸೋಽಹಂ ಭಾವೇನ ಪೂಜಯೇತ್ ॥
\section{ಮಂಟಪಾರ್ಚನಂ}
ಮಹಾಮಾಣಿಕ್ಯ ವೈಡೂರ್ಯ ಕಾಂಚನಸ್ತಂಭ ಭೂಷಿತಮ್ ।\\
ಮುಕ್ತಾದಾಮ ಸಮಾಕೀರ್ಣಂ ರತ್ನಭಿತ್ತಿ ಪರಿಷ್ಕೃತಮ್ ॥

ಮಂದವಾಯು ಗತಿ ಕ್ರಾಂತಂ ಗಂಧಧೂಪ ಮನೋಹರಮ್ ।\\
ರತ್ನಚಾಮರಘಂಟಾದಿ ವಿತಾನೈರುಪಶೋಭಿತಮ್ ॥

ಜಾತೀ ಚಂಪಕ ಕಹ್ಲಾರ ಕೇತಕೀ ಮಲ್ಲಿಕಾದಿಭಿಃ ।\\
ಬದ್ಧಾಭಿಶ್ಚಿತ್ರಮಾಲಾಭಿಃ ಸುವೃತಾಭಿರಲಂಕೃತಮ್ ।\\
ಚತುರ್ದ್ವಾರ ಸಮಾಯುಕ್ತಂ ಧ್ಯಾಯೇತ್ ಸೌಭಾಗ್ಯ ಮಂಟಪಮ್॥

ಚಿತ್ರಮಂಟಪಾಯ ನಮಃ ।
ಯೋಗಮಂಟಪಾಯ ನಮಃ ।
ಭೋಗಮಂಟಪಾಯ ನಮಃ ।
ರತ್ನಮಂಟಪಾಯ ನಮಃ ।
ಪ್ರವಾಲಮಂಟಪಾಯ ನಮಃ ।
ಮೌಕ್ತಿಕಮಂಟಪಾಯ ನಮಃ ।
ಪುಷ್ಪಮಂಟಪಾಯ ನಮಃ ।
ನವರತ್ನಖಚಿತಸೌಭಾಗ್ಯಮಂಟಪಾಯ ನಮಃ ।
ಮೋಕ್ಷಲಕ್ಷ್ಮೀವಿಲಾಸಮಂಟಪಾಯ ನಮಃ ॥\\
ತನ್ಮಧ್ಯೇ ಶ್ರೀಮಹಾಕಾಳೀ ಮಹಾಲಕ್ಷ್ಮೀ ಮಹಾಸರಸ್ವತೀ ಸ್ವರೂಪಿಣ್ಯೈ ದುರ್ಗಾಪರಮೇಶ್ವರ್ಯೈ ನಮಃ ॥\\
ನವರತ್ನಖಚಿತಸೌಭಾಗ್ಯ ಮಂಟಪಂ ಕಲ್ಪಯಾಮಿ॥

\section{ದ್ವಾರಪೂಜಾ}
\as{ಓಂ ಪೂರ್ವ}ದ್ವಾರೇ ಶ್ರಿಯೈ  ನಮಃ ।\\
ಭದ್ರಾಯೈ ನಮಃ ।
ಸುಭದ್ರಾಯೈ ನಮಃ ।\\
\as{ಓಂ ದಕ್ಷಿಣ}ದ್ವಾರೇ ಶ್ರಿಯೈ  ನಮಃ ।\\
ಬಲಾಯ ನಮಃ ।
ಪ್ರಬಲಾಯ ನಮಃ ।\\
\as{ಓಂ ಪಶ್ಚಿಮ}ದ್ವಾರೇ ಶ್ರಿಯೈ  ನಮಃ ।\\
ಜಯಾ ನಮಃ ।
ವಿಜಯಾಯೈ ನಮಃ ।\\
\as{ಓಂ ಉತ್ತರ}ದ್ವಾರೇ ಶ್ರಿಯೈ  ನಮಃ ।\\
ಚಂಡಾಯೈ ನಮಃ ।
ಪ್ರಚಂಡಾಯೈ ನಮಃ ।\\
ಪೂರ್ವಸಮುದ್ರಾಯ  ನಮಃ ।
ದಕ್ಷಿಣಸಮುದ್ರಾಯ  ನಮಃ ।\\
ಪಶ್ಚಿಮಸಮುದ್ರಾಯ  ನಮಃ ।
ಉತ್ತರಸಮುದ್ರಾಯ  ನಮಃ ।\\
ಕೃತಯುಗಾಯ ನಮಃ ।
ತ್ರೇತಾಯುಗಾಯ ನಮಃ ।\\
ದ್ವಾಪರಯುಗಾಯ ನಮಃ ।
ಕಲಿಯುಗಾಯ ನಮಃ ।\\
ದ್ವಾರಪಾಲಪೂಜಾಂ ಸಮರ್ಪಯಾಮಿ॥
\section{ಪೀಠಪೂಜಾ}
 ಓಂ ಆಧಾರಶಕ್ತ್ಯೈ ನಮಃ । ಓಂ ಮೂಲಪ್ರಕೃತ್ಯೈ ನಮಃ । ಓಂ ಆದಿಕೂರ್ಮಾಯ ನಮಃ । ಓಂ ಅನಂತಾಯ ನಮಃ । ಓಂ ಪೃಥಿವ್ಯೈ ನಮಃ । ಓಂ ಕ್ಷೀರಸಮುದ್ರಾಯ ನಮಃ । ಓಂ ರತ್ನದ್ವೀಪಾಯ ನಮಃ । ಓಂ ಹೇಮಪ್ರಾಕಾರಾಯ ನಮಃ । ಓಂ ಕಲ್ಪವೃಕ್ಷಾಯ ನಮಃ । ಓಂ ರತ್ನಮಂಡಪಾಯ ನಮಃ । ಓಂ ಶ್ವೇತಚ್ಛತ್ರಾಯ ನಮಃ । ಓಂ ರತ್ನ ಸಿಂಹಾಸನಾಯ ನಮಃ । ಓಂ ಐರಾವತಾಯ ನಮಃ । ಓಂ ಪುಂಡರೀಕಾಯ ನಮಃ । ಓಂ ವಾಮನಾಯ ನಮಃ । ಓಂ ಕುಮುದಾಯ ನಮಃ । ಓಂ ಅಂಜನಾಯ ನಮಃ । ಓಂ ಪುಷ್ಪದಂತಾಯ ನಮಃ । ಓಂ ಸಾರ್ವಭೌಮಾಯ ನಮಃ । ಓಂ ಸುಪ್ರತೀಕಾಯ ನಮಃ । \as{ಪ್ರಥಮಪಾದೇ} ಓಂ  ಧರ್ಮಾಯ  ವೃಷರೂಪಾಯ  ರಕ್ತವರ್ಣಾಯ ನಮಃ । \as{ದ್ವಿತೀಯಪಾದೇ} ಓಂ ಜ್ಞಾನಾಯ  ಸಿಂಹರೂಪಾಯ ಶ್ಯಾಮವರ್ಣಾಯ ನಮಃ । \as{ತೃತೀಯಪಾದೇ} ಓಂ ವೈರಾಗ್ಯಾಯ  ಭೂತರೂಪಾಯ  ಪೀತವರ್ಣಾಯ ನಮಃ ।  \as{ಚತುರ್ಥಪಾದೇ} ಓಂ ಐಶ್ವರ್ಯಾಯ  ಗಜರೂಪಾಯ  ನೀಲವರ್ಣಾಯ ನಮಃ ।\as{ ಪೀಠಾಧಃ ಪ್ರಥಮಪಾದತಲೇ} ಓಂ ಅಧರ್ಮಾಯ ವೃಷರೂಪಾಯ  ರಕ್ತವರ್ಣಾಯ ನಮಃ । \as{ ದ್ವಿತೀಯಪಾದತಲೇ} ಓಂ ಅಜ್ಞಾನಾಯ ಸಿಂಹರೂಪಾಯ ಶ್ಯಾಮವರ್ಣಾಯ ನಮಃ । \as{ತೃತೀಯಪಾದತಲೇ} ಓಂ ಅವೈರಾಗ್ಯಾಯ ಭೂತರೂಪಾಯ ಪೀತವರ್ಣಾಯ ನಮಃ । \as{ಚತುರ್ಥಪಾದತಲೇ} ಓಂ ಅನೈಶ್ವರ್ಯಾಯ ಗಜರೂಪಾಯ ನೀಲವರ್ಣಾಯ ನಮಃ । \as{ಪೀಠಮಧ್ಯೇ} ಓಂ ಅಂ ಅನಂತಾಯ ನಮಃ । ಓಂ ಆನಂದಕಂದಾಯ ನಮಃ । ಓಂ ಪಂ ಪದ್ಮಾಯ ನಮಃ । ಓಂ ಪತ್ರೇಭ್ಯೋ ನಮಃ । ಓಂ ಕೇಸರೇಭ್ಯೋ ನಮಃ । ಓಂ ಕರ್ಣಿಕಾಯೈ ನಮಃ । ಓಂ ಬೀಜೇಭ್ಯೋ ನಮಃ ।  ಓಂ ಅಂ ಸೂರ್ಯಮಂಡಲಾಯ ವಸುಪ್ರದ ದ್ವಾದಶಕಲಾತ್ಮನೇ ನಮಃ । ಓಂ ಉಂ ಸೋಮಮಂಡಲಾಯ ಕಾಮಪ್ರದ ಷೋಡಶ ನಮಃ । ಓಂ ಮಂ ವಹ್ನಿಮಂಡಲಾಯ ಧರ್ಮಪ್ರದ ದಶಕಲಾತ್ಮನೇ ನಮಃ । ಓಂ ಹಂ ಧ್ರುವಮಂಡಲಾಯ ನಮಃ । ಓಂ ಸಂ ಸತ್ವಾಯ ನಮಃ । ಓಂ ರಂ ರಜಸೇ ನಮಃ । ಓಂ ತಂ ತಮಸೇ ನಮಃ । ಓಂ ಅಂ ಆತ್ಮನೇ ನಮಃ । ಓಂ ಆಂ ಅಂತರಾತ್ಮನೇ ನಮಃ । ಓಂ ಮಂ ಮಾಯಾತತ್ವಾತ್ಮನೇ ನಮಃ । ಓಂ ವಿಂ ವಿಷ್ಣುತತ್ವಾತ್ಮನೇ ನಮಃ । ಓಂ ಶಂ ಶಕ್ತಿತತ್ವಾತ್ಮನೇ ನಮಃ । ಓಂ ವಿಂ ವಿದ್ಯಾತತ್ವಾತ್ಮನೇ ನಮಃ । ಓಂ ಚತುಃಷಷ್ಟಿಯೋಗಿನೀಭ್ಯೋ ನಮಃ । ಓಂ ಬ್ರಾಹ್ಮ್ಯೈ ನಮಃ । ಓಂ ಮಾಹೇಶ್ವರ್ಯೈ ನಮಃ । ಓಂ ಕೌಮಾರ್ಯೈ ನಮಃ । ಓಂ ವೈಷ್ಣವ್ಯೈ ನಮಃ । ಓಂ ವಾರಾಹ್ಯೈ ನಮಃ । ಓಂ ಇಂದ್ರಾಣ್ಯೈ ನಮಃ । ಓಂ ನಾರಸಿಂಹ್ಯೈ ನಮಃ । ಓಂ ಚಾಮುಂಡಾಯೈ ನಮಃ । ಓಂ ಮಹಾಲಕ್ಷ್ಮ್ಯೈ ನಮಃ । ಓಂ ಮಹಾಸರಸ್ವತ್ಯೈ ನಮಃ । ಓಂ ವಿದ್ಯುತೇ ನಮಃ । ಓಂ ಶಾಂತ್ಯೈ ನಮಃ । ಓಂ ಪುಷ್ಟ್ಯೈ ನಮಃ । ಓಂ ತುಷ್ಟ್ಯೈ ನಮಃ । ಓಂ ಮೇಧಾಯೈ ನಮಃ । ಓಂ ಅನುಗ್ರಹಾಯೈ ನಮಃ । ಓಂ ಪ್ರಭಾಯೈ ನಮಃ । ಓಂ ಮಾಯಾಯೈ ನಮಃ । ಓಂ ಜಯಾಯೈ ನಮಃ । ಓಂ ಸೂಕ್ಷ್ಮಾಯೈ ನಮಃ । ಓಂ ವಿಶುದ್ಧಾಯೈ ನಮಃ । ಓಂ ನಂದಿನ್ಯೈ ನಮಃ । ಓಂ ಸುಪ್ರಭಾಯೈ ನಮಃ । ಓಂ ವಿಜಯಾಯೈ ನಮಃ । ಓಂ ಸರ್ವಾತ್ಮಿಕಾಯೈ ನಮಃ । ಓಂ ಸಿದ್ಧಿದಾಯೈ ನಮಃ । ಓಂ ಗಂ ಗಣಪತಯೇ ನಮಃ । ಓಂ ವಂ ವಟುಕಭೈರವಾಯ ನಮಃ । ಓಂ ಕಾಮರೂಪಪೀಠಾಯ ನಮಃ । ಓಂ ಜಾಲಂಧರಪೀಠಾಯ ನಮಃ । ಓಂ ಪೂರ್ಣಗಿರಿಪೀಠಾಯ ನಮಃ । ಓಂ ಓಡ್ಯಾಣಪೀಠಾಯ ನಮಃ । ಓಂ ಕಲ್ಪವೃಕ್ಷವನವಾಟಿಕಾಯೈ ನಮಃ । ಓಂ ಬ್ರಹ್ಮಣೇ ಪೃಥಿವ್ಯಧಿಪತಯೇ ನಮಃ । ಓಂ ವಿಷ್ಣವೇ ಪಾತಾಳಾಧಿಪತಯೇ ನಮಃ । ಓಂ ರುದ್ರಾಯ ತೇಜೋಽಧಿಪತಯೇ ನಮಃ । ಓಂ ಈಶ್ವರಾಯ ವಿದ್ಯಾಧಿಪತಯೇ ನಮಃ । ಪಂಚವಕ್ತ್ರಾಯ ಸದಾಶಿವಾಯ ನಮಃ । ಓಂ ಭಗವತ್ಯೈ ಸರ್ವೇಶ್ವರ್ಯೈ  ಸರ್ವಾತ್ಮಿಕಾಯೈ ಸರ್ವಭೂತಾತ್ಮಿಕಾಯೈ ಓಂ ಐಂಹ್ರೀಂಶ್ರೀಂ ಇತಿ ಮಾಲಾತ್ಮಿಕಾಯೈ ಮಹಾಕಾಳೀ ಮಹಾಲಕ್ಷ್ಮೀ ಮಹಾಸರಸ್ವತ್ಯಾತ್ಮಿಕಾಯೈ ಶ್ರೀದುರ್ಗಾಪರಮೇಶ್ವರ್ಯೈ ನಮಃ । ಸುವರ್ಣಪೀಠಂ ಕಲ್ಪಯಾಮಿ ॥
\newpage
\section{ದಿಕ್ಪಾಲಕಪೂಜಾ}
\as{ಓಂ ಲಂ} ಇಂದ್ರಾಯ ಸುರಾಧಿಪತಯೇ ವಜ್ರಹಸ್ತಾಯ ಶ್ವೇತವರ್ಣಾಯ ಐರಾವತವಾಹನಾಯ ಸಾಂಗಾಯ ಸಾಯುಧಾಯ ಸವಾಹನಾಯ ಸಪರಿವಾರಾಯ ಸಶಕ್ತಿಕಾಯ  ಶ್ರೀದುರ್ಗಾಪರಮೇಶ್ವರೀಪಾರ್ಷದಾಯ ನಮಃ~॥\\
\as{ಓಂ ರಂ} ಅಗ್ನಯೇ ತೇಜೋಽಧಿಪತಯೇ ಶಕ್ತಿಹಸ್ತಾಯ ರಕ್ತವರ್ಣಾಯ ಮೇಷವಾಹನಾಯ ಸಾಂಗಾಯ\\
\as{ಓಂ ಟಂ} ಯಮಾಯ ಪ್ರೇತಾಧಿಪತಯೇ ದಂಡಹಸ್ತಾಯ ನೀಲವರ್ಣಾಯ ಮಹಿಷವಾಹನಾಯ ಸಾಂಗಾಯ ******** ನಮಃ~॥\\
\as{ಓಂ ಕ್ಷಂ} ನಿರ್ಋತಯೇ ರಕ್ಷೋಧಿಪತಯೇ ಖಡ್ಗಹಸ್ತಾಯ ಹಿರಣ್ಯವರ್ಣಾಯ ನರವಾಹನಾಯ ಸಾಂಗಾಯ ******** ನಮಃ~॥\\
\as{ಓಂ ವಂ} ವರುಣಾಯ ಜಲಾಧಿಪತಯೇ ಪಾಶಹಸ್ತಾಯ ನೀಲಶುಭ್ರವರ್ಣಾಯ ಮಕರವಾಹನಾಯ ಸಾಂಗಾಯ ****** ನಮಃ~॥\\
\as{ಓಂ ಯಂ} ವಾಯವೇ ಸರ್ವಪ್ರಾಣಾಧಿಪತಯೇ ಅಂಕುಶಹಸ್ತಾಯ ಮೇಘವರ್ಣಾಯ ಹರಿಣವಾಹನಾಯ ಸಾಂಗಾಯ  ***** ನಮಃ~॥\\
\as{ಓಂ ಸಂ} ಸೋಮಾಯ ನಕ್ಷತ್ರಾಧಿಪತಯೇ ಗದಾಹಸ್ತಾಯ ಶುಕ್ಲವರ್ಣಾಯ ಅಶ್ವವಾಹನಾಯ ಸಾಂಗಾಯ ****** ನಮಃ~॥\\
\as{ಓಂ ಯಂ} ಈಶಾನಾಯ ವಿದ್ಯಾಧಿಪತಯೇ ತ್ರಿಶೂಲಹಸ್ತಾಯ ಹೇಮವರ್ಣಾಯ ವೃಷಭವಾಹನಾಯ ಸಾಂಗಾಯ **** ನಮಃ~॥\\
\as{ಓಂ ಐಂ} ಬ್ರಹ್ಮಣೇ ಸರ್ವಲೋಕಾಧಿಪತಯೇ ಪದ್ಮಹಸ್ತಾಯ ಶುಭ್ರವರ್ಣಾಯ ಹಂಸವಾಹನಾಯ  ಸಾಂಗಾಯ ********* ನಮಃ~॥\\
\as{ಓಂ ವಂ}  ಅನಂತಾಯ ಪಾತಾಲಾ(ನಾಗಾ)ಧಿಪತಯೇ ಚಕ್ರಹಸ್ತಾಯ ನೀಲವರ್ಣಾಯ ಗರುಡವಾಹನಾಯ  ಸಾಂಗಾಯ**** ನಮಃ~॥\\
ಪೀಠಪೂಜಾಂ ಸಮರ್ಪಯಾಮಿ~॥
\section{ನವಶಕ್ತಿಪೂಜಾ}
ಓಂ ಪ್ರಭಾಯೈ ನಮಃ । ಓಂ  ಮಾಯಾಯೈ ನಮಃ । ಓಂ  ಜಯಾಯೈ ನಮಃ । ಓಂ  ಸೂಕ್ಷ್ಮಾಯೈ ನಮಃ । ಓಂ  ವಿಶುದ್ಧಾಯೈ ನಮಃ । ಓಂ  ನಂದಿನ್ಯೈ ನಮಃ । ಓಂ  ಸುಪ್ರಭಾಯೈ ನಮಃ । ಓಂ  ವಿಜಯಾಯೈ ನಮಃ । ಓಂ  ಸರ್ವಸಿದ್ಧಿಪ್ರದಾಯೈ ನಮಃ ।
\section{ಕಲಾವಾಹನಪೂಜಾ}
\subsection{ವಹ್ನಿಕಲಾಃ}
\as{ಓಂ ಹ್ರೀಂ ಯಂ} ಧೂಮ್ರಾರ್ಚಿಷೇ ತ್ವಗಾತ್ಮನೇ ಕಲಾಶಕ್ತ್ಯೈ ನಮಃ~। \as{ಓಂ ಹ್ರೀಂ ರಂ} ಊಷ್ಮಾಯೈ ಚರ್ಮಾತ್ಮನೇ~। \as{ಓಂ ಹ್ರೀಂ ಲಂ} ಜ್ವಲಿನ್ಯೈ ಮಾಂಸಾತ್ಮನೇ~। \as{ಓಂ ಹ್ರೀಂ ವಂ} ಜ್ವಾಲಿನ್ಯೈ ರುಧಿರಾತ್ಮನೇ~। \as{ಓಂ ಹ್ರೀಂ ಶಂ} ವಿಸ್ಫಲಿಂಗಿನ್ಯೈ ಮೇದಾತ್ಮನೇ~। \as{ಓಂ ಹ್ರೀಂ ಷಂ} ಸುಶ್ರಿಯೈ ಮಜ್ಜಾತ್ಮನೇ~। \as{ಓಂ ಹ್ರೀಂ ಸಂ} ಸುರೂಪಾಯೈ ಅಸ್ಥ್ಯಾತ್ಮನೇ~। \as{ಓಂ ಹ್ರೀಂ ಹಂ} ಕಪಿಲಾಯೈ ಶುಕ್ಲಾತ್ಮನೇ~। \as{ಓಂ ಹ್ರೀಂ ಳಂ} ಹವ್ಯವಾಹಾಯೈ ಪ್ರಾಣಾತ್ಮನೇ~। \as{ಓಂ ಹ್ರೀಂ ಕ್ಷಂ} ಕವ್ಯವಾಹಾಯೈ ಶಕ್ತ್ಯಾತ್ಮನೇ ಕಲಾಶಕ್ತ್ಯೈ ನಮಃ~॥\\
\as{ಓಂ ಜಾ॒ತವೇ᳚ದಸೇ ಸುನವಾಮ॒ ಸೋಮ॑ ಮರಾತೀಯ॒ತೋ ನಿದ॑ಹಾತಿ॒ ವೇದಃ॑ ।
ಸ ನಃ॑ ಪರ್-ಷ॒ದತಿ॑ ದು॒ರ್ಗಾಣಿ॒ ವಿಶ್ವಾ᳚ ನಾ॒ವೇವ॒ ಸಿಂಧುಂ᳚ ದುರಿ॒ತಾಽತ್ಯ॒ಗ್ನಿಃ ॥}
\subsection{ಸೂರ್ಯಕಲಾಃ}
\as{ಓಂ ಹ್ರೀಂ ಕಂ ಭಂ} ತಪಿನೀ ಕಲಾಶಕ್ತ್ಯೈ ನಮಃ~। \as{ಓಂ ಹ್ರೀಂ ಖಂ ಬಂ} ತಾಪಿನೀ~। \as{ಓಂ ಹ್ರೀಂ ಗಂ ಫಂ} ಧೂಮ್ರಾ~। \as{ಓಂ ಹ್ರೀಂ ಘಂ ಪಂ} ಮರೀಚಿ~। \as{ಓಂ ಹ್ರೀಂ ಙಂ ನಂ} ಜ್ವಾಲಿನೀ~। \as{ಓಂ ಹ್ರೀಂ ಚಂ ಧಂ} ರುಚಿ~। \as{ಓಂ ಹ್ರೀಂ ಛಂ ದಂ} ಸುಷುಮ್ನಾ~। \as{ಓಂ ಹ್ರೀಂ ಜಂ ಥಂ} ಭೋಗದಾ~। \as{ಓಂ ಹ್ರೀಂ ಝಂ ತಂ} ವಿಶ್ವಾ~। \as{ಓಂ ಹ್ರೀಂ ಞಂ ಣಂ} ಬೋಧಿನೀ~। \as{ಓಂ ಹ್ರೀಂ ಟಂ ಢಂ} ಧಾರಿಣೀ~। \as{ಓಂ ಹ್ರೀಂ ಠಂ ಡಂ} ಕ್ಷಮಾಕಲಾಶಕ್ತ್ಯೈ ನಮಃ~॥\\
 \as{ಓಂ ಹಂ॒ಸಃ ಶು॑ಚಿ॒ಷದ್ವಸು॑ರಂತರಿಕ್ಷ॒ ಸದ್ಧೋತಾ᳚ವೇದಿ॒ಷದತಿ॑ಥಿರ್ದುರೋಣ॒ಸತ್ । ನೃ॒ಷದ್ವ॑ರ॒ಸದೃ॑ತ॒ಸದ್ವ್ಯೋ᳚ಮ॒ ಸದ॒ಬ್ಜಾ ಗೋ॒ಜಾ ಋ॑ತ॒ಜಾ ಅ॑ದ್ರಿ॒ಜಾ ಋ॒ತಮ್ ॥} \\
 \as{ಓಂ ತತ್ಸ॑ವಿ॒ತುರ್ವರೇ᳚ಣ್ಯಂ॒ ಭರ್ಗೋ᳚ ದೇ॒ವಸ್ಯ॑ಧೀಮಹಿ ।\\ ಧಿಯೋ॒ ಯೋ ನಃ॑ ಪ್ರಚೋ॒ದಯಾ᳚ತ್ ॥}
\subsection{ಸೋಮಕಲಾಃ}
\as{ಓಂ ಹ್ರೀಂ ಅಂ} ಅಮೃತಾಕಲಾಶಕ್ತ್ಯೈ ನಮಃ~। \as{ಓಂ ಹ್ರೀಂ ಆಂ} ಮಾನದಾ~। \as{ಓಂ ಹ್ರೀಂ ಇಂ} ಪೂಷಾ~। \as{ಓಂ ಹ್ರೀಂ ಈಂ} ತುಷ್ಟಿ~। \as{ಓಂ ಹ್ರೀಂ ಉಂ} ಪುಷ್ಟಿ~। \as{ಓಂ ಹ್ರೀಂ ಊಂ} ರತಿ~। \as{ಓಂ ಹ್ರೀಂ ಋಂ} ಋದ್ಧಿ~। \as{ಓಂ ಹ್ರೀಂ ೠಂ} ಶಶಿನೀ~। \as{ಓಂ ಹ್ರೀಂ ಲೃಂ} ಚಂದ್ರಿಕಾ~। \as{ಓಂ ಹ್ರೀಂ ಲೄಂ} ಕಾಂತಿ~। \as{ಓಂ ಹ್ರೀಂ ಏಂ} ಜ್ಯೋತ್ಸ್ನಾ~। \as{ಓಂ ಹ್ರೀಂ ಐಂ} ಶ್ರೀ~। \as{ಓಂ ಹ್ರೀಂ ಓಂ} ಪ್ರೀತಿ~। \as{ಓಂ ಹ್ರೀಂ ಔಂ} ಅಂಗದಾ~। \as{ಓಂ ಹ್ರೀಂ ಅಂ} ಪೂರ್ಣಾ~। ಓಂ ಹ್ರೀಂ ಅಃ ಪೂರ್ಣಾಮೃತಾಕಲಾಶಕ್ತ್ಯೈ ನಮಃ~॥\\
\as{ಓಂ ತ್ರ್ಯಂ᳚ಬಕಂ ಯಜಾಮಹೇ ಸು॒ಗಂಧಿಂ᳚ ಪುಷ್ಟಿ॒ವರ್ಧ॑ನಂ ।\\
ಉ॒ರ್ವಾ॒ರು॒ಕಮಿ॑ವ॒ ಬಂಧ॑ನಾನ್ ಮೃ॒ತ್ಯೋರ್ಮು॑ಕ್ಷೀಯ॒ ಮಾಽಮೃತಾ᳚ತ್ ॥}
\subsection{ಬ್ರಹ್ಮಕಲಾಃ} \as{ಓಂ ಹ್ರೀಂ ಕಂ} ಸೃಷ್ಟ್ಯೈ ನಮಃ~। \as{ಓಂ ಹ್ರೀಂ ಖಂ} ಋದ್ಧ್ಯೈ~। \as{ಓಂ ಹ್ರೀಂ ಗಂ} ಸ್ಮೃತ್ಯೈ~। \as{ಓಂ ಹ್ರೀಂ ಘಂ} ಮೇಧಾಯೈ~। \as{ಓಂ ಹ್ರೀಂ ಙಂ} ಕಾಂತ್ಯೈ~। \as{ಓಂ ಹ್ರೀಂ ಚಂ} ಲಕ್ಷ್ಮ್ಯೈ~। \as{ಓಂ ಹ್ರೀಂ ಛಂ} ದ್ಯುತ್ಯೈ~। \as{ಓಂ ಹ್ರೀಂ ಜಂ} ಸ್ಥಿರಾಯೈ~। \as{ಓಂ ಹ್ರೀಂ ಝಂ} ಸ್ಥಿತ್ಯೈ~। \as{ಓಂ ಹ್ರೀಂ ಞಂ} ಸಿದ್ಧ್ಯೈ~॥
\subsection{ವಿಷ್ಣುಕಲಾಃ} \as{ಓಂ ಹ್ರೀಂ ಟಂ} ಜರಾಯೈ ನಮಃ~। \as{ಓಂ ಹ್ರೀಂ ಠಂ} ಪಾಲಿನ್ಯೈ~। \as{ಓಂ ಹ್ರೀಂ ಡಂ} ಶಾಂತ್ಯೈ~। \as{ಓಂ ಹ್ರೀಂ ಢಂ} ಈಶ್ವರ್ಯೈ~। \as{ಓಂ ಹ್ರೀಂ ಣಂ} ರತ್ಯೈ~। \as{ಓಂ ಹ್ರೀಂ ತಂ} ಕಾಮಿಕಾಯೈ~। \as{ಓಂ ಹ್ರೀಂ ಥಂ} ವರದಾಯೈ~। \as{ಓಂ ಹ್ರೀಂ ದಂ} ಆಹ್ಲಾದಿನ್ಯೈ~। \as{ಓಂ ಹ್ರೀಂ ಧಂ} ಪ್ರೀತ್ಯೈ~। \as{ಓಂ ಹ್ರೀಂ ನಂ} ದೀರ್ಘಾಯೈ~॥
\subsection{ರುದ್ರಕಲಾಃ} \as{ಓಂ ಹ್ರೀಂ ಪಂ} ತೀಕ್ಷ್ಣಾಯೈ ನಮಃ~। \as{ಓಂ ಹ್ರೀಂ ಫಂ} ರೌದ್ರ್ಯೈ~। \as{ಓಂ ಹ್ರೀಂ ಬಂ} ಭಯಾಯೈ~। \as{ಓಂ ಹ್ರೀಂ ಭಂ} ನಿದ್ರಾಯೈ~। \as{ಓಂ ಹ್ರೀಂ ಮಂ} ತಂದ್ರ್ಯೈ~। \as{ಓಂ ಹ್ರೀಂ ಯಂ} ಕ್ಷುಧಾಯೈ~। \as{ಓಂ ಹ್ರೀಂ ರಂ} ಕ್ರೋಧಿನ್ಯೈ~। \as{ಓಂ ಹ್ರೀಂ ಲಂ} ಕ್ರಿಯಾಯೈ~। \as{ಓಂ ಹ್ರೀಂ ವಂ} ಉದ್ಗಾರ್ಯೈ~। \as{ಓಂ ಹ್ರೀಂ ಶಂ} ಮೃತ್ಯು~॥
\subsection{ಈಶ್ವರಕಲಾಃ} \as{ಓಂ ಹ್ರೀಂ ಷಂ} ಪೀತಾಯೈ~। \as{ಓಂ ಹ್ರೀಂ ಸಂ} ಶ್ವೇತಾಯೈ~। \as{ಓಂ ಹ್ರೀಂ ಹಂ} ಅರುಣಾಯೈ~। \as{ಓಂ ಹ್ರೀಂ ಕ್ಷಂ} ಅಸಿತಾಯೈ~॥
\subsection{ಸದಾಶಿವಕಲಾಃ} \as{ಓಂ ಹ್ರೀಂ ಅಂ} ನಿವೃತ್ತ್ಯೈ ನಮಃ~। \as{ಓಂ ಹ್ರೀಂ ಆಂ} ಪ್ರತಿಷ್ಠಾಯೈ~। \as{ಓಂ ಹ್ರೀಂ ಇಂ} ವಿದ್ಯಾಯೈ~। \as{ಓಂ ಹ್ರೀಂ ಈಂ} ಶಾಂತ್ಯೈ~। \as{ಓಂ ಹ್ರೀಂ ಉಂ} ಇಂಧಿಕಾಯೈ~। \as{ಓಂ ಹ್ರೀಂ ಊಂ} ದೀಪಿಕಾಯೈ~। \as{ಓಂ ಹ್ರೀಂ ಋಂ} ರೇಚಿಕಾಯೈ~। \as{ಓಂ ಹ್ರೀಂ ೠಂ} ಮೋಚಿಕಾಯೈ~। \as{ಓಂ ಹ್ರೀಂ ಲೃಂ} ಪರಾಯೈ~। \as{ಓಂ ಹ್ರೀಂ ಲೄಂ} ಸೂಕ್ಷ್ಮಾಯೈ~। \as{ಓಂ ಹ್ರೀಂ ಏಂ} ಸೂಕ್ಷ್ಮಾಮೃತಾಯೈ~। \as{ಓಂ ಹ್ರೀಂ ಐಂ} ಜ್ಞಾನಾಯೈ~। \as{ಓಂ ಹ್ರೀಂ ಓಂ} ಜ್ಞಾನಾಮೃತಾಯೈ~। \as{ಓಂ ಹ್ರೀಂ ಔಂ} ಆಪ್ಯಾಯನ್ಯೈ~। \as{ಓಂ ಹ್ರೀಂ ಅಂ} ವ್ಯಾಪಿನ್ಯೈ~। \as{ಓಂ ಹ್ರೀಂ ಅಃ} ವ್ಯೋಮರೂಪಾಯೈ ನಮಃ~॥

ಶ್ರೀದುರ್ಗಾಪರಮೇಶ್ವರ್ಯೈ ನಮಃ ॥\\
ಆವಾಹಿತಾ ಭವ~। ಸಂಸ್ಥಾಪಿತಾ ಭವ~। ಸನ್ನಿಹಿತಾ ಭವ~। ಸನ್ನಿರುದ್ಧಾ ಭವ~। ಸಮ್ಮುಖಾ ಭವ~। ಅವಗುಂಠಿತಾ ಭವ~। ವ್ಯಾಪ್ತಾ ಭವ~। ಸುಪ್ರಸನ್ನಾ ಭವ~।)

\section{ಸಮಷ್ಟಿಧ್ಯಾನಂ}
\as{ಅರುಣಕಮಲಸಂಸ್ಥಾ ತದ್ರಜಃಪುಂಜವರ್ಣಾ\\
ಕರಕಮಲಧೃತೇಷ್ಟಾಭೀತಿಯುಗ್ಮಾಂಬುಜಾ ಚ ।\\
ಮಣಿಮಕುಟವಿಚಿತ್ರಾಲಂಕೃತಾ ಪದ್ಮಹಸ್ತಾ\\
ಸಕಲಭುವನಮಾತಾ ಸಂತತಂ ಶ್ರೀಃ ಶ್ರಿಯೈ ನಃ ॥

ಹೇಮಪ್ರಖ್ಯಾಂ ಇಂದುಖಂಡಾತ್ತಮೌಲಿಂ \\
ಶಂಖಾರಿಷ್ಟಾಭೀತಿ ಹಸ್ತಾಂ  ತ್ರಿಣೇತ್ರಾಂ ।\\
ಹೇಮಾಬ್ಜಸ್ಥಾಂ ಪೀತವಸ್ತ್ರಾಂ ಪ್ರಸನ್ನಾಂ \\
ದೇವೀಂ ದುರ್ಗಾಂ ದಿವ್ಯರೂಪಾಂ ನಮಾಮಿ ॥

ಪಂಚಾಶದ್ವರ್ಣಭೇದೈರ್ವಿಹಿತ ವದನ ದೋಃಪಾದ ಹೃತ್ಕುಕ್ಷಿವಕ್ಷೋ\\
ದೇಶಾಂ ಭಾಸ್ವತ್ಕಪರ್ದಾಕಲಿತ ಶಶಿಕಲಾಮಿಂದು ಕುಂದಾವದಾತಾಂ~।\\
ಅಕ್ಷಸ್ರಕ್ಕುಂಭ ಚಿಂತಾ ಲಿಖಿತ ವರಕರಾಂ ತ್ರೀಕ್ಷಣಾಂ ಪದ್ಮಸಂಸ್ಥಾಂ\\
ಅಚ್ಛಾಕಲ್ಪಾಮತುಚ್ಛ ಸ್ತನಜಘನಭರಾಂ ಭಾರತೀಂ ತಾಂ ನಮಾಮಿ ॥

ಸೂರ್ಯಕೋಟಿ ಪ್ರತೀಕಾಶಾಂ ಮಹಿಷಾಸುರ ಮರ್ದಿನೀಮ್ ।\\
ಸಿಂಹಾರೂಢಾಂ ಮಹಾದುರ್ಗಾಂ ತ್ರಿದಶೈರಭಿಪೂಜಿತಾಮ್ ॥
\newpage
ಅಕ್ಷಸೂತ್ರಾಂಕುಶ ಧರಾಂ ಪಾಶ ಪುಸ್ತಕ ಧಾರಿಣೀಂ ।\\
ಮುಕ್ತಾಹಾರ ಸಮಾಯುಕ್ತಾಂ ದೇವೀಂ ಧ್ಯಾಯೇಚ್ಚತುರ್ಭುಜಾಮ್ ॥

ಪುಸ್ತಕಜಪವಟಹಸ್ತೇ ವರದಾಭಯಚಿಹ್ನ ಚಾರುಬಾಹುಲತೇ ।\\
ಕರ್ಪೂರಾಮಲದೇಹೇ ವಾಗೀಶ್ವರಿ ಶೋಧಯಾಶು ಮಮ ಚೇತಃ ॥}

(ತತ್ತದ್ದಿನಾನುಸಾರೇಣ ಧ್ಯಾಯೇತ್ ॥)\\
(ಪ್ರತಿಪದಿ)\\
\as{ಹಂಸಾರೂಢಾಂ ಶುಕ್ಲವರ್ಣಾಂ ಶುಕ್ಲಮಾಲ್ಯಾದ್ಯಲಂಕೃತಾಂ ।\\
ಚತುರ್ಭುಜಾಂ ಸೃಕ್ ಸೃವೌ ಚ ಕಮಂಡಲ್ವಕ್ಷಮಾಲಿಕಾಂ ।\\
ಬಿಭ್ರತೀಂ ಚಿಂತಯೇದ್ದೇವೀಂ ಪ್ರತಿಪದ್ಯಾಂ ಚ ಪೂಜಯೇತ್ ॥}\\
ಪ್ರಥಮದಿವಸೇ ಮಹಾಕಾಳೀ ಮಹಾಲಕ್ಷ್ಮೀ ಮಹಾಸರಸ್ವತೀ ಸ್ವರೂಪಿಣೀಂ ಹಂಸವಾಹಿನೀಂ ಶೈಲಪುತ್ರೀಮಾವಾಹಯಾಮಿ ॥

(ದ್ವಿತೀಯಾಯಾಂ)\\
\as{ರಕ್ತವರ್ಣಾಂ ಗಜಾರೂಢಾಂ ರಕ್ತಮಾಲ್ಯಾದಿಭೂಷಿತಾಂ ।\\
ಚತುರ್ಭುಜಾಂ ಶಕ್ತಿಶೂಲೇ ಗದಾಮಭಯಧಾರಿಣೀಂ ।\\
ದೇವೀಂ ಚ ಪೂಜಯೇದ್ಧೀಮಾನ್ ದ್ವಿತೀಯಾಯಾಂ ಮುದಾನ್ವಿತಃ ॥}\\
ದ್ವಿತೀಯದಿವಸೇ ಮಹಾಕಾಳೀ ಮಹಾಲಕ್ಷ್ಮೀ ಮಹಾಸರಸ್ವತೀ ಸ್ವರೂಪಿಣೀಂ ಗಜವಾಹಿನೀಂ ಬ್ರಹ್ಮಚಾರಿಣೀಮಾವಾಹಯಾಮಿ ॥

(ತೃತೀಯಾಯಾಂ)\\
\as{ಗೌರಾಂಗೀಂ ದ್ವಿಭುಜಾಂ ದೇವೀಂ ಕಹ್ಲಾರವರಧಾರಿಣೀಂ ।\\
ಚಂಡೀಂ ಚ ಪೂಜಯೇದ್ಧೀಮಾನ್ ಸಿಂಹಾರೂಢಾಂ ತೃತೀಯಕೇ ॥\\}
ತೃತೀಯದಿವಸೇ  ಮಹಾಕಾಳೀ ಮಹಾಲಕ್ಷ್ಮೀ ಮಹಾಸರಸ್ವತೀ ಸ್ವರೂಪಿಣೀಂ ಸಿಂಹವಾಹಿನೀಂ ಚಂಡೀಮಾವಾಹಯಾಮಿ ॥

(ಚತುರ್ಥ್ಯಾಂ)\\
\as{ಗೌರವರ್ಣಾದ್ಯಲಂಕಾರೈಃ ಕ್ಷೌಮವರ್ಣಾಂಬರಾನ್ವಿತಾಂ ।\\
ಪಾಶಾಂಕುಶ ಧರಾಂ ದೇವೀಂ ತಥಾ ಚ ಮೃಗವಾಹಿನೀಂ ॥\\
ಚತುರ್ಭುಜಾಂ ಸುವರ್ಣಾಭಾಂ ಚತುರ್ಥ್ಯಾಂ ಪೂಜಯೇನ್ನೃಪ ॥}\\
ಚತುರ್ಥದಿವಸೇ ಮಹಾಕಾಳೀ ಮಹಾಲಕ್ಷ್ಮೀ ಮಹಾಸರಸ್ವತೀ ಸ್ವರೂಪಿಣೀಂ ಮೃಗವಾಹಿನೀಂ ಕೂಷ್ಮಾಂಡಾಮಾವಾಹಯಾಮಿ ॥

(ಪಂಚಮ್ಯಾಂ)\\
\as{ಸ್ವರ್ಣವಸ್ತ್ರಾದ್ಯಲಂಕಾರೈಃ ಸ್ವರ್ಣವರ್ಣೋಪಶೋಭಿತಾಂ ।\\
ಚತುರ್ಭುಜಾಂ ಸುವರ್ಣಾಭಾಂ ಶಂಖಚಕ್ರಗದಾಂಬುಜಾಂ ॥\\
ಪಂಚಮ್ಯಾಂ ಪೂಜಯೇದ್ದೇವೀಂ ಸದಾ ಮಕರವಾಹಿನೀಂ ॥}\\
ಪಂಚಮದಿವಸೇ ಮಹಾಕಾಳೀ ಮಹಾಲಕ್ಷ್ಮೀ ಮಹಾಸರಸ್ವತೀ ಸ್ವರೂಪಿಣೀಂ ಮಕರವಾಹಿನೀಂ ಸ್ಕಂದಮಾತರಮಾವಾಹಯಾಮಿ ॥

(ಷಷ್ಠ್ಯಾಂ)\\
\as{ಶ್ವೇತಾಂಬರಾಮಾಭರಣೈಃ ಸ್ಫಾಟಿಕೈರುಪಶೋಭಿತಾಂ ।\\
ಬಾಣಕೋದಂಡಖೇಟಾಂಶ್ಚ ಕಮಂಡಲುಧರಾಂ ಶಿವಾಂ ॥\\
ಮಯೂರವಾಹಿನೀಂ ದೇವೀಂ ಷಷ್ಠ್ಯಾಂ ಸಂಪೂಜಯೇನ್ನೃಪ ॥}\\
ಷಷ್ಠದಿವಸೇ ಮಹಾಕಾಳೀ ಮಹಾಲಕ್ಷ್ಮೀ ಮಹಾಸರಸ್ವತೀ ಸ್ವರೂಪಿಣೀಂ ಮಯೂರವಾಹಿನೀಂ ಕತ್ಯಾಯನೀಮಾವಾಹಯಾಮಿ ॥

(ಸಪ್ತಮ್ಯಾಂ)\\
\as{ಶ್ವೇತವರ್ಣಾದ್ಯಲಂಕಾರೈರ್ದ್ವಿಭುಜಾಂ ಪದ್ಮಧಾರಿಣೀಂ ।\\
ಸಪ್ತಮ್ಯಾಂ ಪೂಜಯೇದ್ ಗೌರೀಂ ಅಶ್ವಾರೂಢಾಂ ಸದಾ ನೃಪ ॥}\\
ಸಪ್ತಮದಿವಸೇ ಮಹಾಕಾಳೀ ಮಹಾಲಕ್ಷ್ಮೀ ಮಹಾಸರಸ್ವತೀ ಸ್ವರೂಪಿಣೀಂ ಅಶ್ವಾರೂಢಾಂ ಗೌರೀಮಾವಾಹಯಾಮಿ ॥

(ಅಷ್ಟಮ್ಯಾಂ)\\
\as{ರೌದ್ರವರ್ಣಾದ್ಯಲಂಕಾರೈರ್ಭೂಷಿತಾಂ ಚ ಚತುರ್ಭುಜಾಂ ।\\
ತ್ರಿಶೂಲಂ ಡಮರುಂ ಚೈವ ಚರ್ಮಖಟ್ವಾಂಗಧಾರಿಣೀಂ ।\\
ಅಷ್ಟಮ್ಯಾಂ ಪೂಜಯೇನ್ಮೂರ್ತಿಂ ವೃಷಾರೂಢಾಂ ಪ್ರಯತ್ನತಃ ॥}\\
ಅಷ್ಟಮದಿವಸೇ ಮಹಾಕಾಳೀ ಮಹಾಲಕ್ಷ್ಮೀ ಮಹಾಸರಸ್ವತೀ ಸ್ವರೂಪಿಣೀಂ\\ ವೃಷಭವಾಹಿನೀಂ ತ್ರಿಮೂರ್ತಿಮಾವಾಹಯಾಮಿ ॥

(ನವಮ್ಯಾಂ)\\
\as{ರಕ್ತವರ್ಣಾಂ ರಕ್ತಭೂಷಾಂ ಅಷ್ಟಾದಶಭುಜೈರ್ಯುತಾಂ ।\\
ಸಾಕ್ಷಮಾಲಾಂ ಮಾರ್ಗಣಂ ಚ  ಅರವಿಂದಂ ತಥಾ ಧನುಃ ॥

ಭಿದುರಂ ಚ  ಗದಾಂ ಚೈವ ರಥಾಂಗಂ ಶೂಲಮೇವ ಚ ।\\
ಕುಠಾರಂ ಶಂಖಪೂರ್ಣಂ ಚ ಪಾಶಂ ಶಕ್ತಿಂ ಚ ಯಷ್ಟಿಕಾಂ ॥

ಖೇಟಂ ಶರಾಸನಂ ಪಾತ್ರಂ ಕಮಂಡಲುಧೃತಾಂ ಶಿವಾಂ ।\\
ತ್ರಿಣೇತ್ರಾಂ ಚಂದ್ರಚೂಡಾಂ ಚ ಮಕುಟೇನೋಪಶೋಭಿತಾಂ ॥

ಸೂರ್ಯಕೋಟಿಪ್ರತೀಕಾಶಾಂ ಮಹಿಷಾಸುರಮರ್ದಿನೀಂ ।\\
ಸಿಂಹಾರೂಢಾಂ ಮಹಾದುರ್ಗಾಂ ತ್ರಿದಶೈರಭಿಪೂಜಿತಾಂ ।\\
ನವಮ್ಯಾಂ ಪೂಜಯೇದ್ದೇವೀಂ ಸರ್ವಾಭೀಷ್ಟಪ್ರದಾಯಿನೀಂ ॥}\\
ನವಮದಿವಸೇ ಮಹಾಕಾಳೀ ಮಹಾಲಕ್ಷ್ಮೀ ಮಹಾಸರಸ್ವತೀ ಸ್ವರೂಪಿಣೀಂ ಮಹಾಮಾಯಾಂ ತ್ರಿಗುಣಾತ್ಮಿಕಾಂ ದುರ್ಗಾದೇವೀಮಾವಾಹಯಾಮಿ ॥
\newpage
\as{ವಿದ್ಯುದ್ದಾಮಸಮಪ್ರಭಾಂ ಮೃಗಪತಿಸ್ಕಂಧಸ್ಥಿತಾಂ ಭೀಷಣಾಂ\\
ಕನ್ಯಾಭಿಃ ಕರವಾಲ ಖೇಟ ವಿಲಸದ್ಧಸ್ತಾಭಿ ರಾಸೇವಿತಾಂ।\\
ಹಸ್ತೈಶ್ಚ ಕ್ರಗದಾಸಿಖೇಟವಿಶಿಖಾಂಶ್ಚಾಪಂ ಗುಣಂ ತರ್ಜನೀಂ\\
ಬಿಭ್ರಾಣಾಮನಲಾತ್ಮಿಕಾಂ ಶಶಿಧರಾಂ ದುರ್ಗಾಂ ತ್ರಿನೇತ್ರಾಂ ಭಜೇ॥}\\
ಶ್ರೀ ದುರ್ಗಾಪರಮೇಶ್ವರ್ಯೈ ನಮಃ ॥ ಧ್ಯಾಯಾಮಿ॥


ಸುಭಗೇ ಸುಂದರೇ ಸೌಮ್ಯೇ ಸುಷುಮ್ನೇ ಸುಮನಾವತಿ ।\\
ಮನೋನ್ಮಾಯೇ ಮಹಾಮಾಯೇ ಶೋಭನೇ ಲಲಿತೇ ಶುಭೇ ॥

ಮಹಾಪದ್ಮವನಾಂತಸ್ಥೇ ಕಾರಣಾನಂದ ವಿಗ್ರಹೇ ।\\
ಸರ್ವಭೂತಹಿತೇ ಮಾತರೇಹ್ಯೇಹಿ ಪರಮೇಶ್ವರಿ ॥\\
ದೇವೇಶಿ ಭಕ್ತಸುಲಭೇ ಸರ್ವಾವರಣ ಸಂಯುತೇ ।\\
ಯಾವತ್ವಾಂ ಪೂಜಯಿಷ್ಯಾಮಿ ತಾವತ್ವಂ ಸುಸ್ಥಿರಾ ಭವ ॥\\
\as{ಸಹಸ್ರಶೀರ್ಷಾ ಪುರುಷಃ+++++ದಶಾಂಗುಲಂ॥\\
ಹಿರಣ್ಯವರ್ಣಾಂ ++++++++++ ಮ ಆವಹ ॥} ಆವಾಹನಮ್ ॥

ಅಥ ಕಲ್ಪತರೋ ನಿತ್ಯಂ ವಸಂತ ರತ್ನಮಂದಿರೇ ।\\
ಸ್ವರ್ಣಾಸನಮಿದಂ ದೇವಿ ಸ್ವೀಕುರುಷ್ವ ಶುಭಂಕರಿ ॥\\
\as{ಪುರುಷ ಏವೇದಂ+++++ ನಾ ತಿರೋಹತಿ॥\\
ತಾಂ ಮ+++++++++++ ಪುರುಷಾನಹಮ್॥}ಆಸನಮ್ ॥

ಸರ್ವರಂಜನಶಕ್ತ್ಯಾತ್ಮನ್ ಸರ್ವೋನ್ಮಾದಕರೂಪಿಣಿ ।\\
ಪಾದ್ಯಂ ಗೃಹಾಣ ದೇವೇಶಿ ಶುಂಭಾಸುರ ನಿಬರ್ಹಿಣಿ ॥\\
\as{ಏತಾವಾನಸ್ಯ+++++ ತ್ರಿಪಾದ+++++ಸ್ಯಾಮೃತಂ ದಿವಿ॥\\
ಅಶ್ವಪೂರ್ವಾಂ +++++++++ಜುಷತಾಮ್॥}ಪಾದ್ಯಮ್ ॥


ಸೂರ್ಯೇಂದುವಹ್ನಿಪೀಠೇ ತ್ವಂ ಬಿಂದು ಚಕ್ರನಿವಾಸಿನಿ ।\\
ಬ್ರಹ್ಮಸ್ವರೂಪಿಣೀ ದೇವಿ ಗೃಹಾಣಾರ್ಘ್ಯಂ ನಮೋಽಸ್ತು ತೇ ॥\\
\as{ತ್ರಿಪಾದೂರ್ದ್ಧ್ವ+++++-ಶನೇ ಅಭಿ ॥\\
ಕಾಂ ಸೋಸ್ಮಿತಾಂ +++++++++ ಶ್ರಿಯಮ್ ॥} ಅರ್ಘ್ಯಮ್ ॥


ನಿಶುಂಭಮಥಿನೀ ದೇವಿ ಶುಂಭಾಸುರ ನಿಬರ್ಹಿಣಿ ।\\
ಜಗತ್ಸರ್ವಂ ತ್ವಯಾ ವ್ಯಾಪ್ತಂ ಗೃಹಾಣಾಚಮನೀಯಕಮ್ ॥ ಆಚಮನೀಯಮ್॥\\
\as{ತಸ್ಮಾತ್ ವಿರಾಳ+++++ಪಶ್ಚಾತ್ ಭೂಮಿಮಥೋ ಪುರಃ ॥\\
ಚನ್ದ್ರಾಂ ಪ್ರಭಾಸಾಂ ++++++++ತ್ವಾಂ ವೃಣೇ ॥} ಆಚಮನಮ್ ॥\\
ಆಚಮನಾಂತೇ ಪತ್ರಪುಷ್ಪಾಣಿ ಸಮರ್ಪಯಾಮಿ ॥

ಅಣಿಮಾದಿಗುಣಾಧಾರೇ ಅಕಾರಾದ್ಯಕ್ಷರಾತ್ಮಿಕೇ ।\\
ಅನಂತರೂಪೇ ದೇವೇಶಿ ಮಧುಪರ್ಕಂ ದದಾಮ್ಯಹಮ್ ॥\\
\as{ಮಧುವಾತಾ+++++ತ್ವೋಷಧೀಃ ॥} ಮಧುಪರ್ಕಃ ॥

ಕರ್ಪೂರೋಶೀರಸುರಭಿ ಶೀತಲಂ ನಿರ್ಮಲಂ ಜಲಂ ।\\
ಗಂಗಾಯಾಸ್ತು ಸಮಾನೀತಂ ಗೃಹಾಣಾಚಮನೀಯಕಮ್ ॥\\
ಮಧುಪರ್ಕಾಂತೇ ಪುನರಾಚಮನಂ ಸಮರ್ಪಯಾಮಿ ॥

ಗಂಗಾದಿ ಸರ್ವತೀರ್ಥೇಭ್ಯ ಆನೀತಂ ತೋಯಮುತ್ತಮಂ ।\\
ಭಕ್ತ್ಯಾ ಸಮರ್ಪಯೇ ತುಭ್ಯಂ ಸ್ನಾನಾರ್ಥಂ ಪ್ರತಿಗೃಹ್ಯತಾಮ್ ॥\\
ಮಲಾಪಕರ್ಷಣ ಸ್ನಾನಮ್॥

(ಕ್ಷೀರಂ ದಧಿ ಘೃತಂ ಚೈವ ಶರ್ಕರಾಮಧುಸಂಯುತಮ್ ।\\
ಪಂಚಾಮೃತಂ ಪ್ರದಾಸ್ಯಾಮಿ ಶಂಕರಪ್ರಿಯವಲ್ಲಭೇ ॥ ಪಂಚಾಮೃತಮ್ ॥)

ಕ್ಷೀರಸ್ನಾನಮ್ ॥\\
ಕಾಮಧೇನುಸಮುದ್ಭೂತಂ ಸರ್ವೇಷಾಂ ಜೀವನಂ ಪರಂ।\\
ಪಾವನಂ ಯಜ್ಞಹೇತುಶ್ಚ ಸ್ನಾನಾರ್ಥಂ ಪ್ರತಿಗೃಹ್ಯತಾಂ ॥ \\
\as{ಆಪ್ಯಾಯಸ್ವ++++ಸಂಗಥೇ ॥\\
ದೇವೀಂ ವಾಚಮ++++ಸುಷ್ಟುತೈತು॥}

ದಧಿಸ್ನಾನಮ್ ॥\\
ಪಯಸಾತು ಸಮುದ್ಭೂತಂ ಮಧುರಾಮ್ಲಶಶಿಪ್ರಭಂ ।\\
ದಧ್ಯಾನೀತಂ ಮಯಾ ದತ್ತಂ ಪ್ರೀತ್ಯಾ ಸ್ವೀಕುರು ಶಂಕರ ॥\\
\as{ದಧಿಕ್ರಾವ್ಣೋ+++++ ತಾರಿಷತ್ ॥\\
ಯದ್ವಾಗ್ವದಂತ್ಯ++++ ಪರಮಂ ಜಗಾಮ ॥}

ಘೃತಸ್ನಾನಮ್ ॥\\
ನವನೀತಸಮುತ್ಪನ್ನಂ ಆಯುರಾರೋಗ್ಯವರ್ಧನಂ ।\\
ಘೃತಂ ತುಭ್ಯಂ ಪ್ರದಾಸ್ಯಾಮಿ ಸ್ನಾನಾರ್ಥಂ ಪ್ರತಿಗೃಹ್ಯತಾಂ ॥\\
\as{ಘೃತಂ ಮಿಮಿಕ್ಷೇ +++ಹವ್ಯಮ್ ॥\\
ಅನಂತಾಮಂತಾದಧಿ ನಿರ್ಮಿತಾಂ ಮಹೀಂ ।\\ ಯಸ್ಯಾಂ ದೇವಾ ಅದ-ಧುರ್ಭೋಜನಾನಿ ।\\ ಏಕಾಕ್ಷರಾಂ ದ್ವಿಪದಾँ ಷಟ್ಪದಾಂ ಚ ।\\ ವಾಚಂ ದೇವಾ ಉಪಜೀವಂತಿ ವಿಶ್ವೇ ।}

ಮಧುಸ್ನಾನಮ್ ॥\\
ತರುಪುಷ್ಪಸಮಾಕೃಷ್ಟಂ ಸುಸ್ವಾದು ಮಧುರಂ ಮಧು ।\\
ತೇಜಃಪುಷ್ಟಿಕರಂ ದಿವ್ಯಂ ಸ್ನಾನಾರ್ಥಂ ಪ್ರತಿಗೃಹ್ಯತಾಂ ॥ \\
\as{ಮಧುವಾತಾ ಋತಾಯತೇ ++++ ಸಂತ್ವೋಷಧೀಃ ॥\\
ವಾಚಂ ದೇವಾ ಉಪಜೀವಂತಿ ವಿಶ್ವೇ ।\\ ವಾಚಂ ಗಂಧರ್ವಾಃ ಪಶವೋ ಮನುಷ್ಯಾಃ ।\\ ವಾಚೀಮಾ ವಿಶ್ವಾ ಭುವನಾನ್ಯರ್ಪಿತಾ ।\\ಸಾ ನೋ ಹವಂ ಜುಷತಾಮಿಂದ್ರ ಪತ್ನೀ ।}

ಶರ್ಕರಾಸ್ನಾನಮ್ ॥\\
ಇಕ್ಷಸಾರಸಮುದ್ಭೂತಾ ಶರ್ಕರಾ ಪುಷ್ಟಿಕಾರಿಕಾ ।\\
ಮಲಾಪಹಾರಿಕಾ ದಿವ್ಯಾ ಸ್ನಾನಾರ್ಥಂ ಪ್ರತಿಗೃಹ್ಯತಾಂ ॥\\
\as{ಸ್ವಾದುಃ ಪವಸ್ವ++++ಅದಾಭ್ಯಃ ॥\\
ವಾಗಕ್ಷರಂ ಪ್ರಥಮಜಾ ಋತಸ್ಯ । \\ವೇದಾನಾಂ ಮಾತಾಮೃತಸ್ಯ ನಾಭಿಃ । \\ಸಾ ನೋ ಜುಷಾಣೋಽಪಯಜ್ಞಮಾಗಾತ್ । \\ಅವಂತೀ ದೇವೀ ಸುಹವಾ ಮೇ ಅಸ್ತು ।}

 ಫಲಸ್ನಾನಮ್॥\\
ಸರ್ವಸಾರಸಮುದ್ಭೂತಂ ಶಕ್ತಿಪುಷ್ಟಿಕರಂ ದೃಢಂ ।\\
ಸುಫಲಂ ಕಾರ್ಯಸಿದ್ಧ್ಯರ್ಥಂ ಸ್ನಾನಾರ್ಥಂ ಪ್ರತಿಗೃಹ್ಯತಾಂ ॥\\
\as{ಯಾಃ ಫಲಿನೀ+++++ತ್ವಂ ಹಸಃ ॥\\
ಯಾಮೃಷಯೋ ಮಂತ್ರಕೃತೋ ಮನೀಷಿಣಃ ।\\ ಅನ್ವೈಚ್ಛಂದೇವಾಸ್ತಪಸಾ ಶ್ರಮೇಣ ।\\ ತಾಂ ದೇವೀಂ ವಾಚँ ಹವಿಷಾ ಯಜಾಮಹೇ ।\\ಸಾ ನೋ ದಧಾತು ಸುಕೃತಸ್ಯ ಲೋಕೇ ।}

ಮಲಯಾಚಲಸಂಭೂತಂ ಸುಗಂಧಂ ಶೀತಲಂ ಶುಭಂ ।\\
ಸುಕಾಂತಿದಾಯಕಂ ದಿವ್ಯಂ ಸ್ನಾನಾರ್ಥಂ ಪ್ರತಿಗೃಹ್ಯತಾಂ ॥\\
\as{ಗಂಧದ್ವಾರಾಂ++++ಶ್ರಿಯಮ್ ॥} ಗಂಧೋದಕಸ್ನಾನಮ್ ॥

ಲತಾವೃಕ್ಷಸಮುತ್ಪನ್ನಂ ಸುಗಂಧಂ ಶೋಭನಂ ಪರಂ ।\\
ಸಂತೋಷವರ್ಧನಂ ನಿತ್ಯಂ ಸ್ನಾನಾರ್ಥಂ ಪ್ರತಿಗೃಹ್ಯತಾಂ ॥ \\
\as{ಆಯನೇತೇ ++++++ ಇಮೇ ॥}ಪುಷ್ಪೋದಕಸ್ನಾನಮ್ ॥

ಅಕ್ಷತೈಶ್ವರ್ಯ ವೃದ್ಧ್ಯರ್ಥಂ ಇಕ್ಷುಚಾಪಧರೇಽಂಬಿಕೇ~।\\
ಅಕ್ಷತೋದಕಧಾರಾಭಿಃ ಸ್ನಪಯಾಮಿ ಪ್ರಸೀದ ಮೇ ॥\\
\as{ಉಪಾಸ್ಮೈ ಗಾಯತಾ++++ಇಯಕ್ಷತೇ ॥} ಅಕ್ಷತೋದಕಸ್ನಾನಮ್ ॥

ಸ್ವರ್ಣಕುಂಭಮುಖೋತ್ಸೃಷ್ಟ ಗಾಂಗವಾರಿಭಿರಾಪ್ಲುತೇ।\\
ಸುವರ್ಣೋದಕಧಾರಾಭಿ ರಭಿಷಿಕ್ತಾ ಪ್ರಸೀದ ಮೇ ॥\\
\as{ಹಿರಣ್ಯರೂಪಃ++++ತ್ಯನ್ನಮಸ್ಮೈ ॥}ಸುವರ್ಣೋದಕಸ್ನಾನಮ್॥

ಹೇಲಾಸೃಷ್ಟಾಂಡಕೋಟಿಸ್ತ್ವಂ ಹರಕಾಂತೇ ಹರಿಸ್ತುತೇ~।\\
ಹರಿದ್ರೋದಕ ಧಾರಾಭಿಃ ಸ್ನಪಯಾಮಿ ಮಹಾಶಿವೇ ॥\\
\as{ಹಾ॒ರಿ॒ದ್ರ॒ವೇವ॑ ಪತಥೋ॒ ವನೇದುಪ॒ಸೋಮಂ᳚ ಸು॒ತಂ ಮ॑ಹಿ॒ಷೇವಾವ॑ ಗಚ್ಛಥಃ ।\\ ಸ॒ಜೋಷ॑ಸಾ ಉ॒ಷಸಾ॒ ಸೂರ್ಯೇ᳚ಣ ಚ॒  ತ್ರಿರ್ವ॒ರ್ತಿರ್ಯಾ᳚ತಮಶ್ವಿನಾ ॥} ಹರಿದ್ರೋದಕಸ್ನಾನಮ್ ॥

ಕುಂದಾಭ ದಂತರುಚಿರೇ ಕಂಜಾಭ ರುಚಿರೇಕ್ಷಣೇ~।\\
ಕುಂಕುಮೋದಕ ಧಾರಾಂ ತೇ ಕಲ್ಪಯಾಮಿ ಶಿವಪ್ರಿಯೇ ॥\\
\as{ಪ್ರಣೋ ದೇವೀ++++++ ಮವಿತ್ರ್ಯವತು ॥}ಕುಂಕುಮೋದಕಸ್ನಾನಮ್ ॥

(ಅಥ \as{ನಮಕ ಚಮಕ ಶ್ರೀಸೂಕ್ತಾ}ದಿಭಿಃ ಶುದ್ಧೋದಕಸ್ನಾನಮ್ ॥)
\newpage
ಜಾಹ್ನವೀ ತೋಯಧಾರಾಭಿಃ  ಕರ್ಪೂರೈಲಾ ಸುವಾಸಿತೈಃ ।\\
ಸ್ನಪಯಾಮಿ ಸುರಶ್ರೇಷ್ಠೇ ಅಭೀಷ್ಟಫಲದಾ ಭವ ॥\\
\as{ಯತ್ಪುರುಷೇಣ+++++ ಇದ್ಧ್ಮಃ ಶರದ್ಧವಿಃ॥\\
ಆದಿತ್ಯವರ್ಣೇ +++++++++ಬಾಹ್ಯಾ ಅಲಕ್ಷ್ಮೀಃ ॥}ಶುದ್ಧೋದಕಸ್ನಾನಂ॥\\
ಸ್ನಾನಾನಂತರಂ ಆಚಮನೀಯಂ ಸುಗಂಧದ್ರವ್ಯಂ ಚ ಸಮರ್ಪಯಾಮಿ ॥

ವಸ್ತ್ರಂ ಗೃಹಾಣ ದೇವೇಶಿ ದೇವಾಂಗಸದೃಶಂ ನವಂ ।\\
ವಿಶ್ವೇಶ್ವರಿ ಮಹಾಮಾಯೇ ನಾರಾಯಣಿ ನಮೋಽಸ್ತು ತೇ ॥\\
\as{ತಂ ಯಜ್ಞಂ +++++ ಋಷಯಶ್ಚ ಯೇ॥\\
ಉಪೈತು ಮಾಂ ++++++++ ದದಾತು ಮೇ ॥\\
ಯುವಂ ವಸ್ತ್ರಾಣಿ ++++ಸಚೇಥೇ ॥}ವಸ್ತ್ರಯುಗ್ಮಮ್ ॥

ತಕ್ಷಕಾನಂತ ಕರ್ಕೋಟ ನಾಗಯಜ್ಞೋಪವೀತಿನಿ ।\\
ಸುವರ್ಣ ಯಜ್ಞಸೂತ್ರಂ ಚ ದದಾಮಿ ಹರಿಸೇವಿತೇ ॥\\
\as{ತಸ್ಮಾದ್ಯಜ್ಞಾ+++++ಗ್ರಾಮ್ಯಾಶ್ಚ ಯೇ॥\\
ಕ್ಷುತ್ಪಿಪಾಸಾಮ್+++++++++++ ಗೃಹಾತ್॥\\
ಯಜ್ಞೋಪವೀತಮ್ ++++ತೇಜಃ ॥} ಉಪವೀತಮ್ ॥

ವಜ್ರವೈಡೂರ್ಯ ಮಾಣಿಕ್ಯ ರಚಿತಂ ಹೇಮ ಕಂಚುಕಮ್ ।\\
ಗೃಹಾಣ ತ್ವಂ ಮಹಾದೇವಿ ಜಗನ್ಮಾತರ್ನಮೋಽಸ್ತು ತೇ ॥ಕಂಚುಕಾಭರಣಮ್॥

ಮಾಂಗಲ್ಯಂ ಮಣಿಸಂಯುಕ್ತಂ ಮುಕ್ತಾಫಲ ಸುಶೋಭಿತಮ್ ।\\
ಕಂಠಾಲಂಕರಣಾರ್ಥಂ ಚ ಗೃಹಾಣ ಕಂಬುಕಂಧರೇ ॥ಮಾಂಗಲ್ಯಮ್॥

ತಾಟಂಕಯುಗಲಂ ಶ್ರೇಷ್ಠಂ ತಪನೋಡುಪ ಮಂಡಲಮ್ ।\\
ಅರ್ಪಯಾಮಿ ತವ ಪ್ರೀತ್ಯೈ ಸರ್ವಸೌಭಾಗ್ಯದಾಯಕಮ್ ॥\\ತಾಟಂಕಯುಗಲಂ॥

ಮೂಲ ಭೂತಾತ್ಮಿಕೇ ದೇವಿ ಮೂಲಾಧಾರ ನಿವಾಸಿನಿ ।\\
ಗೃಹಾಣಾಭರಣಂ ದೇವಿ ಸರ್ವಾಲಂಕಾರಭೂಷಿತೇ ॥\\
\as{ಹಿರಣ್ಯರೂಪಃ+++++ತ್ಯನ್ನಮಸ್ಮೈ ॥ }ಆಭರಣಮ್॥

ಸರ್ವೇಶ್ವರಿ ಶಿವಾರಾಧ್ಯೇ ಶಿವಾನಂದೇ ಶಿವಾತ್ಮಕೇ ।\\
ಕರ್ಪೂರಕುಂಕುಮೈರ್ಯುಕ್ತಂ ಗೃಹಾಣ ಗಂಧಮುತ್ತಮಮ್ ॥\\
\as{ತಸ್ಮಾದ್ಯಜ್ಞಾತ್+++++ ಸ್ಮಾದಜಾಯತ॥\\
ಗಂಧದ್ವಾರಾಂ+++++++++ ಶ್ರಿಯಮ್ ॥\\
ಗಂಧದ್ವಾರಾಂ+++++++++ ಶ್ರಿಯಮ್ ॥}ಗಂಧಃ ॥

ಸೌಭಾಗ್ಯ ಶುಭದೇ ದೇವಿ ಸರ್ವಮಂಗಲದಾಯಿನಿ ।\\
ಗಂಡಾಲಂಕರಣಾರ್ಥಾಯ ಹರಿದ್ರಾಂ ಪ್ರದದಾಮ್ಯಹಮ್ ॥ಹರಿದ್ರಾ॥

ಕುಂಕುಂಮಂ ಸರ್ವಸೌಭಾಗ್ಯ ಸೂಚಕಂ ಫಾಲಭೂಷಣಮ್~।\\
ಸ್ವೀಕುರುಷ್ವ ಜಗನ್ಮಾತಃ ಜಪಾಕುಸುಮ ಭಾಸ್ವರಮ್ ॥ಕುಂಕುಮಮ್॥

ಕಜ್ಜಲಂ ನವನೀತಾಢ್ಯಂ ಗೃಹಾಣ ಸುವಿಲೋಚನೇ ।\\
ಅನಂಗಜನಕಾಪಾಂಗವೀಕ್ಷಣೇ ಶಿವ ವಲ್ಲಭೇ ॥ಕಜ್ಜಲಮ್॥

ವಿದ್ಯುತ್ಕೃಶಾನು ಸಂಕಾಶಂ ಜಪಾಕುಸುಮ ಸನ್ನಿಭಮ್ ।\\
ಸಿಂದೂರಂ ತೇ ಪ್ರದಾಸ್ಯಾಮಿ ಗೃಹಾಣ ಜಗದೀಶ್ವರಿ ॥ಸಿಂದೂರಮ್॥

ಶಾಂಭವಾನಂದಜನನಿ ಪರಮಾನಂದ ರೂಪಿಣಿ ।\\
ಅಕ್ಷತಾಂಶ್ಚ ಮಯಾನೀತಾನ್ ಗೃಹಾಣ ಪರಮೇಶ್ವರಿ ॥\\
\as{ಅರ್ಚತಪ್ರಾರ್ಚತ+++++ ಧೃಷ್ಣ್ವರದಚತ ॥}ಅಕ್ಷತಾಃ ॥
\newpage
ಕಸ್ತೂರೀ ಕೇಸರಂ ಚೈಲಾ ಲವಂಗಶ್ಚಂದನಾದಿಕಂ ।\\
ನಾನಾಪರಿಮಲದ್ರವ್ಯಂ ಗೃಹಾಣ ಹರವಲ್ಲಭೇ ॥ಸುಗಂಧದ್ರವ್ಯಮ್॥

ಕರವೀರಜಾತೀಕುಸುಮೈಶ್ಚಂಪಕೈರ್ಬಕುಲೈಃ ಶುಭೈಃ ।\\
ಶತಪತ್ರೈಶ್ಚ ಕಹ್ಲಾರೈರರ್ಚಯೇ ಪರಮೇಶ್ವರಿ ॥\\
\as{ಮನಸಃ +++++++++++ ಯಶಃ ॥\\
ಆಯನೇತೇ++++++ಗೃಹಾ ಇಮೇ ॥}ಪುಷ್ಪಾಣಿ ॥

ಪುಷ್ಪೈರ್ನಾನಾವಿಧೈರ್ದಿವ್ಯೈಃ ವೇಣೀಶೋಭಾವಿವೃದ್ಧಯೇ ।\\
ಶ್ರದ್ಧಯಾ ನಿರ್ಮಿತಾ ದೇವಿ ಮಾಲೇಯಂ ಪ್ರತಿಗೃಹ್ಯತಾಮ್ ॥ಪುಷ್ಪಮಾಲಾ॥ 

\subsection{ಅಂಗಪೂಜಾ}
ವಾರಾಹ್ಯೈ ನಮಃ । ಪಾದೌ  ಪೂಜಯಾಮಿ ॥\\
ನಾರಸಿಂಹ್ಯೈ ನಮಃ । ಗುಲ್ಫೌ  ಪೂಜಯಾಮಿ ॥\\
ರಕ್ತಬೀಜನಿಪಾತಿನ್ಯೈ ನಮಃ । ಜಂಘೇ  ಪೂಜಯಾಮಿ ॥\\
ಬ್ರಾಹ್ಮ್ಯೈ ನಮಃ । ಜಾನುನೀ  ಪೂಜಯಾಮಿ ॥\\
ಶಿವದೂತ್ಯೈ ನಮಃ । ಊರೂ  ಪೂಜಯಾಮಿ ॥\\
ಕೌಮಾರ್ಯೈ ನಮಃ । ಜಘನದ್ವಯಂ  ಪೂಜಯಾಮಿ ॥\\
ಕಾಲ್ಯೈ ನಮಃ । ಕಟಿಂ  ಪೂಜಯಾಮಿ ॥\\
ಗೌರ್ಯೈ ನಮಃ । ಗುಹ್ಯಂ  ಪೂಜಯಾಮಿ ॥\\
ಭವಾನ್ಯೈ ನಮಃ । ನಾಭಿಂ  ಪೂಜಯಾಮಿ ॥\\
ಶರ್ವಾಣ್ಯೈ ನಮಃ । ಉದರಂ  ಪೂಜಯಾಮಿ ॥\\
ಹೈಮವತ್ಯೈ ನಮಃ । ಹೃದಯಂ  ಪೂಜಯಾಮಿ ॥\\
ಮನ್ಮಥವಾಸಿನ್ಯೈ ನಮಃ । ಸ್ತನದ್ವಯಂ  ಪೂಜಯಾಮಿ ॥\\
ಚಂಡಿಕಾಯೈ ನಮಃ । ಕಕ್ಷೌ  ಪೂಜಯಾಮಿ ॥\\
ಶಾಕಂಭರ್ಯೈ ನಮಃ । ಕಂಠಂ  ಪೂಜಯಾಮಿ ॥\\
ಕಾಲ್ಯೈ ನಮಃ । ಸ್ಕಂಧೌ  ಪೂಜಯಾಮಿ ॥\\
ಸರ್ವಾಸ್ತ್ರಧಾರಿಣ್ಯೈ ನಮಃ । ಹಸ್ತಾನ್  ಪೂಜಯಾಮಿ ॥\\
ಸರ್ವಮಂಗಲಾಯೈ ನಮಃ । ಮುಖಂ  ಪೂಜಯಾಮಿ ॥\\
ವೇದಸ್ವರೂಪಣ್ಯೈ ನಮಃ । ನಾಸಿಕಾಂ  ಪೂಜಯಾಮಿ ॥\\
ರಕ್ತದಂತಿಕಾಯೈ ನಮಃ । ದಂತಾನ್  ಪೂಜಯಾಮಿ ॥\\
ಬಿಂದುನಾದಸ್ವರೂಪಿಣ್ಯೈ ನಮಃ । ವಕ್ತ್ರಂ  ಪೂಜಯಾಮಿ ॥\\
ಅರವಿಂದಲೋಚನಾಯೈ ನಮಃ । ನೇತ್ರತ್ರಯಂ  ಪೂಜಯಾಮಿ ॥\\
ಕಾರುಣ್ಯಮೂರ್ತ್ಯೈ ನಮಃ । ಕರ್ಣೌ  ಪೂಜಯಾಮಿ ॥\\
ಚಂದ್ರಚೂಡಾಯೈ ನಮಃ । ಲಲಾಟಂ  ಪೂಜಯಾಮಿ ॥\\
ಸರ್ವೇಶ್ವರ್ಯೈ ನಮಃ । ಶಿರಃ  ಪೂಜಯಾಮಿ ॥\\
ಸರ್ವಾಭೀಷ್ಟಪ್ರಾದಾಯಿನ್ಯೈ ನಮಃ । ಸರ್ವಾಂಗಾನಿ ಪೂಜಯಾಮಿ ॥
\subsection{ಅಥ ಪತ್ರಪೂಜಾ}
ಕಾಮಾಕ್ಷ್ಯೈ ನಮಃ । ಧತ್ತೂರ ಪತ್ರಂ ಸಮರ್ಪಯಾಮಿ ॥\\
ಮಂಗಲಾಯೈ ನಮಃ । ಬಿಲ್ವ ಪತ್ರಂ ಸಮರ್ಪಯಾಮಿ ॥\\
ಮಾಧವ್ಯೈ ನಮಃ । ಮರುಗ ಪತ್ರಂ ಸಮರ್ಪಯಾಮಿ ॥\\
ಗಿರಿಜಾಯೈ ನಮಃ । ಅಪಾಮಾರ್ಗ ಪತ್ರಂ ಸಮರ್ಪಯಾಮಿ ॥\\
ನಿರ್ಗುಣಾಯೈ ನಮಃ । ನಿರ್ಗುಂಡೀ ಪತ್ರಂ ಸಮರ್ಪಯಾಮಿ ॥\\
ನಾಗಹಾರಾಯೈ ನಮಃ । ವೇಣು ಪತ್ರಂ ಸಮರ್ಪಯಾಮಿ ॥\\
ವಿಷ್ಣುಪ್ರಿಯಾಯೈ ನಮಃ । ವಿಷ್ಣುಕ್ರಾಂತಿ ಪತ್ರಂ ಸಮರ್ಪಯಾಮಿ ॥\\
ಸರ್ವದಾಯೈ ನಮಃ । ಶತಪತ್ರ ಪತ್ರಂ ಸಮರ್ಪಯಾಮಿ ॥\\
ಯಶಸ್ವಿನ್ಯೈ ನಮಃ । ದೂರ್ವಾ ಪತ್ರಂ ಸಮರ್ಪಯಾಮಿ ॥\\
ಸರ್ವಮಂತ್ರಾತ್ಮಿಕಾಯೈ ನಮಃ । ಸಮಸ್ತಪತ್ರಾಣಿ  ಸಮರ್ಪಯಾಮಿ ॥
\subsection{ಪುಷ್ಪಪೂಜಾ}
ಉಮಾಯೈ ನಮಃ । ಪುನ್ನಾಗ ಪುಷ್ಪಂ ಸಮರ್ಪಯಾಮಿ ॥\\
ಕಾತ್ಯಾಯನ್ಯೈ ನಮಃ । ಚಂಪಕ ಪುಷ್ಪಂ ಸಮರ್ಪಯಾಮಿ ॥\\
ಗೌರ್ಯೈ ನಮಃ । ಜಾತೀ ಪುಷ್ಪಂ ಸಮರ್ಪಯಾಮಿ ॥\\
ಕಾಲ್ಯೈ ನಮಃ । ಕೇತಕೀ ಪುಷ್ಪಂ ಸಮರ್ಪಯಾಮಿ ॥\\
ಹೈಮವತ್ಯೈ ನಮಃ । ಕರವೀರ ಪುಷ್ಪಂ ಸಮರ್ಪಯಾಮಿ ॥\\
ಈಶ್ವರ್ಯೈ ನಮಃ । ಉತ್ಪಲ ಪುಷ್ಪಂ ಸಮರ್ಪಯಾಮಿ ॥\\
ಭವಾನ್ಯೈ ನಮಃ । ಮಲ್ಲಿಕಾ ಪುಷ್ಪಂ ಸಮರ್ಪಯಾಮಿ ॥\\
ರುದ್ರಾಣ್ಯೈ ನಮಃ । ಯೂಥಿಕಾ ಪುಷ್ಪಂ ಸಮರ್ಪಯಾಮಿ ॥\\
ಲೋಕಮಾತ್ರೇ ನಮಃ । ಕಮಲ ಪುಷ್ಪಂ ಸಮರ್ಪಯಾಮಿ ॥\\
ಸರ್ವಮಂಗಲಾಯೈ ನಮಃ । ಸಮಸ್ತಪುಷ್ಪಾಣಿ ಸಮರ್ಪಯಾಮಿ ॥
\section{ಆವರಣಪೂಜಾ}
\subsection{ಪ್ರಥಮಾವರಣಮ್}
ನಂದಾಯೈ  ನಮಃ ।
ರಕ್ತದಂತಿಕಾಯೈ  ನಮಃ ।
ಶಾಕಂಭರ್ಯೈ  ನಮಃ ।
ಶಿವದೂತ್ಯೈ  ನಮಃ ।
ಭೀಮಾಯೈ  ನಮಃ ।
ಭ್ರಾಮರ್ಯೈ  ನಮಃ ।
ಭವಾನ್ಯೈ  ನಮಃ ।
ಶರ್ವಾಣ್ಯೈ  ನಮಃ ।
ಹೈಮವತ್ಯೈ  ನಮಃ ।
ಚಂಡಿಕಾಯೈ  ನಮಃ ।
ಅಂಬಿಕಾಯೈ  ನಮಃ ।
ಸರ್ವೇಶ್ವರ್ಯೈ  ನಮಃ ।\\
\as{ಚಂಡಿಕೇಶ್ವರಿ ಚಾಮುಂಡೇ ದುರ್ಗೇ ದುರ್ಗತಿನಾಶಿನಿ ।\\
ಭಕ್ತ್ತ್ಯಾ ಸಮರ್ಪಯೇ ತುಭ್ಯಂ  ಪ್ರಥಮಾವರಣಾರ್ಚನಮ್ ॥}
\newpage
\subsection{ದ್ವಿತೀಯಾವರಣಮ್}
ಓಂ ಲಂ ಇಂದ್ರಾಯ ನಮಃ । ಓಂ ರಂ ಅಗ್ನಯೇ ನಮಃ । ಓಂ ಮಂ ಯಮಾಯ ನಮಃ । ಓಂ ಕ್ಷಂ ನಿರ್ಋತಯೇ ನಮಃ ।ಓಂ ವಂ ವರುಣಾಯ ನಮಃ ।ಓಂ ಯಂ ವಾಯವೇ ನಮಃ । ಓಂ ಕುಂ ಕುಬೇರಾಯ ನಮಃ । ಓಂ ಹಂ ಈಶಾನಾಯ ನಮಃ । ಓಂ ಆಂ ಬ್ರಹ್ಮಣೇ ನಮಃ । ಓಂ ಹ್ರೀಂ ಅನಂತಾಯ ನಮಃ । ಓಂ ನಿಯತ್ಯೈ ನಮಃ । ಓಂ ಕಾಲಾಯ ॥\\
ಓಂ ಬ್ರಾಹ್ಮ್ಯೈ ನಮಃ । ಓಂ ಮಾಹೇಶ್ವರ್ಯೈ ನಮಃ । ಓಂ ಕೌಮಾರ್ಯೈ ನಮಃ । ಓಂ ವೈಷ್ಣವ್ಯೈ ನಮಃ । ಓಂ ವಾರಾಹ್ಯೈ ನಮಃ । ಓಂ ಮಾಹೇಂದ್ರ್ಯೈ ನಮಃ । ಓಂ ಚಾಮುಂಡಾಯೈ ನಮಃ । ಓಂ ಮಹಾಲಕ್ಷ್ಮ್ಯೈ ನಮಃ ॥\\
\as{ಚಂಡಿಕೇಶ್ವರಿ ಚಾಮುಂಡೇ ದುರ್ಗೇ ದುರ್ಗತಿನಾಶಿನಿ ।\\
ಭಕ್ತ್ತ್ಯಾ ಸಮರ್ಪಯೇ ತುಭ್ಯಂ ದ್ವಿತೀಯಾವರಣಾರ್ಚನಮ್ ॥}
\subsection{ತೃತೀಯಾವರಣಮ್}
ಸೂರ್ಯಾಯ  ನಮಃ ।
ಚಂದ್ರಾಯ  ನಮಃ ।
ಅಂಗಾರಕಾಯ  ನಮಃ ।
ಬುಧಾಯ  ನಮಃ ।
ಗುರವೇ  ನಮಃ ।
ಶುಕ್ರಾಯ  ನಮಃ ।
ಶನೈಶ್ಚರಾಯ  ನಮಃ ।
ರಾಹವೇ  ನಮಃ ।
ಕೇತವೇ  ನಮಃ ।
ತ್ರಿಶೂಲಾಯ  ನಮಃ ।
ಪರಶವೇ  ನಮಃ ।
ಶಂಖಾಯ  ನಮಃ ।
ಘಂಟಾಯ  ನಮಃ ।
ಪಾಶಾಯ  ನಮಃ ।
ಶಕ್ತಯೇ  ನಮಃ ।
ದಂಡಾಯ  ನಮಃ ।
ಪಾನಪಾತ್ರಾಯ  ನಮಃ ।\\
\as{ಚಂಡಿಕೇಶ್ವರಿ ಚಾಮುಂಡೇ ದುರ್ಗೇ ದುರ್ಗತಿನಾಶಿನಿ ।\\
ಭಕ್ತ್ತ್ಯಾ ಸಮರ್ಪಯೇ ತುಭ್ಯಂ ತೃತೀಯಾವರಣಾರ್ಚನಮ್ ॥}
\subsection{ಚತುರ್ಥಾವರಣಮ್}
ಅಕ್ಷಮಾಲಾಯೈ  ನಮಃ ।
ಕಮಲಾಯ  ನಮಃ ।
ಬಾಣಾಯ  ನಮಃ ।
ಅಶನಯೇ  ನಮಃ ।
ಕುಲಿಶಾಯ  ನಮಃ ।
ಗದಾಯೈ  ನಮಃ ।
ಚಕ್ರಾಯ  ನಮಃ ।
ತ್ರಿಶೂಲಾಯ  ನಮಃ ।
ಪದ್ಮಾಯ  ನಮಃ ।
ಕಾಲ್ಯೈ  ನಮಃ ।
ಕಾಲರಾತ್ರ್ಯೈ  ನಮಃ ।
ಭುವನೇಶ್ವರ್ಯ್ಯೈ  ನಮಃ ।
ಘೋರರೂಪಿಣ್ಯೈ  ನಮಃ ।
ಕಾತ್ಯಾಯನ್ಯೈ  ನಮಃ ।
ಕೌಮಾರ್ಯೈ  ನಮಃ ।\\
\as{ಚಂಡಿಕೇಶ್ವರಿ ಚಾಮುಂಡೇ ದುರ್ಗೇ ದುರ್ಗತಿನಾಶಿನಿ ।\\
ಭಕ್ತ್ತ್ಯಾ ಸಮರ್ಪಯೇ ತುಭ್ಯಂ ತುರೀಯಾವರಣಾರ್ಚನಮ್ ॥}
\subsection{ಪಂಚಮಾವರಣಮ್}
ಮಾಧವಾಯ  ನಮಃ ।
ಶುಕ್ರಾಯ  ನಮಃ ।
ಶುಚಯೇ  ನಮಃ ।
ನಭಸೇ  ನಮಃ ।
ನಭಸ್ಯಾಯ  ನಮಃ ।
ಊರ್ಜಾಯ  ನಮಃ ।
ಸಹಸೇ  ನಮಃ ।
ಸಹಸ್ಯಾಯ  ನಮಃ ।
ತಪಸೇ  ನಮಃ ।
ಮಂಗಳಾಯ  ನಮಃ ।\\
\as{ಚಂಡಿಕೇಶ್ವರಿ ಚಾಮುಂಡೇ ದುರ್ಗೇ ದುರ್ಗತಿನಾಶಿನಿ ।\\
ಭಕ್ತ್ತ್ಯಾ ಸಮರ್ಪಯೇ ತುಭ್ಯಂ  ಪಂಚಮಾವರಣಾರ್ಚನಮ್ ॥}
\subsection{ಷಷ್ಠಾವರಣಮ್}
ಮೇಷಾಯ  ನಮಃ ।
ವೃಷಭಾಯ  ನಮಃ ।
ಮಿಥುನಾಯ  ನಮಃ ।
ಕಟಕಾಯ  ನಮಃ ।
ಸಿಂಹಾಯ  ನಮಃ ।
ಕನ್ಯಾಯೈ  ನಮಃ ।
ತುಲಾಯೈ  ನಮಃ ।
ವೃಶ್ಚಿಕಾಯ  ನಮಃ ।
ಧನುಷೇ  ನಮಃ ।
ಮಕರಾಯ  ನಮಃ ।
ಕುಂಭಾಯ  ನಮಃ ।
ಮೀನಾಯ  ನಮಃ ।\\
\as{ಚಂಡಿಕೇಶ್ವರಿ ಚಾಮುಂಡೇ ದುರ್ಗೇ ದುರ್ಗತಿನಾಶಿನಿ ।\\
ಭಕ್ತ್ತ್ಯಾ ಸಮರ್ಪಯೇ ತುಭ್ಯಂ ಷಷ್ಠಾಖ್ಯಾವರಣಾರ್ಚನಮ್ ॥}
\subsection{ಸಪ್ತಮಾವರಣಮ್}
ಓಂ ಅಸಿತಾಂಗಭೈರವಾಯ ನಮಃ । ಓಂ ರುರುಭೈರವಾಯ ನಮಃ । ಓಂ ಚಂಡಭೈರವಾಯ ನಮಃ । ಓಂ ಕ್ರೋಧಭೈರವಾಯ ನಮಃ । ಓಂ ಉನ್ಮತ್ತಭೈರವಾಯ ನಮಃ । ಓಂ ಕಪಾಲಭೈರವಾಯ ನಮಃ । ಓಂ ಭೀಷಣಭೈರವಾಯ ನಮಃ । ಓಂ ಸಂಹಾರಭೈರವಾಯ ನಮಃ ॥\\%। ಓಂ ವರಪ್ರದಭೈರವಾಯ ನಮಃ
\as{ಚಂಡಿಕೇಶ್ವರಿ ಚಾಮುಂಡೇ ದುರ್ಗೇ ದುರ್ಗತಿನಾಶಿನಿ ।\\
ಭಕ್ತ್ತ್ಯಾ ಸಮರ್ಪಯೇ ತುಭ್ಯಂ ಸಪ್ತಮಾವರಣಾರ್ಚನಮ್ ॥}
\subsection{ಅಷ್ಟಮಾವರಣಮ್}
ಮಾಲಿನ್ಯೈ  ನಮಃ ।
ಕಪಾಲಿನ್ಯೈ  ನಮಃ ।
ವರದಾಯೈ  ನಮಃ ।
ಭೋಗದಾಯೈ  ನಮಃ ।
ರಾಜಲಕ್ಷ್ಮ್ಯೈ  ನಮಃ ।
ವೀರಲಕ್ಷ್ಮ್ಯೈ  ನಮಃ ।
ಭೋಗಲಕ್ಷ್ಮ್ಯೈ  ನಮಃ ।
ಆನಂದಲಕ್ಷ್ಮ್ಯೈ  ನಮಃ ।
ಗಜಲಕ್ಷ್ಮ್ಯೈ  ನಮಃ ।
ಗೃಹಲಕ್ಷ್ಮ್ಯೈ  ನಮಃ ।
ಮೋಕ್ಷಲಕ್ಷ್ಮ್ಯೈ  ನಮಃ ।
ಜಯಲಕ್ಷ್ಮ್ಯೈ  ನಮಃ ।\\
\as{ಚಂಡಿಕೇಶ್ವರಿ ಚಾಮುಂಡೇ ದುರ್ಗೇ ದುರ್ಗತಿನಾಶಿನಿ ।\\
ಭಕ್ತ್ತ್ಯಾ ಸಮರ್ಪಯೇ ತುಭ್ಯಂ ಅಷ್ಟಮಾವರಣಾರ್ಚನಮ್ ॥}
\subsection{ನವಮಾವರಣಮ್}
ಮಯೂರವಾಹಿನ್ಯೈ  ನಮಃ ।
ವೃಷವಾಹಿನ್ಯೈ  ನಮಃ ।
ಹಂಸವಾಹಿನ್ಯೈ  ನಮಃ ।
ಗರುಡವಾಹಿನ್ಯೈ  ನಮಃ ।
ಮೃಗವಾಹಿನ್ಯೈ  ನಮಃ ।
ಸಿಂಹವಾಹಿನ್ಯೈ  ನಮಃ ।
ಅಶ್ವವಾಹಿನ್ಯೈ  ನಮಃ ।
ಗಜವಾಹಿನ್ಯೈ  ನಮಃ ।
ಮಕರವಾಹಿನ್ಯೈ  ನಮಃ ।\\
\as{ಚಂಡಿಕೇಶ್ವರಿ ಚಾಮುಂಡೇ ದುರ್ಗೇ ದುರ್ಗತಿನಾಶಿನಿ ।\\
ಭಕ್ತ್ತ್ಯಾ ಸಮರ್ಪಯೇ ತುಭ್ಯಂ ನವಮಾವರಣಾರ್ಚನಮ್ ॥}
\section{ಪತ್ರಪೂಜಾ}
\as{ಚಂಪಕಾಶೋಕ ಪುನ್ನಾಗ ಬೃಹತೀ ಕರವೀರಕೈಃ।\\
ಕುಂದಕಹ್ಲಾರಪತ್ರೈಸ್ತ್ವಾಂ ಪೂಜಯಾಮಿ ಮಹೇಶ್ವರಿ ॥}\\
ಕಾಮಾಕ್ಷ್ಯೈ ನಮಃ । ಚಂಪಕಪತ್ರಂ ಸಮರ್ಪಯಾಮಿ ।\\
ಸರ್ವದಾಯೈ ನಮಃ । ಅಶೋಕಪತ್ರಂ ಸಮರ್ಪಯಾಮಿ ।\\
ಮಾಧವ್ಯೈ ನಮಃ । ಪುನ್ನಾಗಪತ್ರಂ ಸಮರ್ಪಯಾಮಿ ।\\
ಮಂಗಳಾತ್ಮಿಕಾಯೈ ನಮಃ । ಕುಂದಪತ್ರಂ ಸಮರ್ಪಯಾಮಿ ।\\
ಶಿವಪ್ರಿಯಾಯೈ ನಮಃ । ಕಹ್ಲಾರಪತ್ರಂ ಸಮರ್ಪಯಾಮಿ ।\\
ಸರ್ವೇಶ್ವರ್ಯೈ ನಮಃ । ಮರುಗಪತ್ರಂ ಸಮರ್ಪಯಾಮಿ ।\\
ಶಿವದೂತ್ಯೈ ನಮಃ । ಮಲ್ಲಿಕಾಪತ್ರಂ ಸಮರ್ಪಯಾಮಿ ।\\
ಕಾಳ್ಯೈ ನಮಃ । ಕೇತಕೀಪತ್ರಂ ಸಮರ್ಪಯಾಮಿ ।\\
ಲಕ್ಷ್ಮ್ಯೈ ನಮಃ । ಕಮಲಪತ್ರಂ ಸಮರ್ಪಯಾಮಿ ।\\
ಸರಸ್ವತ್ಯೈ ನಮಃ । ಸೇವಂತಿಕಾಪತ್ರಂ ಸಮರ್ಪಯಾಮಿ ।\\
ಚಾಮುಂಡಾಯೈ ನಮಃ । ಸರ್ವಪತ್ರಂ ಸಮರ್ಪಯಾಮಿ ।

\section{ಬಿಲ್ವಪತ್ರಪೂಜಾ}
\as{ಅಮೃತೋದ್ಭವಂ ಚ ಶ್ರೀವೃಕ್ಷಂ ಮಹಾದೇವೀ ಪ್ರಿಯಂ ಸದಾ ।\\
ಬಿಲ್ವಪತ್ರಂ ಪ್ರಯಚ್ಛಾಮಿ ಪವಿತ್ರಂ ತೇ ಸುರೇಶ್ವರಿ ॥}\\
ಸತ್ಯಾಯೈ ನಮಃ । ಬಿಲ್ವಪತ್ರಂ ಸಮರ್ಪಯಾಮಿ ॥\\
ಸುರೇಶ್ವರ್ಯೈ ನಮಃ । ಬಿಲ್ವಪತ್ರಂ ಸಮರ್ಪಯಾಮಿ ॥\\
ಮಾಯಾಯೈ ನಮಃ । ಬಿಲ್ವಪತ್ರಂ ಸಮರ್ಪಯಾಮಿ ॥\\
ಅಪರ್ಣಾಯೈ ನಮಃ । ಬಿಲ್ವಪತ್ರಂ ಸಮರ್ಪಯಾಮಿ ॥\\
ಚಂಡಿಕಾಯೈ ನಮಃ । ಬಿಲ್ವಪತ್ರಂ ಸಮರ್ಪಯಾಮಿ ॥

\section{ಪಲ್ಲವಪೂಜಾ}
\as{ಶಾಂತಾಗೀಂ ಶ್ಯಾಮಲಾಂ ಸರ್ವಾಂ ದುಷ್ಟಾಸುರನಿಬರ್ಹಿಣೀಮ್ ।\\
ಪೂಜಾಂ ಕರೋಮಿ ಚಾರ್ವಂಗೀಂ ಪಲ್ಲವೈರ್ನಂದನೋದ್ಭವೈಃ ॥\\}
ಮನೋರಮಾಯೈ ನಮಃ । ಚೂತಪಲ್ಲವಂ ಸಮರ್ಪಯಾಮಿ ।\\
ಲೋಕಮಾತ್ರೇ ನಮಃ । ಅಶೋಕಪಲ್ಲವಂ ಸಮರ್ಪಯಾಮಿ ।\\
ಜಗಜ್ಜನನ್ಯೈ ನಮಃ । ನಂದ್ಯಾವರ್ತಪಲ್ಲವಂ ಸಮರ್ಪಯಾಮಿ ।\\
ಸರ್ವೇಶ್ವರ್ಯೈ ನಮಃ । ಪಾರಿಜಾತಪಲ್ಲವಂ ಸಮರ್ಪಯಾಮಿ ।\\
ದುರ್ಗಾಯೈ ನಮಃ । ನವಮಲ್ಲಿಕಾಪಲ್ಲವಂ ಸಮರ್ಪಯಾಮಿ ।
\section{ದೂರ್ವಾಂಕುರಪೂಜಾ}
\as{ಮಂತ್ರಾಕ್ಷರಮಯೀಂ ಲಕ್ಷ್ಮೀಂ ಸಾಕ್ಷಾತ್ ಸಂಪತ್ಪ್ರದಾಯಿನೀಮ್ ।\\
ದೂರ್ವಾಂಕುರಾನ್  ಪ್ರದಾಸ್ಯಾಮಿ ಅಷ್ಟಗಂಧೇನ ಸಂಯುತಾನ್ ॥}\\
ಚಾಮುಂಡಾಯೈ ನಮಃ । ದೂರ್ವಾಂಕುರಂ ಸಮರ್ಪಯಾಮಿ ।\\
ದುರ್ಗಾಯೈ ನಮಃ । ದೂರ್ವಾಂಕುರಂ ಸಮರ್ಪಯಾಮಿ ।\\
ಕಾಲ್ಯೈ ನಮಃ । ದೂರ್ವಾಂಕುರಂ ಸಮರ್ಪಯಾಮಿ ।\\
ಲಕ್ಷ್ಮ್ಯೈ ನಮಃ । ದೂರ್ವಾಂಕುರಂ ಸಮರ್ಪಯಾಮಿ ।\\
ಸರಸ್ವತ್ಯೈ ನಮಃ । ದೂರ್ವಾಂಕುರಂ ಸಮರ್ಪಯಾಮಿ ।\\
ಲಲಿತಾಯೈ ನಮಃ । ದೂರ್ವಾಂಕುರಂ ಸಮರ್ಪಯಾಮಿ ।\\
ಅನಂತಾಯೈ ನಮಃ । ದೂರ್ವಾಂಕುರಂ ಸಮರ್ಪಯಾಮಿ ।\\
ಅನ್ನಪೂರ್ಣಾಯೈ ನಮಃ । ದೂರ್ವಾಂಕುರಂ ಸಮರ್ಪಯಾಮಿ ।\\
ಅನಂತಶಕ್ತ್ಯೈ ನಮಃ । ದೂರ್ವಾಂಕುರಂ ಸಮರ್ಪಯಾಮಿ ।\\
ಆರ್ಯಾಯೈ ನಮಃ । ದೂರ್ವಾಂಕುರಂ ಸಮರ್ಪಯಾಮಿ ।\\
ಅಕಾರಾದಿಕ್ಷಕಾರಾಂತ ಮಾತೃಕಾವರ್ಣರೂಪಿಣ್ಯೈ ನಮಃ ।\\ ದೂರ್ವಾಂಕುರಂ ಸಮರ್ಪಯಾಮಿ ।
\newpage
\section{ಪುಷ್ಪಪೂಜಾ}
\as{ಉತ್ಫುಲ್ಲಮಲ್ಲಿಕಾ ಜಾತೀ ಕೇತಕೀ ಕರವೀರಕೈಃ ।\\
ಕಹ್ಲಾರ ಚಂಪಕಾಪುಷ್ಪೈಃ ಪೂಜಯಾಮ್ಯದ್ಯ ಚಂಡಿಕೇ ॥}\\
ಕಾತ್ಯಾಯನ್ಯೈ ನಮಃ । ಮಲ್ಲಿಕಾಪುಷ್ಪಂ ಸಮರ್ಪಯಾಮಿ ।\\
ಉಮಾಯೈ ನಮಃ । ಜಾತೀಪುಷ್ಪಂ ಸಮರ್ಪಯಾಮಿ ।\\
ಗೌರ್ಯೈ ನಮಃ । ಕೇತಕೀಪುಷ್ಪಂ ಸಮರ್ಪಯಾಮಿ ।\\
ಹೈಮವತ್ಯೈ ನಮಃ । ಕರವೀರಪುಷ್ಪಂ ಸಮರ್ಪಯಾಮಿ ।\\
ಶರ್ವಾಣ್ಯೈ ನಮಃ । ಯೂಥಿಕಾಪುಷ್ಪಂ ಸಮರ್ಪಯಾಮಿ ।\\
ಭವಾನ್ಯೈ ನಮಃ । ಕಹ್ಲಾರಪುಷ್ಪಂ ಸಮರ್ಪಯಾಮಿ ।\\
ರುದ್ರಾಣ್ಯೈ ನಮಃ । ಚಂಪಕಾಪುಷ್ಪಂ ಸಮರ್ಪಯಾಮಿ ।\\
ಸರ್ವಮಂಗಳಾಯೈ ನಮಃ । ಪೂಗಪುಷ್ಪಂ ಸಮರ್ಪಯಾಮಿ ।\\
ದುರ್ಗಾಯೈ ನಮಃ । ಜಪಾಪುಷ್ಪಂ ಸಮರ್ಪಯಾಮಿ ।\\
ಕಾಳ್ಯೈ ನಮಃ । ಕುಂದಪುಷ್ಪಂ ಸಮರ್ಪಯಾಮಿ ।\\
ಲಕ್ಷ್ಮ್ಯೈ ನಮಃ । ಕಮಲಪುಷ್ಪಂ ಸಮರ್ಪಯಾಮಿ ।\\
ಸರಸ್ವತ್ಯೈ ನಮಃ । ಸರ್ವಪುಷ್ಪಂ ಸಮರ್ಪಯಾಮಿ ।
\section{ಚತುಃಷಷ್ಟಿಯೋಗಿನೀ ಪೂಜಾ}
\begin{multicols}{2} 
ದಿವ್ಯಯೋಗಿನ್ಯೈ ನಮಃ ।\\
ಮಹಾಯೋಗಿನ್ಯೈ ।\\
ಸಿದ್ಧಯೋಗಿನ್ಯೈ ।\\
ಗಣೇಶ್ವರೀಯೋಗಿನ್ಯೈ ।\\
ಪ್ರೇತಾಶಿನೀಯೋಗಿನ್ಯೈ ।\\
ಟಂಕಾರೀಯೋಗಿನ್ಯೈ ।\\
ಶಾಕಿನೀಯೋಗಿನ್ಯೈ ।\\
ಡಾಕಿನೀಯೋಗಿನ್ಯೈ ।\\
ಕಾಲರಾತ್ರೀಯೋಗಿನ್ಯೈ ।\\
ನಿಶಾಚರೀಯೋಗಿನ್ಯೈ ।\\
ರೌದ್ರವೇತಾಳೀಯೋಗಿನ್ಯೈ ।\\
ಹುಂಕಾರೀಯೋಗಿನ್ಯೈ ।\\
ಗರ್ಜಿನೀಯೋಗಿನ್ಯೈ ।\\
ಭುವನೇಶ್ವರೀಯೋಗಿನ್ಯೈ ।\\
ಊರ್ಧ್ವಕೇಶೀಯೋಗಿನ್ಯೈ ।\\
ವಿಷಭಂಜನೀಯೋಗಿನ್ಯೈ ।\\
ವಿರೂಪಾಯೋಗಿನ್ಯೈ ।\\
ಶುಷ್ಕಾಂಗೀಯೋಗಿನ್ಯೈ ।\\
ನರಭೋಜಿನೀ ಯೋಗಿನ್ಯೈ ।\\
ಫಟ್ಕಾಳೀಯೋಗಿನ್ಯೈ ।\\
ವೀರಭದ್ರೇಶೀಯೋಗಿನ್ಯೈ ।\\
ಧೂಮ್ರಾಕ್ಷೀಯೋಗಿನ್ಯೈ ।\\
ಕಲಹಪ್ರಿಯಾಯೋಗಿನ್ಯೈ ।\\
ರಾಕ್ಷಸೀಯೋಗಿನ್ಯೈ ।\\
ಘೋರರಕ್ತಾಕ್ಷೀಯೋಗಿನ್ಯೈ ।\\
ವಿರೂಪಾಕ್ಷೀಯೋಗಿನ್ಯೈ ।\\
ಭಯಂಕರೀಯೋಗಿನ್ಯೈ ।\\
ವೀರಕೌಮಾರೀಯೋಗಿನ್ಯೈ ।\\
ಚಂಡೀಯೋಗಿನ್ಯೈ ।\\
ಮುಂಡಧಾರಿಣೀಯೋಗಿನ್ಯೈ ।\\
ಭಾಸುರೀಯೋಗಿನ್ಯೈ ।\\
ವಾರಾಹೀಯೋಗಿನ್ಯೈ ।\\
ರೌದ್ರಘಂಟಾಯೋಗಿನ್ಯೈ ।\\
ತ್ರಿಪುರಾಂತಕೀಯೋಗಿನ್ಯೈ ।\\
ಕಂಟಕೀಯೋಗಿನ್ಯೈ ।\\
ಮಾಲಿನೀಯೋಗಿನ್ಯೈ ।\\
ಮರ್ದಿನೀಯೋಗಿನ್ಯೈ ।\\
ಯಮದೂತೀಯೋಗಿನ್ಯೈ ।\\
ಕಾರ್ಮುಕೀಯೋಗಿನ್ಯೈ ।\\
ಕಾಕದೃಷ್ಟೀಯೋಗಿನ್ಯೈ ।\\
ಅಧೋಮುಖೀಯೋಗಿನ್ಯೈ ।\\
ರುಂಡಧಾರಿಣೀಯೋಗಿನ್ಯೈ ।\\
ವ್ಯಾಘ್ರಚರ್ಮಧರಾಯೋಗಿನ್ಯೈ ।\\
ಕಿಂಕಾರಿಣೀಯೋಗಿನ್ಯೈ ।\\
ಪ್ರತಪ್ತಭೂಷಣೀಯೋಗಿನ್ಯೈ ।\\
ಭೈರವೀಯೋಗಿನ್ಯೈ ।\\
ಭಾಸಿನೀಯೋಗಿನ್ಯೈ ।\\
ಕ್ರೋಧಿನೀಯೋಗಿನ್ಯೈ ।\\
ಮಂತ್ರಯೋಗಿನ್ಯೈ ।\\
ಕಾಲಾಗ್ನಿಗ್ರಾಹಿಣೀಯೋಗಿನ್ಯೈ ।\\
ಚಕ್ರಿಣೀಯೋಗಿನ್ಯೈ ।\\
ಕಂಕಾಲೀಯೋಗಿನ್ಯೈ ।\\
ಸಹಸ್ರಾಕ್ಷೀಯೋಗಿನ್ಯೈ ।\\
ಧೂರ್ಜಟೀಯೋಗಿನ್ಯೈ ।\\
ವಿಕರಾಳೀಯೋಗಿನ್ಯೈ ।\\
ಘೋರರೂಪಾಯೋಗಿನ್ಯೈ ।\\
ಖಟ್ವಾಂಗೀಯೋಗಿನ್ಯೈ ।\\
ಊರ್ಧ್ವ್ವಲಂಬೋಷ್ಠೀಯೋಗಿನ್ಯೈ ।\\
ಧೂಮ್ರಾಂಗೀಯೋಗಿನ್ಯೈ ।\\
ಉಗ್ರಾಯೋಗಿನ್ಯೈ ।\\
ಮಾಲಿನೀಯೋಗಿನ್ಯೈ ।\\
ಶುಭಾಯೋಗಿನ್ಯೈ ।\\
ಮಹೇಶ್ವರ್ಯೈ ।\\
ಮಹಾಯೋಗಿನ್ಯೈ ನಮಃ।
\end{multicols}
\section{ನಿತ್ಯಾದೇವೀ ನಾಮಪೂಜಾ}
\begin{multicols}{2}
\as{೪ ಅಂ} ಕಾಮೇಶ್ವರ್ಯೈ ನಮಃ ।\\
\as{೪ ಆಂ} ಭಗಮಾಲಿನ್ಯೈ ನಮಃ ।\\
\as{೪ ಇಂ} ನಿತ್ಯಕ್ಲಿನ್ನಾಯೈ ನಮಃ ।\\
\as{೪ ಈಂ} ಭೇರುಂಡಾಯೈ ನಮಃ ।\\
\as{೪ ಉಂ} ವಹ್ನಿವಾಸಿನ್ಯೈ ನಮಃ ।\\
\as{೪ ಊಂ} ಮಹಾವಜ್ರೇಶ್ವರ್ಯೈನಮಃ।\\
\as{೪ ಋಂ} ಶಿವಾದೂತ್ಯೈ ನಮಃ ।\\
\as{೪ ೠಂ} ತ್ವರಿತಾಯೈ ನಮಃ ।\\
\as{೪ ಲೃಂ} ಕುಲಸುಂದರ್ಯೈ ನಮಃ ।\\
\as{೪ ಲೄಂ} ನಿತ್ಯಾಯೈ ನಮಃ ।\\
\as{೪ ಏಂ} ನೀಲಪತಾಕಾಯೈ ನಮಃ ।\\
\as{೪ ಐಂ} ವಿಜಯಾಯೈ ನಮಃ ।\\
\as{೪ ಓಂ} ಸರ್ವಮಂಗಳಾಯೈ ನಮಃ ।\\
\as{೪ ಔಂ} ಜ್ವಾಲಾಮಾಲಿನ್ಯೈ ನಮಃ ।\\
\as{೪ ಅಂ} ಚಿತ್ರಾಯೈ ನಮಃ ।\\
\as{೪ ಅಃ} ಮಹಾನಿತ್ಯಾಯೈ ನಮಃ ।
\end{multicols}
\section{ದುರ್ಗಾಷ್ಟೋತ್ತರಶತನಾಮಾವಲಿಃ}
\as{ಮಾತರ್ಮೇ ಮಧುಕೈಟಭಘ್ನಿ ಮಹಿಷಪ್ರಾಣಾಪಹಾರೋದ್ಯಮೇ \\
ಹೇಲಾನಿರ್ಜಿತ ಧೂಮ್ರಲೋಚನವಧೇ ಹೇ ಚಂಡಮುಂಡಾರ್ದಿನಿ ।\\
ನಿಃಶೇಷೀಕೃತ ರಕ್ತಬೀಜದನುಜೇ ನಿತ್ಯೇ ನಿಶುಂಭಾಪಹೇ ।\\
ಶುಂಭಧ್ವಂಸಿನಿ ಸಂಹರಾಶು ದುರಿತಂ ದುರ್ಗೇ ನಮಸ್ತೇಽಂಬಿಕೇ ॥}
\begin{multicols}{2}
ಓಂ ದುರ್ಗಾಯೈ ನಮಃ ।\\
ಓಂ ದಾರಿದ್ರ್ಯಶಮನ್ಯೈ ನಮಃ ।\\
ಓಂ ದುರಿತಘ್ನ್ಯೈ ನಮಃ ।\\
ಓಂ ಲಕ್ಷ್ಮ್ಯೈ ನಮಃ ।\\
ಓಂ ಲಜ್ಜಾಯೈ ನಮಃ ।\\
ಓಂ ಮಹಾವಿದ್ಯಾಯೈ ನಮಃ ।\\
ಓಂ ಶ್ರದ್ಧಾಯೈ ನಮಃ ।\\
ಓಂ ಪುಷ್ಟ್ಯೈ ನಮಃ ।\\
ಓಂ ಸ್ವಧಾಯೈ ನಮಃ ।\\
ಓಂ ಧ್ರುವಾಯೈ ನಮಃ । ೧೦\\
ಓಂ ಮಹಾರಾತ್ರ್ಯೈ ನಮಃ ।\\
ಓಂ ಮಹಾಮಾಯಾಯೈ ನಮಃ ।\\
ಓಂ ಮೇಧಾಯೈ ನಮಃ ।\\
ಓಂ ಮಾತ್ರೇ ನಮಃ ।\\
ಓಂ ಸರಸ್ವತ್ಯೈ ನಮಃ ।\\
ಓಂ ಶಿವಾಯೈ ನಮಃ ।\\
ಓಂ ಶಶಿಧರಾಯೈ ನಮಃ ।\\
ಓಂ ಶಾಂತಾಯೈ ನಮಃ ।\\
ಓಂ ಶಾಂಭವ್ಯೈ ನಮಃ ।\\
ಓಂ ಭೂತಿದಾಯಿನ್ಯೈ ನಮಃ । ೨೦\\
ಓಂ ತಾಮಸ್ಯೈ ನಮಃ ।\\
ಓಂ ನಿಯತಾಯೈ ನಮಃ ।\\
ಓಂ ನಾರ್ಯೈ ನಮಃ ।\\
ಓಂ ಕಾಲ್ಯೈ ನಮಃ ।\\
ಓಂ ನಾರಾಯಣ್ಯೈ ನಮಃ ।\\
ಓಂ ಕಲಾಯೈ ನಮಃ ।\\
ಓಂ ಬ್ರಾಹ್ಮ್ಯೈ ನಮಃ ।\\
ಓಂ ವೀಣಾಧರಾಯೈ ನಮಃ ।\\
ಓಂ ವಾಣ್ಯೈ ನಮಃ ।\\
ಓಂ ಶಾರದಾಯೈ ನಮಃ । ೩೦\\
ಓಂ ಹಂಸವಾಹಿನ್ಯೈ ನಮಃ ।\\
ಓಂ ತ್ರಿಶೂಲಿನ್ಯೈ ನಮಃ ।\\
ಓಂ ತ್ರಿನೇತ್ರಾಯೈ ನಮಃ ।\\
ಓಂ ಈಶಾನಾಯೈ ನಮಃ ।\\
ಓಂ ತ್ರಯ್ಯೈ ನಮಃ ।\\
ಓಂ ತ್ರಯತಮಾಯೈ ನಮಃ ।\\
ಓಂ ಶುಭಾಯೈ ನಮಃ ।\\
ಓಂ ಶಂಖಿನ್ಯೈ ನಮಃ ।\\
ಓಂ ಚಕ್ರಿಣ್ಯೈ ನಮಃ ।\\
ಓಂ ಘೋರಾಯೈ ನಮಃ । ೪೦\\
ಓಂ ಕರಾಲ್ಯೈ ನಮಃ ।\\
ಓಂ ಮಾಲಿನ್ಯೈ ನಮಃ ।\\
ಓಂ ಮತ್ಯೈ ನಮಃ ।\\
ಓಂ ಮಾಹೇಶ್ವರ್ಯೈ ನಮಃ ।\\
ಓಂ ಮಹೇಷ್ವಾಸಾಯೈ ನಮಃ ।\\
ಓಂ ಮಹಿಷಘ್ನ್ಯೈ ನಮಃ ।\\
ಓಂ ಮಧುವ್ರತಾಯೈ ನಮಃ ।\\
ಓಂ ಮಯೂರವಾಹಿನ್ಯೈ ನಮಃ ।\\
ಓಂ ನೀಲಾಯೈ ನಮಃ ।\\
ಓಂ ಭಾರತ್ಯೈ ನಮಃ । ೫೦\\
ಓಂ ಭಾಸ್ವರಾಂಬರಾಯೈ ನಮಃ ।\\
ಓಂ ಪೀತಾಂಬರಧರಾಯೈ ನಮಃ ।\\
ಓಂ ಪೀತಾಯೈ ನಮಃ ।\\
ಓಂ ಕೌಮಾರ್ಯೈ ನಮಃ ।\\
ಓಂ ಪೀವರಸ್ತನ್ಯೈ ನಮಃ ।\\
ಓಂ ರಜನ್ಯೈ ನಮಃ ।\\
ಓಂ ರಾಧಿನ್ಯೈ ನಮಃ ।\\
ಓಂ ರಕ್ತಾಯೈ ನಮಃ ।\\
ಓಂ ಗದಿನ್ಯೈ ನಮಃ ।\\
ಓಂ ಘಂಟಿನ್ಯೈ ನಮಃ । ೬೦\\
ಓಂ ಪ್ರಭಾಯೈ ನಮಃ ।\\
ಓಂ ಶುಂಭಘ್ನ್ಯೈ ನಮಃ ।\\
ಓಂ ಸುಭಗಾಯೈ ನಮಃ ।\\
ಓಂ ಸುಭ್ರುವೇ ನಮಃ ।\\
ಓಂ ನಿಶುಂಭಪ್ರಾಣಹಾರಿಣ್ಯೈ ನಮಃ ।\\
ಓಂ ಕಾಮಾಕ್ಷ್ಯೈ ನಮಃ ।\\
ಓಂ ಕಾಮುಕಾಯೈ ನಮಃ ।\\
ಓಂ ಕನ್ಯಾಯೈ ನಮಃ ।\\
ಓಂ ರಕ್ತಬೀಜನಿಪಾತಿನ್ಯೈ ನಮಃ ।\\
ಓಂ ಸಹಸ್ರವದನಾಯೈ ನಮಃ । ೭೦\\
ಓಂ ಸಂಧ್ಯಾಯೈ ನಮಃ ।\\
ಓಂ ಸಾಕ್ಷಿಣ್ಯೈ ನಮಃ ।\\
ಓಂ ಶಾಂಕರ್ಯೈ ನಮಃ ।\\
ಓಂ ದ್ಯುತಯೇ ನಮಃ ।\\
ಓಂ ಭಾರ್ಗವ್ಯೈ ನಮಃ ।\\
ಓಂ ವಾರುಣ್ಯೈ ನಮಃ ।\\
ಓಂ ವಿದ್ಯಾಯೈ ನಮಃ ।\\
ಓಂ ಧರಾಯೈ ನಮಃ ।\\
ಓಂ ಧರಾಸುರಾರ್ಚಿತಾಯೈ ನಮಃ ।\\
ಓಂ ಗಾಯತ್ರ್ಯೈ ನಮಃ । ೮೦\\
ಓಂ ಗಾಯಕ್ಯೈ ನಮಃ ।\\
ಓಂ ಗಂಗಾಯೈ ನಮಃ ।\\
ಓಂ ದುರ್ಗಾಯೈ ನಮಃ ।\\
ಓಂ ಗೀತಘನಸ್ವನಾಯೈ ನಮಃ ।\\
ಓಂ ಛಂದೋಮಯಾಯೈ ನಮಃ ।\\
ಓಂ ಮಹ್ಯೈ ನಮಃ ।\\
ಓಂ ಛಾಯಾಯೈ ನಮಃ ।\\
ಓಂ ಚಾರ್ವಾಂಗ್ಯೈ ನಮಃ ।\\
ಓಂ ಚಂದನಪ್ರಿಯಾಯೈ ನಮಃ ।\\
ಓಂ ಜನನ್ಯೈ ನಮಃ । ೯೦\\
ಓಂ ಜಾಹ್ನವ್ಯೈ ನಮಃ ।\\
ಓಂ ಜಾತಾಯೈ ನಮಃ ।\\
ಓಂ ಶಾನ್ಙ್ಕರ್ಯೈ ನಮಃ ।\\
ಓಂ ಹತರಾಕ್ಷಸ್ಯೈ ನಮಃ ।\\
ಓಂ ವಲ್ಲರ್ಯೈ ನಮಃ ।\\
ಓಂ ವಲ್ಲಭಾಯೈ ನಮಃ ।\\
ಓಂ ವಲ್ಲ್ಯೈ ನಮಃ ।\\
ಓಂ ವಲ್ಲ್ಯಲಂಕೃತಮಧ್ಯಮಾಯೈ ನಮಃ ।\\
ಓಂ ಹರೀತಕ್ಯೈ ನಮಃ ।\\
ಓಂ ಹಯಾರೂಢಾಯೈ ನಮಃ । ೧೦೦\\
ಓಂ ಭೂತ್ಯೈ ನಮಃ ।\\
ಓಂ ಹರಿಹರಪ್ರಿಯಾಯೈ ನಮಃ ।\\
ಓಂ ವಜ್ರಹಸ್ತಾಯೈ ನಮಃ ।\\
ಓಂ ವರಾರೋಹಾಯೈ ನಮಃ ।\\
ಓಂ ಸರ್ವಸಿದ್ಧ್ಯೈ ನಮಃ ।\\
ಓಂ ವರಪ್ರದಾಯೈ ನಮಃ ।\\
ಓಂ ಸಿಂದೂರವರ್ಣಾಯೈ ನಮಃ ।\\
ಓಂ ಶ್ರೀದುರ್ಗಾದೇವ್ಯೈ ನಮಃ । ೧೦೮
\end{multicols}
ಇತಿ ಶ್ರೀದುರ್ಗಾಷ್ಟೋತ್ತರಶತನಾಮಾವಲಿಃ ಸಮಾಪ್ತಾ ।
\section{ಆರ್ಯಾಷ್ಟೋತ್ತರಶತನಾಮಾವಲಿಃ}
\as{ಹೇಮಪ್ರಖ್ಯಾಮಿಂದು ಖಂಡಾತ್ತಮೌಲಿಂ\\
ಶಂಖಾರಿಷ್ಟಾಭೀತಿ ಹಸ್ತಾಂ ತ್ರಿಣೇತ್ರಾಂ ।\\
ಹೇಮಾಬ್ಜಸ್ಥಾಂ ಪೀತವಸ್ತ್ರಾಂ ಪ್ರಸನ್ನಾಂ\\
ದೇವೀಂ ದುರ್ಗಾಂ ದಿವ್ಯರೂಪಾಂ ನಮಾಮಿ ॥}
\begin{multicols}{2}
ಓಂ ಆರ್ಯಾಯೈ ನಮಃ ।\\
ಓಂ ಕಾತ್ಯಾಯನ್ಯೈ ನಮಃ ।\\
ಓಂ ಗೌರ್ಯೈ ನಮಃ ।\\
ಓಂ ಕುಮಾರ್ಯೈ ನಮಃ ।\\
ಓಂ ವಿಂಧ್ಯವಾಸಿನ್ಯೈ ನಮಃ ।\\
ಓಂ ವಾಗೀಶ್ವರ್ಯೈ ನಮಃ ।\\
ಓಂ ಮಹಾದೇವ್ಯೈ ನಮಃ ।\\
ಓಂ ಕಾಲ್ಯೈ ನಮಃ ।\\
ಓಂ ಕಂಕಾಲಧಾರಿಣ್ಯೈ ನಮಃ ।\\
ಓಂ ಘೋಣಸಾಭರಣಾಯೈ ನಮಃ । ೧೦\\
ಓಂ ಉಗ್ರಾಯೈ ನಮಃ ।\\
ಓಂ ಸ್ಥೂಲಜಂಘಾಯೈ ನಮಃ ।\\
ಓಂ ಮಹೇಶ್ವರ್ಯೈ ನಮಃ ।\\
ಓಂ ಖಟ್ವಾಂಗಧಾರಿಣ್ಯೈ ನಮಃ ।\\
ಓಂ ಚಂಡ್ಯೈ ನಮಃ ।\\
ಓಂ ಭೀಷಣಾಯೈ ನಮಃ ।\\
ಓಂ ಮಹಿಷಾಂತಕಾಯೈ ನಮಃ ।\\
ಓಂ ರಕ್ಷಿಣ್ಯೈ ನಮಃ ।\\
ಓಂ ರಮಣ್ಯೈ ನಮಃ ।\\
ಓಂ ರಾಜ್ಞ್ಯೈ ನಮಃ । ೨೦\\
ಓಂ ರಜನ್ಯೈ ನಮಃ ।\\
ಓಂ ಶೋಷಿಣ್ಯೈ ನಮಃ ।\\
ಓಂ ರತ್ಯೈ ನಮಃ ।\\
ಓಂ ಗಭಸ್ತಿನ್ಯೈ ನಮಃ ।\\
ಓಂ ಗಂಧಿನ್ಯೈ ನಮಃ ।\\
ಓಂ ದುರ್ಗಾಯೈ ನಮಃ ।\\
ಓಂ ಗಾಂಧಾರ್ಯೈ ನಮಃ ।\\
ಓಂ ಕಲಹಪ್ರಿಯಾಯೈ ನಮಃ ।\\
ಓಂ ವಿಕರಾಲ್ಯೈ ನಮಃ ।\\
ಓಂ ಮಹಾಕಾಲ್ಯೈ ನಮಃ । ೩೦\\
ಓಂ ಭದ್ರಕಾಲ್ಯೈ ನಮಃ ।\\
ಓಂ ತರಂಗಿಣ್ಯೈ ನಮಃ ।\\
ಓಂ ಮಾಲಿನ್ಯೈ ನಮಃ ।\\
ಓಂ ದಾಹಿನ್ಯೈ ನಮಃ ।\\
ಓಂ ಕೃಷ್ಣಾಯೈ ನಮಃ ।\\
ಓಂ ಛೇದಿನ್ಯೈ ನಮಃ ।\\
ಓಂ ಭೇದಿನ್ಯೈ ನಮಃ ।\\
ಓಂ ಅಗ್ರಣ್ಯೈ ನಮಃ ।\\
ಓಂ ಗ್ರಾಮಣ್ಯೈ ನಮಃ ।\\
ಓಂ ನಿದ್ರಾಯೈ ನಮಃ । ೪೦\\
ಓಂ ವಿಮಾನಿನ್ಯೈ ನಮಃ ।\\
ಓಂ ಶೀಘ್ರಗಾಮಿನ್ಯೈ ನಮಃ ।\\
ಓಂ ಚಂಡವೇಗಾಯೈ ನಮಃ ।\\
ಓಂ ಮಹಾನಾದಾಯೈ ನಮಃ ।\\
ಓಂ ವಜ್ರಿಣ್ಯೈ ನಮಃ ।\\
ಓಂ ಭದ್ರಾಯೈ ನಮಃ ।\\
ಓಂ ಪ್ರಜೇಶ್ವರ್ಯೈ ನಮಃ ।\\
ಓಂ ಕರಾಲ್ಯೈ ನಮಃ ।\\
ಓಂ ಭೈರವ್ಯೈ ನಮಃ ।\\
ಓಂ ರೌದ್ರ್ಯೈ ನಮಃ । ೫೦\\
ಓಂ ಅಟ್ಟಹಾಸಿನ್ಯೈ ನಮಃ ।\\
ಓಂ ಕಪಾಲಿನ್ಯೈ ನಮಃ ।\\
ಓಂ ಚಾಮುಂಡಾಯೈ ನಮಃ ।\\
ಓಂ ರಕ್ತಚಾಮುಂಡಾಯೈ ನಮಃ ।\\
ಓಂ ಅಘೋರಾಯೈ ನಮಃ ।\\
ಓಂ ಘೋರರೂಪಿಣ್ಯೈ ನಮಃ ।\\
ಓಂ ವಿರೂಪಾಯೈ ನಮಃ ।\\
ಓಂ ಮಹಾರೂಪಾಯೈ ನಮಃ ।\\
ಓಂ ಸ್ವರೂಪಾಯೈ ನಮಃ ।\\
ಓಂ ಸುಪ್ರತೇಜಸ್ವಿನ್ಯೈ ನಮಃ ।\\
ಓಂ ಅಜಾಯೈ ನಮಃ । ೬೦\\
ಓಂ ವಿಜಯಾಯೈ ನಮಃ ।\\
ಓಂ ಚಿತ್ರಾಯೈ ನಮಃ ।\\
ಓಂ ಅಜಿತಾಯೈ ನಮಃ ।\\
ಓಂ ಅಪರಾಜಿತಾಯೈ ನಮಃ ।\\
ಓಂ ಧರಣ್ಯೈ ನಮಃ ।\\
ಓಂ ಧಾತ್ರ್ಯೈ ನಮಃ ।\\
ಓಂ ಪವಮಾನ್ಯೈ ನಮಃ ।\\
ಓಂ ವಸುಂಧರಾಯೈ ನಮಃ ।\\
ಓಂ ಸುವರ್ಣಾಯೈ ನಮಃ ।\\
ಓಂ ರಕ್ತಾಕ್ಷ್ಯೈ ನಮಃ । ೭೦\\
ಓಂ ಕಪರ್ದಿನ್ಯೈ ನಮಃ ।\\
ಓಂ ಸಿಂಹವಾಹಿನ್ಯೈ ನಮಃ ।\\
ಓಂ ಕದ್ರವೇ ನಮಃ ।\\
ಓಂ ವಿಜಿತಾಯೈ ನಮಃ ।\\
ಓಂ ಸತ್ಯವಾಣ್ಯೈ ನಮಃ ।\\
ಓಂ ಅರುಂಧತ್ಯೈ ನಮಃ ।\\
ಓಂ ಕೌಶಿಕ್ಯೈ ನಮಃ ।\\
ಓಂ ಮಹಾಲಕ್ಷ್ಮ್ಯೈ ನಮಃ ।\\
ಓಂ ವಿದ್ಯಾಯೈ ನಮಃ ।\\
ಓಂ ಮೇಧಾಯೈ ನಮಃ । ೮೦\\
ಓಂ ಸರಸ್ವತ್ಯೈ ನಮಃ ।\\
ಓಂ ತ್ರ್ಯಂಬಕಾಯೈ ನಮಃ ।\\
ಓಂ ತ್ರಿಸನ್ಖ್ಯಾಯೈ ನಮಃ ।\\
ಓಂ ತ್ರಿಮೂರ್ತ್ಯೈ ನಮಃ ।\\
ಓಂ ತ್ರಿಪುರಾಂತಕಾಯೈ ನಮಃ ।\\
ಓಂ ಬ್ರಾಹ್ಮ್ಯೈ ನಮಃ ।\\
ಓಂ ನಾರಸಿಂಹ್ಯೈ ನಮಃ ।\\
ಓಂ ವಾರಾಹ್ಯೈ ನಮಃ ।\\
ಓಂ ಇಂದ್ರಾಣ್ಯೈ ನಮಃ ।\\
ಓಂ ವೇದಮಾತೃಕಾಯೈ ನಮಃ । ೯೦\\
ಓಂ ಪಾರ್ವತ್ಯೈ ನಮಃ ।\\
ಓಂ ತಾಮಸ್ಯೈ ನಮಃ ।\\
ಓಂ ಸಿದ್ಧಾಯೈ ನಮಃ ।\\
ಓಂ ಗುಹ್ಯಾಯೈ ನಮಃ ।\\
ಓಂ ಇಜ್ಯಾಯೈ ನಮಃ ।\\
ಓಂ ಉಷಾಯೈ ನಮಃ ।\\
ಓಂ ಉಮಾಯೈ ನಮಃ ।\\
ಓಂ ಅಂಬಿಕಾಯೈ ನಮಃ ।\\
ಓಂ ಭ್ರಾಮರ್ಯೈ ನಮಃ ।\\
ಓಂ ವೀರಾಯೈ ನಮಃ । ೧೦೦\\
ಓಂ ಹಾಹಾಹುಂಕಾರನಾದಿನ್ಯೈ ನಮಃ ।\\
ಓಂ ನಾರಾಯಣ್ಯೈ ನಮಃ ।\\
ಓಂ ವಿಶ್ವರೂಪಾಯೈ ನಮಃ ।\\
ಓಂ ಮೇರುಮಂದಿರವಾಸಿನ್ಯೈ ನಮಃ ।\\
ಓಂ ಶರಣಾಗತದೀನಾರ್ತ\\ಪರಿತ್ರಾಣಪರಾಯಣಾಯೈ ನಮಃ ।\\
ಓಂ ತ್ರಿನೇತ್ರಾಯೈ ನಮಃ ।\\
ಓಂ ಶಶಿಧರಾಯೈ ನಮಃ ।\\
ಓಂ ಆರ್ಯಾಯೈ ನಮಃ । ೧೦೮
\end{multicols}
\section{ಸರಸ್ವತ್ಯಷ್ಟೋತ್ತರ ಶತನಾಮಾವಲಿಃ}
\begin{multicols}{2}
ಯೋಗನಿದ್ರಾಯೈ\\
ದೇವಜಾತ್ಯೈ\\
ಶುಂಭಾಯೈ\\
ನಿಶುಂಭಾಯೈ\\
ಶೈಲಜಾತ್ಯೈ\\
ಧೂಮ್ರಾಕ್ಷ್ಯೈ\\
ಚಾಮುಂಡ್ಯೈ\\
ದುರ್ಗಾಯೈ\\
ನವಶಕ್ತ್ಯಾತ್ಮಿಕಾಯೈ\\
ತ್ರಿಗುಣಾತ್ಮಕದುರ್ಗಾಯೈ ೧೦\\
ಧನಂಜಯಾಯೈ\\
ಸುರಶ್ರೇಷ್ಠಾಯೈ\\
ರಕ್ತದಂತಾಯೈ\\
ಮೃಡಾಯೈ\\
ದ್ವಿರದವಾಸಿನ್ಯೈ\\
ದ್ಯುತ್ಯೈ\\
ರೌದ್ರ್ಯೈ\\
ಮಹಾದೇವ್ಯೈ\\
ಶಂಕರವಲ್ಲಭಾಯೈ\\
ಬ್ರಾಹ್ಮ್ಯೈ ೨೦\\
ಭೀಮಾಯೈ\\
ಶಿವದೂತ್ಯೈ\\
ಕೌಶಿಕ್ಯೈ\\
ಸರ್ವಶಕ್ತಿಸಮನ್ವಿತಾಯೈ\\
ಕುಮಾರ್ಯೈ\\
ತ್ರಿಮೂರ್ತ್ಯೈ\\
ಕಲ್ಪೋಪಮಾಯೈ\\
ಕಲ್ಪ್ಯೈ\\
ಚಂಡಿಕಾಯೈ\\
ಸುಭದ್ರಾಯೈ ೩೦\\
ಗಜಾರೂಢಾಯೈ\\
ಸಿಂಹಾರೂಢಾಯೈ\\
ಖಟ್ವಾಂಗಚರ್ಮತ್ರಿಶೂಲಾರವಿಂದಾಯೈ\\
ಡಮರುಗಗದಾಭಯಧಾರಿಣ್ಯೈ\\
ಅಕ್ಷಮಾಲಾರವಿಂದಾಯೈ\\
ಧೃತ್ಯೈ\\
ಕಾಲ್ಯೈ\\
ವೇದಗರ್ಭಾಯೈ\\
ಸ್ವರ್ಣಧಾರಿಣ್ಯೈ\\
ಖಡ್ಗಾಯೈ ೪೦\\
ವಜ್ರಾಯೈ\\
ತ್ರಿಶೂಲಾಯೈ\\
ಸರ್ಪಾಯೈ\\
ಅಂಬಿಕಾಯೈ\\
ಭವಾಯೈ\\
ಕೌಮಾರ್ಯೈ\\
ವಿದ್ಯಾಯೈ\\
ಸುವಾಸಿನ್ಯೈ\\
ಶುಂಭನಿಶುಂಭನಿಪಾತಿನ್ಯೈ\\
ಮಧುಕೈಟಭಮಹಿಷಾಸುರಮರ್ದಿನ್ಯೈ ೫೦\\
ಚಂಡಮುಂಡರಕ್ತಬೀಜನಿಹಂತ್ರ್ಯೈ\\
ಜಲದುರ್ಗಾಯೈ\\
ಸ್ಥಲದುರ್ಗಾಯೈ\\
ಗರಿದುರ್ಗಾಯೈ\\
ಅಗ್ನಿದುರ್ಗಾಯೈ\\
ನವದುರ್ಗಾಯೈ\\
ಶ್ರೀದುರ್ಗಾಯೈ\\
ಅನೆಕಾಭೇದ್ಯದುರ್ಗಾಯೈ\\
ಮೂಷಿಕವಾಹಿನ್ಯೈ\\
ಮಯೂರವಾಹಿನ್ಯೈ ೭೦\\
ರಥಾರೂಢಾಯೈ\\
ಪಂಚಾನನವಾಹಿನ್ಯೈ\\
ರಕ್ತವರ್ಣಾಯೈ\\
ಶುಕ್ಲವರ್ಣಾಯೈ\\
ರೌದ್ರವರ್ಣಾಯೈ\\
ಕ್ರೋಢವರ್ಣಾಯೈ\\
ಪೀತವರ್ಣಾಯೈ\\
ಶೋಣಿತವರ್ಣಾಯೈ\\
ಗೌರವರ್ಣಾಯೈ\\
ಸುವರ್ಣವರ್ಣಾಯೈ ೮೦\\
ಶಕ್ತಿಶೂಲಗದಾಭಯಧಾರಿಣ್ಯೈ\\
ಶಂಖಚಕ್ರಮಾತುಲುಂಗಧಾರಿಣ್ಯೈ\\
ಬಾಣಕೋದಂಡಖಡ್ಗಧಾರಿಣ್ಯೈ\\
ಅಂಬುಜಧ್ವಜಧಾರಿಣ್ಯೈ\\
ಅಷ್ಟಾದಶಭುಜಾವಲಂಬಿನ್ಯೈ\\
ತಾಪತ್ರಯವಿಧ್ವ್ವಂಸಿನ್ಯೈ\\
ಶತ್ರುಭಂಜಿನ್ಯೈ\\
ಸರ್ವಾರಿಷ್ಟವಿನಾಶಿನ್ಯೈ\\
ತ್ರಿದಶವಂದಿತಾಯೈ\\
ಸರ್ವದೇವಸ್ವರೂಪಿಣ್ಯೈ ೯೦\\
ಶಾಂಕರ್ಯೈ\\
ರಕ್ತದಂತಿಕಾಯೈ\\
ಅಷ್ಟಕೌಮಾರ್ಯೈ\\
ಜಗತ್ಪ್ರತಿಷ್ಠಿತಾಯೈ\\
ಶತಾಕ್ಷಿಣ್ಯೈ\\
ಪ್ರಕೃತ್ಯೈ\\
ಮಾನ್ಯೈ\\
ತನುಮಧ್ಯಾಯೈ\\
ವಿಷ್ಣುಪ್ರಿಯಾಯೈ\\
ಚೇತನಾಧಿಷ್ಠಾನಾಯೈ ೧೦೦\\
ಜಗದಾತ್ಮಿಕಾಯೈ\\
ಅಂಬಿಕಾಯೈ\\
ಶ್ರೇಯೋರೂಪಿಣ್ಯೈ\\
ಸರ್ವಾಶ್ರಯಾಯೈ\\
ಸ್ವಧಾಯೈ\\
ಸದಾತ್ಮಿಕಾಯೈ\\
ಮೇಧಾಯೈ\\
ಪತತ್ರಿಹಂತ್ರಿಣ್ಯೈ\\
ಸ್ವರೂಪಿಣ್ಯೈ\\
ಕ್ಲೇಶಾಯೈ\\
ಕ್ಲೇಶಹಾರಿಣ್ಯೈ\\
ದಾರಿದ್ರ್ಯದುಃಖಭಯಹಾರಿಣ್ಯೈ\\
ಮಹಾಮಾಯಾಯೈ\\
ಮಹಾಲಕ್ಷ್ಮ್ಯೈ\\
ಸರ್ವಾಭರಣಭೂಷಿತಾಯೈ
\end{multicols}
ಗುಗ್ಗುಲಂ ಘೃತಸಂಯುಕ್ತಂ ನಾನಾಗಂಧೈಃ ಸಮನ್ವಿತಮ್ ।\\
ಧೂಪಂ ಗೃಹಾಣ ದೇವೇಶಿ ಭದ್ರಕಾಳಿ ನಮೋಽಸ್ತು ತೇ ॥\\
\as{ಯತ್ಪುರುಷಂ ವ್ಯದಧುಃ ಪಾದಾ ಉಚ್ಯೇತೇ॥\\
ಕರ್ದಮೇನ++++++++ಪದ್ಮಮಾಲಿನೀಮ್ ॥} ಧೂಪಃ ॥

ದಶವರ್ತಿಯುತಂ ದೇವಿ ಗೋಘೃತೇನ ಸಮನ್ವಿತಂ ।\\
ದೀಪಂ ಗೃಹಾಣ ದೇವೇಶಿ ಜ್ಞಾನದಾತ್ರಿ ನಮೋಽಸ್ತು ತೇ ॥\\
\as{ಬ್ರಾಹ್ಮಣೋಽಸ್ಯ +++++ಶೂದ್ರೋ ಅಜಾಯತ॥\\
ಆಪಃ ಸೃಜನ್ತು ++++++++ಮೇ ಕುಲೇ॥}ದೀಪಃ ॥

ಗಾಯತ್ರ್ಯಾ ಸಂಪ್ರೋಕ್ಷ್ಯ -\\
ಪಾಯಸಂ ಸಘೃತಂ ಕ್ಷೌದ್ರಂ ಮಾಷಾಪೂಪಾದಿ ಸಂಯುತಂ ।\\
ದಿವ್ಯಾನ್ನಂ ಚ ಘೃತೋಪೇತಂ ನಾನಾ ಭಕ್ಷ್ಯ ಸಮನ್ವಿತಮ್ ॥

ಷಡ್ರಸಾನಿ ಚ ಭೋಜ್ಯಾನಿ ವ್ಯಂಜನಾನಿ ಬಹೂನಿ ಚ ॥\\
ಕದಲೀ ನಾಲಿಕೇರಾಢ್ಯಂ ಸೋಪದಂಶಂ ಸಶರ್ಕರಮ್ ।

ನೈವೇದ್ಯಂ ಗೃಹ್ಯತಾಂ ದೇವಿ ಸುಪ್ರಸನ್ನಾಖಿಲೇಶ್ವರಿ ॥\\
ಮಯಾ ನಿವೇದಿತಂ ಸರ್ವಂ ವಿಶ್ವಮೂರ್ತೇ ನಮೋಽಸ್ತು ತೇ ॥\\
\as{ಚಂದ್ರಮಾ ಮನಸೋ+++++  ಪ್ರಾಣಾದ್ವಾಯುರಜಾಯತ॥\\
ಆರ್ದ್ರಾಂ ಪುಷ್ಕರಿಣೀಂ ++++ ಮ ಆವಹ॥} ನೈವೇದ್ಯಮ್ ॥

ಏಲೋಶೀರ ಲವಂಗಾದಿ ಮೃಗನಾಭಿ ವಿಮಿಶ್ರಿತಮ್ ।\\
ಪಾನೀಯಂ ಗೃಹ್ಯತಾಂ ಮಧ್ಯೇ ಪ್ರಸನ್ನಾ ವರದಾ ಭವ ॥\\
\as{ಸ್ವಾದುಃ ಪವಸ್ವ ++++++ಅದಾಭ್ಯಃ ॥}ಸ್ವಾದೂದಕಮ್ ॥

ಮಲ್ಲಿಕಾ ಮಾಲತೀ ಜಾತೀ ಕೇತಕ್ಯಾದಿ ಸುಗಂಧಿತಮ್ ।\\
ಉತ್ತರಾಪೋಶನಂ ದೇವಿ ಗೃಹಾಣ ವಿಭವಪ್ರದೇ ॥ಉತ್ತರಾಪೋಶನಮ್ ॥

ಮಹಾದೇವಿ ಮಹಾಮಾಯೇ ಮಹಾರೌದ್ರೇ ಮಹಾಪ್ರಭೇ ।\\
ಹಸ್ತಪ್ರಕ್ಷಾಳನಾರ್ಥಂ ಚ ಸಲಿಲಂ ಪ್ರತಿಗೃಹ್ಯತಾಮ್ ॥ಹಸ್ತಪ್ರಕ್ಷಾಳನಮ್ ॥

ಜಾಹ್ನವೀ ಸಿಂಧು ಕಾವೇರೀ ಜಲತುಲ್ಯಮಿದಂ ಜಲಮ್ ।\\
ದತ್ತಂ ಚ ಮುಖಶುದ್ಧ್ಯರ್ಥಂ ಪ್ರೀತ್ಯಾ ಸ್ವೀಕುರು ಶಂಕರಿ ॥\\ಮುಖಪ್ರಕ್ಷಾಲನಮ್॥

ಗಂಗಾದಿ ಸರ್ವತೀರ್ಥಾಂಬು ಸುಗಂಧ ದ್ರವ್ಯ ಮಿಶ್ರಿತಮ್ ।\\
ಪುನರಾಚಮನಾರ್ಥಂ ತು ಗೃಹಾಣ ಪರಮೇಶ್ವರಿ ॥ಪುನರಾಚಮನಮ್॥

ಗಂಧಕೇಸರ ಕರ್ಪೂರ ಚೂರ್ಣಂ ಏಲಾ ವಿಮಿಶ್ರಿತಂ ।\\
ಕರಸ್ಯೋದ್ವರ್ತನಾರ್ಥಂ ಚ ಗೃಹಾಣ ಪರಮೇಶ್ವರಿ ॥ಕರೋದ್ವರ್ತನಮ್॥

ನಾಲಿಕೇರ ದಾಡಿಮಾನಿ ರಂಭಾದ್ರಾಕ್ಷಾಫಲಾನಿ ಚ ।\\
ನಾರಂಗಾಂಜೂರ ಪನಸ ಚೂತ ಖರ್ಜೂರಕೈಃ ಸಹ ।\\
ಫಲಾನಿ ಸ್ವೀಕುರು ಪ್ರೀತ್ಯಾ ಕಾಮದೇ ಸುರಪೂಜಿತೇ ॥

ತಾಂಬೂಲಂ ಪೂಗಶಕಲಂ ಮೃಗನಾಭಿ ಸಮನ್ವಿತಮ್ ।\\
ಕರ್ಪೂರಚೂರ್ಣಸಂಯುಕ್ತಂ ತಾಂಬೂಲಂ ಪ್ರತಿಗೃಹ್ಯತಾಮ್ ॥\\ತಾಂಬೂಲಮ್॥

ನೀರಾಜನಂ ಪ್ರಯಚ್ಛಾಮಿ ವಿಶ್ವರೂಪಧರೇ ಶಿವೇ ।\\
ತ್ರಾಹಿ ಮಾಂ ನರಕಾದ್ಘೋರಾದ್ ಜ್ಞಾನಮೂರ್ತೇ ನಮೋಽಸ್ತು ತೇ ॥\\
\as{ನಾಭ್ಯಾ +++++ಲೋಕಾನ್ ಅಕಲ್ಪಯನ್॥\\
ಆರ್ದ್ರಾಂ ಯಃ ಕರಿಣೀಂ+++++ಮ ಆವಹ॥}ನೀರಾಜನಮ್ ॥

ದುರ್ಗೇ ದೇವಿ ನಮಸ್ತುಭ್ಯಂ ನಮಸ್ತ್ರೈಲೋಕ್ಯಪೂಜಿತೇ ।\\
ದಿವ್ಯಪುಷ್ಪಾಂಜಲಿಂ ದೇವಿ ಗೃಹಾಣ ಪರಮೇಶ್ವರಿ ॥\\
\as{ಜಾತವೇದಸೇ++++++ತ್ಯಗ್ನಿಃ ॥\\
ದೇವೀಂ ವಾಚ+++++++ಸುಷ್ಟುತೈತು ॥\\
ಪಾವಕಾ ನಃ +++++ರಾಜತಿ ॥\\
ರಾಜಾಧಿರಾಜಾಯ+++++ಮಹೇಶ್ವರಃ ॥}ಮಂತ್ರಪುಷ್ಪಮ್॥

ಕ್ಷೀರಾಬ್ಧಿ ಸಂಭವೇ ಮಾತಃ ಸರ್ವಮಂಗಳದೇವತೇ ।\\
ಸರ್ವಶಕ್ತಿಸ್ವರೂಪೇ ತ್ವಂ ಸ್ವೀಕುರುಷ್ವ ಪ್ರದಕ್ಷಿಣಮ್ ॥\\
\as{ಸಪ್ತಾಸ್ಯಾಸನ್+++++ ಅಬಧ್ನನ್ಪುರುಷಂ ಪಶುಂ॥\\
ತಾಂ ಮ ಆವಹ ++++++ಪುರುಷಾನಹಮ್ ॥}ಪ್ರದಕ್ಷಿಣಮ್॥

ದಾಮೋದರಿ ನಮಸ್ತುಭ್ಯಂ ನಮಃ ಸ್ತ್ರೈಲೋಕ್ಯನಾಯಿಕೇ ।\\
ನಮಸ್ತೇಽಸ್ತು ಮಹಾದೇವಿ ತ್ರಾಹಿ ಮಾಂ ಪರಮೇಶ್ವರಿ ॥

ಆಧಾರಭೂತಾ ಜಗತಸ್ತ್ವಮೇಕಾ\\
        ಮಹೀಸ್ವರೂಪೇಣ ಯತಃ ಸ್ಥಿತಾಸಿ ।\\
ಅಪಾಂ ಸ್ವರೂಪಸ್ಥಿತಯಾ ತ್ವಯೈತ-\\
      ದಾಪ್ಯಾಯತೇ ಕೃತ್ಸ್ನಮಲಂಘ್ಯವೀರ್ಯೇ ॥

ತ್ವಂ ವೈಷ್ಣವೀಶಕ್ತಿರನಂತವೀರ್ಯಾ\\
      ವಿಶ್ವಸ್ಯ ಬೀಜಂ ಪರಮಾಸಿ ಮಾಯಾ ।\\
ಸಮ್ಮೋಹಿತಂ ದೇವಿ ಸಮಸ್ತಮೇತತ್\\
      ತ್ವಂ ವೈ ಪ್ರಸನ್ನಾ ಭುವಿ ಮುಕ್ತಿಹೇತುಃ ॥

ವಿದ್ಯಾಃ ಸಮಸ್ತಾಸ್ತವ ದೇವಿ ಭೇದಾಃ\\
        ಸ್ತ್ರಿಯಃ ಸಮಸ್ತಾಃ ಸಕಲಾ ಜಗತ್ಸು ।\\
ತ್ವಯೈಕಯಾ ಪೂರಿತಮಂಬಯೈತತ್\\
        ಕಾ ತೇ ಸ್ತುತಿಃ ಸ್ತವ್ಯಪರಾಪರೋಕ್ತಿಃ ॥

ಸರ್ವಭೂತಾ ಯದಾ ದೇವೀ ಭುಕ್ತಿಮುಕ್ತಿಪ್ರದಾಯಿನೀ ।\\
ತ್ವಂ ಸ್ತುತಾ ಸ್ತುತಯೇ ಕಾ ವಾ ಭವಂತು ಪರಮೋಕ್ತಯಃ ॥

ಸರ್ವಸ್ಯ ಬುದ್ಧಿರೂಪೇಣ ಜನಸ್ಯ ಹೃದಿ ಸಂಸ್ಥಿತೇ ।\\
ಸ್ವರ್ಗಾಪವರ್ಗದೇ ದೇವಿ ನಾರಾಯಣಿ ನಮೋಽಸ್ತು ತೇ ॥

ಕಲಾಕಾಷ್ಠಾದಿರೂಪೇಣ ಪರಿಣಾಮಪ್ರದಾಯಿನಿ ।\\
ವಿಶ್ವಸ್ಯೋಪರತೌ ಶಕ್ತೇ ನಾರಾಯಣಿ ನಮೋಽಸ್ತು ತೇ ॥

ಸರ್ವಮಂಗಲಮಾಂಗಲ್ಯೇ ಶಿವೇ ಸರ್ವಾರ್ಥಸಾಧಿಕೇ ।\\
ಶರಣ್ಯೇ ತ್ರ್ಯಂಬಕೇ ಗೌರಿ ನಾರಾಯಣಿ ನಮೋಽಸ್ತು ತೇ ॥

ಸೃಷ್ಟಿಸ್ಥಿತಿವಿನಾಶಾನಾಂ ಶಕ್ತಿಭೂತೇ ಸನಾತನಿ ।\\
ಗುಣಾಶ್ರಯೇ ಗುಣಮಯೇ ನಾರಾಯಣಿ ನಮೋಽಸ್ತು ತೇ ॥

ಶರಣಾಗತದೀನಾರ್ತಪರಿತ್ರಾಣಪರಾಯಣೇ ।\\
ಸರ್ವಸ್ಯಾರ್ತಿಹರೇ ದೇವಿ ನಾರಾಯಣಿ ನಮೋಽಸ್ತು ತೇ ॥

ಹಂಸಯುಕ್ತವಿಮಾನಸ್ಥೇ ಬ್ರಹ್ಮಾಣೀರೂಪಧಾರಿಣಿ ।\\
ಕೌಶಾಂಭಃಕ್ಷರಿಕೇ ದೇವಿ ನಾರಾಯಣಿ ನಮೋಽಸ್ತು ತೇ ॥

ತ್ರಿಶೂಲಚಂದ್ರಾಹಿಧರೇ ಮಹಾವೃಷಭವಾಹಿನಿ ।\\
ಮಾಹೇಶ್ವರೀಸ್ವರೂಪೇಣ ನಾರಾಯಣಿ ನಮೋಽಸ್ತುತೇ ॥

ಮಯೂರಕುಕ್ಕುಟವೃತೇ ಮಹಾಶಕ್ತಿಧರೇಽನಘೇ ।\\
ಕೌಮಾರೀರೂಪಸಂಸ್ಥಾನೇ ನಾರಾಯಣಿ ನಮೋಽಸ್ತು ತೇ ॥

ಶಂಖಚಕ್ರಗದಾಶಾರ್ಙ್ಗಗೃಹೀತಪರಮಾಯುಧೇ ।\\
ಪ್ರಸೀದ ವೈಷ್ಣವೀರೂಪೇ ನಾರಾಯಣಿ ನಮೋಽಸ್ತು ತೇ ॥

ಗೃಹೀತೋಗ್ರಮಹಾಚಕ್ರೇ ದಂಷ್ಟ್ರೋದ್ಧೃತವಸುಂಧರೇ ।\\
ವರಾಹರೂಪಿಣಿ ಶಿವೇ ನಾರಾಯಣಿ ನಮೋಽಸ್ತು ತೇ ॥

ನೃಸಿಂಹರೂಪೇಣೋಗ್ರೇಣ ಹಂತುಂ ದೈತ್ಯಾನ್ ಕೃತೋದ್ಯಮೇ ।\\
ತ್ರೈಲೋಕ್ಯತ್ರಾಣಸಹಿತೇ ನಾರಾಯಣಿ ನಮೋಽಸ್ತು ತೇ ॥

ಕಿರೀಟಿನಿ ಮಹಾವಜ್ರೇ ಸಹಸ್ರನಯನೋಜ್ಜ್ವಲೇ ।\\
ವೃತ್ರಪ್ರಾಣಹರೇ ಚೈಂದ್ರಿ ನಾರಾಯಣಿ ನಮೋಽಸ್ತು ತೇ ॥

ಶಿವದೂತೀ ಸ್ವರೂಪೇಣ ಹತದೈತ್ಯ ಮಹಾಬಲೇ ।\\
ಘೋರರೂಪೇ ಮಹಾರಾವೇ ನಾರಾಯಣಿ ನಮೋಽಸ್ತು ತೇ ॥

ದಂಷ್ಟ್ರಾ ಕರಾಲವದನೇ ಶಿರೋಮಾಲಾವಿಭೂಷಣೇ ।\\
ಚಾಮುಂಡೇ ಮುಂಡಮಥನೇ ನಾರಾಯಣಿ ನಮೋಽಸ್ತು ತೇ ॥

ಲಕ್ಷ್ಮಿ ಲಜ್ಜೇ ಮಹಾವಿದ್ಯೇ ಶ್ರದ್ಧೇ ಪುಷ್ಟಿ ಸ್ವಧೇ ಧ್ರುವೇ ।\\
ಮಹಾರಾತ್ರಿ ಮಹಾಮಾಯೇ ನಾರಾಯಣಿ ನಮೋಽಸ್ತು ತೇ ॥

ಮೇಧೇ ಸರಸ್ವತಿ ವರೇ ಭೂತಿ ಬಾಭ್ರವಿ ತಾಮಸಿ ।\\
ನಿಯತೇ ತ್ವಂ ಪ್ರಸೀದೇಶೇ ನಾರಾಯಣಿ ನಮೋಽಸ್ತುತೇ ॥

ಸರ್ವಸ್ವರೂಪೇ ಸರ್ವೇಶೇ ಸರ್ವಶಕ್ತಿಸಮನ್ವಿತೇ ।\\
ಭಯೇಭ್ಯಸ್ತ್ರಾಹಿ ನೋ ದೇವಿ ದುರ್ಗೇ ದೇವಿ ನಮೋಽಸ್ತು ತೇ ॥

ಏತತ್ತೇ ವದನಂ ಸೌಮ್ಯಂ ಲೋಚನತ್ರಯಭೂಷಿತಂ ।\\
ಪಾತು ನಃ ಸರ್ವಭೂತೇಭ್ಯಃ ಕಾತ್ಯಾಯನಿ ನಮೋಽಸ್ತು ತೇ ॥

ಜ್ವಾಲಾ ಕರಾಲಮತ್ಯುಗ್ರಮಶೇಷಾಸುರ ಸೂದನಂ ।\\
ತ್ರಿಶೂಲಂ ಪಾತು ನೋ ಭೀತೇರ್ಭದ್ರಕಾಲಿ ನಮೋಽಸ್ತು ತೇ ॥

ಹಿನಸ್ತಿ ದೈತ್ಯತೇಜಾಂಸಿ ಸ್ವನೇನಾಪೂರ್ಯ ಯಾ ಜಗತ್ ।\\
ಸಾ ಘಂಟಾ ಪಾತು ನೋ ದೇವಿ ಪಾಪೇಭ್ಯೋ ನಃ ಸುತಾನಿವ ॥

ಅಸುರಾಸೃಗ್ವಸಾ ಪಂಕ ಚರ್ಚಿತಸ್ತೇ ಕರೋಜ್ಜ್ವಲಃ ।\\
ಶುಭಾಯ ಖಡ್ಗೋ ಭವತು ಚಂಡಿಕೇ ತ್ವಾಂ ನತಾ ವಯಂ ॥

ರೋಗಾನಶೇಷಾನಪಹಂಸಿ ತುಷ್ಟಾ\\
        ರುಷ್ಟಾ ತು ಕಾಮಾನ್ ಸಕಲಾನಭೀಷ್ಟಾನ್ ।\\
ತ್ವಾಮಾಶ್ರಿತಾನಾಂ ನ ವಿಪನ್ನರಾಣಾಂ\\
        ತ್ವಾಮಾಶ್ರಿತಾ ಹ್ಯಾಶ್ರಯತಾಂ ಪ್ರಯಾಂತಿ ॥

ಜಯ ತ್ವಂ ದೇವಿ ಚಾಮುಂಡೇ ಜಯ ಭೂತಾರ್ತಿಹಾರಿಣಿ ।\\
ಜಯ ಸರ್ವಗತೇ ದೇವಿ ಕಾಲರಾತ್ರಿ ನಮೋಽಸ್ತು ತೇ ॥

ಜಯಂತೀ ಮಂಗಲಾ ಕಾಲೀ ಭದ್ರಕಾಲೀ ಕಪಾಲಿನೀ ।\\
ದುರ್ಗಾ ಕ್ಷಮಾ ಶಿವಾ ಧಾತ್ರೀ ಸ್ವಾಹಾ ಸ್ವಧಾ ನಮೋಽಸ್ತು ತೇ ॥\\
\as{ಯಜ್ಞೇನ ಯಜ್ಞಮಯ+++++-ಪೂರ್ವೇ ಸಾಧ್ಯಾಃ ಸಂತಿ ದೇವಾಃ ॥\\
ಯಃ ಶುಚಿಃ ++++++++ಸತತಂ ಜಪೇತ್ ॥}ನಮಸ್ಕಾರಾಃ ॥

ನಮಸ್ತೇ ದೇವದೇವೇಶಿ ಭಕ್ತಾಭೀಷ್ಟಪ್ರದಾಯಿನಿ ।\\
ಪ್ರಸನ್ನಾರ್ಘ್ಯಂ ಪ್ರದಾಸ್ಯಾಮಿ ಗೃಹಾಣ ಪರಮೇಶ್ವರಿ ॥

ಕುಂಕುಮೇನ ಸಮಾಯುಕ್ತಂ ಚಂದನೇನ ವಿಮಿಶ್ರಿತಂ ।\\
ಬಿಲ್ವಪತ್ರೇಣ ಸಹಿತಂ ಗೃಹಾಣಾರ್ಘ್ಯಂ ನಮೋಽಸ್ತು ತೇ ॥\\
\as{ಕಾತ್ಯಾಯನಾಯ ವಿದ್ಮಹೇ ಕನ್ಯಕುಮಾರಿ ಧೀಮಹಿ ।\\ ತನ್ನೋ ದುರ್ಗಿಃ ಪ್ರಚೋದಯಾತ್ ॥}\\
ಮಹಾಕಾಲ್ಯೈ ದುರ್ಗ್ಯೈ ನಮಃ । ಇದಮರ್ಘ್ಯಮರ್ಘ್ಯಮ್ ।

ಕ್ಷೀರವಾರಿಧಿಜೇ ತುಭ್ಯಂ ನಮೋ ಭಾರ್ಗವ ನಂದಿನಿ ।\\
ಸರ್ವದೇವಸ್ವರೂಪಿಣ್ಯೈ ಮಹಾಲಕ್ಷ್ಮ್ಯೈ ನಮೋ ನಮಃ ॥ \\
\as{ಮಹಾದೇವ್ಯೈ ಚ ವಿದ್ಮಹೇ ವಿಷ್ಣುಪತ್ನ್ಯೈ ಚ ಧೀಮಹಿ ।\\ ತನ್ನೋ ಲಕ್ಷ್ಮೀಃ ಪ್ರಚೋದಯಾತ್ ॥}\\
ಮಹಾಲಕ್ಷ್ಮ್ಯೈ ದುರ್ಗ್ಯೈ ನಮಃ । ಇದಮರ್ಘ್ಯಮರ್ಘ್ಯಮ್ ।
\newpage
ದಾಮೋದರಿ ನಮಸ್ತುಭ್ಯಂ ನಮಸ್ತ್ರೈಲೋಕ್ಯನಾಯಕಿ ।\\
ನಮಸ್ತೇಽಸ್ತು ಮಹಾ ದೇವಿ ತ್ರಾಹಿ ಮಾಂ ಚ ಸರಸ್ವತಿ ॥\\
\as{ಬ್ರಹ್ಮಪತ್ನ್ಯೈ ಚ ವಿದ್ಮಹೇ ವೇದಮಾತ್ರೇ ಚ ಧೀಮಹಿ ।\\ ತನ್ನಃ ಸರಸ್ವತೀ ಪ್ರಚೋದಯಾತ್ ॥}\\
ಮಹಾಸರಸ್ವತ್ಯೈ ದುರ್ಗ್ಯೈ ನಮಃ । ಇದಮರ್ಘ್ಯಮರ್ಘ್ಯಮ್ ।
ಪ್ರಸನ್ನಾರ್ಘ್ಯಮ್॥

ನೃತ್ಯಂ ಗೀತಂ ಚ ವಾದ್ಯಂ ಚ ಛತ್ರಚಾಮರಮೇವ ಚ ।\\
ಆಂದೋಲನಾದಿಕಂ ಕೃತ್ವಾ ಉಪಚಾರಾನ್ ಪ್ರಕಲ್ಪಯೇತ್ ॥
\section{ಪುನಃ ಪ್ರಸನ್ನಪೂಜಾ}
ದುರ್ಗಾಯೈ ನಮಃ । \as{ಧ್ಯಾನಂ ಸಮರ್ಪಯಾಮಿ ॥}\\
ಕಾತ್ಯಾಯನ್ಯೈ  ನಮಃ । \as{ಆವಾಹನಂ ಸಮರ್ಪಯಾಮಿ ॥}\\
ಗೌರ್ಯೈ ನಮಃ । \as{ಆಸನಂ ಸಮರ್ಪಯಾಮಿ ॥}\\
ಮಹಿಷಮರ್ದಿನ್ಯೈ  ನಮಃ । \as{ಪಾದ್ಯಂ ಸಮರ್ಪಯಾಮಿ ॥}\\
ಚಂಡಿಕಾಯೈ  ನಮಃ । \as{ಅರ್ಘ್ಯಂ ಸಮರ್ಪಯಾಮಿ ॥}\\
ಶಿವದೂತ್ಯೈ  ನಮಃ । \as{ಆಚಮನಂ ಸಮರ್ಪಯಾಮಿ ॥}\\
ಮಹಾಲಕ್ಷ್ಮ್ಯೈ ನಮಃ । \as{ಮಧುಪರ್ಕ ಸಮರ್ಪಯಾಮಿ ॥}\\
ಮಹಾಸರಸ್ವತ್ಯೈ ನಮಃ । \as{ಸ್ನಾನ ಸಮರ್ಪಯಾಮಿ ॥}\\
ಸರ್ವೇಶ್ವರ್ಯೈ ನಮಃ । \as{ಪಂಚಾಮೃತಂ ಸಮರ್ಪಯಾಮಿ ॥}\\
ಕಾಲ್ಯೈ  ನಮಃ । \as{ಶುದ್ಧೋದಕಸ್ನಾನ ಸಮರ್ಪಯಾಮಿ ॥}\\
ಪರಮೇಶ್ವರ್ಯೈ  ನಮಃ । \as{ವಸ್ತ್ರಂ ಸಮರ್ಪಯಾಮಿ ॥}\\
ಮಹಾಭೂತ್ಯೈ ನಮಃ । \as{ಉಪವೇತಂ ಸಮರ್ಪಯಾಮಿ ॥}\\
ಹೈಮವತ್ಯೈ ನಮಃ । \as{ಕಂಚುಕಂ ಸಮರ್ಪಯಾಮಿ ॥}\\
ದಾಕ್ಷಾಯಣ್ಯೈ ನಮಃ । \as{ಆಭರಣಂ ಸಮರ್ಪಯಾಮಿ ॥}\\
ದಿವ್ಯಚಕ್ಷುಷೇ ನಮಃ । \as{ಅಷ್ಟಗಂಧಂ ಸಮರ್ಪಯಾಮಿ ॥}\\
ಸರ್ವಾಭರಣಭೂಷಿತಾಯೈ ನಮಃ । \as{ಅಕ್ಷತಾನ್ ಸಮರ್ಪಯಾಮಿ ॥}\\
ಸರ್ವಮಂಗಲಾಯೈ  ನಮಃ । \as{ಹರಿದ್ರಾಂ ಸಮರ್ಪಯಾಮಿ ॥}\\
ವರದಾಯೈ  ನಮಃ । \as{ಕುಂಕುಮಂ ಸಮರ್ಪಯಾಮಿ ॥}\\
ಗಾಯತ್ರ್ಯೈ ನಮಃ । \as{ಸಿಂದೂರಂ ಸಮರ್ಪಯಾಮಿ ॥}\\
ಸಾವಿತ್ರ್ಯೈ ನಮಃ । \as{ಪುಷ್ಪಂ ಸಮರ್ಪಯಾಮಿ ॥}\\
ಅಂಬಿಕಾಯೈ ನಮಃ । \as{ಧೂಪಂ ಸಮರ್ಪಯಾಮಿ ॥}\\
ವಿಜಯಾಯೈ ನಮಃ । \as{ದೀಪಂ ಸಮರ್ಪಯಾಮಿ ॥}\\
ಜ್ಞಾನಮೂರ್ತ್ಯೈ  ನಮಃ । \as{ನೈವೇದ್ಯಂ ಸಮರ್ಪಯಾಮಿ ॥}\\
ಮಾಹೇಶ್ವರ್ಯೈ ನಮಃ । \as{ತಾಂಬೂಲಂ ಸಮರ್ಪಯಾಮಿ ॥}\\
ಶಿವಶಕ್ತ್ಯೈ ನಮಃ । \as{ದೀಪಮಾಲಾಂ ಸಮರ್ಪಯಾಮಿ ॥}\\
ಶಾರದಾಯೈ ನಮಃ । \as{ನೀರಾಜನಂ ಸಮರ್ಪಯಾಮಿ ॥}\\
ಲಲಿತಾಯೈ ನಮಃ । \as{ಮಂತ್ರಪುಷ್ಪಂ ಸಮರ್ಪಯಾಮಿ ॥}\\
ಚಾಮುಂಡಾಯೈ ನಮಃ । \as{ಪ್ರದಕ್ಷಿಣಂ ಸಮರ್ಪಯಾಮಿ ॥}\\
ತ್ರಿಪುರಸುಂದರ್ಯೈ ನಮಃ । \as{ನಮಸ್ಕಾರಾನ್ ಸಮರ್ಪಯಾಮಿ ॥}

\as{ಓಂ ಕುಬೇರಾಯ ವೈ ಶ್ರವಣಾಯ, ಮಹಾರಾಜಾಯ ನಮಃ ॥}\\
ಕಾಲ್ಯೈ ನಮಃ॥\as{ ಛತ್ರಂ ಧಾರಯಾಮಿ॥}

\as{ಓಂ ತದಪ್ಯೇಷಃ ಶ್ಲೋಕೋಭಿಗೀತೋ ಮರುತಃ ಪರಿವೇಷ್ಟಾರೋ ಮರುತ್ತಸ್ಯಾವಸನ್ ಗೃಹೇ॥}\\
ಸುಮಾಲಿನ್ಯೈ ನಮಃ॥\as{ ಚಾಮರೇಣ ವೀಜಯಾಮಿ॥}
\newpage
\as{ಓಂ ಸಹಸ್ರಮಾಖ್ಯಾತ್ರೇ ದದ್ಯಾಚ್ಛತಂ ಪ್ರತಿಗರಿತ್ರ ಏತೇ ಚೈವಾಽಽಸನೇ॥}\\
ಮಹಾಸರಸ್ವತ್ಯೈ ನಮಃ॥\as{ ಆಂದೋಲಿಕಾಮಾರೋಹಯಾಮಿ॥}

\as{ಓಂ ಶ್ರಿಯ ಏವೈನಂ ತಚ್ಛ್ರಿಯಮಾದಧಾತಿ ಸಂತತಮೃಚಾ ವಷಟ್ಕೃತ್ಯಂ ಸಂತತ್ಯೈ ಸಂಧೀಯತೇ॥}\\ 
 ಲೋಕಮಾತ್ರೇ ನಮಃ॥\as{ಗೀತಂ ಗಾಯಾಮಿ॥}

\as{ಓಂ ಪರ್ಯಾಪ್ತ್ಯಾ ಅನಂತರಾಯಾಯ ಸರ್ವಸ್ತೋಮೋತಿ ರಾತ್ರ ಉತ್ತಮಮಹರ್ಭವತಿ॥}\\
 ಮೃಡಾನ್ಯೈ ನಮಃ॥\as{ನಾಟ್ಯಂ ನಟಾಮಿ॥}

\as{ಓಂ ಸಾಮ್ರಾಜ್ಯಂ ಭೌಜ್ಯಂ ಸ್ವಾರಾಜ್ಯಂ ವೈರಾಜ್ಯಂ ಪಾರಮೇಷ್ಟ್ಯಂ ರಾಜ್ಯಂ ಮಹಾರಾಜ್ಯಮಾಧಿಪತ್ಯಮಯಂ॥}\\
 ಮಹಾಲಕ್ಷ್ಮ್ಯೈ ನಮಃ॥\as{ ವಾದ್ಯಂ ವಾದಯಾಮಮಿ॥}

\as{ಓಂ ಆವಿಕ್ಷಿತಸ್ಯ ಕಾಮಪ್ರೇರ್ವಿಶ್ವೇದೇವಾಃ ಸಭಾಸದ ಇತಿ॥}\\
ನಿತ್ಯಸ್ವರೂಪಿಣ್ಯೈ ನಮಃ॥\as{ ವ್ಯಜನೇನ ವೀಜಯಾಮಿ॥}

\as{ಓಂ ಪೃಥಿವ್ಯೈ ಸಮುದ್ರಪರ್ಯಂತಾಯಾ ಏಕರಾಳಿತಿ॥}\\
ಜ್ಞಾನದಾಯಿನ್ಯೈ ನಮಃ॥\as{ ದರ್ಪಣಂ ದರ್ಶಯಾಮಿ॥}

\as{ಓಂ ಶ್ವೇತಶ್ಚಾಶ್ವತರೀರಥೋ ಹೋತುಃ  ಪುತ್ರಕಾಮಾ ಹಾಪ್ಯಾಖ್ಯಾಪಯೇರನ್ ಲಭಂತೇ ಹ ಪುತ್ರಾನ್ ಲಭಂತೇ ಹ ಪುತ್ರಾನ್॥}\\
ಸಾವಿತ್ರ್ಯೈ ನಮಃ॥\as{ ಅಶ್ವಮಾರೋಹಯಾಮಿ॥}

\as{ಓಂ ಪ್ರಜಯಾ ಪಶುಭಿರ್ಯ ಏವಂ ವೇದ॥}\\
ಮಾಹೇಂದ್ರ್ಯೈ ನಮಃ॥\as{ ಗಜಮಾರೋಹಯಾಮಿ॥}
\newpage
\as{ಓಂ ಅನೂರಾಧಾನ್ ಹವಿಷಾ ವರ್ಧಯಂತಃ । \\ಶತಂ ಜೀವೇವ ಶರದಃ ಸವೀರಾಃ॥}\\
ಬ್ರಾಹ್ಮ್ಯೈ ನಮಃ॥\as{ ಹಂಸಮಾರೋಹಯಾಮಿ॥}

\as{ಓಂ ದಿವಂ ಚ ಪೃಥಿವೀಂ ಚಾಂತರಿಕ್ಷಮಥೋ ಸುವಃ ॥}\\
ವೈಷ್ಣವ್ಯೈ ನಮಃ ॥ \as{ಗರುಡಮಾರೋಹಯಾಮಿ ॥}

\as{ಓಂ ಸಮಂತ ಪರ್ಯಾಯಾಸ್ವಂತರ ಪರಾದಾತ್ ॥}\\
ಮಾಹೇಶ್ವರ್ಯೈ ನಮಃ ॥\as{ವೃಷಭಮಾರೋಹಯಾಮಿ ॥}

\as{ಓಂ ಏತೇನ ವೈ ದೇವಾ ಜೈತ್ವಾನಿ ಜಿತ್ವಾ ॥}\\
ಕೌಮಾರ್ಯೈ ನಮಃ ॥\as{ಮಯೂರಮಾರೋಹಯಾಮಿ ॥}

\as{ಓಂ ಜೀವಾತ್ವೈ ಪುಣ್ಯಾಯ ।}\\
ಜಾಹ್ನವ್ಯೈ ನಮಃ॥\as{ಮಕರಮಾರೋಹಯಾಮಿ ॥}

\as{ಓಂ ಉರುದುಹಾನ್ ಯಜಮಾನಾಯ ಯಜ್ಞಮ್ ॥}\\
ಸ್ವಾಹಾಯೈ ನಮಃ ॥\as{ಮೃಗಮಾರೋಹಯಾಮಿ ॥}

\as{ಓಂ ಪೃಥುರಥೋ ದಕ್ಷಿಣಯಾ ಅಯೋಜೀಃ ॥}\\
ಚಂಡಿಕಾಯೈ ನಮಃ ॥\as{ರಥಮಾರೋಹಯಾಮಿ ॥}

\as{ಓಂ ಸರ್ವಸ್ಯಾಪ್ತ್ಯೈ ಸರ್ವಸ್ಯ ಜಿತ್ಯೈ ಸರ್ವಮೇವ \\ತೇನಾಪ್ನೋತಿ ಸರ್ವಂ ಜಯತಿ ॥}\\
ಮಾಹೇಶ್ವರ್ಯೈ ನಮಃ ॥ \as{ಸಮಸ್ತ ರಾಜೋಪಚಾರ ದೇವೋಪಚಾರಾನ್ ಸಮರ್ಪಯಾಮಿ ॥}
\newpage
\section{ಅಷ್ಟಾಂಗನಮಸ್ಕಾರಾಃ}
ಪ್ರಭಾಯೈ ನಮಃ । ಉರಸಾ ನಮಸ್ಕರೋಮಿ ॥\\
ಮಾಯಾಯೈ ನಮಃ । ಶಿರಸಾ ನಮಸ್ಕರೋಮಿ ॥\\
ಜಯಾಯೈ ನಮಃ । ದೃಷ್ಟ್ಯಾ ನಮಸ್ಕರೋಮಿ ॥\\
ಸೂಕ್ಷ್ಮಾಯೈ ನಮಃ । ಮನಸಾ ನಮಸ್ಕರೋಮಿ ॥\\
ವಿಶುದ್ಧಾಯೈ ನಮಃ । ವಚಸಾ ನಮಸ್ಕರೋಮಿ ॥\\
ನಂದಿನ್ಯೈ ನಮಃ । ಪದ್ಭ್ಯಾಂ ನಮಸ್ಕರೋಮಿ ॥\\
ಸುಪ್ರಭಾಯೈ ನಮಃ । ಕರಾಭ್ಯಾಂ ನಮಸ್ಕರೋಮಿ ॥\\
ವಿಜಯಾಯೈ ನಮಃ । ಕರ್ಣಾಭ್ಯಾಂ ನಮಸ್ಕರೋಮಿ ॥\\
ಸರ್ವಸಿದ್ಧಿಪ್ರದಾಯೈ ನಮಃ । ಸಾಷ್ಟಾಂಗಂ ನಮಸ್ಕರೋಮಿ ॥
\section{ಪ್ರಾರ್ಥನಾ}
ಜ್ಞಾನತೋಽಜ್ಞಾನತೋ ವಾಪಿ ಯನ್ನ್ಯೂನಮಧಿಕಂ ಕೃತಮ್ ।\\
ತ್ವರಯಾ ಯತ್ಪ್ರಮಾದಾದ್ವಾ ತತ್ಸರ್ವಂ ಕ್ಷಮ್ಯತಾಂ ತ್ವಯಾ ॥

ಅಪರಾಧಶತಂ ಕೃತ್ವಾ ಜಗದಂಬೇತಿ ಚೋಚ್ಚರೇತ್ ।\\
ಯಾಂ ಗತಿಂ ಸಮವಾಪ್ನೋತಿ ನ ತಾಂ ಬ್ರಹ್ಮಾದಯಃ ಸುರಾಃ ॥

ಸಾಪರಾಧೋಽಸ್ಮಿ ಶರಣಂ ಪ್ರಾಪ್ತಸ್ತ್ವಾಂ ಜಗದಂಬಿಕೇ ।\\
ಇದಾನೀಮನುಕಂಪ್ಯೋಹಂ ಯಥೇಚ್ಛಸಿ ತಥಾ ಕುರು ॥

ಅಜ್ಞಾನಾದ್ವಿಸ್ಮೃತೇರ್ಭ್ರಾಂತ್ಯಾ ಯನ್ನ್ಯೂನಮಧಿಕಂ ಕೃತಮ್ ।\\
ತತ್ಸರ್ವಂ ಕ್ಷಮ್ಯತಾಂ ದೇವಿ ಪ್ರಸೀದ ಪರಮೇಶ್ವರಿ ॥

ಕಾಮೇಶ್ವರಿ ಜಗನ್ಮಾತಃ ಸಚ್ಚಿದಾನಂದ ವಿಗ್ರಹೇ ।\\
ಗೃಹಾಣಾರ್ಚಾಮಿಮಾಂ ಪ್ರೀತ್ಯಾ ಪ್ರಸೀದ ಪರಮೇಶ್ವರಿ ॥

ಸರ್ವರೂಪಮಯೀ ದೇವೀ ಸರ್ವಂ ದೇವೀಮಯಂ ಜಗತ್ ।\\
ಅತೋಽಹಂ ವಿಶ್ವರೂಪಾಂ ತ್ವಾಂ ನಮಾಮಿ ಪರಮೇಶ್ವರಿ ॥

ಯದಕ್ಷರಪದಭ್ರಷ್ಟಂ ಮಂತ್ರಹೀನಂ ಚ ಯದ್ಭವೇತ್ ।\\
ಪೂರ್ಣಂ ಭವತು ತತ್ಸರ್ವಂ ತ್ವತ್ಪ್ರಸಾದಾತ್ ಸುರೇಶ್ವರಿ ॥

ಪ್ರಸೀದ ಭಗವತ್ಯಂಬ ಪ್ರಸೀದ ಭಕ್ತವತ್ಸಲೇ ।\\
ಪ್ರಸಾದಂ ಕುರು ಮೇ ಶೀಘ್ರಂ ದುರ್ಗೇ ದೇವಿ ನಮೋಽಸ್ತು ತೇ ॥

ಯಸ್ಯ ಸ್ಮೃತ್ಯಾ ಚ ನಾಮೋಕ್ತ್ಯಾ  ತಪಃ ಪೂಜಾ ಕ್ರಿಯಾ ದಿಷು ।\\
ನ್ಯೂನಂ ಸಂಪೂರ್ಣತಾಂ ಯಾತಿ ಸದ್ಯೋ ವಂದೇ ತಮಚ್ಯುತಂ ॥

ಮಂತ್ರಹೀನಂ ಕ್ರಿಯಾಹೀನಂ ಭಕ್ತಿಹೀನಂ ಮಹೇಶ್ವರಿ ।\\
ಯತ್ಪೂಜಿತಂ ಮಯಾ ದೇವಿ ಪರಿಪೂರ್ಣಂ ತದಸ್ತು ಮೇ ॥

ವರ್ತಮಾನೇ ವ್ಯಾವಹಾರಿಕೇ****** ಶುಭತಿಥೌ ** ಕಾಲೇ 
ಮಯಾ ಕೃತ ಸ್ಕಾಂದಪುರಾಣೋಕ್ತ ನವರಾತ್ರಾಂಗ ಪ್ರತಿಪದ್ದಿನ 
ಪೂಜಾರಾಧನೇನ ಶ್ರೀ ಮಹಾಕಾಳೀ ಮಹಾಲಕ್ಷ್ಮೀ ಮಹಾಸರಸ್ವತ್ಯಾತ್ಮಿಕಾ ಶ್ರೀ ದುರ್ಗಾಪರಮೇಶ್ವರೀ ಪ್ರೀಯತಾಮ್ ॥
\section{ಕುಮಾರೀ ಪೂಜಾ}
ಪ್ರಾಣಾನಾಯಮ್ಯ ಸಂಕಲ್ಪ್ಯ....\\
ಮಮಾಭೀಷ್ಟ ಸಿದ್ಧ್ಯರ್ಥಂ ಶ್ರೀ ಮಹಾಕಾಳೀ ಮಹಾಲಕ್ಷ್ಮೀ ಮಹಾಸರಸ್ವತ್ಯಾತ್ಮಕ ಶ್ರೀ ದುರ್ಗಾಪರಮೇಶ್ವರೀ ಪ್ರೀತ್ಯರ್ಥಂ ಆಚರಿತ ವ್ರತ ಸಂಪೂರ್ಣತಾವಾಪ್ತ್ಯರ್ಥಂ ಕುಮಾರೀ ಪೂಜನಂ ಕರಿಷ್ಯೇ ॥\\
(ಕೌಮಾರೀ ಶಾಂಭವೀ ದುರ್ಗಾ ಚಂಡಿಕಾ ಕಾಳಿಕಾಂಬಿಕಾ ।\\
ಕಲ್ಯಾಣೀ ರೋಹಿಣೀ ಗೌರೀ ನವದುರ್ಗಾಃ ಪ್ರಕೀರ್ತಿತಾಃ ॥)\\
೧\as{ಮಂತ್ರಾಕ್ಷರಮಯೀಂ ಲಕ್ಷ್ಮೀಂ ಮಾತೃಕಾರೂಪಧಾರಿಣೀಮ್ ।\\
ನವದುರ್ಗಾತ್ಮಿಕಾಂ ಸಾಕ್ಷಾತ್ ಕನ್ಯಾಮಾವಾಹಯಾಮ್ಯಹಮ್ ॥}\\
ಓಂ ಭಗವತೀಂ ಕೌಮಾರೀಂ ಆವಾಹಯಾಮಿ ॥\\
೨\as{ಮಂತ್ರಾಕ್ಷರಮಯೀಂ+++++ಕನ್ಯಾಮಾವಾಹಯಾಮ್ಯಹಮ್ ॥}\\
ಓಂ ಭಗವತೀಂ ಶಾಂಭವೀಂ ಆವಾಹಯಾಮಿ ॥\\
೩\as{ಮಂತ್ರಾಕ್ಷರಮಯೀಂ+++++ಕನ್ಯಾಮಾವಾಹಯಾಮ್ಯಹಮ್ ॥}\\
ಓಂ ಭಗವತೀಂ ದುರ್ಗಾಂ ಆವಾಹಯಾಮಿ ॥\\
೪\as{ಮಂತ್ರಾಕ್ಷರಮಯೀಂ+++++ಕನ್ಯಾಮಾವಾಹಯಾಮ್ಯಹಮ್ ॥}\\
ಓಂ ಭಗವತೀಂ ಚಂಡಿಕಾಂ ಆವಾಹಯಾಮಿ ॥\\
೫\as{ಮಂತ್ರಾಕ್ಷರಮಯೀಂ+++++ಕನ್ಯಾಮಾವಾಹಯಾಮ್ಯಹಮ್ ॥}\\
ಓಂ ಭಗವತೀಂ ಕಾಳಿಕಾಂ ಆವಾಹಯಾಮಿ ॥\\
೬\as{ಮಂತ್ರಾಕ್ಷರಮಯೀಂ+++++ಕನ್ಯಾಮಾವಾಹಯಾಮ್ಯಹಮ್ ॥}\\
ಓಂ ಭಗವತೀಂ ಅಂಬಿಕಾಂ ಆವಾಹಯಾಮಿ ॥\\
೭\as{ಮಂತ್ರಾಕ್ಷರಮಯೀಂ+++++ಕನ್ಯಾಮಾವಾಹಯಾಮ್ಯಹಮ್ ॥}\\
ಓಂ ಭಗವತೀಂ ಕಲ್ಯಾಣೀಂ ಆವಾಹಯಾಮಿ ॥\\
೮\as{ಮಂತ್ರಾಕ್ಷರಮಯೀಂ+++++ಕನ್ಯಾಮಾವಾಹಯಾಮ್ಯಹಮ್ ॥}\\
ಓಂ ಭಗವತೀಂ ರೋಹಿಣೀಂ ಆವಾಹಯಾಮಿ ॥\\
೯\as{ಮಂತ್ರಾಕ್ಷರಮಯೀಂ+++++ಕನ್ಯಾಮಾವಾಹಯಾಮ್ಯಹಮ್ ॥}\\
ಓಂ ಭಗವತೀಂ ಗೌರೀಂ ಆವಾಹಯಾಮಿ ॥

ತತಃ ಧ್ಯಾನಾದ್ಯಪಚಾರಾನ್ ಸಮರ್ಪ್ಯ ವಸ್ತ್ರ ಫಲ ಮಂಗಳದ್ರವ್ಯಾದಿಯುಕ್ತಂ ಉಪಾಯನಂ ದತ್ವಾ ನೀರಾಜನಂ ಸಮರ್ಪ್ಯ ಪ್ರಾರ್ಥಯೇತ್ ॥
\newpage
\section{ದುರ್ಗಾಪದುದ್ಧಾರಸ್ತವರಾಜಃ}
ನಮಸ್ತೇ ಶರಣ್ಯೇ ಶಿವೇ ಸಾನುಕಂಪೇ\\ ನಮಸ್ತೇ ಜಗದ್ವ್ಯಾಪಿಕೇ ವಿಶ್ವರೂಪೇ ।\\
ನಮಸ್ತೇ ಜಗದ್ವಂದ್ಯಪಾದಾರವಿಂದೇ\\ ನಮಸ್ತೇ ಜಗತ್ತಾರಿಣಿ ತ್ರಾಹಿ ದುರ್ಗೇ ॥ ೧॥

ನಮಸ್ತೇ ಜಗಚ್ಚಿಂತ್ಯಮಾನಸ್ವರೂಪೇ\\ ನಮಸ್ತೇ ಮಹಾಯೋಗಿವಿಜ್ಞಾನರೂಪೇ ।\\
ನಮಸ್ತೇ ನಮಸ್ತೇ ಸದಾನಂದ ರೂಪೇ\\ ನಮಸ್ತೇ ಜಗತ್ತಾರಿಣಿ ತ್ರಾಹಿ ದುರ್ಗೇ ॥ ೨॥

ಅನಾಥಸ್ಯ ದೀನಸ್ಯ ತೃಷ್ಣಾತುರಸ್ಯ\\ ಭಯಾರ್ತಸ್ಯ ಭೀತಸ್ಯ ಬದ್ಧಸ್ಯ ಜಂತೋಃ ।\\
ತ್ವಮೇಕಾ ಗತಿರ್ದೇವಿ ನಿಸ್ತಾರಕರ್ತ್ರೀ\\ ನಮಸ್ತೇ ಜಗತ್ತಾರಿಣಿ ತ್ರಾಹಿ ದುರ್ಗೇ ॥ ೩॥

ಅರಣ್ಯೇ ರಣೇ ದಾರುಣೇ ಶುತ್ರುಮಧ್ಯೇ\\ ಜಲೇ ಸಂಕಟೇ ರಾಜಗೇಹೇ ಪ್ರವಾತೇ ।\\
ತ್ವಮೇಕಾ ಗತಿರ್ದೇವಿ ನಿಸ್ತಾರ ಹೇತುಃ\\ನಮಸ್ತೇ ಜಗತ್ತಾರಿಣಿ ತ್ರಾಹಿ ದುರ್ಗೇ ॥ ೪॥

ಅಪಾರೇ ಮಹದುಸ್ತರೇಽತ್ಯಂತಘೋರೇ\\ ವಿಪತ್ ಸಾಗರೇ ಮಜ್ಜತಾಂ ದೇಹಭಾಜಾಂ ।\\
ತ್ವಮೇಕಾ ಗತಿರ್ದೇವಿ ನಿಸ್ತಾರನೌಕಾ \\ನಮಸ್ತೇ ಜಗತ್ತಾರಿಣಿ ತ್ರಾಹಿ ದುರ್ಗೇ ॥ ೫॥

ನಮಶ್ಚಂಡಿಕೇ ಚಂಡದುರ್ದಂಡಲೀಲಾ\\ಸಮುತ್ಖಂಡಿತಾ ಖಂಡಲಾಶೇಷಶತ್ರೋಃ ।\\
ತ್ವಮೇಕಾ ಗತಿರ್ವಿಘ್ನಸಂದೋಹಹರ್ತ್ರೀ\\ ನಮಸ್ತೇ ಜಗತ್ತಾರಿಣಿ ತ್ರಾಹಿ ದುರ್ಗೇ ॥ ೬॥

ತ್ವಮೇಕಾ ಸದಾರಾಧಿತಾ ಸತ್ಯವಾದಿನ್ಯನೇಕಾಖಿಲಾ\\ ಕ್ರೋಧನಾ ಕ್ರೋಧನಿಷ್ಠಾ ।\\
ಇಡಾ ಪಿಂಗಲಾ ತ್ವಂ ಸುಷುಮ್ನಾ ಚ ನಾಡೀ\\  ನಮಸ್ತೇ ಜಗತ್ತಾರಿಣಿ ತ್ರಾಹಿ ದುರ್ಗೇ ॥ ೭॥

ನಮೋ ದೇವಿ ದುರ್ಗೇ ಶಿವೇ ಭೀಮನಾದೇ\\ ಸದಾಸರ್ವಸಿದ್ಧಿಪ್ರದಾತೃಸ್ವರೂಪೇ ।\\
ವಿಭೂತಿಃ ಸತಾಂ ಕಾಲರಾತ್ರಿಸ್ವರೂಪೇ\\ ನಮಸ್ತೇ ಜಗತ್ತಾರಿಣಿ ತ್ರಾಹಿ ದುರ್ಗೇ ॥ ೮॥

ಶರಣಮಸಿ ಸುರಾಣಾಂ ಸಿದ್ಧವಿದ್ಯಾಧರಾಣಾಂ\\
     ಮುನಿಮನುಜಪಶೂನಾಂ ದಸ್ಯುಭಿಸ್ತ್ರಾಸಿತಾನಾಂ ।\\
ನೃಪತಿಗೃಹಗತಾನಾಂ ವ್ಯಾಧಿಭಿಃ ಪೀಡಿತಾನಾಂ\\
     ತ್ವಮಸಿ ಶರಣಮೇಕಾ ದೇವಿ ದುರ್ಗೇ ಪ್ರಸೀದ ॥ ೯॥

ಇದಂ ಸ್ತೋತ್ರಂ ಮಯಾ ಪ್ರೋಕ್ತಮಾಪದುದ್ಧಾರಹೇತುಕಂ ।\\
ತ್ರಿಸಂಧ್ಯಮೇಕಸಂಧ್ಯಂ ವಾ ಪಠನಾದ್ಘೋರಸಂಕಟಾತ್ ॥ ೧೦॥

ಮುಚ್ಯತೇ ನಾತ್ರ ಸಂದೇಹೋ ಭುವಿ ಸ್ವರ್ಗೇ ರಸಾತಲೇ ।\\
ಸರ್ವಂ ವಾ ಶ್ಲೋಕಮೇಕಂ ವಾ ಯಃ ಪಠೇದ್ಭಕ್ತಿಮಾನ್ ಸದಾ ॥ ೧೧॥

ಸ ಸರ್ವ ದುಷ್ಕೃತಂ ತ್ಯಕ್ತ್ವಾ ಪ್ರಾಪ್ನೋತಿ ಪರಮಂ ಪದಂ ।\\
ಪಠನಾದಸ್ಯ ದೇವೇಶಿ ಕಿಂ ನ ಸಿದ್ಧ್ಯತಿ ಭೂತಲೇ ।\\
ಸ್ತವರಾಜಮಿದಂ ದೇವಿ ಸಂಕ್ಷೇಪಾತ್ಕಥಿತಂ ಮಯಾ ॥ ೧೨॥
\authorline{ಇತಿ ಶ್ರೀಸಿದ್ಧೇಶ್ವರೀತಂತ್ರೇ ಉಮಾಮಹೇಶ್ವರಸಂವಾದೇ ಶ್ರೀದುರ್ಗಾಪದುದ್ಧಾರಸ್ತೋತ್ರಮ್॥}
\begin{center}{\LARGE\bfseries ಸದ್ಗುರುಚರಣಾರವಿಂದಾರ್ಪಣಮಸ್ತು\\ಓಂ ತತ್ಸತ್\\********}\end{center}
