\documentclass[12pt,a4paper]{report}
\input setphd
\begin{document}
\begin{center}
\thispagestyle{empty}
{\fontsize{25pt}{27pt}\fontseries{bx}\selectfont वाक्यपदीयस्य वाक्यकाण्डे पाश्चात्यविज्ञाने च वाक्यविज्ञानस्य तौलनिकमध्ययनम्}\\
\vfill
{\large नव्यव्याकरणविभागे {\bf "विद्यावारिधि"(PhD) }उपाध्यर्थं\\ कर्णाटकसंसकृतविश्वविद्यालये समर्पितः\\}
\vfill
{\Large\bfseries शोधप्रस्तावः\\Research Proposal}
\vfill

\begin{tabular}{ c @{\hspace{2cm}}c  }
Guide/मार्गदर्शिका & Researcher/अनुसन्धाता\\
{\bf Prof.\ Shivani } & {\bf Srinivasa K.L.}\\
{\bf प्रो. शिवानी } & {\bf श्रीनिवासः के. एल्}.\\
विभागाध्यक्षा, व्याकरणविभागः& शोधच्छात्रः, नव्यव्याकरणविभागः \\
कर्णाटकसंस्कृतविश्वविद्यालयः & कर्णाटकसंस्कृतविश्वविद्यालयः\\
बेङ्गलूरु - 560018 & बेङ्गलूरु - 560018
\end{tabular}
\vfill
\begin{figure}[h]
\centering
\includegraphics[width=5cm]{KSU_Logo}

\end{figure}

{\Large\bfseries कर्णाटकसंस्कृतविश्वविद्यालयः\\
 Karnataka Samskrit University\\}
Pampa Mahakavi Road, Chamarajapete\\
Bengaluru, Karnataka - 560018\\
{\bf 2022}

%\author{श्रीनिवासः के. एल्	}
%\date{ }
%\maketitle
\end{center} 
\newpage
%\linespread{2.5}
\begin{spacing}{2}
\fontsize{14pt}{16pt}\selectfont
\input  Synopsis
\end{spacing}
\end{document}
