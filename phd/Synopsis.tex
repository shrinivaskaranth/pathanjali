\tableofcontents
\newpage
श्रीगुरुभ्यो नमः~॥ \hfill                                                      श्रीगणेशाय नमः~॥
\section*{\LARGE\centering उपोद्घातः}
\addcontentsline{toc}{section}{उपोद्घातः}
नमः पाणिनिकात्यायन-पतञ्जलिभ्यः शब्दविद्यासम्प्रदायकर्तृभ्यो
नमो वंशर्षिभ्यो नमो महद्भ्यो नमो गुरुभ्यः॥
"वेदोऽखिलो धर्ममूलम्" इति वचनाद् ज्ञायते यत् समस्तेऽस्मिन् भरतवर्षे धर्मव्यवस्था वेदवृक्षाश्रितेति~। तस्य च ऋग्यजुःसामाथर्वेति चतुर्धा विभक्तस्य संरक्षणं शिक्षा- व्याकरण-च्छन्दो-निरुक्त-ज्यौतिष-कल्पेति षड्भिरङ्गैरकारि~। स्पष्टं चैतत् - षडङ्गविदः तत्तथा दधीमहे \footnote{गो.ब्रा.(१-२७)} इति गोपथब्राह्मणोक्त्या~। स्मर्यते च पाणिनीय शिक्षायां "छन्दः पादौ तु वेदस्य हस्तौ कल्पोऽथ पठ्यते~। ज्योतिषामयनं चक्षुर्निरुक्तं श्रोत्रमुच्यते~॥ शिक्षा घ्राणन्तु वेदस्य मुखं व्याकरणं स्मृतम्~। तस्मात् साङ्गमधीत्यैव ब्रह्मलोके महीयते~॥"\footnote{(पा.शि. ४१-४२)}
अत्रोक्तेष्वङ्गेषु व्याकरणस्य मुखस्थानीयत्वमुक्तम् , येन सर्वो व्यवहारः शब्दप्रधानको विशुद्धः सफलत्वमाप्नुयात्~। पस्पशाह्निके भाष्यकारैरपि 'प्रधानं च षट्स्वङ्गेषु व्याकरणम्' इति व्याकरणस्य प्राधान्यमुपापादि~।

\section*{व्याकरणशास्त्रोपक्रमः तद्भेदाः तस्याध्येतव्यता च~।}
\addcontentsline{toc}{section}{व्याकरणशास्त्रोपक्रमः तद्भेदाः तस्याध्येतव्यता च~।}
विदितमेवैतदस्माभिरिन्द्रियमनोगोचरेऽस्मिन् जगतीतले नामरूपाभ्यामेव व्यवहारो वरीवर्तीति~। तत्र वाच्यवाचकभावरूपेण विभक्तयोः शब्दार्थयोः मध्ये सत्यामप्यशक्यगणनायामर्थानां तदपेक्षया शब्दानां बहुल्यं व्यवहारतो जानीमहे, यतो हि एकस्याप्यर्थस्य प्रतिपत्तये नैकान् शब्दान् प्रयुञ्जते लोकाः~। तथा च निरुक्ते जलपर्यायप्रकथनसन्दर्भे "इत उत्तरमेकशतमुदकनामानि" इति ग्रन्थकाराः~। सत्येवं वयं शब्दानां आनन्त्यं निश्चप्रचं वक्तुं शक्नुमः~। शब्दानन्त्यं प्रतिपिपादयिषवो भाष्यकाराः निदर्शनमेकं उदाहार्षुः -
"बृहस्पतिरिन्द्राय दिव्यं वर्षसहस्रं प्रतिपदोक्तानां शब्दानां शब्दपारायणं प्रोवाच~। नान्तं जगाम~। बृहस्पतिश्च प्रवक्ता~। इन्द्रश्चाध्येता~। दिव्यवर्षसहस्रमध्ययनकालः~। नान्तं जगाम~। किं पुनरद्यत्वे ?"\footnote{पस्पशायाम्} इति~।

एवमनियमितानां शब्दानां नानारूपाणि अपशब्देभ्यः पृथक्कृत्य नियमेष्वासञ्जनमित्येतत् सामान्यानां गगनकुसुमायितमिति झटिति मतिपथमारोहति~। अतः महता प्रयत्नेन तपोबलेन च पूर्वसूरिभिरैन्द्रादिव्याकरणानि प्रावर्तिषत~। तथाच श्रूयते -"ऐन्द्रं चान्द्रं काशकृत्स्नं कौमारं शाकटायनम्~। सारस्वतं चापिशलं शाकलं पाणिनीयकम्~॥" इति~। तेष्वन्यतमं पाणिनीयं व्याकरणं प्रशस्ततमं विराजते~। अष्टभिरध्यायैः द्वात्रिंशत्पादैश्च सङ्ग्रथितः ग्रन्थोऽयं ३९८३ सू्त्रैः, गणपठितैः २००० धातुभिः, शास्त्रीयकार्यार्थं सङ्गृहीतैः नानाशब्दैश्च सम्मिलितः सन् , सत्स्वपि व्याकरणान्तरेषु राकाशशाङ्कवत् सर्वतेजोऽभिभावी प्रकाशते~। व्याकरणशास्त्रं प्रवर्तयिषुणा पाणिनिना, तपसा महेश्वरं सन्तोष्य, तदनुग्रहेण चतुर्दशसूत्रीमुपलभ्य विरचितोऽष्टाध्यायीत्यभिधः संसकृतवाङ्मये बृहत्तमः सूत्रग्रन्थोऽयं कात्यायनप्रणीतैः सहस्राधिकैः वार्तिकैः, पतञ्जलिमहर्षिप्रणीतेन भाष्येण चालमकारि~। तथाच शब्दशास्त्रमिदं त्रिमुनिव्याकरणमित्येव प्रसिद्धिपथमारूढम्~। माहेश्वरसूत्रमूलकत्वात् पाणिनीयस्यैव वेदाङ्गत्वमिति नागेशोऽभिप्रैति~। तथा चोक्तं शेखरे - "श्रुतिमूलकत्वादस्यैव वेदाङ्गत्वम्" इति~। 

लोके - "शक्तिग्रहं व्याकरणोपमान-कोशाप्तवाक्याद्व्यवहारतश्च~। वाक्यस्य शेषाद्विवृतेर्वदन्ति सान्निध्यतः सिद्धपदस्य वृद्धाः~॥" इति प्रसिद्धेषु शक्तिग्रहोपायेष्वन्यतमस्य व्याकरणस्य परमप्रयोजनन्तु शब्दसाधुत्वप्रतिपादनम्~। अत एवोक्तं हरिणा - "साधुत्वज्ञानविषया सैषा व्याकरणस्मृतिः~॥" इति~। एवं प्रकृतिप्रत्ययविभागपूर्वकसाधुत्वज्ञानपूर्वकप्रयोगे धर्मनियमः इति च भाष्यकाराः~। उक्तं चाभियुक्तैः "एकः शब्दः सम्यग्ज्ञातः सुप्रयुक्तः स्वर्गे लोके कामधुग्भवति" इति~। अतः "ब्राह्मणेन निष्कारणं षडङ्गो वेदोऽध्येयो ज्ञेयश्च" इति महाभाष्यानुवचनेन च वेदाङ्गत्वाद्व्याकरणमध्येयम्~।
\vspace{-15pt}
\section*{व्याकरणपरम्परा अध्ययनोपकारकाः ग्रन्थाश्च~।}
\addcontentsline{toc}{section}{व्याकरणपरम्परा अध्ययनोपकारकाः ग्रन्थाश्च~।}
\vspace{-15pt}
पस्पशायां भाष्ये "लक्ष्यं लक्षणं चैतत्समुदितं व्याकरणं भवति" इति वचनात् शब्दानां लक्षणद्वारा व्युत्पत्तिप्रदर्शनेन तेषां साधुत्वं प्रतिपिपादयिषितमस्मिञ्छ्रास्त्रे इति विदितम्~। सत्येवं शास्त्रस्यास्य त्रिधा प्रसिद्धा परम्परा चकास्ति~।
\vspace{-45pt}
\begin{itemize}
\itemsep=0pt
\item सूत्रात्मकपरम्परा
\item प्रक्रियात्मकपरम्परा
\item दार्शनिकपरम्परा 
\end{itemize}
चेति~।

\subsection*{सूत्रात्मकपरम्परा}
\addcontentsline{toc}{subsection}{सूत्रात्मकपरम्परा}
सूत्रात्मकपरम्परायां पाणिनीयाष्टाध्याय्यां सूत्रक्रममनुसृत्य व्याख्यानं प्रवृत्तम्~। अस्यां च परम्परायां वामनजयादित्याभ्यां विरचिता काशिका मूर्धन्यस्थानमारोहति~। ग्रन्थेस्मिऽस्मिन् महाभाष्यस्य सारमुद्धृत्य अनुवृत्तिः वृत्तिः उदाहरणानि प्रत्युदाहरणानि शङ्कासमाधानपुरःसरं स्पष्टतया न्यरूपिषत~। महाभाष्ये विद्यमानां संवादशैलीं विहाय सिद्धान्त एव न्यरूपि, येन बालानां शास्त्रोपक्रमे सौकर्यं लभ्येत~। एवमेवास्यां परम्परायां प्रथमावृत्तिः शब्दकौस्तुभः इत्यादयो ग्रन्थाः चकासते~।

\subsection*{प्रक्रियात्मकपरम्परा}
\addcontentsline{toc}{subsection}{प्रक्रियात्मकपरम्परा}
परन्तु अधीतेष्वपि एषु ग्रन्थेषु सत्यपि सूत्रार्थपरिज्ञाने लक्ष्यसंस्कारकाले तत्र तत्र विक्षिप्तानां त्तत्तद्विधिबोधकानां सूत्राणामेकत्र समन्वयोऽध्येतॄणां दुष्करः समजनि~। किञ्चास्यां काशिकायां तत्र तत्र भाष्यानभिमतं विपरिवृत्तसूत्रस्वरूपं व्याख्यानं च दृश्यते~। अतः भाष्यमतमनुसृत्य सामप्रदायिकसूत्रपाठसमन्विताः प्रक्रियाकौमुदी, रूपावतारः, प्रक्रियारत्नं, रूपमाला, प्रक्रियादीपिका, सिद्धान्तकौमुदीत्यादयो ग्रन्थाः प्राणायिषत~। तत्र सम्प्रति निर्दुष्टवृत्युदाहरणाद्युपेता भट्टोजीदीक्षितैर्विरचिता सिद्धान्तकौमुदी मूर्धन्यस्थानं भजते~। अत्र लक्ष्यानुरोधं प्रक्रियानिर्वाहाय सूत्रक्रमे परिवृतेऽपि ग्रन्थोऽयं सञ्ज्ञा-परिभाषा-शब्दाधिकारादि-प्रकरणमनुसृत्य शास्त्रमधिजिगांसूनां सुकुमारमतीनां सौकर्यातिशयं वितन्वन् अन्वर्थनामा चकास्ति~। महाभाष्यसिद्धान्तावलम्बनपूर्वकः अष्टाध्यायीसूत्रव्याख्यानभूतोऽयं ग्रन्थो विदुषां प्रीतिपात्रतामावहति। अत एवोच्यते-\\
\vspace{-15pt}
\begin{verse}	 कौमुदी यदि कण्ठस्था वृथा भाष्ये परिश्रमः।\\
	 कौमुदी यदि न कण्ठस्था वृथा भाष्ये परिश्रमः॥इति॥
\end{verse}
\subsection*{दार्शनिकपरम्परा~।}
\addcontentsline{toc}{subsection}{दार्शनिकपरम्परा~।}
उपर्युक्तपरम्पराद्वयव्यरिक्तपरम्पराऽपि प्रवृत्ता यत्र व्याकरणशास्त्रनिहितदार्शनिकविचाराः उपलभ्यन्ते~। भारतीयानां मेधाविनां प्रज्ञाशक्त्या, विचारपरतया, आमूलाग्रशोधबुद्ध्या च परम्परेयं पुष्टिमावहति~। अत्र च शब्दो नाम कः? तस्य स्वरूपं किं ? शब्दादर्थप्रतीतिः कथम्? शब्दार्थयोः सम्बन्धः कः? सम्बन्धश्चायं कथमस्मद्बुद्धिग्राह्यः ? शब्दप्रतिपाद्याः अर्थाः के? पदवाक्यभाषाणामुत्पत्तिः कीदृशी? शब्देषु विद्यमानानां अक्षराणां स्थानं किम्? व्याकरणशास्त्रपरिभाषितानां द्रव्यगुणजातिक्रियाकालकारकाणां तेषां परस्परसम्बन्धस्यापि निर्णयः कथम्? श्रुते शब्दे श्रोतुः जायमानस्य शाब्दज्ञानस्याकारः कः ? इत्यादयो विचाराः व्याकरणदर्शने विचार्यन्ते~। जीवब्रह्मणोः जगन्मोक्षयोश्च विचारः पाणिनिसूत्रेषु सूक्ष्मतया विमर्शितोऽपि व्याडेः लक्षश्लोकात्मके सङ्ग्रहनामके ग्रन्थे विस्तृततया निरूपित इति श्रूयते~। अयमेव च शास्त्रस्यास्य दार्शनिकदृष्ट्या विमर्शकः प्रथमो ग्रन्थः, उदाहृतश्च भाष्यकारैः पस्पशायां - "सङ्ग्रहे एतत्प्राधान्येन परीक्षितं नित्यः शब्दः कार्यो वेति~।	तत्र त्वेष निर्णयः~।" इति~। कालगर्भे ग्रन्थोऽयं विनष्टः इति तु दौर्भाग्यमेवास्तिकानाम्~। उक्तं च हरिणा -
\begin{verse}
प्रायेण संक्षेपरुचीन् अल्पविद्यापरिग्रहान्~।\\
संप्राप्य वैयाकरणान् संग्रहे ऽस्तम् उपागते~॥

कृते ऽथ पतञ्जलिना गुरुणा तीर्थदर्शिना~।\\
सर्वेषां न्यायबीजानां महाभाष्ये निबन्धने~॥

अलब्धगाधे गाम्भीर्याद् उत्तान इव सौष्ठवात्~।\\
तस्मिन्नकृतबुद्धीनाम् नैवावास्थितनिश्चयः~॥

वैजिसौभवहर्यक्षैः शुष्कतर्कानुसारिभिः~।\\
आर्षे विप्लाविते ग्रन्थे संग्रहप्रतिकञ्चुके~॥

यः पातञ्जलिशिष्येभ्यो भ्रष्टो व्याकरणागमः~।\\
कालेन दाक्षिणात्येषु ग्रन्थमात्रे व्यवस्थितः~॥

पर्वताद् आगमं लब्ध्वा भाष्यबीजानुसारिभिः~।\\
स नीतो बहुमार्गत्वं चन्द्राचार्यादिभिः पुनः~॥

न्यायप्रस्थानमार्गांस्तान् अभ्यस्य स्वं च दर्शनम्~।\\
प्रणीतो गुरुणास्माकमयमागमसंग्रहः~॥\footnote{वा.प.२काण्डसमाप्तौ}
\end{verse}
अत्र 'पर्वतादागमं लब्ध्वा' इति विषये पर्वतेषु निहितादिति वा पर्वतेत्याचार्याद्वेति विमर्शकानामभिप्रायभेदो दृश्यते~। एवं शब्दतत्वं सङ्गृह्य केनचित् क्रमेणाबद्ध्य व्यवस्थापितः सम्प्रत्युपलभ्यमानः ग्रन्थः भर्तृहरिविरचितः वाक्यपदीयाभिधः~। इममेव ग्रन्थमनुसृत्य मण्डनमिश्रस्य 'स्फोटसिद्धिः' कौण्डभट्टस्य 'वैयाकरणभूषणसारः' नागेशस्य 'वैयाकरणसिद्धान्तलघुमञ्जूषा' इति ग्रन्थाः विरचिताः~।

\vspace{-15pt}
\section*{भर्तृहरिः वाक्यपदीयञ्च} 
\addcontentsline{toc}{section}{भर्तृहरिः वाक्यपदीयञ्च}
\vspace{-15pt}
उत्तरोत्तरं मुनीनां प्रामाण्यमिति त्रिमुनिव्याकरणे महाभाष्यकाराणां स्थानं महद्वैशिष्ट्यं वहति~। तदनन्तरं  व्याख्याकाराणां मध्ये अतितमां प्राधान्यं वहति भर्तृहरिः~। अत एव नैके व्याख्यातारः भगवानित्युपाधिना भर्तृहरिं उल्लिखन्ति~। चन्द्राचार्यः वसुरातश्च काश्मीरदेश वासिनाविति श्रूयते~। तथा च वसुरातशिष्यो भर्तृहरिरपि काश्मीराभिजनः स्यात्~। अस्य कालस्तु ३५०तमात् ईशवीयाब्दात् प्रागिति विदुषामभिप्रायः~। 

भर्तृहरिणा प्रणीतेषु नैकेषु ग्रन्थेषु वाक्यपदीयं विद्वल्लोके माननीयतां प्राप्नोत्~। व्याकरणशास्त्रसम्बद्धः दार्शनिकोऽयं ग्रन्थः इति महावैयाकरणैः जिनेन्द्रबुद्धि-भट्टोजीदीक्षित-नागेशादिभिः स्वेषु ग्रन्थेषु तत्र तत्र प्रमाणतया वाक्यपदीयश्लोका उपनिबद्धाः दृश्यन्ते~। किञ्च महामाहेश्वरः अभिनवगुप्तोऽपि ईश्वरप्रत्यभिज्ञाग्रन्थेष्वपि अत्रत्यान् श्लोकान् उदाहरति इत्यस्य महद्वैशिष्ट्यं ज्ञायते~। 

अनुष्टुब्वृत्ते उपनिबद्धो ग्रन्थोऽयं ब्रह्मकाण्डं वाक्यकाण्डं पदकाण्डमिति काण्डत्रयात्मकः~। प्रथमकाण्डंं भूमिकारूपम्, द्वितीयतृतीययोः पदवाक्यविचाराः विमृष्टाः~। तथाच वाक्यपदमधिकृत्य विरचितः इति "अधिकृत्य कृते ग्रन्थे" इत्यधिकारीयेण "शिशुक्रन्दयमसभद्वन्द्वेन्द्रजननादिभ्यश्छः" इति सूत्रेण छप्रत्यये वाक्यपदीयमित्यस्य नाम~।

\vspace{-15pt}
\section*{वाक्यपदीये वाक्यकाण्डम्}
\addcontentsline{toc}{section}{वाक्यपदीये वाक्यकाण्डम्}
\vspace{-15pt}
काण्डेऽस्मिन् पदवाक्ययोः विवेचनं प्रारभ्यते ।  अत्र प्रतिपादितविचाराणां एवं संग्रहः कर्तुं शक्यः । वाक्यस्वरूपे मतभेदः, वाक्यस्फोटः, अभिहितान्वयवादः, अन्विताभिधानवादः, प्रतिभायाः वाक्यार्थत्वं, वाक्ये पदविभागस्य काल्पनिकत्वं, पदापेक्षया वाक्यादर्थप्रतीतिविचारः, पदविभागाशयः, शब्दवाच्यार्थविचारः, शब्दस्यापि वाच्यार्थत्वं, लोके अर्थप्राधान्यं, साङ्ख्यजैनबौद्धादिदर्शनेषु शब्दस्वरूपम्, वाक्यार्थभूतप्रतिभायाः विभागः, प्रत्यक्षापेक्षया शब्दादेवार्थपरिच्छेदः, नामपदेषु प्रकृतिप्रत्ययार्थविचारः, वाचकत्वद्योतकत्वविचारः, स्वार्थिकप्रत्ययविश्लेषणम् , निपातार्थविचारः, समासार्थविचारः, विभिन्नपक्षाश्रयणम् , शब्दार्थयोः गौणमुख्यविचारः, बौद्धपदार्थविमर्शः,नानार्थकशब्देषु अर्थनिर्णयोपायः, उत्सर्गापवादस्थले वाक्यार्थनिरूपणम् , प्रवृत्तिनिमित्तभेदात् शब्दभेदविचारः, समुदाये वाक्यपरिसमाप्तिः, अखण्डवाक्यस्य वाचकत्वं, वाक्ये क्रियाप्रधानत्वं, क्रियालक्षणम् , क्रियास्थले जातिव्यक्तिविचारः, इत्यादयो विचाराः काण्डेस्मिन् निरूपिताः ।

\section*{प्रकृतसंशोधनविचारः}
\addcontentsline{toc}{section}{प्रकृतसंशोधनविचारः}
भारतीयानां प्रज्ञाशक्तेः आमूलचूडं अध्ययनशीलतायाः निकषोपलमिव नैकेषु संशोधनेषु सत्स्वपि पाश्चात्यजीवनशैल्याः प्रभावेण आङ्ग्लविद्याभ्यासवशादपि अस्मदीयानां तत्र अनादरः समजनीति कारणात् तेषां संशोधनानां प्रकाश एव न जातः । सन्दर्भेऽस्मिन् परकीयानां संशोधनानां प्रसिद्धिश्च सुकराऽभूत्  इति भाषासौकर्याभावात् तेषां तदवगमनमपि दुष्करमिति  च हेतुना अस्माकं संशोधनकार्याणि अप्रकाशितान्येवातिष्ठन् । सत्येवं पाश्चिमात्यसंशोधनक्रमस्य भारतीयसंशोधनपद्धतेश्च तुलनात्मकाध्ययनेन अस्मिन्क्षेत्रे विमर्शस्य अवकाशो विद्यते इति धिया , वाक्यविज्ञाने पाश्चात्यभारतीयदृष्टिकोनयोः अभिप्रायभेदः सङ्ग्रहीतुमैच्छम् । तेन च अस्माकं परम्परायाः श्रेष्ठत्वं प्रकाशितं च भवति ।किंच यथा हरिणोक्तं \\{\bf प्रज्ञा विवेकं लभते भिन्नैरागमदर्शनैः ।\\ कियद्वा शक्यमुन्नेतुं स्वतर्कमनुधावता ॥}\footnote{वा.प.२काण्डसमाप्तौ।}\\ तथा अस्माकं बुद्धिवैशद्यमपि लभ्येत । भगवतो भर्तृहरेः वाक्यपदीये वाक्यार्थविचारे सम्यगेव न्यरूपि , यतो हि नागेशादयो वैयाकरणाः अन्ये च  विद्वांसः नैकान् श्लोकान् इत एव उदाहरन्ति । अतः तत्र संशोधनं सफलं भवति विश्वसिमि ।
\vspace{-15pt}
\section*{सन्दर्भग्रन्थसूची}
\addcontentsline{toc}{section}{सन्दर्भग्रन्थसूची}
\begin{itemize}
\itemsep=0pt
\item वाक्यपदीयम्(काण्डद्वयोपेतं पुण्यराजकृतप्रकाशटीकायुतम्)
\item वाक्यपदीयम् (महामहोपाध्यायैः रङ्गनाथशर्मभिः सङ्गृहीतम्)
\item वैयाकरणसिद्धान्तलघुमञ्जूषा (नागेशभट्टविरचितः)
\item वैयाकरणभूषणसारः(कौण्डभट्टविरचितः)
\end{itemize}
\vspace{-15pt}
\section*{Online sources to be reffered}
\addcontentsline{toc}{section}{Online sources to be reffered}
\begin{itemize}
\itemsep=0pt
\item www.inflib.net
\item www.googlebook.com
\item www.archive.org
\item www.sasnkritdocuments.org
\item Wikipidia
\item Latex
\item Sanskrit 2003
\end{itemize}
