\chapter*{\center ಕವಚಾನಿ}
\section{ಗಾಯತ್ರೀಹೃದಯಂ }
\thispagestyle{empty}
ಅಸ್ಯ ಶ್ರೀ ಗಾಯತ್ರೀಹೃದಯಸ್ಯ ನಾರಾಯಣ ಋಷಿಃ~। ಗಾಯತ್ರೀ ಛಂದಃ~। ಪರಮೇಶ್ವರೀ ದೇವತಾ~। ಜಪೇ ವಿನಿಯೋಗಃ ॥ ಅಥ ನ್ಯಾಸಃ ॥\\
ದ್ಯೌರ್ಮೂರ್ಧ್ನಿ ದೈವತಂ~। ದಂತಪಂಕ್ತಾವಶ್ವಿನೌ~। ಉಭೇ ಸಂಧ್ಯೇ ಚೋಷ್ಠೌ~। ಮುಖಮಗ್ನಿಃ~। ಜಿಹ್ವಾ ಸರಸ್ವತೀ~। ಗ್ರೀವಾಯಾಂ ತು ಬೃಹಸ್ಪತಿಃ~। ಸ್ತನಯೋರ್ವಸವೋಽಷ್ಟೌ~। ಬಾಹ್ವೋರ್ಮರುತಃ~। ಹೃದಯೇ ಪರ್ಜನ್ಯಃ~। ಆಕಾಶಮುದರಂ~। ನಾಭಾವಂತರಿಕ್ಷಂ~। ಕಟ್ಯೋ ರಿಂದ್ರಾಗ್ನೀ~। ಜಘನೇ ವಿಜ್ಞಾನಘನಃ ಪ್ರಜಾಪತಿಃ~। ಕೈಲಾಸ ಮಲಯೇ ಊರೂ~। ವಿಶ್ವೇದೇವಾ ಜಾನ್ವೋಃ~। ಜಂಘಾಯಾಂ ಕೌಶಿಕಃ~। ಗುಹ್ಯಮಯನೇ~। ಊರೂ ಪಿತರಃ~। ಪಾದೌ ಪೃಥಿವೀ~। ವನಸ್ಪತ ಯೋಽಙ್ಗುಲೀಷು~। ಋಷಯೋ ರೋಮಾಣಿ~। ನಖಾನಿ ಮುಹೂರ್ತಾನಿ~। ಅಸ್ಥಿಷು ಗ್ರಹಾಃ~। ಅಸೃಙ್ಮಾಂಸಮೃತವಃ~। ಸಂವತ್ಸರಾ ವೈ ನಿಮಿಷಂ~। ಅಹೋರಾತ್ರಾವಾದಿತ್ಯಶ್ಚಂದ್ರಮಾಃ~। ಪ್ರವರಾಂ ದಿವ್ಯಾಂ ಗಾಯತ್ರೀಂ ಸಹಸ್ರನೇತ್ರಾಂ ಶರಣಮಹಂ ಪ್ರಪದ್ಯೇ~। ಓಂ ತತ್ಸವಿತುರ್ವರೇಣ್ಯಾಯ ನಮಃ~। ಓಂ ತತ್ಪೂರ್ವಾಜಯಾಯ ನಮಃ~। ತತ್ಪ್ರಾತರಾದಿತ್ಯಾಯ ನಮಃ~। ತತ್ಪ್ರಾತರಾದಿತ್ಯಪ್ರತಿಷ್ಠಾಯೈ ನಮಃ~। ಪ್ರಾತರಧೀಯಾನೋ ರಾತ್ರಿಕೃತಂ ಪಾಪಂ ನಾಶಯತಿ~। ಸಾಯಮಧೀಯಾನೋ ದಿವಸಕೃತಂ ಪಾಪಂ ನಾಶಯತಿ~। ಸಾಯಂಪ್ರಾತರಧೀಯಾನೋ ಅಪಾಪೋ ಭವತಿ~। ಸರ್ವತೀರ್ಥೇಷು ಸ್ನಾತೋ ಭವತಿ~। ಸರ್ವೈರ್ದೇವೈರ್ಜ್ಞಾತೋ ಭವತಿ~। ಅವಾಚ್ಯವಚನಾತ್ಪೂತೋ ಭವತಿ~। ಅಭಕ್ಷ್ಯಭಕ್ಷಣಾತ್ಪೂತೋ ಭವತಿ~। ಅಭೋಜ್ಯಭೋಜನಾತ್ಪೂತೋ ಭವತಿ~। ಅಚೋಷ್ಯ ಚೋಷಣಾತ್ಪೂತೋ ಭವತಿ~। ಅಸಾಧ್ಯಸಾಧನಾತ್ಪೂತೋ ಭವತಿ~। ದುಷ್ಪ್ರತಿಗ್ರಹಶತಸಹಸ್ರಾತ್ಪೂತೋ ಭವತಿ~। ಸರ್ವಪ್ರತಿಗ್ರಹಾತ್ಪೂತೋ ಭವತಿ~। ಪಂಕ್ತಿದೂಷಣಾತ್ಪೂತೋ ಭವತಿ~। ಅನೃತವಚನಾತ್ಪೂತೋ ಭವತಿ~। ಅಥಾಬ್ರಹ್ಮಚಾರೀ ಬ್ರಹ್ಮಚಾರೀ ಭವತಿ~। ಅನೇನ ಹೃದಯೇನಾಧೀತೇನ ಕ್ರತುಸಹಸ್ರೇಣೇಷ್ಟಂ ಭವತಿ~। ಷಷ್ಟಿಶತ ಸಹಸ್ರಗಾಯತ್ರ್ಯಾ ಜಪ್ಯಾನಿ ಫಲಾನಿ ಭವಂತಿ~। ಅಷ್ಟೌ ಬ್ರಾಹ್ಮಣಾನ್ಸಮ್ಯಗ್ಗ್ರಾಹಯೇತ್~। ತಸ್ಯ ಸಿದ್ಧಿರ್ಭವತಿ~। ಯ ಇದಂ ನಿತ್ಯಮಧೀಯಾನೋ ಬ್ರಾಹ್ಮಣಃ ಪ್ರಾತಃ ಶುಚಿಃ ಸರ್ವಪಾಪೈಃ ಪ್ರಮುಚ್ಯತ ಇತಿ~। ಬ್ರಹ್ಮಲೋಕೇ ಮಹೀಯತೇ~। ಇತ್ಯಾಹ ಭಗವಾನ್ ಶ್ರೀನಾರಾಯಣಃ ॥\\
\authorline{ಇತಿ ಶ್ರೀಮದ್ದೇವೀಭಾಗವತೇ ಮಹಾಪುರಾಣೇ ಗಾಯತ್ರೀಹೃದಯಂ ॥}
\section{ಗಾಯತ್ರೀ ಕವಚಂ}
ಅಸ್ಯ ಶ್ರೀ ಗಾಯತ್ತ್ರೀ ಕವಚಸ್ಯ ಬ್ರಹ್ಮವಿಷ್ಣುಮಹೇಶ್ವರಾಃ ಋಷಯಃ~। ಋಗ್ಯಜುಸ್ಸಾಮಾಥರ್ವಶ್ಛಂದಾಂಸಿ~। ಪರಬ್ರಹ್ಮರೂಪಾ ಗಾಯತ್ರೀ ದೇವತಾ~। ತದ್ಬೀಜಂ~। ಭರ್ಗಃ ಶಕ್ತಿಃ~। ಧಿಯಃ ಕೀಲಕಂ~। ಜಪೇ ವಿನಿಯೋಗಃ ॥

ಗಾಯತ್ರೀ ಪೂರ್ವತಃ ಪಾತು ಸಾವಿತ್ರೀ ಪಾತು ದಕ್ಷಿಣೇ~।\\
ಬ್ರಹ್ಮ ಸಂಧ್ಯಾ ತು ಮೇ ಪಶ್ಚಾದುತ್ತರಾಯಾಂ ಸರಸ್ವತೀ॥೧॥

ಪಾರ್ವತೀ ಮೇ ದಿಶಂ ರಾಕ್ಷೇ ತ್ಪಾವಕೀಂ ಜಲಶಾಯಿನೀ~।\\
ಯಾತುಧಾನೀಂ ದಿಶಂ ರಕ್ಷೇ ದ್ಯಾತುಧಾನಭಯಂಕರೀ॥೨॥

ಪಾವಮಾನೀಂ ದಿಶಂ ರಕ್ಷೇತ್ಪವಮಾನ ವಿಲಾಸಿನೀ~।\\
ದಿಶಂ ರೌದ್ರೀಂಚ ಮೇ ಪಾತು ರುದ್ರಾಣೀ ರುದ್ರ ರೂಪಿಣೀ॥೩॥

ಊರ್ಧ್ವಂ ಬ್ರಹ್ಮಾಣೀ ಮೇ ರಕ್ಷೇ ದಧಸ್ತಾ ದ್ವೈಷ್ಣವೀ ತಥಾ~।\\
ಏವಂ ದಶ ದಿಶೋ ರಕ್ಷೇ ತ್ಸರ್ವಾಂಗಂ ಭುವನೇಶ್ವರೀ॥೪॥

ತತ್ಪದಂ ಪಾತು ಮೇ ಪಾದೌ ಜಂಘೇ ಮೇ ಸವಿತುಃ ಪದಂ~।\\
ವರೇಣ್ಯಂ ಕಟಿ ದೇಶೇ ತು ನಾಭಿಂ ಭರ್ಗ ಸ್ತಥೈವ ಚ॥೫॥

ದೇವಸ್ಯ ಮೇ ತದ್ಧೃದಯಂ ಧೀಮಹೀತಿ ಚ ಗಲ್ಲಯೋಃ~।\\
ಧಿಯಃ ಪದಂ ಚ ಮೇನೇತ್ರೇ ಯಃ ಪದಂ ಮೇ ಲಲಾಟಕಂ॥೬॥

ನಃ ಪಾತು ಮೇ ಪದಂ ಮೂರ್ಧ್ನಿ ಶಿಖಾಯಾಂ ಮೇ ಪ್ರಚೋದಯಾತ್~।\\
ತತ್ಪದಂ ಪಾತು ಮೂರ್ಧಾನಂ ಸಕಾರಃ ಪಾತು ಫಾಲಕಂ॥೭॥

ಚಕ್ಷುಷೀ ತು ವಿಕಾರಾರ್ಣಸ್ತುಕಾರಸ್ತು ಕಪೋಲಯೋಃ~।\\
ನಾಸಾಪುಟಂ ವಕಾರಾರ್ಣೋ ರೇಕಾರಸ್ತು ಮುಖೇ ತಥಾ ॥೮॥

ಣಿಕಾರ ಊರ್ಧ್ವ ಮೋಷ್ಠಂತು ಯಕಾರಸ್ತ್ವಧರೋಷ್ಠಕಂ~।\\
ಆಸ್ಯಮಧ್ಯೇ ಭಕಾರಾರ್ಣೋ ರ್ಗೋಕಾರ ಶ್ಚುಬುಕೇ ತಥಾ ॥೯॥

ದೇಕಾರಃ ಕಂಠ ದೇಶೇತು ವಕಾರಃ ಸ್ಕಂಧ ದೇಶಕಂ~।\\
ಸ್ಯಕಾರೋ ದಕ್ಷಿಣಂ ಹಸ್ತಂ ಧೀಕಾರೋ ವಾಮ ಹಸ್ತಕಂ॥೧೦॥

ಮಕಾರೋ ಹೃದಯಂ ರಕ್ಷೇದ್ಧಿಕಾರ ಉದರೇ ತಥಾ~।\\
ಧಿಕಾರೋ ನಾಭಿ ದೇಶೇ ತು ಯೋಕಾರಸ್ತು ಕಟಿಂ ತಥಾ॥೧೧॥

ಗುಹ್ಯಂ ರಕ್ಷತು ಯೋಕಾರ ಊರೂ ದ್ವೌ ನಃ ಪದಾಕ್ಷರಂ~।\\
ಪ್ರಕಾರೋ ಜಾನುನೀ ರಕ್ಷೇಚ್ಚೋಕಾರೋ ಜಂಘ ದೇಶಕಂ॥೧೨॥

ದಕಾರಂ ಗುಲ್ಫ ದೇಶೇ ತು ಯಾಕಾರಃ ಪದಯುಗ್ಮಕಂ~।\\
ತಕಾರೋ ವ್ಯಂಜನಂ ಚೈವ ಸರ್ವಾಂಗಂ ಮೇ ಸದಾವತು ॥೧೩॥

ಇದಂ ತು ಕವಚಂ ದಿವ್ಯಂ ಬಾಧಾ ಶತ ವಿನಾಶನಂ~।\\
ಚತುಃಷಷ್ಟಿ ಕಲಾ ವಿದ್ಯಾದಾಯಕಂ ಮೋಕ್ಷಕಾರಕಂ॥೧೪॥

ಮುಚ್ಯತೇ ಸರ್ವ ಪಾಪೇಭ್ಯಃ ಪರಂ ಬ್ರಹ್ಮಾಧಿಗಚ್ಛತಿ~।\\
ಪಠನಾ ಚ್ಛ್ರವಣಾ ದ್ವಾಪಿ ಗೋ ಸಹಸ್ರ ಫಲಂ ಲಭೇತ್ ॥೧೫॥
\authorline{॥ಶ್ರೀ ದೇವೀಭಾಗವತಾಂತರ್ಗತಂ ಗಾಯತ್ರೀ ಕವಚಂ ಸಂಪೂರ್ಣಂ ॥}
\section{ಶ್ರೀ ತ್ರೈಲೋಕ್ಯಮೋಹನಕವಚಂ }
(ಉತ್ಕೀಲನಮಂತ್ರಂ ಪಠಿತ್ವಾ ಕವಚಂ ಪಠೇತ್)\\
ಓಂ ಹಸಕಲಹ್ರೀಂ ಹ್ರೀಂಹ್ರೀಂಹ್ರೀಂ ಸಕಲಹ್ರೀಂ ಕ್ಲೀಂಕ್ಲೀಂಕ್ಲೀಂ  ಶ್ರೀಂಶ್ರೀಂಶ್ರೀಂ ಕ್ರೀಂಕ್ರೀಂಕ್ರೀಂ ರುದ್ರಸೂಚ್ಯಗ್ರೇಣ ಮೂಲವಿದ್ಯಾಶಾಪಂ ಉತ್ಕೀಲಯೋತ್ಕೀಲಯ ಸ್ವಾಹಾ ॥\\
ಶ್ರೀ ದೇವ್ಯುವಾಚ ॥\\
ದೇವ ದೇವ ಜಗನ್ನಾಥ ಸಚ್ಚಿದಾನಂದವಿಗ್ರಹ~।\\
ಪಂಚಕೃತ್ಯ ಪರೇಶಾನ ಪರಮಾನಂದದಾಯಕ ॥

ಶ್ರೀಮತ್ತ್ರಿಪುರಸುಂದರ್ಯಾ ಯಾ ಯಾ ವಿದ್ಯಾಸ್ತ್ವಯೋದಿತಾಃ~।\\
ಕೃಪಯಾ ಕಥಿತಾಃ ಸರ್ವಾಃ ಶ್ರುತಾಶ್ಚಾಧಿಗತಾ ಮಯಾ ॥

ಪ್ರಾಣನಾಥಾಧುನಾ ಬ್ರೂಹಿ ಕವಚಂ ಮಂತ್ರವಿಗ್ರಹಂ~।\\
ತ್ರೈಲೋಕ್ಯಮೋಹನಂ ಚೇತಿ ನಾಮತಃ ಕಥಿತಂ ಪುರಾ~।\\
ಇದಾನೀಂ ಶ್ರೋತುಮಿಚ್ಛಾಮಿ ಸರ್ವಾರ್ಥಂ ಕವಚಂ ಸ್ಫುಟಂ ॥

ಈಶ್ವರ ಉವಾಚ ॥\\
ಶೃಣು ದೇವಿ ಪ್ರವಕ್ಷ್ಯಾಮಿ ಸುಂದರಿ ಪ್ರಾಣವಲ್ಲಭೇ~।\\
ತ್ರೈಲೋಕ್ಯಮೋಹನಂ ನಾಮ ಸರ್ವವಿದ್ಯೌಘವಿಗ್ರಹಂ ॥

ಯದ್ಧೃತ್ವಾ ದಾನವಾನ್ ವಿಷ್ಣುಃ ನಿಜಘಾನ ಮುಹುರ್ಮುಹುಃ~।\\
ಸೃಷ್ಟಿಂ ವಿತನುತೇ ಬ್ರಹ್ಮಾ ಯದ್ಧೃತ್ವಾ ಪಠನಾದ್ಯತಃ ॥

ಸಂಹರ್ತಾಹಂ ಯತೋ ದೇವಿ ದೇವೇಶೋ ವಾಸವೋ ಯತಃ~।\\
ಧನಾಧಿಪಃ ಕುಬೇರೋಽಪಿ ಯತಃ ಸರ್ವೇ ದಿಗೀಶ್ವರಾಃ ॥

ನ ದೇಯಂ ಯದಶಿಷ್ಯೇಭ್ಯೋ ದೇಯಂ ಶಿಷ್ಯೇಭ್ಯ ಏವ ಚ~।\\
ಅಭಕ್ತೇಭ್ಯೋಽಪಿ ಪುತ್ರೇಭ್ಯೋ ದತ್ವಾ ಮೃತ್ಯುಮವಾಪ್ನುಯಾತ್ ॥

ಶ್ರೀಮತ್ತ್ರಿಪುರಸುಂದರ್ಯಾಃ ಕವಚಸ್ಯ ಋಷಿಃ ಶಿವಃ~।\\
ಛಂದೋ ವಿರಾಡ್ ದೇವತಾ ಚ ಶ್ರೀಮತ್ತ್ರಿಪುರಸುಂದರೀ~।\\
ಧರ್ಮಾರ್ಥಕಾಮಮೋಕ್ಷೇಷು ವಿನಿಯೋಗಃ ಪ್ರಕೀರ್ತಿತಃ ॥\\
({\bfseries ಓಂಐಂಹ್ರೀಂಶ್ರೀಂ})\\
ಶಿರೋ ಮೇ ವಾಗ್ಭವಂ ಪಾತು ಕಏಈಲಹ್ರೀಂ ಸ್ವರೂಪಕಂ~।\\
ಹಸಕಲಹ್ರೀಂ ಲಲಾಟಂ ಚ ಪಾತು ಕಾಮೇಶ್ವರೀ ಮಮ~।\\
ಹಕಹಲಹ್ರೀಂ ದೃಶೌ ಚ ಪಾತು ಕಾಮೇಶಮಧ್ಯಮಂ~।\\
ಕಹಯಲಹ್ರೀಂ ಶ್ರುತೀ ತು ಪಾತು ಕಾಮಂ ತುರೀಯಕಂ~।\\
ಹಕಲಸಹ್ರೀಂ ಶಕ್ತ್ಯಾಖ್ಯಂ ಪಾತು ಜಿಹ್ವಾಂ ಚ ಪಂಚಮಂ~॥\\
ಕಏಈಲಹ್ರೀಂ ಹಸಕಲಹ್ರೀಂ ಹಕಹಲಹ್ರೀಂ\\
ಕಹಯಲಹ್ರೀಂ ಹಕಲಸಹ್ರೀಂ  ಪರಮಾತ್ಮರೂಪಿಣೀ~।\\
ವದನಂ ಸಕಲಂ ಪಾತು ಪಂಚಕೂಟೈಸ್ತು ಪಂಚಮೀ ॥

ಕಏಈಲಹ್ರೀಂ ಘ್ರಾಣಂ ಮೇ ಪಾತು ವಾಗ್ಭವಸಂಜ್ಞಕಂ~।\\
ಹಸಕಹಲಹ್ರೀಂ ಕಂಠಂ ಪಾತು ಕಾಮೇಶಸಂಜ್ಞಕಂ~।\\
ಸಕಲಹ್ರೀಂ ಶಕ್ತಿಕೂಟಂ ಸ್ಕಂಧೌ ಪಾತು ಸದಾ ಮಮ~॥\\
ಕಏಈಲಹ್ರೀಂ ಹಸಕಹಲಹ್ರೀಂ ಸಕಲಹ್ರೀಂ ಚ~।\\
ಕಾಮೇನೋಪಾಸಿತಾ ವಿದ್ಯಾ ಕಕ್ಷದೇಶೇ ಸದಾವತು ॥

ಕ್ಲೀಂಹ್ರೀಂಶ್ರೀಂಐಂಸೌಃ ಓಂಹ್ರೀಂಶ್ರೀಂ ಐಂಕ್ಲೀಂಸೌಃ\\ ಶ್ರೀಂಹ್ರೀಂಕ್ಲೀಂಐಂಸೌಃ॥\\
ಕಾಮಾದಿಷೋಡಶೀ ಪಾತು ಭುಜೌ ತ್ರಿಪುರಸುಂದರೀ ॥

ಓಂಕ್ಲೀಂಹ್ರೀಂಶ್ರೀಂ ಐಂಕ್ಲೀಂಸೌಃ  ಶ್ರೀಂಹ್ರೀಂಕ್ಲೀಂ\\ ಸ್ತ್ರೀಂಐಂಕ್ರೋಂಕ್ರೀಂಶ್ರೀಂಹೂಂ~।\\
ತಾರಾದಿಷೋಡಶೀ ಪಾತು ಮಣಿಬಂಧದ್ವಯಂ ತಥಾ ॥

ಶ್ರೀಂಹ್ರೀಂಐಂಕ್ಲೀಂಸೌಃ ಓಂಹ್ರೀಂಶ್ರೀಂ ಐಂಕ್ಲೀಂಸೌಃ\\ ಸೌಃಐಂಕ್ಲೀಂಹ್ರೀಂಶ್ರೀಂ~।\\
ರಮಾದಿಷೋಡಶೀ ಪಾತು ಕರೌ ತ್ರಿಪುರಸುಂದರೀ ॥

ಹ್ರೀಂಹ್ಸೌಃ ಓಂಐಂಹ್ರೀಂಶ್ರೀಂ ಐಂಕ್ಲೀಂಸೌಃ \\ಸ್ತ್ರೀಂಕ್ರೋಂಐಂಹ್ರೀಂ ಹೂಂಹೂಂಶ್ರೀಂ~।\\
ಮಾಯಾದಿಕಾ ತು ಹೃತ್ಪಾತು ಶ್ರೀಮತ್ತ್ರಿಪುರಸುಂದರೀ ॥

ಐಂಹ್ರೀಂಶ್ರೀಂಕ್ಲೀಂಸೌಃ ಐಂಕ್ಲೀಂಸೌಃ ಸೌಃಕ್ಲೀಂಐಂ ಸೌಃಕ್ಲೀಂಶ್ರೀಂಹ್ರೀಂಐಂ~।\\
ವಾಗಾದಿಷೋಡಶೀ ಪಾತು ಸ್ತನೌ ಮೇ ಸುಂದರೀ ಪರಾ ॥

ಕಏಶ್ರೀಂಕಏಈಲಹ್ರೀಂ ಕ್ಲೀಂಹಸಕಹಲಹ್ರೀಂ ಸೌಃಸಕಲಹ್ರೀಂ~।\\
ನಖವರ್ಣಾಖ್ಯವಿದ್ಯೇಯಂ ಪಾರ್ಶ್ವೌ ಪಾತ್ವಪರಾಜಿತಾ ॥

ಹ್ರೀಂಹ್ಸೌಂಸ್ಹೌಂ ಹ್ರೀಂಸ್ಹೌಂಹ್ಸೌಂ ಕ್ಲೀಂಹಸಕಹಲಹ್ರೀಂಹ್ರೀಂಹ್ರೀಂ\\ ಸೌಃಸೌಃ ಹಹಸಕಹಲಹ್ರೀಂ ಕ್ಲೀಂಹ್ಸೌಂಸ್ಹೌಂ ಹ್ರೀಂಸ್ಹೌಂಹ್ಸೌಂಹ್ರೀಂ~।\\
ಏಕತ್ರಿಂಶದ್ವರ್ಣರೂಪಾ ಮಹಾಪುರುಷಪೂಜಿತಾ ॥\\
ಮಹಾಗುಹ್ಯಸ್ವರೂಪಾ ಚ ಕೇವಲಾನಂದಚಿನ್ಮಯೀ~।\\
ಕಟಿದೇಶಂ ಸದಾ ಪಾತು ಪರಬ್ರಹ್ಮಸ್ವರೂಪಿಣೀ ॥

ಹಸಕಲಹ್ರೀಂ ಪೃಷ್ಠದೇಶೇ ದೇವೀರಕ್ಷತು ವೈ ಸದಾ~।\\
ಹಸಕಹಲಹ್ರೀಂ ಕುಕ್ಷಿದೇಶಂ ಮಹಾವಿದ್ಯಾ ಚ ಪಾತು ಮಾಂ~।\\
ಸಕಲಹ್ರೀಂ ಶಕ್ತಿಕೂಟಂ ಪಾತು ವಕ್ಷಸ್ಥಲಂ ಮಮ~॥\\
ಹಸಕಲಹ್ರೀಂ ಹಸಕಹಲಹ್ರೀಂ ಸಕಲಹ್ರೀಂ ಮೇ~।\\
ಲೋಪಾಮುದ್ರಾಪಂಚದಶೀ ಮಧ್ಯದೇಶಂ ಸದಾವತು ॥

ಕಹಏಈಲಹ್ರೀಂ ನಾಭಿಂ ಪಾತು ಹಕಏಈಲಹ್ರೀಂ ಕಟಿಂ ಪಾತು~।\\
ಸಕ್ಥಿನೀ ಮೇ ಸದಾ ಪಾತು ಸಕಏಈಲಹ್ರೀಂ ಸದಾ॥\\
ಕಹಏಈಲಹ್ರೀಂ ಹಕಏಈಲಹ್ರೀಂ ಸಕಏಈಲಹ್ರೀಂ ಮೇ~।\\
ವಸುಚಂದ್ರಾಮಾನವೀ ಮಾಂ ಸಾ ಸದಾ ಸರ್ವತೋಽವತು ॥

ಸಹಕಏಈಲಹ್ರೀಂ ಮೇ ಊರುಯುಗ್ಮಂ ಸದಾವತು~।\\
ಸಹಕಹಏಈಲಹ್ರೀಂ ಗುಹ್ಯಂ ಪಾತು ವರಪ್ರದಾ~।\\
ಹಸಕಏಈಲಹ್ರೀಂ ತು ಜಾನುನೀ ಪಾತು ಮೇ ಸದಾ~॥\\
ಸಹಕಏಈಲಹ್ರೀಂ ಸಹಕಹಏಈಲಹ್ರೀಂ ಹಸಕಏಈಲಹ್ರೀಂ~।\\
ಚಂದ್ರವಿದ್ಯಾ ಚ ಪಾತು ಮಾಂ ಪಕ್ಷಾದ್ಯಕ್ಷರವರ್ಣಕಾ~।\\
ಜಲಜೇ ಭಯಸಂಘಾತೇ ಸದಾ ಮಾಂ ಪರಿರಕ್ಷತು ॥

ಹಸಕಏಈಲಹ್ರೀಂ ಗುಲ್ಫಯುಗ್ಮಂ ಮಮ ವೈ ಸರ್ವದಾವತು~।\\
ಹಸಕಹಏಈಲಹ್ರೀಂ ಪಾದೌ ಪಾಯಾತ್ ಸನಾತನೀ~।\\
ಸಹಕಏಈಲಹ್ರೀಂ ಮೇ ಪ್ರಪದೌ ಪಾತು ಸರ್ವದಾ~॥\\
ಹಸಕಏಈಲಹ್ರೀಂ ಹಸಕಹಏಈಲಹ್ರೀಂ\\
ಸಹಕಏಈಲಹ್ರೀಂ ಸದಾ ಕುಬೇರೇಣ ಪ್ರಪೂಜಿತಾ~।\\
ದ್ವಾವಿಂಶತ್ಯಕ್ಷರೀ ವಿದ್ಯಾ ಸರ್ವಾಂಗಂ ಮೇ ಸದಾವತು ॥

ಕಏಈಲಹ್ರೀಂ ಪ್ರಾಚ್ಯಾಂ ತು ತ್ರಿಪುರಾ ಪರಿರಕ್ಷತು~।\\
ಹಸಕಹಲಹ್ರೀಂ ಪಾತು ವಹ್ನಿಕೋಣೇ ನಿರಂತರಂ~।\\
ಸಹಸಕಲಹ್ರೀಂ ಯಾಮ್ಯಾಂ ತು ಪಾತು ಮೇ ಸರ್ವಸಿದ್ಧಿದಾ~॥\\
ಕಏಈಲಹ್ರೀಂ ಹಸಕಹಲಹ್ರೀಂ ಸಹಸಕಲಹ್ರೀಂ ತು~।\\
ಅಗಸ್ತ್ಯವಿದ್ಯಾ ಸಾ ಸೇವ್ಯಾ ಚಕ್ರಸ್ಥಾ ಮಾಂ ಸದಾವತು ॥

ಸಏಈಲಹ್ರೀಂ ಚ ನಿತ್ಯಂ ನೈರೃತ್ಯಾಂ ಮಾಂ ಸದಾವತು~।\\
ಸಹಕಹಲಹ್ರೀಂ ಚೈವ ಪ್ರತೀಚ್ಯಾಂ ಪಾತು ಪಾರ್ವತೀ ॥\\
ಸಕಲಹ್ರೀಂ ತು ವಾಯವ್ಯೇ ಸದಾ ಮಾಂ ಪರಿರಕ್ಷತು~।\\
ಸಏಈಲಹ್ರೀಂ ಸಹಕಹಲಹ್ರೀಂ ಸಕಲಹ್ರೀಂ ತು~॥\\
ನಂದ್ಯಾರಾಧಿತವಿದ್ಯೇಯಂ ಸರ್ವಾಂಗೇ ಮಾಂ ಸದಾವತು ॥

ಹಸಕಲಹ್ರೀಂ ಉತ್ತರೇ ಚ ಪಾತು ಮಾಂ ಜಗದೀಶ್ವರೀ~।\\
ಸಹಕಲಹ್ರೀಂ ಈಶದಿಶಿ ಶಿವಪತ್ನೀ ಚ ಪಾತು ಮಾಂ~।\\
ಸಕಹಲಹ್ರೀಂ ಸುಂದರೀ ಊರ್ಧ್ವೇ ಮಾಂ ಪಾತು ಸರ್ವದಾ~॥\\
ಹಸಕಲಹ್ರೀಂ ಸಹಕಲಹ್ರೀಂ ಸಕಹಲಹ್ರೀಂ ಮಾಂ~।\\
ಅಧೋ ರಕ್ಷತು ಮೇ ನಿತ್ಯಂ ಸೂರ್ಯಪೂಜ್ಯಾ ಮಹೋದಯಾ ॥

ಕಏಈಲಹ್ರೀಂ ಹಸಕಹಲಹ್ರೀಂ ಸಕಲಹ್ರೀಂ ಮೇ~।\\
ಸರ್ವಾಂಗಂ ಶಕ್ರಸಂಪೂಜ್ಯಾ ಸತತಂ ಪರಿರಕ್ಷತು ॥

ಕಏಈಲಹ್ರೀಂ ಹಕಹಲಹ್ರೀಂ ಹಸಕಲಹ್ರೀಂ ಚ~।\\
ಬ್ರಹ್ಮಾಣೀ ಮಾಂ ಸದಾ ಪಾಯಾತ್ ಶ್ರೀಮತ್ತ್ರಿಪುರಸುಂದರೀ ॥

ಹಸಕಲಹ್ರೀಂ ಹಸಕಹಲಹ್ರೀಂ ಸಕಲಹ್ರೀಂ \\ಹಸಕಲ ಹಸಕಹಲ ಸಕಲಹ್ರೀಂ ಚ ಶಾಂಕರೀ~।\\
ಚತುಃಕೂಟಾ ಮಹಾವಿದ್ಯಾ ಪಾತಾಲೇ ಮಾಂ ಸದಾವತು ॥

ಹಸಕಲಹ್ರೀಂ ಆಧಾರಂ ಹಸಕಹಲಹ್ರೀಂ ಚ ಲಿಂಗಕೇ~।\\
ಸಕಲಹ್ರೀಂ ಪಾತು ನಾಭಿಂ ಸಹಕಲಹ್ರೀಂ ಅನಾಹತಂ ।\\
ಸಹಕಹಲಹ್ರೀಂ ಕಂಠಂ ಸಹಸಕಲಹ್ರೀಂ ತಥಾ~॥\\
ಹಸಕಲಹ್ರೀಂ ಹಸಕಹಲಹ್ರೀಂ ಸಕಲಹ್ರೀಂ \\ಸಹಕಲಹ್ರೀಂ ಸಹಕಹಲಹ್ರೀಂ ಸಹಸಕಲಹ್ರೀಂ~।\\
ಮನೋಭವಾ ಸದಾ ಪಾತು ರದಸಂಖ್ಯಾ ಮಹಾಪ್ರಭಾ~।\\
ಷಟ್ಕೂಟಾ ವೈಷ್ಣವೀ ಸಾ ವೈ ಪಾತು ಮಾಂ ಸುಂದರೀ ಪರಾ ॥

ಕಏಈಲಹರೀ ಹಸಕಹಲರೀ ಸಕಲಹರೀ~।\\
ದುರ್ವಾಸಸಾ ಪ್ರಪೂಜ್ಯಾ ಚ ದಿಕ್ಷು ವಿದ್ಯಾ ಸದಾವತು ॥

ಕಹಏಈಲಹ್ರೀಂ ಹಲಏಈಲಹ್ರೀಂ ಸಕಏಈಲಹ್ರೀಂ~।\\
ಕ್ರೋಧೇನ ಪೂಜಿತಾ ನಿತ್ಯಂ ವಿದಿಕ್ಷು ಪರಿರಕ್ಷತು ॥

ಹಸಕಲಹ್ರೀಂ ಹಸಕಹಹಲಹ್ರೀಂ ಸಕಲಹ್ರೀಂ~।\\
ಮಹಾಜ್ಞಾನಮಯೀ ಪಾತು ನಿತ್ಯಂ ಮಾಂ ಷೋಡಶೀ ಪರಾ ॥

ಶ್ರೀಂಹ್ರೀಂಕ್ಲೀಂಐಂಸೌಃ ಓಂಹ್ರೀಂಶ್ರೀಂ ಕಏಈಲಹ್ರೀಂ\\ ಹಸಕಹಲಹ್ರೀಂ ಸಕಲಹ್ರೀಂ ಸೌಃಐಂಕ್ಲೀಂಹ್ರೀಂಶ್ರೀಂ~।\\
ಸರ್ವಾಂಗಂ ಮೇ ಮಹಾವಿದ್ಯಾ ಬೀಜರೂಪಾ ಚ ಷೋಡಶೀ ॥

ಓಂ ಕ್ಲೀಂಹ್ರೀಂಶ್ರೀಂ ಐಂಕ್ಲೀಂಸೌಃ ಕಏಈಲಹ್ರೀಂ\\ಹಸಕಹಲಹ್ರೀಂ ಸಕಲಹ್ರೀಂ ಸ್ತ್ರೀಂಐಂಕ್ರೋಂಕ್ರೀಂ \\ಈಂ ಹೂಂ ಷೋಡಶಸ್ವರರೂಪಿಣಿ  ಶ್ರೀಮತ್ತ್ರಿಪುರಸುಂದರಿ \\ಹ್ರಾಂಹ್ರೀಂಹ್ರೂಂ ಫಟ್ ಸರ್ವಸಿದ್ಧಿಂ ಪ್ರಯಚ್ಛ ಪ್ರಯಚ್ಛ ಸ್ವಾಹಾ~।\\
ಶ್ರೀಮಹಾಷೋಡಶೀ ವಿದ್ಯಾನಾಖ್ಯಾತಾ ಭುವನತ್ರಯೇ~।\\
ಜ್ಞಾನೇನ ಮೃತ್ಯುಹಾ ಸಾ ಮಾಂ ಶಿರಸ್ಥಾ ಸರ್ವತೋಽವತು ॥\\
ಶ್ರೀಮಹಾಷೋಡಶೀ ಪೂರ್ಣಾ ಮಹಾದೇವೇನ ಪೂಜಿತಾ~।\\
ಯಸ್ಯಾ ವಿಜ್ಞಾನಮಾತ್ರೇಣ ಮೃತ್ಯೋರ್ಮೃತ್ಯುರ್ಭವೇತ್ಸ್ವಯಂ ॥

ಕ್ಲೀಂಐಂಶ್ರೀಂ ಕಏಈಲಹ್ರೀಂ ಹಸಕಹಲಹ್ರೀಂ ಸಕಲಹ್ರೀಂ~।\\
ಕಾಮವಾಗೀಶ್ವರೀ ಲಕ್ಷ್ಮೀಸ್ತ್ರಿಕೂಟಾ ಪರಮೇಶ್ವರೀ ॥

ಐಂ ಕಏಈಲಹ್ರೀಂ ಕ್ಲೀಂ ಹಸಕಹಲಹ್ರೀಂ ಸೌಃ ಸಕಲಹ್ರೀಂ\\ಸೋಹಂ ಹೌಂ ಹಂಸಃ ಹ್ರೀಂ ಸಕಲಹ್ರೀಂ ಸೌಃ \\ ಹಸಕಹಲಹ್ರೀಂ ಕ್ಲೀಂ ಕಏಈಲಹ್ರೀಂ ಐಂ ಬ್ರಹ್ಮಸ್ವರೂಪಿಣೀ~।\\
ನೇತ್ರವೇದಯುತೈರ್ವರ್ಣೈರ್ಯುತಾ ಸಾ ಸರ್ವತೋಽವತು~॥\\
ಬ್ರಹ್ಮಸ್ವರೂಪಿಣೀ ಚೇಯಂ ಪರಮಾನಂದಚಿದ್ಘನಾ~।\\
ಅಷ್ಟಾದಶಾಕ್ಷರೀ ವಿದ್ಯಾ ಸದಾ ಮಾಂ ಪರಿರಕ್ಷತು ॥({\bfseries ಶ್ರೀಂಹ್ರೀಂಐಂ})

ಇತಿ ತೇ ಕಥಿತಂ ದೇವಿ ಬ್ರಹ್ಮವಿದ್ಯಾಕಲೇಬರಂ~।\\
ತ್ರೈಲೋಕ್ಯಮೋಹನಂ ನಾಮ ಕವಚಂ ಬ್ರಹ್ಮರೂಪಕಂ ॥

ಸಪ್ತಲಕ್ಷಮಹಾವಿದ್ಯಾಃ ತಂತ್ರಾದೌ ಕಥಿತಾಃ ಪ್ರಿಯೇ~।\\
ತಾಸಾಂ ಸಾರಾತ್ಸಾರತಯಾ ಯಾ ಯಾ ವಿದ್ಯಾಃ ಸುಗೋಪಿತಾಃ ॥

ಬಹುನಾತ್ರಕಿಮುಕ್ತೇನ ಶ್ರೀಮಹಾಷೋಡಶೀ ಪರಾ~।\\
ಪ್ರಕಾಶಿತಾ ಮಯಾ ದೇವಿ ಯಾಂ ಪೃಚ್ಛಸಿ ಪುನಃ ಪುನಃ ॥

ಮಹಾವಿದ್ಯಾಮಯಂ ಬ್ರಹ್ಮಕವಚಂ ಮನ್ಮುಖೋದಿತಂ~।\\
ಗುರುಮಭ್ಯರ್ಚ್ಯ ವಿಧಿವತ್ ಕವಚಂ ಪ್ರಪಠೇತ್ತತಃ ॥

ದೇವೀಮಭ್ಯರ್ಚ್ಯ ವಿಧಿವತ್ ಪುರಶ್ಚರ್ಯಾಂ ಸಮಾಚರೇತ್~।\\
ಅಷ್ಟೋತ್ತರಶತಂ ಜಪ್ತ್ವಾ ದಶಾಂಶಂ ಹವನಾದಿಕಂ ॥

ತತಃ ಸುಸಿದ್ಧಕವಚಃ ಪುಣ್ಯಾತ್ಮಾ ಮದನೋಪಮಃ~।\\
ಮಂತ್ರಸಿದ್ಧಿರ್ಭವೇತ್ತಸ್ಯ ಪುರಶ್ಚರ್ಯಾಂ ವಿನಾ ತತಃ ॥

ಗದ್ಯಪದ್ಯಮಯೀ ವಾಣೀ ತಸ್ಯ ವಕ್ತ್ರಾತ್ ಪ್ರಜಾಯತೇ~।\\
ವಕ್ತ್ರೇ ತಸ್ಯವಸೇದ್ವಾಣೀ ಕಮಲಾ ನಿಶ್ಚಲಾ ಗೃಹೇ ॥

ಪುಷ್ಪಾಂಜಲ್ಯಷ್ಟಕಂ ದತ್ವಾ ಮೂಲೇನೈವ ಪಠೇತ್ ಸಕೃತ್~।\\
ಅಪಿ ವರ್ಷಸಹಸ್ರಾಣಾಂ ಪೂಜಾಫಲಮವಾಪ್ನುಯಾತ್ ॥

ಆತ್ಮಾನಂ ತನ್ಮಯಂ ಕೃತ್ವಾ ಯಃ ಪಠೇತ್ ಕವಚಂ ಪರಂ~।\\
ಯಂ ಯಂ ಪಶ್ಯತಿ ವೈ ಶೀಘ್ರಂ ಸ ಸ ದಾಸೋ ಭವೇದ್ಧೃವಂ ॥

ವಿಲಿಖ್ಯ ಭೂರ್ಜೇ ಘುಟಿಕಾಂ ಸ್ವರ್ಣಸ್ಥಾಂ ಧಾರಯೇದ್ಯದಿ~।\\
ಕಂಠೇ ವಾ ಯದಿ ವಾ ಬಾಹೌ ಸ ಕುರ್ಯಾದ್ದಾಸವಜ್ಜಗತ್ ॥

ತ್ರಿಲೋಕೀಂ ಕ್ಷೋಭಯತ್ಯೇವ ತ್ರೈಲೋಕ್ಯವಿಜಯೀ ಭವೇತ್~।\\
ತದ್ಗಾತ್ರಂ ಪ್ರಾಪ್ಯ ಶಸ್ತ್ರಾಣಿ ಬ್ರಹ್ಮಾಸ್ತ್ರಾದೀನಿ ಪಾರ್ವತಿ ॥

ಮಾಲ್ಯಾನಿ ಕುಸುಮಾನೀವ ಸುಖದಾನಿ ಭವಂತಿ ಹಿ~।\\
ಸ್ಪರ್ಧಾಂ ನಿರಸ್ಯ ಭವನೇ ಲಕ್ಷ್ಮೀರ್ವಾಣೀ ವಸೇತ್ತತಃ ॥

ಇದಂ ಕವಚಮಜ್ಞಾತ್ವಾ ಯೋ ಜಪೇತ್ಸುಂದರೀಂ ಪರಾಂ~।\\
ನವಲಕ್ಷಂ ಪ್ರಜಪ್ತ್ವಾಽಪಿ ತಸ್ಯ ವಿದ್ಯಾ ನ ಸಿದ್ಧ್ಯತಿ ॥

ಸ ಶಸ್ತ್ರಘಾತಮಾಪ್ನೋತಿ ಸೋಽಚಿರಾನ್ಮೃತ್ಯುಮಾಪ್ನುಯಾತ್~।\\
ಇದಮೇವ ಪರಂ ಯಸ್ಮಾದ್ ಭುಕ್ತಿಮುಕ್ತಿಪ್ರದಾಯಕಂ~।\\
ತಸ್ಮಾತ್ಸರ್ವಪ್ರಯತ್ನೇನ ಪಠನೀಯಂ ಮುುಮುಕ್ಷುಭಿಃ ॥
\authorline{ಇತಿ ಶ್ರೀರುದ್ರಯಾಮಲೇ ಗೌರೀಶ್ವರಸಂವಾದೇ ಶ್ರೀರಾಜರಾಜೇಶ್ವರೀ \\ಮಹಾತ್ರಿಪುರಸುಂದರ್ಯಾಃ ತ್ರೈಲೋಕ್ಯಮೋಹನಂ ನಾಮ ಕವಚಂ ಸಂಪೂರ್ಣಂ॥}
\section{ಶಿವಕವಚಂ }
ಅಸ್ಯ ಶ್ರೀ ಶಿವಕವಚ ಸ್ತೋತ್ರಮಹಾಮಂತ್ರಸ್ಯ ಋಷಭಯೋಗೀಶ್ವರ ಋಷಿಃ~। ಅನುಷ್ಟುಪ್ ಛಂದಃ~। ಶ್ರೀಸದಾಶಿವೋ ದೇವತಾ~। ಓಂ ಬೀಜಂ~। ನಮಃ ಶಕ್ತಿಃ~। ಶಿವಾಯೇತಿ ಕೀಲಕಂ~। ಸದಾಶಿವಪ್ರೀತ್ಯರ್ಥೇ ಜಪೇ ವಿನಿಯೋಗಃ~।

ಓಂ ನಮೋ ಭಗವತೇ ಜ್ವಲ ಜ್ವಲ ಮಹಾರುದ್ರಾಯ ಶ್ರೀಂ ಹ್ರೀಂ ಕ್ಲೀಂ(*)\\೧. * ಹ್ರಾಂ ಸರ್ವಶಕ್ತಿ ಧಾಮ್ನೇ
೨. * ನಂ ತೃಪ್ತಿಶಕ್ತಿಧಾಮ್ನೇ\\
೩. * ಮಂ ಅನಾದಿಬೋಧಶಕ್ತಿಧಾಮ್ನೇ
೪. * ಶಿಂ ಸ್ವತಂತ್ರಶಕ್ತಿಧಾಮ್ನೇ\\
೫. * ವಾಂ ಅಲುಪ್ತಶಕ್ತಿಧಾಮ್ನೇ
೬. * ಯಂ ಅನಂತಶಕ್ತಿಧಾಮ್ನೇ ಇತಿ ನ್ಯಾಸಃ

\dhyana{ವಜ್ರದಂಷ್ಟ್ರಂ ತ್ರಿನಯನಂ ಕಾಲಕಂಠಮರಿಂದಮಂ~।\\
ಸಹಸ್ರಕರಮತ್ಯುಗ್ರಂ ವಂದೇ ಶಂಭುಂ ಉಮಾಪತಿಂ ॥}

ಋಷಭ ಉವಾಚ ॥\\
ನಮಸ್ಕೃತ್ಯ ಮಹಾದೇವಂ ವಿಶ್ವವ್ಯಾಪಿನಮೀಶ್ವರಂ~।\\
ವಕ್ಷ್ಯೇ ಶಿವಮಯಂ ವರ್ಮ ಸರ್ವರಕ್ಷಾಕರಂ ನೃಣಾಂ ॥೧॥

ಶುಚೌ ದೇಶೇ ಸಮಾಸೀನೋ ಯಥಾವತ್ಕಲ್ಪಿತಾಸನಃ~।\\
ಜಿತೇಂದ್ರಿಯೋ ಜಿತಪ್ರಾಣಶ್ಚಿಂತಯೇಚ್ಛಿವಮವ್ಯಯಂ ॥೨॥

ಹೃತ್ಪುಂಡರೀಕಾಂತರಸನ್ನಿವಿಷ್ಟಂ\\ಸ್ವತೇಜಸಾ ವ್ಯಾಪ್ತನಭೋಽವಕಾಶಂ~।\\
ಅತೀಂದ್ರಿಯಂ ಸೂಕ್ಷ್ಮಮನಂತಮಾದ್ಯಂ\\ಧ್ಯಾಯೇತ್ ಪರಾನಂದಮಯಂ ಮಹೇಶಂ ॥೩॥

ಧ್ಯಾನಾವಧೂತಾಖಿಲಕರ್ಮಬಂಧ\\ಶ್ಚಿರಂ ಚಿದಾನಂದನಿಮಗ್ನಚೇತಾಃ~।\\
ಷಡಕ್ಷರನ್ಯಾಸ ಸಮಾಹಿತಾತ್ಮಾ\\ಶೈವೇನ ಕುರ್ಯಾತ್ಕವಚೇನ ರಕ್ಷಾಂ ॥೪॥

ಮಾಂ ಪಾತು ದೇವೋಽಖಿಲದೇವತಾತ್ಮಾ\\ಸಂಸಾರಕೂಪೇ ಪತಿತಂ ಗಭೀರೇ~।\\
ತನ್ನಾಮ ದಿವ್ಯಂ ಪರಮಂತ್ರಮೂಲಂ\\ಧುನೋತು ಮೇ ಸರ್ವಮಘಂ ಹೃದಿಸ್ಥಂ ॥೫॥

ಸರ್ವತ್ರ ಮಾಂ ರಕ್ಷತು ವಿಶ್ವಮೂರ್ತಿ\\ರ್ಜ್ಯೋತಿರ್ಮಯಾನಂದಘನಶ್ಚಿದಾತ್ಮಾ~।\\
ಅಣೋರಣೀಯಾನುರುಶಕ್ತಿರೇಕಃ\\ಸ ಈಶ್ವರಃ ಪಾತು ಭಯಾದಶೇಷಾತ್ ॥೬॥

ಯೋ ಭೂಸ್ವರೂಪೇಣ ಬಿಭರ್ತಿ ವಿಶ್ವಂ\\ಪಾಯಾತ್ಸ ಭೂಮೇರ್ಗಿರಿಶೋಽಷ್ಟಮೂರ್ತಿಃ~।\\
ಯೋಽಪಾಂ ಸ್ವರೂಪೇಣ ನೃಣಾಂ ಕರೋತಿ\\ಸಂಜೀವನಂ ಸೋಽವತು ಮಾಂ ಜಲೇಭ್ಯಃ ॥೭॥

ಕಲ್ಪಾವಸಾನೇ ಭುವನಾನಿ ದಗ್ಧ್ವಾ\\ಸರ್ವಾಣಿ ಯೋ ನೃತ್ಯತಿ ಭೂರಿಲೀಲಃ~।\\
ಸ ಕಾಲರುದ್ರೋಽವತು ಮಾಂ ದವಾಗ್ನೇ\\ರ್ವಾತ್ಯಾದಿಭೀತೇರಖಿಲಾಚ್ಚ ತಾಪಾತ್ ॥೮॥

ಪ್ರದೀಪ್ತವಿದ್ಯುತ್ಕನಕಾವಭಾಸೋ\\ವಿದ್ಯಾವರಾಭೀತಿಕುಠಾರಪಾಣಿಃ~।\\
ಚತುರ್ಮುಖಸ್ತತ್ಪುರುಷಸ್ತ್ರಿನೇತ್ರಃ\\ಪ್ರಾಚ್ಯಾಂ ಸ್ಥಿತೋ ರಕ್ಷತು ಮಾಮಜಸ್ರಂ ॥೯॥

ಕುಠಾರಖೇಟಾಂಕುಶಶೂಲಢಕ್ಕಾ\\ಕಪಾಲಪಾಶಾಕ್ಷಗುಣಾನ್ ದಧಾನಃ~।\\
ಚತುರ್ಮುಖೋ ನೀಲರುಚಿಸ್ತ್ರಿನೇತ್ರಃ\\ಪಾಯಾದಘೋರೋ ದಿಶಿ ದಕ್ಷಿಣಸ್ಯಾಂ ॥೧೦॥

ಕುಂದೇಂದುಶಂಖಸ್ಫಟಿಕಾವಭಾಸೋ\\ವೇದಾಕ್ಷಮಾಲಾವರದಾಭಯಾಂಕಃ~।\\
ತ್ರ್ಯಕ್ಷಶ್ಚತುರ್ವಕ್ತ್ರ ಉರುಪ್ರಭಾವಃ\\ಸದ್ಯೋಽಧಿಜಾತೋಽವತು ಮಾಂ ಪ್ರತೀಚ್ಯಾಂ ॥೧೧॥

 ವರಾಕ್ಷಮಾಲಾಭಯಟಂಕಹಸ್ತಃ\\ಸರೋಜಕಿಂಜಲ್ಕಸಮಾನವರ್ಣಃ~।\\
ತ್ರಿಲೋಚನಶ್ಚಾರುಚತುರ್ಮುಖೋ ಮಾಂ\\ಪಾಯಾದುದೀಚ್ಯಾಂ ದಿಶಿ ವಾಮದೇವಃ ॥೧೨॥
\newpage
ವೇದಾಭಯೇಷ್ಟಾಂಕುಶಟಂಕಪಾಶ\\ಕಪಾಲಢಕ್ಕಾಕ್ಷರಶೂಲಪಾಣಿಃ~।\\
ಸಿತದ್ಯುತಿಃ ಪಂಚಮುಖೋಽವತಾನ್ಮಾಂ\\ಈಶಾನ ಊರ್ಧ್ವಂ ಪರಮಪ್ರಕಾಶಃ ॥೧೩॥

ಮೂರ್ಧಾನಮವ್ಯಾನ್ಮಮ ಚಂದ್ರಮೌಲಿ\\ರ್ಭಾಲಂ ಮಮಾವ್ಯಾದಥ ಭಾಲನೇತ್ರಃ~।\\
ನೇತ್ರೇ ಮಮಾವ್ಯಾದ್ಭಗನೇತ್ರಹಾರೀ\\ನಾಸಾಂ ಸದಾ ರಕ್ಷತು ವಿಶ್ವನಾಥಃ ॥೧೪॥

ಪಾಯಾಚ್ಛ್ರುತೀ ಮೇ ಶ್ರುತಿಗೀತಕೀರ್ತಿಃ\\ಕಪೋಲಮವ್ಯಾತ್ಸತತಂ ಕಪಾಲೀ~।\\
ವಕ್ತ್ರಂ ಸದಾ ರಕ್ಷತು ಪಂಚವಕ್ತ್ರೋ\\ಜಿಹ್ವಾಂ ಸದಾ ರಕ್ಷತು ವೇದಜಿಹ್ವಃ ॥೧೫॥

ಕಂಠಂ ಗಿರೀಶೋಽವತು ನೀಲಕಂಠಃ\\ಪಾಣಿದ್ವಯಂ ಪಾತು ಪಿನಾಕಪಾಣಿಃ~।\\
ದೋರ್ಮೂಲಮವ್ಯಾನ್ಮಮ ಧರ್ಮಬಾಹು\\ರ್ವಕ್ಷಃಸ್ಥಲಂ ದಕ್ಷಮಖಾಂತಕೋಽವ್ಯಾತ್ ॥೧೬॥

ಮಮೋದರಂ ಪಾತು ಗಿರೀಂದ್ರಧನ್ವಾ\\ಮಧ್ಯಂ ಮಮಾವ್ಯಾನ್ಮದನಾಂತಕಾರೀ~।\\
ಹೇರಂಬತಾತೋ ಮಮ ಪಾತು ನಾಭಿಂ\\ಪಾಯಾತ್ಕಟಿಂ ಧೂರ್ಜಟಿರೀಶ್ವರೋ ಮೇ ॥೧೭॥
\newpage
ಊರುದ್ವಯಂ ಪಾತು ಕುಬೇರಮಿತ್ರೋ\\ಜಾನುದ್ವಯಂ ಮೇ ಜಗದೀಶ್ವರೋಽವ್ಯಾತ್~।\\
ಜಂಘಾಯುಗಂ ಪುಂಗವಕೇತುರವ್ಯಾತ್\\ಪಾದೌ ಮಮಾವ್ಯಾತ್ಸುರವಂದ್ಯಪಾದಃ ॥೧೮॥

ಮಹೇಶ್ವರಃ ಪಾತು ದಿನಾದಿಯಾಮೇ\\ಮಾಂ ಮಧ್ಯಯಾಮೇಽವತು ವಾಮದೇವಃ~।\\
ತ್ರಿಲೋಚನಃ ಪಾತು ತೃತೀಯಯಾಮೇ\\ವೃಷಧ್ವಜಃ ಪಾತು ದಿನಾಂತ್ಯಯಾಮೇ ॥೨೦॥

ಪಾಯಾನ್ನಿಶಾದೌ ಶಶಿಶೇಖರೋ ಮಾಂ\\ಗಂಗಾಧರೋ ರಕ್ಷತು ಮಾಂ ನಿಶೀಥೇ~।\\
ಗೌರೀಪತಿಃ ಪಾತು ನಿಶಾವಸಾನೇ\\ಮೃತ್ಯುಂಜಯೋ ರಕ್ಷತು ಸರ್ವಕಾಲಂ ॥೨೧॥

ಅಂತಃಸ್ಥಿತಂ ರಕ್ಷತು ಶಂಕರೋ ಮಾಂ\\ಸ್ಥಾಣುಃ ಸದಾ ಪಾತು ಬಹಿಃಸ್ಥಿತಂ ಮಾಂ~।\\
ತದಂತರೇ ಪಾತು ಪತಿಃ ಪಶೂನಾಂ\\ಸದಾಶಿವೋ ರಕ್ಷತು ಮಾಂ ಸಮಂತಾತ್ ॥೨೨॥

ತಿಷ್ಠಂತಮವ್ಯಾದ್ ಭುವನೈಕನಾಥಃ\\ಪಾಯಾದ್ವ್ರಜಂತಂ ಪ್ರಮಥಾಧಿನಾಥಃ~।\\
ವೇದಾಂತವೇದ್ಯೋಽವತು ಮಾಂ ನಿಷಣ್ಣಂ\\ ಮಾಮವ್ಯಯಃ ಪಾತು ಶಿವಃ ಶಯಾನಂ ॥೨೩॥
\newpage
ಮಾರ್ಗೇಷು ಮಾಂ ರಕ್ಷತು ನೀಲಕಂಠಃ\\ಶೈಲಾದಿದುರ್ಗೇಷು ಪುರತ್ರಯಾರಿಃ~।\\
ಅರಣ್ಯವಾಸಾದಿ ಮಹಾಪ್ರವಾಸೇ\\ಪಾಯಾನ್ಮೃಗವ್ಯಾಧ ಉದಾರಶಕ್ತಿಃ ॥೨೪॥

ಕಲ್ಪಾಂತಕಾಲೋಗ್ರಪಟುಪ್ರಕೋಪ\\ಸ್ಫುಟಾಟ್ಟಹಾಸೋಚ್ಚಲಿತಾಂಡಕೋಶಃ~।\\
ಘೋರಾರಿಸೇನಾರ್ಣವದುರ್ನಿವಾರ\\ಮಹಾಭಯಾದ್ರಕ್ಷತು ವೀರಭದ್ರಃ ॥೨೫॥

ಪತ್ತ್ಯಶ್ವಮಾತಂಗರಥಾವರೂಥಿನೀ\\ಸಹಸ್ರಲಕ್ಷಾಯುತ ಕೋಟಿಭೀಷಣಂ~।\\
ಅಕ್ಷೌಹಿಣೀನಾಂ ಶತಮಾತತಾಯಿನಾಂ\\ಛಿಂದ್ಯಾನ್ಮೃಡೋ ಘೋರಕುಠಾರ ಧಾರಯಾ ॥೨೬॥

ನಿಹಂತು ದಸ್ಯೂನ್ಪ್ರಲಯಾನಲಾರ್ಚಿಃ\\ಜ್ವಲತ್ತ್ರಿಶೂಲಂ ತ್ರಿಪುರಾಂತಕಸ್ಯ~।\\
ಶಾರ್ದೂಲಸಿಂಹರ್ಕ್ಷವೃಕಾದಿಹಿಂಸ್ರಾನ್\\ಸಂತ್ರಾಸಯತ್ವೀಶಧನುಃ ಪಿನಾಕಃ ॥೨೭॥

ದುಃಸ್ವಪ್ನದುಃಶಕುನದುರ್ಗತಿದೌರ್ಮನಸ್ಯ\\ದುರ್ಭಿಕ್ಷದುರ್ವ್ಯಸನದುಃಸಹದುರ್ಯಶಾಂಸಿ~।\\
ಉತ್ಪಾತತಾಪವಿಷಭೀತಿಮಸದ್ಗ್ರಹಾರ್ತಿಂ\\ವ್ಯಾಧೀಂಶ್ಚ ನಾಶಯತು ಮೇ ಜಗತಾಮಧೀಶಃ ॥೨೮॥
\newpage
ಓಂ ನಮೋ ಭಗವತೇ ಸದಾಶಿವಾಯ ಸಕಲ ತತ್ವಾತ್ಮಕಾಯ ಸರ್ವಮಂತ್ರ ಸ್ವರೂಪಾಯ ಸರ್ವಯಂತ್ರಾಧಿಷ್ಠಿತಾಯ ಸರ್ವತಂತ್ರ ಸ್ವರೂಪಾಯ ಸರ್ವತತ್ವ ವಿದೂರಾಯ ಬ್ರಹ್ಮ ರುದ್ರಾವತಾರಿಣೇ ನೀಲಕಂಠಾಯ ಪಾರ್ವತೀ ಮನೋಹರ ಪ್ರಿಯಾಯ ಸೋಮ ಸೂರ್ಯಾಗ್ನಿ ಲೋಚನಾಯ ಭಸ್ಮೋದ್ಧೂಲಿತ ವಿಗ್ರಹಾಯ ಮಹಾಮಣಿ ಮುಕುಟ ಧಾರಣಾಯ ಮಾಣಿಕ್ಯ ಭೂಷಣಾಯ ಸೃಷ್ಟಿ ಸ್ಥಿತಿ ಪ್ರಲಯ ಕಾಲ ರೌದ್ರಾವತಾರಾಯ ದಕ್ಷಾಧ್ವರ ಧ್ವಂಸಕಾಯ ಮಹಾಕಾಲ ಭೇದನಾಯ ಮೂಲಾಧಾರೈಕ ನಿಲಯಾಯ ತತ್ವಾತೀತಾಯ ಗಂಗಾಧರಾಯ ಸರ್ವದೇವಾಧಿ ದೇವಾಯ ಷಡಾಶ್ರಯಾಯ ವೇದಾಂತ ಸಾರಾಯ ತ್ರಿವರ್ಗ ಸಾಧನಾಯ ಅನಂತ ಕೋಟಿ ಬ್ರಹ್ಮಾಂಡ ನಾಯಕಾಯ ಅನಂತ ವಾಸುಕಿ ತಕ್ಷಕ ಕರ್ಕೋಟಕ ಶಂಖ ಕುಲಿಕ ಪದ್ಮ ಮಹಾಪದ್ಮೇತಿ ಅಷ್ಟ ಮಹಾನಾಗಕುಲ ಭೂಷಣಾಯ ಪ್ರಣವ ಸ್ವರೂಪಾಯ ಚಿದಾಕಾಶಾಯ ಆಕಾಶ ದಿಕ್ ಸ್ವರೂಪಾಯ ಗ್ರಹ ನಕ್ಷತ್ರ ಮಾಲಿನೇ ಸಕಲಾಯ ಕಲಂಕ ರಹಿತಾಯ ಸಕಲ ಲೋಕೈಕ ಕರ್ತ್ರೇ ಸಕಲ ಲೋಕೈಕ ಭರ್ತ್ರೇ ಸಕಲ ಲೋಕೈಕ ಸಂಹರ್ತ್ರೇ ಸಕಲ ಲೋಕೈಕ ಗುರವೇ ಸಕಲ ಲೋಕೈಕ ಸಾಕ್ಷಿಣೇ ಸಕಲ ನಿಗಮಗುಹ್ಯಾಯ ಸಕಲ ವೇದಾಂತಪಾರಗಾಯ ಸಕಲ ಲೋಕೈಕ ವರಪ್ರದಾಯ ಸಕಲ ಲೋಕೈಕ ಶಂಕರಾಯ ಸಕಲ ದುರಿತಾರ್ತಿ ಭಂಜನಾಯ ಸಕಲ ಜಗದಭಯಂಕರಾಯ ಶಶಾಂಕ ಶೇಖರಾಯ ಶಾಶ್ವತ ನಿಜಾವಾಸಾಯ ನಿರಾಕಾರಾಯ ನಿರಾಭಾಸಾಯ ನಿರಾಮಯಾಯ ನಿರ್ಮಲಾಯ ನಿರ್ಮದಾಯ ನಿಶ್ಚಿಂತಾಯ ನಿರಹಂಕಾರಾಯ ನಿರಂಕುಶಾಯ ನಿಷ್ಕಲಂಕಾಯ ನಿರ್ಗುಣಾಯ ನಿಷ್ಕಾಮಾಯ ನಿರುಪಪ್ಲವಾಯ ನಿರುಪದ್ರವಾಯ ನಿರವದ್ಯಾಯ ನಿರಂತರಾಯ ನಿಷ್ಕಾರಣಾಯ ನಿರಾತಂಕಾಯ ನಿಷ್ಪ್ರಪಂಚಾಯ ನಿಸ್ಸಂಗಾಯ ನಿರ್ದ್ವಂದ್ವಾಯ ನಿರಾಧಾರಾಯ ನೀರಾಗಾಯ ನಿಷ್ಕ್ರೋಧಾಯ ನಿರ್ಲೋಪಾಯ ನಿಷ್ಪಾಪಾಯ ನಿರ್ಭಯಾಯ ನಿರ್ವಿಕಲ್ಪಾಯ ನಿರ್ಭೇದಾಯ ನಿಷ್ಕ್ರಿಯಾಯ ನಿಸ್ತುಲಾಯ ನಿಃಸಂಶಯಾಯ ನಿರಂಜನಾಯ ನಿರುಪಮ ವಿಭವಾಯ ನಿತ್ಯ ಶುದ್ಧ ಬುದ್ಧ ಮುಕ್ತ ಪರಿಪೂರ್ಣ ಸಚ್ಚಿದಾನಂದಾದ್ವಯಾಯ ಪರಮ ಶಾಂತ ಸ್ವರೂಪಾಯ ಪರಮ ಶಾಂತ ಪ್ರಕಾಶಾಯ ತೇಜೋರೂಪಾಯ ತೇಜೋಮಯಾಯ ತೇಜೋಽಧಿಪತಯೇ ಜಯ ಜಯ ರುದ್ರ ಮಹಾರುದ್ರ ಮಹಾರೌದ್ರ ಭದ್ರಾವತಾರ ಮಹಾಭೈರವ ಕಾಲಭೈರವ ಕಲ್ಪಾಂತಭೈರವ ಕಪಾಲ ಮಾಲಾಧರ ಖಟ್ವಾಂಗ ಚರ್ಮಖಡ್ಗಧರ ಪಾಶಾಂಕುಶ ಡಮರುಶೂಲ ಚಾಪ ಬಾಣ ಗದಾ ಶಕ್ತಿ ಭಿಂದಿಪಾಲ ತೋಮರ ಮುಸಲ ಮುದ್ಗರ ಪಾಶ ಪರಿಘ ಭುಶುಂಡೀ ಶತಘ್ನೀ ಚಕ್ರಾದ್ಯಾಯುಧ ಭೀಷಣಾಕಾರ ಸಹಸ್ರಮುಖ ದಂಷ್ಟ್ರಾಕರಾಲವದನ ವಿಕಟಾಟ್ಟಹಾಸ ವಿಸ್ಫಾರಿತ ಬ್ರಹ್ಮಾಂಡಮಂಡಲ ನಾಗೇಂದ್ರಕುಂಡಲ ನಾಗೇಂದ್ರಹಾರ ನಾಗೇಂದ್ರವಲಯ ನಾಗೇಂದ್ರಚರ್ಮಧರ ನಗೇಂದ್ರನಿಕೇತನ ಮೃತ್ಯುಂಜಯ ತ್ರ್ಯಂಬಕ ತ್ರಿಪುರಾಂತಕ ವಿಶ್ವರೂಪ ವಿರೂಪಾಕ್ಷ ವಿಶ್ವೇಶ್ವರ ವೃಷಭವಾಹನ ವಿಷವಿಭೂಷಣ ವಿಶ್ವತೋಮುಖ ಸರ್ವತೋಮುಖ ಮಾಂ ರಕ್ಷ ರಕ್ಷ ಜ್ವಲ ಜ್ವಲ ಪ್ರಜ್ವಲ ಪ್ರಜ್ವಲ ಮಹಾಮೃತ್ಯುಭಯಂ ಶಮಯ ಶಮಯ ಅಪಮೃತ್ಯುಭಯಂ ನಾಶಯ ನಾಶಯ ರೋಗಭಯಂ ಉತ್ಸಾದಯೋತ್ಸಾದಯ ವಿಷಸರ್ಪಭಯಂ ಶಮಯ ಶಮಯ ಚೋರಾನ್ ಮಾರಯ ಮಾರಯ ಮಮ ಶತ್ರೂನ್ ಉಚ್ಚಾಟಯೋಚ್ಚಾಟಯ ತ್ರಿಶೂಲೇನ ವಿದಾರಯ ವಿದಾರಯ ಕುಠಾರೇಣ ಭಿಂಧಿ ಭಿಂಧಿ ಖಡ್ಗೇನ ಛಿಂಧಿ ಛಿಂಧಿ ಖಟ್ವಾಂಗೇನ ವಿಪೋಥಯ ವಿಪೋಥಯ ಮುಸಲೇನ ನಿಷ್ಪೇಷಯ ನಿಷ್ಪೇಷಯ ಬಾಣೈಃ ಸಂತಾಡಯ ಸಂತಾಡಯ ಯಕ್ಷ ರಕ್ಷಾಂಸಿ ಭೀಷಯ ಭೀಷಯ ಅಶೇಷ ಭೂತಾನ್ ವಿದ್ರಾವಯ ವಿದ್ರಾವಯ ಕೂಷ್ಮಾಂಡ ಭೂತ ವೇತಾಲ ಮಾರೀಗಣ ಬ್ರಹ್ಮರಾಕ್ಷಸಗಣಾನ್ ಸಂತ್ರಾಸಯ ಸಂತ್ರಾಸಯ ಮಮ ಅಭಯಂ ಕುರು ಕುರು ಮಮ ಪಾಪಂ ಶೋಧಯ ಶೋಧಯ ವಿತ್ರಸ್ತಂ ಮಾಂ ಆಶ್ವಾಸಯ ಆಶ್ವಾಸಯ ನರಕ ಮಹಾಭಯಾನ್ ಮಾಂ ಉದ್ಧರ ಉದ್ಧರ ಅಮೃತ ಕಟಾಕ್ಷ ವೀಕ್ಷಣೇನ ಮಾಂ ಆಲೋಕಯ ಆಲೋಕಯ ಸಂಜೀವಯ ಸಂಜೀವಯ ಕ್ಷುತ್ತೃಷ್ಣಾರ್ತಂ ಮಾಂ ಆಪ್ಯಾಯಯ ಆಪ್ಯಾಯಯ ದುಃಖಾತುರಂ ಮಾಂ ಆನಂದಯ ಆನಂದಯ ಶಿವಕವಚೇನ ಮಾಂ ಆಚ್ಛಾದಯ ಆಚ್ಛಾದಯ ಹರ ಹರ ಮೃತ್ಯುಂಜಯ ತ್ರ್ಯಂಬಕ ಸದಾಶಿವ ಪರಮಶಿವ ನಮಸ್ತೇ ನಮಸ್ತೇ ನಮಃ॥
\section{ಶ್ರೀದತ್ತಾತ್ರೇಯ ವಜ್ರಕವಚಂ}
ಅಸ್ಯ ಶ್ರೀದತ್ತಾತ್ರೇಯವಜ್ರಕವಚಸ್ತೋತ್ರಮಂತ್ರಸ್ಯ । ಕಿರಾತರೂಪೀ ಮಹಾರುದ್ರ ಋಷಿಃ । ಅನುಷ್ಟುಪ್ ಛಂದಃ । ಶ್ರೀದತ್ತಾತ್ರೇಯೋ ದೇವತಾ । ದ್ರಾಂ ಬೀಜಂ । ಆಂ ಶಕ್ತಿಃ । ಕ್ರೌಂ ಕೀಲಕಂ । ಓಂ ಆತ್ಮನೇ ನಮಃ ಓಂ ದ್ರೀಂ ಮನಸೇ ನಮಃ । ಓಂ ಆಂ ದ್ರೀಂ ಶ್ರೀಂ ಸೌಃ ಓಂ ಕ್ಲಾಂ ಕ್ಲೀಂ ಕ್ಲೂಂ ಕ್ಲೈಂ ಕ್ಲೌಂ ಕ್ಲಃ ।  ಶ್ರೀದತ್ತಾತ್ರೇಯಪ್ರಸಾದಸಿದ್ಧ್ಯರ್ಥೇ ಜಪೇ ವಿನಿಯೋಗಃ ।\\
ಓಂ ದ್ರಾಂ ದ್ರೀಂ ಇತ್ಯಾದಿನಾ ನ್ಯಾಸಃ ।\\
\dhyana{ಜಗದಂಕುರಕಂದಾಯ ಸಚ್ಚಿದಾನಂದಮೂರ್ತಯೇ ।\\
ದತ್ತಾತ್ರೇಯಾಯ ಯೋಗೀಂದ್ರಚಂದ್ರಾಯ ಪರಮಾತ್ಮನೇ ॥

ಕದಾ ಯೋಗೀ ಕದಾ ಭೋಗೀ ಕದಾ ನಗ್ನಃ ಪಿಶಾಚವತ್ ।\\
ದತ್ತಾತ್ರೇಯೋ ಹರಿಃ ಸಾಕ್ಷಾದ್ಭುಕ್ತಿಮುಕ್ತಿಪ್ರದಾಯಕಃ ॥

ವಾರಾಣಸೀಪುರಸ್ನಾಯೀ ಕೋಲ್ಹಾಪುರಜಪಾದರಃ ।\\
ಮಾಹುರೀಪುರಭಿಕ್ಷಾಶೀ ಸಹ್ಯಶಾಯೀ ದಿಗಂಬರಃ ॥

ಇಂದ್ರನೀಲ ಸಮಾಕಾರ ಶ್ಚಂದ್ರಕಾಂತಿ ಸಮದ್ಯುತಿಃ ।\\
ವೈಡೂರ್ಯಸದೃಶ ಸ್ಫೂರ್ತಿಶ್ಚಲತ್ಕಿಂಚಿಜ್ಜಟಾಧರಃ ॥

ಸ್ನಿಗ್ಧಧಾವಲ್ಯ ಯುಕ್ತಾಕ್ಷೋ ಽತ್ಯಂತನೀಲಕನೀನಿಕಃ ।\\
ಭ್ರೂವಕ್ಷಃಶ್ಮಶ್ರುನೀಲಾಂಕಃ ಶಶಾಂಕಸದೃಶಾನನಃ ॥

ಹಾಸನಿರ್ಜಿತ ನೀಹಾರಃ ಕಂಠನಿರ್ಜಿತ ಕಂಬುಕಃ ।\\
ಮಾಂಸಲಾಂಸೋ ದೀರ್ಘಬಾಹುಃ ಪಾಣಿನಿರ್ಜಿತಪಲ್ಲವಃ ॥

ವಿಶಾಲಪೀನವಕ್ಷಾಶ್ಚ ತಾಮ್ರಪಾಣಿರ್ದಲೋದರಃ ।\\
ಪೃಥುಲಶ್ರೋಣಿಲಲಿತೋ ವಿಶಾಲಜಘನಸ್ಥಲಃ ॥

ರಂಭಾ ಸ್ತಂಭೋಪಮಾನೋರುರ್ಜಾನುಪೂರ್ವೈಕಜಂಘಕಃ ।\\
ಗೂಢಗುಲ್ಫಃ ಕೂರ್ಮಪೃಷ್ಠ ಲಸತ್ಪಾದೋಪರಿಸ್ಥಲಃ ॥

ರಕ್ತಾರವಿಂದಸದೃಶ ರಮಣೀಯ ಪದಾಧರಃ ।\\
ಚರ್ಮಾಂಬರಧರೋ ಯೋಗೀ ಸ್ಮರ್ತೃಗಾಮೀ ಕ್ಷಣೇ ಕ್ಷಣೇ ॥

ಜ್ಞಾನೋಪದೇಶನಿರತೋ ವಿಪದ್ಧರಣದೀಕ್ಷಿತಃ ।\\
ಸಿದ್ಧಾಸನಸಮಾಸೀನ ಋಜುಕಾಯೋ ಹಸನ್ಮುಖಃ ॥

ವಾಮಹಸ್ತೇನ ವರದೋ ದಕ್ಷಿಣೇನಾಭಯಂಕರಃ ।\\
ಬಾಲೋನ್ಮತ್ತಪಿಶಾಚೀಭಿಃ ಕ್ವಚಿದ್ಯುಕ್ತಃ ಪರೀಕ್ಷಿತಃ ॥

ತ್ಯಾಗೀ ಭೋಗೀ ಮಹಾಯೋಗೀ ನಿತ್ಯಾನಂದೋ ನಿರಂಜನಃ ।\\
ಸರ್ವರೂಪೀ ಸರ್ವದಾತಾ ಸರ್ವಗಃ ಸರ್ವಕಾಮದಃ ॥

ಭಸ್ಮೋದ್ಧೂಲಿತಸರ್ವಾಂಗೋ ಮಹಾಪಾತಕನಾಶನಃ ।\\
ಭುಕ್ತಿಪ್ರದೋ ಮುಕ್ತಿದಾತಾ ಜೀವನ್ಮುಕ್ತೋ ನ ಸಂಶಯಃ ॥

ಏವಂ ಧ್ಯಾತ್ವಾಽನನ್ಯಚಿತ್ತೋ ಮದ್ವಜ್ರಕವಚಂ ಪಠೇತ್ ।\\
ಮಾಮೇವ ಪಶ್ಯನ್ಸರ್ವತ್ರ ಸ ಮಯಾ ಸಹ ಸಂಚರೇತ್ ॥

ದಿಗಂಬರಂ ಭಸ್ಮಸುಗಂಧಲೇಪನಂ ಚಕ್ರಂ ತ್ರಿಶೂಲಂ ಡಮರುಂ ಗದಾಯುಧಂ ।\\
ಪದ್ಮಾಸನಂ ಯೋಗಿಮುನೀಂದ್ರವಂದಿತಂ ದತ್ತೇತಿ ನಾಮಸ್ಮರಣೇನ ನಿತ್ಯಂ ॥}

ಓಂ ದತ್ತಾತ್ರೇಯಃ ಶಿರಃ ಪಾತು ಸಹಸ್ರಾಬ್ಜೇಷು ಸಂಸ್ಥಿತಃ ।\\
ಭಾಲಂ ಪಾತ್ವಾನಸೂಯೇಯಶ್ಚಂದ್ರಮಂಡಲಮಧ್ಯಗಃ ॥೨೮॥

ಕೂರ್ಚಂ ಮನೋಮಯಃ ಪಾತು ಹಂ ಕ್ಷಂ ದ್ವಿದಲಪದ್ಮಭೂಃ ।\\
ಜ್ಯೋತೀರೂಪೋಽಕ್ಷಿಣೀ ಪಾತು ಪಾತು ಶಬ್ದಾತ್ಮಕಃ ಶ್ರುತೀ ॥೨೯॥

ನಾಸಿಕಾಂ ಪಾತು ಗಂಧಾತ್ಮಾ ಮುಖಂ ಪಾತು ರಸಾತ್ಮಕಃ ।\\
ಜಿಹ್ವಾಂ ವೇದಾತ್ಮಕಃ ಪಾತು ದಂತೋಷ್ಠೌ ಪಾತು ಧಾರ್ಮಿಕಃ ॥೩೦॥

ಕಪೋಲಾವತ್ರಿಭೂಃ ಪಾತು ಪಾತ್ವಶೇಷಂ ಮಮಾತ್ಮವಿತ್ ।\\
ಸ್ವರಾತ್ಮಾ ಷೋಡಶಾರಾಬ್ಜಸ್ಥಿತಃ ಸ್ವಾತ್ಮಾಽವತಾದ್ಗಲಂ ॥೩೧॥

ಸ್ಕಂಧೌ ಚಂದ್ರಾನುಜಃ ಪಾತು ಭುಜೌ ಪಾತು ಕೃತಾದಿಭೂಃ ।\\
ಜತ್ರುಣೀ ಶತ್ರುಜಿತ್ ಪಾತು ಪಾತು ವಕ್ಷಃಸ್ಥಲಂ ಹರಿಃ ॥೩೨॥

ಕಾದಿಠಾಂತ ದ್ವಾದಶಾರ ಪದ್ಮಗೋ ಮರುದಾತ್ಮಕಃ ।\\
ಯೋಗೀಶ್ವರೇಶ್ವರಃ ಪಾತು ಹೃದಯಂ ಹೃದಯಸ್ಥಿತಃ ॥೩೩॥

ಪಾರ್ಶ್ವೇ ಹರಿಃ ಪಾರ್ಶ್ವವರ್ತೀ ಪಾತು ಪಾರ್ಶ್ವಸ್ಥಿತಃ ಸ್ಮೃತಃ ।\\
ಹಠಯೋಗಾದಿ ಯೋಗಜ್ಞಃ ಕುಕ್ಷೀ ಪಾತು ಕೃಪಾನಿಧಿಃ ॥೩೪॥

ಡಕಾರಾದಿ ಫಕಾರಾಂತ ದಶಾರಸರಸೀರುಹೇ ।\\
ನಾಭಿಸ್ಥಲೇ ವರ್ತಮಾನೋ ನಾಭಿಂ ವಹ್ನ್ಯಾತ್ಮಕೋಽವತು ॥೩೫॥

ವಹ್ನಿತತ್ವಮಯೋ ಯೋಗೀ ರಕ್ಷತಾನ್ಮಣಿಪೂರಕಂ ।\\
ಕಟಿಂ ಕಟಿಸ್ಥಬ್ರಹ್ಮಾಂಡವಾಸುದೇವಾತ್ಮಕೋಽವತು ॥೩೬॥

ವಕಾರಾದಿಲಕಾರಾಂತಷಟ್ಪತ್ರಾಂಬುಜಬೋಧಕಃ ।\\
ಜಲತತ್ವಮಯೋ ಯೋಗೀ ಸ್ವಾಧಿಷ್ಠಾನಂ ಮಮಾವತು ॥೩೭॥

ಸಿದ್ಧಾಸನಸಮಾಸೀನ ಊರೂ ಸಿದ್ಧೇಶ್ವರೋಽವತು ।\\
ವಾದಿಸಾಂತಚತುಷ್ಪತ್ರಸರೋರುಹನಿಬೋಧಕಃ ॥೩೮॥

ಮೂಲಾಧಾರಂ ಮಹೀರೂಪೋ ರಕ್ಷತಾದ್ವೀರ್ಯನಿಗ್ರಹೀ ।\\
ಪೃಷ್ಠಂ ಚ ಸರ್ವತಃ ಪಾತು ಜಾನುನ್ಯಸ್ತಕರಾಂಬುಜಃ ॥೩೯॥

ಜಂಘೇ ಪಾತ್ವವಧೂತೇಂದ್ರಃ ಪಾತ್ವಂಘ್ರೀ ತೀರ್ಥಪಾವನಃ ।\\
ಸರ್ವಾಂಗಂ ಪಾತು ಸರ್ವಾತ್ಮಾ ರೋಮಾಣ್ಯವತು ಕೇಶವಃ ॥೪೦॥

ಚರ್ಮ ಚರ್ಮಾಂಬರಃ ಪಾತು ರಕ್ತಂ ಭಕ್ತಿಪ್ರಿಯೋಽವತು ।\\
ಮಾಂಸಂ ಮಾಂಸಕರಃ ಪಾತು ಮಜ್ಜಾಂ ಮಜ್ಜಾತ್ಮಕೋಽವತು ॥೪೧॥

ಅಸ್ಥೀನಿ ಸ್ಥಿರಧೀಃ ಪಾಯಾನ್ಮೇಧಾಂ ವೇಧಾಃ ಪ್ರಪಾಲಯೇತ್ ।\\
ಶುಕ್ರಂ ಸುಖಕರಃ ಪಾತು ಚಿತ್ತಂ ಪಾತು ದೃಢಾಕೃತಿಃ ॥೪೨॥

ಮನೋಬುದ್ಧಿಮಹಂಕಾರಂ ಹೃಷೀಕೇಶಾತ್ಮಕೋಽವತು ।\\
ಕರ್ಮೇಂದ್ರಿಯಾಣಿ ಪಾತ್ವೀಶಃ ಪಾತು ಜ್ಞಾನೇಂದ್ರಿಯಾಣ್ಯಜಃ ॥೪೩॥

ಬಂಧೂನ್ ಬಂಧೂತ್ತಮಃ ಪಾಯಾಚ್ಛತ್ರುಭ್ಯಃ ಪಾತು ಶತ್ರುಜಿತ್ ।\\
ಗೃಹಾರಾಮಧನಕ್ಷೇತ್ರಪುತ್ರಾದೀನ್ ಶಂಕರೋಽವತು ॥೪೪॥

ಭಾರ್ಯಾಂ ಪ್ರಕೃತಿವಿತ್ ಪಾತು ಪಶ್ವಾದೀನ್ಪಾತು ಶಾರ್ಙ್ಗಭೃತ್ ।\\
ಪ್ರಾಣಾನ್ಪಾತು ಪ್ರಧಾನಜ್ಞೋ ಭಕ್ಷ್ಯಾದೀನ್ಪಾತು ಭಾಸ್ಕರಃ ॥೪೫॥

ಸುಖಂ ಚಂದ್ರಾತ್ಮಕಃ ಪಾತು ದುಃಖಾತ್ ಪಾತು ಪುರಾಂತಕಃ ।\\
ಪಶೂನ್ಪಶುಪತಿಃ ಪಾತು ಭೂತಿಂ ಭೂತೇಶ್ವರೋ ಮಮ ॥೪೬॥

ಪ್ರಾಚ್ಯಾಂ ವಿಷಹರಃ ಪಾತು ಪಾತ್ವಾಗ್ನೇಯ್ಯಾಂ ಮಖಾತ್ಮಕಃ ।\\
ಯಾಮ್ಯಾಂ ಧರ್ಮಾತ್ಮಕಃ ಪಾತು ನೈರ್ಋತ್ಯಾಂ ಸರ್ವವೈರಿಹೃತ್ ॥೪೭॥

ವರಾಹಃ ಪಾತು ವಾರುಣ್ಯಾಂ ವಾಯವ್ಯಾಂ ಪ್ರಾಣದೋಽವತು ।\\
ಕೌಬೇರ್ಯಾಂ ಧನದಃ ಪಾತು ಪಾತ್ವೈಶಾನ್ಯಾಂ ಮಹಾಗುರುಃ ॥೪೮॥

ಊರ್ಧ್ವಂ ಪಾತು ಮಹಾಸಿದ್ಧಃ ಪಾತ್ವಧಸ್ತಾಜ್ಜಟಾಧರಃ ।\\
ರಕ್ಷಾಹೀನಂ ತು ಯತ್ಸ್ಥಾನಂ ರಕ್ಷತ್ವಾದಿಮುನೀಶ್ವರಃ ॥೪೯॥

\authorline{॥ಇತಿ ಶ್ರೀರುದ್ರಯಾಮಲೇ ಹಿಮವತ್ಖಂಡೇ ಉಮಾಮಹೇಶ್ವರಸಂವಾದೇ ಶ್ರೀದತ್ತಾತ್ರೇಯವಜ್ರಕವಚಸ್ತೋತ್ರಂ ॥}
\section{ ಶ್ರೀ ಅನಘಾಕವಚಾಷ್ಟಕಂ }
ಶಿರೋ ಮೇ ಅನಘಾ ಪಾತು ಭಾಲಂ ಮೇ ದತ್ತಭಾಮಿನೀ~।\\
ಭ್ರೂಮಧ್ಯಂ ಯೋಗಿನೀ ಪಾತು ನೇತ್ರೇ ಪಾತು ಸುದರ್ಶಿನೀ॥೧॥

ನಾಸಾರಂಧ್ರದ್ವಯಂ ಪಾತು ಯೋಗೇಶೀ ಭಕ್ತವತ್ಸಲಾ~।\\
ಮುಖಂ ಮೇ ಮಧುವಾಕ್ಪಾತು ದತ್ತಚಿತ್ತವಿಹಾರಿಣೀ॥೨॥

ತ್ರಿಕಂಠೀ ಪಾತು ಮೇ ಕಂಠಂ ವಾಚಂ ವಾಚಸ್ಪತಿಪ್ರಿಯಾ~।\\
ಸ್ಕಂಧೌ ಮೇ ತ್ರಿಗುಣಾ ಪಾತು ಭುಜೌ ಕಮಲಧಾರಿಣೀ॥೩॥

ಕರೌ ಸೇವಾರತಾ ಪಾತು ಹೃದಯಂ ಮಂದಹಾಸಿನೀ~।\\
ಉದರಂ ಅನ್ನದಾ ಪಾತು ಸ್ವಯಂಜಾ ನಾಭಿಮಂಡಲಂ॥೪॥

ಕಮನೀಯಾ ಕಟಿಂ ಪಾತು ಗುಹ್ಯಂ ಗುಹ್ಯೇಶ್ವರೀ ಸದಾ~।\\
ಊರೂ ಮೇ ಪಾತು ಜಂಭಘ್ನೀ ಜಾನುನೀ ರೇಣುಕೇಷ್ಟದಾ ॥೫॥
\newpage

ಪಾದೌ ಪಾದಸ್ಥಿತಾ ಪಾತು ಪುತ್ರದಾ ವೈ ಖಿಲಂ ವಪುಃ~।\\
ವಾಮಗಾ ಪಾತು ವಾಮಾಂಗಂ ದಕ್ಷಾಂಗಂ ಗುರುಗಾಮಿನೀ॥೬॥

ಗೃಹಂ ಮೇ ದತ್ತಗೃಹಿಣೀ ಬಾಹ್ಯೇ ಸರ್ವಾತ್ಮಿಕಾವತು~।\\
ತ್ರಿಕಾಲೇ ಸರ್ವದಾ ರಕ್ಷೇತ್ ಪತಿಶುಶ್ರೂಷಣೋತ್ಸುಕಾ ॥೭॥

ಜಾಯಾಂ ಮೇ ದತ್ತವಾಮಾಂಗೀ ಅಷ್ಟಪುತ್ರಾ ಸುತೋವತು~।\\
ಗೋತ್ರಮತ್ರಿಸ್ನುಷಾ ರಕ್ಷೇದ್ ಅನಘಾ ಭಕ್ತರಕ್ಷಣೀ॥೮॥

ಯಃ ಪಠೇದನಘಾಕವಚಂ ನಿತ್ಯಂ ಭಕ್ತಿಯುತೋ ನರಃ~।\\
ತಸ್ಮೈ ಭವತ್ಯನಘಾಂಬಾ ವರದಾ ಸರ್ವಭಾಗ್ಯದಾ॥೯॥
\section{ಶ್ರೀ ದಕ್ಷಿಣಾಮೂರ್ತಿ ಕವಚಂ}
ಪಾರ್ವತ್ಯುವಾಚ ॥\\
ನಮಸ್ತೇಸ್ತು ತ್ರಯೀನಾಥ ಪರಮಾನಂದ ಕಾರಕ~।\\
ಕವಚಂ ದಕ್ಷಿಣಾಮೂರ್ತೇಃ ಕೃಪಯಾ ವದ ಮೇ ಪ್ರಭೋ ॥

ಈಶ್ವರ ಉವಾಚ ॥\\
ವಕ್ಷ್ಯೇಹಂ ದೇವ ದೇವೇಶಿ ದಕ್ಷಿಣಾಮೂರ್ತಿರವ್ಯಯಂ~।\\
ಕವಚಂ ಸರ್ವಪಾಪಘ್ನಂ ವೇದಾನಾಂ ಜ್ಞಾನಗೋಚರಂ॥

ಅಣಿಮಾದಿ ಮಹಾಸಿದ್ಧಿವಿಧಾನಚತುರಂ ಶುಭಂ~।\\
ವೇದಶಾಸ್ತ್ರಪುರಾಣಾನಿ ಕವಿತಾತರ್ಕ ಏವ ಚ ॥

ಬಹುಧಾ ದೇವಿ ಜಾಯಂತೇ ಕವಚಸ್ಯ ಪ್ರಭಾವತಃ~।\\
ಋಷಿರ್ಬ್ರಹ್ಮಾ ಸಮುದ್ದಿಷ್ಟಶ್ಛಂದೋಽನುಷ್ಟುಬುದಾಹೃತಂ॥

ದೇವತಾ ದಕ್ಷಿಣಾಮೂರ್ತಿಃ ಪರಮಾತ್ಮಾ ಸದಾಶಿವಃ~।\\
ಬೀಜಂ ವೇದಾದಿಕಂ ಚೈವ ಸ್ವಾಹಾ ಶಕ್ತಿರುದಾಹೃತಾ~।\\
ಸರ್ವಜ್ಞತ್ವೇಪಿ ದೇವೇಶಿ ವಿನಿಯೋಗಃ ಪ್ರಚಕ್ಷ್ಯತೇ॥

\dhyana{ಅದ್ವಂದ್ವನೇತ್ರಮಮಲೇಂದು ಕಲಾವತಂಸಂ\\
ಹಂಸಾವಲಂಬಿತಸಮಾನಜಟಾಕಲಾಪಂ~।\\
ಆನೀಲಕಂಠಮುಪಕಂಠಮುನಿಪ್ರವೀರಾನ್\\
ಅಧ್ಯಾಪಯಂತಮವಲೋಕಯ ಲೋಕನಾಥಂ॥}

ಶಿರೋ ಮೇ ದಕ್ಷಿಣಾಮೂರ್ತಿರವ್ಯಾತ್ ಫಾಲಂ ಮಹೇಶ್ವರಃ~।\\
ದೃಶೌ ಪಾತು ಮಹಾದೇವಃ ಶ್ರವಣೇ ಚಂದ್ರಶೇಖರಃ॥೧॥

ಕಪೋಲೌ ಪಾತು ಮೇ ರುದ್ರೋ ನಾಸಾಂ ಪಾತು ಜಗದ್ಗುರುಃ~।\\
ಮುಖಂ ಗೌರೀಪತಿಃ ಪಾತು ರಸನಾಂ ವೇದರೂಪಧೃತ್~।\\
ದಶನಾನ್ ತ್ರಿಪುರಧ್ವಂಸೀ ಚೌಷ್ಠಂ ಪನ್ನಗಭೂಷಣಃ ॥೨॥

ಅಧರಂ ಪಾತು ವಿಶ್ವಾತ್ಮಾ ಹನೂ ಪಾತು ಜಗನ್ಮಯಃ~।\\
ಚುಬುಕಂ ದೇವದೇವಸ್ತು ಪಾತು ಕಂಠಂ ಜಟಾಧರಃ॥೩॥

ಸ್ಕಂಧೌ ಮೇ ಪಾತು ಶುದ್ಧಾತ್ಮಾ ಕರೌ ಪಾತು ಯಮಾಂತಕಃ~।\\
ಕುಚಾಗ್ರಂ ಕರಮಧ್ಯಂ ಚ ನಖರಾನ್ ಶಂಕರಃ ಸ್ವಯಂ ॥೪॥

ಹೃನ್ಮೇ ಪಶುಪತಿಃ ಪಾತು ಪಾರ್ಶ್ವೇ ಪರಮಪೂರುಷಃ~।\\
ಮಧ್ಯಮಂ ಪಾತು ಶರ್ವೋ ಮೇ ನಾಭಿಂ ನಾರಾಯಣಪ್ರಿಯಃ ॥೫॥

ಕಟಿಂ ಪಾತು ಜಗದ್ಭರ್ತಾ ಸಕ್ಥಿನೀ ಚ ಮೃಡಃ ಸ್ವಯಂ~।\\
ಕೃತಿವಾಸಾಃ ಸ್ವಯಂ ಗುಹ್ಯಮೂರೂ ಪಾತು ಪಿನಾಕಧೃತ್॥೬॥

ಜಾನುನೀ ತ್ರ್ಯಂಬಕಃ ಪಾತು ಜಂಘೇ ಪಾತು ಸದಾಶಿವಃ~।\\
ಸ್ಮರಾರಿಃ ಪಾತು ಮೇ ಪಾದೌ ಪಾತು ಸರ್ವಾಂಗಮೀಶ್ವರಃ ॥೭॥

ಇತೀದಂ ಕವಚಂ ದೇವಿ ಪರಮಾನಂದ ದಾಯಕಂ~।\\
ಪ್ರಾತಃ ಕಾಲೇ ಶುಚಿರ್ಭೂತ್ವಾ ತ್ರಿವಾರಂ ಸರ್ವದಾ ಪಠೇತ್ ॥೮॥

ನಿತ್ಯಂ ಪೂಜಾ ಸಮಾಯುಕ್ತಃ ಸಂವತ್ಸರಮತಂದ್ರಿತಃ~।\\
ಜಪೇತ್ ತ್ರಿಸಂಧ್ಯಂ ಯೋ ವಿದ್ವಾನ್ ವೇದಶಾಸ್ತ್ರಾರ್ಥಪಾರಗಃ॥೯॥

ಗದ್ಯಪದ್ಯೈಸ್ತಥಾ ಚಾಪಿ ನಾಟಕಾಃ ಸ್ವಯಮೇವ ಹಿ~।\\
ನಿರ್ಗಚ್ಛಂತಿ ಮುಖಾಂಭೋಜಾತ್ ಸತ್ಯಮೇತನ್ನ ಸಂಶಯಃ ॥೧೦॥
\authorline{ಇತಿ ಶ್ರೀ ಬ್ರಹ್ಮವೈವರ್ತಮಹಾಪುರಾಣೇ  ಶ್ರೀ ದಕ್ಷಿಣಾಮೂರ್ತಿಕವಚಂ ಸಂಪೂರ್ಣಂ}
\section{ಶ್ರೀಗಣೇಶಕವಚಂ }
ಗೌರ್ಯುವಾಚ~॥\\
ಏಷೋಽತಿಚಪಲೋ ದೈತ್ಯಾನ್ಬಾಲ್ಯೇಽಪಿ ನಾಶಯತ್ಯಹೋ~।\\
ಅಗ್ರೇ ಕಿಂ ಕರ್ಮ ಕರ್ತೇತಿ ನ ಜಾನೇ ಮುನಿಸತ್ತಮ ॥೧॥

ದೈತ್ಯಾ ನಾನಾವಿಧಾ ದುಷ್ಟಾಃ ಸಾಧುದೇವದ್ರುಹಃ ಖಲಾಃ~।\\
ಅತೋಽಸ್ಯ ಕಂಠೇ ಕಿಂಚಿತ್ತ್ವಂ ರಕ್ಷಾರ್ಥಂ ಬದ್ಧುಮರ್ಹಸಿ ॥೨॥

ಮುನಿರುವಾಚ~॥\\
\dhyana{ಧ್ಯಾಯೇತ್ಸಿಂಹಗತಂ ವಿನಾಯಕಮಮುಂ ದಿಗ್ಬಾಹುಮಾದ್ಯೇ ಯುಗೇ\\
ತ್ರೇತಾಯಾಂ ತು ಮಯೂರವಾಹನಮಮುಂ ಷಡ್ಬಾಹುಕಂ ಸಿದ್ಧಿದಂ~।\\
ದ್ವಾಪಾರೇ ತು ಗಜಾನನಂ ಯುಗಭುಜಂ ರಕ್ತಾಂಗರಾಗಂ ವಿಭುಂ\\
ತುರ್ಯೇ ತು ದ್ವಿಭುಜಂ ಸಿತಾಂಗರುಚಿರಂ ಸರ್ವಾರ್ಥದಂ ಸರ್ವದಾ ॥೩॥}

ವಿನಾಯಕಃ ಶಿಖಾಂ ಪಾತು ಪರಮಾತ್ಮಾ ಪರಾತ್ಪರಃ~।\\
ಅತಿಸುಂದರಕಾಯಸ್ತು ಮಸ್ತಕಂ ಸುಮಹೋತ್ಕಟಃ ॥೪॥

ಲಲಾಟಂ ಕಶ್ಯಪಃ ಪಾತು ಭ್ರೂಯುಗಂ ತು ಮಹೋದರಃ~।\\
ನಯನೇ ಭಾಲಚಂದ್ರಸ್ತು ಗಜಾಸ್ಯಸ್ತ್ವೋಷ್ಠಪಲ್ಲವೌ ॥೫॥

ಜಿಹ್ವಾಂ ಪಾತು ಗಣಕ್ರೀಡಶ್ಚಿಬುಕಂ ಗಿರಿಜಾಸುತಃ~।\\
ವಾಚಂ ವಿನಾಯಕಃ ಪಾತು ದಂತಾನ್ ರಕ್ಷತು ವಿಘ್ನಹಾ ॥೬॥

ಶ್ರವಣೌ ಪಾಶಪಾಣಿಸ್ತು ನಾಸಿಕಾಂ ಚಿಂತಿತಾರ್ಥದಃ~।\\
ಗಣೇಶಸ್ತು ಮುಖಂ ಕಂಠಂ ಪಾತು ದೇವೋ ಗಣಂಜಯಃ ॥೭॥

ಸ್ಕಂಧೌ ಪಾತು ಗಜಸ್ಕಂಧಃ ಸ್ತನೌ ವಿಘ್ನವಿನಾಶನಃ~।\\
ಹೃದಯಂ ಗಣನಾಥಸ್ತು ಹೇರಂಬೋ ಜಠರಂ ಮಹಾನ್ ॥೮॥

ಧರಾಧರಃ ಪಾತು ಪಾರ್ಶ್ವೌ ಪೃಷ್ಠಂ ವಿಘ್ನಹರಃ ಶುಭಃ~।\\
ಲಿಂಗಂ ಗುಹ್ಯಂ ಸದಾ ಪಾತು ವಕ್ರತುಂಡೋ ಮಹಾಬಲಃ ॥೯॥

ಗಣಕ್ರೀಡೋ ಜಾನುಜಂಘೇ ಊರೂ ಮಂಗಲಮೂರ್ತಿಮಾನ್~।\\
ಏಕದಂತೋ ಮಹಾಬುದ್ಧಿಃ ಪಾದೌ ಗುಲ್ಫೌ ಸದಾಽವತು ॥೧೦॥

ಕ್ಷಿಪ್ರಪ್ರಸಾದನೋ ಬಾಹೂ ಪಾಣೀ ಆಶಾಪ್ರಪೂರಕಃ~।\\
ಅಂಗುಲೀಶ್ಚ ನಖಾನ್ಪಾತು ಪದ್ಮಹಸ್ತೋಽರಿನಾಶನಃ ॥೧೧॥

ಸರ್ವಾಂಗಾನಿ ಮಯೂರೇಶೋ ವಿಶ್ವವ್ಯಾಪೀ ಸದಾಽವತು~।\\
ಅನುಕ್ತಮಪಿ ಯತ್ಸ್ಥಾನಂ ಧೂಮ್ರಕೇತುಃ ಸದಾಽವತು ॥೧೨॥

ಆಮೋದಸ್ತ್ವಗ್ರತಃ ಪಾತು ಪ್ರಮೋದಃ ಪೃಷ್ಠತೋಽವತು~।\\
ಪ್ರಾಚ್ಯಾಂ ರಕ್ಷತು ಬುದ್ಧೀಶ ಆಗ್ನೇಯ್ಯಾಂ ಸಿದ್ಧಿದಾಯಕಃ ॥೧೩॥

ದಕ್ಷಿಣಸ್ಯಾಮುಮಾಪುತ್ರೋ ನೈರೃತ್ಯಾಂ ತು ಗಣೇಶ್ವರಃ~।\\
ಪ್ರತೀಚ್ಯಾಂ ವಿಘ್ನಹರ್ತಾಽವ್ಯಾದ್ವಾಯವ್ಯಾಂ ಗಜಕರ್ಣಕಃ ॥೧೪॥

ಕೌಬೇರ್ಯಾಂ ನಿಧಿಪಃ ಪಾಯಾದೀಶಾನ್ಯಾಮೀಶನಂದನಃ~।\\
ದಿವಾಽವ್ಯಾದೇಕದಂತಸ್ತು ರಾತ್ರೌ ಸಂಧ್ಯಾಸು ವಿಘ್ನಹೃತ್ ॥೧೫॥

ರಾಕ್ಷಸಾಸುರವೇತಾಲಗ್ರಹಭೂತಪಿಶಾಚತಃ~।\\
ಪಾಶಾಂಕುಶಧರಃ ಪಾತು ರಜಃಸತ್ತ್ವತಮಃಸ್ಮೃತೀಃ ॥೧೬॥

ಜ್ಞಾನಂ ಧರ್ಮಂ ಚ ಲಕ್ಷ್ಮೀಂ ಚ ಲಜ್ಜಾಂ ಕೀರ್ತಿಂ ತಥಾ ಕುಲಂ~।\\
ವಪುರ್ಧನಂ ಚ ಧಾನ್ಯಂ ಚ ಗೃಹಾನ್ ದಾರಾನ್ಸುತಾನ್ಸಖೀನ್ ॥೧೭॥

ಸರ್ವಾಯುಧಧರಃ ಪೌತ್ರಾನ್ ಮಯೂರೇಶೋಽವತಾತ್ಸದಾ~।\\
ಕಪಿಲೋಽಜಾದಿಕಂ ಪಾತು ಗಜಾಶ್ವಾನ್ವಿಕಟೋಽವತು ॥೧೮॥

ಭೂರ್ಜಪತ್ರೇ ಲಿಖಿತ್ವೇದಂ ಯಃ ಕಂಠೇ ಧಾರಯೇತ್ಸುಧೀಃ~।\\
ನ ಭಯಂ ಜಾಯತೇ ತಸ್ಯ ಯಕ್ಷರಕ್ಷಃಪಿಶಾಚತಃ ॥೧೯॥

ತ್ರಿಸಂಧ್ಯಂ ಜಪತೇ ಯಸ್ತು ವಜ್ರಸಾರತನುರ್ಭವೇತ್~।\\
ಯಾತ್ರಾಕಾಲೇ ಪಠೇದ್ಯಸ್ತು ನಿರ್ವಿಘ್ನೇನ ಫಲಂ ಲಭೇತ್ ॥೨೦॥

ಯುದ್ಧಕಾಲೇ ಪಠೇದ್ಯಸ್ತು ವಿಜಯಂ ಚಾಪ್ನುಯಾದ್ದ್ರುತಂ~।\\
ಮಾರಣೋಚ್ಚಾಟನಾಕರ್ಷಸ್ತಂಭಮೋಹನಕರ್ಮಣಿ ॥೨೧॥

ಸಪ್ತವಾರಂ ಜಪೇದೇತದ್ದಿನಾನಾಮೇಕವಿಂಶತಿಂ~।\\
ತತ್ತತ್ಫಲಮವಾಪ್ನೋತಿ ಸಾಧಕೋ ನಾತ್ರಸಂಶಯಃ ॥೨೨॥

ಏಕವಿಂಶತಿವಾರಂ ಚ ಪಠೇತ್ತಾವದ್ದಿನಾನಿ ಯಃ~।\\
ಕಾರಾಗೃಹಗತಂ ಸದ್ಯೋ ರಾಜ್ಞಾ ವಧ್ಯಂ ಚ ಮೋಚಯೇತ್ ॥೨೩॥

ರಾಜದರ್ಶನವೇಲಾಯಾಂ ಪಠೇದೇತತ್ತ್ರಿವಾರತಃ~।\\
ಸ ರಾಜಾನಂ ವಶಂ ನೀತ್ವಾ ಪ್ರಕೃತೀಶ್ಚ ಸಭಾಂ ಜಯೇತ್ ॥೨೪॥

ಇದಂ ಗಣೇಶಕವಚಂ ಕಶ್ಯಪೇನ ಸಮೀರಿತಂ~।\\
ಮುದ್ಗಲಾಯ ಚ ತೇ ನಾಥ ಮಾಂಡವ್ಯಾಯ ಮಹರ್ಷಯೇ ॥೨೫॥

ಮಹ್ಯಂ ಸ ಪ್ರಾಹ ಕೃಪಯಾ ಕವಚಂ ಸರ್ವಸಿದ್ಧಿದಂ~।\\
ನ ದೇಯಂ ಭಕ್ತಿಹೀನಾಯ ದೇಯಂ ಶ್ರದ್ಧಾವತೇ ಶುಭಂ ॥೨೬॥

ಯಸ್ಯಾನೇನ ಕೃತಾ ರಕ್ಷಾ ನ ಬಾಧಾಸ್ಯ ಭವೇತ್ಕ್ವಚಿತ್~।\\
ರಾಕ್ಷಸಾಸುರವೇತಾಲದೈತ್ಯದಾನವಸಂಭವಾ ॥೨೭॥

\authorline {ಇತಿ ಶ್ರೀಗಣೇಶಪುರಾಣೇ ಗಣೇಶಕವಚಂ ಸಂಪೂರ್ಣಂ ॥}
\section{ಸರಸ್ವತೀ ಕವಚಮ್\\ (ರುದ್ರಯಾಮಲಾಂತರ್ಗತಮ್ )}

ಅಸ್ಯ ಶ್ರೀ ಸರಸ್ವತೀ ಕವಚಮಂತ್ರಸ್ಯ ಆಶ್ವಲಾಯನ ಋಷಿಃ । ಅನುಷ್ಟುಪ್ \\ಛಂದಃ । ಶ್ರೀ ಸರಸ್ವತೀ ದೇವತಾ । ಐಂ ಬೀಜಂ । ಹ್ರೀಂ ಶಕ್ತಿಃ । ಕ್ಲೀಂ ಕೀಲಕಂ । \\ಸರಸ್ವತೀ ಪ್ರಸಾದಸಿದ್ಧ್ಯರ್ಥೇ ಜಪೇ ವಿನಿಯೋಗಃ ॥

\dhyana{ದೋರ್ಭಿರ್ಯುಕ್ತಾ ಚತುರ್ಭಿಃ ಸ್ಫಟಿಕಮಣಿಮಯೀಮಕ್ಷಮಾಲಾಂದಧಾನಾ\\
ಹಸ್ತೇನೈಕೇನ ಪದ್ಮಂ ಸಿತಮಪಿ ಚ ಶುಕಂ ಪುಸ್ತಕಂ ಚಾಪರೇಣ ।\\
ಭಾಸಾ ಕುಂದೇಂದುಶಂಖಸ್ಫಟಿಕಮಣಿನಿಭಾ ಭಾಸಮಾನಾಽಸಮಾನಾ\\
ಸಾ ಮೇ ವಾಗ್ದೇವತೇಯಂ ನಿವಸತು ವದನೇ ಸರ್ವದಾ ಸುಪ್ರಸನ್ನಾ ॥}


ಸರಸ್ವತೀ ಶಿರಃ ಪಾತು ಫಾಲಂ ಫಾಲಾಕ್ಷಸೋದರೀ ।\\
ಶ್ರುತೀ ಶ್ರುತಿಮಯೀ ಪಾತು ನೇತ್ರೇರ್ಕೇಂದ್ವಗ್ನಿಲೋಚನಾ ॥೧॥

ಘ್ರಾಣಂ ಪ್ರಾಣನಿಧಿಃ ಪಾತು ಕಪೋಲೌ ಕಾಮಿತಾರ್ಥದಾ ।\\
ವಕ್ತ್ರಂ ವಿದ್ಯಾತ್ಮಿಕಾ ಪಾತು ಸ್ಕಂಧೌ ಸ್ಕಂದಸಮರ್ಚಿತಾ ॥೨॥

ಭುಜೌ ಚತುರ್ಭುಜಾ ಪಾತು ಕರೌ ಕಾಂಕ್ಷಿತದಾಯಿಕಾ ।\\
ಪಾರ್ಶ್ವೌ ಮೇ ಪಾತು ದೇವೇಶೀ ವಕ್ಷೋ ಬ್ರಹ್ಮಮುಖಾಸನಾ ॥೩॥

ಕುಕ್ಷಿಮಕ್ಷರರೂಪಾ ಚ ನಾಭಿಂ ನಾಭಿಜವಲ್ಲಭಾ ।\\
ಮಧ್ಯಂ ಸುಮಧ್ಯಮಾ ಪಾತು ಗುಹ್ಯಂ ಸರ್ವಾಂಗ ಸುಂದರೀ ॥೪॥

ಊರೂ ಮೇ ಪಾತು ವಾಗ್ದೇವೀ ಜಾನುನೀ ಜಗದೀಶ್ವರೀ ।\\
ಜಂಘೇ ಪಾತು ಮಹಾದೇವೀ ಗುಲ್ಫೌ ಮೇ ಗುಣರೂಪಿಣೀ॥೫॥

ಪಾದೌ ವೇದಾತ್ಮಿಕಾ ಪಾತು ಸರ್ವಾಂಗಂ ಮಾತೃಕಾತ್ಮಿಕಾ ।\\
ಇತೀದಂ ಕವಚಂ ದಿವ್ಯಂ ವಾಗ್ದೇವ್ಯಾಃ ಪ್ರೀತಿಕಾರಣಂ ॥೬॥\\
ಫಲಶ್ರುತಿಃ॥\\
ಜಡಾನಾಂ ಬುದ್ಧಿದಂ ಪುಣ್ಯಂ ಮೂಕಾನಾಂ ವಾಕ್ಪ್ರದಾಯಕಂ ।\\
ಅಂಧಾನಾಂ ದೃಷ್ಟಿದಂ ಚೈವ ಸರ್ವಜ್ಞಾನಪ್ರದಾಯಕಂ ॥೭॥

ವಾಗ್ವಶ್ಯಜನಕಂ ನೄಣಾಂ ತಥಾ ಭೂ-ಪಾಲಮೋಹನಂ ।\\
ವಾಕ್‌ಸ್ತಂಭಕಾರಕಂ ಚೈವ ಸಭಾಯಾಂ ಪ್ರತಿವಾದಿನಾಂ ॥೮॥

ಪುತ್ರಪ್ರದಮಪುತ್ರಾಣಾಂ ಧನದಂ ಧನಕಾಮಿನಾಂ ।\\
ಮೋಕ್ಷದಂ ಮೋಕ್ಷಕಾಮಾನಾಂ ಮಂತ್ರಸಿದ್ಧಿ ಪ್ರದಾಯಕಂ ॥೯॥

ಬಹುನಾ ಕಿಮಿಹೋಕ್ತೇನ ಸತ್ಯಂ ಸತ್ಯಂ ಮುನೀಶ್ವರ ।\\
ಆಶ್ವಲಾಯನಸಂಪ್ರೋಕ್ತಂ ಷಣ್ಮಾಸಂ ಜಪತಾಂ ನೃಣಾಂ ॥೧೦॥

ಕವಿತಾವಾಕ್ಪಟುತ್ವಂ ಚ ಜಾಯತೇ ನಾತ್ರ ಸಂಶಯಃ ।\\
ಶುಕ್ರವಾರೇ ವಿಶೇಷೇಣ ಜಪ್ತವ್ಯಂ ದೇವಿ ಸಾಧಕೈಃ ॥೧೧॥

ಸತ್ಯಂ ಸಾರಸ್ವತಂ ಚೈವ ಸ್ಥಿರತಾಮೇತಿ ತತ್ಕುಲೇ ।\\
ಪೌರ್ಣಮಾಸ್ಯಾಮಮಾವಸ್ಯಾಂ ದೇವಿ ಸಾರಸ್ವತೇ ತಥಾ ॥೧೨॥

ಯೋಗೇ ವಿಶೇಷೇ ಜಪ್ತವ್ಯಂ ವಿದ್ಯಾರ್ಥಿಭಿರತಂದ್ರಿತೈಃ ।\\
ಅನೇನ ಕವಚೇನೈವ ನ್ಯಸ್ತಾಂಗೋ ಮೂಲಮಂತ್ರಕಂ ॥೧೩॥

ಅಷ್ಟೋತ್ತರಶತಂ ಜಪ್ತ್ವಾ  ವಾಚಂ ಚೈತಾಂ ಚ ಭಕ್ಷಯೇತ್ ।\\
ಪ್ರಾತಃ ಕಾಲೇ ತು ಮಾಸೈಕಂ ವಾಕ್ಸಿದ್ಧಿರತುಲಾ ಭವೇತ್ ॥೧೪॥

ಗೋಮಯೇನ ಮೃದಾ ವಾಪಿ ನಿರ್ಮಾಯ ಪ್ರತಿವಾದಿನಮ್ ।\\
ವಾಮಪಾದೇನ ಚಾಕ್ರಮ್ಯ ತಜ್ಜಿಹ್ವಾಂ ಕವಚಂ ಜಪೇತ್ ॥೧೫॥

ಮೂಕೋ ವೈ ಜಾಯತೇ ಶೀಘ್ರಮುನ್ಮತ್ತೋ ವಾ ಭವೇದ್ಧ್ರುವಮ್ ।\\
ಯಂ ಯಂ ಕಾಮಯತೇ ಕಾಮಂ ತಂತಮುದ್ದಿಶ್ಯ ಪಾರ್ವತಿ ॥೧೬॥

ಅಷ್ಟೋತ್ತರಶತಂ ಜಪ್ತ್ವಾ ಫಲಂ ವಿಂದತಿ ಮಾನವಃ ।\\
ಅಶ್ವತ್ಥೇ ರಾಜವಶ್ಯಾರ್ಥೀ ತೇಜಸೇಽಭಿಮುಖೋ ರವೇಃ ॥೧೭॥

ಕನ್ಯಾರ್ಥೀ ಚಂಡಿಕಾಗೇಹೇ ಗೇಹೇ ಶತ್ರುಕೃತೇ ಮಮ ।\\
ಶ್ರೀಕಾಮೋ ಬಿಲ್ವಮೂಲೇ ತು ಉದ್ಯಾನೇ ಸ್ತ್ರೀವಶೀ ಭವೇತ್ ।\\
ಪುತ್ರಾರ್ಥೀ ದಕ್ಷಿಣಾಮೂರ್ತೇಃ ಸನ್ನಿಧೌ ಮಮ ಪಾರ್ವತೀ ॥೧೮॥
\authorline{ಇತಿ ಶ್ರೀರುದ್ರಯಾಮಲೇ ಉಮಾಮಹೇಶ್ವರಸಂವಾದೇ \\ಸರಸ್ವತೀಕವಚಂ ಸಂಪೂರ್ಣಂ ॥}
\section{ಶ್ರೀ ಲಕ್ಷ್ಮೀನಾರಾಯಣ ಕವಚಂ }
ಲಕ್ಷ್ಮೀನಾರಾಯಣ  ಕವಚಸ್ಯ ಶಿವ ಋಷಿಃ । ತ್ರಿಷ್ಟುಪ್ ಛಂದಃ । ಲಕ್ಷ್ಮೀನಾರಾಯಣೋ ದೇವತಾ ।  ಶ್ರೀಂ ಬೀಜಂ । ಹ್ರೀಂ ಶಕ್ತಿಃ । ಓಂ ಕೀಲಕಮ್~। ಭೋಗಾಪವರ್ಗಸಿದ್ಧ್ಯರ್ಥೇ ಜಪೇ ವಿನಿಯೋಗಃ ॥
\newpage
\dhyana{ಪೂರ್ಣೇಂದುವದನಂ ಪೀತವಸನಂ ಕಮಲಾಸನಂ~।\\
ಲಕ್ಷ್ಮ್ಯಾ  ಶ್ರಿತಂ ಚತುರ್ಬಾಹುಂ ಲಕ್ಷ್ಮೀನಾರಾಯಣಂ ಭಜೇ }॥೯॥

ಓಂ ವಾಸುದೇವೋಽವತು ಮೇ ಮಸ್ತಕಂ ಸಶಿರೋರುಹಂ~।\\
ಹ್ರೀಂ ಲಲಾಟಂ ಸದಾ ಪಾತು ಲಕ್ಷ್ಮೀವಿಷ್ಣುಃ ಸಮಂತತಃ ॥೧೦॥

ಹ್ಸೌಃ ನೇತ್ರೇಽವತಾಲ್ಲಕ್ಷ್ಮೀಗೋವಿಂದೋ ಜಗತಾಂ ಪತಿಃ~।\\
ಹ್ರೀಂ ನಾಸಾಂ ಸರ್ವದಾ ಪಾತು ಲಕ್ಷ್ಮೀದಾಮೋದರಃ ಪ್ರಭುಃ ॥೧೧॥

ಶ್ರೀಂ ಮುಖಂ ಸತತಂ ಪಾತು ದೇವೋ ಲಕ್ಷ್ಮೀತ್ರಿವಿಕ್ರಮಃ~।\\
ಲಕ್ಷ್ಮೀ ಕಂಠಂ ಸದಾ ಪಾತು ದೇವೋ ಲಕ್ಷ್ಮೀಜನಾರ್ದನಃ ॥೧೨॥

ನಾರಾಯಣಾಯ ಬಾಹೂ ಮೇ ಪಾತು ಲಕ್ಷ್ಮೀಗದಾಗ್ರಜಃ~।\\
ನಮಃ ಪಾರ್ಶ್ವೌ ಸದಾ ಪಾತು ಲಕ್ಷ್ಮೀನಂದೈಕನಂದನಃ ॥೧೩॥

ಅಂ ಆಂ ಇಂ ಈಂ ಪಾತು ವಕ್ಷೋ ಓಂ ಲಕ್ಷ್ಮೀತ್ರಿಪುರೇಶ್ವರಃ~।\\
ಉಂ ಊಂ ಋಂ ೠಂ ಪಾತು ಕುಕ್ಷಿಂ ಹ್ರೀಂ ಲಕ್ಷ್ಮೀಗರುಡಧ್ವಜಃ ॥೧೪॥

ಲೃಂ ಲೄಂ ಏಂ ಐಂ ಪಾತು ಪೃಷ್ಠಂ ಹ್ಸೌಃ ಲಕ್ಷ್ಮೀನೃಸಿಂಹಕಃ~।\\
ಓಂ ಔಂ ಅಂ ಅಃ ಪಾತು ನಾಭಿಂ ಹ್ರೀಂ ಲಕ್ಷ್ಮೀವಿಷ್ಟರಶ್ರವಾಃ ॥೧೫॥

ಕಂ ಖಂ ಗಂ ಘಂ ಗುದಂ ಪಾತು ಶ್ರೀಂ ಲಕ್ಷ್ಮೀಕೈಟಭಾಂತಕಃ~।\\
ಚಂ ಛಂ ಜಂ ಝಂ ಪಾತು ಶಿಶ್ನಂ ಲಕ್ಷ್ಮೀ ಲಕ್ಷ್ಮೀಶ್ವರಃ ಪ್ರಭುಃ ॥೧೬॥

ಟಂ ಠಂ ಡಂ ಢಂ ಕಟಿಂ ಪಾತು ನಾರಾಯಣಾಯ ನಾಯಕಃ~।\\
ತಂ ಥಂ ದಂ ಧಂ ಪಾತು ಚೋರೂ ನಮೋ ಲಕ್ಷ್ಮೀಜಗತ್ಪತಿಃ ॥೧೭॥

ಪಂ ಫಂ ಬಂ ಭಂ ಪಾತು ಜಾನೂ ಓಂ ಹ್ರೀಂ ಲಕ್ಷ್ಮೀಚತುರ್ಭುಜಃ~।\\
ಯಂ ರಂ ಲಂ ವಂ ಪಾತು ಜಂಘೇ ಹ್ಸೌಃ ಲಕ್ಷ್ಮೀಗದಾಧರಃ ॥೧೮॥

ಶಂ ಷಂ ಸಂ ಹಂ ಪಾತು ಗುಲ್ಫೌ ಹ್ರೀಂ ಶ್ರೀಂ ಲಕ್ಷ್ಮೀರಥಾಂಗಭೃತ್~।\\
ಳಂ ಕ್ಷಃ ಪಾದೌ ಸದಾ ಪಾತು ಮೂಲಂ ಲಕ್ಷ್ಮೀಸಹಸ್ರಪಾತ್ ॥೧೯॥

ಙಂ ಞಂ ಣಂ ನಂ ಮಂ ಮೇ ಪಾತು ಲಕ್ಷ್ಮೀಶಃ ಸಕಲಂ ವಪುಃ~।\\
ಇಂದ್ರೋ ಮಾಂ ಪೂರ್ವತಃ ಪಾತು ವಹ್ನಿರ್ವಹ್ನೌ ಸದಾವತು ॥೨೦॥

ಯಮೋ ಮಾಂ ದಕ್ಷಿಣೇ ಪಾತು ನೈರೃತ್ಯಾಂ ನಿರೃತಿಶ್ಚ ಮಾಂ~।\\
ವರುಣಃ ಪಶ್ಚಿಮೇಽವ್ಯಾನ್ಮಾಂ ವಾಯವ್ಯೇಽವತು ಮಾಂ ಮರುತ್ ॥೨೧॥

ಉತ್ತರೇ ಧನದಃ ಪಾಯಾದೈಶಾನ್ಯಾಮೀಶ್ವರೋಽವತು~।\\
ವಜ್ರ ಶಕ್ತಿ ದಂಡ ಖಡ್ಗ ಪಾಶ ಯಷ್ಟಿ ಧ್ವಜಾಂಕಿತಾಃ ॥೨೨॥

ಸಶೂಲಾಃ ಸರ್ವದಾ ಪಾಂತು ದಿಗೀಶಾಃ ಪರಮಾರ್ಥದಾಃ~।\\
ಅನಂತಃ ಪಾತ್ವಧೋ ನಿತ್ಯಮೂರ್ಧ್ವೇ ಬ್ರಹ್ಮಾವತಾಚ್ಚ ಮಾಂ ॥೨೩॥

ದಶದಿಕ್ಷು ಸದಾ ಪಾತು ಲಕ್ಷ್ಮೀನಾರಾಯಣಃ ಪ್ರಭುಃ~।\\
ಪ್ರಭಾತೇ ಪಾತು ಮಾಂ ವಿಷ್ಣುರ್ಮಧ್ಯಾಹ್ನೇ ವಾಸುದೇವಕಃ ॥೨೪॥

ದಾಮೋದರೋಽವತಾತ್ ಸಾಯಂ ನಿಶಾದೌ ನರಸಿಂಹಕಃ~।\\
ಸಂಕರ್ಷಣೋಽರ್ಧರಾತ್ರೇಽವ್ಯಾತ್ ಪ್ರಭಾತೇಽವ್ಯಾತ್ ತ್ರಿವಿಕ್ರಮಃ ॥೨೫॥

ಅನಿರುದ್ಧಃ ಸರ್ವಕಾಲಂ ವಿಷ್ವಕ್ಸೇನಶ್ಚ ಸರ್ವತಃ~।\\
ರಣೇ ರಾಜಕುಲೇ ದ್ಯೂತೇ ವಿವಾದೇ ಶತ್ರುಸಂಕಟೇ~।\\
ಓಂ ಹ್ರೀಂ ಹ್ಸೌಃ ಹ್ರೀಂ ಶ್ರೀಂ ಮೂಲಂ ಲಕ್ಷ್ಮೀನಾರಾಯಣೋಽವತು ॥೨೬॥

ಓಂಓಂಓಂರಣರಾಜಚೌರರಿಪುತಃ ಪಾಯಾಚ್ಚ ಮಾಂ ಕೇಶವಃ\\
ಹ್ರೀಂಹ್ರೀಂಹ್ರೀಂಹಹಹಾ ಹಸೌಃ ಹಸಹಸೌಃ ವಹ್ನೇರ್ವತಾನ್ಮಾಧವಃ~।\\
ಹ್ರೀಂಹ್ರೀಂಹ್ರೀಂಜಲಪರ್ವತಾಗ್ನಿಭಯತಃ ಪಾಯಾದನಂತೋ ವಿಭುಃ\\
ಶ್ರೀಂಶ್ರೀಂಶ್ರೀಂಶಶಶಾಲಲಂ ಪ್ರತಿದಿನಂ ಲಕ್ಷ್ಮೀಧವಃ ಪಾತು ಮಾಂ ॥೨೭॥

ಇತೀದಂ ಕವಚಂ ದಿವ್ಯಂ ವಜ್ರಪಂಜರಕಾಭಿಧಂ~।\\
ಲಕ್ಷ್ಮೀನಾರಾಯಣಸ್ಯೇಷ್ಟಂ ಚತುರ್ವರ್ಗಫಲಪ್ರದಂ ॥೨೮॥

ಸರ್ವಸೌಭಾಗ್ಯನಿಲಯಂ ಸರ್ವಸಾರಸ್ವತಪ್ರದಂ~।\\
ಲಕ್ಷ್ಮೀಸಂವನನಂ ತತ್ವಂ ಪರಮಾರ್ಥರಸಾಯನಂ ॥೨೯॥

ಮಂತ್ರಗರ್ಭಂ ಜಗತ್ಸಾರಂ ರಹಸ್ಯಂ ತ್ರಿದಿವೌಕಸಾಂ~।\\
ದಶವಾರಂ ಪಠೇದ್ರಾತ್ರೌ ರತಾಂತೇ ವೈಷ್ಣವೋತ್ತಮಃ ॥೩೦॥

ಸ್ವಪ್ನೇ ವರಪ್ರದಂ ಪಶ್ಯೇಲ್ಲಕ್ಷ್ಮೀನಾರಾಯಣಂ ಸುಧೀಃ~।\\
ತ್ರಿಸಂಧ್ಯಂ ಯಃ ಪಠೇನ್ನಿತ್ಯಂ ಕವಚಂ ಮನ್ಮುಖೋದಿತಂ ॥೩೧॥

ಸ ಯಾತಿ ಪರಮಂ ಧಾಮ ವೈಷ್ಣವಂ ವೈಷ್ಣವೇಶ್ವರಃ~।\\
ಮಹಾಚೀನಪದಸ್ಥೋಽಪಿ ಯಃ ಪಠೇದಾತ್ಮಚಿಂತಕಃ ॥೩೨॥

ಆನಂದಪೂರಿತಸ್ತೂರ್ಣಂ ಲಭೇದ್ ಮೋಕ್ಷಂ ಸ ಸಾಧಕಃ~।\\
ಗಂಧಾಷ್ಟಕೇನ ವಿಲಿಖೇದ್ರವೌ ಭೂರ್ಜೇ ಜಪನ್ಮನುಂ ॥೩೩॥

ಪೀತಸೂತ್ರೇಣ ಸಂವೇಷ್ಟ್ಯ ಸೌವರ್ಣೇನಾಥ ವೇಷ್ಟಯೇತ್~।\\
ಧಾರಯೇದ್ಗುಟಿಕಾಂ ಮೂರ್ಧ್ನಿ ಲಕ್ಷ್ಮೀನಾರಾಯಣಂ ಸ್ಮರನ್ ॥೩೪॥

ರಣೇ ರಿಪೂನ್ ವಿಜಿತ್ಯಾಶು ಕಲ್ಯಾಣೀ ಗೃಹಮಾವಿಶೇತ್~।\\
ವಂಧ್ಯಾ ವಾ ಕಾಕವಂಧ್ಯಾ ವಾ ಮೃತವತ್ಸಾ ಚ ಯಾಂಗನಾ ॥೩೫॥

ಸಾ ಬಧ್ನೀಯಾತ್ ಕಂಠದೇಶೇ ಲಭೇತ್ ಪುತ್ರಾಂಶ್ಚಿರಾಯುಷಃ~।\\
ಗುರೂಪದೇಶತೋ ಧೃತ್ವಾ ಗುರುಂ ಧ್ಯಾತ್ವಾ ಮನುಂ ಜಪನ್ ॥೩೬॥

ವರ್ಣಲಕ್ಷಪುರಶ್ಚರ್ಯಾ ಫಲಮಾಪ್ನೋತಿ ಸಾಧಕಃ~।\\
ಬಹುನೋಕ್ತೇನ ಕಿಂ ದೇವಿ ಕವಚಸ್ಯಾಸ್ಯ ಪಾರ್ವತಿ ॥೩೭॥

ವಿನಾನೇನ ನ ಸಿದ್ಧಿಃ ಸ್ಯಾನ್ಮಂತ್ರಸ್ಯಾಸ್ಯ ಮಹೇಶ್ವರಿ~।\\
ಸರ್ವಾಗಮರಹಸ್ಯಾಢ್ಯಂ ತತ್ವಾತ್ ತತ್ವಂ ಪರಾತ್ ಪರಂ ॥೩೮॥

ಅಭಕ್ತಾಯ ನ ದಾತವ್ಯಂ ಕುಚೈಲಾಯ ದುರಾತ್ಮನೇ~।\\
ದೀಕ್ಷಿತಾಯ ಕುಲೀನಾಯ ಸ್ವಶಿಷ್ಯಾಯ ಮಹಾತ್ಮನೇ ॥೩೯॥

ಮಹಾಚೀನಪದಸ್ಥಾಯ ದಾತವ್ಯಂ ಕವಚೋತ್ತಮಂ~।\\
ಗುಹ್ಯಂ ಗೋಪ್ಯಂ ಮಹಾದೇವಿ ಲಕ್ಷ್ಮೀನಾರಾಯಣಪ್ರಿಯಂ~।\\
ವಜ್ರಪಂಜರಕಂ ವರ್ಮ ಗೋಪನೀಯಂ ಸ್ವಯೋನಿವತ್ ॥೪೦॥
\authorline{॥ಇತಿ ಶ್ರೀ ಲಕ್ಷ್ಮೀನಾರಾಯಣ ಕವಚಂ ಸಂಪೂರ್ಣಂ ॥}

\section{ಶ್ರೀಬಾಲಾತ್ರಿಪುರಸುಂದರೀ ಕವಚಂ}
ಶ್ರೀ ಪಾರ್ವತ್ಯುವಾಚ ॥\\
ದೇವ ದೇವ ಮಹಾದೇವ ಶಂಕರ ಪ್ರಾಣ ವಲ್ಲಭ~।\\
ಕವಚಂ ಶ್ರೋತುಮಿಚ್ಛಾಮಿ ಬಾಲಾಯಾ ವದ ಮೇ ಪ್ರಭೋ ॥೧॥

ಶ್ರೀ ಮಹೇಶ್ವರ ಉವಾಚ ॥\\
ಶ್ರೀಬಾಲಾಕವಚಂ ದೇವಿ ಮಹಾಪ್ರಾಣಾಧಿಕಂ ಪರಂ~।\\
ವಕ್ಷ್ಯಾಮಿ ಸಾವಧಾನಾ ತ್ವಂ ಶೃಣುಷ್ವಾವಹಿತಾ ಪ್ರಿಯೇ ॥೨॥

ಅಸ್ಯ ಶ್ರೀಬಾಲಾಕವಚಸ್ತೋತ್ರ ಮಹಾಮಂತ್ರಸ್ಯ ಶ್ರೀ ದಕ್ಷಿಣಾಮುರ್ತಿರ್ಋಷಿಃ~। ಪಂಕ್ತಿಶ್ಛಂದಃ~। ಬಾಲಾತ್ರಿಪುರಸುಂದರೀ ದೇವತಾ~। ಐಂ ಬೀಜಂ~। ಸೌಃ ಶಕ್ತಿಃ~। ಕ್ಲೀಂ ಕೀಲಕಂ~। ಶ್ರೀಬಾಲಾತ್ರಿಪುರಸುಂದರೀದೇವತಾ ಪ್ರಸಾದಸಿದ್ಧ್ಯರ್ಥೇ ಜಪೇ ವಿನಿಯೋಗಃ॥

\dhyana{ಅರುಣ ಕಿರಣ ಜಾಲೈರಂಚಿತಾಶಾವಕಾಶಾ\\
ವಿಧೃತಜಪವಟೀಕಾ ಪುಸ್ತಕಾಭೀತಿಹಸ್ತಾ~।\\
ಇತರಕರವರಾಢ್ಯಾ ಫುಲ್ಲಕಹ್ಲಾರಸಂಸ್ಥಾ\\
ನಿವಸತು ಹೃದಿ ಬಾಲಾ ನಿತ್ಯಕಲ್ಯಾಣಶೀಲಾ }॥೩॥

ಐಂ ವಾಗ್ಭವಂ ಪಾತು ಶೀರ್ಷಂ ಕ್ಲೀಂ ಕಾಮಸ್ತು ತಥಾ ಹೃದಿ~।\\
ಸೌಃ ಶಕ್ತಿಬೀಜಂ ಚ ಪಾತು ನಾಭೌ ಗುಹ್ಯೇ ಚ ಪಾದಯೋಃ ॥೪॥

ಐಂ ಕ್ಲೀಂ ಸೌಃ ವದನೇ ಪಾತು ಬಾಲಾ ಮಾಂ ಸರ್ವಸಿದ್ಧಯೇ~।\\
ಹ್‌ಸ್‌ರೈಂ ಹ್‌ಸ್‌ಕ್ಲ್ರೀಂ ಹ್‌ಸ್‌ರ್‌ಸೌಃ ಪಾತು ಸ್ಕಂಧೇ ಭೈರವೀ ಕಂಠದೇಶತಃ ॥೫॥

ಸುಂದರೀ ನಾಸದೇಶೇವ್ಯಾಚ್ಚರ್ಚೇ ಕಾಮಕಲಾ ಸದಾ~।\\
ಭ್ರೂನಾಸಯೋರಂತರಾಲೇ ಮಹಾತ್ರಿಪುರಸುಂದರೀ ॥೬॥

ಲಲಾಟೇ ಸುಭಗಾ ಪಾತು ಭಗಾ ಮಾಂ ಕಂಠದೇಶತಃ~।\\
ಭಗೋದಯಾ ತು ಹೃದಯೇ ಉದರೇ ಭಗಸರ್ಪಿಣೀ ॥೭॥

ಭಗಮಾಲಾ ನಾಭಿದೇಶೇ ಲಿಂಗೇ ಪಾತು ಮನೋಭವಾ~।\\
ಗುಹ್ಯೇ ಪಾತು ಮಹಾವೀರಾ ರಾಜರಾಜೇಶ್ವರೀ ಶಿವಾ ॥೮॥

ಚೈತನ್ಯರೂಪಿಣೀ ಪಾತು ಪಾದಯೋರ್ಜಗದಂಬಿಕಾ~।\\
ನಾರಾಯಣೀ ಸರ್ವಗಾತ್ರೇ ಸರ್ವಕಾರ್ಯಶುಭಂಕರೀ ॥೯॥

ಬ್ರಹ್ಮಾಣೀ ಪಾತು ಮಾಂ ಪೂರ್ವೇ ದಕ್ಷಿಣೇ ವೈಷ್ಣವೀ ತಥಾ~।\\
ಪಶ್ಚಿಮೇ ಪಾತು ವಾರಾಹೀ ಹ್ಯುತ್ತರೇ ತು ಮಹೇಶ್ವರೀ ॥೧೦॥

ಆಗ್ನೇಯ್ಯಾಂ ಪಾತು ಕೌಮಾರೀ ಮಹಾಲಕ್ಷ್ಮೀಶ್ಚ ನೈರ್ಋತೇ~।\\
ವಾಯವ್ಯೇ ಪಾತು ಚಾಮುಂಡಾ ಚೇಂದ್ರಾಣೀ ಪಾತು ಚೇಶಕೇ ॥೧೧॥

ಜಲೇ ಪಾತು ಮಹಾಮಾಯಾ ಪೃಥಿವ್ಯಾಂ ಸರ್ವಮಂಗಲಾ~।\\
ಸ್ಕ್ಲೀಂ ಮಾಂ ಸರ್ವತಃ ಪಾತು ಸಕಲಹ್ರೀಂ ಪಾತು ಸಂಧಿಷು ॥೧೨॥

ಜಲೇ ಸ್ಥಲೇ ತಥಾಕಾಶೇ ದಿಕ್ಷು ರಾಜಗೃಹೇ ತಥಾ~।\\
ಕ್ಷೂಂಕ್ಷೇಂಮಾಂ ತ್ವರಿತಾಪಾತು ಸಹ್ರೀಂಸಕ್ಲೀಂ ಮನೋಭವಾ ॥೧೩॥

ಹಂಸಃ ಪಾಯಾನ್ಮಹಾದೇವೀ ಪರಂ ನಿಷ್ಕಲದೇವತಾ~।\\
ವಿಜಯಾ ಮಂಗಲಾ ದೂತೀ ಕಲ್ಯಾಣೀ ಭಗಮಾಲಿನೀ ॥೧೪॥

ಜ್ವಾಲಾಮಾಲಿನಿನಿತ್ಯಾ ಚ ಸರ್ವದಾ ಪಾತು ಮಾಂ ಶಿವಾ~।\\
ಇತ್ಯೇತತ್ಕವಚಂ ದೇವಿ ಬಾಲಾದೇವ್ಯಾಃ ಪ್ರಕೀರ್ತಿತಂ~।\\
ಸರ್ವಸ್ವಂ ಮೇ ತವ ಪ್ರೀತ್ಯಾ ಪ್ರಾಣವದ್ರಕ್ಷಿತಂ ಕುರು॥೧೫॥

\authorline{ಇತಿ ಶ್ರೀರುದ್ರಯಾಮಲೇ ಶ್ರೀಶಿವಪಾರ್ವತಿಸಂವಾದೇ\\ ಶ್ರೀಬಾಲಾತ್ರಿಪುರಸುಂದರೀ ಮಂತ್ರಕವಚಂ ಸಂಪೂರ್ಣಂ ॥}

\section{ಶೀತಲಾ ಕವಚಮ್}
ಅಸ್ಯ ಶ್ರೀ ಶೀತಲಾಕವಚಸ್ಯ ಮಹೇಶ್ವರಃ ಋಷಿಃ । ಅನುಷ್ಟುಪ್ ಛನ್ದಃ ।
ಶೀತಲಾ ದೇವತಾ~। ಲಕ್ಷ್ಮೀಬೀಜಂ । ರಮಾ ಶಕ್ತಿಃ ।ತಾರಂ ಕೀಲಕಮ್~।
ಲೂತಾವಿಸ್ಫೋಟಕಾದೀನಾಂ ಶಾಂತ್ಯರ್ಥೇ ಜಪೇ ವಿನಿಯೋಗಃ ।

\dhyana{ಉದ್ಯತ್ಸೂರ್ಯನಿಭಾಂ ನವೇಂದುಮುಕುಟಾಂ ಸೂರ್ಯಾಗ್ನಿನೇತ್ರೋಜ್ಜ್ವಲಾಂ\\
ನಾನಾಗಂಧ ವಿಲೇಪನಾಂ ಮೃದುತನುಂ ದಿವ್ಯಾಂಬರಾಲಂಕೃತಾಮ್ ।\\
ದೋರ್ಭ್ಯಾಂ ಸಂದಧತೀಂ ವರಾಭಯಯುಗಂ ವಾಹೇ ಸ್ಥಿತಾಂ ರಾಸಭೇ\\
ಭಕ್ತಾಭೀಷ್ಟ-ಫಲ-ಪ್ರದಾಂ ಭಗವತೀಂ ಶ್ರೀಶೀತಲಾಂ ತ್ವಾಂ ಭಜೇ ॥}

ಶೀತಲಾ ಪಾತು ಮೇ ಪ್ರಾಣೇ ರುನುಕೀ ಪಾತು ಚಾಪಾನೇ ।\\
ಸಮಾನೇ ಝುನುಕೀ ಪಾತು ಉದಾನೇ ಪಾತು ಮನ್ದಲಾ ॥೧॥

ವ್ಯಾನೇ ಚ ಸೇಢಲಾ ಪಾತು ಮನೋ ಮೇ ಶಾಂಕರೀ ತಥಾ ।\\
ಪಾತು ಮಾಮಿಂದ್ರಿಯಾನ್ ಸರ್ವಾನ್ ಶ್ರೀದುರ್ಗಾ ವಿನ್ಧ್ಯವಾಸಿನೀ ॥೨॥

ಮಮ ಪಾತು ಶಿರೋ ದುರ್ಗಾ ಕಮಲಾ ಪಾತು ಮಸ್ತಕಮ್ ।\\
ಹ್ರೀಂ ಮೇ ಪಾತು ಭ್ರುವೋರ್ಮಧ್ಯೇ ಭವಾನೀ ಭುವನೇಶ್ವರೀ ॥\\
ಪಾತು ಮೇ ಮಧುಮತೀ ದೇವೀ ಓಂಕಾರಂ ಭೃಕುಟಿದ್ವಯಮ್ ॥೩॥

ನಾಸಿಕಾಂ ಶಾರದಾ ಪಾತು ತಮಸಾ ವರ್ತ್ಮಸಂಯುತಮ್ ।\\
ನೇತ್ರೇ ಜ್ವಾಲಾಮುಖೀ ಪಾತು ಭೀಷಣಾ ಪಾತು ಶ್ರುತೀ ಮೇ ॥೪॥

ಕಪೋಲೌ ಕಾಲಿಕಾ ಪಾತು ಸುಮುಖೀ ಪಾತು ಚೋಷ್ಠಯೋಃ ।\\
ಸನ್ಧ್ಯಯೋಃ ತ್ರಿಪುರಾ ಪಾತು ದನ್ತೇ ಚ ರಕ್ತದನ್ತಿಕಾ ॥೫॥

ಜಿಹ್ವಾಂ ಸರಸ್ವತೀ ಪಾತು ತಾಲುಕೇ ಚ ವಾಗ್ವಾದಿನೀ ।\\
ಕಣ್ಠೇ ಪಾತು ತು ಮಾತಂಗೀ ಗ್ರೀವಾಯಾಂ ಭದ್ರಕಾಲಿಕಾ ॥೬॥

ಸ್ಕನ್ಧೌ ಚ ಪಾತು ಮೇ ಛಿನ್ನಾ ಕಕುದಂ ಸ್ಕನ್ದ-ಮಾತರಃ ।\\
ಬಾಹುಯುಗ್ಮೌ ಚ ಮೇ ಪಾತು ಶ್ರೀದೇವೀ ಬಗಲಾಮುಖೀ ॥೭॥

ಕರೌ ಮೇ ಭೈರವೀ ಪಾತು ಪೃಷ್ಠೇ ಪಾತು ಧನುರ್ಧರೀ~।\\
ವಕ್ಷಃಸ್ಥಲೇ ಚ ಮೇ ಪಾತು ದುರ್ಗಾ ಮಹಿಷಮರ್ದಿನೀ ॥೮॥

ಹೃದಯೇ ಲಲಿತಾ ಪಾತು ಕುಕ್ಷೌ ಪಾತು ಮಹೇಶ್ವರೀ ।\\
ಪಾರ್ಶ್ವೌ ಚ ಗಿರಿಜಾ ಪಾತು ಚಾನ್ನಪೂರ್ಣಾ ತು ಚೋದರಮ್ ॥೯॥

ನಾಭಿಂ ನಾರಾಯಣೀ ಪಾತು ಕಟಿಂ ಮೇ ಸರ್ವಮಂಗಲಾ ।\\
ಜಂಘಯೋ ರ್ಮೇ ಸದಾ ಪಾತು ದೇವೀ ಕಾತ್ಯಾಯನೀ ಪರಾ ॥೧೦॥

ಬ್ರಹ್ಮಾಣೀ ಶಿಶ್ನಂ ಪಾತು ವೃಷಣಂ ಪಾತು ಕಪಾಲಿನೀ ।\\
ಗುಹ್ಯಂ ಗುಹ್ಯೇಶ್ವರೀ ಪಾತು ಜಾನುನೋರ್ಜಗದೀಶ್ವರೀ ॥೧೧॥

ಪಾತು ಗುಲ್ಫೌ ತು ಕೌಮಾರೀ ಪಾದಪೃಷ್ಠಂ ತು ವೈಷ್ಣವೀ~।\\
ವಾರಾಹೀ ಪಾತು ಪಾದಾಗ್ರೇ ಇಂದ್ರಾಣೀ ಸರ್ವಮರ್ಮಸು ॥೧೨॥

ಮಾರ್ಗೇ ರಕ್ಷತು ಚಾಮುಂಡಾ ವನೇ ತು ವನವಾಸಿನೀ ।\\
ಜಲೇ ಚ ವಿಜಯಾ ರಕ್ಷೇತ್ ವಹ್ನೌ ಮೇ ಚಾಪರಾಜಿತಾ ॥೧೩॥

ರಣೇ ಕ್ಷೇಮಂಕರೀ ರಕ್ಷೇತ್ ಸರ್ವತ್ರ ಸರ್ವಮಂಗಲಾ ।\\
ಭವಾನೀ ಪಾತು ಬಂಧೂನ್ ಮೇ ಭಾರ್ಯಾಂ ರಕ್ಷತು ಚಾಂಬಿಕಾ ॥೧೪॥

ಪುತ್ರಾನ್ ರಕ್ಷತು ಮಾಹೇಂದ್ರೀ ಕನ್ಯಕಾಂ ಪಾತು ಶಾಂಭವೀ ।\\
ಗೃಹೇಷು ಸರ್ವಕಲ್ಯಾಣೀ ಪಾತು ನಿತ್ಯಂ ಮಹೇಶ್ವರೀ ॥೧೫॥

ಪೂರ್ವೇ ಕಾದಂಬರೀ ಪಾತು ವಹ್ನೌ ಶುಕ್ಲೇಶ್ವರೀ ತಥಾ ।\\
ದಕ್ಷಿಣೇ ಕರಾಲಿನೀ ಪಾತು, ಪ್ರೇತಾರೂಢಾ ತು ನೈರ್ಋತೇ ॥೧೬॥

ಪಾಶಹಸ್ತಾ ಪಶ್ಚಿಮೇ ಪಾಯಾತ್ ವಾಯವ್ಯೇ ಮೃಗವಾಹಿನೀ ।\\
ಪಾತು ಮೇ ಚೋತ್ತರೇ ದೇವೀ ಯಕ್ಷಿಣೀ ಸಿಂಹವಾಹಿನೀ ।\\
ಈಶಾನೇ ಶೂಲಿನೀ ಪಾತು ಊರ್ಧ್ವೇ ಚ ಖಗಗಾಮಿನೀ ॥೧೭॥

ಅಧಸ್ತಾತ್ ವೈಷ್ಣವೀ ಪಾತು, ಸರ್ವತ್ರ ನಾರಸಿಂಹಿಕಾ ।\\
ಪ್ರಭಾತೇ ಸುಂದರೀ ಪಾತು ಮಧ್ಯಾಹ್ನೇ ಜಗದಮ್ಬಿಕಾ ॥೧೮॥

ಸಾಯಾಹ್ನೇ ಚಂಡಿಕಾ ಪಾತು ನಿಶೀಥೇಽತ್ರ ನಿಶಾಚರೀ ।\\
ನಿಶಾಂತೇ ಖೇಚರೀ ಪಾತು ಸರ್ವದಾ ದಿವ್ಯಯೋಗಿನೀ ॥೧೯॥

ವಾಯೌ ಮಾಂ ಪಾತು ವೇತಾಲೀ ವಾಹನೇ ವಜ್ರಧಾರಿಣೀ ।\\
ಸಿಂಹಾ ಸಿಂಹಾಸನೇ ಪಾತು ಶಯ್ಯಾಂ ಚ ಭಗಮಾಲಿನೀ ॥೨೦॥

ಸರ್ವರೋಗೇಷು ಮಾಂ ಪಾತು ಕಾಲರಾತ್ರಿಸ್ವರೂಪಿಣೀ ।\\
ಯಕ್ಷೇಭ್ಯೋ ಯಾಕಿನೀ ಪಾತು ರಾಕ್ಷಸೇ ಡಾಕಿನೀ ತಥಾ ॥೨೧॥

ಭೂತಪ್ರೇತಪಿಶಾಚೇಭ್ಯೋ ಹಾಕಿನೀ ಪಾತು ಮಾಂ ಸದಾ ।\\
ಮಂತ್ರಂ ಮಂತ್ರಾಭಿಚಾರೇಷು ಶಾಕಿನೀ ಪಾತು ಮಾಂ ಸದಾ ॥೨೨॥

ಸರ್ವತ್ರ ಸರ್ವದಾ ಪಾತು ಶ್ರೀದೇವೀ ಗಿರಿಜಾತ್ಮಜಾ ।\\
ಇತ್ಯೇತತ್ ಕಥಿತಂ ಗುಹ್ಯಂ ಶೀತಲಾಕವಚಮುತ್ತಮಮ್ ॥೨೩॥

\section{ಅನ್ನಪೂರ್ಣಾ ಕವಚಂ}
ದೇವ್ಯುವಾಚ॥\\
ಭವತಾ ತ್ವನ್ನಪೂರ್ಣಾಯಾ ಯಾ ಯಾ ವಿದ್ಯಾಃ ಸುದುರ್ಲಭಾಃ~।\\
ಕೃಪಯಾ ಕಥಿತಾಃ ಸರ್ವಾಃ ಶ್ರುತಾಶ್ಚಾಧಿಗತಾ ಮಯಾ ॥೧ ॥

ಸಾಂಪ್ರತಂ ಶ್ರೋತುಮಿಚ್ಛಾಮಿ ಕವಚಂ ಮಂತ್ರ ವಿಗ್ರಹಂ~।\\
ಈಶ್ವರ ಉವಾಚ॥\\
ಶೃಣು ಪಾರ್ವತಿ ವಕ್ಷ್ಯಾಮಿ ಸಾವಧಾನಾವಧಾರಯ॥೨ ॥

ಬ್ರಹ್ಮವಿದ್ಯಾ ಸ್ವರೂಪಂ ಚ ಮಹದೈಶ್ವರ್ಯದಾಯಕಂ~।\\
ಪಠನಾದ್ಧಾರಣಾನ್ಮರ್ತ್ಯ ಸ್ತ್ರೈಲೋಕ್ಯೈಶ್ವರ್ಯ ಭಾಗ್ಭವೇತ್॥೩ ॥

ತ್ರೈಲೋಕ್ಯ ರಕ್ಷಣಸ್ಯಾಸ್ಯ ಕವಚಸ್ಯ ಋಷಿಃ ಶಿವಃ~।\\
ಛಂದೋ ವಿರಾಡ್ ದೇವತಾ ಸ್ಯಾದನ್ನಪೂರ್ಣಾ ಸಮೃದ್ಧಿದಾ॥೪ ॥

ಧರ್ಮಾರ್ಥಕಾಮಮೋಕ್ಷೇಷು ವಿನಿಯೋಗಃ ಪ್ರಕೀರ್ತಿತಃ~।\\
ಹ್ರೀಂ ನಮೋ ಭಗವತ್ಯಂತೇ ಮಾಹೇಶ್ವರಿ ಪದಂ ತತಃ॥೫ ॥
\newpage
ಅನ್ನಪೂರ್ಣೇ ತತಃ ಸ್ವಾಹಾ ಚೈಷಾ ಸಪ್ತದಶಾಕ್ಷರೀ~।\\
ಪಾತು ಮಾಮನ್ನಪೂರ್ಣಾ ಸಾ ಯಾ ಖ್ಯಾತಾ ಭುವನತ್ರಯೇ ॥೬ ॥

ವಿಮಾಯಾ ಪ್ರಣವಾದ್ಯೈಷಾ ತಥಾ ಸಪ್ತದಶಾಕ್ಷರೀ~।\\
ಪಾತ್ವನ್ನಪೂರ್ಣಾ ಸರ್ವಾಂಗೇ ರತ್ನಕುಂಭಾನ್ನಪಾತ್ರದಾ॥೭ ॥

ಶ್ರೀಬೀಜಾದ್ಯಾ ತಥೈವೈಷಾ ದ್ವಿರಂಧ್ರಾರ್ಣಾ ತಥಾ ಮುಖಂ~।\\
ಪ್ರಣವಾದ್ಯಾ ಭ್ರುವೌ ಪಾತು ಕಂಠಂ ವಾಗ್ಬೀಜಪೂರ್ವಿಕಾ॥೮ ॥

ಕಾಮಬೀಜಾದಿಕಾ ಚೈಷಾ ಹೃದಯಂ ತು ಮಹೇಶ್ವರೀ~।\\
ಐಂಶ್ರೀಂಹ್ರೀಂ ಚ ನಮೋಽಂತೇ ತು ಭಗವತಿ ಪದಂ ತತಃ॥೯ ॥

ಮಾಹೇಶ್ವರಿ ಪದಂ ಚಾನ್ನಪೂರ್ಣೇ ಸ್ವಾಹೇತಿ ಪಾತು ಮೇ~।\\
ನಾಭಿಮೇಕೋನವಿಂಶಾರ್ಣಾ ಪಾಯಾನ್ಮಾಹೇಶ್ವರೀ ಸದಾ॥೧೦ ॥

ತಾರಂ ಮಾಯಾ ರಮಾ ಕಾಮಃ ಷೋಡಶಾರ್ಣಾ ತತಃ ಪರಂ~।\\
ಶಿರಃಸ್ಥಾ ಸರ್ವದಾ ಪಾತು ವಿಂಶತ್ಯರ್ಣಾತ್ಮಿಕಾ ಪರಾ॥೧೧ ॥

ಅನ್ನಪೂರ್ಣಾ ಮಹಾವಿದ್ಯಾ ಹ್ರೀಂ ಪಾತು ಭುವನೇಶ್ವರೀ~।\\
ಶಿರಃ ಶ್ರೀಂ ಹ್ರೀಂ ತಥಾ ಕ್ಲೀಂ ಚ ತ್ರಿಪುಟಾ ಪಾತು ಮೇಗುದಂ॥೧೨॥

ಷಡ್‌ದೀರ್ಘ ಭಾಜಾ ಬೀಜೇನ ಷಡಂಗಾನಿ ಪುನಂತು ಮಾಂ~।\\
ಇಂದ್ರೋ ಮಾಂ ಪಾತು ಪೂರ್ವಂ ಚ ವಹ್ನಿಕೋಣೇನಲೋವತು॥೧೩ ॥

ಯಮೋಮಾಂ ದಕ್ಷಿಣೇ ಪಾತು ನೈರ್ಋತ್ಯಾಂ ನಿರ್ಋತಿಶ್ಚ ಮಾಂ~।\\
ಪಶ್ಚಿಮೇ ವರುಣಃ ಪಾತು ವಾಯವ್ಯಾಂ ಪವನೋವತು॥೧೪ ॥

ಕುಬೇರಶ್ಚೋತ್ತರೇ ಪಾತು ಚೈಶಾನ್ಯಾಂ ಶಂಕರೋಽವತು~।\\
ಊರ್ಧ್ವಾಧಃ ಪಾತು ಸತತಂ ಬ್ರಹ್ಮಾನಂತೋ ಯಥಾಕ್ರಮಾತ್॥೧೫ ॥

ಪಾಂತು ವಜ್ರಾದ್ಯಾಯುಧಾನಿ ದಶದಿಕ್ಷು ಯಥಾಕ್ರಮಂ~।\\
ಇತಿ ತೇ ಕಥಿತಂ ಪುಣ್ಯಂ ತ್ರೈಲೋಕ್ಯ ರಕ್ಷಣಂ ಪರಂ॥೧೬ ॥

ಯದ್ಧೃತ್ವಾ ಪಠನಾದ್ದೇವಾಃ ಸರ್ವೈಶ್ವರ್ಯಮವಾಪ್ನುಯುಃ~।\\
ಬ್ರಹ್ಮಾ ವಿಷ್ಣುಶ್ಚ ರುದ್ರಶ್ಚ ಧಾರಣಾತ್ ಪಠನಾದ್ಯತಃ॥೧೭ ॥

ಸೃಜತ್ಯವತಿ ಹಂತ್ಯೇವ ಕಲ್ಪೇ ಕಲ್ಪೇ ಪೃಥಕ್ ಪೃಥಕ್~।\\
ಪುಷ್ಪಾಂಜಲ್ಯಷ್ಟಕಂ ದೇವ್ಯೈ ಮೂಲೇನೈವ ಸಮರ್ಪಯೇತ್॥೧೮ ॥

ಕವಚಸ್ಯಾಸ್ಯ ಪಠನಾತ್ ಪೂಜಾಯಾಃ ಫಲಮಾಪ್ನುಯಾತ್~।\\
ವಾಣೀ ವಕ್ತ್ರೇ ವಸೇತ್ತಸ್ಯ ಸತ್ಯಂ ಸತ್ಯಂ ನ ಸಂಶಯಃ॥೧೯ ॥

ಅಷ್ಟೋತ್ತರ ಶತಂ ಚಾಸ್ಯ ಪುರಶ್ಚರ್ಯಾ ವಿಧಿಃ ಸ್ಮೃತಃ~।\\
ಭೂರ್ಜೇ ವಿಲಿಖ್ಯ ಗುಟಿಕಾಂ ಸ್ವರ್ಣಸ್ಥಾಂ ಧಾರಯೇದ್ಯದಿ॥೨೦ ॥

ಕಂಠೇ ವಾ ದಕ್ಷಿಣೇ ಬಾಹೌ ಸೋಪಿ ಪುಣ್ಯವತಾಂ ವರಃ~।\\
ಸರ್ವಾಣ್ಯಸ್ತ್ರಾಣಿ ಶಸ್ತ್ರಾಣಿ ತದ್ಗಾತ್ರಂ ಪ್ರಾಪ್ಯ ಪಾರ್ವತಿ~।\\
ಮಾಲ್ಯಾನಿ ಕುಸುಮಾನೀವ ಸುಖದಾನಿ ಭವಂತಿ ಹಿ॥೨೧ ॥
\authorline{॥ಇತಿ ಭೈರವ ತಂತ್ರೇ ಅನ್ನಪೂರ್ಣಾ ಕವಚಂ॥}
\section{ನಿತ್ಯಾಕವಚಂ}
ಸಮಸ್ತಾಪದ್ವಿಮುಕ್ತ್ಯರ್ಥಂ ಸರ್ವಸಂಪದವಾಪ್ತಯೇ ।\\
ಭೂತಪ್ರೇತಪಿಶಾಚಾದಿಪೀಡಾಶಾಂತ್ಯೈ ಸುಖಾಪ್ತಯೇ ॥೧॥

ಸಮಸ್ತರೋಗನಾಶಾಯ ಸಮರೇ ವಿಜಯಾಯ ಚ ।\\
ಚೋರಸಿಂಹದ್ವೀಪಿಗಜ ಗವಯಾದಿಭಯಾನಕೇ ॥೨॥
\newpage
ಅರಣ್ಯೇ ಶೈಲಗಹನೇ ಮಾರ್ಗೇ ದುರ್ಭಿಕ್ಷಕೇ ತಥಾ ।\\
ಸಲಿಲಾಗ್ನಿ ಮರುತ್ಪೀಡಾಸ್ವಬ್ಧೌ ಪೋತಾದಿಸಂಕಟೇ ॥೩॥

ಪ್ರಜಪ್ಯ ನಿತ್ಯಾಕವಚಂ ಸಕೃತ್ಸರ್ವಂ ತರತ್ಯಸೌ ।\\
ಸುಖೀ ಜೀವತಿ ನಿರ್ದ್ವಂದ್ವೋ ನಿಃಸಪತ್ನೋ ಜಿತೇಂದ್ರಿಯಃ ॥೪॥

ಶೃಣು ತತ್ ಕವಚಂ ದೇವಿ ವಕ್ಷ್ಯೇ ತವ ತವಾತ್ಮಕಂ ।\\
ಯೇನಾಹಮಪಿ ಯುದ್ಧೇಷು ದೇವಾಸುರಜಯೀ ಸದಾ ॥೫॥

ಸರ್ವತಃ ಸರ್ವದಾತ್ಮಾನಂ ಲಲಿತಾ ಪಾತು ಸರ್ವಗಾ ।\\
ಕಾಮೇಶೀ ಪುರತಃ ಪಾತು ಭಗಮಾಲಾ ತ್ವನಂತರಾಂ ॥೬॥

ದಿಶಂ ಪಾತು ತಥಾ ದಕ್ಷಪಾರ್ಶ್ವಂ ಮೇ ಪಾತು ಸರ್ವದಾ ।\\
ನಿತ್ಯಕ್ಲಿನ್ನಾಥ ಭೇರುಂಡಾ ದಿಶಂ ಪಾತು ಸದಾ ಮಮ ॥೭॥

ತಥೈವ ಪಶ್ಚಿಮಂ ಭಾಗಂ ರಕ್ಷೇತ್ ಸಾ ವಹ್ನಿವಾಸಿನೀ ।\\
ಮಹಾವಜ್ರೇಶ್ವರೀ ರಕ್ಷೇದನಂತರದಿಶಂ ಸದಾ ॥೮॥

ವಾಮಪಾರ್ಶ್ವೇ ಸದಾ ಪಾತು ದೂತೀ ಮೇ ತ್ವರಿತಾ ತತಃ ।\\
ಪಾಲಯೇತ್ತು ದಿಶಂ ಚಾನ್ಯಾಂ ರಕ್ಷೇನ್ಮಾಂ ಕುಲಸುಂದರೀ ॥೯॥

ನಿತ್ಯಾ ಮಾಮೂರ್ಧ್ವತಃ ಪಾತು ಸಾಧೋ ಮೇ ಪಾತು ಸರ್ವದಾ ।\\
ನಿತ್ಯಾ ನೀಲಪತಾಕಾಖ್ಯಾ ವಿಜಯಾ ಸರ್ವತಶ್ಚ ಮಾಂ ॥೧೦॥

ಕರೋತು ಮೇ ಮಂಗಲಾನಿ ಸರ್ವದಾ ಸರ್ವಮಂಗಲಾ ।\\
ದೇಹೇಂದ್ರಿಯಮನಃಪ್ರಾಣಾನ್ ಜ್ವಾಲಾಮಾಲಿನಿವಿಗ್ರಹಾ ॥೧೧॥

ಪಾಲಯೇದನಿಶಂ ಚಿತ್ರಾ ಚಿತ್ತಂ ಮೇ ಪಾತು ಸರ್ವದಾ ।\\
ಕಾಮಾತ್ ಕ್ರೋಧಾತ್ತಥಾ ಲೋಭಾನ್ಮೋಹಾತ್ಪಾಯಾನ್ಮದಾದಪಿ ॥೧೨॥

ಪಾಪಾನ್ಮತ್ಸರತಃ ಶೋಕಾತ್ ಸಂಶಯಾತ್ ಸರ್ವತಃ ಸದಾ ।\\
ಸ್ತೈಮಿತ್ಯಾಚ್ಚ ಸಮುದ್ಯೋಗಾದಶುಭೇಷು ತು ಕರ್ಮಸು ॥೧೩॥

ಅಸತ್ಯಾತ್ ಕ್ರೂರಚಿಂತಾತೋ ಹಿಂಸಾತಶ್ಚೌರ್ಯತಸ್ತಥಾ ।\\
ರಕ್ಷಂತು ಮಾಂ ಸರ್ವದಾ ತಾಃ ಕುರ್ವಂತ್ವಿಚ್ಛಾಂ ಶುಭೇಷು ಚ ॥೧೪॥

ನಿತ್ಯಾಃ ಷೋಡಶ ಮಾಂ ಪಾಂತು ಗಜಾರೂಢಾಃ ಸ್ವಶಕ್ತಿಭಿಃ ।\\
ತಥಾ ಹಯಸಮಾರೂಢಾಃ ಪಾಂತು ಮಾಂ ಸರ್ವತಃ ಸದಾ ॥೧೫॥

ಸಿಂಹಾರೂಢಾಸ್ತಥಾ ಪಾಂತು ಮಾಂ ತರಕ್ಷುಗತಾ ಅಪಿ ।\\
ರಥಾರೂಢಾಶ್ಚ ಮಾಂ ಪಾಂತು ಸರ್ವತಃ ಸರ್ವದಾ ರಣೇ ॥೧೬॥

ತಾರ್ಕ್ಷ್ಯಾರೂಢಾಶ್ಚ ಮಾಂ ಪಾಂತು ತಥಾ ವ್ಯೋಮಗತಾಸ್ತಥಾ ।\\
ಭೂಗತಾಃ ಸರ್ವದಾ ಪಾಂತು ಮಾಂಚ ಸರ್ವತ್ರ ಸರ್ವದಾ ॥೧೭॥

ಭೂತಪ್ರೇತಪಿಶಾಚಾಪಸ್ಮಾರಕೃತ್ಯಾದಿಕಾನ್ ಗದಾನ್ ।\\
ದ್ರಾವಯಂತು ಸ್ವಶಕ್ತೀನಾಂ ಭೀಷಣೈರಾಯುಧೈರ್ಮಮ ॥೧೮॥

ಗಜಾಶ್ವದ್ವೀಪಿಪಂಚಾಸ್ಯತಾರ್ಕ್ಷ್ಯಾರೂಢಾಖಿಲಾಯುಧಾಃ ।\\
ಅಸಂಖ್ಯಾಃ ಶಕ್ತಯೋ ದೇವ್ಯಾಃ ಪಾಂತು ಮಾಂ ಸರ್ವತಃ ಸದಾ ॥೧೯॥

ಸಾಯಂ ಪ್ರಾತರ್ಜಪೇನ್ನಿತ್ಯಾಕವಚಂ ಸರ್ವರಕ್ಷಕಂ ।\\
ಕದಾಚಿನ್ನಾಶುಭಂ ಪಶ್ಯೇನ್ನ ಶೃಣೋತಿ ಚ ಮತ್ಸಮಃ ॥೨೦॥
\authorline{ಇತಿ  ನಿತ್ಯಾಕವಚಂ ಸಂಪೂರ್ಣಂ ॥}
\newpage
\section{ಶ್ರೀಕಾಲಿಕಾಕವಚಂ}
\addcontentsline{toc}{section}{ಶ್ರೀಕಾಲಿಕಾಕವಚಂ}
ಶ್ರೀಸದಾಶಿವ ಉವಾಚ ।\\
ಕಥಿತಂ ಪರಮಂ ಬ್ರಹ್ಮ ಪ್ರಕೃತೇಃ ಸ್ತವನಂ ಮಹತ್ ।\\
ಆದ್ಯಾಯಾಃ ಶ್ರೀಕಾಲಿಕಾಯಾಃ ಕವಚಂ ಶೃಣು ಸಾಂಪ್ರತಂ ॥೧॥

ತ್ರೈಲೋಕ್ಯವಿಜಯಸ್ಯಾಸ್ಯ ಕವಚಸ್ಯ ಋಷಿಃ ಶಿವಃ ।\\
ಛಂದೋಽನುಷ್ಟುಬ್ದೇವತಾ ಚ ಆದ್ಯಾ ಕಾಲೀ ಪ್ರಕೀರ್ತಿತಾ ॥೨॥

ಮಾಯಾಬೀಜಂ ಬೀಜಮಿತಿ ರಮಾ ಶಕ್ತಿರುದಾಹೃತಾ ।\\
ಕ್ರೀಂ ಕೀಲಕಂ ಕಾಮ್ಯಸಿದ್ಧೌ ವಿನಿಯೋಗಃ ಪ್ರಕೀರ್ತಿತಃ ॥೩॥

ಹ್ರೀಮಾದ್ಯಾ ಮೇ ಶಿರಃ ಪಾತು ಶ್ರೀಂ ಕಾಲೀ ವದನಂ ಮಮ ।\\
ಹೃದಯಂ ಕ್ರೀಂ ಪರಾ ಶಕ್ತಿ ಪಾಯಾತ್ಕಂಠಂ ಪರಾತ್ಪರಾ ॥೪॥

ನೇತ್ರೇ ಪಾತು ಜಗದ್ಧಾತ್ರೀ ಕರ್ಣೌ ರಕ್ಷತು ಶಂಕರೀ ।\\
ಘ್ರಾಣಂ ಪಾತು ಮಹಾಮಾಯಾ ರಸನಾಂ ಸರ್ವಮಂಗಲಾ ॥೫॥

ದಂತಾನ್ ರಕ್ಷತು ಕೌಮಾರೀ ಕಪೋಲೌ ಕಮಲಾಲಯಾ ।\\
ಓಷ್ಠಾಧರೌ ಕ್ಷಮಾ ರಕ್ಷೇಚ್ಚಿಬುಕಂ ಚಾರುಹಾಸಿನೀ ॥೬॥

ಗ್ರೀವಾಂ ಪಾಯಾತ್ಕುಲೇಶಾನೀ ಕಕುತ್ಪಾತು ಕೃಪಾಮಯೀ ।\\
ದ್ವೌ ಬಾಹೂ ಬಾಹುದಾ ರಕ್ಷೇತ್ಕರೌ ಕೈವಲ್ಯದಾಯಿನೀ ॥೭॥

ಸ್ಕಂಧೌ ಕಪರ್ದಿನೀ ಪಾತು ಪೃಷ್ಠಂ ತ್ರೈಲೋಕ್ಯತಾರಿಣೀ ।\\
ಪಾರ್ಶ್ವೇ ಪಾಯಾದಪರ್ಣಾ ಮೇ ಕಟಿಂ ಮೇ ಕಮಠಾಸನಾ ॥೮॥

ನಾಭೌ ಪಾತು ವಿಶಾಲಾಕ್ಷೀ ಪ್ರಜಸ್ಥಾನಂ ಪ್ರಭಾವತೀ ।\\
ಊರೂ ರಕ್ಷತು ಕಲ್ಯಾಣೀ ಪಾದೌ ಮೇ ಪಾತು ಪಾರ್ವತೀ ॥೯॥

ಜಯದುರ್ಗಾವತು ಪ್ರಾಣಾನ್ಸರ್ವಾಂಗಂ ಸರ್ವಸಿದ್ಧಿದಾ ।\\
ರಕ್ಷಾಹೀನಂ ತು ಯತ್ಸ್ಥಾನಂ ವರ್ಜಿತಂ ಕವಚೇನ ಚ ॥೧೦॥

ತತ್ಸರ್ವಂ ಮೇ ಸದಾ ರಕ್ಷೇದಾದ್ಯಾ ಕಾಲೀ ಸನಾತನೀ ।\\
ಇತಿ ತೇ ಕಥಿತಂ ದಿವ್ಯಂ ತ್ರೈಲೋಕ್ಯವಿಜಯಾಭಿಧಂ ॥೧೧॥

ಕವಚಂ ಕಾಲಿಕಾದೇವ್ಯಾ ಆದ್ಯಾಯಾಃ ಪರಮಾದ್ಭುತಂ ।\\
ಪೂಜಾಕಾಲೇ ಪಠೇದ್ಯಸ್ತು ಆದ್ಯಾಧಿಕೃತಮಾನಸಃ ॥೧೨॥
\authorline{ಇತಿ ಮಹಾನಿರ್ವಾಣತಂತ್ರೇ ಶ್ರೀಕಾಲಿಕಾಕವಚಂ ಸಂಪೂರ್ಣಂ ।}
\section{॥ ಚಂಡೀಕವಚಮ್ ॥}
ಅಸ್ಯ ಶ್ರೀಚಂಡೀಕವಚಸ್ಯ~। ಬ್ರಹ್ಮಾ ಋಷಿಃ~। ಅನುಷ್ಟುಪ್ ಛಂದಃ~।\\ ಚಾಮುಂಡಾ ದೇವತಾ~। ಅಂಗನ್ಯಾಸೋಕ್ತಮಾತರೋ ಬೀಜಮ್~।\\ದಿಗ್ಬಂಧದೇವತಾಸ್ತತ್ವಮ್~। ಶ್ರೀಜಗದಂಬಾಪ್ರೀತ್ಯರ್ಥೇ ಜಪೇ ವಿನಿಯೋಗಃ ॥

ಓಂ ನಮಶ್ಚಂಡಿಕಾಯೈ ॥ ಮಾರ್ಕಂಡೇಯ ಉವಾಚ ॥\\
ಯದ್ಗುಹ್ಯಂ ಪರಮಂ ಲೋಕೇ ಸರ್ವರಕ್ಷಾಕರಂ ನೃಣಾಮ್~।\\
ಯನ್ನ ಕಸ್ಯಚಿದಾಖ್ಯಾತಂ ತನ್ಮೇ ಬ್ರೂಹಿ ಪಿತಾಮಹ ॥೧॥

ಬ್ರಹ್ಮೋವಾಚ ॥\\
ಅಸ್ತಿ ಗುಹ್ಯತಮಂ ವಿಪ್ರ ಸರ್ವಭೂತೋಪಕಾರಕಮ್~।\\
ದೇವ್ಯಾಸ್ತು ಕವಚಂ ಪುಣ್ಯಂ ತಚ್ಛೃಣುಷ್ವ ಮಹಾಮುನೇ ॥೨॥

ಪ್ರಥಮಂ ಶೈಲಪುತ್ರೀತಿ ದ್ವಿತೀಯಂ ಬ್ರಹ್ಮಚಾರಿಣೀ~।\\
ತೃತೀಯಂ ಚಂದ್ರಘಂಟೇತಿ ಕೂಷ್ಮಾಂಡೇತಿ ಚತುರ್ಥಕಮ್ ॥೩॥

ಪಂಚಮಂ ಸ್ಕಂದಮಾತೇತಿ ಷಷ್ಠಂ ಕಾತ್ಯಾಯನೀತಿ ಚ~।\\
ಸಪ್ತಮಂ ಕಾಲರಾತ್ರಿಶ್ಚ ಮಹಾಗೌರೀತಿ ಚಾಷ್ಟಮಮ್ ॥೪॥

ನವಮಂ ಸಿದ್ಧಿದಾತ್ರೀ ಚ ನವದುರ್ಗಾಃ ಪ್ರಕೀರ್ತಿತಾಃ~।\\
ಉಕ್ತಾನ್ಯೇತಾನಿ ನಾಮಾನಿ ಬ್ರಹ್ಮಣೈವ ಮಹಾತ್ಮನಾ ॥೫॥

ಅಗ್ನಿನಾ ದಹ್ಯಮಾನಸ್ತು ಶತ್ರುಮಧ್ಯೇ ಗತೋ ರಣೇ~।\\
ವಿಷಮೇ ದುರ್ಗಮೇ ಚೈವ ಭಯಾರ್ತಾಃ ಶರಣಂ ಗತಾಃ ॥೬॥

ನ ತೇಷಾಂ ಜಾಯತೇ ಕಿಂಚಿದಶುಭಂ ರಣಸಂಕಟೇ~।\\
ನಾಪದಂ ತಸ್ಯ ಪಶ್ಯಾಮಿ ಶೋಕದುಃಖಭಯಂ ನಹಿ ॥೭॥

ಯೈಸ್ತು ಭಕ್ತ್ಯಾ ಸ್ಮೃತಾ ನೂನಂ ತೇಷಾಂ ಸಿದ್ಧಿಃ ಪ್ರಜಾಯತೇ~।\\
ಪ್ರೇತಸಂಸ್ಥಾ ತು ಚಾಮುಂಡಾ ವಾರಾಹೀ ಮಹಿಷಾಸನಾ ॥೮॥

ಐಂದ್ರೀ ಗಜಸಮಾರೂಢಾ ವೈಷ್ಣವೀ ಗರುಡಾಸನಾ~।\\
ಮಾಹೇಶ್ವರೀ ವೃಷಾರೂಢಾ ಕೌಮಾರೀ ಶಿಖಿವಾಹನಾ ॥೯॥

ಬ್ರಾಹ್ಮೀ ಹಂಸಸಮಾರೂಢಾ ಸರ್ವಾಭರಣಭೂಷಿತಾ~।\\
ನಾನಾಽಭರಣಶೋಭಾಢ್ಯಾ ನಾನಾರತ್ನೋಪಶೋಭಿತಾಃ ॥೧೦॥

ದೃಶ್ಯಂತೇ ರಥಮಾರೂಢಾ ದೇವ್ಯಃ ಕ್ರೋಧಸಮಾಕುಲಾಃ~।\\
ಶಂಖಂ ಚಕ್ರಂ ಗದಾಂ ಶಕ್ತಿಂ ಹಲಂ ಚ ಮುಸಲಾಯುಧಮ್ ॥೧೧॥

ಖೇಟಕಂ ತೋಮರಂ ಚೈವ ಪರಶುಂ ಪಾಶಮೇವ ಚ~।\\
ಕುಂತಾಯುಧಂ ತ್ರಿಶೂಲಂ ಚ ಶಾರ್ಙ್ಗಮಾಯುಧಮುತ್ತಮಮ್ ॥೧೨॥

ದೈತ್ಯಾನಾಂ ದೇಹನಾಶಾಯ ಭಕ್ತಾನಾಮಭಯಾಯ ಚ~।\\
ಧಾರಯಂತ್ಯಾಯುಧಾನೀತ್ಥಂ ದೇವಾನಾಂ ಚ ಹಿತಾಯ ವೈ ॥೧೩॥

ಮಹಾಬಲೇ ಮಹೋತ್ಸಾಹೇ ಮಹಾಭಯವಿನಾಶಿನಿ~।\\
ತ್ರಾಹಿ ಮಾಂ ದೇವಿ ದುಷ್ಪ್ರೇಕ್ಷ್ಯೇ ಶತ್ರೂಣಾಂ ಭಯವರ್ಧಿನಿ ॥೧೪॥

ಪ್ರಾಚ್ಯಾಂ ರಕ್ಷತು ಮಾಮೈಂದ್ರೀ ಆಗ್ನೇಯ್ಯಾಮಗ್ನಿದೇವತಾ~।\\
ದಕ್ಷಿಣೇಽವತು ವಾರಾಹೀ ನೈರ್‌ಋತ್ಯಾಂ ಖಡ್ಗಧಾರಿಣೀ ॥೧೫॥

ಪ್ರತೀಚ್ಯಾಂ ವಾರುಣೀ ರಕ್ಷೇದ್ವಾಯವ್ಯಾಂ ಮೃಗವಾಹಿನೀ~।\\
ಉದೀಚ್ಯಾಂ ರಕ್ಷ ಕೌಬೇರಿ ಈಶಾನ್ಯಾಂ ಶೂಲಧಾರಿಣಿ ॥೧೬॥

ಊರ್ಧ್ವಂ ಬ್ರಹ್ಮಾಣೀ ಮೇ ರಕ್ಷೇದಧಸ್ತಾದ್ವೈಷ್ಣವೀ ತಥಾ~।\\
ಏವಂ ದಶ ದಿಶೋ ರಕ್ಷೇಚ್ಚಾಮುಂಡಾ ಶವವಾಹನಾ ॥೧೭॥

ಜಯಾ ಮೇ ಚಾಗ್ರತಃ ಸ್ಥಾತು ವಿಜಯಾ ಸ್ಥಾತು ಪೃಷ್ಠತಃ~।\\
ಅಜಿತಾ ವಾಮಪಾರ್ಶ್ವೇ ತು ದಕ್ಷಿಣೇ ಚಾಪರಾಜಿತಾ ॥೧೮॥

ಶಿಖಾಂ ಮೇ ದ್ಯೋತಿನೀ ರಕ್ಷೇದುಮಾ ಮೂರ್ಧ್ನಿ ವ್ಯವಸ್ಥಿತಾ~।\\
ಮಾಲಾಧರೀ ಲಲಾಟೇ ಚ ಭ್ರುವೌ ರಕ್ಷೇದ್ಯಶಸ್ವಿನೀ ॥೧೯॥

ತ್ರಿನೇತ್ರಾ ಚ ಭ್ರುವೋರ್ಮಧ್ಯೇ ಯಮಘಂಟಾ ಚ ನಾಸಿಕೇ~।\\
ಶಂಖಿನೀ ಚಕ್ಷುಷೋರ್ಮಧ್ಯೇ ಶ್ರೋತ್ರಯೋರ್ದ್ವಾರವಾಸಿನೀ ॥೨೦॥

ಕಪೋಲೌ ಕಾಲಿಕಾ ರಕ್ಷೇತ್ಕರ್ಣಮೂಲೇ ತು ಶಾಂಕರೀ~।\\
ನಾಸಿಕಾಯಾಂ ಸುಗಂಧಾ ಚ ಉತ್ತರೋಷ್ಠೇ ಚ ಚರ್ಚಿಕಾ ॥೨೧॥

ಅಧರೇ ಚಾಮೃತಕಲಾ ಜಿಹ್ವಾಯಾಂ ಚ ಸರಸ್ವತೀ~।\\
ದಂತಾನ್ ರಕ್ಷತು ಕೌಮಾರೀ ಕಂಠಮಧ್ಯೇ ತು ಚಂಡಿಕಾ ॥೨೨॥

ಘಂಟಿಕಾಂ ಚಿತ್ರಘಂಟಾ ಚ ಮಹಾಮಾಯಾ ಚ ತಾಲುಕೇ~।\\
ಕಾಮಾಕ್ಷೀ ಚಿಬುಕಂ ರಕ್ಷೇದ್ವಾಚಂ ಮೇ ಸರ್ವಮಂಗಲಾ ॥೨೩॥

ಗ್ರೀವಾಯಾಂ ಭದ್ರಕಾಲೀ ಚ ಪೃಷ್ಠವಂಶೇ ಧನುರ್ಧರೀ~।\\
ನೀಲಗ್ರೀವಾ ಬಹಿಃಕಂಠೇ ನಲಿಕಾಂ ನಲಕೂಬರೀ ॥೨೪॥

ಖಡ್ಗಧಾರಿಣ್ಯುಭೌ ಸ್ಕಂಧೌ ಬಾಹೂ ಮೇ ವಜ್ರಧಾರಿಣೀ~।\\
ಹಸ್ತಯೋರ್ದಂಡಿನೀ ರಕ್ಷೇದಂಬಿಕಾ ಚಾಂಗುಲೀಸ್ತಥಾ ॥೨೫॥

ನಖಾಂಛೂಲೇಶ್ವರೀ ರಕ್ಷೇತ್ ಕುಕ್ಷೌ ರಕ್ಷೇನ್ನಲೇಶ್ವರೀ~।\\
ಸ್ತನೌ ರಕ್ಷೇನ್ಮಹಾಲಕ್ಷ್ಮೀರ್ಮನಃ ಶೋಕವಿನಾಶಿನೀ ॥೨೬॥

ಹೃದಯಂ ಲಲಿತಾದೇವೀ ಉದರಂ ಶೂಲಧಾರಿಣೀ~।\\
ನಾಭೌ ಚ ಕಾಮಿನೀ ರಕ್ಷೇದ್ಗುಹ್ಯಂ ಗುಹ್ಯೇಶ್ವರೀ ತಥಾ ॥೨೭॥

ಕಟ್ಯಾಂ ಭಗವತೀ ರಕ್ಷೇಜ್ಜಾನುನೀ ವಿಂಧ್ಯವಾಸಿನೀ~।\\
ಭೂತನಾಥಾ ಚ ಮೇಢ್ರಂ ಮೇ ಊರೂ ಮಹಿಷವಾಹಿನೀ ॥೨೮॥

ಜಂಘೇ ಮಹಾಬಲಾ ಪ್ರೋಕ್ತಾ ಸರ್ವಕಾಮಪ್ರದಾಯಿನೀ~।\\
ಗುಲ್ಫಯೋರ್ನಾರಸಿಂಹೀ ಚ ಪಾದೌ ಚಾಮಿತತೇಜಸೀ ॥೨೯॥

ಪಾದಾಂಗುಲೀಃ ಶ್ರೀರ್ಮೇ ರಕ್ಷೇತ್ಪಾದಾಧಸ್ತಲವಾಸಿನೀ~।\\
ನಖಾಂದಂಷ್ಟ್ರಾಃ ಕರಾಲೀ ಚ ಕೇಶಾಂಶ್ಚೈವೋರ್ಧ್ವಕೇಶಿನೀ ॥೩೦॥

ರೋಮಕೂಪೇಷು ಕೌಬೇರೀ ತ್ವಚಂ ವಾಗೀಶ್ವರೀ ತಥಾ~।\\
ರಕ್ತಮಜ್ಜಾವಸಾಮಾಂಸಾನ್ಯಸ್ಥಿಮೇದಾಂಸಿ ಪಾರ್ವತೀ ॥೩೧॥

ಅಂತ್ರಾಣಿ ಕಾಲರಾತ್ರಿಶ್ಚ ಪಿತ್ತಂ ಚ ಮುಕುಟೇಶ್ವರೀ~।\\
ಪದ್ಮಾವತೀ ಪದ್ಮಕೋಶೇ ಕಫೇ ಚೂಡಾಮಣಿಸ್ತಥಾ ॥೩೨॥

ಜ್ವಾಲಾಮುಖೀ ನಖಜ್ವಾಲಾ ಅಭೇದ್ಯಾ ಸರ್ವಸಂಧಿಷು~।\\
ಶುಕ್ರಂ ಬ್ರಹ್ಮಾಣೀ ಮೇ ರಕ್ಷೇಚ್ಛಾಯಾಂ ಛತ್ರೇಶ್ವರೀ ತಥಾ ॥೩೩॥

ಅಹಂಕಾರಂ ಮನೋ ಬುದ್ಧಿಂ ರಕ್ಷ ಮೇ ಧರ್ಮಚಾರಿಣಿ~।\\
ಪ್ರಾಣಾಪಾನೌ ತಥಾ ವ್ಯಾನಂ ಸಮಾನೋದಾನಮೇವ ಚ ॥೩೪॥

ಯಶಃ ಕೀರ್ತಿಂ ಚ ಲಕ್ಷ್ಮೀಂ ಚ ಸದಾ ರಕ್ಷತು ವೈಷ್ಣವೀ~।\\
ಗೋತ್ರಮಿಂದ್ರಾಣೀ ಮೇ ರಕ್ಷೇತ್ಪಶೂನ್ಮೇ ರಕ್ಷ ಚಂಡಿಕೇ ॥೩೫॥

ಪುತ್ರಾನ್ ರಕ್ಷೇನ್ಮಹಾಲಕ್ಷ್ಮೀರ್ಭಾರ್ಯಾಂ ರಕ್ಷತು ಭೈರವೀ~।\\
ಮಾರ್ಗಂ ಕ್ಷೇಮಕರೀ ರಕ್ಷೇದ್ವಿಜಯಾ ಸರ್ವತಃ ಸ್ಥಿತಾ ॥೩೬॥

ರಕ್ಷಾಹೀನಂ ತು ಯತ್ಸ್ಥಾನಂ ವರ್ಜಿತಂ ಕವಚೇನ ತು~।\\
ತತ್ಸರ್ವಂ ರಕ್ಷ ಮೇ ದೇವಿ ಜಯಂತೀ ಪಾಪನಾಶಿನೀ ॥೩೭॥

ಪದಮೇಕಂ ನ ಗಚ್ಛೇತ್ತು ಯದೀಚ್ಛೇಚ್ಛುಭಮಾತ್ಮನಃ~।\\
ಕವಚೇನಾವೃತೋ ನಿತ್ಯಂ ಯತ್ರ ಯತ್ರಾಧಿಗಚ್ಛತಿ ॥೩೮॥

ತತ್ರ ತತ್ರಾರ್ಥ ಲಾಭಶ್ಚ ವಿಜಯಃ ಸಾರ್ವಕಾಮಿಕಃ~।\\
ಯಂ ಯಂ ಕಾಮಯತೇ ಕಾಮಂ ತಂ ತಂ ಪ್ರಾಪ್ನೋತಿ ನಿಶ್ಚಿತಮ್ ॥೩೯॥

ಪರಮೈಶ್ವರ್ಯಮತುಲಂ ಪ್ರಾಪ್ಸ್ಯತೇ ಭೂತಲೇ ಪುಮಾನ್~।\\
ನಿರ್ಭಯೋ ಜಾಯತೇ ಮರ್ತ್ಯಃ ಸಂಗ್ರಾಮೇಷ್ವ ಪರಾಜಿತಃ ॥೪೦॥

ತ್ರೈಲೋಕ್ಯೇ ತು ಭವೇತ್ಪೂಜ್ಯಃ ಕವಚೇನಾವೃತಃ ಪುಮಾನ್~।\\
ಇದಂ ತು ದೇವ್ಯಾಃ ಕವಚಂ ದೇವಾನಾಮಪಿ ದುರ್ಲಭಮ್ ॥೪೧॥

ಯಃ ಪಠೇತ್ಪ್ರಯತೋ ನಿತ್ಯಂ ತ್ರಿಸಂಧ್ಯಂ ಶ್ರದ್ಧಯಾನ್ವಿತಃ~।\\
ದೈವೀ ಕಲಾ ಭವೇತ್ತಸ್ಯ ತ್ರೈಲೋಕ್ಯೇಪ್ಯಪರಾಜಿತಃ ॥೪೨॥

ಜೀವೇದ್ವರ್ಷಶತಂ ಸಾಗ್ರಮಪಮೃತ್ಯು ವಿವರ್ಜಿತಃ~।\\
ನಶ್ಯಂತಿ ವ್ಯಾಧಯಃ ಸರ್ವೇ ಲೂತಾವಿಸ್ಫೋಟಕಾದಯಃ ॥೪೩॥

ಸ್ಥಾವರಂ ಜಂಗಮಂ ವಾಪಿ ಕೃತ್ರಿಮಂ ಚಾಪಿ ಯದ್ವಿಷಮ್~।\\
ಆಭಿಚಾರಾಣಿ ಸರ್ವಾಣಿ ಮಂತ್ರಯಂತ್ರಾಣಿ ಭೂತಲೇ ॥೪೪॥

ಭೂಚರಾಃ ಖೇಚರಾಶ್ಚೈವ ಜಲಜಾಶ್ಚೋಪದೇಶಿಕಾಃ~।\\
ಸಹಜಾಃ ಕುಲಜಾ ಮಾಲಾಃ ಶಾಕಿನೀ ಡಾಕಿನೀ ತಥಾ ॥೪೫॥

ಅಂತರಿಕ್ಷಚರಾ ಘೋರಾ ಡಾಕಿನ್ಯಶ್ಚ ಮಹಾಬಲಾಃ~।\\
ಗ್ರಹಭೂತಪಿಶಾಚಾಶ್ಚ ಯಕ್ಷಗಂಧರ್ವರಾಕ್ಷಸಾಃ ॥೪೬॥

ಬ್ರಹ್ಮರಾಕ್ಷಸವೇತಾಲಾಃ ಕೂಷ್ಮಾಂಡಾ ಭೈರವಾದಯಃ~।\\
ನಶ್ಯಂತಿ ದರ್ಶನಾತ್ತಸ್ಯ ಕವಚೇ ಹೃದಿ ಸಂಸ್ಥಿತೇ ॥೪೭॥

ಮಾನೋನ್ನತಿರ್ಭವೇದ್ರಾಜ್ಞಸ್ತೇಜೋವೃದ್ಧಿಕರಂ ಪರಮ್~।\\
ಯಶಸಾ ವರ್ಧತೇ ಸೋಽಪಿ ಕೀರ್ತಿಮಂಡಿತಭೂತಲೇ ॥೪೮॥

ಜಪೇತ್ಸಪ್ತಶತೀಂ ಚಂಡೀಂ ಕೃತ್ವಾ ತು ಕವಚಂ ಪುರಾ~।\\
ಯಾವದ್ಭೂಮಂಡಲಂ ಧತ್ತೇ ಸಶೈಲವನಕಾನನಮ್ ॥೪೯॥

ತಾವತ್ತಿಷ್ಠತಿ ಮೇದಿನ್ಯಾಂ ಸಂತತಿಃ ಪುತ್ರಪೌತ್ರಿಕೀ~।\\
ದೇಹಾಂತೇ ಪರಮಂ ಸ್ಥಾನಂ ಯತ್ಸುರೈರಪಿ ದುರ್ಲಭಮ್~।\\
ಪ್ರಾಪ್ನೋತಿ ಪುರುಷೋ ನಿತ್ಯಂ ಮಹಾಮಾಯಾಪ್ರಸಾದತಃ ॥೫೦॥
\authorline{ಇತಿ ಶ್ರೀವಾರಾಹಪುರಾಣೇ ಹರಿಹರಬ್ರಹ್ಮವಿರಚಿತಂ ದೇವ್ಯಾಃ ಕವಚಂ ಸಂಪೂರ್ಣಮ್ ॥}
\section{ಅರ್ಗಲಾ ಸ್ತೋತ್ರಮ್}
ಅಸ್ಯ ಶ್ರೀಅರ್ಗಲಾ ಸ್ತೋತ್ರಮಂತ್ರಸ್ಯ ವಿಷ್ಣುರ್ಋಷಿಃ । ಅನುಷ್ಟುಪ್ಛಂದಃ ।  ಶ್ರೀಮಹಾಲಕ್ಷ್ಮೀರ್ದೇವತಾ ।  ಶ್ರೀಜಗದಂಬಾ ಪ್ರೀತ್ಯರ್ಥೇ ಜಪೇ ವಿನಿಯೋಗಃ ॥

ಓಂ ನಮಶ್ಚಂಡಿಕಾಯೈ ॥\\
ಜಯಂತೀ ಮಂಗಲಾ ಕಾಲೀ ಭದ್ರಕಾಲೀ ಕಪಾಲಿನೀ ।\\
ದುರ್ಗಾ ಕ್ಷಮಾ ಶಿವಾ ಧಾತ್ರೀ ಸ್ವಾಹಾ ಸ್ವಧಾ ನಮೋಽಸ್ತು ತೇ ॥೧॥

ಮಧುಕೈಟಭವಿದ್ರಾವಿ ವಿಧಾತೃವರದೇ ನಮಃ ।\\
ರೂಪಂ ದೇಹಿ ಜಯಂ ದೇಹಿ ಯಶೋ ದೇಹಿ ದ್ವಿಷೋ ಜಹಿ ॥೨॥

ಮಹಿಷಾಸುರನಿರ್ನಾಶವಿಧಾತ್ರಿ ವರದೇ ನಮಃ ।\\
ರೂಪಂ ದೇಹಿ ಜಯಂ ದೇಹಿ ಯಶೋ ದೇಹಿ ದ್ವಿಷೋ ಜಹಿ ॥೩॥

ವಂದಿತಾಂಘ್ರಿಯುಗೇ ದೇವಿ ಸರ್ವಸೌಭಾಗ್ಯದಾಯಿನಿ ।\\
ರೂಪಂ ದೇಹಿ ಜಯಂ ದೇಹಿ ಯಶೋ ದೇಹಿ ದ್ವಿಷೋ ಜಹಿ ॥೪॥

ರಕ್ತಬೀಜವಧೇ ದೇವಿ ಚಂಡಮುಂಡವಿನಾಶಿನಿ ।\\
ರೂಪಂ ದೇಹಿ ಜಯಂ ದೇಹಿ ಯಶೋ ದೇಹಿ ದ್ವಿಷೋ ಜಹಿ ॥೫॥

ಅಚಿಂತ್ಯರೂಪಚರಿತೇ ಸರ್ವಶತ್ರುವಿನಾಶಿನಿ ।\\
ರೂಪಂ ದೇಹಿ ಜಯಂ ದೇಹಿ ಯಶೋ ದೇಹಿ ದ್ವಿಷೋ ಜಹಿ ॥೬॥

ನತೇಭ್ಯಃ ಸರ್ವದಾ ಭಕ್ತ್ಯಾ ಚಂಡಿಕೇ ದುರಿತಾಪಹೇ ।\\
ರೂಪಂ ದೇಹಿ ಜಯಂ ದೇಹಿ ಯಶೋ ದೇಹಿ ದ್ವಿಷೋ ಜಹಿ ॥೭॥

ಸ್ತುವದ್ಭ್ಯೋ ಭಕ್ತಿಪೂರ್ವಂ ತ್ವಾಂ ಚಂಡಿಕೇ ವ್ಯಾಧಿನಾಶಿನಿ ।\\
ರೂಪಂ ದೇಹಿ ಜಯಂ ದೇಹಿ ಯಶೋ ದೇಹಿ ದ್ವಿಷೋ ಜಹಿ ॥೮॥

ಚಂಡಿಕೇ ಸತತಂ ಯೇ ತ್ವಾಮರ್ಚಯಂತೀಹ ಭಕ್ತಿತಃ ।\\
ರೂಪಂ ದೇಹಿ ಜಯಂ ದೇಹಿ ಯಶೋ ದೇಹಿ ದ್ವಿಷೋ ಜಹಿ ॥೯॥
\newpage
ದೇಹಿ ಸೌಭಾಗ್ಯಮಾರೋಗ್ಯಂ ದೇಹಿ ದೇವಿ ಪರಂ ಸುಖಂ ।\\
ರೂಪಂ ದೇಹಿ ಜಯಂ ದೇಹಿ ಯಶೋ ದೇಹಿ ದ್ವಿಷೋ ಜಹಿ ॥೧೦॥

ವಿಧೇಹಿ ದ್ವಿಷತಾಂ ನಾಶಂ ವಿಧೇಹಿ ಬಲಮುಚ್ಚಕೈಃ ।\\
ರೂಪಂ ದೇಹಿ ಜಯಂ ದೇಹಿ ಯಶೋ ದೇಹಿ ದ್ವಿಷೋ ಜಹಿ ॥೧೧॥

ವಿಧೇಹಿ ದೇವಿ ಕಲ್ಯಾಣಂ ವಿಧೇಹಿ ಪರಮಾಂ ಶ್ರಿಯಂ ।\\
ರೂಪಂ ದೇಹಿ ಜಯಂ ದೇಹಿ ಯಶೋ ದೇಹಿ ದ್ವಿಷೋ ಜಹಿ ॥೧೨॥

ವಿದ್ಯಾವಂತಂ ಯಶಸ್ವಂತಂ ಲಕ್ಷ್ಮೀವಂತಂ ಜನಂ ಕುರು ।\\
ರೂಪಂ ದೇಹಿ ಜಯಂ ದೇಹಿ ಯಶೋ ದೇಹಿ ದ್ವಿಷೋ ಜಹಿ ॥೧೩॥

ಪ್ರಚಂಡದೈತ್ಯದರ್ಪಘ್ನೇ ಚಂಡಿಕೇ ಪ್ರಣತಾಯ ಮೇ ।\\
ರೂಪಂ ದೇಹಿ ಜಯಂ ದೇಹಿ ಯಶೋ ದೇಹಿ ದ್ವಿಷೋ ಜಹಿ ॥೧೪॥

ಚತುರ್ಭುಜೇ ಚತುರ್ವಕ್ತ್ರಸಂಸ್ತುತೇ ಪರಮೇಶ್ವರಿ ।\\
ರೂಪಂ ದೇಹಿ ಜಯಂ ದೇಹಿ ಯಶೋ ದೇಹಿ ದ್ವಿಷೋ ಜಹಿ ॥೧೫॥

ಕೃಷ್ಣೇನ ಸಂಸ್ತುತೇ ದೇವಿ ಶಶ್ವದ್ಭಕ್ತ್ಯಾ ತ್ವಮಂಬಿಕೇ ।\\
ರೂಪಂ ದೇಹಿ ಜಯಂ ದೇಹಿ ಯಶೋ ದೇಹಿ ದ್ವಿಷೋ ಜಹಿ ॥೧೬॥

ಹಿಮಾಚಲಸುತಾನಾಥಸಂಸ್ತುತೇ ಪರಮೇಶ್ವರಿ ।\\
ರೂಪಂ ದೇಹಿ ಜಯಂ ದೇಹಿ ಯಶೋ ದೇಹಿ ದ್ವಿಷೋ ಜಹಿ ॥೧೭॥

ಸುರಾಸುರಶಿರೋರತ್ನನಿಘೃಷ್ಟಚರಣೇಽಂಬಿಕೇ ।\\
ರೂಪಂ ದೇಹಿ ಜಯಂ ದೇಹಿ ಯಶೋ ದೇಹಿ ದ್ವಿಷೋ ಜಹಿ ॥೧೮॥

ಇಂದ್ರಾಣೀಪತಿಸದ್ಭಾವಪೂಜಿತೇ ಪರಮೇಶ್ವರಿ ।\\
ರೂಪಂ ದೇಹಿ ಜಯಂ ದೇಹಿ ಯಶೋ ದೇಹಿ ದ್ವಿಷೋ ಜಹಿ ॥೧೯॥

ದೇವಿ ಭಕ್ತಜನೋದ್ದಾಮದತ್ತಾನಂದೋದಯೇಂಬಿಕೇ ।\\
ರೂಪಂ ದೇಹಿ ಜಯಂ ದೇಹಿ ಯಶೋ ದೇಹಿ ದ್ವಿಷೋ ಜಹಿ ॥೨೦॥

ಪುತ್ರಾನ್ ದೇಹಿ ಧನಂ ದೇಹಿ ಸರ್ವಕಾಮಾಂಶ್ಚ ದೇಹಿ ಮೇ ।\\
ರೂಪಂ ದೇಹಿ ಜಯಂ ದೇಹಿ ಯಶೋ ದೇಹಿ ದ್ವಿಷೋ ಜಹಿ ॥೨೧॥

ಪತ್ನೀಂ ಮನೋರಮಾಂ ದೇಹಿ ಮನೋವೃತ್ತಾನು ಸಾರಿಣೀಂ~।\\
ತಾರಿಣೀಂ ದುರ್ಗಸಂಸಾರಸಾಗರಸ್ಯ ಕುಲೋದ್ಭವಾಂ॥೨೨॥

ಇದಂ ಸ್ತೋತ್ರಂ ಪಠಿತ್ವಾ ತು ಮಹಾಸ್ತೋತ್ರಂ ಪಠೇನ್ನರಃ ।\\
ಸ ತು ಸಪ್ತಶತೀ ಸಂಖ್ಯಾವರಮಾಪ್ನೋತಿ ಸಂಪದಾಂ ॥೨೩॥
\authorline{॥ ಮಾರ್ಕಂಡೇಯಪುರಾಣೇ ಅರ್ಗಲಾ ಸ್ತೋತ್ರಂ ॥}
\section{ಕೀಲಕಮ್}
ಓಂ ಅಸ್ಯ ಶ್ರೀಕೀಲಕಮಂತ್ರಸ್ಯ ಶಿವಋಷಿಃ । ಅನುಷ್ಟುಪ್ ಛಂದಃ । ಶ್ರೀಮಹಾಸರಸ್ವತೀ ದೇವತಾ । ಶ್ರೀಜಗದಂಬಾಪ್ರೀತ್ಯರ್ಥಂ ಜಪೇ ವಿನಿಯೋಗಃ ।

ಓಂ ನಮಶ್ಚಂಡಿಕಾಯೈ ॥ ಮಾರ್ಕಂಡೇಯ ಉವಾಚ ॥\\
ವಿಶುದ್ಧಜ್ಞಾನದೇಹಾಯ ತ್ರಿವೇದೀದಿವ್ಯಚಕ್ಷುಷೇ ।\\
ಶ್ರೇಯಃಪ್ರಾಪ್ತಿನಿಮಿತ್ತಾಯ ನಮಃ ಸೋಮಾರ್ಧಧಾರಿಣೇ ॥೧॥

ಸರ್ವಮೇತದ್ವಿನಾ ಯಸ್ತು ಮಂತ್ರಾಣಾಮಪಿ ಕೀಲಕಂ ।\\
ಸೋಽಪಿ ಕ್ಷೇಮಮವಾಪ್ನೋತಿ ಸತತಂ ಜಾಪ್ಯತತ್ಪರಃ ॥೨॥

ಸಿದ್ಧ್ಯಂತ್ಯುಚ್ಚಾಟನಾದೀನಿ ವಸ್ತೂನಿ ಸಕಲಾನ್ಯಪಿ ।\\
ಏತೇನ ಸ್ತುವತಾಂ ನಿತ್ಯಂ ಸ್ತೋತ್ರಮಾತ್ರೇಣ ಸಿದ್ಧ್ಯತಿ ॥೩॥

ನ ಮಂತ್ರೋ ನೌಷಧಂ ತತ್ರ ನ ಕಿಂಚಿದಪಿ ವಿದ್ಯತೇ ।\\
ವಿನಾ ಜಾಪ್ಯೇನ ಸಿದ್ಧ್ಯೇತ ಸರ್ವಮುಚ್ಚಾಟನಾದಿಕಂ ॥೪॥

ಸಮಗ್ರಾಣ್ಯಪಿ ಸಿದ್ಧ್ಯಂತಿ ಲೋಕಶಂಕಾಮಿಮಾಂ ಹರಃ ।\\
ಕೃತ್ವಾ ನಿಮಂತ್ರಯಾಮಾಸ ಸರ್ವಮೇವಮಿದಂ ಶುಭಂ ॥೫॥

ಸ್ತೋತ್ರಂ ವೈ ಚಂಡಿಕಾಯಾಸ್ತು ತಚ್ಚ ಗುಹ್ಯಂ ಚಕಾರ ಸಃ ।\\
ಸಮಾಪ್ತಿರ್ನ ಚ ಪುಣ್ಯಸ್ಯ ತಾಂ ಯಥಾವನ್ನಿಯಂತ್ರಣಾಂ ॥೬॥

ಸೋಽಪಿ ಕ್ಷೇಮಮವಾಪ್ನೋತಿ ಸರ್ವಮೇವ ನ ಸಂಶಯಃ ।\\
ಕೃಷ್ಣಾಯಾಂ ವಾ ಚತುರ್ದಶ್ಯಾಮಷ್ಟಮ್ಯಾಂ ವಾ ಸಮಾಹಿತಃ ॥೭॥

ದದಾತಿ ಪ್ರತಿಗೃಹ್ಣಾತಿ ನಾನ್ಯಥೈಷಾ ಪ್ರಸೀದತಿ ।\\
ಇತ್ಥಂ ರೂಪೇಣ ಕೀಲೇನ ಮಹಾದೇವೇನ ಕೀಲಿತಂ ॥೮॥

ಯೋ ನಿಷ್ಕೀಲಾಂ ವಿಧಾಯೈನಾಂ ನಿತ್ಯಂ ಜಪತಿ ಸುಸ್ಫುಟಂ ।\\
ಸಸಿದ್ಧಃ ಸಗಣಃ ಸೋಽಪಿ ಗಂಧರ್ವೋ ಜಾಯತೇ ವನೇ ॥೯॥

ನ ಚೈವಾಪ್ಯಟತಸ್ತಸ್ಯ ಭಯಂ ಕ್ವಾಪಿ ಹಿ ಜಾಯತೇ ।\\
ನಾಪಮೃತ್ಯುವಶಂ ಯಾತಿ ಮೃತೋ ಮೋಕ್ಷಮವಾಪ್ನುಯಾತ್ ॥೧೦॥

ಜ್ಞಾತ್ವಾ ಪ್ರಾರಭ್ಯ ಕುರ್ವೀತ ಹ್ಯಕುರ್ವಾಣೋ ವಿನಶ್ಯತಿ ।\\
ತತೋ ಜ್ಞಾತ್ವೈವ ಸಂಪನ್ನಮಿದಂ ಪ್ರಾರಭ್ಯತೇ ಬುಧೈಃ ॥೧೧॥

ಸೌಭಾಗ್ಯಾದಿ ಚ ಯತ್ಕಿಂಚಿದ್ ದೃಶ್ಯತೇ ಲಲನಾಜನೇ ।\\
ತತ್ಸರ್ವಂ ತತ್ಪ್ರಸಾದೇನ ತೇನ ಜಾಪ್ಯಮಿದಂ ಶುಭಂ ॥೧೨॥

ಶನೈಸ್ತು ಜಪ್ಯಮಾನೇಽಸ್ಮಿನ್ ಸ್ತೋತ್ರೇ ಸಂಪತ್ತಿರುಚ್ಚಕೈಃ ।\\
ಭವತ್ಯೇವ ಸಮಗ್ರಾಪಿ ತತಃ ಪ್ರಾರಭ್ಯಮೇವ ತತ್ ॥೧೩॥

ಐಶ್ವರ್ಯಂ ಯತ್ಪ್ರಸಾದೇನ ಸೌಭಾಗ್ಯಾರೋಗ್ಯಸಂಪದಃ ।\\
ಶತ್ರುಹಾನಿಃ ಪರೋ ಮೋಕ್ಷಃ ಸ್ತೂಯತೇ ಸಾ ನ ಕಿಂ ಜನೈಃ ॥೧೪॥
\authorline{॥ಭಗವತ್ಯಾಃ ಕೀಲಕಸ್ತೋತ್ರಂ ॥}
\section{ಬ್ರಹ್ಮಸ್ತುತಿಃ}
ಬ್ರಹ್ಮೋವಾಚ ॥\\
ತ್ವಂ ಸ್ವಾಹಾ ತ್ವಂ ಸ್ವಧಾ ತ್ವಂ ಹಿ ವಷಟ್ಕಾರ ಸ್ವರಾತ್ಮಿಕಾ ।\\
ಸುಧಾ ತ್ವಮಕ್ಷರೇ ನಿತ್ಯೇ ತ್ರಿಧಾ ಮಾತ್ರಾತ್ಮಿಕಾ ಸ್ಥಿತಾ ॥೨॥

ಅರ್ಧಮಾತ್ರಾ ಸ್ಥಿತಾ ನಿತ್ಯಾ ಯಾನುಚ್ಚಾರ್ಯಾ ವಿಶೇಷತಃ ।\\
ತ್ವಮೇವ ಸಂಧ್ಯಾ ಸಾವಿತ್ರೀ ತ್ವಂ ದೇವಿ ಜನನೀ ಪರಾ ॥೩॥

ತ್ವಯೈತದ್ಧಾರ್ಯತೇ ವಿಶ್ವಂ ತ್ವಯೈತತ್ಸೃಜ್ಯತೇ ಜಗತ್ ।\\
ತ್ವಯೈತತ್ಪಾಲ್ಯತೇ ದೇವಿ ತ್ವಮತ್ಸ್ಯಂತೇ ಚ ಸರ್ವದಾ ॥೪॥

ವಿಸೃಷ್ಟೌ ಸೃಷ್ಟಿರೂಪಾ ತ್ವಂ ಸ್ಥಿತಿರೂಪಾ ಚ ಪಾಲನೇ ।\\
ತಥಾ ಸಂಹೃತಿರೂಪಾಂತೇ ಜಗತೋಽಸ್ಯ ಜಗನ್ಮಯೇ ॥೫॥

ಮಹಾವಿದ್ಯಾ ಮಹಾಮಾಯಾ ಮಹಾಮೇಧಾ ಮಹಾಸ್ಮೃತಿಃ ।\\
ಮಹಾಮೋಹಾ ಚ ಭವತೀ ಮಹಾದೇವೀ ಮಹೇಶ್ವರೀ ॥೬॥

ಪ್ರಕೃತಿಸ್ತ್ವಂ ಚ ಸರ್ವಸ್ಯ ಗುಣತ್ರಯ ವಿಭಾವಿನೀ ।\\
ಕಾಲರಾತ್ರಿರ್ಮಹಾರಾತ್ರಿರ್ಮೋಹರಾತ್ರಿಶ್ಚ ದಾರುಣಾ ॥೭॥

ತ್ವಂ ಶ್ರೀಸ್ತ್ವಮೀಶ್ವರೀ ತ್ವಂ ಹ್ರೀಸ್ತ್ವಂ ಬುದ್ಧಿರ್ಬೋಧಲಕ್ಷಣಾ ।\\
ಲಜ್ಜಾ ಪುಷ್ಟಿಸ್ತಥಾ ತುಷ್ಟಿಸ್ತ್ವಂ ಶಾಂತಿಃ ಕ್ಷಾಂತಿರೇವ ಚ ॥೮॥
\newpage
ಖಡ್ಗಿನೀ ಶೂಲಿನೀ ಘೋರಾ ಗದಿನೀ ಚಕ್ರಿಣೀ ತಥಾ ।\\
ಶಂಖಿನೀ ಚಾಪಿನೀ ಬಾಣಭುಶುಂಡೀಪರಿಘಾಯುಧಾ ॥೯॥

ಸೌಮ್ಯಾ ಸೌಮ್ಯತರಾಶೇಷ ಸೌಮ್ಯೇಭ್ಯಸ್ತ್ವತಿ ಸುಂದರೀ ।\\
ಪರಾಪರಾಣಾಂ ಪರಮಾ ತ್ವಮೇವ ಪರಮೇಶ್ವರೀ ॥೧೦॥

ಯಚ್ಚ ಕಿಂಚಿತ್ ಕ್ವಚಿದ್ವಸ್ತು ಸದಸದ್ವಾಖಿಲಾತ್ಮಿಕೇ ।\\
ತಸ್ಯ ಸರ್ವಸ್ಯ ಯಾ ಶಕ್ತಿಃ ಸಾ ತ್ವಂ ಕಿಂ ಸ್ತೂಯಸೇ ಮಯಾ ॥೧೧॥

ಯಯಾ ತ್ವಯಾ ಜಗತ್‌ಸ್ರಷ್ಟಾ ಜಗತ್ಪಾತ್ಯತ್ತಿ ಯೋ ಜಗತ್ ।\\
ಸೋಽಪಿ ನಿದ್ರಾವಶಂ ನೀತಃ ಕಸ್ತ್ವಾಂ ಸ್ತೋತುಮಿಹೇಶ್ವರಃ ॥೧೨॥

ವಿಷ್ಣುಃ ಶರೀರಗ್ರಹಣಮಹಮೀಶಾನ ಏವ ಚ ।\\
ಕಾರಿತಾಸ್ತೇ ಯತೋಽತಸ್ತ್ವಾಂ ಕಃ ಸ್ತೋತುಂ ಶಕ್ತಿಮಾನ್ಭವೇತ್ ॥೧೩॥

ಸಾ ತ್ವಮಿತ್ಥಂ ಪ್ರಭಾವೈಃ ಸ್ವೈರುದಾರೈರ್ದೇವಿ ಸಂಸ್ತುತಾ ।\\
ಮೋಹಯೈತೌ ದುರಾಧರ್ಷಾವಸುರೌ ಮಧುಕೈಟಭೌ ॥೧೪॥

ಪ್ರಬೋಧಂ ಚ ಜಗತ್ಸ್ವಾಮೀ ನೀಯತಾಮಚ್ಯುತೋ ಲಘು ।\\
ಬೋಧಶ್ಚ ಕ್ರಿಯತಾಮಸ್ಯ ಹಂತುಮೇತೌ ಮಹಾಸುರೌ ॥೧೫॥


\section{ಶಕ್ರಾದಿಸ್ತುತಿಃ}
          ಋಷಿರುವಾಚ ॥೧॥\\
     ಶಕ್ರಾದಯಃ ಸುರಗಣಾ ನಿಹತೇಽತಿವೀರ್ಯೇ\\
ತಸ್ಮಿನ್ ದುರಾತ್ಮನಿ ಸುರಾರಿಬಲೇ ಚ ದೇವ್ಯಾ ।\\
     ತಾಂ ತುಷ್ಟುವುಃ ಪ್ರಣತಿನಮ್ರಶಿರೋಧರಾಂಸಾ\\
ವಾಗ್ಭಿಃ ಪ್ರಹರ್ಷಪುಲಕೋದ್ಗಮಚಾರುದೇಹಾಃ ॥೨॥

     ದೇವ್ಯಾ ಯಯಾ ತತಮಿದಂ ಜಗದಾತ್ಮಶಕ್ತ್ಯಾ\\
ನಿಶ್ಶೇಷದೇವಗಣಶಕ್ತಿಸಮೂಹಮೂರ್ತ್ಯಾ ।\\
     ತಾಮಂಬಿಕಾಮಖಿಲದೇವಮಹರ್ಷಿಪೂಜ್ಯಾಂ\\
ಭಕ್ತ್ಯಾ ನತಾಃ ಸ್ಮ ವಿದಧಾತು ಶುಭಾನಿ ಸಾ ನಃ ॥೩॥

     ಯಸ್ಯಾಃ ಪ್ರಭಾವಮತುಲಂ ಭಗವಾನನಂತೋ\\
ಬ್ರಹ್ಮಾ ಹರಶ್ಚ ನ ಹಿ ವಕ್ತುಮಲಂ ಬಲಂ ಚ ।\\
     ಸಾ ಚಂಡಿಕಾಖಿಲಜಗತ್ಪರಿಪಾಲನಾಯ\\
ನಾಶಾಯ ಚಾಶುಭಭಯಸ್ಯ ಮತಿಂ ಕರೋತು ॥೪॥

     ಯಾ ಶ್ರೀಃ ಸ್ವಯಂ ಸುಕೃತಿನಾಂ ಭವನೇಷ್ವಲಕ್ಷ್ಮೀಃ\\
ಪಾಪಾತ್ಮನಾಂ ಕೃತಧಿಯಾಂ ಹೃದಯೇಷು ಬುದ್ಧಿಃ ।\\
     ಶ್ರದ್ಧಾ ಸತಾಂ ಕುಲಜನಪ್ರಭವಸ್ಯ ಲಜ್ಜಾ\\
ತಾಂ ತ್ವಾಂ ನತಾಃ ಸ್ಮ ಪರಿಪಾಲಯ ದೇವಿ ವಿಶ್ವಂ ॥೫॥

     ಕಿಂ ವರ್ಣಯಾಮ ತವ ರೂಪಮಚಿಂತ್ಯಮೇತತ್\\
ಕಿಂ ಚಾತಿವೀರ್ಯಮಸುರ ಕ್ಷಯಕಾರಿ ಭೂರಿ ।\\
     ಕಿಂ ಚಾಹವೇಷು ಚರಿತಾನಿ ತವಾದ್ಭುತಾನಿ\\
ಸರ್ವೇಷು ದೇವ್ಯಸುರ ದೇವ ಗಣಾದಿಕೇಷು ॥೬॥

     ಹೇತುಃ ಸಮಸ್ತ ಜಗತಾಂ ತ್ರಿಗುಣಾಪಿ ದೋಷೈ-\\
ರ್ನ ಜ್ಞಾಯಸೇ ಹರಿಹರಾದಿಭಿರಪ್ಯಪಾರಾ ।\\
     ಸರ್ವಾಶ್ರಯಾಖಿಲಮಿದಂ ಜಗದಂಶಭೂತ-\\
ಮವ್ಯಾಕೃತಾ ಹಿ ಪರಮಾ ಪ್ರಕೃತಿಸ್ತ್ವಮಾದ್ಯಾ ॥೭॥
\newpage
     ಯಸ್ಯಾಃ ಸಮಸ್ತ ಸುರತಾ ಸಮುದೀರಣೇನ\\
ತೃಪ್ತಿಂ ಪ್ರಯಾತಿ ಸಕಲೇಷು ಮಖೇಷು ದೇವಿ ।\\
     ಸ್ವಾಹಾಸಿ ವೈ ಪಿತೃಗಣಸ್ಯ ಚ ತೃಪ್ತಿಹೇತು-\\
ರುಚ್ಚಾರ್ಯಸೇ ತ್ವಮತ ಏವ ಜನೈಃ ಸ್ವಧಾ ಚ ॥೮॥

     ಯಾ ಮುಕ್ತಿಹೇತುರವಿಚಿಂತ್ಯಮಹಾವ್ರತಾ ತ್ವಂ\\
ಅಭ್ಯಸ್ಯಸೇ ಸುನಿಯತೇಂದ್ರಿಯ ತತ್ತ್ವಸಾರೈಃ ।\\
     ಮೋಕ್ಷಾರ್ಥಿಭಿರ್ಮುನಿಭಿರಸ್ತಸಮಸ್ತದೋಷೈ-\\
ರ್ವಿದ್ಯಾಸಿ ಸಾ ಭಗವತೀ ಪರಮಾ ಹಿ ದೇವಿ ॥೯॥

     ಶಬ್ದಾತ್ಮಿಕಾ ಸುವಿಮಲರ್ಗ್ಯಜುಷಾಂ ನಿಧಾನ-\\
ಮುದ್ಗೀಥರಮ್ಯಪದಪಾಠವತಾಂ ಚ ಸಾಮ್ನಾಂ ।\\
     ದೇವಿ ತ್ರಯೀ ಭಗವತೀ ಭವಭಾವನಾಯ\\
ವಾರ್ತಾಸಿ ಸರ್ವಜಗತಾಂ ಪರಮಾರ್ತಿ ಹಂತ್ರೀ ॥೧೦॥

     ಮೇಧಾಸಿ ದೇವಿ ವಿದಿತಾಖಿಲಶಾಸ್ತ್ರಸಾರಾ\\
ದುರ್ಗಾಸಿ ದುರ್ಗಭವಸಾಗರ ನೌರಸಂಗಾ ।\\
     ಶ್ರೀಃ ಕೈಟಭಾರಿ ಹೃದಯೈಕ ಕೃತಾಧಿವಾಸಾ\\
ಗೌರೀ ತ್ವಮೇವ ಶಶಿಮೌಲಿಕೃತಪ್ರತಿಷ್ಠಾ ॥೧೧॥

     ಈಷತ್ಸಹಾಸಮಮಲಂ ಪರಿಪೂರ್ಣಚಂದ್ರ-\\
ಬಿಂಬಾನುಕಾರಿ ಕನಕೋತ್ತಮ ಕಾಂತಿಕಾಂತಂ ।\\
     ಅತ್ಯದ್ಭುತಂ ಪ್ರಹೃತಮಾತ್ತರುಷಾ ತಥಾಪಿ\\
ವಕ್ತ್ರಂ ವಿಲೋಕ್ಯ ಸಹಸಾ ಮಹಿಷಾಸುರೇಣ ॥೧೨॥
\newpage
     ದೃಷ್ಟ್ವಾ ತು ದೇವಿ ಕುಪಿತಂ ಭ್ರುಕುಟೀಕರಾಲ-\\
ಮುದ್ಯಚ್ಛಶಾಂಕ ಸದೃಶಚ್ಛವಿ ಯನ್ನ ಸದ್ಯಃ ।\\
     ಪ್ರಾಣಾನ್ಮುಮೋಚ ಮಹಿಷಸ್ತದತೀವ ಚಿತ್ರಂ\\
ಕೈರ್ಜೀವ್ಯತೇ ಹಿ ಕುಪಿತಾಂತಕ ದರ್ಶನೇನ ॥೧೩॥

     ದೇವಿ ಪ್ರಸೀದ ಪರಮಾ ಭವತೀ ಭವಾಯ\\
ಸದ್ಯೋ ವಿನಾಶಯಸಿ ಕೋಪವತೀ ಕುಲಾನಿ ।\\
     ವಿಜ್ಞಾತಮೇತದಧುನೈವ ಯದಸ್ತಮೇತ-\\
ನ್ನೀತಂ ಬಲಂ ಸುವಿಪುಲಂ ಮಹಿಷಾಸುರಸ್ಯ ॥೧೪॥

     ತೇ ಸಮ್ಮತಾ ಜನಪದೇಷು ಧನಾನಿ ತೇಷಾಂ\\
ತೇಷಾಂ ಯಶಾಂಸಿ ನ ಚ ಸೀದತಿ ಬಂಧುವರ್ಗಃ ।\\
     ಧನ್ಯಾಸ್ತ ಏವ ನಿಭೃತಾತ್ಮಜ ಭೃತ್ಯ ದಾರಾ\\
ಯೇಷಾಂ ಸದಾಭ್ಯುದಯದಾ ಭವತೀ ಪ್ರಸನ್ನಾ ॥೧೫॥

     ಧರ್ಮ್ಯಾಣಿ ದೇವಿ ಸಕಲಾನಿ ಸದೈವ ಕರ್ಮಾ-\\
ಣ್ಯತ್ಯಾದೃತಃ ಪ್ರತಿದಿನಂ ಸುಕೃತೀ ಕರೋತಿ ।\\
     ಸ್ವರ್ಗಂ ಪ್ರಯಾತಿ ಚ ತತೋ ಭವತೀಪ್ರಸಾದಾ-\\
ಲ್ಲೋಕತ್ರಯೇಽಪಿ ಫಲದಾ ನನು ದೇವಿ ತೇನ ॥೧೬॥

     ದುರ್ಗೇ ಸ್ಮೃತಾ ಹರಸಿ ಭೀತಿಮಶೇಷಜಂತೋಃ\\
ಸ್ವಸ್ಥೈಃ ಸ್ಮೃತಾ ಮತಿಮತೀವ ಶುಭಾಂ ದದಾಸಿ ।\\
     ದಾರಿದ್ರ್ಯ ದುಃಖ ಭಯಹಾರಿಣಿ ಕಾ ತ್ವದನ್ಯಾ\\
ಸರ್ವೋಪಕಾರ ಕರಣಾಯ ಸದಾಽಽರ್ದ್ರಚಿತ್ತಾ ॥೧೭॥
\newpage
     ಏಭಿರ್ಹತೈರ್ಜಗದುಪೈತಿ ಸುಖಂ ತಥೈತೇ\\
ಕುರ್ವಂತು ನಾಮ ನರಕಾಯ ಚಿರಾಯ ಪಾಪಂ ।\\
     ಸಂಗ್ರಾಮಮೃತ್ಯುಮಧಿಗಮ್ಯ ದಿವಂ ಪ್ರಯಾಂತು\\
ಮತ್ವೇತಿ ನೂನಮಹಿತಾನ್ವಿನಿಹಂಸಿ ದೇವಿ ॥೧೮॥

     ದೃಷ್ಟ್ವೈವ ಕಿಂ ನ ಭವತೀ ಪ್ರಕರೋತಿ ಭಸ್ಮ\\
ಸರ್ವಾಸುರಾನರಿಷು ಯತ್ಪ್ರಹಿಣೋಷಿ ಶಸ್ತ್ರಂ ।\\
     ಲೋಕಾನ್ಪ್ರಯಾಂತು ರಿಪವೋಽಪಿ ಹಿ ಶಸ್ತ್ರಪೂತಾ\\
ಇತ್ಥಂ ಮತಿರ್ಭವತಿ ತೇಷ್ವಹಿತೇಷು ಸಾಧ್ವೀ ॥೧೯॥

     ಖಡ್ಗಪ್ರಭಾ ನಿಕರ ವಿಸ್ಫುರಣೈಸ್ತಥೋಗ್ರೈಃ\\
ಶೂಲಾಗ್ರಕಾಂತಿನಿವಹೇನ ದೃಶೋಽಸುರಾಣಾಂ ।\\
     ಯನ್ನಾಗತಾ ವಿಲಯಮಂಶುಮದಿಂದುಖಂಡ-\\
ಯೋಗ್ಯಾನನಂ ತವ ವಿಲೋಕಯತಾಂ ತದೇತತ್ ॥೨೦॥

     ದುರ್ವೃತ್ತವೃತ್ತಶಮನಂ ತವ ದೇವಿ ಶೀಲಂ\\
ರೂಪಂ ತಥೈತದವಿಚಿಂತ್ಯಮತುಲ್ಯಮನ್ಯೈಃ ।\\
     ವೀರ್ಯಂ ಚ ಹಂತೃಹೃತದೇವಪರಾಕ್ರಮಾಣಾಂ\\
ವೈರಿಷ್ವಪಿ ಪ್ರಕಟಿತೈವ ದಯಾ ತ್ವಯೇತ್ಥಂ ॥೨೧॥

     ಕೇನೋಪಮಾ ಭವತು ತೇಽಸ್ಯ ಪರಾಕ್ರಮಸ್ಯ\\
ರೂಪಂ ಚ ಶತ್ರುಭಯಕಾರ್ಯತಿಹಾರಿ ಕುತ್ರ ।\\
     ಚಿತ್ತೇ ಕೃಪಾ ಸಮರನಿಷ್ಠುರತಾ ಚ ದೃಷ್ಟಾ\\
ತ್ವಯ್ಯೇವ ದೇವಿ ವರದೇ ಭುವನತ್ರಯೇಽಪಿ ॥೨೨॥
\newpage
     ತ್ರೈಲೋಕ್ಯಮೇತದಖಿಲಂ ರಿಪುನಾಶನೇನ\\
ತ್ರಾತಂ ತ್ವಯಾ ಸಮರಮೂರ್ಧನಿ ತೇಽಪಿ ಹತ್ವಾ ।\\
     ನೀತಾ ದಿವಂ ರಿಪುಗಣಾ ಭಯಮಪ್ಯಪಾಸ್ತ-\\
ಮಸ್ಮಾಕಮುನ್ಮದಸುರಾರಿಭವಂ ನಮಸ್ತೇ ॥೨೩॥

ಶೂಲೇನ ಪಾಹಿ ನೋ ದೇವಿ ಪಾಹಿ ಖಡ್ಗೇನ ಚಾಂಬಿಕೇ ।\\
ಘಂಟಾಸ್ವನೇನ ನಃ ಪಾಹಿ ಚಾಪಜ್ಯಾನಿಃಸ್ವನೇನ ಚ ॥೨೪॥

ಪ್ರಾಚ್ಯಾಂ ರಕ್ಷ ಪ್ರತೀಚ್ಯಾಂ ಚ ಚಂಡಿಕೇ ರಕ್ಷ ದಕ್ಷಿಣೇ ।\\
ಭ್ರಾಮಣೇನಾತ್ಮಶೂಲಸ್ಯ ಉತ್ತರಸ್ಯಾಂ ತಥೇಶ್ವರಿ ॥೨೫॥

ಸೌಮ್ಯಾನಿ ಯಾನಿ ರೂಪಾಣಿ ತ್ರೈಲೋಕ್ಯೇ ವಿಚರಂತಿ ತೇ ।\\
ಯಾನಿ ಚಾತ್ಯಂತ ಘೋರಾಣಿ ತೈ ರಕ್ಷಾಸ್ಮಾಂಸ್ತಥಾ ಭುವಂ ॥೨೬॥

ಖಡ್ಗಶೂಲಗದಾದೀನಿ ಯಾನಿ ಚಾಸ್ತ್ರಾಣಿ ತೇಽಮ್ಬಿಕೇ ।\\
ಕರಪಲ್ಲವಸಂಗೀನಿ ತೈರಸ್ಮಾನ್ರಕ್ಷ ಸರ್ವತಃ ॥೨೭॥

 ಋಷಿರುವಾಚ ॥೨೮॥\\
ಏವಂ ಸ್ತುತಾ ಸುರೈರ್ದಿವ್ಯೈಃ ಕುಸುಮೈರ್ನಂದನೋದ್ಭವೈಃ ।\\
ಅರ್ಚಿತಾ ಜಗತಾಂ ಧಾತ್ರೀ ತಥಾ ಗಂಧಾನುಲೇಪನೈಃ ॥೨೯॥

ಭಕ್ತ್ಯಾ ಸಮಸ್ತೈಸ್ತ್ರಿದಶೈರ್ದಿವ್ಯೈರ್ಧೂಪೈಃ ಸುಧೂಪಿತಾ ।\\
ಪ್ರಾಹ ಪ್ರಸಾದಸುಮುಖೀ ಸಮಸ್ತಾನ್ ಪ್ರಣತಾನ್ ಸುರಾನ್ ॥೩೦॥

ದೇವ್ಯುವಾಚ ॥೩೧॥\\
ವ್ರಿಯತಾಂ ತ್ರಿದಶಾಃ ಸರ್ವೇ ಯದಸ್ಮತ್ತೋಽಭಿವಾಂಛಿತಂ ॥೩೨॥

ದೇವಾ ಊಚುಃ ॥೩೩॥\\
\newpage
ಭಗವತ್ಯಾ ಕೃತಂ ಸರ್ವಂ ನ ಕಿಂಚಿದವಶಿಷ್ಯತೇ ।\\
ಯದಯಂ ನಿಹತಃ ಶತ್ರುರಸ್ಮಾಕಂ ಮಹಿಷಾಸುರಃ ॥೩೪॥

ಯದಿ ಚಾಪಿ ವರೋ ದೇಯಸ್ತ್ವಯಾಽಸ್ಮಾಕಂ ಮಹೇಶ್ವರಿ ।\\
ಸಂಸ್ಮೃತಾಽಸಂಸ್ಮೃತಾ ತ್ವಂ ನೋ ಹಿಂಸೇಥಾಃ ಪರಮಾಪದಃ ॥೩೫॥

ಯಶ್ಚ ಮರ್ತ್ಯಃ ಸ್ತವೈರೇಭಿಸ್ತ್ವಾಂ ಸ್ತೋಷ್ಯತ್ಯಮಲಾನನೇ ॥೩೬॥

ತಸ್ಯ ವಿತ್ತರ್ದ್ಧಿವಿಭವೈರ್ಧನದಾರಾದಿ ಸಂಪದಾಂ ।\\
ವೃದ್ಧಯೇಽಸ್ಮತ್ಪ್ರಸನ್ನಾ ತ್ವಂ ಭವೇಥಾಃ ಸರ್ವದಾಂಬಿಕೇ ॥೩೭॥


\section{ನಾರಾಯಣೀಸ್ತುತಿಃ}
ಓಂ ಋಷಿರುವಾಚ ॥೧॥\\
ದೇವ್ಯಾ ಹತೇ ತತ್ರ ಮಹಾಸುರೇಂದ್ರೇ\\
        ಸೇಂದ್ರಾಃ ಸುರಾ ವಹ್ನಿಪುರೋಗಮಾಸ್ತಾಂ ।\\
ಕಾತ್ಯಾಯನೀಂ ತುಷ್ಟುವುರಿಷ್ಟಲಾಭಾದ್\\
      ವಿಕಾಶಿವಕ್ತ್ರಾಬ್ಜವಿಕಾಶಿತಾಶಾಃ ॥೨॥

ದೇವಿ ಪ್ರಪನ್ನಾರ್ತಿಹರೇ ಪ್ರಸೀದ\\
        ಪ್ರಸೀದ ಮಾತರ್ಜಗತೋಽಖಿಲಸ್ಯ ।\\
ಪ್ರಸೀದ ವಿಶ್ವೇಶ್ವರಿ ಪಾಹಿ ವಿಶ್ವಂ\\
        ತ್ವಮೀಶ್ವರೀ ದೇವಿ ಚರಾಚರಸ್ಯ ॥೩॥

ಆಧಾರಭೂತಾ ಜಗತಸ್ತ್ವಮೇಕಾ\\
        ಮಹೀಸ್ವರೂಪೇಣ ಯತಃ ಸ್ಥಿತಾಸಿ ।\\
ಅಪಾಂ ಸ್ವರೂಪಸ್ಥಿತಯಾ ತ್ವಯೈತ-\\
      ದಾಪ್ಯಾಯತೇ ಕೃತ್ಸ್ನಮಲಂಘ್ಯವೀರ್ಯೇ ॥೪॥

ತ್ವಂ ವೈಷ್ಣವೀಶಕ್ತಿರನಂತವೀರ್ಯಾ\\
      ವಿಶ್ವಸ್ಯ ಬೀಜಂ ಪರಮಾಸಿ ಮಾಯಾ ।\\
ಸಮ್ಮೋಹಿತಂ ದೇವಿ ಸಮಸ್ತಮೇತತ್\\
      ತ್ವಂ ವೈ ಪ್ರಸನ್ನಾ ಭುವಿ ಮುಕ್ತಿಹೇತುಃ ॥೫॥

ವಿದ್ಯಾಃ ಸಮಸ್ತಾಸ್ತವ ದೇವಿ ಭೇದಾಃ\\
        ಸ್ತ್ರಿಯಃ ಸಮಸ್ತಾಃ ಸಕಲಾ ಜಗತ್ಸು ।\\
ತ್ವಯೈಕಯಾ ಪೂರಿತಮಂಬಯೈತತ್\\
        ಕಾ ತೇ ಸ್ತುತಿಃ ಸ್ತವ್ಯಪರಾಪರೋಕ್ತಿಃ ॥೬॥

ಸರ್ವಭೂತಾ ಯದಾ ದೇವೀ ಭುಕ್ತಿಮುಕ್ತಿಪ್ರದಾಯಿನೀ ।\\
ತ್ವಂ ಸ್ತುತಾ ಸ್ತುತಯೇ ಕಾ ವಾ ಭವಂತು ಪರಮೋಕ್ತಯಃ ॥೭॥

ಸರ್ವಸ್ಯ ಬುದ್ಧಿರೂಪೇಣ ಜನಸ್ಯ ಹೃದಿ ಸಂಸ್ಥಿತೇ ।\\
ಸ್ವರ್ಗಾಪವರ್ಗದೇ ದೇವಿ ನಾರಾಯಣಿ ನಮೋಽಸ್ತು ತೇ ॥೮॥

ಕಲಾಕಾಷ್ಠಾದಿರೂಪೇಣ ಪರಿಣಾಮಪ್ರದಾಯಿನಿ ।\\
ವಿಶ್ವಸ್ಯೋಪರತೌ ಶಕ್ತೇ ನಾರಾಯಣಿ ನಮೋಽಸ್ತು ತೇ ॥೯॥

ಸರ್ವಮಂಗಲಮಾಂಗಲ್ಯೇ ಶಿವೇ ಸರ್ವಾರ್ಥಸಾಧಿಕೇ ।\\
ಶರಣ್ಯೇ ತ್ರ್ಯಂಬಕೇ ಗೌರಿ ನಾರಾಯಣಿ ನಮೋಽಸ್ತು ತೇ ॥೧೦॥

ಸೃಷ್ಟಿಸ್ಥಿತಿವಿನಾಶಾನಾಂ ಶಕ್ತಿಭೂತೇ ಸನಾತನಿ ।\\
ಗುಣಾಶ್ರಯೇ ಗುಣಮಯೇ ನಾರಾಯಣಿ ನಮೋಽಸ್ತು ತೇ ॥೧೧॥

ಶರಣಾಗತದೀನಾರ್ತಪರಿತ್ರಾಣಪರಾಯಣೇ ।\\
ಸರ್ವಸ್ಯಾರ್ತಿಹರೇ ದೇವಿ ನಾರಾಯಣಿ ನಮೋಽಸ್ತು ತೇ ॥೧೨॥

ಹಂಸಯುಕ್ತವಿಮಾನಸ್ಥೇ ಬ್ರಹ್ಮಾಣೀರೂಪಧಾರಿಣಿ ।\\
ಕೌಶಾಂಭಃಕ್ಷರಿಕೇ ದೇವಿ ನಾರಾಯಣಿ ನಮೋಽಸ್ತು ತೇ ॥೧೩॥

ತ್ರಿಶೂಲಚಂದ್ರಾಹಿಧರೇ ಮಹಾವೃಷಭವಾಹಿನಿ ।\\
ಮಾಹೇಶ್ವರೀಸ್ವರೂಪೇಣ ನಾರಾಯಣಿ ನಮೋಽಸ್ತುತೇ ॥೧೪॥

ಮಯೂರಕುಕ್ಕುಟವೃತೇ ಮಹಾಶಕ್ತಿಧರೇಽನಘೇ ।\\
ಕೌಮಾರೀರೂಪಸಂಸ್ಥಾನೇ ನಾರಾಯಣಿ ನಮೋಽಸ್ತು ತೇ ॥೧೫॥

ಶಂಖಚಕ್ರಗದಾಶಾರ್ಙ್ಗಗೃಹೀತಪರಮಾಯುಧೇ ।\\
ಪ್ರಸೀದ ವೈಷ್ಣವೀರೂಪೇ ನಾರಾಯಣಿ ನಮೋಽಸ್ತು ತೇ ॥೧೬॥

ಗೃಹೀತೋಗ್ರಮಹಾಚಕ್ರೇ ದಂಷ್ಟ್ರೋದ್ಧೃತವಸುಂಧರೇ ।\\
ವರಾಹರೂಪಿಣಿ ಶಿವೇ ನಾರಾಯಣಿ ನಮೋಽಸ್ತು ತೇ ॥೧೭॥

ನೃಸಿಂಹರೂಪೇಣೋಗ್ರೇಣ ಹಂತುಂ ದೈತ್ಯಾನ್ ಕೃತೋದ್ಯಮೇ ।\\
ತ್ರೈಲೋಕ್ಯತ್ರಾಣಸಹಿತೇ ನಾರಾಯಣಿ ನಮೋಽಸ್ತು ತೇ ॥೧೮॥

ಕಿರೀಟಿನಿ ಮಹಾವಜ್ರೇ ಸಹಸ್ರನಯನೋಜ್ಜ್ವಲೇ ।\\
ವೃತ್ರಪ್ರಾಣಹರೇ ಚೈಂದ್ರಿ ನಾರಾಯಣಿ ನಮೋಽಸ್ತು ತೇ ॥೧೯॥

ಶಿವದೂತೀ ಸ್ವರೂಪೇಣ ಹತದೈತ್ಯ ಮಹಾಬಲೇ ।\\
ಘೋರರೂಪೇ ಮಹಾರಾವೇ ನಾರಾಯಣಿ ನಮೋಽಸ್ತು ತೇ ॥೨೦॥

ದಂಷ್ಟ್ರಾ ಕರಾಲವದನೇ ಶಿರೋಮಾಲಾವಿಭೂಷಣೇ ।\\
ಚಾಮುಂಡೇ ಮುಂಡಮಥನೇ ನಾರಾಯಣಿ ನಮೋಽಸ್ತು ತೇ ॥೨೧॥

ಲಕ್ಷ್ಮಿ ಲಜ್ಜೇ ಮಹಾವಿದ್ಯೇ ಶ್ರದ್ಧೇ ಪುಷ್ಟಿ ಸ್ವಧೇ ಧ್ರುವೇ ।\\
ಮಹಾರಾತ್ರಿ ಮಹಾಮಾಯೇ ನಾರಾಯಣಿ ನಮೋಽಸ್ತು ತೇ ॥೨೨॥

ಮೇಧೇ ಸರಸ್ವತಿ ವರೇ ಭೂತಿ ಬಾಭ್ರವಿ ತಾಮಸಿ ।\\
ನಿಯತೇ ತ್ವಂ ಪ್ರಸೀದೇಶೇ ನಾರಾಯಣಿ ನಮೋಽಸ್ತುತೇ ॥೨೩॥

ಸರ್ವಸ್ವರೂಪೇ ಸರ್ವೇಶೇ ಸರ್ವಶಕ್ತಿಸಮನ್ವಿತೇ ।\\
ಭಯೇಭ್ಯಸ್ತ್ರಾಹಿ ನೋ ದೇವಿ ದುರ್ಗೇ ದೇವಿ ನಮೋಽಸ್ತು ತೇ ॥೨೪॥

ಏತತ್ತೇ ವದನಂ ಸೌಮ್ಯಂ ಲೋಚನತ್ರಯಭೂಷಿತಂ ।\\
ಪಾತು ನಃ ಸರ್ವಭೂತೇಭ್ಯಃ ಕಾತ್ಯಾಯನಿ ನಮೋಽಸ್ತು ತೇ ॥೨೫॥

ಜ್ವಾಲಾ ಕರಾಲಮತ್ಯುಗ್ರಮಶೇಷಾಸುರ ಸೂದನಂ ।\\
ತ್ರಿಶೂಲಂ ಪಾತು ನೋ ಭೀತೇರ್ಭದ್ರಕಾಲಿ ನಮೋಽಸ್ತು ತೇ ॥೨೬॥

ಹಿನಸ್ತಿ ದೈತ್ಯತೇಜಾಂಸಿ ಸ್ವನೇನಾಪೂರ್ಯ ಯಾ ಜಗತ್ ।\\
ಸಾ ಘಂಟಾ ಪಾತು ನೋ ದೇವಿ ಪಾಪೇಭ್ಯೋ ನಃ ಸುತಾನಿವ ॥೨೭॥

ಅಸುರಾಸೃಗ್ವಸಾ ಪಂಕ ಚರ್ಚಿತಸ್ತೇ ಕರೋಜ್ಜ್ವಲಃ ।\\
ಶುಭಾಯ ಖಡ್ಗೋ ಭವತು ಚಂಡಿಕೇ ತ್ವಾಂ ನತಾ ವಯಂ ॥೨೮॥

ರೋಗಾನಶೇಷಾನಪಹಂಸಿ ತುಷ್ಟಾ\\
        ರುಷ್ಟಾ ತು ಕಾಮಾನ್ ಸಕಲಾನಭೀಷ್ಟಾನ್ ।\\
ತ್ವಾಮಾಶ್ರಿತಾನಾಂ ನ ವಿಪನ್ನರಾಣಾಂ\\
        ತ್ವಾಮಾಶ್ರಿತಾ ಹ್ಯಾಶ್ರಯತಾಂ ಪ್ರಯಾಂತಿ ॥೨೯॥

ಏತತ್ಕೃತಂ ಯತ್ಕದನಂ ತ್ವಯಾದ್ಯ\\
        ಧರ್ಮದ್ವಿಷಾಂ ದೇವಿ ಮಹಾಸುರಾಣಾಂ ।\\
ರೂಪೈರನೇಕೈರ್ಬಹುಧಾತ್ಮಮೂರ್ತಿಂ\\
        ಕೃತ್ವಾಂಬಿಕೇ ತತ್ಪ್ರಕರೋತಿ ಕಾನ್ಯಾ ॥೩೦॥
\newpage
ವಿದ್ಯಾಸು ಶಾಸ್ತ್ರೇಷು ವಿವೇಕದೀಪೇ-\\
      ಷ್ವಾದ್ಯೇಷು ವಾಕ್ಯೇಷು ಚ ಕಾ ತ್ವದನ್ಯಾ ।\\
ಮಮತ್ವಗರ್ತೇಽತಿಮಹಾಂಧಕಾರೇ\\
      ವಿಭ್ರಾಮಯಸ್ಯೇತದತೀವ ವಿಶ್ವಂ ॥೩೧॥

ರಕ್ಷಾಂಸಿ ಯತ್ರೋಗ್ರವಿಷಾಶ್ಚ ನಾಗಾ\\
        ಯತ್ರಾರಯೋ ದಸ್ಯುಬಲಾನಿ ಯತ್ರ ।\\
ದಾವಾನಲೋ ಯತ್ರ ತಥಾಬ್ಧಿಮಧ್ಯೇ\\
        ತತ್ರ ಸ್ಥಿತಾ ತ್ವಂ ಪರಿಪಾಸಿ ವಿಶ್ವಂ ॥೩೨॥

ವಿಶ್ವೇಶ್ವರಿ ತ್ವಂ ಪರಿಪಾಸಿ ವಿಶ್ವಂ\\
        ವಿಶ್ವಾತ್ಮಿಕಾ ಧಾರಯಸೀಹ ವಿಶ್ವಂ ।\\
ವಿಶ್ವೇಶವಂದ್ಯಾ ಭವತೀ ಭವಂತಿ\\
        ವಿಶ್ವಾಶ್ರಯಾ ಯೇ ತ್ವಯಿ ಭಕ್ತಿನಮ್ರಾಃ ॥೩೩॥

ದೇವಿ ಪ್ರಸೀದ ಪರಿಪಾಲಯ ನೋಽರಿಭೀತೇ-\\
      ರ್ನಿತ್ಯಂ ಯಥಾಸುರವಧಾದಧುನೈವ ಸದ್ಯಃ ।\\
ಪಾಪಾನಿ ಸರ್ವಜಗತಾಂ ಪ್ರಶಮಂ ನಯಾಶು\\
        ಉತ್ಪಾತಪಾಕಜನಿತಾಂಶ್ಚ ಮಹೋಪಸರ್ಗಾನ್ ॥೩೪॥

ಪ್ರಣತಾನಾಂ ಪ್ರಸೀದ ತ್ವಂ ದೇವಿ ವಿಶ್ವಾರ್ತಿಹಾರಿಣಿ ।\\
ತ್ರೈಲೋಕ್ಯವಾಸಿನಾಮೀಡ್ಯೇ ಲೋಕಾನಾಂ ವರದಾ ಭವ ॥೩೫॥

ದೇವ್ಯುವಾಚ ॥೩೬॥\\
ವರದಾಹಂ ಸುರಗಣಾ ವರಂ ಯಂ ಮನಸೇಚ್ಛಥ ।\\
ತಂ ವೃಣುಧ್ವಂ ಪ್ರಯಚ್ಛಾಮಿ ಜಗತಾಮುಪಕಾರಕಂ ॥೩೭॥

ದೇವಾ ಊಚುಃ ॥೩೮॥\\
ಸರ್ವಾಬಾಧಾಪ್ರಶಮನಂ ತ್ರೈಲೋಕ್ಯಸ್ಯಾಖಿಲೇಶ್ವರಿ ।\\
ಏವಮೇವ ತ್ವಯಾ ಕಾರ್ಯಮಸ್ಮದ್ವೈರಿವಿನಾಶನಂ ॥೩೯॥
\section{ಶ್ರೀಮೇಧಾದಕ್ಷಿಣಾಮೂರ್ತಿಸಹಸ್ರನಾಮಸ್ತೋತ್ರಂ}
\addcontentsline{toc}{section}{ಶ್ರೀಮೇಧಾದಕ್ಷಿಣಾಮೂರ್ತಿಸಹಸ್ರನಾಮಸ್ತೋತ್ರಂ}
ಅಸ್ಯ ಶ್ರೀ ಮೇಧಾದಕ್ಷಿಣಾಮೂರ್ತಿಸಹಸ್ರನಾಮಸ್ತೋತ್ರಸ್ಯ ಬ್ರಹ್ಮಾ ಋಷಿಃ~। ಗಾಯತ್ರೀ ಛಂದಃ~। ದಕ್ಷಿಣಾಮೂರ್ತಿರ್ದೇವತಾ~। ಓಂ ಬೀಜಂ~। ಸ್ವಾಹಾ ಶಕ್ತಿಃ~। ನಮಃ ಕೀಲಕಂ~। ದಕ್ಷಿಣಾಮೂರ್ತಿಪ್ರೀತ್ಯರ್ಥೇ ಜಪೇ ವಿನಿಯೋಗಃ~॥

\dhyana{ಸಿದ್ಧಿತೋಯನಿಧೇರ್ಮಧ್ಯೇ ರತ್ನದ್ವೀಪೇ ಮನೋರಮೇ~।\\
ಕದಂಬವನಿಕಾಮಧ್ಯೇ ಶ್ರೀಮದ್ವಟತರೋರಧಃ ॥೧॥

	ಆಸೀನಮಾದ್ಯಂ ಪುರುಷಮಾದಿಮಧ್ಯಾಂತವರ್ಜಿತಂ~।\\
	ಶುದ್ಧಸ್ಫಟಿಕ ಗೋಕ್ಷೀರ ಶರತ್ಪೂರ್ಣೇಂದು ಶೇಖರಂ ॥೨॥

ದಕ್ಷಿಣೇ ಚಾಕ್ಷಮಾಲಾಂ ಚ ವಹ್ನಿಂ ವೈ ವಾಮಹಸ್ತಕೇ~।\\
ಜಟಾಮಂಡಲ ಸಂಲಗ್ನ ಶೀತಾಂಶುಕರ ಮಂಡಿತಂ ॥೩॥

	ನಾಗಹಾರಧರಂ ಚಾರುಕಂಕಣೈಃ ಕಟಿಸೂತ್ರಕೈಃ~।\\
	ವಿರಾಜಮಾನ ವೃಷಭಂ ವ್ಯಾಘ್ರ ಚರ್ಮಾಂಬರಾವೃತಂ ॥೪॥

ಚಿಂತಾಮಣಿ ಮಹಾಬೃಂದೈಃ ಕಲ್ಪಕೈಃ ಕಾಮಧೇನುಭಿಃ~।\\
ಚತುಷ್ಷಷ್ಟಿ ಕಲಾವಿದ್ಯಾ ಮೂರ್ತಿಭಿಃ ಶ್ರುತಿಮಸ್ತಕೈಃ ॥೫॥

	ರತ್ನಸಿಂಹಾಸನೇ ಸಾಧುದ್ವೀಪಿಚರ್ಮ ಸಮಾಯುತಂ~।\\
	ತತ್ರಾಷ್ಟದಲಪದ್ಮಸ್ಯ ಕರ್ಣಿಕಾಯಾಂ ಸುಶೋಭನೇ ॥೬॥
\newpage
ವೀರಾಸನೇ ಸಮಾಸೀನಂ ಲಂಬದಕ್ಷಪದಾಂಬುಜಂ~।\\
ಜ್ಞಾನಮುದ್ರಾಂ ಪುಸ್ತಕಂ ಚ ವರಾಭೀತಿಧರಂ ಹರಂ ॥೭॥

	ಪಾದಮೂಲ ಸಮಾಕ್ರಾಂತ ಮಹಾಪಸ್ಮಾರ ವೈಭವಂ~।\\
	ರುದ್ರಾಕ್ಷಮಾಲಾಭರಣ ಭೂಷಿತಂ ಭೂತಿಭಾಸುರಂ ॥೮॥

ಗಜಚರ್ಮೋತ್ತರೀಯಂ ಚ ಮಂದಸ್ಮಿತ ಮುಖಾಂಬುಜಂ~।\\
ಸಿದ್ಧವೃಂದೈ ರ್ಯೋಗಿವೃಂದೈ ರ್ಮುನಿವೃಂದೈ ರ್ನಿಷೇವಿತಂ ॥೯॥

	ಆರಾಧ್ಯಮಾನವೃಷಭಂ ಅಗ್ನೀಂದುರವಿಲೋಚನಂ~।\\
	ಪೂರಯಂತಂ ಕೃಪಾದೃಷ್ಟ್ಯಾ  ಪುಮರ್ಥಾನಾಶ್ರಿತೇ ಜನೇ~।\\
ಏವಂ ವಿಭಾವಯೇದೀಶಂ ಸರ್ವವಿದ್ಯಾ ಕಲಾನಿಧಿಂ ॥೧೦॥}

	ದೇವದೇವೋ ಮಹಾದೇವೋ ದೇವಾನಾಮಪಿ ದೇಶಿಕಃ~।\\
	ದಕ್ಷಿಣಾಮೂರ್ತಿರೀಶಾನೋ ದಯಾಪೂರಿತ ದಿಙ್ಮುಖಃ ॥೧॥

ಕೈಲಾಸಶಿಖರೋತ್ತುಂಗ ಕಮನೀಯನಿಜಾಕೃತಿಃ~।\\
ವಟದ್ರುಮತಟೀ ದಿವ್ಯಕನಕಾಸನ ಸಂಸ್ಥಿತಃ ॥೨॥

	ಕಟೀತಟ ಪಟೀಭೂತ ಕರಿಚರ್ಮೋಜ್ಜ್ವಲಾಕೃತಿಃ~।\\
	ಪಾಟೀರಾಪಾಂಡುರಾಕಾರ ಪರಿಪೂರ್ಣಸುಧಾಧಿಪಃ~।೩॥

ಜಟಾಕೋಟೀರಘಟಿತ ಸುಧಾಕರ ಸುಧಾಪ್ಲುತಃ~।\\
ಪಶ್ಯಲ್ಲಲಾಟಸುಭಗ ಸುಂದರ ಭ್ರೂವಿಲಾಸವಾನ್ ॥೪॥

	ಕಟಾಕ್ಷಸರಣೀ ನಿರ್ಯತ್ಕರುಣಾಪೂರ್ಣ ಲೋಚನಃ~।\\
	ಕರ್ಣಾಲೋಲ ತಟಿದ್ವರ್ಣ ಕುಂಡಲೋಜ್ಜ್ವಲ ಗಂಡಭೂಃ ॥೫॥
\newpage
ತಿಲಪ್ರಸೂನ ಸಂಕಾಶ ನಾಸಿಕಾಪುಟ ಭಾಸುರಃ~।\\
ಮಂದಸ್ಮಿತ ಸ್ಫುರನ್ಮುಗ್ಧ ಮಹನೀಯ ಮುಖಾಂಬುಜಃ ॥೬॥

	ಕುಂದಕುಡ್ಮಲ ಸಂಸ್ಪರ್ಧಿ ದಂತಪಂಕ್ತಿ ವಿರಾಜಿತಃ~।\\
	ಸಿಂದೂರಾರುಣ ಸುಸ್ನಿಗ್ಧ ಕೋಮಲಾಧರ ಪಲ್ಲವಃ ॥೭॥

ಶಂಖಾಟೋಪ ಗಲದ್ದಿವ್ಯ ಗಳವೈಭವಮಂಜುಲಃ~।\\
ಕರಕಂದಲಿತ ಜ್ಞಾನಮುದ್ರಾ ರುದ್ರಾಕ್ಷಮಾಲಿಕಃ ॥೮॥

	ಅನ್ಯಹಸ್ತ ತಲನ್ಯಸ್ತ ವೀಣಾ ಪುಸ್ತೋಲ್ಲಸದ್ವಪುಃ~।\\
	ವಿಶಾಲ ರುಚಿರೋರಸ್ಕ ವಲಿಮತ್ಪಲ್ಲವೋದರಃ ॥೯॥

ಬೃಹತ್ಕಟಿ ನಿತಂಬಾಢ್ಯಃ ಪೀವರೋರು ದ್ವಯಾನ್ವಿತಃ~।\\
ಜಂಘಾವಿಜಿತ ತೂಣೀರಸ್ತುಂಗಗುಲ್ಫ ಯುಗೋಜ್ಜ್ವಲಃ ॥೧೦॥

	ಮೃದು ಪಾಟಲ ಪಾದಾಬ್ಜಶ್ಚಂದ್ರಾಭ ನಖದೀಧಿತಿಃ~।\\
	ಅಪಸವ್ಯೋರು ವಿನ್ಯಸ್ತ ಸವ್ಯಪಾದ ಸರೋರುಹಃ ॥೧೧॥

ಘೋರಾಪಸ್ಮಾರ ನಿಕ್ಷಿಪ್ತ ಧೀರದಕ್ಷ ಪದಾಂಬುಜಃ~।\\
ಸನಕಾದಿ ಮುನಿಧ್ಯೇಯಃ ಸರ್ವಾಭರಣ ಭೂಷಿತಃ ॥೧೨॥

	ದಿವ್ಯಚಂದನ ಲಿಪ್ತಾಂಗಶ್ಚಾರುಹಾಸ ಪರಿಷ್ಕೃತಃ~।\\
	ಕರ್ಪೂರ ಧವಲಾಕಾರಃ ಕಂದರ್ಪಶತ ಸುಂದರಃ ॥೧೩॥

ಕಾತ್ಯಾಯನೀ ಪ್ರೇಮನಿಧಿಃ ಕರುಣಾರಸ ವಾರಿಧಿಃ~।\\
ಕಾಮಿತಾರ್ಥಪ್ರದಃ ಶ್ರೀಮತ್ಕಮಲಾ ವಲ್ಲಭಪ್ರಿಯಃ ॥೧೪॥

	ಕಟಾಕ್ಷಿತಾತ್ಮವಿಜ್ಞಾನಃ ಕೈವಲ್ಯಾನಂದಕಂದಲಃ~।\\
	ಮಂದಹಾಸ ಸಮಾನೇಂದುಃ ಛಿನ್ನಾಜ್ಞಾನ ತಮಸ್ತತಿಃ ॥೧೫॥

ಸಂಸಾರಾನಲ ಸಂತಪ್ತಜನತಾಮೃತ ಸಾಗರಃ~।\\
ಗಂಭೀರ ಹೃದಯಾಂಭೋಜ ನಭೋಮಣಿ ನಿಭಾಕೃತಿಃ ॥೧೬॥

	ನಿಶಾಕರಕರಾಕಾರ ವಶೀಕೃತಜಗತ್ತ್ರಯಃ~।\\
	ತಾಪಸಾರಾಧ್ಯ ಪಾದಾಬ್ಜಸ್ತರುಣಾನಂದ ವಿಗ್ರಹಃ ॥೧೭॥

ಭೂತಿ ಭೂಷಿತ ಸರ್ವಾಂಗೋ ಭೂತಾಧಿಪತಿರೀಶ್ವರಃ~।\\
ವದನೇಂದು ಸ್ಮಿತಜ್ಯೋತ್ಸ್ನಾನಿಲೀನ ತ್ರಿಪುರಾಕೃತಿಃ ॥೧೮॥

	ತಾಪತ್ರಯತ ಮೋಭಾನುಃ ಪಾಪಾರಣ್ಯ ದವಾನಲಃ~।\\
	ಸಂಸಾರ ಸಾಗರೋದ್ಧರ್ತಾ ಹಂಸಾಗ್ರ್ಯೋಪಾಸ್ಯ ವಿಗ್ರಹಃ ॥೧೯॥

ಲಲಾಟ ಹುತಭುಗ್ದಗ್ಧ ಮನೋಭವ ಶುಭಾಕೃತಿಃ~।\\
ತುಚ್ಛೀಕೃತ ಜಗಜ್ಜಾಲ ಸ್ತುಷಾರಕರ ಶೀತಲಃ ॥೨೦॥

	ಅಸ್ತಂಗತ ಸಮಸ್ತೇಚ್ಛೋ ನಿಸ್ತುಲಾನಂದ ಮಂಥರಃ~।\\
	ಧೀರೋದಾತ್ತ ಗುಣಾಧಾರ ಉದಾರ ವರವೈಭವಃ ॥೨೧॥

ಅಪಾರ ಕರುಣಾಮೂರ್ತಿರಜ್ಞಾನ ಧ್ವಾಂತಭಾಸ್ಕರಃ~।\\
ಭಕ್ತಮಾನಸ ಹಂಸಾಗ್ರ್ಯ ಭವಾಮಯ ಭಿಷಕ್ತಮಃ ॥೨೨॥

	ಯೋಗೀಂದ್ರಪೂಜ್ಯಪಾದಾಬ್ಜೋ ಯೋಗಪಟ್ಟೋಲ್ಲಸತ್ಕಟಿಃ~।\\
	ಶುದ್ಧಸ್ಫಟಿಕಸಂಕಾಶೋ ಬದ್ಧಪನ್ನಗಭೂಷಣಃ ॥೨೩॥

ನಾನಾಮುನಿ ಸಮಾಕೀರ್ಣೋ ನಾಸಾಗ್ರನ್ಯಸ್ತಲೋಚನಃ~।\\
ವೇದಮೂರ್ಧೈಕಸಂವೇದ್ಯೋ ನಾದಧ್ಯಾನಪರಾಯಣಃ ॥೨೪॥

	ಧರಾಧರೇಂದು ರಾನಂದಸಂದೋಹ ರಸಸಾಗರಃ~।\\
	ದ್ವೈತವೃಂದವಿಮೋಹಾಂಧ್ಯ ಪರಾಕೃತ ದೃಗದ್ಭುತಃ ॥೨೫॥

ಪ್ರತ್ಯಗಾತ್ಮಾ ಪರಂಜ್ಯೋತಿಃ ಪುರಾಣಃ ಪರಮೇಶ್ವರಃ~।\\
ಪ್ರಪಂಚೋಪಶಮಃ ಪ್ರಾಜ್ಞಃ ಪುಣ್ಯಕೀರ್ತಿಃ ಪುರಾತನಃ ॥೨೬॥

	ಸರ್ವಾಧಿಷ್ಠಾನ ಸನ್ಮಾತ್ರಸ್ಸ್ವಾತ್ಮ ಬಂಧಹರೋ ಹರಃ~।\\
	ಸರ್ವಪ್ರೇಮನಿಜಾಹಾಸಃ ಸರ್ವಾನುಗ್ರಹಕೃತ್ ಶಿವಃ ॥೨೭॥

ಸರ್ವೇಂದ್ರಿಯಗುಣಾಭಾಸಃ ಸರ್ವಭೂತಗುಣಾಶ್ರಯಃ~।\\
ಸಚ್ಚಿದಾನಂದಪೂರ್ಣಾತ್ಮಾ ಸ್ವೇ ಮಹಿಮ್ನಿ ಪ್ರತಿಷ್ಠಿತಃ ॥೨೮॥

	ಸರ್ವಭೂತಾಂತರಸ್ಸಾಕ್ಷೀ ಸರ್ವಜ್ಞಸ್ಸರ್ವಕಾಮದಃ~।\\
	ಸನಕಾದಿಮಹಾಯೋಗಿಸಮಾರಾಧಿತಪಾದುಕಃ ॥೨೯॥

ಆದಿದೇವೋ ದಯಾಸಿಂಧುಃ ಶಿಕ್ಷಿತಾಸುರವಿಗ್ರಹಃ~।\\
ಯಕ್ಷಕಿನ್ನರಗಂಧರ್ವಸ್ತೂಯಮಾನಾತ್ಮವೈಭವಃ ॥೩೦॥

	ಬ್ರಹ್ಮಾದಿದೇವವಿನುತೋ ಯೋಗಮಾಯಾನಿಯೋಜಕಃ~।\\
	ಶಿವಯೋಗೀ ಶಿವಾನಂದಃ ಶಿವಭಕ್ತಸಮುದ್ಧರಃ ॥೩೧॥

ವೇದಾಂತಸಾರಸಂದೋಹಃ ಸರ್ವಸತ್ತ್ವಾವಲಂಬನಃ~।\\
ವಟಮೂಲಾಶ್ರಯೋ ವಾಗ್ಮೀ ಮಾನ್ಯೋ ಮಲಯಜಪ್ರಿಯಃ ॥೩೨॥

	ಸುಶೀಲೋ ವಾಂಛಿತಾರ್ಥಜ್ಞಃ ಪ್ರಸನ್ನವದನೇಕ್ಷಣಃ ।\\
	ನೃತ್ತಗೀತಕಲಾಭಿಜ್ಞಃ ಕರ್ಮವಿತ್ ಕರ್ಮಮೋಚಕಃ ॥೩೩॥

ಕರ್ಮಸಾಕ್ಷೀ ಕರ್ಮಮಯಃ ಕರ್ಮಣಾಂ ಚ ಫಲಪ್ರದಃ~।\\
ಜ್ಞಾನದಾತಾ ಸದಾಚಾರಃ ಸರ್ವೋಪದ್ರವಮೋಚಕಃ ॥೩೪॥

	ಅನಾಥನಾಥೋ ಭಗವಾನಾಶ್ರಿತಾಮರಪಾದಪಃ~।\\
	ವರಪ್ರದಃ ಪ್ರಕಾಶಾತ್ಮಾ ಸರ್ವಭೂತಹಿತೇ ರತಃ ॥೩೫॥

ವ್ಯಾಘ್ರಚರ್ಮಾಸನಾಸೀನ ಆದಿಕರ್ತಾ ಮಹೇಶ್ವರಃ~।\\
ಸುವಿಕ್ರಮಃ ಸರ್ವಗತೋ ವಿಶಿಷ್ಟಜನವತ್ಸಲಃ ॥೩೬॥

	ಚಿಂತಾಶೋಕಪ್ರಶಮನೋ ಜಗದಾನಂದಕಾರಕಃ~।\\
	ರಶ್ಮಿಮಾನ್ ಭುವನೇಶಶ್ಚ ದೇವಾಸುರಸುಪೂಜಿತಃ ॥೩೭॥

ಮೃತ್ಯುಂಜಯೋ ವ್ಯೋಮಕೇಶಃ ಷಟ್ತ್ರಿಂಶತ್ತತ್ತ್ವಸಂಗ್ರಹಃ~।\\
ಅಜ್ಞಾತಸಂಭವೋ ಭಿಕ್ಷುರದ್ವಿತೀಯೋ ದಿಗಂಬರಃ ॥೩೮॥

	ಸಮಸ್ತದೇವತಾಮೂರ್ತಿಃ ಸೋಮಸೂರ್ಯಾಗ್ನಿಲೋಚನಃ~।\\
	ಸರ್ವಸಾಮ್ರಾಜ್ಯನಿಪುಣೋ ಧರ್ಮಮಾರ್ಗಪ್ರವರ್ತಕಃ ॥೩೯॥

ವಿಶ್ವಾಧಿಕಃ ಪಶುಪತಿಃ ಪಶುಪಾಶವಿಮೋಚಕಃ~।\\
ಅಷ್ಟಮೂರ್ತಿರ್ದೀಪ್ತಮೂರ್ತಿಃ ನಾಮೋಚ್ಚಾರಣಮುಕ್ತಿದಃ ॥೪೦॥

	ಸಹಸ್ರಾದಿತ್ಯಸಂಕಾಶಃ ಸದಾಷೋಡಶವಾರ್ಷಿಕಃ~।\\
	ದಿವ್ಯಕೇಲೀಸಮಾಯುಕ್ತೋ ದಿವ್ಯಮಾಲ್ಯಾಂಬರಾವೃತಃ ॥೪೧॥

ಅನರ್ಘರತ್ನಸಂಪೂರ್ಣೋ ಮಲ್ಲಿಕಾಕುಸುಮಪ್ರಿಯಃ~।\\
ತಪ್ತಚಾಮೀಕರಾಕಾರೋ ಜಿತದಾವಾನಲಾಕೃತಿಃ ॥೪೨॥

	ನಿರಂಜನೋ ನಿರ್ವಿಕಾರೋ ನಿಜಾವಾಸೋ ನಿರಾಕೃತಿಃ~।\\
	ಜಗದ್ಗುರುರ್ಜಗತ್ಕರ್ತಾ ಜಗದೀಶೋ ಜಗತ್ಪತಿಃ ॥೪೩॥

ಕಾಮಹಂತಾ ಕಾಮಮೂರ್ತಿಃ ಕಲ್ಯಾಣವೃಷವಾಹನಃ~।\\
ಗಂಗಾಧರೋ ಮಹಾದೇವೋ ದೀನಬಂಧವಿಮೋಚಕಃ ॥೪೪॥

	ಧೂರ್ಜಟಿಃ ಖಂಡಪರಶುಃ ಸದ್ಗುಣೋ ಗಿರಿಜಾಸಖಃ~।\\
	ಅವ್ಯಯೋ ಭೂತಸೇನೇಶಃ ಪಾಪಘ್ನಃ ಪುಣ್ಯದಾಯಕಃ ॥೪೫॥

ಉಪದೇಷ್ಟಾ ದೃಢಪ್ರಜ್ಞೋ ರುದ್ರೋ ರೋಗವಿನಾಶನಃ~।\\
ನಿತ್ಯಾನಂದೋ ನಿರಾಧಾರೋ ಹರೋ ದೇವಶಿಖಾಮಣಿಃ ॥೪೬॥

	ಪ್ರಣತಾರ್ತಿಹರಃ ಸೋಮಃ ಸಾಂದ್ರಾನಂದೋ ಮಹಾಮತಿಃ~।\\
	ಆಶ್ಚರ್ಯವೈಭವೋ ದೇವಃ ಸಂಸಾರಾರ್ಣವತಾರಕಃ ॥೪೭॥

ಯಜ್ಞೇಶೋ ರಾಜರಾಜೇಶೋ ಭಸ್ಮರುದ್ರಾಕ್ಷಲಾಂಛನಃ~।\\
ಅನಂತಸ್ತಾರಕಃ ಸ್ಥಾಣುಃ ಸರ್ವವಿದ್ಯೇಶ್ವರೋ ಹರಿಃ ॥೪೮॥

	ವಿಶ್ವರೂಪೋ ವಿರೂಪಾಕ್ಷಃ ಪ್ರಭುಃ ಪರಿವೃಢೋ ದೃಢಃ~।\\
	ಭವ್ಯೋ ಜಿತಾರಿಷಡ್ವರ್ಗೋ ಮಹೋದಾರೋ ವಿಷಾಶನಃ ॥೪೯॥

ಸುಕೀರ್ತಿರಾದಿಪುರುಷೋ ಜರಾಮರಣವರ್ಜಿತಃ~।\\
ಪ್ರಮಾಣಭೂತೋ ದುರ್ಜ್ಞೇಯಃ ಪುಣ್ಯಃ ಪರಪುರಂಜಯಃ ॥೫೦॥

	ಗುಣಾಕಾರೋ ಗುಣಶ್ರೇಷ್ಠಃ ಸಚ್ಚಿದಾನಂದವಿಗ್ರಹಃ~।\\
	ಸುಖದಃ ಕಾರಣಂ ಕರ್ತಾ ಭವಬಂಧವಿಮೋಚಕಃ ॥೫೧॥

ಅನಿರ್ವಿಣ್ಣೋ ಗುಣಗ್ರಾಹೀ ನಿಷ್ಕಲಂಕಃ ಕಲಂಕಹಾ~।\\
ಪುರುಷಃ ಶಾಶ್ವತೋ ಯೋಗೀ ವ್ಯಕ್ತಾವ್ಯಕ್ತಃ ಸನಾತನಃ ॥೫೨॥

	ಚರಾಚರಾತ್ಮಾ ಸೂಕ್ಷ್ಮಾತ್ಮಾ ವಿಶ್ವಕರ್ಮಾ ತಮೋಽಪಹೃತ್~।\\
	ಭುಜಂಗಭೂಷಣೋ ಭರ್ಗಸ್ತರುಣಃ ಕರುಣಾಲಯಃ ॥೫೩॥

ಅಣಿಮಾದಿಗುಣೋಪೇತೋ ಲೋಕವಶ್ಯವಿಧಾಯಕಃ~।\\
ಯೋಗಪಟ್ಟಧರೋ ಮುಕ್ತೋ ಮುಕ್ತಾನಾಂ ಪರಮಾ ಗತಿಃ ॥೫೪॥

	ಗುರುರೂಪಧರಃ ಶ್ರೀಮತ್ಪರಮಾನಂದಸಾಗರಃ~।\\
	ಸಹಸ್ರಬಾಹುಃ ಸರ್ವೇಶಃ ಸಹಸ್ರಾವಯವಾನ್ವಿತಃ ॥೫೫॥

ಸಹಸ್ರಮೂರ್ಧಾ ಸರ್ವಾತ್ಮಾ ಸಹಸ್ರಾಕ್ಷಃ ಸಹಸ್ರಪಾತ್~।\\
ನಿರಾಭಾಸಃ ಸೂಕ್ಷ್ಮತನುರ್ಹೃದಿ ಜ್ಞಾತಃ ಪರಾತ್ಪರಃ ॥೫೬॥

	ಸರ್ವಾತ್ಮಗಃ ಸರ್ವಸಾಕ್ಷೀ ನಿಃಸಂಗೋ ನಿರುಪದ್ರವಃ~।\\
	ನಿಷ್ಕಲಃ ಸಕಲಾಧ್ಯಕ್ಷಶ್ಚಿನ್ಮಯಸ್ತಮಸಃ ಪರಃ ॥೫೭॥

ಜ್ಞಾನವೈರಾಗ್ಯಸಂಪನ್ನೋ ಯೋಗಾನಂದಮಯಃ ಶಿವಃ~।\\
ಶಾಶ್ವತೈಶ್ವರ್ಯಸಂಪೂರ್ಣೋ ಮಹಾಯೋಗೀಶ್ವರೇಶ್ವರಃ ॥೫೮॥

	ಸಹಸ್ರಶಕ್ತಿಸಂಯುಕ್ತಃ ಪುಣ್ಯಕಾಯೋ ದುರಾಸದಃ~।\\
	ತಾರಕಬ್ರಹ್ಮಸಂಪೂರ್ಣಸ್ತಪಸ್ವಿಜನಸಂವೃತಃ ॥೫೯॥

ವಿಧೀಂದ್ರಾಮರಸಂಪೂಜ್ಯೋ ಜ್ಯೋತಿಷಾಂ ಜ್ಯೋತಿರುತ್ತಮಃ~।\\
ನಿರಕ್ಷರೋ ನಿರಾಲಂಬಃ ಸ್ವಾತ್ಮಾರಾಮೋ ವಿಕರ್ತನಃ ॥೬೦॥

	ನಿರವದ್ಯೋ ನಿರಾತಂಕೋ ಭೀಮೋ ಭೀಮಪರಾಕ್ರಮಃ~।\\
	ವೀರಭದ್ರಃ ಪುರಾರಾತಿರ್ಜಲಂಧರಶಿರೋಹರಃ ॥೬೧॥

ಅಂಧಕಾಸುರಸಂಹರ್ತಾ ಭಗನೇತ್ರಭಿದದ್ಭುತಃ~।\\
ವಿಶ್ವಗ್ರಾಸೋಽಧರ್ಮಶತ್ರುರ್ಬ್ರಹ್ಮಜ್ಞಾನೈಕಮಂಥರಃ ॥೬೨॥

	ಅಗ್ರೇಸರಸ್ತೀರ್ಥಭೂತಃ ಸಿತಭಸ್ಮಾವಕುಂಠನಃ~।\\
	ಅಕುಂಠಮೇಧಾಃ ಶ್ರೀಕಂಠೋ ವೈಕುಂಠಪರಮಪ್ರಿಯಃ ॥೬೩॥

ಲಲಾಟೋಜ್ಜ್ವಲನೇತ್ರಾಬ್ಜಸ್ತುಷಾರಕರಶೇಖರಃ~।\\
ಗಜಾಸುರಶಿರಶ್ಛೇತ್ತಾ ಗಂಗೋದ್ಭಾಸಿತಮೂರ್ಧಜಃ ॥೬೪॥

	ಕಲ್ಯಾಣಾಚಲಕೋದಂಡಃ ಕಮಲಾಪತಿಸಾಯಕಃ~।\\
	ವಾರಾಂ ಶೇವಧಿ ತೂಣೀರಃ ಸರೋಜಾಸನ ಸಾರಥಿಃ ॥೬೫॥

ತ್ರಯೀತುರಂಗಸಂಕ್ರಾಂತೋ ವಾಸುಕಿಜ್ಯಾವಿರಾಜಿತಃ~।\\
ರವೀಂದುಚರಣಾಚಾರಿಧರಾರಥವಿರಾಜಿತಃ ॥೬೬॥

	ತ್ರಯ್ಯಂತಪ್ರಗ್ರಹೋದಾರಚಾರುಘಂಟಾರವೋಜ್ಜ್ವಲಃ~।\\
	ಉತ್ತಾನಪರ್ವಲೋಮಾಢ್ಯೋ ಲೀಲಾವಿಜಿತಮನ್ಮಥಃ ॥೬೭॥

ಜಾತುಪ್ರಪನ್ನಜನತಾ ಜೀವನೋಪಾಯನೋತ್ಸುಕಃ~।\\
ಸಂಸಾರಾರ್ಣವನಿರ್ಮಗ್ನ ಸಮುದ್ಧರಣಪಂಡಿತಃ ॥೬೮॥

	ಮದದ್ವಿರದಧಿಕ್ಕಾರಿಗತಿಮಂಜುಲವೈಭವಃ~।\\
	ಮತ್ತಕೋಕಿಲಮಾಧುರ್ಯರಸನಿರ್ಭರಗೀರ್ಗಣಃ ॥೬೯॥

ಕೈವಲ್ಯೋದಧಿಕಲ್ಲೋಲಲೀಲಾತಾಂಡವಪಂಡಿತಃ~।\\
ವಿಷ್ಣುರ್ಜಿಷ್ಣುರ್ವಾಸುದೇವಃ ಪ್ರಭವಿಷ್ಣುಃ ಪುರಾತನಃ ॥೭೦॥

	ವರ್ಧಿಷ್ಣುರ್ವರದೋ ವೈದ್ಯೋ ಹರಿರ್ನಾರಾಯಣೋಽಚ್ಯುತಃ~।\\
	ಅಜ್ಞಾನವನದಾವಾಗ್ನಿಃ ಪ್ರಜ್ಞಾಪ್ರಾಸಾದಭೂಪತಿಃ ॥೭೧॥

ಸರ್ಪಭೂಷಿತಸರ್ವಾಂಗಃ ಕರ್ಪೂರೋಜ್ಜ್ವಲಿತಾಕೃತಿಃ~।\\
ಅನಾದಿಮಧ್ಯನಿಧನೋ ಗಿರೀಶೋ ಗಿರಿಜಾಪತಿಃ ॥೭೨॥

	ವೀತರಾಗೋ ವಿನೀತಾತ್ಮಾ ತಪಸ್ವೀ ಭೂತಭಾವನಃ~।\\
	ದೇವಾಸುರಗುರುಧ್ಯೇಯೋ ದೇವಾಸುರನಮಸ್ಕೃತಃ ॥೭೩॥

ದೇವಾದಿದೇವೋ ದೇವರ್ಷಿರ್ದೇವಾಸುರವರಪ್ರದಃ~।\\
ಸರ್ವದೇವಮಯೋಽಚಿಂತ್ಯೋ ದೇವಾತ್ಮಾ ಚಾತ್ಮಸಂಭವಃ ॥೭೪॥

	ನಿರ್ಲೇಪೋ ನಿಷ್ಪ್ರಪಂಚಾತ್ಮಾ ನಿರ್ವಿಘ್ನೋ ವಿಘ್ನನಾಶಕಃ~।\\
	ಏಕಜ್ಯೋತಿರ್ನಿರಾತಂಕೋ ವ್ಯಾಪ್ತಮೂರ್ತಿರನಾಕುಲಃ ॥೭೫॥

ನಿರವದ್ಯಪದೋಪಾಧಿರ್ವಿದ್ಯಾರಾಶಿರನುತ್ತಮಃ~।\\
ನಿತ್ಯಾನಂದಃ ಸುರಾಧ್ಯಕ್ಷೋ ನಿಃಸಂಕಲ್ಪೋ ನಿರಂಜನಃ ॥೭೬॥

	ನಿಷ್ಕಲಂಕೋ ನಿರಾಕಾರೋ ನಿಷ್ಪ್ರಪಂಚೋ ನಿರಾಮಯಃ~।\\
	ವಿದ್ಯಾಧರೋ ವಿಯತ್ಕೇಶೋ ಮಾರ್ಕಂಡೇಯವರಪ್ರದಃ ॥೭೭॥

ಭೈರವೋ ಭೈರವೀನಾಥಃ ಕಾಮದಃ ಕಮಲಾಸನಃ~।\\
ವೇದವೇದ್ಯಃ ಸುರಾನಂದೋ ಲಸಜ್ಜ್ಯೋತಿಃ ಪ್ರಭಾಕರಃ ॥೭೮॥

	ಚೂಡಾಮಣಿಃ ಸುರಾಧೀಶೋ ಯಜ್ಞಗೇಯೋ ಹರಿಪ್ರಿಯಃ~।\\
	ನಿರ್ಲೇಪೋ ನೀತಿಮಾನ್ ಸೂತ್ರೀ ಶ್ರೀಹಾಲಾಹಲಸುಂದರಃ ॥೭೯॥

ಧರ್ಮದಕ್ಷೋ ಮಹಾರಾಜಃ ಕಿರೀಟೀ ವಂದಿತೋ ಗುಹಃ~।\\
ಮಾಧವೋ ಯಾಮಿನೀನಾಥಃ ಶಂಬರಃ ಶಬರೀಪ್ರಿಯಃ ॥೮೦॥

	ಸಂಗೀತವೇತ್ತಾ ಲೋಕಜ್ಞಃ ಶಾಂತಃ ಕಲಶಸಂಭವಃ~।\\
	ಬ್ರಹ್ಮಣ್ಯೋ ವರದೋ ನಿತ್ಯಃ ಶೂಲೀ ಗುರುವರೋ ಹರಃ ॥೮೧॥

ಮಾರ್ತಾಂಡಃ ಪುಂಡರೀಕಾಕ್ಷೋ ಲೋಕನಾಯಕವಿಕ್ರಮಃ~।\\
ಮುಕುಂದಾರ್ಚ್ಯೋ ವೈದ್ಯನಾಥಃ ಪುರಂದರವರಪ್ರದಃ ॥೮೨॥

	ಭಾಷಾವಿಹೀನೋ ಭಾಷಾಜ್ಞೋ ವಿಘ್ನೇಶೋ ವಿಘ್ನನಾಶನಃ~।\\
	ಕಿನ್ನರೇಶೋ ಬೃಹದ್ಭಾನುಃ ಶ್ರೀನಿವಾಸಃ ಕಪಾಲಭೃತ್ ॥೮೩॥

ವಿಜಯೋ ಭೂತಭಾವಜ್ಞೋ ಭೀಮಸೇನೋ ದಿವಾಕರಃ~।\\
ಬಿಲ್ವಪ್ರಿಯೋ ವಸಿಷ್ಠೇಶಃ ಸರ್ವಮಾರ್ಗಪ್ರವರ್ತಕಃ ॥೮೪॥

	ಓಷಧೀಶೋ ವಾಮದೇವೋ ಗೋವಿಂದೋ ನೀಲಲೋಹಿತಃ~।\\
	ಷಡರ್ಧನಯನಃ ಶ್ರೀಮನ್ಮಹಾದೇವೋ ವೃಷಧ್ವಜಃ ॥೮೫॥

ಕರ್ಪೂರದೀಪಿಕಾಲೋಲಃ ಕರ್ಪೂರರಸಚರ್ಚಿತಃ~।\\
ಅವ್ಯಾಜಕರುಣಾಮೂರ್ತಿಸ್ತ್ಯಾಗರಾಜಃ ಕ್ಷಪಾಕರಃ ॥೮೬॥

	ಆಶ್ಚರ್ಯವಿಗ್ರಹಃ ಸೂಕ್ಷ್ಮಃ ಸಿದ್ಧೇಶಃ ಸ್ವರ್ಣಭೈರವಃ~।\\
	ದೇವರಾಜಃ ಕೃಪಾಸಿಂಧುರದ್ವಯೋಽಮಿತವಿಕ್ರಮಃ ॥೮೭॥

ನಿರ್ಭೇದೋ ನಿತ್ಯಸತ್ವಸ್ಥೋ ನಿರ್ಯೋಗಕ್ಷೇಮ ಆತ್ಮವಾನ್~।\\
ನಿರಪಾಯೋ ನಿರಾಸಂಗೋ ನಿಃಶಬ್ದೋ ನಿರುಪಾಧಿಕಃ ॥೮೮॥

	ಭವಃ ಸರ್ವೇಶ್ವರಃ ಸ್ವಾಮೀ ಭವಭೀತಿವಿಭಂಜನಃ~।\\
	ದಾರಿದ್ರ್ಯತೃಣಕೂಟಾಗ್ನಿರ್ದಾರಿತಾಸುರಸಂತತಿಃ ॥೮೯॥

ಮುಕ್ತಿದೋ ಮುದಿತೋಽಕುಬ್ಜೋ ಧಾರ್ಮಿಕೋ ಭಕ್ತವತ್ಸಲಃ~।\\
ಅಭ್ಯಾಸಾತಿಶಯಜ್ಞೇಯಶ್ಚಂದ್ರಮೌಲಿಃ ಕಲಾಧರಃ ॥೯೦॥

	ಮಹಾಬಲೋ ಮಹಾವೀರ್ಯೋ ವಿಭುಃ ಶ್ರೀಶಃ ಶುಭಪ್ರದಃ~।\\
	ಸಿದ್ಧಃ ಪುರಾಣಪುರುಷೋ ರಣಮಂಡಲಭೈರವಃ ॥೯೧॥

ಸದ್ಯೋಜಾತೋ ವಟಾರಣ್ಯವಾಸೀ ಪುರುಷವಲ್ಲಭಃ~।\\
ಹರಿಕೇಶೋ ಮಹಾತ್ರಾತಾ ನೀಲಗ್ರೀವಸ್ಸುಮಂಗಲಃ ॥೯೨॥

	ಹಿರಣ್ಯಬಾಹುಸ್ತೀಕ್ಷ್ಣಾಂಶುಃ ಕಾಮೇಶಃ ಸೋಮವಿಗ್ರಹಃ~।\\
	ಸರ್ವಾತ್ಮಾ ಸರ್ವಕರ್ತಾ ಚ ತಾಂಡವೋ ಮುಂಡಮಾಲಿಕಃ ॥೯೩॥

ಅಗ್ರಗಣ್ಯಃ ಸುಗಂಭೀರೋ ದೇಶಿಕೋ ವೈದಿಕೋತ್ತಮಃ~।\\
ಪ್ರಸನ್ನದೇವೋ ವಾಗೀಶಶ್ಚಿಂತಾತಿಮಿರಭಾಸ್ಕರಃ ॥೯೪॥

	ಗೌರೀಪತಿಸ್ತುಂಗಮೌಲಿರ್ಮಖರಾಜೋ ಮಹಾಕವಿಃ~।\\
	ಶ್ರೀಧರಸ್ಸರ್ವಸಿದ್ಧೇಶೋ ವಿಶ್ವನಾಥೋ ದಯಾನಿಧಿಃ ॥೯೫॥

ಅಂತರ್ಮುಖೋ ಬಹಿರ್ದೃಷ್ಟಿಃ ಸಿದ್ಧವೇಷಮನೋಹರಃ~।\\
ಕೃತಿವಾಸಾಃ ಕೃಪಾಸಿಂಧುರ್ಮಂತ್ರಸಿದ್ಧೋ ಮತಿಪ್ರದಃ ॥೯೬॥

	ಮಹೋತ್ಕೃಷ್ಟಃ ಪುಣ್ಯಕರೋ ಜಗತ್ಸಾಕ್ಷೀ ಸದಾಶಿವಃ~।\\
	ಮಹಾಕ್ರತುರ್ಮಹಾಯಜ್ವಾ ವಿಶ್ವಕರ್ಮಾ ತಪೋನಿಧಿಃ ॥೯೭॥

ಛಂದೋಮಯೋ ಮಹಾಜ್ಞಾನೀ ಸರ್ವಜ್ಞೋ ದೇವವಂದಿತಃ~।\\
ಸಾರ್ವಭೌಮಸ್ಸದಾನಂದಃ ಕರುಣಾಮೃತವಾರಿಧಿಃ ॥೯೮॥

	ಕಾಲಕಾಲಃ ಕಲಿಧ್ವಂಸೀ ಜರಾಮರಣನಾಶಕಃ~।\\
	ಶಿತಿಕಂಠಶ್ಚಿದಾನಂದೋ ಯೋಗಿನೀಗಣಸೇವಿತಃ ॥೯೯॥

ಚಂಡೀಶಃ ಶುಕಸಂವೇದ್ಯಃ ಪುಣ್ಯಶ್ಲೋಕೋ ದಿವಸ್ಪತಿಃ~।\\
ಸ್ಥಾಯೀ ಸಕಲತತ್ತ್ವಾತ್ಮಾ ಸದಾಸೇವಕವರ್ಧನಃ ॥೧೦೦॥

	ರೋಹಿತಾಶ್ವಃ ಕ್ಷಮಾರೂಪೀ ತಪ್ತಚಾಮೀಕರಪ್ರಭಃ~।\\
	ತ್ರಿಯಂಬಕೋ ವರರುಚಿರ್ದೇವದೇವಶ್ಚತುರ್ಭುಜಃ ॥೧೦೧~।

ವಿಶ್ವಂಭರೋ ವಿಚಿತ್ರಾಂಗೋ ವಿಧಾತಾ ಪುರಶಾಸನಃ~।\\
ಸುಬ್ರಹ್ಮಣ್ಯೋ ಜಗತ್ಸ್ವಾಮೀ ರೋಹಿತಾಕ್ಷಃ ಶಿವೋತ್ತಮಃ ॥೧೦೨॥

	ನಕ್ಷತ್ರಮಾಲಾಭರಣೋ ಮಘವಾನ್ ಅಘನಾಶನಃ~।\\
	ವಿಧಿಕರ್ತಾ ವಿಧಾನಜ್ಞಃ ಪ್ರಧಾನಪುರುಷೇಶ್ವರಃ ॥೧೦೩॥

ಚಿಂತಾಮಣಿಃ ಸುರಗುರುರ್ಧ್ಯೇಯೋ ನೀರಾಜನಪ್ರಿಯಃ~।\\
ಗೋವಿಂದೋ ರಾಜರಾಜೇಶೋ ಬಹುಪುಷ್ಪಾರ್ಚನಪ್ರಿಯಃ ॥೧೦೪॥

ಸರ್ವಾನಂದೋ ದಯಾರೂಪೀ ಶೈಲಜಾಸುಮನೋಹರಃ~।\\
ಸುವಿಕ್ರಮಃ ಸರ್ವಗತೋ ಹೇತುಸಾಧನವರ್ಜಿತಃ ॥೧೦೫॥

	ವೃಷಾಂಕೋ ರಮಣೀಯಾಂಗಃ ಸದಂಘ್ರಿಃ ಸಾಮಪಾರಗಃ~।\\
	ಮಂತ್ರಾತ್ಮಾ ಕೋಟಿಕಂದರ್ಪಸೌಂದರ್ಯರಸವಾರಿಧಿಃ ॥೧೦೬ ॥

ಯಜ್ಞೇಶೋ ಯಜ್ಞಪುರುಷಃ ಸೃಷ್ಟಿಸ್ಥಿತ್ಯಂತಕಾರಣಂ~।\\
ಪರಹಂಸೈಕಜಿಜ್ಞಾಸ್ಯಃ ಸ್ವಪ್ರಕಾಶಸ್ವರೂಪವಾನ್ ॥೧೦೭॥

	ಮುನಿಮೃಗ್ಯೋ ದೇವಮೃಗ್ಯೋ ಮೃಗಹಸ್ತೋ ಮೃಗೇಶ್ವರಃ~।\\
	ಮೃಗೇಂದ್ರಚರ್ಮವಸನೋ ನರಸಿಂಹನಿಪಾತನಃ ॥೧೦೮॥

ಮುನಿವಂದ್ಯೋ ಮುನಿಶ್ರೇಷ್ಠೋ ಮುನಿಬೃಂದನಿಷೇವಿತಃ~।\\
ದುಷ್ಟಮೃತ್ಯುರದುಷ್ಟೇಹೋ ಮೃತ್ಯುಹಾ ಮೃತ್ಯುಪೂಜಿತಃ ॥೧೦೯॥

	ಅವ್ಯಕ್ತೋಽಮ್ಬುಜಜನ್ಮಾದಿಕೋಟಿಕೋಟಿಸುಪೂಜಿತಃ~।\\
	ಲಿಂಗಮೂರ್ತಿರಲಿಂಗಾತ್ಮಾ ಲಿಂಗಾತ್ಮಾ ಲಿಂಗವಿಗ್ರಹಃ ॥೧೧೦॥

ಯಜುರ್ಮೂರ್ತಿಃ ಸಾಮಮೂರ್ತಿರೃಙ್ಮೂರ್ತಿರ್ಮೂರ್ತಿವರ್ಜಿತಃ~।\\
ವಿಶ್ವೇಶೋ ಗಜಚರ್ಮೈಕಚೇಲಾಂಚಿತಕಟೀತಟಃ ॥೧೧೧॥

	ಪಾವನಾಂತೇವಸದ್ಯೋಗಿಜನಸಾರ್ಥಸುಧಾಕರಃ~।\\
	ಅನಂತಸೋಮಸೂರ್ಯಾಗ್ನಿಮಂಡಲಪ್ರತಿಮಪ್ರಭಃ ॥೧೧೨॥

ಚಿಂತಾಶೋಕಪ್ರಶಮನಃ ಸರ್ವವಿದ್ಯಾವಿಶಾರದಃ~।\\
ಭಕ್ತವಿಜ್ಞಪ್ತಿಸಂಧಾತಾ ಕರ್ತಾ ಗಿರಿವರಾಕೃತಿಃ ॥೧೧೩॥

	ಜ್ಞಾನಪ್ರದೋ ಮನೋವಾಸಃ ಕ್ಷೇಮ್ಯೋ ಮೋಹವಿನಾಶನಃ~।\\
	ಸುರೋತ್ತಮಶ್ಚಿತ್ರಭಾನುಃ ಸದಾವೈಭವತತ್ಪರಃ ॥೧೧೪॥

ಸುಹೃದಗ್ರೇಸರಃ ಸಿದ್ಧಜ್ಞಾನಮುದ್ರೋ ಗಣಾಧಿಪಃ~।\\
ಆಗಮಶ್ಚರ್ಮವಸನೋ ವಾಂಛಿತಾರ್ಥಫಲಪ್ರದಃ ॥೧೧೫॥

	ಅಂತರ್ಹಿತೋಽಸಮಾನಶ್ಚ ದೇವಸಿಂಹಾಸನಾಧಿಪಃ~।\\
	ವಿವಾದಹಂತಾ ಸರ್ವಾತ್ಮಾ ಕಾಲಃ ಕಾಲವಿವರ್ಜಿತಃ ॥೧೧೬॥

ವಿಶ್ವಾತೀತೋ ವಿಶ್ವಕರ್ತಾ ವಿಶ್ವೇಶೋ ವಿಶ್ವಕಾರಣಂ~।\\
ಯೋಗಿಧ್ಯೇಯೋ ಯೋಗನಿಷ್ಠೋ ಯೋಗಾತ್ಮಾ ಯೋಗವಿತ್ತಮಃ ॥೧೧೭॥

	ಓಂಕಾರರೂಪೋ ಭಗವಾನ್ ಬಿಂದುನಾದಮಯಃ ಶಿವಃ~।\\
	ಚತುರ್ಮುಖಾದಿಸಂಸ್ತುತ್ಯಶ್ಚತುರ್ವರ್ಗಫಲಪ್ರದಃ ॥೧೧೮॥

ಸಹ್ಯಾಚಲಗುಹಾವಾಸೀ ಸಾಕ್ಷಾನ್ಮೋಕ್ಷರಸಾಮೃತಃ~।\\
ದಕ್ಷಾಧ್ವರಸಮುಚ್ಛೇತ್ತಾ ಪಕ್ಷಪಾತವಿವರ್ಜಿತಃ ॥೧೧೯॥

	ಓಂಕಾರವಾಚಕಃ ಶಂಭುಃ ಶಂಕರಃ ಶಶಿಶೀತಲಃ~।\\
	ಪಂಕಜಾಸನಸಂಸೇವ್ಯಃ ಕಿಂಕರಾಮರವತ್ಸಲಃ ॥೧೨೦॥

ನತದೌರ್ಭಾಗ್ಯತೂಲಾಗ್ನಿಃ ಕೃತಕೌತುಕಮಂಗಲಃ~।\\
ತ್ರಿಲೋಕಮೋಹನಃ ಶ್ರೀಮತ್ತ್ರಿಪುಂಡ್ರಾಂಕಿತಮಸ್ತಕಃ ॥೧೨೧॥

	ಕ್ರೌಂಚಾರಿಜನಕಃ ಶ್ರೀಮದ್ಗಣನಾಥಸುತಾನ್ವಿತಃ~।\\
	ಅದ್ಭುತಾನಂತವರದೋಽಪರಿಚ್ಛಿನ್ನಾತ್ಮವೈಭವಃ ॥೧೨೨॥

ಇಷ್ಟಾಪೂರ್ತಪ್ರಿಯಃ ಶರ್ವ ಏಕವೀರಃ ಪ್ರಿಯಂವದಃ~।\\
ಊಹಾಪೋಹವಿನಿರ್ಮುಕ್ತ ಓಂಕಾರೇಶ್ವರಪೂಜಿತಃ ॥೧೨೩॥

	ರುದ್ರಾಕ್ಷವಕ್ಷಾ ರುದ್ರಾಕ್ಷರೂಪೋ ರುದ್ರಾಕ್ಷಪಕ್ಷಕಃ~।\\
	ಭುಜಗೇಂದ್ರಲಸತ್ಕಂಠೋ ಭುಜಂಗಾಭರಣಪ್ರಿಯಃ ॥೧೨೪॥

ಕಲ್ಯಾಣರೂಪಃ ಕಲ್ಯಾಣಃ ಕಲ್ಯಾಣಗುಣಸಂಶ್ರಯಃ~।\\
ಸುಂದರಭ್ರೂಃ ಸುನಯನಃ ಸುಲಲಾಟಃ ಸುಕಂಧರಃ ॥೧೨೫॥

	ವಿದ್ವಜ್ಜನಾಶ್ರಯೋ ವಿದ್ವಜ್ಜನಸ್ತವ್ಯಪರಾಕ್ರಮಃ~।\\
	ವಿನೀತವತ್ಸಲೋ ನೀತಿಸ್ವರೂಪೋ ನೀತಿಸಂಶ್ರಯಃ ॥೧೨೬॥

ಅತಿರಾಗೀ ವೀತರಾಗೀ ರಾಗಹೇತುರ್ವಿರಾಗವಿತ್~।\\
ರಾಗಹಾ ರಾಗಶಮನೋ ರಾಗದೋ ರಾಗಿರಾಗವಿತ್ ॥೧೨೭॥

	ಮನೋನ್ಮನೋ ಮನೋರೂಪೋ ಬಲಪ್ರಮಥನೋ ಬಲಃ~।\\
	ವಿದ್ಯಾಕರೋ ಮಹಾವಿದ್ಯೋ ವಿದ್ಯಾವಿದ್ಯಾವಿಶಾರದಃ ॥೧೨೮॥

ವಸಂತಕೃದ್ವಸಂತಾತ್ಮಾ ವಸಂತೇಶೋ ವಸಂತದಃ~।\\
ಪ್ರಾವೃಟ್ಕೃತ್ ಪ್ರಾವೃಡಾಕಾರಃ ಪ್ರಾವೃಟ್ಕಾಲಪ್ರವರ್ತಕಃ ॥೧೨೯॥

	ಶರನ್ನಾಥೋ ಶರತ್ಕಾಲನಾಶಕಃ ಶರದಾಶ್ರಯಃ~।\\
	ಕುಂದಮಂದಾರಪುಷ್ಪೌಘಲಸದ್ವಾಯುನಿಷೇವಿತಃ ॥೧೩೦॥

ದಿವ್ಯದೇಹಪ್ರಭಾಕೂಟಸಂದೀಪಿತದಿಗಂತರಃ~।\\
ದೇವಾಸುರಗುರುಸ್ತವ್ಯೋ ದೇವಾಸುರನಮಸ್ಕೃತಃ ॥೧೩೧॥

	ವಾಮಾಂಗಭಾಗವಿಲಸಚ್ಛ್ಯಾಮಲಾವೀಕ್ಷಣಪ್ರಿಯಃ~।\\
	ಕೀರ್ತ್ಯಾಧಾರಃ ಕೀರ್ತಿಕರಃ ಕೀರ್ತಿಹೇತುರಹೇತುಕಃ ॥೧೩೨॥

ಶರಣಾಗತದೀನಾರ್ತಪರಿತ್ರಾಣಪರಾಯಣಃ~।\\
ಮಹಾಪ್ರೇತಾಸನಾಸೀನೋ ಜಿತಸರ್ವಪಿತಾಮಹಃ ॥೧೩೩॥

	ಮುಕ್ತಾದಾಮಪರೀತಾಂಗೋ ನಾನಾಗಾನವಿಶಾರದಃ~।\\
	ವಿಷ್ಣುಬ್ರಹ್ಮಾದಿವಂದ್ಯಾಂಘ್ರಿರ್ನಾನಾದೇಶೈಕನಾಯಕಃ ॥೧೩೪॥

ಧೀರೋದಾತ್ತೋ ಮಹಾಧೀರೋ ಧೈರ್ಯದೋ ಧೈರ್ಯವರ್ಧಕಃ~।\\
ವಿಜ್ಞಾನಮಯ ಆನಂದಮಯಃ ಪ್ರಾಣಮಯೋಽನ್ನದಃ ॥೧೩೫॥

	ಭವಾಬ್ಧಿತರಣೋಪಾಯಃ ಕವಿರ್ದುಃಸ್ವಪ್ನನಾಶನಃ~।\\
	ಗೌರೀವಿಲಾಸಸದನಃ ಪಿಶಾಚಾನುಚರಾವೃತಃ ॥೧೩೬॥

ದಕ್ಷಿಣಾಪ್ರೇಮಸಂತುಷ್ಟೋ ದಾರಿದ್ರ್ಯವಡವಾನಲಃ~।\\
ಅದ್ಭುತಾನಂತ ಸಂಗ್ರಾಮೋ ಢಕ್ಕಾವಾದನ ತತ್ಪರಃ ॥೧೩೭॥

	ಪ್ರಾಚ್ಯಾತ್ಮಾ ದಕ್ಷಿಣಾಕಾರಃ ಪ್ರತೀಚ್ಯಾತ್ಮೋತ್ತರಾಕೃತಿಃ~।\\
	ಊರ್ಧ್ವಾದ್ಯನ್ಯದಿಗಾಕಾರೋ ಮರ್ಮಜ್ಞಃ ಸರ್ವಶಿಕ್ಷಕಃ ॥೧೩೮॥

ಯುಗಾವಹೋ ಯುಗಾಧೀಶೋ ಯುಗಾತ್ಮಾ ಯುಗನಾಯಕಃ~।\\
ಜಂಗಮಃ ಸ್ಥಾವರಾಕಾರಃ ಕೈಲಾಸಶಿಖರಪ್ರಿಯಃ ॥೧೩೯॥

	ಹಸ್ತರಾಜತ್ಪುಂಡರೀಕಃ ಪುಂಡರೀಕನಿಭೇಕ್ಷಣಃ~।\\
	ಲೀಲಾವಿಡಂಬಿತವಪುರ್ಭಕ್ತಮಾನಸಮಂಡಿತಃ ॥೧೪೦॥

ಬೃಂದಾರಕಪ್ರಿಯತಮೋ ಬೃಂದಾರಕವರಾರ್ಚಿತಃ~।\\
ನಾನಾವಿಧಾನೇಕರತ್ನಲಸತ್ಕುಂಡಲಮಂಡಿತಃ ॥೧೪೧॥

	ನಿಃಸೀಮಮಹಿಮಾ ನಿತ್ಯಲೀಲಾವಿಗ್ರಹರೂಪಧೃತ್~।\\
	ಚಂದನದ್ರವದಿಗ್ಧಾಂಗಶ್ಚಾಂಪೇಯಕುಸುಮಾರ್ಚಿತಃ ॥೧೪೨॥

ಸಮಸ್ತಭಕ್ತಸುಖದಃ ಪರಮಾಣುರ್ಮಹಾಹ್ರದಃ~।\\
ಅಲೌಕಿಕೋ ದುಷ್ಪ್ರಧರ್ಷಃ ಕಪಿಲಃ ಕಾಲಕಂಧರಃ ॥೧೪೩॥

	ಕರ್ಪೂರಗೌರಃ ಕುಶಲಃ ಸತ್ಯಸಂಧೋ ಜಿತೇಂದ್ರಿಯಃ~।\\
	ಶಾಶ್ವತೈಶ್ವರ್ಯವಿಭವಃ ಪೋಷಕಃ ಸುಸಮಾಹಿತಃ ॥೧೪೪॥

ಮಹರ್ಷಿನಾಥಿತೋ ಬ್ರಹ್ಮಯೋನಿಃ ಸರ್ವೋತ್ತಮೋತ್ತಮಃ~।\\
ಭೂಮಿಭಾರಾರ್ತಿಸಂಹರ್ತಾ ಷಡೂರ್ಮಿರಹಿತೋ ಮೃಡಃ ॥೧೪೫॥

	ತ್ರಿವಿಷ್ಟಪೇಶ್ವರಃ ಸರ್ವಹೃದಯಾಂಬುಜಮಧ್ಯಗಃ~।\\
	ಸಹಸ್ರದಲಪದ್ಮಸ್ಥಃ ಸರ್ವವರ್ಣೋಪಶೋಭಿತಃ ॥೧೪೬॥

ಪುಣ್ಯಮೂರ್ತಿಃ ಪುಣ್ಯಲಭ್ಯಃ ಪುಣ್ಯಶ್ರವಣಕೀರ್ತನಃ~।\\
ಸೂರ್ಯಮಂಡಲಮಧ್ಯಸ್ಥಶ್ಚಂದ್ರಮಂಡಲಮಧ್ಯಗಃ ॥೧೪೭॥

	ಸದ್ಭಕ್ತಧ್ಯಾನನಿಗಲಃ ಶರಣಾಗತಪಾಲಕಃ~।\\
	ಶ್ವೇತಾತಪತ್ರರುಚಿರಃ ಶ್ವೇತಚಾಮರವೀಜಿತಃ ॥೧೪೮॥

ಸರ್ವಾವಯವಸಂಪೂರ್ಣಃ ಸರ್ವಲಕ್ಷಣಲಕ್ಷಿತಃ~।\\
ಸರ್ವಮಂಗಲಮಾಂಗಲ್ಯಃ ಸರ್ವಕಾರಣಕಾರಣಃ ॥೧೪೯॥

	ಅಮೋದೋ ಮೋದಜನಕಃ ಸರ್ಪರಾಜೋತ್ತರೀಯಕಃ~।\\
	ಕಪಾಲೀ ಕೋವಿದಃ ಸಿದ್ಧಕಾಂತಿಸಂವಲಿತಾನನಃ ॥೧೫೦॥

ಸರ್ವಸದ್ಗುರುಸಂಸೇವ್ಯೋ ದಿವ್ಯಚಂದನಚರ್ಚಿತಃ~।\\
ವಿಲಾಸಿನೀಕೃತೋಲ್ಲಾಸ ಇಚ್ಛಾಶಕ್ತಿನಿಷೇವಿತಃ ॥೧೫೧॥

	ಅನಂತಾನಂದಸುಖದೋ ನಂದನಃ ಶ್ರೀನಿಕೇತನಃ~।\\
	ಅಮೃತಾಬ್ಧಿಕೃತಾವಾಸೋ ನಿತ್ಯಕ್ಲೀಬೋ ನಿರಾಮಯಃ ॥೧೫೨॥

ಅನಪಾಯೋಽನಂತದೃಷ್ಟಿರಪ್ರಮೇಯೋಽಜರೋಽಮರಃ~।\\
ತಮೋಮೋಹಪ್ರತಿಹತಿರಪ್ರತರ್ಕ್ಯೋಽಮೃತೋಽಕ್ಷರಃ ॥೧೫೩॥

	ಅಮೋಘಬುದ್ಧಿರಾಧಾರ ಆಧಾರಾಧೇಯವರ್ಜಿತಃ~।\\
	ಈಷಣಾತ್ರಯನಿರ್ಮುಕ್ತ ಇಹಾಮುತ್ರವಿವರ್ಜಿತಃ ॥೧೫೪॥

ಋಗ್ಯಜುಃಸಾಮನಯನೋ ಬುದ್ಧಿಸಿದ್ಧಿಸಮೃದ್ಧಿದಃ~।\\
ಔದಾರ್ಯನಿಧಿರಾಪೂರ್ಣ ಐಹಿಕಾಮುಷ್ಮಿಕಪ್ರದಃ ॥೧೫೫॥

	ಶುದ್ಧಸನ್ಮಾತ್ರಸಂವಿದ್ಧೀಸ್ವರೂಪಃ ಸುಖವಿಗ್ರಹಃ~।\\
	ದರ್ಶನಪ್ರಥಮಾಭಾಸೋ ದೃಷ್ಟಿದೃಶ್ಯವಿವರ್ಜಿತಃ ॥೧೫೬॥

ಅಗ್ರಗಣ್ಯೋಽಚಿಂತ್ಯರೂಪಃ ಕಲಿಕಲ್ಮಷನಾಶನಃ~।\\
ವಿಮರ್ಶರೂಪೋ ವಿಮಲೋ ನಿತ್ಯರೂಪೋ ನಿರಾಶ್ರಯಃ ॥೧೫೭॥

ನಿತ್ಯಶುದ್ಧೋ ನಿತ್ಯಬುದ್ಧೋ ನಿತ್ಯಮುಕ್ತೋಽಪರಾಕೃತಃ~।\\
ಮೈತ್ರ್ಯಾದಿವಾಸನಾಲಭ್ಯೋ ಮಹಾಪ್ರಲಯಸಂಸ್ಥಿತಃ ॥೧೫೮॥

ಮಹಾಕೈಲಾಸನಿಲಯಃ ಪ್ರಜ್ಞಾನಘನವಿಗ್ರಹಃ~।\\
ಶ್ರೀಮಾನ್ ವ್ಯಾಘ್ರಪುರಾವಾಸೋ ಭುಕ್ತಿಮುಕ್ತಿಪ್ರದಾಯಕಃ ॥೧೫೯॥

	ಜಗದ್ಯೋನಿರ್ಜಗತ್ಸಾಕ್ಷೀ ಜಗದೀಶೋ ಜಗನ್ಮಯಃ~।\\
	ಜಪೋ ಜಪಪರೋ ಜಪ್ಯೋ ವಿದ್ಯಾಸಿಂಹಾಸನಪ್ರಭುಃ ॥೧೬೦॥

ತತ್ತ್ವಾನಾಂ ಪ್ರಕೃತಿಸ್ತತ್ತ್ವಂ ತತ್ತ್ವಂಪದನಿರೂಪಿತಃ~।\\
ದಿಕ್ಕಾಲಾದ್ಯನವಚ್ಛಿನ್ನಃ ಸಹಜಾನಂದಸಾಗರಃ ॥೧೬೧॥

	ಪ್ರಕೃತಿಃ ಪ್ರಾಕೃತಾತೀತೋ ವಿಜ್ಞಾನೈಕರಸಾಕೃತಿಃ~।\\
	ನಿಃಶಂಕಮತಿದೂರಸ್ಥಶ್ಚೈತ್ಯಚೇತನಚಿಂತನಃ ॥೧೬೨॥

ತಾರಕಾನಾಂ ಹೃದಂತಸ್ಥಸ್ತಾರಕಸ್ತಾರಕಾಂತಕಃ~।\\
ಧ್ಯಾನೈಕಪ್ರಕಟೋ ಧ್ಯೇಯೋ ಧ್ಯಾನೀ ಧ್ಯಾನವಿಭೂಷಣಃ ॥೧೬೩॥

	ಪರಂ ವ್ಯೋಮ ಪರಂ ಧಾಮ ಪರಮಾತ್ಮಾ ಪರಂ ಪದಂ~।\\
	ಪೂರ್ಣಾನಂದಃ ಸದಾನಂದೋ ನಾದಮಧ್ಯಪ್ರತಿಷ್ಠಿತಃ ॥೧೬೪॥

ಪ್ರಮಾವಿಪರ್ಯಯಾತೀತಃ ಪ್ರಣತಾಜ್ಞಾನನಾಶಕಃ~।\\
ಬಾಣಾರ್ಚಿತಾಂಘ್ರಿರ್ಬಹುದೋ ಬಾಲಕೇಲಿಕುತೂಹಲೀ ॥೧೬೫॥

	ಬ್ರಹ್ಮರೂಪೀ ಬ್ರಹ್ಮಪದಂ ಬ್ರಹ್ಮವಿದ್ ಬ್ರಾಹ್ಮಣಪ್ರಿಯಃ~।\\
	ಭ್ರೂಕ್ಷೇಪದತ್ತಲಕ್ಷ್ಮೀಕೋ ಭ್ರೂಮಧ್ಯಧ್ಯಾನಲಕ್ಷಿತಃ ॥೧೬೬॥

ಯಶಸ್ಕರೋ ರತ್ನಗರ್ಭೋ ಮಹಾರಾಜ್ಯಸುಖಪ್ರದಃ~।\\
ಶಬ್ದಬ್ರಹ್ಮ ಶಮಪ್ರಾಪ್ಯೋ ಲಾಭಕೃಲ್ಲೋಕವಿಶ್ರುತಃ ॥೧೬೭॥

	ಶಾಸ್ತಾ ಶಿವಾದ್ರಿನಿಲಯಃ ಶರಣ್ಯೋ ಯಾಜಕಪ್ರಿಯಃ~।\\
	ಸಂಸಾರವೈದ್ಯಃ ಸರ್ವಜ್ಞಃ ಸಭೇಷಜವಿಭೇಷಜಃ ॥೧೬೮॥

ಮನೋವಚೋಭಿರಗ್ರಾಹ್ಯಃ ಪಂಚಕೋಶವಿಲಕ್ಷಣಃ~।\\
ಅವಸ್ಥಾತ್ರಯನಿರ್ಮುಕ್ತಸ್ತ್ವವಸ್ಥಾಸಾಕ್ಷಿತುರ್ಯಕಃ ॥೧೬೯॥

	ಪಂಚಭೂತಾದಿದೂರಸ್ಥಃ ಪ್ರತ್ಯಗೇಕರಸೋಽವ್ಯಯಃ~।\\
	ಷಟ್ಚಕ್ರಾಂತರ್ಗತೋಲ್ಲಾಸೀ ಷಡ್ವಿಕಾರವಿವರ್ಜಿತಃ ॥೧೭೦॥

ವಿಜ್ಞಾನಘನಸಂಪೂರ್ಣೋ ವೀಣಾವಾದನತತ್ಪರಃ~।\\
ನೀಹಾರಾಕಾರಗೌರಾಂಗೋ ಮಹಾಲಾವಣ್ಯವಾರಿಧಿಃ ॥೧೭೧॥

	ಪರಾಭಿಚಾರಶಮನಃ ಷಡಧ್ವೋಪರಿಸಂಸ್ಥಿತಃ~।\\
	ಸುಷುಮ್ನಾಮಾರ್ಗಸಂಚಾರೀ ಬಿಸತಂತುನಿಭಾಕೃತಿಃ ॥೧೭೨॥

ಪಿನಾಕೀ ಲಿಂಗರೂಪಶ್ರೀಃ ಮಂಗಲಾವಯವೋಜ್ಜ್ವಲಃ~।\\
ಕ್ಷೇತ್ರಾಧಿಪಃ ಸುಸಂವೇದ್ಯಃ ಶ್ರೀಪ್ರದೋ ವಿಭವಪ್ರದಃ ॥೧೭೩॥

	ಸರ್ವವಶ್ಯಕರಃ ಸರ್ವದೋಷಹಾ ಪುತ್ರಪೌತ್ರದಃ~।\\
	ತೈಲದೀಪಪ್ರಿಯಸ್ತೈಲಪಕ್ವಾನ್ನಪ್ರೀತಮಾನಸಃ ॥೧೭೪॥

ತೈಲಾಭಿಷೇಕಸಂತುಷ್ಟಸ್ತಿಲಭಕ್ಷಣತತ್ಪರಃ~।\\
ಆಪಾದಕನಿಕಾಮುಕ್ತಾಭೂಷಾಶತಮನೋಹರಃ ॥೧೭೫॥

	ಶಾಣೋಲ್ಲೀಢಮಣಿಶ್ರೇಣೀರಮ್ಯಾಂಘ್ರಿನಖಮಂಡಲಃ~।\\
	ಮಣಿಮಂಜೀರಕಿರಣಕಿಂಜಲ್ಕಿತಪದಾಂಬುಜಃ ॥೧೭೬॥

ಅಪಸ್ಮಾರೋಪರಿನ್ಯಸ್ತಸವ್ಯಪಾದಸರೋರುಹಃ~।\\
ಕಂದರ್ಪತೂಣಾಭಜಂಘೋ ಗುಲ್ಫೋದಂಚಿತನೂಪುರಃ ॥೧೭೭॥

	ಕರಿಹಸ್ತೋಪಮೇಯೋರುರಾದರ್ಶೋಜ್ಜ್ವಲಜಾನುಭೃತ್~।\\
	ವಿಶಂಕಟಕಟಿನ್ಯಸ್ತವಾಚಾಲಮಣಿಮೇಖಲಃ ॥೧೭೮॥

ಆವರ್ತನಾಭಿರೋಮಾಲಿವಲಿಮತ್ಪಲ್ಲವೋದರಃ~।\\
ಮುಕ್ತಾಹಾರಲಸತ್ತುಂಗವಿಪುಲೋರಸ್ಕರಂಜಿತಃ ॥೧೭೯॥

	ವೀರಾಸನಸಮಾಸೀನೋ ವೀಣಾಪುಸ್ತೋಲ್ಲಸತ್ಕರಃ~।\\
	ಅಕ್ಷಮಾಲಾಲಸತ್ಪಾಣಿಶ್ಚಿನ್ಮುದ್ರಿತಕರಾಂಬುಜಃ ॥೧೮೦॥

ಮಾಣಿಕ್ಯಕಂಕಣೋಲ್ಲಾಸಿ ಕರಾಂಬುಜವಿರಾಜಿತಃ~।\\
ಅನರ್ಘ್ಯರತ್ನ ಗ್ರೈವೇಯ ವಿಲಸತ್ಕಂಬುಕಂಧರಃ ॥೧೮೧॥

	ಅನಾಕಲಿತಸಾದೃಶ್ಯಚಿಬುಕಶ್ರೀವಿರಾಜಿತಃ~।\\
	ಮುಗ್ಧಸ್ಮಿತಪರೀಪಾಕಪ್ರಕಾಶಿತರದಾಂಕುರಃ ॥೧೮೨॥

ಚಾರುಚಾಂಪೇಯಪುಷ್ಪಾಭನಾಸಿಕಾಪುಟರಂಜಿತಃ~।\\
ವರವಜ್ರಶಿಲಾದರ್ಶಪರಿಭಾವಿಕಪೋಲಭೂಃ ॥೧೮೩॥

	ಕರ್ಣದ್ವಯೋಲ್ಲಸದ್ದಿವ್ಯಮಣಿಕುಂಡಲಮಂಡಿತಃ~।\\
	ಕರುಣಾಲಹರೀಪೂರ್ಣಕರ್ಣಾಂತಾಯತಲೋಚನಃ ॥೧೮೪॥

ಅರ್ಧಚಂದ್ರಾಭನಿಟಿಲಪಾಟೀರತಿಲಕೋಜ್ಜ್ವಲಃ~।\\
ಚಾರುಚಾಮೀಕರಾಕಾರಜಟಾಚರ್ಚಿತಚಂದನಃ~।\\
ಕೈಲಾಸಶಿಖರಸ್ಫರ್ಧಿಕಮನೀಯನಿಜಾಕೃತಿಃ ॥೧೮೫॥
\authorline{ಇತಿ ಶ್ರೀದಕ್ಷಿಣಾಮೂರ್ತಿಸಹಸ್ರನಾಮಸ್ತೋತ್ರಂ ಸಂಪೂರ್ಣಂ ॥}

\section{ದತ್ತಾತ್ರೇಯಸಹಸ್ರನಾಮಸ್ತೋತ್ರಮ್ ॥}
\addcontentsline{toc}{section}{ದತ್ತಾತ್ರೇಯಸಹಸ್ರನಾಮಸ್ತೋತ್ರಮ್ ॥}
ಶ್ರೀದತ್ತಾತ್ರೇಯಾಯ ಸಚ್ಚಿದಾನಂದಾಯ ಸರ್ವಾಂತರಾತ್ಮನೇ ಸದ್ಗುರವೇ ಪರಬ್ರಹ್ಮಣೇ ನಮಃ ।\\
ಓಂ ಅಸ್ಯ ಶ್ರೀದತ್ತಾತ್ರೇಯಸಹಸ್ರನಾಮಸ್ತೋತ್ರಮಂತ್ರಸ್ಯ ಬ್ರಹ್ಮಾಋಷಿಃ~। ಅನುಷ್ಟುಪ್ಛಂದಃ~। ಶ್ರೀದತ್ತಪುರುಷಃ ಪರಮಾತ್ಮಾ ದೇವತಾ~। ಓಂ ಹಂಸಹಂಸಾಯ ವಿದ್ಮಹೇ ಇತಿ ಬೀಜಂ~। ಸೋಽಹಂ ಸೋಽಹಂ ಚ ಧೀಮಹಿ ಇತಿ ಶಕ್ತಿಃ~। ಹಂಸಃ ಸೋಽಹಂ ಚ ಪ್ರಚೋದಯಾತ್ ಇತಿ ಕೀಲಕಂ~।ಶ್ರೀಪರಮಪುರುಷ ಪರಮಹಂಸ ಪರಮಾತ್ಮ ಪ್ರೀತ್ಯರ್ಥೇ ಜಪೇ ವಿನಿಯೋಗಃ ॥

ಅಥಃ ನ್ಯಾಸಃ ।\\
ಓಂ ಹಂಸಾಂ ಗಣೇಶಾಯ ಅಂಗುಷ್ಠಾಭ್ಯಾಂ ನಮಃ ।\\
ಓಂ ಹಂಸೀ ಪ್ರಜಾಪತಯೇ ತರ್ಜನೀಭ್ಯಾಂ ನಮಃ ।\\
ಓಂ ಹಂಸೂಂ ಮಹಾವಿಷ್ಣವೇ ಮಧ್ಯಮಾಭ್ಯಾಂ ನಮಃ ।\\
ಓಂ ಹಂಸೈಂ ಶಂಭವೇ ಅನಾಮಿಕಾಭ್ಯಾಂ ನಮಃ ।\\
ಓಂ ಹಂಸೌಂ ಜೀವಾತ್ಮನೇ ಕನಿಷ್ಠಿಕಾಮ್ಯಾಂ ನಗಃ ।\\
ಓಂ ಹಂಸಃ ಪರಮಾತ್ಮನೇ ಕರತಲಕರಪೃಷ್ಠಾಭ್ಯಾಂ ನಮಃ ।\\
ಏವಂ ಹೃದಯಾದಿಷಡಂಗನ್ಯಾಸಃ ।\\
ಓಂ ಹಂಸಃ ಸೋಽಹಂ ಹಂಸಃ ಇತಿ ದಿಗ್ಬಂಧಃ ॥
\newpage
\as{ಬಾಲಾರ್ಕಪ್ರಭಮಿಂದ್ರನೀಲಜಟಿಲಂ ಭಸ್ಮಾಂಗರಾಗೋಜ್ಜ್ವಲಂ\\
ಶಾಂತಂ ನಾದವಿಲೀನಚಿತ್ತಪವನಂ ಶಾರ್ದೂಲಚರ್ಮಾಂಬರಂ ।\\
ಬ್ರಹ್ಮಜ್ಞೈಃ ಸನಕಾದಿಭಿಃ ಪರಿವೃತಂ ಸಿದ್ಧೈರ್ಮಹಾಯೋಗಿಭಿಃ\\
ದತ್ತಾತ್ರೇಯಮುಪಾಸ್ಮಹೇ ಹೃದಿ ಮುದಾ ಧ್ಯೇಯಂ ಸದಾ ಯೋಗಿನಾಂ ॥}

ಓಂ ಶ್ರೀಮಾನ್ ದೇವೋ ವಿರೂಪಾಕ್ಷೋ ಪುರಾಣಪುರುಷೋತ್ತಮಃ ।\\
ಬ್ರಹ್ಮಾ ಪರೋ ಯತೀನಾಥೋ ದೀನಬಂಧುಃ ಕೃಪಾನಿಧಿಃ ॥೧॥

ಸಾರಸ್ವತೋ ಮುನಿರ್ಮುಖ್ಯಸ್ತೇಜಸ್ವೀ ಭಕ್ತವತ್ಸಲಃ ।\\
ಧರ್ಮೋ ಧರ್ಮಮಯೋ ಧರ್ಮೀ ಧರ್ಮದೋ ಧರ್ಮಭಾವನಃ ॥೨॥

ಭಾಗ್ಯದೋ ಭೋಗದೋ ಭೋಗೀ ಭಾಗ್ಯವಾನ್ ಭಾನುರಂಜನಃ ।\\
ಭಾಸ್ಕರೋ ಭಯಹಾ ಭರ್ತಾ ಭಾವಭೂರ್ಭವತಾರಣಃ ॥೩॥

ಕೃಷ್ಣೋ ಲಕ್ಷ್ಮೀಪತಿರ್ದೇವಃ ಪಾರಿಜಾತಾಪಹಾರಕಃ ।\\
ಸಿಂಹಾದ್ರಿನಿಲಯಃ ಶಂಭುರ್ವೇಂಕಟಾಚಲವಾಸಕಃ ॥೪॥

ಕೋಲ್ಹಾಪುರಃ ಶ್ರೀಜಪವಾನ್ ಮಾಹುರಾರ್ಜಿತಭಿಕ್ಷುಕಃ ।\\
ಸೇತುತೀರ್ಥವಿಶುದ್ಧಾತ್ಮಾ ರಾಮಧ್ಯಾನಪರಾಯಣಃ ॥೫॥

ರಾಮಾರ್ಚಿತೋ ರಾಮಗುರುಃ ರಾಮಾತ್ಮಾ ರಾಮದೈವತಃ ।\\
ಶ್ರೀರಾಮಶಿಷ್ಯೋ ರಾಮಜ್ಞೋ ರಾಮೈಕಾಕ್ಷರತತ್ಪರಃ ॥೬॥

ಶ್ರೀರಾಮಮಂತ್ರವಿಖ್ಯಾತೋ ರಾಮಮಂತ್ರಾಬ್ಧಿಪಾರಗಃ ।\\
ರಾಮಭಕ್ತೋ ರಾಮಸಖಾ ರಾಮವಾನ್ ರಾಮಹರ್ಷಣಃ ॥೭॥

ಅನಸೂಯಾತ್ಮಜೋ ದೇವದತ್ತಶ್ಚಾತ್ರೇಯನಾಮಕಃ ।\\
ಸುರೂಪಃ ಸುಮತಿಃ ಪ್ರಾಜ್ಞಃ ಶ್ರೀದೋ ವೈಕುಂಠವಲ್ಲಭಃ ॥೮॥

ವಿರಜಸ್ಥಾನಕಃ ಶ್ರೇಷ್ಠಃ ಸರ್ವೋ ನಾರಾಯಣಃ ಪ್ರಭುಃ ।\\
ಕರ್ಮಜ್ಞಃ ಕರ್ಮನಿರತೋ ನೃಸಿಂಹೋ ವಾಮನೋಽಚ್ಯುತಃ ॥೯॥

ಕವಿಃ ಕಾವ್ಯೋ ಜಗನ್ನಾಥೋ ಜಗನ್ಮೂರ್ತಿರನಾಮಯಃ ।\\
ಮತ್ಸ್ಯಃ ಕೂರ್ಮೋ ವರಾಹಶ್ಚ ಹರಿಃ ಕೃಷ್ಣೋ ಮಹಾಸ್ಮಯಃ ॥೧೦॥

ರಾಮೋ ರಾಮೋ ರಘುಪತಿರ್ಬುದ್ಧಃ ಕಲ್ಕೀ ಜನಾರ್ದನಃ ।\\
ಗೋವಿಂದೋ ಮಾಧವೋ ವಿಷ್ಣುಃ ಶ್ರೀಧರೋ ದೇವನಾಯಕಃ ॥೧೧॥

ತ್ರಿವಿಕ್ರಮಃ ಕೇಶವಶ್ಚ ವಾಸುದೇವೋ ಮಹೇಶ್ವರಃ ।\\
ಸಂಕರ್ಷಣಃ ಪದ್ಮನಾಭೋ ದಾಮೋದರಪರಃ ಶುಚಿಃ ॥೧೨॥

ಶ್ರೀಶೈಲವನಚಾರೀ ಚ ಭಾರ್ಗವಸ್ಥಾನಕೋವಿದಃ ।\\
ಶೇಷಾಚಲನಿವಾಸೀ ಚ ಸ್ವಾಮೀ ಪುಷ್ಕರಿಣೀಪ್ರಿಯಃ ॥೧೩॥

ಕುಂಭಕೋಣನಿವಾಸೀ ಚ ಕಾಂಚಿವಾಸೀ ರಸೇಶ್ವರಃ ।\\
ರಸಾನುಭೋಕ್ತಾ ಸಿದ್ಧೇಶಃ ಸಿದ್ಧಿಮಾನ್ ಸಿದ್ಧವತ್ಸಲಃ ॥೧೪॥

ಸಿದ್ಧರೂಪಃ ಸಿದ್ಧವಿಧಿಃ ಸಿದ್ಧಾಚಾರಪ್ರವರ್ತಕಃ ।\\
ರಸಾಹಾರೋ ವಿಷಾಹಾರೋ ಗಂಧಕಾದಿ ಪ್ರಸೇವಕಃ ॥೧೫॥

ಯೋಗೀ ಯೋಗಪರೋ ರಾಜಾ ಧೃತಿಮಾನ್ ಮತಿಮಾನ್ಸುಖೀ ।\\
ಬುದ್ಧಿಮಾನ್ನೀತಿಮಾನ್ ಬಾಲೋ ಹ್ಯುನ್ಮತ್ತೋ ಜ್ಞಾನಸಾಗರಃ ॥೧೬॥

ಯೋಗಿಸ್ತುತೋ ಯೋಗಿಚಂದ್ರೋ ಯೋಗಿವಂದ್ಯೋ ಯತೀಶ್ವರಃ ।\\
ಯೋಗಾದಿಮಾನ್ ಯೋಗರೂಪೋ ಯೋಗೀಶೋ ಯೋಗಿಪೂಜಿತಃ ॥೧೭॥

ಕಾಷ್ಠಾಯೋಗೀ ದೃಢಪ್ರಜ್ಞೋ ಲಂಬಿಕಾಯೋಗವಾನ್ ದೃಢಃ ।\\
ಖೇಚರಶ್ಚ ಖಗಃ ಪೂಷಾ ರಶ್ಮಿವಾನ್ಭೂತಭಾವನಃ ॥೧೮॥

ಬ್ರಹ್ಮಜ್ಞಃ ಸನಕಾದಿಭ್ಯಃ ಶ್ರೀಪತಿಃ ಕಾರ್ಯಸಿದ್ಧಿಮಾನ್ ।\\
ಸ್ಪೃಷ್ಟಾಸ್ಪೃಷ್ಟವಿಹೀನಾತ್ಮಾ ಯೋಗಜ್ಞೋ ಯೋಗಮೂರ್ತಿಮಾನ್ ॥೧೯॥

ಮೋಕ್ಷಶ್ರೀರ್ಮೋಕ್ಷದೋ ಮೋಕ್ಷೀ ಮೋಕ್ಷರೂಪೋ ವಿಶೇಷವಾನ್ ।\\
ಸುಖಪ್ರದಃ ಸುಖಃ ಸೌಖ್ಯಃ ಸುಖರೂಪಃ ಸುಖಾತ್ಮಕಃ ॥೨೦॥

ರಾತ್ರಿರೂಪೋ ದಿವಾರೂಪಃ ಸಂಧ್ಯಾಽಽತ್ಮಾ ಕಾಲರೂಪಕಃ ।\\
ಕಾಲಃ ಕಾಲವಿವರ್ಣಶ್ಚ ಬಾಲಃ ಪ್ರಭುರತುಲ್ಯಕಃ ॥೨೧॥

ಸಹಸ್ರಶೀರ್ಷಾ ಪುರುಷೋ ವೇದಾತ್ಮಾ ವೇದಪಾರಗಃ ।\\
ಸಹಸ್ರಚರಣೋಽನಂತಃ ಸಹಸ್ರಾಕ್ಷೋ ಜಿತೇಂದ್ರಿಯಃ ॥೨೨॥

ಸ್ಥೂಲಸೂಕ್ಷ್ಮೋ ನಿರಾಕಾರೋ ನಿರ್ಮೋಹೋ ಭಕ್ತಮೋಹವಾನ್ ।\\
ಮಹೀಯಾನ್ಪರಮಾಣುಶ್ಚ ಜಿತಕ್ರೋಧೋ ಭಯಾಪಹಃ ॥೨೩॥

ಯೋಗಾನಂದಪ್ರದಾತಾ ಚ ಯೋಗೋ ಯೋಗವಿಶಾರದಃ ।\\
ನಿತ್ಯೋ ನಿತ್ಯಾತ್ಮವಾನ್ ಯೋಗೀ ನಿತ್ಯಪೂರ್ಣೋ ನಿರಾಮಯಃ ॥೨೪॥

ದತ್ತಾತ್ರೇಯೋ ದೇಯದತ್ತೋ ಯೋಗೀ ಪರಮಭಾಸ್ಕರಃ ।\\
ಅವಧೂತಃ ಸರ್ವನಾಥಃ ಸತ್ಕರ್ತಾ ಪುರುಷೋತ್ತಮಃ ॥೨೫॥

ಜ್ಞಾನೀ ಲೋಕವಿಭುಃ ಕಾಂತಃ ಶೀತೋಷ್ಣಸಮಬುದ್ಧಿಕಃ ।\\
ವಿದ್ವೇಷಿ ಜನಸಂಹರ್ತಾ ಧರ್ಮಬುದ್ಧಿವಿಚಕ್ಷಣಃ ॥೨೬॥

ನಿತ್ಯತೃಪ್ತೋ ವಿಶೋಕಶ್ಚ ದ್ವಿಭುಜಃ ಕಾಮರೂಪಕಃ ।\\
ಕಲ್ಯಾಣೋಽಭಿಜನೋ ಧೀರೋ ವಿಶಿಷ್ಟಃ ಸುವಿಚಕ್ಷಣಃ ॥೨೭॥

ಶ್ರೀಮದ್ಭಾಗವತಾರ್ಥಜ್ಞೋ ರಾಮಾಯಣವಿಶೇಷವಾನ್ ।\\
ಅಷ್ಟಾದಶಪುರಾಣಜ್ಞೋ ಷಡ್ದರ್ಶನವಿಜೃಂಭಕಃ ॥೨೮॥

ನಿರ್ವಿಕಲ್ಪಃ ಸುರಶ್ರೇಷ್ಠೋ ಹ್ಯುತ್ತಮೋ ಲೋಕಪೂಜಿತಃ ।\\
ಗುಣಾತೀತಃ ಪೂರ್ಣಗುಣೋ ಬ್ರಹ್ಮಣ್ಯೋ ದ್ವಿಜಸಂವೃತಃ ॥೨೯॥

ದಿಗಂಬರೋ ಮಹಾಜ್ಞೇಯೋ ವಿಶ್ವಾತ್ಮಾಽಽತ್ಮಪರಾಯಣಃ ।\\
ವೇದಾಂತಶ್ರವಣೋ ವೇದೀ ಕಲಾವಾನ್ನಿಷ್ಕಲಂಕವಾನ್ ॥೩೦॥

ಮಿತಭಾಷ್ಯಮಿತಭಾಷೀ ಚ ಸೌಮ್ಯೋ ರಾಮೋ ಜಯಃ ಶಿವಃ ।\\
ಸರ್ವಜಿತ್ ಸರ್ವತೋಭದ್ರೋ ಜಯಕಾಂಕ್ಷೀ ಸುಖಾವಹಃ ॥೩೧॥

ಪ್ರತ್ಯರ್ಥಿಕೀರ್ತಿಸಂಹರ್ತಾ ಮಂದರಾರ್ಚಿತಪಾದುಕಃ ।\\
ವೈಕುಂಠವಾಸೀ ದೇವೇಶೋ ವಿರಜಾಸ್ನಾತಮಾನಸಃ ॥೩೨॥

ಶ್ರೀಮೇರುನಿಲಯೋ ಯೋಗೀ ಬಾಲಾರ್ಕಸಮಕಾಂತಿಮಾನ್ ।\\
ರಕ್ತಾಂಗಃ ಶ್ಯಾಮಲಾಂಗಶ್ಚ ಬಹುವೇಷೋ ಬಹುಪ್ರಿಯಃ ॥೩೩॥

ಮಹಾಲಕ್ಷ್ಮ್ಯನ್ನಪೂರ್ಣೇಶಃ ಸ್ವಧಾಕಾರೋ ಯತೀಶ್ವರಃ ।\\
ಸ್ವರ್ಣರೂಪಃ ಸ್ವರ್ಣದಾಯೀ ಮೂಲಿಕಾಯಂತ್ರಕೋವಿದಃ ॥೩೪॥

ಆನೀತಮೂಲಿಕಾಯಂತ್ರೋ ಭಕ್ತಾಭೀಷ್ಟಪ್ರದೋ ಮಹಾನ್ ।\\
ಶಾಂತಾಕಾರೋ ಮಹಾಮಾಯೋ ಮಾಹುರಸ್ಥೋ ಜಗನ್ಮಯಃ ॥೩೫॥

ಬದ್ಧಾಶನಶ್ಚ ಸೂಕ್ಷ್ಮಾಂಶೀ ಮಿತಾಹಾರೋ ನಿರುದ್ಯಮಃ ।\\
ಧ್ಯಾನಾತ್ಮಾ ಧ್ಯಾನಯೋಗಾತ್ಮಾ ಧ್ಯಾನಸ್ಥೋ ಧ್ಯಾನಸತ್ಪ್ರಿಯಃ ॥೩೬॥

ಸತ್ಯಧ್ಯಾನಃ ಸತ್ಯಮಯಃ ಸತ್ಯರೂಪೋ ನಿಜಾಕೃತಿಃ ।\\
ತ್ರಿಲೋಕಗುರುರೇಕಾತ್ಮಾ ಭಸ್ಮೋದ್ಧೂಲಿತವಿಗ್ರಹಃ ॥೩೭॥

ಪ್ರಿಯಾಪ್ರಿಯಸಮಃ ಪೂರ್ಣೋ ಲಾಭಾಲಾಭಸಮಪ್ರಿಯಃ ।\\
ಸುಖದುಃಖಸಮೋ ಹ್ರೀಮಾನ್ ಹಿತಾಹಿತಸಮಃ ಪರಃ ॥೩೮॥

ಗುರುರ್ಬ್ರಹ್ಮಾ ಚ ವಿಷ್ಣುಶ್ಚ ಮಹಾವಿಷ್ಣುಃ ಸನಾತನಃ ।\\
ಸದಾಶಿವೋ ಮಹೇಂದ್ರಶ್ಚ ಗೋವಿಂದೋ ಮಧುಸೂದನಃ ॥೩೯॥

ಕರ್ತಾ ಕಾರಯಿತಾ ರುದ್ರಃ ಸರ್ವಚಾರೀ ತು ಯಾಚಕಃ ।\\
ಸಂಪತ್ಪ್ರದೋ ವೃಷ್ಟಿರೂಪೋ ಮೇಘರೂಪಸ್ತಪಃಪ್ರಿಯಃ ॥೪೦॥

ತಪೋಮೂರ್ತಿ ಸ್ತಪೋರಾಶಿ ಸ್ತಪಸ್ವೀ ಚ ತಪೋಧನಃ ।\\
ತಪೋಮಯ ಸ್ತಪಃಶುದ್ಧೋ ಜನಕೋ ವಿಶ್ವಸೃಗ್ವಿಧಿಃ ॥೪೧॥

ತಪಃಸಿದ್ಧ ಸ್ತಪಃಸಾಧ್ಯ ಸ್ತಪಃಕರ್ತಾ ತಪಃಕ್ರತುಃ ।\\
ತಪಃಶಮ ಸ್ತಪಃಕೀರ್ತಿ ಸ್ತಪೋದಾರ ಸ್ತಪೋಽತ್ಯಯಃ ॥೪೨॥

ತಪೋರೇತ ಸ್ತಪೋಜ್ಯೋತಿ ಸ್ತಪಾತ್ಮಾ ಚಾತ್ರಿನಂದನಃ ।\\
ನಿಷ್ಕಲ್ಮಷೋ ನಿಷ್ಕಪಟೋ ನಿರ್ವಿಘ್ನೋ ಧರ್ಮಭೀರುಕಃ ॥೪೩॥

ವೈದ್ಯುತಸ್ತಾರಕಃ ಕರ್ಮವೈದಿಕೋ ಬ್ರಾಹ್ಮಣೋ ಯತಿಃ ।\\
ನಕ್ಷತ್ರತೇಜಾ ದೀಪ್ತಾತ್ಮಾ ಪರಿಶುದ್ಧೋ ವಿಮತ್ಸರಃ ॥೪೪॥

ಜಟೀ ಕೃಷ್ಣಾಜಿನಪದೋ ವ್ಯಾಘ್ರಚರ್ಮಧರೋ ವಶೀ ।\\
ಜಿತೇಂದ್ರಿಯಶ್ಚೀರವಾಸಾಃ ಶುಕ್ಲವಸ್ತ್ರಾಂಬರೋ ಹರಿಃ ॥೪೫॥

ಚಂದ್ರಾನುಜ ಶ್ಚಂದ್ರಮುಖಃ ಶುಕಯೋಗೀ ವರಪ್ರದಃ ।\\
ದಿವ್ಯಯೋಗೀ ಪಂಚತಪೋ ಮಾಸರ್ತುವತ್ಸರಾನನಃ ॥೪೬॥

ಭೂತಜ್ಞೋ ವರ್ತಮಾನಜ್ಞೋ ಭಾವಿಜ್ಞೋ ಧರ್ಮವತ್ಸಲಃ ।\\
ಪ್ರಜಾಹಿತಃ ಸರ್ವಹಿತ ಅನಿಂದ್ಯೋ ಲೋಕವಂದಿತಃ ॥೪೭॥

ಆಕುಂಚಯೋಗ ಸಂಬದ್ಧ ಮಲಮೂತ್ರ ರಸಾದಿಕಃ ।\\
ಕನಕೀಭೂತ ಮಲವಾನ್ ರಾಜಯೋಗ ವಿಚಕ್ಷಣಃ ॥೪೮॥

ಶಕಟಾದಿ ವಿಶೇಷಜ್ಞೋ ಲಂಬಿಕಾನೀತಿತತ್ಪರಃ ।\\
ಪ್ರಪಂಚರೂಪೀ ಬಲವಾನ್ ಏಕಕೌಪೀನವಸ್ತ್ರಕಃ ॥೪೯॥

ದಿಗಂಬರಃ ಸೋತ್ತರೀಯಃ ಸಜಟಃ ಸಕಮಂಡಲುಃ ।\\
ನಿರ್ದಂಡಶ್ಚಾಸಿದಂಡಶ್ಚ ಸ್ತ್ರೀವೇಷಃ ಪುರುಷಾಕೃತಿಃ ॥೫೦॥

ತುಲಸೀಕಾಷ್ಠಮಾಲೀ ಚ ರೌದ್ರಃ ಸ್ಫಟಿಕಮಾಲಿಕಃ ।\\
ನಿರ್ಮಾಲಿಕಃ ಶುದ್ಧತರಃ ಸ್ವೇಚ್ಛಾ ಅಮರವಾನ್ ಪರಃ ॥೫೧॥

ಉರ್ಧ್ವಪುಂಡ್ರಸ್ತ್ರಿಪುಂಡ್ರಾಂಕೋ ದ್ವಂದ್ವಹೀನಃ ಸುನಿರ್ಮಲಃ ।\\
ನಿರ್ಜಟಃ ಸುಜಟೋ ಹೇಯೋ ಭಸ್ಮಶಾಯೀ ಸುಭೋಗವಾನ್ ॥೫೨॥

ಮೂತ್ರಸ್ಪರ್ಶೋ ಮಲಸ್ಪರ್ಶೋ ಜಾತಿಹೀನಃ ಸುಜಾತಿಕಃ ।\\
ಅಭಕ್ಷ್ಯಭಕ್ಷೋ ನಿರ್ಭಕ್ಷೋ ಜಗದ್ವಂದಿತ ದೇಹವಾನ್ ॥೫೩॥

ಭೂಷಣೋ ದೂಷಣಸಮಃ ಕಾಲಾಕಾಲೋ ದಯಾನಿಧಿಃ ।\\
ಬಾಲಪ್ರಿಯೋ ಬಾಲರುಚಿರ್ಬಾಲವಾನತಿಬಾಲಕಃ ॥೫೪॥

ಬಾಲಕ್ರೀಡೋ ಬಾಲರತೋ ಬಾಲಸಂಘವೃತೋ ಬಲೀ ।\\
ಬಾಲಲೀಲಾವಿನೋದಶ್ಚ ಕರ್ಣಾಕರ್ಷಣಕಾರಕಃ ॥೫೫॥

ಕ್ರಯಾನೀತವಣಿಕ್ಪಣ್ಯೋ ಗುಡಸೂಪಾದಿಭಕ್ಷಕಃ ।\\
ಬಾಲವದ್ಗೀತಹೃಷ್ಟಶ್ಚ ಮುಷ್ಟಿಯುದ್ಧಕರಶ್ಚಲಃ ॥೫೬॥

ಅದೃಶ್ಯೋ ದೃಶ್ಯಮಾನಶ್ಚ ದ್ವಂದ್ವಯುದ್ಧಪ್ರವರ್ತಕಃ ।\\
ಪಲಾಯಮಾನೋ ಬಾಲಾಢ್ಯೋ ಬಾಲಹಾಸಃ ಸುಸಂಗತಃ ॥೫೭॥

ಪ್ರತ್ಯಾಗತಃ ಪುನರ್ಗಚ್ಛಚ್ಚಕ್ರವದ್ಗಮನಾಕುಲಃ ।\\
ಚೋರವದ್ಧೃತಸರ್ವಸ್ವೋ ಜನತಾಽಽರ್ತಿಕದೇಹವಾನ್ ॥೫೮॥

ಪ್ರಹಸನ್ಪ್ರವದಂದತ್ತೋ ದಿವ್ಯಮಂಗಲವಿಗ್ರಹಃ ।\\
ಮಾಯಾಬಾಲಶ್ಚ ಮಾಯಾವೀ ಪೂರ್ಣಲೀಲೋ ಮುನೀಶ್ವರಃ ॥೫೯॥

ಮಾಹುರೇಶೋ ವಿಶುದ್ಧಾತ್ಮಾ ಯಶಸ್ವೀ ಕೀರ್ತಿಮಾನ್ ಯುವಾ ।\\
ಸವಿಕಲ್ಪಃ ಸಚ್ಚಿದಾಭೋ ಗುಣವಾನ್ ಸೌಮ್ಯಭಾವನಃ ॥೬೦॥

ಪಿನಾಕೀ ಶಶಿಮೌಲೀ ಚ ವಾಸುದೇವೋ ದಿವಸ್ಪತಿಃ ।\\
ಸುಶಿರಾಃ ಸೂರ್ಯತೇಜಶ್ಚ ಶ್ರೀಗಂಭೀರೋಷ್ಠ ಉನ್ನತಿಃ ॥೬೧॥

ದಶಪದ್ಮಾ ತ್ರಿಶೀರ್ಷಶ್ಚ ತ್ರಿಭಿರ್ವ್ಯಾಪ್ತೋ ದ್ವಿಶುಕ್ಲವಾನ್ ।\\
ತ್ರಿಸಮಶ್ಚ ತ್ರಿತಾತ್ಮಾಚ ತ್ರಿಲೋಕಶ್ಚ ತ್ರಯಂಬಕಃ ॥೬೨॥

ಚತುರ್ದ್ವಂದ್ವಸ್ತ್ರಿಯವನಸ್ತ್ರಿಕಾಮೋ ಹಂಸವಾಹನಃ ।\\
ಚತುಷ್ಕಲಶ್ಚತುರ್ದಂಷ್ಟ್ರೋ ಗತಿಃ ಶಂಭುಃ ಪ್ರಿಯಾನನಃ ॥೬೩॥

ಚತುರ್ಮತಿರ್ಮಹಾದಂಷ್ಟ್ರೋ ವೇದಾಂಗೀ ಚತುರಾನನಃ ।\\
ಪಂಚಶುದ್ಧೋ ಮಹಾಯೋಗೀ ಮಹಾದ್ವಾದಶವಾನಕಃ ॥೬೪॥

ಚತುರ್ಮುಖೋ ನರತನುರಜೇಯಶ್ಚಾಷ್ಟವಂಶವಾನ್ ।\\
ಚತುರ್ದಶಸಮದ್ವಂದ್ವೋ ಮುಕುರಾಂಕೋ ದಶಾಂಶವಾನ್ ॥೬೫॥

ವೃಷಾಂಕೋ ವೃಷಭಾರೂಢಶ್ಚಂದ್ರತೇಜಾಃ ಸುದರ್ಶನಃ ।\\
ಸಾಮಪ್ರಿಯೋ ಮಹೇಶಾನಶ್ಚಿದಾಕಾರೋಃ ನರೋತ್ತಮಃ ॥೬೬॥

ದಯಾವಾನ್ ಕರುಣಾಪೂರ್ಣೋ ಮಹೇಂದ್ರೋ ಮಾಹುರೇಶ್ವರಃ ।\\
ವೀರಾಸನಸಮಾಸೀನೋ ರಾಮೋ ರಾಮಪರಾಯಣಃ ॥೬೭॥

ಇಂದ್ರೋ ವಹ್ನಿರ್ಯಮಃ ಕಾಲೋ ನಿರೃತಿರ್ವರುಣೋ ಯಮಃ ।\\
ವಾಯುಶ್ಚ ರುದ್ರಶ್ಚೇಶಾನೋ ಲೋಕಪಾಲೋ ಮಹಾಯಶಾಃ ॥೬೮॥

ಯಕ್ಷಗಂಧರ್ವನಾಗಶ್ಚ ಕಿನ್ನರಃ ಶುದ್ಧರೂಪಕಃ ।\\
ವಿದ್ಯಾಧರಶ್ಚಾಹಿಪತಿಶ್ಚಾರಣಃ ಪನ್ನಗೇಶ್ವರಃ ॥೬೯॥

ಚಂಡಿಕೇಶಃ ಪ್ರಚಂಡಶ್ಚ ಘಂಟಾನಾದರತಃ ಪ್ರಿಯಃ ।\\
ವೀಣಾಧ್ವನಿರ್ವೈನತೇಯೋ ನಾರದಸ್ತುಂಬರುರ್ಹರಃ ॥೭೦॥

ವೀಣಾಪ್ರಚಂಡಸೌಂದರ್ಯೋ ರಾಜೀವಾಕ್ಷಶ್ಚ ಮನ್ಮಥಃ ।\\
ಚಂದ್ರೋ ದಿವಾಕರೋ ಗೋಪಃ ಕೇಸರೀ ಸೋಮಸೋದರಃ ॥೭೧॥

ಸನಕಃ ಶುಕಯೋಗೀ ಚ ನಂದೀ ಷಣ್ಮುಖರಾಗಕಃ ।\\
ಗಣೇಶೋ ವಿಘ್ನರಾಜಶ್ಚ ಚಂದ್ರಾಭೋ ವಿಜಯೋ ಜಯಃ ॥೭೨॥

ಅತೀತಕಾಲಚಕ್ರಶ್ಚ ತಾಮಸಃ ಕಾಲದಂಡವಾನ್ ।\\
ವಿಷ್ಣುಚಕ್ರಃ ತ್ರಿಶೂಲೇಂದ್ರೋ ಬ್ರಹ್ಮದಂಡೋ ವಿರುದ್ಧಕಃ ॥೭೩॥

ಬ್ರಹ್ಮಾಸ್ತ್ರರೂಪಃ ಸತ್ಯೇಂದ್ರಃ ಕೀರ್ತಿಮಾನ್ಗೋಪತಿರ್ಭವಃ ।\\
ವಸಿಷ್ಠೋ ವಾಮದೇವಶ್ಚ ಜಾಬಾಲೀ ಕಣ್ವರೂಪಕಃ ॥೭೪॥

ಸಂವರ್ತರೂಪೋ ಮೌದ್ಗಲ್ಯೋ ಮಾರ್ಕಂಡೇಯಶ್ಚ ಕಶ್ಯಪಃ ।\\
ತ್ರಿಜಟೋ ಗಾರ್ಗ್ಯರೂಪೀ ಚ ವಿಷನಾಥೋ ಮಹೋದಯಃ ॥೭೫॥

ತ್ವಷ್ಟಾ ನಿಶಾಕರಃ ಕರ್ಮಕಾಶ್ಯಪಶ್ಚ ತ್ರಿರೂಪವಾನ್ ।\\
ಜಮದಗ್ನಿಃ ಸರ್ವರೂಪಃ ಸರ್ವನಾದೋ ಯತೀಶ್ವರಃ ॥೭೬॥

ಅಶ್ವರೂಪೀ ವೈದ್ಯಪತಿರ್ಗರಕಂಠೋಽಮ್ಬಿಕಾರ್ಚಿತಃ ।\\
ಚಿಂತಾಮಣಿಃ ಕಲ್ಪವೃಕ್ಷೋ ರತ್ನಾದ್ರಿರುದಧಿಪ್ರಿಯಃ ॥೭೭॥

ಮಹಾಮಂಡೂಕರೂಪೀ ಚ ಕಾಲಾಗ್ನಿಸಮವಿಗ್ರಹಃ ।\\
ಆಧಾರಶಕ್ತಿರೂಪೀ ಚ ಕೂರ್ಮಃ ಪಂಚಾಗ್ನಿರೂಪಕಃ ॥೭೮॥

ಕ್ಷೀರಾರ್ಣವೋ ಮಹಾರೂಪೀ ವರಾಹಶ್ಚ ಧೃತಾವನಿಃ ।\\
ಐರಾವತೋ ಜನಃ ಪದ್ಮೋ ವಾಮನಃ ಕುಮುದಾತ್ಮವಾನ್ ॥೭೯॥

ಪುಂಡರೀಕಃ ಪುಷ್ಪದಂತೋ ಮೇಘಚ್ಛನ್ನೋಽಭ್ರಚಾರಕಃ ।\\
ಸಿತೋತ್ಪಲಾಭೋ ದ್ಯುತಿಮಾನ್ ದೃಢೋರಸ್ಕಃ ಸುರಾರ್ಚಿತಃ ॥೮೦॥

ಪದ್ಮನಾಭಃ ಸುನಾಭಶ್ಚ ದಶಶೀರ್ಷಃ ಶತೋದರಃ ।\\
ಅವಾಙ್ಮುಖೋ ಪಂಚವಕ್ತ್ರೋ ರಕ್ಷಾಖ್ಯಾತ್ಮಾ ದ್ವಿರೂಪಕಃ ॥೮೧॥

ಸ್ವರ್ಣಮಂಡಲಸಂಚಾರೀ ವೇದಿಸ್ಥಃ ಸರ್ವಪೂಜಿತಃ ।\\
ಸ್ವಪ್ರಸನ್ನಃ ಪ್ರಸನ್ನಾತ್ಮಾ ಸ್ವಭಕ್ತಾಭಿಮುಖೋ ಮೃದುಃ ॥೮೨॥

ಆವಾಹಿತಃ ಸನ್ನಿಹಿತೋ ವರದೋ ಜ್ಞಾನಿವತ್ಸ್ಥಿತಃ ।\\
ಶಾಲಿಗ್ರಾಮಾತ್ಮಕೋ ಧ್ಯಾತೋ ರತ್ನಸಿಂಹಾಸನಸ್ಥಿತಃ ॥೮೩॥

ಅರ್ಘ್ಯಪ್ರಿಯಃ ಪಾದ್ಯತುಷ್ಟಶ್ಚಾಚಮ್ಯಾರ್ಚಿತಪಾದುಕಃ ।\\
ಪಂಚಾಮೃತಃ ಸ್ನಾನವಿಧಿಃ ಶುದ್ಧೋದಕಸುಸಂಚಿತಃ ॥೮೪॥

ಗಂಧಾಕ್ಷತಸುಸಂಪ್ರೀತಃ ಪುಷ್ಪಾಲಂಕಾರಭೂಷಣಃ ।\\
ಅಂಗಪೂಜಾಪ್ರಿಯಃ ಸರ್ವೋ ಮಹಾಕೀರ್ತಿರ್ಮಹಾಭುಜಃ ॥೮೫॥

ನಾಮಪೂಜಾವಿಶೇಷಜ್ಞಃ ಸರ್ವನಾಮಸ್ವರೂಪಕಃ ।\\
ಧೂಪಿತೋ ದಿವ್ಯಧೂಪಾತ್ಮಾ ದೀಪಿತೋ ಬಹುದೀಪವಾನ್ ॥೮೬॥

ಬಹುನೈವೇದ್ಯಸಂಹೃಷ್ಟೋ ನಿರಾಜನವಿರಾಜಿತಃ ।\\
ಸರ್ವಾತಿರಂಜಿತಾನಂದಃ ಸೌಖ್ಯವಾನ್ ಧವಲಾರ್ಜುನಃ ॥೮೭॥

ವಿರಾಗೋ ನಿರ್ವಿರಾಗಶ್ಚ ಯಜ್ಞಾರ್ಚಾಂಗೋ ವಿಭೂತಿಕಃ ।\\
ಉನ್ಮತ್ತೋ ಭ್ರಾಂತಚಿತ್ತಶ್ಚ ಶುಭಚಿತ್ತಃ ಶುಭಾಹುತಿಃ ॥೮೮॥

ಸುರೈರಿಷ್ಟೋ ಲಘಿಷ್ಠಶ್ಚ ಬಂಹಿಷ್ಠೋ ಬಹುದಾಯಕಃ ।\\
ಮಹಿಷ್ಠಃ ಸುಮಹೌಜಾಶ್ಚ ಬಲಿಷ್ಠಃ ಸುಪ್ರತಿಷ್ಠಿತಃ ॥೮೯॥

ಕಾಶೀಗಂಗಾಂಬುಮಜ್ಜಶ್ಚ ಕುಲಶ್ರೀಮಂತ್ರಜಾಪಕಃ ।\\
ಚಿಕುರಾನ್ವಿತಭಾಲಶ್ಚ ಸರ್ವಾಂಗಾಲಿಪ್ತಭೂತಿಕಃ ॥೯೦॥

ಅನಾದಿನಿಧನೋ ಜ್ಯೋತಿರ್ಭಾರ್ಗವಾದ್ಯಃ ಸನಾತನಃ ।\\
ತಾಪತ್ರಯೋಪಶಮನೋ ಮಾನವಾಸೋ ಮಹೋದಯಃ ॥೯೧॥

ಜ್ಯೇಷ್ಠಃ ಶ್ರೇಷ್ಠೋ ಮಹಾರೌದ್ರಃ ಕಾಲಮೂರ್ತಿಃ ಸುನಿಶ್ಚಯಃ ।\\
ಊರ್ಧ್ವಃ ಸಮೂರ್ಧ್ವಲಿಂಗಶ್ಚ ಹಿರಣ್ಯೋ ಹೇಮಲಿಂಗವಾನ್ ॥೯೨॥

ಸುವರ್ಣಃ ಸ್ವರ್ಣಲಿಂಗಶ್ಚ ದಿವ್ಯಸೂತಿರ್ದಿವಸ್ಪತಿಃ ।\\
ದಿವ್ಯಲಿಂಗೋ ಭವೋ ಭವ್ಯಃ ಸರ್ವಲಿಂಗಸ್ತು ಸರ್ವಕಃ ॥೯೩॥

ಶಿವಲಿಂಗಃ ಶಿವೋ ಮಾಯೋ ಜ್ವಲಸ್ತೂಜ್ಜ್ವಲಲಿಂಗವಾನ್ ।\\
ಆತ್ಮಾ ಚೈವಾತ್ಮಲಿಂಗಶ್ಚ ಪರಮೋ ಲಿಂಗಪಾರಗಃ ॥೯೪॥

ಸೋಮಃ ಸೂರ್ಯಃ ಸರ್ವಲಿಂಗಃ ಪಾಣಿಯಂತ್ರಪವಿತ್ರವಾನ್ ।\\
ಸದ್ಯೋಜಾತೋ ತಪೋರೂಪೋ ಭವೋದ್ಭವ ಅನೀಶ್ವರಃ ॥೯೫॥

ತತ್ಸವಿದ್ರೂಪಸವಿತಾ ವರೇಣ್ಯಶ್ಚ ಪ್ರಚೋದಯಾತ್ ।\\
ದೂರದೃಷ್ಟಿರ್ದೂರಗತೋ ದೂರಶ್ರವಣತರ್ಪಿತಃ ॥೯೬॥

ಯೋಗಪೀಠಸ್ಥಿತೋ ವಿದ್ವಾನ್ ನಮಸ್ಕಾರಿತರಾಸಭಃ ।\\
ನಮಸ್ಕೃತಶುನಶ್ಚಾಪಿ ವಜ್ರಕಷ್ಟ್ಯಾತಿಭೀಷಣಃ ॥೯೭॥

ಜ್ವಲನ್ಮುಖಃ ಪ್ರತಿವೀಣಾ ಸಖಡ್ಗೋ ದ್ರಾವಿತಪ್ರಜಃ ।\\
ಪಶುಘ್ನಶ್ಚ ರಸೋನ್ಮತ್ತೋ ರಸೋರ್ಧ್ವಮುಖರಂಜಿತಃ ॥೯೮॥

ರಸಪ್ರಿಯೋ ರಸಾತ್ಮಾ ಚ ರಸರೂಪೀ ರಸೇಶ್ವರಃ ।\\
ರಸಾಧಿದೈವತೋ ಭೌಮೋ ರಸಾಂಗೋ ರಸಭಾವನಃ ॥೯೯॥

ರಸೋನ್ಮಯೋ ರಸಕರೋ ರಸೇಂದ್ರೋ ರಸಪೂಜಕಃ ।\\
ರಸಸಿದ್ಧಃ ಸಿದ್ಧರಸೋ ರಸದ್ರವ್ಯೋ ರಸೋನ್ಮುಖಃ ॥೧೦೦॥

ರಸಾಂಕಿತೋ ರಸಾಪೂರ್ಣೋ ರಸದೋ ರಸಿಕೋ ರಸೀ ।\\
ಗಂಧಕಾದಸ್ತಾಲಕಾದೋ ಗೌರಃಸ್ಫಟಿಕಸೇವನಃ ॥೧೦೧॥

ಕಾರ್ಯಸಿದ್ಧಃ ಕಾರ್ಯರುಚಿರ್ಬಹುಕಾರ್ಯೋ ನ ಕಾರ್ಯವಾನ್ ।\\
ಅಭೇದೀ ಜನಕರ್ತಾ ಚ ಶಂಖಚಕ್ರಗದಾಧರಃ ॥೧೦೨॥

ಕೃಷ್ಣಾಜಿನಕಿರೀಟೀ ಚ ಶ್ರೀಕೃಷ್ಣಾಜಿನಕಂಚುಕಃ ।\\
ಮೃಗಯಾಯೀ ಮೃಗೇಂದ್ರಶ್ಚ ಗಜರೂಪೀ ಗಜೇಶ್ವರಃ ॥೧೦೩॥

ದೃಢವ್ರತಃ ಸತ್ಯವಾದೀ ಕೃತಜ್ಞೋ ಬಲವಾನ್ಬಲಃ ।\\
ಗುಣವಾನ್ ಕಾರ್ಯವಾನ್ ದಾಂತಃ ಕೃತಶೋಭೋ ದುರಾಸದಃ ॥೧೦೪॥

ಸುಕಾಲೋ ಭೂತನಿಹಿತಃ ಸಮರ್ಥಶ್ಚಾಂಡನಾಯಕಃ ।\\
ಸಂಪೂರ್ಣದೃಷ್ಟಿರಕ್ಷುಬ್ಧೋ ಜನೈಕಪ್ರಿಯದರ್ಶನಃ ॥೧೦೫॥

ನಿಯತಾತ್ಮಾ ಪದ್ಮಧರೋ ಬ್ರಹ್ಮವಾಂಶ್ಚಾನಸೂಯಕಃ ।\\
ಉಂಚ್ಛವೃತಿರನೀಶಶ್ಚ ರಾಜಭೋಗೀ ಸುಮಾಲಿಕಃ ॥೧೦೬॥

ಸುಕುಮಾರೋ ಜರಾಹೀನೇ ಚೋರಘ್ನೋ ಮಂಜುಲಕ್ಷಣಃ ।\\
ಸುಪದಃ ಸ್ವಂಗುಲೀಕಶ್ಚ ಸುಜಂಘಃ ಶುಭಜಾನುಕಃ ॥೧೦೭॥

ಶುಭೋರುಃ ಶುಭಲಿಂಗಶ್ಚ ಸುನಾಭೋ ಜಘನೋತ್ತಮಃ ।\\
ಸುಪಾರ್ಶ್ವಃ ಸುಸ್ತನೋ ನೀಲಃ ಸುವಕ್ಷಾಶ್ಚ ಸುಜತ್ರುಕಃ ॥೧೦೮॥

ನೀಲಗ್ರೀವೋ ಮಹಾಸ್ಕಂಧಃ ಸುಭುಜೋ ದಿವ್ಯಜಂಘಕಃ ।\\
ಸುಹಸ್ತರೇಖೋ ಲಕ್ಷ್ಮೀವಾನ್ ದೀರ್ಘಪೃಷ್ಠೋ ಯತಿಶ್ಚಲಃ ॥೧೦೯॥

ಬಿಂಬೋಷ್ಠಃ ಶುಭದಂತಶ್ಚ ವಿದ್ಯುಜ್ಜಿಹ್ವಃ ಸುತಾಲುಕಃ ।\\
ದೀರ್ಘನಾಸಃ ಸುತಾಮ್ರಾಕ್ಷಃ ಸುಕಪೋಲಃ ಸುಕರ್ಣಕಃ ॥೧೧೦॥

ನಿಮೀಲಿತೋನ್ಮೀಲಿತಶ್ಚ ವಿಶಾಲಾಕ್ಷಶ್ಚ ಶುಭ್ರಕಃ ।\\
ಶುಭಮಧ್ಯಃ ಸುಭಾಲಶ್ಚ ಸುಶಿರಾ ನೀಲರೋಮಕಃ ॥೧೧೧॥

ವಿಶಿಷ್ಟಗ್ರಾಮಣಿಸ್ಕಂಧಃ ಶಿಖಿವರ್ಣೋ ವಿಭಾವಸುಃ ।\\
ಕೈಲಾಸೇಶೋ ವಿಚಿತ್ರಜ್ಞೋ ವೈಕುಂಠೇಂದ್ರೋ ವಿಚಿತ್ರವಾನ್ ॥೧೧೨॥

ಮನಸೇಂದ್ರಶ್ಚಕ್ರವಾಲೋ ಮಹೇಂದ್ರೋ ಮಂದಾರಧಿಪಃ ।\\
ಮಲಯೋ ವಿಂಧ್ಯರೂಪಶ್ಚ ಹಿಮವಾನ್ ಮೇರುರೂಪಕಃ ॥೧೧೩॥

ಸುವೇಷೋ ನವ್ಯರೂಪಾತ್ಮಾ ಮೈನಾಕೋ ಗಂಧಮಾದನಃ ।\\
ಸಿಂಹಲಶ್ಚೈವ ವೇದಾದ್ರಿಃ ಶ್ರೀಶೈಲಃ ಕ್ರಕಚಾತ್ಮಕಃ ॥೧೧೪॥

ನಾನಾಚಲಶ್ಚಿತ್ರಕೂಟೋ ದುರ್ವಾಸಾಃ ಪರ್ವತಾತ್ಮಜಃ ।\\
ಯಮುನಾಕೃಷ್ಣವೇಣೀಶೋ ಭದ್ರೇಶೋ ಗೌತಮೀಪತಿಃ ॥೧೧೫॥

ಗೋದಾವರೀಶೋ ಗಂಗಾತ್ಮಾ ಶೋಣಕಃ ಕೌಶಿಕೀಪತಿಃ ।\\
ನರ್ಮದೇಶಸ್ತು ಕಾವೇರೀತಾಮ್ರಪರ್ಣೀಶ್ವರೋ ಜಟೀ ॥೧೧೬॥

ಸರಿದ್ರೂಪಾ ನದಾತ್ಮಾ ಚ ಸಮುದ್ರಃ ಸರಿದೀಶ್ವರಃ ।\\
ಹ್ರಾದಿನೀಶಃ ಪಾವನೀಶೋ ನಲಿನೀಶಃ ಸುಚಕ್ಷುಮಾನ್ ॥೧೧೭॥

ಸೀತಾನದೀಪತಿಃ ಸಿಂಧೂ ರೇವೇಶೋ ಮುರಲೀಪತಿಃ ।\\
ಲವಣೇಕ್ಷುಃ ಕ್ಷೀರನಿಧಿಃ ಸುರಾಬ್ಧಿಃ ಸರ್ಪಿರಂಬುಧಿಃ ॥೧೧೮॥

ದಯಾಬ್ಧಿಶುದ್ಧಜಲಧಿಸ್ತತ್ವರೂಪೋ ಧನಾಧಿಪಃ ।\\
ಭೂಪಾಲಮಧುರಾಗಜ್ಞೋ ಮಾಲತೀರಾಗಕೋವಿದಃ ॥೧೧೯॥

ಪೌಂಡ್ರಕ್ರಿಯಾಜ್ಞಃ ಶ್ರೀರಾಗೋ ನಾನಾರಾಗಾರ್ಣವಾಂತಕಃ ।\\
ವೇದಾದಿರೂಪೋ ಹ್ರೀರೂಪೋ ಕ್ಲಂರೂಪಃ ಕ್ಲೀಂವಿಕಾರಕಃ ॥೧೨೦॥

ವ್ರುಮ್ಮಯಃ ಕ್ಲೀಮ್ಮಯಃ ಪ್ರಖ್ಯೋ ಹುಮ್ಮಯಃ ಕ್ರೋಮ್ಮಯೋ ಭಟಃ ।\\
ಧ್ರೀಮ್ಮಯೋ ಲುಂಗ್ಮಯೋ ಗಾಂಗೋ ಘಮ್ಮಯಃ ಖಮ್ಮಯಃ ಖಗಃ ॥೧೨೧॥

ಖಮ್ಮಯೋ ಜ್ಞಮ್ಮಯಶ್ಚಾಂಗೋ ಬೀಜಾಂಗೋ ಬೀಜಜಮ್ಮಯಃ ।\\
ಝಂಕರಷ್ಟಂಕರಃ ಷ್ಟಂಗೋ ಡಂಕರೀ ಠಂಕರೋಽಣುಕಃ ॥೧೨೨॥

ತಂಕರಸ್ಥಂಕರಸ್ತುಂಗೋ ದ್ರಾಮ್ಮುದ್ರಾರೂಪಕಃ ಸುದಃ ।\\
ದಕ್ಷೋ ದಂಡೀ ದಾನವಘ್ನೋ ಅಪ್ರತಿದ್ವಂದ್ವವಾಮದಃ ॥೧೨೩॥

ಧಂರೂಪೋ ನಂಸ್ವರೂಪಶ್ಚ ಪಂಕಜಾಕ್ಷಶ್ಚ ಫಮ್ಮಯಃ ।\\
ಮಹೇಂದ್ರೋ ಮಧುಭೋಕ್ತಾ ಚ ಮಂದರೇತಾಸ್ತು ಭಮ್ಮಯಃ ॥೧೨೪॥

ರಮ್ಮಯೋ ರಿಂಕರೋ ರಂಗೋ ಲಂಕರಃ ವಮ್ಮಯಃ ಶರಃ ।\\
ಶಂಕರಃಷಣ್ಮುಖೋ ಹಂಸಃ ಶಂಕರಃ ಶಂಕರೋಽಕ್ಷಯಃ ॥೧೨೫॥

ಓಮಿತ್ಯೇಕಾಕ್ಷರಾತ್ಮಾ ಚ ಸರ್ವಬೀಜಸ್ವರೂಪಕಃ ।\\
ಶ್ರೀಕರಃ ಶ್ರೀಪದಃ ಶ್ರೀಶಃ ಶ್ರೀನಿಧಿಃ ಶ್ರೀನಿಕೇತನಃ ॥೧೨೬॥

ಪುರುಷೋತ್ತಮಃ ಸುಖೀ ಯೋಗೀ ದತ್ತಾತ್ರೇಯೋ ಹೃದಿಪ್ರಿಯಃ ।\\
ತತ್ಸಂಯುತಃ ಸದಾಯೋಗೀ ಧೀರತಂತ್ರಸುಸಾಧಕಃ ॥೧೨೭॥

ಪುರುಷೋತ್ತಮೋ ಯತಿಶ್ರೇಷ್ಠೋ ದತ್ತಾತ್ರೇಯಃ ಸಖೀತ್ವವಾನ್ ।\\
ವಸಿಷ್ಠವಾಮದೇವಾಭ್ಯಾಂ ದತ್ತಃ ಪುರುಷಃ ಈರಿತಃ ॥೧೨೮॥

ಯಾವತ್ತಿಷ್ಠತೇ ಹ್ಯಸ್ಮಿನ್ ತಾವತ್ತಿಷ್ಠತಿ ತತ್ಸುಖೀ ।\\
ಯ ಇದಂ ಶೃಣುಯಾನ್ನಿತ್ಯಂ ಬ್ರಹ್ಮಸಾಯುಜ್ಯತಾಂ ವ್ರಜೇತ್ ॥೧೨೯॥

ಭುಕ್ತಿಮುಕ್ತಿಕರಂ ತಸ್ಯ ನಾತ್ರಕಾರ್ಯಾ ವಿಚಾರಣಾ ।\\
ಆಯುಷ್ಮತ್ಪುತ್ರಪೌತ್ರಾಂಶ್ಚ ದತ್ತಾತ್ರೇಯಃ ಪ್ರದರ್ಶಯೇತ್ ॥೧೩೦॥

ಧನ್ಯಂ ಯಶಸ್ಯಮಾಯುಷ್ಯಂ ಪುತ್ರಭಾಗ್ಯವಿವರ್ಧನಂ ।\\
ಕರೋತಿ ಲೇಖನಾದೇವ ಪರಾರ್ಥಂ ವಾ ನ ಸಂಶಯಃ ॥೧೩೧॥

ಯಃ ಕರೋತ್ಯುಪದೇಶಂ ಚ ನಾಮದತ್ತಸಹಸ್ರಕಂ ।\\
ಸ ಚ ಯಾತಿ ಚ ಸಾಯುಜ್ಯಂ ಶ್ರೀಮಾನ್ ಶ್ರೀಮಾನ್ ನ ಸಂಶಯಃ ॥೧೩೨॥

ಪಠನಾಚ್ಛ್ರವಣಾದ್ವಾಪಿ ಸರ್ವಾನ್ಕಾಮಾನವಾಪ್ನುಯಾತ್ ।\\
ಖೇಚರತ್ವಂ ಕಾರ್ಯಸಿದ್ಧಿಂ ಯೋಗಸಿದ್ಧಿಮವಾಪ್ನುಯಾತ್ ॥೧೩೬॥

ಬ್ರಹ್ಮರಾಕ್ಷಸವೇತಾಲೈಃ ಪಿಶಾಚೈಃ ಕಾಮಿನೀಮುಖೈಃ ।\\
ಪೀಡಾಕರೈಃ ಸುಖಕರೈರ್ಗ್ರಹೈರ್ದುಷ್ಟೈರ್ನ ಬಾಧ್ಯತೇ ॥೧೩೪॥

ದೇವೈಃ ಪಿಶಾಚೈರ್ಮುಚ್ಯೇತ ಸಕೃದುಚ್ಚಾರಣೇನ ತು ।\\
ಯಸ್ಮಿಂದೇಶೇ ಸ್ಥಿತಂ ಚೈತತ್ಪುಸ್ತಕಂ ದತ್ತನಾಮಕಂ ॥೧೩೫॥

ಪಂಚಯೋಜನವಿಸ್ತಾರಂ ರಕ್ಷಣಂ ನಾತ್ರ ಸಂಶಯಃ ।\\
ಸರ್ವಬೀಜಸಮಾಯುಕ್ತಂ ಸ್ತೋತ್ರಂ ನಾಮಸಹಸ್ರಕಂ ॥೧೩೬॥

ಸರ್ವಮಂತ್ರಸ್ವರೂಪಂ ಚ ದತ್ತಾತ್ರೇಯಸ್ವರೂಪಕಂ ।\\
ಏಕವಾರಂ ಪಠಿತ್ವಾ ತು ತಾಮ್ರಪಾತ್ರೇ ಜಲಂ ಸ್ಪೃಶೇತ್ ॥೧೩೭॥

ಪೀತ್ವಾ ಚೇತ್ಸರ್ವರೋಗೈಶ್ಚ ಮುಚ್ಯತೇ ನಾತ್ರ ಸಂಶಯಃ ।\\
ಸ್ತ್ರೀವಶ್ಯಂ ಪುರುಷವಶ್ಯಂ ರಾಜವಶ್ಯಂ ಜಯಾವಹಂ ॥೧೩೮॥

ಸಂಪತ್ಪ್ರದಂ ಮೋಕ್ಷಕರಂ ಪಠೇನ್ನಿತ್ಯಮತಂದ್ರಿತಃ ।\\
ಲೀಯತೇಽಸ್ಮಿನ್ಪ್ರಪಂಚಾರ್ಥಾನ್ ವೈರಿಶೋಕಾದಿಕಾರಿತಃ ॥೧೩೯॥

ಪಠನಾತ್ತು ಪ್ರಸನ್ನೋಽಹಂ ಶಂಕರಾಚಾರ್ಯ ಬುದ್ಧಿಮಾನ್ ।\\
ಭವಿಷ್ಯಸಿ ನ ಸಂದೇಹಃ ಪಠಿತಃ ಪ್ರಾತರೇವ ಮಾಂ ॥೧೪೦॥

ಉಪದೇಕ್ಷ್ಯೇ ಸರ್ವಯೋಗಾನ್ ಲಂಬಿಕಾದಿಬಹೂನ್ವರಾನ್ ।\\
ದತ್ತಾತ್ರೇಯಸ್ತು ಚೇತ್ಯುಕ್ತ್ವಾ ಸ್ವಪ್ನೇ ಚಾಂತರಧೀಯತ ॥೧೪೧॥

ಸ್ವಪ್ನಾದುತ್ಥಾಯ ಚಾಚಾರ್ಯಃ ಶಂಕರೋ ವಿಸ್ಮಯಂ ಗತಃ ।\\
ಸ್ವಪ್ನೋಪದೇಶಿತಂ ಸ್ತೋತ್ರಂ ದತ್ತಾತ್ರೇಯೇನ ಯೋಗಿನಾ ॥೧೪೨॥

ಸಹಸ್ರನಾಮಕಂ ದಿವ್ಯಂ ಪಠಿತ್ವಾ ಯೋಗವಾನ್ಭವೇತ್ ।\\
ಜ್ಞಾನಯೋಗಯತಿತ್ವಂ ಚ ಪರಾಕಾಯಪ್ರವೇಶನಂ ॥೧೪೩॥

ಬಹುವಿದ್ಯಾಖೇಚರತ್ವಂ ದೀರ್ಘಾಯುಸ್ತತ್ಪ್ರಸಾದತಃ ।\\
ತದಾರಭ್ಯ ಭುವಿ ಶ್ರೇಷ್ಠಃ ಪ್ರಸಿದ್ಧಶ್ಚಾಭವದ್ಯತೀ ॥೧೪೪॥
\authorline{ಇತಿ ಶ್ರೀಶಂಕರಾಚಾರ್ಯಸ್ವಪ್ನಾವಸ್ಥಾಯಾಂ ದತ್ತಾತ್ರೇಯೋಪದೇಶಿತಂ ಸಕಲಪುರಾಣವೇದೋಕ್ತಪ್ರಪಂಚಾರ್ಥಸಾರವತ್ಸ್ತೋತ್ರಂ ಸಂಪೂರ್ಣಂ ॥೧೪೫॥}
\section{ ಶ್ರೀ ಲಲಿತಾಸಹಸ್ರನಾಮ ಸ್ತೋತ್ರಂ }
\addcontentsline{toc}{section}{ ಶ್ರೀ ಲಲಿತಾಸಹಸ್ರನಾಮ ಸ್ತೋತ್ರಂ }
ಅಸ್ಯ ಶ್ರೀಲಲಿತಾಸಹಸ್ರನಾಮಸ್ತೋತ್ರಮಾಲಾ ಮಂತ್ರಸ್ಯ~। ವಶಿನ್ಯಾದಿವಾಗ್ದೇವತಾ ಋಷಯಃ~। ಅನುಷ್ಟುಪ್ ಛಂದಃ~। ಶ್ರೀಲಲಿತಾಪರಮೇಶ್ವರೀ ದೇವತಾ~। ಶ್ರೀಮದ್ವಾಗ್ಭವಕೂಟೇತಿ ಬೀಜಂ~। ಮಧ್ಯಕೂಟೇತಿ ಶಕ್ತಿಃ~। ಶಕ್ತಿಕೂಟೇತಿ ಕೀಲಕಂ~। ವಾಕ್ಸಿದ್ಧ್ಯರ್ಥೇ ಜಪೇ ವಿನಿಯೋಗಃ~।

\dhyana{ಸಿಂದೂರಾರುಣ ವಿಗ್ರಹಾಂ ತ್ರಿನಯನಾಂ ಮಾಣಿಕ್ಯಮೌಲಿ ಸ್ಫುರತ್\\
ತಾರಾ ನಾಯಕ ಶೇಖರಾಂ ಸ್ಮಿತಮುಖೀಮಾಪೀನವಕ್ಷೋರುಹಾಂ~।\\
ಪಾಣಿಭ್ಯಾಮಲಿಪೂರ್ಣ ರತ್ನ ಚಷಕಂ ರಕ್ತೋತ್ಪಲಂ ಬಿಭ್ರತೀಂ\\
ಸೌಮ್ಯಾಂ ರತ್ನ ಘಟಸ್ಥ ರಕ್ತಚರಣಾಂ ಧ್ಯಾಯೇತ್ ಪರಾಮಂಬಿಕಾಂ ॥

ಅರುಣಾಂ ಕರುಣಾ ತರಂಗಿತಾಕ್ಷೀಂ ಧೃತ ಪಾಶಾಂಕುಶ ಪುಷ್ಪ ಬಾಣಚಾಪಾಂ~।\\
ಅಣಿಮಾದಿಭಿರಾವೃತಾಂ ಮಯೂಖೈರಹಮಿತ್ಯೇವ ವಿಭಾವಯೇ ಭವಾನೀಂ ॥

ಧ್ಯಾಯೇತ್ ಪದ್ಮಾಸನಸ್ಥಾಂ ವಿಕಸಿತವದನಾಂ ಪದ್ಮಪತ್ರಾಯತಾಕ್ಷೀಂ\\
ಹೇಮಾಭಾಂ ಪೀತವಸ್ತ್ರಾಂ ಕರಕಲಿತಲಸದ್ಧೇಮಪದ್ಮಾಂ ವರಾಂಗೀಂ~।\\
ಸರ್ವಾಲಂಕಾರಯುಕ್ತಾಂ ಸತತಮಭಯದಾಂ ಭಕ್ತನಮ್ರಾಂ ಭವಾನೀಂ\\
ಶ್ರೀವಿದ್ಯಾಂ ಶಾಂತಮೂರ್ತಿಂ ಸಕಲ ಸುರನುತಾಂ ಸರ್ವ ಸಂಪತ್ಪ್ರದಾತ್ರೀಂ ॥

ಸಕುಂಕುಮವಿಲೇಪನಾಮಲಿಕಚುಂಬಿಕಸ್ತೂರಿಕಾಂ\\
ಸಮಂದಹಸಿತೇಕ್ಷಣಾಂ ಸಶರಚಾಪಪಾಶಾಂಕುಶಾಂ~।\\
ಅಶೇಷಜನಮೋಹಿನೀಂ ಅರುಣಮಾಲ್ಯಭೂಷಾಂಬರಾಂ\\
ಜಪಾಕುಸುಮಭಾಸುರಾಂ ಜಪವಿಧೌ ಸ್ಮರೇದಂಬಿಕಾಂ ॥}

{\bfseries ಓಂ ಐಂಹ್ರೀಂಶ್ರೀಂ}\\
ಶ್ರೀಮಾತಾ ಶ್ರೀಮಹಾರಾಜ್ಞೀ ಶ್ರೀಮತ್ಸಿಂಹಾಸನೇಶ್ವರೀ~।\\
ಚಿದಗ್ನಿ-ಕುಂಡ-ಸಂಭೂತಾ ದೇವಕಾರ್ಯ-ಸಮುದ್ಯತಾ ॥೧॥

ಉದ್ಯದ್ಭಾನು-ಸಹಸ್ರಾಭಾ ಚತುರ್ಬಾಹು-ಸಮನ್ವಿತಾ~।\\
ರಾಗಸ್ವರೂಪ-ಪಾಶಾಢ್ಯಾ ಕ್ರೋಧಾಕಾರಾಂಕುಶೋಜ್ಜ್ವಲಾ ॥೨॥

ಮನೋರೂಪೇಕ್ಷು-ಕೋದಂಡಾ ಪಂಚತನ್ಮಾತ್ರ-ಸಾಯಕಾ~।\\
ನಿಜಾರುಣ-ಪ್ರಭಾಪೂರ-ಮಜ್ಜದ್‍ಬ್ರಹ್ಮಾಂಡ-ಮಂಡಲಾ ॥೩॥

ಚಂಪಕಾಶೋಕ-ಪುನ್ನಾಗ-ಸೌಗಂಧಿಕ-ಲಸತ್ಕಚಾ~।\\
ಕುರುವಿಂದಮಣಿ-ಶ್ರೇಣೀ-ಕನತ್ಕೋಟೀರ-ಮಂಡಿತಾ ॥೪॥

ಅಷ್ಟಮೀಚಂದ್ರ-ವಿಭ್ರಾಜ-ದಲಿಕಸ್ಥಲ-ಶೋಭಿತಾ~।\\
ಮುಖಚಂದ್ರ-ಕಲಂಕಾಭ-ಮೃಗನಾಭಿ-ವಿಶೇಷಕಾ ॥೫॥

ವದನಸ್ಮರ-ಮಾಂಗಲ್ಯ-ಗೃಹತೋರಣ-ಚಿಲ್ಲಿಕಾ~।\\
ವಕ್ತ್ರಲಕ್ಷ್ಮೀ-ಪರೀವಾಹ-ಚಲನ್ಮೀನಾಭ-ಲೋಚನಾ ॥೬॥

ನವಚಂಪಕ-ಪುಷ್ಪಾಭ-ನಾಸಾದಂಡ-ವಿರಾಜಿತಾ~।\\
ತಾರಾಕಾಂತಿ-ತಿರಸ್ಕಾರಿ-ನಾಸಾಭರಣ-ಭಾಸುರಾ ॥೭॥

ಕದಂಬಮಂಜರೀ-ಕ್ಲೃಪ್ತ-ಕರ್ಣಪೂರ-ಮನೋಹರಾ~।\\
ತಾಟಂಕ-ಯುಗಲೀ-ಭೂತ-ತಪನೋಡುಪ-ಮಂಡಲಾ ॥೮॥

ಪದ್ಮರಾಗಶಿಲಾದರ್ಶ-ಪರಿಭಾವಿ-ಕಪೋಲಭೂಃ~।\\
ನವವಿದ್ರುಮ-ಬಿಂಬಶ್ರೀ-ನ್ಯಕ್ಕಾರಿ-ರದನಚ್ಛದಾ ॥೯॥

ಶುದ್ಧವಿದ್ಯಾಂಕುರಾಕಾರ-ದ್ವಿಜಪಂಕ್ತಿ-ದ್ವಯೋಜ್ಜ್ವಲಾ~।\\
ಕರ್ಪೂರವೀಟಿಕಾಮೋದ-ಸಮಾಕರ್ಷದ್ದಿಗಂತರಾ ॥೧೦॥

ನಿಜ-ಸಲ್ಲಾಪ-ಮಾಧುರ್ಯ-ವಿನಿರ್ಭರ್ತ್ಸಿತ-ಕಚ್ಛಪೀ~।\\
ಮಂದಸ್ಮಿತ-ಪ್ರಭಾಪೂರ-ಮಜ್ಜತ್ಕಾಮೇಶ-ಮಾನಸಾ ॥೧೧॥

ಅನಾಕಲಿತ-ಸಾದೃಶ್ಯ-ಚುಬುಕಶ್ರೀ-ವಿರಾಜಿತಾ~।\\
ಕಾಮೇಶ-ಬದ್ಧ-ಮಾಂಗಲ್ಯ-ಸೂತ್ರ-ಶೋಭಿತ-ಕಂಧರಾ ॥೧೨॥

ಕನಕಾಂಗದ-ಕೇಯೂರ-ಕಮನೀಯ-ಭುಜಾನ್ವಿತಾ~।\\
ರತ್ನಗ್ರೈವೇಯ-ಚಿಂತಾಕ-ಲೋಲ-ಮುಕ್ತಾ-ಫಲಾನ್ವಿತಾ ॥೧೩॥

ಕಾಮೇಶ್ವರ-ಪ್ರೇಮರತ್ನ-ಮಣಿ-ಪ್ರತಿಪಣ-ಸ್ತನೀ~।\\
ನಾಭ್ಯಾಲವಾಲ-ರೋಮಾಲಿ-ಲತಾ-ಫಲ-ಕುಚದ್ವಯೀ ॥೧೪॥

ಲಕ್ಷ್ಯರೋಮ-ಲತಾಧಾರತಾ-ಸಮುನ್ನೇಯ-ಮಧ್ಯಮಾ~।\\
ಸ್ತನಭಾರ-ದಲನ್ಮಧ್ಯ-ಪಟ್ಟಬಂಧ-ವಲಿತ್ರಯಾ ॥೧೫॥

ಅರುಣಾರುಣಕೌಸುಂಭ-ವಸ್ತ್ರ-ಭಾಸ್ವತ್ಕಟೀತಟೀ~।\\
ರತ್ನ-ಕಿಂಕಿಣಿಕಾ-ರಮ್ಯ-ರಶನಾ-ದಾಮ-ಭೂಷಿತಾ ॥೧೬॥

ಕಾಮೇಶ-ಜ್ಞಾತ-ಸೌಭಾಗ್ಯ-ಮಾರ್ದವೋರು-ದ್ವಯಾನ್ವಿತಾ~।\\
ಮಾಣಿಕ್ಯ-ಮುಕುಟಾಕಾರ-ಜಾನುದ್ವಯ-ವಿರಾಜಿತಾ ॥೧೭॥

ಇಂದ್ರಗೋಪ-ಪರಿಕ್ಷಿಪ್ತಸ್ಮರತೂಣಾಭ-ಜಂಘಿಕಾ~।\\
ಗೂಢಗುಲ್ಫಾ ಕೂರ್ಮಪೃಷ್ಠ-ಜಯಿಷ್ಣು-ಪ್ರಪದಾನ್ವಿತಾ ॥೧೮॥

ನಖ-ದೀಧಿತಿ-ಸಂಛನ್ನ-ನಮಜ್ಜನ-ತಮೋಗುಣಾ~।\\
ಪದದ್ವಯ-ಪ್ರಭಾಜಾಲ-ಪರಾಕೃತ-ಸರೋರುಹಾ ॥೧೯॥

ಶಿಂಜಾನ-ಮಣಿಮಂಜೀರ-ಮಂಡಿತ-ಶ್ರೀ-ಪದಾಂಬುಜಾ~।\\
ಮರಾಲೀ-ಮಂದಗಮನಾ ಮಹಾಲಾವಣ್ಯ-ಶೇವಧಿಃ ॥೨೦॥

ಸರ್ವಾರುಣಾಽನವದ್ಯಾಂಗೀ ಸರ್ವಾಭರಣ-ಭೂಷಿತಾ~।\\
ಶಿವ-ಕಾಮೇಶ್ವರಾಂಕಸ್ಥಾ ಶಿವಾ ಸ್ವಾಧೀನ-ವಲ್ಲಭಾ ॥೨೧॥

ಸುಮೇರು-ಮಧ್ಯ-ಶೃಂಗಸ್ಥಾ ಶ್ರೀಮನ್ನಗರ-ನಾಯಿಕಾ~।\\
ಚಿಂತಾಮಣಿ-ಗೃಹಾಂತಸ್ಥಾ ಪಂಚ-ಬ್ರಹ್ಮಾಸನ-ಸ್ಥಿತಾ ॥೨೨॥

ಮಹಾಪದ್ಮಾಟವೀ-ಸಂಸ್ಥಾ ಕದಂಬವನ-ವಾಸಿನೀ~।\\
ಸುಧಾಸಾಗರ-ಮಧ್ಯಸ್ಥಾ ಕಾಮಾಕ್ಷೀ ಕಾಮದಾಯಿನೀ ॥೨೩॥

ದೇವರ್ಷಿ-ಗಣ-ಸಂಘಾತ-ಸ್ತೂಯಮಾನಾತ್ಮ-ವೈಭವಾ~।\\
ಭಂಡಾಸುರ-ವಧೋದ್ಯುಕ್ತ-ಶಕ್ತಿಸೇನಾ-ಸಮನ್ವಿತಾ ॥೨೪॥

ಸಂಪತ್ಕರೀ-ಸಮಾರೂಢ-ಸಿಂಧುರ-ವ್ರಜ-ಸೇವಿತಾ~।\\
ಅಶ್ವಾರೂಢಾಧಿಷ್ಠಿತಾಶ್ವ-ಕೋಟಿ-ಕೋಟಿಭಿರಾವೃತಾ ॥೨೫॥

ಚಕ್ರರಾಜ-ರಥಾರೂಢ-ಸರ್ವಾಯುಧ-ಪರಿಷ್ಕೃತಾ~।\\
ಗೇಯಚಕ್ರ-ರಥಾರೂಢ-ಮಂತ್ರಿಣೀ-ಪರಿಸೇವಿತಾ ॥೨೬॥

ಕಿರಿಚಕ್ರ-ರಥಾರೂಢ-ದಂಡನಾಥಾ-ಪುರಸ್ಕೃತಾ~।\\
ಜ್ವಾಲಾ-ಮಾಲಿನಿಕಾಕ್ಷಿಪ್ತ-ವಹ್ನಿಪ್ರಾಕಾರ-ಮಧ್ಯಗಾ ॥೨೭॥

ಭಂಡಸೈನ್ಯ-ವಧೋದ್ಯುಕ್ತ-ಶಕ್ತಿ-ವಿಕ್ರಮ-ಹರ್ಷಿತಾ~।\\
ನಿತ್ಯಾ-ಪರಾಕ್ರಮಾಟೋಪ-ನಿರೀಕ್ಷಣ-ಸಮುತ್ಸುಕಾ ॥೨೮॥

ಭಂಡಪುತ್ರ-ವಧೋದ್ಯುಕ್ತ-ಬಾಲಾ-ವಿಕ್ರಮ-ನಂದಿತಾ~।\\
ಮಂತ್ರಿಣ್ಯಂಬಾ-ವಿರಚಿತ-ವಿಷಂಗ-ವಧ-ತೋಷಿತಾ ॥೨೯॥

ವಿಶುಕ್ರ-ಪ್ರಾಣಹರಣ-ವಾರಾಹೀ-ವೀರ್ಯ-ನಂದಿತಾ~।\\
ಕಾಮೇಶ್ವರ-ಮುಖಾಲೋಕ-ಕಲ್ಪಿತ-ಶ್ರೀಗಣೇಶ್ವರಾ ॥೩೦॥

ಮಹಾಗಣೇಶ-ನಿರ್ಭಿನ್ನ-ವಿಘ್ನಯಂತ್ರ-ಪ್ರಹರ್ಷಿತಾ~।\\
ಭಂಡಾಸುರೇಂದ್ರ-ನಿರ್ಮುಕ್ತ-ಶಸ್ತ್ರ-ಪ್ರತ್ಯಸ್ತ್ರ-ವರ್ಷಿಣೀ ॥೩೧॥

ಕರಾಂಗುಲಿ-ನಖೋತ್ಪನ್ನ-ನಾರಾಯಣ-ದಶಾಕೃತಿಃ~।\\
ಮಹಾ-ಪಾಶುಪತಾಸ್ತ್ರಾಗ್ನಿ-ನಿರ್ದಗ್ಧಾಸುರ-ಸೈನಿಕಾ ॥೩೨॥

ಕಾಮೇಶ್ವರಾಸ್ತ್ರ-ನಿರ್ದಗ್ಧ-ಸಭಂಡಾಸುರ-ಶೂನ್ಯಕಾ~।\\
ಬ್ರಹ್ಮೋಪೇಂದ್ರ-ಮಹೇಂದ್ರಾದಿ-ದೇವ-ಸಂಸ್ತುತ-ವೈಭವಾ ॥೩೩॥

ಹರ-ನೇತ್ರಾಗ್ನಿ-ಸಂದಗ್ಧ-ಕಾಮ-ಸಂಜೀವನೌಷಧಿಃ~।\\
ಶ್ರೀಮದ್ವಾಗ್ಭವ-ಕೂಟೈಕ-ಸ್ವರೂಪ-ಮುಖ-ಪಂಕಜಾ ॥೩೪॥

ಕಂಠಾಧಃ-ಕಟಿ-ಪರ್ಯಂತ-ಮಧ್ಯಕೂಟ-ಸ್ವರೂಪಿಣೀ~।\\
ಶಕ್ತಿ-ಕೂಟೈಕತಾಪನ್ನ-ಕಟ್ಯಧೋಭಾಗ ಧಾರಿಣೀ ॥೩೫॥

ಮೂಲ-ಮಂತ್ರಾತ್ಮಿಕಾ ಮೂಲಕೂಟತ್ರಯ-ಕಲೇಬರಾ~।\\
ಕುಲಾಮೃತೈಕ-ರಸಿಕಾ ಕುಲಸಂಕೇತ-ಪಾಲಿನೀ ॥೩೬॥

ಕುಲಾಂಗನಾ ಕುಲಾಂತಸ್ಥಾ ಕೌಲಿನೀ ಕುಲಯೋಗಿನೀ~।\\
ಅಕುಲಾ ಸಮಯಾಂತಸ್ಥಾ ಸಮಯಾಚಾರ-ತತ್ಪರಾ ॥೩೭॥

ಮೂಲಾಧಾರೈಕ-ನಿಲಯಾ ಬ್ರಹ್ಮಗ್ರಂಥಿ-ವಿಭೇದಿನೀ \as{(೧೦೦)}~।\\
ಮಣಿ-ಪೂರಾಂತರುದಿತಾ ವಿಷ್ಣುಗ್ರಂಥಿ-ವಿಭೇದಿನೀ ॥೩೮॥

ಆಜ್ಞಾ-ಚಕ್ರಾಂತರಾಲಸ್ಥಾ ರುದ್ರಗ್ರಂಥಿ-ವಿಭೇದಿನೀ~।\\
ಸಹಸ್ರಾರಾಂಬುಜಾರೂಢಾ ಸುಧಾ-ಸಾರಾಭಿವರ್ಷಿಣೀ ॥೩೯॥

ತಡಿಲ್ಲತಾ-ಸಮರುಚಿಃ ಷಟ್‍ಚಕ್ರೋಪರಿ-ಸಂಸ್ಥಿತಾ~।\\
ಮಹಾಸಕ್ತಿಃ ಕುಂಡಲಿನೀ ಬಿಸತಂತು-ತನೀಯಸೀ ॥೪೦॥

ಭವಾನೀ ಭಾವನಾಗಮ್ಯಾ ಭವಾರಣ್ಯ-ಕುಠಾರಿಕಾ~।\\
ಭದ್ರಪ್ರಿಯಾ ಭದ್ರಮೂರ್ತಿರ್ಭಕ್ತ-ಸೌಭಾಗ್ಯದಾಯಿನೀ ॥೪೧॥

ಭಕ್ತಿಪ್ರಿಯಾ ಭಕ್ತಿಗಮ್ಯಾ ಭಕ್ತಿವಶ್ಯಾ ಭಯಾಪಹಾ~।\\
ಶಾಂಭವೀ ಶಾರದಾರಾಧ್ಯಾ ಶರ್ವಾಣೀ ಶರ್ಮದಾಯಿನೀ ॥೪೨॥

ಶಾಂಕರೀ ಶ್ರೀಕರೀ ಸಾಧ್ವೀ ಶರಚ್ಚಂದ್ರ-ನಿಭಾನನಾ~।\\
ಶಾತೋದರೀ ಶಾಂತಿಮತೀ ನಿರಾಧಾರಾ ನಿರಂಜನಾ ॥೪೩॥

ನಿರ್ಲೇಪಾ ನಿರ್ಮಲಾ ನಿತ್ಯಾ ನಿರಾಕಾರಾ ನಿರಾಕುಲಾ~।\\
ನಿರ್ಗುಣಾ ನಿಷ್ಕಲಾ ಶಾಂತಾ ನಿಷ್ಕಾಮಾ ನಿರುಪಪ್ಲವಾ ॥೪೪॥

ನಿತ್ಯಮುಕ್ತಾ ನಿರ್ವಿಕಾರಾ ನಿಷ್ಪ್ರಪಂಚಾ ನಿರಾಶ್ರಯಾ~।\\
ನಿತ್ಯಶುದ್ಧಾ ನಿತ್ಯಬುದ್ಧಾ ನಿರವದ್ಯಾ ನಿರಂತರಾ ॥೪೫॥

ನಿಷ್ಕಾರಣಾ ನಿಷ್ಕಲಂಕಾ ನಿರುಪಾಧಿರ್ನಿರೀಶ್ವರಾ~।\\
ನೀರಾಗಾ ರಾಗಮಥನೀ ನಿರ್ಮದಾ ಮದನಾಶಿನೀ ॥೪೬॥

ನಿಶ್ಚಿಂತಾ ನಿರಹಂಕಾರಾ ನಿರ್ಮೋಹಾ ಮೋಹನಾಶಿನೀ~।\\
ನಿರ್ಮಮಾ ಮಮತಾಹಂತ್ರೀ ನಿಷ್ಪಾಪಾ ಪಾಪನಾಶಿನೀ ॥೪೭॥

ನಿಷ್ಕ್ರೋಧಾ ಕ್ರೋಧಶಮನೀ ನಿರ್ಲೋಭಾ ಲೋಭನಾಶಿನೀ~।\\
ನಿಃಸಂಶಯಾ ಸಂಶಯಘ್ನೀ ನಿರ್ಭವಾ ಭವನಾಶಿನೀ ॥೪೮॥

ನಿರ್ವಿಕಲ್ಪಾ ನಿರಾಬಾಧಾ ನಿರ್ಭೇದಾ ಭೇದನಾಶಿನೀ~।\\
ನಿರ್ನಾಶಾ ಮೃತ್ಯುಮಥಿನೀ ನಿಷ್ಕ್ರಿಯಾ ನಿಷ್ಪರಿಗ್ರಹಾ~।೪೯॥

ನಿಸ್ತುಲಾ ನೀಲಚಿಕುರಾ ನಿರಪಾಯಾ ನಿರತ್ಯಯಾ~।\\
ದುರ್ಲಭಾ ದುರ್ಗಮಾ ದುರ್ಗಾ ದುಃಖಹಂತ್ರೀ ಸುಖಪ್ರದಾ ॥೫೦॥

ದುಷ್ಟದೂರಾ ದುರಾಚಾರ-ಶಮನೀ ದೋಷವರ್ಜಿತಾ~।\\
ಸರ್ವಜ್ಞಾ ಸಾಂದ್ರಕರುಣಾ ಸಮಾನಾಧಿಕ-ವರ್ಜಿತಾ ॥೫೧॥

ಸರ್ವಶಕ್ತಿಮಯೀ ಸರ್ವ-ಮಂಗಲಾ \as{(೨೦೦)} ಸದ್ಗತಿಪ್ರದಾ~।\\
ಸರ್ವೇಶ್ವರೀ ಸರ್ವಮಯೀ ಸರ್ವಮಂತ್ರ-ಸ್ವರೂಪಿಣೀ ॥೫೨॥

ಸರ್ವ-ಯಂತ್ರಾತ್ಮಿಕಾ ಸರ್ವ-ತಂತ್ರರೂಪಾ ಮನೋನ್ಮನೀ~।\\
ಮಾಹೇಶ್ವರೀ ಮಹಾದೇವೀ ಮಹಾಲಕ್ಷ್ಮೀರ್ಮೃಡಪ್ರಿಯಾ ॥೫೩॥

ಮಹಾರೂಪಾ ಮಹಾಪೂಜ್ಯಾ ಮಹಾಪಾತಕ-ನಾಶಿನೀ~।\\
ಮಹಾಮಾಯಾ ಮಹಾಸತ್ತ್ವಾ ಮಹಾಶಕ್ತಿರ್ಮಹಾರತಿಃ ॥೫೪॥

ಮಹಾಭೋಗಾ ಮಹೈಶ್ವರ್ಯಾ ಮಹಾವೀರ್ಯಾ ಮಹಾಬಲಾ~।\\
ಮಹಾಬುದ್ಧಿರ್ಮಹಾಸಿದ್ಧಿರ್ಮಹಾಯೋಗೇಶ್ವರೇಶ್ವರೀ ॥೫೫॥

ಮಹಾತಂತ್ರಾ ಮಹಾಮಂತ್ರಾ ಮಹಾಯಂತ್ರಾ ಮಹಾಸನಾ~।\\
ಮಹಾಯಾಗ-ಕ್ರಮಾರಾಧ್ಯಾ ಮಹಾಭೈರವ-ಪೂಜಿತಾ ॥೫೬॥

ಮಹೇಶ್ವರ-ಮಹಾಕಲ್ಪ-ಮಹಾತಾಂಡವ-ಸಾಕ್ಷಿಣೀ~।\\
ಮಹಾಕಾಮೇಶ-ಮಹಿಷೀ ಮಹಾತ್ರಿಪುರ-ಸುಂದರೀ ॥೫೭॥

ಚತುಃಷಷ್ಟ್ಯುಪಚಾರಾಢ್ಯಾ ಚತುಃಷಷ್ಟಿಕಲಾಮಯೀ~।\\
ಮಹಾಚತುಃ-ಷಷ್ಟಿಕೋಟಿ-ಯೋಗಿನೀ-ಗಣಸೇವಿತಾ ॥೫೮॥

ಮನುವಿದ್ಯಾ ಚಂದ್ರವಿದ್ಯಾ ಚಂದ್ರಮಂಡಲ-ಮಧ್ಯಗಾ~।\\
ಚಾರುರೂಪಾ ಚಾರುಹಾಸಾ ಚಾರುಚಂದ್ರ-ಕಲಾಧರಾ ॥೫೯॥

ಚರಾಚರ-ಜಗನ್ನಾಥಾ ಚಕ್ರರಾಜ-ನಿಕೇತನಾ~।\\
ಪಾರ್ವತೀ ಪದ್ಮನಯನಾ ಪದ್ಮರಾಗ-ಸಮಪ್ರಭಾ ॥೬೦॥

ಪಂಚ-ಪ್ರೇತಾಸನಾಸೀನಾ ಪಂಚಬ್ರಹ್ಮ-ಸ್ವರೂಪಿಣೀ~।\\
ಚಿನ್ಮಯೀ ಪರಮಾನಂದಾ ವಿಜ್ಞಾನ-ಘನರೂಪಿಣೀ ॥೬೧॥

ಧ್ಯಾನ-ಧ್ಯಾತೃ-ಧ್ಯೇಯರೂಪಾ ಧರ್ಮಾಧರ್ಮ-ವಿವರ್ಜಿತಾ~।\\
ವಿಶ್ವರೂಪಾ ಜಾಗರಿಣೀ ಸ್ವಪಂತೀ ತೈಜಸಾತ್ಮಿಕಾ ॥೬೨॥

ಸುಪ್ತಾ ಪ್ರಾಜ್ಞಾತ್ಮಿಕಾ ತುರ್ಯಾ ಸರ್ವಾವಸ್ಥಾ-ವಿವರ್ಜಿತಾ~।\\
ಸೃಷ್ಟಿಕರ್ತ್ರೀ ಬ್ರಹ್ಮರೂಪಾ ಗೋಪ್ತ್ರೀ ಗೋವಿಂದರೂಪಿಣೀ ॥೬೩॥

ಸಂಹಾರಿಣೀ ರುದ್ರರೂಪಾ ತಿರೋಧಾನ-ಕರೀಶ್ವರೀ~।\\
ಸದಾಶಿವಾಽನುಗ್ರಹದಾ ಪಂಚಕೃತ್ಯ-ಪರಾಯಣಾ ॥೬೪॥

ಭಾನುಮಂಡಲ-ಮಧ್ಯಸ್ಥಾ ಭೈರವೀ ಭಗಮಾಲಿನೀ~।\\
ಪದ್ಮಾಸನಾ ಭಗವತೀ ಪದ್ಮನಾಭ-ಸಹೋದರೀ ॥೬೫॥

ಉನ್ಮೇಷ-ನಿಮಿಷೋತ್ಪನ್ನ-ವಿಪನ್ನ-ಭುವನಾವಲಿಃ~।\\
ಸಹಸ್ರ-ಶೀರ್ಷವದನಾ ಸಹಸ್ರಾಕ್ಷೀ ಸಹಸ್ರಪಾತ್ ॥೬೬॥

ಆಬ್ರಹ್ಮ-ಕೀಟ-ಜನನೀ ವರ್ಣಾಶ್ರಮ-ವಿಧಾಯಿನೀ~।\\
ನಿಜಾಜ್ಞಾರೂಪ-ನಿಗಮಾ ಪುಣ್ಯಾಪುಣ್ಯ-ಫಲಪ್ರದಾ ॥೬೭॥

ಶ್ರುತಿ-ಸೀಮಂತ-ಸಿಂದೂರೀ-ಕೃತ-ಪಾದಾಬ್ಜ-ಧೂಲಿಕಾ~।\\
ಸಕಲಾಗಮ-ಸಂದೋಹ-ಶುಕ್ತಿ-ಸಂಪುಟ-ಮೌಕ್ತಿಕಾ ॥೬೮॥

ಪುರುಷಾರ್ಥಪ್ರದಾ ಪೂರ್ಣಾ ಭೋಗಿನೀ ಭುವನೇಶ್ವರೀ~।\\
ಅಂಬಿಕಾಽಽನಾದಿ-ನಿಧನಾ ಹರಿಬ್ರಹ್ಮೇಂದ್ರ-ಸೇವಿತಾ ॥೬೯॥

ನಾರಾಯಣೀ ನಾದರೂಪಾ ನಾಮರೂಪ-ವಿವರ್ಜಿತಾ \as{(೩೦೦)}।\\
ಹ್ರೀಂಕಾರೀ ಹ್ರೀಂಮತೀ ಹೃದ್ಯಾ ಹೇಯೋಪಾದೇಯ-ವರ್ಜಿತಾ ॥೭೦॥

ರಾಜರಾಜಾರ್ಚಿತಾ ರಾಜ್ಞೀ ರಮ್ಯಾ ರಾಜೀವಲೋಚನಾ~।\\
ರಂಜನೀ ರಮಣೀ ರಸ್ಯಾ ರಣತ್ಕಿಂಕಿಣಿ-ಮೇಖಲಾ ॥೭೧॥

ರಮಾ ರಾಕೇಂದುವದನಾ ರತಿರೂಪಾ ರತಿಪ್ರಿಯಾ~।\\
ರಕ್ಷಾಕರೀ ರಾಕ್ಷಸಘ್ನೀ ರಾಮಾ ರಮಣಲಂಪಟಾ ॥೭೨॥

ಕಾಮ್ಯಾ ಕಾಮಕಲಾರೂಪಾ ಕದಂಬ-ಕುಸುಮ-ಪ್ರಿಯಾ~।\\
ಕಲ್ಯಾಣೀ ಜಗತೀಕಂದಾ ಕರುಣಾ-ರಸ-ಸಾಗರಾ ॥೭೩॥

ಕಲಾವತೀ ಕಲಾಲಾಪಾ ಕಾಂತಾ ಕಾದಂಬರೀಪ್ರಿಯಾ~।\\
ವರದಾ ವಾಮನಯನಾ ವಾರುಣೀ-ಮದ-ವಿಹ್ವಲಾ ॥೭೪॥

ವಿಶ್ವಾಧಿಕಾ ವೇದವೇದ್ಯಾ ವಿಂಧ್ಯಾಚಲ-ನಿವಾಸಿನೀ~।\\
ವಿಧಾತ್ರೀ ವೇದಜನನೀ ವಿಷ್ಣುಮಾಯಾ ವಿಲಾಸಿನೀ ॥೭೫॥

ಕ್ಷೇತ್ರಸ್ವರೂಪಾ ಕ್ಷೇತ್ರೇಶೀ ಕ್ಷೇತ್ರ-ಕ್ಷೇತ್ರಜ್ಞ-ಪಾಲಿನೀ~।\\
ಕ್ಷಯವೃದ್ಧಿ-ವಿನಿರ್ಮುಕ್ತಾ ಕ್ಷೇತ್ರಪಾಲ-ಸಮರ್ಚಿತಾ ॥೭೬॥

ವಿಜಯಾ ವಿಮಲಾ ವಂದ್ಯಾ ವಂದಾರು-ಜನ-ವತ್ಸಲಾ~।\\
ವಾಗ್ವಾದಿನೀ ವಾಮಕೇಶೀ ವಹ್ನಿಮಂಡಲ-ವಾಸಿನೀ ॥೭೭॥

ಭಕ್ತಿಮತ್-ಕಲ್ಪಲತಿಕಾ ಪಶುಪಾಶ-ವಿಮೋಚಿನೀ~।\\
ಸಂಹೃತಾಶೇಷ-ಪಾಷಂಡಾ ಸದಾಚಾರ-ಪ್ರವರ್ತಿಕಾ ॥೭೮॥

ತಾಪತ್ರಯಾಗ್ನಿ-ಸಂತಪ್ತ-ಸಮಾಹ್ಲಾದನ ಚಂದ್ರಿಕಾ~।\\
ತರುಣೀ ತಾಪಸಾರಾಧ್ಯಾ ತನುಮಧ್ಯಾ ತಮೋಽಪಹಾ ॥೭೯॥

ಚಿತಿಸ್ತತ್ಪದ-ಲಕ್ಷ್ಯಾರ್ಥಾ ಚಿದೇಕರಸ-ರೂಪಿಣೀ~।\\
ಸ್ವಾತ್ಮಾನಂದ-ಲವೀಭೂತ-ಬ್ರಹ್ಮಾದ್ಯಾನಂದ-ಸಂತತಿಃ ॥೮೦॥

ಪರಾ ಪ್ರತ್ಯಕ್ಚಿತೀರೂಪಾ ಪಶ್ಯಂತೀ ಪರದೇವತಾ~।\\
ಮಧ್ಯಮಾ ವೈಖರೀರೂಪಾ ಭಕ್ತ-ಮಾನಸ-ಹಂಸಿಕಾ ॥೮೧॥

ಕಾಮೇಶ್ವರ-ಪ್ರಾಣನಾಡೀ ಕೃತಜ್ಞಾ ಕಾಮಪೂಜಿತಾ~।\\
ಶೃಂಗಾರ-ರಸ-ಸಂಪೂರ್ಣಾ ಜಯಾ ಜಾಲಂಧರ-ಸ್ಥಿತಾ ॥೮೨॥

ಓಡ್ಯಾಣಪೀಠ-ನಿಲಯಾ ಬಿಂದು-ಮಂಡಲವಾಸಿನೀ~।\\
ರಹೋಯಾಗ-ಕ್ರಮಾರಾಧ್ಯಾ ರಹಸ್ತರ್ಪಣ-ತರ್ಪಿತಾ ॥೮೩॥

ಸದ್ಯಃಪ್ರಸಾದಿನೀ ವಿಶ್ವ-ಸಾಕ್ಷಿಣೀ ಸಾಕ್ಷಿವರ್ಜಿತಾ~।\\
ಷಡಂಗದೇವತಾ-ಯುಕ್ತಾ ಷಾಡ್ಗುಣ್ಯ-ಪರಿಪೂರಿತಾ ॥೮೪॥

ನಿತ್ಯಕ್ಲಿನ್ನಾ ನಿರುಪಮಾ ನಿರ್ವಾಣ-ಸುಖ-ದಾಯಿನೀ~।\\
ನಿತ್ಯಾ-ಷೋಡಶಿಕಾ-ರೂಪಾ ಶ್ರೀಕಂಠಾರ್ಧ-ಶರೀರಿಣೀ ॥೮೫॥

ಪ್ರಭಾವತೀ ಪ್ರಭಾರೂಪಾ ಪ್ರಸಿದ್ಧಾ ಪರಮೇಶ್ವರೀ~।\\
ಮೂಲಪ್ರಕೃತಿರವ್ಯಕ್ತಾ ವ್ಯಕ್ತಾವ್ಯಕ್ತ-ಸ್ವರೂಪಿಣೀ ॥೮೬॥

ವ್ಯಾಪಿನೀ \as{(೪೦೦)} ವಿವಿಧಾಕಾರಾ ವಿದ್ಯಾವಿದ್ಯಾ-ಸ್ವರೂಪಿಣೀ~।\\
ಮಹಾಕಾಮೇಶ-ನಯನ-ಕುಮುದಾಹ್ಲಾದ-ಕೌಮುದೀ ॥೮೭॥

ಭಕ್ತ-ಹಾರ್ದ-ತಮೋಭೇದ-ಭಾನುಮದ್ಭಾನು-ಸಂತತಿಃ~।\\
ಶಿವದೂತೀ ಶಿವಾರಾಧ್ಯಾ ಶಿವಮೂರ್ತಿಃ ಶಿವಂಕರೀ ॥೮೮॥

ಶಿವಪ್ರಿಯಾ ಶಿವಪರಾ ಶಿಷ್ಟೇಷ್ಟಾ ಶಿಷ್ಟಪೂಜಿತಾ~।\\
ಅಪ್ರಮೇಯಾ ಸ್ವಪ್ರಕಾಶಾ ಮನೋವಾಚಾಮಗೋಚರಾ ॥೮೯॥

ಚಿಚ್ಛಕ್ತಿಶ್ ಚೇತನಾರೂಪಾ ಜಡಶಕ್ತಿರ್ಜಡಾತ್ಮಿಕಾ~।\\
ಗಾಯತ್ರೀ ವ್ಯಾಹೃತಿಃ ಸಂಧ್ಯಾ ದ್ವಿಜಬೃಂದ-ನಿಷೇವಿತಾ ॥೯೦॥

ತತ್ತ್ವಾಸನಾ ತತ್ತ್ವಮಯೀ ಪಂಚ-ಕೋಶಾಂತರ-ಸ್ಥಿತಾ~।\\
ನಿಃಸೀಮಮಹಿಮಾ ನಿತ್ಯ-ಯೌವನಾ ಮದಶಾಲಿನೀ ॥೯೧॥

ಮದಘೂರ್ಣಿತ-ರಕ್ತಾಕ್ಷೀ ಮದಪಾಟಲ-ಗಂಡಭೂಃ~।\\
ಚಂದನ-ದ್ರವ-ದಿಗ್ಧಾಂಗೀ ಚಾಂಪೇಯ-ಕುಸುಮ-ಪ್ರಿಯಾ ॥೯೨॥

ಕುಶಲಾ ಕೋಮಲಾಕಾರಾ ಕುರುಕುಲ್ಲಾ ಕುಲೇಶ್ವರೀ~।\\
ಕುಲಕುಂಡಾಲಯಾ ಕೌಲ-ಮಾರ್ಗ-ತತ್ಪರ-ಸೇವಿತಾ ॥೯೩॥

ಕುಮಾರ-ಗಣನಾಥಾಂಬಾ ತುಷ್ಟಿಃ ಪುಷ್ಟಿರ್ಮತಿರ್ಧೃತಿಃ~।\\
ಶಾಂತಿಃ ಸ್ವಸ್ತಿಮತೀ ಕಾಂತಿರ್ನಂದಿನೀ ವಿಘ್ನನಾಶಿನೀ ॥೯೪॥

ತೇಜೋವತೀ ತ್ರಿನಯನಾ ಲೋಲಾಕ್ಷೀ-ಕಾಮರೂಪಿಣೀ~।\\
ಮಾಲಿನೀ ಹಂಸಿನೀ ಮಾತಾ ಮಲಯಾಚಲ-ವಾಸಿನೀ ॥೯೫॥

ಸುಮುಖೀ ನಲಿನೀ ಸುಭ್ರೂಃ ಶೋಭನಾ ಸುರನಾಯಿಕಾ~।\\
ಕಾಲಕಂಠೀ ಕಾಂತಿಮತೀ ಕ್ಷೋಭಿಣೀ ಸೂಕ್ಷ್ಮರೂಪಿಣೀ ॥೯೬॥

ವಜ್ರೇಶ್ವರೀ ವಾಮದೇವೀ ವಯೋಽವಸ್ಥಾ-ವಿವರ್ಜಿತಾ~।\\
ಸಿದ್ಧೇಶ್ವರೀ ಸಿದ್ಧವಿದ್ಯಾ ಸಿದ್ಧಮಾತಾ ಯಶಸ್ವಿನೀ ॥೯೭॥

ವಿಶುದ್ಧಿಚಕ್ರ-ನಿಲಯಾಽಽರಕ್ತವರ್ಣಾ ತ್ರಿಲೋಚನಾ~।\\
ಖಟ್‍ವಾಂಗಾದಿ-ಪ್ರಹರಣಾ ವದನೈಕ-ಸಮನ್ವಿತಾ ॥೯೮॥

ಪಾಯಸಾನ್ನಪ್ರಿಯಾ ತ್ವಕ್ಸ್ಥಾ ಪಶುಲೋಕ-ಭಯಂಕರೀ~।\\
ಅಮೃತಾದಿ-ಮಹಾಶಕ್ತಿ-ಸಂವೃತಾ ಡಾಕಿನೀಶ್ವರೀ ॥೯೯॥

ಅನಾಹತಾಬ್ಜ-ನಿಲಯಾ ಶ್ಯಾಮಾಭಾ ವದನದ್ವಯಾ~।\\
ದಂಷ್ಟ್ರೋಜ್ಜ್ವಲಾಽಕ್ಷ-ಮಾಲಾದಿ-ಧರಾ ರುಧಿರಸಂಸ್ಥಿತಾ ॥೧೦೦॥

ಕಾಲರಾತ್ರ್ಯಾದಿ-ಶಕ್ತ್ಯೌಘ-ವೃತಾ ಸ್ನಿಗ್ಧೌದನಪ್ರಿಯಾ~।\\
ಮಹಾವೀರೇಂದ್ರ-ವರದಾ ರಾಕಿಣ್ಯಂಬಾ-ಸ್ವರೂಪಿಣೀ ॥೧೦೧॥

ಮಣಿಪೂರಾಬ್ಜ-ನಿಲಯಾ ವದನತ್ರಯ-ಸಂಯುತಾ~।\\
ವಜ್ರಾದಿಕಾಯುಧೋಪೇತಾ ಡಾಮರ್ಯಾದಿಭಿರಾವೃತಾ ॥೧೦೨॥

ರಕ್ತವರ್ಣಾ ಮಾಂಸನಿಷ್ಠಾ \as{(೫೦೦)} ಗುಡಾನ್ನ-ಪ್ರೀತ-ಮಾನಸಾ~।\\
ಸಮಸ್ತಭಕ್ತ-ಸುಖದಾ ಲಾಕಿನ್ಯಂಬಾ-ಸ್ವರೂಪಿಣೀ ॥೧೦೩॥

ಸ್ವಾಧಿಷ್ಠಾನಾಂಬುಜ-ಗತಾ ಚತುರ್ವಕ್ತ್ರ-ಮನೋಹರಾ~।\\
ಶೂಲಾದ್ಯಾಯುಧ-ಸಂಪನ್ನಾ ಪೀತವರ್ಣಾಽತಿಗರ್ವಿತಾ ॥೧೦೪॥

ಮೇದೋನಿಷ್ಠಾ ಮಧುಪ್ರೀತಾ ಬಂದಿನ್ಯಾದಿ-ಸಮನ್ವಿತಾ~।\\
ದಧ್ಯನ್ನಾಸಕ್ತ-ಹೃದಯಾ ಕಾಕಿನೀ-ರೂಪ-ಧಾರಿಣೀ ॥೧೦೫॥

ಮೂಲಾಧಾರಾಂಬುಜಾರೂಢಾ ಪಂಚ-ವಕ್ತ್ರಾಽಸ್ಥಿ-ಸಂಸ್ಥಿತಾ~।\\
ಅಂಕುಶಾದಿ-ಪ್ರಹರಣಾ ವರದಾದಿ-ನಿಷೇವಿತಾ ॥೧೦೬॥

ಮುದ್ಗೌದನಾಸಕ್ತ-ಚಿತ್ತಾ ಸಾಕಿನ್ಯಂಬಾ-ಸ್ವರೂಪಿಣೀ~।\\
ಆಜ್ಞಾ-ಚಕ್ರಾಬ್ಜ-ನಿಲಯಾ ಶುಕ್ಲವರ್ಣಾ ಷಡಾನನಾ ॥೧೦೭॥

ಮಜ್ಜಾಸಂಸ್ಥಾ ಹಂಸವತೀ-ಮುಖ್ಯ-ಶಕ್ತಿ-ಸಮನ್ವಿತಾ~।\\
ಹರಿದ್ರಾನ್ನೈಕ-ರಸಿಕಾ ಹಾಕಿನೀ-ರೂಪ-ಧಾರಿಣೀ ॥೧೦೮॥

ಸಹಸ್ರದಲ-ಪದ್ಮಸ್ಥಾ ಸರ್ವ-ವರ್ಣೋಪ-ಶೋಭಿತಾ~।\\
ಸರ್ವಾಯುಧಧರಾ ಶುಕ್ಲ-ಸಂಸ್ಥಿತಾ ಸರ್ವತೋಮುಖೀ ॥೧೦೯॥

ಸರ್ವೌದನ-ಪ್ರೀತಚಿತ್ತಾ ಯಾಕಿನ್ಯಂಬಾ-ಸ್ವರೂಪಿಣೀ~।\\
ಸ್ವಾಹಾ ಸ್ವಧಾಽಮತಿರ್ಮೇಧಾ ಶ್ರುತಿಃ ಸ್ಮೃತಿರನುತ್ತಮಾ ॥೧೧೦॥

ಪುಣ್ಯಕೀರ್ತಿಃ ಪುಣ್ಯಲಭ್ಯಾ ಪುಣ್ಯಶ್ರವಣ-ಕೀರ್ತನಾ~।\\
ಪುಲೋಮಜಾರ್ಚಿತಾ ಬಂಧ-ಮೋಚನೀ ಬಂಧುರಾಲಕಾ ॥೧೧೧॥

ವಿಮರ್ಶರೂಪಿಣೀ ವಿದ್ಯಾ ವಿಯದಾದಿ-ಜಗತ್ಪ್ರಸೂಃ~।\\
ಸರ್ವವ್ಯಾಧಿ-ಪ್ರಶಮನೀ ಸರ್ವಮೃತ್ಯು-ನಿವಾರಿಣೀ ॥೧೧೨॥

ಅಗ್ರಗಣ್ಯಾಽಚಿಂತ್ಯರೂಪಾ ಕಲಿಕಲ್ಮಷ-ನಾಶಿನೀ~।\\
ಕಾತ್ಯಾಯನೀ ಕಾಲಹಂತ್ರೀ ಕಮಲಾಕ್ಷ-ನಿಷೇವಿತಾ ॥೧೧೩॥

ತಾಂಬೂಲ-ಪೂರಿತ-ಮುಖೀ ದಾಡಿಮೀ-ಕುಸುಮ-ಪ್ರಭಾ~।\\
ಮೃಗಾಕ್ಷೀ ಮೋಹಿನೀ ಮುಖ್ಯಾ ಮೃಡಾನೀ ಮಿತ್ರರೂಪಿಣೀ ॥೧೧೪॥

ನಿತ್ಯತೃಪ್ತಾ ಭಕ್ತನಿಧಿರ್ನಿಯಂತ್ರೀ ನಿಖಿಲೇಶ್ವರೀ~।\\
ಮೈತ್ರ್ಯಾದಿ-ವಾಸನಾಲಭ್ಯಾ ಮಹಾಪ್ರಲಯ-ಸಾಕ್ಷಿಣೀ ॥೧೧೫॥

ಪರಾ ಶಕ್ತಿಃ ಪರಾ ನಿಷ್ಠಾ ಪ್ರಜ್ಞಾನಘನ-ರೂಪಿಣೀ~।\\
ಮಾಧ್ವೀಪಾನಾಲಸಾ ಮತ್ತಾ ಮಾತೃಕಾ-ವರ್ಣ-ರೂಪಿಣೀ ॥೧೧೬॥

ಮಹಾಕೈಲಾಸ-ನಿಲಯಾ ಮೃಣಾಲ-ಮೃದು-ದೋರ್ಲತಾ~।\\
ಮಹನೀಯಾ ದಯಾಮೂರ್ತಿರ್ಮಹಾಸಾಮ್ರಾಜ್ಯ-ಶಾಲಿನೀ ॥೧೧೭॥

ಆತ್ಮವಿದ್ಯಾ ಮಹಾವಿದ್ಯಾ ಶ್ರೀವಿದ್ಯಾ ಕಾಮಸೇವಿತಾ~।\\
ಶ್ರೀ-ಷೋಡಶಾಕ್ಷರೀ-ವಿದ್ಯಾ ತ್ರಿಕೂಟಾ ಕಾಮಕೋಟಿಕಾ ॥೧೧೮॥

ಕಟಾಕ್ಷ-ಕಿಂಕರೀ-ಭೂತ-ಕಮಲಾ-ಕೋಟಿ-ಸೇವಿತಾ~।\\
ಶಿರಃಸ್ಥಿತಾ ಚಂದ್ರನಿಭಾ ಭಾಲಸ್ಥೇಂದ್ರ-ಧನುಃಪ್ರಭಾ ॥೧೧೯॥

ಹೃದಯಸ್ಥಾ ರವಿಪ್ರಖ್ಯಾ ತ್ರಿಕೋಣಾಂತರ-ದೀಪಿಕಾ~।\\
ದಾಕ್ಷಾಯಣೀ ದೈತ್ಯಹಂತ್ರೀ ದಕ್ಷಯಜ್ಞ-ವಿನಾಶಿನೀ \as{(೬೦೦)} ॥೧೨೦॥

ದರಾಂದೋಲಿತ-ದೀರ್ಘಾಕ್ಷೀ ದರ-ಹಾಸೋಜ್ಜ್ವಲನ್ಮುಖೀ~।\\
ಗುರುಮೂರ್ತಿರ್ಗುಣನಿಧಿರ್ಗೋಮಾತಾ ಗುಹಜನ್ಮಭೂಃ ॥೧೨೧॥

ದೇವೇಶೀ ದಂಡನೀತಿಸ್ಥಾ ದಹರಾಕಾಶ-ರೂಪಿಣೀ~।\\
ಪ್ರತಿಪನ್ಮುಖ್ಯ-ರಾಕಾಂತ-ತಿಥಿ-ಮಂಡಲ-ಪೂಜಿತಾ ॥೧೨೨॥

ಕಲಾತ್ಮಿಕಾ ಕಲಾನಾಥಾ ಕಾವ್ಯಾಲಾಪ-ವಿನೋದಿನೀ~।\\
ಸಚಾಮರ-ರಮಾ-ವಾಣೀ-ಸವ್ಯ-ದಕ್ಷಿಣ-ಸೇವಿತಾ ॥೧೨೩॥

ಆದಿಶಕ್ತಿರಮೇಯಾಽಽತ್ಮಾ ಪರಮಾ ಪಾವನಾಕೃತಿಃ~।\\
ಅನೇಕಕೋಟಿ-ಬ್ರಹ್ಮಾಂಡ-ಜನನೀ ದಿವ್ಯವಿಗ್ರಹಾ ॥೧೨೪॥

ಕ್ಲೀಂಕಾರೀ ಕೇವಲಾ ಗುಹ್ಯಾ ಕೈವಲ್ಯ-ಪದದಾಯಿನೀ~।\\
ತ್ರಿಪುರಾ ತ್ರಿಜಗದ್ವಂದ್ಯಾ ತ್ರಿಮೂರ್ತಿಸ್ತ್ರಿದಶೇಶ್ವರೀ ॥೧೨೫॥

ತ್ರ್ಯಕ್ಷರೀ ದಿವ್ಯ-ಗಂಧಾಢ್ಯಾ ಸಿಂದೂರ-ತಿಲಕಾಂಚಿತಾ~।\\
ಉಮಾ ಶೈಲೇಂದ್ರತನಯಾ ಗೌರೀ ಗಂಧರ್ವ-ಸೇವಿತಾ ॥೧೨೬॥

ವಿಶ್ವಗರ್ಭಾ ಸ್ವರ್ಣಗರ್ಭಾ ವರದಾ ವಾಗಧೀಶ್ವರೀ~।\\
ಧ್ಯಾನಗಮ್ಯಾಽಪರಿಚ್ಛೇದ್ಯಾ ಜ್ಞಾನದಾ ಜ್ಞಾನವಿಗ್ರಹಾ ॥೧೨೭॥

ಸರ್ವವೇದಾಂತ-ಸಂವೇದ್ಯಾ ಸತ್ಯಾನಂದ-ಸ್ವರೂಪಿಣೀ~।\\
ಲೋಪಾಮುದ್ರಾರ್ಚಿತಾ ಲೀಲಾ-ಕ್ಲೃಪ್ತ-ಬ್ರಹ್ಮಾಂಡ-ಮಂಡಲಾ ॥೧೨೮॥

ಅದೃಶ್ಯಾ ದೃಶ್ಯರಹಿತಾ ವಿಜ್ಞಾತ್ರೀ ವೇದ್ಯವರ್ಜಿತಾ~।\\
ಯೋಗಿನೀ ಯೋಗದಾ ಯೋಗ್ಯಾ ಯೋಗಾನಂದಾ ಯುಗಂಧರಾ ॥೧೨೯॥

ಇಚ್ಛಾಶಕ್ತಿ-ಜ್ಞಾನಶಕ್ತಿ-ಕ್ರಿಯಾಶಕ್ತಿ-ಸ್ವರೂಪಿಣೀ~।\\
ಸರ್ವಾಧಾರಾ ಸುಪ್ರತಿಷ್ಠಾ ಸದಸದ್ರೂಪ-ಧಾರಿಣೀ ॥೧೩೦॥

ಅಷ್ಟಮೂರ್ತಿರಜಾಜೈತ್ರೀ ಲೋಕಯಾತ್ರಾ-ವಿಧಾಯಿನೀ~।\\
ಏಕಾಕಿನೀ ಭೂಮರೂಪಾ ನಿರ್ದ್ವೈತಾ ದ್ವೈತವರ್ಜಿತಾ ॥೧೩೧॥

ಅನ್ನದಾ ವಸುದಾ ವೃದ್ಧಾ ಬ್ರಹ್ಮಾತ್ಮೈಕ್ಯ-ಸ್ವರೂಪಿಣೀ~।\\
ಬೃಹತೀ ಬ್ರಾಹ್ಮಣೀ ಬ್ರಾಹ್ಮೀ ಬ್ರಹ್ಮಾನಂದಾ ಬಲಿಪ್ರಿಯಾ ॥೧೩೨॥

ಭಾಷಾರೂಪಾ ಬೃಹತ್ಸೇನಾ ಭಾವಾಭಾವ-ವಿವರ್ಜಿತಾ~।\\
ಸುಖಾರಾಧ್ಯಾ ಶುಭಕರೀ ಶೋಭನಾ ಸುಲಭಾ ಗತಿಃ ॥೧೩೩॥

ರಾಜ-ರಾಜೇಶ್ವರೀ ರಾಜ್ಯ-ದಾಯಿನೀ ರಾಜ್ಯ-ವಲ್ಲಭಾ~।\\
ರಾಜತ್ಕೃಪಾ ರಾಜಪೀಠ-ನಿವೇಶಿತ-ನಿಜಾಶ್ರಿತಾ ॥೧೩೪॥

ರಾಜ್ಯಲಕ್ಷ್ಮೀಃ ಕೋಶನಾಥಾ ಚತುರಂಗ-ಬಲೇಶ್ವರೀ~।\\
ಸಾಮ್ರಾಜ್ಯ-ದಾಯಿನೀ ಸತ್ಯಸಂಧಾ ಸಾಗರಮೇಖಲಾ ॥೧೩೫॥

ದೀಕ್ಷಿತಾ ದೈತ್ಯಶಮನೀ ಸರ್ವಲೋಕ-ವಶಂಕರೀ~।\\
ಸರ್ವಾರ್ಥದಾತ್ರೀ ಸಾವಿತ್ರೀ ಸಚ್ಚಿದಾನಂದ-ರೂಪಿಣೀ \as{(೭೦೦)} ॥೧೩೬॥

ದೇಶ-ಕಾಲಾಪರಿಚ್ಛಿನ್ನಾ ಸರ್ವಗಾ ಸರ್ವಮೋಹಿನೀ~।\\
ಸರಸ್ವತೀ ಶಾಸ್ತ್ರಮಯೀ ಗುಹಾಂಬಾ ಗುಹ್ಯರೂಪಿಣೀ ॥೧೩೭॥

ಸರ್ವೋಪಾಧಿ-ವಿನಿರ್ಮುಕ್ತಾ ಸದಾಶಿವ-ಪತಿವ್ರತಾ~।\\
ಸಂಪ್ರದಾಯೇಶ್ವರೀ ಸಾಧ್ವೀ ಗುರುಮಂಡಲ-ರೂಪಿಣೀ ॥೧೩೮॥

ಕುಲೋತ್ತೀರ್ಣಾ ಭಗಾರಾಧ್ಯಾ ಮಾಯಾ ಮಧುಮತೀ ಮಹೀ~।\\
ಗಣಾಂಬಾ ಗುಹ್ಯಕಾರಾಧ್ಯಾ ಕೋಮಲಾಂಗೀ ಗುರುಪ್ರಿಯಾ ॥೧೩೯॥

ಸ್ವತಂತ್ರಾ ಸರ್ವತಂತ್ರೇಶೀ ದಕ್ಷಿಣಾಮೂರ್ತಿ-ರೂಪಿಣೀ~।\\
ಸನಕಾದಿ-ಸಮಾರಾಧ್ಯಾ ಶಿವಜ್ಞಾನ-ಪ್ರದಾಯಿನೀ ॥೧೪೦॥

ಚಿತ್ಕಲಾಽಽನಂದ-ಕಲಿಕಾ ಪ್ರೇಮರೂಪಾ ಪ್ರಿಯಂಕರೀ~।\\
ನಾಮಪಾರಾಯಣ-ಪ್ರೀತಾ ನಂದಿವಿದ್ಯಾ ನಟೇಶ್ವರೀ ॥೧೪೧॥

ಮಿಥ್ಯಾ-ಜಗದಧಿಷ್ಠಾನಾ ಮುಕ್ತಿದಾ ಮುಕ್ತಿರೂಪಿಣೀ~।\\
ಲಾಸ್ಯಪ್ರಿಯಾ ಲಯಕರೀ ಲಜ್ಜಾ ರಂಭಾದಿವಂದಿತಾ ॥೧೪೨॥

ಭವದಾವ-ಸುಧಾವೃಷ್ಟಿಃ ಪಾಪಾರಣ್ಯ-ದವಾನಲಾ~।\\
ದೌರ್ಭಾಗ್ಯ-ತೂಲವಾತೂಲಾ ಜರಾಧ್ವಾಂತ-ರವಿಪ್ರಭಾ ॥೧೪೩॥

ಭಾಗ್ಯಾಬ್ಧಿ-ಚಂದ್ರಿಕಾ ಭಕ್ತ-ಚಿತ್ತಕೇಕಿ-ಘನಾಘನಾ~।\\
ರೋಗಪರ್ವತ-ದಂಭೋಲಿರ್ಮೃತ್ಯುದಾರು-ಕುಠಾರಿಕಾ ॥೧೪೪॥

ಮಹೇಶ್ವರೀ ಮಹಾಕಾಲೀ ಮಹಾಗ್ರಾಸಾ ಮಹಾಶನಾ~।\\
ಅಪರ್ಣಾ ಚಂಡಿಕಾ ಚಂಡಮುಂಡಾಸುರ-ನಿಷೂದಿನೀ ॥೧೪೫॥

ಕ್ಷರಾಕ್ಷರಾತ್ಮಿಕಾ ಸರ್ವ-ಲೋಕೇಶೀ ವಿಶ್ವಧಾರಿಣೀ~।\\
ತ್ರಿವರ್ಗದಾತ್ರೀ ಸುಭಗಾ ತ್ರ್ಯಂಬಕಾ ತ್ರಿಗುಣಾತ್ಮಿಕಾ ॥೧೪೬॥

ಸ್ವರ್ಗಾಪವರ್ಗದಾ ಶುದ್ಧಾ ಜಪಾಪುಷ್ಪ-ನಿಭಾಕೃತಿಃ~।\\
ಓಜೋವತೀ ದ್ಯುತಿಧರಾ ಯಜ್ಞರೂಪಾ ಪ್ರಿಯವ್ರತಾ ॥೧೪೭॥

ದುರಾರಾಧ್ಯಾ ದುರಾಧರ್ಷಾ ಪಾಟಲೀ-ಕುಸುಮ-ಪ್ರಿಯಾ~।\\
ಮಹತೀ ಮೇರುನಿಲಯಾ ಮಂದಾರ-ಕುಸುಮ-ಪ್ರಿಯಾ ॥೧೪೮॥

ವೀರಾರಾಧ್ಯಾ ವಿರಾಡ್ರೂಪಾ ವಿರಜಾ ವಿಶ್ವತೋಮುಖೀ~।\\
ಪ್ರತ್ಯಗ್ರೂಪಾ ಪರಾಕಾಶಾ ಪ್ರಾಣದಾ ಪ್ರಾಣರೂಪಿಣೀ ॥೧೪೯॥

ಮಾರ್ತಾಂಡ-ಭೈರವಾರಾಧ್ಯಾ ಮಂತ್ರಿಣೀನ್ಯಸ್ತ-ರಾಜ್ಯಧೂಃ~।\\
ತ್ರಿಪುರೇಶೀ ಜಯತ್ಸೇನಾ ನಿಸ್ತ್ರೈಗುಣ್ಯಾ ಪರಾಪರಾ ॥೧೫೦॥

ಸತ್ಯ-ಜ್ಞಾನಾನಂದ-ರೂಪಾ ಸಾಮರಸ್ಯ-ಪರಾಯಣಾ~।\\
ಕಪರ್ದಿನೀ ಕಲಾಮಾಲಾ ಕಾಮಧುಕ್ ಕಾಮರೂಪಿಣೀ ॥೧೫೧॥

ಕಲಾನಿಧಿಃ ಕಾವ್ಯಕಲಾ ರಸಜ್ಞಾ ರಸಶೇವಧಿಃ \as{(೮೦೦)}~।\\
ಪುಷ್ಟಾ ಪುರಾತನಾ ಪೂಜ್ಯಾ ಪುಷ್ಕರಾ ಪುಷ್ಕರೇಕ್ಷಣಾ ॥೧೫೨॥

ಪರಂಜ್ಯೋತಿಃ ಪರಂಧಾಮ ಪರಮಾಣುಃ ಪರಾತ್ಪರಾ~।\\
ಪಾಶಹಸ್ತಾ ಪಾಶಹಂತ್ರೀ ಪರಮಂತ್ರ-ವಿಭೇದಿನೀ ॥೧೫೩॥

ಮೂರ್ತಾಽಮೂರ್ತಾಽನಿತ್ಯತೃಪ್ತಾ ಮುನಿಮಾನಸ-ಹಂಸಿಕಾ~।\\
ಸತ್ಯವ್ರತಾ ಸತ್ಯರೂಪಾ ಸರ್ವಾಂತರ್ಯಾಮಿನೀ ಸತೀ ॥೧೫೪॥

ಬ್ರಹ್ಮಾಣೀ ಬ್ರಹ್ಮಜನನೀ ಬಹುರೂಪಾ ಬುಧಾರ್ಚಿತಾ~।\\
ಪ್ರಸವಿತ್ರೀ ಪ್ರಚಂಡಾಽಽಜ್ಞಾ ಪ್ರತಿಷ್ಠಾ ಪ್ರಕಟಾಕೃತಿಃ ॥೧೫೫॥

ಪ್ರಾಣೇಶ್ವರೀ ಪ್ರಾಣದಾತ್ರೀ ಪಂಚಾಶತ್ಪೀಠ-ರೂಪಿಣೀ~।\\
ವಿಶೃಂಖಲಾ ವಿವಿಕ್ತಸ್ಥಾ ವೀರಮಾತಾ ವಿಯತ್ಪ್ರಸೂಃ ॥೧೫೬॥

ಮುಕುಂದಾ ಮುಕ್ತಿನಿಲಯಾ ಮೂಲವಿಗ್ರಹ-ರೂಪಿಣೀ~।\\
ಭಾವಜ್ಞಾ ಭವರೋಗಘ್ನೀ ಭವಚಕ್ರ-ಪ್ರವರ್ತಿನೀ ॥೧೫೭॥

ಛಂದಃಸಾರಾ ಶಾಸ್ತ್ರಸಾರಾ ಮಂತ್ರಸಾರಾ ತಲೋದರೀ~।\\
ಉದಾರಕೀರ್ತಿರುದ್ದಾಮವೈಭವಾ ವರ್ಣರೂಪಿಣೀ ॥೧೫೮॥

ಜನ್ಮಮೃತ್ಯು-ಜರಾತಪ್ತ-ಜನವಿಶ್ರಾಂತಿ-ದಾಯಿನೀ~।\\
ಸರ್ವೋಪನಿಷದುದ್ಘುಷ್ಟಾ ಶಾಂತ್ಯತೀತ-ಕಲಾತ್ಮಿಕಾ ॥೧೫೯॥

ಗಂಭೀರಾ ಗಗನಾಂತಸ್ಥಾ ಗರ್ವಿತಾ ಗಾನಲೋಲುಪಾ~।\\
ಕಲ್ಪನಾ-ರಹಿತಾ ಕಾಷ್ಠಾಽಕಾಂತಾ ಕಾಂತಾರ್ಧ-ವಿಗ್ರಹಾ ॥೧೬೦॥

ಕಾರ್ಯಕಾರಣ-ನಿರ್ಮುಕ್ತಾ ಕಾಮಕೇಲಿ-ತರಂಗಿತಾ~।\\
ಕನತ್ಕನಕತಾ-ಟಂಕಾ ಲೀಲಾ-ವಿಗ್ರಹ-ಧಾರಿಣೀ ॥೧೬೧॥

ಅಜಾ ಕ್ಷಯವಿನಿರ್ಮುಕ್ತಾ ಮುಗ್ಧಾ ಕ್ಷಿಪ್ರ-ಪ್ರಸಾದಿನೀ~।\\
ಅಂತರ್ಮುಖ-ಸಮಾರಾಧ್ಯಾ ಬಹಿರ್ಮುಖ-ಸುದುರ್ಲಭಾ ॥೧೬೨॥

ತ್ರಯೀ ತ್ರಿವರ್ಗನಿಲಯಾ ತ್ರಿಸ್ಥಾ ತ್ರಿಪುರಮಾಲಿನೀ~।\\
ನಿರಾಮಯಾ ನಿರಾಲಂಬಾ ಸ್ವಾತ್ಮಾರಾಮಾ ಸುಧಾಸೃತಿಃ ॥೧೬೩॥

ಸಂಸಾರಪಂಕ-ನಿರ್ಮಗ್ನ-ಸಮುದ್ಧರಣ-ಪಂಡಿತಾ~।\\
ಯಜ್ಞಪ್ರಿಯಾ ಯಜ್ಞಕರ್ತ್ರೀ ಯಜಮಾನ-ಸ್ವರೂಪಿಣೀ ॥೧೬೪॥

ಧರ್ಮಾಧಾರಾ ಧನಾಧ್ಯಕ್ಷಾ ಧನಧಾನ್ಯ-ವಿವರ್ಧಿನೀ~।\\
ವಿಪ್ರಪ್ರಿಯಾ ವಿಪ್ರರೂಪಾ ವಿಶ್ವಭ್ರಮಣ-ಕಾರಿಣೀ ॥೧೬೫॥

ವಿಶ್ವಗ್ರಾಸಾ ವಿದ್ರುಮಾಭಾ ವೈಷ್ಣವೀ ವಿಷ್ಣುರೂಪಿಣೀ~।\\
ಅಯೋನಿರ್ಯೋನಿನಿಲಯಾ ಕೂಟಸ್ಥಾ ಕುಲರೂಪಿಣೀ ॥೧೬೬॥

ವೀರಗೋಷ್ಠೀಪ್ರಿಯಾ ವೀರಾ ನೈಷ್ಕರ್ಮ್ಯಾ \as{(೯೦೦)} ನಾದರೂಪಿಣೀ~।\\
ವಿಜ್ಞಾನಕಲನಾ ಕಲ್ಯಾ ವಿದಗ್ಧಾ ಬೈಂದವಾಸನಾ ॥೧೬೭॥

ತತ್ತ್ವಾಧಿಕಾ ತತ್ತ್ವಮಯೀ ತತ್ತ್ವಮರ್ಥ-ಸ್ವರೂಪಿಣೀ~।\\
ಸಾಮಗಾನಪ್ರಿಯಾ ಸೌಮ್ಯಾ ಸದಾಶಿವ-ಕುಟುಂಬಿನೀ ॥೧೬೮॥

ಸವ್ಯಾಪಸವ್ಯ-ಮಾರ್ಗಸ್ಥಾ ಸರ್ವಾಪದ್ವಿನಿವಾರಿಣೀ~।\\
ಸ್ವಸ್ಥಾ ಸ್ವಭಾವಮಧುರಾ ಧೀರಾ ಧೀರಸಮರ್ಚಿತಾ ॥೧೬೯॥

ಚೈತನ್ಯಾರ್ಘ್ಯ-ಸಮಾರಾಧ್ಯಾ ಚೈತನ್ಯ-ಕುಸುಮಪ್ರಿಯಾ~।\\
ಸದೋದಿತಾ ಸದಾತುಷ್ಟಾ ತರುಣಾದಿತ್ಯ-ಪಾಟಲಾ ॥೧೭೦॥

ದಕ್ಷಿಣಾ-ದಕ್ಷಿಣಾರಾಧ್ಯಾ ದರಸ್ಮೇರ-ಮುಖಾಂಬುಜಾ~।\\
ಕೌಲಿನೀ-ಕೇವಲಾಽನರ್ಘ್ಯ-ಕೈವಲ್ಯ-ಪದದಾಯಿನೀ ॥೧೭೧॥

ಸ್ತೋತ್ರಪ್ರಿಯಾ ಸ್ತುತಿಮತೀ ಶ್ರುತಿ-ಸಂಸ್ತುತ-ವೈಭವಾ~।\\
ಮನಸ್ವಿನೀ ಮಾನವತೀ ಮಹೇಶೀ ಮಂಗಲಾಕೃತಿಃ ॥೧೭೨॥

ವಿಶ್ವಮಾತಾ ಜಗದ್ಧಾತ್ರೀ ವಿಶಾಲಾಕ್ಷೀ ವಿರಾಗಿಣೀ~।\\
ಪ್ರಗಲ್ಭಾ ಪರಮೋದಾರಾ ಪರಾಮೋದಾ ಮನೋಮಯೀ ॥೧೭೩॥

ವ್ಯೋಮಕೇಶೀ ವಿಮಾನಸ್ಥಾ ವಜ್ರಿಣೀ ವಾಮಕೇಶ್ವರೀ~।\\
ಪಂಚಯಜ್ಞ-ಪ್ರಿಯಾ ಪಂಚ-ಪ್ರೇತ-ಮಂಚಾಧಿಶಾಯಿನೀ ॥೧೭೪॥

ಪಂಚಮೀ ಪಂಚಭೂತೇಶೀ ಪಂಚ-ಸಂಖ್ಯೋಪಚಾರಿಣೀ~।\\
ಶಾಶ್ವತೀ ಶಾಶ್ವತೈಶ್ವರ್ಯಾ ಶರ್ಮದಾ ಶಂಭುಮೋಹಿನೀ ॥೧೭೫॥

ಧರಾ ಧರಸುತಾ ಧನ್ಯಾ ಧರ್ಮಿಣೀ ಧರ್ಮವರ್ಧಿನೀ~।\\
ಲೋಕಾತೀತಾ ಗುಣಾತೀತಾ ಸರ್ವಾತೀತಾ ಶಮಾತ್ಮಿಕಾ ॥೧೭೬॥

ಬಂಧೂಕ-ಕುಸುಮಪ್ರಖ್ಯಾ ಬಾಲಾ ಲೀಲಾವಿನೋದಿನೀ~।\\
ಸುಮಂಗಲೀ ಸುಖಕರೀ ಸುವೇಷಾಢ್ಯಾ ಸುವಾಸಿನೀ ॥೧೭೭॥

ಸುವಾಸಿನ್ಯರ್ಚನ-ಪ್ರೀತಾಽಽಶೋಭನಾ ಶುದ್ಧಮಾನಸಾ~।\\
ಬಿಂದು-ತರ್ಪಣ-ಸಂತುಷ್ಟಾ ಪೂರ್ವಜಾ ತ್ರಿಪುರಾಂಬಿಕಾ ॥೧೭೮॥

ದಶಮುದ್ರಾ-ಸಮಾರಾಧ್ಯಾ ತ್ರಿಪುರಾಶ್ರೀ-ವಶಂಕರೀ~।\\
ಜ್ಞಾನಮುದ್ರಾ ಜ್ಞಾನಗಮ್ಯಾ ಜ್ಞಾನಜ್ಞೇಯ-ಸ್ವರೂಪಿಣೀ ॥೧೭೯॥

ಯೋನಿಮುದ್ರಾ ತ್ರಿಖಂಡೇಶೀ ತ್ರಿಗುಣಾಂಬಾ ತ್ರಿಕೋಣಗಾ~।\\
ಅನಘಾಽದ್ಭುತ-ಚಾರಿತ್ರಾ ವಾಂಛಿತಾರ್ಥ-ಪ್ರದಾಯಿನೀ ॥೧೮೦॥

ಅಭ್ಯಾಸಾತಿಶಯ-ಜ್ಞಾತಾ ಷಡಧ್ವಾತೀತ-ರೂಪಿಣೀ~।\\
ಅವ್ಯಾಜ-ಕರುಣಾ-ಮೂರ್ತಿರಜ್ಞಾನ-ಧ್ವಾಂತ-ದೀಪಿಕಾ ॥೧೮೧॥

ಆಬಾಲ-ಗೋಪ-ವಿದಿತಾ ಸರ್ವಾನುಲ್ಲಂಘ್ಯ-ಶಾಸನಾ~।\\
ಶ್ರೀಚಕ್ರರಾಜ-ನಿಲಯಾ ಶ್ರೀಮತ್-ತ್ರಿಪುರಸುಂದರೀ ॥೧೮೨॥

ಶ್ರೀಶಿವಾ ಶಿವ-ಶಕ್ತ್ಯೈಕ್ಯ-ರೂಪಿಣೀ ಲಲಿತಾಂಬಿಕಾ\as{(೧೦೦೦)। ಶ್ರೀಂಹ್ರೀಂಐಂ ಓಂ}\\
ಏವಂ ಶ್ರೀಲಲಿತಾ ದೇವ್ಯಾ ನಾಮ್ನಾಂ ಸಾಹಸ್ರಕಂ ಜಗುಃ ॥೧೮೩॥
\section{ಸೀತಾಷ್ಟೋತ್ತರಶತನಾಮಸ್ತೋತ್ರಂ}
\addcontentsline{toc}{section}{ಸೀತಾಷ್ಟೋತ್ತರಶತನಾಮಸ್ತೋತ್ರಂ}
ಸೀತಾ ಸೀರಧ್ವಜಸುತಾ ಸೀಮಾತೀತಗುಣೋಜ್ಜ್ವಲಾ ।\\
ಸೌಂದರ್ಯಸಾರಸರ್ವಸ್ವಭೂತಾ ಸೌಭಾಗ್ಯದಾಯಿನೀ ॥೧॥

ದೇವೀ ದೇವಾರ್ಚಿತಪದಾ ದಿವ್ಯಾ ದಶರಥಸ್ನುಷಾ ।\\
ರಾಮಾ ರಾಮಪ್ರಿಯಾ ರಮ್ಯಾ ರಾಕೇಂದುವದನೋಜ್ವಲಾ ॥೨॥

ವೀರ್ಯಶುಲ್ಕಾ ವೀರಪತ್ನೀ ವಿಯನ್ಮಧ್ಯಾ ವರಪ್ರದಾ ।\\
ಪತಿವ್ರತಾ ಪಂಕ್ತಿಕಂಠನಾಶಿನೀ ಪಾವನಸ್ಮೃತಿಃ ॥೩॥

ವಂದಾರುವತ್ಸಲಾ ವೀರಮಾತಾ ವೃತರಘೂತ್ತಮಾ ।\\
ಸಂಪತ್ಕರೀ ಸದಾತುಷ್ಟಾ ಸಾಕ್ಷಿಣೀ ಸಾಧುಸಮ್ಮತಾ ॥೪॥

ನಿತ್ಯಾ ನಿಯತಸಂಸ್ಥಾನಾ ನಿತ್ಯಾನಂದಾ ನುತಿಪ್ರಿಯಾ ।\\
ಪೃಥ್ವೀ ಪೃಥ್ವೀಸುತಾ ಪುತ್ರದಾಯಿನೀ ಪ್ರಕೃತಿಃ ಪರಾ ॥೫॥

ಹನುಮತ್ಸ್ವಾಮಿನೀ ಹೃದ್ಯಾ ಹೃದಯಸ್ಥಾ ಹತಾಶುಭಾ ।\\
ಹಂಸಯುಕ್ತಾ ಹಂಸಗತಿಃ ಹರ್ಷಯುಕ್ತಾ ಹತಾಸುರಾ ॥೬॥

ಸಾರರೂಪಾ ಸಾರವಚಾಃ ಸಾಧ್ವೀ ಚ ಸರಮಾಪ್ರಿಯಾ ।\\
ತ್ರಿಲೋಕವಂದ್ಯಾ ತ್ರಿಜಟಾಸೇವ್ಯಾ ತ್ರಿಪಥಗಾರ್ಚಿನೀ ॥೭॥

ತ್ರಾಣಪ್ರದಾ ತ್ರಾತಕಾಕಾ ತೃಣೀಕೃತದಶಾನನಾ ।\\
ಅನಸೂಯಾಂಗರಾಗಾಂಕಾಽನಸೂಯಾ ಸುರವಂದಿತಾ ॥೮॥

ಅಶೋಕವನಿಕಾಸ್ಥಾನಾಽಶೋಕಾ ಶೋಕವಿನಾಶಿನೀ ।\\
ಸೂರ್ಯವಂಶಸ್ನುಷಾ ಸೂರ್ಯಮಂಡಲಾಂತಃಸ್ಥವಲ್ಲಭಾ ॥೯॥
\newpage
ಶ್ರುತಮಾತ್ರಾಘಹರಣಾ ಶ್ರುತಿಸನ್ನಿಹಿತೇಕ್ಷಣಾ ।\\
ಪುಷ್ಪಪ್ರಿಯಾ ಪುಷ್ಪಕಸ್ಥಾ ಪುಣ್ಯಲಭ್ಯಾ ಪುರಾತನಾ ॥೧೦॥

ಪುರುಷಾರ್ಥಪ್ರದಾ ಪೂಜ್ಯಾ ಪೂತನಾಮ್ನೀ ಪರಂತಪಾ ।\\
ಪದ್ಮಪ್ರಿಯಾ ಪದ್ಮಹಸ್ತಾ ಪದ್ಮಾ ಪದ್ಮಮುಖೀ ಶುಭಾ ॥೧೧॥

ಜನಶೋಕಹರಾ ಜನ್ಮಮೃತ್ಯುಶೋಕವಿನಾಶಿನೀ ।\\
ಜಗದ್ರೂಪಾ ಜಗದ್ವಂದ್ಯಾ ಜಯದಾ ಜನಕಾತ್ಮಜಾ ॥೧೨॥

ನಾಥನೀಯಕಟಾಕ್ಷಾ ಚ ನಾಥಾ ನಾಥೈಕತತ್ಪರಾ ।\\
ನಕ್ಷತ್ರನಾಥವದನಾ ನಷ್ಟದೋಷಾ ನಯಾವಹಾ ॥೧೩॥

ವಹ್ನಿಪಾಪಹರಾ ವಹ್ನಿಶೈತ್ಯಕೃದ್ವೃದ್ಧಿದಾಯಿನೀ ।\\
ವಾಲ್ಮೀಕಿಗೀತವಿಭವಾ ವಚೋಽತೀತಾ ವರಾಂಗನಾ ॥೧೪॥

ಭಕ್ತಿಗಮ್ಯಾ ಭವ್ಯಗುಣಾ ಭಾಂತೀ ಭರತವಂದಿತಾ ।\\
ಸುವರ್ಣಾಂಗೀ ಸುಖಕರೀ ಸುಗ್ರೀವಾಂಗದಸೇವಿತಾ ॥೧೫॥

ವೈದೇಹೀ ವಿನತಾಘೌಘನಾಶಿನೀ ವಿಧಿವಂದಿತಾ ।\\
ಲೋಕಮಾತಾ ಲೋಚನಾಂತಃಸ್ಥಿತಕಾರುಣ್ಯಸಾಗರಾ ॥\\
ಶ್ರೀರಾಮವಲ್ಲಭಾ ಸಾ ನಃ ಪಾಯಾದಾರ್ತಾನುಪಾಶ್ರಿತಾನ್ ॥೧೬॥

ಕೃತಾಕೃತಜಗದ್ಧೇತುಃ ಕೃತರಾಜ್ಯಾಭಿಷೇಕಕಾ ।\\
ಇದಮಷ್ಟೋತ್ತರಶತಂ ಸೀತಾನಾಮ್ನಾಂ ತು ಯಾ ವಧೂಃ ॥೧೭॥

ಧನಧಾನ್ಯಸಮೃದ್ಧಾ ಚ ದೀರ್ಘಸೌಭಾಗ್ಯದರ್ಶಿನೀ ।\\
ಪುಂಸಾಮಪಿ ಸ್ತೋತ್ರಮಿದಂ ಪಠನಾತ್ಸರ್ವಸಿದ್ಧಿದಂ ॥೧೮॥
\authorline{ಇತಿ ಬ್ರಹ್ಮಯಾಮಲೇ ರಾಮರಹಸ್ಯಗತಂ ಸೀತಾಷ್ಟೋತ್ತರಶತನಾಮಸ್ತೋತ್ರಂ ಸಂಪೂರ್ಣಂ ।}
ಓಂ ಜನಕಜಾಯೈ ವಿದ್ಮಹೇ ರಾಮಪ್ರಿಯಾಯೈ ಧೀಮಹಿ । ತನ್ನಃ ಸೀತಾ ಪ್ರಚೋದಯಾತ್॥
ಓಂ ಜನಕನಂದಿನ್ಯೈ ವಿದ್ಮಹೇ ಭೂಮಿಜಾಯೈ ಚ ಧೀಮಹಿ । ತನ್ನಃ ಸೀತಾ ಪ್ರಚೋದಯಾತ್॥
\section{ಶ್ರೀದುರ್ಗಾಷ್ಟೋತ್ತರಶತನಾಮಸ್ತೋತ್ರಂ}
\addcontentsline{toc}{section}{ಶ್ರೀದುರ್ಗಾಷ್ಟೋತ್ತರಶತನಾಮಸ್ತೋತ್ರಂ}

ಅಸ್ಯಶ್ರೀ ದುರ್ಗಾಷ್ಟೋತ್ತರಶತನಾಮಾಸ್ತೋತ್ರ ಮಾಲಾಮಂತ್ರಸ್ಯ ಬ್ರಹ್ಮವಿಷ್ಣುಮಹೇಶ್ವರಾಃ ಋಷಯಃ । ಅನುಷ್ಟುಪ್ಛಂದಃ । ಶ್ರೀದುರ್ಗಾಪರಮೇಶ್ವರೀ ದೇವತಾ । ಹ್ರಾಂ ಬೀಜಂ । ಹ್ರೀಂ ಶಕ್ತಿಃ । ಹ್ರೂಂ ಕೀಲಕಂ । ಸರ್ವಾಭೀಷ್ಟಸಿಧ್ಯರ್ಥೇ ಜಪಹೋಮಾರ್ಚನೇ ವಿನಿಯೋಗಃ ।

ಓಂ ಸತ್ಯಾ ಸಾಧ್ಯಾ ಭವಪ್ರೀತಾ ಭವಾನೀ ಭವಮೋಚನೀ ।\\
ಆರ್ಯಾ ದುರ್ಗಾ ಜಯಾ ಚಾದ್ಯಾ ತ್ರಿಣೇತ್ರಾಶೂಲಧಾರಿಣೀ ॥

ಪಿನಾಕಧಾರಿಣೀ ಚಿತ್ರಾ ಚಂಡಘಂಟಾ ಮಹಾತಪಾಃ ।\\
ಮನೋ ಬುದ್ಧಿ ರಹಂಕಾರಾ ಚಿದ್ರೂಪಾ ಚ ಚಿದಾಕೃತಿಃ ॥

ಅನಂತಾ ಭಾವಿನೀ ಭವ್ಯಾ ಹ್ಯಭವ್ಯಾ ಚ ಸದಾಗತಿಃ ।\\
ಶಾಂಭವೀ ದೇವಮಾತಾ ಚ ಚಿಂತಾ ರತ್ನಪ್ರಿಯಾ ತಥಾ ॥

ಸರ್ವವಿದ್ಯಾ ದಕ್ಷಕನ್ಯಾ ದಕ್ಷಯಜ್ಞವಿನಾಶಿನೀ ।\\
ಅಪರ್ಣಾಽನೇಕವರ್ಣಾ ಚ ಪಾಟಲಾ ಪಾಟಲಾವತೀ ॥

ಪಟ್ಟಾಂಬರಪರೀಧಾನಾ ಕಲಮಂಜೀರರಂಜಿನೀ ।\\
ಈಶಾನೀ ಚ ಮಹಾರಾಜ್ಞೀ ಹ್ಯಪ್ರಮೇಯಪರಾಕ್ರಮಾ ।\\
ರುದ್ರಾಣೀ ಕ್ರೂರರೂಪಾ ಚ ಸುಂದರೀ ಸುರಸುಂದರೀ ॥

ವನದುರ್ಗಾ ಚ ಮಾತಂಗೀ ಮತಂಗಮುನಿಕನ್ಯಕಾ ।\\
ಬ್ರಾಹ್ಮೀ ಮಾಹೇಶ್ವರೀ ಚೈಂದ್ರೀ ಕೌಮಾರೀ ವೈಷ್ಣವೀ ತಥಾ ॥

ಚಾಮುಂಡಾ ಚೈವ ವಾರಾಹೀ ಲಕ್ಷ್ಮೀಶ್ಚ ಪುರುಷಾಕೃತಿಃ ।\\
ವಿಮಲಾ ಜ್ಞಾನರೂಪಾ ಚ ಕ್ರಿಯಾ ನಿತ್ಯಾ ಚ ಬುದ್ಧಿದಾ ॥

ಬಹುಲಾ ಬಹುಲಪ್ರೇಮಾ ಮಹಿಷಾಸುರಮರ್ದಿನೀ ।\\
ಮಧುಕೈಟಭ ಹಂತ್ರೀ ಚ ಚಂಡಮುಂಡವಿನಾಶಿನೀ ॥

ಸರ್ವಶಾಸ್ತ್ರಮಯೀ ಚೈವ ಸರ್ವದಾನವಘಾತಿನೀ ।\\
ಅನೇಕಶಸ್ತ್ರಹಸ್ತಾ ಚ ಸರ್ವಶಸ್ತ್ರಾಸ್ತ್ರಧಾರಿಣೀ ॥

ಭದ್ರಕಾಲೀ ಸದಾಕನ್ಯಾ ಕೈಶೋರೀ ಯುವತಿರ್ಯತಿಃ ।\\
ಪ್ರೌಢಾಽಪ್ರೌಢಾ ವೃದ್ಧಮಾತಾ ಘೋರರೂಪಾ ಮಹೋದರೀ ॥

ಬಲಪ್ರದಾ ಘೋರರೂಪಾ ಮಹೋತ್ಸಾಹಾ ಮಹಾಬಲಾ ।\\
ಅಗ್ನಿಜ್ವಾಲಾ ರೌದ್ರಮುಖೀ ಕಾಲರಾತ್ರಿಃ ತಪಸ್ವಿನೀ ॥

ನಾರಾಯಣೀ ಮಹಾದೇವೀ ವಿಷ್ಣುಮಾಯಾ ಶಿವಾತ್ಮಿಕಾ ।\\
ಶಿವದೂತೀ ಕರಾಲೀ ಚ ಹ್ಯನಂತಾ ಪರಮೇಶ್ವರೀ ॥

ಕಾತ್ಯಾಯನೀ ಮಹಾವಿದ್ಯಾ ಮಹಾಮೇಧಾಸ್ವರೂಪಿಣೀ ।\\
ಗೌರೀ ಸರಸ್ವತೀ ಚೈವ ಸಾವಿತ್ರೀ ಬ್ರಹ್ಮವಾದಿನೀ ।\\
ಸರ್ವತತ್ತ್ವೈಕನಿಲಯಾ ವೇದಮಂತ್ರಸ್ವರೂಪಿಣೀ ॥

\section{ಶ್ರೀರಾಮಾಷ್ಟೋತ್ತರಶತನಾಮಸ್ತೋತ್ರಂ}
\addcontentsline{toc}{section}{ಶ್ರೀರಾಮಾಷ್ಟೋತ್ತರಶತನಾಮಸ್ತೋತ್ರಂ}

ರಾಮೋ ರಾವಣಸಂಹಾರಕೃತಮಾನುಷವಿಗ್ರಹಃ ।\\
ಕೌಸಲ್ಯಾಸುಕೃತವ್ರಾತಫಲಂ ದಶರಥಾತ್ಮಜಃ ॥೧॥

ಲಕ್ಷ್ಮಣಾರ್ಚಿತಪಾದಾಬ್ಜಃ ಸರ್ವಲೋಕಪ್ರಿಯಂಕರಃ ।\\
ಸಾಕೇತವಾಸಿನೇತ್ರಾಬ್ಜ ಸಂಪ್ರೀಣನದಿವಾಕರಃ ॥೨॥

ವಿಶ್ವಾಮಿತ್ರಪ್ರಿಯಶ್ಶಾಂತಃ ತಾಟಕಾಧ್ವಾಂತಭಾಸ್ಕರಃ ।\\
ಸುಬಾಹುರಾಕ್ಷಸರಿಪುಃ ಕೌಶಿಕಾಧ್ವರಪಾಲಕಃ ॥೩॥

ಅಹಲ್ಯಾಪಾಪಸಂಹರ್ತಾ ಜನಕೇಂದ್ರಪ್ರಿಯಾತಿಥಿಃ ।\\
ಪುರಾರಿಚಾಪದಲನೋ ವೀರಲಕ್ಷ್ಮೀಸಮಾಶ್ರಯಃ ॥೪॥

ಸೀತಾವರಣಮಾಲ್ಯಾಢ್ಯೋ ಜಾಮದಗ್ನ್ಯಮದಾಪಹಃ ।\\
ವೈದೇಹೀಕೃತಶೃಂಗಾರಃ ಪಿತೃಪ್ರೀತಿವಿವರ್ಧನಃ ॥೫॥

ತಾತಾಜ್ಞೋತ್ಸೃಷ್ಟಹಸ್ತಸ್ಥರಾಜ್ಯಸ್ಸತ್ಯಪ್ರತಿಶ್ರವಃ ।\\
ತಮಸಾತೀರಸಂವಾಸೀ ಗುಹಾನುಗ್ರಹತತ್ಪರಃ ॥೬॥

ಸುಮಂತ್ರಸೇವಿತಪದೋ ಭರದ್ವಾಜಪ್ರಿಯಾತಿಥಿಃ ।\\
ಚಿತ್ರಕೂಟಪ್ರಿಯಾವಾಸಃ ಪಾದುಕಾನ್ಯಸ್ತಭೂಭರಃ ॥೭॥

ಅನಸೂಯಾಂಗರಾಗಾಂಕಸೀತಾಸಾಹಿತ್ಯಶೋಭಿತಃ ।\\
ದಂಡಕಾರಣ್ಯಸಂಚಾರೀ ವಿರಾಧಸ್ವರ್ಗದಾಯಕಃ ॥೮॥

ರಕ್ಷಃಕಾಲಾಂತಕಸ್ಸರ್ವಮುನಿಸಂಘಮುದಾವಹಃ ।\\
ಪ್ರತಿಜ್ಞಾತಾಸುರವಧಃ ಶರಭಂಗಗತಿಪ್ರದಃ ॥೯॥

ಅಗಸ್ತ್ಯಾರ್ಪಿತಬಾಣಾಸಖಡ್ಗತೂಣೀರಮಂಡಿತಃ ।\\
ಪ್ರಾಪ್ತಪಂಚವಟೀವಾಸೋ ಗೃಧ್ರರಾಜಸಹಾಯವಾನ್ ॥೧೦॥

ಕಾಮಿಶೂರ್ಪಣಖಾಕರ್ಣನಾಸಾಚ್ಛೇದನಿಯಾಮಕಃ ।\\
ಖರಾದಿರಾಕ್ಷಸವ್ರಾತಖಂಡನಾವಿತಸಜ್ಜನಃ ॥೧೧॥

ಸೀತಾಸಂಶ್ಲಿಷ್ಟಕಾಯಾಭಾಜಿತವಿದ್ಯುದ್ಯುತಾಂಬುದಃ ।\\
ಮಾರೀಚಹಂತಾ ಮಾಯಾಢ್ಯೋ ಜಟಾಯುರ್ಮೋಕ್ಷದಾಯಕಃ ॥೧೨॥

ಕಬಂಧಬಾಹುದಲನಶ್ಶಬರೀಪ್ರಾರ್ಥಿತಾತಿಥಿಃ ।\\
ಹನುಮದ್ವಂದಿತಪದಸ್ಸುಗ್ರೀವಸುಹೃದವ್ಯಯಃ ॥೧೩॥

ದೈತ್ಯಕಂಕಾಲವಿಕ್ಷೇಪೀ ಸಪ್ತತಾಲಪ್ರಭೇದಕಃ ।\\
ಏಕೇಷುಹತವಾಲೀ ಚ ತಾರಾಸಂಸ್ತುತಸದ್ಗುಣಃ ॥೧೪॥

ಕಪೀಂದ್ರೀಕೃತಸುಗ್ರೀವಸ್ಸರ್ವವಾನರಪೂಜಿತಃ ।\\
ವಾಯುಸೂನುಸಮಾನೀತಸೀತಾಸಂದೇಶನಂದಿತಃ ॥೧೫॥

ಜೈತ್ರಯಾತ್ರೋತ್ಸವಃ ಜಿಷ್ಣುರ್ವಿಷ್ಣುರೂಪೋ ನಿರಾಕೃತಿಃ ।\\
ಕಂಪಿತಾಂಭೋನಿಧಿಸ್ಸಂಪತ್ಪ್ರದಸ್ಸೇತುನಿಬಂಧನಃ ॥೧೬॥

ಲಂಕಾವಿಭೇದನಪಟುರ್ನಿಶಾಚರವಿನಾಶಕಃ ।\\
ಕುಂಭಕರ್ಣಾಖ್ಯಕುಂಭೀಂದ್ರಮೃಗರಾಜಪರಾಕ್ರಮಃ ॥೧೭॥

ಮೇಘನಾದವಧೋದ್ಯುಕ್ತಲಕ್ಷ್ಮಣಾಸ್ತ್ರಬಲಪ್ರದಃ ।\\
ದಶಗ್ರೀವಾಂಧತಾಮಿಸ್ರಪ್ರಮಾಪಣಪ್ರಭಾಕರಃ ॥೧೮॥

ಇಂದ್ರಾದಿದೇವತಾಸ್ತುತ್ಯಶ್ಚಂದ್ರಾಭಮುಖಮಂಡಲಃ ।\\
ವಿಭೀಷಣಾರ್ಪಿತನಿಶಾಚರರಾಜ್ಯೋ ವೃಷಪ್ರಿಯಃ ॥೧೯॥

ವೈಶ್ವಾನರಸ್ತುತಗುಣಾವನಿಪುತ್ರೀಸಮಾಗತಃ ।\\
ಪುಷ್ಪಕಸ್ಥಾನಸುಭಗಃ ಪುಣ್ಯವತ್ಪ್ರಾಪ್ಯದರ್ಶನಃ ॥೨೦॥

ರಾಜ್ಯಾಭಿಷಿಕ್ತೋ ರಾಜೇಂದ್ರೋ ರಾಜೀವಸದೃಶೇಕ್ಷಣಃ ।\\
ಲೋಕತಾಪಪರಿಹಂತಾ ಧರ್ಮಸಂಸ್ಥಾಪನೋದ್ಯತಃ ॥೨೧॥

ಶರಣ್ಯಃ ಕೀರ್ತಿಮಾನ್ನಿತ್ಯೋ ವದಾನ್ಯಃ ಕರುಣಾರ್ಣವಃ ।\\
ಸಂಸಾರಸಿಂಧುಸಮ್ಮಗ್ನತಾರಕಾಖ್ಯಾಮಹೋಜ್ಜ್ವಲಃ ॥೨೨॥

ಮಧುರೋ ಮಧುರೋಕ್ತಿಶ್ಚ ಮಧುರಾನಾಯಕಾಗ್ರಜಃ ।\\
ಶಂಬೂಕದತ್ತಸ್ವರ್ಲೋಕಶ್ಶಂಬರಾರಾತಿಸುಂದರಃ ॥೨೩॥

ಅಶ್ವಮೇಧಮಹಾಯಾಜೀ ವಾಲ್ಮೀಕಿಪ್ರೀತಿಮಾನ್ವಶೀ ।\\
ಸ್ವಯಂರಾಮಾಯಣಶ್ರೋತಾ ಪುತ್ರಪ್ರಾಪ್ತಿಪ್ರಮೋದಿತಃ ॥೨೪॥

ಬ್ರಹ್ಮಾದಿಸ್ತುತಮಾಹಾತ್ಮ್ಯೋ ಬ್ರಹ್ಮರ್ಷಿಗಣಪೂಜಿತಃ ।\\
ವರ್ಣಾಶ್ರಮರತೋ ವರ್ಣಾಶ್ರಮಧರ್ಮನಿಯಾಮಕಃ ॥೨೫॥

ರಕ್ಷಾಪರೋ ರಾಜವಂಶಪ್ರತಿಷ್ಠಾಪನತತ್ಪರಃ ।\\
ಗಂಧರ್ವಹಿಂಸಾಸಂಹಾರೀ ಧೃತಿಮಾಂದೀನವತ್ಸಲಃ ॥೨೬॥

ಜ್ಞಾನೋಪದೇಷ್ಟಾ ವೇದಾಂತವೇದ್ಯೋ ಭಕ್ತಪ್ರಿಯಂಕರಃ ।\\
ವೈಕುಂಠವಾಸೀ ಪಾಯಾನ್ನಶ್ಚರಾಚರವಿಮುಕ್ತಿದಃ ॥೨೭॥
\authorline{ಇತಿ ಶ್ರೀರಾಮರಹಸ್ಯೋಕ್ತಂ ಶ್ರೀರಾಮಾಷ್ಟೋತ್ತರಶತನಾಮಸ್ತ್ತೋರಂ ಸಂಪೂರ್ಣಂ ।}
ಓಂ ರಘುವಂಶ್ಯಾಯ ವಿದ್ಮಹೇ ಸೀತಾವಲ್ಲಭಾಯ ಧೀಮಹಿ । ತನ್ನೋ ರಾಮಃ ಪ್ರಚೋದಯಾತ್॥
%===============================================
\chapter*{\center ಭಾವನಾಚಕ್ರನ್ಯಾಸಃ}
\thispagestyle{empty}
\section{ಗುರುಪ್ರಾರ್ಥನಾ}
ಓಂ ಆಬ್ರಹ್ಮಲೋಕಾದಾಶೇಷಾತ್ ಆಲೋಕಾಲೋಕಪರ್ವತಾತ್।\\
ಯೇ ವಸಂತಿ ದ್ವಿಜಾ ದೇವಾಃ ತೇಭ್ಯೋ ನಿತ್ಯಂ ನಮಾಮ್ಯಹಂ॥

ಓಂ ನಮೋ ಬ್ರಹ್ಮಾದಿಭ್ಯೋ ಬ್ರಹ್ಮವಿದ್ಯಾಸಂಪ್ರದಾಯಕರ್ತೃಭ್ಯೋ\\ ವಂಶರ್ಷಿಭ್ಯೋ ಮಹದ್ಭ್ಯೋ ನಮೋ ಗುರುಭ್ಯಃ~॥

ಸರ್ವೋಪಪ್ಲವರಹಿತಃ ಪ್ರಜ್ಞಾನಘನಃ ಪ್ರತ್ಯಗರ್ಥೋ ಬ್ರಹ್ಮೈವಾಹಮಸ್ಮಿ ॥

ಸಚ್ಚಿದಾನಂದ ರೂಪಾಯ ಬಿಂದು ನಾದಾಂತರಾತ್ಮನೇ~।\\
ಆದಿಮಧ್ಯಾಂತ ಶೂನ್ಯಾಯ ಗುರೂಣಾಂ ಗುರವೇ ನಮಃ ॥

ಶುದ್ಧಸ್ಫಟಿಕಸಂಕಾಶಂ ದ್ವಿನೇತ್ರಂ ಕರುಣಾನಿಧಿಂ~।\\
ವರಾಭಯಪ್ರದಂ ವಂದೇ ಶ್ರೀಗುರುಂ ಶಿವರೂಪಿಣಂ ॥

ಗುರುರ್ಬ್ರಹ್ಮಾ ಗ್ರುರುರ್ವಿಷ್ಣುಃ ಗುರುರ್ದೇವೋ ಮಹೇಶ್ವರಃ~।\\
ಗುರುಃ ಸಾಕ್ಷಾತ್ ಪರಂ ಬ್ರಹ್ಮ ತಸ್ಮೈ ಶ್ರೀ ಗುರವೇ ನಮಃ ॥

ಶುಕ್ಲಾಂಬರಧರಂ ವಿಷ್ಣುಂ ಶಶಿವರ್ಣಂ ಚತುರ್ಭುಜಂ~।\\
ಪ್ರಸನ್ನವದನಂ ಧ್ಯಾಯೇತ್ ಸರ್ವವಿಘ್ನೋಪಶಾಂತಯೇ ॥

ಮಂಜುಶಿಂಜಿತ ಮಂಜೀರಂ ವಾಮಮರ್ಧಂ ಮಹೇಶಿತುಃ~।\\
ಆಶ್ರಯಾಮಿ ಜಗನ್ಮೂಲಂ ಯನ್ಮೂಲಂ ವಚಸಾಮಪಿ ॥

ಶ್ರೀ ವಿದ್ಯಾಂ ಜಗತಾಂ ಧಾತ್ರೀಂ ಸರ್ಗಸ್ಥಿತಿ ಲಯೇಶ್ವರೀಂ~।\\
ನಮಾಮಿ ಲಲಿತಾಂ ನಿತ್ಯಂ ಭಕ್ತಾನಾಮಿಷ್ಟದಾಯಿನೀಂ ॥

ಶ್ರೀನಾಥಾದಿ ಗುರುತ್ರಯಂ ಗಣಪತಿಂ ಪೀಠತ್ರಯಂ ಭೈರವಂ\\
ಸಿದ್ಧೌಘಂ ಬಟುಕತ್ರಯಂ ಪದಯುಗಂ ದೂತೀಕ್ರಮಂ ಮಂಡಲಂ~।\\
ವೀರದ್ವ್ಯಷ್ಟ ಚತುಷ್ಕಷಷ್ಟಿನವಕಂ ವೀರಾವಲೀಪಂಚಕಂ\\
ಶ್ರೀಮನ್ಮಾಲಿನಿಮಂತ್ರರಾಜಸಹಿತಂ ವಂದೇ ಗುರೋರ್ಮಂಡಲಂ॥

\section{ಗುರುಪಾದುಕಾಮಂತ್ರಃ}
\dhyana{ಓಂಐಂಹ್ರೀಂಶ್ರೀಂಐಂಕ್ಲೀಂಸೌಃ ಹಂಸಃ ಶಿವಃ ಸೋಽಹಂ ಹ್‌ಸ್‌ಖ್‌ಫ್ರೇಂ ಹಸಕ್ಷಮಲವರಯೂಂ ಹ್‌ಸೌಃ ಸಹಕ್ಷಮಲವರಯೀಂ ಸ್‌ಹೌಃ ಹಂಸಃ ಶಿವಃ ಸೋಽಹಂ~॥} ಸ್ವರೂಪ ನಿರೂಪಣ ಹೇತವೇ ಶ್ರೀಗುರವೇ ನಮಃ~। ಶ್ರೀಪಾದುಕಾಂ ಪೂಜಯಾಮಿ ನಮಃ ॥\\
\dhyana{೭ ಸೋಽಹಂ ಹಂಸಃ ಶಿವಃ ಹ್‌ಸ್‌ಖ್‌ಫ್ರೇಂ ಹಸಕ್ಷಮಲವರಯೂಂ ಹ್‌ಸೌಃ ಸಹಕ್ಷಮಲವರಯೀಂ ಸ್‌ಹೌಃ ಸೋಽಹಂ ಹಂಸಃ ಶಿವಃ॥} ಸ್ವಚ್ಛಪ್ರಕಾಶ ವಿಮರ್ಶಹೇತವೇ ಪರಮಗುರವೇ ನಮಃ।ಶ್ರೀಪಾದುಕಾಂ ಪೂಜಯಾಮಿ ನಮಃ॥\\
\dhyana{೭ ಹಂಸಃಶಿವಃ ಸೋಽಹಂಹಂಸಃ ಹ್‌ಸ್‌ಖ್‌ಫ್ರೇಂ ಹಸಕ್ಷಮಲವರಯೂಂ ಹ್‌ಸೌಃ ಸಹಕ್ಷಮಲವರಯೀಂ ಸ್‌ಹೌಃ ಹಂಸಃ ಶಿವಃ ಸೋಽಹಂ ಹಂಸಃ॥} ಸ್ವಾತ್ಮಾರಾಮ ಪರಮಾನಂದ ಪಂಜರ ವಿಲೀನ ತೇಜಸೇ ಪರಮೇಷ್ಠಿಗುರವೇ ನಮಃ~। ಶ್ರೀಪಾದುಕಾಂ ಪೂಜಯಾಮಿ ನಮಃ॥

ಪೃಥ್ವೀತಿ ಮಂತ್ರಸ್ಯ ಮೇರುಪೃಷ್ಠ ಋಷಿಃ~। ಸುತಲಂ ಛಂದಃ~।\\ಆದಿಕೂರ್ಮೋ ದೇವತಾ ॥ ಆಸನೇ ವಿನಿಯೋಗಃ ॥\\
ಪೃಥ್ವಿ ತ್ವಯಾ ಧೃತಾ ಲೋಕಾ ದೇವಿ ತ್ವಂ ವಿಷ್ಣುನಾ ಧೃತಾ~।\\
ತ್ವಂ ಚ ಧಾರಯ ಮಾಂ ದೇವಿ ಪವಿತ್ರಂ ಕುರು ಚಾಸನಂ ॥

ಅಪಸರ್ಪಂತು ತೇ ಭೂತಾಃ ಯೇ ಭೂತಾ ಭೂಮಿ ಸಂಸ್ಥಿತಾಃ~।\\
ಯೇ ಭೂತಾಃ ವಿಘ್ನಕರ್ತಾರಸ್ತೇ ನಶ್ಯಂತು ಶಿವಾಜ್ಞಯಾ ॥

ಅಪಕ್ರಾಮಂತು ಭೂತಾನಿ ಪಿಶಾಚಾಃ ಸರ್ವತೋ ದಿಶಂ।\\
ಸರ್ವೇಷಾಮವಿರೋಧೇನ ಪೂಜಾ ಕರ್ಮಸಮಾರಭೇ ॥

ವಾಮಪಾದಾದಿ ಪೃಷ್ಠೇ ಚ ಧರಣೀತಾಡನತ್ರಯಮ್ ।\\
ವೀರಭದ್ರ ವಿರೂಪಾಕ್ಷ ವಿಶ್ವರೂಪತ್ರಯಂ ಸ್ಮರೇತ್ ॥

ಸ್ಯೋನಾ ಪೃಥಿವೀತ್ಯಸ್ಯ ಮೇಧಾತಿಥಿಃ ಕಾಣ್ವ ಋಷಿಃ । ಗಾಯತ್ರೀ ಛಂದಃ । ಪೃಥಿವೀ ದೇವತಾ । ಭೂಪ್ರಾರ್ಥನೇ ವಿನಿಯೋಗಃ ॥\\
ಸ್ಯೋ॒ನಾ ಪೃ॑ಥಿವಿ ಭವಾನೃಕ್ಷ॒ರಾ ನಿ॒ವೇಶ॑ನೀ~।\\ ಯಚ್ಛಾ᳚ನಃ॒ ಶರ್ಮ॑ ಸ॒ಪ್ರಥಃ॑ ॥

ಧನುರ್ಧರಾಯೈ ಚ ವಿದ್ಮಹೇ ಸರ್ವಸಿದ್ಧ್ಯೈ ಚ ಧೀಮಹಿ~।\\ ತನ್ನೋ ಧರಾ ಪ್ರಚೋದಯಾತ್ ॥

ಲಂ ಪೃಥಿವ್ಯೈ ನಮಃ~। ರಂ ರಕ್ತಾಸನಾಯ ನಮಃ~। ವಿಂ ವಿಮಲಾಸನಾಯ ನಮಃ~। ಯಂ ಯೋಗಾಸನಾಯ ನಮಃ~। ಕೂರ್ಮಾಸನಾಯ ನಮಃ~। ಅನಂತಾಸನಾಯ ನಮಃ~। ವೀರಾಸನಾಯ ನಮಃ~। ಖಡ್ಗಾಸನಾಯ ನಮಃ~। ಶರಾಸನಾಯ ನಮಃ~। ಪಂ ಪದ್ಮಾಸನಾಯ ನಮಃ~। ಪರಮಸುಖಾಸನಾಯ ನಮಃ॥

೪ ರಕ್ತದ್ವಾದಶಶಕ್ತಿಯುಕ್ತಾಯ ದ್ವೀಪನಾಥಾಯ ನಮಃ ॥

೪ ಶ್ರೀಲಲಿತಾಮಹಾತ್ರಿಪುರಸುಂದರಿ ಆತ್ಮಾನಂ ರಕ್ಷ ರಕ್ಷ ॥

ಓಂ ಗುಂ ಗುರುಭ್ಯೋ ನಮಃ~। ಪರಮಗುರುಭ್ಯೋ ನಮಃ~। ಪರಮೇಷ್ಠಿ\\ಗುರುಭ್ಯೋ ನಮಃ~। ಗಂ ಗಣಪತಯೇ ನಮಃ~। ದುಂ ದುರ್ಗಾಯೈ ನಮಃ~। ಸಂ ಸರಸ್ವತ್ಯೈ ನಮಃ~। ವಂ ವಟುಕಾಯ ನಮಃ~। ಕ್ಷಂ ಕ್ಷೇತ್ರಪಾಲಾಯ ನಮಃ~। ಯಾಂ ಯೋಗಿನೀಭ್ಯೋ ನಮಃ~। ಅಂ ಆತ್ಮನೇ ನಮಃ~। ಪಂ ಪರಮಾತ್ಮನೇ ನಮಃ~। ಸಂ ಸರ್ವಾತ್ಮನೇ ನಮಃ ॥

೪ ಓಂ ನಮೋ ಭಗವತಿ ತಿರಸ್ಕರಿಣಿ ಮಹಾಮಾಯೇ ಮಹಾನಿದ್ರೇ ಸಕಲ \\ಪಶುಜನ ಮನಶ್ಚಕ್ಷುಃಶ್ರೋತ್ರತಿರಸ್ಕರಣಂ ಕುರು ಕುರು ಸ್ವಾಹಾ ॥

೪ ಓಂ ಹಸಂತಿ ಹಸಿತಾಲಾಪೇ ಮಾತಂಗಿ ಪರಿಚಾರಿಕೇ~।\\
ಮಮ ವಿಘ್ನಾಪದಾಂ ನಾಶಂ ಕುರು ಕುರು ಠಃಠಃಠಃ ಹುಂ ಫಟ್ ಸ್ವಾಹಾ ॥

೪ ಓಂ ನಮೋ ಭಗವತಿ ಜ್ವಾಲಾಮಾಲಿನಿ ದೇವದೇವಿ ಸರ್ವಭೂತ ಸಂಹಾರ ಕಾರಿಕೇ ಜಾತವೇದಸಿ ಜ್ವಲಂತಿ ಜ್ವಲ ಜ್ವಲ ಪ್ರಜ್ವಲ ಪ್ರಜ್ವಲ ಹ್ರಾಂ ಹ್ರೀಂ ಹ್ರೂಂ ರರ ರರ ರರರ ಹುಂ ಫಟ್ ಸ್ವಾಹಾ~। ಸಹಸ್ರಾರ ಹುಂ ಫಟ್~।\\ ಭೂರ್ಭುವಃಸುವರೋಮಿತಿ ದಿಗ್ಬಂಧಃ ॥

೪ ಸಮಸ್ತ ಪ್ರಕಟ ಗುಪ್ತ ಗುಪ್ತತರ ಸಂಪ್ರದಾಯ ಕುಲೋತ್ತೀರ್ಣ ನಿಗರ್ಭ ರಹಸ್ಯಾ\-ತಿರಹಸ್ಯ ಪರಾಪರಾತಿರಹಸ್ಯ ಯೋಗಿನೀ ದೇವತಾಭ್ಯೋ ನಮಃ ॥

೪ ಐಂ ಹ್ರಃ ಅಸ್ತ್ರಾಯ ಫಟ್ ॥

೪ ಮೂಲಶೃಂಗಾಟಕಾತ್ ಸುಷುಮ್ನಾಪಥೇನ ಜೀವಶಿವಂ ಪರಮಶಿವಪದೇ \\ಯೋಜಯಾಮಿ ಸ್ವಾಹಾ~।\\
ಯಂ ೮ ಸಂಕೋಚಶರೀರಂ ಶೋಷಯ ಶೋಷಯ ಸ್ವಾಹಾ~।\\
ರಂ ೮ ಸಂಕೋಚಶರೀರಂ ದಹ ದಹ ಪಚ ಪಚ ಸ್ವಾಹಾ~।\\
ವಂ ೮ ಪರಮಶಿವಾಮೃತಂ ವರ್ಷಯ ವರ್ಷಯ ಸ್ವಾಹಾ~।\\
ಲಂ ೮ ಶಾಂಭವಶರೀರಮುತ್ಪಾದಯೋತ್ಪಾದಯ ಸ್ವಾಹಾ~।\\
ಹಂಸಃ ಸೋಹಂ ಅವತರ ಅವತರ ಶಿವಪದಾತ್ ಜೀವಶಿವ\\ ಸುಷುಮ್ನಾಪಥೇನ ಪ್ರವಿಶ ಮೂಲಶೃಂಗಾಟಕಂ ಉಲ್ಲಸೋಲ್ಲಸ\\ ಜ್ವಲ ಜ್ವಲ ಪ್ರಜ್ವಲ ಪ್ರಜ್ವಲ ಹಂಸಃ ಸೋಹಂ ಸ್ವಾಹಾ ॥\\
೪ ಆಂ ಸೋಹಂ (ಇತಿ ತ್ರಿಃ ಹೃದಿ) ಇತಿ ಭೂತಶುದ್ಧಿಃ ॥

ತತಃ ಪ್ರಾಣಾನಾಯಮ್ಯ, ದೇಶಕಾಲೌ ಸಂಕೀರ್ತ್ಯ,\\
ಶುಭಪುಣ್ಯತಿಥೌ ಶ್ರೀಮಹಾಭಟ್ಟಾರಕ ಕಾಮೇಶ್ವರಸಹಿತ ಶ್ರೀಶಕ್ತಿಚಕ್ರೈಕನಾಯಿಕಾ ಶ್ರೀಮಹಾಭಟ್ಟಾರಿಕಾ ಶ್ರೀ ಲಲಿತಾಮಹಾತ್ರಿಪುರಸುಂದರೀ ಪ್ರೀತ್ಯರ್ಥೇ ಶ್ರೀಗುರೋರಾಜ್ಞಯಾ ಶ್ರೀಚಕ್ರ ಅಂತರಾರಾಧನಂ ಕರಿಷ್ಯೇ ॥


\section{ಲಘುಪ್ರಾಣಪ್ರತಿಷ್ಠಾ}
ಹೃದಿ ಹಸ್ತಂ ದತ್ವಾ\\ಆಂ ಸೋಹಂ~। ಆಂ ಹ್ರೀಂ ಕ್ರೋಂ ಯರಲವಶಷಸಹೋಂ ॥ ಮಮ ಪ್ರಾಣಾ ಇಹ ಪ್ರಾಣಾಃ~। ಜೀವ ಇಹ ಸ್ಥಿತಃ~। ಸರ್ವೇಂದ್ರಿಯಾಣಿ~। ಮಮ ವಾಙ್ಮನಸ್ತ್ವಕ್ಚಕ್ಷುಃ ಶ್ರೋತ್ರ ಜಿಹ್ವಾಘ್ರಾಣಪ್ರಾಣಾ ಇಹೈವಾಗತ್ಯ ಸುಖಂ ಚಿರಂ ತಿಷ್ಠಂತು ಸ್ವಾಹಾ ॥

ಮೂಲವಿದ್ಯಾನ್ಯಾಸಂ ಷೋಡಶಾಕ್ಷರೀ ನ್ಯಾಸಂ ಚ ವಿಧಾಯ,

ಶರೀರಂ ಚಿಂತಯೇದಾದೌ ನಿಜಂ ಶ್ರೀಚಕ್ರರೂಪಕಮ್ ।\\
ತ್ವಗಾದ್ಯಾಕಾರನಿರ್ಮುಕ್ತಂ ಜ್ವಲತ್ಕಾಲಾಗ್ನಿಸನ್ನಿಭಮ್ ॥\\
ತತಃ ಸ್ವಾತ್ಮಾನಂ ದೇವೀರೂಪಂ ವಿಭಾವಯೇತ್ ॥

\as{ಅ॒ಣೋರಣೀ॑ಯಾನ್ಮಹ॒ತೋ ಮಹೀ॑ಯಾನಾ॒ತ್ಮಾ ಗುಹಾ॑ಯಾಂ॒ ನಿಹಿ॑ತೋಽಸ್ಯ ಜಂ॒ತೋಃ ।
ತಮ॑ಕ್ರತುಂ ಪಶ್ಯತಿ ವೀತಶೋ॒ಕೋ ಧಾ॒ತುಃ ಪ್ರ॒ಸಾದಾ᳚ನ್ಮಹಿ॒ಮಾನ॑ಮೀಶಂ ॥}

ವಿವೇಕವೃತ್ಯವಚ್ಛಿನ್ನ ಚಿಚ್ಛಕ್ತಿರೂಪಸುಷುಮ್ನಾತ್ಮನೇ ಶ್ರೀಗುರವೇ ನಮಃ ।\\(ಇತಿ ಬ್ರಹ್ಮರಂಧ್ರಂ ಸ್ಪೃಷ್ಟ್ವಾ)

\as{ದಿವ್ಯೌಘಾಃ}\\
ದಕ್ಷಶ್ರೋತ್ರರೂಪ ಪಯಸ್ವಿನ್ಯಾತ್ಮನೇ ಪ್ರಕಾಶಾನಂದನಾಥಾಯ ನಮಃ ।\\
ವಾಮಶ್ರೋತ್ರರೂಪ ಶಂಖಿನ್ಯಾತ್ಮನೇ ವಿಮರ್ಶಾನಂದನಾಥಾಯ ನಮಃ ।\\
ಜಿಹ್ವಾರೂಪ ಸರಸ್ವತ್ಯಾತ್ಮನೇ ಆನಂದಾನಂದನಾಥಾಯ ನಮಃ ।

\as{ಸಿದ್ಧೌಘಾಃ}\\
ದಕ್ಷನೇತ್ರರೂಪ ಪೂಷಾತ್ಮನೇ ಜ್ಞಾನಾನಂದನಾಥಾಯ ನಮಃ ।\\
ವಾಮನೇತ್ರರೂಪ ಗಾಂಧಾರ್ಯಾತ್ಮನೇ ಸತ್ಯಾನಂದನಾಥಾಯ ನಮಃ ।\\
ಧ್ವಜರೂಪ ಕುಹ್ವಾತ್ಮನೇ ಪೂರ್ಣಾನಂದನಾಥಾಯ ನಮಃ ।

\as{ಮಾನವೌಘಾಃ}\\
ದಕ್ಷನಾಸಾರೂಪ ಪಿಂಗಲಾತ್ಮನೇ ಸ್ವಭಾವಾನಂದನಾಥಾಯ ನಮಃ ।\\
ವಾಮನಾಸಾರೂಪ ಇಡಾತ್ಮನೇ ಪ್ರತಿಭಾನಂದನಾಥಾಯ ನಮಃ ।\\
ಪಾಯುರೂಪ ಅಲಂಬುಸಾತ್ಮನೇ ಸುಭಗಾನಂದನಾಥಾಯ ನಮಃ ।

\as{ಸ॒ಪ್ತ ಪ್ರಾ॒ಣಾಃ ಪ್ರ॒ಭವಂ॑ತಿ॒ ತಸ್ಮಾ᳚ಥ್ಸ॒ಪ್ತಾರ್ಚಿಷ॑ಸ್ಸ॒ಮಿಧ॑ಸ್ಸ॒ಪ್ತಜಿ॒ಹ್ವಾಃ ।
ಸ॒ಪ್ತ ಇ॒ಮೇ ಲೋ॒ಕಾ ಯೇಷು॒ ಚರಂ॑ತಿ ಪ್ರಾ॒ಣಾ ಗು॒ಹಾಶ॑ಯಾ॒ನ್ನಿಹಿ॑ತಾಃ ಸ॒ಪ್ತ ಸ॑ಪ್ತ ॥}\\
ನವಚಕ್ರರೂಪ ಶ್ರೀಚಕ್ರಾತ್ಮನೇ ದೇಹಾಯ ನಮಃ ।\\
ಪಿತೃರೂಪ ಅಸ್ಥ್ಯಾದ್ಯವಯವಾತ್ಮನೇ ವಾರಾಹ್ಯೈ ನಮಃ ।\\
ಮಾತೃರೂಪ ಮಾಂಸಾದ್ಯವಯವಾತ್ಮನೇ ಬಲಿದೇವತಾಯೈ ಕುರುಕುಲ್ಲಾಯೈ ನಮಃ ।(ಇತಿ ತ್ರಿಃ ವ್ಯಾಪಕಂ ಕೃತ್ವಾ)\\
ದೇಹಪಶ್ಚಾದ್ಭಾಗರೂಪ ಧರ್ಮಾತ್ಮನೇ ಇಕ್ಷುಸಾಗರಾಯ ನಮಃ ।\\
ದೇಹದಕ್ಷಿಣಭಾಗರೂಪ ಅರ್ಥಾತ್ಮನೇ ಸುರಾಸಾಗರಾಯ ನಮಃ ।\\
ದೇಹಪ್ರಾಗ್ಭಾಗರೂಪ ಕಾಮಾತ್ಮನೇ ಘೃತಸಾಗರಾಯ ನಮಃ ।\\
ದೇಹಉದಗ್ಭಾಗರೂಪ ಮೋಕ್ಷಾತ್ಮನೇ ಕ್ಷೀರಸಾಗರಾಯ ನಮಃ ॥\\
ದೇಹಾತ್ಮನೇ ನವರತ್ನದ್ವೀಪಾಯ ನಮಃ ।\\
ಮಾಂಸಾತ್ಮನೇ ಪುಷ್ಯರಾಗರತ್ನಾಯ ನಮಃ ।\\
ರೋಮಾತ್ಮನೇ ನೀಲರತ್ನಾಯ ನಮಃ ।\\
ತ್ವಗಾತ್ಮನೇ ವೈಡೂರ್ಯರತ್ನಾಯ ನಮಃ ।\\
ರುಧಿರಾತ್ಮನೇ ವಿದ್ರುಮರತ್ನಾಯ ನಮಃ ।\\
ಮಜ್ಜಾತ್ಮನೇ ಮರಕತರತ್ನಾಯ ನಮಃ ।\\
ಅಸ್ಥ್ಯಾತ್ಮನೇ ವಜ್ರರತ್ನಾಯ ನಮಃ ।\\
ಶುಕ್ಲಾತ್ಮನೇ ಮೌಕ್ತಿಕರತ್ನಾಯ ನಮಃ ।\\
ಮೇದ ಆತ್ಮನೇ ಗೋಮೇಧಕರತ್ನಾಯ ನಮಃ ।\\
ಓಜ ಆತ್ಮನೇ ಪದ್ಮರಾಗರತ್ನಾಯ ನಮಃ ।

ಮಾಂಸಾಧಿದೇವತಾಯೈ ಕಾಲಚಕ್ರೇಶ್ವರ್ಯೈ ನಮಃ ।\\
ರೋಮಾಧಿದೇವತಾಯೈ ಮುದ್ರಾಚಕ್ರೇಶ್ವರ್ಯೈ ನಮಃ ।\\
ತ್ವಗಧಿದೇವತಾಯೈ ಮಾತೃಕಾಚಕ್ರೇಶ್ವರ್ಯೈ ನಮಃ ।\\
ರುಧಿರಾಧಿದೇವತಾಯೈ ರತ್ನಚಕ್ರೇಶ್ವರ್ಯೈ ನಮಃ ।\\
ಮಜ್ಜಾಧಿದೇವತಾಯೈ ಗುರುಚಕ್ರೇಶ್ವರ್ಯೈ ನಮಃ ।\\
ಅಸ್ಥ್ಯಧಿದೇವತಾಯೈ ತತ್ವಚಕ್ರೇಶ್ವರ್ಯೈ ನಮಃ ।\\
ಶುಕ್ಲಾಧಿದೇವತಾಯೈ ದೇಶಚಕ್ರೇಶ್ವರ್ಯೈ ನಮಃ ।\\
ಮೇದೋಽಧಿದೇವತಾಯೈ ಗ್ರಹಚಕ್ರೇಶ್ವರ್ಯೈ ನಮಃ ।\\
ಓಜೋಽಧಿದೇವತಾಯೈ ಮೂರ್ತಿಚಕ್ರೇಶ್ವರ್ಯೈ ನಮಃ ।\\
ಸಂಕಲ್ಪಾತ್ಮಭ್ಯಃ ಕಲ್ಪತರುಭ್ಯೋ ನಮಃ ।\\
ತೇಜಆತ್ಮನೇ ಕಲ್ಪಕೋದ್ಯಾನಾಯ ನಮಃ ।

ಮಧುರರಸಾತ್ಮನೇ ವಸಂತರ್ತವೇ ನಮಃ ।\\
ಆಮ್ಲರಸಾತ್ಮನೇ ಗ್ರೀಷ್ಮರ್ತವೇ ನಮಃ ।\\
ತಿಕ್ತರಸಾತ್ಮನೇ ವರ್ಷರ್ತವೇ ನಮಃ ।\\
ಕಟುರಸಾತ್ಮನೇ ಶರದೃತವೇ ನಮಃ ।\\
ಕಷಾಯರಸಾತ್ಮನೇ ಹೇಮಂತರ್ತವೇ ನಮಃ ।\\
ಲವಣರಸಾತ್ಮನೇ ಶಿಶಿರರ್ತವೇ ನಮಃ ।

ಇಂದ್ರಿಯಾತ್ಮಭ್ಯೋಽಶ್ವೇಭ್ಯೋ ನಮಃ ।\\
ಇಂದ್ರಿಯಾರ್ಥಾತ್ಮಭ್ಯೋ ಗಜೇಭ್ಯೋ ನಮಃ ।\\
ಕರುಣಾತ್ಮನೇ ತೋಯಪರಿಘಾಯ ನಮಃ ।\\
ಓಜಃಪುಂಜಾತ್ಮನೇ ಮಾಣಿಕ್ಯಮಂಡಪಾಯ ನಮಃ ।\\
ಜ್ಞಾನಾತ್ಮನೇ ವಿಶೇಷಾರ್ಘ್ಯಾಯ ನಮಃ ।\\
ಜ್ಞೇಯಾತ್ಮನೇ ಹವಿಷೇ ನಮಃ ।\\
ಜ್ಞಾತ್ರಾತ್ಮನೇ ಸ್ವಾತ್ಮನೇ ನಮಃ ।\\
ಚಿದಾತ್ಮನೇ ಶ್ರೀಮಹಾತ್ರಿಪುರಸುಂದರ್ಯೈ ನಮಃ ।

(ಇತಿ ತತ್ತದನುಸಂಧಾನಪೂರ್ವಕಂ ಮನಸಾ ನತ್ವಾ, ಜ್ಞಾತೃಜ್ಞಾನಜ್ಞೇಯಾನಾಂ ನಾಮರೂಪವಿಲಾಪನಾನುಸಂಧಾನೇನ ಚಿನ್ಮಾತ್ರರೂಪತಾ ವಿಭಾವನೇನ ಕ್ಷಣಂ ವಿಶ್ರಮ್ಯ)\\
 \as{(ಲಲಿತಾಸಹಸ್ರನಾಮ ೧ - ೧೦೦)\\
  ಓಂ ಐಂಹ್ರೀಂಶ್ರೀಂ}\\
ಶ್ರೀಮಾತಾ ಶ್ರೀಮಹಾರಾಜ್ಞೀ ಶ್ರೀಮತ್ಸಿಂಹಾಸನೇಶ್ವರೀ~।\\
ಚಿದಗ್ನಿ - ಕುಂಡ - ಸಂಭೂತಾ ದೇವಕಾರ್ಯ - ಸಮುದ್ಯತಾ ॥೧॥

ಉದ್ಯದ್ಭಾನು - ಸಹಸ್ರಾಭಾ ಚತುರ್ಬಾಹು - ಸಮನ್ವಿತಾ~।\\
ರಾಗಸ್ವರೂಪ - ಪಾಶಾಢ್ಯಾ ಕ್ರೋಧಾಕಾರಾಂಕುಶೋಜ್ಜ್ವಲಾ ॥೨॥

ಮನೋರೂಪೇಕ್ಷು - ಕೋದಂಡಾ ಪಂಚತನ್ಮಾತ್ರ - ಸಾಯಕಾ~।\\
ನಿಜಾರುಣ - ಪ್ರಭಾಪೂರ - ಮಜ್ಜದ್‍ಬ್ರಹ್ಮಾಂಡ - ಮಂಡಲಾ ॥೩॥

ಚಂಪಕಾಶೋಕ - ಪುನ್ನಾಗ - ಸೌಗಂಧಿಕ - ಲಸತ್ಕಚಾ~।\\
ಕುರುವಿಂದಮಣಿ - ಶ್ರೇಣೀ - ಕನತ್ಕೋಟೀರ - ಮಂಡಿತಾ ॥೪॥

ಅಷ್ಟಮೀಚಂದ್ರ - ವಿಭ್ರಾಜ - ದಲಿಕಸ್ಥಲ - ಶೋಭಿತಾ~।\\
ಮುಖಚಂದ್ರ - ಕಲಂಕಾಭ - ಮೃಗನಾಭಿ - ವಿಶೇಷಕಾ ॥೫॥

ವದನಸ್ಮರ - ಮಾಂಗಲ್ಯ - ಗೃಹತೋರಣ - ಚಿಲ್ಲಿಕಾ~।\\
ವಕ್ತ್ರಲಕ್ಷ್ಮೀ - ಪರೀವಾಹ - ಚಲನ್ಮೀನಾಭ - ಲೋಚನಾ ॥೬॥

ನವಚಂಪಕ - ಪುಷ್ಪಾಭ - ನಾಸಾದಂಡ - ವಿರಾಜಿತಾ~।\\
ತಾರಾಕಾಂತಿ - ತಿರಸ್ಕಾರಿ - ನಾಸಾಭರಣ - ಭಾಸುರಾ ॥೭॥

ಕದಂಬಮಂಜರೀ - ಕ್ಲೃಪ್ತ - ಕರ್ಣಪೂರ - ಮನೋಹರಾ~।\\
ತಾಟಂಕ - ಯುಗಲೀ - ಭೂತ - ತಪನೋಡುಪ - ಮಂಡಲಾ ॥೮॥

ಪದ್ಮರಾಗಶಿಲಾದರ್ಶ - ಪರಿಭಾವಿ - ಕಪೋಲಭೂಃ~।\\
ನವವಿದ್ರುಮ - ಬಿಂಬಶ್ರೀ - ನ್ಯಕ್ಕಾರಿ - ರದನಚ್ಛದಾ ॥೯॥

ಶುದ್ಧವಿದ್ಯಾಂಕುರಾಕಾರ - ದ್ವಿಜಪಂಕ್ತಿ - ದ್ವಯೋಜ್ಜ್ವಲಾ~।\\
ಕರ್ಪೂರವೀಟಿಕಾಮೋದ - ಸಮಾಕರ್ಷದ್ದಿಗಂತರಾ ॥೧೦॥

ನಿಜ - ಸಲ್ಲಾಪ - ಮಾಧುರ್ಯ - ವಿನಿರ್ಭರ್ತ್ಸಿತ - ಕಚ್ಛಪೀ~।\\
ಮಂದಸ್ಮಿತ - ಪ್ರಭಾಪೂರ - ಮಜ್ಜತ್ಕಾಮೇಶ - ಮಾನಸಾ ॥೧೧॥

ಅನಾಕಲಿತ - ಸಾದೃಶ್ಯ - ಚುಬುಕಶ್ರೀ - ವಿರಾಜಿತಾ~।\\
ಕಾಮೇಶ - ಬದ್ಧ - ಮಾಂಗಲ್ಯ - ಸೂತ್ರ - ಶೋಭಿತ - ಕಂಧರಾ ॥೧೨॥

ಕನಕಾಂಗದ - ಕೇಯೂರ - ಕಮನೀಯ - ಭುಜಾನ್ವಿತಾ~।\\
ರತ್ನಗ್ರೈವೇಯ - ಚಿಂತಾಕ - ಲೋಲ - ಮುಕ್ತಾ - ಫಲಾನ್ವಿತಾ ॥೧೩॥

ಕಾಮೇಶ್ವರ - ಪ್ರೇಮರತ್ನ - ಮಣಿ - ಪ್ರತಿಪಣ - ಸ್ತನೀ~।\\
ನಾಭ್ಯಾಲವಾಲ - ರೋಮಾಲಿ - ಲತಾ - ಫಲ - ಕುಚದ್ವಯೀ ॥೧೪॥

ಲಕ್ಷ್ಯರೋಮ - ಲತಾಧಾರತಾ - ಸಮುನ್ನೇಯ - ಮಧ್ಯಮಾ~।\\
ಸ್ತನಭಾರ - ದಲನ್ಮಧ್ಯ - ಪಟ್ಟಬಂಧ - ವಲಿತ್ರಯಾ ॥೧೫॥

ಅರುಣಾರುಣಕೌಸುಂಭ - ವಸ್ತ್ರ - ಭಾಸ್ವತ್ಕಟೀತಟೀ~।\\
ರತ್ನ - ಕಿಂಕಿಣಿಕಾ - ರಮ್ಯ - ರಶನಾ - ದಾಮ - ಭೂಷಿತಾ ॥೧೬॥

ಕಾಮೇಶ - ಜ್ಞಾತ - ಸೌಭಾಗ್ಯ - ಮಾರ್ದವೋರು - ದ್ವಯಾನ್ವಿತಾ~।\\
ಮಾಣಿಕ್ಯ - ಮುಕುಟಾಕಾರ - ಜಾನುದ್ವಯ - ವಿರಾಜಿತಾ ॥೧೭॥

ಇಂದ್ರಗೋಪ - ಪರಿಕ್ಷಿಪ್ತಸ್ಮರತೂಣಾಭ - ಜಂಘಿಕಾ~।\\
ಗೂಢಗುಲ್ಫಾ ಕೂರ್ಮಪೃಷ್ಠ - ಜಯಿಷ್ಣು - ಪ್ರಪದಾನ್ವಿತಾ ॥೧೮॥

ನಖ - ದೀಧಿತಿ - ಸಂಛನ್ನ - ನಮಜ್ಜನ - ತಮೋಗುಣಾ~।\\
ಪದದ್ವಯ - ಪ್ರಭಾಜಾಲ - ಪರಾಕೃತ - ಸರೋರುಹಾ ॥೧೯॥

ಶಿಂಜಾನ - ಮಣಿಮಂಜೀರ - ಮಂಡಿತ - ಶ್ರೀ - ಪದಾಂಬುಜಾ~।\\
ಮರಾಲೀ - ಮಂದಗಮನಾ ಮಹಾಲಾವಣ್ಯ - ಶೇವಧಿಃ ॥೨೦॥

ಸರ್ವಾರುಣಾಽನವದ್ಯಾಂಗೀ ಸರ್ವಾಭರಣ - ಭೂಷಿತಾ~।\\
ಶಿವ - ಕಾಮೇಶ್ವರಾಂಕಸ್ಥಾ ಶಿವಾ ಸ್ವಾಧೀನ - ವಲ್ಲಭಾ ॥೨೧॥

ಸುಮೇರು - ಮಧ್ಯ - ಶೃಂಗಸ್ಥಾ ಶ್ರೀಮನ್ನಗರ - ನಾಯಿಕಾ~।\\
ಚಿಂತಾಮಣಿ - ಗೃಹಾಂತಸ್ಥಾ ಪಂಚ - ಬ್ರಹ್ಮಾಸನ - ಸ್ಥಿತಾ ॥೨೨॥

ಮಹಾಪದ್ಮಾಟವೀ - ಸಂಸ್ಥಾ ಕದಂಬವನ - ವಾಸಿನೀ~।\\
ಸುಧಾಸಾಗರ - ಮಧ್ಯಸ್ಥಾ ಕಾಮಾಕ್ಷೀ ಕಾಮದಾಯಿನೀ ॥೨೩॥

ದೇವರ್ಷಿ - ಗಣ - ಸಂಘಾತ - ಸ್ತೂಯಮಾನಾತ್ಮ - ವೈಭವಾ~।\\
ಭಂಡಾಸುರ - ವಧೋದ್ಯುಕ್ತ - ಶಕ್ತಿಸೇನಾ - ಸಮನ್ವಿತಾ ॥೨೪॥

ಸಂಪತ್ಕರೀ - ಸಮಾರೂಢ - ಸಿಂಧುರ - ವ್ರಜ - ಸೇವಿತಾ~।\\
ಅಶ್ವಾರೂಢಾಧಿಷ್ಠಿತಾಶ್ವ - ಕೋಟಿ - ಕೋಟಿಭಿರಾವೃತಾ ॥೨೫॥

ಚಕ್ರರಾಜ - ರಥಾರೂಢ - ಸರ್ವಾಯುಧ - ಪರಿಷ್ಕೃತಾ~।\\
ಗೇಯಚಕ್ರ - ರಥಾರೂಢ - ಮಂತ್ರಿಣೀ - ಪರಿಸೇವಿತಾ ॥೨೬॥

ಕಿರಿಚಕ್ರ - ರಥಾರೂಢ - ದಂಡನಾಥಾ - ಪುರಸ್ಕೃತಾ~।\\
ಜ್ವಾಲಾ - ಮಾಲಿನಿಕಾಕ್ಷಿಪ್ತ - ವಹ್ನಿಪ್ರಾಕಾರ - ಮಧ್ಯಗಾ ॥೨೭॥

ಭಂಡಸೈನ್ಯ - ವಧೋದ್ಯುಕ್ತ - ಶಕ್ತಿ - ವಿಕ್ರಮ - ಹರ್ಷಿತಾ~।\\
ನಿತ್ಯಾ - ಪರಾಕ್ರಮಾಟೋಪ - ನಿರೀಕ್ಷಣ - ಸಮುತ್ಸುಕಾ ॥೨೮॥

ಭಂಡಪುತ್ರ - ವಧೋದ್ಯುಕ್ತ - ಬಾಲಾ - ವಿಕ್ರಮ - ನಂದಿತಾ~।\\
ಮಂತ್ರಿಣ್ಯಂಬಾ - ವಿರಚಿತ - ವಿಷಂಗ - ವಧ - ತೋಷಿತಾ ॥೨೯॥

ವಿಶುಕ್ರ - ಪ್ರಾಣಹರಣ - ವಾರಾಹೀ - ವೀರ್ಯ - ನಂದಿತಾ~।\\
ಕಾಮೇಶ್ವರ - ಮುಖಾಲೋಕ - ಕಲ್ಪಿತ - ಶ್ರೀಗಣೇಶ್ವರಾ ॥೩೦॥

ಮಹಾಗಣೇಶ - ನಿರ್ಭಿನ್ನ - ವಿಘ್ನಯಂತ್ರ - ಪ್ರಹರ್ಷಿತಾ~।\\
ಭಂಡಾಸುರೇಂದ್ರ - ನಿರ್ಮುಕ್ತ - ಶಸ್ತ್ರ - ಪ್ರತ್ಯಸ್ತ್ರ - ವರ್ಷಿಣೀ ॥೩೧॥

ಕರಾಂಗುಲಿ - ನಖೋತ್ಪನ್ನ - ನಾರಾಯಣ - ದಶಾಕೃತಿಃ~।\\
ಮಹಾ - ಪಾಶುಪತಾಸ್ತ್ರಾಗ್ನಿ - ನಿರ್ದಗ್ಧಾಸುರ - ಸೈನಿಕಾ ॥೩೨॥

ಕಾಮೇಶ್ವರಾಸ್ತ್ರ - ನಿರ್ದಗ್ಧ - ಸಭಂಡಾಸುರ - ಶೂನ್ಯಕಾ~।\\
ಬ್ರಹ್ಮೋಪೇಂದ್ರ - ಮಹೇಂದ್ರಾದಿ - ದೇವ - ಸಂಸ್ತುತ - ವೈಭವಾ ॥೩೩॥

ಹರ - ನೇತ್ರಾಗ್ನಿ - ಸಂದಗ್ಧ - ಕಾಮ - ಸಂಜೀವನೌಷಧಿಃ~।\\
ಶ್ರೀಮದ್ವಾಗ್ಭವ - ಕೂಟೈಕ - ಸ್ವರೂಪ - ಮುಖ - ಪಂಕಜಾ ॥೩೪॥

ಕಂಠಾಧಃ - ಕಟಿ - ಪರ್ಯಂತ - ಮಧ್ಯಕೂಟ - ಸ್ವರೂಪಿಣೀ~।\\
ಶಕ್ತಿ - ಕೂಟೈಕತಾಪನ್ನ - ಕಟ್ಯಧೋಭಾಗ ಧಾರಿಣೀ ॥೩೫॥

ಮೂಲ - ಮಂತ್ರಾತ್ಮಿಕಾ ಮೂಲಕೂಟತ್ರಯ - ಕಲೇಬರಾ~।\\
ಕುಲಾಮೃತೈಕ - ರಸಿಕಾ ಕುಲಸಂಕೇತ - ಪಾಲಿನೀ ॥೩೬॥

ಕುಲಾಂಗನಾ ಕುಲಾಂತಸ್ಥಾ ಕೌಲಿನೀ ಕುಲಯೋಗಿನೀ~।\\
ಅಕುಲಾ ಸಮಯಾಂತಸ್ಥಾ ಸಮಯಾಚಾರ - ತತ್ಪರಾ ॥೩೭॥

ಮೂಲಾಧಾರೈಕ - ನಿಲಯಾ ಬ್ರಹ್ಮಗ್ರಂಥಿ - ವಿಭೇದಿನೀ \as{(೧೦೦)}~।

\as{ (ಲಲಿತಾಷ್ಟೋತ್ತರ ೧ - ೧೦)\\
ಓಂ ಐಂಹ್ರೀಂಶ್ರೀಂ}\\
ರಜತಾಚಲಶೃಂಗಾಗ್ರಮಧ್ಯಸ್ಥಾಯೈ ನಮೋ ನಮಃ ।\\
ಹಿಮಾಚಲಮಹಾವಂಶಪಾವನಾಯೈ ನಮೋ ನಮಃ ।\\
ಶಂಕರಾರ್ಧಾಂಗಸೌಂದರ್ಯಶರೀರಾಯೈ ನಮೋ ನಮಃ ।\\
ಲಸನ್ಮರಕತಸ್ವಚ್ಛವಿಗ್ರಹಾಯೈ ನಮೋ ನಮಃ ।\\
ಮಹಾತಿಶಯಸೌಂದರ್ಯಲಾವಣ್ಯಾಯೈ ನಮೋ ನಮಃ ।\\
ಶಶಾಂಕಶೇಖರಪ್ರಾಣವಲ್ಲಭಾಯೈ ನಮೋ ನಮಃ ।\\
ಸದಾಪಂಚದಶಾತ್ಮೈಕ್ಯಸ್ವರೂಪಾಯೈ ನಮೋ ನಮಃ ।\\
ವಜ್ರಮಾಣಿಕ್ಯಕಟಕಕಿರೀಟಾಯೈ ನಮೋ ನಮಃ ।\\
ಕಸ್ತೂರೀತಿಲಕೋಲ್ಲಾಸನಿಟಿಲಾಯೈ ನಮೋ ನಮಃ ।\\
ಭಸ್ಮರೇಖಾಂಕಿತಲಸನ್ಮಸ್ತಕಾಯೈ ನಮೋ ನಮಃ ।

\as{ಅತ॑ಸ್ಸಮು॒ದ್ರಾ ಗಿ॒ರಯ॑ಶ್ಚ॒ ಸರ್ವೇ॒ಽಸ್ಮಾಥ್ಸ್ಯಂದಂ॑ತೇ॒ ಸಿಂಧ॑ವ॒ಸ್ಸರ್ವ॑ರೂಪಾಃ ।
ಅತ॑ಶ್ಚ॒ ವಿಶ್ವಾ॒ ಓಷ॑ಧಯೋ॒ ರಸಾ᳚ಚ್ಚ॒ ಯೇನೈ॑ಷ ಭೂ॒ತಸ್ತಿ॑ಷ್ಠತ್ಯಂತರಾ॒ತ್ಮಾ ॥}
\section{(ಹೃದಿ ಹಸ್ತಂ ದತ್ವಾ)\\
ಪಂಚದಶನಿತ್ಯಾಭ್ಯೋ ನಮಃ ।}
ಚತ್ವಾರಿಂಶದಧಿಕ ಚತುರ್ದಶಶತಶ್ವಾಸಾತ್ಮನೇ ಪ್ರತಿಪತ್ತಿಥಿರೂಪ ಕಾಮೇಶ್ವರೀನಿತ್ಯಾಯೈ ನಮಃ ।\\
ತದುತ್ತರ ಚತ್ವಾರಿಂಶದಧಿಕ ಚತುರ್ದಶಶತಶ್ವಾಸಾತ್ಮನೇ ದ್ವಿತೀಯಾತಿಥಿರೂಪ ಭಗಮಾಲಿನೀನಿತ್ಯಾಯೈ ನಮಃ ।\\
ತದುತ್ತರ ಚತ್ವಾರಿಂಶದಧಿಕ ಚತುರ್ದಶಶತಶ್ವಾಸಾತ್ಮನೇ ತೃತೀಯಾತಿಥಿರೂಪ ನಿತ್ಯಕ್ಲಿನ್ನಾನಿತ್ಯಾಯೈ ನಮಃ ।\\
ತದುತ್ತರ ಚತ್ವಾರಿಂಶದಧಿಕ ಚತುರ್ದಶಶತಶ್ವಾಸಾತ್ಮನೇ ಚತುರ್ಥೀತಿಥಿರೂಪ ಭೇರುಂಡಾನಿತ್ಯಾಯೈ ನಮಃ ।\\
ತದುತ್ತರ ಚತ್ವಾರಿಂಶದಧಿಕ ಚತುರ್ದಶಶತಶ್ವಾಸಾತ್ಮನೇ ಪಂಚಮೀತಿಥಿರೂಪ ವಹ್ನಿವಾಸಿನೀನಿತ್ಯಾಯೈ ನಮಃ ।\\
ತದುತ್ತರ ಚತ್ವಾರಿಂಶದಧಿಕ ಚತುರ್ದಶಶತಶ್ವಾಸಾತ್ಮನೇ ಷಷ್ಠೀತಿಥಿರೂಪ ಮಹಾವಜ್ರೇಶ್ವರೀನಿತ್ಯಾಯೈ ನಮಃ ।\\
ತದುತ್ತರ ಚತ್ವಾರಿಂಶದಧಿಕ ಚತುರ್ದಶಶತಶ್ವಾಸಾತ್ಮನೇ ಸಪ್ತಮೀತಿಥಿರೂಪ ಶಿವಾದೂತೀನಿತ್ಯಾಯೈ ನಮಃ ।\\
ತದುತ್ತರ ಚತ್ವಾರಿಂಶದಧಿಕ ಚತುರ್ದಶಶತಶ್ವಾಸಾತ್ಮನೇ ಅಷ್ಟಮೀತಿಥಿರೂಪ ತ್ವರಿತಾನಿತ್ಯಾಯೈ ನಮಃ ।\\
ತದುತ್ತರ ಚತ್ವಾರಿಂಶದಧಿಕ ಚತುರ್ದಶಶತಶ್ವಾಸಾತ್ಮನೇ ನವಮೀತಿಥಿರೂಪ ಕುಲಸುಂದರೀನಿತ್ಯಾಯೈ ನಮಃ ।\\
ತದುತ್ತರ ಚತ್ವಾರಿಂಶದಧಿಕ ಚತುರ್ದಶಶತಶ್ವಾಸಾತ್ಮನೇ ದಶಮೀತಿಥಿರೂಪ ನಿತ್ಯಾನಿತ್ಯಾಯೈ ನಮಃ ।\\
ತದುತ್ತರ ಚತ್ವಾರಿಂಶದಧಿಕ ಚತುರ್ದಶಶತಶ್ವಾಸಾತ್ಮನೇ ಏಕಾದಶೀತಿಥಿರೂಪ ನೀಲಪತಾಕಾನಿತ್ಯಾಯೈ ನಮಃ ।\\
ತದುತ್ತರ ಚತ್ವಾರಿಂಶದಧಿಕ ಚತುರ್ದಶಶತಶ್ವಾಸಾತ್ಮನೇ ದ್ವಾದಶೀತಿಥಿರೂಪ ವಿಜಯಾನಿತ್ಯಾಯೈ ನಮಃ ।\\
ತದುತ್ತರ ಚತ್ವಾರಿಂಶದಧಿಕ ಚತುರ್ದಶಶತಶ್ವಾಸಾತ್ಮನೇ ತ್ರಯೋದಶೀತಿಥಿರೂಪ ಸರ್ವಮಂಗಳಾನಿತ್ಯಾಯೈ ನಮಃ ।\\
ತದುತ್ತರ ಚತ್ವಾರಿಂಶದಧಿಕ ಚತುರ್ದಶಶತಶ್ವಾಸಾತ್ಮನೇ ಚತುರ್ದಶೀತಿಥಿರೂಪ ಜ್ವಾಲಾಮಾಲಿನೀನಿತ್ಯಾಯೈ ನಮಃ ।\\
ತದುತ್ತರ ಚತ್ವಾರಿಂಶದಧಿಕ ಚತುರ್ದಶಶತಶ್ವಾಸಾತ್ಮನೇ ಪೌರ್ಣಮಾಸೀತಿಥಿರೂಪ ಚಿತ್ರಾನಿತ್ಯಾಯೈ ನಮಃ ।

\as{(ಸಹಸ್ರನಾಮ ೧೦೧ - ೨೦೦)\\}
ಮಣಿ - ಪೂರಾಂತರುದಿತಾ ವಿಷ್ಣುಗ್ರಂಥಿ - ವಿಭೇದಿನೀ ॥೩೮॥

ಆಜ್ಞಾ - ಚಕ್ರಾಂತರಾಲಸ್ಥಾ ರುದ್ರಗ್ರಂಥಿ - ವಿಭೇದಿನೀ~।\\
ಸಹಸ್ರಾರಾಂಬುಜಾರೂಢಾ ಸುಧಾ - ಸಾರಾಭಿವರ್ಷಿಣೀ ॥೩೯॥

ತಡಿಲ್ಲತಾ - ಸಮರುಚಿಃ ಷಟ್‍ಚಕ್ರೋಪರಿ - ಸಂಸ್ಥಿತಾ~।\\
ಮಹಾಸಕ್ತಿಃ ಕುಂಡಲಿನೀ ಬಿಸತಂತು - ತನೀಯಸೀ ॥೪೦॥

ಭವಾನೀ ಭಾವನಾಗಮ್ಯಾ ಭವಾರಣ್ಯ - ಕುಠಾರಿಕಾ~।\\
ಭದ್ರಪ್ರಿಯಾ ಭದ್ರಮೂರ್ತಿರ್ಭಕ್ತ - ಸೌಭಾಗ್ಯದಾಯಿನೀ ॥೪೧॥

ಭಕ್ತಿಪ್ರಿಯಾ ಭಕ್ತಿಗಮ್ಯಾ ಭಕ್ತಿವಶ್ಯಾ ಭಯಾಪಹಾ~।\\
ಶಾಂಭವೀ ಶಾರದಾರಾಧ್ಯಾ ಶರ್ವಾಣೀ ಶರ್ಮದಾಯಿನೀ ॥೪೨॥

ಶಾಂಕರೀ ಶ್ರೀಕರೀ ಸಾಧ್ವೀ ಶರಚ್ಚಂದ್ರ - ನಿಭಾನನಾ~।\\
ಶಾತೋದರೀ ಶಾಂತಿಮತೀ ನಿರಾಧಾರಾ ನಿರಂಜನಾ ॥೪೩॥

ನಿರ್ಲೇಪಾ ನಿರ್ಮಲಾ ನಿತ್ಯಾ ನಿರಾಕಾರಾ ನಿರಾಕುಲಾ~।\\
ನಿರ್ಗುಣಾ ನಿಷ್ಕಲಾ ಶಾಂತಾ ನಿಷ್ಕಾಮಾ ನಿರುಪಪ್ಲವಾ ॥೪೪॥

ನಿತ್ಯಮುಕ್ತಾ ನಿರ್ವಿಕಾರಾ ನಿಷ್ಪ್ರಪಂಚಾ ನಿರಾಶ್ರಯಾ~।\\
ನಿತ್ಯಶುದ್ಧಾ ನಿತ್ಯಬುದ್ಧಾ ನಿರವದ್ಯಾ ನಿರಂತರಾ ॥೪೫॥

ನಿಷ್ಕಾರಣಾ ನಿಷ್ಕಲಂಕಾ ನಿರುಪಾಧಿರ್ನಿರೀಶ್ವರಾ~।\\
ನೀರಾಗಾ ರಾಗಮಥನೀ ನಿರ್ಮದಾ ಮದನಾಶಿನೀ ॥೪೬॥

ನಿಶ್ಚಿಂತಾ ನಿರಹಂಕಾರಾ ನಿರ್ಮೋಹಾ ಮೋಹನಾಶಿನೀ~।\\
ನಿರ್ಮಮಾ ಮಮತಾಹಂತ್ರೀ ನಿಷ್ಪಾಪಾ ಪಾಪನಾಶಿನೀ ॥೪೭॥

ನಿಷ್ಕ್ರೋಧಾ ಕ್ರೋಧಶಮನೀ ನಿರ್ಲೋಭಾ ಲೋಭನಾಶಿನೀ~।\\
ನಿಃಸಂಶಯಾ ಸಂಶಯಘ್ನೀ ನಿರ್ಭವಾ ಭವನಾಶಿನೀ ॥೪೮॥

ನಿರ್ವಿಕಲ್ಪಾ ನಿರಾಬಾಧಾ ನಿರ್ಭೇದಾ ಭೇದನಾಶಿನೀ~।\\
ನಿರ್ನಾಶಾ ಮೃತ್ಯುಮಥಿನೀ ನಿಷ್ಕ್ರಿಯಾ ನಿಷ್ಪರಿಗ್ರಹಾ~।೪೯॥

ನಿಸ್ತುಲಾ ನೀಲಚಿಕುರಾ ನಿರಪಾಯಾ ನಿರತ್ಯಯಾ~।\\
ದುರ್ಲಭಾ ದುರ್ಗಮಾ ದುರ್ಗಾ ದುಃಖಹಂತ್ರೀ ಸುಖಪ್ರದಾ ॥೫೦॥

ದುಷ್ಟದೂರಾ ದುರಾಚಾರ - ಶಮನೀ ದೋಷವರ್ಜಿತಾ~।\\
ಸರ್ವಜ್ಞಾ ಸಾಂದ್ರಕರುಣಾ ಸಮಾನಾಧಿಕ - ವರ್ಜಿತಾ ॥೫೧॥

ಸರ್ವಶಕ್ತಿಮಯೀ ಸರ್ವ - ಮಂಗಲಾ \as{(೨೦೦)}

\as{(ಅಷ್ಟೋತ್ತರ ೧೧ - ೨೦)\\}
ವಿಕಚಾಂಭೋರುಹದಲಲೋಚನಾಯೈ ನಮೋ ನಮಃ ।\\
ಶರಚ್ಚಾಂಪೇಯಪುಷ್ಪಾಭನಾಸಿಕಾಯೈ ನಮೋ ನಮಃ ।\\
ಲಸತ್ಕಾಂಚನತಾಟಂಕಯುಗಲಾಯೈ ನಮೋ ನಮಃ ।\\
ಮಣಿದರ್ಪಣಸಂಕಾಶಕಪೋಲಾಯೈ ನಮೋ ನಮಃ ।\\
ತಾಂಬೂಲಪೂರಿತಸ್ಮೇರವದನಾಯೈ ನಮೋ ನಮಃ ।\\
ಸುಪಕ್ವದಾಡಿಮೀಬೀಜರದನಾಯೈ ನಮೋ ನಮಃ ।\\
ಕಂಬುಪೂಗಸಮಚ್ಛಾಯಕಂಧರಾಯೈ ನಮೋ ನಮಃ ।\\
ಸ್ಥೂಲಮುಕ್ತಾಫಲೋದಾರಸುಹಾರಾಯೈ ನಮೋ ನಮಃ ।\\
ಗಿರೀಶಬದ್ಧಮಾಂಗಲ್ಯಮಂಗಲಾಯೈ ನಮೋ ನಮಃ ।\\
ಪದ್ಮಪಾಶಾಂಕುಶಲಸತ್ಕರಾಬ್ಜಾಯೈ ನಮೋ ನಮಃ ।	

\as{ಬ್ರ॒ಹ್ಮಾ ದೇ॒ವಾನಾಂ᳚ ಪದ॒ವೀಃ ಕ॑ವೀ॒ನಾಮೃಷಿ॒ರ್ವಿಪ್ರಾ॑ಣಾಂ ಮಹಿ॒ಷೋ ಮೃ॒ಗಾಣಾಂ᳚ ।
ಶ್ಯೇ॒ನೋ ಗೃದ್ಧ್ರಾ॑ಣಾ॒ಁ ಸ್ವಧಿ॑ತಿ॒ರ್ವನಾ॑ನಾ॒ಁ ಸೋಮಃ॑ ಪ॒ವಿತ್ರ॒ಮತ್ಯೇ॑ತಿ॒ ರೇಭನ್॑ ॥}
\section{ಚತುರಶ್ರಾದ್ಯರೇಖಾಯೈ ನಮಃ ।\\ (ಇತಿ ವ್ಯಾಪಕಂ ನ್ಯಸ್ಯ)}
ದಕ್ಷಾಂಸಪೃಷ್ಠರೂಪ ಶಾಂತರಸಾತ್ಮನೇ ಅಣಿಮಾಸಿದ್ಧ್ಯೈ ನಮಃ ।\\
ದಕ್ಷಪಾಣ್ಯಂಗುಲ್ಯಗ್ರರೂಪ ಅದ್ಭುತರಸಾತ್ಮನೇ ಲಘಿಮಾಸಿದ್ಧ್ಯೈ ನಮಃ ।\\
ದಕ್ಷಸ್ಫಿಗ್ರೂಪ ಕರುಣರಸಾತ್ಮನೇ ಮಹಿಮಾಸಿದ್ಧ್ಯೈ ನಮಃ ।\\
ದಕ್ಷಪಾದಾಂಗುಲ್ಯಗ್ರರೂಪ ವೀರರಸಾತ್ಮನೇ ಈಶಿತ್ವಸಿದ್ಧ್ಯೈ ನಮಃ ।\\
ವಾಮಪಾದಾಂಗುಲ್ಯಗ್ರರೂಪ ಹಾಸ್ಯರಸಾತ್ಮನೇ ವಶಿತ್ವಸಿದ್ಧ್ಯೈ ನಮಃ ।\\
ವಾಮಸ್ಫಿಗ್ರೂಪ ಬೀಭತ್ಸರಸಾತ್ಮನೇ ಪ್ರಾಕಾಮ್ಯಸಿದ್ಧ್ಯೈ ನಮಃ ।\\
ವಾಮಪಾಣ್ಯಂಗುಲ್ಯಗ್ರರೂಪ ರೌದ್ರರಸಾತ್ಮನೇ ಭುಕ್ತಿಸಿದ್ಧ್ಯೈ ನಮಃ ।\\
ವಾಮಾಂಸಪೃಷ್ಠರೂಪ ಭಯಾನಕರಸಾತ್ಮನೇ ಇಚ್ಛಾಸಿದ್ಧ್ಯೈ ನಮಃ ।\\
ಚೂಲೀಮೂಲರೂಪ ಶೃಂಗಾರರಸಾತ್ಮನೇ ಪ್ರಾಪ್ತಿಸಿದ್ಧ್ಯೈ ನಮಃ ।\\
ಚೂಲೀಪೃಷ್ಠರೂಪ ನಿಯತ್ಯಾತ್ಮನೇ ಸರ್ವಕಾಮಸಿದ್ಧ್ಯೈ ನಮಃ ।

\as{ಅ॒ಜಾಮೇಕಾಂ॒ ಲೋಹಿ॑ತಶುಕ್ಲಕೃ॒ಷ್ಣಾಂ ಬ॒ಹ್ವೀಂ ಪ್ರ॒ಜಾಂ ಜ॒ನಯಂ॑ತೀಁ॒ ಸರೂ॑ಪಾಂ ।
ಅ॒ಜೋ ಹ್ಯೇಕೋ॑ ಜು॒ಷಮಾ॑ಣೋಽನು॒ಶೇತೇ॒ ಜಹಾ᳚ತ್ಯೇನಾಂ ಭು॒ಕ್ತಭೋ॑ಗಾ॒ಮಜೋ᳚ಽನ್ಯಃ ॥}
\section{ಚತುರಶ್ರ ಮಧ್ಯರೇಖಾಯೈ ನಮಃ ।\\ (ಇತಿ ವ್ಯಾಪಕಂ ನ್ಯಸ್ಯ)}
ಪಾದಾಂಗುಷ್ಠದ್ವಯರೂಪ ಕಾಮಾತ್ಮನೇ ಬ್ರಾಹ್ಮ್ಯೈ ನಮಃ ।\\
ದಕ್ಷಪಾರ್ಶ್ವರೂಪ ಕ್ರೋಧಾತ್ಮನೇ ಮಾಹೇಶ್ವರ್ಯೈ ನಮಃ ।\\
ಮೂರ್ಧರೂಪ ಲೋಭಾತ್ಮನೇ ಕೌಮಾರ್ಯೈ ನಮಃ ।\\
ವಾಮಪಾರ್ಶ್ವರೂಪ ಮೋಹಾತ್ಮನೇ ವೈಷ್ಣವ್ಯೈ ನಮಃ ।\\
ವಾಮಜಾನುರೂಪ ಮದಾತ್ಮನೇ ವಾರಾಹ್ಯೈ ನಮಃ ।\\
ದಕ್ಷಜಾನುರೂಪ ಮಾತ್ಸರ್ಯಾತ್ಮನೇ ಮಾಹೇಂದ್ರ್ಯೈ ನಮಃ ।\\
ದಕ್ಷಬಹಿರಂಸರೂಪ ಪುಣ್ಯಾತ್ಮನೇ ಚಾಮುಂಡಾಯೈ ನಮಃ ।\\
ವಾಮಬಹಿರಂಸರೂಪ ಪಾಪಾತ್ಮನೇ ಮಹಾಲಕ್ಷ್ಮ್ಯೈ ನಮಃ ।

\as{ಹಁ॒ಸಶ್ಶು॑ಚಿ॒ಷದ್ವಸು॑ರಂತರಿಕ್ಷ॒ಸದ್ಧೋತಾ॑ ವೇದಿ॒ಷದತಿ॑ಥಿರ್ದುರೋಣ॒ಸತ್ ।
ನೃ॒ಷದ್ವ॑ರ॒ ಸದೃ॑ತ॒ ಸದ್ವ್ಯೋ॑ಮ॒ ಸದ॒ಬ್ಜಾ ಗೋ॒ಜಾ ಋ॑ತ॒ಜಾ ಅ॑ದ್ರಿ॒ಜಾ ಋ॒ತಂ ಬೃ॒ಹತ್ ॥}
\section{ಚತುರಶ್ರಾಂತ್ಯರೇಖಾಯೈ ನಮಃ ।\\ (ಇತಿ ವ್ಯಾಪಕಂ ನ್ಯಸ್ಯ)}
ಪಾದಾಂಗುಷ್ಠದ್ವಯರೂಪ ಸಹಸ್ರದಲಕಮಲಾತ್ಮನೇ ಸರ್ವಸಂಕ್ಷೋಭಿಣೀಮುದ್ರಾಯೈ ನಮಃ ।\\
ದಕ್ಷಪಾರ್ಶ್ವರೂಪ ಮೂಲಾಧಾರಾತ್ಮನೇ ಸರ್ವವಿದ್ರಾವಿಣೀಮುದ್ರಾಯೈ ನಮಃ ।\\
ಮೂರ್ಧರೂಪ ಸ್ವಾಧಿಷ್ಠಾನಾತ್ಮನೇ ಸರ್ವಾಕರ್ಷಿಣೀಮುದ್ರಾಯೈ ನಮಃ ।\\
ವಾಮಪಾರ್ಶ್ವರೂಪ ಮಣಿಪೂರಾತ್ಮನೇ ಸರ್ವವಶಂಕರೀಮುದ್ರಾಯೈ ನಮಃ ।\\
ವಾಮಜಾನುರೂಪ ಅನಾಹತಾತ್ಮನೇ ಸರ್ವೋನ್ಮಾದಿನೀಮುದ್ರಾಯೈ ನಮಃ ।\\
ದಕ್ಷಜಾನುರೂಪ ವಿಶುದ್ಧ್ಯಾತ್ಮನೇ ಸರ್ವಮಹಾಂಕುಶಾಮುದ್ರಾಯೈ ನಮಃ ।\\
ದಕ್ಷಾಂತರಂಸರೂಪ ಇಂದ್ರಯೋನ್ಯಾತ್ಮನೇ ಸರ್ವಖೇಚರೀಮುದ್ರಾಯೈ ನಮಃ ।\\
ವಾಮಾಂತರಂಸರೂಪ ಆಜ್ಞಾತ್ಮನೇ ಸರ್ವಬೀಜಾಮುದ್ರಾಯೈ ನಮಃ ।\\
ದ್ವಾದಶಾಂತರೂಪ ಊರ್ಧ್ವಸಹಸ್ರದಲಕಮಲಾತ್ಮನೇ ಸರ್ವಯೋನಿಮುದ್ರಾಯೈ ನಮಃ ।\\
ಪಾದಾಂಗುಷ್ಠದ್ವಯರೂಪ ಆಧಾರನವಕಾತ್ಮನೇ ಸರ್ವತ್ರಿಖಂಡಾಮುದ್ರಾಯೈ ನಮಃ ।\\
ಹೃದ್ರೂಪ ತ್ರೈಲೋಕ್ಯಮೋಹನಚಕ್ರೇಶ್ವರ್ಯೈ ತ್ರಿಪುರಾಯೈ ನಮಃ ।\\
ಪ್ರಕಟಯೋಗಿನೀರೂಪ ಸ್ವಾತ್ಮಾತ್ಮನೇ ಅಣಿಮಾಸಿದ್ಧ್ಯೈ ನಮಃ ।\\
ಅಪರಿಚ್ಛಿನ್ನರೂಪ ಸ್ವಾತ್ಮಾತ್ಮನೇ ಸರ್ವಸಂಕ್ಷೋಭಿಣೀಮುದ್ರಾಯೈ ನಮಃ ।

\as{(ಸಹಸ್ರನಾಮ ೨೦೧ - ೩೦೦)\\}
ಸದ್ಗತಿಪ್ರದಾ~।\\
ಸರ್ವೇಶ್ವರೀ ಸರ್ವಮಯೀ ಸರ್ವಮಂತ್ರ - ಸ್ವರೂಪಿಣೀ ॥೫೨॥

ಸರ್ವ - ಯಂತ್ರಾತ್ಮಿಕಾ ಸರ್ವ - ತಂತ್ರರೂಪಾ ಮನೋನ್ಮನೀ~।\\
ಮಾಹೇಶ್ವರೀ ಮಹಾದೇವೀ ಮಹಾಲಕ್ಷ್ಮೀರ್ಮೃಡಪ್ರಿಯಾ ॥೫೩॥

ಮಹಾರೂಪಾ ಮಹಾಪೂಜ್ಯಾ ಮಹಾಪಾತಕ - ನಾಶಿನೀ~।\\
ಮಹಾಮಾಯಾ ಮಹಾಸತ್ತ್ವಾ ಮಹಾಶಕ್ತಿರ್ಮಹಾರತಿಃ ॥೫೪॥

ಮಹಾಭೋಗಾ ಮಹೈಶ್ವರ್ಯಾ ಮಹಾವೀರ್ಯಾ ಮಹಾಬಲಾ~।\\
ಮಹಾಬುದ್ಧಿರ್ಮಹಾಸಿದ್ಧಿರ್ಮಹಾಯೋಗೇಶ್ವರೇಶ್ವರೀ ॥೫೫॥

ಮಹಾತಂತ್ರಾ ಮಹಾಮಂತ್ರಾ ಮಹಾಯಂತ್ರಾ ಮಹಾಸನಾ~।\\
ಮಹಾಯಾಗ - ಕ್ರಮಾರಾಧ್ಯಾ ಮಹಾಭೈರವ - ಪೂಜಿತಾ ॥೫೬॥

ಮಹೇಶ್ವರ - ಮಹಾಕಲ್ಪ - ಮಹಾತಾಂಡವ - ಸಾಕ್ಷಿಣೀ~।\\
ಮಹಾಕಾಮೇಶ - ಮಹಿಷೀ ಮಹಾತ್ರಿಪುರ - ಸುಂದರೀ ॥೫೭॥

ಚತುಃಷಷ್ಟ್ಯುಪಚಾರಾಢ್ಯಾ ಚತುಃಷಷ್ಟಿಕಲಾಮಯೀ~।\\
ಮಹಾಚತುಃ - ಷಷ್ಟಿಕೋಟಿ - ಯೋಗಿನೀ - ಗಣಸೇವಿತಾ ॥೫೮॥

ಮನುವಿದ್ಯಾ ಚಂದ್ರವಿದ್ಯಾ ಚಂದ್ರಮಂಡಲ - ಮಧ್ಯಗಾ~।\\
ಚಾರುರೂಪಾ ಚಾರುಹಾಸಾ ಚಾರುಚಂದ್ರ - ಕಲಾಧರಾ ॥೫೯॥

ಚರಾಚರ - ಜಗನ್ನಾಥಾ ಚಕ್ರರಾಜ - ನಿಕೇತನಾ~।\\
ಪಾರ್ವತೀ ಪದ್ಮನಯನಾ ಪದ್ಮರಾಗ - ಸಮಪ್ರಭಾ ॥೬೦॥

ಪಂಚ - ಪ್ರೇತಾಸನಾಸೀನಾ ಪಂಚಬ್ರಹ್ಮ - ಸ್ವರೂಪಿಣೀ~।\\
ಚಿನ್ಮಯೀ ಪರಮಾನಂದಾ ವಿಜ್ಞಾನ - ಘನರೂಪಿಣೀ ॥೬೧॥

ಧ್ಯಾನ - ಧ್ಯಾತೃ - ಧ್ಯೇಯರೂಪಾ ಧರ್ಮಾಧರ್ಮ - ವಿವರ್ಜಿತಾ~।\\
ವಿಶ್ವರೂಪಾ ಜಾಗರಿಣೀ ಸ್ವಪಂತೀ ತೈಜಸಾತ್ಮಿಕಾ ॥೬೨॥

ಸುಪ್ತಾ ಪ್ರಾಜ್ಞಾತ್ಮಿಕಾ ತುರ್ಯಾ ಸರ್ವಾವಸ್ಥಾ - ವಿವರ್ಜಿತಾ~।\\
ಸೃಷ್ಟಿಕರ್ತ್ರೀ ಬ್ರಹ್ಮರೂಪಾ ಗೋಪ್ತ್ರೀ ಗೋವಿಂದರೂಪಿಣೀ ॥೬೩॥

ಸಂಹಾರಿಣೀ ರುದ್ರರೂಪಾ ತಿರೋಧಾನ - ಕರೀಶ್ವರೀ~।\\
ಸದಾಶಿವಾಽನುಗ್ರಹದಾ ಪಂಚಕೃತ್ಯ - ಪರಾಯಣಾ ॥೬೪॥

ಭಾನುಮಂಡಲ - ಮಧ್ಯಸ್ಥಾ ಭೈರವೀ ಭಗಮಾಲಿನೀ~।\\
ಪದ್ಮಾಸನಾ ಭಗವತೀ ಪದ್ಮನಾಭ - ಸಹೋದರೀ ॥೬೫॥

ಉನ್ಮೇಷ - ನಿಮಿಷೋತ್ಪನ್ನ - ವಿಪನ್ನ - ಭುವನಾವಲಿಃ~।\\
ಸಹಸ್ರ - ಶೀರ್ಷವದನಾ ಸಹಸ್ರಾಕ್ಷೀ ಸಹಸ್ರಪಾತ್ ॥೬೬॥

ಆಬ್ರಹ್ಮ - ಕೀಟ - ಜನನೀ ವರ್ಣಾಶ್ರಮ - ವಿಧಾಯಿನೀ~।\\
ನಿಜಾಜ್ಞಾರೂಪ - ನಿಗಮಾ ಪುಣ್ಯಾಪುಣ್ಯ - ಫಲಪ್ರದಾ ॥೬೭॥

ಶ್ರುತಿ - ಸೀಮಂತ - ಸಿಂದೂರೀ - ಕೃತ - ಪಾದಾಬ್ಜ - ಧೂಲಿಕಾ~।\\
ಸಕಲಾಗಮ - ಸಂದೋಹ - ಶುಕ್ತಿ - ಸಂಪುಟ - ಮೌಕ್ತಿಕಾ ॥೬೮॥

ಪುರುಷಾರ್ಥಪ್ರದಾ ಪೂರ್ಣಾ ಭೋಗಿನೀ ಭುವನೇಶ್ವರೀ~।\\
ಅಂಬಿಕಾಽಽನಾದಿ - ನಿಧನಾ ಹರಿಬ್ರಹ್ಮೇಂದ್ರ - ಸೇವಿತಾ ॥೬೯॥

ನಾರಾಯಣೀ ನಾದರೂಪಾ ನಾಮರೂಪ - ವಿವರ್ಜಿತಾ \as{(೩೦೦)}।\\

\as{(ಅಷ್ಟೋತ್ತರ ೨೧ - ೩೦)\\}
ಪದ್ಮಕೈರವಮಂದಾರಸುಮಾಲಿನ್ಯೈ ನಮೋ ನಮಃ ।\\
ಸುವರ್ಣಕುಂಭಯುಗ್ಮಾಭಸುಕುಚಾಯೈ ನಮೋ ನಮಃ ।\\
ರಮಣೀಯಚತುರ್ಬಾಹುಸಂಯುಕ್ತಾಯೈ ನಮೋ ನಮಃ ।\\
ಕನಕಾಂಗದಕೇಯೂರಭೂಷಿತಾಯೈ ನಮೋ ನಮಃ ।\\
ಬೃಹತ್ಸೌವರ್ಣಸೌಂದರ್ಯವಸನಾಯೈ ನಮೋ ನಮಃ ।\\
ಬೃಹನ್ನಿತಂಬವಿಲಸಜ್ಜಘನಾಯೈ ನಮೋ ನಮಃ ।\\
ಸೌಭಾಗ್ಯಜಾತಶೃಂಗಾರಮಧ್ಯಮಾಯೈ ನಮೋ ನಮಃ ।\\
ದಿವ್ಯಭೂಷಣಸಂದೋಹರಂಜಿತಾಯೈ ನಮೋ ನಮಃ ।\\
ಪಾರಿಜಾತಗುಣಾಧಿಕ್ಯಪದಾಬ್ಜಾಯೈ ನಮೋ ನಮಃ ।\\
ಸುಪದ್ಮರಾಗಸಂಕಾಶಚರಣಾಯೈ ನಮೋ ನಮಃ ।

\as{ಘೃ॒ತಂ ಮಿ॑ಮಿಕ್ಷಿರೇ ಘೃ॒ತಮ॑ಸ್ಯ॒ಯೋನಿ॑ರ್ಘೃ॒ತೇಶ್ರಿ॒ತೋ ಘೃ॒ತಮು॑ವಸ್ಯ॒ ಧಾಮ॑ ।
ಅ॒ನು॒ಷ್ವ॒ಧಮಾವ॑ಹ ಮಾ॒ದಯ॑ಸ್ವ॒ ಸ್ವಾಹಾ॑ಕೃತಂ ವೃಷಭ ವಕ್ಷಿ ಹ॒ವ್ಯಂ ॥}
\section{ಷೋಡಶದಲಕಮಲಾಯ ನಮಃ ।\\ (ಇತಿ ವ್ಯಾಪಕಂ ನ್ಯಸ್ಯ)}
ದಕ್ಷಶ್ರೋತ್ರಪೃಷ್ಠರೂಪ ಪೃಥಿವ್ಯಾತ್ಮನೇ ಕಾಮಾಕರ್ಷಣೀ ನಿತ್ಯಾಕಲಾಯೈ ನಮಃ ।\\
ದಕ್ಷಾಂಸರೂಪಾಬಾತ್ಮನೇ ಬುದ್ಧ್ಯಾಕರ್ಷಣೀ ನಿತ್ಯಾಕಲಾಯೈ ನಮಃ ।\\
ದಕ್ಷಕೂರ್ಪರರೂಪ ತೇಜ ಆತ್ಮನೇ ಅಹಂಕಾರಾಕರ್ಷಣೀ ನಿತ್ಯಾಕಲಾಯೈ ನಮಃ ।\\
ದಕ್ಷಕರಪೃಷ್ಠರೂಪ ವಾಯ್ವಾತ್ಮನೇ ಶಬ್ದಾಕರ್ಷಣೀ ನಿತ್ಯಾಕಲಾಯೈ ನಮಃ ।\\
ದಕ್ಷೋರುರೂಪ ಆಕಾಶಾತ್ಮನೇ ಸ್ಪರ್ಶಾಕರ್ಷಣೀ ನಿತ್ಯಾಕಲಾಯೈ ನಮಃ ।\\
ದಕ್ಷಜಾನುರೂಪ ಶ್ರೋತ್ರಾತ್ಮನೇ ರೂಪಾಕರ್ಷಣೀ ನಿತ್ಯಾಕಲಾಯೈ ನಮಃ ।\\
ದಕ್ಷಗುಲ್ಫರೂಪ ತ್ವಗಾತ್ಮನೇ ರಸಾಕರ್ಷಣೀ ನಿತ್ಯಾಕಲಾಯೈ ನಮಃ ।\\
ದಕ್ಷಪಾದತಲರೂಪ ಚಕ್ಷುರಾತ್ಮನೇ ಗಂಧಾಕರ್ಷಣೀ ನಿತ್ಯಾಕಲಾಯೈ ನಮಃ ।\\
ವಾಮಪಾದತಲರೂಪ ಜಿಹ್ವಾತ್ಮನೇ ಚಿತ್ತಾಕರ್ಷಣೀ ನಿತ್ಯಾಕಲಾಯೈ ನಮಃ ।\\
ವಾಮಗುಲ್ಫರೂಪ ಘ್ರಾಣಾತ್ಮನೇ ಧೈರ್ಯಾಕರ್ಷಣೀ ನಿತ್ಯಾಕಲಾಯೈ ನಮಃ ।\\
ವಾಮಜಾನುರೂಪ ವಾಗಾತ್ಮನೇ ಸ್ಮೃತ್ಯಾಕರ್ಷಣೀ ನಿತ್ಯಾಕಲಾಯೈ ನಮಃ ।\\
ವಾಮೋರುರೂಪ ಪಾಣ್ಯಾತ್ಮನೇ ನಾಮಾಕರ್ಷಣೀ ನಿತ್ಯಾಕಲಾಯೈ ನಮಃ ।\\
ವಾಮಕರಪೃಷ್ಠರೂಪ ಪಾದಾತ್ಮನೇ ಬೀಜಾಕರ್ಷಣೀ ನಿತ್ಯಾಕಲಾಯೈ ನಮಃ ।\\
ವಾಮಕೂರ್ಪರರೂಪ ಪಾಯ್ವಾತ್ಮನೇ ಆತ್ಮಾಕರ್ಷಣೀ ನಿತ್ಯಾಕಲಾಯೈ ನಮಃ ।\\
ವಾಮಾಂಸರೂಪ ಉಪಸ್ಥಾತ್ಮನೇ ಅಮೃತಾಕರ್ಷಣೀ ನಿತ್ಯಾಕಲಾಯೈ ನಮಃ ।\\
ವಾಮಶ್ರೋತ್ರಪೃಷ್ಠರೂಪ ವಿಕೃತಮನಆತ್ಮನೇ ಶರೀರಾಕರ್ಷಣೀ ನಿತ್ಯಾಕಲಾಯೈ ನಮಃ ।\\
ಹೃದ್ರೂಪ ಸರ್ವಾಶಾಪರಿಪೂರಕಚಕ್ರೇಶ್ವರ್ಯೈ ತ್ರಿಪುರೇಶ್ಯೈ ನಮಃ ।\\
ಗುಪ್ತಯೋಗಿನೀರೂಪ ಸ್ವಾತ್ಮಾತ್ಮನೇ ಲಘಿಮಾಸಿದ್ಧ್ಯೈ ನಮಃ ।\\
ಅಪರಿಚ್ಛಿನ್ನರೂಪ ಸ್ವಾತ್ಮಾತ್ಮನೇ ಸರ್ವವಿದ್ರಾವಿಣೀಮುದ್ರಾಯೈ ನಮಃ ।

\as{(ಸಹಸ್ರನಾಮ ೩೦೧ - ೪೦೦)\\}
ಹ್ರೀಂಕಾರೀ ಹ್ರೀಂಮತೀ ಹೃದ್ಯಾ ಹೇಯೋಪಾದೇಯ - ವರ್ಜಿತಾ ॥೭೦॥

ರಾಜರಾಜಾರ್ಚಿತಾ ರಾಜ್ಞೀ ರಮ್ಯಾ ರಾಜೀವಲೋಚನಾ~।\\
ರಂಜನೀ ರಮಣೀ ರಸ್ಯಾ ರಣತ್ಕಿಂಕಿಣಿ - ಮೇಖಲಾ ॥೭೧॥

ರಮಾ ರಾಕೇಂದುವದನಾ ರತಿರೂಪಾ ರತಿಪ್ರಿಯಾ~।\\
ರಕ್ಷಾಕರೀ ರಾಕ್ಷಸಘ್ನೀ ರಾಮಾ ರಮಣಲಂಪಟಾ ॥೭೨॥

ಕಾಮ್ಯಾ ಕಾಮಕಲಾರೂಪಾ ಕದಂಬ - ಕುಸುಮ - ಪ್ರಿಯಾ~।\\
ಕಲ್ಯಾಣೀ ಜಗತೀಕಂದಾ ಕರುಣಾ - ರಸ - ಸಾಗರಾ ॥೭೩॥

ಕಲಾವತೀ ಕಲಾಲಾಪಾ ಕಾಂತಾ ಕಾದಂಬರೀಪ್ರಿಯಾ~।\\
ವರದಾ ವಾಮನಯನಾ ವಾರುಣೀ - ಮದ - ವಿಹ್ವಲಾ ॥೭೪॥

ವಿಶ್ವಾಧಿಕಾ ವೇದವೇದ್ಯಾ ವಿಂಧ್ಯಾಚಲ - ನಿವಾಸಿನೀ~।\\
ವಿಧಾತ್ರೀ ವೇದಜನನೀ ವಿಷ್ಣುಮಾಯಾ ವಿಲಾಸಿನೀ ॥೭೫॥

ಕ್ಷೇತ್ರಸ್ವರೂಪಾ ಕ್ಷೇತ್ರೇಶೀ ಕ್ಷೇತ್ರ - ಕ್ಷೇತ್ರಜ್ಞ - ಪಾಲಿನೀ~।\\
ಕ್ಷಯವೃದ್ಧಿ - ವಿನಿರ್ಮುಕ್ತಾ ಕ್ಷೇತ್ರಪಾಲ - ಸಮರ್ಚಿತಾ ॥೭೬॥

ವಿಜಯಾ ವಿಮಲಾ ವಂದ್ಯಾ ವಂದಾರು - ಜನ - ವತ್ಸಲಾ~।\\
ವಾಗ್ವಾದಿನೀ ವಾಮಕೇಶೀ ವಹ್ನಿಮಂಡಲ - ವಾಸಿನೀ ॥೭೭॥

ಭಕ್ತಿಮತ್ - ಕಲ್ಪಲತಿಕಾ ಪಶುಪಾಶ - ವಿಮೋಚಿನೀ~।\\
ಸಂಹೃತಾಶೇಷ - ಪಾಷಂಡಾ ಸದಾಚಾರ - ಪ್ರವರ್ತಿಕಾ ॥೭೮॥

ತಾಪತ್ರಯಾಗ್ನಿ - ಸಂತಪ್ತ - ಸಮಾಹ್ಲಾದನ ಚಂದ್ರಿಕಾ~।\\
ತರುಣೀ ತಾಪಸಾರಾಧ್ಯಾ ತನುಮಧ್ಯಾ ತಮೋಽಪಹಾ ॥೭೯॥

ಚಿತಿಸ್ತತ್ಪದ - ಲಕ್ಷ್ಯಾರ್ಥಾ ಚಿದೇಕರಸ - ರೂಪಿಣೀ~।\\
ಸ್ವಾತ್ಮಾನಂದ - ಲವೀಭೂತ - ಬ್ರಹ್ಮಾದ್ಯಾನಂದ - ಸಂತತಿಃ ॥೮೦॥

ಪರಾ ಪ್ರತ್ಯಕ್ಚಿತೀರೂಪಾ ಪಶ್ಯಂತೀ ಪರದೇವತಾ~।\\
ಮಧ್ಯಮಾ ವೈಖರೀರೂಪಾ ಭಕ್ತ - ಮಾನಸ - ಹಂಸಿಕಾ ॥೮೧॥

ಕಾಮೇಶ್ವರ - ಪ್ರಾಣನಾಡೀ ಕೃತಜ್ಞಾ ಕಾಮಪೂಜಿತಾ~।\\
ಶೃಂಗಾರ - ರಸ - ಸಂಪೂರ್ಣಾ ಜಯಾ ಜಾಲಂಧರ - ಸ್ಥಿತಾ ॥೮೨॥

ಓಡ್ಯಾಣಪೀಠ - ನಿಲಯಾ ಬಿಂದು - ಮಂಡಲವಾಸಿನೀ~।\\
ರಹೋಯಾಗ - ಕ್ರಮಾರಾಧ್ಯಾ ರಹಸ್ತರ್ಪಣ - ತರ್ಪಿತಾ ॥೮೩॥

ಸದ್ಯಃಪ್ರಸಾದಿನೀ ವಿಶ್ವ - ಸಾಕ್ಷಿಣೀ ಸಾಕ್ಷಿವರ್ಜಿತಾ~।\\
ಷಡಂಗದೇವತಾ - ಯುಕ್ತಾ ಷಾಡ್ಗುಣ್ಯ - ಪರಿಪೂರಿತಾ ॥೮೪॥

ನಿತ್ಯಕ್ಲಿನ್ನಾ ನಿರುಪಮಾ ನಿರ್ವಾಣ - ಸುಖ - ದಾಯಿನೀ~।\\
ನಿತ್ಯಾ - ಷೋಡಶಿಕಾ - ರೂಪಾ ಶ್ರೀಕಂಠಾರ್ಧ - ಶರೀರಿಣೀ ॥೮೫॥

ಪ್ರಭಾವತೀ ಪ್ರಭಾರೂಪಾ ಪ್ರಸಿದ್ಧಾ ಪರಮೇಶ್ವರೀ~।\\
ಮೂಲಪ್ರಕೃತಿರವ್ಯಕ್ತಾ ವ್ಯಕ್ತಾವ್ಯಕ್ತ - ಸ್ವರೂಪಿಣೀ ॥೮೬॥

ವ್ಯಾಪಿನೀ \as{(೪೦೦)}

\as{(ಅಷ್ಟೋತ್ತರ ೪೦ - ೫೦)\\}
ಕಾಮಕೋಟಿಮಹಾಪದ್ಮಪೀಠಸ್ಥಾಯೈ ನಮೋ ನಮಃ ।\\
ಶ್ರೀಕಂಠನೇತ್ರಕುಮುದಚಂದ್ರಿಕಾಯೈ ನಮೋ ನಮಃ ।\\
ಸಚಾಮರ ರಮಾವಾಣೀವೀಜಿತಾಯೈ ನಮೋ ನಮಃ ।\\
ಭಕ್ತರಕ್ಷಣದಾಕ್ಷಿಣ್ಯಕಟಾಕ್ಷಾಯೈ ನಮೋ ನಮಃ ।\\
ಭೂತೇಶಾಲಿಂಗನೋದ್ಭೂತಪುಲಕಾಂಗ್ಯೈ ನಮೋ ನಮಃ ।\\
ಅನಂಗಜನಕಾಪಾಂಗವೀಕ್ಷಣಾಯೈ ನಮೋ ನಮಃ ।\\
ಬ್ರಹ್ಮೋಪೇಂದ್ರಶಿರೋರತ್ನರಂಜಿತಾಯೈ ನಮೋ ನಮಃ ।\\
ಶಚೀಮುಖ್ಯಾಮರವಧೂಸೇವಿತಾಯೈ ನಮೋ ನಮಃ ।\\
ಲೀಲಾಕಲ್ಪಿತಬ್ರಹ್ಮಾಂಡಮಂಡಲಾಯೈ ನಮೋ ನಮಃ ।\\
ಅಮೃತಾದಿಮಹಾಶಕ್ತಿಸಂವೃತಾಯೈ ನಮೋ ನಮಃ ।

\as{ಸ॒ಮು॒ದ್ರಾದೂ॒ರ್ಮಿರ್ಮಧು॑ಮಾಁ॒ ಉದಾ॑ರದುಪಾಁ॒ ಶುನಾ॒ ಸಮ॑ಮೃತ॒ತ್ವಮಾ॑ನಟ್ ।
ಘೃ॒ತಸ್ಯ॒ ನಾಮ॒ ಗುಹ್ಯಂ॒ ಯದಸ್ತಿ॑ ಜಿ॒ಹ್ವಾ ದೇ॒ವಾನಾ॑ಮ॒ಮೃತ॑ಸ್ಯ॒ ನಾಭಿಃ॑ ॥}
\section{ಅಷ್ಟದಲಪದ್ಮಾಯ ನಮಃ ।\\ (ಇತಿ ವ್ಯಾಪಕಂ ನ್ಯಸ್ಯ)}
ದಕ್ಷಶಂಖರೂಪ ವಚನಾತ್ಮನೇ ಅನಂಗಕುಸುಮಾದೇವ್ಯೈ ನಮಃ ।\\
ದಕ್ಷಬಾಹುಮೂಲರೂಪಾದಾನಾತ್ಮನೇ ಅನಂಗಮೇಖಲಾದೇವ್ಯೈ ನಮಃ ।\\
ದಕ್ಷೋರುರೂಪ ಗಮನಾತ್ಮನೇ ಅನಂಗಮದನಾದೇವ್ಯೈ ನಮಃ ।\\
ದಕ್ಷಗುಲ್ಫರೂಪ ವಿಸರ್ಗಾತ್ಮನೇ ಅನಂಗಮದನಾತುರಾದೇವ್ಯೈ ನಮಃ ।\\
ವಾಮಗುಲ್ಫರೂಪ ಆನಂದಾತ್ಮನೇ ಅನಂಗರೇಖಾದೇವ್ಯೈ ನಮಃ ।\\
ವಾಮೋರುರೂಪ ಹಾನಾಖ್ಯಬುದ್ಧ್ಯಾತ್ಮನೇ ಅನಂಗವೇಗಿನೀದೇವ್ಯೈ ನಮಃ ।\\
ವಾಮಬಾಹುಮೂಲರೂಪ ಉಪಾದಾನಾಖ್ಯಬುದ್ಧ್ಯಾತ್ಮನೇ ಅನಂಗಾಂಕುಶಾದೇವ್ಯೈ ನಮಃ ।\\
ವಾಮಶಂಖರೂಪ ಉಪೇಕ್ಷಾಖ್ಯಬುದ್ಧ್ಯಾತ್ಮನೇ ಅನಂಗಮಾಲಿನೀದೇವ್ಯೈ ನಮಃ ।\\
ಹೃದ್ರೂಪ ಸರ್ವಸಂಕ್ಷೋಭಣಚಕ್ರೇಶ್ವರ್ಯೈ ತ್ರಿಪುರಸುಂದರ್ಯೈ ನಮಃ ।\\
ಗುಪ್ತತರಯೋಗಿನೀರೂಪ ಸ್ವಾತ್ಮಾತ್ಮನೇ ಮಹಿಮಾಸಿದ್ಧ್ಯೈ ನಮಃ ।\\
ಅಪರಿಚ್ಛಿನ್ನ ಸ್ವಾತ್ಮಾತ್ಮನೇ ಸರ್ವಾಕರ್ಷಿಣೀಮುದ್ರಾಯೈ ನಮಃ ।

\as{(ಸಹಸ್ರನಾಮ ೪೦೧ - ೫೦೦)\\}
ವಿವಿಧಾಕಾರಾ ವಿದ್ಯಾವಿದ್ಯಾ - ಸ್ವರೂಪಿಣೀ~।\\
ಮಹಾಕಾಮೇಶ - ನಯನ - ಕುಮುದಾಹ್ಲಾದ - ಕೌಮುದೀ ॥೮೭॥

ಭಕ್ತ - ಹಾರ್ದ - ತಮೋಭೇದ - ಭಾನುಮದ್ಭಾನು - ಸಂತತಿಃ~।\\
ಶಿವದೂತೀ ಶಿವಾರಾಧ್ಯಾ ಶಿವಮೂರ್ತಿಃ ಶಿವಂಕರೀ ॥೮೮॥

ಶಿವಪ್ರಿಯಾ ಶಿವಪರಾ ಶಿಷ್ಟೇಷ್ಟಾ ಶಿಷ್ಟಪೂಜಿತಾ~।\\
ಅಪ್ರಮೇಯಾ ಸ್ವಪ್ರಕಾಶಾ ಮನೋವಾಚಾಮಗೋಚರಾ ॥೮೯॥

ಚಿಚ್ಛಕ್ತಿಶ್ ಚೇತನಾರೂಪಾ ಜಡಶಕ್ತಿರ್ಜಡಾತ್ಮಿಕಾ~।\\
ಗಾಯತ್ರೀ ವ್ಯಾಹೃತಿಃ ಸಂಧ್ಯಾ ದ್ವಿಜಬೃಂದ - ನಿಷೇವಿತಾ ॥೯೦॥

ತತ್ತ್ವಾಸನಾ ತತ್ತ್ವಮಯೀ ಪಂಚ - ಕೋಶಾಂತರ - ಸ್ಥಿತಾ~।\\
ನಿಃಸೀಮಮಹಿಮಾ ನಿತ್ಯ - ಯೌವನಾ ಮದಶಾಲಿನೀ ॥೯೧॥

ಮದಘೂರ್ಣಿತ - ರಕ್ತಾಕ್ಷೀ ಮದಪಾಟಲ - ಗಂಡಭೂಃ~।\\
ಚಂದನ - ದ್ರವ - ದಿಗ್ಧಾಂಗೀ ಚಾಂಪೇಯ - ಕುಸುಮ - ಪ್ರಿಯಾ ॥೯೨॥

ಕುಶಲಾ ಕೋಮಲಾಕಾರಾ ಕುರುಕುಲ್ಲಾ ಕುಲೇಶ್ವರೀ~।\\
ಕುಲಕುಂಡಾಲಯಾ ಕೌಲ - ಮಾರ್ಗ - ತತ್ಪರ - ಸೇವಿತಾ ॥೯೩॥

ಕುಮಾರ - ಗಣನಾಥಾಂಬಾ ತುಷ್ಟಿಃ ಪುಷ್ಟಿರ್ಮತಿರ್ಧೃತಿಃ~।\\
ಶಾಂತಿಃ ಸ್ವಸ್ತಿಮತೀ ಕಾಂತಿರ್ನಂದಿನೀ ವಿಘ್ನನಾಶಿನೀ ॥೯೪॥

ತೇಜೋವತೀ ತ್ರಿನಯನಾ ಲೋಲಾಕ್ಷೀ - ಕಾಮರೂಪಿಣೀ~।\\
ಮಾಲಿನೀ ಹಂಸಿನೀ ಮಾತಾ ಮಲಯಾಚಲ - ವಾಸಿನೀ ॥೯೫॥

ಸುಮುಖೀ ನಲಿನೀ ಸುಭ್ರೂಃ ಶೋಭನಾ ಸುರನಾಯಿಕಾ~।\\
ಕಾಲಕಂಠೀ ಕಾಂತಿಮತೀ ಕ್ಷೋಭಿಣೀ ಸೂಕ್ಷ್ಮರೂಪಿಣೀ ॥೯೬॥

ವಜ್ರೇಶ್ವರೀ ವಾಮದೇವೀ ವಯೋಽವಸ್ಥಾ - ವಿವರ್ಜಿತಾ~।\\
ಸಿದ್ಧೇಶ್ವರೀ ಸಿದ್ಧವಿದ್ಯಾ ಸಿದ್ಧಮಾತಾ ಯಶಸ್ವಿನೀ ॥೯೭॥

ವಿಶುದ್ಧಿಚಕ್ರ - ನಿಲಯಾಽಽರಕ್ತವರ್ಣಾ ತ್ರಿಲೋಚನಾ~।\\
ಖಟ್‍ವಾಂಗಾದಿ - ಪ್ರಹರಣಾ ವದನೈಕ - ಸಮನ್ವಿತಾ ॥೯೮॥

ಪಾಯಸಾನ್ನಪ್ರಿಯಾ ತ್ವಕ್ಸ್ಥಾ ಪಶುಲೋಕ - ಭಯಂಕರೀ~।\\
ಅಮೃತಾದಿ - ಮಹಾಶಕ್ತಿ - ಸಂವೃತಾ ಡಾಕಿನೀಶ್ವರೀ ॥೯೯॥

ಅನಾಹತಾಬ್ಜ - ನಿಲಯಾ ಶ್ಯಾಮಾಭಾ ವದನದ್ವಯಾ~।\\
ದಂಷ್ಟ್ರೋಜ್ಜ್ವಲಾಽಕ್ಷ - ಮಾಲಾದಿ - ಧರಾ ರುಧಿರಸಂಸ್ಥಿತಾ ॥೧೦೦॥

ಕಾಲರಾತ್ರ್ಯಾದಿ - ಶಕ್ತ್ಯೌಘ - ವೃತಾ ಸ್ನಿಗ್ಧೌದನಪ್ರಿಯಾ~।\\
ಮಹಾವೀರೇಂದ್ರ - ವರದಾ ರಾಕಿಣ್ಯಂಬಾ - ಸ್ವರೂಪಿಣೀ ॥೧೦೧॥

ಮಣಿಪೂರಾಬ್ಜ - ನಿಲಯಾ ವದನತ್ರಯ - ಸಂಯುತಾ~।\\
ವಜ್ರಾದಿಕಾಯುಧೋಪೇತಾ ಡಾಮರ್ಯಾದಿಭಿರಾವೃತಾ ॥೧೦೨॥

ರಕ್ತವರ್ಣಾ ಮಾಂಸನಿಷ್ಠಾ \as{(೫೦೦)}

\as{(ಅಷ್ಟೋತ್ತರ ೪೧ - ೫೦)\\}
ಏಕಾತಪತ್ರಸಾಮ್ರಾಜ್ಯದಾಯಿಕಾಯೈ ನಮೋ ನಮಃ ।\\
ಸನಕಾದಿಸಮಾರಾಧ್ಯಪಾದುಕಾಯೈ ನಮೋ ನಮಃ ।\\
ದೇವರ್ಷಿಭಿಸ್ಸ್ತೂಯಮಾನವೈಭವಾಯೈ ನಮೋ ನಮಃ ।\\
ಕಲಶೋದ್ಭವದುರ್ವಾಸಃಪೂಜಿತಾಯೈ ನಮೋ ನಮಃ ।\\
ಮತ್ತೇಭವಕ್ತ್ರಷಡ್ವಕ್ತ್ರವತ್ಸಲಾಯೈ ನಮೋ ನಮಃ ।\\
ಚಕ್ರರಾಜಮಹಾಯಂತ್ರಮಧ್ಯವರ್ತಿನ್ಯೈ ನಮೋ ನಮಃ ।\\
ಚಿದಗ್ನಿಕುಂಡಸಂಭೂತಸುದೇಹಾಯೈ ನಮೋ ನಮಃ ।\\
ಶಶಾಂಕಖಂಡಸಂಯುಕ್ತಮಕುಟಾಯೈ ನಮೋ ನಮಃ ।\\
ಮತ್ತಹಂಸವಧೂಮಂದಗಮನಾಯೈ ನಮೋ ನಮಃ ।\\
ವಂದಾರುಜನಸಂದೋಹವಂದಿತಾಯೈ ನಮೋ ನಮಃ ।

\as{ಓಂ ॥ ಹಿರ॑ಣ್ಯವರ್ಣಾಂ॒ ಹರಿ॑ಣೀಂ ಸು॒ವರ್ಣ॑ರಜ॒ತಸ್ರ॑ಜಾಂ ।\\
ಚಂ॒ದ್ರಾಂ ಹಿ॒ರಣ್ಮ॑ಯೀಂ ಲ॒ಕ್ಷ್ಮೀಂ ಜಾತ॑ವೇದೋ ಮ॒ ಆವ॑ಹ ॥ ೧॥}

೪ ಕಕಾರರೂಪಾಯೈ ನಮಃ ।\\
೪ ಕಲ್ಯಾಣ್ಯೈ ನಮಃ ।\\
೪ ಕಲ್ಯಾಣಗುಣಶಾಲಿನ್ಯೈ ನಮಃ ।\\
೪ ಕಲ್ಯಾಣಶೈಲನಿಲಯಾಯೈ ನಮಃ ।\\
೪ ಕಮನೀಯಾಯೈ ನಮಃ ।\\
೪ ಕಲಾವತ್ಯೈ ನಮಃ ।\\
೪ ಕಮಲಾಕ್ಷ್ಯೈ ನಮಃ ।\\
೪ ಕಲ್ಮಷಘ್ನ್ಯೈ ನಮಃ ।\\
೪ ಕರುಣಾಮೃತಸಾಗರಾಯೈ ನಮಃ ।\\
೪ ಕದಂಬಕಾನನಾವಾಸಾಯೈ ನಮಃ ।\\
೪ ಕದಂಬಕುಸುಮಪ್ರಿಯಾಯೈ ನಮಃ ।\\
೪ ಕಂದರ್ಪವಿದ್ಯಾಯೈ ನಮಃ ।\\
೪ ಕಂದರ್ಪಜನಕಾಪಾಂಗವೀಕ್ಷಣಾಯೈ ನಮಃ ।\\
೪ ಕರ್ಪೂರವೀಟೀಸೌರಭ್ಯಕಲ್ಲೋಲಿತಕಕುಪ್ತಟಾಯೈ ನಮಃ ।\\
೪ ಕಲಿದೋಷಹರಾಯೈ ನಮಃ ।\\
೪ ಕಂಜಲೋಚನಾಯೈ ನಮಃ ।\\
೪ ಕಮ್ರವಿಗ್ರಹಾಯೈ ನಮಃ ।\\
೪ ಕರ್ಮಾದಿಸಾಕ್ಷಿಣ್ಯೈ ನಮಃ ।\\
೪ ಕಾರಯಿತ್ರ್ಯೈ ನಮಃ ।\\
೪ ಕರ್ಮಫಲಪ್ರದಾಯೈ ನಮಃ ।

\as{ತಾಂ ಮ॒ ಆವ॑ಹ॒ ಜಾತ॑ವೇದೋ ಲ॒ಕ್ಷ್ಮೀಮನ॑ಪಗಾ॒ಮಿನೀಂ᳚ ।\\
ಯಸ್ಯಾಂ॒ ಹಿರ॑ಣ್ಯಂ ವಿಂ॒ದೇಯಂ॒ ಗಾಮಶ್ವಂ॒ ಪುರು॑ಷಾನ॒ಹಂ ॥ ೨॥}

೪ ಏಕಾರರೂಪಾಯೈ ನಮಃ ।\\
೪ ಏಕಾಕ್ಷರ್ಯೈ ನಮಃ ।\\
೪ ಏಕಾನೇಕಾಕ್ಷರಾಕೃತ್ಯೈ ನಮಃ ।\\
೪ ಏತತ್ತದಿತ್ಯನಿರ್ದೇಶ್ಯಾಯೈ ನಮಃ ।\\
೪ ಏಕಾನಂದಚಿದಾಕೃತ್ಯೈ ನಮಃ ।\\
೪ ಏವಮಿತ್ಯಾಗಮಾಬೋಧ್ಯಾಯೈ ನಮಃ ।\\
೪ ಏಕಭಕ್ತಿಮದರ್ಚಿತಾಯೈ ನಮಃ ।\\
೪ ಏಕಾಗ್ರಚಿತ್ತನಿರ್ಧ್ಯಾತಾಯೈ ನಮಃ ।\\
೪ ಏಷಣಾರಹಿತಾದೃತಾಯೈ ನಮಃ ।\\
೪ ಏಲಾಸುಗಂಧಿಚಿಕುರಾಯೈ ನಮಃ ।\\
೪ ಏನಃಕೂಟವಿನಾಶಿನ್ಯೈ ನಮಃ ।\\
೪ ಏಕಭೋಗಾಯೈ ನಮಃ ।\\
೪ ಏಕರಸಾಯೈ ನಮಃ ।\\
೪ ಏಕೈಶ್ವರ್ಯಪ್ರದಾಯಿನ್ಯೈ ನಮಃ ।\\
೪ ಏಕಾತಪತ್ರಸಾಮ್ರಾಜ್ಯಪ್ರದಾಯೈ ನಮಃ ।\\
೪ ಏಕಾಂತಪೂಜಿತಾಯೈ ನಮಃ ।\\
೪ ಏಧಮಾನಪ್ರಭಾಯೈ ನಮಃ ।\\
೪ ಏಜದನೇಕಜಗದೀಶ್ವರ್ಯೈ ನಮಃ ।\\
೪ ಏಕವೀರಾದಿಸಂಸೇವ್ಯಾಯೈ ನಮಃ ।\\
೪ ಏಕಪ್ರಾಭವಶಾಲಿನ್ಯೈ ನಮಃ ।


\as{ಅ॒ಶ್ವ॒ಪೂ॒ರ್ವಾಂ ರ॑ಥಮ॒ಧ್ಯಾಂ ಹ॒ಸ್ತಿನಾ᳚ದಪ್ರ॒ಬೋಧಿ॑ನೀಂ ।\\
ಶ್ರಿಯಂ॑ ದೇ॒ವೀಮುಪ॑ಹ್ವಯೇ॒ ಶ್ರೀರ್ಮಾ᳚ದೇ॒ವೀರ್ಜು॑ಷತಾಂ ॥ ೩॥}

೪ ಈಕಾರರೂಪಾಯೈ ನಮಃ ।\\
೪ ಈಶಿತ್ರ್ಯೈ ನಮಃ ।\\
೪ ಈಪ್ಸಿತಾರ್ಥಪ್ರದಾಯಿನ್ಯೈ ನಮಃ ।\\
೪ ಈದೃಗಿತ್ಯಾವಿನಿರ್ದೇಶ್ಯಾಯೈ ನಮಃ ।\\
೪ ಈಶ್ವರತ್ವವಿಧಾಯಿನ್ಯೈ ನಮಃ ।\\
೪ ಈಶಾನಾದಿಬ್ರಹ್ಮಮಯ್ಯೈ ನಮಃ ।\\
೪ ಈಶಿತ್ವಾದ್ಯಷ್ಟಸಿದ್ಧಿದಾಯೈ ನಮಃ ।\\
೪ ಈಕ್ಷಿತ್ರ್ಯೈ ನಮಃ ।\\
೪ ಈಕ್ಷಣಸೃಷ್ಟಾಂಡಕೋಟ್ಯೈ ನಮಃ ।\\
೪ ಈಶ್ವರವಲ್ಲಭಾಯೈ ನಮಃ ।\\
೪ ಈಡಿತಾಯೈ ನಮಃ ।\\
೪ ಈಶ್ವರಾರ್ಧಾಂಗಶರೀರಾಯೈ ನಮಃ ।\\
೪ ಈಶಾಧಿದೇವತಾಯೈ ನಮಃ ।\\
೪ ಈಶ್ವರಪ್ರೇರಣಕರ್ಯೈ ನಮಃ ।\\
೪ ಈಶತಾಂಡವಸಾಕ್ಷಿಣ್ಯೈ ನಮಃ ।\\
೪ ಈಶ್ವರೋತ್ಸಂಗನಿಲಯಾಯೈ ನಮಃ ।\\
೪ ಈತಿಬಾಧಾವಿನಾಶಿನ್ಯೈ ನಮಃ ।\\
೪ ಈಹಾವಿರಹಿತಾಯೈ ನಮಃ ।\\
೪ ಈಶಶಕ್ತ್ಯೈ ನಮಃ ।\\
೪ ಈಷತ್ಸ್ಮಿತಾನನಾಯೈ ನಮಃ ।

\as{ಕಾಂ॒ ಸೋ॒ಸ್ಮಿ॒ತಾಂ ಹಿರ॑ಣ್ಯಪ್ರಾ॒ಕಾರಾ॑ಮಾ॒ರ್ದ್ರಾಂ ಜ್ವಲಂ॑ತೀಂ ತೃ॒ಪ್ತಾಂ ತ॒ರ್ಪಯಂ॑ತೀಂ ।\\
ಪ॒ದ್ಮೇ॒ ಸ್ಥಿ॒ತಾಂ ಪ॒ದ್ಮವ॑ರ್ಣಾಂ॒ ತಾಮಿ॒ಹೋಪ॑ಹ್ವಯೇ॒ ಶ್ರಿಯಂ ॥ ೪॥}

೪ ಲಕಾರರೂಪಾಯೈ ನಮಃ ।\\
೪ ಲಲಿತಾಯೈ ನಮಃ ।\\
೪ ಲಕ್ಷ್ಮೀವಾಣೀನಿಷೇವಿತಾಯೈ ನಮಃ ।\\
೪ ಲಾಕಿನ್ಯೈ ನಮಃ ।\\
೪ ಲಲನಾರೂಪಾಯೈ ನಮಃ ।\\
೪ ಲಸದ್ದಾಡಿಮಪಾಟಲಾಯೈ ನಮಃ ।\\
೪ ಲಲಂತಿಕಾಲಸತ್ಫಾಲಾಯೈ ನಮಃ ।\\
೪ ಲಲಾಟನಯನಾರ್ಚಿತಾಯೈ ನಮಃ ।\\
೪ ಲಕ್ಷಣೋಜ್ಜ್ವಲದಿವ್ಯಾಂಗ್ಯೈ ನಮಃ ।\\
೪ ಲಕ್ಷಕೋಟ್ಯಂಡನಾಯಿಕಾಯೈ ನಮಃ ।\\
೪ ಲಕ್ಷ್ಯಾರ್ಥಾಯೈ ನಮಃ ।\\
೪ ಲಕ್ಷಣಾಗಮ್ಯಾಯೈ ನಮಃ ।\\
೪ ಲಬ್ಧಕಾಮಾಯೈ ನಮಃ ।\\
೪ ಲತಾತನವೇ ನಮಃ ।\\
೪ ಲಲಾಮರಾಜದಲಿಕಾಯೈ ನಮಃ ।\\
೪ ಲಂಬಿಮುಕ್ತಾಲತಾಂಚಿತಾಯೈ ನಮಃ ।\\
೪ ಲಂಬೋದರಪ್ರಸುವೇ ನಮಃ ।\\
೪ ಲಭ್ಯಾಯೈ ನಮಃ ।\\
೪ ಲಜ್ಜಾಢ್ಯಾಯೈ ನಮಃ ।\\
೪ ಲಯವರ್ಜಿತಾಯೈ ನಮಃ ।

\as{ಚಂ॒ದ್ರಾಂ ಪ್ರ॑ಭಾ॒ಸಾಂ ಯ॒ಶಸಾ॒ ಜ್ವಲಂ॑ತೀಂ॒ ಶ್ರಿಯಂ॑ ಲೋ॒ಕೇ ದೇ॒ವಜು॑ಷ್ಟಾಮುದಾ॒ರಾಂ ।\\
ತಾಂ ಪ॒ದ್ಮಿನೀ॑ಮೀಂ॒ ಶರ॑ಣಮ॒ಹಂ ಪ್ರಪ॑ದ್ಯೇಽಲ॒ಕ್ಷ್ಮೀರ್ಮೇ॑ ನಶ್ಯತಾಂ॒ ತ್ವಾಂ ವೃ॑ಣೇ ॥ ೫॥}

೪ ಹ್ರೀಂಕಾರರೂಪಾಯೈ ನಮಃ ।\\
೪ ಹ್ರೀಂಕಾರನಿಲಯಾಯೈ ನಮಃ ।\\
೪ ಹ್ರೀಂಪದಪ್ರಿಯಾಯೈ ನಮಃ ।\\
೪ ಹ್ರೀಂಕಾರಬೀಜಾಯೈ ನಮಃ ।\\
೪ ಹ್ರೀಂಕಾರಮಂತ್ರಾಯೈ ನಮಃ ।\\
೪ ಹ್ರೀಂಕಾರಲಕ್ಷಣಾಯೈ ನಮಃ ।\\
೪ ಹ್ರೀಂಕಾರಜಪಸುಪ್ರೀತಾಯೈ ನಮಃ ।\\
೪ ಹ್ರೀಂಮತ್ಯೈ ನಮಃ ।\\
೪ ಹ್ರೀಂವಿಭೂಷಣಾಯೈ ನಮಃ ।\\
೪ ಹ್ರೀಂಶೀಲಾಯೈ ನಮಃ ।\\
೪ ಹ್ರೀಂಪದಾರಾಧ್ಯಾಯೈ ನಮಃ ।\\
೪ ಹ್ರೀಂಗರ್ಭಾಯೈ ನಮಃ ।\\
೪ ಹ್ರೀಂಪದಾಭಿಧಾಯೈ ನಮಃ ।\\
೪ ಹ್ರೀಂಕಾರವಾಚ್ಯಾಯೈ ನಮಃ ।\\
೪ ಹ್ರೀಂಕಾರಪೂಜ್ಯಾಯೈ ನಮಃ ।\\
೪ ಹ್ರೀಂಕಾರಪೀಠಿಕಾಯೈ ನಮಃ ।\\
೪ ಹ್ರೀಂಕಾರವೇದ್ಯಾಯೈ ನಮಃ ।\\
೪ ಹ್ರೀಂಕಾರಚಿಂತ್ಯಾಯೈ ನಮಃ ।\\
೪ ಹ್ರೀಂ ನಮಃ ।\\
೪ ಹ್ರೀಂಶರೀರಿಣ್ಯೈ ನಮಃ । 

\as{ವ॒ಯಂ ನಾಮ॒ ಪ್ರಬ್ರ॑ವಾಮಾ ಘೃ॒ತೇನಾ॒ಸ್ಮಿನ್ ಯ॒ಜ್ಞೇ ಧಾ॑ರಯಾಮಾ॒ ನಮೋ॑ಭಿಃ ।
ಉಪ॑ ಬ್ರ॒ಹ್ಮಾ ಶೃ॑ಣವ ಚ್ಛ॒ಸ್ಯಮಾ॑ನಂ॒ ಚತು॑ಶ್ಶೃಂಗೋಽವಮೀದ್ಗೌ॒ರ ಏ॒ತತ್ ॥}
\section{ಚತುರ್ದಶಾರಚಕ್ರಾಯ ನಮಃ ।\\ (ಇತಿ ವ್ಯಾಪಕಂ ನ್ಯಸ್ಯ)}
ಲಲಾಟಮಧ್ಯಭಾಗರೂಪ ಅಲಂಬುಸಾತ್ಮನೇ ಸರ್ವಸಂಕ್ಷೋಭಿಣೀಶಕ್ತ್ಯೈ ನಮಃ ।\\
ಲಲಾಟದಕ್ಷಭಾಗರೂಪ ಕುಹ್ವಾತ್ಮನೇ ಸರ್ವವಿದ್ರಾವಿಣೀಶಕ್ತ್ಯೈ ನಮಃ ।\\
ದಕ್ಷಗಂಡರೂಪ ವಿಶ್ವೋದರಾತ್ಮನೇ ಸರ್ವಾಕರ್ಷಿಣೀಶಕ್ತ್ಯೈ ನಮಃ ।\\
ದಕ್ಷಾಂಸರೂಪ ವಾರಣಾತ್ಮನೇ ಸರ್ವಾಹ್ಲಾದಿನೀಶಕ್ತ್ಯೈ ನಮಃ ।\\
ದಕ್ಷಪಾರ್ಶ್ವರೂಪ ಹಸ್ತಿಜಿಹ್ವಾತ್ಮನೇ ಸರ್ವಸಮ್ಮೋಹಿನೀಶಕ್ತ್ಯೈ ನಮಃ ।\\
ದಕ್ಷೋರುರೂಪ ಯಶೋವತ್ಯಾತ್ಮನೇ ಸರ್ವಸ್ತಂಭಿನೀಶಕ್ತ್ಯೈ ನಮಃ ।\\
ದಕ್ಷಜಂಘಾರೂಪ ಪಯಸ್ವಿನ್ಯಾತ್ಮನೇ ಸರ್ವಜೃಂಭಿಣೀಶಕ್ತ್ಯೈ ನಮಃ ।\\
ವಾಮಜಂಘಾರೂಪ ಗಾಂಧಾರ್ಯತ್ಮನೇ ಸರ್ವವಶಂಕರೀಶಕ್ತ್ಯೈ ನಮಃ ।\\
ವಾಮೋರುರೂಪ ಪೂಷಾತ್ಮನೇ ಸರ್ವರಂಜನೀಶಕ್ತ್ಯೈ ನಮಃ ।\\
ವಾಮಪಾರ್ಶ್ವರೂಪ ಶಂಖಿನ್ಯಾತ್ಮನೇ ಸರ್ವೋನ್ಮಾದಿನೀಶಕ್ತ್ಯೈ ನಮಃ ।\\
ವಾಮಾಂಸರೂಪ ಸರಸ್ವತ್ಯಾತ್ಮನೇ ಸರ್ವಾರ್ಥಸಾಧಿನೀಶಕ್ತ್ಯೈ ನಮಃ ।\\
ವಾಮಗಂಡರೂಪ ಇಡಾತ್ಮನೇ ಸರ್ವಸಂಪತ್ತಿಪೂರಣೀಶಕ್ತ್ಯೈ ನಮಃ ।\\
ಲಲಾಟವಾಮಭಾಗರೂಪ ಪಿಂಗಲಾತ್ಮನೇ ಸರ್ವಮಂತ್ರಮಯೀಶಕ್ತ್ಯೈ ನಮಃ ।\\
ಶಿರಃಪೃಷ್ಠಭಾಗರೂಪ ಸುಷುಮ್ನಾತ್ಮನೇ ಸರ್ವದ್ವಂದ್ವಕ್ಷಯಂಕರೀಶಕ್ತ್ಯೈ ನಮಃ ।\\
ಹೃದ್ರೂಪ ಸರ್ವಸೌಭಾಗ್ಯದಾಯಕಚಕ್ರೇಶ್ವರ್ಯೈ ತ್ರಿಪುರವಾಸಿನ್ಯೈ ನಮಃ ।\\
ಸಂಪ್ರದಾಯ ಯೋಗಿನೀರೂಪ ಸ್ವಾತ್ಮಾತ್ಮನೇ ಈಶಿತ್ವಸಿದ್ಧ್ಯೈ ನಮಃ ।\\
ಅಪರಿಚ್ಛಿನ್ನರೂಪ ಸ್ವಾತ್ಮಾತ್ಮನೇ ಸರ್ವವಶಂಕರೀಮುದ್ರಾಯೈ ನಮಃ ।

\as{(ಸಹಸ್ರನಾಮ ೫೦೧ - ೬೦೦)\\}
 ಗುಡಾನ್ನ - ಪ್ರೀತ - ಮಾನಸಾ~।\\
ಸಮಸ್ತಭಕ್ತ - ಸುಖದಾ ಲಾಕಿನ್ಯಂಬಾ - ಸ್ವರೂಪಿಣೀ ॥೧೦೩॥

ಸ್ವಾಧಿಷ್ಠಾನಾಂಬುಜ - ಗತಾ ಚತುರ್ವಕ್ತ್ರ - ಮನೋಹರಾ~।\\
ಶೂಲಾದ್ಯಾಯುಧ - ಸಂಪನ್ನಾ ಪೀತವರ್ಣಾಽತಿಗರ್ವಿತಾ ॥೧೦೪॥

ಮೇದೋನಿಷ್ಠಾ ಮಧುಪ್ರೀತಾ ಬಂದಿನ್ಯಾದಿ - ಸಮನ್ವಿತಾ~।\\
ದಧ್ಯನ್ನಾಸಕ್ತ - ಹೃದಯಾ ಕಾಕಿನೀ - ರೂಪ - ಧಾರಿಣೀ ॥೧೦೫॥

ಮೂಲಾಧಾರಾಂಬುಜಾರೂಢಾ ಪಂಚ - ವಕ್ತ್ರಾಽಸ್ಥಿ - ಸಂಸ್ಥಿತಾ~।\\
ಅಂಕುಶಾದಿ - ಪ್ರಹರಣಾ ವರದಾದಿ - ನಿಷೇವಿತಾ ॥೧೦೬॥

ಮುದ್ಗೌದನಾಸಕ್ತ - ಚಿತ್ತಾ ಸಾಕಿನ್ಯಂಬಾ - ಸ್ವರೂಪಿಣೀ~।\\
ಆಜ್ಞಾ - ಚಕ್ರಾಬ್ಜ - ನಿಲಯಾ ಶುಕ್ಲವರ್ಣಾ ಷಡಾನನಾ ॥೧೦೭॥

ಮಜ್ಜಾಸಂಸ್ಥಾ ಹಂಸವತೀ - ಮುಖ್ಯ - ಶಕ್ತಿ - ಸಮನ್ವಿತಾ~।\\
ಹರಿದ್ರಾನ್ನೈಕ - ರಸಿಕಾ ಹಾಕಿನೀ - ರೂಪ - ಧಾರಿಣೀ ॥೧೦೮॥

ಸಹಸ್ರದಲ - ಪದ್ಮಸ್ಥಾ ಸರ್ವ - ವರ್ಣೋಪ - ಶೋಭಿತಾ~।\\
ಸರ್ವಾಯುಧಧರಾ ಶುಕ್ಲ - ಸಂಸ್ಥಿತಾ ಸರ್ವತೋಮುಖೀ ॥೧೦೯॥

ಸರ್ವೌದನ - ಪ್ರೀತಚಿತ್ತಾ ಯಾಕಿನ್ಯಂಬಾ - ಸ್ವರೂಪಿಣೀ~।\\
ಸ್ವಾಹಾ ಸ್ವಧಾಽಮತಿರ್ಮೇಧಾ ಶ್ರುತಿಃ ಸ್ಮೃತಿರನುತ್ತಮಾ ॥೧೧೦॥

ಪುಣ್ಯಕೀರ್ತಿಃ ಪುಣ್ಯಲಭ್ಯಾ ಪುಣ್ಯಶ್ರವಣ - ಕೀರ್ತನಾ~।\\
ಪುಲೋಮಜಾರ್ಚಿತಾ ಬಂಧ - ಮೋಚನೀ ಬಂಧುರಾಲಕಾ ॥೧೧೧॥

ವಿಮರ್ಶರೂಪಿಣೀ ವಿದ್ಯಾ ವಿಯದಾದಿ - ಜಗತ್ಪ್ರಸೂಃ~।\\
ಸರ್ವವ್ಯಾಧಿ - ಪ್ರಶಮನೀ ಸರ್ವಮೃತ್ಯು - ನಿವಾರಿಣೀ ॥೧೧೨॥

ಅಗ್ರಗಣ್ಯಾಽಚಿಂತ್ಯರೂಪಾ ಕಲಿಕಲ್ಮಷ - ನಾಶಿನೀ~।\\
ಕಾತ್ಯಾಯನೀ ಕಾಲಹಂತ್ರೀ ಕಮಲಾಕ್ಷ - ನಿಷೇವಿತಾ ॥೧೧೩॥

ತಾಂಬೂಲ - ಪೂರಿತ - ಮುಖೀ ದಾಡಿಮೀ - ಕುಸುಮ - ಪ್ರಭಾ~।\\
ಮೃಗಾಕ್ಷೀ ಮೋಹಿನೀ ಮುಖ್ಯಾ ಮೃಡಾನೀ ಮಿತ್ರರೂಪಿಣೀ ॥೧೧೪॥

ನಿತ್ಯತೃಪ್ತಾ ಭಕ್ತನಿಧಿರ್ನಿಯಂತ್ರೀ ನಿಖಿಲೇಶ್ವರೀ~।\\
ಮೈತ್ರ್ಯಾದಿ - ವಾಸನಾಲಭ್ಯಾ ಮಹಾಪ್ರಲಯ - ಸಾಕ್ಷಿಣೀ ॥೧೧೫॥

ಪರಾ ಶಕ್ತಿಃ ಪರಾ ನಿಷ್ಠಾ ಪ್ರಜ್ಞಾನಘನ - ರೂಪಿಣೀ~।\\
ಮಾಧ್ವೀಪಾನಾಲಸಾ ಮತ್ತಾ ಮಾತೃಕಾ - ವರ್ಣ - ರೂಪಿಣೀ ॥೧೧೬॥

ಮಹಾಕೈಲಾಸ - ನಿಲಯಾ ಮೃಣಾಲ - ಮೃದು - ದೋರ್ಲತಾ~।\\
ಮಹನೀಯಾ ದಯಾಮೂರ್ತಿರ್ಮಹಾಸಾಮ್ರಾಜ್ಯ - ಶಾಲಿನೀ ॥೧೧೭॥

ಆತ್ಮವಿದ್ಯಾ ಮಹಾವಿದ್ಯಾ ಶ್ರೀವಿದ್ಯಾ ಕಾಮಸೇವಿತಾ~।\\
ಶ್ರೀ - ಷೋಡಶಾಕ್ಷರೀ - ವಿದ್ಯಾ ತ್ರಿಕೂಟಾ ಕಾಮಕೋಟಿಕಾ ॥೧೧೮॥

ಕಟಾಕ್ಷ - ಕಿಂಕರೀ - ಭೂತ - ಕಮಲಾ - ಕೋಟಿ - ಸೇವಿತಾ~।\\
ಶಿರಃಸ್ಥಿತಾ ಚಂದ್ರನಿಭಾ ಭಾಲಸ್ಥೇಂದ್ರ - ಧನುಃಪ್ರಭಾ ॥೧೧೯॥

ಹೃದಯಸ್ಥಾ ರವಿಪ್ರಖ್ಯಾ ತ್ರಿಕೋಣಾಂತರ - ದೀಪಿಕಾ~।\\
ದಾಕ್ಷಾಯಣೀ ದೈತ್ಯಹಂತ್ರೀ ದಕ್ಷಯಜ್ಞ - ವಿನಾಶಿನೀ \as{(೬೦೦)} ॥೧೨೦॥

\as{(ಅಷ್ಟೋತ್ತರ ೫೧ - ೬೦)\\}
ಅಂತರ್ಮುಖಜನಾನಂದಫಲದಾಯೈ ನಮೋ ನಮಃ ।\\
ಪತಿವ್ರತಾಂಗನಾಭೀಷ್ಟಫಲದಾಯೈ ನಮೋ ನಮಃ ।\\
ಅವ್ಯಾಜಕರುಣಾಪೂರಪೂರಿತಾಯೈ ನಮೋ ನಮಃ ।\\
ನಿತಾಂತಸಚ್ಚಿದಾನಂದಸಂಯುಕ್ತಾಯೈ ನಮೋ ನಮಃ ।\\
ಸಹಸ್ರಸೂರ್ಯಸಂಯುಕ್ತಪ್ರಕಾಶಾಯೈ ನಮೋ ನಮಃ ।\\
ರತ್ನಚಿಂತಾಮಣಿಗೃಹಮಧ್ಯಸ್ಥಾಯೈ ನಮೋ ನಮಃ ।\\
ಹಾನಿವೃದ್ಧಿಗುಣಾಧಿಕ್ಯರಹಿತಾಯೈ ನಮೋ ನಮಃ ।\\
ಮಹಾಪದ್ಮಾಟವೀಮಧ್ಯನಿವಾಸಾಯೈ ನಮೋ ನಮಃ ।\\
ಜಾಗ್ರತ್ಸ್ವಪ್ನಸುಷುಪ್ತೀನಾಂ ಸಾಕ್ಷಿಭೂತ್ಯೈ ನಮೋ ನಮಃ ।\\
ಮಹಾಪಾಪೌಘಪಾಪಾನಾಂ ವಿನಾಶಿನ್ಯೈ ನಮೋ ನಮಃ ।

\as{ಚ॒ತ್ವಾರಿ॒ ಶೃಂಗಾ॒ ತ್ರಯೋ॑ ಅಸ್ಯ॒ ಪಾದಾ॒ ದ್ವೇ ಶೀ॒ರ್ಷೇ ಸ॒ಪ್ತ ಹಸ್ತಾ॑ಸೋ ಅ॒ಸ್ಯ ।
ತ್ರಿಧಾ॑ಬ॒ದ್ಧೋ ವೃ॑ಷ॒ಭೋ ರೋ॑ರವೀತಿ ಮ॒ಹೋ ದೇ॒ವೋ ಮರ್ತ್ಯಾ॒ँ ಆವಿ॑ವೇಶ ॥}
\section{ಬಹಿರ್ದಶಾರಚಕ್ರಾಯ ನಮಃ ।\\ (ಇತಿ ವ್ಯಾಪಕಂ ನ್ಯಸ್ಯ)}
ದಕ್ಷಾಕ್ಷಿರೂಪ ಪ್ರಾಣಾತ್ಮನೇ ಸರ್ವಸಿದ್ಧಿಪ್ರದಾದೇವ್ಯೈ ನಮಃ ।\\
ನಾಸಾಮೂಲರೂಪ ಅಪಾನಾತ್ಮನೇ ಸರ್ವಸಂಪತ್ಪ್ರದಾದೇವ್ಯೈ ನಮಃ ।\\
ವಾಮನೇತ್ರರೂಪ ವ್ಯಾನಾತ್ಮನೇ ಸರ್ವಪ್ರಿಯಂಕರೀದೇವ್ಯೈ ನಮಃ ।\\
ಕುಕ್ಷೀಶಕೋಣರೂಪ ಉದಾನಾತ್ಮನೇ ಸರ್ವಮಂಗಳಕಾರಿಣೀದೇವ್ಯೈ ನಮಃ ।\\
ಕುಕ್ಷಿವಾಯುಕೋಣರೂಪ ಸಮಾನಾತ್ಮನೇ ಸರ್ವಕಾಮಪ್ರದಾದೇವ್ಯೈ ನಮಃ ।\\
ವಾಮಜಾನುರೂಪ ನಾಗಾತ್ಮನೇ ಸರ್ವದುಃಖವಿಮೋಚನೀದೇವ್ಯೈ ನಮಃ ।\\
ಗುದರೂಪ ಕೂರ್ಮಾತ್ಮನೇ ಸರ್ವಮೃತ್ಯುಪ್ರಶಮನೀದೇವ್ಯೈ ನಮಃ ।\\
ದಕ್ಷಜಾನುರೂಪ ಕೃಕರಾತ್ಮನೇ ಸರ್ವವಿಘ್ನನಿವಾರಿಣೀದೇವ್ಯೈ ನಮಃ ।\\
ಕುಕ್ಷಿನಿರ್ಋತಿಕೋಣರೂಪ ದೇವದತ್ತಾತ್ಮನೇ ಸರ್ವಾಂಗಸುಂದರೀದೇವ್ಯೈ ನಮಃ ।\\
ಕುಕ್ಷುವಹ್ನಿಕೋಣರೂಪ ಧನಂಜಯಾತ್ಮನೇ ಸರ್ವಸೌಭಾಗ್ಯದಾಯಿನೀದೇವ್ಯೈ ನಮಃ ।\\
ಹೃದ್ರೂಪ ಸರ್ವಾರ್ಥಸಾಧಕಚಕ್ರೇಶ್ವರ್ಯೈ ತ್ರಿಪುರಾಶ್ರಿಯೈ ನಮಃ ।\\
ಕುಲೋತ್ತೀರ್ಣಯೋಗಿನೀರೂಪ ಸ್ವಾತ್ಮಾತ್ಮನೇ ವಶಿತ್ವಸಿದ್ಧ್ಯೈ ನಮಃ ।\\
ಅಪರಿಚ್ಛಿನ್ನ ಸ್ವಾತ್ಮಾತ್ಮನೇ ಸರ್ವೋನ್ಮಾದಿನೀ ಮುದ್ರಾಯೈ ನಮಃ ।

\as{(ಸಹಸ್ರನಾಮ ೬೦೧ - ೭೦೦)\\}
ದರಾಂದೋಲಿತ - ದೀರ್ಘಾಕ್ಷೀ ದರ - ಹಾಸೋಜ್ಜ್ವಲನ್ಮುಖೀ~।\\
ಗುರುಮೂರ್ತಿರ್ಗುಣನಿಧಿರ್ಗೋಮಾತಾ ಗುಹಜನ್ಮಭೂಃ ॥೧೨೧॥

ದೇವೇಶೀ ದಂಡನೀತಿಸ್ಥಾ ದಹರಾಕಾಶ - ರೂಪಿಣೀ~।\\
ಪ್ರತಿಪನ್ಮುಖ್ಯ - ರಾಕಾಂತ - ತಿಥಿ - ಮಂಡಲ - ಪೂಜಿತಾ ॥೧೨೨॥

ಕಲಾತ್ಮಿಕಾ ಕಲಾನಾಥಾ ಕಾವ್ಯಾಲಾಪ - ವಿನೋದಿನೀ~।\\
ಸಚಾಮರ - ರಮಾ - ವಾಣೀ - ಸವ್ಯ - ದಕ್ಷಿಣ - ಸೇವಿತಾ ॥೧೨೩॥

ಆದಿಶಕ್ತಿರಮೇಯಾಽಽತ್ಮಾ ಪರಮಾ ಪಾವನಾಕೃತಿಃ~।\\
ಅನೇಕಕೋಟಿ - ಬ್ರಹ್ಮಾಂಡ - ಜನನೀ ದಿವ್ಯವಿಗ್ರಹಾ ॥೧೨೪॥

ಕ್ಲೀಂಕಾರೀ ಕೇವಲಾ ಗುಹ್ಯಾ ಕೈವಲ್ಯ - ಪದದಾಯಿನೀ~।\\
ತ್ರಿಪುರಾ ತ್ರಿಜಗದ್ವಂದ್ಯಾ ತ್ರಿಮೂರ್ತಿಸ್ತ್ರಿದಶೇಶ್ವರೀ ॥೧೨೫॥

ತ್ರ್ಯಕ್ಷರೀ ದಿವ್ಯ - ಗಂಧಾಢ್ಯಾ ಸಿಂದೂರ - ತಿಲಕಾಂಚಿತಾ~।\\
ಉಮಾ ಶೈಲೇಂದ್ರತನಯಾ ಗೌರೀ ಗಂಧರ್ವ - ಸೇವಿತಾ ॥೧೨೬॥

ವಿಶ್ವಗರ್ಭಾ ಸ್ವರ್ಣಗರ್ಭಾ ವರದಾ ವಾಗಧೀಶ್ವರೀ~।\\
ಧ್ಯಾನಗಮ್ಯಾಽಪರಿಚ್ಛೇದ್ಯಾ ಜ್ಞಾನದಾ ಜ್ಞಾನವಿಗ್ರಹಾ ॥೧೨೭॥

ಸರ್ವವೇದಾಂತ - ಸಂವೇದ್ಯಾ ಸತ್ಯಾನಂದ - ಸ್ವರೂಪಿಣೀ~।\\
ಲೋಪಾಮುದ್ರಾರ್ಚಿತಾ ಲೀಲಾ - ಕ್ಲೃಪ್ತ - ಬ್ರಹ್ಮಾಂಡ - ಮಂಡಲಾ ॥೧೨೮॥

ಅದೃಶ್ಯಾ ದೃಶ್ಯರಹಿತಾ ವಿಜ್ಞಾತ್ರೀ ವೇದ್ಯವರ್ಜಿತಾ~।\\
ಯೋಗಿನೀ ಯೋಗದಾ ಯೋಗ್ಯಾ ಯೋಗಾನಂದಾ ಯುಗಂಧರಾ ॥೧೨೯॥

ಇಚ್ಛಾಶಕ್ತಿ - ಜ್ಞಾನಶಕ್ತಿ - ಕ್ರಿಯಾಶಕ್ತಿ - ಸ್ವರೂಪಿಣೀ~।\\
ಸರ್ವಾಧಾರಾ ಸುಪ್ರತಿಷ್ಠಾ ಸದಸದ್ರೂಪ - ಧಾರಿಣೀ ॥೧೩೦॥

ಅಷ್ಟಮೂರ್ತಿರಜಾಜೈತ್ರೀ ಲೋಕಯಾತ್ರಾ - ವಿಧಾಯಿನೀ~।\\
ಏಕಾಕಿನೀ ಭೂಮರೂಪಾ ನಿರ್ದ್ವೈತಾ ದ್ವೈತವರ್ಜಿತಾ ॥೧೩೧॥

ಅನ್ನದಾ ವಸುದಾ ವೃದ್ಧಾ ಬ್ರಹ್ಮಾತ್ಮೈಕ್ಯ - ಸ್ವರೂಪಿಣೀ~।\\
ಬೃಹತೀ ಬ್ರಾಹ್ಮಣೀ ಬ್ರಾಹ್ಮೀ ಬ್ರಹ್ಮಾನಂದಾ ಬಲಿಪ್ರಿಯಾ ॥೧೩೨॥

ಭಾಷಾರೂಪಾ ಬೃಹತ್ಸೇನಾ ಭಾವಾಭಾವ - ವಿವರ್ಜಿತಾ~।\\
ಸುಖಾರಾಧ್ಯಾ ಶುಭಕರೀ ಶೋಭನಾ ಸುಲಭಾ ಗತಿಃ ॥೧೩೩॥

ರಾಜ - ರಾಜೇಶ್ವರೀ ರಾಜ್ಯ - ದಾಯಿನೀ ರಾಜ್ಯ - ವಲ್ಲಭಾ~।\\
ರಾಜತ್ಕೃಪಾ ರಾಜಪೀಠ - ನಿವೇಶಿತ - ನಿಜಾಶ್ರಿತಾ ॥೧೩೪॥

ರಾಜ್ಯಲಕ್ಷ್ಮೀಃ ಕೋಶನಾಥಾ ಚತುರಂಗ - ಬಲೇಶ್ವರೀ~।\\
ಸಾಮ್ರಾಜ್ಯ - ದಾಯಿನೀ ಸತ್ಯಸಂಧಾ ಸಾಗರಮೇಖಲಾ ॥೧೩೫॥

ದೀಕ್ಷಿತಾ ದೈತ್ಯಶಮನೀ ಸರ್ವಲೋಕ - ವಶಂಕರೀ~।\\
ಸರ್ವಾರ್ಥದಾತ್ರೀ ಸಾವಿತ್ರೀ ಸಚ್ಚಿದಾನಂದ - ರೂಪಿಣೀ \as{(೭೦೦)} ॥೧೩೬॥

\as{(ಅಷ್ಟೋತ್ತರ ೬೧ - ೭೦)\\}
ದುಷ್ಟಭೀತಿಮಹಾಭೀತಿಭಂಜನಾಯೈ ನಮೋ ನಮಃ ।\\
ಸಮಸ್ತದೇವದನುಜಪ್ರೇರಕಾಯೈ ನಮೋ ನಮಃ ।\\
ಸಮಸ್ತಹೃದಯಾಂಭೋಜನಿಲಯಾಯೈ ನಮೋ ನಮಃ ।\\
ಅನಾಹತಮಹಾಪದ್ಮಮಂದಿರಾಯೈ ನಮೋ ನಮಃ ।\\
ಸಹಸ್ರಾರಸರೋಜಾತವಾಸಿತಾಯೈ ನಮೋ ನಮಃ ।\\
ಪುನರಾವೃತ್ತಿರಹಿತಪುರಸ್ಥಾಯೈ ನಮೋ ನಮಃ ।\\
ವಾಣೀಗಾಯತ್ರೀಸಾವಿತ್ರೀಸನ್ನುತಾಯೈ ನಮೋ ನಮಃ ।\\
ರಮಾಭೂಮಿಸುತಾರಾಧ್ಯಪದಾಬ್ಜಾಯೈ ನಮೋ ನಮಃ ।\\
ಲೋಪಾಮುದ್ರಾರ್ಚಿತಶ್ರೀಮಚ್ಚರಣಾಯೈ ನಮೋ ನಮಃ ।\\
ಸಹಸ್ರರತಿಸೌಂದರ್ಯಶರೀರಾಯೈ ನಮೋ ನಮಃ ।

\as{ತ್ರಿಧಾ॑ಹಿ॒ತಂ ಪ॒ಣಿಭಿ॑ರ್ಗು॒ಹ್ಯಮಾ॑ನಂ॒ ಗವಿ॑ ದೇ॒ವಾಸೋ॑ ಘೃ॒ತಮನ್ವ॑ವಿಂದನ್ ।
ಇಂದ್ರ॒ ಏಕ॒ಁ ಸೂರ್ಯ॒ ಏಕಂ॑ ಜಜಾನ ವೇ॒ನಾದೇಕಁ॑ ಸ್ವ॒ಧಯಾ॒ ನಿಷ್ಟ॑ತಕ್ಷುಃ ॥}
\section{ಅಂತರ್ದಶಾರಚಕ್ರಾಯ ನಮಃ ।\\ (ಇತಿ ವ್ಯಾಪಕಂ ನ್ಯಸ್ಯ)}
ದಕ್ಷನಾಸಾರೂಪ ರೇಚಕಾಗ್ನ್ಯಾತ್ಮನೇ ಸರ್ವಜ್ಞಾದೇವ್ಯೈ ನಮಃ ।\\
ದಕ್ಷಸೃಗ್ರೂಪ ಪಾಚಕಾಗ್ನ್ಯಾತ್ಮನೇ ಸರ್ವಶಕ್ತಿದೇವ್ಯೈ ನಮಃ ।\\
ದಕ್ಷಸ್ತನರೂಪ ಶೋಷಕಾಗ್ನ್ಯಾತ್ಮನೇ ಸರ್ವೈಶ್ವರ್ಯಪ್ರದಾದೇವ್ಯೈ ನಮಃ ।\\
ದಕ್ಷವೃಷಣರೂಪ ದಾಹಕಾಗ್ನ್ಯಾತ್ಮನೇ ಸರ್ವಜ್ಞಾನಮಯೀದೇವ್ಯೈ ನಮಃ ।\\
ಸೀವಿನೀರೂಪ ಪ್ಲಾವಕಾಗ್ನ್ಯಾತ್ಮನೇ ಸರ್ವವ್ಯಾಧಿವಿನಾಶಿನೀದೇವ್ಯೈ ನಮಃ ।\\
ವಾಮವೃಷಣರೂಪ ಕ್ಷಾರಕಾಗ್ನ್ಯಾತ್ಮನೇ ಸರ್ವಾಧಾರಸ್ವರೂಪಾದೇವ್ಯೈ ನಮಃ ।\\
ವಾಮಸ್ತನರೂಪ ಉದ್ಗಾರಕಾಗ್ನ್ಯಾತ್ಮನೇ ಸರ್ವಪಾಪಹರಾದೇವ್ಯೈ ನಮಃ ।\\
ವಾಮಸೃಗ್ರೂಪ ಕ್ಷೋಭಕಾಗ್ನ್ಯಾತ್ಮನೇ ಸರ್ವಾನಂದಮಯೀದೇವ್ಯೈ ನಮಃ ।\\
ವಾಮನಾಸಾರೂಪಜೃಂಭಕಾಗ್ನ್ಯಾತ್ಮನೇ ಸರ್ವರಕ್ಷಾಸ್ವರೂಪಿಣೀದೇವ್ಯೈ ನಮಃ ।\\
ನಾಸಾಗ್ರರೂಪ ಮೋಹಕಾಗ್ನ್ಯಾತ್ಮನೇ ಸರ್ವೇಪ್ಸಿತಫಲಪ್ರದಾದೇವ್ಯೈ ನಮಃ ।\\
ಹೃದ್ರೂಪ ಸರ್ವರಕ್ಷಾಕರಚಕ್ರೇಶ್ವರ್ಯೈ ತ್ರಿಪುರಮಾಲಿನ್ಯೈ ನಮಃ ।\\
ನಿಗರ್ಭಯೋಗಿನೀರೂಪ ಸ್ವಾತ್ಮಾತ್ಮನೇ ಪ್ರಾಕಾಮ್ಯಸಿದ್ಧ್ಯೈ ನಮಃ ।\\
ಅಪರಿಚ್ಛಿನ್ನರೂಪ ಸ್ವಾತ್ಮಾತ್ಮನೇ ಸರ್ವಮಹಾಂಕುಶಾಮುದ್ರಾಯೈ ನಮಃ ।

\as{(ಸಹಸ್ರನಾಮ ೭೦೧ - ೮೦೦)\\}
ದೇಶ - ಕಾಲಾಪರಿಚ್ಛಿನ್ನಾ ಸರ್ವಗಾ ಸರ್ವಮೋಹಿನೀ~।\\
ಸರಸ್ವತೀ ಶಾಸ್ತ್ರಮಯೀ ಗುಹಾಂಬಾ ಗುಹ್ಯರೂಪಿಣೀ ॥೧೩೭॥

ಸರ್ವೋಪಾಧಿ - ವಿನಿರ್ಮುಕ್ತಾ ಸದಾಶಿವ - ಪತಿವ್ರತಾ~।\\
ಸಂಪ್ರದಾಯೇಶ್ವರೀ ಸಾಧ್ವೀ ಗುರುಮಂಡಲ - ರೂಪಿಣೀ ॥೧೩೮॥

ಕುಲೋತ್ತೀರ್ಣಾ ಭಗಾರಾಧ್ಯಾ ಮಾಯಾ ಮಧುಮತೀ ಮಹೀ~।\\
ಗಣಾಂಬಾ ಗುಹ್ಯಕಾರಾಧ್ಯಾ ಕೋಮಲಾಂಗೀ ಗುರುಪ್ರಿಯಾ ॥೧೩೯॥

ಸ್ವತಂತ್ರಾ ಸರ್ವತಂತ್ರೇಶೀ ದಕ್ಷಿಣಾಮೂರ್ತಿ - ರೂಪಿಣೀ~।\\
ಸನಕಾದಿ - ಸಮಾರಾಧ್ಯಾ ಶಿವಜ್ಞಾನ - ಪ್ರದಾಯಿನೀ ॥೧೪೦॥

ಚಿತ್ಕಲಾಽಽನಂದ - ಕಲಿಕಾ ಪ್ರೇಮರೂಪಾ ಪ್ರಿಯಂಕರೀ~।\\
ನಾಮಪಾರಾಯಣ - ಪ್ರೀತಾ ನಂದಿವಿದ್ಯಾ ನಟೇಶ್ವರೀ ॥೧೪೧॥

ಮಿಥ್ಯಾ - ಜಗದಧಿಷ್ಠಾನಾ ಮುಕ್ತಿದಾ ಮುಕ್ತಿರೂಪಿಣೀ~।\\
ಲಾಸ್ಯಪ್ರಿಯಾ ಲಯಕರೀ ಲಜ್ಜಾ ರಂಭಾದಿವಂದಿತಾ ॥೧೪೨॥

ಭವದಾವ - ಸುಧಾವೃಷ್ಟಿಃ ಪಾಪಾರಣ್ಯ - ದವಾನಲಾ~।\\
ದೌರ್ಭಾಗ್ಯ - ತೂಲವಾತೂಲಾ ಜರಾಧ್ವಾಂತ - ರವಿಪ್ರಭಾ ॥೧೪೩॥

ಭಾಗ್ಯಾಬ್ಧಿ - ಚಂದ್ರಿಕಾ ಭಕ್ತ - ಚಿತ್ತಕೇಕಿ - ಘನಾಘನಾ~।\\
ರೋಗಪರ್ವತ - ದಂಭೋಲಿರ್ಮೃತ್ಯುದಾರು - ಕುಠಾರಿಕಾ ॥೧೪೪॥

ಮಹೇಶ್ವರೀ ಮಹಾಕಾಲೀ ಮಹಾಗ್ರಾಸಾ ಮಹಾಶನಾ~।\\
ಅಪರ್ಣಾ ಚಂಡಿಕಾ ಚಂಡಮುಂಡಾಸುರ - ನಿಷೂದಿನೀ ॥೧೪೫॥

ಕ್ಷರಾಕ್ಷರಾತ್ಮಿಕಾ ಸರ್ವ - ಲೋಕೇಶೀ ವಿಶ್ವಧಾರಿಣೀ~।\\
ತ್ರಿವರ್ಗದಾತ್ರೀ ಸುಭಗಾ ತ್ರ್ಯಂಬಕಾ ತ್ರಿಗುಣಾತ್ಮಿಕಾ ॥೧೪೬॥

ಸ್ವರ್ಗಾಪವರ್ಗದಾ ಶುದ್ಧಾ ಜಪಾಪುಷ್ಪ - ನಿಭಾಕೃತಿಃ~।\\
ಓಜೋವತೀ ದ್ಯುತಿಧರಾ ಯಜ್ಞರೂಪಾ ಪ್ರಿಯವ್ರತಾ ॥೧೪೭॥

ದುರಾರಾಧ್ಯಾ ದುರಾಧರ್ಷಾ ಪಾಟಲೀ - ಕುಸುಮ - ಪ್ರಿಯಾ~।\\
ಮಹತೀ ಮೇರುನಿಲಯಾ ಮಂದಾರ - ಕುಸುಮ - ಪ್ರಿಯಾ ॥೧೪೮॥

ವೀರಾರಾಧ್ಯಾ ವಿರಾಡ್ರೂಪಾ ವಿರಜಾ ವಿಶ್ವತೋಮುಖೀ~।\\
ಪ್ರತ್ಯಗ್ರೂಪಾ ಪರಾಕಾಶಾ ಪ್ರಾಣದಾ ಪ್ರಾಣರೂಪಿಣೀ ॥೧೪೯॥

ಮಾರ್ತಾಂಡ - ಭೈರವಾರಾಧ್ಯಾ ಮಂತ್ರಿಣೀನ್ಯಸ್ತ - ರಾಜ್ಯಧೂಃ~।\\
ತ್ರಿಪುರೇಶೀ ಜಯತ್ಸೇನಾ ನಿಸ್ತ್ರೈಗುಣ್ಯಾ ಪರಾಪರಾ ॥೧೫೦॥

ಸತ್ಯ - ಜ್ಞಾನಾನಂದ - ರೂಪಾ ಸಾಮರಸ್ಯ - ಪರಾಯಣಾ~।\\
ಕಪರ್ದಿನೀ ಕಲಾಮಾಲಾ ಕಾಮಧುಕ್ ಕಾಮರೂಪಿಣೀ ॥೧೫೧॥

ಕಲಾನಿಧಿಃ ಕಾವ್ಯಕಲಾ ರಸಜ್ಞಾ ರಸಶೇವಧಿಃ \as{(೮೦೦)}~।

\as{ಅಷ್ಟೋತ್ತರ (೭೧ - ೮೦)\\}
ಭಾವನಾಮಾತ್ರಸಂತುಷ್ಟಹೃದಯಾಯೈ ನಮೋ ನಮಃ ।\\
ಸತ್ಯಸಂಪೂರ್ಣವಿಜ್ಞಾನಸಿದ್ಧಿದಾಯೈ ನಮೋ ನಮಃ ।\\
ಶ್ರೀಲೋಚನಕೃತೋಲ್ಲಾಸಫಲದಾಯೈ ನಮೋ ನಮಃ ।\\
ಶ್ರೀಸುಧಾಬ್ಧಿಮಣಿದ್ವೀಪಮಧ್ಯಗಾಯೈ ನಮೋ ನಮಃ ।\\
ದಕ್ಷಾಧ್ವರವಿನಿರ್ಭೇದಸಾಧನಾಯೈ ನಮೋ ನಮಃ ।\\
ಶ್ರೀನಾಥಸೋದರೀಭೂತಶೋಭಿತಾಯೈ ನಮೋ ನಮಃ ।\\
ಚಂದ್ರಶೇಖರಭಕ್ತಾರ್ತಿಭಂಜನಾಯೈ ನಮೋ ನಮಃ ।\\
ಸರ್ವೋಪಾಧಿವಿನಿರ್ಮುಕ್ತಚೈತನ್ಯಾಯೈ ನಮೋ ನಮಃ ।\\
ನಾಮಪಾರಯಣಾಭೀಷ್ಟಫಲದಾಯೈ ನಮೋ ನಮಃ ।\\
ಸೃಷ್ಟಿಸ್ಥಿತಿತಿರೋಧಾನಸಂಕಲ್ಪಾಯೈ ನಮೋ ನಮಃ ।

\as{ಆ॒ದಿ॒ತ್ಯವ॑ರ್ಣೇ॒ ತಪ॒ಸೋಽಧಿ॑ಜಾ॒ತೋ ವನ॒ಸ್ಪತಿ॒ಸ್ತವ॑ ವೃ॒ಕ್ಷೋಽಥ ಬಿ॒ಲ್ವಃ ।\\
ತಸ್ಯ॒ ಫಲಾ᳚ನಿ॒ ತಪ॒ಸಾ ನು॑ದಂತು ಮಾ॒ಯಾಂತ॑ರಾ॒ಯಾಶ್ಚ॑ ಬಾ॒ಹ್ಯಾ ಅ॑ಲ॒ಕ್ಷ್ಮೀಃ ॥ ೬॥}

೪ ಹಕಾರರೂಪಾಯೈ ನಮಃ ।\\
೪ ಹಲಧೃತ್ಪೂಜಿತಾಯೈ ನಮಃ ।\\
೪ ಹರಿಣೇಕ್ಷಣಾಯೈ ನಮಃ ।\\
೪ ಹರಪ್ರಿಯಾಯೈ ನಮಃ ।\\
೪ ಹರಾರಾಧ್ಯಾಯೈ ನಮಃ ।\\
೪ ಹರಿಬ್ರಹ್ಮೇಂದ್ರವಂದಿತಾಯೈ ನಮಃ ।\\
೪ ಹಯಾರೂಢಾಸೇವಿತಾಂಘ್ರ್ಯೈ ನಮಃ ।\\
೪ ಹಯಮೇಧಸಮರ್ಚಿತಾಯೈ ನಮಃ ।\\
೪ ಹರ್ಯಕ್ಷವಾಹನಾಯೈ ನಮಃ ।\\
೪ ಹಂಸವಾಹನಾಯೈ ನಮಃ ।\\
೪ ಹತದಾನವಾಯೈ ನಮಃ ।\\
೪ ಹತ್ತ್ಯಾದಿಪಾಪಶಮನ್ಯೈ ನಮಃ ।\\
೪ ಹರಿದಶ್ವಾದಿಸೇವಿತಾಯೈ ನಮಃ ।\\
೪ ಹಸ್ತಿಕುಂಭೋತ್ತುಂಗಕುಚಾಯೈ ನಮಃ ।\\
೪ ಹಸ್ತಿಕೃತ್ತಿಪ್ರಿಯಾಂಗನಾಯೈ ನಮಃ ।\\
೪ ಹರಿದ್ರಾಕುಂಕುಮಾದಿಗ್ಧಾಯೈ ನಮಃ ।\\
೪ ಹರ್ಯಶ್ವಾದ್ಯಮರಾರ್ಚಿತಾಯೈ ನಮಃ ।\\
೪ ಹರಿಕೇಶಸಖ್ಯೈ ನಮಃ ।\\
೪ ಹಾದಿವಿದ್ಯಾಯೈ ನಮಃ ।\\
೪ ಹಾಲಾಮದಾಲಸಾಯೈ ನಮಃ ।

\as{ಉಪೈ॑ತು॒ ಮಾಂ ದೇ॑ವಸ॒ಖಃ ಕೀ॒ರ್ತಿಶ್ಚ॒ ಮಣಿ॑ನಾ ಸ॒ಹ ।\\
ಪ್ರಾ॒ದು॒ರ್ಭೂ॒ತೋಽಸ್ಮಿ॑ ರಾಷ್ಟ್ರೇ॒ಽಸ್ಮಿನ್ ಕೀ॒ರ್ತಿಮೃ॑ದ್ಧಿಂ ದ॒ದಾತು॑ ಮೇ ॥ ೭॥}

೪ ಸಕಾರರೂಪಾಯೈ ನಮಃ ।\\
೪ ಸರ್ವಜ್ಞಾಯೈ ನಮಃ ।\\
೪ ಸರ್ವೇಶ್ಯೈ ನಮಃ ।\\
೪ ಸರ್ವಮಂಗಲಾಯೈ ನಮಃ ।\\
೪ ಸರ್ವಕರ್ತ್ರ್ಯೈ ನಮಃ ।\\
೪ ಸರ್ವಭರ್ತ್ರ್ಯೈ ನಮಃ ।\\
೪ ಸರ್ವಹಂತ್ರ್ಯೈ ನಮಃ ।\\
೪ ಸನಾತನ್ಯೈ ನಮಃ ।\\
೪ ಸರ್ವಾನವದ್ಯಾಯೈ ನಮಃ ।\\
೪ ಸರ್ವಾಂಗಸುಂದರ್ಯೈ ನಮಃ ।\\
೪ ಸರ್ವಸಾಕ್ಷಿಣ್ಯೈ ನಮಃ ।\\
೪ ಸರ್ವಾತ್ಮಿಕಾಯೈ ನಮಃ ।\\
೪ ಸರ್ವಸೌಖ್ಯದಾತ್ರ್ಯೈ ನಮಃ ।\\
೪ ಸರ್ವವಿಮೋಹಿನ್ಯೈ ನಮಃ ।\\
೪ ಸರ್ವಾಧಾರಾಯೈ ನಮಃ ।\\
೪ ಸರ್ವಗತಾಯೈ ನಮಃ ।\\
೪ ಸರ್ವಾವಗುಣವರ್ಜಿತಾಯೈ ನಮಃ ।\\
೪ ಸರ್ವಾರುಣಾಯೈ ನಮಃ ।\\
೪ ಸರ್ವಮಾತ್ರೇ ನಮಃ ।\\
೪ ಸರ್ವಭೂಷಣಭೂಷಿತಾಯೈ ನಮಃ ।

\as{ಕ್ಷುತ್ಪಿ॑ಪಾ॒ಸಾಮ॑ಲಾಂ ಜ್ಯೇ॒ಷ್ಠಾಮ॑ಲ॒ಕ್ಷ್ಮೀಂ ನಾ॑ಶಯಾ॒ಮ್ಯಹಂ ।\\
ಅಭೂ॑ತಿ॒ಮಸ॑ಮೃದ್ಧಿಂ॒ ಚ ಸರ್ವಾಂ॒ ನಿರ್ಣು॑ದ ಮೇ॒ ಗೃಹಾ᳚ತ್ ॥ ೮॥}

೪ ಕಕಾರಾರ್ಥಾಯೈ ನಮಃ ।\\
೪ ಕಾಲಹಂತ್ರ್ಯೈ ನಮಃ ।\\
೪ ಕಾಮೇಶ್ಯೈ ನಮಃ ।\\
೪ ಕಾಮಿತಾರ್ಥದಾಯೈ ನಮಃ ।\\
೪ ಕಾಮಸಂಜೀವಿನ್ಯೈ ನಮಃ ।\\
೪ ಕಲ್ಯಾಯೈ ನಮಃ ।\\
೪ ಕಠಿನಸ್ತನಮಂಡಲಾಯೈ ನಮಃ ।\\
೪ ಕರಭೋರವೇ ನಮಃ ।\\
೪ ಕಲಾನಾಥಮುಖ್ಯೈ ನಮಃ\\
೪ ಕಚಜಿತಾಂಬುದಾಯೈ ನಮಃ ।\\
೪ ಕಟಾಕ್ಷಸ್ಯಂದಿಕರುಣಾಯೈ ನಮಃ ।\\
೪ ಕಪಾಲಿಪ್ರಾಣನಾಯಿಕಾಯೈ ನಮಃ ।\\
೪ ಕಾರುಣ್ಯವಿಗ್ರಹಾಯೈ ನಮಃ ।\\
೪ ಕಾಂತಾಯೈ ನಮಃ ।\\
೪ ಕಾಂತಿಧೂತಜಪಾವಲ್ಯೈ ನಮಃ ।\\
೪ ಕಲಾಲಾಪಾಯೈ ನಮಃ ।\\
೪ ಕಂಬುಕಂಠ್ಯೈ ನಮಃ ।\\
೪ ಕರನಿರ್ಜಿತಪಲ್ಲವಾಯೈ ನಮಃ ।\\
೪ ಕಲ್ಪವಲ್ಲೀಸಮಭುಜಾಯೈ ನಮಃ ।\\
೪ ಕಸ್ತೂರೀತಿಲಕಾಂಚಿತಾಯೈ ನಮಃ ।

\as{ಗಂ॒ಧ॒ದ್ವಾ॒ರಾಂ ದು॑ರಾಧ॒ರ್ಷಾಂ॒ ನಿ॒ತ್ಯಪು॑ಷ್ಟಾಂ ಕರೀ॒ಷಿಣೀಂ᳚ ।\\
ಈ॒ಶ್ವರೀ॑ँ ಸರ್ವ॑ಭೂತಾ॒ನಾಂ॒ ತಾಮಿ॒ಹೋಪ॑ಹ್ವಯೇ॒ ಶ್ರಿಯಂ ॥ ೯॥}

೪ ಹಕಾರಾರ್ಥಾಯೈ ನಮಃ ।\\
೪ ಹಂಸಗತ್ಯೈ ನಮಃ ।\\
೪ ಹಾಟಕಾಭರಣೋಜ್ಜ್ವಲಾಯೈ ನಮಃ ।\\
೪ ಹಾರಹಾರಿಕುಚಾಭೋಗಾಯೈ ನಮಃ ।\\
೪ ಹಾಕಿನ್ಯೈ ನಮಃ ।\\
೪ ಹಲ್ಯವರ್ಜಿತಾಯೈ ನಮಃ ।\\
೪ ಹರಿತ್ಪತಿಸಮಾರಾಧ್ಯಾಯೈ ನಮಃ ।\\
೪ ಹಠಾತ್ಕಾರಹತಾಸುರಾಯೈ ನಮಃ ।\\
೪ ಹರ್ಷಪ್ರದಾಯೈ ನಮಃ ।\\
೪ ಹವಿರ್ಭೋಕ್ತ್ರ್ಯೈ ನಮಃ ।\\
೪ ಹಾರ್ದಸಂತಮಸಾಪಹಾಯೈ ನಮಃ ।\\
೪ ಹಲ್ಲೀಸಲಾಸ್ಯಸಂತುಷ್ಟಾಯೈ ನಮಃ ।\\
೪ ಹಂಸಮಂತ್ರಾರ್ಥರೂಪಿಣ್ಯೈ ನಮಃ ।\\
೪ ಹಾನೋಪಾದಾನನಿರ್ಮುಕ್ತಾಯೈ ನಮಃ ।\\
೪ ಹರ್ಷಿಣ್ಯೈ ನಮಃ ।\\
೪ ಹರಿಸೋದರ್ಯೈ ನಮಃ ।\\
೪ ಹಾಹಾಹೂಹೂಮುಖಸ್ತುತ್ಯಾಯೈ ನಮಃ ।\\
೪ ಹಾನಿವೃದ್ಧಿವಿವರ್ಜಿತಾಯೈ ನಮಃ ।\\
೪ ಹಯ್ಯಂಗವೀನಹೃದಯಾಯೈ ನಮಃ ।\\
೪ ಹರಿಗೋಪಾರುಣಾಂಶುಕಾಯೈ ನಮಃ ।

\as{ಮನ॑ಸಃ॒ ಕಾಮ॒ಮಾಕೂ᳚ತಿಂ ವಾ॒ಚಃ ಸ॒ತ್ಯಮ॑ಶೀಮಹಿ ।\\
ಪ॒ಶೂ॒ನಾಂ ರೂ॒ಪಮನ್ನ॑ಸ್ಯ॒ ಮಯಿ॒ ಶ್ರೀಃ ಶ್ರ॑ಯತಾಂ॒ ಯಶಃ॑ ॥ ೧೦॥}

೪ ಲಕಾರಾಖ್ಯಾಯೈ ನಮಃ ।\\
೪ ಲತಾಪೂಜ್ಯಾಯೈ ನಮಃ ।\\
೪ ಲಯಸ್ಥಿತ್ಯುದ್ಭವೇಶ್ವರ್ಯೈ ನಮಃ ।\\
೪ ಲಾಸ್ಯದರ್ಶನಸಂತುಷ್ಟಾಯೈ ನಮಃ ।\\
೪ ಲಾಭಾಲಾಭವಿವರ್ಜಿತಾಯೈ ನಮಃ ।\\
೪ ಲಂಘ್ಯೇತರಾಜ್ಞಾಯೈ ನಮಃ ।\\
೪ ಲಾವಣ್ಯಶಾಲಿನ್ಯೈ ನಮಃ ।\\
೪ ಲಘುಸಿದ್ಧಿದಾಯೈ ನಮಃ ।\\
೪ ಲಾಕ್ಷಾರಸಸವರ್ಣಾಭಾಯೈ ನಮಃ ।\\
೪ ಲಕ್ಷ್ಮಣಾಗ್ರಜಪೂಜಿತಾಯೈ ನಮಃ ।\\
೪ ಲಭ್ಯೇತರಾಯೈ ನಮಃ ।\\
೪ ಲಬ್ಧಭಕ್ತಿಸುಲಭಾಯೈ ನಮಃ ।\\
೪ ಲಾಂಗಲಾಯುಧಾಯೈ ನಮಃ ।\\
೪ ಲಗ್ನಚಾಮರಹಸ್ತ ಶ್ರೀಶಾರದಾ ಪರಿವೀಜಿತಾಯೈ ನಮಃ ।\\
೪ ಲಜ್ಜಾಪದಸಮಾರಾಧ್ಯಾಯೈ ನಮಃ ।\\
೪ ಲಂಪಟಾಯೈ ನಮಃ ।\\
೪ ಲಕುಲೇಶ್ವರ್ಯೈ ನಮಃ ।\\
೪ ಲಬ್ಧಮಾನಾಯೈ ನಮಃ ।\\
೪ ಲಬ್ಧರಸಾಯೈ ನಮಃ ।\\
೪ ಲಬ್ಧಸಂಪತ್ಸಮುನ್ನತ್ಯೈ ನಮಃ ।

\as{ಕ॒ರ್ದಮೇ॑ನ ಪ್ರ॑ಜಾಭೂ॒ತಾ॒ ಮ॒ಯಿ॒ ಸಂಭ॑ವ ಕ॒ರ್ದಮ ।\\
ಶ್ರಿಯಂ॑ ವಾ॒ಸಯ॑ ಮೇ ಕು॒ಲೇ ಮಾ॒ತರಂ॑ ಪದ್ಮ॒ಮಾಲಿ॑ನೀಂ ॥ ೧೧॥}

೪ ಹ್ರೀಂಕಾರಿಣ್ಯೈ ನಮಃ ।\\
೪ ಹ್ರೀಂಕಾರಾದ್ಯಾಯೈ ನಮಃ ।\\
೪ ಹ್ರೀಂಮಧ್ಯಾಯೈ ನಮಃ ।\\
೪ ಹ್ರೀಂಶಿಖಾಮಣಯೇ ನಮಃ ।\\
೪ ಹ್ರೀಂಕಾರಕುಂಡಾಗ್ನಿಶಿಖಾಯೈ ನಮಃ ।\\
೪ ಹ್ರೀಂಕಾರಶಶಿಚಂದ್ರಿಕಾಯೈ ನಮಃ ।\\
೪ ಹ್ರೀಂಕಾರಭಾಸ್ಕರರುಚ್ಯೈ ನಮಃ ।\\
೪ ಹ್ರೀಂಕಾರಾಂಭೋದಚಂಚಲಾಯೈ ನಮಃ ।\\
೪ ಹ್ರೀಂಕಾರಕಂದಾಂಕುರಿಕಾಯೈ ನಮಃ ।\\
೪ ಹ್ರೀಂಕಾರೈಕಪರಾಯಣಾಯೈ ನಮಃ ।\\
೪ ಹ್ರೀಂಕಾರದೀರ್ಘಿಕಾಹಂಸ್ಯೈ ನಮಃ ।\\
೪ ಹ್ರೀಂಕಾರೋದ್ಯಾನಕೇಕಿನ್ಯೈ ನಮಃ ।\\
೪ ಹ್ರೀಂಕಾರಾರಣ್ಯಹರಿಣ್ಯೈ ನಮಃ ।\\
೪ ಹ್ರೀಂಕಾರಾವಾಲವಲ್ಲರ್ಯೈ ನಮಃ ।\\
೪ ಹ್ರೀಂಕಾರಪಂಜರಶುಕ್ಯೈ ನಮಃ ।\\
೪ ಹ್ರೀಂಕಾರಾಂಗಣದೀಪಿಕಾಯೈ ನಮಃ ।\\
೪ ಹ್ರೀಂಕಾರಕಂದರಾಸಿಂಹ್ಯೈ ನಮಃ ।\\
೪ ಹ್ರೀಂಕಾರಾಂಭೋಜಭೃಂಗಿಕಾಯೈ ನಮಃ ।\\
೪ ಹ್ರೀಂಕಾರಸುಮನೋಮಾಧ್ವ್ಯೈ ನಮಃ ।\\
೪ ಹ್ರೀಂಕಾರತರುಮಂಜರ್ಯೈ ನಮಃ ।

\as{ಯೋ ದೇ॒ವಾನಾಂ᳚ ಪ್ರಥ॒ಮಂ ಪು॒ರಸ್ತಾ॒ದ್ವಿಶ್ವಾ॒ಧಿಯೋ॑ ರು॒ದ್ರೋ ಮ॒ಹರ್ಷಿಃ॑ ।
ಹಿ॒ರ॒ಣ್ಯ॒ಗ॒ರ್ಭಂ ಪ॑ಶ್ಯತ॒ ಜಾಯ॑ಮಾನಁ॒ ಸನೋ॑ ದೇ॒ವಶ್ಶು॒ಭಯಾ॒ ಸ್ಮೃತ್ಯಾ॒ ಸಂಯು॑ನಕ್ತು ॥}
\section{ಅಷ್ಟಕೋಣಚಕ್ರಾಯ ನಮಃ ।\\ (ಇತಿ ವ್ಯಾಪಕಂ ನ್ಯಸ್ಯ)}
ಚುಬುಕದಕ್ಷಭಾಗರೂಪ ಶೀತಾತ್ಮನೇ ವಶಿನೀವಾಗ್ದೇವತಾಯೈ ನಮಃ ।\\
ಕಂಠದಕ್ಷಭಾಗರೂಪ ಉಷ್ಣಾತ್ಮನೇ ಕಾಮೇಶ್ವರೀವಾಗ್ದೇವತಾಯೈ ನಮಃ ।\\
ಹೃದಯದಕ್ಷಭಾಗರೂಪ ಸುಖಾತ್ಮನೇ ಮೋದಿನೀವಾಗ್ದೇವತಾಯೈ ನಮಃ ।\\
ನಾಭಿದಕ್ಷಭಾಗರೂಪ ದುಃಖಾತ್ಮನೇ ವಿಮಲಾವಾಗ್ದೇವತಾಯೈ ನಮಃ ।\\
ನಾಭಿವಾಮಭಾಗರೂಪ ಇಚ್ಛಾತ್ಮನೇ ಅರುಣಾವಾಗ್ದೇವತಾಯೈ ನಮಃ ।\\
ಹೃದಯವಾಮಭಾಗರೂಪ ಸತ್ವಗುಣಾತ್ಮನೇ ಜಯಿನೀವಾಗ್ದೇವತಾಯೈ ನಮಃ ।\\
ಕಂಠವಾಮಭಾಗರೂಪ ರಜೋಗುಣಾತ್ಮನೇ ಸರ್ವೇಶ್ವರೀವಾಗ್ದೇವತಾಯೈ ನಮಃ ।\\
ಚುಬುಕವಾಮಭಾಗರೂಪ ತಮೋಗುಣಾತ್ಮನೇ ಕೌಳಿನೀವಾಗ್ದೇವತಾಯೈ ನಮಃ ।\\
ಹೃದ್ರೂಪ ಸರ್ವರೋಗಹರ ಚಕ್ರೇಶ್ವರ್ಯೈ ತ್ರಿಪುರಾಸಿದ್ಧಾಯೈ ನಮಃ ।\\
ರಹಸ್ಯಯೋಗಿನೀ ರೂಪ ಸ್ವಾತ್ಮಾತ್ಮನೇ ಭುಕ್ತಿಸಿದ್ಧ್ಯೈ ನಮಃ ।\\
ಅಪರಿಚ್ಛಿನ್ನರೂಪ ಸ್ವಾತ್ಮಾತ್ಮನೇ ಸರ್ವಖೇಚರೀಮುದ್ರಾಯೈ ನಮಃ ।

\as{(ಸಹಸ್ರನಾಮ ೮೦೧ - ೯೦೦)\\}
ಪುಷ್ಟಾ ಪುರಾತನಾ ಪೂಜ್ಯಾ ಪುಷ್ಕರಾ ಪುಷ್ಕರೇಕ್ಷಣಾ ॥೧೫೨॥

ಪರಂಜ್ಯೋತಿಃ ಪರಂಧಾಮ ಪರಮಾಣುಃ ಪರಾತ್ಪರಾ~।\\
ಪಾಶಹಸ್ತಾ ಪಾಶಹಂತ್ರೀ ಪರಮಂತ್ರ - ವಿಭೇದಿನೀ ॥೧೫೩॥

ಮೂರ್ತಾಽಮೂರ್ತಾಽನಿತ್ಯತೃಪ್ತಾ ಮುನಿಮಾನಸ - ಹಂಸಿಕಾ~।\\
ಸತ್ಯವ್ರತಾ ಸತ್ಯರೂಪಾ ಸರ್ವಾಂತರ್ಯಾಮಿನೀ ಸತೀ ॥೧೫೪॥

ಬ್ರಹ್ಮಾಣೀ ಬ್ರಹ್ಮಜನನೀ ಬಹುರೂಪಾ ಬುಧಾರ್ಚಿತಾ~।\\
ಪ್ರಸವಿತ್ರೀ ಪ್ರಚಂಡಾಽಽಜ್ಞಾ ಪ್ರತಿಷ್ಠಾ ಪ್ರಕಟಾಕೃತಿಃ ॥೧೫೫॥

ಪ್ರಾಣೇಶ್ವರೀ ಪ್ರಾಣದಾತ್ರೀ ಪಂಚಾಶತ್ಪೀಠ - ರೂಪಿಣೀ~।\\
ವಿಶೃಂಖಲಾ ವಿವಿಕ್ತಸ್ಥಾ ವೀರಮಾತಾ ವಿಯತ್ಪ್ರಸೂಃ ॥೧೫೬॥

ಮುಕುಂದಾ ಮುಕ್ತಿನಿಲಯಾ ಮೂಲವಿಗ್ರಹ - ರೂಪಿಣೀ~।\\
ಭಾವಜ್ಞಾ ಭವರೋಗಘ್ನೀ ಭವಚಕ್ರ - ಪ್ರವರ್ತಿನೀ ॥೧೫೭॥

ಛಂದಃಸಾರಾ ಶಾಸ್ತ್ರಸಾರಾ ಮಂತ್ರಸಾರಾ ತಲೋದರೀ~।\\
ಉದಾರಕೀರ್ತಿರುದ್ದಾಮವೈಭವಾ ವರ್ಣರೂಪಿಣೀ ॥೧೫೮॥

ಜನ್ಮಮೃತ್ಯು - ಜರಾತಪ್ತ - ಜನವಿಶ್ರಾಂತಿ - ದಾಯಿನೀ~।\\
ಸರ್ವೋಪನಿಷದುದ್ಘುಷ್ಟಾ ಶಾಂತ್ಯತೀತ - ಕಲಾತ್ಮಿಕಾ ॥೧೫೯॥

ಗಂಭೀರಾ ಗಗನಾಂತಸ್ಥಾ ಗರ್ವಿತಾ ಗಾನಲೋಲುಪಾ~।\\
ಕಲ್ಪನಾ - ರಹಿತಾ ಕಾಷ್ಠಾಽಕಾಂತಾ ಕಾಂತಾರ್ಧ - ವಿಗ್ರಹಾ ॥೧೬೦॥

ಕಾರ್ಯಕಾರಣ - ನಿರ್ಮುಕ್ತಾ ಕಾಮಕೇಲಿ - ತರಂಗಿತಾ~।\\
ಕನತ್ಕನಕತಾ - ಟಂಕಾ ಲೀಲಾ - ವಿಗ್ರಹ - ಧಾರಿಣೀ ॥೧೬೧॥

ಅಜಾ ಕ್ಷಯವಿನಿರ್ಮುಕ್ತಾ ಮುಗ್ಧಾ ಕ್ಷಿಪ್ರ - ಪ್ರಸಾದಿನೀ~।\\
ಅಂತರ್ಮುಖ - ಸಮಾರಾಧ್ಯಾ ಬಹಿರ್ಮುಖ - ಸುದುರ್ಲಭಾ ॥೧೬೨॥

ತ್ರಯೀ ತ್ರಿವರ್ಗನಿಲಯಾ ತ್ರಿಸ್ಥಾ ತ್ರಿಪುರಮಾಲಿನೀ~।\\
ನಿರಾಮಯಾ ನಿರಾಲಂಬಾ ಸ್ವಾತ್ಮಾರಾಮಾ ಸುಧಾಸೃತಿಃ ॥೧೬೩॥

ಸಂಸಾರಪಂಕ - ನಿರ್ಮಗ್ನ - ಸಮುದ್ಧರಣ - ಪಂಡಿತಾ~।\\
ಯಜ್ಞಪ್ರಿಯಾ ಯಜ್ಞಕರ್ತ್ರೀ ಯಜಮಾನ - ಸ್ವರೂಪಿಣೀ ॥೧೬೪॥

ಧರ್ಮಾಧಾರಾ ಧನಾಧ್ಯಕ್ಷಾ ಧನಧಾನ್ಯ - ವಿವರ್ಧಿನೀ~।\\
ವಿಪ್ರಪ್ರಿಯಾ ವಿಪ್ರರೂಪಾ ವಿಶ್ವಭ್ರಮಣ - ಕಾರಿಣೀ ॥೧೬೫॥

ವಿಶ್ವಗ್ರಾಸಾ ವಿದ್ರುಮಾಭಾ ವೈಷ್ಣವೀ ವಿಷ್ಣುರೂಪಿಣೀ~।\\
ಅಯೋನಿರ್ಯೋನಿನಿಲಯಾ ಕೂಟಸ್ಥಾ ಕುಲರೂಪಿಣೀ ॥೧೬೬॥

ವೀರಗೋಷ್ಠೀಪ್ರಿಯಾ ವೀರಾ ನೈಷ್ಕರ್ಮ್ಯಾ \as{(೯೦೦)}

\as{(ಅಷ್ಟೋತ್ತರ ೮೦ ೯೦)\\}
ಶ್ರೀಷೋಡಶಾಕ್ಷರೀಮಂತ್ರಮಧ್ಯಗಾಯೈ ನಮೋ ನಮಃ ।\\
ಅನಾದ್ಯಂತಸ್ವಯಂಭೂತದಿವ್ಯಮೂರ್ತ್ಯೈ ನಮೋ ನಮಃ ।\\
ಭಕ್ತಹಂಸಪರೀಮುಖ್ಯವಿಯೋಗಾಯೈ ನಮೋ ನಮಃ ।\\
ಮಾತೃಮಂಡಲಸಂಯುಕ್ತಲಲಿತಾಯೈ ನಮೋ ನಮಃ ।\\
ಭಂಡದೈತ್ಯಮಹಾಸತ್ತ್ವನಾಶನಾಯೈ ನಮೋ ನಮಃ ।\\
ಕ್ರೂರಭಂಡಶಿರಚ್ಛೇದನಿಪುಣಾಯೈ ನಮೋ ನಮಃ ।\\
ಧಾತ್ರಚ್ಯುತಸುರಾಧೀಶಸುಖದಾಯೈ ನಮೋ ನಮಃ ।\\
ಚಂಡಮುಂಡನಿಶುಂಭಾದಿಖಂಡನಾಯೈ ನಮೋ ನಮಃ ।\\
ರಕ್ತಾಕ್ಷರಕ್ತಜಿಹ್ವಾದಿಶಿಕ್ಷಣಾಯೈ ನಮೋ ನಮಃ ।\\
ಮಹಿಷಾಸುರದೋರ್ವೀರ್ಯನಿಗ್ರಹಾಯೈ ನಮೋ ನಮಃ ।

\as{ಯಸ್ಮಾ॒ತ್ಪರಂ॒ ನಾಪ॑ರ॒ಮಸ್ತಿ॒ ಕಿಂಚಿ॒ದ್ಯಸ್ಮಾ॒ನ್ನಾಣೀ॑ಯೋ॒ ನ ಜ್ಯಾಯೋ᳚ಽಸ್ತಿ॒ ಕಶ್ಚಿ॑ತ್ ।
ವೃ॒ಕ್ಷ ಇ॑ವ ಸ್ತಬ್ಧೋ ದಿ॒ವಿ ತಿ॑ಷ್ಠ॒ತ್ಯೇಕ॒ಸ್ತೇನೇ॒ದಂಪೂ॒ರ್ಣಂ ಪುರು॑ಷೇಣ॒ ಸರ್ವಂ᳚ ॥}

ಹೃದಯತ್ರಿಕೋಣಾಧೋಭಾಗರೂಪಪಂಚತನ್ಮಾತ್ರಾತ್ಮಕೇಭ್ಯಃ ಸರ್ವಜಂಭನಬಾಣೇಭ್ಯೋ ನಮಃ ॥\\
ತದ್ದಕ್ಷಭಾಗರೂಪಮನ ಆತ್ಮಕಾಭ್ಯಾಂ ಸರ್ವಮೋಹನಧನುರ್ಭ್ಯಾಂ ನಮಃ ।\\
ತದೂರ್ಧ್ವಭಾಗರೂಪರಾಗಾತ್ಮಕಾಭ್ಯಾಂ ಸರ್ವವಶಂಕರಪಾಶಾಭ್ಯಾಂ ನಮಃ ।\\
ತದ್ವಾಮಭಾಗರೂಪದ್ವೇಷಾತ್ಮಕಾಭ್ಯಾಂ ಸರ್ವಸ್ತಂಭಕರಾಂಕುಶಾಭ್ಯಾಂ ನಮಃ ॥
\section{ತ್ರಿಕೋಣಚಕ್ರಾಯ ನಮಃ ।\\ (ಇತಿ ವ್ಯಾಪಕಂ ನ್ಯಸ್ಯ)}
ಹೃದಯತ್ರಿಕೋಣಾಗ್ರಭಾಗರೂಪ ಮಹತ್ತತ್ವಾತ್ಮನೇ ಕಾಮೇಶ್ವರೀ ದೇವ್ಯೈ ನಮಃ ।\\
ತದ್ದಕ್ಷಕೋಣರೂಪಾಂಹಕಾರಾತ್ಮನೇ ವಜ್ರೇಶ್ವರೀ ದೇವ್ಯೈ ನಮಃ ।\\
ತದ್ವಾಮಕೋಣರೂಪಾವ್ಯಕ್ತಾತ್ಮನೇ ಭಗಮಾಲಿನೀ ದೇವ್ಯೈ ನಮಃ ।\\
ಹೃದ್ರೂಪ ಸರ್ವಸಿದ್ಧಿಪ್ರದಚಕ್ರೇಶ್ವರ್ಯೈ ತ್ರಿಪುರಾಂಬಾಯೈ ನಮಃ ।\\
ಅತಿರಹಸ್ಯಯೋಗಿನೀರೂಪ ಸ್ವಾತ್ಮಾತ್ಮನೇ ಇಚ್ಛಾಸಿದ್ಧ್ಯೈ ನಮಃ ।\\
ಅಪರಿಚ್ಛಿನ್ನರೂಪ ಸ್ವಾತ್ಮಾತ್ಮನೇ ಸರ್ವಬೀಜಮುದ್ರಾಯೈ ನಮಃ ।

\as{(ಸಹಸ್ರನಾಮ ೯೦೧ - ೧೦೦೦)\\}
ನಾದರೂಪಿಣೀ~।\\
ವಿಜ್ಞಾನಕಲನಾ ಕಲ್ಯಾ ವಿದಗ್ಧಾ ಬೈಂದವಾಸನಾ ॥೧೬೭॥

ತತ್ತ್ವಾಧಿಕಾ ತತ್ತ್ವಮಯೀ ತತ್ತ್ವಮರ್ಥ - ಸ್ವರೂಪಿಣೀ~।\\
ಸಾಮಗಾನಪ್ರಿಯಾ ಸೌಮ್ಯಾ ಸದಾಶಿವ - ಕುಟುಂಬಿನೀ ॥೧೬೮॥

ಸವ್ಯಾಪಸವ್ಯ - ಮಾರ್ಗಸ್ಥಾ ಸರ್ವಾಪದ್ವಿನಿವಾರಿಣೀ~।\\
ಸ್ವಸ್ಥಾ ಸ್ವಭಾವಮಧುರಾ ಧೀರಾ ಧೀರಸಮರ್ಚಿತಾ ॥೧೬೯॥

ಚೈತನ್ಯಾರ್ಘ್ಯ - ಸಮಾರಾಧ್ಯಾ ಚೈತನ್ಯ - ಕುಸುಮಪ್ರಿಯಾ~।\\
ಸದೋದಿತಾ ಸದಾತುಷ್ಟಾ ತರುಣಾದಿತ್ಯ - ಪಾಟಲಾ ॥೧೭೦॥

ದಕ್ಷಿಣಾ - ದಕ್ಷಿಣಾರಾಧ್ಯಾ ದರಸ್ಮೇರ - ಮುಖಾಂಬುಜಾ~।\\
ಕೌಲಿನೀ - ಕೇವಲಾಽನರ್ಘ್ಯ - ಕೈವಲ್ಯ - ಪದದಾಯಿನೀ ॥೧೭೧॥

ಸ್ತೋತ್ರಪ್ರಿಯಾ ಸ್ತುತಿಮತೀ ಶ್ರುತಿ - ಸಂಸ್ತುತ - ವೈಭವಾ~।\\
ಮನಸ್ವಿನೀ ಮಾನವತೀ ಮಹೇಶೀ ಮಂಗಲಾಕೃತಿಃ ॥೧೭೨॥

ವಿಶ್ವಮಾತಾ ಜಗದ್ಧಾತ್ರೀ ವಿಶಾಲಾಕ್ಷೀ ವಿರಾಗಿಣೀ~।\\
ಪ್ರಗಲ್ಭಾ ಪರಮೋದಾರಾ ಪರಾಮೋದಾ ಮನೋಮಯೀ ॥೧೭೩॥

ವ್ಯೋಮಕೇಶೀ ವಿಮಾನಸ್ಥಾ ವಜ್ರಿಣೀ ವಾಮಕೇಶ್ವರೀ~।\\
ಪಂಚಯಜ್ಞ - ಪ್ರಿಯಾ ಪಂಚ - ಪ್ರೇತ - ಮಂಚಾಧಿಶಾಯಿನೀ ॥೧೭೪॥

ಪಂಚಮೀ ಪಂಚಭೂತೇಶೀ ಪಂಚ - ಸಂಖ್ಯೋಪಚಾರಿಣೀ~।\\
ಶಾಶ್ವತೀ ಶಾಶ್ವತೈಶ್ವರ್ಯಾ ಶರ್ಮದಾ ಶಂಭುಮೋಹಿನೀ ॥೧೭೫॥

ಧರಾ ಧರಸುತಾ ಧನ್ಯಾ ಧರ್ಮಿಣೀ ಧರ್ಮವರ್ಧಿನೀ~।\\
ಲೋಕಾತೀತಾ ಗುಣಾತೀತಾ ಸರ್ವಾತೀತಾ ಶಮಾತ್ಮಿಕಾ ॥೧೭೬॥

ಬಂಧೂಕ - ಕುಸುಮಪ್ರಖ್ಯಾ ಬಾಲಾ ಲೀಲಾವಿನೋದಿನೀ~।\\
ಸುಮಂಗಲೀ ಸುಖಕರೀ ಸುವೇಷಾಢ್ಯಾ ಸುವಾಸಿನೀ ॥೧೭೭॥

ಸುವಾಸಿನ್ಯರ್ಚನ - ಪ್ರೀತಾಽಽಶೋಭನಾ ಶುದ್ಧಮಾನಸಾ~।\\
ಬಿಂದು - ತರ್ಪಣ - ಸಂತುಷ್ಟಾ ಪೂರ್ವಜಾ ತ್ರಿಪುರಾಂಬಿಕಾ ॥೧೭೮॥

ದಶಮುದ್ರಾ - ಸಮಾರಾಧ್ಯಾ ತ್ರಿಪುರಾಶ್ರೀ - ವಶಂಕರೀ~।\\
ಜ್ಞಾನಮುದ್ರಾ ಜ್ಞಾನಗಮ್ಯಾ ಜ್ಞಾನಜ್ಞೇಯ - ಸ್ವರೂಪಿಣೀ ॥೧೭೯॥

ಯೋನಿಮುದ್ರಾ ತ್ರಿಖಂಡೇಶೀ ತ್ರಿಗುಣಾಂಬಾ ತ್ರಿಕೋಣಗಾ~।\\
ಅನಘಾಽದ್ಭುತ - ಚಾರಿತ್ರಾ ವಾಂಛಿತಾರ್ಥ - ಪ್ರದಾಯಿನೀ ॥೧೮೦॥

ಅಭ್ಯಾಸಾತಿಶಯ - ಜ್ಞಾತಾ ಷಡಧ್ವಾತೀತ - ರೂಪಿಣೀ~।\\
ಅವ್ಯಾಜ - ಕರುಣಾ - ಮೂರ್ತಿರಜ್ಞಾನ - ಧ್ವಾಂತ - ದೀಪಿಕಾ ॥೧೮೧॥

ಆಬಾಲ - ಗೋಪ - ವಿದಿತಾ ಸರ್ವಾನುಲ್ಲಂಘ್ಯ - ಶಾಸನಾ~।\\
ಶ್ರೀಚಕ್ರರಾಜ - ನಿಲಯಾ ಶ್ರೀಮತ್ - ತ್ರಿಪುರಸುಂದರೀ ॥೧೮೨॥

ಶ್ರೀಶಿವಾ ಶಿವಶಕ್ತ್ಯೈಕ್ಯ  ರೂಪಿಣೀ ಲಲಿತಾಂಬಿಕಾ\as{(೧೦೦೦)। ಶ್ರೀಂಹ್ರೀಂಐಂ ಓಂ}

\as{(ಅಷ್ಟೋತ್ತರ ೯೦ - ೧೦೮)\\}
ಅಭ್ರಕೇಶಮಹೋತ್ಸಾಹಕಾರಣಾಯೈ ನಮೋ ನಮಃ ।\\
ಮಹೇಶಯುಕ್ತನಟನತತ್ಪರಾಯೈ ನಮೋ ನಮಃ ।\\
ನಿಜಭರ್ತೃಮುಖಾಂಭೋಜಚಿಂತನಾಯೈ ನಮೋ ನಮಃ ।\\
ವೃಷಭಧ್ವಜವಿಜ್ಞಾನಭಾವನಾಯೈ ನಮೋ ನಮಃ ।\\
ಜನ್ಮಮೃತ್ಯುಜರಾರೋಗಭಂಜನಾಯೈ ನಮೋ ನಮಃ ।\\
ವಿಧೇಯಮುಕ್ತವಿಜ್ಞಾನಸಿದ್ಧಿದಾಯೈ ನಮೋ ನಮಃ ।\\
ಕಾಮಕ್ರೋಧಾದಿಷಡ್ವರ್ಗನಾಶನಾಯೈ ನಮೋ ನಮಃ ।\\
ರಾಜರಾಜಾರ್ಚಿತಪದಸರೋಜಾಯೈ ನಮೋ ನಮಃ ।\\
ಸರ್ವವೇದಾಂತಸಂಸಿದ್ಧಸುತತ್ವಾಯೈ ನಮೋ ನಮಃ ।\\
ಶ್ರೀವೀರಭಕ್ತವಿಜ್ಞಾನವಿಧಾನಾಯೈ ನಮೋ ನಮಃ ।\\
ಅಶೇಷದುಷ್ಟದನುಜಸೂದನಾಯೈ ನಮೋ ನಮಃ ।\\
ಸಾಕ್ಷಾಚ್ಛ್ರೀದಕ್ಷಿಣಾಮೂರ್ತಿಮನೋಜ್ಞಾಯೈ ನಮೋ ನಮಃ ।\\
ಹಯಮೇಧಾಗ್ರಸಂಪೂಜ್ಯಮಹಿಮಾಯೈ ನಮೋ ನಮಃ ।\\
ದಕ್ಷಪ್ರಜಾಪತಿಸುತಾವೇಷಾಢ್ಯಾಯೈ ನಮೋ ನಮಃ ।\\
ಸುಮಬಾಣೇಕ್ಷುಕೋದಂಡಮಂಡಿತಾಯೈ ನಮೋ ನಮಃ ।\\
ನಿತ್ಯಯೌವನಮಾಂಗಲ್ಯಮಂಗಲಾಯೈ ನಮೋ ನಮಃ ।\\
ಮಹಾದೇವಸಮಾಯುಕ್ತಶರೀರಾಯೈ ನಮೋ ನಮಃ ।\\
ಮಹಾದೇವರತೌತ್ಸುಕ್ಯಮಹಾದೇವ್ಯೈ ನಮೋ ನಮಃ ।

\as{ಆಪಃ॑ ಸೃ॒ಜಂತು॑ ಸ್ನಿ॒ಗ್ಧಾ॒ನಿ॒ ಚಿ॒ಕ್ಲೀ॒ತ ವ॑ಸ ಮೇ॒ ಗೃಹೇ ।\\
ನಿ ಚ॑ ದೇ॒ವೀಂ ಮಾ॒ತರಂ॒ ಶ್ರಿಯಂ॑ ವಾ॒ಸಯ॑ ಮೇ ಕು॒ಲೇ ॥ ೧೨॥}

೪ ಸಕಾರಾಖ್ಯಾಯೈ ನಮಃ ।\\
೪ ಸಮರಸಾಯೈ ನಮಃ ।\\
೪ ಸಕಲಾಗಮಸಂಸ್ತುತಾಯೈ ನಮಃ ।\\
೪ ಸರ್ವವೇದಾಂತ ತಾತ್ಪರ್ಯಭೂಮ್ಯೈ ನಮಃ ।\\
೪ ಸದಸದಾಶ್ರಯಾಯೈ ನಮಃ ।\\
೪ ಸಕಲಾಯೈ ನಮಃ ।\\
೪ ಸಚ್ಚಿದಾನಂದಾಯೈ ನಮಃ ।\\
೪ ಸಾಧ್ಯಾಯೈ ನಮಃ ।\\
೪ ಸದ್ಗತಿದಾಯಿನ್ಯೈ ನಮಃ ।\\
೪ ಸನಕಾದಿಮುನಿಧ್ಯೇಯಾಯೈ ನಮಃ ।\\
೪ ಸದಾಶಿವಕುಟುಂಬಿನ್ಯೈ ನಮಃ ।\\
೪ ಸಕಲಾಧಿಷ್ಠಾನರೂಪಾಯೈ ನಮಃ ।\\
೪ ಸತ್ಯರೂಪಾಯೈ ನಮಃ ।\\
೪ ಸಮಾಕೃತ್ಯೈ ನಮಃ ।\\
೪ ಸರ್ವಪ್ರಪಂಚನಿರ್ಮಾತ್ರ್ಯೈ ನಮಃ ।\\
೪ ಸಮಾನಾಧಿಕವರ್ಜಿತಾಯೈ ನಮಃ ।\\
೪ ಸರ್ವೋತ್ತುಂಗಾಯೈ ನಮಃ ।\\
೪ ಸಂಗಹೀನಾಯೈ ನಮಃ ।\\
೪ ಸಗುಣಾಯೈ ನಮಃ ।\\
೪ ಸಕಲೇಷ್ಟದಾಯೈ ನಮಃ ।


\as{ಆ॒ರ್ದ್ರಾಂ ಪು॒ಷ್ಕರಿ॑ಣೀಂ ಪು॒ಷ್ಟಿಂ॒ ಪಿಂ॒ಗ॒ಲಾಂ ಪ॑ದ್ಮಮಾ॒ಲಿನೀಂ ।\\
ಚಂ॒ದ್ರಾಂ ಹಿ॒ರಣ್ಮ॑ಯೀಂ ಲ॒ಕ್ಷ್ಮೀಂ ಜಾತ॑ವೇದೋ ಮ॒ ಆವ॑ಹ ॥ ೧೩॥}

೪ ಕಕಾರಿಣ್ಯೈ ನಮಃ ।\\
೪ ಕಾವ್ಯಲೋಲಾಯೈ ನಮಃ ।\\
೪ ಕಾಮೇಶ್ವರಮನೋಹರಾಯೈ ನಮಃ ।\\
೪ ಕಾಮೇಶ್ವರಪ್ರಾಣನಾಡ್ಯೈ ನಮಃ ।\\
೪ ಕಾಮೇಶೋತ್ಸಂಗವಾಸಿನ್ಯೈ ನಮಃ ।\\
೪ ಕಾಮೇಶ್ವರಾಲಿಂಗಿತಾಂಗ್ಯೈ ನಮಃ ।\\
೪ ಕಾಮೇಶ್ವರಸುಖಪ್ರದಾಯೈ ನಮಃ ।\\
೪ ಕಾಮೇಶ್ವರಪ್ರಣಯಿನ್ಯೈ ನಮಃ ।\\
೪ ಕಾಮೇಶ್ವರವಿಲಾಸಿನ್ಯೈ ನಮಃ ।\\
೪ ಕಾಮೇಶ್ವರತಪಸ್ಸಿದ್ಧ್ಯೈ ನಮಃ ।\\
೪ ಕಾಮೇಶ್ವರಮನಃಪ್ರಿಯಾಯೈ ನಮಃ ।\\
೪ ಕಾಮೇಶ್ವರಪ್ರಾಣನಾಥಾಯೈ ನಮಃ ।\\
೪ ಕಾಮೇಶ್ವರವಿಮೋಹಿನ್ಯೈ ನಮಃ ।\\
೪ ಕಾಮೇಶ್ವರಬ್ರಹ್ಮವಿದ್ಯಾಯೈ ನಮಃ ।\\
೪ ಕಾಮೇಶ್ವರಗೃಹೇಶ್ವರ್ಯೈ ನಮಃ ।\\
೪ ಕಾಮೇಶ್ವರಾಹ್ಲಾದಕರ್ಯೈ ನಮಃ ।\\
೪ ಕಾಮೇಶ್ವರಮಹೇಶ್ವರ್ಯೈ ನಮಃ ।\\
೪ ಕಾಮೇಶ್ವರ್ಯೈ ನಮಃ ।\\
೪ ಕಾಮಕೋಟಿನಿಲಯಾಯೈ ನಮಃ ।\\
೪ ಕಾಂಕ್ಷಿತಾರ್ಥದಾಯೈ ನಮಃ ।

\as{ಆ॒ರ್ದ್ರಾಂ ಯಃ॒ ಕರಿ॑ಣೀಂ ಯ॒ಷ್ಟಿಂ॒ ಸು॒ವ॒ರ್ಣಾಂ ಹೇ॑ಮಮಾ॒ಲಿನೀಂ ।\\
ಸೂ॒ರ್ಯಾಂ ಹಿ॒ರಣ್ಮ॑ಯೀಂ ಲ॒ಕ್ಷ್ಮೀಂ॒ ಜಾತ॑ವೇದೋ ಮ॒ ಆವಹ ॥ ೧೪॥}

೪ ಲಕಾರಿಣ್ಯೈ ನಮಃ ।\\
೪ ಲಬ್ಧರೂಪಾಯೈ ನಮಃ ।\\
೪ ಲಬ್ಧಧಿಯೇ ನಮಃ ।\\
೪ ಲಬ್ಧವಾಂಛಿತಾಯೈ ನಮಃ ।\\
೪ ಲಬ್ಧಪಾಪಮನೋದೂರಾಯೈ ನಮಃ ।\\
೪ ಲಬ್ಧಾಹಂಕಾರದುರ್ಗಮಾಯೈ ನಮಃ ।\\
೪ ಲಬ್ಧಶಕ್ತ್ಯೈ ನಮಃ ।\\
೪ ಲಬ್ಧದೇಹಾಯೈ ನಮಃ ।\\
೪ ಲಬ್ಧೈಶ್ವರ್ಯಸಮುನ್ನತ್ಯೈ ನಮಃ ।\\
೪ ಲಬ್ಧಬುದ್ಧಯೇ ನಮಃ ।\\
೪ ಲಬ್ಧಲೀಲಾಯೈ ನಮಃ ।\\
೪ ಲಬ್ಧಯೌವನಶಾಲಿನ್ಯೈ ನಮಃ ।\\
೪ ಲಬ್ಧಾತಿಶಯಸರ್ವಾಂಗಸೌಂದರ್ಯಾಯೈ ನಮಃ ।\\
೪ ಲಬ್ಧವಿಭ್ರಮಾಯೈ ನಮಃ ।\\
೪ ಲಬ್ಧರಾಗಾಯೈ ನಮಃ ।\\
೪ ಲಬ್ಧಪತ್ಯೈ ನಮಃ ।\\
೪ ಲಬ್ಧನಾನಾಗಮಸ್ಥಿತ್ಯೈ ನಮಃ ।\\
೪ ಲಬ್ಧಭೋಗಾಯೈ ನಮಃ ।\\
೪ ಲಬ್ಧಸುಖಾಯೈ ನಮಃ ।\\
೪ ಲಬ್ಧಹರ್ಷಾಭಿಪೂರಿತಾಯೈ ನಮಃ ।

\as{ತಾಂ ಮ॒ ಆವ॑ಹ॒ ಜಾತ॑ವೇದೋ ಲ॒ಕ್ಷ್ಮೀಮನ॑ಪಗಾ॒ಮಿನೀಂ᳚ ।\\
ಯಸ್ಯಾಂ॒ ಹಿ॑ರಣ್ಯಂ॒ ಪ್ರಭೂ॑ತಂ॒ ಗಾವೋ॑ ದಾ॒ಸ್ಯೋಽಶ್ವಾ᳚ನ್ವಿಂ॒ದೇಯಂ॒ ಪುರು॑ಷಾನ॒ಹಂ ॥ ೧೫॥}

೪ ಹ್ರೀಂಕಾರಮೂರ್ತಯೇ ನಮಃ ।\\
೪ ಹ್ರೀಂಕಾರಸೌಧಶೃಂಗಕಪೋತಿಕಾಯೈ ನಮಃ ।\\
೪ ಹ್ರೀಂಕಾರದುಗ್ಧಾಬ್ಧಿಸುಧಾಯೈ ನಮಃ ।\\
೪ ಹ್ರೀಂಕಾರಕಮಲೇಂದಿರಾಯೈ ನಮಃ ।\\
೪ ಹ್ರೀಂಕರಮಣಿದೀಪಾರ್ಚಿಷೇ ನಮಃ ।\\
೪ ಹ್ರೀಂಕಾರತರುಶಾರಿಕಾಯೈ ನಮಃ ।\\
೪ ಹ್ರೀಂಕಾರಪೇಟಕಮಣಯೇ ನಮಃ ।\\
೪ ಹ್ರೀಂಕಾರಾದರ್ಶಬಿಂಬಿತಾಯೈ ನಮಃ ।\\
೪ ಹ್ರೀಂಕಾರಕೋಶಾಸಿಲತಾಯೈ ನಮಃ ।\\
೪ ಹ್ರೀಂಕಾರಾಸ್ಥಾನನರ್ತಕ್ಯೈ ನಮಃ ।\\
೪ ಹ್ರೀಂಕಾರಶುಕ್ತಿಕಾ ಮುಕ್ತಾಮಣಯೇ ನಮಃ ।\\
೪ ಹ್ರೀಂಕಾರಬೋಧಿತಾಯೈ ನಮಃ ।\\
೪ ಹ್ರೀಂಕಾರಮಯಸೌವರ್ಣಸ್ತಂಭವಿದ್ರುಮಪುತ್ರಿಕಾಯೈ ನಮಃ ।\\
೪ ಹ್ರೀಂಕಾರವೇದೋಪನಿಷದೇ ನಮಃ ।\\
೪ ಹ್ರೀಂಕಾರಾಧ್ವರದಕ್ಷಿಣಾಯೈ ನಮಃ ।\\
೪ ಹ್ರೀಂಕಾರನಂದನಾರಾಮನವಕಲ್ಪಕ ವಲ್ಲರ್ಯೈ ನಮಃ ।\\
೪ ಹ್ರೀಂಕಾರಹಿಮವದ್ಗಂಗಾಯೈ ನಮಃ ।\\
೪ ಹ್ರೀಂಕಾರಾರ್ಣವಕೌಸ್ತುಭಾಯೈ ನಮಃ ।\\
೪ ಹ್ರೀಂಕಾರಮಂತ್ರಸರ್ವಸ್ವಾಯೈ ನಮಃ ।\\
೪ ಹ್ರೀಂಕಾರಪರಸೌಖ್ಯದಾಯೈ ನಮಃ ।

\as{ನ॒ ಕರ್ಮ॑ಣಾ ನ ಪ್ರ॒ಜಯಾ॒ ಧನೇ॑ನ॒ ತ್ಯಾಗೇ॑ನೈಕೇ ಅಮೃತ॒ತ್ವಮಾ॑ನ॒ಶುಃ ।
ಪರೇ॑ಣ॒ ನಾಕಂ॒ ನಿಹಿ॑ತಂ॒ ಗುಹಾ॑ಯಾಂ ವಿ॒ಭ್ರಾಜ॑ದೇ॒ತದ್ಯತ॑ಯೋ ವಿ॒ಶಂತಿ॑ ।
ವೇ॒ದಾಂ॒ತ॒ವಿ॒ಜ್ಞಾನ॒ಸುನಿ॑ಶ್ಚಿತಾ॒ರ್ಥಾಸ್ಸಂನ್ಯಾ॑ಸಯೋ॒ಗಾದ್ಯತ॑ಯಶ್ಶುದ್ಧ॒ಸತ್ತ್ವಾಃ᳚ ।
ತೇ ಬ್ರ॑ಹ್ಮಲೋ॒ಕೇ ತು॒ ಪರಾಂ᳚ತಕಾಲೇ॒ ಪರಾ॑ಮೃತಾ॒ತ್ಪರಿ॑ಮುಚ್ಯಂತಿ॒ ಸರ್ವೇ᳚ ।
ದ॒ಹ್ರಂ॒ ವಿ॒ಪಾ॒ಪಂ ಪ॒ರಮೇ᳚ಽಶ್ಮಭೂತಂ॒ ಯತ್ಪುಂ॑ಡರೀ॒ಕಂ ಪು॒ರಮ॑ಧ್ಯಸಁ॒ಸ್ಥಂ ।
ತ॒ತ್ರಾ॒ಪಿ॒ ದ॒ಹ್ರಂ ಗ॒ಗನಂ॑ ವಿಶೋಕ॒ಸ್ತಸ್ಮಿ॑ನ್ ಯದಂ॒ತಸ್ತ॒ದುಪಾ॑ಸಿತ॒ವ್ಯಂ ।
ಯೋ ವೇದಾದೌ ಸ್ವ॑ರಃ ಪ್ರೋ॒ಕ್ತೋ॒ ವೇ॒ದಾಂತೇ॑ ಚ ಪ್ರ॒ತಿಷ್ಠಿ॑ತಃ ।
ತಸ್ಯ॑ ಪ್ರ॒ಕೃತಿ॑ಲೀನ॒ಸ್ಯ॒ ಯಃ॒ ಪರ॑ಸ್ಸ ಮ॒ಹೇಶ್ವ॑ರಃ  ॥}
\section{ಬಿಂದುಚಕ್ರಾಯ ನಮಃ ।\\ (ಇತಿ ವ್ಯಾಪಕಂ ನ್ಯಸ್ಯ)}
ಹೃನ್ಮಧ್ಯರೂಪ ನಿರುಪಾಧಿಕ ಸಂವಿನ್ಮಾತ್ರರೂಪ ಕಾಮೇಶ್ವರಾಂಕ ನಿಲಯಾಯೈ ಸಚ್ಚಿದಾನಂದೈಕ ಬ್ರಹ್ಮಾತ್ಮಿಕಾಯೈ ಪರದೇವತಾಯೈ ಲಲಿತಾಯೈ ಮಹಾತ್ರಿಪುರಸುಂದರ್ಯೈ ನಮಃ ॥

ನಿರುಪಾಧಿಕ ಚೈತನ್ಯಮೇವ ಸಚ್ಚಿದಾನಂದಾತ್ಮಕಂ ಅನ್ತಃಕರಣಪ್ರತಿಬಿಂಬಿತಂ ಸತ್ ತದಹಮೇವೇತ್ಯನುಸಂಧಾನಂ ಲಲಿತಾಯಾ ಲೌಹಿತ್ಯಮಿತಿ ವಿಭಾವ್ಯ ॥

ಅಭೇದಸಂಬಂಧೇನ ಸತ್ವಚಿತ್ವಾದಿ ಧರ್ಮವಿಶಿಷ್ಟಸಂವಿದಃ ಕೇವಲಸಂವಿದಶ್ಚ ತಾದಾತ್ಮ್ಯಸಂಬಂಧರೂಪಂ ಕಾಮೇಶ್ವರಾಂಕಯಂತ್ರಣಂ ವಿಶೇಷಣಂ ವಿಭಾವ್ಯ ॥

ಉಪಾಧ್ಯಭಾವರೂಪಶುಕ್ಲತ್ವೋಪಲಕ್ಷಿತಾ ಸತೀ ಶುದ್ಧಸಂವಿದೇವ ಶುಕ್ಲಚರಣಃ ॥

ಚಿತ್ವವಿಶಿಷ್ಟಸಂವಿತ್ ಪ್ರಾಥಮಿಕ ಪರಾಹಂತಾತ್ಮಕ ಮೃತ್ಯುರೂಪೇಣ ರಾಗೇಣೋಪಲಕ್ಷಿತಾ ಸತೀ ರಕ್ತಚರಣಃ ॥

ಅಹಮಾಕಾರವೃತ್ತಿನಿರೂಪಿತಾ ವಿಷಯತಾ ಚರಣಯೋರ್ಮಿಥೋ ವಿಶೇಷಣವಿಶೇಷ್ಯಭಾವರೂಪೈವ ತದುಭಯಸಾಮರಸ್ಯಮಿತಿ ವಿಭಾವ್ಯ ॥

ಹೃದ್ರೂಪಸರ್ವಾನಂದಮಯಚಕ್ರೇಶ್ವರ್ಯೈ ಮಹಾತ್ರಿಪುರಸುಂದರ್ಯೈ ನಮಃ ॥

ಪರಾಪರರಹಸ್ಯ ಯೋಗಿನೀರೂಪ ಸ್ವಾತ್ಮಾತ್ಮನೇ ಪ್ರಾಪ್ತಿಸಿದ್ಧ್ಯೈ ನಮಃ ॥

ಅಪರಿಚ್ಛಿನ್ನರೂಪಸ್ವಾತ್ಮಾತ್ಮನೇ ಸರ್ವಯೋನಿಮುದ್ರಾಯೈ ನಮಃ ॥\\
ಇತಿ ತತ್ತತ್ಸ್ಥಾನಸ್ಪರ್ಶಪೂರ್ವಕಂ ಸಮ್ಯಗನುಸಂಧಾಯ ಉಪಚಾರಾನ್\\ ಸಮರ್ಪಯೇತ್ ॥
ತದ್ಯಥಾ  -

ಏವಮಪರಿಚ್ಛಿನ್ನತಯಾ ಭಾವಿತಾಯಾ ಲಲಿತಾಯಾಃ ಸ್ವೇ ಮಹಿಮ್ನ್ಯೇವ ಪ್ರತಿಷ್ಠಿತಮಾಸನಮನುಸಂದಧಾಮಿ॥

ವಿಯದಾದಿ ಸ್ಥೂಲಪ್ರಪಂಚರೂಪ ಪಾದಗತಸ್ಯ ಮಲಸ್ಯ ಸಚ್ಚಿದಾನಂದೈಕ ರೂಪತ್ವ ಭಾವನಾಜಲೇನ ಕ್ಷಾಲನಂ ಪಾದ್ಯಂ ಭಾವಯಾಮಿ ॥

ಸೂಕ್ಷ್ಮ ಪ್ರಪಂಚರೂಪಹಸ್ತಗತಸ್ಯ ತಸ್ಯ ಕ್ಷಾಲನಮರ್ಘ್ಯಂ ಚಿಂತಯಾಮಿ ॥

ಭಾವನಾರೂಪಾಣಾಮಪಾಮಪಿ ಕಬಲೀಕಾರರೂಪಮಾಚಮನಂ ಭಾವಯಾಮಿ ॥

ಸತ್ತ್ವ ಚಿತ್ತ್ವ ಆನಂದತ್ವಾದ್ಯಖಿಲಾವಯವಾವಚ್ಛೇದೇನ ಭಾವನಾಜಲಸಂಪರ್ಕರೂಪಂ ಸ್ನಾನಮನುಚಿಂತಯಾಮಿ ॥

ತೇಷ್ವೇವಾವಯವೇಷು ಪ್ರಸಕ್ತಾಯಾ ಭಾವನಾತ್ಮಕವೃತ್ತಿವಿಶೇಷ್ಯತಾಯಾಃ ಪ್ರೋಂಛನಂ ವೃತ್ತ್ಯವಿಷಯತ್ವಭಾವನೇನ ವಸ್ತ್ರಂ ಕಲ್ಪಯಾಮಿ ॥

ನಿರ್ವಿಷಯತ್ವ ನಿರಂಜನತ್ವ ಅಶೋಕತ್ವ ಅಮೃತತ್ವಾದ್ಯನೇಕ ಧರ್ಮರೂಪ ಆಭರಣಾನಿ ಧರ್ಮ್ಯಭೇದಭಾವನೇನ ಸಮರ್ಪಯಾಮಿ ॥

ಸ್ವಶರೀರಘಟಕ ಪಾರ್ಥಿವಭಾಗಾನಾಂ ಜಡತಾಪನಯನೇನ ಚಿನ್ಮಾತ್ರಾವಶೇಷರೂಪಂ ಗಂಧಂ ಪ್ರಯಚ್ಛಾಮಿ ॥

ಆಕಾಶಭಾಗಾನಾಂ ತಥಾ ಭಾವನೇನ ಪುಷ್ಪಾಣಿ ಸಮರ್ಪಯಾಮಿ ॥

ವಾಯವ್ಯಭಾಗಾನಾಂ ತಥಾ ಭಾವನಯಾ ಧೂಪಯಾಮಿ ॥

ತೈಜಸಭಾಗಾನಾಂ ತಥಾಕರಣೇನೋದ್ದೀಪಯಾಮಿ ॥

ಅಮೃತಭಾಗಾಂಸ್ತಥಾ ವಿಭಾವ್ಯ ನಿವೇದಯಾಮಿ ॥

ಷೋಡಶಾಂತೇಂದುಮಂಡಲಸ್ಯ ತಥಾ ಭಾವನೇನ ತಾಂಬೂಲಕಲ್ಪನಾಮಾಚರಾಮಿ ॥

ಪರಾಪಶ್ಯಂತ್ಯಾದಿನಿಖಿಲಶಬ್ದಾನಾಂ ನಾದದ್ವಾರಾ ಬ್ರಹ್ಮಣ್ಯುಪಸಂಹಾರಚಿಂತನೇನ ಸ್ತೌಮಿ ॥

ವಿಷಯೇಷು ಧಾವಮಾನಾನಾಂ ಚಿತ್ತವೃತ್ತೀನಾಂ ವಿಷಯಜಡತಾ ನಿರಾಸೇನ ಬ್ರಹ್ಮಣಿ ಪ್ರವಿಲಾಪನೇನ ಪ್ರದಕ್ಷಿಣೀಕರೋಮಿ ॥

ತಾಸಾಂ ವಿಷಯೇಭ್ಯಃ ಪರಾವರ್ತನೇನ ಬ್ರಹ್ಮೈಕಪ್ರವಣತಯಾ ಪ್ರಣಮಾಮಿ॥\\(ಇತ್ಯುಪಚರ್ಯಂ ಜುಹುಯಾತ್ ॥)

ವಿಹಿತಾವಿಹಿತವಿಷಯಾಃ ವೃತ್ತಯಃ ಉತ್ಪನ್ನಾಃ ಅಹಂ ತ್ವಂ ಗುರುರ್ದೇವತೇತ್ಯಾದಯಃ ತಾಃ ಸರ್ವಾಃ ಚಕ್ರರಾಜಸ್ಥ ಅನಂತಶಕ್ತಿಕದಂಬರೂಪಾಃ ತತ್ತತ್ಸೂಕ್ಷ್ಮರೂಪಾಃ ಯೇ ಯೇ ಸಂಸ್ಕಾರಾಃ ತತ್ತತ್ಸರ್ವಂ ಚಿನ್ಮಾತ್ರಮೇವೇತಿ ವಿಭಾವನಯಾ  ನಿರ್ವ್ಯುತ್ಥಾನಂ ಸ್ವಾತ್ಮನಿ ಜುಹೋಮಿ ॥

ಪ್ರಕೃತಭಾವನಾಸು ಯೇ ಗುರುಚರಣಾದಿ ಶಕ್ತಿಕದಂಬಾಂತಾ ವಿಷಯಾಸ್ತೇ ಸರ್ವೇಽಪಿ ಚಿನ್ಮಾತ್ರರೂಪಾಃ, ನ ಪರಸ್ಪರಂ ಭಿದ್ಯಂತೇ ಇತಿ ಭಾವನಯಾ ತರ್ಪಯಾಮಿ ॥

ತಿಥಿಚಕ್ರಮುಕ್ತರೂಪಂ ಕಾಲಚಕ್ರಂ ದೇಶಚಕ್ರಂ ಚ ಸರ್ವಮಸ್ತಿ ಭಾತಿ ಪ್ರಿಯಂ ಚ, ನ ತು ನಾಮರೂಪವತ್ । ಅತಃ ಸರ್ವಂ ಬ್ರಹ್ಮೈವೇತಿ ವಿಭಾವಯಾಮಿ ॥

ಅಥವಾ ಪೂರ್ವಲಿಖಿತಾಂ ನಿತ್ಯಾಭಾವನಾಮಿಹೈವ ಶ್ವಾಸಪ್ರವಿಲಾಪನಫಲಿಕಾಂ ಕುರ್ಯಾತ್ । ತೇನ ಮನಃ ಪವನಾತ್ಮನಾಂ ಐಕ್ಯನಿಭಾಲನೇನ ತ್ರೀನ್ಮುಹೂರ್ತಾನ್ ದ್ವಾವೇಕಂ ವಾ ಮುಹೂರ್ತಮವಿಚ್ಛಿನ್ನಂ ವ್ಯಾಪಯೇತ್ । ತಸ್ಯ ದೇವತಾತ್ಮೈಕ್ಯಸಿದ್ಧಿಃ ಚಿಂತಿತಕಾರ್ಯಾಣ್ಯಯತ್ನೇನ ಸಿದ್ಧ್ಯಂತಿ ॥
\begin{center}{\LARGE\bfseries ಸದ್ಗುರುಚರಣಾರವಿಂದಾರ್ಪಣಮಸ್ತು}\end{center}
